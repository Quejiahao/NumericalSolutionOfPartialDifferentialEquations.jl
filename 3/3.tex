% !TeX program	= xelatex
% !TeX encoding	= UTF-8

%-------------------- 文类 --------------------
\documentclass[UTF8, a4paper, 12pt, oneside, onecolumn]{article}

%-------------------- 宏包 --------------------
\input{../../template/usepackage}

%-------------------- 杂项 --------------------
\input{../../template/misc}
\def\homeworkName{双曲型偏微分方程的差分方法}
\hypersetup
{
	% 颜色
	colorlinks	= true,
	linkcolor	= black,
	urlcolor	= red,
	citecolor	= black,
	anchorcolor	= blue,
}

%-------------------- 字体设置 --------------------
\input{../../template/fonts}

%-------------------- 标题样式 --------------------
\input{../title}

%-------------------- 自定义符号 --------------------
\input{../../template/symbols}

%-------------------- 自定义环境 --------------------
\input{../../template/environments}

%-------------------- \item 编号 --------------------
\input{../../template/itemstyle}

%-------------------- 代码样式设置 --------------------
\input{../../template/codestyle}

%-------------------- PDF 元信息, 建模时注释掉 --------------------
\input{../pdfinfo}

%-------------------- 页眉页脚 --------------------
\input{../headersfooters}
\usepackage[ntheorem]{empheq}

%-------------------- 正文 --------------------
\begin{document}

\thispagestyle{plain}

%\columnseprule = 1pt	% 栏线
\begin{center}
	{\zihao{-2}\heiti \homeworkName} \\
	\vspace{1.5ex}
	{\zihao{-4}\fangsong 阙嘉豪\textsuperscript{\hyperref[auth:1]{1}}} \\
	{\zihao{6}\songti \label{auth:1}(1. 北京师范大学 数学科学学院, 北京~~100875)}
\end{center}

\zihao{5}

%\watermark{60}{10}{\currenttime}

\section{一阶双曲方程模型问题}

考虑 $\Omega := (0, 1)$ 上具有周期边界条件的一阶线性双曲方程初值问题:
\begin{equation}\label{equ:ModelProblemOne}
	\lb\begin{aligned}
		u_t + u_{x} &= 0,	&	0 < x < x_{\max },&~t > 0,\\%		\label{equ:ModelProblemOne1}\\
		u(x, 0) &= u^0(x),	&	0 \leq x \leq x_{\max }.&	\\
		% u(x, t) &= u\(x_{\max }, t\),	&	&~t > 0.%		\label{equ:ModelProblemOne3}
	\end{aligned}\rd
\end{equation}
其中 $x_{\max }$ 为给定的区间最大值, $u^0$ 为给定的初值函数. 还需给定边界条件, 后面的数值算例中会分别给出. 下面分别使用迎风格式, Lax-Wendroff 格式和 Beam-Warming 格式来求解问题 \eqref{equ:ModelProblemOne}.

\section{差分逼近}

\subsection{迎风格式}

对初值问题 \eqref{equ:ModelProblemOne}, 用关于时间的向前差分算子 $\dfrac{\D_{+t}}{\D t}$ 逼近 $\dfrac{\pa }{\pa t}$, 用 $\dfrac{\D_{-x}}{\D x}$ 逼近 $\dfrac{\pa }{\pa x}$ 后得到显式差分格式:
\begin{equation}
	\lb\begin{aligned}
		U_{j}^{m + 1} &= (1 - \nu ) U_j^m + \nu U_{j - 1}^m,	&	0 &\leq j \leq N,	&	0	\leq m \leq M,\\
		U_j^0 &= u^0(jh),	&	0 &\leq j \leq N,	&
	\end{aligned}\rd
\end{equation}
其中 $\nu = \dfrac{\tau }{h}$, $h = \dfrac{1}{N}$, $M = \l[\dfrac{t_{\max}}{\tau }\r]$.

\subsubsection{截断误差}

设问题 \eqref{equ:ModelProblemOne} 的解 $u$ 充分光滑, 利用 Taylor 展开式可得:
\begin{align*}
	T_j^m &= \dfrac{u_j^{m + 1} - u_j^m}{\tau } + \dfrac{u_j^{m} - u_{j - 1}^m}{h} - \l[\dfrac{\pa u}{\pa t} + \dfrac{\pa u}{\pa x}\r]_j^m\\
	&= \dfrac{1}{2} \l[\tau u_{tt} - hu_{xx}\r]_j^m + \dfrac{1}{6} \l[\tau u_{ttt} + hu_{xxx}\r]_j^m + \cdots\\
	&= -\l[\dfrac{1}{2} h(1 - \nu )u_{xx} + \dfrac{1}{6} h^2 \(1 - \nu^2 \) u_{xxx} + \cdots\r]\\
	&= O(h),
\end{align*}
可见迎风格式的局部阶段误差只有一阶精度. 更确切地说, 只要问题 \eqref{equ:ModelProblemOne} 的解 $u$ 的二阶导数有界, 就有
$$T_h = \max \lrb{\lrv{T_j^m}} = O(\tau + h).$$

\subsubsection{稳定性, 收敛性}

差分逼近解的误差 $e_j^m = U_j^m - u_j^m$ 满足方程
$$e_j^m = \(1 - |\nu|\) e_j^{m} + |\nu |e_{j - 1}^{m} - \tau T_j^m.$$
迎风格式满足 CFL 条件时, 即 $|\nu | \leq 1$ 时, 格式满足最大值原理. 于是有估计:
$$\max_j \lrv{e_j^{m + 1}} \leq \max_j \lrv{e_j^m} + \tau \max_j \lrv{T_j^m} \leq \max_j \lrv{e_j^0} + t_{\max } \max_{j, m} \lrv{T_{j}^m},\quad \forall (m + 1) \tau \leq t_{\max }.$$
即 $|\nu | \leq 1$ 时格式 $\mathbb{L}^\infty$ 稳定.

又将 Fourier 波形 $U_j^m := \lambda_k^m \e^{\i kjh}$ 代入格式, 有
$$\lrv{\lambda_k}^2 = \(\(1 - |\nu |\) + |\nu|\cos kh\)^2 + \( |\nu | \sin kh\)^2 = 1 - 4|\nu |\(1 - |\nu |\)\sin^2 \dfrac{1}{2}kh \leq 1,$$
即 $|\nu | \leq 1$ 时, 任取 $k \in \Z$ 都有 $\lrv{\lambda_k} \leq 1$, 格式 $\mathbb{L}^2$ 稳定.

从而, 当问题 \eqref{equ:ModelProblemOne} 的解 $u$ 的二阶导数有界时, 沿着任意满足 CFL 条件的加密路径迎风格式为收敛的且具有一阶逼近精度.

\subsection{Lax-Wendroff 格式}

由 Lax-Wendroff 格式及初值问题 \eqref{equ:ModelProblemOne} 得到递推公式:
\begin{equation}
	\lb\begin{aligned}
		U_{j}^{m + 1} &= -\dfrac{1}{2}\nu (1 - \nu ) U_{j + 1}^m + \(1 - \nu^2 \) U_{j}^m	&&&\\
		&\quad + \dfrac{1}{2} \nu (1 + \nu ) U_{j - 1}^m,	&	0 &\leq j \leq N,	&	0	\leq m \leq M,\\
		U_j^0 &= u^0(jh),	&	0 &\leq j \leq N.	&
	\end{aligned}\rd
\end{equation}

\subsubsection{截断误差}

由迎风格式已知
\begin{equation}\label{equ:lax_generate}
	\dfrac{u_j^{m + 1} - u_j^m}{\tau } + \dfrac{u_j^{m} - u_{j - 1}^m}{h} = - \dfrac{1}{2} h(1 - \nu )\l[ u_{xx} \r]_j^m + O\(h^2\).
\end{equation}
将 \eqref{equ:lax_generate} 式中 $\l[ u_{xx} \r]_j^m$ 用二阶中心差商 $\dfrac{\de_x^2 u_j^m}{h^2}$ 替代:
$$\l[ u_{xx} \r]_j^m = \dfrac{\de_x^2 u_j^m}{h^2} + O\(h^2 \),$$
则可得截断误差为二阶的 Lax-Wendroff 格式.

\subsubsection{稳定性, 收敛性}

易知, Lax-Wendroff 格式的 CFL 条件为 $|\nu | \leq 1$. 由于格式右端的系数不可能同号, Lax-Wendroff 格式不可能满足最大值原理, 故没有 $\mathbb{L}^\infty$ 稳定性.

又将 Fourier 波形 $U_j^m := \lambda_k^m \e^{\i kjh}$ 代入格式, 有
$$\lrv{\lambda_k}^2 = 1 - 4\nu^2 \(1 - \nu^2 \) \sin^4 \dfrac{kh}{2}.$$
又 $|\nu | \leq 1$ 时, 任取 $k \in \Z$ 都有 $\lrv{\lambda_k }^2 \leq 1$. 故 $|\nu | \leq 1$ 时 Lax-Wendroff 格式 $\mathbb{L}^2$ 稳定.

\subsection{Beam-Warming 格式}

由 Lax-Wendroff 格式及初值问题 \eqref{equ:ModelProblemOne} 得到递推公式:
\begin{equation}
	\lb\begin{aligned}
		U_{j}^{m + 1} &= \dfrac{1}{2} (1 - \nu )(2 - \nu ) U_{j}^m + \nu \(2 - \nu \) U_{j - 1}^m	&&&\\
		&\quad - \dfrac{1}{2} \nu (1 - \nu ) U_{j - 2}^m,	&	0 &\leq j \leq N,	&	0	\leq m \leq M,\\
		U_j^0 &= u^0(jh),	&	0 &\leq j \leq N.	&
	\end{aligned}\rd
\end{equation}

\subsubsection{截断误差}

将 \eqref{equ:lax_generate} 式中 $\l[ u_{xx} \r]_j^m$ 用二阶中心差商 $\dfrac{\de_x^2 u_{j - 1}^m}{h^2}$ 替代:
$$\l[ u_{xx} \r]_j^m = \dfrac{\de_x^2 u_{j - 1}^m}{h^2} + O\(h^2 \),$$
则可得截断误差为二阶的 Beam-Warming 格式.

\subsubsection{稳定性, 收敛性}

易知, Beam-Warming 格式的 CFL 条件为 $|\nu | \leq 1$. 由于格式右端的系数不可能同号, Beam-Warming 格式不可能满足最大值原理, 故没有 $\mathbb{L}^\infty$ 稳定性.

又将 Fourier 波形 $U_j^m := \lambda_k^m \e^{\i kjh}$ 代入格式, 有
$$\lrv{\lambda_k}^2 = 1 - 4\nu (2 - \nu ) \(1 - \nu^2 \) \sin^4 \dfrac{kh}{2}.$$
又 $0 \leq \nu \leq 2$ 时, 任取 $k \in \Z$ 都有 $\lrv{\lambda_k }^2 \leq 1$. 故 $|\nu | \leq 2$ 时 Beam-Warming 格式 $\mathbb{L}^2$ 稳定.

\section{数值实验}

下面用三种数值格式分别求解正弦波问题:
\begin{equation}\label{equ:sine_wave_con}
	\lb \begin{aligned}
		u_t + u_{x} &= 0,	&	0 < x &< x_{\max },&~t > 0,\\
		u^0(x) &= \sin(\pi x),	&	x_{\max } &= 1,&\\
		u(x, t) &= u\(x_{\max }, t\),	&&	&~t > 0.
	\end{aligned}\rd
\end{equation}
和方波问题:
\begin{equation}\label{equ:squa_wave_con}
	\lb \begin{aligned}
		u_t + u_{x} &= 0,\quad	0 < x < x_{\max }, t > 0,\\
		u^0(x) &= \lb \begin{array}{ll}
			3,	&	0 \leq x \leq 1,\\
			1,	&	1 < x \leq 3,
		\end{array} \rd\\
		u(0, t) &= 3,\quad u(3, t) = 1.
	\end{aligned} \rd
\end{equation}
解析解分别为 $u(x, t) = \sin\( \pi (x - t)\)$ 和
$$u(x, t) = \lb \begin{array}{ll}
	3,	&	t \leq x \leq t + 1,\\
	1,	&	t + 1 < x \leq 3,
\end{array} \rd \quad t < 2.$$

\subsection{迎风格式}

取 $\nu = 0.5 \leq 1$, 满足 CFL 条件. 对正弦波问题 \eqref{equ:sine_wave_con} 求解得到误差及收敛阶如表 \ref{tab:upwind_err} 所示.

\begin{table}[H]\centering\heiti\zihao{-5}
	\caption{迎风格式不同步长时的 $\mathbb{L}^2$, $\mathbb{L}^\infty$ 误差及收敛阶}\label{tab:upwind_err}
	\begin{tabular}{|c|c|c|c|c|}\hline
		收敛阶	&	$\mathbb{L}^2$ 误差	&	$h$	&	$\mathbb{L}^\infty$ 误差		&	收敛阶\\\hline
					&	$1.68359 \times 10^{-1}$	&	$2^{-4}$	&	$2.35319 \times 10^{-1}$	&			\\\hline
		0.820918	&	$9.53050 \times 10^{-2}$	&	$2^{-5}$	&	$1.34378 \times 10^{-1}$	&	0.808319\\\hline
		0.904664	&	$5.09078 \times 10^{-2}$	&	$2^{-6}$	&	$7.19393 \times 10^{-2}$	&	0.901445\\\hline
		0.950960	&	$2.63340 \times 10^{-2}$	&	$2^{-7}$	&	$3.72347 \times 10^{-2}$	&	0.950134\\\hline
		0.975148	&	$1.33958 \times 10^{-2}$	&	$2^{-8}$	&	$1.89436 \times 10^{-2}$	&	0.974938\\\hline
		0.987492	&	$6.75621 \times 10^{-3}$	&	$2^{-9}$	&	$9.55461 \times 10^{-3}$	&	0.987439\\\hline
		0.993726	&	$3.39283 \times 10^{-3}$	&	$2^{-10}$	&	$4.79817 \times 10^{-3}$	&	0.993713\\\hline
		0.996858	&	$1.70011 \times 10^{-3}$	&	$2^{-11}$	&	$2.40432 \times 10^{-3}$	&	0.996855\\\hline
		0.998428	&	$8.50984 \times 10^{-4}$	&	$2^{-12}$	&	$1.20347 \times 10^{-3}$	&	0.998427\\\hline
	\end{tabular}
\end{table}

由数值结果可以看出解序列逐步收敛到模型问题的解, 收敛阶趋于 1, 与理论结果相符. $h = 2^{-7}$ 和 $h = 2^{-11}$ 时差分逼近解 $U$ 与真解 $u$ 在 $t = t_{\max }$ 时刻图像如图 \ref{fig:upwind_Uu} 所示. 可以看出 $h = 2^{-7}$ 时出现了耗散.

\begin{figure}[H]\centering\zihao{-5}
	\resizebox{0.4\linewidth}{!}{%% Creator: Matplotlib, PGF backend
%%
%% To include the figure in your LaTeX document, write
%%   \input{<filename>.pgf}
%%
%% Make sure the required packages are loaded in your preamble
%%   \usepackage{pgf}
%%
%% Figures using additional raster images can only be included by \input if
%% they are in the same directory as the main LaTeX file. For loading figures
%% from other directories you can use the `import` package
%%   \usepackage{import}
%%
%% and then include the figures with
%%   \import{<path to file>}{<filename>.pgf}
%%
%% Matplotlib used the following preamble
%%   \usepackage{fontspec}
%%   \setmainfont{DejaVuSerif.ttf}[Path=\detokenize{/Users/quejiahao/.julia/conda/3/lib/python3.9/site-packages/matplotlib/mpl-data/fonts/ttf/}]
%%   \setsansfont{DejaVuSans.ttf}[Path=\detokenize{/Users/quejiahao/.julia/conda/3/lib/python3.9/site-packages/matplotlib/mpl-data/fonts/ttf/}]
%%   \setmonofont{DejaVuSansMono.ttf}[Path=\detokenize{/Users/quejiahao/.julia/conda/3/lib/python3.9/site-packages/matplotlib/mpl-data/fonts/ttf/}]
%%
\begingroup%
\makeatletter%
\begin{pgfpicture}%
\pgfpathrectangle{\pgfpointorigin}{\pgfqpoint{12.000000in}{8.000000in}}%
\pgfusepath{use as bounding box, clip}%
\begin{pgfscope}%
\pgfsetbuttcap%
\pgfsetmiterjoin%
\definecolor{currentfill}{rgb}{1.000000,1.000000,1.000000}%
\pgfsetfillcolor{currentfill}%
\pgfsetlinewidth{0.000000pt}%
\definecolor{currentstroke}{rgb}{1.000000,1.000000,1.000000}%
\pgfsetstrokecolor{currentstroke}%
\pgfsetdash{}{0pt}%
\pgfpathmoveto{\pgfqpoint{0.000000in}{0.000000in}}%
\pgfpathlineto{\pgfqpoint{12.000000in}{0.000000in}}%
\pgfpathlineto{\pgfqpoint{12.000000in}{8.000000in}}%
\pgfpathlineto{\pgfqpoint{0.000000in}{8.000000in}}%
\pgfpathclose%
\pgfusepath{fill}%
\end{pgfscope}%
\begin{pgfscope}%
\pgfsetbuttcap%
\pgfsetmiterjoin%
\definecolor{currentfill}{rgb}{1.000000,1.000000,1.000000}%
\pgfsetfillcolor{currentfill}%
\pgfsetlinewidth{0.000000pt}%
\definecolor{currentstroke}{rgb}{0.000000,0.000000,0.000000}%
\pgfsetstrokecolor{currentstroke}%
\pgfsetstrokeopacity{0.000000}%
\pgfsetdash{}{0pt}%
\pgfpathmoveto{\pgfqpoint{1.396958in}{1.247073in}}%
\pgfpathlineto{\pgfqpoint{11.921260in}{1.247073in}}%
\pgfpathlineto{\pgfqpoint{11.921260in}{7.921260in}}%
\pgfpathlineto{\pgfqpoint{1.396958in}{7.921260in}}%
\pgfpathclose%
\pgfusepath{fill}%
\end{pgfscope}%
\begin{pgfscope}%
\pgfpathrectangle{\pgfqpoint{1.396958in}{1.247073in}}{\pgfqpoint{10.524301in}{6.674186in}}%
\pgfusepath{clip}%
\pgfsetrectcap%
\pgfsetroundjoin%
\pgfsetlinewidth{0.501875pt}%
\definecolor{currentstroke}{rgb}{0.000000,0.000000,0.000000}%
\pgfsetstrokecolor{currentstroke}%
\pgfsetstrokeopacity{0.100000}%
\pgfsetdash{}{0pt}%
\pgfpathmoveto{\pgfqpoint{1.694816in}{1.247073in}}%
\pgfpathlineto{\pgfqpoint{1.694816in}{7.921260in}}%
\pgfusepath{stroke}%
\end{pgfscope}%
\begin{pgfscope}%
\pgfsetbuttcap%
\pgfsetroundjoin%
\definecolor{currentfill}{rgb}{0.000000,0.000000,0.000000}%
\pgfsetfillcolor{currentfill}%
\pgfsetlinewidth{0.501875pt}%
\definecolor{currentstroke}{rgb}{0.000000,0.000000,0.000000}%
\pgfsetstrokecolor{currentstroke}%
\pgfsetdash{}{0pt}%
\pgfsys@defobject{currentmarker}{\pgfqpoint{0.000000in}{0.000000in}}{\pgfqpoint{0.000000in}{0.034722in}}{%
\pgfpathmoveto{\pgfqpoint{0.000000in}{0.000000in}}%
\pgfpathlineto{\pgfqpoint{0.000000in}{0.034722in}}%
\pgfusepath{stroke,fill}%
}%
\begin{pgfscope}%
\pgfsys@transformshift{1.694816in}{1.247073in}%
\pgfsys@useobject{currentmarker}{}%
\end{pgfscope}%
\end{pgfscope}%
\begin{pgfscope}%
\definecolor{textcolor}{rgb}{0.000000,0.000000,0.000000}%
\pgfsetstrokecolor{textcolor}%
\pgfsetfillcolor{textcolor}%
\pgftext[x=1.694816in,y=1.198462in,,top]{\color{textcolor}\sffamily\fontsize{18.000000}{21.600000}\selectfont $\displaystyle 0$}%
\end{pgfscope}%
\begin{pgfscope}%
\pgfpathrectangle{\pgfqpoint{1.396958in}{1.247073in}}{\pgfqpoint{10.524301in}{6.674186in}}%
\pgfusepath{clip}%
\pgfsetrectcap%
\pgfsetroundjoin%
\pgfsetlinewidth{0.501875pt}%
\definecolor{currentstroke}{rgb}{0.000000,0.000000,0.000000}%
\pgfsetstrokecolor{currentstroke}%
\pgfsetstrokeopacity{0.100000}%
\pgfsetdash{}{0pt}%
\pgfpathmoveto{\pgfqpoint{3.275000in}{1.247073in}}%
\pgfpathlineto{\pgfqpoint{3.275000in}{7.921260in}}%
\pgfusepath{stroke}%
\end{pgfscope}%
\begin{pgfscope}%
\pgfsetbuttcap%
\pgfsetroundjoin%
\definecolor{currentfill}{rgb}{0.000000,0.000000,0.000000}%
\pgfsetfillcolor{currentfill}%
\pgfsetlinewidth{0.501875pt}%
\definecolor{currentstroke}{rgb}{0.000000,0.000000,0.000000}%
\pgfsetstrokecolor{currentstroke}%
\pgfsetdash{}{0pt}%
\pgfsys@defobject{currentmarker}{\pgfqpoint{0.000000in}{0.000000in}}{\pgfqpoint{0.000000in}{0.034722in}}{%
\pgfpathmoveto{\pgfqpoint{0.000000in}{0.000000in}}%
\pgfpathlineto{\pgfqpoint{0.000000in}{0.034722in}}%
\pgfusepath{stroke,fill}%
}%
\begin{pgfscope}%
\pgfsys@transformshift{3.275000in}{1.247073in}%
\pgfsys@useobject{currentmarker}{}%
\end{pgfscope}%
\end{pgfscope}%
\begin{pgfscope}%
\definecolor{textcolor}{rgb}{0.000000,0.000000,0.000000}%
\pgfsetstrokecolor{textcolor}%
\pgfsetfillcolor{textcolor}%
\pgftext[x=3.275000in,y=1.198462in,,top]{\color{textcolor}\sffamily\fontsize{18.000000}{21.600000}\selectfont $\displaystyle 1$}%
\end{pgfscope}%
\begin{pgfscope}%
\pgfpathrectangle{\pgfqpoint{1.396958in}{1.247073in}}{\pgfqpoint{10.524301in}{6.674186in}}%
\pgfusepath{clip}%
\pgfsetrectcap%
\pgfsetroundjoin%
\pgfsetlinewidth{0.501875pt}%
\definecolor{currentstroke}{rgb}{0.000000,0.000000,0.000000}%
\pgfsetstrokecolor{currentstroke}%
\pgfsetstrokeopacity{0.100000}%
\pgfsetdash{}{0pt}%
\pgfpathmoveto{\pgfqpoint{4.855183in}{1.247073in}}%
\pgfpathlineto{\pgfqpoint{4.855183in}{7.921260in}}%
\pgfusepath{stroke}%
\end{pgfscope}%
\begin{pgfscope}%
\pgfsetbuttcap%
\pgfsetroundjoin%
\definecolor{currentfill}{rgb}{0.000000,0.000000,0.000000}%
\pgfsetfillcolor{currentfill}%
\pgfsetlinewidth{0.501875pt}%
\definecolor{currentstroke}{rgb}{0.000000,0.000000,0.000000}%
\pgfsetstrokecolor{currentstroke}%
\pgfsetdash{}{0pt}%
\pgfsys@defobject{currentmarker}{\pgfqpoint{0.000000in}{0.000000in}}{\pgfqpoint{0.000000in}{0.034722in}}{%
\pgfpathmoveto{\pgfqpoint{0.000000in}{0.000000in}}%
\pgfpathlineto{\pgfqpoint{0.000000in}{0.034722in}}%
\pgfusepath{stroke,fill}%
}%
\begin{pgfscope}%
\pgfsys@transformshift{4.855183in}{1.247073in}%
\pgfsys@useobject{currentmarker}{}%
\end{pgfscope}%
\end{pgfscope}%
\begin{pgfscope}%
\definecolor{textcolor}{rgb}{0.000000,0.000000,0.000000}%
\pgfsetstrokecolor{textcolor}%
\pgfsetfillcolor{textcolor}%
\pgftext[x=4.855183in,y=1.198462in,,top]{\color{textcolor}\sffamily\fontsize{18.000000}{21.600000}\selectfont $\displaystyle 2$}%
\end{pgfscope}%
\begin{pgfscope}%
\pgfpathrectangle{\pgfqpoint{1.396958in}{1.247073in}}{\pgfqpoint{10.524301in}{6.674186in}}%
\pgfusepath{clip}%
\pgfsetrectcap%
\pgfsetroundjoin%
\pgfsetlinewidth{0.501875pt}%
\definecolor{currentstroke}{rgb}{0.000000,0.000000,0.000000}%
\pgfsetstrokecolor{currentstroke}%
\pgfsetstrokeopacity{0.100000}%
\pgfsetdash{}{0pt}%
\pgfpathmoveto{\pgfqpoint{6.435367in}{1.247073in}}%
\pgfpathlineto{\pgfqpoint{6.435367in}{7.921260in}}%
\pgfusepath{stroke}%
\end{pgfscope}%
\begin{pgfscope}%
\pgfsetbuttcap%
\pgfsetroundjoin%
\definecolor{currentfill}{rgb}{0.000000,0.000000,0.000000}%
\pgfsetfillcolor{currentfill}%
\pgfsetlinewidth{0.501875pt}%
\definecolor{currentstroke}{rgb}{0.000000,0.000000,0.000000}%
\pgfsetstrokecolor{currentstroke}%
\pgfsetdash{}{0pt}%
\pgfsys@defobject{currentmarker}{\pgfqpoint{0.000000in}{0.000000in}}{\pgfqpoint{0.000000in}{0.034722in}}{%
\pgfpathmoveto{\pgfqpoint{0.000000in}{0.000000in}}%
\pgfpathlineto{\pgfqpoint{0.000000in}{0.034722in}}%
\pgfusepath{stroke,fill}%
}%
\begin{pgfscope}%
\pgfsys@transformshift{6.435367in}{1.247073in}%
\pgfsys@useobject{currentmarker}{}%
\end{pgfscope}%
\end{pgfscope}%
\begin{pgfscope}%
\definecolor{textcolor}{rgb}{0.000000,0.000000,0.000000}%
\pgfsetstrokecolor{textcolor}%
\pgfsetfillcolor{textcolor}%
\pgftext[x=6.435367in,y=1.198462in,,top]{\color{textcolor}\sffamily\fontsize{18.000000}{21.600000}\selectfont $\displaystyle 3$}%
\end{pgfscope}%
\begin{pgfscope}%
\pgfpathrectangle{\pgfqpoint{1.396958in}{1.247073in}}{\pgfqpoint{10.524301in}{6.674186in}}%
\pgfusepath{clip}%
\pgfsetrectcap%
\pgfsetroundjoin%
\pgfsetlinewidth{0.501875pt}%
\definecolor{currentstroke}{rgb}{0.000000,0.000000,0.000000}%
\pgfsetstrokecolor{currentstroke}%
\pgfsetstrokeopacity{0.100000}%
\pgfsetdash{}{0pt}%
\pgfpathmoveto{\pgfqpoint{8.015550in}{1.247073in}}%
\pgfpathlineto{\pgfqpoint{8.015550in}{7.921260in}}%
\pgfusepath{stroke}%
\end{pgfscope}%
\begin{pgfscope}%
\pgfsetbuttcap%
\pgfsetroundjoin%
\definecolor{currentfill}{rgb}{0.000000,0.000000,0.000000}%
\pgfsetfillcolor{currentfill}%
\pgfsetlinewidth{0.501875pt}%
\definecolor{currentstroke}{rgb}{0.000000,0.000000,0.000000}%
\pgfsetstrokecolor{currentstroke}%
\pgfsetdash{}{0pt}%
\pgfsys@defobject{currentmarker}{\pgfqpoint{0.000000in}{0.000000in}}{\pgfqpoint{0.000000in}{0.034722in}}{%
\pgfpathmoveto{\pgfqpoint{0.000000in}{0.000000in}}%
\pgfpathlineto{\pgfqpoint{0.000000in}{0.034722in}}%
\pgfusepath{stroke,fill}%
}%
\begin{pgfscope}%
\pgfsys@transformshift{8.015550in}{1.247073in}%
\pgfsys@useobject{currentmarker}{}%
\end{pgfscope}%
\end{pgfscope}%
\begin{pgfscope}%
\definecolor{textcolor}{rgb}{0.000000,0.000000,0.000000}%
\pgfsetstrokecolor{textcolor}%
\pgfsetfillcolor{textcolor}%
\pgftext[x=8.015550in,y=1.198462in,,top]{\color{textcolor}\sffamily\fontsize{18.000000}{21.600000}\selectfont $\displaystyle 4$}%
\end{pgfscope}%
\begin{pgfscope}%
\pgfpathrectangle{\pgfqpoint{1.396958in}{1.247073in}}{\pgfqpoint{10.524301in}{6.674186in}}%
\pgfusepath{clip}%
\pgfsetrectcap%
\pgfsetroundjoin%
\pgfsetlinewidth{0.501875pt}%
\definecolor{currentstroke}{rgb}{0.000000,0.000000,0.000000}%
\pgfsetstrokecolor{currentstroke}%
\pgfsetstrokeopacity{0.100000}%
\pgfsetdash{}{0pt}%
\pgfpathmoveto{\pgfqpoint{9.595734in}{1.247073in}}%
\pgfpathlineto{\pgfqpoint{9.595734in}{7.921260in}}%
\pgfusepath{stroke}%
\end{pgfscope}%
\begin{pgfscope}%
\pgfsetbuttcap%
\pgfsetroundjoin%
\definecolor{currentfill}{rgb}{0.000000,0.000000,0.000000}%
\pgfsetfillcolor{currentfill}%
\pgfsetlinewidth{0.501875pt}%
\definecolor{currentstroke}{rgb}{0.000000,0.000000,0.000000}%
\pgfsetstrokecolor{currentstroke}%
\pgfsetdash{}{0pt}%
\pgfsys@defobject{currentmarker}{\pgfqpoint{0.000000in}{0.000000in}}{\pgfqpoint{0.000000in}{0.034722in}}{%
\pgfpathmoveto{\pgfqpoint{0.000000in}{0.000000in}}%
\pgfpathlineto{\pgfqpoint{0.000000in}{0.034722in}}%
\pgfusepath{stroke,fill}%
}%
\begin{pgfscope}%
\pgfsys@transformshift{9.595734in}{1.247073in}%
\pgfsys@useobject{currentmarker}{}%
\end{pgfscope}%
\end{pgfscope}%
\begin{pgfscope}%
\definecolor{textcolor}{rgb}{0.000000,0.000000,0.000000}%
\pgfsetstrokecolor{textcolor}%
\pgfsetfillcolor{textcolor}%
\pgftext[x=9.595734in,y=1.198462in,,top]{\color{textcolor}\sffamily\fontsize{18.000000}{21.600000}\selectfont $\displaystyle 5$}%
\end{pgfscope}%
\begin{pgfscope}%
\pgfpathrectangle{\pgfqpoint{1.396958in}{1.247073in}}{\pgfqpoint{10.524301in}{6.674186in}}%
\pgfusepath{clip}%
\pgfsetrectcap%
\pgfsetroundjoin%
\pgfsetlinewidth{0.501875pt}%
\definecolor{currentstroke}{rgb}{0.000000,0.000000,0.000000}%
\pgfsetstrokecolor{currentstroke}%
\pgfsetstrokeopacity{0.100000}%
\pgfsetdash{}{0pt}%
\pgfpathmoveto{\pgfqpoint{11.175917in}{1.247073in}}%
\pgfpathlineto{\pgfqpoint{11.175917in}{7.921260in}}%
\pgfusepath{stroke}%
\end{pgfscope}%
\begin{pgfscope}%
\pgfsetbuttcap%
\pgfsetroundjoin%
\definecolor{currentfill}{rgb}{0.000000,0.000000,0.000000}%
\pgfsetfillcolor{currentfill}%
\pgfsetlinewidth{0.501875pt}%
\definecolor{currentstroke}{rgb}{0.000000,0.000000,0.000000}%
\pgfsetstrokecolor{currentstroke}%
\pgfsetdash{}{0pt}%
\pgfsys@defobject{currentmarker}{\pgfqpoint{0.000000in}{0.000000in}}{\pgfqpoint{0.000000in}{0.034722in}}{%
\pgfpathmoveto{\pgfqpoint{0.000000in}{0.000000in}}%
\pgfpathlineto{\pgfqpoint{0.000000in}{0.034722in}}%
\pgfusepath{stroke,fill}%
}%
\begin{pgfscope}%
\pgfsys@transformshift{11.175917in}{1.247073in}%
\pgfsys@useobject{currentmarker}{}%
\end{pgfscope}%
\end{pgfscope}%
\begin{pgfscope}%
\definecolor{textcolor}{rgb}{0.000000,0.000000,0.000000}%
\pgfsetstrokecolor{textcolor}%
\pgfsetfillcolor{textcolor}%
\pgftext[x=11.175917in,y=1.198462in,,top]{\color{textcolor}\sffamily\fontsize{18.000000}{21.600000}\selectfont $\displaystyle 6$}%
\end{pgfscope}%
\begin{pgfscope}%
\definecolor{textcolor}{rgb}{0.000000,0.000000,0.000000}%
\pgfsetstrokecolor{textcolor}%
\pgfsetfillcolor{textcolor}%
\pgftext[x=6.659109in,y=0.900964in,,top]{\color{textcolor}\sffamily\fontsize{18.000000}{21.600000}\selectfont $\displaystyle x$}%
\end{pgfscope}%
\begin{pgfscope}%
\pgfpathrectangle{\pgfqpoint{1.396958in}{1.247073in}}{\pgfqpoint{10.524301in}{6.674186in}}%
\pgfusepath{clip}%
\pgfsetrectcap%
\pgfsetroundjoin%
\pgfsetlinewidth{0.501875pt}%
\definecolor{currentstroke}{rgb}{0.000000,0.000000,0.000000}%
\pgfsetstrokecolor{currentstroke}%
\pgfsetstrokeopacity{0.100000}%
\pgfsetdash{}{0pt}%
\pgfpathmoveto{\pgfqpoint{1.396958in}{1.435966in}}%
\pgfpathlineto{\pgfqpoint{11.921260in}{1.435966in}}%
\pgfusepath{stroke}%
\end{pgfscope}%
\begin{pgfscope}%
\pgfsetbuttcap%
\pgfsetroundjoin%
\definecolor{currentfill}{rgb}{0.000000,0.000000,0.000000}%
\pgfsetfillcolor{currentfill}%
\pgfsetlinewidth{0.501875pt}%
\definecolor{currentstroke}{rgb}{0.000000,0.000000,0.000000}%
\pgfsetstrokecolor{currentstroke}%
\pgfsetdash{}{0pt}%
\pgfsys@defobject{currentmarker}{\pgfqpoint{0.000000in}{0.000000in}}{\pgfqpoint{0.034722in}{0.000000in}}{%
\pgfpathmoveto{\pgfqpoint{0.000000in}{0.000000in}}%
\pgfpathlineto{\pgfqpoint{0.034722in}{0.000000in}}%
\pgfusepath{stroke,fill}%
}%
\begin{pgfscope}%
\pgfsys@transformshift{1.396958in}{1.435966in}%
\pgfsys@useobject{currentmarker}{}%
\end{pgfscope}%
\end{pgfscope}%
\begin{pgfscope}%
\definecolor{textcolor}{rgb}{0.000000,0.000000,0.000000}%
\pgfsetstrokecolor{textcolor}%
\pgfsetfillcolor{textcolor}%
\pgftext[x=0.876267in, y=1.340995in, left, base]{\color{textcolor}\sffamily\fontsize{18.000000}{21.600000}\selectfont $\displaystyle -1.0$}%
\end{pgfscope}%
\begin{pgfscope}%
\pgfpathrectangle{\pgfqpoint{1.396958in}{1.247073in}}{\pgfqpoint{10.524301in}{6.674186in}}%
\pgfusepath{clip}%
\pgfsetrectcap%
\pgfsetroundjoin%
\pgfsetlinewidth{0.501875pt}%
\definecolor{currentstroke}{rgb}{0.000000,0.000000,0.000000}%
\pgfsetstrokecolor{currentstroke}%
\pgfsetstrokeopacity{0.100000}%
\pgfsetdash{}{0pt}%
\pgfpathmoveto{\pgfqpoint{1.396958in}{3.010066in}}%
\pgfpathlineto{\pgfqpoint{11.921260in}{3.010066in}}%
\pgfusepath{stroke}%
\end{pgfscope}%
\begin{pgfscope}%
\pgfsetbuttcap%
\pgfsetroundjoin%
\definecolor{currentfill}{rgb}{0.000000,0.000000,0.000000}%
\pgfsetfillcolor{currentfill}%
\pgfsetlinewidth{0.501875pt}%
\definecolor{currentstroke}{rgb}{0.000000,0.000000,0.000000}%
\pgfsetstrokecolor{currentstroke}%
\pgfsetdash{}{0pt}%
\pgfsys@defobject{currentmarker}{\pgfqpoint{0.000000in}{0.000000in}}{\pgfqpoint{0.034722in}{0.000000in}}{%
\pgfpathmoveto{\pgfqpoint{0.000000in}{0.000000in}}%
\pgfpathlineto{\pgfqpoint{0.034722in}{0.000000in}}%
\pgfusepath{stroke,fill}%
}%
\begin{pgfscope}%
\pgfsys@transformshift{1.396958in}{3.010066in}%
\pgfsys@useobject{currentmarker}{}%
\end{pgfscope}%
\end{pgfscope}%
\begin{pgfscope}%
\definecolor{textcolor}{rgb}{0.000000,0.000000,0.000000}%
\pgfsetstrokecolor{textcolor}%
\pgfsetfillcolor{textcolor}%
\pgftext[x=0.876267in, y=2.915095in, left, base]{\color{textcolor}\sffamily\fontsize{18.000000}{21.600000}\selectfont $\displaystyle -0.5$}%
\end{pgfscope}%
\begin{pgfscope}%
\pgfpathrectangle{\pgfqpoint{1.396958in}{1.247073in}}{\pgfqpoint{10.524301in}{6.674186in}}%
\pgfusepath{clip}%
\pgfsetrectcap%
\pgfsetroundjoin%
\pgfsetlinewidth{0.501875pt}%
\definecolor{currentstroke}{rgb}{0.000000,0.000000,0.000000}%
\pgfsetstrokecolor{currentstroke}%
\pgfsetstrokeopacity{0.100000}%
\pgfsetdash{}{0pt}%
\pgfpathmoveto{\pgfqpoint{1.396958in}{4.584167in}}%
\pgfpathlineto{\pgfqpoint{11.921260in}{4.584167in}}%
\pgfusepath{stroke}%
\end{pgfscope}%
\begin{pgfscope}%
\pgfsetbuttcap%
\pgfsetroundjoin%
\definecolor{currentfill}{rgb}{0.000000,0.000000,0.000000}%
\pgfsetfillcolor{currentfill}%
\pgfsetlinewidth{0.501875pt}%
\definecolor{currentstroke}{rgb}{0.000000,0.000000,0.000000}%
\pgfsetstrokecolor{currentstroke}%
\pgfsetdash{}{0pt}%
\pgfsys@defobject{currentmarker}{\pgfqpoint{0.000000in}{0.000000in}}{\pgfqpoint{0.034722in}{0.000000in}}{%
\pgfpathmoveto{\pgfqpoint{0.000000in}{0.000000in}}%
\pgfpathlineto{\pgfqpoint{0.034722in}{0.000000in}}%
\pgfusepath{stroke,fill}%
}%
\begin{pgfscope}%
\pgfsys@transformshift{1.396958in}{4.584167in}%
\pgfsys@useobject{currentmarker}{}%
\end{pgfscope}%
\end{pgfscope}%
\begin{pgfscope}%
\definecolor{textcolor}{rgb}{0.000000,0.000000,0.000000}%
\pgfsetstrokecolor{textcolor}%
\pgfsetfillcolor{textcolor}%
\pgftext[x=1.062934in, y=4.489196in, left, base]{\color{textcolor}\sffamily\fontsize{18.000000}{21.600000}\selectfont $\displaystyle 0.0$}%
\end{pgfscope}%
\begin{pgfscope}%
\pgfpathrectangle{\pgfqpoint{1.396958in}{1.247073in}}{\pgfqpoint{10.524301in}{6.674186in}}%
\pgfusepath{clip}%
\pgfsetrectcap%
\pgfsetroundjoin%
\pgfsetlinewidth{0.501875pt}%
\definecolor{currentstroke}{rgb}{0.000000,0.000000,0.000000}%
\pgfsetstrokecolor{currentstroke}%
\pgfsetstrokeopacity{0.100000}%
\pgfsetdash{}{0pt}%
\pgfpathmoveto{\pgfqpoint{1.396958in}{6.158267in}}%
\pgfpathlineto{\pgfqpoint{11.921260in}{6.158267in}}%
\pgfusepath{stroke}%
\end{pgfscope}%
\begin{pgfscope}%
\pgfsetbuttcap%
\pgfsetroundjoin%
\definecolor{currentfill}{rgb}{0.000000,0.000000,0.000000}%
\pgfsetfillcolor{currentfill}%
\pgfsetlinewidth{0.501875pt}%
\definecolor{currentstroke}{rgb}{0.000000,0.000000,0.000000}%
\pgfsetstrokecolor{currentstroke}%
\pgfsetdash{}{0pt}%
\pgfsys@defobject{currentmarker}{\pgfqpoint{0.000000in}{0.000000in}}{\pgfqpoint{0.034722in}{0.000000in}}{%
\pgfpathmoveto{\pgfqpoint{0.000000in}{0.000000in}}%
\pgfpathlineto{\pgfqpoint{0.034722in}{0.000000in}}%
\pgfusepath{stroke,fill}%
}%
\begin{pgfscope}%
\pgfsys@transformshift{1.396958in}{6.158267in}%
\pgfsys@useobject{currentmarker}{}%
\end{pgfscope}%
\end{pgfscope}%
\begin{pgfscope}%
\definecolor{textcolor}{rgb}{0.000000,0.000000,0.000000}%
\pgfsetstrokecolor{textcolor}%
\pgfsetfillcolor{textcolor}%
\pgftext[x=1.062934in, y=6.063297in, left, base]{\color{textcolor}\sffamily\fontsize{18.000000}{21.600000}\selectfont $\displaystyle 0.5$}%
\end{pgfscope}%
\begin{pgfscope}%
\pgfpathrectangle{\pgfqpoint{1.396958in}{1.247073in}}{\pgfqpoint{10.524301in}{6.674186in}}%
\pgfusepath{clip}%
\pgfsetrectcap%
\pgfsetroundjoin%
\pgfsetlinewidth{0.501875pt}%
\definecolor{currentstroke}{rgb}{0.000000,0.000000,0.000000}%
\pgfsetstrokecolor{currentstroke}%
\pgfsetstrokeopacity{0.100000}%
\pgfsetdash{}{0pt}%
\pgfpathmoveto{\pgfqpoint{1.396958in}{7.732368in}}%
\pgfpathlineto{\pgfqpoint{11.921260in}{7.732368in}}%
\pgfusepath{stroke}%
\end{pgfscope}%
\begin{pgfscope}%
\pgfsetbuttcap%
\pgfsetroundjoin%
\definecolor{currentfill}{rgb}{0.000000,0.000000,0.000000}%
\pgfsetfillcolor{currentfill}%
\pgfsetlinewidth{0.501875pt}%
\definecolor{currentstroke}{rgb}{0.000000,0.000000,0.000000}%
\pgfsetstrokecolor{currentstroke}%
\pgfsetdash{}{0pt}%
\pgfsys@defobject{currentmarker}{\pgfqpoint{0.000000in}{0.000000in}}{\pgfqpoint{0.034722in}{0.000000in}}{%
\pgfpathmoveto{\pgfqpoint{0.000000in}{0.000000in}}%
\pgfpathlineto{\pgfqpoint{0.034722in}{0.000000in}}%
\pgfusepath{stroke,fill}%
}%
\begin{pgfscope}%
\pgfsys@transformshift{1.396958in}{7.732368in}%
\pgfsys@useobject{currentmarker}{}%
\end{pgfscope}%
\end{pgfscope}%
\begin{pgfscope}%
\definecolor{textcolor}{rgb}{0.000000,0.000000,0.000000}%
\pgfsetstrokecolor{textcolor}%
\pgfsetfillcolor{textcolor}%
\pgftext[x=1.062934in, y=7.637397in, left, base]{\color{textcolor}\sffamily\fontsize{18.000000}{21.600000}\selectfont $\displaystyle 1.0$}%
\end{pgfscope}%
\begin{pgfscope}%
\pgfpathrectangle{\pgfqpoint{1.396958in}{1.247073in}}{\pgfqpoint{10.524301in}{6.674186in}}%
\pgfusepath{clip}%
\pgfsetbuttcap%
\pgfsetroundjoin%
\pgfsetlinewidth{1.003750pt}%
\definecolor{currentstroke}{rgb}{0.000000,0.605603,0.978680}%
\pgfsetstrokecolor{currentstroke}%
\pgfsetdash{}{0pt}%
\pgfpathmoveto{\pgfqpoint{1.694816in}{4.581871in}}%
\pgfpathlineto{\pgfqpoint{1.927517in}{4.137159in}}%
\pgfpathlineto{\pgfqpoint{2.082651in}{3.845472in}}%
\pgfpathlineto{\pgfqpoint{2.237786in}{3.560899in}}%
\pgfpathlineto{\pgfqpoint{2.315353in}{3.422141in}}%
\pgfpathlineto{\pgfqpoint{2.392920in}{3.286181in}}%
\pgfpathlineto{\pgfqpoint{2.470487in}{3.153349in}}%
\pgfpathlineto{\pgfqpoint{2.548054in}{3.023963in}}%
\pgfpathlineto{\pgfqpoint{2.625621in}{2.898337in}}%
\pgfpathlineto{\pgfqpoint{2.703188in}{2.776771in}}%
\pgfpathlineto{\pgfqpoint{2.780755in}{2.659560in}}%
\pgfpathlineto{\pgfqpoint{2.858322in}{2.546985in}}%
\pgfpathlineto{\pgfqpoint{2.935889in}{2.439318in}}%
\pgfpathlineto{\pgfqpoint{3.013456in}{2.336818in}}%
\pgfpathlineto{\pgfqpoint{3.091023in}{2.239732in}}%
\pgfpathlineto{\pgfqpoint{3.168591in}{2.148294in}}%
\pgfpathlineto{\pgfqpoint{3.246158in}{2.062725in}}%
\pgfpathlineto{\pgfqpoint{3.323725in}{1.983229in}}%
\pgfpathlineto{\pgfqpoint{3.401292in}{1.910000in}}%
\pgfpathlineto{\pgfqpoint{3.478859in}{1.843213in}}%
\pgfpathlineto{\pgfqpoint{3.556426in}{1.783029in}}%
\pgfpathlineto{\pgfqpoint{3.633993in}{1.729593in}}%
\pgfpathlineto{\pgfqpoint{3.711560in}{1.683034in}}%
\pgfpathlineto{\pgfqpoint{3.789127in}{1.643465in}}%
\pgfpathlineto{\pgfqpoint{3.866694in}{1.610979in}}%
\pgfpathlineto{\pgfqpoint{3.944261in}{1.585657in}}%
\pgfpathlineto{\pgfqpoint{4.021828in}{1.567558in}}%
\pgfpathlineto{\pgfqpoint{4.099396in}{1.556726in}}%
\pgfpathlineto{\pgfqpoint{4.176963in}{1.553188in}}%
\pgfpathlineto{\pgfqpoint{4.254530in}{1.556951in}}%
\pgfpathlineto{\pgfqpoint{4.332097in}{1.568008in}}%
\pgfpathlineto{\pgfqpoint{4.409664in}{1.586330in}}%
\pgfpathlineto{\pgfqpoint{4.487231in}{1.611875in}}%
\pgfpathlineto{\pgfqpoint{4.564798in}{1.644580in}}%
\pgfpathlineto{\pgfqpoint{4.642365in}{1.684367in}}%
\pgfpathlineto{\pgfqpoint{4.719932in}{1.731140in}}%
\pgfpathlineto{\pgfqpoint{4.797499in}{1.784786in}}%
\pgfpathlineto{\pgfqpoint{4.875066in}{1.845176in}}%
\pgfpathlineto{\pgfqpoint{4.952633in}{1.912164in}}%
\pgfpathlineto{\pgfqpoint{5.030200in}{1.985589in}}%
\pgfpathlineto{\pgfqpoint{5.107768in}{2.065275in}}%
\pgfpathlineto{\pgfqpoint{5.185335in}{2.151029in}}%
\pgfpathlineto{\pgfqpoint{5.262902in}{2.242644in}}%
\pgfpathlineto{\pgfqpoint{5.340469in}{2.339901in}}%
\pgfpathlineto{\pgfqpoint{5.418036in}{2.442564in}}%
\pgfpathlineto{\pgfqpoint{5.495603in}{2.550386in}}%
\pgfpathlineto{\pgfqpoint{5.573170in}{2.663108in}}%
\pgfpathlineto{\pgfqpoint{5.650737in}{2.780458in}}%
\pgfpathlineto{\pgfqpoint{5.728304in}{2.902154in}}%
\pgfpathlineto{\pgfqpoint{5.805871in}{3.027901in}}%
\pgfpathlineto{\pgfqpoint{5.883438in}{3.157397in}}%
\pgfpathlineto{\pgfqpoint{5.961005in}{3.290331in}}%
\pgfpathlineto{\pgfqpoint{6.038573in}{3.426382in}}%
\pgfpathlineto{\pgfqpoint{6.116140in}{3.565222in}}%
\pgfpathlineto{\pgfqpoint{6.271274in}{3.849925in}}%
\pgfpathlineto{\pgfqpoint{6.426408in}{4.141700in}}%
\pgfpathlineto{\pgfqpoint{6.659109in}{4.586462in}}%
\pgfpathlineto{\pgfqpoint{6.891810in}{5.031174in}}%
\pgfpathlineto{\pgfqpoint{7.046945in}{5.322861in}}%
\pgfpathlineto{\pgfqpoint{7.202079in}{5.607434in}}%
\pgfpathlineto{\pgfqpoint{7.279646in}{5.746193in}}%
\pgfpathlineto{\pgfqpoint{7.357213in}{5.882152in}}%
\pgfpathlineto{\pgfqpoint{7.434780in}{6.014985in}}%
\pgfpathlineto{\pgfqpoint{7.512347in}{6.144370in}}%
\pgfpathlineto{\pgfqpoint{7.589914in}{6.269997in}}%
\pgfpathlineto{\pgfqpoint{7.667481in}{6.391562in}}%
\pgfpathlineto{\pgfqpoint{7.745048in}{6.508774in}}%
\pgfpathlineto{\pgfqpoint{7.822615in}{6.621348in}}%
\pgfpathlineto{\pgfqpoint{7.900182in}{6.729015in}}%
\pgfpathlineto{\pgfqpoint{7.977750in}{6.831515in}}%
\pgfpathlineto{\pgfqpoint{8.055317in}{6.928601in}}%
\pgfpathlineto{\pgfqpoint{8.132884in}{7.020039in}}%
\pgfpathlineto{\pgfqpoint{8.210451in}{7.105609in}}%
\pgfpathlineto{\pgfqpoint{8.288018in}{7.185104in}}%
\pgfpathlineto{\pgfqpoint{8.365585in}{7.258333in}}%
\pgfpathlineto{\pgfqpoint{8.443152in}{7.325121in}}%
\pgfpathlineto{\pgfqpoint{8.520719in}{7.385304in}}%
\pgfpathlineto{\pgfqpoint{8.598286in}{7.438740in}}%
\pgfpathlineto{\pgfqpoint{8.675853in}{7.485299in}}%
\pgfpathlineto{\pgfqpoint{8.753420in}{7.524869in}}%
\pgfpathlineto{\pgfqpoint{8.830987in}{7.557354in}}%
\pgfpathlineto{\pgfqpoint{8.908554in}{7.582677in}}%
\pgfpathlineto{\pgfqpoint{8.986122in}{7.600776in}}%
\pgfpathlineto{\pgfqpoint{9.063689in}{7.611607in}}%
\pgfpathlineto{\pgfqpoint{9.141256in}{7.615146in}}%
\pgfpathlineto{\pgfqpoint{9.218823in}{7.611382in}}%
\pgfpathlineto{\pgfqpoint{9.296390in}{7.600326in}}%
\pgfpathlineto{\pgfqpoint{9.373957in}{7.582003in}}%
\pgfpathlineto{\pgfqpoint{9.451524in}{7.556458in}}%
\pgfpathlineto{\pgfqpoint{9.529091in}{7.523753in}}%
\pgfpathlineto{\pgfqpoint{9.606658in}{7.483966in}}%
\pgfpathlineto{\pgfqpoint{9.684225in}{7.437194in}}%
\pgfpathlineto{\pgfqpoint{9.761792in}{7.383548in}}%
\pgfpathlineto{\pgfqpoint{9.839359in}{7.323158in}}%
\pgfpathlineto{\pgfqpoint{9.916927in}{7.256169in}}%
\pgfpathlineto{\pgfqpoint{9.994494in}{7.182744in}}%
\pgfpathlineto{\pgfqpoint{10.072061in}{7.103058in}}%
\pgfpathlineto{\pgfqpoint{10.149628in}{7.017304in}}%
\pgfpathlineto{\pgfqpoint{10.227195in}{6.925689in}}%
\pgfpathlineto{\pgfqpoint{10.304762in}{6.828432in}}%
\pgfpathlineto{\pgfqpoint{10.382329in}{6.725769in}}%
\pgfpathlineto{\pgfqpoint{10.459896in}{6.617947in}}%
\pgfpathlineto{\pgfqpoint{10.537463in}{6.505225in}}%
\pgfpathlineto{\pgfqpoint{10.615030in}{6.387875in}}%
\pgfpathlineto{\pgfqpoint{10.692597in}{6.266180in}}%
\pgfpathlineto{\pgfqpoint{10.770164in}{6.140432in}}%
\pgfpathlineto{\pgfqpoint{10.847731in}{6.010936in}}%
\pgfpathlineto{\pgfqpoint{10.925299in}{5.878002in}}%
\pgfpathlineto{\pgfqpoint{11.002866in}{5.741951in}}%
\pgfpathlineto{\pgfqpoint{11.080433in}{5.603112in}}%
\pgfpathlineto{\pgfqpoint{11.235567in}{5.318408in}}%
\pgfpathlineto{\pgfqpoint{11.390701in}{5.026633in}}%
\pgfpathlineto{\pgfqpoint{11.623402in}{4.581871in}}%
\pgfpathlineto{\pgfqpoint{11.623402in}{4.581871in}}%
\pgfusepath{stroke}%
\end{pgfscope}%
\begin{pgfscope}%
\pgfpathrectangle{\pgfqpoint{1.396958in}{1.247073in}}{\pgfqpoint{10.524301in}{6.674186in}}%
\pgfusepath{clip}%
\pgfsetbuttcap%
\pgfsetroundjoin%
\pgfsetlinewidth{1.003750pt}%
\definecolor{currentstroke}{rgb}{0.888874,0.435649,0.278123}%
\pgfsetstrokecolor{currentstroke}%
\pgfsetdash{}{0pt}%
\pgfpathmoveto{\pgfqpoint{1.694816in}{4.584167in}}%
\pgfpathlineto{\pgfqpoint{1.927517in}{4.122230in}}%
\pgfpathlineto{\pgfqpoint{2.082651in}{3.819216in}}%
\pgfpathlineto{\pgfqpoint{2.237786in}{3.523570in}}%
\pgfpathlineto{\pgfqpoint{2.315353in}{3.379402in}}%
\pgfpathlineto{\pgfqpoint{2.392920in}{3.238137in}}%
\pgfpathlineto{\pgfqpoint{2.470487in}{3.100115in}}%
\pgfpathlineto{\pgfqpoint{2.548054in}{2.965668in}}%
\pgfpathlineto{\pgfqpoint{2.625621in}{2.835120in}}%
\pgfpathlineto{\pgfqpoint{2.703188in}{2.708785in}}%
\pgfpathlineto{\pgfqpoint{2.780755in}{2.586969in}}%
\pgfpathlineto{\pgfqpoint{2.858322in}{2.469964in}}%
\pgfpathlineto{\pgfqpoint{2.935889in}{2.358052in}}%
\pgfpathlineto{\pgfqpoint{3.013456in}{2.251504in}}%
\pgfpathlineto{\pgfqpoint{3.091023in}{2.150574in}}%
\pgfpathlineto{\pgfqpoint{3.168591in}{2.055508in}}%
\pgfpathlineto{\pgfqpoint{3.246158in}{1.966533in}}%
\pgfpathlineto{\pgfqpoint{3.323725in}{1.883865in}}%
\pgfpathlineto{\pgfqpoint{3.401292in}{1.807701in}}%
\pgfpathlineto{\pgfqpoint{3.478859in}{1.738227in}}%
\pgfpathlineto{\pgfqpoint{3.556426in}{1.675608in}}%
\pgfpathlineto{\pgfqpoint{3.633993in}{1.619997in}}%
\pgfpathlineto{\pgfqpoint{3.711560in}{1.571526in}}%
\pgfpathlineto{\pgfqpoint{3.789127in}{1.530313in}}%
\pgfpathlineto{\pgfqpoint{3.866694in}{1.496457in}}%
\pgfpathlineto{\pgfqpoint{3.944261in}{1.470040in}}%
\pgfpathlineto{\pgfqpoint{4.021828in}{1.451125in}}%
\pgfpathlineto{\pgfqpoint{4.099396in}{1.439758in}}%
\pgfpathlineto{\pgfqpoint{4.176963in}{1.435966in}}%
\pgfpathlineto{\pgfqpoint{4.254530in}{1.439758in}}%
\pgfpathlineto{\pgfqpoint{4.332097in}{1.451125in}}%
\pgfpathlineto{\pgfqpoint{4.409664in}{1.470040in}}%
\pgfpathlineto{\pgfqpoint{4.487231in}{1.496457in}}%
\pgfpathlineto{\pgfqpoint{4.564798in}{1.530313in}}%
\pgfpathlineto{\pgfqpoint{4.642365in}{1.571526in}}%
\pgfpathlineto{\pgfqpoint{4.719932in}{1.619997in}}%
\pgfpathlineto{\pgfqpoint{4.797499in}{1.675608in}}%
\pgfpathlineto{\pgfqpoint{4.875066in}{1.738227in}}%
\pgfpathlineto{\pgfqpoint{4.952633in}{1.807701in}}%
\pgfpathlineto{\pgfqpoint{5.030200in}{1.883865in}}%
\pgfpathlineto{\pgfqpoint{5.107768in}{1.966533in}}%
\pgfpathlineto{\pgfqpoint{5.185335in}{2.055508in}}%
\pgfpathlineto{\pgfqpoint{5.262902in}{2.150574in}}%
\pgfpathlineto{\pgfqpoint{5.340469in}{2.251504in}}%
\pgfpathlineto{\pgfqpoint{5.418036in}{2.358052in}}%
\pgfpathlineto{\pgfqpoint{5.495603in}{2.469964in}}%
\pgfpathlineto{\pgfqpoint{5.573170in}{2.586969in}}%
\pgfpathlineto{\pgfqpoint{5.650737in}{2.708785in}}%
\pgfpathlineto{\pgfqpoint{5.728304in}{2.835120in}}%
\pgfpathlineto{\pgfqpoint{5.805871in}{2.965668in}}%
\pgfpathlineto{\pgfqpoint{5.883438in}{3.100115in}}%
\pgfpathlineto{\pgfqpoint{5.961005in}{3.238137in}}%
\pgfpathlineto{\pgfqpoint{6.038573in}{3.379402in}}%
\pgfpathlineto{\pgfqpoint{6.116140in}{3.523570in}}%
\pgfpathlineto{\pgfqpoint{6.271274in}{3.819216in}}%
\pgfpathlineto{\pgfqpoint{6.426408in}{4.122230in}}%
\pgfpathlineto{\pgfqpoint{6.659109in}{4.584167in}}%
\pgfpathlineto{\pgfqpoint{6.891810in}{5.046104in}}%
\pgfpathlineto{\pgfqpoint{7.046945in}{5.349117in}}%
\pgfpathlineto{\pgfqpoint{7.202079in}{5.644764in}}%
\pgfpathlineto{\pgfqpoint{7.279646in}{5.788931in}}%
\pgfpathlineto{\pgfqpoint{7.357213in}{5.930196in}}%
\pgfpathlineto{\pgfqpoint{7.434780in}{6.068218in}}%
\pgfpathlineto{\pgfqpoint{7.512347in}{6.202665in}}%
\pgfpathlineto{\pgfqpoint{7.589914in}{6.333213in}}%
\pgfpathlineto{\pgfqpoint{7.667481in}{6.459548in}}%
\pgfpathlineto{\pgfqpoint{7.745048in}{6.581364in}}%
\pgfpathlineto{\pgfqpoint{7.822615in}{6.698369in}}%
\pgfpathlineto{\pgfqpoint{7.900182in}{6.810281in}}%
\pgfpathlineto{\pgfqpoint{7.977750in}{6.916830in}}%
\pgfpathlineto{\pgfqpoint{8.055317in}{7.017759in}}%
\pgfpathlineto{\pgfqpoint{8.132884in}{7.112826in}}%
\pgfpathlineto{\pgfqpoint{8.210451in}{7.201800in}}%
\pgfpathlineto{\pgfqpoint{8.288018in}{7.284469in}}%
\pgfpathlineto{\pgfqpoint{8.365585in}{7.360632in}}%
\pgfpathlineto{\pgfqpoint{8.443152in}{7.430107in}}%
\pgfpathlineto{\pgfqpoint{8.520719in}{7.492725in}}%
\pgfpathlineto{\pgfqpoint{8.598286in}{7.548337in}}%
\pgfpathlineto{\pgfqpoint{8.675853in}{7.596807in}}%
\pgfpathlineto{\pgfqpoint{8.753420in}{7.638020in}}%
\pgfpathlineto{\pgfqpoint{8.830987in}{7.671876in}}%
\pgfpathlineto{\pgfqpoint{8.908554in}{7.698293in}}%
\pgfpathlineto{\pgfqpoint{8.986122in}{7.717208in}}%
\pgfpathlineto{\pgfqpoint{9.063689in}{7.728576in}}%
\pgfpathlineto{\pgfqpoint{9.141256in}{7.732368in}}%
\pgfpathlineto{\pgfqpoint{9.218823in}{7.728576in}}%
\pgfpathlineto{\pgfqpoint{9.296390in}{7.717208in}}%
\pgfpathlineto{\pgfqpoint{9.373957in}{7.698293in}}%
\pgfpathlineto{\pgfqpoint{9.451524in}{7.671876in}}%
\pgfpathlineto{\pgfqpoint{9.529091in}{7.638020in}}%
\pgfpathlineto{\pgfqpoint{9.606658in}{7.596807in}}%
\pgfpathlineto{\pgfqpoint{9.684225in}{7.548337in}}%
\pgfpathlineto{\pgfqpoint{9.761792in}{7.492725in}}%
\pgfpathlineto{\pgfqpoint{9.839359in}{7.430107in}}%
\pgfpathlineto{\pgfqpoint{9.916927in}{7.360632in}}%
\pgfpathlineto{\pgfqpoint{9.994494in}{7.284469in}}%
\pgfpathlineto{\pgfqpoint{10.072061in}{7.201800in}}%
\pgfpathlineto{\pgfqpoint{10.149628in}{7.112826in}}%
\pgfpathlineto{\pgfqpoint{10.227195in}{7.017759in}}%
\pgfpathlineto{\pgfqpoint{10.304762in}{6.916830in}}%
\pgfpathlineto{\pgfqpoint{10.382329in}{6.810281in}}%
\pgfpathlineto{\pgfqpoint{10.459896in}{6.698369in}}%
\pgfpathlineto{\pgfqpoint{10.537463in}{6.581364in}}%
\pgfpathlineto{\pgfqpoint{10.615030in}{6.459548in}}%
\pgfpathlineto{\pgfqpoint{10.692597in}{6.333213in}}%
\pgfpathlineto{\pgfqpoint{10.770164in}{6.202665in}}%
\pgfpathlineto{\pgfqpoint{10.847731in}{6.068218in}}%
\pgfpathlineto{\pgfqpoint{10.925299in}{5.930196in}}%
\pgfpathlineto{\pgfqpoint{11.002866in}{5.788931in}}%
\pgfpathlineto{\pgfqpoint{11.080433in}{5.644764in}}%
\pgfpathlineto{\pgfqpoint{11.235567in}{5.349117in}}%
\pgfpathlineto{\pgfqpoint{11.390701in}{5.046104in}}%
\pgfpathlineto{\pgfqpoint{11.623402in}{4.584167in}}%
\pgfpathlineto{\pgfqpoint{11.623402in}{4.584167in}}%
\pgfusepath{stroke}%
\end{pgfscope}%
\begin{pgfscope}%
\pgfsetrectcap%
\pgfsetmiterjoin%
\pgfsetlinewidth{1.003750pt}%
\definecolor{currentstroke}{rgb}{0.000000,0.000000,0.000000}%
\pgfsetstrokecolor{currentstroke}%
\pgfsetdash{}{0pt}%
\pgfpathmoveto{\pgfqpoint{1.396958in}{1.247073in}}%
\pgfpathlineto{\pgfqpoint{1.396958in}{7.921260in}}%
\pgfusepath{stroke}%
\end{pgfscope}%
\begin{pgfscope}%
\pgfsetrectcap%
\pgfsetmiterjoin%
\pgfsetlinewidth{1.003750pt}%
\definecolor{currentstroke}{rgb}{0.000000,0.000000,0.000000}%
\pgfsetstrokecolor{currentstroke}%
\pgfsetdash{}{0pt}%
\pgfpathmoveto{\pgfqpoint{1.396958in}{1.247073in}}%
\pgfpathlineto{\pgfqpoint{11.921260in}{1.247073in}}%
\pgfusepath{stroke}%
\end{pgfscope}%
\begin{pgfscope}%
\pgfsetbuttcap%
\pgfsetmiterjoin%
\definecolor{currentfill}{rgb}{1.000000,1.000000,1.000000}%
\pgfsetfillcolor{currentfill}%
\pgfsetlinewidth{1.003750pt}%
\definecolor{currentstroke}{rgb}{0.000000,0.000000,0.000000}%
\pgfsetstrokecolor{currentstroke}%
\pgfsetdash{}{0pt}%
\pgfpathmoveto{\pgfqpoint{10.511589in}{6.787373in}}%
\pgfpathlineto{\pgfqpoint{11.796260in}{6.787373in}}%
\pgfpathlineto{\pgfqpoint{11.796260in}{7.796260in}}%
\pgfpathlineto{\pgfqpoint{10.511589in}{7.796260in}}%
\pgfpathclose%
\pgfusepath{stroke,fill}%
\end{pgfscope}%
\begin{pgfscope}%
\pgfsetbuttcap%
\pgfsetmiterjoin%
\pgfsetlinewidth{2.258437pt}%
\definecolor{currentstroke}{rgb}{0.000000,0.605603,0.978680}%
\pgfsetstrokecolor{currentstroke}%
\pgfsetdash{}{0pt}%
\pgfpathmoveto{\pgfqpoint{10.711589in}{7.493818in}}%
\pgfpathlineto{\pgfqpoint{11.211589in}{7.493818in}}%
\pgfusepath{stroke}%
\end{pgfscope}%
\begin{pgfscope}%
\definecolor{textcolor}{rgb}{0.000000,0.000000,0.000000}%
\pgfsetstrokecolor{textcolor}%
\pgfsetfillcolor{textcolor}%
\pgftext[x=11.411589in,y=7.406318in,left,base]{\color{textcolor}\sffamily\fontsize{18.000000}{21.600000}\selectfont $\displaystyle U$}%
\end{pgfscope}%
\begin{pgfscope}%
\pgfsetbuttcap%
\pgfsetmiterjoin%
\pgfsetlinewidth{2.258437pt}%
\definecolor{currentstroke}{rgb}{0.888874,0.435649,0.278123}%
\pgfsetstrokecolor{currentstroke}%
\pgfsetdash{}{0pt}%
\pgfpathmoveto{\pgfqpoint{10.711589in}{7.126875in}}%
\pgfpathlineto{\pgfqpoint{11.211589in}{7.126875in}}%
\pgfusepath{stroke}%
\end{pgfscope}%
\begin{pgfscope}%
\definecolor{textcolor}{rgb}{0.000000,0.000000,0.000000}%
\pgfsetstrokecolor{textcolor}%
\pgfsetfillcolor{textcolor}%
\pgftext[x=11.411589in,y=7.039375in,left,base]{\color{textcolor}\sffamily\fontsize{18.000000}{21.600000}\selectfont $\displaystyle u$}%
\end{pgfscope}%
\end{pgfpicture}%
\makeatother%
\endgroup%
}\quad
	\resizebox{0.4\linewidth}{!}{%% Creator: Matplotlib, PGF backend
%%
%% To include the figure in your LaTeX document, write
%%   \input{<filename>.pgf}
%%
%% Make sure the required packages are loaded in your preamble
%%   \usepackage{pgf}
%%
%% Figures using additional raster images can only be included by \input if
%% they are in the same directory as the main LaTeX file. For loading figures
%% from other directories you can use the `import` package
%%   \usepackage{import}
%%
%% and then include the figures with
%%   \import{<path to file>}{<filename>.pgf}
%%
%% Matplotlib used the following preamble
%%   \usepackage{fontspec}
%%   \setmainfont{DejaVuSerif.ttf}[Path=\detokenize{/Users/quejiahao/.julia/conda/3/lib/python3.9/site-packages/matplotlib/mpl-data/fonts/ttf/}]
%%   \setsansfont{DejaVuSans.ttf}[Path=\detokenize{/Users/quejiahao/.julia/conda/3/lib/python3.9/site-packages/matplotlib/mpl-data/fonts/ttf/}]
%%   \setmonofont{DejaVuSansMono.ttf}[Path=\detokenize{/Users/quejiahao/.julia/conda/3/lib/python3.9/site-packages/matplotlib/mpl-data/fonts/ttf/}]
%%
\begingroup%
\makeatletter%
\begin{pgfpicture}%
\pgfpathrectangle{\pgfpointorigin}{\pgfqpoint{12.000000in}{8.000000in}}%
\pgfusepath{use as bounding box, clip}%
\begin{pgfscope}%
\pgfsetbuttcap%
\pgfsetmiterjoin%
\definecolor{currentfill}{rgb}{1.000000,1.000000,1.000000}%
\pgfsetfillcolor{currentfill}%
\pgfsetlinewidth{0.000000pt}%
\definecolor{currentstroke}{rgb}{1.000000,1.000000,1.000000}%
\pgfsetstrokecolor{currentstroke}%
\pgfsetdash{}{0pt}%
\pgfpathmoveto{\pgfqpoint{0.000000in}{0.000000in}}%
\pgfpathlineto{\pgfqpoint{12.000000in}{0.000000in}}%
\pgfpathlineto{\pgfqpoint{12.000000in}{8.000000in}}%
\pgfpathlineto{\pgfqpoint{0.000000in}{8.000000in}}%
\pgfpathclose%
\pgfusepath{fill}%
\end{pgfscope}%
\begin{pgfscope}%
\pgfsetbuttcap%
\pgfsetmiterjoin%
\definecolor{currentfill}{rgb}{1.000000,1.000000,1.000000}%
\pgfsetfillcolor{currentfill}%
\pgfsetlinewidth{0.000000pt}%
\definecolor{currentstroke}{rgb}{0.000000,0.000000,0.000000}%
\pgfsetstrokecolor{currentstroke}%
\pgfsetstrokeopacity{0.000000}%
\pgfsetdash{}{0pt}%
\pgfpathmoveto{\pgfqpoint{1.396958in}{1.247073in}}%
\pgfpathlineto{\pgfqpoint{11.921260in}{1.247073in}}%
\pgfpathlineto{\pgfqpoint{11.921260in}{7.921260in}}%
\pgfpathlineto{\pgfqpoint{1.396958in}{7.921260in}}%
\pgfpathclose%
\pgfusepath{fill}%
\end{pgfscope}%
\begin{pgfscope}%
\pgfpathrectangle{\pgfqpoint{1.396958in}{1.247073in}}{\pgfqpoint{10.524301in}{6.674186in}}%
\pgfusepath{clip}%
\pgfsetrectcap%
\pgfsetroundjoin%
\pgfsetlinewidth{0.501875pt}%
\definecolor{currentstroke}{rgb}{0.000000,0.000000,0.000000}%
\pgfsetstrokecolor{currentstroke}%
\pgfsetstrokeopacity{0.100000}%
\pgfsetdash{}{0pt}%
\pgfpathmoveto{\pgfqpoint{1.694816in}{1.247073in}}%
\pgfpathlineto{\pgfqpoint{1.694816in}{7.921260in}}%
\pgfusepath{stroke}%
\end{pgfscope}%
\begin{pgfscope}%
\pgfsetbuttcap%
\pgfsetroundjoin%
\definecolor{currentfill}{rgb}{0.000000,0.000000,0.000000}%
\pgfsetfillcolor{currentfill}%
\pgfsetlinewidth{0.501875pt}%
\definecolor{currentstroke}{rgb}{0.000000,0.000000,0.000000}%
\pgfsetstrokecolor{currentstroke}%
\pgfsetdash{}{0pt}%
\pgfsys@defobject{currentmarker}{\pgfqpoint{0.000000in}{0.000000in}}{\pgfqpoint{0.000000in}{0.034722in}}{%
\pgfpathmoveto{\pgfqpoint{0.000000in}{0.000000in}}%
\pgfpathlineto{\pgfqpoint{0.000000in}{0.034722in}}%
\pgfusepath{stroke,fill}%
}%
\begin{pgfscope}%
\pgfsys@transformshift{1.694816in}{1.247073in}%
\pgfsys@useobject{currentmarker}{}%
\end{pgfscope}%
\end{pgfscope}%
\begin{pgfscope}%
\definecolor{textcolor}{rgb}{0.000000,0.000000,0.000000}%
\pgfsetstrokecolor{textcolor}%
\pgfsetfillcolor{textcolor}%
\pgftext[x=1.694816in,y=1.198462in,,top]{\color{textcolor}\sffamily\fontsize{18.000000}{21.600000}\selectfont $\displaystyle 0$}%
\end{pgfscope}%
\begin{pgfscope}%
\pgfpathrectangle{\pgfqpoint{1.396958in}{1.247073in}}{\pgfqpoint{10.524301in}{6.674186in}}%
\pgfusepath{clip}%
\pgfsetrectcap%
\pgfsetroundjoin%
\pgfsetlinewidth{0.501875pt}%
\definecolor{currentstroke}{rgb}{0.000000,0.000000,0.000000}%
\pgfsetstrokecolor{currentstroke}%
\pgfsetstrokeopacity{0.100000}%
\pgfsetdash{}{0pt}%
\pgfpathmoveto{\pgfqpoint{3.275000in}{1.247073in}}%
\pgfpathlineto{\pgfqpoint{3.275000in}{7.921260in}}%
\pgfusepath{stroke}%
\end{pgfscope}%
\begin{pgfscope}%
\pgfsetbuttcap%
\pgfsetroundjoin%
\definecolor{currentfill}{rgb}{0.000000,0.000000,0.000000}%
\pgfsetfillcolor{currentfill}%
\pgfsetlinewidth{0.501875pt}%
\definecolor{currentstroke}{rgb}{0.000000,0.000000,0.000000}%
\pgfsetstrokecolor{currentstroke}%
\pgfsetdash{}{0pt}%
\pgfsys@defobject{currentmarker}{\pgfqpoint{0.000000in}{0.000000in}}{\pgfqpoint{0.000000in}{0.034722in}}{%
\pgfpathmoveto{\pgfqpoint{0.000000in}{0.000000in}}%
\pgfpathlineto{\pgfqpoint{0.000000in}{0.034722in}}%
\pgfusepath{stroke,fill}%
}%
\begin{pgfscope}%
\pgfsys@transformshift{3.275000in}{1.247073in}%
\pgfsys@useobject{currentmarker}{}%
\end{pgfscope}%
\end{pgfscope}%
\begin{pgfscope}%
\definecolor{textcolor}{rgb}{0.000000,0.000000,0.000000}%
\pgfsetstrokecolor{textcolor}%
\pgfsetfillcolor{textcolor}%
\pgftext[x=3.275000in,y=1.198462in,,top]{\color{textcolor}\sffamily\fontsize{18.000000}{21.600000}\selectfont $\displaystyle 1$}%
\end{pgfscope}%
\begin{pgfscope}%
\pgfpathrectangle{\pgfqpoint{1.396958in}{1.247073in}}{\pgfqpoint{10.524301in}{6.674186in}}%
\pgfusepath{clip}%
\pgfsetrectcap%
\pgfsetroundjoin%
\pgfsetlinewidth{0.501875pt}%
\definecolor{currentstroke}{rgb}{0.000000,0.000000,0.000000}%
\pgfsetstrokecolor{currentstroke}%
\pgfsetstrokeopacity{0.100000}%
\pgfsetdash{}{0pt}%
\pgfpathmoveto{\pgfqpoint{4.855183in}{1.247073in}}%
\pgfpathlineto{\pgfqpoint{4.855183in}{7.921260in}}%
\pgfusepath{stroke}%
\end{pgfscope}%
\begin{pgfscope}%
\pgfsetbuttcap%
\pgfsetroundjoin%
\definecolor{currentfill}{rgb}{0.000000,0.000000,0.000000}%
\pgfsetfillcolor{currentfill}%
\pgfsetlinewidth{0.501875pt}%
\definecolor{currentstroke}{rgb}{0.000000,0.000000,0.000000}%
\pgfsetstrokecolor{currentstroke}%
\pgfsetdash{}{0pt}%
\pgfsys@defobject{currentmarker}{\pgfqpoint{0.000000in}{0.000000in}}{\pgfqpoint{0.000000in}{0.034722in}}{%
\pgfpathmoveto{\pgfqpoint{0.000000in}{0.000000in}}%
\pgfpathlineto{\pgfqpoint{0.000000in}{0.034722in}}%
\pgfusepath{stroke,fill}%
}%
\begin{pgfscope}%
\pgfsys@transformshift{4.855183in}{1.247073in}%
\pgfsys@useobject{currentmarker}{}%
\end{pgfscope}%
\end{pgfscope}%
\begin{pgfscope}%
\definecolor{textcolor}{rgb}{0.000000,0.000000,0.000000}%
\pgfsetstrokecolor{textcolor}%
\pgfsetfillcolor{textcolor}%
\pgftext[x=4.855183in,y=1.198462in,,top]{\color{textcolor}\sffamily\fontsize{18.000000}{21.600000}\selectfont $\displaystyle 2$}%
\end{pgfscope}%
\begin{pgfscope}%
\pgfpathrectangle{\pgfqpoint{1.396958in}{1.247073in}}{\pgfqpoint{10.524301in}{6.674186in}}%
\pgfusepath{clip}%
\pgfsetrectcap%
\pgfsetroundjoin%
\pgfsetlinewidth{0.501875pt}%
\definecolor{currentstroke}{rgb}{0.000000,0.000000,0.000000}%
\pgfsetstrokecolor{currentstroke}%
\pgfsetstrokeopacity{0.100000}%
\pgfsetdash{}{0pt}%
\pgfpathmoveto{\pgfqpoint{6.435367in}{1.247073in}}%
\pgfpathlineto{\pgfqpoint{6.435367in}{7.921260in}}%
\pgfusepath{stroke}%
\end{pgfscope}%
\begin{pgfscope}%
\pgfsetbuttcap%
\pgfsetroundjoin%
\definecolor{currentfill}{rgb}{0.000000,0.000000,0.000000}%
\pgfsetfillcolor{currentfill}%
\pgfsetlinewidth{0.501875pt}%
\definecolor{currentstroke}{rgb}{0.000000,0.000000,0.000000}%
\pgfsetstrokecolor{currentstroke}%
\pgfsetdash{}{0pt}%
\pgfsys@defobject{currentmarker}{\pgfqpoint{0.000000in}{0.000000in}}{\pgfqpoint{0.000000in}{0.034722in}}{%
\pgfpathmoveto{\pgfqpoint{0.000000in}{0.000000in}}%
\pgfpathlineto{\pgfqpoint{0.000000in}{0.034722in}}%
\pgfusepath{stroke,fill}%
}%
\begin{pgfscope}%
\pgfsys@transformshift{6.435367in}{1.247073in}%
\pgfsys@useobject{currentmarker}{}%
\end{pgfscope}%
\end{pgfscope}%
\begin{pgfscope}%
\definecolor{textcolor}{rgb}{0.000000,0.000000,0.000000}%
\pgfsetstrokecolor{textcolor}%
\pgfsetfillcolor{textcolor}%
\pgftext[x=6.435367in,y=1.198462in,,top]{\color{textcolor}\sffamily\fontsize{18.000000}{21.600000}\selectfont $\displaystyle 3$}%
\end{pgfscope}%
\begin{pgfscope}%
\pgfpathrectangle{\pgfqpoint{1.396958in}{1.247073in}}{\pgfqpoint{10.524301in}{6.674186in}}%
\pgfusepath{clip}%
\pgfsetrectcap%
\pgfsetroundjoin%
\pgfsetlinewidth{0.501875pt}%
\definecolor{currentstroke}{rgb}{0.000000,0.000000,0.000000}%
\pgfsetstrokecolor{currentstroke}%
\pgfsetstrokeopacity{0.100000}%
\pgfsetdash{}{0pt}%
\pgfpathmoveto{\pgfqpoint{8.015550in}{1.247073in}}%
\pgfpathlineto{\pgfqpoint{8.015550in}{7.921260in}}%
\pgfusepath{stroke}%
\end{pgfscope}%
\begin{pgfscope}%
\pgfsetbuttcap%
\pgfsetroundjoin%
\definecolor{currentfill}{rgb}{0.000000,0.000000,0.000000}%
\pgfsetfillcolor{currentfill}%
\pgfsetlinewidth{0.501875pt}%
\definecolor{currentstroke}{rgb}{0.000000,0.000000,0.000000}%
\pgfsetstrokecolor{currentstroke}%
\pgfsetdash{}{0pt}%
\pgfsys@defobject{currentmarker}{\pgfqpoint{0.000000in}{0.000000in}}{\pgfqpoint{0.000000in}{0.034722in}}{%
\pgfpathmoveto{\pgfqpoint{0.000000in}{0.000000in}}%
\pgfpathlineto{\pgfqpoint{0.000000in}{0.034722in}}%
\pgfusepath{stroke,fill}%
}%
\begin{pgfscope}%
\pgfsys@transformshift{8.015550in}{1.247073in}%
\pgfsys@useobject{currentmarker}{}%
\end{pgfscope}%
\end{pgfscope}%
\begin{pgfscope}%
\definecolor{textcolor}{rgb}{0.000000,0.000000,0.000000}%
\pgfsetstrokecolor{textcolor}%
\pgfsetfillcolor{textcolor}%
\pgftext[x=8.015550in,y=1.198462in,,top]{\color{textcolor}\sffamily\fontsize{18.000000}{21.600000}\selectfont $\displaystyle 4$}%
\end{pgfscope}%
\begin{pgfscope}%
\pgfpathrectangle{\pgfqpoint{1.396958in}{1.247073in}}{\pgfqpoint{10.524301in}{6.674186in}}%
\pgfusepath{clip}%
\pgfsetrectcap%
\pgfsetroundjoin%
\pgfsetlinewidth{0.501875pt}%
\definecolor{currentstroke}{rgb}{0.000000,0.000000,0.000000}%
\pgfsetstrokecolor{currentstroke}%
\pgfsetstrokeopacity{0.100000}%
\pgfsetdash{}{0pt}%
\pgfpathmoveto{\pgfqpoint{9.595734in}{1.247073in}}%
\pgfpathlineto{\pgfqpoint{9.595734in}{7.921260in}}%
\pgfusepath{stroke}%
\end{pgfscope}%
\begin{pgfscope}%
\pgfsetbuttcap%
\pgfsetroundjoin%
\definecolor{currentfill}{rgb}{0.000000,0.000000,0.000000}%
\pgfsetfillcolor{currentfill}%
\pgfsetlinewidth{0.501875pt}%
\definecolor{currentstroke}{rgb}{0.000000,0.000000,0.000000}%
\pgfsetstrokecolor{currentstroke}%
\pgfsetdash{}{0pt}%
\pgfsys@defobject{currentmarker}{\pgfqpoint{0.000000in}{0.000000in}}{\pgfqpoint{0.000000in}{0.034722in}}{%
\pgfpathmoveto{\pgfqpoint{0.000000in}{0.000000in}}%
\pgfpathlineto{\pgfqpoint{0.000000in}{0.034722in}}%
\pgfusepath{stroke,fill}%
}%
\begin{pgfscope}%
\pgfsys@transformshift{9.595734in}{1.247073in}%
\pgfsys@useobject{currentmarker}{}%
\end{pgfscope}%
\end{pgfscope}%
\begin{pgfscope}%
\definecolor{textcolor}{rgb}{0.000000,0.000000,0.000000}%
\pgfsetstrokecolor{textcolor}%
\pgfsetfillcolor{textcolor}%
\pgftext[x=9.595734in,y=1.198462in,,top]{\color{textcolor}\sffamily\fontsize{18.000000}{21.600000}\selectfont $\displaystyle 5$}%
\end{pgfscope}%
\begin{pgfscope}%
\pgfpathrectangle{\pgfqpoint{1.396958in}{1.247073in}}{\pgfqpoint{10.524301in}{6.674186in}}%
\pgfusepath{clip}%
\pgfsetrectcap%
\pgfsetroundjoin%
\pgfsetlinewidth{0.501875pt}%
\definecolor{currentstroke}{rgb}{0.000000,0.000000,0.000000}%
\pgfsetstrokecolor{currentstroke}%
\pgfsetstrokeopacity{0.100000}%
\pgfsetdash{}{0pt}%
\pgfpathmoveto{\pgfqpoint{11.175917in}{1.247073in}}%
\pgfpathlineto{\pgfqpoint{11.175917in}{7.921260in}}%
\pgfusepath{stroke}%
\end{pgfscope}%
\begin{pgfscope}%
\pgfsetbuttcap%
\pgfsetroundjoin%
\definecolor{currentfill}{rgb}{0.000000,0.000000,0.000000}%
\pgfsetfillcolor{currentfill}%
\pgfsetlinewidth{0.501875pt}%
\definecolor{currentstroke}{rgb}{0.000000,0.000000,0.000000}%
\pgfsetstrokecolor{currentstroke}%
\pgfsetdash{}{0pt}%
\pgfsys@defobject{currentmarker}{\pgfqpoint{0.000000in}{0.000000in}}{\pgfqpoint{0.000000in}{0.034722in}}{%
\pgfpathmoveto{\pgfqpoint{0.000000in}{0.000000in}}%
\pgfpathlineto{\pgfqpoint{0.000000in}{0.034722in}}%
\pgfusepath{stroke,fill}%
}%
\begin{pgfscope}%
\pgfsys@transformshift{11.175917in}{1.247073in}%
\pgfsys@useobject{currentmarker}{}%
\end{pgfscope}%
\end{pgfscope}%
\begin{pgfscope}%
\definecolor{textcolor}{rgb}{0.000000,0.000000,0.000000}%
\pgfsetstrokecolor{textcolor}%
\pgfsetfillcolor{textcolor}%
\pgftext[x=11.175917in,y=1.198462in,,top]{\color{textcolor}\sffamily\fontsize{18.000000}{21.600000}\selectfont $\displaystyle 6$}%
\end{pgfscope}%
\begin{pgfscope}%
\definecolor{textcolor}{rgb}{0.000000,0.000000,0.000000}%
\pgfsetstrokecolor{textcolor}%
\pgfsetfillcolor{textcolor}%
\pgftext[x=6.659109in,y=0.900964in,,top]{\color{textcolor}\sffamily\fontsize{18.000000}{21.600000}\selectfont $\displaystyle x$}%
\end{pgfscope}%
\begin{pgfscope}%
\pgfpathrectangle{\pgfqpoint{1.396958in}{1.247073in}}{\pgfqpoint{10.524301in}{6.674186in}}%
\pgfusepath{clip}%
\pgfsetrectcap%
\pgfsetroundjoin%
\pgfsetlinewidth{0.501875pt}%
\definecolor{currentstroke}{rgb}{0.000000,0.000000,0.000000}%
\pgfsetstrokecolor{currentstroke}%
\pgfsetstrokeopacity{0.100000}%
\pgfsetdash{}{0pt}%
\pgfpathmoveto{\pgfqpoint{1.396958in}{1.435966in}}%
\pgfpathlineto{\pgfqpoint{11.921260in}{1.435966in}}%
\pgfusepath{stroke}%
\end{pgfscope}%
\begin{pgfscope}%
\pgfsetbuttcap%
\pgfsetroundjoin%
\definecolor{currentfill}{rgb}{0.000000,0.000000,0.000000}%
\pgfsetfillcolor{currentfill}%
\pgfsetlinewidth{0.501875pt}%
\definecolor{currentstroke}{rgb}{0.000000,0.000000,0.000000}%
\pgfsetstrokecolor{currentstroke}%
\pgfsetdash{}{0pt}%
\pgfsys@defobject{currentmarker}{\pgfqpoint{0.000000in}{0.000000in}}{\pgfqpoint{0.034722in}{0.000000in}}{%
\pgfpathmoveto{\pgfqpoint{0.000000in}{0.000000in}}%
\pgfpathlineto{\pgfqpoint{0.034722in}{0.000000in}}%
\pgfusepath{stroke,fill}%
}%
\begin{pgfscope}%
\pgfsys@transformshift{1.396958in}{1.435966in}%
\pgfsys@useobject{currentmarker}{}%
\end{pgfscope}%
\end{pgfscope}%
\begin{pgfscope}%
\definecolor{textcolor}{rgb}{0.000000,0.000000,0.000000}%
\pgfsetstrokecolor{textcolor}%
\pgfsetfillcolor{textcolor}%
\pgftext[x=0.876267in, y=1.340995in, left, base]{\color{textcolor}\sffamily\fontsize{18.000000}{21.600000}\selectfont $\displaystyle -1.0$}%
\end{pgfscope}%
\begin{pgfscope}%
\pgfpathrectangle{\pgfqpoint{1.396958in}{1.247073in}}{\pgfqpoint{10.524301in}{6.674186in}}%
\pgfusepath{clip}%
\pgfsetrectcap%
\pgfsetroundjoin%
\pgfsetlinewidth{0.501875pt}%
\definecolor{currentstroke}{rgb}{0.000000,0.000000,0.000000}%
\pgfsetstrokecolor{currentstroke}%
\pgfsetstrokeopacity{0.100000}%
\pgfsetdash{}{0pt}%
\pgfpathmoveto{\pgfqpoint{1.396958in}{3.010066in}}%
\pgfpathlineto{\pgfqpoint{11.921260in}{3.010066in}}%
\pgfusepath{stroke}%
\end{pgfscope}%
\begin{pgfscope}%
\pgfsetbuttcap%
\pgfsetroundjoin%
\definecolor{currentfill}{rgb}{0.000000,0.000000,0.000000}%
\pgfsetfillcolor{currentfill}%
\pgfsetlinewidth{0.501875pt}%
\definecolor{currentstroke}{rgb}{0.000000,0.000000,0.000000}%
\pgfsetstrokecolor{currentstroke}%
\pgfsetdash{}{0pt}%
\pgfsys@defobject{currentmarker}{\pgfqpoint{0.000000in}{0.000000in}}{\pgfqpoint{0.034722in}{0.000000in}}{%
\pgfpathmoveto{\pgfqpoint{0.000000in}{0.000000in}}%
\pgfpathlineto{\pgfqpoint{0.034722in}{0.000000in}}%
\pgfusepath{stroke,fill}%
}%
\begin{pgfscope}%
\pgfsys@transformshift{1.396958in}{3.010066in}%
\pgfsys@useobject{currentmarker}{}%
\end{pgfscope}%
\end{pgfscope}%
\begin{pgfscope}%
\definecolor{textcolor}{rgb}{0.000000,0.000000,0.000000}%
\pgfsetstrokecolor{textcolor}%
\pgfsetfillcolor{textcolor}%
\pgftext[x=0.876267in, y=2.915095in, left, base]{\color{textcolor}\sffamily\fontsize{18.000000}{21.600000}\selectfont $\displaystyle -0.5$}%
\end{pgfscope}%
\begin{pgfscope}%
\pgfpathrectangle{\pgfqpoint{1.396958in}{1.247073in}}{\pgfqpoint{10.524301in}{6.674186in}}%
\pgfusepath{clip}%
\pgfsetrectcap%
\pgfsetroundjoin%
\pgfsetlinewidth{0.501875pt}%
\definecolor{currentstroke}{rgb}{0.000000,0.000000,0.000000}%
\pgfsetstrokecolor{currentstroke}%
\pgfsetstrokeopacity{0.100000}%
\pgfsetdash{}{0pt}%
\pgfpathmoveto{\pgfqpoint{1.396958in}{4.584167in}}%
\pgfpathlineto{\pgfqpoint{11.921260in}{4.584167in}}%
\pgfusepath{stroke}%
\end{pgfscope}%
\begin{pgfscope}%
\pgfsetbuttcap%
\pgfsetroundjoin%
\definecolor{currentfill}{rgb}{0.000000,0.000000,0.000000}%
\pgfsetfillcolor{currentfill}%
\pgfsetlinewidth{0.501875pt}%
\definecolor{currentstroke}{rgb}{0.000000,0.000000,0.000000}%
\pgfsetstrokecolor{currentstroke}%
\pgfsetdash{}{0pt}%
\pgfsys@defobject{currentmarker}{\pgfqpoint{0.000000in}{0.000000in}}{\pgfqpoint{0.034722in}{0.000000in}}{%
\pgfpathmoveto{\pgfqpoint{0.000000in}{0.000000in}}%
\pgfpathlineto{\pgfqpoint{0.034722in}{0.000000in}}%
\pgfusepath{stroke,fill}%
}%
\begin{pgfscope}%
\pgfsys@transformshift{1.396958in}{4.584167in}%
\pgfsys@useobject{currentmarker}{}%
\end{pgfscope}%
\end{pgfscope}%
\begin{pgfscope}%
\definecolor{textcolor}{rgb}{0.000000,0.000000,0.000000}%
\pgfsetstrokecolor{textcolor}%
\pgfsetfillcolor{textcolor}%
\pgftext[x=1.062934in, y=4.489196in, left, base]{\color{textcolor}\sffamily\fontsize{18.000000}{21.600000}\selectfont $\displaystyle 0.0$}%
\end{pgfscope}%
\begin{pgfscope}%
\pgfpathrectangle{\pgfqpoint{1.396958in}{1.247073in}}{\pgfqpoint{10.524301in}{6.674186in}}%
\pgfusepath{clip}%
\pgfsetrectcap%
\pgfsetroundjoin%
\pgfsetlinewidth{0.501875pt}%
\definecolor{currentstroke}{rgb}{0.000000,0.000000,0.000000}%
\pgfsetstrokecolor{currentstroke}%
\pgfsetstrokeopacity{0.100000}%
\pgfsetdash{}{0pt}%
\pgfpathmoveto{\pgfqpoint{1.396958in}{6.158267in}}%
\pgfpathlineto{\pgfqpoint{11.921260in}{6.158267in}}%
\pgfusepath{stroke}%
\end{pgfscope}%
\begin{pgfscope}%
\pgfsetbuttcap%
\pgfsetroundjoin%
\definecolor{currentfill}{rgb}{0.000000,0.000000,0.000000}%
\pgfsetfillcolor{currentfill}%
\pgfsetlinewidth{0.501875pt}%
\definecolor{currentstroke}{rgb}{0.000000,0.000000,0.000000}%
\pgfsetstrokecolor{currentstroke}%
\pgfsetdash{}{0pt}%
\pgfsys@defobject{currentmarker}{\pgfqpoint{0.000000in}{0.000000in}}{\pgfqpoint{0.034722in}{0.000000in}}{%
\pgfpathmoveto{\pgfqpoint{0.000000in}{0.000000in}}%
\pgfpathlineto{\pgfqpoint{0.034722in}{0.000000in}}%
\pgfusepath{stroke,fill}%
}%
\begin{pgfscope}%
\pgfsys@transformshift{1.396958in}{6.158267in}%
\pgfsys@useobject{currentmarker}{}%
\end{pgfscope}%
\end{pgfscope}%
\begin{pgfscope}%
\definecolor{textcolor}{rgb}{0.000000,0.000000,0.000000}%
\pgfsetstrokecolor{textcolor}%
\pgfsetfillcolor{textcolor}%
\pgftext[x=1.062934in, y=6.063297in, left, base]{\color{textcolor}\sffamily\fontsize{18.000000}{21.600000}\selectfont $\displaystyle 0.5$}%
\end{pgfscope}%
\begin{pgfscope}%
\pgfpathrectangle{\pgfqpoint{1.396958in}{1.247073in}}{\pgfqpoint{10.524301in}{6.674186in}}%
\pgfusepath{clip}%
\pgfsetrectcap%
\pgfsetroundjoin%
\pgfsetlinewidth{0.501875pt}%
\definecolor{currentstroke}{rgb}{0.000000,0.000000,0.000000}%
\pgfsetstrokecolor{currentstroke}%
\pgfsetstrokeopacity{0.100000}%
\pgfsetdash{}{0pt}%
\pgfpathmoveto{\pgfqpoint{1.396958in}{7.732368in}}%
\pgfpathlineto{\pgfqpoint{11.921260in}{7.732368in}}%
\pgfusepath{stroke}%
\end{pgfscope}%
\begin{pgfscope}%
\pgfsetbuttcap%
\pgfsetroundjoin%
\definecolor{currentfill}{rgb}{0.000000,0.000000,0.000000}%
\pgfsetfillcolor{currentfill}%
\pgfsetlinewidth{0.501875pt}%
\definecolor{currentstroke}{rgb}{0.000000,0.000000,0.000000}%
\pgfsetstrokecolor{currentstroke}%
\pgfsetdash{}{0pt}%
\pgfsys@defobject{currentmarker}{\pgfqpoint{0.000000in}{0.000000in}}{\pgfqpoint{0.034722in}{0.000000in}}{%
\pgfpathmoveto{\pgfqpoint{0.000000in}{0.000000in}}%
\pgfpathlineto{\pgfqpoint{0.034722in}{0.000000in}}%
\pgfusepath{stroke,fill}%
}%
\begin{pgfscope}%
\pgfsys@transformshift{1.396958in}{7.732368in}%
\pgfsys@useobject{currentmarker}{}%
\end{pgfscope}%
\end{pgfscope}%
\begin{pgfscope}%
\definecolor{textcolor}{rgb}{0.000000,0.000000,0.000000}%
\pgfsetstrokecolor{textcolor}%
\pgfsetfillcolor{textcolor}%
\pgftext[x=1.062934in, y=7.637397in, left, base]{\color{textcolor}\sffamily\fontsize{18.000000}{21.600000}\selectfont $\displaystyle 1.0$}%
\end{pgfscope}%
\begin{pgfscope}%
\pgfpathrectangle{\pgfqpoint{1.396958in}{1.247073in}}{\pgfqpoint{10.524301in}{6.674186in}}%
\pgfusepath{clip}%
\pgfsetbuttcap%
\pgfsetroundjoin%
\pgfsetlinewidth{1.003750pt}%
\definecolor{currentstroke}{rgb}{0.000000,0.605603,0.978680}%
\pgfsetstrokecolor{currentstroke}%
\pgfsetdash{}{0pt}%
\pgfpathmoveto{\pgfqpoint{1.694816in}{4.584157in}}%
\pgfpathlineto{\pgfqpoint{1.956605in}{4.066226in}}%
\pgfpathlineto{\pgfqpoint{2.097195in}{3.793039in}}%
\pgfpathlineto{\pgfqpoint{2.213546in}{3.571594in}}%
\pgfpathlineto{\pgfqpoint{2.320201in}{3.373394in}}%
\pgfpathlineto{\pgfqpoint{2.417159in}{3.197974in}}%
\pgfpathlineto{\pgfqpoint{2.504422in}{3.044538in}}%
\pgfpathlineto{\pgfqpoint{2.586837in}{2.903928in}}%
\pgfpathlineto{\pgfqpoint{2.664405in}{2.775757in}}%
\pgfpathlineto{\pgfqpoint{2.741972in}{2.651943in}}%
\pgfpathlineto{\pgfqpoint{2.814691in}{2.540090in}}%
\pgfpathlineto{\pgfqpoint{2.882562in}{2.439593in}}%
\pgfpathlineto{\pgfqpoint{2.950433in}{2.343053in}}%
\pgfpathlineto{\pgfqpoint{3.013456in}{2.257106in}}%
\pgfpathlineto{\pgfqpoint{3.076480in}{2.174860in}}%
\pgfpathlineto{\pgfqpoint{3.139503in}{2.096446in}}%
\pgfpathlineto{\pgfqpoint{3.197678in}{2.027573in}}%
\pgfpathlineto{\pgfqpoint{3.255854in}{1.962165in}}%
\pgfpathlineto{\pgfqpoint{3.314029in}{1.900310in}}%
\pgfpathlineto{\pgfqpoint{3.367356in}{1.846803in}}%
\pgfpathlineto{\pgfqpoint{3.420684in}{1.796413in}}%
\pgfpathlineto{\pgfqpoint{3.474011in}{1.749198in}}%
\pgfpathlineto{\pgfqpoint{3.522490in}{1.709076in}}%
\pgfpathlineto{\pgfqpoint{3.570970in}{1.671659in}}%
\pgfpathlineto{\pgfqpoint{3.619449in}{1.636983in}}%
\pgfpathlineto{\pgfqpoint{3.667929in}{1.605082in}}%
\pgfpathlineto{\pgfqpoint{3.716408in}{1.575984in}}%
\pgfpathlineto{\pgfqpoint{3.760040in}{1.552216in}}%
\pgfpathlineto{\pgfqpoint{3.803671in}{1.530759in}}%
\pgfpathlineto{\pgfqpoint{3.847303in}{1.511630in}}%
\pgfpathlineto{\pgfqpoint{3.890934in}{1.494843in}}%
\pgfpathlineto{\pgfqpoint{3.934565in}{1.480412in}}%
\pgfpathlineto{\pgfqpoint{3.978197in}{1.468347in}}%
\pgfpathlineto{\pgfqpoint{4.021828in}{1.458657in}}%
\pgfpathlineto{\pgfqpoint{4.065460in}{1.451350in}}%
\pgfpathlineto{\pgfqpoint{4.109091in}{1.446431in}}%
\pgfpathlineto{\pgfqpoint{4.152723in}{1.443904in}}%
\pgfpathlineto{\pgfqpoint{4.196354in}{1.443771in}}%
\pgfpathlineto{\pgfqpoint{4.239986in}{1.446033in}}%
\pgfpathlineto{\pgfqpoint{4.283617in}{1.450686in}}%
\pgfpathlineto{\pgfqpoint{4.327249in}{1.457729in}}%
\pgfpathlineto{\pgfqpoint{4.370880in}{1.467155in}}%
\pgfpathlineto{\pgfqpoint{4.414512in}{1.478957in}}%
\pgfpathlineto{\pgfqpoint{4.458143in}{1.493127in}}%
\pgfpathlineto{\pgfqpoint{4.501775in}{1.509653in}}%
\pgfpathlineto{\pgfqpoint{4.545406in}{1.528523in}}%
\pgfpathlineto{\pgfqpoint{4.589038in}{1.549722in}}%
\pgfpathlineto{\pgfqpoint{4.632669in}{1.573235in}}%
\pgfpathlineto{\pgfqpoint{4.681149in}{1.602051in}}%
\pgfpathlineto{\pgfqpoint{4.729628in}{1.633674in}}%
\pgfpathlineto{\pgfqpoint{4.778107in}{1.668074in}}%
\pgfpathlineto{\pgfqpoint{4.826587in}{1.705219in}}%
\pgfpathlineto{\pgfqpoint{4.875066in}{1.745073in}}%
\pgfpathlineto{\pgfqpoint{4.923546in}{1.787599in}}%
\pgfpathlineto{\pgfqpoint{4.976873in}{1.837417in}}%
\pgfpathlineto{\pgfqpoint{5.030200in}{1.890362in}}%
\pgfpathlineto{\pgfqpoint{5.083528in}{1.946375in}}%
\pgfpathlineto{\pgfqpoint{5.141703in}{2.010903in}}%
\pgfpathlineto{\pgfqpoint{5.199878in}{2.078919in}}%
\pgfpathlineto{\pgfqpoint{5.258054in}{2.150330in}}%
\pgfpathlineto{\pgfqpoint{5.321077in}{2.231411in}}%
\pgfpathlineto{\pgfqpoint{5.384100in}{2.316234in}}%
\pgfpathlineto{\pgfqpoint{5.447124in}{2.404665in}}%
\pgfpathlineto{\pgfqpoint{5.514995in}{2.503770in}}%
\pgfpathlineto{\pgfqpoint{5.582866in}{2.606712in}}%
\pgfpathlineto{\pgfqpoint{5.655585in}{2.721050in}}%
\pgfpathlineto{\pgfqpoint{5.733152in}{2.847353in}}%
\pgfpathlineto{\pgfqpoint{5.810719in}{2.977839in}}%
\pgfpathlineto{\pgfqpoint{5.893134in}{3.120714in}}%
\pgfpathlineto{\pgfqpoint{5.980397in}{3.276323in}}%
\pgfpathlineto{\pgfqpoint{6.077356in}{3.453878in}}%
\pgfpathlineto{\pgfqpoint{6.184011in}{3.654075in}}%
\pgfpathlineto{\pgfqpoint{6.300361in}{3.877270in}}%
\pgfpathlineto{\pgfqpoint{6.440952in}{4.151962in}}%
\pgfpathlineto{\pgfqpoint{6.644565in}{4.555270in}}%
\pgfpathlineto{\pgfqpoint{6.920898in}{5.102107in}}%
\pgfpathlineto{\pgfqpoint{7.061488in}{5.375294in}}%
\pgfpathlineto{\pgfqpoint{7.177839in}{5.596739in}}%
\pgfpathlineto{\pgfqpoint{7.284494in}{5.794939in}}%
\pgfpathlineto{\pgfqpoint{7.381453in}{5.970360in}}%
\pgfpathlineto{\pgfqpoint{7.468716in}{6.123796in}}%
\pgfpathlineto{\pgfqpoint{7.551131in}{6.264405in}}%
\pgfpathlineto{\pgfqpoint{7.628698in}{6.392576in}}%
\pgfpathlineto{\pgfqpoint{7.706265in}{6.516390in}}%
\pgfpathlineto{\pgfqpoint{7.778984in}{6.628244in}}%
\pgfpathlineto{\pgfqpoint{7.846855in}{6.728740in}}%
\pgfpathlineto{\pgfqpoint{7.914726in}{6.825281in}}%
\pgfpathlineto{\pgfqpoint{7.977750in}{6.911228in}}%
\pgfpathlineto{\pgfqpoint{8.040773in}{6.993474in}}%
\pgfpathlineto{\pgfqpoint{8.103796in}{7.071888in}}%
\pgfpathlineto{\pgfqpoint{8.161971in}{7.140761in}}%
\pgfpathlineto{\pgfqpoint{8.220147in}{7.206169in}}%
\pgfpathlineto{\pgfqpoint{8.278322in}{7.268024in}}%
\pgfpathlineto{\pgfqpoint{8.331649in}{7.321531in}}%
\pgfpathlineto{\pgfqpoint{8.384977in}{7.371920in}}%
\pgfpathlineto{\pgfqpoint{8.438304in}{7.419135in}}%
\pgfpathlineto{\pgfqpoint{8.486783in}{7.459258in}}%
\pgfpathlineto{\pgfqpoint{8.535263in}{7.496674in}}%
\pgfpathlineto{\pgfqpoint{8.583742in}{7.531350in}}%
\pgfpathlineto{\pgfqpoint{8.632222in}{7.563252in}}%
\pgfpathlineto{\pgfqpoint{8.680701in}{7.592349in}}%
\pgfpathlineto{\pgfqpoint{8.724333in}{7.616118in}}%
\pgfpathlineto{\pgfqpoint{8.767964in}{7.637574in}}%
\pgfpathlineto{\pgfqpoint{8.811596in}{7.656703in}}%
\pgfpathlineto{\pgfqpoint{8.855227in}{7.673490in}}%
\pgfpathlineto{\pgfqpoint{8.898859in}{7.687921in}}%
\pgfpathlineto{\pgfqpoint{8.942490in}{7.699987in}}%
\pgfpathlineto{\pgfqpoint{8.986122in}{7.709676in}}%
\pgfpathlineto{\pgfqpoint{9.029753in}{7.716984in}}%
\pgfpathlineto{\pgfqpoint{9.073385in}{7.721902in}}%
\pgfpathlineto{\pgfqpoint{9.117016in}{7.724429in}}%
\pgfpathlineto{\pgfqpoint{9.160647in}{7.724562in}}%
\pgfpathlineto{\pgfqpoint{9.204279in}{7.722301in}}%
\pgfpathlineto{\pgfqpoint{9.247910in}{7.717647in}}%
\pgfpathlineto{\pgfqpoint{9.291542in}{7.710604in}}%
\pgfpathlineto{\pgfqpoint{9.335173in}{7.701178in}}%
\pgfpathlineto{\pgfqpoint{9.378805in}{7.689376in}}%
\pgfpathlineto{\pgfqpoint{9.422436in}{7.675207in}}%
\pgfpathlineto{\pgfqpoint{9.466068in}{7.658681in}}%
\pgfpathlineto{\pgfqpoint{9.509699in}{7.639811in}}%
\pgfpathlineto{\pgfqpoint{9.553331in}{7.618611in}}%
\pgfpathlineto{\pgfqpoint{9.596962in}{7.595099in}}%
\pgfpathlineto{\pgfqpoint{9.645442in}{7.566282in}}%
\pgfpathlineto{\pgfqpoint{9.693921in}{7.534659in}}%
\pgfpathlineto{\pgfqpoint{9.742401in}{7.500259in}}%
\pgfpathlineto{\pgfqpoint{9.790880in}{7.463114in}}%
\pgfpathlineto{\pgfqpoint{9.839359in}{7.423260in}}%
\pgfpathlineto{\pgfqpoint{9.887839in}{7.380734in}}%
\pgfpathlineto{\pgfqpoint{9.941166in}{7.330917in}}%
\pgfpathlineto{\pgfqpoint{9.994494in}{7.277972in}}%
\pgfpathlineto{\pgfqpoint{10.047821in}{7.221959in}}%
\pgfpathlineto{\pgfqpoint{10.105996in}{7.157430in}}%
\pgfpathlineto{\pgfqpoint{10.164172in}{7.089414in}}%
\pgfpathlineto{\pgfqpoint{10.222347in}{7.018003in}}%
\pgfpathlineto{\pgfqpoint{10.285370in}{6.936922in}}%
\pgfpathlineto{\pgfqpoint{10.348393in}{6.852099in}}%
\pgfpathlineto{\pgfqpoint{10.411417in}{6.763669in}}%
\pgfpathlineto{\pgfqpoint{10.479288in}{6.664564in}}%
\pgfpathlineto{\pgfqpoint{10.547159in}{6.561621in}}%
\pgfpathlineto{\pgfqpoint{10.619878in}{6.447283in}}%
\pgfpathlineto{\pgfqpoint{10.697445in}{6.320981in}}%
\pgfpathlineto{\pgfqpoint{10.775012in}{6.190494in}}%
\pgfpathlineto{\pgfqpoint{10.857427in}{6.047619in}}%
\pgfpathlineto{\pgfqpoint{10.944690in}{5.892010in}}%
\pgfpathlineto{\pgfqpoint{11.041649in}{5.714456in}}%
\pgfpathlineto{\pgfqpoint{11.148304in}{5.514259in}}%
\pgfpathlineto{\pgfqpoint{11.264655in}{5.291063in}}%
\pgfpathlineto{\pgfqpoint{11.405245in}{5.016371in}}%
\pgfpathlineto{\pgfqpoint{11.608858in}{4.613063in}}%
\pgfpathlineto{\pgfqpoint{11.623402in}{4.584157in}}%
\pgfpathlineto{\pgfqpoint{11.623402in}{4.584157in}}%
\pgfusepath{stroke}%
\end{pgfscope}%
\begin{pgfscope}%
\pgfpathrectangle{\pgfqpoint{1.396958in}{1.247073in}}{\pgfqpoint{10.524301in}{6.674186in}}%
\pgfusepath{clip}%
\pgfsetbuttcap%
\pgfsetroundjoin%
\pgfsetlinewidth{1.003750pt}%
\definecolor{currentstroke}{rgb}{0.888874,0.435649,0.278123}%
\pgfsetstrokecolor{currentstroke}%
\pgfsetdash{}{0pt}%
\pgfpathmoveto{\pgfqpoint{1.694816in}{4.584167in}}%
\pgfpathlineto{\pgfqpoint{1.956605in}{4.064987in}}%
\pgfpathlineto{\pgfqpoint{2.097195in}{3.791142in}}%
\pgfpathlineto{\pgfqpoint{2.213546in}{3.569162in}}%
\pgfpathlineto{\pgfqpoint{2.320201in}{3.370485in}}%
\pgfpathlineto{\pgfqpoint{2.417159in}{3.194641in}}%
\pgfpathlineto{\pgfqpoint{2.504422in}{3.040835in}}%
\pgfpathlineto{\pgfqpoint{2.586837in}{2.899887in}}%
\pgfpathlineto{\pgfqpoint{2.664405in}{2.771407in}}%
\pgfpathlineto{\pgfqpoint{2.741972in}{2.647294in}}%
\pgfpathlineto{\pgfqpoint{2.814691in}{2.535170in}}%
\pgfpathlineto{\pgfqpoint{2.882562in}{2.434432in}}%
\pgfpathlineto{\pgfqpoint{2.950433in}{2.337658in}}%
\pgfpathlineto{\pgfqpoint{3.013456in}{2.251504in}}%
\pgfpathlineto{\pgfqpoint{3.076480in}{2.169059in}}%
\pgfpathlineto{\pgfqpoint{3.139503in}{2.090456in}}%
\pgfpathlineto{\pgfqpoint{3.197678in}{2.021417in}}%
\pgfpathlineto{\pgfqpoint{3.255854in}{1.955850in}}%
\pgfpathlineto{\pgfqpoint{3.314029in}{1.893846in}}%
\pgfpathlineto{\pgfqpoint{3.367356in}{1.840210in}}%
\pgfpathlineto{\pgfqpoint{3.420684in}{1.789699in}}%
\pgfpathlineto{\pgfqpoint{3.474011in}{1.742370in}}%
\pgfpathlineto{\pgfqpoint{3.522490in}{1.702150in}}%
\pgfpathlineto{\pgfqpoint{3.570970in}{1.664643in}}%
\pgfpathlineto{\pgfqpoint{3.619449in}{1.629884in}}%
\pgfpathlineto{\pgfqpoint{3.667929in}{1.597905in}}%
\pgfpathlineto{\pgfqpoint{3.716408in}{1.568736in}}%
\pgfpathlineto{\pgfqpoint{3.760040in}{1.544911in}}%
\pgfpathlineto{\pgfqpoint{3.803671in}{1.523402in}}%
\pgfpathlineto{\pgfqpoint{3.847303in}{1.504227in}}%
\pgfpathlineto{\pgfqpoint{3.890934in}{1.487400in}}%
\pgfpathlineto{\pgfqpoint{3.934565in}{1.472933in}}%
\pgfpathlineto{\pgfqpoint{3.978197in}{1.460839in}}%
\pgfpathlineto{\pgfqpoint{4.021828in}{1.451125in}}%
\pgfpathlineto{\pgfqpoint{4.065460in}{1.443800in}}%
\pgfpathlineto{\pgfqpoint{4.109091in}{1.438869in}}%
\pgfpathlineto{\pgfqpoint{4.152723in}{1.436336in}}%
\pgfpathlineto{\pgfqpoint{4.196354in}{1.436203in}}%
\pgfpathlineto{\pgfqpoint{4.239986in}{1.438469in}}%
\pgfpathlineto{\pgfqpoint{4.283617in}{1.443134in}}%
\pgfpathlineto{\pgfqpoint{4.327249in}{1.450193in}}%
\pgfpathlineto{\pgfqpoint{4.370880in}{1.459641in}}%
\pgfpathlineto{\pgfqpoint{4.414512in}{1.471472in}}%
\pgfpathlineto{\pgfqpoint{4.458143in}{1.485675in}}%
\pgfpathlineto{\pgfqpoint{4.501775in}{1.502241in}}%
\pgfpathlineto{\pgfqpoint{4.545406in}{1.521156in}}%
\pgfpathlineto{\pgfqpoint{4.589038in}{1.542406in}}%
\pgfpathlineto{\pgfqpoint{4.632669in}{1.565975in}}%
\pgfpathlineto{\pgfqpoint{4.681149in}{1.594861in}}%
\pgfpathlineto{\pgfqpoint{4.729628in}{1.626560in}}%
\pgfpathlineto{\pgfqpoint{4.778107in}{1.661043in}}%
\pgfpathlineto{\pgfqpoint{4.826587in}{1.698277in}}%
\pgfpathlineto{\pgfqpoint{4.875066in}{1.738227in}}%
\pgfpathlineto{\pgfqpoint{4.923546in}{1.780855in}}%
\pgfpathlineto{\pgfqpoint{4.976873in}{1.830792in}}%
\pgfpathlineto{\pgfqpoint{5.030200in}{1.883865in}}%
\pgfpathlineto{\pgfqpoint{5.083528in}{1.940012in}}%
\pgfpathlineto{\pgfqpoint{5.141703in}{2.004696in}}%
\pgfpathlineto{\pgfqpoint{5.199878in}{2.072875in}}%
\pgfpathlineto{\pgfqpoint{5.258054in}{2.144458in}}%
\pgfpathlineto{\pgfqpoint{5.321077in}{2.225735in}}%
\pgfpathlineto{\pgfqpoint{5.384100in}{2.310762in}}%
\pgfpathlineto{\pgfqpoint{5.447124in}{2.399405in}}%
\pgfpathlineto{\pgfqpoint{5.514995in}{2.498749in}}%
\pgfpathlineto{\pgfqpoint{5.582866in}{2.601939in}}%
\pgfpathlineto{\pgfqpoint{5.655585in}{2.716552in}}%
\pgfpathlineto{\pgfqpoint{5.733152in}{2.843159in}}%
\pgfpathlineto{\pgfqpoint{5.810719in}{2.973960in}}%
\pgfpathlineto{\pgfqpoint{5.893134in}{3.117179in}}%
\pgfpathlineto{\pgfqpoint{5.980397in}{3.273163in}}%
\pgfpathlineto{\pgfqpoint{6.077356in}{3.451145in}}%
\pgfpathlineto{\pgfqpoint{6.184011in}{3.651824in}}%
\pgfpathlineto{\pgfqpoint{6.300361in}{3.875557in}}%
\pgfpathlineto{\pgfqpoint{6.440952in}{4.150911in}}%
\pgfpathlineto{\pgfqpoint{6.644565in}{4.555191in}}%
\pgfpathlineto{\pgfqpoint{6.920898in}{5.103346in}}%
\pgfpathlineto{\pgfqpoint{7.061488in}{5.377192in}}%
\pgfpathlineto{\pgfqpoint{7.177839in}{5.599171in}}%
\pgfpathlineto{\pgfqpoint{7.284494in}{5.797849in}}%
\pgfpathlineto{\pgfqpoint{7.381453in}{5.973692in}}%
\pgfpathlineto{\pgfqpoint{7.468716in}{6.127498in}}%
\pgfpathlineto{\pgfqpoint{7.551131in}{6.268447in}}%
\pgfpathlineto{\pgfqpoint{7.628698in}{6.396927in}}%
\pgfpathlineto{\pgfqpoint{7.706265in}{6.521039in}}%
\pgfpathlineto{\pgfqpoint{7.778984in}{6.633163in}}%
\pgfpathlineto{\pgfqpoint{7.846855in}{6.733902in}}%
\pgfpathlineto{\pgfqpoint{7.914726in}{6.830675in}}%
\pgfpathlineto{\pgfqpoint{7.977750in}{6.916830in}}%
\pgfpathlineto{\pgfqpoint{8.040773in}{6.999274in}}%
\pgfpathlineto{\pgfqpoint{8.103796in}{7.077877in}}%
\pgfpathlineto{\pgfqpoint{8.161971in}{7.146917in}}%
\pgfpathlineto{\pgfqpoint{8.220147in}{7.212483in}}%
\pgfpathlineto{\pgfqpoint{8.278322in}{7.274487in}}%
\pgfpathlineto{\pgfqpoint{8.331649in}{7.328123in}}%
\pgfpathlineto{\pgfqpoint{8.384977in}{7.378635in}}%
\pgfpathlineto{\pgfqpoint{8.438304in}{7.425964in}}%
\pgfpathlineto{\pgfqpoint{8.486783in}{7.466183in}}%
\pgfpathlineto{\pgfqpoint{8.535263in}{7.503690in}}%
\pgfpathlineto{\pgfqpoint{8.583742in}{7.538450in}}%
\pgfpathlineto{\pgfqpoint{8.632222in}{7.570429in}}%
\pgfpathlineto{\pgfqpoint{8.680701in}{7.599597in}}%
\pgfpathlineto{\pgfqpoint{8.724333in}{7.623423in}}%
\pgfpathlineto{\pgfqpoint{8.767964in}{7.644931in}}%
\pgfpathlineto{\pgfqpoint{8.811596in}{7.664106in}}%
\pgfpathlineto{\pgfqpoint{8.855227in}{7.680934in}}%
\pgfpathlineto{\pgfqpoint{8.898859in}{7.695400in}}%
\pgfpathlineto{\pgfqpoint{8.942490in}{7.707495in}}%
\pgfpathlineto{\pgfqpoint{8.986122in}{7.717208in}}%
\pgfpathlineto{\pgfqpoint{9.029753in}{7.724533in}}%
\pgfpathlineto{\pgfqpoint{9.073385in}{7.729464in}}%
\pgfpathlineto{\pgfqpoint{9.117016in}{7.731997in}}%
\pgfpathlineto{\pgfqpoint{9.160647in}{7.732131in}}%
\pgfpathlineto{\pgfqpoint{9.204279in}{7.729864in}}%
\pgfpathlineto{\pgfqpoint{9.247910in}{7.725200in}}%
\pgfpathlineto{\pgfqpoint{9.291542in}{7.718140in}}%
\pgfpathlineto{\pgfqpoint{9.335173in}{7.708692in}}%
\pgfpathlineto{\pgfqpoint{9.378805in}{7.696861in}}%
\pgfpathlineto{\pgfqpoint{9.422436in}{7.682658in}}%
\pgfpathlineto{\pgfqpoint{9.466068in}{7.666092in}}%
\pgfpathlineto{\pgfqpoint{9.509699in}{7.647177in}}%
\pgfpathlineto{\pgfqpoint{9.553331in}{7.625927in}}%
\pgfpathlineto{\pgfqpoint{9.596962in}{7.602358in}}%
\pgfpathlineto{\pgfqpoint{9.645442in}{7.573472in}}%
\pgfpathlineto{\pgfqpoint{9.693921in}{7.541773in}}%
\pgfpathlineto{\pgfqpoint{9.742401in}{7.507291in}}%
\pgfpathlineto{\pgfqpoint{9.790880in}{7.470057in}}%
\pgfpathlineto{\pgfqpoint{9.839359in}{7.430107in}}%
\pgfpathlineto{\pgfqpoint{9.887839in}{7.387478in}}%
\pgfpathlineto{\pgfqpoint{9.941166in}{7.337541in}}%
\pgfpathlineto{\pgfqpoint{9.994494in}{7.284469in}}%
\pgfpathlineto{\pgfqpoint{10.047821in}{7.228321in}}%
\pgfpathlineto{\pgfqpoint{10.105996in}{7.163637in}}%
\pgfpathlineto{\pgfqpoint{10.164172in}{7.095458in}}%
\pgfpathlineto{\pgfqpoint{10.222347in}{7.023875in}}%
\pgfpathlineto{\pgfqpoint{10.285370in}{6.942599in}}%
\pgfpathlineto{\pgfqpoint{10.348393in}{6.857571in}}%
\pgfpathlineto{\pgfqpoint{10.411417in}{6.768928in}}%
\pgfpathlineto{\pgfqpoint{10.479288in}{6.669585in}}%
\pgfpathlineto{\pgfqpoint{10.547159in}{6.566394in}}%
\pgfpathlineto{\pgfqpoint{10.619878in}{6.451781in}}%
\pgfpathlineto{\pgfqpoint{10.697445in}{6.325174in}}%
\pgfpathlineto{\pgfqpoint{10.775012in}{6.194373in}}%
\pgfpathlineto{\pgfqpoint{10.857427in}{6.051154in}}%
\pgfpathlineto{\pgfqpoint{10.944690in}{5.895171in}}%
\pgfpathlineto{\pgfqpoint{11.041649in}{5.717189in}}%
\pgfpathlineto{\pgfqpoint{11.148304in}{5.516509in}}%
\pgfpathlineto{\pgfqpoint{11.264655in}{5.292776in}}%
\pgfpathlineto{\pgfqpoint{11.405245in}{5.017422in}}%
\pgfpathlineto{\pgfqpoint{11.608858in}{4.613142in}}%
\pgfpathlineto{\pgfqpoint{11.623402in}{4.584167in}}%
\pgfpathlineto{\pgfqpoint{11.623402in}{4.584167in}}%
\pgfusepath{stroke}%
\end{pgfscope}%
\begin{pgfscope}%
\pgfsetrectcap%
\pgfsetmiterjoin%
\pgfsetlinewidth{1.003750pt}%
\definecolor{currentstroke}{rgb}{0.000000,0.000000,0.000000}%
\pgfsetstrokecolor{currentstroke}%
\pgfsetdash{}{0pt}%
\pgfpathmoveto{\pgfqpoint{1.396958in}{1.247073in}}%
\pgfpathlineto{\pgfqpoint{1.396958in}{7.921260in}}%
\pgfusepath{stroke}%
\end{pgfscope}%
\begin{pgfscope}%
\pgfsetrectcap%
\pgfsetmiterjoin%
\pgfsetlinewidth{1.003750pt}%
\definecolor{currentstroke}{rgb}{0.000000,0.000000,0.000000}%
\pgfsetstrokecolor{currentstroke}%
\pgfsetdash{}{0pt}%
\pgfpathmoveto{\pgfqpoint{1.396958in}{1.247073in}}%
\pgfpathlineto{\pgfqpoint{11.921260in}{1.247073in}}%
\pgfusepath{stroke}%
\end{pgfscope}%
\begin{pgfscope}%
\pgfsetbuttcap%
\pgfsetmiterjoin%
\definecolor{currentfill}{rgb}{1.000000,1.000000,1.000000}%
\pgfsetfillcolor{currentfill}%
\pgfsetlinewidth{1.003750pt}%
\definecolor{currentstroke}{rgb}{0.000000,0.000000,0.000000}%
\pgfsetstrokecolor{currentstroke}%
\pgfsetdash{}{0pt}%
\pgfpathmoveto{\pgfqpoint{10.511589in}{6.787373in}}%
\pgfpathlineto{\pgfqpoint{11.796260in}{6.787373in}}%
\pgfpathlineto{\pgfqpoint{11.796260in}{7.796260in}}%
\pgfpathlineto{\pgfqpoint{10.511589in}{7.796260in}}%
\pgfpathclose%
\pgfusepath{stroke,fill}%
\end{pgfscope}%
\begin{pgfscope}%
\pgfsetbuttcap%
\pgfsetmiterjoin%
\pgfsetlinewidth{2.258437pt}%
\definecolor{currentstroke}{rgb}{0.000000,0.605603,0.978680}%
\pgfsetstrokecolor{currentstroke}%
\pgfsetdash{}{0pt}%
\pgfpathmoveto{\pgfqpoint{10.711589in}{7.493818in}}%
\pgfpathlineto{\pgfqpoint{11.211589in}{7.493818in}}%
\pgfusepath{stroke}%
\end{pgfscope}%
\begin{pgfscope}%
\definecolor{textcolor}{rgb}{0.000000,0.000000,0.000000}%
\pgfsetstrokecolor{textcolor}%
\pgfsetfillcolor{textcolor}%
\pgftext[x=11.411589in,y=7.406318in,left,base]{\color{textcolor}\sffamily\fontsize{18.000000}{21.600000}\selectfont $\displaystyle U$}%
\end{pgfscope}%
\begin{pgfscope}%
\pgfsetbuttcap%
\pgfsetmiterjoin%
\pgfsetlinewidth{2.258437pt}%
\definecolor{currentstroke}{rgb}{0.888874,0.435649,0.278123}%
\pgfsetstrokecolor{currentstroke}%
\pgfsetdash{}{0pt}%
\pgfpathmoveto{\pgfqpoint{10.711589in}{7.126875in}}%
\pgfpathlineto{\pgfqpoint{11.211589in}{7.126875in}}%
\pgfusepath{stroke}%
\end{pgfscope}%
\begin{pgfscope}%
\definecolor{textcolor}{rgb}{0.000000,0.000000,0.000000}%
\pgfsetstrokecolor{textcolor}%
\pgfsetfillcolor{textcolor}%
\pgftext[x=11.411589in,y=7.039375in,left,base]{\color{textcolor}\sffamily\fontsize{18.000000}{21.600000}\selectfont $\displaystyle u$}%
\end{pgfscope}%
\end{pgfpicture}%
\makeatother%
\endgroup%
}
	\caption{迎风格式差分逼近解 $U$ 与真解 $u$}\label{fig:upwind_Uu}
\end{figure}

取 $\nu = 2 \geq 1$, 不满足 CFL 条件. $h = 2^{-7}$ 和 $h = 2^{-11}$ 时差分逼近解 $U$ 与真解 $u$ 在 $t = t_{\max }$ 时刻图像如图 \ref{fig:upwind_Uu_noCFL} 所示. 可以看到出现了震荡和错误解.

\begin{figure}[H]\centering\zihao{-5}
	\resizebox{0.4\linewidth}{!}{%% Creator: Matplotlib, PGF backend
%%
%% To include the figure in your LaTeX document, write
%%   \input{<filename>.pgf}
%%
%% Make sure the required packages are loaded in your preamble
%%   \usepackage{pgf}
%%
%% Figures using additional raster images can only be included by \input if
%% they are in the same directory as the main LaTeX file. For loading figures
%% from other directories you can use the `import` package
%%   \usepackage{import}
%%
%% and then include the figures with
%%   \import{<path to file>}{<filename>.pgf}
%%
%% Matplotlib used the following preamble
%%   \usepackage{fontspec}
%%   \setmainfont{DejaVuSerif.ttf}[Path=\detokenize{/Users/quejiahao/.julia/conda/3/lib/python3.9/site-packages/matplotlib/mpl-data/fonts/ttf/}]
%%   \setsansfont{DejaVuSans.ttf}[Path=\detokenize{/Users/quejiahao/.julia/conda/3/lib/python3.9/site-packages/matplotlib/mpl-data/fonts/ttf/}]
%%   \setmonofont{DejaVuSansMono.ttf}[Path=\detokenize{/Users/quejiahao/.julia/conda/3/lib/python3.9/site-packages/matplotlib/mpl-data/fonts/ttf/}]
%%
\begingroup%
\makeatletter%
\begin{pgfpicture}%
\pgfpathrectangle{\pgfpointorigin}{\pgfqpoint{12.000000in}{8.000000in}}%
\pgfusepath{use as bounding box, clip}%
\begin{pgfscope}%
\pgfsetbuttcap%
\pgfsetmiterjoin%
\definecolor{currentfill}{rgb}{1.000000,1.000000,1.000000}%
\pgfsetfillcolor{currentfill}%
\pgfsetlinewidth{0.000000pt}%
\definecolor{currentstroke}{rgb}{1.000000,1.000000,1.000000}%
\pgfsetstrokecolor{currentstroke}%
\pgfsetdash{}{0pt}%
\pgfpathmoveto{\pgfqpoint{0.000000in}{0.000000in}}%
\pgfpathlineto{\pgfqpoint{12.000000in}{0.000000in}}%
\pgfpathlineto{\pgfqpoint{12.000000in}{8.000000in}}%
\pgfpathlineto{\pgfqpoint{0.000000in}{8.000000in}}%
\pgfpathclose%
\pgfusepath{fill}%
\end{pgfscope}%
\begin{pgfscope}%
\pgfsetbuttcap%
\pgfsetmiterjoin%
\definecolor{currentfill}{rgb}{1.000000,1.000000,1.000000}%
\pgfsetfillcolor{currentfill}%
\pgfsetlinewidth{0.000000pt}%
\definecolor{currentstroke}{rgb}{0.000000,0.000000,0.000000}%
\pgfsetstrokecolor{currentstroke}%
\pgfsetstrokeopacity{0.000000}%
\pgfsetdash{}{0pt}%
\pgfpathmoveto{\pgfqpoint{1.396958in}{1.247073in}}%
\pgfpathlineto{\pgfqpoint{11.921260in}{1.247073in}}%
\pgfpathlineto{\pgfqpoint{11.921260in}{7.921260in}}%
\pgfpathlineto{\pgfqpoint{1.396958in}{7.921260in}}%
\pgfpathclose%
\pgfusepath{fill}%
\end{pgfscope}%
\begin{pgfscope}%
\pgfpathrectangle{\pgfqpoint{1.396958in}{1.247073in}}{\pgfqpoint{10.524301in}{6.674186in}}%
\pgfusepath{clip}%
\pgfsetrectcap%
\pgfsetroundjoin%
\pgfsetlinewidth{0.501875pt}%
\definecolor{currentstroke}{rgb}{0.000000,0.000000,0.000000}%
\pgfsetstrokecolor{currentstroke}%
\pgfsetstrokeopacity{0.100000}%
\pgfsetdash{}{0pt}%
\pgfpathmoveto{\pgfqpoint{1.694816in}{1.247073in}}%
\pgfpathlineto{\pgfqpoint{1.694816in}{7.921260in}}%
\pgfusepath{stroke}%
\end{pgfscope}%
\begin{pgfscope}%
\pgfsetbuttcap%
\pgfsetroundjoin%
\definecolor{currentfill}{rgb}{0.000000,0.000000,0.000000}%
\pgfsetfillcolor{currentfill}%
\pgfsetlinewidth{0.501875pt}%
\definecolor{currentstroke}{rgb}{0.000000,0.000000,0.000000}%
\pgfsetstrokecolor{currentstroke}%
\pgfsetdash{}{0pt}%
\pgfsys@defobject{currentmarker}{\pgfqpoint{0.000000in}{0.000000in}}{\pgfqpoint{0.000000in}{0.034722in}}{%
\pgfpathmoveto{\pgfqpoint{0.000000in}{0.000000in}}%
\pgfpathlineto{\pgfqpoint{0.000000in}{0.034722in}}%
\pgfusepath{stroke,fill}%
}%
\begin{pgfscope}%
\pgfsys@transformshift{1.694816in}{1.247073in}%
\pgfsys@useobject{currentmarker}{}%
\end{pgfscope}%
\end{pgfscope}%
\begin{pgfscope}%
\definecolor{textcolor}{rgb}{0.000000,0.000000,0.000000}%
\pgfsetstrokecolor{textcolor}%
\pgfsetfillcolor{textcolor}%
\pgftext[x=1.694816in,y=1.198462in,,top]{\color{textcolor}\sffamily\fontsize{18.000000}{21.600000}\selectfont $\displaystyle 0$}%
\end{pgfscope}%
\begin{pgfscope}%
\pgfpathrectangle{\pgfqpoint{1.396958in}{1.247073in}}{\pgfqpoint{10.524301in}{6.674186in}}%
\pgfusepath{clip}%
\pgfsetrectcap%
\pgfsetroundjoin%
\pgfsetlinewidth{0.501875pt}%
\definecolor{currentstroke}{rgb}{0.000000,0.000000,0.000000}%
\pgfsetstrokecolor{currentstroke}%
\pgfsetstrokeopacity{0.100000}%
\pgfsetdash{}{0pt}%
\pgfpathmoveto{\pgfqpoint{3.275000in}{1.247073in}}%
\pgfpathlineto{\pgfqpoint{3.275000in}{7.921260in}}%
\pgfusepath{stroke}%
\end{pgfscope}%
\begin{pgfscope}%
\pgfsetbuttcap%
\pgfsetroundjoin%
\definecolor{currentfill}{rgb}{0.000000,0.000000,0.000000}%
\pgfsetfillcolor{currentfill}%
\pgfsetlinewidth{0.501875pt}%
\definecolor{currentstroke}{rgb}{0.000000,0.000000,0.000000}%
\pgfsetstrokecolor{currentstroke}%
\pgfsetdash{}{0pt}%
\pgfsys@defobject{currentmarker}{\pgfqpoint{0.000000in}{0.000000in}}{\pgfqpoint{0.000000in}{0.034722in}}{%
\pgfpathmoveto{\pgfqpoint{0.000000in}{0.000000in}}%
\pgfpathlineto{\pgfqpoint{0.000000in}{0.034722in}}%
\pgfusepath{stroke,fill}%
}%
\begin{pgfscope}%
\pgfsys@transformshift{3.275000in}{1.247073in}%
\pgfsys@useobject{currentmarker}{}%
\end{pgfscope}%
\end{pgfscope}%
\begin{pgfscope}%
\definecolor{textcolor}{rgb}{0.000000,0.000000,0.000000}%
\pgfsetstrokecolor{textcolor}%
\pgfsetfillcolor{textcolor}%
\pgftext[x=3.275000in,y=1.198462in,,top]{\color{textcolor}\sffamily\fontsize{18.000000}{21.600000}\selectfont $\displaystyle 1$}%
\end{pgfscope}%
\begin{pgfscope}%
\pgfpathrectangle{\pgfqpoint{1.396958in}{1.247073in}}{\pgfqpoint{10.524301in}{6.674186in}}%
\pgfusepath{clip}%
\pgfsetrectcap%
\pgfsetroundjoin%
\pgfsetlinewidth{0.501875pt}%
\definecolor{currentstroke}{rgb}{0.000000,0.000000,0.000000}%
\pgfsetstrokecolor{currentstroke}%
\pgfsetstrokeopacity{0.100000}%
\pgfsetdash{}{0pt}%
\pgfpathmoveto{\pgfqpoint{4.855183in}{1.247073in}}%
\pgfpathlineto{\pgfqpoint{4.855183in}{7.921260in}}%
\pgfusepath{stroke}%
\end{pgfscope}%
\begin{pgfscope}%
\pgfsetbuttcap%
\pgfsetroundjoin%
\definecolor{currentfill}{rgb}{0.000000,0.000000,0.000000}%
\pgfsetfillcolor{currentfill}%
\pgfsetlinewidth{0.501875pt}%
\definecolor{currentstroke}{rgb}{0.000000,0.000000,0.000000}%
\pgfsetstrokecolor{currentstroke}%
\pgfsetdash{}{0pt}%
\pgfsys@defobject{currentmarker}{\pgfqpoint{0.000000in}{0.000000in}}{\pgfqpoint{0.000000in}{0.034722in}}{%
\pgfpathmoveto{\pgfqpoint{0.000000in}{0.000000in}}%
\pgfpathlineto{\pgfqpoint{0.000000in}{0.034722in}}%
\pgfusepath{stroke,fill}%
}%
\begin{pgfscope}%
\pgfsys@transformshift{4.855183in}{1.247073in}%
\pgfsys@useobject{currentmarker}{}%
\end{pgfscope}%
\end{pgfscope}%
\begin{pgfscope}%
\definecolor{textcolor}{rgb}{0.000000,0.000000,0.000000}%
\pgfsetstrokecolor{textcolor}%
\pgfsetfillcolor{textcolor}%
\pgftext[x=4.855183in,y=1.198462in,,top]{\color{textcolor}\sffamily\fontsize{18.000000}{21.600000}\selectfont $\displaystyle 2$}%
\end{pgfscope}%
\begin{pgfscope}%
\pgfpathrectangle{\pgfqpoint{1.396958in}{1.247073in}}{\pgfqpoint{10.524301in}{6.674186in}}%
\pgfusepath{clip}%
\pgfsetrectcap%
\pgfsetroundjoin%
\pgfsetlinewidth{0.501875pt}%
\definecolor{currentstroke}{rgb}{0.000000,0.000000,0.000000}%
\pgfsetstrokecolor{currentstroke}%
\pgfsetstrokeopacity{0.100000}%
\pgfsetdash{}{0pt}%
\pgfpathmoveto{\pgfqpoint{6.435367in}{1.247073in}}%
\pgfpathlineto{\pgfqpoint{6.435367in}{7.921260in}}%
\pgfusepath{stroke}%
\end{pgfscope}%
\begin{pgfscope}%
\pgfsetbuttcap%
\pgfsetroundjoin%
\definecolor{currentfill}{rgb}{0.000000,0.000000,0.000000}%
\pgfsetfillcolor{currentfill}%
\pgfsetlinewidth{0.501875pt}%
\definecolor{currentstroke}{rgb}{0.000000,0.000000,0.000000}%
\pgfsetstrokecolor{currentstroke}%
\pgfsetdash{}{0pt}%
\pgfsys@defobject{currentmarker}{\pgfqpoint{0.000000in}{0.000000in}}{\pgfqpoint{0.000000in}{0.034722in}}{%
\pgfpathmoveto{\pgfqpoint{0.000000in}{0.000000in}}%
\pgfpathlineto{\pgfqpoint{0.000000in}{0.034722in}}%
\pgfusepath{stroke,fill}%
}%
\begin{pgfscope}%
\pgfsys@transformshift{6.435367in}{1.247073in}%
\pgfsys@useobject{currentmarker}{}%
\end{pgfscope}%
\end{pgfscope}%
\begin{pgfscope}%
\definecolor{textcolor}{rgb}{0.000000,0.000000,0.000000}%
\pgfsetstrokecolor{textcolor}%
\pgfsetfillcolor{textcolor}%
\pgftext[x=6.435367in,y=1.198462in,,top]{\color{textcolor}\sffamily\fontsize{18.000000}{21.600000}\selectfont $\displaystyle 3$}%
\end{pgfscope}%
\begin{pgfscope}%
\pgfpathrectangle{\pgfqpoint{1.396958in}{1.247073in}}{\pgfqpoint{10.524301in}{6.674186in}}%
\pgfusepath{clip}%
\pgfsetrectcap%
\pgfsetroundjoin%
\pgfsetlinewidth{0.501875pt}%
\definecolor{currentstroke}{rgb}{0.000000,0.000000,0.000000}%
\pgfsetstrokecolor{currentstroke}%
\pgfsetstrokeopacity{0.100000}%
\pgfsetdash{}{0pt}%
\pgfpathmoveto{\pgfqpoint{8.015550in}{1.247073in}}%
\pgfpathlineto{\pgfqpoint{8.015550in}{7.921260in}}%
\pgfusepath{stroke}%
\end{pgfscope}%
\begin{pgfscope}%
\pgfsetbuttcap%
\pgfsetroundjoin%
\definecolor{currentfill}{rgb}{0.000000,0.000000,0.000000}%
\pgfsetfillcolor{currentfill}%
\pgfsetlinewidth{0.501875pt}%
\definecolor{currentstroke}{rgb}{0.000000,0.000000,0.000000}%
\pgfsetstrokecolor{currentstroke}%
\pgfsetdash{}{0pt}%
\pgfsys@defobject{currentmarker}{\pgfqpoint{0.000000in}{0.000000in}}{\pgfqpoint{0.000000in}{0.034722in}}{%
\pgfpathmoveto{\pgfqpoint{0.000000in}{0.000000in}}%
\pgfpathlineto{\pgfqpoint{0.000000in}{0.034722in}}%
\pgfusepath{stroke,fill}%
}%
\begin{pgfscope}%
\pgfsys@transformshift{8.015550in}{1.247073in}%
\pgfsys@useobject{currentmarker}{}%
\end{pgfscope}%
\end{pgfscope}%
\begin{pgfscope}%
\definecolor{textcolor}{rgb}{0.000000,0.000000,0.000000}%
\pgfsetstrokecolor{textcolor}%
\pgfsetfillcolor{textcolor}%
\pgftext[x=8.015550in,y=1.198462in,,top]{\color{textcolor}\sffamily\fontsize{18.000000}{21.600000}\selectfont $\displaystyle 4$}%
\end{pgfscope}%
\begin{pgfscope}%
\pgfpathrectangle{\pgfqpoint{1.396958in}{1.247073in}}{\pgfqpoint{10.524301in}{6.674186in}}%
\pgfusepath{clip}%
\pgfsetrectcap%
\pgfsetroundjoin%
\pgfsetlinewidth{0.501875pt}%
\definecolor{currentstroke}{rgb}{0.000000,0.000000,0.000000}%
\pgfsetstrokecolor{currentstroke}%
\pgfsetstrokeopacity{0.100000}%
\pgfsetdash{}{0pt}%
\pgfpathmoveto{\pgfqpoint{9.595734in}{1.247073in}}%
\pgfpathlineto{\pgfqpoint{9.595734in}{7.921260in}}%
\pgfusepath{stroke}%
\end{pgfscope}%
\begin{pgfscope}%
\pgfsetbuttcap%
\pgfsetroundjoin%
\definecolor{currentfill}{rgb}{0.000000,0.000000,0.000000}%
\pgfsetfillcolor{currentfill}%
\pgfsetlinewidth{0.501875pt}%
\definecolor{currentstroke}{rgb}{0.000000,0.000000,0.000000}%
\pgfsetstrokecolor{currentstroke}%
\pgfsetdash{}{0pt}%
\pgfsys@defobject{currentmarker}{\pgfqpoint{0.000000in}{0.000000in}}{\pgfqpoint{0.000000in}{0.034722in}}{%
\pgfpathmoveto{\pgfqpoint{0.000000in}{0.000000in}}%
\pgfpathlineto{\pgfqpoint{0.000000in}{0.034722in}}%
\pgfusepath{stroke,fill}%
}%
\begin{pgfscope}%
\pgfsys@transformshift{9.595734in}{1.247073in}%
\pgfsys@useobject{currentmarker}{}%
\end{pgfscope}%
\end{pgfscope}%
\begin{pgfscope}%
\definecolor{textcolor}{rgb}{0.000000,0.000000,0.000000}%
\pgfsetstrokecolor{textcolor}%
\pgfsetfillcolor{textcolor}%
\pgftext[x=9.595734in,y=1.198462in,,top]{\color{textcolor}\sffamily\fontsize{18.000000}{21.600000}\selectfont $\displaystyle 5$}%
\end{pgfscope}%
\begin{pgfscope}%
\pgfpathrectangle{\pgfqpoint{1.396958in}{1.247073in}}{\pgfqpoint{10.524301in}{6.674186in}}%
\pgfusepath{clip}%
\pgfsetrectcap%
\pgfsetroundjoin%
\pgfsetlinewidth{0.501875pt}%
\definecolor{currentstroke}{rgb}{0.000000,0.000000,0.000000}%
\pgfsetstrokecolor{currentstroke}%
\pgfsetstrokeopacity{0.100000}%
\pgfsetdash{}{0pt}%
\pgfpathmoveto{\pgfqpoint{11.175917in}{1.247073in}}%
\pgfpathlineto{\pgfqpoint{11.175917in}{7.921260in}}%
\pgfusepath{stroke}%
\end{pgfscope}%
\begin{pgfscope}%
\pgfsetbuttcap%
\pgfsetroundjoin%
\definecolor{currentfill}{rgb}{0.000000,0.000000,0.000000}%
\pgfsetfillcolor{currentfill}%
\pgfsetlinewidth{0.501875pt}%
\definecolor{currentstroke}{rgb}{0.000000,0.000000,0.000000}%
\pgfsetstrokecolor{currentstroke}%
\pgfsetdash{}{0pt}%
\pgfsys@defobject{currentmarker}{\pgfqpoint{0.000000in}{0.000000in}}{\pgfqpoint{0.000000in}{0.034722in}}{%
\pgfpathmoveto{\pgfqpoint{0.000000in}{0.000000in}}%
\pgfpathlineto{\pgfqpoint{0.000000in}{0.034722in}}%
\pgfusepath{stroke,fill}%
}%
\begin{pgfscope}%
\pgfsys@transformshift{11.175917in}{1.247073in}%
\pgfsys@useobject{currentmarker}{}%
\end{pgfscope}%
\end{pgfscope}%
\begin{pgfscope}%
\definecolor{textcolor}{rgb}{0.000000,0.000000,0.000000}%
\pgfsetstrokecolor{textcolor}%
\pgfsetfillcolor{textcolor}%
\pgftext[x=11.175917in,y=1.198462in,,top]{\color{textcolor}\sffamily\fontsize{18.000000}{21.600000}\selectfont $\displaystyle 6$}%
\end{pgfscope}%
\begin{pgfscope}%
\definecolor{textcolor}{rgb}{0.000000,0.000000,0.000000}%
\pgfsetstrokecolor{textcolor}%
\pgfsetfillcolor{textcolor}%
\pgftext[x=6.659109in,y=0.900964in,,top]{\color{textcolor}\sffamily\fontsize{18.000000}{21.600000}\selectfont $\displaystyle x$}%
\end{pgfscope}%
\begin{pgfscope}%
\pgfpathrectangle{\pgfqpoint{1.396958in}{1.247073in}}{\pgfqpoint{10.524301in}{6.674186in}}%
\pgfusepath{clip}%
\pgfsetrectcap%
\pgfsetroundjoin%
\pgfsetlinewidth{0.501875pt}%
\definecolor{currentstroke}{rgb}{0.000000,0.000000,0.000000}%
\pgfsetstrokecolor{currentstroke}%
\pgfsetstrokeopacity{0.100000}%
\pgfsetdash{}{0pt}%
\pgfpathmoveto{\pgfqpoint{1.396958in}{1.757355in}}%
\pgfpathlineto{\pgfqpoint{11.921260in}{1.757355in}}%
\pgfusepath{stroke}%
\end{pgfscope}%
\begin{pgfscope}%
\pgfsetbuttcap%
\pgfsetroundjoin%
\definecolor{currentfill}{rgb}{0.000000,0.000000,0.000000}%
\pgfsetfillcolor{currentfill}%
\pgfsetlinewidth{0.501875pt}%
\definecolor{currentstroke}{rgb}{0.000000,0.000000,0.000000}%
\pgfsetstrokecolor{currentstroke}%
\pgfsetdash{}{0pt}%
\pgfsys@defobject{currentmarker}{\pgfqpoint{0.000000in}{0.000000in}}{\pgfqpoint{0.034722in}{0.000000in}}{%
\pgfpathmoveto{\pgfqpoint{0.000000in}{0.000000in}}%
\pgfpathlineto{\pgfqpoint{0.034722in}{0.000000in}}%
\pgfusepath{stroke,fill}%
}%
\begin{pgfscope}%
\pgfsys@transformshift{1.396958in}{1.757355in}%
\pgfsys@useobject{currentmarker}{}%
\end{pgfscope}%
\end{pgfscope}%
\begin{pgfscope}%
\definecolor{textcolor}{rgb}{0.000000,0.000000,0.000000}%
\pgfsetstrokecolor{textcolor}%
\pgfsetfillcolor{textcolor}%
\pgftext[x=0.876267in, y=1.662384in, left, base]{\color{textcolor}\sffamily\fontsize{18.000000}{21.600000}\selectfont $\displaystyle -1.0$}%
\end{pgfscope}%
\begin{pgfscope}%
\pgfpathrectangle{\pgfqpoint{1.396958in}{1.247073in}}{\pgfqpoint{10.524301in}{6.674186in}}%
\pgfusepath{clip}%
\pgfsetrectcap%
\pgfsetroundjoin%
\pgfsetlinewidth{0.501875pt}%
\definecolor{currentstroke}{rgb}{0.000000,0.000000,0.000000}%
\pgfsetstrokecolor{currentstroke}%
\pgfsetstrokeopacity{0.100000}%
\pgfsetdash{}{0pt}%
\pgfpathmoveto{\pgfqpoint{1.396958in}{3.063456in}}%
\pgfpathlineto{\pgfqpoint{11.921260in}{3.063456in}}%
\pgfusepath{stroke}%
\end{pgfscope}%
\begin{pgfscope}%
\pgfsetbuttcap%
\pgfsetroundjoin%
\definecolor{currentfill}{rgb}{0.000000,0.000000,0.000000}%
\pgfsetfillcolor{currentfill}%
\pgfsetlinewidth{0.501875pt}%
\definecolor{currentstroke}{rgb}{0.000000,0.000000,0.000000}%
\pgfsetstrokecolor{currentstroke}%
\pgfsetdash{}{0pt}%
\pgfsys@defobject{currentmarker}{\pgfqpoint{0.000000in}{0.000000in}}{\pgfqpoint{0.034722in}{0.000000in}}{%
\pgfpathmoveto{\pgfqpoint{0.000000in}{0.000000in}}%
\pgfpathlineto{\pgfqpoint{0.034722in}{0.000000in}}%
\pgfusepath{stroke,fill}%
}%
\begin{pgfscope}%
\pgfsys@transformshift{1.396958in}{3.063456in}%
\pgfsys@useobject{currentmarker}{}%
\end{pgfscope}%
\end{pgfscope}%
\begin{pgfscope}%
\definecolor{textcolor}{rgb}{0.000000,0.000000,0.000000}%
\pgfsetstrokecolor{textcolor}%
\pgfsetfillcolor{textcolor}%
\pgftext[x=0.876267in, y=2.968485in, left, base]{\color{textcolor}\sffamily\fontsize{18.000000}{21.600000}\selectfont $\displaystyle -0.5$}%
\end{pgfscope}%
\begin{pgfscope}%
\pgfpathrectangle{\pgfqpoint{1.396958in}{1.247073in}}{\pgfqpoint{10.524301in}{6.674186in}}%
\pgfusepath{clip}%
\pgfsetrectcap%
\pgfsetroundjoin%
\pgfsetlinewidth{0.501875pt}%
\definecolor{currentstroke}{rgb}{0.000000,0.000000,0.000000}%
\pgfsetstrokecolor{currentstroke}%
\pgfsetstrokeopacity{0.100000}%
\pgfsetdash{}{0pt}%
\pgfpathmoveto{\pgfqpoint{1.396958in}{4.369556in}}%
\pgfpathlineto{\pgfqpoint{11.921260in}{4.369556in}}%
\pgfusepath{stroke}%
\end{pgfscope}%
\begin{pgfscope}%
\pgfsetbuttcap%
\pgfsetroundjoin%
\definecolor{currentfill}{rgb}{0.000000,0.000000,0.000000}%
\pgfsetfillcolor{currentfill}%
\pgfsetlinewidth{0.501875pt}%
\definecolor{currentstroke}{rgb}{0.000000,0.000000,0.000000}%
\pgfsetstrokecolor{currentstroke}%
\pgfsetdash{}{0pt}%
\pgfsys@defobject{currentmarker}{\pgfqpoint{0.000000in}{0.000000in}}{\pgfqpoint{0.034722in}{0.000000in}}{%
\pgfpathmoveto{\pgfqpoint{0.000000in}{0.000000in}}%
\pgfpathlineto{\pgfqpoint{0.034722in}{0.000000in}}%
\pgfusepath{stroke,fill}%
}%
\begin{pgfscope}%
\pgfsys@transformshift{1.396958in}{4.369556in}%
\pgfsys@useobject{currentmarker}{}%
\end{pgfscope}%
\end{pgfscope}%
\begin{pgfscope}%
\definecolor{textcolor}{rgb}{0.000000,0.000000,0.000000}%
\pgfsetstrokecolor{textcolor}%
\pgfsetfillcolor{textcolor}%
\pgftext[x=1.062934in, y=4.274586in, left, base]{\color{textcolor}\sffamily\fontsize{18.000000}{21.600000}\selectfont $\displaystyle 0.0$}%
\end{pgfscope}%
\begin{pgfscope}%
\pgfpathrectangle{\pgfqpoint{1.396958in}{1.247073in}}{\pgfqpoint{10.524301in}{6.674186in}}%
\pgfusepath{clip}%
\pgfsetrectcap%
\pgfsetroundjoin%
\pgfsetlinewidth{0.501875pt}%
\definecolor{currentstroke}{rgb}{0.000000,0.000000,0.000000}%
\pgfsetstrokecolor{currentstroke}%
\pgfsetstrokeopacity{0.100000}%
\pgfsetdash{}{0pt}%
\pgfpathmoveto{\pgfqpoint{1.396958in}{5.675657in}}%
\pgfpathlineto{\pgfqpoint{11.921260in}{5.675657in}}%
\pgfusepath{stroke}%
\end{pgfscope}%
\begin{pgfscope}%
\pgfsetbuttcap%
\pgfsetroundjoin%
\definecolor{currentfill}{rgb}{0.000000,0.000000,0.000000}%
\pgfsetfillcolor{currentfill}%
\pgfsetlinewidth{0.501875pt}%
\definecolor{currentstroke}{rgb}{0.000000,0.000000,0.000000}%
\pgfsetstrokecolor{currentstroke}%
\pgfsetdash{}{0pt}%
\pgfsys@defobject{currentmarker}{\pgfqpoint{0.000000in}{0.000000in}}{\pgfqpoint{0.034722in}{0.000000in}}{%
\pgfpathmoveto{\pgfqpoint{0.000000in}{0.000000in}}%
\pgfpathlineto{\pgfqpoint{0.034722in}{0.000000in}}%
\pgfusepath{stroke,fill}%
}%
\begin{pgfscope}%
\pgfsys@transformshift{1.396958in}{5.675657in}%
\pgfsys@useobject{currentmarker}{}%
\end{pgfscope}%
\end{pgfscope}%
\begin{pgfscope}%
\definecolor{textcolor}{rgb}{0.000000,0.000000,0.000000}%
\pgfsetstrokecolor{textcolor}%
\pgfsetfillcolor{textcolor}%
\pgftext[x=1.062934in, y=5.580686in, left, base]{\color{textcolor}\sffamily\fontsize{18.000000}{21.600000}\selectfont $\displaystyle 0.5$}%
\end{pgfscope}%
\begin{pgfscope}%
\pgfpathrectangle{\pgfqpoint{1.396958in}{1.247073in}}{\pgfqpoint{10.524301in}{6.674186in}}%
\pgfusepath{clip}%
\pgfsetrectcap%
\pgfsetroundjoin%
\pgfsetlinewidth{0.501875pt}%
\definecolor{currentstroke}{rgb}{0.000000,0.000000,0.000000}%
\pgfsetstrokecolor{currentstroke}%
\pgfsetstrokeopacity{0.100000}%
\pgfsetdash{}{0pt}%
\pgfpathmoveto{\pgfqpoint{1.396958in}{6.981758in}}%
\pgfpathlineto{\pgfqpoint{11.921260in}{6.981758in}}%
\pgfusepath{stroke}%
\end{pgfscope}%
\begin{pgfscope}%
\pgfsetbuttcap%
\pgfsetroundjoin%
\definecolor{currentfill}{rgb}{0.000000,0.000000,0.000000}%
\pgfsetfillcolor{currentfill}%
\pgfsetlinewidth{0.501875pt}%
\definecolor{currentstroke}{rgb}{0.000000,0.000000,0.000000}%
\pgfsetstrokecolor{currentstroke}%
\pgfsetdash{}{0pt}%
\pgfsys@defobject{currentmarker}{\pgfqpoint{0.000000in}{0.000000in}}{\pgfqpoint{0.034722in}{0.000000in}}{%
\pgfpathmoveto{\pgfqpoint{0.000000in}{0.000000in}}%
\pgfpathlineto{\pgfqpoint{0.034722in}{0.000000in}}%
\pgfusepath{stroke,fill}%
}%
\begin{pgfscope}%
\pgfsys@transformshift{1.396958in}{6.981758in}%
\pgfsys@useobject{currentmarker}{}%
\end{pgfscope}%
\end{pgfscope}%
\begin{pgfscope}%
\definecolor{textcolor}{rgb}{0.000000,0.000000,0.000000}%
\pgfsetstrokecolor{textcolor}%
\pgfsetfillcolor{textcolor}%
\pgftext[x=1.062934in, y=6.886787in, left, base]{\color{textcolor}\sffamily\fontsize{18.000000}{21.600000}\selectfont $\displaystyle 1.0$}%
\end{pgfscope}%
\begin{pgfscope}%
\pgfpathrectangle{\pgfqpoint{1.396958in}{1.247073in}}{\pgfqpoint{10.524301in}{6.674186in}}%
\pgfusepath{clip}%
\pgfsetbuttcap%
\pgfsetroundjoin%
\pgfsetlinewidth{1.003750pt}%
\definecolor{currentstroke}{rgb}{0.000000,0.605603,0.978680}%
\pgfsetstrokecolor{currentstroke}%
\pgfsetdash{}{0pt}%
\pgfpathmoveto{\pgfqpoint{1.694816in}{4.972369in}}%
\pgfpathlineto{\pgfqpoint{1.772383in}{3.889244in}}%
\pgfpathlineto{\pgfqpoint{1.849950in}{4.663964in}}%
\pgfpathlineto{\pgfqpoint{1.927517in}{3.695544in}}%
\pgfpathlineto{\pgfqpoint{2.005084in}{4.218985in}}%
\pgfpathlineto{\pgfqpoint{2.082651in}{3.681520in}}%
\pgfpathlineto{\pgfqpoint{2.160219in}{3.630439in}}%
\pgfpathlineto{\pgfqpoint{2.237786in}{3.716331in}}%
\pgfpathlineto{\pgfqpoint{2.315353in}{3.134575in}}%
\pgfpathlineto{\pgfqpoint{2.392920in}{3.593308in}}%
\pgfpathlineto{\pgfqpoint{2.470487in}{2.798408in}}%
\pgfpathlineto{\pgfqpoint{2.548054in}{3.420897in}}%
\pgfpathlineto{\pgfqpoint{2.625621in}{2.448518in}}%
\pgfpathlineto{\pgfqpoint{2.703188in}{3.279738in}}%
\pgfpathlineto{\pgfqpoint{2.780755in}{2.221996in}}%
\pgfpathlineto{\pgfqpoint{2.858322in}{2.860638in}}%
\pgfpathlineto{\pgfqpoint{2.935889in}{2.423238in}}%
\pgfpathlineto{\pgfqpoint{3.013456in}{2.094396in}}%
\pgfpathlineto{\pgfqpoint{3.091023in}{2.787293in}}%
\pgfpathlineto{\pgfqpoint{3.168591in}{1.505436in}}%
\pgfpathlineto{\pgfqpoint{3.246158in}{2.763936in}}%
\pgfpathlineto{\pgfqpoint{3.323725in}{1.440023in}}%
\pgfpathlineto{\pgfqpoint{3.401292in}{2.340988in}}%
\pgfpathlineto{\pgfqpoint{3.478859in}{1.623530in}}%
\pgfpathlineto{\pgfqpoint{3.556426in}{1.925915in}}%
\pgfpathlineto{\pgfqpoint{3.633993in}{1.701734in}}%
\pgfpathlineto{\pgfqpoint{3.711560in}{1.716802in}}%
\pgfpathlineto{\pgfqpoint{3.789127in}{1.643649in}}%
\pgfpathlineto{\pgfqpoint{3.866694in}{1.642572in}}%
\pgfpathlineto{\pgfqpoint{3.944261in}{1.559310in}}%
\pgfpathlineto{\pgfqpoint{4.021828in}{1.597179in}}%
\pgfpathlineto{\pgfqpoint{4.099396in}{1.538046in}}%
\pgfpathlineto{\pgfqpoint{4.176963in}{1.521811in}}%
\pgfpathlineto{\pgfqpoint{4.254530in}{1.597052in}}%
\pgfpathlineto{\pgfqpoint{4.332097in}{1.446956in}}%
\pgfpathlineto{\pgfqpoint{4.409664in}{1.674362in}}%
\pgfpathlineto{\pgfqpoint{4.487231in}{1.435966in}}%
\pgfpathlineto{\pgfqpoint{4.564798in}{1.732432in}}%
\pgfpathlineto{\pgfqpoint{4.642365in}{1.492070in}}%
\pgfpathlineto{\pgfqpoint{4.719932in}{1.791739in}}%
\pgfpathlineto{\pgfqpoint{4.797499in}{1.581008in}}%
\pgfpathlineto{\pgfqpoint{4.875066in}{1.893831in}}%
\pgfpathlineto{\pgfqpoint{4.952633in}{1.657029in}}%
\pgfpathlineto{\pgfqpoint{5.030200in}{2.065752in}}%
\pgfpathlineto{\pgfqpoint{5.107768in}{1.735209in}}%
\pgfpathlineto{\pgfqpoint{5.185335in}{2.239413in}}%
\pgfpathlineto{\pgfqpoint{5.262902in}{1.898822in}}%
\pgfpathlineto{\pgfqpoint{5.340469in}{2.368330in}}%
\pgfpathlineto{\pgfqpoint{5.418036in}{2.115350in}}%
\pgfpathlineto{\pgfqpoint{5.495603in}{2.541153in}}%
\pgfpathlineto{\pgfqpoint{5.573170in}{2.294638in}}%
\pgfpathlineto{\pgfqpoint{5.650737in}{2.783191in}}%
\pgfpathlineto{\pgfqpoint{5.728304in}{2.482619in}}%
\pgfpathlineto{\pgfqpoint{5.805871in}{3.009578in}}%
\pgfpathlineto{\pgfqpoint{5.883438in}{2.735822in}}%
\pgfpathlineto{\pgfqpoint{5.961005in}{3.216372in}}%
\pgfpathlineto{\pgfqpoint{6.038573in}{3.010076in}}%
\pgfpathlineto{\pgfqpoint{6.116140in}{3.450110in}}%
\pgfpathlineto{\pgfqpoint{6.193707in}{3.271447in}}%
\pgfpathlineto{\pgfqpoint{6.271274in}{3.723152in}}%
\pgfpathlineto{\pgfqpoint{6.348841in}{3.515189in}}%
\pgfpathlineto{\pgfqpoint{6.426408in}{4.008822in}}%
\pgfpathlineto{\pgfqpoint{6.503975in}{3.810793in}}%
\pgfpathlineto{\pgfqpoint{6.581542in}{4.191860in}}%
\pgfpathlineto{\pgfqpoint{6.659109in}{4.253036in}}%
\pgfpathlineto{\pgfqpoint{6.736676in}{4.265523in}}%
\pgfpathlineto{\pgfqpoint{6.814243in}{4.705245in}}%
\pgfpathlineto{\pgfqpoint{6.891810in}{4.471657in}}%
\pgfpathlineto{\pgfqpoint{6.969377in}{4.903990in}}%
\pgfpathlineto{\pgfqpoint{7.046945in}{4.966452in}}%
\pgfpathlineto{\pgfqpoint{7.124512in}{4.877728in}}%
\pgfpathlineto{\pgfqpoint{7.202079in}{5.511269in}}%
\pgfpathlineto{\pgfqpoint{7.279646in}{4.988785in}}%
\pgfpathlineto{\pgfqpoint{7.357213in}{5.755727in}}%
\pgfpathlineto{\pgfqpoint{7.434780in}{5.410429in}}%
\pgfpathlineto{\pgfqpoint{7.512347in}{5.776406in}}%
\pgfpathlineto{\pgfqpoint{7.589914in}{5.850313in}}%
\pgfpathlineto{\pgfqpoint{7.667481in}{5.911636in}}%
\pgfpathlineto{\pgfqpoint{7.745048in}{6.105400in}}%
\pgfpathlineto{\pgfqpoint{7.822615in}{6.114339in}}%
\pgfpathlineto{\pgfqpoint{7.900182in}{6.408995in}}%
\pgfpathlineto{\pgfqpoint{7.977750in}{6.144079in}}%
\pgfpathlineto{\pgfqpoint{8.055317in}{6.802011in}}%
\pgfpathlineto{\pgfqpoint{8.132884in}{6.220913in}}%
\pgfpathlineto{\pgfqpoint{8.210451in}{6.907274in}}%
\pgfpathlineto{\pgfqpoint{8.288018in}{6.649828in}}%
\pgfpathlineto{\pgfqpoint{8.365585in}{6.675416in}}%
\pgfpathlineto{\pgfqpoint{8.443152in}{7.170808in}}%
\pgfpathlineto{\pgfqpoint{8.520719in}{6.553547in}}%
\pgfpathlineto{\pgfqpoint{8.598286in}{7.353177in}}%
\pgfpathlineto{\pgfqpoint{8.675853in}{6.757193in}}%
\pgfpathlineto{\pgfqpoint{8.753420in}{7.273416in}}%
\pgfpathlineto{\pgfqpoint{8.830987in}{6.982887in}}%
\pgfpathlineto{\pgfqpoint{8.908554in}{7.272444in}}%
\pgfpathlineto{\pgfqpoint{8.986122in}{7.048863in}}%
\pgfpathlineto{\pgfqpoint{9.063689in}{7.273535in}}%
\pgfpathlineto{\pgfqpoint{9.141256in}{7.217125in}}%
\pgfpathlineto{\pgfqpoint{9.218823in}{7.021107in}}%
\pgfpathlineto{\pgfqpoint{9.296390in}{7.549635in}}%
\pgfpathlineto{\pgfqpoint{9.373957in}{6.696402in}}%
\pgfpathlineto{\pgfqpoint{9.451524in}{7.732368in}}%
\pgfpathlineto{\pgfqpoint{9.529091in}{6.565909in}}%
\pgfpathlineto{\pgfqpoint{9.606658in}{7.665094in}}%
\pgfpathlineto{\pgfqpoint{9.684225in}{6.574136in}}%
\pgfpathlineto{\pgfqpoint{9.761792in}{7.466323in}}%
\pgfpathlineto{\pgfqpoint{9.839359in}{6.622054in}}%
\pgfpathlineto{\pgfqpoint{9.916927in}{7.213238in}}%
\pgfpathlineto{\pgfqpoint{9.994494in}{6.613461in}}%
\pgfpathlineto{\pgfqpoint{10.072061in}{7.039789in}}%
\pgfpathlineto{\pgfqpoint{10.149628in}{6.438165in}}%
\pgfpathlineto{\pgfqpoint{10.227195in}{6.948098in}}%
\pgfpathlineto{\pgfqpoint{10.304762in}{6.240260in}}%
\pgfpathlineto{\pgfqpoint{10.382329in}{6.722231in}}%
\pgfpathlineto{\pgfqpoint{10.459896in}{6.185961in}}%
\pgfpathlineto{\pgfqpoint{10.537463in}{6.344050in}}%
\pgfpathlineto{\pgfqpoint{10.615030in}{6.158963in}}%
\pgfpathlineto{\pgfqpoint{10.692597in}{5.981970in}}%
\pgfpathlineto{\pgfqpoint{10.770164in}{6.044764in}}%
\pgfpathlineto{\pgfqpoint{10.847731in}{5.666560in}}%
\pgfpathlineto{\pgfqpoint{10.925299in}{5.867193in}}%
\pgfpathlineto{\pgfqpoint{11.002866in}{5.399575in}}%
\pgfpathlineto{\pgfqpoint{11.080433in}{5.565617in}}%
\pgfpathlineto{\pgfqpoint{11.158000in}{5.286705in}}%
\pgfpathlineto{\pgfqpoint{11.235567in}{5.072501in}}%
\pgfpathlineto{\pgfqpoint{11.313134in}{5.295293in}}%
\pgfpathlineto{\pgfqpoint{11.390701in}{4.546106in}}%
\pgfpathlineto{\pgfqpoint{11.468268in}{5.210623in}}%
\pgfpathlineto{\pgfqpoint{11.545835in}{4.167135in}}%
\pgfpathlineto{\pgfqpoint{11.623402in}{4.972369in}}%
\pgfpathlineto{\pgfqpoint{11.623402in}{4.972369in}}%
\pgfusepath{stroke}%
\end{pgfscope}%
\begin{pgfscope}%
\pgfpathrectangle{\pgfqpoint{1.396958in}{1.247073in}}{\pgfqpoint{10.524301in}{6.674186in}}%
\pgfusepath{clip}%
\pgfsetbuttcap%
\pgfsetroundjoin%
\pgfsetlinewidth{1.003750pt}%
\definecolor{currentstroke}{rgb}{0.888874,0.435649,0.278123}%
\pgfsetstrokecolor{currentstroke}%
\pgfsetdash{}{0pt}%
\pgfpathmoveto{\pgfqpoint{1.694816in}{4.369556in}}%
\pgfpathlineto{\pgfqpoint{1.927517in}{3.986267in}}%
\pgfpathlineto{\pgfqpoint{2.082651in}{3.734843in}}%
\pgfpathlineto{\pgfqpoint{2.237786in}{3.489532in}}%
\pgfpathlineto{\pgfqpoint{2.315353in}{3.369910in}}%
\pgfpathlineto{\pgfqpoint{2.392920in}{3.252696in}}%
\pgfpathlineto{\pgfqpoint{2.470487in}{3.138173in}}%
\pgfpathlineto{\pgfqpoint{2.548054in}{3.026616in}}%
\pgfpathlineto{\pgfqpoint{2.625621in}{2.918295in}}%
\pgfpathlineto{\pgfqpoint{2.703188in}{2.813470in}}%
\pgfpathlineto{\pgfqpoint{2.780755in}{2.712393in}}%
\pgfpathlineto{\pgfqpoint{2.858322in}{2.615309in}}%
\pgfpathlineto{\pgfqpoint{2.935889in}{2.522451in}}%
\pgfpathlineto{\pgfqpoint{3.013456in}{2.434043in}}%
\pgfpathlineto{\pgfqpoint{3.091023in}{2.350297in}}%
\pgfpathlineto{\pgfqpoint{3.168591in}{2.271417in}}%
\pgfpathlineto{\pgfqpoint{3.246158in}{2.197590in}}%
\pgfpathlineto{\pgfqpoint{3.323725in}{2.128997in}}%
\pgfpathlineto{\pgfqpoint{3.401292in}{2.065800in}}%
\pgfpathlineto{\pgfqpoint{3.478859in}{2.008154in}}%
\pgfpathlineto{\pgfqpoint{3.556426in}{1.956197in}}%
\pgfpathlineto{\pgfqpoint{3.633993in}{1.910054in}}%
\pgfpathlineto{\pgfqpoint{3.711560in}{1.869836in}}%
\pgfpathlineto{\pgfqpoint{3.789127in}{1.835639in}}%
\pgfpathlineto{\pgfqpoint{3.866694in}{1.807548in}}%
\pgfpathlineto{\pgfqpoint{3.944261in}{1.785628in}}%
\pgfpathlineto{\pgfqpoint{4.021828in}{1.769933in}}%
\pgfpathlineto{\pgfqpoint{4.099396in}{1.760502in}}%
\pgfpathlineto{\pgfqpoint{4.176963in}{1.757355in}}%
\pgfpathlineto{\pgfqpoint{4.254530in}{1.760502in}}%
\pgfpathlineto{\pgfqpoint{4.332097in}{1.769933in}}%
\pgfpathlineto{\pgfqpoint{4.409664in}{1.785628in}}%
\pgfpathlineto{\pgfqpoint{4.487231in}{1.807548in}}%
\pgfpathlineto{\pgfqpoint{4.564798in}{1.835639in}}%
\pgfpathlineto{\pgfqpoint{4.642365in}{1.869836in}}%
\pgfpathlineto{\pgfqpoint{4.719932in}{1.910054in}}%
\pgfpathlineto{\pgfqpoint{4.797499in}{1.956197in}}%
\pgfpathlineto{\pgfqpoint{4.875066in}{2.008154in}}%
\pgfpathlineto{\pgfqpoint{4.952633in}{2.065800in}}%
\pgfpathlineto{\pgfqpoint{5.030200in}{2.128997in}}%
\pgfpathlineto{\pgfqpoint{5.107768in}{2.197590in}}%
\pgfpathlineto{\pgfqpoint{5.185335in}{2.271417in}}%
\pgfpathlineto{\pgfqpoint{5.262902in}{2.350297in}}%
\pgfpathlineto{\pgfqpoint{5.340469in}{2.434043in}}%
\pgfpathlineto{\pgfqpoint{5.418036in}{2.522451in}}%
\pgfpathlineto{\pgfqpoint{5.495603in}{2.615309in}}%
\pgfpathlineto{\pgfqpoint{5.573170in}{2.712393in}}%
\pgfpathlineto{\pgfqpoint{5.650737in}{2.813470in}}%
\pgfpathlineto{\pgfqpoint{5.728304in}{2.918295in}}%
\pgfpathlineto{\pgfqpoint{5.805871in}{3.026616in}}%
\pgfpathlineto{\pgfqpoint{5.883438in}{3.138173in}}%
\pgfpathlineto{\pgfqpoint{5.961005in}{3.252696in}}%
\pgfpathlineto{\pgfqpoint{6.038573in}{3.369910in}}%
\pgfpathlineto{\pgfqpoint{6.116140in}{3.489532in}}%
\pgfpathlineto{\pgfqpoint{6.271274in}{3.734843in}}%
\pgfpathlineto{\pgfqpoint{6.426408in}{3.986267in}}%
\pgfpathlineto{\pgfqpoint{6.659109in}{4.369556in}}%
\pgfpathlineto{\pgfqpoint{6.891810in}{4.752846in}}%
\pgfpathlineto{\pgfqpoint{7.046945in}{5.004269in}}%
\pgfpathlineto{\pgfqpoint{7.202079in}{5.249580in}}%
\pgfpathlineto{\pgfqpoint{7.279646in}{5.369202in}}%
\pgfpathlineto{\pgfqpoint{7.357213in}{5.486416in}}%
\pgfpathlineto{\pgfqpoint{7.434780in}{5.600939in}}%
\pgfpathlineto{\pgfqpoint{7.512347in}{5.712496in}}%
\pgfpathlineto{\pgfqpoint{7.589914in}{5.820818in}}%
\pgfpathlineto{\pgfqpoint{7.667481in}{5.925643in}}%
\pgfpathlineto{\pgfqpoint{7.745048in}{6.026719in}}%
\pgfpathlineto{\pgfqpoint{7.822615in}{6.123803in}}%
\pgfpathlineto{\pgfqpoint{7.900182in}{6.216662in}}%
\pgfpathlineto{\pgfqpoint{7.977750in}{6.305070in}}%
\pgfpathlineto{\pgfqpoint{8.055317in}{6.388815in}}%
\pgfpathlineto{\pgfqpoint{8.132884in}{6.467696in}}%
\pgfpathlineto{\pgfqpoint{8.210451in}{6.541522in}}%
\pgfpathlineto{\pgfqpoint{8.288018in}{6.610116in}}%
\pgfpathlineto{\pgfqpoint{8.365585in}{6.673312in}}%
\pgfpathlineto{\pgfqpoint{8.443152in}{6.730958in}}%
\pgfpathlineto{\pgfqpoint{8.520719in}{6.782916in}}%
\pgfpathlineto{\pgfqpoint{8.598286in}{6.829059in}}%
\pgfpathlineto{\pgfqpoint{8.675853in}{6.869277in}}%
\pgfpathlineto{\pgfqpoint{8.753420in}{6.903473in}}%
\pgfpathlineto{\pgfqpoint{8.830987in}{6.931565in}}%
\pgfpathlineto{\pgfqpoint{8.908554in}{6.953484in}}%
\pgfpathlineto{\pgfqpoint{8.986122in}{6.969179in}}%
\pgfpathlineto{\pgfqpoint{9.063689in}{6.978611in}}%
\pgfpathlineto{\pgfqpoint{9.141256in}{6.981758in}}%
\pgfpathlineto{\pgfqpoint{9.218823in}{6.978611in}}%
\pgfpathlineto{\pgfqpoint{9.296390in}{6.969179in}}%
\pgfpathlineto{\pgfqpoint{9.373957in}{6.953484in}}%
\pgfpathlineto{\pgfqpoint{9.451524in}{6.931565in}}%
\pgfpathlineto{\pgfqpoint{9.529091in}{6.903473in}}%
\pgfpathlineto{\pgfqpoint{9.606658in}{6.869277in}}%
\pgfpathlineto{\pgfqpoint{9.684225in}{6.829059in}}%
\pgfpathlineto{\pgfqpoint{9.761792in}{6.782916in}}%
\pgfpathlineto{\pgfqpoint{9.839359in}{6.730958in}}%
\pgfpathlineto{\pgfqpoint{9.916927in}{6.673312in}}%
\pgfpathlineto{\pgfqpoint{9.994494in}{6.610116in}}%
\pgfpathlineto{\pgfqpoint{10.072061in}{6.541522in}}%
\pgfpathlineto{\pgfqpoint{10.149628in}{6.467696in}}%
\pgfpathlineto{\pgfqpoint{10.227195in}{6.388815in}}%
\pgfpathlineto{\pgfqpoint{10.304762in}{6.305070in}}%
\pgfpathlineto{\pgfqpoint{10.382329in}{6.216662in}}%
\pgfpathlineto{\pgfqpoint{10.459896in}{6.123803in}}%
\pgfpathlineto{\pgfqpoint{10.537463in}{6.026719in}}%
\pgfpathlineto{\pgfqpoint{10.615030in}{5.925643in}}%
\pgfpathlineto{\pgfqpoint{10.692597in}{5.820818in}}%
\pgfpathlineto{\pgfqpoint{10.770164in}{5.712496in}}%
\pgfpathlineto{\pgfqpoint{10.847731in}{5.600939in}}%
\pgfpathlineto{\pgfqpoint{10.925299in}{5.486416in}}%
\pgfpathlineto{\pgfqpoint{11.002866in}{5.369202in}}%
\pgfpathlineto{\pgfqpoint{11.080433in}{5.249580in}}%
\pgfpathlineto{\pgfqpoint{11.235567in}{5.004269in}}%
\pgfpathlineto{\pgfqpoint{11.390701in}{4.752846in}}%
\pgfpathlineto{\pgfqpoint{11.623402in}{4.369556in}}%
\pgfpathlineto{\pgfqpoint{11.623402in}{4.369556in}}%
\pgfusepath{stroke}%
\end{pgfscope}%
\begin{pgfscope}%
\pgfsetrectcap%
\pgfsetmiterjoin%
\pgfsetlinewidth{1.003750pt}%
\definecolor{currentstroke}{rgb}{0.000000,0.000000,0.000000}%
\pgfsetstrokecolor{currentstroke}%
\pgfsetdash{}{0pt}%
\pgfpathmoveto{\pgfqpoint{1.396958in}{1.247073in}}%
\pgfpathlineto{\pgfqpoint{1.396958in}{7.921260in}}%
\pgfusepath{stroke}%
\end{pgfscope}%
\begin{pgfscope}%
\pgfsetrectcap%
\pgfsetmiterjoin%
\pgfsetlinewidth{1.003750pt}%
\definecolor{currentstroke}{rgb}{0.000000,0.000000,0.000000}%
\pgfsetstrokecolor{currentstroke}%
\pgfsetdash{}{0pt}%
\pgfpathmoveto{\pgfqpoint{1.396958in}{1.247073in}}%
\pgfpathlineto{\pgfqpoint{11.921260in}{1.247073in}}%
\pgfusepath{stroke}%
\end{pgfscope}%
\begin{pgfscope}%
\pgfsetbuttcap%
\pgfsetmiterjoin%
\definecolor{currentfill}{rgb}{1.000000,1.000000,1.000000}%
\pgfsetfillcolor{currentfill}%
\pgfsetlinewidth{1.003750pt}%
\definecolor{currentstroke}{rgb}{0.000000,0.000000,0.000000}%
\pgfsetstrokecolor{currentstroke}%
\pgfsetdash{}{0pt}%
\pgfpathmoveto{\pgfqpoint{10.511589in}{6.787373in}}%
\pgfpathlineto{\pgfqpoint{11.796260in}{6.787373in}}%
\pgfpathlineto{\pgfqpoint{11.796260in}{7.796260in}}%
\pgfpathlineto{\pgfqpoint{10.511589in}{7.796260in}}%
\pgfpathclose%
\pgfusepath{stroke,fill}%
\end{pgfscope}%
\begin{pgfscope}%
\pgfsetbuttcap%
\pgfsetmiterjoin%
\pgfsetlinewidth{2.258437pt}%
\definecolor{currentstroke}{rgb}{0.000000,0.605603,0.978680}%
\pgfsetstrokecolor{currentstroke}%
\pgfsetdash{}{0pt}%
\pgfpathmoveto{\pgfqpoint{10.711589in}{7.493818in}}%
\pgfpathlineto{\pgfqpoint{11.211589in}{7.493818in}}%
\pgfusepath{stroke}%
\end{pgfscope}%
\begin{pgfscope}%
\definecolor{textcolor}{rgb}{0.000000,0.000000,0.000000}%
\pgfsetstrokecolor{textcolor}%
\pgfsetfillcolor{textcolor}%
\pgftext[x=11.411589in,y=7.406318in,left,base]{\color{textcolor}\sffamily\fontsize{18.000000}{21.600000}\selectfont $\displaystyle U$}%
\end{pgfscope}%
\begin{pgfscope}%
\pgfsetbuttcap%
\pgfsetmiterjoin%
\pgfsetlinewidth{2.258437pt}%
\definecolor{currentstroke}{rgb}{0.888874,0.435649,0.278123}%
\pgfsetstrokecolor{currentstroke}%
\pgfsetdash{}{0pt}%
\pgfpathmoveto{\pgfqpoint{10.711589in}{7.126875in}}%
\pgfpathlineto{\pgfqpoint{11.211589in}{7.126875in}}%
\pgfusepath{stroke}%
\end{pgfscope}%
\begin{pgfscope}%
\definecolor{textcolor}{rgb}{0.000000,0.000000,0.000000}%
\pgfsetstrokecolor{textcolor}%
\pgfsetfillcolor{textcolor}%
\pgftext[x=11.411589in,y=7.039375in,left,base]{\color{textcolor}\sffamily\fontsize{18.000000}{21.600000}\selectfont $\displaystyle u$}%
\end{pgfscope}%
\end{pgfpicture}%
\makeatother%
\endgroup%
}\quad
	\resizebox{0.4\linewidth}{!}{%% Creator: Matplotlib, PGF backend
%%
%% To include the figure in your LaTeX document, write
%%   \input{<filename>.pgf}
%%
%% Make sure the required packages are loaded in your preamble
%%   \usepackage{pgf}
%%
%% Figures using additional raster images can only be included by \input if
%% they are in the same directory as the main LaTeX file. For loading figures
%% from other directories you can use the `import` package
%%   \usepackage{import}
%%
%% and then include the figures with
%%   \import{<path to file>}{<filename>.pgf}
%%
%% Matplotlib used the following preamble
%%   \usepackage{fontspec}
%%   \setmainfont{DejaVuSerif.ttf}[Path=\detokenize{/Users/quejiahao/.julia/conda/3/lib/python3.9/site-packages/matplotlib/mpl-data/fonts/ttf/}]
%%   \setsansfont{DejaVuSans.ttf}[Path=\detokenize{/Users/quejiahao/.julia/conda/3/lib/python3.9/site-packages/matplotlib/mpl-data/fonts/ttf/}]
%%   \setmonofont{DejaVuSansMono.ttf}[Path=\detokenize{/Users/quejiahao/.julia/conda/3/lib/python3.9/site-packages/matplotlib/mpl-data/fonts/ttf/}]
%%
\begingroup%
\makeatletter%
\begin{pgfpicture}%
\pgfpathrectangle{\pgfpointorigin}{\pgfqpoint{12.000000in}{8.000000in}}%
\pgfusepath{use as bounding box, clip}%
\begin{pgfscope}%
\pgfsetbuttcap%
\pgfsetmiterjoin%
\definecolor{currentfill}{rgb}{1.000000,1.000000,1.000000}%
\pgfsetfillcolor{currentfill}%
\pgfsetlinewidth{0.000000pt}%
\definecolor{currentstroke}{rgb}{1.000000,1.000000,1.000000}%
\pgfsetstrokecolor{currentstroke}%
\pgfsetdash{}{0pt}%
\pgfpathmoveto{\pgfqpoint{0.000000in}{0.000000in}}%
\pgfpathlineto{\pgfqpoint{12.000000in}{0.000000in}}%
\pgfpathlineto{\pgfqpoint{12.000000in}{8.000000in}}%
\pgfpathlineto{\pgfqpoint{0.000000in}{8.000000in}}%
\pgfpathclose%
\pgfusepath{fill}%
\end{pgfscope}%
\begin{pgfscope}%
\pgfsetbuttcap%
\pgfsetmiterjoin%
\definecolor{currentfill}{rgb}{1.000000,1.000000,1.000000}%
\pgfsetfillcolor{currentfill}%
\pgfsetlinewidth{0.000000pt}%
\definecolor{currentstroke}{rgb}{0.000000,0.000000,0.000000}%
\pgfsetstrokecolor{currentstroke}%
\pgfsetstrokeopacity{0.000000}%
\pgfsetdash{}{0pt}%
\pgfpathmoveto{\pgfqpoint{3.128011in}{1.247073in}}%
\pgfpathlineto{\pgfqpoint{11.921260in}{1.247073in}}%
\pgfpathlineto{\pgfqpoint{11.921260in}{7.921260in}}%
\pgfpathlineto{\pgfqpoint{3.128011in}{7.921260in}}%
\pgfpathclose%
\pgfusepath{fill}%
\end{pgfscope}%
\begin{pgfscope}%
\pgfpathrectangle{\pgfqpoint{3.128011in}{1.247073in}}{\pgfqpoint{8.793249in}{6.674186in}}%
\pgfusepath{clip}%
\pgfsetrectcap%
\pgfsetroundjoin%
\pgfsetlinewidth{0.501875pt}%
\definecolor{currentstroke}{rgb}{0.000000,0.000000,0.000000}%
\pgfsetstrokecolor{currentstroke}%
\pgfsetstrokeopacity{0.100000}%
\pgfsetdash{}{0pt}%
\pgfpathmoveto{\pgfqpoint{3.376876in}{1.247073in}}%
\pgfpathlineto{\pgfqpoint{3.376876in}{7.921260in}}%
\pgfusepath{stroke}%
\end{pgfscope}%
\begin{pgfscope}%
\pgfsetbuttcap%
\pgfsetroundjoin%
\definecolor{currentfill}{rgb}{0.000000,0.000000,0.000000}%
\pgfsetfillcolor{currentfill}%
\pgfsetlinewidth{0.501875pt}%
\definecolor{currentstroke}{rgb}{0.000000,0.000000,0.000000}%
\pgfsetstrokecolor{currentstroke}%
\pgfsetdash{}{0pt}%
\pgfsys@defobject{currentmarker}{\pgfqpoint{0.000000in}{0.000000in}}{\pgfqpoint{0.000000in}{0.034722in}}{%
\pgfpathmoveto{\pgfqpoint{0.000000in}{0.000000in}}%
\pgfpathlineto{\pgfqpoint{0.000000in}{0.034722in}}%
\pgfusepath{stroke,fill}%
}%
\begin{pgfscope}%
\pgfsys@transformshift{3.376876in}{1.247073in}%
\pgfsys@useobject{currentmarker}{}%
\end{pgfscope}%
\end{pgfscope}%
\begin{pgfscope}%
\definecolor{textcolor}{rgb}{0.000000,0.000000,0.000000}%
\pgfsetstrokecolor{textcolor}%
\pgfsetfillcolor{textcolor}%
\pgftext[x=3.376876in,y=1.198462in,,top]{\color{textcolor}\sffamily\fontsize{18.000000}{21.600000}\selectfont $\displaystyle 0$}%
\end{pgfscope}%
\begin{pgfscope}%
\pgfpathrectangle{\pgfqpoint{3.128011in}{1.247073in}}{\pgfqpoint{8.793249in}{6.674186in}}%
\pgfusepath{clip}%
\pgfsetrectcap%
\pgfsetroundjoin%
\pgfsetlinewidth{0.501875pt}%
\definecolor{currentstroke}{rgb}{0.000000,0.000000,0.000000}%
\pgfsetstrokecolor{currentstroke}%
\pgfsetstrokeopacity{0.100000}%
\pgfsetdash{}{0pt}%
\pgfpathmoveto{\pgfqpoint{4.697149in}{1.247073in}}%
\pgfpathlineto{\pgfqpoint{4.697149in}{7.921260in}}%
\pgfusepath{stroke}%
\end{pgfscope}%
\begin{pgfscope}%
\pgfsetbuttcap%
\pgfsetroundjoin%
\definecolor{currentfill}{rgb}{0.000000,0.000000,0.000000}%
\pgfsetfillcolor{currentfill}%
\pgfsetlinewidth{0.501875pt}%
\definecolor{currentstroke}{rgb}{0.000000,0.000000,0.000000}%
\pgfsetstrokecolor{currentstroke}%
\pgfsetdash{}{0pt}%
\pgfsys@defobject{currentmarker}{\pgfqpoint{0.000000in}{0.000000in}}{\pgfqpoint{0.000000in}{0.034722in}}{%
\pgfpathmoveto{\pgfqpoint{0.000000in}{0.000000in}}%
\pgfpathlineto{\pgfqpoint{0.000000in}{0.034722in}}%
\pgfusepath{stroke,fill}%
}%
\begin{pgfscope}%
\pgfsys@transformshift{4.697149in}{1.247073in}%
\pgfsys@useobject{currentmarker}{}%
\end{pgfscope}%
\end{pgfscope}%
\begin{pgfscope}%
\definecolor{textcolor}{rgb}{0.000000,0.000000,0.000000}%
\pgfsetstrokecolor{textcolor}%
\pgfsetfillcolor{textcolor}%
\pgftext[x=4.697149in,y=1.198462in,,top]{\color{textcolor}\sffamily\fontsize{18.000000}{21.600000}\selectfont $\displaystyle 1$}%
\end{pgfscope}%
\begin{pgfscope}%
\pgfpathrectangle{\pgfqpoint{3.128011in}{1.247073in}}{\pgfqpoint{8.793249in}{6.674186in}}%
\pgfusepath{clip}%
\pgfsetrectcap%
\pgfsetroundjoin%
\pgfsetlinewidth{0.501875pt}%
\definecolor{currentstroke}{rgb}{0.000000,0.000000,0.000000}%
\pgfsetstrokecolor{currentstroke}%
\pgfsetstrokeopacity{0.100000}%
\pgfsetdash{}{0pt}%
\pgfpathmoveto{\pgfqpoint{6.017422in}{1.247073in}}%
\pgfpathlineto{\pgfqpoint{6.017422in}{7.921260in}}%
\pgfusepath{stroke}%
\end{pgfscope}%
\begin{pgfscope}%
\pgfsetbuttcap%
\pgfsetroundjoin%
\definecolor{currentfill}{rgb}{0.000000,0.000000,0.000000}%
\pgfsetfillcolor{currentfill}%
\pgfsetlinewidth{0.501875pt}%
\definecolor{currentstroke}{rgb}{0.000000,0.000000,0.000000}%
\pgfsetstrokecolor{currentstroke}%
\pgfsetdash{}{0pt}%
\pgfsys@defobject{currentmarker}{\pgfqpoint{0.000000in}{0.000000in}}{\pgfqpoint{0.000000in}{0.034722in}}{%
\pgfpathmoveto{\pgfqpoint{0.000000in}{0.000000in}}%
\pgfpathlineto{\pgfqpoint{0.000000in}{0.034722in}}%
\pgfusepath{stroke,fill}%
}%
\begin{pgfscope}%
\pgfsys@transformshift{6.017422in}{1.247073in}%
\pgfsys@useobject{currentmarker}{}%
\end{pgfscope}%
\end{pgfscope}%
\begin{pgfscope}%
\definecolor{textcolor}{rgb}{0.000000,0.000000,0.000000}%
\pgfsetstrokecolor{textcolor}%
\pgfsetfillcolor{textcolor}%
\pgftext[x=6.017422in,y=1.198462in,,top]{\color{textcolor}\sffamily\fontsize{18.000000}{21.600000}\selectfont $\displaystyle 2$}%
\end{pgfscope}%
\begin{pgfscope}%
\pgfpathrectangle{\pgfqpoint{3.128011in}{1.247073in}}{\pgfqpoint{8.793249in}{6.674186in}}%
\pgfusepath{clip}%
\pgfsetrectcap%
\pgfsetroundjoin%
\pgfsetlinewidth{0.501875pt}%
\definecolor{currentstroke}{rgb}{0.000000,0.000000,0.000000}%
\pgfsetstrokecolor{currentstroke}%
\pgfsetstrokeopacity{0.100000}%
\pgfsetdash{}{0pt}%
\pgfpathmoveto{\pgfqpoint{7.337694in}{1.247073in}}%
\pgfpathlineto{\pgfqpoint{7.337694in}{7.921260in}}%
\pgfusepath{stroke}%
\end{pgfscope}%
\begin{pgfscope}%
\pgfsetbuttcap%
\pgfsetroundjoin%
\definecolor{currentfill}{rgb}{0.000000,0.000000,0.000000}%
\pgfsetfillcolor{currentfill}%
\pgfsetlinewidth{0.501875pt}%
\definecolor{currentstroke}{rgb}{0.000000,0.000000,0.000000}%
\pgfsetstrokecolor{currentstroke}%
\pgfsetdash{}{0pt}%
\pgfsys@defobject{currentmarker}{\pgfqpoint{0.000000in}{0.000000in}}{\pgfqpoint{0.000000in}{0.034722in}}{%
\pgfpathmoveto{\pgfqpoint{0.000000in}{0.000000in}}%
\pgfpathlineto{\pgfqpoint{0.000000in}{0.034722in}}%
\pgfusepath{stroke,fill}%
}%
\begin{pgfscope}%
\pgfsys@transformshift{7.337694in}{1.247073in}%
\pgfsys@useobject{currentmarker}{}%
\end{pgfscope}%
\end{pgfscope}%
\begin{pgfscope}%
\definecolor{textcolor}{rgb}{0.000000,0.000000,0.000000}%
\pgfsetstrokecolor{textcolor}%
\pgfsetfillcolor{textcolor}%
\pgftext[x=7.337694in,y=1.198462in,,top]{\color{textcolor}\sffamily\fontsize{18.000000}{21.600000}\selectfont $\displaystyle 3$}%
\end{pgfscope}%
\begin{pgfscope}%
\pgfpathrectangle{\pgfqpoint{3.128011in}{1.247073in}}{\pgfqpoint{8.793249in}{6.674186in}}%
\pgfusepath{clip}%
\pgfsetrectcap%
\pgfsetroundjoin%
\pgfsetlinewidth{0.501875pt}%
\definecolor{currentstroke}{rgb}{0.000000,0.000000,0.000000}%
\pgfsetstrokecolor{currentstroke}%
\pgfsetstrokeopacity{0.100000}%
\pgfsetdash{}{0pt}%
\pgfpathmoveto{\pgfqpoint{8.657967in}{1.247073in}}%
\pgfpathlineto{\pgfqpoint{8.657967in}{7.921260in}}%
\pgfusepath{stroke}%
\end{pgfscope}%
\begin{pgfscope}%
\pgfsetbuttcap%
\pgfsetroundjoin%
\definecolor{currentfill}{rgb}{0.000000,0.000000,0.000000}%
\pgfsetfillcolor{currentfill}%
\pgfsetlinewidth{0.501875pt}%
\definecolor{currentstroke}{rgb}{0.000000,0.000000,0.000000}%
\pgfsetstrokecolor{currentstroke}%
\pgfsetdash{}{0pt}%
\pgfsys@defobject{currentmarker}{\pgfqpoint{0.000000in}{0.000000in}}{\pgfqpoint{0.000000in}{0.034722in}}{%
\pgfpathmoveto{\pgfqpoint{0.000000in}{0.000000in}}%
\pgfpathlineto{\pgfqpoint{0.000000in}{0.034722in}}%
\pgfusepath{stroke,fill}%
}%
\begin{pgfscope}%
\pgfsys@transformshift{8.657967in}{1.247073in}%
\pgfsys@useobject{currentmarker}{}%
\end{pgfscope}%
\end{pgfscope}%
\begin{pgfscope}%
\definecolor{textcolor}{rgb}{0.000000,0.000000,0.000000}%
\pgfsetstrokecolor{textcolor}%
\pgfsetfillcolor{textcolor}%
\pgftext[x=8.657967in,y=1.198462in,,top]{\color{textcolor}\sffamily\fontsize{18.000000}{21.600000}\selectfont $\displaystyle 4$}%
\end{pgfscope}%
\begin{pgfscope}%
\pgfpathrectangle{\pgfqpoint{3.128011in}{1.247073in}}{\pgfqpoint{8.793249in}{6.674186in}}%
\pgfusepath{clip}%
\pgfsetrectcap%
\pgfsetroundjoin%
\pgfsetlinewidth{0.501875pt}%
\definecolor{currentstroke}{rgb}{0.000000,0.000000,0.000000}%
\pgfsetstrokecolor{currentstroke}%
\pgfsetstrokeopacity{0.100000}%
\pgfsetdash{}{0pt}%
\pgfpathmoveto{\pgfqpoint{9.978240in}{1.247073in}}%
\pgfpathlineto{\pgfqpoint{9.978240in}{7.921260in}}%
\pgfusepath{stroke}%
\end{pgfscope}%
\begin{pgfscope}%
\pgfsetbuttcap%
\pgfsetroundjoin%
\definecolor{currentfill}{rgb}{0.000000,0.000000,0.000000}%
\pgfsetfillcolor{currentfill}%
\pgfsetlinewidth{0.501875pt}%
\definecolor{currentstroke}{rgb}{0.000000,0.000000,0.000000}%
\pgfsetstrokecolor{currentstroke}%
\pgfsetdash{}{0pt}%
\pgfsys@defobject{currentmarker}{\pgfqpoint{0.000000in}{0.000000in}}{\pgfqpoint{0.000000in}{0.034722in}}{%
\pgfpathmoveto{\pgfqpoint{0.000000in}{0.000000in}}%
\pgfpathlineto{\pgfqpoint{0.000000in}{0.034722in}}%
\pgfusepath{stroke,fill}%
}%
\begin{pgfscope}%
\pgfsys@transformshift{9.978240in}{1.247073in}%
\pgfsys@useobject{currentmarker}{}%
\end{pgfscope}%
\end{pgfscope}%
\begin{pgfscope}%
\definecolor{textcolor}{rgb}{0.000000,0.000000,0.000000}%
\pgfsetstrokecolor{textcolor}%
\pgfsetfillcolor{textcolor}%
\pgftext[x=9.978240in,y=1.198462in,,top]{\color{textcolor}\sffamily\fontsize{18.000000}{21.600000}\selectfont $\displaystyle 5$}%
\end{pgfscope}%
\begin{pgfscope}%
\pgfpathrectangle{\pgfqpoint{3.128011in}{1.247073in}}{\pgfqpoint{8.793249in}{6.674186in}}%
\pgfusepath{clip}%
\pgfsetrectcap%
\pgfsetroundjoin%
\pgfsetlinewidth{0.501875pt}%
\definecolor{currentstroke}{rgb}{0.000000,0.000000,0.000000}%
\pgfsetstrokecolor{currentstroke}%
\pgfsetstrokeopacity{0.100000}%
\pgfsetdash{}{0pt}%
\pgfpathmoveto{\pgfqpoint{11.298512in}{1.247073in}}%
\pgfpathlineto{\pgfqpoint{11.298512in}{7.921260in}}%
\pgfusepath{stroke}%
\end{pgfscope}%
\begin{pgfscope}%
\pgfsetbuttcap%
\pgfsetroundjoin%
\definecolor{currentfill}{rgb}{0.000000,0.000000,0.000000}%
\pgfsetfillcolor{currentfill}%
\pgfsetlinewidth{0.501875pt}%
\definecolor{currentstroke}{rgb}{0.000000,0.000000,0.000000}%
\pgfsetstrokecolor{currentstroke}%
\pgfsetdash{}{0pt}%
\pgfsys@defobject{currentmarker}{\pgfqpoint{0.000000in}{0.000000in}}{\pgfqpoint{0.000000in}{0.034722in}}{%
\pgfpathmoveto{\pgfqpoint{0.000000in}{0.000000in}}%
\pgfpathlineto{\pgfqpoint{0.000000in}{0.034722in}}%
\pgfusepath{stroke,fill}%
}%
\begin{pgfscope}%
\pgfsys@transformshift{11.298512in}{1.247073in}%
\pgfsys@useobject{currentmarker}{}%
\end{pgfscope}%
\end{pgfscope}%
\begin{pgfscope}%
\definecolor{textcolor}{rgb}{0.000000,0.000000,0.000000}%
\pgfsetstrokecolor{textcolor}%
\pgfsetfillcolor{textcolor}%
\pgftext[x=11.298512in,y=1.198462in,,top]{\color{textcolor}\sffamily\fontsize{18.000000}{21.600000}\selectfont $\displaystyle 6$}%
\end{pgfscope}%
\begin{pgfscope}%
\definecolor{textcolor}{rgb}{0.000000,0.000000,0.000000}%
\pgfsetstrokecolor{textcolor}%
\pgfsetfillcolor{textcolor}%
\pgftext[x=7.524635in,y=0.900964in,,top]{\color{textcolor}\sffamily\fontsize{18.000000}{21.600000}\selectfont $\displaystyle x$}%
\end{pgfscope}%
\begin{pgfscope}%
\pgfpathrectangle{\pgfqpoint{3.128011in}{1.247073in}}{\pgfqpoint{8.793249in}{6.674186in}}%
\pgfusepath{clip}%
\pgfsetrectcap%
\pgfsetroundjoin%
\pgfsetlinewidth{0.501875pt}%
\definecolor{currentstroke}{rgb}{0.000000,0.000000,0.000000}%
\pgfsetstrokecolor{currentstroke}%
\pgfsetstrokeopacity{0.100000}%
\pgfsetdash{}{0pt}%
\pgfpathmoveto{\pgfqpoint{3.128011in}{1.907163in}}%
\pgfpathlineto{\pgfqpoint{11.921260in}{1.907163in}}%
\pgfusepath{stroke}%
\end{pgfscope}%
\begin{pgfscope}%
\pgfsetbuttcap%
\pgfsetroundjoin%
\definecolor{currentfill}{rgb}{0.000000,0.000000,0.000000}%
\pgfsetfillcolor{currentfill}%
\pgfsetlinewidth{0.501875pt}%
\definecolor{currentstroke}{rgb}{0.000000,0.000000,0.000000}%
\pgfsetstrokecolor{currentstroke}%
\pgfsetdash{}{0pt}%
\pgfsys@defobject{currentmarker}{\pgfqpoint{0.000000in}{0.000000in}}{\pgfqpoint{0.034722in}{0.000000in}}{%
\pgfpathmoveto{\pgfqpoint{0.000000in}{0.000000in}}%
\pgfpathlineto{\pgfqpoint{0.034722in}{0.000000in}}%
\pgfusepath{stroke,fill}%
}%
\begin{pgfscope}%
\pgfsys@transformshift{3.128011in}{1.907163in}%
\pgfsys@useobject{currentmarker}{}%
\end{pgfscope}%
\end{pgfscope}%
\begin{pgfscope}%
\definecolor{textcolor}{rgb}{0.000000,0.000000,0.000000}%
\pgfsetstrokecolor{textcolor}%
\pgfsetfillcolor{textcolor}%
\pgftext[x=1.925977in, y=1.812193in, left, base]{\color{textcolor}\sffamily\fontsize{18.000000}{21.600000}\selectfont $\displaystyle -6.0×10^{227}$}%
\end{pgfscope}%
\begin{pgfscope}%
\pgfpathrectangle{\pgfqpoint{3.128011in}{1.247073in}}{\pgfqpoint{8.793249in}{6.674186in}}%
\pgfusepath{clip}%
\pgfsetrectcap%
\pgfsetroundjoin%
\pgfsetlinewidth{0.501875pt}%
\definecolor{currentstroke}{rgb}{0.000000,0.000000,0.000000}%
\pgfsetstrokecolor{currentstroke}%
\pgfsetstrokeopacity{0.100000}%
\pgfsetdash{}{0pt}%
\pgfpathmoveto{\pgfqpoint{3.128011in}{2.798063in}}%
\pgfpathlineto{\pgfqpoint{11.921260in}{2.798063in}}%
\pgfusepath{stroke}%
\end{pgfscope}%
\begin{pgfscope}%
\pgfsetbuttcap%
\pgfsetroundjoin%
\definecolor{currentfill}{rgb}{0.000000,0.000000,0.000000}%
\pgfsetfillcolor{currentfill}%
\pgfsetlinewidth{0.501875pt}%
\definecolor{currentstroke}{rgb}{0.000000,0.000000,0.000000}%
\pgfsetstrokecolor{currentstroke}%
\pgfsetdash{}{0pt}%
\pgfsys@defobject{currentmarker}{\pgfqpoint{0.000000in}{0.000000in}}{\pgfqpoint{0.034722in}{0.000000in}}{%
\pgfpathmoveto{\pgfqpoint{0.000000in}{0.000000in}}%
\pgfpathlineto{\pgfqpoint{0.034722in}{0.000000in}}%
\pgfusepath{stroke,fill}%
}%
\begin{pgfscope}%
\pgfsys@transformshift{3.128011in}{2.798063in}%
\pgfsys@useobject{currentmarker}{}%
\end{pgfscope}%
\end{pgfscope}%
\begin{pgfscope}%
\definecolor{textcolor}{rgb}{0.000000,0.000000,0.000000}%
\pgfsetstrokecolor{textcolor}%
\pgfsetfillcolor{textcolor}%
\pgftext[x=1.925977in, y=2.703092in, left, base]{\color{textcolor}\sffamily\fontsize{18.000000}{21.600000}\selectfont $\displaystyle -4.0×10^{227}$}%
\end{pgfscope}%
\begin{pgfscope}%
\pgfpathrectangle{\pgfqpoint{3.128011in}{1.247073in}}{\pgfqpoint{8.793249in}{6.674186in}}%
\pgfusepath{clip}%
\pgfsetrectcap%
\pgfsetroundjoin%
\pgfsetlinewidth{0.501875pt}%
\definecolor{currentstroke}{rgb}{0.000000,0.000000,0.000000}%
\pgfsetstrokecolor{currentstroke}%
\pgfsetstrokeopacity{0.100000}%
\pgfsetdash{}{0pt}%
\pgfpathmoveto{\pgfqpoint{3.128011in}{3.688962in}}%
\pgfpathlineto{\pgfqpoint{11.921260in}{3.688962in}}%
\pgfusepath{stroke}%
\end{pgfscope}%
\begin{pgfscope}%
\pgfsetbuttcap%
\pgfsetroundjoin%
\definecolor{currentfill}{rgb}{0.000000,0.000000,0.000000}%
\pgfsetfillcolor{currentfill}%
\pgfsetlinewidth{0.501875pt}%
\definecolor{currentstroke}{rgb}{0.000000,0.000000,0.000000}%
\pgfsetstrokecolor{currentstroke}%
\pgfsetdash{}{0pt}%
\pgfsys@defobject{currentmarker}{\pgfqpoint{0.000000in}{0.000000in}}{\pgfqpoint{0.034722in}{0.000000in}}{%
\pgfpathmoveto{\pgfqpoint{0.000000in}{0.000000in}}%
\pgfpathlineto{\pgfqpoint{0.034722in}{0.000000in}}%
\pgfusepath{stroke,fill}%
}%
\begin{pgfscope}%
\pgfsys@transformshift{3.128011in}{3.688962in}%
\pgfsys@useobject{currentmarker}{}%
\end{pgfscope}%
\end{pgfscope}%
\begin{pgfscope}%
\definecolor{textcolor}{rgb}{0.000000,0.000000,0.000000}%
\pgfsetstrokecolor{textcolor}%
\pgfsetfillcolor{textcolor}%
\pgftext[x=1.925977in, y=3.593991in, left, base]{\color{textcolor}\sffamily\fontsize{18.000000}{21.600000}\selectfont $\displaystyle -2.0×10^{227}$}%
\end{pgfscope}%
\begin{pgfscope}%
\pgfpathrectangle{\pgfqpoint{3.128011in}{1.247073in}}{\pgfqpoint{8.793249in}{6.674186in}}%
\pgfusepath{clip}%
\pgfsetrectcap%
\pgfsetroundjoin%
\pgfsetlinewidth{0.501875pt}%
\definecolor{currentstroke}{rgb}{0.000000,0.000000,0.000000}%
\pgfsetstrokecolor{currentstroke}%
\pgfsetstrokeopacity{0.100000}%
\pgfsetdash{}{0pt}%
\pgfpathmoveto{\pgfqpoint{3.128011in}{4.579861in}}%
\pgfpathlineto{\pgfqpoint{11.921260in}{4.579861in}}%
\pgfusepath{stroke}%
\end{pgfscope}%
\begin{pgfscope}%
\pgfsetbuttcap%
\pgfsetroundjoin%
\definecolor{currentfill}{rgb}{0.000000,0.000000,0.000000}%
\pgfsetfillcolor{currentfill}%
\pgfsetlinewidth{0.501875pt}%
\definecolor{currentstroke}{rgb}{0.000000,0.000000,0.000000}%
\pgfsetstrokecolor{currentstroke}%
\pgfsetdash{}{0pt}%
\pgfsys@defobject{currentmarker}{\pgfqpoint{0.000000in}{0.000000in}}{\pgfqpoint{0.034722in}{0.000000in}}{%
\pgfpathmoveto{\pgfqpoint{0.000000in}{0.000000in}}%
\pgfpathlineto{\pgfqpoint{0.034722in}{0.000000in}}%
\pgfusepath{stroke,fill}%
}%
\begin{pgfscope}%
\pgfsys@transformshift{3.128011in}{4.579861in}%
\pgfsys@useobject{currentmarker}{}%
\end{pgfscope}%
\end{pgfscope}%
\begin{pgfscope}%
\definecolor{textcolor}{rgb}{0.000000,0.000000,0.000000}%
\pgfsetstrokecolor{textcolor}%
\pgfsetfillcolor{textcolor}%
\pgftext[x=2.969332in, y=4.484891in, left, base]{\color{textcolor}\sffamily\fontsize{18.000000}{21.600000}\selectfont $\displaystyle 0$}%
\end{pgfscope}%
\begin{pgfscope}%
\pgfpathrectangle{\pgfqpoint{3.128011in}{1.247073in}}{\pgfqpoint{8.793249in}{6.674186in}}%
\pgfusepath{clip}%
\pgfsetrectcap%
\pgfsetroundjoin%
\pgfsetlinewidth{0.501875pt}%
\definecolor{currentstroke}{rgb}{0.000000,0.000000,0.000000}%
\pgfsetstrokecolor{currentstroke}%
\pgfsetstrokeopacity{0.100000}%
\pgfsetdash{}{0pt}%
\pgfpathmoveto{\pgfqpoint{3.128011in}{5.470761in}}%
\pgfpathlineto{\pgfqpoint{11.921260in}{5.470761in}}%
\pgfusepath{stroke}%
\end{pgfscope}%
\begin{pgfscope}%
\pgfsetbuttcap%
\pgfsetroundjoin%
\definecolor{currentfill}{rgb}{0.000000,0.000000,0.000000}%
\pgfsetfillcolor{currentfill}%
\pgfsetlinewidth{0.501875pt}%
\definecolor{currentstroke}{rgb}{0.000000,0.000000,0.000000}%
\pgfsetstrokecolor{currentstroke}%
\pgfsetdash{}{0pt}%
\pgfsys@defobject{currentmarker}{\pgfqpoint{0.000000in}{0.000000in}}{\pgfqpoint{0.034722in}{0.000000in}}{%
\pgfpathmoveto{\pgfqpoint{0.000000in}{0.000000in}}%
\pgfpathlineto{\pgfqpoint{0.034722in}{0.000000in}}%
\pgfusepath{stroke,fill}%
}%
\begin{pgfscope}%
\pgfsys@transformshift{3.128011in}{5.470761in}%
\pgfsys@useobject{currentmarker}{}%
\end{pgfscope}%
\end{pgfscope}%
\begin{pgfscope}%
\definecolor{textcolor}{rgb}{0.000000,0.000000,0.000000}%
\pgfsetstrokecolor{textcolor}%
\pgfsetfillcolor{textcolor}%
\pgftext[x=2.112644in, y=5.375790in, left, base]{\color{textcolor}\sffamily\fontsize{18.000000}{21.600000}\selectfont $\displaystyle 2.0×10^{227}$}%
\end{pgfscope}%
\begin{pgfscope}%
\pgfpathrectangle{\pgfqpoint{3.128011in}{1.247073in}}{\pgfqpoint{8.793249in}{6.674186in}}%
\pgfusepath{clip}%
\pgfsetrectcap%
\pgfsetroundjoin%
\pgfsetlinewidth{0.501875pt}%
\definecolor{currentstroke}{rgb}{0.000000,0.000000,0.000000}%
\pgfsetstrokecolor{currentstroke}%
\pgfsetstrokeopacity{0.100000}%
\pgfsetdash{}{0pt}%
\pgfpathmoveto{\pgfqpoint{3.128011in}{6.361660in}}%
\pgfpathlineto{\pgfqpoint{11.921260in}{6.361660in}}%
\pgfusepath{stroke}%
\end{pgfscope}%
\begin{pgfscope}%
\pgfsetbuttcap%
\pgfsetroundjoin%
\definecolor{currentfill}{rgb}{0.000000,0.000000,0.000000}%
\pgfsetfillcolor{currentfill}%
\pgfsetlinewidth{0.501875pt}%
\definecolor{currentstroke}{rgb}{0.000000,0.000000,0.000000}%
\pgfsetstrokecolor{currentstroke}%
\pgfsetdash{}{0pt}%
\pgfsys@defobject{currentmarker}{\pgfqpoint{0.000000in}{0.000000in}}{\pgfqpoint{0.034722in}{0.000000in}}{%
\pgfpathmoveto{\pgfqpoint{0.000000in}{0.000000in}}%
\pgfpathlineto{\pgfqpoint{0.034722in}{0.000000in}}%
\pgfusepath{stroke,fill}%
}%
\begin{pgfscope}%
\pgfsys@transformshift{3.128011in}{6.361660in}%
\pgfsys@useobject{currentmarker}{}%
\end{pgfscope}%
\end{pgfscope}%
\begin{pgfscope}%
\definecolor{textcolor}{rgb}{0.000000,0.000000,0.000000}%
\pgfsetstrokecolor{textcolor}%
\pgfsetfillcolor{textcolor}%
\pgftext[x=2.112644in, y=6.266689in, left, base]{\color{textcolor}\sffamily\fontsize{18.000000}{21.600000}\selectfont $\displaystyle 4.0×10^{227}$}%
\end{pgfscope}%
\begin{pgfscope}%
\pgfpathrectangle{\pgfqpoint{3.128011in}{1.247073in}}{\pgfqpoint{8.793249in}{6.674186in}}%
\pgfusepath{clip}%
\pgfsetrectcap%
\pgfsetroundjoin%
\pgfsetlinewidth{0.501875pt}%
\definecolor{currentstroke}{rgb}{0.000000,0.000000,0.000000}%
\pgfsetstrokecolor{currentstroke}%
\pgfsetstrokeopacity{0.100000}%
\pgfsetdash{}{0pt}%
\pgfpathmoveto{\pgfqpoint{3.128011in}{7.252559in}}%
\pgfpathlineto{\pgfqpoint{11.921260in}{7.252559in}}%
\pgfusepath{stroke}%
\end{pgfscope}%
\begin{pgfscope}%
\pgfsetbuttcap%
\pgfsetroundjoin%
\definecolor{currentfill}{rgb}{0.000000,0.000000,0.000000}%
\pgfsetfillcolor{currentfill}%
\pgfsetlinewidth{0.501875pt}%
\definecolor{currentstroke}{rgb}{0.000000,0.000000,0.000000}%
\pgfsetstrokecolor{currentstroke}%
\pgfsetdash{}{0pt}%
\pgfsys@defobject{currentmarker}{\pgfqpoint{0.000000in}{0.000000in}}{\pgfqpoint{0.034722in}{0.000000in}}{%
\pgfpathmoveto{\pgfqpoint{0.000000in}{0.000000in}}%
\pgfpathlineto{\pgfqpoint{0.034722in}{0.000000in}}%
\pgfusepath{stroke,fill}%
}%
\begin{pgfscope}%
\pgfsys@transformshift{3.128011in}{7.252559in}%
\pgfsys@useobject{currentmarker}{}%
\end{pgfscope}%
\end{pgfscope}%
\begin{pgfscope}%
\definecolor{textcolor}{rgb}{0.000000,0.000000,0.000000}%
\pgfsetstrokecolor{textcolor}%
\pgfsetfillcolor{textcolor}%
\pgftext[x=2.112644in, y=7.157588in, left, base]{\color{textcolor}\sffamily\fontsize{18.000000}{21.600000}\selectfont $\displaystyle 6.0×10^{227}$}%
\end{pgfscope}%
\begin{pgfscope}%
\pgfpathrectangle{\pgfqpoint{3.128011in}{1.247073in}}{\pgfqpoint{8.793249in}{6.674186in}}%
\pgfusepath{clip}%
\pgfsetbuttcap%
\pgfsetroundjoin%
\pgfsetlinewidth{1.003750pt}%
\definecolor{currentstroke}{rgb}{0.000000,0.605603,0.978680}%
\pgfsetstrokecolor{currentstroke}%
\pgfsetdash{}{0pt}%
\pgfpathmoveto{\pgfqpoint{3.376876in}{4.551499in}}%
\pgfpathlineto{\pgfqpoint{3.378902in}{4.731764in}}%
\pgfpathlineto{\pgfqpoint{3.380927in}{4.301928in}}%
\pgfpathlineto{\pgfqpoint{3.382952in}{4.986153in}}%
\pgfpathlineto{\pgfqpoint{3.384977in}{4.043547in}}%
\pgfpathlineto{\pgfqpoint{3.387003in}{5.246692in}}%
\pgfpathlineto{\pgfqpoint{3.389028in}{3.783663in}}%
\pgfpathlineto{\pgfqpoint{3.391053in}{5.502238in}}%
\pgfpathlineto{\pgfqpoint{3.393079in}{3.536809in}}%
\pgfpathlineto{\pgfqpoint{3.395104in}{5.735634in}}%
\pgfpathlineto{\pgfqpoint{3.397129in}{3.321741in}}%
\pgfpathlineto{\pgfqpoint{3.399154in}{5.927735in}}%
\pgfpathlineto{\pgfqpoint{3.401180in}{3.156689in}}%
\pgfpathlineto{\pgfqpoint{3.403205in}{6.062515in}}%
\pgfpathlineto{\pgfqpoint{3.405230in}{3.054300in}}%
\pgfpathlineto{\pgfqpoint{3.407255in}{6.131672in}}%
\pgfpathlineto{\pgfqpoint{3.409281in}{3.017852in}}%
\pgfpathlineto{\pgfqpoint{3.411306in}{6.137283in}}%
\pgfpathlineto{\pgfqpoint{3.413331in}{3.039959in}}%
\pgfpathlineto{\pgfqpoint{3.415357in}{6.091629in}}%
\pgfpathlineto{\pgfqpoint{3.417382in}{3.104211in}}%
\pgfpathlineto{\pgfqpoint{3.419407in}{6.014188in}}%
\pgfpathlineto{\pgfqpoint{3.421432in}{3.189322in}}%
\pgfpathlineto{\pgfqpoint{3.423458in}{5.926679in}}%
\pgfpathlineto{\pgfqpoint{3.425483in}{3.274534in}}%
\pgfpathlineto{\pgfqpoint{3.427508in}{5.847603in}}%
\pgfpathlineto{\pgfqpoint{3.429533in}{3.344704in}}%
\pgfpathlineto{\pgfqpoint{3.431559in}{5.787911in}}%
\pgfpathlineto{\pgfqpoint{3.433584in}{3.393574in}}%
\pgfpathlineto{\pgfqpoint{3.435609in}{5.749037in}}%
\pgfpathlineto{\pgfqpoint{3.437635in}{3.424300in}}%
\pgfpathlineto{\pgfqpoint{3.439660in}{5.723816in}}%
\pgfpathlineto{\pgfqpoint{3.441685in}{3.447179in}}%
\pgfpathlineto{\pgfqpoint{3.443710in}{5.699912in}}%
\pgfpathlineto{\pgfqpoint{3.445736in}{3.475360in}}%
\pgfpathlineto{\pgfqpoint{3.447761in}{5.664620in}}%
\pgfpathlineto{\pgfqpoint{3.449786in}{3.519920in}}%
\pgfpathlineto{\pgfqpoint{3.451811in}{5.609506in}}%
\pgfpathlineto{\pgfqpoint{3.453837in}{3.585886in}}%
\pgfpathlineto{\pgfqpoint{3.455862in}{5.533406in}}%
\pgfpathlineto{\pgfqpoint{3.457887in}{3.670448in}}%
\pgfpathlineto{\pgfqpoint{3.459913in}{5.442855in}}%
\pgfpathlineto{\pgfqpoint{3.461938in}{3.763933in}}%
\pgfpathlineto{\pgfqpoint{3.463963in}{5.349797in}}%
\pgfpathlineto{\pgfqpoint{3.465988in}{3.853205in}}%
\pgfpathlineto{\pgfqpoint{3.468014in}{5.267351in}}%
\pgfpathlineto{\pgfqpoint{3.470039in}{3.926396in}}%
\pgfpathlineto{\pgfqpoint{3.472064in}{5.204986in}}%
\pgfpathlineto{\pgfqpoint{3.474089in}{3.977404in}}%
\pgfpathlineto{\pgfqpoint{3.476115in}{5.164734in}}%
\pgfpathlineto{\pgfqpoint{3.478140in}{4.008637in}}%
\pgfpathlineto{\pgfqpoint{3.480165in}{5.139745in}}%
\pgfpathlineto{\pgfqpoint{3.482191in}{4.031007in}}%
\pgfpathlineto{\pgfqpoint{3.484216in}{5.115782in}}%
\pgfpathlineto{\pgfqpoint{3.486241in}{4.061036in}}%
\pgfpathlineto{\pgfqpoint{3.488266in}{5.075312in}}%
\pgfpathlineto{\pgfqpoint{3.490292in}{4.115862in}}%
\pgfpathlineto{\pgfqpoint{3.492317in}{5.003016in}}%
\pgfpathlineto{\pgfqpoint{3.494342in}{4.207653in}}%
\pgfpathlineto{\pgfqpoint{3.496367in}{4.891005in}}%
\pgfpathlineto{\pgfqpoint{3.498393in}{4.339188in}}%
\pgfpathlineto{\pgfqpoint{3.500418in}{4.742077in}}%
\pgfpathlineto{\pgfqpoint{3.502443in}{4.502040in}}%
\pgfpathlineto{\pgfqpoint{3.504469in}{4.569913in}}%
\pgfpathlineto{\pgfqpoint{3.506494in}{4.678049in}}%
\pgfpathlineto{\pgfqpoint{3.508519in}{4.396026in}}%
\pgfpathlineto{\pgfqpoint{3.510544in}{4.843739in}}%
\pgfpathlineto{\pgfqpoint{3.512570in}{4.244308in}}%
\pgfpathlineto{\pgfqpoint{3.514595in}{4.976395in}}%
\pgfpathlineto{\pgfqpoint{3.516620in}{4.134785in}}%
\pgfpathlineto{\pgfqpoint{3.518645in}{5.059993in}}%
\pgfpathlineto{\pgfqpoint{3.520671in}{4.078466in}}%
\pgfpathlineto{\pgfqpoint{3.522696in}{5.089172in}}%
\pgfpathlineto{\pgfqpoint{3.524721in}{4.074851in}}%
\pgfpathlineto{\pgfqpoint{3.526747in}{5.070065in}}%
\pgfpathlineto{\pgfqpoint{3.528772in}{4.112828in}}%
\pgfpathlineto{\pgfqpoint{3.530797in}{5.017753in}}%
\pgfpathlineto{\pgfqpoint{3.532822in}{4.174632in}}%
\pgfpathlineto{\pgfqpoint{3.534848in}{4.951221in}}%
\pgfpathlineto{\pgfqpoint{3.536873in}{4.241570in}}%
\pgfpathlineto{\pgfqpoint{3.538898in}{4.887435in}}%
\pgfpathlineto{\pgfqpoint{3.540923in}{4.299652in}}%
\pgfpathlineto{\pgfqpoint{3.542949in}{4.836455in}}%
\pgfpathlineto{\pgfqpoint{3.544974in}{4.343331in}}%
\pgfpathlineto{\pgfqpoint{3.546999in}{4.799145in}}%
\pgfpathlineto{\pgfqpoint{3.549025in}{4.376172in}}%
\pgfpathlineto{\pgfqpoint{3.551050in}{4.768162in}}%
\pgfpathlineto{\pgfqpoint{3.553075in}{4.408302in}}%
\pgfpathlineto{\pgfqpoint{3.555100in}{4.731839in}}%
\pgfpathlineto{\pgfqpoint{3.557126in}{4.451547in}}%
\pgfpathlineto{\pgfqpoint{3.559151in}{4.679594in}}%
\pgfpathlineto{\pgfqpoint{3.561176in}{4.513944in}}%
\pgfpathlineto{\pgfqpoint{3.563201in}{4.607011in}}%
\pgfpathlineto{\pgfqpoint{3.565227in}{4.595530in}}%
\pgfpathlineto{\pgfqpoint{3.567252in}{4.518803in}}%
\pgfpathlineto{\pgfqpoint{3.569277in}{4.686905in}}%
\pgfpathlineto{\pgfqpoint{3.571303in}{4.428569in}}%
\pgfpathlineto{\pgfqpoint{3.573328in}{4.771150in}}%
\pgfpathlineto{\pgfqpoint{3.575353in}{4.355334in}}%
\pgfpathlineto{\pgfqpoint{3.577378in}{4.828580in}}%
\pgfpathlineto{\pgfqpoint{3.579404in}{4.317882in}}%
\pgfpathlineto{\pgfqpoint{3.581429in}{4.842859in}}%
\pgfpathlineto{\pgfqpoint{3.583454in}{4.328697in}}%
\pgfpathlineto{\pgfqpoint{3.585479in}{4.806505in}}%
\pgfpathlineto{\pgfqpoint{3.587505in}{4.389466in}}%
\pgfpathlineto{\pgfqpoint{3.589530in}{4.723992in}}%
\pgfpathlineto{\pgfqpoint{3.591555in}{4.489650in}}%
\pgfpathlineto{\pgfqpoint{3.593581in}{4.611359in}}%
\pgfpathlineto{\pgfqpoint{3.595606in}{4.608699in}}%
\pgfpathlineto{\pgfqpoint{3.597631in}{4.492328in}}%
\pgfpathlineto{\pgfqpoint{3.599656in}{4.721319in}}%
\pgfpathlineto{\pgfqpoint{3.601682in}{4.392030in}}%
\pgfpathlineto{\pgfqpoint{3.603707in}{4.804277in}}%
\pgfpathlineto{\pgfqpoint{3.605732in}{4.330194in}}%
\pgfpathlineto{\pgfqpoint{3.607757in}{4.842703in}}%
\pgfpathlineto{\pgfqpoint{3.609783in}{4.315831in}}%
\pgfpathlineto{\pgfqpoint{3.611808in}{4.834006in}}%
\pgfpathlineto{\pgfqpoint{3.613833in}{4.345025in}}%
\pgfpathlineto{\pgfqpoint{3.615859in}{4.788216in}}%
\pgfpathlineto{\pgfqpoint{3.617884in}{4.402487in}}%
\pgfpathlineto{\pgfqpoint{3.619909in}{4.724640in}}%
\pgfpathlineto{\pgfqpoint{3.621934in}{4.466424in}}%
\pgfpathlineto{\pgfqpoint{3.623960in}{4.665845in}}%
\pgfpathlineto{\pgfqpoint{3.625985in}{4.515252in}}%
\pgfpathlineto{\pgfqpoint{3.628010in}{4.630761in}}%
\pgfpathlineto{\pgfqpoint{3.630035in}{4.534147in}}%
\pgfpathlineto{\pgfqpoint{3.632061in}{4.628983in}}%
\pgfpathlineto{\pgfqpoint{3.634086in}{4.519461in}}%
\pgfpathlineto{\pgfqpoint{3.636111in}{4.657962in}}%
\pgfpathlineto{\pgfqpoint{3.638137in}{4.479700in}}%
\pgfpathlineto{\pgfqpoint{3.640162in}{4.703934in}}%
\pgfpathlineto{\pgfqpoint{3.642187in}{4.432784in}}%
\pgfpathlineto{\pgfqpoint{3.644212in}{4.746249in}}%
\pgfpathlineto{\pgfqpoint{3.646238in}{4.400460in}}%
\pgfpathlineto{\pgfqpoint{3.648263in}{4.763767in}}%
\pgfpathlineto{\pgfqpoint{3.650288in}{4.401614in}}%
\pgfpathlineto{\pgfqpoint{3.652313in}{4.741315in}}%
\pgfpathlineto{\pgfqpoint{3.654339in}{4.446559in}}%
\pgfpathlineto{\pgfqpoint{3.656364in}{4.674190in}}%
\pgfpathlineto{\pgfqpoint{3.658389in}{4.534091in}}%
\pgfpathlineto{\pgfqpoint{3.660415in}{4.569314in}}%
\pgfpathlineto{\pgfqpoint{3.662440in}{4.652225in}}%
\pgfpathlineto{\pgfqpoint{3.664465in}{4.442679in}}%
\pgfpathlineto{\pgfqpoint{3.666490in}{4.782341in}}%
\pgfpathlineto{\pgfqpoint{3.668516in}{4.313945in}}%
\pgfpathlineto{\pgfqpoint{3.670541in}{4.905389in}}%
\pgfpathlineto{\pgfqpoint{3.672566in}{4.199970in}}%
\pgfpathlineto{\pgfqpoint{3.674591in}{5.008085in}}%
\pgfpathlineto{\pgfqpoint{3.676617in}{4.109416in}}%
\pgfpathlineto{\pgfqpoint{3.678642in}{5.087021in}}%
\pgfpathlineto{\pgfqpoint{3.680667in}{4.040274in}}%
\pgfpathlineto{\pgfqpoint{3.682693in}{5.149286in}}%
\pgfpathlineto{\pgfqpoint{3.684718in}{3.981183in}}%
\pgfpathlineto{\pgfqpoint{3.686743in}{5.209307in}}%
\pgfpathlineto{\pgfqpoint{3.688768in}{3.916162in}}%
\pgfpathlineto{\pgfqpoint{3.690794in}{5.282926in}}%
\pgfpathlineto{\pgfqpoint{3.692819in}{3.831228in}}%
\pgfpathlineto{\pgfqpoint{3.694844in}{5.380669in}}%
\pgfpathlineto{\pgfqpoint{3.696869in}{3.720653in}}%
\pgfpathlineto{\pgfqpoint{3.698895in}{5.502508in}}%
\pgfpathlineto{\pgfqpoint{3.700920in}{3.590698in}}%
\pgfpathlineto{\pgfqpoint{3.702945in}{5.635998in}}%
\pgfpathlineto{\pgfqpoint{3.704971in}{3.459407in}}%
\pgfpathlineto{\pgfqpoint{3.706996in}{5.758589in}}%
\pgfpathlineto{\pgfqpoint{3.709021in}{3.352321in}}%
\pgfpathlineto{\pgfqpoint{3.711046in}{5.843571in}}%
\pgfpathlineto{\pgfqpoint{3.713072in}{3.295327in}}%
\pgfpathlineto{\pgfqpoint{3.715097in}{5.867875in}}%
\pgfpathlineto{\pgfqpoint{3.717122in}{3.306837in}}%
\pgfpathlineto{\pgfqpoint{3.719147in}{5.819284in}}%
\pgfpathlineto{\pgfqpoint{3.721173in}{3.391773in}}%
\pgfpathlineto{\pgfqpoint{3.723198in}{5.700739in}}%
\pgfpathlineto{\pgfqpoint{3.725223in}{3.539348in}}%
\pgfpathlineto{\pgfqpoint{3.727249in}{5.530264in}}%
\pgfpathlineto{\pgfqpoint{3.729274in}{3.725457in}}%
\pgfpathlineto{\pgfqpoint{3.731299in}{5.336422in}}%
\pgfpathlineto{\pgfqpoint{3.733324in}{3.919043in}}%
\pgfpathlineto{\pgfqpoint{3.735350in}{5.150613in}}%
\pgfpathlineto{\pgfqpoint{3.737375in}{4.090543in}}%
\pgfpathlineto{\pgfqpoint{3.739400in}{4.998530in}}%
\pgfpathlineto{\pgfqpoint{3.741425in}{4.219852in}}%
\pgfpathlineto{\pgfqpoint{3.743451in}{4.893411in}}%
\pgfpathlineto{\pgfqpoint{3.745476in}{4.301336in}}%
\pgfpathlineto{\pgfqpoint{3.747501in}{4.833160in}}%
\pgfpathlineto{\pgfqpoint{3.749527in}{4.344332in}}%
\pgfpathlineto{\pgfqpoint{3.751552in}{4.802260in}}%
\pgfpathlineto{\pgfqpoint{3.753577in}{4.368979in}}%
\pgfpathlineto{\pgfqpoint{3.755602in}{4.777890in}}%
\pgfpathlineto{\pgfqpoint{3.757628in}{4.398614in}}%
\pgfpathlineto{\pgfqpoint{3.759653in}{4.738418in}}%
\pgfpathlineto{\pgfqpoint{3.761678in}{4.451061in}}%
\pgfpathlineto{\pgfqpoint{3.763703in}{4.671653in}}%
\pgfpathlineto{\pgfqpoint{3.765729in}{4.531464in}}%
\pgfpathlineto{\pgfqpoint{3.769779in}{4.628816in}}%
\pgfpathlineto{\pgfqpoint{3.771805in}{4.483506in}}%
\pgfpathlineto{\pgfqpoint{3.773830in}{4.717267in}}%
\pgfpathlineto{\pgfqpoint{3.775855in}{4.412359in}}%
\pgfpathlineto{\pgfqpoint{3.777880in}{4.761865in}}%
\pgfpathlineto{\pgfqpoint{3.779906in}{4.403330in}}%
\pgfpathlineto{\pgfqpoint{3.781931in}{4.727113in}}%
\pgfpathlineto{\pgfqpoint{3.783956in}{4.488725in}}%
\pgfpathlineto{\pgfqpoint{3.788007in}{4.688379in}}%
\pgfpathlineto{\pgfqpoint{3.790032in}{4.327332in}}%
\pgfpathlineto{\pgfqpoint{3.792057in}{5.004091in}}%
\pgfpathlineto{\pgfqpoint{3.794083in}{3.959083in}}%
\pgfpathlineto{\pgfqpoint{3.796108in}{5.418106in}}%
\pgfpathlineto{\pgfqpoint{3.798133in}{3.508083in}}%
\pgfpathlineto{\pgfqpoint{3.800158in}{5.895668in}}%
\pgfpathlineto{\pgfqpoint{3.802184in}{3.015591in}}%
\pgfpathlineto{\pgfqpoint{3.804209in}{6.390717in}}%
\pgfpathlineto{\pgfqpoint{3.806234in}{2.530626in}}%
\pgfpathlineto{\pgfqpoint{3.808259in}{6.853155in}}%
\pgfpathlineto{\pgfqpoint{3.810285in}{2.102533in}}%
\pgfpathlineto{\pgfqpoint{3.812310in}{7.236084in}}%
\pgfpathlineto{\pgfqpoint{3.814335in}{1.774263in}}%
\pgfpathlineto{\pgfqpoint{3.816361in}{7.501779in}}%
\pgfpathlineto{\pgfqpoint{3.818386in}{1.577288in}}%
\pgfpathlineto{\pgfqpoint{3.820411in}{7.625796in}}%
\pgfpathlineto{\pgfqpoint{3.822436in}{1.528485in}}%
\pgfpathlineto{\pgfqpoint{3.824462in}{7.599140in}}%
\pgfpathlineto{\pgfqpoint{3.826487in}{1.628857in}}%
\pgfpathlineto{\pgfqpoint{3.828512in}{7.428719in}}%
\pgfpathlineto{\pgfqpoint{3.830537in}{1.863841in}}%
\pgfpathlineto{\pgfqpoint{3.832563in}{7.136322in}}%
\pgfpathlineto{\pgfqpoint{3.834588in}{2.205031in}}%
\pgfpathlineto{\pgfqpoint{3.836613in}{6.756190in}}%
\pgfpathlineto{\pgfqpoint{3.838639in}{2.613316in}}%
\pgfpathlineto{\pgfqpoint{3.840664in}{6.331144in}}%
\pgfpathlineto{\pgfqpoint{3.842689in}{3.043498in}}%
\pgfpathlineto{\pgfqpoint{3.844714in}{5.907279in}}%
\pgfpathlineto{\pgfqpoint{3.846740in}{3.450186in}}%
\pgfpathlineto{\pgfqpoint{3.848765in}{5.527623in}}%
\pgfpathlineto{\pgfqpoint{3.850790in}{3.794357in}}%
\pgfpathlineto{\pgfqpoint{3.852815in}{5.225646in}}%
\pgfpathlineto{\pgfqpoint{3.854841in}{4.049450in}}%
\pgfpathlineto{\pgfqpoint{3.856866in}{5.019927in}}%
\pgfpathlineto{\pgfqpoint{3.858891in}{4.205576in}}%
\pgfpathlineto{\pgfqpoint{3.860917in}{4.911390in}}%
\pgfpathlineto{\pgfqpoint{3.862942in}{4.270572in}}%
\pgfpathlineto{\pgfqpoint{3.864967in}{4.884131in}}%
\pgfpathlineto{\pgfqpoint{3.866992in}{4.267252in}}%
\pgfpathlineto{\pgfqpoint{3.869018in}{4.909997in}}%
\pgfpathlineto{\pgfqpoint{3.871043in}{4.227211in}}%
\pgfpathlineto{\pgfqpoint{3.873068in}{4.956052in}}%
\pgfpathlineto{\pgfqpoint{3.875093in}{4.182564in}}%
\pgfpathlineto{\pgfqpoint{3.877119in}{4.993084in}}%
\pgfpathlineto{\pgfqpoint{3.879144in}{4.157762in}}%
\pgfpathlineto{\pgfqpoint{3.881169in}{5.002891in}}%
\pgfpathlineto{\pgfqpoint{3.883195in}{4.163746in}}%
\pgfpathlineto{\pgfqpoint{3.885220in}{4.982259in}}%
\pgfpathlineto{\pgfqpoint{3.887245in}{4.196123in}}%
\pgfpathlineto{\pgfqpoint{3.889270in}{4.942484in}}%
\pgfpathlineto{\pgfqpoint{3.891296in}{4.237928in}}%
\pgfpathlineto{\pgfqpoint{3.893321in}{4.904537in}}%
\pgfpathlineto{\pgfqpoint{3.895346in}{4.266149in}}%
\pgfpathlineto{\pgfqpoint{3.897372in}{4.891350in}}%
\pgfpathlineto{\pgfqpoint{3.899397in}{4.260046in}}%
\pgfpathlineto{\pgfqpoint{3.901422in}{4.919544in}}%
\pgfpathlineto{\pgfqpoint{3.903447in}{4.208707in}}%
\pgfpathlineto{\pgfqpoint{3.905473in}{4.993186in}}%
\pgfpathlineto{\pgfqpoint{3.907498in}{4.115495in}}%
\pgfpathlineto{\pgfqpoint{3.909523in}{5.101500in}}%
\pgfpathlineto{\pgfqpoint{3.911548in}{3.997983in}}%
\pgfpathlineto{\pgfqpoint{3.913574in}{5.221288in}}%
\pgfpathlineto{\pgfqpoint{3.915599in}{3.883354in}}%
\pgfpathlineto{\pgfqpoint{3.917624in}{5.323357in}}%
\pgfpathlineto{\pgfqpoint{3.919650in}{3.800658in}}%
\pgfpathlineto{\pgfqpoint{3.921675in}{5.380973in}}%
\pgfpathlineto{\pgfqpoint{3.923700in}{3.772293in}}%
\pgfpathlineto{\pgfqpoint{3.925725in}{5.377771in}}%
\pgfpathlineto{\pgfqpoint{3.927751in}{3.807341in}}%
\pgfpathlineto{\pgfqpoint{3.929776in}{5.312661in}}%
\pgfpathlineto{\pgfqpoint{3.931801in}{3.898798in}}%
\pgfpathlineto{\pgfqpoint{3.933826in}{5.200230in}}%
\pgfpathlineto{\pgfqpoint{3.935852in}{4.025573in}}%
\pgfpathlineto{\pgfqpoint{3.937877in}{5.066504in}}%
\pgfpathlineto{\pgfqpoint{3.939902in}{4.158639in}}%
\pgfpathlineto{\pgfqpoint{3.941928in}{4.941365in}}%
\pgfpathlineto{\pgfqpoint{3.943953in}{4.269454in}}%
\pgfpathlineto{\pgfqpoint{3.945978in}{4.849946in}}%
\pgfpathlineto{\pgfqpoint{3.948003in}{4.338079in}}%
\pgfpathlineto{\pgfqpoint{3.950029in}{4.805623in}}%
\pgfpathlineto{\pgfqpoint{3.952054in}{4.358548in}}%
\pgfpathlineto{\pgfqpoint{3.954079in}{4.806694in}}%
\pgfpathlineto{\pgfqpoint{3.956104in}{4.339891in}}%
\pgfpathlineto{\pgfqpoint{3.958130in}{4.837703in}}%
\pgfpathlineto{\pgfqpoint{3.960155in}{4.302592in}}%
\pgfpathlineto{\pgfqpoint{3.962180in}{4.874909in}}%
\pgfpathlineto{\pgfqpoint{3.964206in}{4.271655in}}%
\pgfpathlineto{\pgfqpoint{3.966231in}{4.894111in}}%
\pgfpathlineto{\pgfqpoint{3.968256in}{4.268508in}}%
\pgfpathlineto{\pgfqpoint{3.970281in}{4.878368in}}%
\pgfpathlineto{\pgfqpoint{3.972307in}{4.304278in}}%
\pgfpathlineto{\pgfqpoint{3.974332in}{4.823198in}}%
\pgfpathlineto{\pgfqpoint{3.976357in}{4.376531in}}%
\pgfpathlineto{\pgfqpoint{3.978382in}{4.737655in}}%
\pgfpathlineto{\pgfqpoint{3.980408in}{4.470436in}}%
\pgfpathlineto{\pgfqpoint{3.982433in}{4.641028in}}%
\pgfpathlineto{\pgfqpoint{3.984458in}{4.563921in}}%
\pgfpathlineto{\pgfqpoint{3.986484in}{4.556271in}}%
\pgfpathlineto{\pgfqpoint{3.988509in}{4.635120in}}%
\pgfpathlineto{\pgfqpoint{3.990534in}{4.502286in}}%
\pgfpathlineto{\pgfqpoint{3.992559in}{4.669723in}}%
\pgfpathlineto{\pgfqpoint{3.994585in}{4.487546in}}%
\pgfpathlineto{\pgfqpoint{3.996610in}{4.665864in}}%
\pgfpathlineto{\pgfqpoint{3.998635in}{4.507079in}}%
\pgfpathlineto{\pgfqpoint{4.000660in}{4.635023in}}%
\pgfpathlineto{\pgfqpoint{4.002686in}{4.543758in}}%
\pgfpathlineto{\pgfqpoint{4.004711in}{4.598656in}}%
\pgfpathlineto{\pgfqpoint{4.006736in}{4.573462in}}%
\pgfpathlineto{\pgfqpoint{4.008762in}{4.581668in}}%
\pgfpathlineto{\pgfqpoint{4.010787in}{4.572450in}}%
\pgfpathlineto{\pgfqpoint{4.012812in}{4.604789in}}%
\pgfpathlineto{\pgfqpoint{4.014837in}{4.524606in}}%
\pgfpathlineto{\pgfqpoint{4.016863in}{4.678280in}}%
\pgfpathlineto{\pgfqpoint{4.018888in}{4.426297in}}%
\pgfpathlineto{\pgfqpoint{4.020913in}{4.798894in}}%
\pgfpathlineto{\pgfqpoint{4.022938in}{4.287370in}}%
\pgfpathlineto{\pgfqpoint{4.024964in}{4.950974in}}%
\pgfpathlineto{\pgfqpoint{4.026989in}{4.128071in}}%
\pgfpathlineto{\pgfqpoint{4.029014in}{5.111229in}}%
\pgfpathlineto{\pgfqpoint{4.031040in}{3.972993in}}%
\pgfpathlineto{\pgfqpoint{4.033065in}{5.255567in}}%
\pgfpathlineto{\pgfqpoint{4.035090in}{3.844012in}}%
\pgfpathlineto{\pgfqpoint{4.037115in}{5.365809in}}%
\pgfpathlineto{\pgfqpoint{4.039141in}{3.754467in}}%
\pgfpathlineto{\pgfqpoint{4.041166in}{5.434186in}}%
\pgfpathlineto{\pgfqpoint{4.043191in}{3.706294in}}%
\pgfpathlineto{\pgfqpoint{4.045216in}{5.464388in}}%
\pgfpathlineto{\pgfqpoint{4.047242in}{3.690819in}}%
\pgfpathlineto{\pgfqpoint{4.049267in}{5.469064in}}%
\pgfpathlineto{\pgfqpoint{4.051292in}{3.692695in}}%
\pgfpathlineto{\pgfqpoint{4.053318in}{5.464829in}}%
\pgfpathlineto{\pgfqpoint{4.055343in}{3.695498in}}%
\pgfpathlineto{\pgfqpoint{4.057368in}{5.466551in}}%
\pgfpathlineto{\pgfqpoint{4.059393in}{3.687084in}}%
\pgfpathlineto{\pgfqpoint{4.061419in}{5.482762in}}%
\pgfpathlineto{\pgfqpoint{4.063444in}{3.663073in}}%
\pgfpathlineto{\pgfqpoint{4.065469in}{5.513523in}}%
\pgfpathlineto{\pgfqpoint{4.067494in}{3.627534in}}%
\pgfpathlineto{\pgfqpoint{4.069520in}{5.551148in}}%
\pgfpathlineto{\pgfqpoint{4.071545in}{3.590991in}}%
\pgfpathlineto{\pgfqpoint{4.073570in}{5.583237in}}%
\pgfpathlineto{\pgfqpoint{4.075596in}{3.566657in}}%
\pgfpathlineto{\pgfqpoint{4.077621in}{5.596834in}}%
\pgfpathlineto{\pgfqpoint{4.079646in}{3.566252in}}%
\pgfpathlineto{\pgfqpoint{4.081671in}{5.582284in}}%
\pgfpathlineto{\pgfqpoint{4.083697in}{3.596730in}}%
\pgfpathlineto{\pgfqpoint{4.085722in}{5.535730in}}%
\pgfpathlineto{\pgfqpoint{4.087747in}{3.658708in}}%
\pgfpathlineto{\pgfqpoint{4.089772in}{5.459703in}}%
\pgfpathlineto{\pgfqpoint{4.091798in}{3.746806in}}%
\pgfpathlineto{\pgfqpoint{4.093823in}{5.361965in}}%
\pgfpathlineto{\pgfqpoint{4.095848in}{3.851464in}}%
\pgfpathlineto{\pgfqpoint{4.097874in}{5.253224in}}%
\pgfpathlineto{\pgfqpoint{4.099899in}{3.961496in}}%
\pgfpathlineto{\pgfqpoint{4.101924in}{5.144511in}}%
\pgfpathlineto{\pgfqpoint{4.103949in}{4.066578in}}%
\pgfpathlineto{\pgfqpoint{4.105975in}{5.044981in}}%
\pgfpathlineto{\pgfqpoint{4.108000in}{4.159073in}}%
\pgfpathlineto{\pgfqpoint{4.110025in}{4.960541in}}%
\pgfpathlineto{\pgfqpoint{4.112050in}{4.234901in}}%
\pgfpathlineto{\pgfqpoint{4.114076in}{4.893455in}}%
\pgfpathlineto{\pgfqpoint{4.116101in}{4.293499in}}%
\pgfpathlineto{\pgfqpoint{4.118126in}{4.842771in}}%
\pgfpathlineto{\pgfqpoint{4.120152in}{4.337090in}}%
\pgfpathlineto{\pgfqpoint{4.122177in}{4.805277in}}%
\pgfpathlineto{\pgfqpoint{4.124202in}{4.369581in}}%
\pgfpathlineto{\pgfqpoint{4.126227in}{4.776665in}}%
\pgfpathlineto{\pgfqpoint{4.128253in}{4.395401in}}%
\pgfpathlineto{\pgfqpoint{4.130278in}{4.752642in}}%
\pgfpathlineto{\pgfqpoint{4.132303in}{4.418488in}}%
\pgfpathlineto{\pgfqpoint{4.134328in}{4.729800in}}%
\pgfpathlineto{\pgfqpoint{4.136354in}{4.441581in}}%
\pgfpathlineto{\pgfqpoint{4.138379in}{4.706163in}}%
\pgfpathlineto{\pgfqpoint{4.140404in}{4.465849in}}%
\pgfpathlineto{\pgfqpoint{4.142430in}{4.681373in}}%
\pgfpathlineto{\pgfqpoint{4.144455in}{4.490871in}}%
\pgfpathlineto{\pgfqpoint{4.146480in}{4.656564in}}%
\pgfpathlineto{\pgfqpoint{4.148505in}{4.514897in}}%
\pgfpathlineto{\pgfqpoint{4.150531in}{4.633983in}}%
\pgfpathlineto{\pgfqpoint{4.152556in}{4.535315in}}%
\pgfpathlineto{\pgfqpoint{4.154581in}{4.616468in}}%
\pgfpathlineto{\pgfqpoint{4.156606in}{4.549203in}}%
\pgfpathlineto{\pgfqpoint{4.158632in}{4.606884in}}%
\pgfpathlineto{\pgfqpoint{4.160657in}{4.553879in}}%
\pgfpathlineto{\pgfqpoint{4.162682in}{4.607624in}}%
\pgfpathlineto{\pgfqpoint{4.164708in}{4.547330in}}%
\pgfpathlineto{\pgfqpoint{4.166733in}{4.620250in}}%
\pgfpathlineto{\pgfqpoint{4.168758in}{4.528492in}}%
\pgfpathlineto{\pgfqpoint{4.170783in}{4.645304in}}%
\pgfpathlineto{\pgfqpoint{4.172809in}{4.497349in}}%
\pgfpathlineto{\pgfqpoint{4.174834in}{4.682284in}}%
\pgfpathlineto{\pgfqpoint{4.176859in}{4.454899in}}%
\pgfpathlineto{\pgfqpoint{4.178884in}{4.729731in}}%
\pgfpathlineto{\pgfqpoint{4.180910in}{4.403018in}}%
\pgfpathlineto{\pgfqpoint{4.182935in}{4.785400in}}%
\pgfpathlineto{\pgfqpoint{4.184960in}{4.344274in}}%
\pgfpathlineto{\pgfqpoint{4.186986in}{4.846454in}}%
\pgfpathlineto{\pgfqpoint{4.189011in}{4.281717in}}%
\pgfpathlineto{\pgfqpoint{4.191036in}{4.909686in}}%
\pgfpathlineto{\pgfqpoint{4.193061in}{4.218644in}}%
\pgfpathlineto{\pgfqpoint{4.195087in}{4.971780in}}%
\pgfpathlineto{\pgfqpoint{4.197112in}{4.158313in}}%
\pgfpathlineto{\pgfqpoint{4.199137in}{5.029622in}}%
\pgfpathlineto{\pgfqpoint{4.201162in}{4.103603in}}%
\pgfpathlineto{\pgfqpoint{4.203188in}{5.080663in}}%
\pgfpathlineto{\pgfqpoint{4.205213in}{4.056637in}}%
\pgfpathlineto{\pgfqpoint{4.207238in}{5.123298in}}%
\pgfpathlineto{\pgfqpoint{4.209264in}{4.018423in}}%
\pgfpathlineto{\pgfqpoint{4.211289in}{5.157178in}}%
\pgfpathlineto{\pgfqpoint{4.213314in}{3.988609in}}%
\pgfpathlineto{\pgfqpoint{4.215339in}{5.183371in}}%
\pgfpathlineto{\pgfqpoint{4.217365in}{3.965430in}}%
\pgfpathlineto{\pgfqpoint{4.219390in}{5.204282in}}%
\pgfpathlineto{\pgfqpoint{4.221415in}{3.945929in}}%
\pgfpathlineto{\pgfqpoint{4.223440in}{5.223308in}}%
\pgfpathlineto{\pgfqpoint{4.225466in}{3.926412in}}%
\pgfpathlineto{\pgfqpoint{4.227491in}{5.244270in}}%
\pgfpathlineto{\pgfqpoint{4.229516in}{3.903106in}}%
\pgfpathlineto{\pgfqpoint{4.231542in}{5.270723in}}%
\pgfpathlineto{\pgfqpoint{4.233567in}{3.872840in}}%
\pgfpathlineto{\pgfqpoint{4.235592in}{5.305298in}}%
\pgfpathlineto{\pgfqpoint{4.237617in}{3.833650in}}%
\pgfpathlineto{\pgfqpoint{4.239643in}{5.349204in}}%
\pgfpathlineto{\pgfqpoint{4.241668in}{3.785141in}}%
\pgfpathlineto{\pgfqpoint{4.243693in}{5.401992in}}%
\pgfpathlineto{\pgfqpoint{4.245718in}{3.728594in}}%
\pgfpathlineto{\pgfqpoint{4.247744in}{5.461597in}}%
\pgfpathlineto{\pgfqpoint{4.249769in}{3.666794in}}%
\pgfpathlineto{\pgfqpoint{4.251794in}{5.524595in}}%
\pgfpathlineto{\pgfqpoint{4.253820in}{3.603698in}}%
\pgfpathlineto{\pgfqpoint{4.255845in}{5.586609in}}%
\pgfpathlineto{\pgfqpoint{4.257870in}{3.543998in}}%
\pgfpathlineto{\pgfqpoint{4.259895in}{5.642736in}}%
\pgfpathlineto{\pgfqpoint{4.261921in}{3.492704in}}%
\pgfpathlineto{\pgfqpoint{4.263946in}{5.687959in}}%
\pgfpathlineto{\pgfqpoint{4.265971in}{3.454749in}}%
\pgfpathlineto{\pgfqpoint{4.267996in}{5.717515in}}%
\pgfpathlineto{\pgfqpoint{4.270022in}{3.434639in}}%
\pgfpathlineto{\pgfqpoint{4.272047in}{5.727243in}}%
\pgfpathlineto{\pgfqpoint{4.274072in}{3.436097in}}%
\pgfpathlineto{\pgfqpoint{4.276098in}{5.713951in}}%
\pgfpathlineto{\pgfqpoint{4.278123in}{3.461691in}}%
\pgfpathlineto{\pgfqpoint{4.280148in}{5.675792in}}%
\pgfpathlineto{\pgfqpoint{4.282173in}{3.512455in}}%
\pgfpathlineto{\pgfqpoint{4.284199in}{5.612623in}}%
\pgfpathlineto{\pgfqpoint{4.286224in}{3.587577in}}%
\pgfpathlineto{\pgfqpoint{4.288249in}{5.526253in}}%
\pgfpathlineto{\pgfqpoint{4.290274in}{3.684237in}}%
\pgfpathlineto{\pgfqpoint{4.292300in}{5.420495in}}%
\pgfpathlineto{\pgfqpoint{4.294325in}{3.797688in}}%
\pgfpathlineto{\pgfqpoint{4.296350in}{5.300937in}}%
\pgfpathlineto{\pgfqpoint{4.298376in}{3.921624in}}%
\pgfpathlineto{\pgfqpoint{4.300401in}{5.174449in}}%
\pgfpathlineto{\pgfqpoint{4.302426in}{4.048793in}}%
\pgfpathlineto{\pgfqpoint{4.304451in}{5.048468in}}%
\pgfpathlineto{\pgfqpoint{4.306477in}{4.171765in}}%
\pgfpathlineto{\pgfqpoint{4.308502in}{4.930230in}}%
\pgfpathlineto{\pgfqpoint{4.310527in}{4.283677in}}%
\pgfpathlineto{\pgfqpoint{4.312552in}{4.826075in}}%
\pgfpathlineto{\pgfqpoint{4.314578in}{4.378826in}}%
\pgfpathlineto{\pgfqpoint{4.316603in}{4.740987in}}%
\pgfpathlineto{\pgfqpoint{4.318628in}{4.452995in}}%
\pgfpathlineto{\pgfqpoint{4.320654in}{4.678406in}}%
\pgfpathlineto{\pgfqpoint{4.322679in}{4.503494in}}%
\pgfpathlineto{\pgfqpoint{4.324704in}{4.640322in}}%
\pgfpathlineto{\pgfqpoint{4.326729in}{4.528968in}}%
\pgfpathlineto{\pgfqpoint{4.328755in}{4.627538in}}%
\pgfpathlineto{\pgfqpoint{4.330780in}{4.529081in}}%
\pgfpathlineto{\pgfqpoint{4.332805in}{4.639999in}}%
\pgfpathlineto{\pgfqpoint{4.334830in}{4.504208in}}%
\pgfpathlineto{\pgfqpoint{4.336856in}{4.677069in}}%
\pgfpathlineto{\pgfqpoint{4.338881in}{4.455203in}}%
\pgfpathlineto{\pgfqpoint{4.340906in}{4.737704in}}%
\pgfpathlineto{\pgfqpoint{4.342932in}{4.383283in}}%
\pgfpathlineto{\pgfqpoint{4.344957in}{4.820521in}}%
\pgfpathlineto{\pgfqpoint{4.346982in}{4.290000in}}%
\pgfpathlineto{\pgfqpoint{4.349007in}{4.923800in}}%
\pgfpathlineto{\pgfqpoint{4.351033in}{4.177236in}}%
\pgfpathlineto{\pgfqpoint{4.353058in}{5.045502in}}%
\pgfpathlineto{\pgfqpoint{4.355083in}{4.047169in}}%
\pgfpathlineto{\pgfqpoint{4.357108in}{5.183339in}}%
\pgfpathlineto{\pgfqpoint{4.359134in}{3.902166in}}%
\pgfpathlineto{\pgfqpoint{4.361159in}{5.334908in}}%
\pgfpathlineto{\pgfqpoint{4.363184in}{3.744619in}}%
\pgfpathlineto{\pgfqpoint{4.365210in}{5.497871in}}%
\pgfpathlineto{\pgfqpoint{4.367235in}{3.576765in}}%
\pgfpathlineto{\pgfqpoint{4.369260in}{5.670131in}}%
\pgfpathlineto{\pgfqpoint{4.371285in}{3.400541in}}%
\pgfpathlineto{\pgfqpoint{4.373311in}{5.849921in}}%
\pgfpathlineto{\pgfqpoint{4.375336in}{3.217544in}}%
\pgfpathlineto{\pgfqpoint{4.377361in}{6.035793in}}%
\pgfpathlineto{\pgfqpoint{4.379386in}{3.029115in}}%
\pgfpathlineto{\pgfqpoint{4.381412in}{6.226456in}}%
\pgfpathlineto{\pgfqpoint{4.383437in}{2.836568in}}%
\pgfpathlineto{\pgfqpoint{4.385462in}{6.420485in}}%
\pgfpathlineto{\pgfqpoint{4.387488in}{2.641539in}}%
\pgfpathlineto{\pgfqpoint{4.389513in}{6.615930in}}%
\pgfpathlineto{\pgfqpoint{4.391538in}{2.446393in}}%
\pgfpathlineto{\pgfqpoint{4.393563in}{6.809907in}}%
\pgfpathlineto{\pgfqpoint{4.395589in}{2.254623in}}%
\pgfpathlineto{\pgfqpoint{4.397614in}{6.998249in}}%
\pgfpathlineto{\pgfqpoint{4.399639in}{2.071118in}}%
\pgfpathlineto{\pgfqpoint{4.401664in}{7.175327in}}%
\pgfpathlineto{\pgfqpoint{4.403690in}{1.902221in}}%
\pgfpathlineto{\pgfqpoint{4.405715in}{7.334151in}}%
\pgfpathlineto{\pgfqpoint{4.407740in}{1.755463in}}%
\pgfpathlineto{\pgfqpoint{4.409766in}{7.466793in}}%
\pgfpathlineto{\pgfqpoint{4.411791in}{1.638984in}}%
\pgfpathlineto{\pgfqpoint{4.413816in}{7.565127in}}%
\pgfpathlineto{\pgfqpoint{4.415841in}{1.560649in}}%
\pgfpathlineto{\pgfqpoint{4.417867in}{7.621804in}}%
\pgfpathlineto{\pgfqpoint{4.419892in}{1.527029in}}%
\pgfpathlineto{\pgfqpoint{4.421917in}{7.631288in}}%
\pgfpathlineto{\pgfqpoint{4.423942in}{1.542395in}}%
\pgfpathlineto{\pgfqpoint{4.425968in}{7.590766in}}%
\pgfpathlineto{\pgfqpoint{4.427993in}{1.607948in}}%
\pgfpathlineto{\pgfqpoint{4.430018in}{7.500747in}}%
\pgfpathlineto{\pgfqpoint{4.432044in}{1.721426in}}%
\pgfpathlineto{\pgfqpoint{4.434069in}{7.365234in}}%
\pgfpathlineto{\pgfqpoint{4.436094in}{1.877164in}}%
\pgfpathlineto{\pgfqpoint{4.438119in}{7.191426in}}%
\pgfpathlineto{\pgfqpoint{4.440145in}{2.066597in}}%
\pgfpathlineto{\pgfqpoint{4.442170in}{6.989039in}}%
\pgfpathlineto{\pgfqpoint{4.444195in}{2.279105in}}%
\pgfpathlineto{\pgfqpoint{4.446220in}{6.769337in}}%
\pgfpathlineto{\pgfqpoint{4.448246in}{2.503052in}}%
\pgfpathlineto{\pgfqpoint{4.450271in}{6.544051in}}%
\pgfpathlineto{\pgfqpoint{4.452296in}{2.726883in}}%
\pgfpathlineto{\pgfqpoint{4.454322in}{6.324300in}}%
\pgfpathlineto{\pgfqpoint{4.456347in}{2.940154in}}%
\pgfpathlineto{\pgfqpoint{4.458372in}{6.119634in}}%
\pgfpathlineto{\pgfqpoint{4.460397in}{3.134407in}}%
\pgfpathlineto{\pgfqpoint{4.462423in}{5.937252in}}%
\pgfpathlineto{\pgfqpoint{4.464448in}{3.303836in}}%
\pgfpathlineto{\pgfqpoint{4.466473in}{5.781462in}}%
\pgfpathlineto{\pgfqpoint{4.468498in}{3.445702in}}%
\pgfpathlineto{\pgfqpoint{4.470524in}{5.653408in}}%
\pgfpathlineto{\pgfqpoint{4.472549in}{3.560441in}}%
\pgfpathlineto{\pgfqpoint{4.474574in}{5.551130in}}%
\pgfpathlineto{\pgfqpoint{4.476600in}{3.651434in}}%
\pgfpathlineto{\pgfqpoint{4.478625in}{5.469974in}}%
\pgfpathlineto{\pgfqpoint{4.480650in}{3.724413in}}%
\pgfpathlineto{\pgfqpoint{4.482675in}{5.403364in}}%
\pgfpathlineto{\pgfqpoint{4.484701in}{3.786534in}}%
\pgfpathlineto{\pgfqpoint{4.486726in}{5.343855in}}%
\pgfpathlineto{\pgfqpoint{4.488751in}{3.845226in}}%
\pgfpathlineto{\pgfqpoint{4.490776in}{5.284337in}}%
\pgfpathlineto{\pgfqpoint{4.492802in}{3.906998in}}%
\pgfpathlineto{\pgfqpoint{4.494827in}{5.219153in}}%
\pgfpathlineto{\pgfqpoint{4.496852in}{3.976443in}}%
\pgfpathlineto{\pgfqpoint{4.498878in}{5.144924in}}%
\pgfpathlineto{\pgfqpoint{4.500903in}{4.055652in}}%
\pgfpathlineto{\pgfqpoint{4.502928in}{5.060849in}}%
\pgfpathlineto{\pgfqpoint{4.504953in}{4.144208in}}%
\pgfpathlineto{\pgfqpoint{4.506979in}{4.968415in}}%
\pgfpathlineto{\pgfqpoint{4.509004in}{4.239768in}}%
\pgfpathlineto{\pgfqpoint{4.511029in}{4.870554in}}%
\pgfpathlineto{\pgfqpoint{4.513054in}{4.339112in}}%
\pgfpathlineto{\pgfqpoint{4.515080in}{4.770455in}}%
\pgfpathlineto{\pgfqpoint{4.517105in}{4.439398in}}%
\pgfpathlineto{\pgfqpoint{4.519130in}{4.670327in}}%
\pgfpathlineto{\pgfqpoint{4.521156in}{4.539293in}}%
\pgfpathlineto{\pgfqpoint{4.523181in}{4.570445in}}%
\pgfpathlineto{\pgfqpoint{4.525206in}{4.639685in}}%
\pgfpathlineto{\pgfqpoint{4.527231in}{4.468739in}}%
\pgfpathlineto{\pgfqpoint{4.529257in}{4.743754in}}%
\pgfpathlineto{\pgfqpoint{4.531282in}{4.361069in}}%
\pgfpathlineto{\pgfqpoint{4.533307in}{4.856373in}}%
\pgfpathlineto{\pgfqpoint{4.535332in}{4.242126in}}%
\pgfpathlineto{\pgfqpoint{4.537358in}{4.982947in}}%
\pgfpathlineto{\pgfqpoint{4.539383in}{4.106777in}}%
\pgfpathlineto{\pgfqpoint{4.541408in}{5.127965in}}%
\pgfpathlineto{\pgfqpoint{4.543434in}{3.951525in}}%
\pgfpathlineto{\pgfqpoint{4.545459in}{5.293626in}}%
\pgfpathlineto{\pgfqpoint{4.547484in}{3.775713in}}%
\pgfpathlineto{\pgfqpoint{4.549509in}{5.478880in}}%
\pgfpathlineto{\pgfqpoint{4.551535in}{3.582171in}}%
\pgfpathlineto{\pgfqpoint{4.553560in}{5.679147in}}%
\pgfpathlineto{\pgfqpoint{4.555585in}{3.377097in}}%
\pgfpathlineto{\pgfqpoint{4.557610in}{5.886833in}}%
\pgfpathlineto{\pgfqpoint{4.559636in}{3.169178in}}%
\pgfpathlineto{\pgfqpoint{4.561661in}{6.092525in}}%
\pgfpathlineto{\pgfqpoint{4.563686in}{2.968146in}}%
\pgfpathlineto{\pgfqpoint{4.565712in}{6.286595in}}%
\pgfpathlineto{\pgfqpoint{4.567737in}{2.783118in}}%
\pgfpathlineto{\pgfqpoint{4.569762in}{6.460797in}}%
\pgfpathlineto{\pgfqpoint{4.571787in}{2.621172in}}%
\pgfpathlineto{\pgfqpoint{4.573813in}{6.609436in}}%
\pgfpathlineto{\pgfqpoint{4.575838in}{2.486513in}}%
\pgfpathlineto{\pgfqpoint{4.577863in}{6.729794in}}%
\pgfpathlineto{\pgfqpoint{4.579888in}{2.380474in}}%
\pgfpathlineto{\pgfqpoint{4.581914in}{6.821727in}}%
\pgfpathlineto{\pgfqpoint{4.583939in}{2.302288in}}%
\pgfpathlineto{\pgfqpoint{4.585964in}{6.886579in}}%
\pgfpathlineto{\pgfqpoint{4.587990in}{2.250388in}}%
\pgfpathlineto{\pgfqpoint{4.590015in}{6.925798in}}%
\pgfpathlineto{\pgfqpoint{4.592040in}{2.223754in}}%
\pgfpathlineto{\pgfqpoint{4.594065in}{6.939730in}}%
\pgfpathlineto{\pgfqpoint{4.596091in}{2.222869in}}%
\pgfpathlineto{\pgfqpoint{4.598116in}{6.927014in}}%
\pgfpathlineto{\pgfqpoint{4.600141in}{2.249910in}}%
\pgfpathlineto{\pgfqpoint{4.602166in}{6.884820in}}%
\pgfpathlineto{\pgfqpoint{4.604192in}{2.308095in}}%
\pgfpathlineto{\pgfqpoint{4.606217in}{6.809891in}}%
\pgfpathlineto{\pgfqpoint{4.608242in}{2.400332in}}%
\pgfpathlineto{\pgfqpoint{4.610268in}{6.700078in}}%
\pgfpathlineto{\pgfqpoint{4.612293in}{2.527606in}}%
\pgfpathlineto{\pgfqpoint{4.614318in}{6.555913in}}%
\pgfpathlineto{\pgfqpoint{4.616343in}{2.687589in}}%
\pgfpathlineto{\pgfqpoint{4.618369in}{6.381703in}}%
\pgfpathlineto{\pgfqpoint{4.620394in}{2.873930in}}%
\pgfpathlineto{\pgfqpoint{4.622419in}{6.185784in}}%
\pgfpathlineto{\pgfqpoint{4.624444in}{3.076490in}}%
\pgfpathlineto{\pgfqpoint{4.626470in}{5.979806in}}%
\pgfpathlineto{\pgfqpoint{4.628495in}{3.282499in}}%
\pgfpathlineto{\pgfqpoint{4.630520in}{5.777190in}}%
\pgfpathlineto{\pgfqpoint{4.632546in}{3.478394in}}%
\pgfpathlineto{\pgfqpoint{4.634571in}{5.591117in}}%
\pgfpathlineto{\pgfqpoint{4.636596in}{3.651887in}}%
\pgfpathlineto{\pgfqpoint{4.638621in}{5.432518in}}%
\pgfpathlineto{\pgfqpoint{4.640647in}{3.793804in}}%
\pgfpathlineto{\pgfqpoint{4.642672in}{5.308495in}}%
\pgfpathlineto{\pgfqpoint{4.644697in}{3.899322in}}%
\pgfpathlineto{\pgfqpoint{4.646722in}{5.221488in}}%
\pgfpathlineto{\pgfqpoint{4.648748in}{3.968388in}}%
\pgfpathlineto{\pgfqpoint{4.650773in}{5.169268in}}%
\pgfpathlineto{\pgfqpoint{4.652798in}{4.005319in}}%
\pgfpathlineto{\pgfqpoint{4.654824in}{5.145691in}}%
\pgfpathlineto{\pgfqpoint{4.656849in}{4.017761in}}%
\pgfpathlineto{\pgfqpoint{4.658874in}{5.141981in}}%
\pgfpathlineto{\pgfqpoint{4.660899in}{4.015228in}}%
\pgfpathlineto{\pgfqpoint{4.662925in}{5.148280in}}%
\pgfpathlineto{\pgfqpoint{4.664950in}{4.007530in}}%
\pgfpathlineto{\pgfqpoint{4.666975in}{5.155199in}}%
\pgfpathlineto{\pgfqpoint{4.669000in}{4.003300in}}%
\pgfpathlineto{\pgfqpoint{4.671026in}{5.155160in}}%
\pgfpathlineto{\pgfqpoint{4.673051in}{4.008811in}}%
\pgfpathlineto{\pgfqpoint{4.675076in}{5.143386in}}%
\pgfpathlineto{\pgfqpoint{4.677102in}{4.027211in}}%
\pgfpathlineto{\pgfqpoint{4.679127in}{5.118429in}}%
\pgfpathlineto{\pgfqpoint{4.681152in}{4.058241in}}%
\pgfpathlineto{\pgfqpoint{4.683177in}{5.082190in}}%
\pgfpathlineto{\pgfqpoint{4.685203in}{4.098491in}}%
\pgfpathlineto{\pgfqpoint{4.687228in}{5.039396in}}%
\pgfpathlineto{\pgfqpoint{4.689253in}{4.142171in}}%
\pgfpathlineto{\pgfqpoint{4.691278in}{4.996594in}}%
\pgfpathlineto{\pgfqpoint{4.693304in}{4.182323in}}%
\pgfpathlineto{\pgfqpoint{4.695329in}{4.960776in}}%
\pgfpathlineto{\pgfqpoint{4.697354in}{4.212307in}}%
\pgfpathlineto{\pgfqpoint{4.699380in}{4.937852in}}%
\pgfpathlineto{\pgfqpoint{4.701405in}{4.227297in}}%
\pgfpathlineto{\pgfqpoint{4.703430in}{4.931261in}}%
\pgfpathlineto{\pgfqpoint{4.705455in}{4.225473in}}%
\pgfpathlineto{\pgfqpoint{4.707481in}{4.941055in}}%
\pgfpathlineto{\pgfqpoint{4.709506in}{4.208597in}}%
\pgfpathlineto{\pgfqpoint{4.711531in}{4.963724in}}%
\pgfpathlineto{\pgfqpoint{4.713556in}{4.181749in}}%
\pgfpathlineto{\pgfqpoint{4.715582in}{4.992912in}}%
\pgfpathlineto{\pgfqpoint{4.717607in}{4.152167in}}%
\pgfpathlineto{\pgfqpoint{4.719632in}{5.020968in}}%
\pgfpathlineto{\pgfqpoint{4.721658in}{4.127388in}}%
\pgfpathlineto{\pgfqpoint{4.723683in}{5.041023in}}%
\pgfpathlineto{\pgfqpoint{4.725708in}{4.113077in}}%
\pgfpathlineto{\pgfqpoint{4.727733in}{5.049105in}}%
\pgfpathlineto{\pgfqpoint{4.729759in}{4.111102in}}%
\pgfpathlineto{\pgfqpoint{4.731784in}{5.045738in}}%
\pgfpathlineto{\pgfqpoint{4.733809in}{4.118403in}}%
\pgfpathlineto{\pgfqpoint{4.735834in}{5.036512in}}%
\pgfpathlineto{\pgfqpoint{4.737860in}{4.127030in}}%
\pgfpathlineto{\pgfqpoint{4.739885in}{5.031405in}}%
\pgfpathlineto{\pgfqpoint{4.741910in}{4.125450in}}%
\pgfpathlineto{\pgfqpoint{4.743936in}{5.042916in}}%
\pgfpathlineto{\pgfqpoint{4.745961in}{4.100860in}}%
\pgfpathlineto{\pgfqpoint{4.747986in}{5.083455in}}%
\pgfpathlineto{\pgfqpoint{4.750011in}{4.041944in}}%
\pgfpathlineto{\pgfqpoint{4.752037in}{5.162599in}}%
\pgfpathlineto{\pgfqpoint{4.754062in}{3.941412in}}%
\pgfpathlineto{\pgfqpoint{4.756087in}{5.284917in}}%
\pgfpathlineto{\pgfqpoint{4.758112in}{3.797702in}}%
\pgfpathlineto{\pgfqpoint{4.760138in}{5.448840in}}%
\pgfpathlineto{\pgfqpoint{4.762163in}{3.615483in}}%
\pgfpathlineto{\pgfqpoint{4.764188in}{5.646777in}}%
\pgfpathlineto{\pgfqpoint{4.766214in}{3.404964in}}%
\pgfpathlineto{\pgfqpoint{4.768239in}{5.866304in}}%
\pgfpathlineto{\pgfqpoint{4.770264in}{3.180316in}}%
\pgfpathlineto{\pgfqpoint{4.772289in}{6.091997in}}%
\pgfpathlineto{\pgfqpoint{4.774315in}{2.957725in}}%
\pgfpathlineto{\pgfqpoint{4.776340in}{6.307368in}}%
\pgfpathlineto{\pgfqpoint{4.778365in}{2.753581in}}%
\pgfpathlineto{\pgfqpoint{4.780391in}{6.496451in}}%
\pgfpathlineto{\pgfqpoint{4.782416in}{2.583176in}}%
\pgfpathlineto{\pgfqpoint{4.784441in}{6.644807in}}%
\pgfpathlineto{\pgfqpoint{4.786466in}{2.459979in}}%
\pgfpathlineto{\pgfqpoint{4.788492in}{6.740004in}}%
\pgfpathlineto{\pgfqpoint{4.790517in}{2.395340in}}%
\pgfpathlineto{\pgfqpoint{4.792542in}{6.771823in}}%
\pgfpathlineto{\pgfqpoint{4.794567in}{2.398282in}}%
\pgfpathlineto{\pgfqpoint{4.796593in}{6.732538in}}%
\pgfpathlineto{\pgfqpoint{4.798618in}{2.475083in}}%
\pgfpathlineto{\pgfqpoint{4.800643in}{6.617520in}}%
\pgfpathlineto{\pgfqpoint{4.802669in}{2.628479in}}%
\pgfpathlineto{\pgfqpoint{4.804694in}{6.426200in}}%
\pgfpathlineto{\pgfqpoint{4.806719in}{2.856586in}}%
\pgfpathlineto{\pgfqpoint{4.808744in}{6.163178in}}%
\pgfpathlineto{\pgfqpoint{4.810770in}{3.151881in}}%
\pgfpathlineto{\pgfqpoint{4.812795in}{5.839031in}}%
\pgfpathlineto{\pgfqpoint{4.814820in}{3.500713in}}%
\pgfpathlineto{\pgfqpoint{4.816845in}{5.470363in}}%
\pgfpathlineto{\pgfqpoint{4.818871in}{3.883790in}}%
\pgfpathlineto{\pgfqpoint{4.820896in}{5.078736in}}%
\pgfpathlineto{\pgfqpoint{4.822921in}{4.277848in}}%
\pgfpathlineto{\pgfqpoint{4.824947in}{4.688431in}}%
\pgfpathlineto{\pgfqpoint{4.826972in}{4.658364in}}%
\pgfpathlineto{\pgfqpoint{4.828997in}{4.323380in}}%
\pgfpathlineto{\pgfqpoint{4.831022in}{5.002829in}}%
\pgfpathlineto{\pgfqpoint{4.833048in}{4.003896in}}%
\pgfpathlineto{\pgfqpoint{4.835073in}{5.293803in}}%
\pgfpathlineto{\pgfqpoint{4.837098in}{3.744009in}}%
\pgfpathlineto{\pgfqpoint{4.839123in}{5.521011in}}%
\pgfpathlineto{\pgfqpoint{4.841149in}{3.550101in}}%
\pgfpathlineto{\pgfqpoint{4.843174in}{5.681890in}}%
\pgfpathlineto{\pgfqpoint{4.845199in}{3.421207in}}%
\pgfpathlineto{\pgfqpoint{4.847225in}{5.780457in}}%
\pgfpathlineto{\pgfqpoint{4.849250in}{3.350887in}}%
\pgfpathlineto{\pgfqpoint{4.851275in}{5.824836in}}%
\pgfpathlineto{\pgfqpoint{4.853300in}{3.330116in}}%
\pgfpathlineto{\pgfqpoint{4.855326in}{5.824180in}}%
\pgfpathlineto{\pgfqpoint{4.857351in}{3.350323in}}%
\pgfpathlineto{\pgfqpoint{4.859376in}{5.785882in}}%
\pgfpathlineto{\pgfqpoint{4.861401in}{3.405727in}}%
\pgfpathlineto{\pgfqpoint{4.863427in}{5.713866in}}%
\pgfpathlineto{\pgfqpoint{4.865452in}{3.494311in}}%
\pgfpathlineto{\pgfqpoint{4.867477in}{5.608386in}}%
\pgfpathlineto{\pgfqpoint{4.869503in}{3.617271in}}%
\pgfpathlineto{\pgfqpoint{4.871528in}{5.467253in}}%
\pgfpathlineto{\pgfqpoint{4.873553in}{3.777229in}}%
\pgfpathlineto{\pgfqpoint{4.875578in}{5.288009in}}%
\pgfpathlineto{\pgfqpoint{4.877604in}{3.975893in}}%
\pgfpathlineto{\pgfqpoint{4.879629in}{5.070229in}}%
\pgfpathlineto{\pgfqpoint{4.881654in}{4.211973in}}%
\pgfpathlineto{\pgfqpoint{4.883679in}{4.817211in}}%
\pgfpathlineto{\pgfqpoint{4.885705in}{4.480029in}}%
\pgfpathlineto{\pgfqpoint{4.887730in}{4.536504in}}%
\pgfpathlineto{\pgfqpoint{4.889755in}{4.770593in}}%
\pgfpathlineto{\pgfqpoint{4.891781in}{4.239175in}}%
\pgfpathlineto{\pgfqpoint{4.893806in}{5.071423in}}%
\pgfpathlineto{\pgfqpoint{4.895831in}{3.938151in}}%
\pgfpathlineto{\pgfqpoint{4.897856in}{5.369416in}}%
\pgfpathlineto{\pgfqpoint{4.899882in}{3.646233in}}%
\pgfpathlineto{\pgfqpoint{4.901907in}{5.652473in}}%
\pgfpathlineto{\pgfqpoint{4.903932in}{3.374516in}}%
\pgfpathlineto{\pgfqpoint{4.905957in}{5.910684in}}%
\pgfpathlineto{\pgfqpoint{4.907983in}{3.131690in}}%
\pgfpathlineto{\pgfqpoint{4.910008in}{6.136482in}}%
\pgfpathlineto{\pgfqpoint{4.912033in}{2.924405in}}%
\pgfpathlineto{\pgfqpoint{4.914059in}{6.323837in}}%
\pgfpathlineto{\pgfqpoint{4.916084in}{2.758417in}}%
\pgfpathlineto{\pgfqpoint{4.918109in}{6.466925in}}%
\pgfpathlineto{\pgfqpoint{4.920134in}{2.639916in}}%
\pgfpathlineto{\pgfqpoint{4.922160in}{6.558976in}}%
\pgfpathlineto{\pgfqpoint{4.924185in}{2.576338in}}%
\pgfpathlineto{\pgfqpoint{4.926210in}{6.591952in}}%
\pgfpathlineto{\pgfqpoint{4.928235in}{2.576099in}}%
\pgfpathlineto{\pgfqpoint{4.930261in}{6.557441in}}%
\pgfpathlineto{\pgfqpoint{4.932286in}{2.647088in}}%
\pgfpathlineto{\pgfqpoint{4.934311in}{6.448707in}}%
\pgfpathlineto{\pgfqpoint{4.936337in}{2.794200in}}%
\pgfpathlineto{\pgfqpoint{4.938362in}{6.263379in}}%
\pgfpathlineto{\pgfqpoint{4.940387in}{3.016651in}}%
\pgfpathlineto{\pgfqpoint{4.942412in}{6.005920in}}%
\pgfpathlineto{\pgfqpoint{4.944438in}{3.305949in}}%
\pgfpathlineto{\pgfqpoint{4.946463in}{5.688986in}}%
\pgfpathlineto{\pgfqpoint{4.948488in}{3.645369in}}%
\pgfpathlineto{\pgfqpoint{4.950513in}{5.333027in}}%
\pgfpathlineto{\pgfqpoint{4.952539in}{4.011326in}}%
\pgfpathlineto{\pgfqpoint{4.954564in}{4.963957in}}%
\pgfpathlineto{\pgfqpoint{4.956589in}{4.376555in}}%
\pgfpathlineto{\pgfqpoint{4.960640in}{4.714410in}}%
\pgfpathlineto{\pgfqpoint{4.962665in}{4.293706in}}%
\pgfpathlineto{\pgfqpoint{4.964690in}{5.003276in}}%
\pgfpathlineto{\pgfqpoint{4.966716in}{4.034889in}}%
\pgfpathlineto{\pgfqpoint{4.968741in}{5.229980in}}%
\pgfpathlineto{\pgfqpoint{4.970766in}{3.841074in}}%
\pgfpathlineto{\pgfqpoint{4.972791in}{5.391378in}}%
\pgfpathlineto{\pgfqpoint{4.974817in}{3.710482in}}%
\pgfpathlineto{\pgfqpoint{4.976842in}{5.493753in}}%
\pgfpathlineto{\pgfqpoint{4.978867in}{3.632968in}}%
\pgfpathlineto{\pgfqpoint{4.980893in}{5.550284in}}%
\pgfpathlineto{\pgfqpoint{4.982918in}{3.593278in}}%
\pgfpathlineto{\pgfqpoint{4.984943in}{5.577283in}}%
\pgfpathlineto{\pgfqpoint{4.986968in}{3.575047in}}%
\pgfpathlineto{\pgfqpoint{4.988994in}{5.590231in}}%
\pgfpathlineto{\pgfqpoint{4.991019in}{3.564501in}}%
\pgfpathlineto{\pgfqpoint{4.993044in}{5.600540in}}%
\pgfpathlineto{\pgfqpoint{4.995069in}{3.553040in}}%
\pgfpathlineto{\pgfqpoint{4.997095in}{5.613759in}}%
\pgfpathlineto{\pgfqpoint{4.999120in}{3.538197in}}%
\pgfpathlineto{\pgfqpoint{5.001145in}{5.629436in}}%
\pgfpathlineto{\pgfqpoint{5.003171in}{3.523008in}}%
\pgfpathlineto{\pgfqpoint{5.005196in}{5.642425in}}%
\pgfpathlineto{\pgfqpoint{5.007221in}{3.514169in}}%
\pgfpathlineto{\pgfqpoint{5.009246in}{5.645082in}}%
\pgfpathlineto{\pgfqpoint{5.011272in}{3.519664in}}%
\pgfpathlineto{\pgfqpoint{5.013297in}{5.629656in}}%
\pgfpathlineto{\pgfqpoint{5.015322in}{3.546513in}}%
\pgfpathlineto{\pgfqpoint{5.017347in}{5.590251in}}%
\pgfpathlineto{\pgfqpoint{5.019373in}{3.599200in}}%
\pgfpathlineto{\pgfqpoint{5.021398in}{5.523968in}}%
\pgfpathlineto{\pgfqpoint{5.023423in}{3.678992in}}%
\pgfpathlineto{\pgfqpoint{5.025449in}{5.431117in}}%
\pgfpathlineto{\pgfqpoint{5.027474in}{3.784142in}}%
\pgfpathlineto{\pgfqpoint{5.029499in}{5.314674in}}%
\pgfpathlineto{\pgfqpoint{5.031524in}{3.910698in}}%
\pgfpathlineto{\pgfqpoint{5.033550in}{5.179291in}}%
\pgfpathlineto{\pgfqpoint{5.035575in}{4.053581in}}%
\pgfpathlineto{\pgfqpoint{5.037600in}{5.030219in}}%
\pgfpathlineto{\pgfqpoint{5.039625in}{4.207593in}}%
\pgfpathlineto{\pgfqpoint{5.041651in}{4.872420in}}%
\pgfpathlineto{\pgfqpoint{5.043676in}{4.368146in}}%
\pgfpathlineto{\pgfqpoint{5.045701in}{4.710017in}}%
\pgfpathlineto{\pgfqpoint{5.047727in}{4.531631in}}%
\pgfpathlineto{\pgfqpoint{5.049752in}{4.546089in}}%
\pgfpathlineto{\pgfqpoint{5.051777in}{4.695480in}}%
\pgfpathlineto{\pgfqpoint{5.053802in}{4.382736in}}%
\pgfpathlineto{\pgfqpoint{5.055828in}{4.858012in}}%
\pgfpathlineto{\pgfqpoint{5.057853in}{4.221279in}}%
\pgfpathlineto{\pgfqpoint{5.059878in}{5.018194in}}%
\pgfpathlineto{\pgfqpoint{5.061903in}{4.062530in}}%
\pgfpathlineto{\pgfqpoint{5.063929in}{5.175371in}}%
\pgfpathlineto{\pgfqpoint{5.065954in}{3.907062in}}%
\pgfpathlineto{\pgfqpoint{5.067979in}{5.328979in}}%
\pgfpathlineto{\pgfqpoint{5.070005in}{3.755504in}}%
\pgfpathlineto{\pgfqpoint{5.072030in}{5.478241in}}%
\pgfpathlineto{\pgfqpoint{5.074055in}{3.608866in}}%
\pgfpathlineto{\pgfqpoint{5.076080in}{5.621823in}}%
\pgfpathlineto{\pgfqpoint{5.078106in}{3.468895in}}%
\pgfpathlineto{\pgfqpoint{5.080131in}{5.757485in}}%
\pgfpathlineto{\pgfqpoint{5.082156in}{3.338393in}}%
\pgfpathlineto{\pgfqpoint{5.084181in}{5.881815in}}%
\pgfpathlineto{\pgfqpoint{5.086207in}{3.221403in}}%
\pgfpathlineto{\pgfqpoint{5.088232in}{5.990155in}}%
\pgfpathlineto{\pgfqpoint{5.090257in}{3.123142in}}%
\pgfpathlineto{\pgfqpoint{5.092283in}{6.076821in}}%
\pgfpathlineto{\pgfqpoint{5.094308in}{3.049630in}}%
\pgfpathlineto{\pgfqpoint{5.096333in}{6.135623in}}%
\pgfpathlineto{\pgfqpoint{5.098358in}{3.007040in}}%
\pgfpathlineto{\pgfqpoint{5.100384in}{6.160607in}}%
\pgfpathlineto{\pgfqpoint{5.102409in}{3.000903in}}%
\pgfpathlineto{\pgfqpoint{5.104434in}{6.146846in}}%
\pgfpathlineto{\pgfqpoint{5.106459in}{3.035389in}}%
\pgfpathlineto{\pgfqpoint{5.108485in}{6.091049in}}%
\pgfpathlineto{\pgfqpoint{5.110510in}{3.112837in}}%
\pgfpathlineto{\pgfqpoint{5.112535in}{5.991855in}}%
\pgfpathlineto{\pgfqpoint{5.114561in}{3.233647in}}%
\pgfpathlineto{\pgfqpoint{5.116586in}{5.849766in}}%
\pgfpathlineto{\pgfqpoint{5.118611in}{3.396495in}}%
\pgfpathlineto{\pgfqpoint{5.120636in}{5.666837in}}%
\pgfpathlineto{\pgfqpoint{5.122662in}{3.598688in}}%
\pgfpathlineto{\pgfqpoint{5.124687in}{5.446325in}}%
\pgfpathlineto{\pgfqpoint{5.126712in}{3.836443in}}%
\pgfpathlineto{\pgfqpoint{5.128737in}{5.192539in}}%
\pgfpathlineto{\pgfqpoint{5.130763in}{4.104889in}}%
\pgfpathlineto{\pgfqpoint{5.132788in}{4.910993in}}%
\pgfpathlineto{\pgfqpoint{5.134813in}{4.397753in}}%
\pgfpathlineto{\pgfqpoint{5.138864in}{4.706853in}}%
\pgfpathlineto{\pgfqpoint{5.140889in}{4.295428in}}%
\pgfpathlineto{\pgfqpoint{5.142914in}{5.021641in}}%
\pgfpathlineto{\pgfqpoint{5.144940in}{3.982535in}}%
\pgfpathlineto{\pgfqpoint{5.146965in}{5.329095in}}%
\pgfpathlineto{\pgfqpoint{5.148990in}{3.684303in}}%
\pgfpathlineto{\pgfqpoint{5.151015in}{5.614151in}}%
\pgfpathlineto{\pgfqpoint{5.153041in}{3.416466in}}%
\pgfpathlineto{\pgfqpoint{5.155066in}{5.860729in}}%
\pgfpathlineto{\pgfqpoint{5.157091in}{3.195083in}}%
\pgfpathlineto{\pgfqpoint{5.159117in}{6.053189in}}%
\pgfpathlineto{\pgfqpoint{5.161142in}{3.034965in}}%
\pgfpathlineto{\pgfqpoint{5.163167in}{6.177951in}}%
\pgfpathlineto{\pgfqpoint{5.165192in}{2.948090in}}%
\pgfpathlineto{\pgfqpoint{5.167218in}{6.224958in}}%
\pgfpathlineto{\pgfqpoint{5.169243in}{2.942328in}}%
\pgfpathlineto{\pgfqpoint{5.171268in}{6.188730in}}%
\pgfpathlineto{\pgfqpoint{5.173293in}{3.020638in}}%
\pgfpathlineto{\pgfqpoint{5.175319in}{6.068900in}}%
\pgfpathlineto{\pgfqpoint{5.177344in}{3.180786in}}%
\pgfpathlineto{\pgfqpoint{5.179369in}{5.870256in}}%
\pgfpathlineto{\pgfqpoint{5.181395in}{3.415511in}}%
\pgfpathlineto{\pgfqpoint{5.183420in}{5.602422in}}%
\pgfpathlineto{\pgfqpoint{5.185445in}{3.712965in}}%
\pgfpathlineto{\pgfqpoint{5.187470in}{5.279313in}}%
\pgfpathlineto{\pgfqpoint{5.189496in}{4.057336in}}%
\pgfpathlineto{\pgfqpoint{5.191521in}{4.918449in}}%
\pgfpathlineto{\pgfqpoint{5.193546in}{4.429606in}}%
\pgfpathlineto{\pgfqpoint{5.195571in}{4.540108in}}%
\pgfpathlineto{\pgfqpoint{5.197597in}{4.808519in}}%
\pgfpathlineto{\pgfqpoint{5.199622in}{4.166194in}}%
\pgfpathlineto{\pgfqpoint{5.201647in}{5.171900in}}%
\pgfpathlineto{\pgfqpoint{5.203673in}{3.818718in}}%
\pgfpathlineto{\pgfqpoint{5.205698in}{5.498401in}}%
\pgfpathlineto{\pgfqpoint{5.207723in}{3.517817in}}%
\pgfpathlineto{\pgfqpoint{5.209748in}{5.769672in}}%
\pgfpathlineto{\pgfqpoint{5.211774in}{3.279459in}}%
\pgfpathlineto{\pgfqpoint{5.213799in}{5.972715in}}%
\pgfpathlineto{\pgfqpoint{5.215824in}{3.113135in}}%
\pgfpathlineto{\pgfqpoint{5.217849in}{6.102009in}}%
\pgfpathlineto{\pgfqpoint{5.219875in}{3.020039in}}%
\pgfpathlineto{\pgfqpoint{5.221900in}{6.160891in}}%
\pgfpathlineto{\pgfqpoint{5.223925in}{2.992269in}}%
\pgfpathlineto{\pgfqpoint{5.225951in}{6.161678in}}%
\pgfpathlineto{\pgfqpoint{5.227976in}{3.013449in}}%
\pgfpathlineto{\pgfqpoint{5.230001in}{6.124249in}}%
\pgfpathlineto{\pgfqpoint{5.232026in}{3.060935in}}%
\pgfpathlineto{\pgfqpoint{5.234052in}{6.073115in}}%
\pgfpathlineto{\pgfqpoint{5.236077in}{3.109364in}}%
\pgfpathlineto{\pgfqpoint{5.238102in}{6.033405in}}%
\pgfpathlineto{\pgfqpoint{5.240127in}{3.134948in}}%
\pgfpathlineto{\pgfqpoint{5.242153in}{6.026502in}}%
\pgfpathlineto{\pgfqpoint{5.244178in}{3.119670in}}%
\pgfpathlineto{\pgfqpoint{5.246203in}{6.066251in}}%
\pgfpathlineto{\pgfqpoint{5.248229in}{3.054462in}}%
\pgfpathlineto{\pgfqpoint{5.250254in}{6.156577in}}%
\pgfpathlineto{\pgfqpoint{5.252279in}{2.940648in}}%
\pgfpathlineto{\pgfqpoint{5.254304in}{6.291069in}}%
\pgfpathlineto{\pgfqpoint{5.256330in}{2.789307in}}%
\pgfpathlineto{\pgfqpoint{5.258355in}{6.454624in}}%
\pgfpathlineto{\pgfqpoint{5.260380in}{2.618730in}}%
\pgfpathlineto{\pgfqpoint{5.262405in}{6.626741in}}%
\pgfpathlineto{\pgfqpoint{5.264431in}{2.450575in}}%
\pgfpathlineto{\pgfqpoint{5.266456in}{6.785668in}}%
\pgfpathlineto{\pgfqpoint{5.268481in}{2.305677in}}%
\pgfpathlineto{\pgfqpoint{5.270507in}{6.912391in}}%
\pgfpathlineto{\pgfqpoint{5.272532in}{2.200485in}}%
\pgfpathlineto{\pgfqpoint{5.274557in}{6.993567in}}%
\pgfpathlineto{\pgfqpoint{5.276582in}{2.144917in}}%
\pgfpathlineto{\pgfqpoint{5.278608in}{7.022799in}}%
\pgfpathlineto{\pgfqpoint{5.280633in}{2.141965in}}%
\pgfpathlineto{\pgfqpoint{5.282658in}{7.000191in}}%
\pgfpathlineto{\pgfqpoint{5.284683in}{2.188894in}}%
\pgfpathlineto{\pgfqpoint{5.286709in}{6.930542in}}%
\pgfpathlineto{\pgfqpoint{5.288734in}{2.279451in}}%
\pgfpathlineto{\pgfqpoint{5.290759in}{6.820954in}}%
\pgfpathlineto{\pgfqpoint{5.292785in}{2.406244in}}%
\pgfpathlineto{\pgfqpoint{5.294810in}{6.678651in}}%
\pgfpathlineto{\pgfqpoint{5.296835in}{2.562538in}}%
\pgfpathlineto{\pgfqpoint{5.298860in}{6.509698in}}%
\pgfpathlineto{\pgfqpoint{5.300886in}{2.742975in}}%
\pgfpathlineto{\pgfqpoint{5.302911in}{6.318855in}}%
\pgfpathlineto{\pgfqpoint{5.304936in}{2.943172in}}%
\pgfpathlineto{\pgfqpoint{5.306961in}{6.110417in}}%
\pgfpathlineto{\pgfqpoint{5.308987in}{3.158592in}}%
\pgfpathlineto{\pgfqpoint{5.311012in}{5.889484in}}%
\pgfpathlineto{\pgfqpoint{5.313037in}{3.383314in}}%
\pgfpathlineto{\pgfqpoint{5.315063in}{5.662970in}}%
\pgfpathlineto{\pgfqpoint{5.317088in}{3.609377in}}%
\pgfpathlineto{\pgfqpoint{5.319113in}{5.439792in}}%
\pgfpathlineto{\pgfqpoint{5.321138in}{3.827134in}}%
\pgfpathlineto{\pgfqpoint{5.323164in}{5.229973in}}%
\pgfpathlineto{\pgfqpoint{5.325189in}{4.026649in}}%
\pgfpathlineto{\pgfqpoint{5.327214in}{5.042835in}}%
\pgfpathlineto{\pgfqpoint{5.329239in}{4.199768in}}%
\pgfpathlineto{\pgfqpoint{5.331265in}{4.884826in}}%
\pgfpathlineto{\pgfqpoint{5.333290in}{4.342220in}}%
\pgfpathlineto{\pgfqpoint{5.335315in}{4.757677in}}%
\pgfpathlineto{\pgfqpoint{5.337341in}{4.455032in}}%
\pgfpathlineto{\pgfqpoint{5.339366in}{4.657555in}}%
\pgfpathlineto{\pgfqpoint{5.341391in}{4.544718in}}%
\pgfpathlineto{\pgfqpoint{5.345442in}{4.622071in}}%
\pgfpathlineto{\pgfqpoint{5.347467in}{4.499632in}}%
\pgfpathlineto{\pgfqpoint{5.349492in}{4.699763in}}%
\pgfpathlineto{\pgfqpoint{5.351517in}{4.417255in}}%
\pgfpathlineto{\pgfqpoint{5.353543in}{4.789322in}}%
\pgfpathlineto{\pgfqpoint{5.355568in}{4.318621in}}%
\pgfpathlineto{\pgfqpoint{5.357593in}{4.898180in}}%
\pgfpathlineto{\pgfqpoint{5.359619in}{4.199233in}}%
\pgfpathlineto{\pgfqpoint{5.361644in}{5.027497in}}%
\pgfpathlineto{\pgfqpoint{5.363669in}{4.061509in}}%
\pgfpathlineto{\pgfqpoint{5.365694in}{5.171224in}}%
\pgfpathlineto{\pgfqpoint{5.367720in}{3.914992in}}%
\pgfpathlineto{\pgfqpoint{5.369745in}{5.316627in}}%
\pgfpathlineto{\pgfqpoint{5.371770in}{3.775141in}}%
\pgfpathlineto{\pgfqpoint{5.373795in}{5.446131in}}%
\pgfpathlineto{\pgfqpoint{5.375821in}{3.660934in}}%
\pgfpathlineto{\pgfqpoint{5.377846in}{5.540148in}}%
\pgfpathlineto{\pgfqpoint{5.379871in}{3.591733in}}%
\pgfpathlineto{\pgfqpoint{5.381897in}{5.580374in}}%
\pgfpathlineto{\pgfqpoint{5.383922in}{3.583985in}}%
\pgfpathlineto{\pgfqpoint{5.385947in}{5.552961in}}%
\pgfpathlineto{\pgfqpoint{5.387972in}{3.648298in}}%
\pgfpathlineto{\pgfqpoint{5.389998in}{5.451050in}}%
\pgfpathlineto{\pgfqpoint{5.392023in}{3.787416in}}%
\pgfpathlineto{\pgfqpoint{5.394048in}{5.276210in}}%
\pgfpathlineto{\pgfqpoint{5.396073in}{3.995434in}}%
\pgfpathlineto{\pgfqpoint{5.398099in}{5.038529in}}%
\pgfpathlineto{\pgfqpoint{5.400124in}{4.258414in}}%
\pgfpathlineto{\pgfqpoint{5.402149in}{4.755303in}}%
\pgfpathlineto{\pgfqpoint{5.406200in}{4.448551in}}%
\pgfpathlineto{\pgfqpoint{5.408225in}{4.865985in}}%
\pgfpathlineto{\pgfqpoint{5.410250in}{4.141796in}}%
\pgfpathlineto{\pgfqpoint{5.412276in}{5.164368in}}%
\pgfpathlineto{\pgfqpoint{5.414301in}{3.856748in}}%
\pgfpathlineto{\pgfqpoint{5.416326in}{5.431767in}}%
\pgfpathlineto{\pgfqpoint{5.418351in}{3.610561in}}%
\pgfpathlineto{\pgfqpoint{5.420377in}{5.653987in}}%
\pgfpathlineto{\pgfqpoint{5.422402in}{3.414246in}}%
\pgfpathlineto{\pgfqpoint{5.424427in}{5.823236in}}%
\pgfpathlineto{\pgfqpoint{5.426453in}{3.272526in}}%
\pgfpathlineto{\pgfqpoint{5.428478in}{5.937541in}}%
\pgfpathlineto{\pgfqpoint{5.430503in}{3.185088in}}%
\pgfpathlineto{\pgfqpoint{5.432528in}{5.998940in}}%
\pgfpathlineto{\pgfqpoint{5.434554in}{3.148780in}}%
\pgfpathlineto{\pgfqpoint{5.436579in}{6.011072in}}%
\pgfpathlineto{\pgfqpoint{5.438604in}{3.160064in}}%
\pgfpathlineto{\pgfqpoint{5.440629in}{5.976885in}}%
\pgfpathlineto{\pgfqpoint{5.442655in}{3.216944in}}%
\pgfpathlineto{\pgfqpoint{5.444680in}{5.897218in}}%
\pgfpathlineto{\pgfqpoint{5.446705in}{3.319764in}}%
\pgfpathlineto{\pgfqpoint{5.448731in}{5.770687in}}%
\pgfpathlineto{\pgfqpoint{5.450756in}{3.470640in}}%
\pgfpathlineto{\pgfqpoint{5.452781in}{5.594892in}}%
\pgfpathlineto{\pgfqpoint{5.454806in}{3.671718in}}%
\pgfpathlineto{\pgfqpoint{5.456832in}{5.368521in}}%
\pgfpathlineto{\pgfqpoint{5.458857in}{3.922906in}}%
\pgfpathlineto{\pgfqpoint{5.460882in}{5.093575in}}%
\pgfpathlineto{\pgfqpoint{5.462907in}{4.219912in}}%
\pgfpathlineto{\pgfqpoint{5.464933in}{4.776857in}}%
\pgfpathlineto{\pgfqpoint{5.468983in}{4.430092in}}%
\pgfpathlineto{\pgfqpoint{5.471009in}{4.909607in}}%
\pgfpathlineto{\pgfqpoint{5.473034in}{4.068448in}}%
\pgfpathlineto{\pgfqpoint{5.475059in}{5.272639in}}%
\pgfpathlineto{\pgfqpoint{5.477084in}{3.707864in}}%
\pgfpathlineto{\pgfqpoint{5.479110in}{5.627326in}}%
\pgfpathlineto{\pgfqpoint{5.481135in}{3.361982in}}%
\pgfpathlineto{\pgfqpoint{5.483160in}{5.962152in}}%
\pgfpathlineto{\pgfqpoint{5.485185in}{3.039745in}}%
\pgfpathlineto{\pgfqpoint{5.487211in}{6.270976in}}%
\pgfpathlineto{\pgfqpoint{5.489236in}{2.744515in}}%
\pgfpathlineto{\pgfqpoint{5.491261in}{6.552953in}}%
\pgfpathlineto{\pgfqpoint{5.493287in}{2.475102in}}%
\pgfpathlineto{\pgfqpoint{5.495312in}{6.810633in}}%
\pgfpathlineto{\pgfqpoint{5.497337in}{2.228400in}}%
\pgfpathlineto{\pgfqpoint{5.499362in}{7.046824in}}%
\pgfpathlineto{\pgfqpoint{5.501388in}{2.002730in}}%
\pgfpathlineto{\pgfqpoint{5.503413in}{7.261346in}}%
\pgfpathlineto{\pgfqpoint{5.505438in}{1.800678in}}%
\pgfpathlineto{\pgfqpoint{5.507463in}{7.448898in}}%
\pgfpathlineto{\pgfqpoint{5.509489in}{1.630298in}}%
\pgfpathlineto{\pgfqpoint{5.511514in}{7.598930in}}%
\pgfpathlineto{\pgfqpoint{5.513539in}{1.504097in}}%
\pgfpathlineto{\pgfqpoint{5.515565in}{7.697758in}}%
\pgfpathlineto{\pgfqpoint{5.517590in}{1.435966in}}%
\pgfpathlineto{\pgfqpoint{5.519615in}{7.732368in}}%
\pgfpathlineto{\pgfqpoint{5.521640in}{1.436944in}}%
\pgfpathlineto{\pgfqpoint{5.523666in}{7.694716in}}%
\pgfpathlineto{\pgfqpoint{5.525691in}{1.511216in}}%
\pgfpathlineto{\pgfqpoint{5.527716in}{7.585094in}}%
\pgfpathlineto{\pgfqpoint{5.529741in}{1.653707in}}%
\pgfpathlineto{\pgfqpoint{5.531767in}{7.413330in}}%
\pgfpathlineto{\pgfqpoint{5.533792in}{1.850217in}}%
\pgfpathlineto{\pgfqpoint{5.535817in}{7.197279in}}%
\pgfpathlineto{\pgfqpoint{5.537843in}{2.080236in}}%
\pgfpathlineto{\pgfqpoint{5.539868in}{6.958888in}}%
\pgfpathlineto{\pgfqpoint{5.541893in}{2.321725in}}%
\pgfpathlineto{\pgfqpoint{5.543918in}{6.718928in}}%
\pgfpathlineto{\pgfqpoint{5.545944in}{2.556461in}}%
\pgfpathlineto{\pgfqpoint{5.547969in}{6.491971in}}%
\pgfpathlineto{\pgfqpoint{5.549994in}{2.774348in}}%
\pgfpathlineto{\pgfqpoint{5.552019in}{6.283153in}}%
\pgfpathlineto{\pgfqpoint{5.554045in}{2.975315in}}%
\pgfpathlineto{\pgfqpoint{5.556070in}{6.087770in}}%
\pgfpathlineto{\pgfqpoint{5.558095in}{3.168170in}}%
\pgfpathlineto{\pgfqpoint{5.560121in}{5.893914in}}%
\pgfpathlineto{\pgfqpoint{5.562146in}{3.366668in}}%
\pgfpathlineto{\pgfqpoint{5.564171in}{5.687400in}}%
\pgfpathlineto{\pgfqpoint{5.566196in}{3.583950in}}%
\pgfpathlineto{\pgfqpoint{5.568222in}{5.457538in}}%
\pgfpathlineto{\pgfqpoint{5.570247in}{3.827020in}}%
\pgfpathlineto{\pgfqpoint{5.572272in}{5.201971in}}%
\pgfpathlineto{\pgfqpoint{5.574297in}{4.092981in}}%
\pgfpathlineto{\pgfqpoint{5.576323in}{4.929052in}}%
\pgfpathlineto{\pgfqpoint{5.578348in}{4.368250in}}%
\pgfpathlineto{\pgfqpoint{5.580373in}{4.656946in}}%
\pgfpathlineto{\pgfqpoint{5.582399in}{4.631114in}}%
\pgfpathlineto{\pgfqpoint{5.584424in}{4.409580in}}%
\pgfpathlineto{\pgfqpoint{5.586449in}{4.856975in}}%
\pgfpathlineto{\pgfqpoint{5.588474in}{4.210571in}}%
\pgfpathlineto{\pgfqpoint{5.590500in}{5.024822in}}%
\pgfpathlineto{\pgfqpoint{5.592525in}{4.076843in}}%
\pgfpathlineto{\pgfqpoint{5.594550in}{5.123047in}}%
\pgfpathlineto{\pgfqpoint{5.596575in}{4.013812in}}%
\pgfpathlineto{\pgfqpoint{5.598601in}{5.152894in}}%
\pgfpathlineto{\pgfqpoint{5.600626in}{4.013556in}}%
\pgfpathlineto{\pgfqpoint{5.602651in}{5.128517in}}%
\pgfpathlineto{\pgfqpoint{5.604677in}{4.056549in}}%
\pgfpathlineto{\pgfqpoint{5.606702in}{5.073600in}}%
\pgfpathlineto{\pgfqpoint{5.608727in}{4.116440in}}%
\pgfpathlineto{\pgfqpoint{5.610752in}{5.015517in}}%
\pgfpathlineto{\pgfqpoint{5.612778in}{4.166508in}}%
\pgfpathlineto{\pgfqpoint{5.614803in}{4.978741in}}%
\pgfpathlineto{\pgfqpoint{5.616828in}{4.185917in}}%
\pgfpathlineto{\pgfqpoint{5.618853in}{4.979389in}}%
\pgfpathlineto{\pgfqpoint{5.620879in}{4.163984in}}%
\pgfpathlineto{\pgfqpoint{5.622904in}{5.022408in}}%
\pgfpathlineto{\pgfqpoint{5.624929in}{4.101366in}}%
\pgfpathlineto{\pgfqpoint{5.626955in}{5.102069in}}%
\pgfpathlineto{\pgfqpoint{5.628980in}{4.007999in}}%
\pgfpathlineto{\pgfqpoint{5.631005in}{5.205357in}}%
\pgfpathlineto{\pgfqpoint{5.633030in}{3.898691in}}%
\pgfpathlineto{\pgfqpoint{5.635056in}{5.316988in}}%
\pgfpathlineto{\pgfqpoint{5.637081in}{3.787955in}}%
\pgfpathlineto{\pgfqpoint{5.639106in}{5.424289in}}%
\pgfpathlineto{\pgfqpoint{5.641131in}{3.685830in}}%
\pgfpathlineto{\pgfqpoint{5.643157in}{5.520325in}}%
\pgfpathlineto{\pgfqpoint{5.645182in}{3.596037in}}%
\pgfpathlineto{\pgfqpoint{5.647207in}{5.604335in}}%
\pgfpathlineto{\pgfqpoint{5.649233in}{3.516943in}}%
\pgfpathlineto{\pgfqpoint{5.651258in}{5.679530in}}%
\pgfpathlineto{\pgfqpoint{5.653283in}{3.444750in}}%
\pgfpathlineto{\pgfqpoint{5.655308in}{5.749245in}}%
\pgfpathlineto{\pgfqpoint{5.657334in}{3.377578in}}%
\pgfpathlineto{\pgfqpoint{5.659359in}{5.813071in}}%
\pgfpathlineto{\pgfqpoint{5.661384in}{3.318706in}}%
\pgfpathlineto{\pgfqpoint{5.663410in}{5.864597in}}%
\pgfpathlineto{\pgfqpoint{5.665435in}{3.277578in}}%
\pgfpathlineto{\pgfqpoint{5.667460in}{5.891822in}}%
\pgfpathlineto{\pgfqpoint{5.669485in}{3.267934in}}%
\pgfpathlineto{\pgfqpoint{5.671511in}{5.880367in}}%
\pgfpathlineto{\pgfqpoint{5.673536in}{3.303492in}}%
\pgfpathlineto{\pgfqpoint{5.675561in}{5.818561in}}%
\pgfpathlineto{\pgfqpoint{5.677586in}{3.392523in}}%
\pgfpathlineto{\pgfqpoint{5.679612in}{5.702740in}}%
\pgfpathlineto{\pgfqpoint{5.681637in}{3.533138in}}%
\pgfpathlineto{\pgfqpoint{5.683662in}{5.540931in}}%
\pgfpathlineto{\pgfqpoint{5.685688in}{3.711008in}}%
\pgfpathlineto{\pgfqpoint{5.687713in}{5.353479in}}%
\pgfpathlineto{\pgfqpoint{5.689738in}{3.900510in}}%
\pgfpathlineto{\pgfqpoint{5.691763in}{5.170133in}}%
\pgfpathlineto{\pgfqpoint{5.693789in}{4.069261in}}%
\pgfpathlineto{\pgfqpoint{5.695814in}{5.024176in}}%
\pgfpathlineto{\pgfqpoint{5.697839in}{4.184932in}}%
\pgfpathlineto{\pgfqpoint{5.699864in}{4.945140in}}%
\pgfpathlineto{\pgfqpoint{5.701890in}{4.222499in}}%
\pgfpathlineto{\pgfqpoint{5.703915in}{4.952087in}}%
\pgfpathlineto{\pgfqpoint{5.705940in}{4.169956in}}%
\pgfpathlineto{\pgfqpoint{5.707966in}{5.049296in}}%
\pgfpathlineto{\pgfqpoint{5.709991in}{4.030954in}}%
\pgfpathlineto{\pgfqpoint{5.712016in}{5.225454in}}%
\pgfpathlineto{\pgfqpoint{5.714041in}{3.823761in}}%
\pgfpathlineto{\pgfqpoint{5.716067in}{5.456427in}}%
\pgfpathlineto{\pgfqpoint{5.718092in}{3.577004in}}%
\pgfpathlineto{\pgfqpoint{5.720117in}{5.710646in}}%
\pgfpathlineto{\pgfqpoint{5.722142in}{3.323573in}}%
\pgfpathlineto{\pgfqpoint{5.724168in}{5.955467in}}%
\pgfpathlineto{\pgfqpoint{5.726193in}{3.094461in}}%
\pgfpathlineto{\pgfqpoint{5.728218in}{6.162713in}}%
\pgfpathlineto{\pgfqpoint{5.730244in}{2.914168in}}%
\pgfpathlineto{\pgfqpoint{5.732269in}{6.312075in}}%
\pgfpathlineto{\pgfqpoint{5.734294in}{2.798640in}}%
\pgfpathlineto{\pgfqpoint{5.736319in}{6.391836in}}%
\pgfpathlineto{\pgfqpoint{5.738345in}{2.755758in}}%
\pgfpathlineto{\pgfqpoint{5.740370in}{6.397373in}}%
\pgfpathlineto{\pgfqpoint{5.742395in}{2.787560in}}%
\pgfpathlineto{\pgfqpoint{5.744420in}{6.328552in}}%
\pgfpathlineto{\pgfqpoint{5.746446in}{2.892884in}}%
\pgfpathlineto{\pgfqpoint{5.748471in}{6.187372in}}%
\pgfpathlineto{\pgfqpoint{5.750496in}{3.069153in}}%
\pgfpathlineto{\pgfqpoint{5.752522in}{5.976947in}}%
\pgfpathlineto{\pgfqpoint{5.754547in}{3.312543in}}%
\pgfpathlineto{\pgfqpoint{5.756572in}{5.702170in}}%
\pgfpathlineto{\pgfqpoint{5.758597in}{3.616597in}}%
\pgfpathlineto{\pgfqpoint{5.760623in}{5.371621in}}%
\pgfpathlineto{\pgfqpoint{5.762648in}{3.970074in}}%
\pgfpathlineto{\pgfqpoint{5.764673in}{4.999639in}}%
\pgfpathlineto{\pgfqpoint{5.766698in}{4.355277in}}%
\pgfpathlineto{\pgfqpoint{5.770749in}{4.748010in}}%
\pgfpathlineto{\pgfqpoint{5.772774in}{4.221366in}}%
\pgfpathlineto{\pgfqpoint{5.774800in}{5.119765in}}%
\pgfpathlineto{\pgfqpoint{5.776825in}{3.871035in}}%
\pgfpathlineto{\pgfqpoint{5.778850in}{5.441888in}}%
\pgfpathlineto{\pgfqpoint{5.780875in}{3.583115in}}%
\pgfpathlineto{\pgfqpoint{5.782901in}{5.690698in}}%
\pgfpathlineto{\pgfqpoint{5.784926in}{3.376998in}}%
\pgfpathlineto{\pgfqpoint{5.786951in}{5.852031in}}%
\pgfpathlineto{\pgfqpoint{5.788976in}{3.260978in}}%
\pgfpathlineto{\pgfqpoint{5.791002in}{5.923743in}}%
\pgfpathlineto{\pgfqpoint{5.793027in}{3.231160in}}%
\pgfpathlineto{\pgfqpoint{5.795052in}{5.915272in}}%
\pgfpathlineto{\pgfqpoint{5.797078in}{3.273419in}}%
\pgfpathlineto{\pgfqpoint{5.799103in}{5.844278in}}%
\pgfpathlineto{\pgfqpoint{5.801128in}{3.367911in}}%
\pgfpathlineto{\pgfqpoint{5.803153in}{5.731338in}}%
\pgfpathlineto{\pgfqpoint{5.805179in}{3.494766in}}%
\pgfpathlineto{\pgfqpoint{5.807204in}{5.594313in}}%
\pgfpathlineto{\pgfqpoint{5.809229in}{3.639194in}}%
\pgfpathlineto{\pgfqpoint{5.811254in}{5.444182in}}%
\pgfpathlineto{\pgfqpoint{5.813280in}{3.794382in}}%
\pgfpathlineto{\pgfqpoint{5.815305in}{5.283648in}}%
\pgfpathlineto{\pgfqpoint{5.817330in}{3.961274in}}%
\pgfpathlineto{\pgfqpoint{5.819356in}{5.108941in}}%
\pgfpathlineto{\pgfqpoint{5.821381in}{4.145363in}}%
\pgfpathlineto{\pgfqpoint{5.823406in}{4.914142in}}%
\pgfpathlineto{\pgfqpoint{5.825431in}{4.351626in}}%
\pgfpathlineto{\pgfqpoint{5.827457in}{4.696524in}}%
\pgfpathlineto{\pgfqpoint{5.831507in}{4.460969in}}%
\pgfpathlineto{\pgfqpoint{5.833532in}{4.819126in}}%
\pgfpathlineto{\pgfqpoint{5.835558in}{4.221823in}}%
\pgfpathlineto{\pgfqpoint{5.837583in}{5.052060in}}%
\pgfpathlineto{\pgfqpoint{5.839608in}{4.001352in}}%
\pgfpathlineto{\pgfqpoint{5.841634in}{5.253582in}}%
\pgfpathlineto{\pgfqpoint{5.843659in}{3.825081in}}%
\pgfpathlineto{\pgfqpoint{5.845684in}{5.398911in}}%
\pgfpathlineto{\pgfqpoint{5.847709in}{3.715372in}}%
\pgfpathlineto{\pgfqpoint{5.849735in}{5.469680in}}%
\pgfpathlineto{\pgfqpoint{5.851760in}{3.685243in}}%
\pgfpathlineto{\pgfqpoint{5.853785in}{5.459235in}}%
\pgfpathlineto{\pgfqpoint{5.855810in}{3.734403in}}%
\pgfpathlineto{\pgfqpoint{5.857836in}{5.374911in}}%
\pgfpathlineto{\pgfqpoint{5.859861in}{3.848870in}}%
\pgfpathlineto{\pgfqpoint{5.861886in}{5.236465in}}%
\pgfpathlineto{\pgfqpoint{5.863912in}{4.004399in}}%
\pgfpathlineto{\pgfqpoint{5.865937in}{5.071021in}}%
\pgfpathlineto{\pgfqpoint{5.867962in}{4.172789in}}%
\pgfpathlineto{\pgfqpoint{5.869987in}{4.906002in}}%
\pgfpathlineto{\pgfqpoint{5.872013in}{4.329173in}}%
\pgfpathlineto{\pgfqpoint{5.874038in}{4.762156in}}%
\pgfpathlineto{\pgfqpoint{5.876063in}{4.458143in}}%
\pgfpathlineto{\pgfqpoint{5.878088in}{4.648759in}}%
\pgfpathlineto{\pgfqpoint{5.880114in}{4.556860in}}%
\pgfpathlineto{\pgfqpoint{5.882139in}{4.562409in}}%
\pgfpathlineto{\pgfqpoint{5.884164in}{4.634292in}}%
\pgfpathlineto{\pgfqpoint{5.886190in}{4.489671in}}%
\pgfpathlineto{\pgfqpoint{5.888215in}{4.706920in}}%
\pgfpathlineto{\pgfqpoint{5.890240in}{4.412643in}}%
\pgfpathlineto{\pgfqpoint{5.892265in}{4.792368in}}%
\pgfpathlineto{\pgfqpoint{5.894291in}{4.315611in}}%
\pgfpathlineto{\pgfqpoint{5.896316in}{4.903010in}}%
\pgfpathlineto{\pgfqpoint{5.898341in}{4.190652in}}%
\pgfpathlineto{\pgfqpoint{5.900366in}{5.041610in}}%
\pgfpathlineto{\pgfqpoint{5.902392in}{4.040416in}}%
\pgfpathlineto{\pgfqpoint{5.904417in}{5.200316in}}%
\pgfpathlineto{\pgfqpoint{5.906442in}{3.877301in}}%
\pgfpathlineto{\pgfqpoint{5.908468in}{5.363227in}}%
\pgfpathlineto{\pgfqpoint{5.910493in}{3.719372in}}%
\pgfpathlineto{\pgfqpoint{5.912518in}{5.511622in}}%
\pgfpathlineto{\pgfqpoint{5.914543in}{3.584461in}}%
\pgfpathlineto{\pgfqpoint{5.916569in}{5.630022in}}%
\pgfpathlineto{\pgfqpoint{5.918594in}{3.484440in}}%
\pgfpathlineto{\pgfqpoint{5.920619in}{5.711088in}}%
\pgfpathlineto{\pgfqpoint{5.922644in}{3.421584in}}%
\pgfpathlineto{\pgfqpoint{5.924670in}{5.757704in}}%
\pgfpathlineto{\pgfqpoint{5.926695in}{3.388213in}}%
\pgfpathlineto{\pgfqpoint{5.928720in}{5.781569in}}%
\pgfpathlineto{\pgfqpoint{5.930746in}{3.369731in}}%
\pgfpathlineto{\pgfqpoint{5.932771in}{5.798789in}}%
\pgfpathlineto{\pgfqpoint{5.934796in}{3.350042in}}%
\pgfpathlineto{\pgfqpoint{5.936821in}{5.823927in}}%
\pgfpathlineto{\pgfqpoint{5.938847in}{3.317519in}}%
\pgfpathlineto{\pgfqpoint{5.940872in}{5.864520in}}%
\pgfpathlineto{\pgfqpoint{5.942897in}{3.269523in}}%
\pgfpathlineto{\pgfqpoint{5.944922in}{5.917920in}}%
\pgfpathlineto{\pgfqpoint{5.946948in}{3.213915in}}%
\pgfpathlineto{\pgfqpoint{5.948973in}{5.971588in}}%
\pgfpathlineto{\pgfqpoint{5.950998in}{3.166956in}}%
\pgfpathlineto{\pgfqpoint{5.953024in}{6.006847in}}%
\pgfpathlineto{\pgfqpoint{5.955049in}{3.148181in}}%
\pgfpathlineto{\pgfqpoint{5.957074in}{6.004988in}}%
\pgfpathlineto{\pgfqpoint{5.959099in}{3.173798in}}%
\pgfpathlineto{\pgfqpoint{5.961125in}{5.953835in}}%
\pgfpathlineto{\pgfqpoint{5.963150in}{3.250685in}}%
\pgfpathlineto{\pgfqpoint{5.965175in}{5.852705in}}%
\pgfpathlineto{\pgfqpoint{5.967200in}{3.372886in}}%
\pgfpathlineto{\pgfqpoint{5.969226in}{5.714155in}}%
\pgfpathlineto{\pgfqpoint{5.971251in}{3.521764in}}%
\pgfpathlineto{\pgfqpoint{5.973276in}{5.561922in}}%
\pgfpathlineto{\pgfqpoint{5.975302in}{3.669844in}}%
\pgfpathlineto{\pgfqpoint{5.977327in}{5.425580in}}%
\pgfpathlineto{\pgfqpoint{5.979352in}{3.787252in}}%
\pgfpathlineto{\pgfqpoint{5.981377in}{5.333474in}}%
\pgfpathlineto{\pgfqpoint{5.983403in}{3.848895in}}%
\pgfpathlineto{\pgfqpoint{5.985428in}{5.305953in}}%
\pgfpathlineto{\pgfqpoint{5.987453in}{3.840330in}}%
\pgfpathlineto{\pgfqpoint{5.989478in}{5.350796in}}%
\pgfpathlineto{\pgfqpoint{5.991504in}{3.760749in}}%
\pgfpathlineto{\pgfqpoint{5.993529in}{5.461990in}}%
\pgfpathlineto{\pgfqpoint{5.995554in}{3.622413in}}%
\pgfpathlineto{\pgfqpoint{5.997580in}{5.621963in}}%
\pgfpathlineto{\pgfqpoint{5.999605in}{3.446986in}}%
\pgfpathlineto{\pgfqpoint{6.001630in}{5.806348in}}%
\pgfpathlineto{\pgfqpoint{6.003655in}{3.260097in}}%
\pgfpathlineto{\pgfqpoint{6.005681in}{5.989646in}}%
\pgfpathlineto{\pgfqpoint{6.007706in}{3.085878in}}%
\pgfpathlineto{\pgfqpoint{6.009731in}{6.150087in}}%
\pgfpathlineto{\pgfqpoint{6.011756in}{2.943032in}}%
\pgfpathlineto{\pgfqpoint{6.013782in}{6.272417in}}%
\pgfpathlineto{\pgfqpoint{6.015807in}{2.843294in}}%
\pgfpathlineto{\pgfqpoint{6.017832in}{6.348218in}}%
\pgfpathlineto{\pgfqpoint{6.019858in}{2.792195in}}%
\pgfpathlineto{\pgfqpoint{6.021883in}{6.374272in}}%
\pgfpathlineto{\pgfqpoint{6.023908in}{2.791263in}}%
\pgfpathlineto{\pgfqpoint{6.025933in}{6.350144in}}%
\pgfpathlineto{\pgfqpoint{6.027959in}{2.840328in}}%
\pgfpathlineto{\pgfqpoint{6.029984in}{6.276290in}}%
\pgfpathlineto{\pgfqpoint{6.032009in}{2.938768in}}%
\pgfpathlineto{\pgfqpoint{6.034034in}{6.153602in}}%
\pgfpathlineto{\pgfqpoint{6.036060in}{3.085117in}}%
\pgfpathlineto{\pgfqpoint{6.038085in}{5.984573in}}%
\pgfpathlineto{\pgfqpoint{6.040110in}{3.275305in}}%
\pgfpathlineto{\pgfqpoint{6.042136in}{5.775414in}}%
\pgfpathlineto{\pgfqpoint{6.044161in}{3.500490in}}%
\pgfpathlineto{\pgfqpoint{6.046186in}{5.537938in}}%
\pgfpathlineto{\pgfqpoint{6.048211in}{3.745769in}}%
\pgfpathlineto{\pgfqpoint{6.050237in}{5.289983in}}%
\pgfpathlineto{\pgfqpoint{6.052262in}{3.990819in}}%
\pgfpathlineto{\pgfqpoint{6.054287in}{5.053622in}}%
\pgfpathlineto{\pgfqpoint{6.056312in}{4.212808in}}%
\pgfpathlineto{\pgfqpoint{6.058338in}{4.851262in}}%
\pgfpathlineto{\pgfqpoint{6.060363in}{4.391034in}}%
\pgfpathlineto{\pgfqpoint{6.062388in}{4.700617in}}%
\pgfpathlineto{\pgfqpoint{6.064414in}{4.511954in}}%
\pgfpathlineto{\pgfqpoint{6.066439in}{4.610098in}}%
\pgfpathlineto{\pgfqpoint{6.068464in}{4.572944in}}%
\pgfpathlineto{\pgfqpoint{6.070489in}{4.576253in}}%
\pgfpathlineto{\pgfqpoint{6.072515in}{4.583398in}}%
\pgfpathlineto{\pgfqpoint{6.074540in}{4.584305in}}%
\pgfpathlineto{\pgfqpoint{6.076565in}{4.562523in}}%
\pgfpathlineto{\pgfqpoint{6.078590in}{4.611913in}}%
\pgfpathlineto{\pgfqpoint{6.080616in}{4.534255in}}%
\pgfpathlineto{\pgfqpoint{6.082641in}{4.635214in}}%
\pgfpathlineto{\pgfqpoint{6.084666in}{4.520704in}}%
\pgfpathlineto{\pgfqpoint{6.086692in}{4.635412in}}%
\pgfpathlineto{\pgfqpoint{6.088717in}{4.536039in}}%
\pgfpathlineto{\pgfqpoint{6.090742in}{4.603925in}}%
\pgfpathlineto{\pgfqpoint{6.092767in}{4.582720in}}%
\pgfpathlineto{\pgfqpoint{6.094793in}{4.544478in}}%
\pgfpathlineto{\pgfqpoint{6.096818in}{4.651254in}}%
\pgfpathlineto{\pgfqpoint{6.098843in}{4.471469in}}%
\pgfpathlineto{\pgfqpoint{6.100868in}{4.723587in}}%
\pgfpathlineto{\pgfqpoint{6.102894in}{4.405070in}}%
\pgfpathlineto{\pgfqpoint{6.104919in}{4.779123in}}%
\pgfpathlineto{\pgfqpoint{6.106944in}{4.364580in}}%
\pgfpathlineto{\pgfqpoint{6.108970in}{4.801478in}}%
\pgfpathlineto{\pgfqpoint{6.110995in}{4.362092in}}%
\pgfpathlineto{\pgfqpoint{6.113020in}{4.783877in}}%
\pgfpathlineto{\pgfqpoint{6.115045in}{4.398467in}}%
\pgfpathlineto{\pgfqpoint{6.117071in}{4.731488in}}%
\pgfpathlineto{\pgfqpoint{6.119096in}{4.462880in}}%
\pgfpathlineto{\pgfqpoint{6.121121in}{4.659958in}}%
\pgfpathlineto{\pgfqpoint{6.123146in}{4.536088in}}%
\pgfpathlineto{\pgfqpoint{6.125172in}{4.590613in}}%
\pgfpathlineto{\pgfqpoint{6.127197in}{4.596368in}}%
\pgfpathlineto{\pgfqpoint{6.129222in}{4.543848in}}%
\pgfpathlineto{\pgfqpoint{6.131248in}{4.626269in}}%
\pgfpathlineto{\pgfqpoint{6.133273in}{4.532795in}}%
\pgfpathlineto{\pgfqpoint{6.135298in}{4.618007in}}%
\pgfpathlineto{\pgfqpoint{6.137323in}{4.559290in}}%
\pgfpathlineto{\pgfqpoint{6.139349in}{4.575811in}}%
\pgfpathlineto{\pgfqpoint{6.141374in}{4.613417in}}%
\pgfpathlineto{\pgfqpoint{6.143399in}{4.514447in}}%
\pgfpathlineto{\pgfqpoint{6.145424in}{4.676790in}}%
\pgfpathlineto{\pgfqpoint{6.147450in}{4.454397in}}%
\pgfpathlineto{\pgfqpoint{6.149475in}{4.728528in}}%
\pgfpathlineto{\pgfqpoint{6.151500in}{4.415208in}}%
\pgfpathlineto{\pgfqpoint{6.153526in}{4.752041in}}%
\pgfpathlineto{\pgfqpoint{6.155551in}{4.409121in}}%
\pgfpathlineto{\pgfqpoint{6.157576in}{4.740478in}}%
\pgfpathlineto{\pgfqpoint{6.159601in}{4.436996in}}%
\pgfpathlineto{\pgfqpoint{6.161627in}{4.699095in}}%
\pgfpathlineto{\pgfqpoint{6.163652in}{4.487832in}}%
\pgfpathlineto{\pgfqpoint{6.165677in}{4.643800in}}%
\pgfpathlineto{\pgfqpoint{6.167702in}{4.542044in}}%
\pgfpathlineto{\pgfqpoint{6.169728in}{4.596322in}}%
\pgfpathlineto{\pgfqpoint{6.171753in}{4.577471in}}%
\pgfpathlineto{\pgfqpoint{6.173778in}{4.577510in}}%
\pgfpathlineto{\pgfqpoint{6.175804in}{4.576222in}}%
\pgfpathlineto{\pgfqpoint{6.177829in}{4.600881in}}%
\pgfpathlineto{\pgfqpoint{6.179854in}{4.530216in}}%
\pgfpathlineto{\pgfqpoint{6.181879in}{4.668445in}}%
\pgfpathlineto{\pgfqpoint{6.183905in}{4.443670in}}%
\pgfpathlineto{\pgfqpoint{6.185930in}{4.770103in}}%
\pgfpathlineto{\pgfqpoint{6.187955in}{4.331757in}}%
\pgfpathlineto{\pgfqpoint{6.189980in}{4.886816in}}%
\pgfpathlineto{\pgfqpoint{6.192006in}{4.215865in}}%
\pgfpathlineto{\pgfqpoint{6.194031in}{4.996543in}}%
\pgfpathlineto{\pgfqpoint{6.196056in}{4.116944in}}%
\pgfpathlineto{\pgfqpoint{6.198082in}{5.081084in}}%
\pgfpathlineto{\pgfqpoint{6.200107in}{4.049009in}}%
\pgfpathlineto{\pgfqpoint{6.202132in}{5.131711in}}%
\pgfpathlineto{\pgfqpoint{6.204157in}{4.014817in}}%
\pgfpathlineto{\pgfqpoint{6.206183in}{5.151848in}}%
\pgfpathlineto{\pgfqpoint{6.208208in}{4.005031in}}%
\pgfpathlineto{\pgfqpoint{6.210233in}{5.156022in}}%
\pgfpathlineto{\pgfqpoint{6.212258in}{4.001061in}}%
\pgfpathlineto{\pgfqpoint{6.214284in}{5.165454in}}%
\pgfpathlineto{\pgfqpoint{6.216309in}{3.980671in}}%
\pgfpathlineto{\pgfqpoint{6.218334in}{5.201719in}}%
\pgfpathlineto{\pgfqpoint{6.220360in}{3.924557in}}%
\pgfpathlineto{\pgfqpoint{6.222385in}{5.280431in}}%
\pgfpathlineto{\pgfqpoint{6.224410in}{3.821911in}}%
\pgfpathlineto{\pgfqpoint{6.226435in}{5.406851in}}%
\pgfpathlineto{\pgfqpoint{6.228461in}{3.673338in}}%
\pgfpathlineto{\pgfqpoint{6.230486in}{5.574625in}}%
\pgfpathlineto{\pgfqpoint{6.232511in}{3.490418in}}%
\pgfpathlineto{\pgfqpoint{6.234536in}{5.767826in}}%
\pgfpathlineto{\pgfqpoint{6.236562in}{3.292279in}}%
\pgfpathlineto{\pgfqpoint{6.238587in}{5.965433in}}%
\pgfpathlineto{\pgfqpoint{6.240612in}{3.100472in}}%
\pgfpathlineto{\pgfqpoint{6.242638in}{6.146665in}}%
\pgfpathlineto{\pgfqpoint{6.244663in}{2.933868in}}%
\pgfpathlineto{\pgfqpoint{6.246688in}{6.295461in}}%
\pgfpathlineto{\pgfqpoint{6.248713in}{2.805121in}}%
\pgfpathlineto{\pgfqpoint{6.250739in}{6.402844in}}%
\pgfpathlineto{\pgfqpoint{6.252764in}{2.719579in}}%
\pgfpathlineto{\pgfqpoint{6.254789in}{6.466761in}}%
\pgfpathlineto{\pgfqpoint{6.256814in}{2.676560in}}%
\pgfpathlineto{\pgfqpoint{6.258840in}{6.489924in}}%
\pgfpathlineto{\pgfqpoint{6.260865in}{2.672083in}}%
\pgfpathlineto{\pgfqpoint{6.262890in}{6.476853in}}%
\pgfpathlineto{\pgfqpoint{6.264916in}{2.701705in}}%
\pgfpathlineto{\pgfqpoint{6.266941in}{6.431483in}}%
\pgfpathlineto{\pgfqpoint{6.268966in}{2.762209in}}%
\pgfpathlineto{\pgfqpoint{6.270991in}{6.356333in}}%
\pgfpathlineto{\pgfqpoint{6.273017in}{2.851530in}}%
\pgfpathlineto{\pgfqpoint{6.275042in}{6.253449in}}%
\pgfpathlineto{\pgfqpoint{6.277067in}{2.967088in}}%
\pgfpathlineto{\pgfqpoint{6.279092in}{6.126533in}}%
\pgfpathlineto{\pgfqpoint{6.281118in}{3.103507in}}%
\pgfpathlineto{\pgfqpoint{6.283143in}{5.983063in}}%
\pgfpathlineto{\pgfqpoint{6.285168in}{3.250981in}}%
\pgfpathlineto{\pgfqpoint{6.287194in}{5.835150in}}%
\pgfpathlineto{\pgfqpoint{6.289219in}{3.395395in}}%
\pgfpathlineto{\pgfqpoint{6.291244in}{5.698333in}}%
\pgfpathlineto{\pgfqpoint{6.293269in}{3.520615in}}%
\pgfpathlineto{\pgfqpoint{6.295295in}{5.588327in}}%
\pgfpathlineto{\pgfqpoint{6.297320in}{3.612467in}}%
\pgfpathlineto{\pgfqpoint{6.299345in}{5.516629in}}%
\pgfpathlineto{\pgfqpoint{6.301370in}{3.663164in}}%
\pgfpathlineto{\pgfqpoint{6.303396in}{5.486479in}}%
\pgfpathlineto{\pgfqpoint{6.305421in}{3.674569in}}%
\pgfpathlineto{\pgfqpoint{6.307446in}{5.490717in}}%
\pgfpathlineto{\pgfqpoint{6.309472in}{3.658941in}}%
\pgfpathlineto{\pgfqpoint{6.311497in}{5.512584in}}%
\pgfpathlineto{\pgfqpoint{6.313522in}{3.636557in}}%
\pgfpathlineto{\pgfqpoint{6.315547in}{5.529578in}}%
\pgfpathlineto{\pgfqpoint{6.317573in}{3.630627in}}%
\pgfpathlineto{\pgfqpoint{6.319598in}{5.519416in}}%
\pgfpathlineto{\pgfqpoint{6.321623in}{3.660885in}}%
\pgfpathlineto{\pgfqpoint{6.323648in}{5.466394in}}%
\pgfpathlineto{\pgfqpoint{6.325674in}{3.737777in}}%
\pgfpathlineto{\pgfqpoint{6.327699in}{5.366195in}}%
\pgfpathlineto{\pgfqpoint{6.329724in}{3.859065in}}%
\pgfpathlineto{\pgfqpoint{6.331750in}{5.227563in}}%
\pgfpathlineto{\pgfqpoint{6.333775in}{4.010020in}}%
\pgfpathlineto{\pgfqpoint{6.335800in}{5.070234in}}%
\pgfpathlineto{\pgfqpoint{6.337825in}{4.167260in}}%
\pgfpathlineto{\pgfqpoint{6.339851in}{4.919605in}}%
\pgfpathlineto{\pgfqpoint{6.341876in}{4.305160in}}%
\pgfpathlineto{\pgfqpoint{6.343901in}{4.799717in}}%
\pgfpathlineto{\pgfqpoint{6.345926in}{4.402957in}}%
\pgfpathlineto{\pgfqpoint{6.347952in}{4.726611in}}%
\pgfpathlineto{\pgfqpoint{6.349977in}{4.450411in}}%
\pgfpathlineto{\pgfqpoint{6.352002in}{4.704090in}}%
\pgfpathlineto{\pgfqpoint{6.354028in}{4.450308in}}%
\pgfpathlineto{\pgfqpoint{6.356053in}{4.723129in}}%
\pgfpathlineto{\pgfqpoint{6.358078in}{4.417087in}}%
\pgfpathlineto{\pgfqpoint{6.360103in}{4.765100in}}%
\pgfpathlineto{\pgfqpoint{6.362129in}{4.372037in}}%
\pgfpathlineto{\pgfqpoint{6.364154in}{4.807766in}}%
\pgfpathlineto{\pgfqpoint{6.366179in}{4.336587in}}%
\pgfpathlineto{\pgfqpoint{6.368204in}{4.832162in}}%
\pgfpathlineto{\pgfqpoint{6.370230in}{4.325812in}}%
\pgfpathlineto{\pgfqpoint{6.372255in}{4.828187in}}%
\pgfpathlineto{\pgfqpoint{6.374280in}{4.344185in}}%
\pgfpathlineto{\pgfqpoint{6.376306in}{4.797171in}}%
\pgfpathlineto{\pgfqpoint{6.378331in}{4.384891in}}%
\pgfpathlineto{\pgfqpoint{6.380356in}{4.750628in}}%
\pgfpathlineto{\pgfqpoint{6.382381in}{4.432898in}}%
\pgfpathlineto{\pgfqpoint{6.384407in}{4.705631in}}%
\pgfpathlineto{\pgfqpoint{6.386432in}{4.470738in}}%
\pgfpathlineto{\pgfqpoint{6.388457in}{4.678360in}}%
\pgfpathlineto{\pgfqpoint{6.390482in}{4.485098in}}%
\pgfpathlineto{\pgfqpoint{6.392508in}{4.677937in}}%
\pgfpathlineto{\pgfqpoint{6.394533in}{4.472010in}}%
\pgfpathlineto{\pgfqpoint{6.396558in}{4.702653in}}%
\pgfpathlineto{\pgfqpoint{6.398584in}{4.438884in}}%
\pgfpathlineto{\pgfqpoint{6.400609in}{4.739881in}}%
\pgfpathlineto{\pgfqpoint{6.402634in}{4.402601in}}%
\pgfpathlineto{\pgfqpoint{6.404659in}{4.769853in}}%
\pgfpathlineto{\pgfqpoint{6.406685in}{4.384165in}}%
\pgfpathlineto{\pgfqpoint{6.408710in}{4.772145in}}%
\pgfpathlineto{\pgfqpoint{6.410735in}{4.401577in}}%
\pgfpathlineto{\pgfqpoint{6.412760in}{4.732880in}}%
\pgfpathlineto{\pgfqpoint{6.414786in}{4.463168in}}%
\pgfpathlineto{\pgfqpoint{6.416811in}{4.650300in}}%
\pgfpathlineto{\pgfqpoint{6.418836in}{4.563586in}}%
\pgfpathlineto{\pgfqpoint{6.420862in}{4.536864in}}%
\pgfpathlineto{\pgfqpoint{6.422887in}{4.683822in}}%
\pgfpathlineto{\pgfqpoint{6.424912in}{4.417063in}}%
\pgfpathlineto{\pgfqpoint{6.426937in}{4.795410in}}%
\pgfpathlineto{\pgfqpoint{6.428963in}{4.321492in}}%
\pgfpathlineto{\pgfqpoint{6.430988in}{4.867657in}}%
\pgfpathlineto{\pgfqpoint{6.433013in}{4.278880in}}%
\pgfpathlineto{\pgfqpoint{6.435038in}{4.875749in}}%
\pgfpathlineto{\pgfqpoint{6.437064in}{4.308440in}}%
\pgfpathlineto{\pgfqpoint{6.439089in}{4.807364in}}%
\pgfpathlineto{\pgfqpoint{6.441114in}{4.414803in}}%
\pgfpathlineto{\pgfqpoint{6.443140in}{4.665812in}}%
\pgfpathlineto{\pgfqpoint{6.447190in}{4.468847in}}%
\pgfpathlineto{\pgfqpoint{6.449215in}{4.801829in}}%
\pgfpathlineto{\pgfqpoint{6.451241in}{4.243612in}}%
\pgfpathlineto{\pgfqpoint{6.453266in}{5.030062in}}%
\pgfpathlineto{\pgfqpoint{6.455291in}{4.019357in}}%
\pgfpathlineto{\pgfqpoint{6.457316in}{5.244228in}}%
\pgfpathlineto{\pgfqpoint{6.459342in}{3.820201in}}%
\pgfpathlineto{\pgfqpoint{6.461367in}{5.424856in}}%
\pgfpathlineto{\pgfqpoint{6.463392in}{3.660128in}}%
\pgfpathlineto{\pgfqpoint{6.465418in}{5.563801in}}%
\pgfpathlineto{\pgfqpoint{6.467443in}{3.541592in}}%
\pgfpathlineto{\pgfqpoint{6.469468in}{5.663679in}}%
\pgfpathlineto{\pgfqpoint{6.471493in}{3.457925in}}%
\pgfpathlineto{\pgfqpoint{6.473519in}{5.733897in}}%
\pgfpathlineto{\pgfqpoint{6.475544in}{3.398470in}}%
\pgfpathlineto{\pgfqpoint{6.477569in}{5.784835in}}%
\pgfpathlineto{\pgfqpoint{6.479594in}{3.354549in}}%
\pgfpathlineto{\pgfqpoint{6.481620in}{5.822273in}}%
\pgfpathlineto{\pgfqpoint{6.483645in}{3.324144in}}%
\pgfpathlineto{\pgfqpoint{6.485670in}{5.844007in}}%
\pgfpathlineto{\pgfqpoint{6.487696in}{3.313698in}}%
\pgfpathlineto{\pgfqpoint{6.489721in}{5.839780in}}%
\pgfpathlineto{\pgfqpoint{6.491746in}{3.336450in}}%
\pgfpathlineto{\pgfqpoint{6.493771in}{5.794532in}}%
\pgfpathlineto{\pgfqpoint{6.495797in}{3.407903in}}%
\pgfpathlineto{\pgfqpoint{6.497822in}{5.693805in}}%
\pgfpathlineto{\pgfqpoint{6.499847in}{3.539997in}}%
\pgfpathlineto{\pgfqpoint{6.501872in}{5.529497in}}%
\pgfpathlineto{\pgfqpoint{6.503898in}{3.735938in}}%
\pgfpathlineto{\pgfqpoint{6.505923in}{5.304008in}}%
\pgfpathlineto{\pgfqpoint{6.507948in}{3.987419in}}%
\pgfpathlineto{\pgfqpoint{6.509974in}{5.031425in}}%
\pgfpathlineto{\pgfqpoint{6.511999in}{4.275123in}}%
\pgfpathlineto{\pgfqpoint{6.514024in}{4.735361in}}%
\pgfpathlineto{\pgfqpoint{6.518075in}{4.444137in}}%
\pgfpathlineto{\pgfqpoint{6.520100in}{4.850721in}}%
\pgfpathlineto{\pgfqpoint{6.522125in}{4.184804in}}%
\pgfpathlineto{\pgfqpoint{6.524150in}{5.085873in}}%
\pgfpathlineto{\pgfqpoint{6.526176in}{3.977802in}}%
\pgfpathlineto{\pgfqpoint{6.528201in}{5.262071in}}%
\pgfpathlineto{\pgfqpoint{6.530226in}{3.833729in}}%
\pgfpathlineto{\pgfqpoint{6.532252in}{5.373965in}}%
\pgfpathlineto{\pgfqpoint{6.534277in}{3.752943in}}%
\pgfpathlineto{\pgfqpoint{6.536302in}{5.425634in}}%
\pgfpathlineto{\pgfqpoint{6.538327in}{3.727726in}}%
\pgfpathlineto{\pgfqpoint{6.540353in}{5.427481in}}%
\pgfpathlineto{\pgfqpoint{6.542378in}{3.745995in}}%
\pgfpathlineto{\pgfqpoint{6.544403in}{5.392301in}}%
\pgfpathlineto{\pgfqpoint{6.546429in}{3.795096in}}%
\pgfpathlineto{\pgfqpoint{6.548454in}{5.331944in}}%
\pgfpathlineto{\pgfqpoint{6.550479in}{3.864413in}}%
\pgfpathlineto{\pgfqpoint{6.552504in}{5.255613in}}%
\pgfpathlineto{\pgfqpoint{6.554530in}{3.946096in}}%
\pgfpathlineto{\pgfqpoint{6.556555in}{5.170069in}}%
\pgfpathlineto{\pgfqpoint{6.558580in}{4.034050in}}%
\pgfpathlineto{\pgfqpoint{6.560605in}{5.081244in}}%
\pgfpathlineto{\pgfqpoint{6.562631in}{4.122006in}}%
\pgfpathlineto{\pgfqpoint{6.564656in}{4.996173in}}%
\pgfpathlineto{\pgfqpoint{6.566681in}{4.201875in}}%
\pgfpathlineto{\pgfqpoint{6.568707in}{4.924094in}}%
\pgfpathlineto{\pgfqpoint{6.570732in}{4.263397in}}%
\pgfpathlineto{\pgfqpoint{6.572757in}{4.875929in}}%
\pgfpathlineto{\pgfqpoint{6.574782in}{4.295555in}}%
\pgfpathlineto{\pgfqpoint{6.576808in}{4.862069in}}%
\pgfpathlineto{\pgfqpoint{6.578833in}{4.289400in}}%
\pgfpathlineto{\pgfqpoint{6.580858in}{4.889189in}}%
\pgfpathlineto{\pgfqpoint{6.582883in}{4.241287in}}%
\pgfpathlineto{\pgfqpoint{6.584909in}{4.957304in}}%
\pgfpathlineto{\pgfqpoint{6.586934in}{4.155198in}}%
\pgfpathlineto{\pgfqpoint{6.588959in}{5.058358in}}%
\pgfpathlineto{\pgfqpoint{6.590985in}{4.043029in}}%
\pgfpathlineto{\pgfqpoint{6.593010in}{5.177171in}}%
\pgfpathlineto{\pgfqpoint{6.595035in}{3.922391in}}%
\pgfpathlineto{\pgfqpoint{6.597060in}{5.294789in}}%
\pgfpathlineto{\pgfqpoint{6.599086in}{3.812332in}}%
\pgfpathlineto{\pgfqpoint{6.601111in}{5.393379in}}%
\pgfpathlineto{\pgfqpoint{6.603136in}{3.728202in}}%
\pgfpathlineto{\pgfqpoint{6.605161in}{5.461195in}}%
\pgfpathlineto{\pgfqpoint{6.607187in}{3.677277in}}%
\pgfpathlineto{\pgfqpoint{6.609212in}{5.495973in}}%
\pgfpathlineto{\pgfqpoint{6.611237in}{3.656643in}}%
\pgfpathlineto{\pgfqpoint{6.613263in}{5.505562in}}%
\pgfpathlineto{\pgfqpoint{6.615288in}{3.654157in}}%
\pgfpathlineto{\pgfqpoint{6.617313in}{5.505404in}}%
\pgfpathlineto{\pgfqpoint{6.619338in}{3.652351in}}%
\pgfpathlineto{\pgfqpoint{6.621364in}{5.513539in}}%
\pgfpathlineto{\pgfqpoint{6.623389in}{3.634140in}}%
\pgfpathlineto{\pgfqpoint{6.625414in}{5.544626in}}%
\pgfpathlineto{\pgfqpoint{6.627439in}{3.588587in}}%
\pgfpathlineto{\pgfqpoint{6.629465in}{5.604861in}}%
\pgfpathlineto{\pgfqpoint{6.631490in}{3.514889in}}%
\pgfpathlineto{\pgfqpoint{6.633515in}{5.689425in}}%
\pgfpathlineto{\pgfqpoint{6.635541in}{3.423268in}}%
\pgfpathlineto{\pgfqpoint{6.637566in}{5.783346in}}%
\pgfpathlineto{\pgfqpoint{6.639591in}{3.332412in}}%
\pgfpathlineto{\pgfqpoint{6.641616in}{5.865560in}}%
\pgfpathlineto{\pgfqpoint{6.643642in}{3.264214in}}%
\pgfpathlineto{\pgfqpoint{6.645667in}{5.914969in}}%
\pgfpathlineto{\pgfqpoint{6.647692in}{3.237402in}}%
\pgfpathlineto{\pgfqpoint{6.649717in}{5.916627in}}%
\pgfpathlineto{\pgfqpoint{6.651743in}{3.262015in}}%
\pgfpathlineto{\pgfqpoint{6.653768in}{5.866153in}}%
\pgfpathlineto{\pgfqpoint{6.655793in}{3.336447in}}%
\pgfpathlineto{\pgfqpoint{6.657819in}{5.771011in}}%
\pgfpathlineto{\pgfqpoint{6.659844in}{3.447951in}}%
\pgfpathlineto{\pgfqpoint{6.661869in}{5.648260in}}%
\pgfpathlineto{\pgfqpoint{6.663894in}{3.576449in}}%
\pgfpathlineto{\pgfqpoint{6.665920in}{5.519487in}}%
\pgfpathlineto{\pgfqpoint{6.667945in}{3.700462in}}%
\pgfpathlineto{\pgfqpoint{6.669970in}{5.404468in}}%
\pgfpathlineto{\pgfqpoint{6.671995in}{3.803358in}}%
\pgfpathlineto{\pgfqpoint{6.674021in}{5.315505in}}%
\pgfpathlineto{\pgfqpoint{6.676046in}{3.878005in}}%
\pgfpathlineto{\pgfqpoint{6.678071in}{5.254133in}}%
\pgfpathlineto{\pgfqpoint{6.680097in}{3.928449in}}%
\pgfpathlineto{\pgfqpoint{6.682122in}{5.211180in}}%
\pgfpathlineto{\pgfqpoint{6.684147in}{3.968135in}}%
\pgfpathlineto{\pgfqpoint{6.686172in}{5.170110in}}%
\pgfpathlineto{\pgfqpoint{6.688198in}{4.015272in}}%
\pgfpathlineto{\pgfqpoint{6.690223in}{5.112584in}}%
\pgfpathlineto{\pgfqpoint{6.692248in}{4.086790in}}%
\pgfpathlineto{\pgfqpoint{6.694273in}{5.024497in}}%
\pgfpathlineto{\pgfqpoint{6.696299in}{4.192783in}}%
\pgfpathlineto{\pgfqpoint{6.698324in}{4.900617in}}%
\pgfpathlineto{\pgfqpoint{6.700349in}{4.333166in}}%
\pgfpathlineto{\pgfqpoint{6.702375in}{4.746373in}}%
\pgfpathlineto{\pgfqpoint{6.704400in}{4.497579in}}%
\pgfpathlineto{\pgfqpoint{6.708450in}{4.668585in}}%
\pgfpathlineto{\pgfqpoint{6.710476in}{4.409221in}}%
\pgfpathlineto{\pgfqpoint{6.712501in}{4.827137in}}%
\pgfpathlineto{\pgfqpoint{6.714526in}{4.262869in}}%
\pgfpathlineto{\pgfqpoint{6.716551in}{4.958608in}}%
\pgfpathlineto{\pgfqpoint{6.718577in}{4.147690in}}%
\pgfpathlineto{\pgfqpoint{6.720602in}{5.057474in}}%
\pgfpathlineto{\pgfqpoint{6.722627in}{4.063757in}}%
\pgfpathlineto{\pgfqpoint{6.724653in}{5.129148in}}%
\pgfpathlineto{\pgfqpoint{6.726678in}{4.000582in}}%
\pgfpathlineto{\pgfqpoint{6.728703in}{5.188370in}}%
\pgfpathlineto{\pgfqpoint{6.730728in}{3.940344in}}%
\pgfpathlineto{\pgfqpoint{6.732754in}{5.254618in}}%
\pgfpathlineto{\pgfqpoint{6.734779in}{3.863475in}}%
\pgfpathlineto{\pgfqpoint{6.736804in}{5.345953in}}%
\pgfpathlineto{\pgfqpoint{6.738829in}{3.754915in}}%
\pgfpathlineto{\pgfqpoint{6.740855in}{5.473165in}}%
\pgfpathlineto{\pgfqpoint{6.742880in}{3.609103in}}%
\pgfpathlineto{\pgfqpoint{6.744905in}{5.636013in}}%
\pgfpathlineto{\pgfqpoint{6.746931in}{3.432225in}}%
\pgfpathlineto{\pgfqpoint{6.748956in}{5.822653in}}%
\pgfpathlineto{\pgfqpoint{6.750981in}{3.241079in}}%
\pgfpathlineto{\pgfqpoint{6.753006in}{6.012395in}}%
\pgfpathlineto{\pgfqpoint{6.755032in}{3.058925in}}%
\pgfpathlineto{\pgfqpoint{6.757057in}{6.180898in}}%
\pgfpathlineto{\pgfqpoint{6.759082in}{2.909638in}}%
\pgfpathlineto{\pgfqpoint{6.761107in}{6.306238in}}%
\pgfpathlineto{\pgfqpoint{6.763133in}{2.811880in}}%
\pgfpathlineto{\pgfqpoint{6.765158in}{6.374056in}}%
\pgfpathlineto{\pgfqpoint{6.767183in}{2.774996in}}%
\pgfpathlineto{\pgfqpoint{6.769209in}{6.380357in}}%
\pgfpathlineto{\pgfqpoint{6.771234in}{2.797695in}}%
\pgfpathlineto{\pgfqpoint{6.773259in}{6.331276in}}%
\pgfpathlineto{\pgfqpoint{6.775284in}{2.869768in}}%
\pgfpathlineto{\pgfqpoint{6.777310in}{6.240063in}}%
\pgfpathlineto{\pgfqpoint{6.779335in}{2.976133in}}%
\pgfpathlineto{\pgfqpoint{6.781360in}{6.122350in}}%
\pgfpathlineto{\pgfqpoint{6.783385in}{3.101866in}}%
\pgfpathlineto{\pgfqpoint{6.785411in}{5.991219in}}%
\pgfpathlineto{\pgfqpoint{6.787436in}{3.236641in}}%
\pgfpathlineto{\pgfqpoint{6.789461in}{5.853604in}}%
\pgfpathlineto{\pgfqpoint{6.791487in}{3.377233in}}%
\pgfpathlineto{\pgfqpoint{6.793512in}{5.709052in}}%
\pgfpathlineto{\pgfqpoint{6.795537in}{3.527405in}}%
\pgfpathlineto{\pgfqpoint{6.797562in}{5.551157in}}%
\pgfpathlineto{\pgfqpoint{6.799588in}{3.695294in}}%
\pgfpathlineto{\pgfqpoint{6.801613in}{5.371126in}}%
\pgfpathlineto{\pgfqpoint{6.803638in}{3.889201in}}%
\pgfpathlineto{\pgfqpoint{6.805663in}{5.162281in}}%
\pgfpathlineto{\pgfqpoint{6.807689in}{4.113163in}}%
\pgfpathlineto{\pgfqpoint{6.809714in}{4.924050in}}%
\pgfpathlineto{\pgfqpoint{6.811739in}{4.363727in}}%
\pgfpathlineto{\pgfqpoint{6.813765in}{4.664155in}}%
\pgfpathlineto{\pgfqpoint{6.815790in}{4.628991in}}%
\pgfpathlineto{\pgfqpoint{6.817815in}{4.398259in}}%
\pgfpathlineto{\pgfqpoint{6.819840in}{4.890254in}}%
\pgfpathlineto{\pgfqpoint{6.821866in}{4.147128in}}%
\pgfpathlineto{\pgfqpoint{6.823891in}{5.125834in}}%
\pgfpathlineto{\pgfqpoint{6.825916in}{3.932118in}}%
\pgfpathlineto{\pgfqpoint{6.827941in}{5.315949in}}%
\pgfpathlineto{\pgfqpoint{6.829967in}{3.770286in}}%
\pgfpathlineto{\pgfqpoint{6.831992in}{5.447234in}}%
\pgfpathlineto{\pgfqpoint{6.834017in}{3.670581in}}%
\pgfpathlineto{\pgfqpoint{6.836043in}{5.515579in}}%
\pgfpathlineto{\pgfqpoint{6.838068in}{3.632186in}}%
\pgfpathlineto{\pgfqpoint{6.840093in}{5.526483in}}%
\pgfpathlineto{\pgfqpoint{6.842118in}{3.645482in}}%
\pgfpathlineto{\pgfqpoint{6.844144in}{5.492851in}}%
\pgfpathlineto{\pgfqpoint{6.846169in}{3.695309in}}%
\pgfpathlineto{\pgfqpoint{6.848194in}{5.430942in}}%
\pgfpathlineto{\pgfqpoint{6.850219in}{3.765502in}}%
\pgfpathlineto{\pgfqpoint{6.852245in}{5.355692in}}%
\pgfpathlineto{\pgfqpoint{6.854270in}{3.843351in}}%
\pgfpathlineto{\pgfqpoint{6.856295in}{5.276805in}}%
\pgfpathlineto{\pgfqpoint{6.858321in}{3.922660in}}%
\pgfpathlineto{\pgfqpoint{6.860346in}{5.196772in}}%
\pgfpathlineto{\pgfqpoint{6.862371in}{4.004529in}}%
\pgfpathlineto{\pgfqpoint{6.864396in}{5.111323in}}%
\pgfpathlineto{\pgfqpoint{6.866422in}{4.095704in}}%
\pgfpathlineto{\pgfqpoint{6.868447in}{5.012138in}}%
\pgfpathlineto{\pgfqpoint{6.870472in}{4.205048in}}%
\pgfpathlineto{\pgfqpoint{6.872497in}{4.890893in}}%
\pgfpathlineto{\pgfqpoint{6.874523in}{4.339291in}}%
\pgfpathlineto{\pgfqpoint{6.876548in}{4.743386in}}%
\pgfpathlineto{\pgfqpoint{6.878573in}{4.499369in}}%
\pgfpathlineto{\pgfqpoint{6.882624in}{4.678479in}}%
\pgfpathlineto{\pgfqpoint{6.884649in}{4.388756in}}%
\pgfpathlineto{\pgfqpoint{6.886674in}{4.862421in}}%
\pgfpathlineto{\pgfqpoint{6.888700in}{4.209398in}}%
\pgfpathlineto{\pgfqpoint{6.890725in}{5.032091in}}%
\pgfpathlineto{\pgfqpoint{6.892750in}{4.054514in}}%
\pgfpathlineto{\pgfqpoint{6.894775in}{5.167362in}}%
\pgfpathlineto{\pgfqpoint{6.896801in}{3.943164in}}%
\pgfpathlineto{\pgfqpoint{6.898826in}{5.251215in}}%
\pgfpathlineto{\pgfqpoint{6.900851in}{3.889495in}}%
\pgfpathlineto{\pgfqpoint{6.902877in}{5.273009in}}%
\pgfpathlineto{\pgfqpoint{6.904902in}{3.900231in}}%
\pgfpathlineto{\pgfqpoint{6.906927in}{5.230108in}}%
\pgfpathlineto{\pgfqpoint{6.908952in}{3.973981in}}%
\pgfpathlineto{\pgfqpoint{6.910978in}{5.127658in}}%
\pgfpathlineto{\pgfqpoint{6.913003in}{4.102297in}}%
\pgfpathlineto{\pgfqpoint{6.915028in}{4.976830in}}%
\pgfpathlineto{\pgfqpoint{6.917053in}{4.271936in}}%
\pgfpathlineto{\pgfqpoint{6.919079in}{4.792259in}}%
\pgfpathlineto{\pgfqpoint{6.921104in}{4.467530in}}%
\pgfpathlineto{\pgfqpoint{6.925155in}{4.673924in}}%
\pgfpathlineto{\pgfqpoint{6.927180in}{4.382820in}}%
\pgfpathlineto{\pgfqpoint{6.929205in}{4.877761in}}%
\pgfpathlineto{\pgfqpoint{6.931230in}{4.184491in}}%
\pgfpathlineto{\pgfqpoint{6.933256in}{5.068213in}}%
\pgfpathlineto{\pgfqpoint{6.935281in}{4.003956in}}%
\pgfpathlineto{\pgfqpoint{6.937306in}{5.237112in}}%
\pgfpathlineto{\pgfqpoint{6.939331in}{3.848099in}}%
\pgfpathlineto{\pgfqpoint{6.941357in}{5.378827in}}%
\pgfpathlineto{\pgfqpoint{6.943382in}{3.721323in}}%
\pgfpathlineto{\pgfqpoint{6.945407in}{5.490171in}}%
\pgfpathlineto{\pgfqpoint{6.947433in}{3.625594in}}%
\pgfpathlineto{\pgfqpoint{6.949458in}{5.570415in}}%
\pgfpathlineto{\pgfqpoint{6.951483in}{3.560397in}}%
\pgfpathlineto{\pgfqpoint{6.953508in}{5.621296in}}%
\pgfpathlineto{\pgfqpoint{6.955534in}{3.522834in}}%
\pgfpathlineto{\pgfqpoint{6.957559in}{5.646765in}}%
\pgfpathlineto{\pgfqpoint{6.959584in}{3.508066in}}%
\pgfpathlineto{\pgfqpoint{6.961609in}{5.652326in}}%
\pgfpathlineto{\pgfqpoint{6.963635in}{3.510194in}}%
\pgfpathlineto{\pgfqpoint{6.965660in}{5.643969in}}%
\pgfpathlineto{\pgfqpoint{6.967685in}{3.523453in}}%
\pgfpathlineto{\pgfqpoint{6.969711in}{5.626935in}}%
\pgfpathlineto{\pgfqpoint{6.971736in}{3.543390in}}%
\pgfpathlineto{\pgfqpoint{6.973761in}{5.604683in}}%
\pgfpathlineto{\pgfqpoint{6.975786in}{3.567659in}}%
\pgfpathlineto{\pgfqpoint{6.977812in}{5.578421in}}%
\pgfpathlineto{\pgfqpoint{6.979837in}{3.596126in}}%
\pgfpathlineto{\pgfqpoint{6.981862in}{5.547365in}}%
\pgfpathlineto{\pgfqpoint{6.983887in}{3.630256in}}%
\pgfpathlineto{\pgfqpoint{6.985913in}{5.509654in}}%
\pgfpathlineto{\pgfqpoint{6.987938in}{3.671995in}}%
\pgfpathlineto{\pgfqpoint{6.989963in}{5.463571in}}%
\pgfpathlineto{\pgfqpoint{6.991989in}{3.722548in}}%
\pgfpathlineto{\pgfqpoint{6.994014in}{5.408646in}}%
\pgfpathlineto{\pgfqpoint{6.996039in}{3.781504in}}%
\pgfpathlineto{\pgfqpoint{6.998064in}{5.346238in}}%
\pgfpathlineto{\pgfqpoint{7.000090in}{3.846575in}}%
\pgfpathlineto{\pgfqpoint{7.002115in}{5.279457in}}%
\pgfpathlineto{\pgfqpoint{7.004140in}{3.914006in}}%
\pgfpathlineto{\pgfqpoint{7.006165in}{5.212484in}}%
\pgfpathlineto{\pgfqpoint{7.008191in}{3.979430in}}%
\pgfpathlineto{\pgfqpoint{7.010216in}{5.149627in}}%
\pgfpathlineto{\pgfqpoint{7.012241in}{4.038818in}}%
\pgfpathlineto{\pgfqpoint{7.014267in}{5.094465in}}%
\pgfpathlineto{\pgfqpoint{7.016292in}{4.089156in}}%
\pgfpathlineto{\pgfqpoint{7.018317in}{5.049392in}}%
\pgfpathlineto{\pgfqpoint{7.020342in}{4.128659in}}%
\pgfpathlineto{\pgfqpoint{7.022368in}{5.015649in}}%
\pgfpathlineto{\pgfqpoint{7.024393in}{4.156535in}}%
\pgfpathlineto{\pgfqpoint{7.026418in}{4.993697in}}%
\pgfpathlineto{\pgfqpoint{7.028443in}{4.172536in}}%
\pgfpathlineto{\pgfqpoint{7.030469in}{4.983660in}}%
\pgfpathlineto{\pgfqpoint{7.032494in}{4.176612in}}%
\pgfpathlineto{\pgfqpoint{7.034519in}{4.985511in}}%
\pgfpathlineto{\pgfqpoint{7.036545in}{4.168930in}}%
\pgfpathlineto{\pgfqpoint{7.038570in}{4.998817in}}%
\pgfpathlineto{\pgfqpoint{7.040595in}{4.150369in}}%
\pgfpathlineto{\pgfqpoint{7.042620in}{5.022046in}}%
\pgfpathlineto{\pgfqpoint{7.044646in}{4.123336in}}%
\pgfpathlineto{\pgfqpoint{7.046671in}{5.051699in}}%
\pgfpathlineto{\pgfqpoint{7.048696in}{4.092600in}}%
\pgfpathlineto{\pgfqpoint{7.050721in}{5.081619in}}%
\pgfpathlineto{\pgfqpoint{7.052747in}{4.065740in}}%
\pgfpathlineto{\pgfqpoint{7.054772in}{5.102869in}}%
\pgfpathlineto{\pgfqpoint{7.056797in}{4.052884in}}%
\pgfpathlineto{\pgfqpoint{7.058823in}{5.104407in}}%
\pgfpathlineto{\pgfqpoint{7.060848in}{4.065612in}}%
\pgfpathlineto{\pgfqpoint{7.062873in}{5.074569in}}%
\pgfpathlineto{\pgfqpoint{7.064898in}{4.115162in}}%
\pgfpathlineto{\pgfqpoint{7.066924in}{5.003084in}}%
\pgfpathlineto{\pgfqpoint{7.068949in}{4.210299in}}%
\pgfpathlineto{\pgfqpoint{7.070974in}{4.883194in}}%
\pgfpathlineto{\pgfqpoint{7.072999in}{4.355344in}}%
\pgfpathlineto{\pgfqpoint{7.075025in}{4.713348in}}%
\pgfpathlineto{\pgfqpoint{7.077050in}{4.548862in}}%
\pgfpathlineto{\pgfqpoint{7.079075in}{4.498047in}}%
\pgfpathlineto{\pgfqpoint{7.081101in}{4.783345in}}%
\pgfpathlineto{\pgfqpoint{7.083126in}{4.247610in}}%
\pgfpathlineto{\pgfqpoint{7.085151in}{5.045990in}}%
\pgfpathlineto{\pgfqpoint{7.087176in}{3.976887in}}%
\pgfpathlineto{\pgfqpoint{7.089202in}{5.320421in}}%
\pgfpathlineto{\pgfqpoint{7.091227in}{3.703204in}}%
\pgfpathlineto{\pgfqpoint{7.093252in}{5.588967in}}%
\pgfpathlineto{\pgfqpoint{7.095277in}{3.443981in}}%
\pgfpathlineto{\pgfqpoint{7.097303in}{5.835014in}}%
\pgfpathlineto{\pgfqpoint{7.099328in}{3.214531in}}%
\pgfpathlineto{\pgfqpoint{7.101353in}{6.044955in}}%
\pgfpathlineto{\pgfqpoint{7.103379in}{3.026455in}}%
\pgfpathlineto{\pgfqpoint{7.105404in}{6.209381in}}%
\pgfpathlineto{\pgfqpoint{7.107429in}{2.886901in}}%
\pgfpathlineto{\pgfqpoint{7.109454in}{6.323362in}}%
\pgfpathlineto{\pgfqpoint{7.111480in}{2.798729in}}%
\pgfpathlineto{\pgfqpoint{7.113505in}{6.385879in}}%
\pgfpathlineto{\pgfqpoint{7.115530in}{2.761412in}}%
\pgfpathlineto{\pgfqpoint{7.117555in}{6.398663in}}%
\pgfpathlineto{\pgfqpoint{7.119581in}{2.772367in}}%
\pgfpathlineto{\pgfqpoint{7.121606in}{6.364805in}}%
\pgfpathlineto{\pgfqpoint{7.123631in}{2.828315in}}%
\pgfpathlineto{\pgfqpoint{7.125657in}{6.287509in}}%
\pgfpathlineto{\pgfqpoint{7.127682in}{2.926320in}}%
\pgfpathlineto{\pgfqpoint{7.129707in}{6.169322in}}%
\pgfpathlineto{\pgfqpoint{7.131732in}{3.064261in}}%
\pgfpathlineto{\pgfqpoint{7.133758in}{6.011988in}}%
\pgfpathlineto{\pgfqpoint{7.135783in}{3.240646in}}%
\pgfpathlineto{\pgfqpoint{7.137808in}{5.816935in}}%
\pgfpathlineto{\pgfqpoint{7.139833in}{3.453876in}}%
\pgfpathlineto{\pgfqpoint{7.141859in}{5.586204in}}%
\pgfpathlineto{\pgfqpoint{7.143884in}{3.701185in}}%
\pgfpathlineto{\pgfqpoint{7.145909in}{5.323547in}}%
\pgfpathlineto{\pgfqpoint{7.147935in}{3.977603in}}%
\pgfpathlineto{\pgfqpoint{7.149960in}{5.035340in}}%
\pgfpathlineto{\pgfqpoint{7.151985in}{4.275235in}}%
\pgfpathlineto{\pgfqpoint{7.154010in}{4.731024in}}%
\pgfpathlineto{\pgfqpoint{7.158061in}{4.422885in}}%
\pgfpathlineto{\pgfqpoint{7.160086in}{4.887936in}}%
\pgfpathlineto{\pgfqpoint{7.162111in}{4.125139in}}%
\pgfpathlineto{\pgfqpoint{7.164137in}{5.174944in}}%
\pgfpathlineto{\pgfqpoint{7.166162in}{3.852461in}}%
\pgfpathlineto{\pgfqpoint{7.168187in}{5.429920in}}%
\pgfpathlineto{\pgfqpoint{7.170213in}{3.618231in}}%
\pgfpathlineto{\pgfqpoint{7.172238in}{5.640783in}}%
\pgfpathlineto{\pgfqpoint{7.174263in}{3.432855in}}%
\pgfpathlineto{\pgfqpoint{7.176288in}{5.799115in}}%
\pgfpathlineto{\pgfqpoint{7.178314in}{3.302528in}}%
\pgfpathlineto{\pgfqpoint{7.180339in}{5.901076in}}%
\pgfpathlineto{\pgfqpoint{7.182364in}{3.228709in}}%
\pgfpathlineto{\pgfqpoint{7.184389in}{5.947515in}}%
\pgfpathlineto{\pgfqpoint{7.186415in}{3.208418in}}%
\pgfpathlineto{\pgfqpoint{7.188440in}{5.943276in}}%
\pgfpathlineto{\pgfqpoint{7.190465in}{3.235283in}}%
\pgfpathlineto{\pgfqpoint{7.192491in}{5.895871in}}%
\pgfpathlineto{\pgfqpoint{7.194516in}{3.301066in}}%
\pgfpathlineto{\pgfqpoint{7.196541in}{5.813844in}}%
\pgfpathlineto{\pgfqpoint{7.198566in}{3.397314in}}%
\pgfpathlineto{\pgfqpoint{7.200592in}{5.705211in}}%
\pgfpathlineto{\pgfqpoint{7.202617in}{3.516736in}}%
\pgfpathlineto{\pgfqpoint{7.204642in}{5.576325in}}%
\pgfpathlineto{\pgfqpoint{7.206667in}{3.654042in}}%
\pgfpathlineto{\pgfqpoint{7.208693in}{5.431379in}}%
\pgfpathlineto{\pgfqpoint{7.210718in}{3.806084in}}%
\pgfpathlineto{\pgfqpoint{7.212743in}{5.272602in}}%
\pgfpathlineto{\pgfqpoint{7.214769in}{3.971355in}}%
\pgfpathlineto{\pgfqpoint{7.216794in}{5.101028in}}%
\pgfpathlineto{\pgfqpoint{7.218819in}{4.149010in}}%
\pgfpathlineto{\pgfqpoint{7.220844in}{4.917623in}}%
\pgfpathlineto{\pgfqpoint{7.222870in}{4.337653in}}%
\pgfpathlineto{\pgfqpoint{7.224895in}{4.724503in}}%
\pgfpathlineto{\pgfqpoint{7.226920in}{4.534183in}}%
\pgfpathlineto{\pgfqpoint{7.228945in}{4.525978in}}%
\pgfpathlineto{\pgfqpoint{7.230971in}{4.732917in}}%
\pgfpathlineto{\pgfqpoint{7.232996in}{4.329197in}}%
\pgfpathlineto{\pgfqpoint{7.235021in}{4.925224in}}%
\pgfpathlineto{\pgfqpoint{7.237047in}{4.144205in}}%
\pgfpathlineto{\pgfqpoint{7.239072in}{5.099800in}}%
\pgfpathlineto{\pgfqpoint{7.241097in}{3.983321in}}%
\pgfpathlineto{\pgfqpoint{7.243122in}{5.243642in}}%
\pgfpathlineto{\pgfqpoint{7.245148in}{3.859824in}}%
\pgfpathlineto{\pgfqpoint{7.247173in}{5.343668in}}%
\pgfpathlineto{\pgfqpoint{7.249198in}{3.786081in}}%
\pgfpathlineto{\pgfqpoint{7.251223in}{5.388770in}}%
\pgfpathlineto{\pgfqpoint{7.253249in}{3.771400in}}%
\pgfpathlineto{\pgfqpoint{7.255274in}{5.371944in}}%
\pgfpathlineto{\pgfqpoint{7.257299in}{3.820034in}}%
\pgfpathlineto{\pgfqpoint{7.259325in}{5.292046in}}%
\pgfpathlineto{\pgfqpoint{7.261350in}{3.929781in}}%
\pgfpathlineto{\pgfqpoint{7.263375in}{5.154723in}}%
\pgfpathlineto{\pgfqpoint{7.265400in}{4.091602in}}%
\pgfpathlineto{\pgfqpoint{7.267426in}{4.972196in}}%
\pgfpathlineto{\pgfqpoint{7.269451in}{4.290453in}}%
\pgfpathlineto{\pgfqpoint{7.271476in}{4.761836in}}%
\pgfpathlineto{\pgfqpoint{7.273501in}{4.507254in}}%
\pgfpathlineto{\pgfqpoint{7.275527in}{4.543727in}}%
\pgfpathlineto{\pgfqpoint{7.277552in}{4.721664in}}%
\pgfpathlineto{\pgfqpoint{7.279577in}{4.337712in}}%
\pgfpathlineto{\pgfqpoint{7.281603in}{4.915065in}}%
\pgfpathlineto{\pgfqpoint{7.283628in}{4.160526in}}%
\pgfpathlineto{\pgfqpoint{7.285653in}{5.073158in}}%
\pgfpathlineto{\pgfqpoint{7.287678in}{4.023610in}}%
\pgfpathlineto{\pgfqpoint{7.289704in}{5.187637in}}%
\pgfpathlineto{\pgfqpoint{7.291729in}{3.932013in}}%
\pgfpathlineto{\pgfqpoint{7.293754in}{5.256673in}}%
\pgfpathlineto{\pgfqpoint{7.295779in}{3.884529in}}%
\pgfpathlineto{\pgfqpoint{7.297805in}{5.284199in}}%
\pgfpathlineto{\pgfqpoint{7.299830in}{3.874904in}}%
\pgfpathlineto{\pgfqpoint{7.301855in}{5.278309in}}%
\pgfpathlineto{\pgfqpoint{7.303881in}{3.893731in}}%
\pgfpathlineto{\pgfqpoint{7.305906in}{5.249184in}}%
\pgfpathlineto{\pgfqpoint{7.307931in}{3.930575in}}%
\pgfpathlineto{\pgfqpoint{7.309956in}{5.207038in}}%
\pgfpathlineto{\pgfqpoint{7.311982in}{3.975855in}}%
\pgfpathlineto{\pgfqpoint{7.314007in}{5.160477in}}%
\pgfpathlineto{\pgfqpoint{7.316032in}{4.022199in}}%
\pgfpathlineto{\pgfqpoint{7.318057in}{5.115474in}}%
\pgfpathlineto{\pgfqpoint{7.320083in}{4.065113in}}%
\pgfpathlineto{\pgfqpoint{7.322108in}{5.075038in}}%
\pgfpathlineto{\pgfqpoint{7.324133in}{4.103011in}}%
\pgfpathlineto{\pgfqpoint{7.326159in}{5.039451in}}%
\pgfpathlineto{\pgfqpoint{7.328184in}{4.136750in}}%
\pgfpathlineto{\pgfqpoint{7.330209in}{5.006914in}}%
\pgfpathlineto{\pgfqpoint{7.332234in}{4.168860in}}%
\pgfpathlineto{\pgfqpoint{7.334260in}{4.974384in}}%
\pgfpathlineto{\pgfqpoint{7.336285in}{4.202673in}}%
\pgfpathlineto{\pgfqpoint{7.338310in}{4.938457in}}%
\pgfpathlineto{\pgfqpoint{7.340335in}{4.241464in}}%
\pgfpathlineto{\pgfqpoint{7.342361in}{4.896178in}}%
\pgfpathlineto{\pgfqpoint{7.344386in}{4.287693in}}%
\pgfpathlineto{\pgfqpoint{7.346411in}{4.845736in}}%
\pgfpathlineto{\pgfqpoint{7.348437in}{4.342377in}}%
\pgfpathlineto{\pgfqpoint{7.350462in}{4.787039in}}%
\pgfpathlineto{\pgfqpoint{7.352487in}{4.404580in}}%
\pgfpathlineto{\pgfqpoint{7.354512in}{4.722131in}}%
\pgfpathlineto{\pgfqpoint{7.356538in}{4.471090in}}%
\pgfpathlineto{\pgfqpoint{7.358563in}{4.655425in}}%
\pgfpathlineto{\pgfqpoint{7.360588in}{4.536304in}}%
\pgfpathlineto{\pgfqpoint{7.362613in}{4.593649in}}%
\pgfpathlineto{\pgfqpoint{7.364639in}{4.592478in}}%
\pgfpathlineto{\pgfqpoint{7.366664in}{4.545395in}}%
\pgfpathlineto{\pgfqpoint{7.368689in}{4.630409in}}%
\pgfpathlineto{\pgfqpoint{7.370715in}{4.520187in}}%
\pgfpathlineto{\pgfqpoint{7.372740in}{4.640601in}}%
\pgfpathlineto{\pgfqpoint{7.374765in}{4.527093in}}%
\pgfpathlineto{\pgfqpoint{7.376790in}{4.614833in}}%
\pgfpathlineto{\pgfqpoint{7.380841in}{4.547851in}}%
\pgfpathlineto{\pgfqpoint{7.382866in}{4.661286in}}%
\pgfpathlineto{\pgfqpoint{7.384891in}{4.438793in}}%
\pgfpathlineto{\pgfqpoint{7.386917in}{4.790102in}}%
\pgfpathlineto{\pgfqpoint{7.388942in}{4.291931in}}%
\pgfpathlineto{\pgfqpoint{7.390967in}{4.952710in}}%
\pgfpathlineto{\pgfqpoint{7.392993in}{4.116378in}}%
\pgfpathlineto{\pgfqpoint{7.395018in}{5.138024in}}%
\pgfpathlineto{\pgfqpoint{7.397043in}{3.924726in}}%
\pgfpathlineto{\pgfqpoint{7.399068in}{5.332508in}}%
\pgfpathlineto{\pgfqpoint{7.401094in}{3.730841in}}%
\pgfpathlineto{\pgfqpoint{7.403119in}{5.522594in}}%
\pgfpathlineto{\pgfqpoint{7.405144in}{3.547388in}}%
\pgfpathlineto{\pgfqpoint{7.407169in}{5.697053in}}%
\pgfpathlineto{\pgfqpoint{7.409195in}{3.383734in}}%
\pgfpathlineto{\pgfqpoint{7.411220in}{5.848674in}}%
\pgfpathlineto{\pgfqpoint{7.413245in}{3.244797in}}%
\pgfpathlineto{\pgfqpoint{7.415271in}{5.974813in}}%
\pgfpathlineto{\pgfqpoint{7.417296in}{3.131109in}}%
\pgfpathlineto{\pgfqpoint{7.419321in}{6.076765in}}%
\pgfpathlineto{\pgfqpoint{7.421346in}{3.039924in}}%
\pgfpathlineto{\pgfqpoint{7.423372in}{6.158298in}}%
\pgfpathlineto{\pgfqpoint{7.425397in}{2.966883in}}%
\pgfpathlineto{\pgfqpoint{7.427422in}{6.223967in}}%
\pgfpathlineto{\pgfqpoint{7.429448in}{2.907561in}}%
\pgfpathlineto{\pgfqpoint{7.431473in}{6.277841in}}%
\pgfpathlineto{\pgfqpoint{7.433498in}{2.858358in}}%
\pgfpathlineto{\pgfqpoint{7.435523in}{6.323050in}}%
\pgfpathlineto{\pgfqpoint{7.437549in}{2.816517in}}%
\pgfpathlineto{\pgfqpoint{7.439574in}{6.362153in}}%
\pgfpathlineto{\pgfqpoint{7.441599in}{2.779466in}}%
\pgfpathlineto{\pgfqpoint{7.443624in}{6.397942in}}%
\pgfpathlineto{\pgfqpoint{7.445650in}{2.744020in}}%
\pgfpathlineto{\pgfqpoint{7.447675in}{6.434087in}}%
\pgfpathlineto{\pgfqpoint{7.449700in}{2.706041in}}%
\pgfpathlineto{\pgfqpoint{7.451726in}{6.475062in}}%
\pgfpathlineto{\pgfqpoint{7.453751in}{2.660979in}}%
\pgfpathlineto{\pgfqpoint{7.455776in}{6.525119in}}%
\pgfpathlineto{\pgfqpoint{7.457801in}{2.605328in}}%
\pgfpathlineto{\pgfqpoint{7.459827in}{6.586524in}}%
\pgfpathlineto{\pgfqpoint{7.461852in}{2.538556in}}%
\pgfpathlineto{\pgfqpoint{7.463877in}{6.657643in}}%
\pgfpathlineto{\pgfqpoint{7.465902in}{2.464791in}}%
\pgfpathlineto{\pgfqpoint{7.467928in}{6.731678in}}%
\pgfpathlineto{\pgfqpoint{7.469953in}{2.393488in}}%
\pgfpathlineto{\pgfqpoint{7.471978in}{6.796729in}}%
\pgfpathlineto{\pgfqpoint{7.474004in}{2.338572in}}%
\pgfpathlineto{\pgfqpoint{7.476029in}{6.837457in}}%
\pgfpathlineto{\pgfqpoint{7.478054in}{2.316029in}}%
\pgfpathlineto{\pgfqpoint{7.480079in}{6.838108in}}%
\pgfpathlineto{\pgfqpoint{7.482105in}{2.340458in}}%
\pgfpathlineto{\pgfqpoint{7.484130in}{6.786140in}}%
\pgfpathlineto{\pgfqpoint{7.486155in}{2.421526in}}%
\pgfpathlineto{\pgfqpoint{7.488180in}{6.675421in}}%
\pgfpathlineto{\pgfqpoint{7.490206in}{2.561384in}}%
\pgfpathlineto{\pgfqpoint{7.492231in}{6.507979in}}%
\pgfpathlineto{\pgfqpoint{7.494256in}{2.753902in}}%
\pgfpathlineto{\pgfqpoint{7.496282in}{6.293694in}}%
\pgfpathlineto{\pgfqpoint{7.498307in}{2.986058in}}%
\pgfpathlineto{\pgfqpoint{7.500332in}{6.047897in}}%
\pgfpathlineto{\pgfqpoint{7.502357in}{3.241199in}}%
\pgfpathlineto{\pgfqpoint{7.504383in}{5.787500in}}%
\pgfpathlineto{\pgfqpoint{7.506408in}{3.503226in}}%
\pgfpathlineto{\pgfqpoint{7.508433in}{5.526797in}}%
\pgfpathlineto{\pgfqpoint{7.510458in}{3.760479in}}%
\pgfpathlineto{\pgfqpoint{7.512484in}{5.274212in}}%
\pgfpathlineto{\pgfqpoint{7.514509in}{4.008093in}}%
\pgfpathlineto{\pgfqpoint{7.516534in}{5.031036in}}%
\pgfpathlineto{\pgfqpoint{7.518560in}{4.248052in}}%
\pgfpathlineto{\pgfqpoint{7.520585in}{4.792596in}}%
\pgfpathlineto{\pgfqpoint{7.522610in}{4.486891in}}%
\pgfpathlineto{\pgfqpoint{7.524635in}{4.551499in}}%
\pgfpathlineto{\pgfqpoint{7.526661in}{4.551499in}}%
\pgfpathlineto{\pgfqpoint{7.528686in}{4.731764in}}%
\pgfpathlineto{\pgfqpoint{7.530711in}{4.301928in}}%
\pgfpathlineto{\pgfqpoint{7.532736in}{4.986153in}}%
\pgfpathlineto{\pgfqpoint{7.534762in}{4.043547in}}%
\pgfpathlineto{\pgfqpoint{7.536787in}{5.246692in}}%
\pgfpathlineto{\pgfqpoint{7.538812in}{3.783663in}}%
\pgfpathlineto{\pgfqpoint{7.540838in}{5.502238in}}%
\pgfpathlineto{\pgfqpoint{7.542863in}{3.536809in}}%
\pgfpathlineto{\pgfqpoint{7.544888in}{5.735634in}}%
\pgfpathlineto{\pgfqpoint{7.546913in}{3.321741in}}%
\pgfpathlineto{\pgfqpoint{7.548939in}{5.927735in}}%
\pgfpathlineto{\pgfqpoint{7.550964in}{3.156689in}}%
\pgfpathlineto{\pgfqpoint{7.552989in}{6.062515in}}%
\pgfpathlineto{\pgfqpoint{7.555014in}{3.054300in}}%
\pgfpathlineto{\pgfqpoint{7.557040in}{6.131672in}}%
\pgfpathlineto{\pgfqpoint{7.559065in}{3.017852in}}%
\pgfpathlineto{\pgfqpoint{7.561090in}{6.137283in}}%
\pgfpathlineto{\pgfqpoint{7.563116in}{3.039959in}}%
\pgfpathlineto{\pgfqpoint{7.565141in}{6.091629in}}%
\pgfpathlineto{\pgfqpoint{7.567166in}{3.104211in}}%
\pgfpathlineto{\pgfqpoint{7.569191in}{6.014188in}}%
\pgfpathlineto{\pgfqpoint{7.571217in}{3.189322in}}%
\pgfpathlineto{\pgfqpoint{7.573242in}{5.926679in}}%
\pgfpathlineto{\pgfqpoint{7.575267in}{3.274534in}}%
\pgfpathlineto{\pgfqpoint{7.577292in}{5.847603in}}%
\pgfpathlineto{\pgfqpoint{7.579318in}{3.344704in}}%
\pgfpathlineto{\pgfqpoint{7.581343in}{5.787911in}}%
\pgfpathlineto{\pgfqpoint{7.583368in}{3.393574in}}%
\pgfpathlineto{\pgfqpoint{7.585394in}{5.749037in}}%
\pgfpathlineto{\pgfqpoint{7.587419in}{3.424300in}}%
\pgfpathlineto{\pgfqpoint{7.589444in}{5.723816in}}%
\pgfpathlineto{\pgfqpoint{7.591469in}{3.447179in}}%
\pgfpathlineto{\pgfqpoint{7.593495in}{5.699912in}}%
\pgfpathlineto{\pgfqpoint{7.595520in}{3.475360in}}%
\pgfpathlineto{\pgfqpoint{7.597545in}{5.664620in}}%
\pgfpathlineto{\pgfqpoint{7.599570in}{3.519920in}}%
\pgfpathlineto{\pgfqpoint{7.601596in}{5.609506in}}%
\pgfpathlineto{\pgfqpoint{7.603621in}{3.585886in}}%
\pgfpathlineto{\pgfqpoint{7.605646in}{5.533406in}}%
\pgfpathlineto{\pgfqpoint{7.607672in}{3.670448in}}%
\pgfpathlineto{\pgfqpoint{7.609697in}{5.442855in}}%
\pgfpathlineto{\pgfqpoint{7.611722in}{3.763933in}}%
\pgfpathlineto{\pgfqpoint{7.613747in}{5.349797in}}%
\pgfpathlineto{\pgfqpoint{7.615773in}{3.853205in}}%
\pgfpathlineto{\pgfqpoint{7.617798in}{5.267351in}}%
\pgfpathlineto{\pgfqpoint{7.619823in}{3.926396in}}%
\pgfpathlineto{\pgfqpoint{7.621848in}{5.204986in}}%
\pgfpathlineto{\pgfqpoint{7.623874in}{3.977404in}}%
\pgfpathlineto{\pgfqpoint{7.625899in}{5.164734in}}%
\pgfpathlineto{\pgfqpoint{7.627924in}{4.008637in}}%
\pgfpathlineto{\pgfqpoint{7.629950in}{5.139745in}}%
\pgfpathlineto{\pgfqpoint{7.631975in}{4.031007in}}%
\pgfpathlineto{\pgfqpoint{7.634000in}{5.115782in}}%
\pgfpathlineto{\pgfqpoint{7.636025in}{4.061036in}}%
\pgfpathlineto{\pgfqpoint{7.638051in}{5.075312in}}%
\pgfpathlineto{\pgfqpoint{7.640076in}{4.115862in}}%
\pgfpathlineto{\pgfqpoint{7.642101in}{5.003016in}}%
\pgfpathlineto{\pgfqpoint{7.644126in}{4.207653in}}%
\pgfpathlineto{\pgfqpoint{7.646152in}{4.891005in}}%
\pgfpathlineto{\pgfqpoint{7.648177in}{4.339188in}}%
\pgfpathlineto{\pgfqpoint{7.650202in}{4.742077in}}%
\pgfpathlineto{\pgfqpoint{7.652228in}{4.502040in}}%
\pgfpathlineto{\pgfqpoint{7.654253in}{4.569913in}}%
\pgfpathlineto{\pgfqpoint{7.656278in}{4.678049in}}%
\pgfpathlineto{\pgfqpoint{7.658303in}{4.396026in}}%
\pgfpathlineto{\pgfqpoint{7.660329in}{4.843739in}}%
\pgfpathlineto{\pgfqpoint{7.662354in}{4.244308in}}%
\pgfpathlineto{\pgfqpoint{7.664379in}{4.976395in}}%
\pgfpathlineto{\pgfqpoint{7.666404in}{4.134785in}}%
\pgfpathlineto{\pgfqpoint{7.668430in}{5.059993in}}%
\pgfpathlineto{\pgfqpoint{7.670455in}{4.078466in}}%
\pgfpathlineto{\pgfqpoint{7.672480in}{5.089172in}}%
\pgfpathlineto{\pgfqpoint{7.674506in}{4.074851in}}%
\pgfpathlineto{\pgfqpoint{7.676531in}{5.070065in}}%
\pgfpathlineto{\pgfqpoint{7.678556in}{4.112828in}}%
\pgfpathlineto{\pgfqpoint{7.680581in}{5.017753in}}%
\pgfpathlineto{\pgfqpoint{7.682607in}{4.174632in}}%
\pgfpathlineto{\pgfqpoint{7.684632in}{4.951221in}}%
\pgfpathlineto{\pgfqpoint{7.686657in}{4.241570in}}%
\pgfpathlineto{\pgfqpoint{7.688682in}{4.887435in}}%
\pgfpathlineto{\pgfqpoint{7.690708in}{4.299652in}}%
\pgfpathlineto{\pgfqpoint{7.692733in}{4.836455in}}%
\pgfpathlineto{\pgfqpoint{7.694758in}{4.343331in}}%
\pgfpathlineto{\pgfqpoint{7.696784in}{4.799145in}}%
\pgfpathlineto{\pgfqpoint{7.698809in}{4.376172in}}%
\pgfpathlineto{\pgfqpoint{7.700834in}{4.768162in}}%
\pgfpathlineto{\pgfqpoint{7.702859in}{4.408302in}}%
\pgfpathlineto{\pgfqpoint{7.704885in}{4.731839in}}%
\pgfpathlineto{\pgfqpoint{7.706910in}{4.451547in}}%
\pgfpathlineto{\pgfqpoint{7.708935in}{4.679594in}}%
\pgfpathlineto{\pgfqpoint{7.710960in}{4.513944in}}%
\pgfpathlineto{\pgfqpoint{7.712986in}{4.607011in}}%
\pgfpathlineto{\pgfqpoint{7.715011in}{4.595530in}}%
\pgfpathlineto{\pgfqpoint{7.717036in}{4.518803in}}%
\pgfpathlineto{\pgfqpoint{7.719062in}{4.686905in}}%
\pgfpathlineto{\pgfqpoint{7.721087in}{4.428569in}}%
\pgfpathlineto{\pgfqpoint{7.723112in}{4.771150in}}%
\pgfpathlineto{\pgfqpoint{7.725137in}{4.355334in}}%
\pgfpathlineto{\pgfqpoint{7.727163in}{4.828580in}}%
\pgfpathlineto{\pgfqpoint{7.729188in}{4.317882in}}%
\pgfpathlineto{\pgfqpoint{7.731213in}{4.842859in}}%
\pgfpathlineto{\pgfqpoint{7.733238in}{4.328697in}}%
\pgfpathlineto{\pgfqpoint{7.735264in}{4.806505in}}%
\pgfpathlineto{\pgfqpoint{7.737289in}{4.389466in}}%
\pgfpathlineto{\pgfqpoint{7.739314in}{4.723992in}}%
\pgfpathlineto{\pgfqpoint{7.741340in}{4.489650in}}%
\pgfpathlineto{\pgfqpoint{7.743365in}{4.611359in}}%
\pgfpathlineto{\pgfqpoint{7.745390in}{4.608699in}}%
\pgfpathlineto{\pgfqpoint{7.747415in}{4.492328in}}%
\pgfpathlineto{\pgfqpoint{7.749441in}{4.721319in}}%
\pgfpathlineto{\pgfqpoint{7.751466in}{4.392030in}}%
\pgfpathlineto{\pgfqpoint{7.753491in}{4.804277in}}%
\pgfpathlineto{\pgfqpoint{7.755516in}{4.330194in}}%
\pgfpathlineto{\pgfqpoint{7.757542in}{4.842703in}}%
\pgfpathlineto{\pgfqpoint{7.759567in}{4.315831in}}%
\pgfpathlineto{\pgfqpoint{7.761592in}{4.834006in}}%
\pgfpathlineto{\pgfqpoint{7.763618in}{4.345025in}}%
\pgfpathlineto{\pgfqpoint{7.765643in}{4.788216in}}%
\pgfpathlineto{\pgfqpoint{7.767668in}{4.402487in}}%
\pgfpathlineto{\pgfqpoint{7.769693in}{4.724640in}}%
\pgfpathlineto{\pgfqpoint{7.771719in}{4.466424in}}%
\pgfpathlineto{\pgfqpoint{7.773744in}{4.665845in}}%
\pgfpathlineto{\pgfqpoint{7.775769in}{4.515252in}}%
\pgfpathlineto{\pgfqpoint{7.777794in}{4.630761in}}%
\pgfpathlineto{\pgfqpoint{7.779820in}{4.534147in}}%
\pgfpathlineto{\pgfqpoint{7.781845in}{4.628983in}}%
\pgfpathlineto{\pgfqpoint{7.783870in}{4.519461in}}%
\pgfpathlineto{\pgfqpoint{7.785896in}{4.657962in}}%
\pgfpathlineto{\pgfqpoint{7.787921in}{4.479700in}}%
\pgfpathlineto{\pgfqpoint{7.789946in}{4.703934in}}%
\pgfpathlineto{\pgfqpoint{7.791971in}{4.432784in}}%
\pgfpathlineto{\pgfqpoint{7.793997in}{4.746249in}}%
\pgfpathlineto{\pgfqpoint{7.796022in}{4.400460in}}%
\pgfpathlineto{\pgfqpoint{7.798047in}{4.763767in}}%
\pgfpathlineto{\pgfqpoint{7.800072in}{4.401614in}}%
\pgfpathlineto{\pgfqpoint{7.802098in}{4.741315in}}%
\pgfpathlineto{\pgfqpoint{7.804123in}{4.446559in}}%
\pgfpathlineto{\pgfqpoint{7.806148in}{4.674190in}}%
\pgfpathlineto{\pgfqpoint{7.808174in}{4.534091in}}%
\pgfpathlineto{\pgfqpoint{7.810199in}{4.569314in}}%
\pgfpathlineto{\pgfqpoint{7.812224in}{4.652225in}}%
\pgfpathlineto{\pgfqpoint{7.814249in}{4.442679in}}%
\pgfpathlineto{\pgfqpoint{7.816275in}{4.782341in}}%
\pgfpathlineto{\pgfqpoint{7.818300in}{4.313945in}}%
\pgfpathlineto{\pgfqpoint{7.820325in}{4.905389in}}%
\pgfpathlineto{\pgfqpoint{7.822350in}{4.199970in}}%
\pgfpathlineto{\pgfqpoint{7.824376in}{5.008085in}}%
\pgfpathlineto{\pgfqpoint{7.826401in}{4.109416in}}%
\pgfpathlineto{\pgfqpoint{7.828426in}{5.087021in}}%
\pgfpathlineto{\pgfqpoint{7.830452in}{4.040274in}}%
\pgfpathlineto{\pgfqpoint{7.832477in}{5.149286in}}%
\pgfpathlineto{\pgfqpoint{7.834502in}{3.981183in}}%
\pgfpathlineto{\pgfqpoint{7.836527in}{5.209307in}}%
\pgfpathlineto{\pgfqpoint{7.838553in}{3.916162in}}%
\pgfpathlineto{\pgfqpoint{7.840578in}{5.282926in}}%
\pgfpathlineto{\pgfqpoint{7.842603in}{3.831228in}}%
\pgfpathlineto{\pgfqpoint{7.844628in}{5.380669in}}%
\pgfpathlineto{\pgfqpoint{7.846654in}{3.720653in}}%
\pgfpathlineto{\pgfqpoint{7.848679in}{5.502508in}}%
\pgfpathlineto{\pgfqpoint{7.850704in}{3.590698in}}%
\pgfpathlineto{\pgfqpoint{7.852730in}{5.635998in}}%
\pgfpathlineto{\pgfqpoint{7.854755in}{3.459407in}}%
\pgfpathlineto{\pgfqpoint{7.856780in}{5.758589in}}%
\pgfpathlineto{\pgfqpoint{7.858805in}{3.352321in}}%
\pgfpathlineto{\pgfqpoint{7.860831in}{5.843571in}}%
\pgfpathlineto{\pgfqpoint{7.862856in}{3.295327in}}%
\pgfpathlineto{\pgfqpoint{7.864881in}{5.867875in}}%
\pgfpathlineto{\pgfqpoint{7.866906in}{3.306837in}}%
\pgfpathlineto{\pgfqpoint{7.868932in}{5.819284in}}%
\pgfpathlineto{\pgfqpoint{7.870957in}{3.391773in}}%
\pgfpathlineto{\pgfqpoint{7.872982in}{5.700739in}}%
\pgfpathlineto{\pgfqpoint{7.875008in}{3.539348in}}%
\pgfpathlineto{\pgfqpoint{7.877033in}{5.530264in}}%
\pgfpathlineto{\pgfqpoint{7.879058in}{3.725457in}}%
\pgfpathlineto{\pgfqpoint{7.881083in}{5.336422in}}%
\pgfpathlineto{\pgfqpoint{7.883109in}{3.919043in}}%
\pgfpathlineto{\pgfqpoint{7.885134in}{5.150613in}}%
\pgfpathlineto{\pgfqpoint{7.887159in}{4.090543in}}%
\pgfpathlineto{\pgfqpoint{7.889184in}{4.998530in}}%
\pgfpathlineto{\pgfqpoint{7.891210in}{4.219852in}}%
\pgfpathlineto{\pgfqpoint{7.893235in}{4.893411in}}%
\pgfpathlineto{\pgfqpoint{7.895260in}{4.301336in}}%
\pgfpathlineto{\pgfqpoint{7.897286in}{4.833160in}}%
\pgfpathlineto{\pgfqpoint{7.899311in}{4.344332in}}%
\pgfpathlineto{\pgfqpoint{7.901336in}{4.802260in}}%
\pgfpathlineto{\pgfqpoint{7.903361in}{4.368979in}}%
\pgfpathlineto{\pgfqpoint{7.905387in}{4.777890in}}%
\pgfpathlineto{\pgfqpoint{7.907412in}{4.398614in}}%
\pgfpathlineto{\pgfqpoint{7.909437in}{4.738418in}}%
\pgfpathlineto{\pgfqpoint{7.911462in}{4.451061in}}%
\pgfpathlineto{\pgfqpoint{7.913488in}{4.671653in}}%
\pgfpathlineto{\pgfqpoint{7.915513in}{4.531464in}}%
\pgfpathlineto{\pgfqpoint{7.919564in}{4.628816in}}%
\pgfpathlineto{\pgfqpoint{7.921589in}{4.483506in}}%
\pgfpathlineto{\pgfqpoint{7.923614in}{4.717267in}}%
\pgfpathlineto{\pgfqpoint{7.925639in}{4.412359in}}%
\pgfpathlineto{\pgfqpoint{7.927665in}{4.761865in}}%
\pgfpathlineto{\pgfqpoint{7.929690in}{4.403330in}}%
\pgfpathlineto{\pgfqpoint{7.931715in}{4.727113in}}%
\pgfpathlineto{\pgfqpoint{7.933740in}{4.488725in}}%
\pgfpathlineto{\pgfqpoint{7.937791in}{4.688379in}}%
\pgfpathlineto{\pgfqpoint{7.939816in}{4.327332in}}%
\pgfpathlineto{\pgfqpoint{7.941842in}{5.004091in}}%
\pgfpathlineto{\pgfqpoint{7.943867in}{3.959083in}}%
\pgfpathlineto{\pgfqpoint{7.945892in}{5.418106in}}%
\pgfpathlineto{\pgfqpoint{7.947917in}{3.508083in}}%
\pgfpathlineto{\pgfqpoint{7.949943in}{5.895668in}}%
\pgfpathlineto{\pgfqpoint{7.951968in}{3.015591in}}%
\pgfpathlineto{\pgfqpoint{7.953993in}{6.390717in}}%
\pgfpathlineto{\pgfqpoint{7.956018in}{2.530626in}}%
\pgfpathlineto{\pgfqpoint{7.958044in}{6.853155in}}%
\pgfpathlineto{\pgfqpoint{7.960069in}{2.102533in}}%
\pgfpathlineto{\pgfqpoint{7.962094in}{7.236084in}}%
\pgfpathlineto{\pgfqpoint{7.964120in}{1.774263in}}%
\pgfpathlineto{\pgfqpoint{7.966145in}{7.501779in}}%
\pgfpathlineto{\pgfqpoint{7.968170in}{1.577288in}}%
\pgfpathlineto{\pgfqpoint{7.970195in}{7.625796in}}%
\pgfpathlineto{\pgfqpoint{7.972221in}{1.528485in}}%
\pgfpathlineto{\pgfqpoint{7.974246in}{7.599140in}}%
\pgfpathlineto{\pgfqpoint{7.976271in}{1.628857in}}%
\pgfpathlineto{\pgfqpoint{7.978296in}{7.428719in}}%
\pgfpathlineto{\pgfqpoint{7.980322in}{1.863841in}}%
\pgfpathlineto{\pgfqpoint{7.982347in}{7.136322in}}%
\pgfpathlineto{\pgfqpoint{7.984372in}{2.205031in}}%
\pgfpathlineto{\pgfqpoint{7.986398in}{6.756190in}}%
\pgfpathlineto{\pgfqpoint{7.988423in}{2.613316in}}%
\pgfpathlineto{\pgfqpoint{7.990448in}{6.331144in}}%
\pgfpathlineto{\pgfqpoint{7.992473in}{3.043498in}}%
\pgfpathlineto{\pgfqpoint{7.994499in}{5.907279in}}%
\pgfpathlineto{\pgfqpoint{7.996524in}{3.450186in}}%
\pgfpathlineto{\pgfqpoint{7.998549in}{5.527623in}}%
\pgfpathlineto{\pgfqpoint{8.000574in}{3.794357in}}%
\pgfpathlineto{\pgfqpoint{8.002600in}{5.225646in}}%
\pgfpathlineto{\pgfqpoint{8.004625in}{4.049450in}}%
\pgfpathlineto{\pgfqpoint{8.006650in}{5.019927in}}%
\pgfpathlineto{\pgfqpoint{8.008676in}{4.205576in}}%
\pgfpathlineto{\pgfqpoint{8.010701in}{4.911390in}}%
\pgfpathlineto{\pgfqpoint{8.012726in}{4.270572in}}%
\pgfpathlineto{\pgfqpoint{8.014751in}{4.884131in}}%
\pgfpathlineto{\pgfqpoint{8.016777in}{4.267252in}}%
\pgfpathlineto{\pgfqpoint{8.018802in}{4.909997in}}%
\pgfpathlineto{\pgfqpoint{8.020827in}{4.227211in}}%
\pgfpathlineto{\pgfqpoint{8.022852in}{4.956052in}}%
\pgfpathlineto{\pgfqpoint{8.024878in}{4.182564in}}%
\pgfpathlineto{\pgfqpoint{8.026903in}{4.993084in}}%
\pgfpathlineto{\pgfqpoint{8.028928in}{4.157762in}}%
\pgfpathlineto{\pgfqpoint{8.030954in}{5.002891in}}%
\pgfpathlineto{\pgfqpoint{8.032979in}{4.163746in}}%
\pgfpathlineto{\pgfqpoint{8.035004in}{4.982259in}}%
\pgfpathlineto{\pgfqpoint{8.037029in}{4.196123in}}%
\pgfpathlineto{\pgfqpoint{8.039055in}{4.942484in}}%
\pgfpathlineto{\pgfqpoint{8.041080in}{4.237928in}}%
\pgfpathlineto{\pgfqpoint{8.043105in}{4.904537in}}%
\pgfpathlineto{\pgfqpoint{8.045130in}{4.266149in}}%
\pgfpathlineto{\pgfqpoint{8.047156in}{4.891350in}}%
\pgfpathlineto{\pgfqpoint{8.049181in}{4.260046in}}%
\pgfpathlineto{\pgfqpoint{8.051206in}{4.919544in}}%
\pgfpathlineto{\pgfqpoint{8.053232in}{4.208707in}}%
\pgfpathlineto{\pgfqpoint{8.055257in}{4.993186in}}%
\pgfpathlineto{\pgfqpoint{8.057282in}{4.115495in}}%
\pgfpathlineto{\pgfqpoint{8.059307in}{5.101500in}}%
\pgfpathlineto{\pgfqpoint{8.061333in}{3.997983in}}%
\pgfpathlineto{\pgfqpoint{8.063358in}{5.221288in}}%
\pgfpathlineto{\pgfqpoint{8.065383in}{3.883354in}}%
\pgfpathlineto{\pgfqpoint{8.067408in}{5.323357in}}%
\pgfpathlineto{\pgfqpoint{8.069434in}{3.800658in}}%
\pgfpathlineto{\pgfqpoint{8.071459in}{5.380973in}}%
\pgfpathlineto{\pgfqpoint{8.073484in}{3.772293in}}%
\pgfpathlineto{\pgfqpoint{8.075510in}{5.377771in}}%
\pgfpathlineto{\pgfqpoint{8.077535in}{3.807341in}}%
\pgfpathlineto{\pgfqpoint{8.079560in}{5.312661in}}%
\pgfpathlineto{\pgfqpoint{8.081585in}{3.898798in}}%
\pgfpathlineto{\pgfqpoint{8.083611in}{5.200230in}}%
\pgfpathlineto{\pgfqpoint{8.085636in}{4.025573in}}%
\pgfpathlineto{\pgfqpoint{8.087661in}{5.066504in}}%
\pgfpathlineto{\pgfqpoint{8.089686in}{4.158639in}}%
\pgfpathlineto{\pgfqpoint{8.091712in}{4.941365in}}%
\pgfpathlineto{\pgfqpoint{8.093737in}{4.269454in}}%
\pgfpathlineto{\pgfqpoint{8.095762in}{4.849946in}}%
\pgfpathlineto{\pgfqpoint{8.097788in}{4.338079in}}%
\pgfpathlineto{\pgfqpoint{8.099813in}{4.805623in}}%
\pgfpathlineto{\pgfqpoint{8.101838in}{4.358548in}}%
\pgfpathlineto{\pgfqpoint{8.103863in}{4.806694in}}%
\pgfpathlineto{\pgfqpoint{8.105889in}{4.339891in}}%
\pgfpathlineto{\pgfqpoint{8.107914in}{4.837703in}}%
\pgfpathlineto{\pgfqpoint{8.109939in}{4.302592in}}%
\pgfpathlineto{\pgfqpoint{8.111964in}{4.874909in}}%
\pgfpathlineto{\pgfqpoint{8.113990in}{4.271655in}}%
\pgfpathlineto{\pgfqpoint{8.116015in}{4.894111in}}%
\pgfpathlineto{\pgfqpoint{8.118040in}{4.268508in}}%
\pgfpathlineto{\pgfqpoint{8.120066in}{4.878368in}}%
\pgfpathlineto{\pgfqpoint{8.122091in}{4.304278in}}%
\pgfpathlineto{\pgfqpoint{8.124116in}{4.823198in}}%
\pgfpathlineto{\pgfqpoint{8.126141in}{4.376531in}}%
\pgfpathlineto{\pgfqpoint{8.128167in}{4.737655in}}%
\pgfpathlineto{\pgfqpoint{8.130192in}{4.470436in}}%
\pgfpathlineto{\pgfqpoint{8.132217in}{4.641028in}}%
\pgfpathlineto{\pgfqpoint{8.134242in}{4.563921in}}%
\pgfpathlineto{\pgfqpoint{8.136268in}{4.556271in}}%
\pgfpathlineto{\pgfqpoint{8.138293in}{4.635120in}}%
\pgfpathlineto{\pgfqpoint{8.140318in}{4.502286in}}%
\pgfpathlineto{\pgfqpoint{8.142344in}{4.669723in}}%
\pgfpathlineto{\pgfqpoint{8.144369in}{4.487546in}}%
\pgfpathlineto{\pgfqpoint{8.146394in}{4.665864in}}%
\pgfpathlineto{\pgfqpoint{8.148419in}{4.507079in}}%
\pgfpathlineto{\pgfqpoint{8.150445in}{4.635023in}}%
\pgfpathlineto{\pgfqpoint{8.152470in}{4.543758in}}%
\pgfpathlineto{\pgfqpoint{8.154495in}{4.598656in}}%
\pgfpathlineto{\pgfqpoint{8.156520in}{4.573462in}}%
\pgfpathlineto{\pgfqpoint{8.158546in}{4.581668in}}%
\pgfpathlineto{\pgfqpoint{8.160571in}{4.572450in}}%
\pgfpathlineto{\pgfqpoint{8.162596in}{4.604789in}}%
\pgfpathlineto{\pgfqpoint{8.164622in}{4.524606in}}%
\pgfpathlineto{\pgfqpoint{8.166647in}{4.678280in}}%
\pgfpathlineto{\pgfqpoint{8.168672in}{4.426297in}}%
\pgfpathlineto{\pgfqpoint{8.170697in}{4.798894in}}%
\pgfpathlineto{\pgfqpoint{8.172723in}{4.287370in}}%
\pgfpathlineto{\pgfqpoint{8.174748in}{4.950974in}}%
\pgfpathlineto{\pgfqpoint{8.176773in}{4.128071in}}%
\pgfpathlineto{\pgfqpoint{8.178798in}{5.111229in}}%
\pgfpathlineto{\pgfqpoint{8.180824in}{3.972993in}}%
\pgfpathlineto{\pgfqpoint{8.182849in}{5.255567in}}%
\pgfpathlineto{\pgfqpoint{8.184874in}{3.844012in}}%
\pgfpathlineto{\pgfqpoint{8.186900in}{5.365809in}}%
\pgfpathlineto{\pgfqpoint{8.188925in}{3.754467in}}%
\pgfpathlineto{\pgfqpoint{8.190950in}{5.434186in}}%
\pgfpathlineto{\pgfqpoint{8.192975in}{3.706294in}}%
\pgfpathlineto{\pgfqpoint{8.195001in}{5.464388in}}%
\pgfpathlineto{\pgfqpoint{8.197026in}{3.690819in}}%
\pgfpathlineto{\pgfqpoint{8.199051in}{5.469064in}}%
\pgfpathlineto{\pgfqpoint{8.201076in}{3.692695in}}%
\pgfpathlineto{\pgfqpoint{8.203102in}{5.464829in}}%
\pgfpathlineto{\pgfqpoint{8.205127in}{3.695498in}}%
\pgfpathlineto{\pgfqpoint{8.207152in}{5.466551in}}%
\pgfpathlineto{\pgfqpoint{8.209178in}{3.687084in}}%
\pgfpathlineto{\pgfqpoint{8.211203in}{5.482762in}}%
\pgfpathlineto{\pgfqpoint{8.213228in}{3.663073in}}%
\pgfpathlineto{\pgfqpoint{8.215253in}{5.513523in}}%
\pgfpathlineto{\pgfqpoint{8.217279in}{3.627534in}}%
\pgfpathlineto{\pgfqpoint{8.219304in}{5.551148in}}%
\pgfpathlineto{\pgfqpoint{8.221329in}{3.590991in}}%
\pgfpathlineto{\pgfqpoint{8.223354in}{5.583237in}}%
\pgfpathlineto{\pgfqpoint{8.225380in}{3.566657in}}%
\pgfpathlineto{\pgfqpoint{8.227405in}{5.596834in}}%
\pgfpathlineto{\pgfqpoint{8.229430in}{3.566252in}}%
\pgfpathlineto{\pgfqpoint{8.231456in}{5.582284in}}%
\pgfpathlineto{\pgfqpoint{8.233481in}{3.596730in}}%
\pgfpathlineto{\pgfqpoint{8.235506in}{5.535730in}}%
\pgfpathlineto{\pgfqpoint{8.237531in}{3.658708in}}%
\pgfpathlineto{\pgfqpoint{8.239557in}{5.459703in}}%
\pgfpathlineto{\pgfqpoint{8.241582in}{3.746806in}}%
\pgfpathlineto{\pgfqpoint{8.243607in}{5.361965in}}%
\pgfpathlineto{\pgfqpoint{8.245632in}{3.851464in}}%
\pgfpathlineto{\pgfqpoint{8.247658in}{5.253224in}}%
\pgfpathlineto{\pgfqpoint{8.249683in}{3.961496in}}%
\pgfpathlineto{\pgfqpoint{8.251708in}{5.144511in}}%
\pgfpathlineto{\pgfqpoint{8.253734in}{4.066578in}}%
\pgfpathlineto{\pgfqpoint{8.255759in}{5.044981in}}%
\pgfpathlineto{\pgfqpoint{8.257784in}{4.159073in}}%
\pgfpathlineto{\pgfqpoint{8.259809in}{4.960541in}}%
\pgfpathlineto{\pgfqpoint{8.261835in}{4.234901in}}%
\pgfpathlineto{\pgfqpoint{8.263860in}{4.893455in}}%
\pgfpathlineto{\pgfqpoint{8.265885in}{4.293499in}}%
\pgfpathlineto{\pgfqpoint{8.267910in}{4.842771in}}%
\pgfpathlineto{\pgfqpoint{8.269936in}{4.337090in}}%
\pgfpathlineto{\pgfqpoint{8.271961in}{4.805277in}}%
\pgfpathlineto{\pgfqpoint{8.273986in}{4.369581in}}%
\pgfpathlineto{\pgfqpoint{8.276012in}{4.776665in}}%
\pgfpathlineto{\pgfqpoint{8.278037in}{4.395401in}}%
\pgfpathlineto{\pgfqpoint{8.280062in}{4.752642in}}%
\pgfpathlineto{\pgfqpoint{8.282087in}{4.418488in}}%
\pgfpathlineto{\pgfqpoint{8.284113in}{4.729800in}}%
\pgfpathlineto{\pgfqpoint{8.286138in}{4.441581in}}%
\pgfpathlineto{\pgfqpoint{8.288163in}{4.706163in}}%
\pgfpathlineto{\pgfqpoint{8.290188in}{4.465849in}}%
\pgfpathlineto{\pgfqpoint{8.292214in}{4.681373in}}%
\pgfpathlineto{\pgfqpoint{8.294239in}{4.490871in}}%
\pgfpathlineto{\pgfqpoint{8.296264in}{4.656564in}}%
\pgfpathlineto{\pgfqpoint{8.298290in}{4.514897in}}%
\pgfpathlineto{\pgfqpoint{8.300315in}{4.633983in}}%
\pgfpathlineto{\pgfqpoint{8.302340in}{4.535315in}}%
\pgfpathlineto{\pgfqpoint{8.304365in}{4.616468in}}%
\pgfpathlineto{\pgfqpoint{8.306391in}{4.549203in}}%
\pgfpathlineto{\pgfqpoint{8.308416in}{4.606884in}}%
\pgfpathlineto{\pgfqpoint{8.310441in}{4.553879in}}%
\pgfpathlineto{\pgfqpoint{8.312467in}{4.607624in}}%
\pgfpathlineto{\pgfqpoint{8.314492in}{4.547330in}}%
\pgfpathlineto{\pgfqpoint{8.316517in}{4.620250in}}%
\pgfpathlineto{\pgfqpoint{8.318542in}{4.528492in}}%
\pgfpathlineto{\pgfqpoint{8.320568in}{4.645304in}}%
\pgfpathlineto{\pgfqpoint{8.322593in}{4.497349in}}%
\pgfpathlineto{\pgfqpoint{8.324618in}{4.682284in}}%
\pgfpathlineto{\pgfqpoint{8.326643in}{4.454899in}}%
\pgfpathlineto{\pgfqpoint{8.328669in}{4.729731in}}%
\pgfpathlineto{\pgfqpoint{8.330694in}{4.403018in}}%
\pgfpathlineto{\pgfqpoint{8.332719in}{4.785400in}}%
\pgfpathlineto{\pgfqpoint{8.334745in}{4.344274in}}%
\pgfpathlineto{\pgfqpoint{8.336770in}{4.846454in}}%
\pgfpathlineto{\pgfqpoint{8.338795in}{4.281717in}}%
\pgfpathlineto{\pgfqpoint{8.340820in}{4.909686in}}%
\pgfpathlineto{\pgfqpoint{8.342846in}{4.218644in}}%
\pgfpathlineto{\pgfqpoint{8.344871in}{4.971780in}}%
\pgfpathlineto{\pgfqpoint{8.346896in}{4.158313in}}%
\pgfpathlineto{\pgfqpoint{8.348921in}{5.029622in}}%
\pgfpathlineto{\pgfqpoint{8.350947in}{4.103603in}}%
\pgfpathlineto{\pgfqpoint{8.352972in}{5.080663in}}%
\pgfpathlineto{\pgfqpoint{8.354997in}{4.056637in}}%
\pgfpathlineto{\pgfqpoint{8.357023in}{5.123298in}}%
\pgfpathlineto{\pgfqpoint{8.359048in}{4.018423in}}%
\pgfpathlineto{\pgfqpoint{8.361073in}{5.157178in}}%
\pgfpathlineto{\pgfqpoint{8.363098in}{3.988609in}}%
\pgfpathlineto{\pgfqpoint{8.365124in}{5.183371in}}%
\pgfpathlineto{\pgfqpoint{8.367149in}{3.965430in}}%
\pgfpathlineto{\pgfqpoint{8.369174in}{5.204282in}}%
\pgfpathlineto{\pgfqpoint{8.371199in}{3.945929in}}%
\pgfpathlineto{\pgfqpoint{8.373225in}{5.223308in}}%
\pgfpathlineto{\pgfqpoint{8.375250in}{3.926412in}}%
\pgfpathlineto{\pgfqpoint{8.377275in}{5.244270in}}%
\pgfpathlineto{\pgfqpoint{8.379301in}{3.903106in}}%
\pgfpathlineto{\pgfqpoint{8.381326in}{5.270723in}}%
\pgfpathlineto{\pgfqpoint{8.383351in}{3.872840in}}%
\pgfpathlineto{\pgfqpoint{8.385376in}{5.305298in}}%
\pgfpathlineto{\pgfqpoint{8.387402in}{3.833650in}}%
\pgfpathlineto{\pgfqpoint{8.389427in}{5.349204in}}%
\pgfpathlineto{\pgfqpoint{8.391452in}{3.785141in}}%
\pgfpathlineto{\pgfqpoint{8.393477in}{5.401992in}}%
\pgfpathlineto{\pgfqpoint{8.395503in}{3.728594in}}%
\pgfpathlineto{\pgfqpoint{8.397528in}{5.461597in}}%
\pgfpathlineto{\pgfqpoint{8.399553in}{3.666794in}}%
\pgfpathlineto{\pgfqpoint{8.401579in}{5.524595in}}%
\pgfpathlineto{\pgfqpoint{8.403604in}{3.603698in}}%
\pgfpathlineto{\pgfqpoint{8.405629in}{5.586609in}}%
\pgfpathlineto{\pgfqpoint{8.407654in}{3.543998in}}%
\pgfpathlineto{\pgfqpoint{8.409680in}{5.642736in}}%
\pgfpathlineto{\pgfqpoint{8.411705in}{3.492704in}}%
\pgfpathlineto{\pgfqpoint{8.413730in}{5.687959in}}%
\pgfpathlineto{\pgfqpoint{8.415755in}{3.454749in}}%
\pgfpathlineto{\pgfqpoint{8.417781in}{5.717515in}}%
\pgfpathlineto{\pgfqpoint{8.419806in}{3.434639in}}%
\pgfpathlineto{\pgfqpoint{8.421831in}{5.727243in}}%
\pgfpathlineto{\pgfqpoint{8.423857in}{3.436097in}}%
\pgfpathlineto{\pgfqpoint{8.425882in}{5.713951in}}%
\pgfpathlineto{\pgfqpoint{8.427907in}{3.461691in}}%
\pgfpathlineto{\pgfqpoint{8.429932in}{5.675792in}}%
\pgfpathlineto{\pgfqpoint{8.431958in}{3.512455in}}%
\pgfpathlineto{\pgfqpoint{8.433983in}{5.612623in}}%
\pgfpathlineto{\pgfqpoint{8.436008in}{3.587577in}}%
\pgfpathlineto{\pgfqpoint{8.438033in}{5.526253in}}%
\pgfpathlineto{\pgfqpoint{8.440059in}{3.684237in}}%
\pgfpathlineto{\pgfqpoint{8.442084in}{5.420495in}}%
\pgfpathlineto{\pgfqpoint{8.444109in}{3.797688in}}%
\pgfpathlineto{\pgfqpoint{8.446135in}{5.300937in}}%
\pgfpathlineto{\pgfqpoint{8.448160in}{3.921624in}}%
\pgfpathlineto{\pgfqpoint{8.450185in}{5.174449in}}%
\pgfpathlineto{\pgfqpoint{8.452210in}{4.048793in}}%
\pgfpathlineto{\pgfqpoint{8.454236in}{5.048468in}}%
\pgfpathlineto{\pgfqpoint{8.456261in}{4.171765in}}%
\pgfpathlineto{\pgfqpoint{8.458286in}{4.930230in}}%
\pgfpathlineto{\pgfqpoint{8.460311in}{4.283677in}}%
\pgfpathlineto{\pgfqpoint{8.462337in}{4.826075in}}%
\pgfpathlineto{\pgfqpoint{8.464362in}{4.378826in}}%
\pgfpathlineto{\pgfqpoint{8.466387in}{4.740987in}}%
\pgfpathlineto{\pgfqpoint{8.468413in}{4.452995in}}%
\pgfpathlineto{\pgfqpoint{8.470438in}{4.678406in}}%
\pgfpathlineto{\pgfqpoint{8.472463in}{4.503494in}}%
\pgfpathlineto{\pgfqpoint{8.474488in}{4.640322in}}%
\pgfpathlineto{\pgfqpoint{8.476514in}{4.528968in}}%
\pgfpathlineto{\pgfqpoint{8.478539in}{4.627538in}}%
\pgfpathlineto{\pgfqpoint{8.480564in}{4.529081in}}%
\pgfpathlineto{\pgfqpoint{8.482589in}{4.639999in}}%
\pgfpathlineto{\pgfqpoint{8.484615in}{4.504208in}}%
\pgfpathlineto{\pgfqpoint{8.486640in}{4.677069in}}%
\pgfpathlineto{\pgfqpoint{8.488665in}{4.455203in}}%
\pgfpathlineto{\pgfqpoint{8.490691in}{4.737704in}}%
\pgfpathlineto{\pgfqpoint{8.492716in}{4.383283in}}%
\pgfpathlineto{\pgfqpoint{8.494741in}{4.820521in}}%
\pgfpathlineto{\pgfqpoint{8.496766in}{4.290000in}}%
\pgfpathlineto{\pgfqpoint{8.498792in}{4.923800in}}%
\pgfpathlineto{\pgfqpoint{8.500817in}{4.177236in}}%
\pgfpathlineto{\pgfqpoint{8.502842in}{5.045502in}}%
\pgfpathlineto{\pgfqpoint{8.504867in}{4.047169in}}%
\pgfpathlineto{\pgfqpoint{8.506893in}{5.183339in}}%
\pgfpathlineto{\pgfqpoint{8.508918in}{3.902166in}}%
\pgfpathlineto{\pgfqpoint{8.510943in}{5.334908in}}%
\pgfpathlineto{\pgfqpoint{8.512969in}{3.744619in}}%
\pgfpathlineto{\pgfqpoint{8.514994in}{5.497871in}}%
\pgfpathlineto{\pgfqpoint{8.517019in}{3.576765in}}%
\pgfpathlineto{\pgfqpoint{8.519044in}{5.670131in}}%
\pgfpathlineto{\pgfqpoint{8.521070in}{3.400541in}}%
\pgfpathlineto{\pgfqpoint{8.523095in}{5.849921in}}%
\pgfpathlineto{\pgfqpoint{8.525120in}{3.217544in}}%
\pgfpathlineto{\pgfqpoint{8.527145in}{6.035793in}}%
\pgfpathlineto{\pgfqpoint{8.529171in}{3.029115in}}%
\pgfpathlineto{\pgfqpoint{8.531196in}{6.226456in}}%
\pgfpathlineto{\pgfqpoint{8.533221in}{2.836568in}}%
\pgfpathlineto{\pgfqpoint{8.535247in}{6.420485in}}%
\pgfpathlineto{\pgfqpoint{8.537272in}{2.641539in}}%
\pgfpathlineto{\pgfqpoint{8.539297in}{6.615930in}}%
\pgfpathlineto{\pgfqpoint{8.541322in}{2.446393in}}%
\pgfpathlineto{\pgfqpoint{8.543348in}{6.809907in}}%
\pgfpathlineto{\pgfqpoint{8.545373in}{2.254623in}}%
\pgfpathlineto{\pgfqpoint{8.547398in}{6.998249in}}%
\pgfpathlineto{\pgfqpoint{8.549423in}{2.071118in}}%
\pgfpathlineto{\pgfqpoint{8.551449in}{7.175327in}}%
\pgfpathlineto{\pgfqpoint{8.553474in}{1.902221in}}%
\pgfpathlineto{\pgfqpoint{8.555499in}{7.334151in}}%
\pgfpathlineto{\pgfqpoint{8.557525in}{1.755463in}}%
\pgfpathlineto{\pgfqpoint{8.559550in}{7.466793in}}%
\pgfpathlineto{\pgfqpoint{8.561575in}{1.638984in}}%
\pgfpathlineto{\pgfqpoint{8.563600in}{7.565127in}}%
\pgfpathlineto{\pgfqpoint{8.565626in}{1.560649in}}%
\pgfpathlineto{\pgfqpoint{8.567651in}{7.621804in}}%
\pgfpathlineto{\pgfqpoint{8.569676in}{1.527029in}}%
\pgfpathlineto{\pgfqpoint{8.571701in}{7.631288in}}%
\pgfpathlineto{\pgfqpoint{8.573727in}{1.542395in}}%
\pgfpathlineto{\pgfqpoint{8.575752in}{7.590766in}}%
\pgfpathlineto{\pgfqpoint{8.577777in}{1.607948in}}%
\pgfpathlineto{\pgfqpoint{8.579803in}{7.500747in}}%
\pgfpathlineto{\pgfqpoint{8.581828in}{1.721426in}}%
\pgfpathlineto{\pgfqpoint{8.583853in}{7.365234in}}%
\pgfpathlineto{\pgfqpoint{8.585878in}{1.877164in}}%
\pgfpathlineto{\pgfqpoint{8.587904in}{7.191426in}}%
\pgfpathlineto{\pgfqpoint{8.589929in}{2.066597in}}%
\pgfpathlineto{\pgfqpoint{8.591954in}{6.989039in}}%
\pgfpathlineto{\pgfqpoint{8.593979in}{2.279105in}}%
\pgfpathlineto{\pgfqpoint{8.596005in}{6.769337in}}%
\pgfpathlineto{\pgfqpoint{8.598030in}{2.503052in}}%
\pgfpathlineto{\pgfqpoint{8.600055in}{6.544051in}}%
\pgfpathlineto{\pgfqpoint{8.602081in}{2.726883in}}%
\pgfpathlineto{\pgfqpoint{8.604106in}{6.324300in}}%
\pgfpathlineto{\pgfqpoint{8.606131in}{2.940154in}}%
\pgfpathlineto{\pgfqpoint{8.608156in}{6.119634in}}%
\pgfpathlineto{\pgfqpoint{8.610182in}{3.134407in}}%
\pgfpathlineto{\pgfqpoint{8.612207in}{5.937252in}}%
\pgfpathlineto{\pgfqpoint{8.614232in}{3.303836in}}%
\pgfpathlineto{\pgfqpoint{8.616257in}{5.781462in}}%
\pgfpathlineto{\pgfqpoint{8.618283in}{3.445702in}}%
\pgfpathlineto{\pgfqpoint{8.620308in}{5.653408in}}%
\pgfpathlineto{\pgfqpoint{8.622333in}{3.560441in}}%
\pgfpathlineto{\pgfqpoint{8.624359in}{5.551130in}}%
\pgfpathlineto{\pgfqpoint{8.626384in}{3.651434in}}%
\pgfpathlineto{\pgfqpoint{8.628409in}{5.469974in}}%
\pgfpathlineto{\pgfqpoint{8.630434in}{3.724413in}}%
\pgfpathlineto{\pgfqpoint{8.632460in}{5.403364in}}%
\pgfpathlineto{\pgfqpoint{8.634485in}{3.786534in}}%
\pgfpathlineto{\pgfqpoint{8.636510in}{5.343855in}}%
\pgfpathlineto{\pgfqpoint{8.638535in}{3.845226in}}%
\pgfpathlineto{\pgfqpoint{8.640561in}{5.284337in}}%
\pgfpathlineto{\pgfqpoint{8.642586in}{3.906998in}}%
\pgfpathlineto{\pgfqpoint{8.644611in}{5.219153in}}%
\pgfpathlineto{\pgfqpoint{8.646637in}{3.976443in}}%
\pgfpathlineto{\pgfqpoint{8.648662in}{5.144924in}}%
\pgfpathlineto{\pgfqpoint{8.650687in}{4.055652in}}%
\pgfpathlineto{\pgfqpoint{8.652712in}{5.060849in}}%
\pgfpathlineto{\pgfqpoint{8.654738in}{4.144208in}}%
\pgfpathlineto{\pgfqpoint{8.656763in}{4.968415in}}%
\pgfpathlineto{\pgfqpoint{8.658788in}{4.239768in}}%
\pgfpathlineto{\pgfqpoint{8.660813in}{4.870554in}}%
\pgfpathlineto{\pgfqpoint{8.662839in}{4.339112in}}%
\pgfpathlineto{\pgfqpoint{8.664864in}{4.770455in}}%
\pgfpathlineto{\pgfqpoint{8.666889in}{4.439398in}}%
\pgfpathlineto{\pgfqpoint{8.668915in}{4.670327in}}%
\pgfpathlineto{\pgfqpoint{8.670940in}{4.539293in}}%
\pgfpathlineto{\pgfqpoint{8.672965in}{4.570445in}}%
\pgfpathlineto{\pgfqpoint{8.674990in}{4.639685in}}%
\pgfpathlineto{\pgfqpoint{8.677016in}{4.468739in}}%
\pgfpathlineto{\pgfqpoint{8.679041in}{4.743754in}}%
\pgfpathlineto{\pgfqpoint{8.681066in}{4.361069in}}%
\pgfpathlineto{\pgfqpoint{8.683091in}{4.856373in}}%
\pgfpathlineto{\pgfqpoint{8.685117in}{4.242126in}}%
\pgfpathlineto{\pgfqpoint{8.687142in}{4.982947in}}%
\pgfpathlineto{\pgfqpoint{8.689167in}{4.106777in}}%
\pgfpathlineto{\pgfqpoint{8.691193in}{5.127965in}}%
\pgfpathlineto{\pgfqpoint{8.693218in}{3.951525in}}%
\pgfpathlineto{\pgfqpoint{8.695243in}{5.293626in}}%
\pgfpathlineto{\pgfqpoint{8.697268in}{3.775713in}}%
\pgfpathlineto{\pgfqpoint{8.699294in}{5.478880in}}%
\pgfpathlineto{\pgfqpoint{8.701319in}{3.582171in}}%
\pgfpathlineto{\pgfqpoint{8.703344in}{5.679147in}}%
\pgfpathlineto{\pgfqpoint{8.705369in}{3.377097in}}%
\pgfpathlineto{\pgfqpoint{8.707395in}{5.886833in}}%
\pgfpathlineto{\pgfqpoint{8.709420in}{3.169178in}}%
\pgfpathlineto{\pgfqpoint{8.711445in}{6.092525in}}%
\pgfpathlineto{\pgfqpoint{8.713471in}{2.968146in}}%
\pgfpathlineto{\pgfqpoint{8.715496in}{6.286595in}}%
\pgfpathlineto{\pgfqpoint{8.717521in}{2.783118in}}%
\pgfpathlineto{\pgfqpoint{8.719546in}{6.460797in}}%
\pgfpathlineto{\pgfqpoint{8.721572in}{2.621172in}}%
\pgfpathlineto{\pgfqpoint{8.723597in}{6.609436in}}%
\pgfpathlineto{\pgfqpoint{8.725622in}{2.486513in}}%
\pgfpathlineto{\pgfqpoint{8.727647in}{6.729794in}}%
\pgfpathlineto{\pgfqpoint{8.729673in}{2.380474in}}%
\pgfpathlineto{\pgfqpoint{8.731698in}{6.821727in}}%
\pgfpathlineto{\pgfqpoint{8.733723in}{2.302288in}}%
\pgfpathlineto{\pgfqpoint{8.735749in}{6.886579in}}%
\pgfpathlineto{\pgfqpoint{8.737774in}{2.250388in}}%
\pgfpathlineto{\pgfqpoint{8.739799in}{6.925798in}}%
\pgfpathlineto{\pgfqpoint{8.741824in}{2.223754in}}%
\pgfpathlineto{\pgfqpoint{8.743850in}{6.939730in}}%
\pgfpathlineto{\pgfqpoint{8.745875in}{2.222869in}}%
\pgfpathlineto{\pgfqpoint{8.747900in}{6.927014in}}%
\pgfpathlineto{\pgfqpoint{8.749925in}{2.249910in}}%
\pgfpathlineto{\pgfqpoint{8.751951in}{6.884820in}}%
\pgfpathlineto{\pgfqpoint{8.753976in}{2.308095in}}%
\pgfpathlineto{\pgfqpoint{8.756001in}{6.809891in}}%
\pgfpathlineto{\pgfqpoint{8.758027in}{2.400332in}}%
\pgfpathlineto{\pgfqpoint{8.760052in}{6.700078in}}%
\pgfpathlineto{\pgfqpoint{8.762077in}{2.527606in}}%
\pgfpathlineto{\pgfqpoint{8.764102in}{6.555913in}}%
\pgfpathlineto{\pgfqpoint{8.766128in}{2.687589in}}%
\pgfpathlineto{\pgfqpoint{8.768153in}{6.381703in}}%
\pgfpathlineto{\pgfqpoint{8.770178in}{2.873930in}}%
\pgfpathlineto{\pgfqpoint{8.772203in}{6.185784in}}%
\pgfpathlineto{\pgfqpoint{8.774229in}{3.076490in}}%
\pgfpathlineto{\pgfqpoint{8.776254in}{5.979806in}}%
\pgfpathlineto{\pgfqpoint{8.778279in}{3.282499in}}%
\pgfpathlineto{\pgfqpoint{8.780305in}{5.777190in}}%
\pgfpathlineto{\pgfqpoint{8.782330in}{3.478394in}}%
\pgfpathlineto{\pgfqpoint{8.784355in}{5.591117in}}%
\pgfpathlineto{\pgfqpoint{8.786380in}{3.651887in}}%
\pgfpathlineto{\pgfqpoint{8.788406in}{5.432518in}}%
\pgfpathlineto{\pgfqpoint{8.790431in}{3.793804in}}%
\pgfpathlineto{\pgfqpoint{8.792456in}{5.308495in}}%
\pgfpathlineto{\pgfqpoint{8.794481in}{3.899322in}}%
\pgfpathlineto{\pgfqpoint{8.796507in}{5.221488in}}%
\pgfpathlineto{\pgfqpoint{8.798532in}{3.968388in}}%
\pgfpathlineto{\pgfqpoint{8.800557in}{5.169268in}}%
\pgfpathlineto{\pgfqpoint{8.802583in}{4.005319in}}%
\pgfpathlineto{\pgfqpoint{8.804608in}{5.145691in}}%
\pgfpathlineto{\pgfqpoint{8.806633in}{4.017761in}}%
\pgfpathlineto{\pgfqpoint{8.808658in}{5.141981in}}%
\pgfpathlineto{\pgfqpoint{8.810684in}{4.015228in}}%
\pgfpathlineto{\pgfqpoint{8.812709in}{5.148280in}}%
\pgfpathlineto{\pgfqpoint{8.814734in}{4.007530in}}%
\pgfpathlineto{\pgfqpoint{8.816759in}{5.155199in}}%
\pgfpathlineto{\pgfqpoint{8.818785in}{4.003300in}}%
\pgfpathlineto{\pgfqpoint{8.820810in}{5.155160in}}%
\pgfpathlineto{\pgfqpoint{8.822835in}{4.008811in}}%
\pgfpathlineto{\pgfqpoint{8.824861in}{5.143386in}}%
\pgfpathlineto{\pgfqpoint{8.826886in}{4.027211in}}%
\pgfpathlineto{\pgfqpoint{8.828911in}{5.118429in}}%
\pgfpathlineto{\pgfqpoint{8.830936in}{4.058241in}}%
\pgfpathlineto{\pgfqpoint{8.832962in}{5.082190in}}%
\pgfpathlineto{\pgfqpoint{8.834987in}{4.098491in}}%
\pgfpathlineto{\pgfqpoint{8.837012in}{5.039396in}}%
\pgfpathlineto{\pgfqpoint{8.839037in}{4.142171in}}%
\pgfpathlineto{\pgfqpoint{8.841063in}{4.996594in}}%
\pgfpathlineto{\pgfqpoint{8.843088in}{4.182323in}}%
\pgfpathlineto{\pgfqpoint{8.845113in}{4.960776in}}%
\pgfpathlineto{\pgfqpoint{8.847139in}{4.212307in}}%
\pgfpathlineto{\pgfqpoint{8.849164in}{4.937852in}}%
\pgfpathlineto{\pgfqpoint{8.851189in}{4.227297in}}%
\pgfpathlineto{\pgfqpoint{8.853214in}{4.931261in}}%
\pgfpathlineto{\pgfqpoint{8.855240in}{4.225473in}}%
\pgfpathlineto{\pgfqpoint{8.857265in}{4.941055in}}%
\pgfpathlineto{\pgfqpoint{8.859290in}{4.208597in}}%
\pgfpathlineto{\pgfqpoint{8.861315in}{4.963724in}}%
\pgfpathlineto{\pgfqpoint{8.863341in}{4.181749in}}%
\pgfpathlineto{\pgfqpoint{8.865366in}{4.992912in}}%
\pgfpathlineto{\pgfqpoint{8.867391in}{4.152167in}}%
\pgfpathlineto{\pgfqpoint{8.869417in}{5.020968in}}%
\pgfpathlineto{\pgfqpoint{8.871442in}{4.127388in}}%
\pgfpathlineto{\pgfqpoint{8.873467in}{5.041023in}}%
\pgfpathlineto{\pgfqpoint{8.875492in}{4.113077in}}%
\pgfpathlineto{\pgfqpoint{8.877518in}{5.049105in}}%
\pgfpathlineto{\pgfqpoint{8.879543in}{4.111102in}}%
\pgfpathlineto{\pgfqpoint{8.881568in}{5.045738in}}%
\pgfpathlineto{\pgfqpoint{8.883593in}{4.118403in}}%
\pgfpathlineto{\pgfqpoint{8.885619in}{5.036512in}}%
\pgfpathlineto{\pgfqpoint{8.887644in}{4.127030in}}%
\pgfpathlineto{\pgfqpoint{8.889669in}{5.031405in}}%
\pgfpathlineto{\pgfqpoint{8.891695in}{4.125450in}}%
\pgfpathlineto{\pgfqpoint{8.893720in}{5.042916in}}%
\pgfpathlineto{\pgfqpoint{8.895745in}{4.100860in}}%
\pgfpathlineto{\pgfqpoint{8.897770in}{5.083455in}}%
\pgfpathlineto{\pgfqpoint{8.899796in}{4.041944in}}%
\pgfpathlineto{\pgfqpoint{8.901821in}{5.162599in}}%
\pgfpathlineto{\pgfqpoint{8.903846in}{3.941412in}}%
\pgfpathlineto{\pgfqpoint{8.905871in}{5.284917in}}%
\pgfpathlineto{\pgfqpoint{8.907897in}{3.797702in}}%
\pgfpathlineto{\pgfqpoint{8.909922in}{5.448840in}}%
\pgfpathlineto{\pgfqpoint{8.911947in}{3.615483in}}%
\pgfpathlineto{\pgfqpoint{8.913973in}{5.646777in}}%
\pgfpathlineto{\pgfqpoint{8.915998in}{3.404964in}}%
\pgfpathlineto{\pgfqpoint{8.918023in}{5.866304in}}%
\pgfpathlineto{\pgfqpoint{8.920048in}{3.180316in}}%
\pgfpathlineto{\pgfqpoint{8.922074in}{6.091997in}}%
\pgfpathlineto{\pgfqpoint{8.924099in}{2.957725in}}%
\pgfpathlineto{\pgfqpoint{8.926124in}{6.307368in}}%
\pgfpathlineto{\pgfqpoint{8.928149in}{2.753581in}}%
\pgfpathlineto{\pgfqpoint{8.930175in}{6.496451in}}%
\pgfpathlineto{\pgfqpoint{8.932200in}{2.583176in}}%
\pgfpathlineto{\pgfqpoint{8.934225in}{6.644807in}}%
\pgfpathlineto{\pgfqpoint{8.936251in}{2.459979in}}%
\pgfpathlineto{\pgfqpoint{8.938276in}{6.740004in}}%
\pgfpathlineto{\pgfqpoint{8.940301in}{2.395340in}}%
\pgfpathlineto{\pgfqpoint{8.942326in}{6.771823in}}%
\pgfpathlineto{\pgfqpoint{8.944352in}{2.398282in}}%
\pgfpathlineto{\pgfqpoint{8.946377in}{6.732538in}}%
\pgfpathlineto{\pgfqpoint{8.948402in}{2.475083in}}%
\pgfpathlineto{\pgfqpoint{8.950427in}{6.617520in}}%
\pgfpathlineto{\pgfqpoint{8.952453in}{2.628479in}}%
\pgfpathlineto{\pgfqpoint{8.954478in}{6.426200in}}%
\pgfpathlineto{\pgfqpoint{8.956503in}{2.856586in}}%
\pgfpathlineto{\pgfqpoint{8.958529in}{6.163178in}}%
\pgfpathlineto{\pgfqpoint{8.960554in}{3.151881in}}%
\pgfpathlineto{\pgfqpoint{8.962579in}{5.839031in}}%
\pgfpathlineto{\pgfqpoint{8.964604in}{3.500713in}}%
\pgfpathlineto{\pgfqpoint{8.966630in}{5.470363in}}%
\pgfpathlineto{\pgfqpoint{8.968655in}{3.883790in}}%
\pgfpathlineto{\pgfqpoint{8.970680in}{5.078736in}}%
\pgfpathlineto{\pgfqpoint{8.972705in}{4.277848in}}%
\pgfpathlineto{\pgfqpoint{8.974731in}{4.688431in}}%
\pgfpathlineto{\pgfqpoint{8.976756in}{4.658364in}}%
\pgfpathlineto{\pgfqpoint{8.978781in}{4.323380in}}%
\pgfpathlineto{\pgfqpoint{8.980807in}{5.002829in}}%
\pgfpathlineto{\pgfqpoint{8.982832in}{4.003896in}}%
\pgfpathlineto{\pgfqpoint{8.984857in}{5.293803in}}%
\pgfpathlineto{\pgfqpoint{8.986882in}{3.744009in}}%
\pgfpathlineto{\pgfqpoint{8.988908in}{5.521011in}}%
\pgfpathlineto{\pgfqpoint{8.990933in}{3.550101in}}%
\pgfpathlineto{\pgfqpoint{8.992958in}{5.681890in}}%
\pgfpathlineto{\pgfqpoint{8.994983in}{3.421207in}}%
\pgfpathlineto{\pgfqpoint{8.997009in}{5.780457in}}%
\pgfpathlineto{\pgfqpoint{8.999034in}{3.350887in}}%
\pgfpathlineto{\pgfqpoint{9.001059in}{5.824836in}}%
\pgfpathlineto{\pgfqpoint{9.003085in}{3.330116in}}%
\pgfpathlineto{\pgfqpoint{9.005110in}{5.824180in}}%
\pgfpathlineto{\pgfqpoint{9.007135in}{3.350323in}}%
\pgfpathlineto{\pgfqpoint{9.009160in}{5.785882in}}%
\pgfpathlineto{\pgfqpoint{9.011186in}{3.405727in}}%
\pgfpathlineto{\pgfqpoint{9.013211in}{5.713866in}}%
\pgfpathlineto{\pgfqpoint{9.015236in}{3.494311in}}%
\pgfpathlineto{\pgfqpoint{9.017261in}{5.608386in}}%
\pgfpathlineto{\pgfqpoint{9.019287in}{3.617271in}}%
\pgfpathlineto{\pgfqpoint{9.021312in}{5.467253in}}%
\pgfpathlineto{\pgfqpoint{9.023337in}{3.777229in}}%
\pgfpathlineto{\pgfqpoint{9.025363in}{5.288009in}}%
\pgfpathlineto{\pgfqpoint{9.027388in}{3.975893in}}%
\pgfpathlineto{\pgfqpoint{9.029413in}{5.070229in}}%
\pgfpathlineto{\pgfqpoint{9.031438in}{4.211973in}}%
\pgfpathlineto{\pgfqpoint{9.033464in}{4.817211in}}%
\pgfpathlineto{\pgfqpoint{9.035489in}{4.480029in}}%
\pgfpathlineto{\pgfqpoint{9.037514in}{4.536504in}}%
\pgfpathlineto{\pgfqpoint{9.039539in}{4.770593in}}%
\pgfpathlineto{\pgfqpoint{9.041565in}{4.239175in}}%
\pgfpathlineto{\pgfqpoint{9.043590in}{5.071423in}}%
\pgfpathlineto{\pgfqpoint{9.045615in}{3.938151in}}%
\pgfpathlineto{\pgfqpoint{9.047641in}{5.369416in}}%
\pgfpathlineto{\pgfqpoint{9.049666in}{3.646233in}}%
\pgfpathlineto{\pgfqpoint{9.051691in}{5.652473in}}%
\pgfpathlineto{\pgfqpoint{9.053716in}{3.374516in}}%
\pgfpathlineto{\pgfqpoint{9.055742in}{5.910684in}}%
\pgfpathlineto{\pgfqpoint{9.057767in}{3.131690in}}%
\pgfpathlineto{\pgfqpoint{9.059792in}{6.136482in}}%
\pgfpathlineto{\pgfqpoint{9.061817in}{2.924405in}}%
\pgfpathlineto{\pgfqpoint{9.063843in}{6.323837in}}%
\pgfpathlineto{\pgfqpoint{9.065868in}{2.758417in}}%
\pgfpathlineto{\pgfqpoint{9.067893in}{6.466925in}}%
\pgfpathlineto{\pgfqpoint{9.069919in}{2.639916in}}%
\pgfpathlineto{\pgfqpoint{9.071944in}{6.558976in}}%
\pgfpathlineto{\pgfqpoint{9.073969in}{2.576338in}}%
\pgfpathlineto{\pgfqpoint{9.075994in}{6.591952in}}%
\pgfpathlineto{\pgfqpoint{9.078020in}{2.576099in}}%
\pgfpathlineto{\pgfqpoint{9.080045in}{6.557441in}}%
\pgfpathlineto{\pgfqpoint{9.082070in}{2.647088in}}%
\pgfpathlineto{\pgfqpoint{9.084095in}{6.448707in}}%
\pgfpathlineto{\pgfqpoint{9.086121in}{2.794200in}}%
\pgfpathlineto{\pgfqpoint{9.088146in}{6.263379in}}%
\pgfpathlineto{\pgfqpoint{9.090171in}{3.016651in}}%
\pgfpathlineto{\pgfqpoint{9.092197in}{6.005920in}}%
\pgfpathlineto{\pgfqpoint{9.094222in}{3.305949in}}%
\pgfpathlineto{\pgfqpoint{9.096247in}{5.688986in}}%
\pgfpathlineto{\pgfqpoint{9.098272in}{3.645369in}}%
\pgfpathlineto{\pgfqpoint{9.100298in}{5.333027in}}%
\pgfpathlineto{\pgfqpoint{9.102323in}{4.011326in}}%
\pgfpathlineto{\pgfqpoint{9.104348in}{4.963957in}}%
\pgfpathlineto{\pgfqpoint{9.106373in}{4.376555in}}%
\pgfpathlineto{\pgfqpoint{9.110424in}{4.714410in}}%
\pgfpathlineto{\pgfqpoint{9.112449in}{4.293706in}}%
\pgfpathlineto{\pgfqpoint{9.114475in}{5.003276in}}%
\pgfpathlineto{\pgfqpoint{9.116500in}{4.034889in}}%
\pgfpathlineto{\pgfqpoint{9.118525in}{5.229980in}}%
\pgfpathlineto{\pgfqpoint{9.120550in}{3.841074in}}%
\pgfpathlineto{\pgfqpoint{9.122576in}{5.391378in}}%
\pgfpathlineto{\pgfqpoint{9.124601in}{3.710482in}}%
\pgfpathlineto{\pgfqpoint{9.126626in}{5.493753in}}%
\pgfpathlineto{\pgfqpoint{9.128651in}{3.632968in}}%
\pgfpathlineto{\pgfqpoint{9.130677in}{5.550284in}}%
\pgfpathlineto{\pgfqpoint{9.132702in}{3.593278in}}%
\pgfpathlineto{\pgfqpoint{9.134727in}{5.577283in}}%
\pgfpathlineto{\pgfqpoint{9.136753in}{3.575047in}}%
\pgfpathlineto{\pgfqpoint{9.138778in}{5.590231in}}%
\pgfpathlineto{\pgfqpoint{9.140803in}{3.564501in}}%
\pgfpathlineto{\pgfqpoint{9.142828in}{5.600540in}}%
\pgfpathlineto{\pgfqpoint{9.144854in}{3.553040in}}%
\pgfpathlineto{\pgfqpoint{9.146879in}{5.613759in}}%
\pgfpathlineto{\pgfqpoint{9.148904in}{3.538197in}}%
\pgfpathlineto{\pgfqpoint{9.150929in}{5.629436in}}%
\pgfpathlineto{\pgfqpoint{9.152955in}{3.523008in}}%
\pgfpathlineto{\pgfqpoint{9.154980in}{5.642425in}}%
\pgfpathlineto{\pgfqpoint{9.157005in}{3.514169in}}%
\pgfpathlineto{\pgfqpoint{9.159031in}{5.645082in}}%
\pgfpathlineto{\pgfqpoint{9.161056in}{3.519664in}}%
\pgfpathlineto{\pgfqpoint{9.163081in}{5.629656in}}%
\pgfpathlineto{\pgfqpoint{9.165106in}{3.546513in}}%
\pgfpathlineto{\pgfqpoint{9.167132in}{5.590251in}}%
\pgfpathlineto{\pgfqpoint{9.169157in}{3.599200in}}%
\pgfpathlineto{\pgfqpoint{9.171182in}{5.523968in}}%
\pgfpathlineto{\pgfqpoint{9.173208in}{3.678992in}}%
\pgfpathlineto{\pgfqpoint{9.175233in}{5.431117in}}%
\pgfpathlineto{\pgfqpoint{9.177258in}{3.784142in}}%
\pgfpathlineto{\pgfqpoint{9.179283in}{5.314674in}}%
\pgfpathlineto{\pgfqpoint{9.181309in}{3.910698in}}%
\pgfpathlineto{\pgfqpoint{9.183334in}{5.179291in}}%
\pgfpathlineto{\pgfqpoint{9.185359in}{4.053581in}}%
\pgfpathlineto{\pgfqpoint{9.187384in}{5.030219in}}%
\pgfpathlineto{\pgfqpoint{9.189410in}{4.207593in}}%
\pgfpathlineto{\pgfqpoint{9.191435in}{4.872420in}}%
\pgfpathlineto{\pgfqpoint{9.193460in}{4.368146in}}%
\pgfpathlineto{\pgfqpoint{9.195486in}{4.710017in}}%
\pgfpathlineto{\pgfqpoint{9.197511in}{4.531631in}}%
\pgfpathlineto{\pgfqpoint{9.199536in}{4.546089in}}%
\pgfpathlineto{\pgfqpoint{9.201561in}{4.695480in}}%
\pgfpathlineto{\pgfqpoint{9.203587in}{4.382736in}}%
\pgfpathlineto{\pgfqpoint{9.205612in}{4.858012in}}%
\pgfpathlineto{\pgfqpoint{9.207637in}{4.221279in}}%
\pgfpathlineto{\pgfqpoint{9.209662in}{5.018194in}}%
\pgfpathlineto{\pgfqpoint{9.211688in}{4.062530in}}%
\pgfpathlineto{\pgfqpoint{9.213713in}{5.175371in}}%
\pgfpathlineto{\pgfqpoint{9.215738in}{3.907062in}}%
\pgfpathlineto{\pgfqpoint{9.217764in}{5.328979in}}%
\pgfpathlineto{\pgfqpoint{9.219789in}{3.755504in}}%
\pgfpathlineto{\pgfqpoint{9.221814in}{5.478241in}}%
\pgfpathlineto{\pgfqpoint{9.223839in}{3.608866in}}%
\pgfpathlineto{\pgfqpoint{9.225865in}{5.621823in}}%
\pgfpathlineto{\pgfqpoint{9.227890in}{3.468895in}}%
\pgfpathlineto{\pgfqpoint{9.229915in}{5.757485in}}%
\pgfpathlineto{\pgfqpoint{9.231940in}{3.338393in}}%
\pgfpathlineto{\pgfqpoint{9.233966in}{5.881815in}}%
\pgfpathlineto{\pgfqpoint{9.235991in}{3.221403in}}%
\pgfpathlineto{\pgfqpoint{9.238016in}{5.990155in}}%
\pgfpathlineto{\pgfqpoint{9.240042in}{3.123142in}}%
\pgfpathlineto{\pgfqpoint{9.242067in}{6.076821in}}%
\pgfpathlineto{\pgfqpoint{9.244092in}{3.049630in}}%
\pgfpathlineto{\pgfqpoint{9.246117in}{6.135623in}}%
\pgfpathlineto{\pgfqpoint{9.248143in}{3.007040in}}%
\pgfpathlineto{\pgfqpoint{9.250168in}{6.160607in}}%
\pgfpathlineto{\pgfqpoint{9.252193in}{3.000903in}}%
\pgfpathlineto{\pgfqpoint{9.254218in}{6.146846in}}%
\pgfpathlineto{\pgfqpoint{9.256244in}{3.035389in}}%
\pgfpathlineto{\pgfqpoint{9.258269in}{6.091049in}}%
\pgfpathlineto{\pgfqpoint{9.260294in}{3.112837in}}%
\pgfpathlineto{\pgfqpoint{9.262320in}{5.991855in}}%
\pgfpathlineto{\pgfqpoint{9.264345in}{3.233647in}}%
\pgfpathlineto{\pgfqpoint{9.266370in}{5.849766in}}%
\pgfpathlineto{\pgfqpoint{9.268395in}{3.396495in}}%
\pgfpathlineto{\pgfqpoint{9.270421in}{5.666837in}}%
\pgfpathlineto{\pgfqpoint{9.272446in}{3.598688in}}%
\pgfpathlineto{\pgfqpoint{9.274471in}{5.446325in}}%
\pgfpathlineto{\pgfqpoint{9.276496in}{3.836443in}}%
\pgfpathlineto{\pgfqpoint{9.278522in}{5.192539in}}%
\pgfpathlineto{\pgfqpoint{9.280547in}{4.104889in}}%
\pgfpathlineto{\pgfqpoint{9.282572in}{4.910993in}}%
\pgfpathlineto{\pgfqpoint{9.284598in}{4.397753in}}%
\pgfpathlineto{\pgfqpoint{9.288648in}{4.706853in}}%
\pgfpathlineto{\pgfqpoint{9.290673in}{4.295428in}}%
\pgfpathlineto{\pgfqpoint{9.292699in}{5.021641in}}%
\pgfpathlineto{\pgfqpoint{9.294724in}{3.982535in}}%
\pgfpathlineto{\pgfqpoint{9.296749in}{5.329095in}}%
\pgfpathlineto{\pgfqpoint{9.298774in}{3.684303in}}%
\pgfpathlineto{\pgfqpoint{9.300800in}{5.614151in}}%
\pgfpathlineto{\pgfqpoint{9.302825in}{3.416466in}}%
\pgfpathlineto{\pgfqpoint{9.304850in}{5.860729in}}%
\pgfpathlineto{\pgfqpoint{9.306876in}{3.195083in}}%
\pgfpathlineto{\pgfqpoint{9.308901in}{6.053189in}}%
\pgfpathlineto{\pgfqpoint{9.310926in}{3.034965in}}%
\pgfpathlineto{\pgfqpoint{9.312951in}{6.177951in}}%
\pgfpathlineto{\pgfqpoint{9.314977in}{2.948090in}}%
\pgfpathlineto{\pgfqpoint{9.317002in}{6.224958in}}%
\pgfpathlineto{\pgfqpoint{9.319027in}{2.942328in}}%
\pgfpathlineto{\pgfqpoint{9.321052in}{6.188730in}}%
\pgfpathlineto{\pgfqpoint{9.323078in}{3.020638in}}%
\pgfpathlineto{\pgfqpoint{9.325103in}{6.068900in}}%
\pgfpathlineto{\pgfqpoint{9.327128in}{3.180786in}}%
\pgfpathlineto{\pgfqpoint{9.329154in}{5.870256in}}%
\pgfpathlineto{\pgfqpoint{9.331179in}{3.415511in}}%
\pgfpathlineto{\pgfqpoint{9.333204in}{5.602422in}}%
\pgfpathlineto{\pgfqpoint{9.335229in}{3.712965in}}%
\pgfpathlineto{\pgfqpoint{9.337255in}{5.279313in}}%
\pgfpathlineto{\pgfqpoint{9.339280in}{4.057336in}}%
\pgfpathlineto{\pgfqpoint{9.341305in}{4.918449in}}%
\pgfpathlineto{\pgfqpoint{9.343330in}{4.429606in}}%
\pgfpathlineto{\pgfqpoint{9.345356in}{4.540108in}}%
\pgfpathlineto{\pgfqpoint{9.347381in}{4.808519in}}%
\pgfpathlineto{\pgfqpoint{9.349406in}{4.166194in}}%
\pgfpathlineto{\pgfqpoint{9.351432in}{5.171900in}}%
\pgfpathlineto{\pgfqpoint{9.353457in}{3.818718in}}%
\pgfpathlineto{\pgfqpoint{9.355482in}{5.498401in}}%
\pgfpathlineto{\pgfqpoint{9.357507in}{3.517817in}}%
\pgfpathlineto{\pgfqpoint{9.359533in}{5.769672in}}%
\pgfpathlineto{\pgfqpoint{9.361558in}{3.279459in}}%
\pgfpathlineto{\pgfqpoint{9.363583in}{5.972715in}}%
\pgfpathlineto{\pgfqpoint{9.365608in}{3.113135in}}%
\pgfpathlineto{\pgfqpoint{9.367634in}{6.102009in}}%
\pgfpathlineto{\pgfqpoint{9.369659in}{3.020039in}}%
\pgfpathlineto{\pgfqpoint{9.371684in}{6.160891in}}%
\pgfpathlineto{\pgfqpoint{9.373710in}{2.992269in}}%
\pgfpathlineto{\pgfqpoint{9.375735in}{6.161678in}}%
\pgfpathlineto{\pgfqpoint{9.377760in}{3.013449in}}%
\pgfpathlineto{\pgfqpoint{9.379785in}{6.124249in}}%
\pgfpathlineto{\pgfqpoint{9.381811in}{3.060935in}}%
\pgfpathlineto{\pgfqpoint{9.383836in}{6.073115in}}%
\pgfpathlineto{\pgfqpoint{9.385861in}{3.109364in}}%
\pgfpathlineto{\pgfqpoint{9.387886in}{6.033405in}}%
\pgfpathlineto{\pgfqpoint{9.389912in}{3.134948in}}%
\pgfpathlineto{\pgfqpoint{9.391937in}{6.026502in}}%
\pgfpathlineto{\pgfqpoint{9.393962in}{3.119670in}}%
\pgfpathlineto{\pgfqpoint{9.395988in}{6.066251in}}%
\pgfpathlineto{\pgfqpoint{9.398013in}{3.054462in}}%
\pgfpathlineto{\pgfqpoint{9.400038in}{6.156577in}}%
\pgfpathlineto{\pgfqpoint{9.402063in}{2.940648in}}%
\pgfpathlineto{\pgfqpoint{9.404089in}{6.291069in}}%
\pgfpathlineto{\pgfqpoint{9.406114in}{2.789307in}}%
\pgfpathlineto{\pgfqpoint{9.408139in}{6.454624in}}%
\pgfpathlineto{\pgfqpoint{9.410164in}{2.618730in}}%
\pgfpathlineto{\pgfqpoint{9.412190in}{6.626741in}}%
\pgfpathlineto{\pgfqpoint{9.414215in}{2.450575in}}%
\pgfpathlineto{\pgfqpoint{9.416240in}{6.785668in}}%
\pgfpathlineto{\pgfqpoint{9.418266in}{2.305677in}}%
\pgfpathlineto{\pgfqpoint{9.420291in}{6.912391in}}%
\pgfpathlineto{\pgfqpoint{9.422316in}{2.200485in}}%
\pgfpathlineto{\pgfqpoint{9.424341in}{6.993567in}}%
\pgfpathlineto{\pgfqpoint{9.426367in}{2.144917in}}%
\pgfpathlineto{\pgfqpoint{9.428392in}{7.022799in}}%
\pgfpathlineto{\pgfqpoint{9.430417in}{2.141965in}}%
\pgfpathlineto{\pgfqpoint{9.432442in}{7.000191in}}%
\pgfpathlineto{\pgfqpoint{9.434468in}{2.188894in}}%
\pgfpathlineto{\pgfqpoint{9.436493in}{6.930542in}}%
\pgfpathlineto{\pgfqpoint{9.438518in}{2.279451in}}%
\pgfpathlineto{\pgfqpoint{9.440544in}{6.820954in}}%
\pgfpathlineto{\pgfqpoint{9.442569in}{2.406244in}}%
\pgfpathlineto{\pgfqpoint{9.444594in}{6.678651in}}%
\pgfpathlineto{\pgfqpoint{9.446619in}{2.562538in}}%
\pgfpathlineto{\pgfqpoint{9.448645in}{6.509698in}}%
\pgfpathlineto{\pgfqpoint{9.450670in}{2.742975in}}%
\pgfpathlineto{\pgfqpoint{9.452695in}{6.318855in}}%
\pgfpathlineto{\pgfqpoint{9.454720in}{2.943172in}}%
\pgfpathlineto{\pgfqpoint{9.456746in}{6.110417in}}%
\pgfpathlineto{\pgfqpoint{9.458771in}{3.158592in}}%
\pgfpathlineto{\pgfqpoint{9.460796in}{5.889484in}}%
\pgfpathlineto{\pgfqpoint{9.462822in}{3.383314in}}%
\pgfpathlineto{\pgfqpoint{9.464847in}{5.662970in}}%
\pgfpathlineto{\pgfqpoint{9.466872in}{3.609377in}}%
\pgfpathlineto{\pgfqpoint{9.468897in}{5.439792in}}%
\pgfpathlineto{\pgfqpoint{9.470923in}{3.827134in}}%
\pgfpathlineto{\pgfqpoint{9.472948in}{5.229973in}}%
\pgfpathlineto{\pgfqpoint{9.474973in}{4.026649in}}%
\pgfpathlineto{\pgfqpoint{9.476998in}{5.042835in}}%
\pgfpathlineto{\pgfqpoint{9.479024in}{4.199768in}}%
\pgfpathlineto{\pgfqpoint{9.481049in}{4.884826in}}%
\pgfpathlineto{\pgfqpoint{9.483074in}{4.342220in}}%
\pgfpathlineto{\pgfqpoint{9.485100in}{4.757677in}}%
\pgfpathlineto{\pgfqpoint{9.487125in}{4.455032in}}%
\pgfpathlineto{\pgfqpoint{9.489150in}{4.657555in}}%
\pgfpathlineto{\pgfqpoint{9.491175in}{4.544718in}}%
\pgfpathlineto{\pgfqpoint{9.495226in}{4.622071in}}%
\pgfpathlineto{\pgfqpoint{9.497251in}{4.499632in}}%
\pgfpathlineto{\pgfqpoint{9.499276in}{4.699763in}}%
\pgfpathlineto{\pgfqpoint{9.501302in}{4.417255in}}%
\pgfpathlineto{\pgfqpoint{9.503327in}{4.789322in}}%
\pgfpathlineto{\pgfqpoint{9.505352in}{4.318621in}}%
\pgfpathlineto{\pgfqpoint{9.507378in}{4.898180in}}%
\pgfpathlineto{\pgfqpoint{9.509403in}{4.199233in}}%
\pgfpathlineto{\pgfqpoint{9.511428in}{5.027497in}}%
\pgfpathlineto{\pgfqpoint{9.513453in}{4.061509in}}%
\pgfpathlineto{\pgfqpoint{9.515479in}{5.171224in}}%
\pgfpathlineto{\pgfqpoint{9.517504in}{3.914992in}}%
\pgfpathlineto{\pgfqpoint{9.519529in}{5.316627in}}%
\pgfpathlineto{\pgfqpoint{9.521554in}{3.775141in}}%
\pgfpathlineto{\pgfqpoint{9.523580in}{5.446131in}}%
\pgfpathlineto{\pgfqpoint{9.525605in}{3.660934in}}%
\pgfpathlineto{\pgfqpoint{9.527630in}{5.540148in}}%
\pgfpathlineto{\pgfqpoint{9.529656in}{3.591733in}}%
\pgfpathlineto{\pgfqpoint{9.531681in}{5.580374in}}%
\pgfpathlineto{\pgfqpoint{9.533706in}{3.583985in}}%
\pgfpathlineto{\pgfqpoint{9.535731in}{5.552961in}}%
\pgfpathlineto{\pgfqpoint{9.537757in}{3.648298in}}%
\pgfpathlineto{\pgfqpoint{9.539782in}{5.451050in}}%
\pgfpathlineto{\pgfqpoint{9.541807in}{3.787416in}}%
\pgfpathlineto{\pgfqpoint{9.543832in}{5.276210in}}%
\pgfpathlineto{\pgfqpoint{9.545858in}{3.995434in}}%
\pgfpathlineto{\pgfqpoint{9.547883in}{5.038529in}}%
\pgfpathlineto{\pgfqpoint{9.549908in}{4.258414in}}%
\pgfpathlineto{\pgfqpoint{9.551934in}{4.755303in}}%
\pgfpathlineto{\pgfqpoint{9.555984in}{4.448551in}}%
\pgfpathlineto{\pgfqpoint{9.558009in}{4.865985in}}%
\pgfpathlineto{\pgfqpoint{9.560035in}{4.141796in}}%
\pgfpathlineto{\pgfqpoint{9.562060in}{5.164368in}}%
\pgfpathlineto{\pgfqpoint{9.564085in}{3.856748in}}%
\pgfpathlineto{\pgfqpoint{9.566110in}{5.431767in}}%
\pgfpathlineto{\pgfqpoint{9.568136in}{3.610561in}}%
\pgfpathlineto{\pgfqpoint{9.570161in}{5.653987in}}%
\pgfpathlineto{\pgfqpoint{9.572186in}{3.414246in}}%
\pgfpathlineto{\pgfqpoint{9.574212in}{5.823236in}}%
\pgfpathlineto{\pgfqpoint{9.576237in}{3.272526in}}%
\pgfpathlineto{\pgfqpoint{9.578262in}{5.937541in}}%
\pgfpathlineto{\pgfqpoint{9.580287in}{3.185088in}}%
\pgfpathlineto{\pgfqpoint{9.582313in}{5.998940in}}%
\pgfpathlineto{\pgfqpoint{9.584338in}{3.148780in}}%
\pgfpathlineto{\pgfqpoint{9.586363in}{6.011072in}}%
\pgfpathlineto{\pgfqpoint{9.588388in}{3.160064in}}%
\pgfpathlineto{\pgfqpoint{9.590414in}{5.976885in}}%
\pgfpathlineto{\pgfqpoint{9.592439in}{3.216944in}}%
\pgfpathlineto{\pgfqpoint{9.594464in}{5.897218in}}%
\pgfpathlineto{\pgfqpoint{9.596490in}{3.319764in}}%
\pgfpathlineto{\pgfqpoint{9.598515in}{5.770687in}}%
\pgfpathlineto{\pgfqpoint{9.600540in}{3.470640in}}%
\pgfpathlineto{\pgfqpoint{9.602565in}{5.594892in}}%
\pgfpathlineto{\pgfqpoint{9.604591in}{3.671718in}}%
\pgfpathlineto{\pgfqpoint{9.606616in}{5.368521in}}%
\pgfpathlineto{\pgfqpoint{9.608641in}{3.922906in}}%
\pgfpathlineto{\pgfqpoint{9.610666in}{5.093575in}}%
\pgfpathlineto{\pgfqpoint{9.612692in}{4.219912in}}%
\pgfpathlineto{\pgfqpoint{9.614717in}{4.776857in}}%
\pgfpathlineto{\pgfqpoint{9.618768in}{4.430092in}}%
\pgfpathlineto{\pgfqpoint{9.620793in}{4.909607in}}%
\pgfpathlineto{\pgfqpoint{9.622818in}{4.068448in}}%
\pgfpathlineto{\pgfqpoint{9.624843in}{5.272639in}}%
\pgfpathlineto{\pgfqpoint{9.626869in}{3.707864in}}%
\pgfpathlineto{\pgfqpoint{9.628894in}{5.627326in}}%
\pgfpathlineto{\pgfqpoint{9.630919in}{3.361982in}}%
\pgfpathlineto{\pgfqpoint{9.632944in}{5.962152in}}%
\pgfpathlineto{\pgfqpoint{9.634970in}{3.039745in}}%
\pgfpathlineto{\pgfqpoint{9.636995in}{6.270976in}}%
\pgfpathlineto{\pgfqpoint{9.639020in}{2.744515in}}%
\pgfpathlineto{\pgfqpoint{9.641046in}{6.552953in}}%
\pgfpathlineto{\pgfqpoint{9.643071in}{2.475102in}}%
\pgfpathlineto{\pgfqpoint{9.645096in}{6.810633in}}%
\pgfpathlineto{\pgfqpoint{9.647121in}{2.228400in}}%
\pgfpathlineto{\pgfqpoint{9.649147in}{7.046824in}}%
\pgfpathlineto{\pgfqpoint{9.651172in}{2.002730in}}%
\pgfpathlineto{\pgfqpoint{9.653197in}{7.261346in}}%
\pgfpathlineto{\pgfqpoint{9.655222in}{1.800678in}}%
\pgfpathlineto{\pgfqpoint{9.657248in}{7.448898in}}%
\pgfpathlineto{\pgfqpoint{9.659273in}{1.630298in}}%
\pgfpathlineto{\pgfqpoint{9.661298in}{7.598930in}}%
\pgfpathlineto{\pgfqpoint{9.663324in}{1.504097in}}%
\pgfpathlineto{\pgfqpoint{9.665349in}{7.697758in}}%
\pgfpathlineto{\pgfqpoint{9.667374in}{1.435966in}}%
\pgfpathlineto{\pgfqpoint{9.669399in}{7.732368in}}%
\pgfpathlineto{\pgfqpoint{9.671425in}{1.436944in}}%
\pgfpathlineto{\pgfqpoint{9.673450in}{7.694716in}}%
\pgfpathlineto{\pgfqpoint{9.675475in}{1.511216in}}%
\pgfpathlineto{\pgfqpoint{9.677500in}{7.585094in}}%
\pgfpathlineto{\pgfqpoint{9.679526in}{1.653707in}}%
\pgfpathlineto{\pgfqpoint{9.681551in}{7.413330in}}%
\pgfpathlineto{\pgfqpoint{9.683576in}{1.850217in}}%
\pgfpathlineto{\pgfqpoint{9.685602in}{7.197279in}}%
\pgfpathlineto{\pgfqpoint{9.687627in}{2.080236in}}%
\pgfpathlineto{\pgfqpoint{9.689652in}{6.958888in}}%
\pgfpathlineto{\pgfqpoint{9.691677in}{2.321725in}}%
\pgfpathlineto{\pgfqpoint{9.693703in}{6.718928in}}%
\pgfpathlineto{\pgfqpoint{9.695728in}{2.556461in}}%
\pgfpathlineto{\pgfqpoint{9.697753in}{6.491971in}}%
\pgfpathlineto{\pgfqpoint{9.699778in}{2.774348in}}%
\pgfpathlineto{\pgfqpoint{9.701804in}{6.283153in}}%
\pgfpathlineto{\pgfqpoint{9.703829in}{2.975315in}}%
\pgfpathlineto{\pgfqpoint{9.705854in}{6.087770in}}%
\pgfpathlineto{\pgfqpoint{9.707880in}{3.168170in}}%
\pgfpathlineto{\pgfqpoint{9.709905in}{5.893914in}}%
\pgfpathlineto{\pgfqpoint{9.711930in}{3.366668in}}%
\pgfpathlineto{\pgfqpoint{9.713955in}{5.687400in}}%
\pgfpathlineto{\pgfqpoint{9.715981in}{3.583950in}}%
\pgfpathlineto{\pgfqpoint{9.718006in}{5.457538in}}%
\pgfpathlineto{\pgfqpoint{9.720031in}{3.827020in}}%
\pgfpathlineto{\pgfqpoint{9.722056in}{5.201971in}}%
\pgfpathlineto{\pgfqpoint{9.724082in}{4.092981in}}%
\pgfpathlineto{\pgfqpoint{9.726107in}{4.929052in}}%
\pgfpathlineto{\pgfqpoint{9.728132in}{4.368250in}}%
\pgfpathlineto{\pgfqpoint{9.730158in}{4.656946in}}%
\pgfpathlineto{\pgfqpoint{9.732183in}{4.631114in}}%
\pgfpathlineto{\pgfqpoint{9.734208in}{4.409580in}}%
\pgfpathlineto{\pgfqpoint{9.736233in}{4.856975in}}%
\pgfpathlineto{\pgfqpoint{9.738259in}{4.210571in}}%
\pgfpathlineto{\pgfqpoint{9.740284in}{5.024822in}}%
\pgfpathlineto{\pgfqpoint{9.742309in}{4.076843in}}%
\pgfpathlineto{\pgfqpoint{9.744334in}{5.123047in}}%
\pgfpathlineto{\pgfqpoint{9.746360in}{4.013812in}}%
\pgfpathlineto{\pgfqpoint{9.748385in}{5.152894in}}%
\pgfpathlineto{\pgfqpoint{9.750410in}{4.013556in}}%
\pgfpathlineto{\pgfqpoint{9.752436in}{5.128517in}}%
\pgfpathlineto{\pgfqpoint{9.754461in}{4.056549in}}%
\pgfpathlineto{\pgfqpoint{9.756486in}{5.073600in}}%
\pgfpathlineto{\pgfqpoint{9.758511in}{4.116440in}}%
\pgfpathlineto{\pgfqpoint{9.760537in}{5.015517in}}%
\pgfpathlineto{\pgfqpoint{9.762562in}{4.166508in}}%
\pgfpathlineto{\pgfqpoint{9.764587in}{4.978741in}}%
\pgfpathlineto{\pgfqpoint{9.766612in}{4.185917in}}%
\pgfpathlineto{\pgfqpoint{9.768638in}{4.979389in}}%
\pgfpathlineto{\pgfqpoint{9.770663in}{4.163984in}}%
\pgfpathlineto{\pgfqpoint{9.772688in}{5.022408in}}%
\pgfpathlineto{\pgfqpoint{9.774714in}{4.101366in}}%
\pgfpathlineto{\pgfqpoint{9.776739in}{5.102069in}}%
\pgfpathlineto{\pgfqpoint{9.778764in}{4.007999in}}%
\pgfpathlineto{\pgfqpoint{9.780789in}{5.205357in}}%
\pgfpathlineto{\pgfqpoint{9.782815in}{3.898691in}}%
\pgfpathlineto{\pgfqpoint{9.784840in}{5.316988in}}%
\pgfpathlineto{\pgfqpoint{9.786865in}{3.787955in}}%
\pgfpathlineto{\pgfqpoint{9.788890in}{5.424289in}}%
\pgfpathlineto{\pgfqpoint{9.790916in}{3.685830in}}%
\pgfpathlineto{\pgfqpoint{9.792941in}{5.520325in}}%
\pgfpathlineto{\pgfqpoint{9.794966in}{3.596037in}}%
\pgfpathlineto{\pgfqpoint{9.796992in}{5.604335in}}%
\pgfpathlineto{\pgfqpoint{9.799017in}{3.516943in}}%
\pgfpathlineto{\pgfqpoint{9.801042in}{5.679530in}}%
\pgfpathlineto{\pgfqpoint{9.803067in}{3.444750in}}%
\pgfpathlineto{\pgfqpoint{9.805093in}{5.749245in}}%
\pgfpathlineto{\pgfqpoint{9.807118in}{3.377578in}}%
\pgfpathlineto{\pgfqpoint{9.809143in}{5.813071in}}%
\pgfpathlineto{\pgfqpoint{9.811168in}{3.318706in}}%
\pgfpathlineto{\pgfqpoint{9.813194in}{5.864597in}}%
\pgfpathlineto{\pgfqpoint{9.815219in}{3.277578in}}%
\pgfpathlineto{\pgfqpoint{9.817244in}{5.891822in}}%
\pgfpathlineto{\pgfqpoint{9.819270in}{3.267934in}}%
\pgfpathlineto{\pgfqpoint{9.821295in}{5.880367in}}%
\pgfpathlineto{\pgfqpoint{9.823320in}{3.303492in}}%
\pgfpathlineto{\pgfqpoint{9.825345in}{5.818561in}}%
\pgfpathlineto{\pgfqpoint{9.827371in}{3.392523in}}%
\pgfpathlineto{\pgfqpoint{9.829396in}{5.702740in}}%
\pgfpathlineto{\pgfqpoint{9.831421in}{3.533138in}}%
\pgfpathlineto{\pgfqpoint{9.833446in}{5.540931in}}%
\pgfpathlineto{\pgfqpoint{9.835472in}{3.711008in}}%
\pgfpathlineto{\pgfqpoint{9.837497in}{5.353479in}}%
\pgfpathlineto{\pgfqpoint{9.839522in}{3.900510in}}%
\pgfpathlineto{\pgfqpoint{9.841548in}{5.170133in}}%
\pgfpathlineto{\pgfqpoint{9.843573in}{4.069261in}}%
\pgfpathlineto{\pgfqpoint{9.845598in}{5.024176in}}%
\pgfpathlineto{\pgfqpoint{9.847623in}{4.184932in}}%
\pgfpathlineto{\pgfqpoint{9.849649in}{4.945140in}}%
\pgfpathlineto{\pgfqpoint{9.851674in}{4.222499in}}%
\pgfpathlineto{\pgfqpoint{9.853699in}{4.952087in}}%
\pgfpathlineto{\pgfqpoint{9.855724in}{4.169956in}}%
\pgfpathlineto{\pgfqpoint{9.857750in}{5.049296in}}%
\pgfpathlineto{\pgfqpoint{9.859775in}{4.030954in}}%
\pgfpathlineto{\pgfqpoint{9.861800in}{5.225454in}}%
\pgfpathlineto{\pgfqpoint{9.863826in}{3.823761in}}%
\pgfpathlineto{\pgfqpoint{9.865851in}{5.456427in}}%
\pgfpathlineto{\pgfqpoint{9.867876in}{3.577004in}}%
\pgfpathlineto{\pgfqpoint{9.869901in}{5.710646in}}%
\pgfpathlineto{\pgfqpoint{9.871927in}{3.323573in}}%
\pgfpathlineto{\pgfqpoint{9.873952in}{5.955467in}}%
\pgfpathlineto{\pgfqpoint{9.875977in}{3.094461in}}%
\pgfpathlineto{\pgfqpoint{9.878002in}{6.162713in}}%
\pgfpathlineto{\pgfqpoint{9.880028in}{2.914168in}}%
\pgfpathlineto{\pgfqpoint{9.882053in}{6.312075in}}%
\pgfpathlineto{\pgfqpoint{9.884078in}{2.798640in}}%
\pgfpathlineto{\pgfqpoint{9.886104in}{6.391836in}}%
\pgfpathlineto{\pgfqpoint{9.888129in}{2.755758in}}%
\pgfpathlineto{\pgfqpoint{9.890154in}{6.397373in}}%
\pgfpathlineto{\pgfqpoint{9.892179in}{2.787560in}}%
\pgfpathlineto{\pgfqpoint{9.894205in}{6.328552in}}%
\pgfpathlineto{\pgfqpoint{9.896230in}{2.892884in}}%
\pgfpathlineto{\pgfqpoint{9.898255in}{6.187372in}}%
\pgfpathlineto{\pgfqpoint{9.900280in}{3.069153in}}%
\pgfpathlineto{\pgfqpoint{9.902306in}{5.976947in}}%
\pgfpathlineto{\pgfqpoint{9.904331in}{3.312543in}}%
\pgfpathlineto{\pgfqpoint{9.906356in}{5.702170in}}%
\pgfpathlineto{\pgfqpoint{9.908382in}{3.616597in}}%
\pgfpathlineto{\pgfqpoint{9.910407in}{5.371621in}}%
\pgfpathlineto{\pgfqpoint{9.912432in}{3.970074in}}%
\pgfpathlineto{\pgfqpoint{9.914457in}{4.999639in}}%
\pgfpathlineto{\pgfqpoint{9.916483in}{4.355277in}}%
\pgfpathlineto{\pgfqpoint{9.920533in}{4.748010in}}%
\pgfpathlineto{\pgfqpoint{9.922558in}{4.221366in}}%
\pgfpathlineto{\pgfqpoint{9.924584in}{5.119765in}}%
\pgfpathlineto{\pgfqpoint{9.926609in}{3.871035in}}%
\pgfpathlineto{\pgfqpoint{9.928634in}{5.441888in}}%
\pgfpathlineto{\pgfqpoint{9.930660in}{3.583115in}}%
\pgfpathlineto{\pgfqpoint{9.932685in}{5.690698in}}%
\pgfpathlineto{\pgfqpoint{9.934710in}{3.376998in}}%
\pgfpathlineto{\pgfqpoint{9.936735in}{5.852031in}}%
\pgfpathlineto{\pgfqpoint{9.938761in}{3.260978in}}%
\pgfpathlineto{\pgfqpoint{9.940786in}{5.923743in}}%
\pgfpathlineto{\pgfqpoint{9.942811in}{3.231160in}}%
\pgfpathlineto{\pgfqpoint{9.944836in}{5.915272in}}%
\pgfpathlineto{\pgfqpoint{9.946862in}{3.273419in}}%
\pgfpathlineto{\pgfqpoint{9.948887in}{5.844278in}}%
\pgfpathlineto{\pgfqpoint{9.950912in}{3.367911in}}%
\pgfpathlineto{\pgfqpoint{9.952938in}{5.731338in}}%
\pgfpathlineto{\pgfqpoint{9.954963in}{3.494766in}}%
\pgfpathlineto{\pgfqpoint{9.956988in}{5.594313in}}%
\pgfpathlineto{\pgfqpoint{9.959013in}{3.639194in}}%
\pgfpathlineto{\pgfqpoint{9.961039in}{5.444182in}}%
\pgfpathlineto{\pgfqpoint{9.963064in}{3.794382in}}%
\pgfpathlineto{\pgfqpoint{9.965089in}{5.283648in}}%
\pgfpathlineto{\pgfqpoint{9.967114in}{3.961274in}}%
\pgfpathlineto{\pgfqpoint{9.969140in}{5.108941in}}%
\pgfpathlineto{\pgfqpoint{9.971165in}{4.145363in}}%
\pgfpathlineto{\pgfqpoint{9.973190in}{4.914142in}}%
\pgfpathlineto{\pgfqpoint{9.975216in}{4.351626in}}%
\pgfpathlineto{\pgfqpoint{9.977241in}{4.696524in}}%
\pgfpathlineto{\pgfqpoint{9.981291in}{4.460969in}}%
\pgfpathlineto{\pgfqpoint{9.983317in}{4.819126in}}%
\pgfpathlineto{\pgfqpoint{9.985342in}{4.221823in}}%
\pgfpathlineto{\pgfqpoint{9.987367in}{5.052060in}}%
\pgfpathlineto{\pgfqpoint{9.989392in}{4.001352in}}%
\pgfpathlineto{\pgfqpoint{9.991418in}{5.253582in}}%
\pgfpathlineto{\pgfqpoint{9.993443in}{3.825081in}}%
\pgfpathlineto{\pgfqpoint{9.995468in}{5.398911in}}%
\pgfpathlineto{\pgfqpoint{9.997494in}{3.715372in}}%
\pgfpathlineto{\pgfqpoint{9.999519in}{5.469680in}}%
\pgfpathlineto{\pgfqpoint{10.001544in}{3.685243in}}%
\pgfpathlineto{\pgfqpoint{10.003569in}{5.459235in}}%
\pgfpathlineto{\pgfqpoint{10.005595in}{3.734403in}}%
\pgfpathlineto{\pgfqpoint{10.007620in}{5.374911in}}%
\pgfpathlineto{\pgfqpoint{10.009645in}{3.848870in}}%
\pgfpathlineto{\pgfqpoint{10.011670in}{5.236465in}}%
\pgfpathlineto{\pgfqpoint{10.013696in}{4.004399in}}%
\pgfpathlineto{\pgfqpoint{10.015721in}{5.071021in}}%
\pgfpathlineto{\pgfqpoint{10.017746in}{4.172789in}}%
\pgfpathlineto{\pgfqpoint{10.019772in}{4.906002in}}%
\pgfpathlineto{\pgfqpoint{10.021797in}{4.329173in}}%
\pgfpathlineto{\pgfqpoint{10.023822in}{4.762156in}}%
\pgfpathlineto{\pgfqpoint{10.025847in}{4.458143in}}%
\pgfpathlineto{\pgfqpoint{10.027873in}{4.648759in}}%
\pgfpathlineto{\pgfqpoint{10.029898in}{4.556860in}}%
\pgfpathlineto{\pgfqpoint{10.031923in}{4.562409in}}%
\pgfpathlineto{\pgfqpoint{10.033948in}{4.634292in}}%
\pgfpathlineto{\pgfqpoint{10.035974in}{4.489671in}}%
\pgfpathlineto{\pgfqpoint{10.037999in}{4.706920in}}%
\pgfpathlineto{\pgfqpoint{10.040024in}{4.412643in}}%
\pgfpathlineto{\pgfqpoint{10.042050in}{4.792368in}}%
\pgfpathlineto{\pgfqpoint{10.044075in}{4.315611in}}%
\pgfpathlineto{\pgfqpoint{10.046100in}{4.903010in}}%
\pgfpathlineto{\pgfqpoint{10.048125in}{4.190652in}}%
\pgfpathlineto{\pgfqpoint{10.050151in}{5.041610in}}%
\pgfpathlineto{\pgfqpoint{10.052176in}{4.040416in}}%
\pgfpathlineto{\pgfqpoint{10.054201in}{5.200316in}}%
\pgfpathlineto{\pgfqpoint{10.056227in}{3.877301in}}%
\pgfpathlineto{\pgfqpoint{10.058252in}{5.363227in}}%
\pgfpathlineto{\pgfqpoint{10.060277in}{3.719372in}}%
\pgfpathlineto{\pgfqpoint{10.062302in}{5.511622in}}%
\pgfpathlineto{\pgfqpoint{10.064328in}{3.584461in}}%
\pgfpathlineto{\pgfqpoint{10.066353in}{5.630022in}}%
\pgfpathlineto{\pgfqpoint{10.068378in}{3.484440in}}%
\pgfpathlineto{\pgfqpoint{10.070403in}{5.711088in}}%
\pgfpathlineto{\pgfqpoint{10.072429in}{3.421584in}}%
\pgfpathlineto{\pgfqpoint{10.074454in}{5.757704in}}%
\pgfpathlineto{\pgfqpoint{10.076479in}{3.388213in}}%
\pgfpathlineto{\pgfqpoint{10.078505in}{5.781569in}}%
\pgfpathlineto{\pgfqpoint{10.080530in}{3.369731in}}%
\pgfpathlineto{\pgfqpoint{10.082555in}{5.798789in}}%
\pgfpathlineto{\pgfqpoint{10.084580in}{3.350042in}}%
\pgfpathlineto{\pgfqpoint{10.086606in}{5.823927in}}%
\pgfpathlineto{\pgfqpoint{10.088631in}{3.317519in}}%
\pgfpathlineto{\pgfqpoint{10.090656in}{5.864520in}}%
\pgfpathlineto{\pgfqpoint{10.092681in}{3.269523in}}%
\pgfpathlineto{\pgfqpoint{10.094707in}{5.917920in}}%
\pgfpathlineto{\pgfqpoint{10.096732in}{3.213915in}}%
\pgfpathlineto{\pgfqpoint{10.098757in}{5.971588in}}%
\pgfpathlineto{\pgfqpoint{10.100783in}{3.166956in}}%
\pgfpathlineto{\pgfqpoint{10.102808in}{6.006847in}}%
\pgfpathlineto{\pgfqpoint{10.104833in}{3.148181in}}%
\pgfpathlineto{\pgfqpoint{10.106858in}{6.004988in}}%
\pgfpathlineto{\pgfqpoint{10.108884in}{3.173798in}}%
\pgfpathlineto{\pgfqpoint{10.110909in}{5.953835in}}%
\pgfpathlineto{\pgfqpoint{10.112934in}{3.250685in}}%
\pgfpathlineto{\pgfqpoint{10.114959in}{5.852705in}}%
\pgfpathlineto{\pgfqpoint{10.116985in}{3.372886in}}%
\pgfpathlineto{\pgfqpoint{10.119010in}{5.714155in}}%
\pgfpathlineto{\pgfqpoint{10.121035in}{3.521764in}}%
\pgfpathlineto{\pgfqpoint{10.123061in}{5.561922in}}%
\pgfpathlineto{\pgfqpoint{10.125086in}{3.669844in}}%
\pgfpathlineto{\pgfqpoint{10.127111in}{5.425580in}}%
\pgfpathlineto{\pgfqpoint{10.129136in}{3.787252in}}%
\pgfpathlineto{\pgfqpoint{10.131162in}{5.333474in}}%
\pgfpathlineto{\pgfqpoint{10.133187in}{3.848895in}}%
\pgfpathlineto{\pgfqpoint{10.135212in}{5.305953in}}%
\pgfpathlineto{\pgfqpoint{10.137237in}{3.840330in}}%
\pgfpathlineto{\pgfqpoint{10.139263in}{5.350796in}}%
\pgfpathlineto{\pgfqpoint{10.141288in}{3.760749in}}%
\pgfpathlineto{\pgfqpoint{10.143313in}{5.461990in}}%
\pgfpathlineto{\pgfqpoint{10.145339in}{3.622413in}}%
\pgfpathlineto{\pgfqpoint{10.147364in}{5.621963in}}%
\pgfpathlineto{\pgfqpoint{10.149389in}{3.446986in}}%
\pgfpathlineto{\pgfqpoint{10.151414in}{5.806348in}}%
\pgfpathlineto{\pgfqpoint{10.153440in}{3.260097in}}%
\pgfpathlineto{\pgfqpoint{10.155465in}{5.989646in}}%
\pgfpathlineto{\pgfqpoint{10.157490in}{3.085878in}}%
\pgfpathlineto{\pgfqpoint{10.159515in}{6.150087in}}%
\pgfpathlineto{\pgfqpoint{10.161541in}{2.943032in}}%
\pgfpathlineto{\pgfqpoint{10.163566in}{6.272417in}}%
\pgfpathlineto{\pgfqpoint{10.165591in}{2.843294in}}%
\pgfpathlineto{\pgfqpoint{10.167617in}{6.348218in}}%
\pgfpathlineto{\pgfqpoint{10.169642in}{2.792195in}}%
\pgfpathlineto{\pgfqpoint{10.171667in}{6.374272in}}%
\pgfpathlineto{\pgfqpoint{10.173692in}{2.791263in}}%
\pgfpathlineto{\pgfqpoint{10.175718in}{6.350144in}}%
\pgfpathlineto{\pgfqpoint{10.177743in}{2.840328in}}%
\pgfpathlineto{\pgfqpoint{10.179768in}{6.276290in}}%
\pgfpathlineto{\pgfqpoint{10.181793in}{2.938768in}}%
\pgfpathlineto{\pgfqpoint{10.183819in}{6.153602in}}%
\pgfpathlineto{\pgfqpoint{10.185844in}{3.085117in}}%
\pgfpathlineto{\pgfqpoint{10.187869in}{5.984573in}}%
\pgfpathlineto{\pgfqpoint{10.189895in}{3.275305in}}%
\pgfpathlineto{\pgfqpoint{10.191920in}{5.775414in}}%
\pgfpathlineto{\pgfqpoint{10.193945in}{3.500490in}}%
\pgfpathlineto{\pgfqpoint{10.195970in}{5.537938in}}%
\pgfpathlineto{\pgfqpoint{10.197996in}{3.745769in}}%
\pgfpathlineto{\pgfqpoint{10.200021in}{5.289983in}}%
\pgfpathlineto{\pgfqpoint{10.202046in}{3.990819in}}%
\pgfpathlineto{\pgfqpoint{10.204071in}{5.053622in}}%
\pgfpathlineto{\pgfqpoint{10.206097in}{4.212808in}}%
\pgfpathlineto{\pgfqpoint{10.208122in}{4.851262in}}%
\pgfpathlineto{\pgfqpoint{10.210147in}{4.391034in}}%
\pgfpathlineto{\pgfqpoint{10.212173in}{4.700617in}}%
\pgfpathlineto{\pgfqpoint{10.214198in}{4.511954in}}%
\pgfpathlineto{\pgfqpoint{10.216223in}{4.610098in}}%
\pgfpathlineto{\pgfqpoint{10.218248in}{4.572944in}}%
\pgfpathlineto{\pgfqpoint{10.220274in}{4.576253in}}%
\pgfpathlineto{\pgfqpoint{10.222299in}{4.583398in}}%
\pgfpathlineto{\pgfqpoint{10.224324in}{4.584305in}}%
\pgfpathlineto{\pgfqpoint{10.226349in}{4.562523in}}%
\pgfpathlineto{\pgfqpoint{10.228375in}{4.611913in}}%
\pgfpathlineto{\pgfqpoint{10.230400in}{4.534255in}}%
\pgfpathlineto{\pgfqpoint{10.232425in}{4.635214in}}%
\pgfpathlineto{\pgfqpoint{10.234451in}{4.520704in}}%
\pgfpathlineto{\pgfqpoint{10.236476in}{4.635412in}}%
\pgfpathlineto{\pgfqpoint{10.238501in}{4.536039in}}%
\pgfpathlineto{\pgfqpoint{10.240526in}{4.603925in}}%
\pgfpathlineto{\pgfqpoint{10.242552in}{4.582720in}}%
\pgfpathlineto{\pgfqpoint{10.244577in}{4.544478in}}%
\pgfpathlineto{\pgfqpoint{10.246602in}{4.651254in}}%
\pgfpathlineto{\pgfqpoint{10.248627in}{4.471469in}}%
\pgfpathlineto{\pgfqpoint{10.250653in}{4.723587in}}%
\pgfpathlineto{\pgfqpoint{10.252678in}{4.405070in}}%
\pgfpathlineto{\pgfqpoint{10.254703in}{4.779123in}}%
\pgfpathlineto{\pgfqpoint{10.256729in}{4.364580in}}%
\pgfpathlineto{\pgfqpoint{10.258754in}{4.801478in}}%
\pgfpathlineto{\pgfqpoint{10.260779in}{4.362092in}}%
\pgfpathlineto{\pgfqpoint{10.262804in}{4.783877in}}%
\pgfpathlineto{\pgfqpoint{10.264830in}{4.398467in}}%
\pgfpathlineto{\pgfqpoint{10.266855in}{4.731488in}}%
\pgfpathlineto{\pgfqpoint{10.268880in}{4.462880in}}%
\pgfpathlineto{\pgfqpoint{10.270905in}{4.659958in}}%
\pgfpathlineto{\pgfqpoint{10.272931in}{4.536088in}}%
\pgfpathlineto{\pgfqpoint{10.274956in}{4.590613in}}%
\pgfpathlineto{\pgfqpoint{10.276981in}{4.596368in}}%
\pgfpathlineto{\pgfqpoint{10.279007in}{4.543848in}}%
\pgfpathlineto{\pgfqpoint{10.281032in}{4.626269in}}%
\pgfpathlineto{\pgfqpoint{10.283057in}{4.532795in}}%
\pgfpathlineto{\pgfqpoint{10.285082in}{4.618007in}}%
\pgfpathlineto{\pgfqpoint{10.287108in}{4.559290in}}%
\pgfpathlineto{\pgfqpoint{10.289133in}{4.575811in}}%
\pgfpathlineto{\pgfqpoint{10.291158in}{4.613417in}}%
\pgfpathlineto{\pgfqpoint{10.293183in}{4.514447in}}%
\pgfpathlineto{\pgfqpoint{10.295209in}{4.676790in}}%
\pgfpathlineto{\pgfqpoint{10.297234in}{4.454397in}}%
\pgfpathlineto{\pgfqpoint{10.299259in}{4.728528in}}%
\pgfpathlineto{\pgfqpoint{10.301285in}{4.415208in}}%
\pgfpathlineto{\pgfqpoint{10.303310in}{4.752041in}}%
\pgfpathlineto{\pgfqpoint{10.305335in}{4.409121in}}%
\pgfpathlineto{\pgfqpoint{10.307360in}{4.740478in}}%
\pgfpathlineto{\pgfqpoint{10.309386in}{4.436996in}}%
\pgfpathlineto{\pgfqpoint{10.311411in}{4.699095in}}%
\pgfpathlineto{\pgfqpoint{10.313436in}{4.487832in}}%
\pgfpathlineto{\pgfqpoint{10.315461in}{4.643800in}}%
\pgfpathlineto{\pgfqpoint{10.317487in}{4.542044in}}%
\pgfpathlineto{\pgfqpoint{10.319512in}{4.596322in}}%
\pgfpathlineto{\pgfqpoint{10.321537in}{4.577471in}}%
\pgfpathlineto{\pgfqpoint{10.323563in}{4.577510in}}%
\pgfpathlineto{\pgfqpoint{10.325588in}{4.576222in}}%
\pgfpathlineto{\pgfqpoint{10.327613in}{4.600881in}}%
\pgfpathlineto{\pgfqpoint{10.329638in}{4.530216in}}%
\pgfpathlineto{\pgfqpoint{10.331664in}{4.668445in}}%
\pgfpathlineto{\pgfqpoint{10.333689in}{4.443670in}}%
\pgfpathlineto{\pgfqpoint{10.335714in}{4.770103in}}%
\pgfpathlineto{\pgfqpoint{10.337739in}{4.331757in}}%
\pgfpathlineto{\pgfqpoint{10.339765in}{4.886816in}}%
\pgfpathlineto{\pgfqpoint{10.341790in}{4.215865in}}%
\pgfpathlineto{\pgfqpoint{10.343815in}{4.996543in}}%
\pgfpathlineto{\pgfqpoint{10.345841in}{4.116944in}}%
\pgfpathlineto{\pgfqpoint{10.347866in}{5.081084in}}%
\pgfpathlineto{\pgfqpoint{10.349891in}{4.049009in}}%
\pgfpathlineto{\pgfqpoint{10.351916in}{5.131711in}}%
\pgfpathlineto{\pgfqpoint{10.353942in}{4.014817in}}%
\pgfpathlineto{\pgfqpoint{10.355967in}{5.151848in}}%
\pgfpathlineto{\pgfqpoint{10.357992in}{4.005031in}}%
\pgfpathlineto{\pgfqpoint{10.360017in}{5.156022in}}%
\pgfpathlineto{\pgfqpoint{10.362043in}{4.001061in}}%
\pgfpathlineto{\pgfqpoint{10.364068in}{5.165454in}}%
\pgfpathlineto{\pgfqpoint{10.366093in}{3.980671in}}%
\pgfpathlineto{\pgfqpoint{10.368119in}{5.201719in}}%
\pgfpathlineto{\pgfqpoint{10.370144in}{3.924557in}}%
\pgfpathlineto{\pgfqpoint{10.372169in}{5.280431in}}%
\pgfpathlineto{\pgfqpoint{10.374194in}{3.821911in}}%
\pgfpathlineto{\pgfqpoint{10.376220in}{5.406851in}}%
\pgfpathlineto{\pgfqpoint{10.378245in}{3.673338in}}%
\pgfpathlineto{\pgfqpoint{10.380270in}{5.574625in}}%
\pgfpathlineto{\pgfqpoint{10.382295in}{3.490418in}}%
\pgfpathlineto{\pgfqpoint{10.384321in}{5.767826in}}%
\pgfpathlineto{\pgfqpoint{10.386346in}{3.292279in}}%
\pgfpathlineto{\pgfqpoint{10.388371in}{5.965433in}}%
\pgfpathlineto{\pgfqpoint{10.390397in}{3.100472in}}%
\pgfpathlineto{\pgfqpoint{10.392422in}{6.146665in}}%
\pgfpathlineto{\pgfqpoint{10.394447in}{2.933868in}}%
\pgfpathlineto{\pgfqpoint{10.396472in}{6.295461in}}%
\pgfpathlineto{\pgfqpoint{10.398498in}{2.805121in}}%
\pgfpathlineto{\pgfqpoint{10.400523in}{6.402844in}}%
\pgfpathlineto{\pgfqpoint{10.402548in}{2.719579in}}%
\pgfpathlineto{\pgfqpoint{10.404573in}{6.466761in}}%
\pgfpathlineto{\pgfqpoint{10.406599in}{2.676560in}}%
\pgfpathlineto{\pgfqpoint{10.408624in}{6.489924in}}%
\pgfpathlineto{\pgfqpoint{10.410649in}{2.672083in}}%
\pgfpathlineto{\pgfqpoint{10.412675in}{6.476853in}}%
\pgfpathlineto{\pgfqpoint{10.414700in}{2.701705in}}%
\pgfpathlineto{\pgfqpoint{10.416725in}{6.431483in}}%
\pgfpathlineto{\pgfqpoint{10.418750in}{2.762209in}}%
\pgfpathlineto{\pgfqpoint{10.420776in}{6.356333in}}%
\pgfpathlineto{\pgfqpoint{10.422801in}{2.851530in}}%
\pgfpathlineto{\pgfqpoint{10.424826in}{6.253449in}}%
\pgfpathlineto{\pgfqpoint{10.426851in}{2.967088in}}%
\pgfpathlineto{\pgfqpoint{10.428877in}{6.126533in}}%
\pgfpathlineto{\pgfqpoint{10.430902in}{3.103507in}}%
\pgfpathlineto{\pgfqpoint{10.432927in}{5.983063in}}%
\pgfpathlineto{\pgfqpoint{10.434953in}{3.250981in}}%
\pgfpathlineto{\pgfqpoint{10.436978in}{5.835150in}}%
\pgfpathlineto{\pgfqpoint{10.439003in}{3.395395in}}%
\pgfpathlineto{\pgfqpoint{10.441028in}{5.698333in}}%
\pgfpathlineto{\pgfqpoint{10.443054in}{3.520615in}}%
\pgfpathlineto{\pgfqpoint{10.445079in}{5.588327in}}%
\pgfpathlineto{\pgfqpoint{10.447104in}{3.612467in}}%
\pgfpathlineto{\pgfqpoint{10.449129in}{5.516629in}}%
\pgfpathlineto{\pgfqpoint{10.451155in}{3.663164in}}%
\pgfpathlineto{\pgfqpoint{10.453180in}{5.486479in}}%
\pgfpathlineto{\pgfqpoint{10.455205in}{3.674569in}}%
\pgfpathlineto{\pgfqpoint{10.457231in}{5.490717in}}%
\pgfpathlineto{\pgfqpoint{10.459256in}{3.658941in}}%
\pgfpathlineto{\pgfqpoint{10.461281in}{5.512584in}}%
\pgfpathlineto{\pgfqpoint{10.463306in}{3.636557in}}%
\pgfpathlineto{\pgfqpoint{10.465332in}{5.529578in}}%
\pgfpathlineto{\pgfqpoint{10.467357in}{3.630627in}}%
\pgfpathlineto{\pgfqpoint{10.469382in}{5.519416in}}%
\pgfpathlineto{\pgfqpoint{10.471407in}{3.660885in}}%
\pgfpathlineto{\pgfqpoint{10.473433in}{5.466394in}}%
\pgfpathlineto{\pgfqpoint{10.475458in}{3.737777in}}%
\pgfpathlineto{\pgfqpoint{10.477483in}{5.366195in}}%
\pgfpathlineto{\pgfqpoint{10.479509in}{3.859065in}}%
\pgfpathlineto{\pgfqpoint{10.481534in}{5.227563in}}%
\pgfpathlineto{\pgfqpoint{10.483559in}{4.010020in}}%
\pgfpathlineto{\pgfqpoint{10.485584in}{5.070234in}}%
\pgfpathlineto{\pgfqpoint{10.487610in}{4.167260in}}%
\pgfpathlineto{\pgfqpoint{10.489635in}{4.919605in}}%
\pgfpathlineto{\pgfqpoint{10.491660in}{4.305160in}}%
\pgfpathlineto{\pgfqpoint{10.493685in}{4.799717in}}%
\pgfpathlineto{\pgfqpoint{10.495711in}{4.402957in}}%
\pgfpathlineto{\pgfqpoint{10.497736in}{4.726611in}}%
\pgfpathlineto{\pgfqpoint{10.499761in}{4.450411in}}%
\pgfpathlineto{\pgfqpoint{10.501787in}{4.704090in}}%
\pgfpathlineto{\pgfqpoint{10.503812in}{4.450308in}}%
\pgfpathlineto{\pgfqpoint{10.505837in}{4.723129in}}%
\pgfpathlineto{\pgfqpoint{10.507862in}{4.417087in}}%
\pgfpathlineto{\pgfqpoint{10.509888in}{4.765100in}}%
\pgfpathlineto{\pgfqpoint{10.511913in}{4.372037in}}%
\pgfpathlineto{\pgfqpoint{10.513938in}{4.807766in}}%
\pgfpathlineto{\pgfqpoint{10.515963in}{4.336587in}}%
\pgfpathlineto{\pgfqpoint{10.517989in}{4.832162in}}%
\pgfpathlineto{\pgfqpoint{10.520014in}{4.325812in}}%
\pgfpathlineto{\pgfqpoint{10.522039in}{4.828187in}}%
\pgfpathlineto{\pgfqpoint{10.524065in}{4.344185in}}%
\pgfpathlineto{\pgfqpoint{10.526090in}{4.797171in}}%
\pgfpathlineto{\pgfqpoint{10.528115in}{4.384891in}}%
\pgfpathlineto{\pgfqpoint{10.530140in}{4.750628in}}%
\pgfpathlineto{\pgfqpoint{10.532166in}{4.432898in}}%
\pgfpathlineto{\pgfqpoint{10.534191in}{4.705631in}}%
\pgfpathlineto{\pgfqpoint{10.536216in}{4.470738in}}%
\pgfpathlineto{\pgfqpoint{10.538241in}{4.678360in}}%
\pgfpathlineto{\pgfqpoint{10.540267in}{4.485098in}}%
\pgfpathlineto{\pgfqpoint{10.542292in}{4.677937in}}%
\pgfpathlineto{\pgfqpoint{10.544317in}{4.472010in}}%
\pgfpathlineto{\pgfqpoint{10.546343in}{4.702653in}}%
\pgfpathlineto{\pgfqpoint{10.548368in}{4.438884in}}%
\pgfpathlineto{\pgfqpoint{10.550393in}{4.739881in}}%
\pgfpathlineto{\pgfqpoint{10.552418in}{4.402601in}}%
\pgfpathlineto{\pgfqpoint{10.554444in}{4.769853in}}%
\pgfpathlineto{\pgfqpoint{10.556469in}{4.384165in}}%
\pgfpathlineto{\pgfqpoint{10.558494in}{4.772145in}}%
\pgfpathlineto{\pgfqpoint{10.560519in}{4.401577in}}%
\pgfpathlineto{\pgfqpoint{10.562545in}{4.732880in}}%
\pgfpathlineto{\pgfqpoint{10.564570in}{4.463168in}}%
\pgfpathlineto{\pgfqpoint{10.566595in}{4.650300in}}%
\pgfpathlineto{\pgfqpoint{10.568621in}{4.563586in}}%
\pgfpathlineto{\pgfqpoint{10.570646in}{4.536864in}}%
\pgfpathlineto{\pgfqpoint{10.572671in}{4.683822in}}%
\pgfpathlineto{\pgfqpoint{10.574696in}{4.417063in}}%
\pgfpathlineto{\pgfqpoint{10.576722in}{4.795410in}}%
\pgfpathlineto{\pgfqpoint{10.578747in}{4.321492in}}%
\pgfpathlineto{\pgfqpoint{10.580772in}{4.867657in}}%
\pgfpathlineto{\pgfqpoint{10.582797in}{4.278880in}}%
\pgfpathlineto{\pgfqpoint{10.584823in}{4.875749in}}%
\pgfpathlineto{\pgfqpoint{10.586848in}{4.308440in}}%
\pgfpathlineto{\pgfqpoint{10.588873in}{4.807364in}}%
\pgfpathlineto{\pgfqpoint{10.590899in}{4.414803in}}%
\pgfpathlineto{\pgfqpoint{10.592924in}{4.665812in}}%
\pgfpathlineto{\pgfqpoint{10.596974in}{4.468847in}}%
\pgfpathlineto{\pgfqpoint{10.599000in}{4.801829in}}%
\pgfpathlineto{\pgfqpoint{10.601025in}{4.243612in}}%
\pgfpathlineto{\pgfqpoint{10.603050in}{5.030062in}}%
\pgfpathlineto{\pgfqpoint{10.605075in}{4.019357in}}%
\pgfpathlineto{\pgfqpoint{10.607101in}{5.244228in}}%
\pgfpathlineto{\pgfqpoint{10.609126in}{3.820201in}}%
\pgfpathlineto{\pgfqpoint{10.611151in}{5.424856in}}%
\pgfpathlineto{\pgfqpoint{10.613177in}{3.660128in}}%
\pgfpathlineto{\pgfqpoint{10.615202in}{5.563801in}}%
\pgfpathlineto{\pgfqpoint{10.617227in}{3.541592in}}%
\pgfpathlineto{\pgfqpoint{10.619252in}{5.663679in}}%
\pgfpathlineto{\pgfqpoint{10.621278in}{3.457925in}}%
\pgfpathlineto{\pgfqpoint{10.623303in}{5.733897in}}%
\pgfpathlineto{\pgfqpoint{10.625328in}{3.398470in}}%
\pgfpathlineto{\pgfqpoint{10.627353in}{5.784835in}}%
\pgfpathlineto{\pgfqpoint{10.629379in}{3.354549in}}%
\pgfpathlineto{\pgfqpoint{10.631404in}{5.822273in}}%
\pgfpathlineto{\pgfqpoint{10.633429in}{3.324144in}}%
\pgfpathlineto{\pgfqpoint{10.635455in}{5.844007in}}%
\pgfpathlineto{\pgfqpoint{10.637480in}{3.313698in}}%
\pgfpathlineto{\pgfqpoint{10.639505in}{5.839780in}}%
\pgfpathlineto{\pgfqpoint{10.641530in}{3.336450in}}%
\pgfpathlineto{\pgfqpoint{10.643556in}{5.794532in}}%
\pgfpathlineto{\pgfqpoint{10.645581in}{3.407903in}}%
\pgfpathlineto{\pgfqpoint{10.647606in}{5.693805in}}%
\pgfpathlineto{\pgfqpoint{10.649631in}{3.539997in}}%
\pgfpathlineto{\pgfqpoint{10.651657in}{5.529497in}}%
\pgfpathlineto{\pgfqpoint{10.653682in}{3.735938in}}%
\pgfpathlineto{\pgfqpoint{10.655707in}{5.304008in}}%
\pgfpathlineto{\pgfqpoint{10.657733in}{3.987419in}}%
\pgfpathlineto{\pgfqpoint{10.659758in}{5.031425in}}%
\pgfpathlineto{\pgfqpoint{10.661783in}{4.275123in}}%
\pgfpathlineto{\pgfqpoint{10.663808in}{4.735361in}}%
\pgfpathlineto{\pgfqpoint{10.667859in}{4.444137in}}%
\pgfpathlineto{\pgfqpoint{10.669884in}{4.850721in}}%
\pgfpathlineto{\pgfqpoint{10.671909in}{4.184804in}}%
\pgfpathlineto{\pgfqpoint{10.673935in}{5.085873in}}%
\pgfpathlineto{\pgfqpoint{10.675960in}{3.977802in}}%
\pgfpathlineto{\pgfqpoint{10.677985in}{5.262071in}}%
\pgfpathlineto{\pgfqpoint{10.680011in}{3.833729in}}%
\pgfpathlineto{\pgfqpoint{10.682036in}{5.373965in}}%
\pgfpathlineto{\pgfqpoint{10.684061in}{3.752943in}}%
\pgfpathlineto{\pgfqpoint{10.686086in}{5.425634in}}%
\pgfpathlineto{\pgfqpoint{10.688112in}{3.727726in}}%
\pgfpathlineto{\pgfqpoint{10.690137in}{5.427481in}}%
\pgfpathlineto{\pgfqpoint{10.692162in}{3.745995in}}%
\pgfpathlineto{\pgfqpoint{10.694187in}{5.392301in}}%
\pgfpathlineto{\pgfqpoint{10.696213in}{3.795096in}}%
\pgfpathlineto{\pgfqpoint{10.698238in}{5.331944in}}%
\pgfpathlineto{\pgfqpoint{10.700263in}{3.864413in}}%
\pgfpathlineto{\pgfqpoint{10.702289in}{5.255613in}}%
\pgfpathlineto{\pgfqpoint{10.704314in}{3.946096in}}%
\pgfpathlineto{\pgfqpoint{10.706339in}{5.170069in}}%
\pgfpathlineto{\pgfqpoint{10.708364in}{4.034050in}}%
\pgfpathlineto{\pgfqpoint{10.710390in}{5.081244in}}%
\pgfpathlineto{\pgfqpoint{10.712415in}{4.122006in}}%
\pgfpathlineto{\pgfqpoint{10.714440in}{4.996173in}}%
\pgfpathlineto{\pgfqpoint{10.716465in}{4.201875in}}%
\pgfpathlineto{\pgfqpoint{10.718491in}{4.924094in}}%
\pgfpathlineto{\pgfqpoint{10.720516in}{4.263397in}}%
\pgfpathlineto{\pgfqpoint{10.722541in}{4.875929in}}%
\pgfpathlineto{\pgfqpoint{10.724567in}{4.295555in}}%
\pgfpathlineto{\pgfqpoint{10.726592in}{4.862069in}}%
\pgfpathlineto{\pgfqpoint{10.728617in}{4.289400in}}%
\pgfpathlineto{\pgfqpoint{10.730642in}{4.889189in}}%
\pgfpathlineto{\pgfqpoint{10.732668in}{4.241287in}}%
\pgfpathlineto{\pgfqpoint{10.734693in}{4.957304in}}%
\pgfpathlineto{\pgfqpoint{10.736718in}{4.155198in}}%
\pgfpathlineto{\pgfqpoint{10.738743in}{5.058358in}}%
\pgfpathlineto{\pgfqpoint{10.740769in}{4.043029in}}%
\pgfpathlineto{\pgfqpoint{10.742794in}{5.177171in}}%
\pgfpathlineto{\pgfqpoint{10.744819in}{3.922391in}}%
\pgfpathlineto{\pgfqpoint{10.746845in}{5.294789in}}%
\pgfpathlineto{\pgfqpoint{10.748870in}{3.812332in}}%
\pgfpathlineto{\pgfqpoint{10.750895in}{5.393379in}}%
\pgfpathlineto{\pgfqpoint{10.752920in}{3.728202in}}%
\pgfpathlineto{\pgfqpoint{10.754946in}{5.461195in}}%
\pgfpathlineto{\pgfqpoint{10.756971in}{3.677277in}}%
\pgfpathlineto{\pgfqpoint{10.758996in}{5.495973in}}%
\pgfpathlineto{\pgfqpoint{10.761021in}{3.656643in}}%
\pgfpathlineto{\pgfqpoint{10.763047in}{5.505562in}}%
\pgfpathlineto{\pgfqpoint{10.765072in}{3.654157in}}%
\pgfpathlineto{\pgfqpoint{10.767097in}{5.505404in}}%
\pgfpathlineto{\pgfqpoint{10.769123in}{3.652351in}}%
\pgfpathlineto{\pgfqpoint{10.771148in}{5.513539in}}%
\pgfpathlineto{\pgfqpoint{10.773173in}{3.634140in}}%
\pgfpathlineto{\pgfqpoint{10.775198in}{5.544626in}}%
\pgfpathlineto{\pgfqpoint{10.777224in}{3.588587in}}%
\pgfpathlineto{\pgfqpoint{10.779249in}{5.604861in}}%
\pgfpathlineto{\pgfqpoint{10.781274in}{3.514889in}}%
\pgfpathlineto{\pgfqpoint{10.783299in}{5.689425in}}%
\pgfpathlineto{\pgfqpoint{10.785325in}{3.423268in}}%
\pgfpathlineto{\pgfqpoint{10.787350in}{5.783346in}}%
\pgfpathlineto{\pgfqpoint{10.789375in}{3.332412in}}%
\pgfpathlineto{\pgfqpoint{10.791401in}{5.865560in}}%
\pgfpathlineto{\pgfqpoint{10.793426in}{3.264214in}}%
\pgfpathlineto{\pgfqpoint{10.795451in}{5.914969in}}%
\pgfpathlineto{\pgfqpoint{10.797476in}{3.237402in}}%
\pgfpathlineto{\pgfqpoint{10.799502in}{5.916627in}}%
\pgfpathlineto{\pgfqpoint{10.801527in}{3.262015in}}%
\pgfpathlineto{\pgfqpoint{10.803552in}{5.866153in}}%
\pgfpathlineto{\pgfqpoint{10.805577in}{3.336447in}}%
\pgfpathlineto{\pgfqpoint{10.807603in}{5.771011in}}%
\pgfpathlineto{\pgfqpoint{10.809628in}{3.447951in}}%
\pgfpathlineto{\pgfqpoint{10.811653in}{5.648260in}}%
\pgfpathlineto{\pgfqpoint{10.813679in}{3.576449in}}%
\pgfpathlineto{\pgfqpoint{10.815704in}{5.519487in}}%
\pgfpathlineto{\pgfqpoint{10.817729in}{3.700462in}}%
\pgfpathlineto{\pgfqpoint{10.819754in}{5.404468in}}%
\pgfpathlineto{\pgfqpoint{10.821780in}{3.803358in}}%
\pgfpathlineto{\pgfqpoint{10.823805in}{5.315505in}}%
\pgfpathlineto{\pgfqpoint{10.825830in}{3.878005in}}%
\pgfpathlineto{\pgfqpoint{10.827855in}{5.254133in}}%
\pgfpathlineto{\pgfqpoint{10.829881in}{3.928449in}}%
\pgfpathlineto{\pgfqpoint{10.831906in}{5.211180in}}%
\pgfpathlineto{\pgfqpoint{10.833931in}{3.968135in}}%
\pgfpathlineto{\pgfqpoint{10.835957in}{5.170110in}}%
\pgfpathlineto{\pgfqpoint{10.837982in}{4.015272in}}%
\pgfpathlineto{\pgfqpoint{10.840007in}{5.112584in}}%
\pgfpathlineto{\pgfqpoint{10.842032in}{4.086790in}}%
\pgfpathlineto{\pgfqpoint{10.844058in}{5.024497in}}%
\pgfpathlineto{\pgfqpoint{10.846083in}{4.192783in}}%
\pgfpathlineto{\pgfqpoint{10.848108in}{4.900617in}}%
\pgfpathlineto{\pgfqpoint{10.850133in}{4.333166in}}%
\pgfpathlineto{\pgfqpoint{10.852159in}{4.746373in}}%
\pgfpathlineto{\pgfqpoint{10.854184in}{4.497579in}}%
\pgfpathlineto{\pgfqpoint{10.858235in}{4.668585in}}%
\pgfpathlineto{\pgfqpoint{10.860260in}{4.409221in}}%
\pgfpathlineto{\pgfqpoint{10.862285in}{4.827137in}}%
\pgfpathlineto{\pgfqpoint{10.864310in}{4.262869in}}%
\pgfpathlineto{\pgfqpoint{10.866336in}{4.958608in}}%
\pgfpathlineto{\pgfqpoint{10.868361in}{4.147690in}}%
\pgfpathlineto{\pgfqpoint{10.870386in}{5.057474in}}%
\pgfpathlineto{\pgfqpoint{10.872411in}{4.063757in}}%
\pgfpathlineto{\pgfqpoint{10.874437in}{5.129148in}}%
\pgfpathlineto{\pgfqpoint{10.876462in}{4.000582in}}%
\pgfpathlineto{\pgfqpoint{10.878487in}{5.188370in}}%
\pgfpathlineto{\pgfqpoint{10.880513in}{3.940344in}}%
\pgfpathlineto{\pgfqpoint{10.882538in}{5.254618in}}%
\pgfpathlineto{\pgfqpoint{10.884563in}{3.863475in}}%
\pgfpathlineto{\pgfqpoint{10.886588in}{5.345953in}}%
\pgfpathlineto{\pgfqpoint{10.888614in}{3.754915in}}%
\pgfpathlineto{\pgfqpoint{10.890639in}{5.473165in}}%
\pgfpathlineto{\pgfqpoint{10.892664in}{3.609103in}}%
\pgfpathlineto{\pgfqpoint{10.894689in}{5.636013in}}%
\pgfpathlineto{\pgfqpoint{10.896715in}{3.432225in}}%
\pgfpathlineto{\pgfqpoint{10.898740in}{5.822653in}}%
\pgfpathlineto{\pgfqpoint{10.900765in}{3.241079in}}%
\pgfpathlineto{\pgfqpoint{10.902791in}{6.012395in}}%
\pgfpathlineto{\pgfqpoint{10.904816in}{3.058925in}}%
\pgfpathlineto{\pgfqpoint{10.906841in}{6.180898in}}%
\pgfpathlineto{\pgfqpoint{10.908866in}{2.909638in}}%
\pgfpathlineto{\pgfqpoint{10.910892in}{6.306238in}}%
\pgfpathlineto{\pgfqpoint{10.912917in}{2.811880in}}%
\pgfpathlineto{\pgfqpoint{10.914942in}{6.374056in}}%
\pgfpathlineto{\pgfqpoint{10.916967in}{2.774996in}}%
\pgfpathlineto{\pgfqpoint{10.918993in}{6.380357in}}%
\pgfpathlineto{\pgfqpoint{10.921018in}{2.797695in}}%
\pgfpathlineto{\pgfqpoint{10.923043in}{6.331276in}}%
\pgfpathlineto{\pgfqpoint{10.925069in}{2.869768in}}%
\pgfpathlineto{\pgfqpoint{10.927094in}{6.240063in}}%
\pgfpathlineto{\pgfqpoint{10.929119in}{2.976133in}}%
\pgfpathlineto{\pgfqpoint{10.931144in}{6.122350in}}%
\pgfpathlineto{\pgfqpoint{10.933170in}{3.101866in}}%
\pgfpathlineto{\pgfqpoint{10.935195in}{5.991219in}}%
\pgfpathlineto{\pgfqpoint{10.937220in}{3.236641in}}%
\pgfpathlineto{\pgfqpoint{10.939246in}{5.853604in}}%
\pgfpathlineto{\pgfqpoint{10.941271in}{3.377233in}}%
\pgfpathlineto{\pgfqpoint{10.943296in}{5.709052in}}%
\pgfpathlineto{\pgfqpoint{10.945321in}{3.527405in}}%
\pgfpathlineto{\pgfqpoint{10.947347in}{5.551157in}}%
\pgfpathlineto{\pgfqpoint{10.949372in}{3.695294in}}%
\pgfpathlineto{\pgfqpoint{10.951397in}{5.371126in}}%
\pgfpathlineto{\pgfqpoint{10.953422in}{3.889201in}}%
\pgfpathlineto{\pgfqpoint{10.955448in}{5.162281in}}%
\pgfpathlineto{\pgfqpoint{10.957473in}{4.113163in}}%
\pgfpathlineto{\pgfqpoint{10.959498in}{4.924050in}}%
\pgfpathlineto{\pgfqpoint{10.961524in}{4.363727in}}%
\pgfpathlineto{\pgfqpoint{10.963549in}{4.664155in}}%
\pgfpathlineto{\pgfqpoint{10.965574in}{4.628991in}}%
\pgfpathlineto{\pgfqpoint{10.967599in}{4.398259in}}%
\pgfpathlineto{\pgfqpoint{10.969625in}{4.890254in}}%
\pgfpathlineto{\pgfqpoint{10.971650in}{4.147128in}}%
\pgfpathlineto{\pgfqpoint{10.973675in}{5.125834in}}%
\pgfpathlineto{\pgfqpoint{10.975700in}{3.932118in}}%
\pgfpathlineto{\pgfqpoint{10.977726in}{5.315949in}}%
\pgfpathlineto{\pgfqpoint{10.979751in}{3.770286in}}%
\pgfpathlineto{\pgfqpoint{10.981776in}{5.447234in}}%
\pgfpathlineto{\pgfqpoint{10.983802in}{3.670581in}}%
\pgfpathlineto{\pgfqpoint{10.985827in}{5.515579in}}%
\pgfpathlineto{\pgfqpoint{10.987852in}{3.632186in}}%
\pgfpathlineto{\pgfqpoint{10.989877in}{5.526483in}}%
\pgfpathlineto{\pgfqpoint{10.991903in}{3.645482in}}%
\pgfpathlineto{\pgfqpoint{10.993928in}{5.492851in}}%
\pgfpathlineto{\pgfqpoint{10.995953in}{3.695309in}}%
\pgfpathlineto{\pgfqpoint{10.997978in}{5.430942in}}%
\pgfpathlineto{\pgfqpoint{11.000004in}{3.765502in}}%
\pgfpathlineto{\pgfqpoint{11.002029in}{5.355692in}}%
\pgfpathlineto{\pgfqpoint{11.004054in}{3.843351in}}%
\pgfpathlineto{\pgfqpoint{11.006080in}{5.276805in}}%
\pgfpathlineto{\pgfqpoint{11.008105in}{3.922660in}}%
\pgfpathlineto{\pgfqpoint{11.010130in}{5.196772in}}%
\pgfpathlineto{\pgfqpoint{11.012155in}{4.004529in}}%
\pgfpathlineto{\pgfqpoint{11.014181in}{5.111323in}}%
\pgfpathlineto{\pgfqpoint{11.016206in}{4.095704in}}%
\pgfpathlineto{\pgfqpoint{11.018231in}{5.012138in}}%
\pgfpathlineto{\pgfqpoint{11.020256in}{4.205048in}}%
\pgfpathlineto{\pgfqpoint{11.022282in}{4.890893in}}%
\pgfpathlineto{\pgfqpoint{11.024307in}{4.339291in}}%
\pgfpathlineto{\pgfqpoint{11.026332in}{4.743386in}}%
\pgfpathlineto{\pgfqpoint{11.028358in}{4.499369in}}%
\pgfpathlineto{\pgfqpoint{11.032408in}{4.678479in}}%
\pgfpathlineto{\pgfqpoint{11.034433in}{4.388756in}}%
\pgfpathlineto{\pgfqpoint{11.036459in}{4.862421in}}%
\pgfpathlineto{\pgfqpoint{11.038484in}{4.209398in}}%
\pgfpathlineto{\pgfqpoint{11.040509in}{5.032091in}}%
\pgfpathlineto{\pgfqpoint{11.042534in}{4.054514in}}%
\pgfpathlineto{\pgfqpoint{11.044560in}{5.167362in}}%
\pgfpathlineto{\pgfqpoint{11.046585in}{3.943164in}}%
\pgfpathlineto{\pgfqpoint{11.048610in}{5.251215in}}%
\pgfpathlineto{\pgfqpoint{11.050636in}{3.889495in}}%
\pgfpathlineto{\pgfqpoint{11.052661in}{5.273009in}}%
\pgfpathlineto{\pgfqpoint{11.054686in}{3.900231in}}%
\pgfpathlineto{\pgfqpoint{11.056711in}{5.230108in}}%
\pgfpathlineto{\pgfqpoint{11.058737in}{3.973981in}}%
\pgfpathlineto{\pgfqpoint{11.060762in}{5.127658in}}%
\pgfpathlineto{\pgfqpoint{11.062787in}{4.102297in}}%
\pgfpathlineto{\pgfqpoint{11.064812in}{4.976830in}}%
\pgfpathlineto{\pgfqpoint{11.066838in}{4.271936in}}%
\pgfpathlineto{\pgfqpoint{11.068863in}{4.792259in}}%
\pgfpathlineto{\pgfqpoint{11.070888in}{4.467530in}}%
\pgfpathlineto{\pgfqpoint{11.074939in}{4.673924in}}%
\pgfpathlineto{\pgfqpoint{11.076964in}{4.382820in}}%
\pgfpathlineto{\pgfqpoint{11.078989in}{4.877761in}}%
\pgfpathlineto{\pgfqpoint{11.081015in}{4.184491in}}%
\pgfpathlineto{\pgfqpoint{11.083040in}{5.068213in}}%
\pgfpathlineto{\pgfqpoint{11.085065in}{4.003956in}}%
\pgfpathlineto{\pgfqpoint{11.087090in}{5.237112in}}%
\pgfpathlineto{\pgfqpoint{11.089116in}{3.848099in}}%
\pgfpathlineto{\pgfqpoint{11.091141in}{5.378827in}}%
\pgfpathlineto{\pgfqpoint{11.093166in}{3.721323in}}%
\pgfpathlineto{\pgfqpoint{11.095192in}{5.490171in}}%
\pgfpathlineto{\pgfqpoint{11.097217in}{3.625594in}}%
\pgfpathlineto{\pgfqpoint{11.099242in}{5.570415in}}%
\pgfpathlineto{\pgfqpoint{11.101267in}{3.560397in}}%
\pgfpathlineto{\pgfqpoint{11.103293in}{5.621296in}}%
\pgfpathlineto{\pgfqpoint{11.105318in}{3.522834in}}%
\pgfpathlineto{\pgfqpoint{11.107343in}{5.646765in}}%
\pgfpathlineto{\pgfqpoint{11.109368in}{3.508066in}}%
\pgfpathlineto{\pgfqpoint{11.111394in}{5.652326in}}%
\pgfpathlineto{\pgfqpoint{11.113419in}{3.510194in}}%
\pgfpathlineto{\pgfqpoint{11.115444in}{5.643969in}}%
\pgfpathlineto{\pgfqpoint{11.117470in}{3.523453in}}%
\pgfpathlineto{\pgfqpoint{11.119495in}{5.626935in}}%
\pgfpathlineto{\pgfqpoint{11.121520in}{3.543390in}}%
\pgfpathlineto{\pgfqpoint{11.123545in}{5.604683in}}%
\pgfpathlineto{\pgfqpoint{11.125571in}{3.567659in}}%
\pgfpathlineto{\pgfqpoint{11.127596in}{5.578421in}}%
\pgfpathlineto{\pgfqpoint{11.129621in}{3.596126in}}%
\pgfpathlineto{\pgfqpoint{11.131646in}{5.547365in}}%
\pgfpathlineto{\pgfqpoint{11.133672in}{3.630256in}}%
\pgfpathlineto{\pgfqpoint{11.135697in}{5.509654in}}%
\pgfpathlineto{\pgfqpoint{11.137722in}{3.671995in}}%
\pgfpathlineto{\pgfqpoint{11.139748in}{5.463571in}}%
\pgfpathlineto{\pgfqpoint{11.141773in}{3.722548in}}%
\pgfpathlineto{\pgfqpoint{11.143798in}{5.408646in}}%
\pgfpathlineto{\pgfqpoint{11.145823in}{3.781504in}}%
\pgfpathlineto{\pgfqpoint{11.147849in}{5.346238in}}%
\pgfpathlineto{\pgfqpoint{11.149874in}{3.846575in}}%
\pgfpathlineto{\pgfqpoint{11.151899in}{5.279457in}}%
\pgfpathlineto{\pgfqpoint{11.153924in}{3.914006in}}%
\pgfpathlineto{\pgfqpoint{11.155950in}{5.212484in}}%
\pgfpathlineto{\pgfqpoint{11.157975in}{3.979430in}}%
\pgfpathlineto{\pgfqpoint{11.160000in}{5.149627in}}%
\pgfpathlineto{\pgfqpoint{11.162026in}{4.038818in}}%
\pgfpathlineto{\pgfqpoint{11.164051in}{5.094465in}}%
\pgfpathlineto{\pgfqpoint{11.166076in}{4.089156in}}%
\pgfpathlineto{\pgfqpoint{11.168101in}{5.049392in}}%
\pgfpathlineto{\pgfqpoint{11.170127in}{4.128659in}}%
\pgfpathlineto{\pgfqpoint{11.172152in}{5.015649in}}%
\pgfpathlineto{\pgfqpoint{11.174177in}{4.156535in}}%
\pgfpathlineto{\pgfqpoint{11.176202in}{4.993697in}}%
\pgfpathlineto{\pgfqpoint{11.178228in}{4.172536in}}%
\pgfpathlineto{\pgfqpoint{11.180253in}{4.983660in}}%
\pgfpathlineto{\pgfqpoint{11.182278in}{4.176612in}}%
\pgfpathlineto{\pgfqpoint{11.184304in}{4.985511in}}%
\pgfpathlineto{\pgfqpoint{11.186329in}{4.168930in}}%
\pgfpathlineto{\pgfqpoint{11.188354in}{4.998817in}}%
\pgfpathlineto{\pgfqpoint{11.190379in}{4.150369in}}%
\pgfpathlineto{\pgfqpoint{11.192405in}{5.022046in}}%
\pgfpathlineto{\pgfqpoint{11.194430in}{4.123336in}}%
\pgfpathlineto{\pgfqpoint{11.196455in}{5.051699in}}%
\pgfpathlineto{\pgfqpoint{11.198480in}{4.092600in}}%
\pgfpathlineto{\pgfqpoint{11.200506in}{5.081619in}}%
\pgfpathlineto{\pgfqpoint{11.202531in}{4.065740in}}%
\pgfpathlineto{\pgfqpoint{11.204556in}{5.102869in}}%
\pgfpathlineto{\pgfqpoint{11.206582in}{4.052884in}}%
\pgfpathlineto{\pgfqpoint{11.208607in}{5.104407in}}%
\pgfpathlineto{\pgfqpoint{11.210632in}{4.065612in}}%
\pgfpathlineto{\pgfqpoint{11.212657in}{5.074569in}}%
\pgfpathlineto{\pgfqpoint{11.214683in}{4.115162in}}%
\pgfpathlineto{\pgfqpoint{11.216708in}{5.003084in}}%
\pgfpathlineto{\pgfqpoint{11.218733in}{4.210299in}}%
\pgfpathlineto{\pgfqpoint{11.220758in}{4.883194in}}%
\pgfpathlineto{\pgfqpoint{11.222784in}{4.355344in}}%
\pgfpathlineto{\pgfqpoint{11.224809in}{4.713348in}}%
\pgfpathlineto{\pgfqpoint{11.226834in}{4.548862in}}%
\pgfpathlineto{\pgfqpoint{11.228860in}{4.498047in}}%
\pgfpathlineto{\pgfqpoint{11.230885in}{4.783345in}}%
\pgfpathlineto{\pgfqpoint{11.232910in}{4.247610in}}%
\pgfpathlineto{\pgfqpoint{11.234935in}{5.045990in}}%
\pgfpathlineto{\pgfqpoint{11.236961in}{3.976887in}}%
\pgfpathlineto{\pgfqpoint{11.238986in}{5.320421in}}%
\pgfpathlineto{\pgfqpoint{11.241011in}{3.703204in}}%
\pgfpathlineto{\pgfqpoint{11.243036in}{5.588967in}}%
\pgfpathlineto{\pgfqpoint{11.245062in}{3.443981in}}%
\pgfpathlineto{\pgfqpoint{11.247087in}{5.835014in}}%
\pgfpathlineto{\pgfqpoint{11.249112in}{3.214531in}}%
\pgfpathlineto{\pgfqpoint{11.251138in}{6.044955in}}%
\pgfpathlineto{\pgfqpoint{11.253163in}{3.026455in}}%
\pgfpathlineto{\pgfqpoint{11.255188in}{6.209381in}}%
\pgfpathlineto{\pgfqpoint{11.257213in}{2.886901in}}%
\pgfpathlineto{\pgfqpoint{11.259239in}{6.323362in}}%
\pgfpathlineto{\pgfqpoint{11.261264in}{2.798729in}}%
\pgfpathlineto{\pgfqpoint{11.263289in}{6.385879in}}%
\pgfpathlineto{\pgfqpoint{11.265314in}{2.761412in}}%
\pgfpathlineto{\pgfqpoint{11.267340in}{6.398663in}}%
\pgfpathlineto{\pgfqpoint{11.269365in}{2.772367in}}%
\pgfpathlineto{\pgfqpoint{11.271390in}{6.364805in}}%
\pgfpathlineto{\pgfqpoint{11.273416in}{2.828315in}}%
\pgfpathlineto{\pgfqpoint{11.275441in}{6.287509in}}%
\pgfpathlineto{\pgfqpoint{11.277466in}{2.926320in}}%
\pgfpathlineto{\pgfqpoint{11.279491in}{6.169322in}}%
\pgfpathlineto{\pgfqpoint{11.281517in}{3.064261in}}%
\pgfpathlineto{\pgfqpoint{11.283542in}{6.011988in}}%
\pgfpathlineto{\pgfqpoint{11.285567in}{3.240646in}}%
\pgfpathlineto{\pgfqpoint{11.287592in}{5.816935in}}%
\pgfpathlineto{\pgfqpoint{11.289618in}{3.453876in}}%
\pgfpathlineto{\pgfqpoint{11.291643in}{5.586204in}}%
\pgfpathlineto{\pgfqpoint{11.293668in}{3.701185in}}%
\pgfpathlineto{\pgfqpoint{11.295694in}{5.323547in}}%
\pgfpathlineto{\pgfqpoint{11.297719in}{3.977603in}}%
\pgfpathlineto{\pgfqpoint{11.299744in}{5.035340in}}%
\pgfpathlineto{\pgfqpoint{11.301769in}{4.275235in}}%
\pgfpathlineto{\pgfqpoint{11.303795in}{4.731024in}}%
\pgfpathlineto{\pgfqpoint{11.307845in}{4.422885in}}%
\pgfpathlineto{\pgfqpoint{11.309870in}{4.887936in}}%
\pgfpathlineto{\pgfqpoint{11.311896in}{4.125139in}}%
\pgfpathlineto{\pgfqpoint{11.313921in}{5.174944in}}%
\pgfpathlineto{\pgfqpoint{11.315946in}{3.852461in}}%
\pgfpathlineto{\pgfqpoint{11.317972in}{5.429920in}}%
\pgfpathlineto{\pgfqpoint{11.319997in}{3.618231in}}%
\pgfpathlineto{\pgfqpoint{11.322022in}{5.640783in}}%
\pgfpathlineto{\pgfqpoint{11.324047in}{3.432855in}}%
\pgfpathlineto{\pgfqpoint{11.326073in}{5.799115in}}%
\pgfpathlineto{\pgfqpoint{11.328098in}{3.302528in}}%
\pgfpathlineto{\pgfqpoint{11.330123in}{5.901076in}}%
\pgfpathlineto{\pgfqpoint{11.332148in}{3.228709in}}%
\pgfpathlineto{\pgfqpoint{11.334174in}{5.947515in}}%
\pgfpathlineto{\pgfqpoint{11.336199in}{3.208418in}}%
\pgfpathlineto{\pgfqpoint{11.338224in}{5.943276in}}%
\pgfpathlineto{\pgfqpoint{11.340250in}{3.235283in}}%
\pgfpathlineto{\pgfqpoint{11.342275in}{5.895871in}}%
\pgfpathlineto{\pgfqpoint{11.344300in}{3.301066in}}%
\pgfpathlineto{\pgfqpoint{11.346325in}{5.813844in}}%
\pgfpathlineto{\pgfqpoint{11.348351in}{3.397314in}}%
\pgfpathlineto{\pgfqpoint{11.350376in}{5.705211in}}%
\pgfpathlineto{\pgfqpoint{11.352401in}{3.516736in}}%
\pgfpathlineto{\pgfqpoint{11.354426in}{5.576325in}}%
\pgfpathlineto{\pgfqpoint{11.356452in}{3.654042in}}%
\pgfpathlineto{\pgfqpoint{11.358477in}{5.431379in}}%
\pgfpathlineto{\pgfqpoint{11.360502in}{3.806084in}}%
\pgfpathlineto{\pgfqpoint{11.362528in}{5.272602in}}%
\pgfpathlineto{\pgfqpoint{11.364553in}{3.971355in}}%
\pgfpathlineto{\pgfqpoint{11.366578in}{5.101028in}}%
\pgfpathlineto{\pgfqpoint{11.368603in}{4.149010in}}%
\pgfpathlineto{\pgfqpoint{11.370629in}{4.917623in}}%
\pgfpathlineto{\pgfqpoint{11.372654in}{4.337653in}}%
\pgfpathlineto{\pgfqpoint{11.374679in}{4.724503in}}%
\pgfpathlineto{\pgfqpoint{11.376704in}{4.534183in}}%
\pgfpathlineto{\pgfqpoint{11.378730in}{4.525978in}}%
\pgfpathlineto{\pgfqpoint{11.380755in}{4.732917in}}%
\pgfpathlineto{\pgfqpoint{11.382780in}{4.329197in}}%
\pgfpathlineto{\pgfqpoint{11.384806in}{4.925224in}}%
\pgfpathlineto{\pgfqpoint{11.386831in}{4.144205in}}%
\pgfpathlineto{\pgfqpoint{11.388856in}{5.099800in}}%
\pgfpathlineto{\pgfqpoint{11.390881in}{3.983321in}}%
\pgfpathlineto{\pgfqpoint{11.392907in}{5.243642in}}%
\pgfpathlineto{\pgfqpoint{11.394932in}{3.859824in}}%
\pgfpathlineto{\pgfqpoint{11.396957in}{5.343668in}}%
\pgfpathlineto{\pgfqpoint{11.398982in}{3.786081in}}%
\pgfpathlineto{\pgfqpoint{11.401008in}{5.388770in}}%
\pgfpathlineto{\pgfqpoint{11.403033in}{3.771400in}}%
\pgfpathlineto{\pgfqpoint{11.405058in}{5.371944in}}%
\pgfpathlineto{\pgfqpoint{11.407084in}{3.820034in}}%
\pgfpathlineto{\pgfqpoint{11.409109in}{5.292046in}}%
\pgfpathlineto{\pgfqpoint{11.411134in}{3.929781in}}%
\pgfpathlineto{\pgfqpoint{11.413159in}{5.154723in}}%
\pgfpathlineto{\pgfqpoint{11.415185in}{4.091602in}}%
\pgfpathlineto{\pgfqpoint{11.417210in}{4.972196in}}%
\pgfpathlineto{\pgfqpoint{11.419235in}{4.290453in}}%
\pgfpathlineto{\pgfqpoint{11.421260in}{4.761836in}}%
\pgfpathlineto{\pgfqpoint{11.423286in}{4.507254in}}%
\pgfpathlineto{\pgfqpoint{11.425311in}{4.543727in}}%
\pgfpathlineto{\pgfqpoint{11.427336in}{4.721664in}}%
\pgfpathlineto{\pgfqpoint{11.429362in}{4.337712in}}%
\pgfpathlineto{\pgfqpoint{11.431387in}{4.915065in}}%
\pgfpathlineto{\pgfqpoint{11.433412in}{4.160526in}}%
\pgfpathlineto{\pgfqpoint{11.435437in}{5.073158in}}%
\pgfpathlineto{\pgfqpoint{11.437463in}{4.023610in}}%
\pgfpathlineto{\pgfqpoint{11.439488in}{5.187637in}}%
\pgfpathlineto{\pgfqpoint{11.441513in}{3.932013in}}%
\pgfpathlineto{\pgfqpoint{11.443538in}{5.256673in}}%
\pgfpathlineto{\pgfqpoint{11.445564in}{3.884529in}}%
\pgfpathlineto{\pgfqpoint{11.447589in}{5.284199in}}%
\pgfpathlineto{\pgfqpoint{11.449614in}{3.874904in}}%
\pgfpathlineto{\pgfqpoint{11.451640in}{5.278309in}}%
\pgfpathlineto{\pgfqpoint{11.453665in}{3.893731in}}%
\pgfpathlineto{\pgfqpoint{11.455690in}{5.249184in}}%
\pgfpathlineto{\pgfqpoint{11.457715in}{3.930575in}}%
\pgfpathlineto{\pgfqpoint{11.459741in}{5.207038in}}%
\pgfpathlineto{\pgfqpoint{11.461766in}{3.975855in}}%
\pgfpathlineto{\pgfqpoint{11.463791in}{5.160477in}}%
\pgfpathlineto{\pgfqpoint{11.465816in}{4.022199in}}%
\pgfpathlineto{\pgfqpoint{11.467842in}{5.115474in}}%
\pgfpathlineto{\pgfqpoint{11.469867in}{4.065113in}}%
\pgfpathlineto{\pgfqpoint{11.471892in}{5.075038in}}%
\pgfpathlineto{\pgfqpoint{11.473918in}{4.103011in}}%
\pgfpathlineto{\pgfqpoint{11.475943in}{5.039451in}}%
\pgfpathlineto{\pgfqpoint{11.477968in}{4.136750in}}%
\pgfpathlineto{\pgfqpoint{11.479993in}{5.006914in}}%
\pgfpathlineto{\pgfqpoint{11.482019in}{4.168860in}}%
\pgfpathlineto{\pgfqpoint{11.484044in}{4.974384in}}%
\pgfpathlineto{\pgfqpoint{11.486069in}{4.202673in}}%
\pgfpathlineto{\pgfqpoint{11.488094in}{4.938457in}}%
\pgfpathlineto{\pgfqpoint{11.490120in}{4.241464in}}%
\pgfpathlineto{\pgfqpoint{11.492145in}{4.896178in}}%
\pgfpathlineto{\pgfqpoint{11.494170in}{4.287693in}}%
\pgfpathlineto{\pgfqpoint{11.496196in}{4.845736in}}%
\pgfpathlineto{\pgfqpoint{11.498221in}{4.342377in}}%
\pgfpathlineto{\pgfqpoint{11.500246in}{4.787039in}}%
\pgfpathlineto{\pgfqpoint{11.502271in}{4.404580in}}%
\pgfpathlineto{\pgfqpoint{11.504297in}{4.722131in}}%
\pgfpathlineto{\pgfqpoint{11.506322in}{4.471090in}}%
\pgfpathlineto{\pgfqpoint{11.508347in}{4.655425in}}%
\pgfpathlineto{\pgfqpoint{11.510372in}{4.536304in}}%
\pgfpathlineto{\pgfqpoint{11.512398in}{4.593649in}}%
\pgfpathlineto{\pgfqpoint{11.514423in}{4.592478in}}%
\pgfpathlineto{\pgfqpoint{11.516448in}{4.545395in}}%
\pgfpathlineto{\pgfqpoint{11.518474in}{4.630409in}}%
\pgfpathlineto{\pgfqpoint{11.520499in}{4.520187in}}%
\pgfpathlineto{\pgfqpoint{11.522524in}{4.640601in}}%
\pgfpathlineto{\pgfqpoint{11.524549in}{4.527093in}}%
\pgfpathlineto{\pgfqpoint{11.526575in}{4.614833in}}%
\pgfpathlineto{\pgfqpoint{11.530625in}{4.547851in}}%
\pgfpathlineto{\pgfqpoint{11.532650in}{4.661286in}}%
\pgfpathlineto{\pgfqpoint{11.534676in}{4.438793in}}%
\pgfpathlineto{\pgfqpoint{11.536701in}{4.790102in}}%
\pgfpathlineto{\pgfqpoint{11.538726in}{4.291931in}}%
\pgfpathlineto{\pgfqpoint{11.540752in}{4.952710in}}%
\pgfpathlineto{\pgfqpoint{11.542777in}{4.116378in}}%
\pgfpathlineto{\pgfqpoint{11.544802in}{5.138024in}}%
\pgfpathlineto{\pgfqpoint{11.546827in}{3.924726in}}%
\pgfpathlineto{\pgfqpoint{11.548853in}{5.332508in}}%
\pgfpathlineto{\pgfqpoint{11.550878in}{3.730841in}}%
\pgfpathlineto{\pgfqpoint{11.552903in}{5.522594in}}%
\pgfpathlineto{\pgfqpoint{11.554928in}{3.547388in}}%
\pgfpathlineto{\pgfqpoint{11.556954in}{5.697053in}}%
\pgfpathlineto{\pgfqpoint{11.558979in}{3.383734in}}%
\pgfpathlineto{\pgfqpoint{11.561004in}{5.848674in}}%
\pgfpathlineto{\pgfqpoint{11.563030in}{3.244797in}}%
\pgfpathlineto{\pgfqpoint{11.565055in}{5.974813in}}%
\pgfpathlineto{\pgfqpoint{11.567080in}{3.131109in}}%
\pgfpathlineto{\pgfqpoint{11.569105in}{6.076765in}}%
\pgfpathlineto{\pgfqpoint{11.571131in}{3.039924in}}%
\pgfpathlineto{\pgfqpoint{11.573156in}{6.158298in}}%
\pgfpathlineto{\pgfqpoint{11.575181in}{2.966883in}}%
\pgfpathlineto{\pgfqpoint{11.577206in}{6.223967in}}%
\pgfpathlineto{\pgfqpoint{11.579232in}{2.907561in}}%
\pgfpathlineto{\pgfqpoint{11.581257in}{6.277841in}}%
\pgfpathlineto{\pgfqpoint{11.583282in}{2.858358in}}%
\pgfpathlineto{\pgfqpoint{11.585308in}{6.323050in}}%
\pgfpathlineto{\pgfqpoint{11.587333in}{2.816517in}}%
\pgfpathlineto{\pgfqpoint{11.589358in}{6.362153in}}%
\pgfpathlineto{\pgfqpoint{11.591383in}{2.779466in}}%
\pgfpathlineto{\pgfqpoint{11.593409in}{6.397942in}}%
\pgfpathlineto{\pgfqpoint{11.595434in}{2.744020in}}%
\pgfpathlineto{\pgfqpoint{11.597459in}{6.434087in}}%
\pgfpathlineto{\pgfqpoint{11.599484in}{2.706041in}}%
\pgfpathlineto{\pgfqpoint{11.601510in}{6.475062in}}%
\pgfpathlineto{\pgfqpoint{11.603535in}{2.660979in}}%
\pgfpathlineto{\pgfqpoint{11.605560in}{6.525119in}}%
\pgfpathlineto{\pgfqpoint{11.607586in}{2.605328in}}%
\pgfpathlineto{\pgfqpoint{11.609611in}{6.586524in}}%
\pgfpathlineto{\pgfqpoint{11.611636in}{2.538556in}}%
\pgfpathlineto{\pgfqpoint{11.613661in}{6.657643in}}%
\pgfpathlineto{\pgfqpoint{11.615687in}{2.464791in}}%
\pgfpathlineto{\pgfqpoint{11.617712in}{6.731678in}}%
\pgfpathlineto{\pgfqpoint{11.619737in}{2.393488in}}%
\pgfpathlineto{\pgfqpoint{11.621762in}{6.796729in}}%
\pgfpathlineto{\pgfqpoint{11.623788in}{2.338572in}}%
\pgfpathlineto{\pgfqpoint{11.625813in}{6.837457in}}%
\pgfpathlineto{\pgfqpoint{11.627838in}{2.316029in}}%
\pgfpathlineto{\pgfqpoint{11.629864in}{6.838108in}}%
\pgfpathlineto{\pgfqpoint{11.631889in}{2.340458in}}%
\pgfpathlineto{\pgfqpoint{11.633914in}{6.786140in}}%
\pgfpathlineto{\pgfqpoint{11.635939in}{2.421526in}}%
\pgfpathlineto{\pgfqpoint{11.637965in}{6.675421in}}%
\pgfpathlineto{\pgfqpoint{11.639990in}{2.561384in}}%
\pgfpathlineto{\pgfqpoint{11.642015in}{6.507979in}}%
\pgfpathlineto{\pgfqpoint{11.644040in}{2.753902in}}%
\pgfpathlineto{\pgfqpoint{11.646066in}{6.293694in}}%
\pgfpathlineto{\pgfqpoint{11.648091in}{2.986058in}}%
\pgfpathlineto{\pgfqpoint{11.650116in}{6.047897in}}%
\pgfpathlineto{\pgfqpoint{11.652142in}{3.241199in}}%
\pgfpathlineto{\pgfqpoint{11.654167in}{5.787500in}}%
\pgfpathlineto{\pgfqpoint{11.656192in}{3.503226in}}%
\pgfpathlineto{\pgfqpoint{11.658217in}{5.526797in}}%
\pgfpathlineto{\pgfqpoint{11.660243in}{3.760479in}}%
\pgfpathlineto{\pgfqpoint{11.662268in}{5.274212in}}%
\pgfpathlineto{\pgfqpoint{11.664293in}{4.008093in}}%
\pgfpathlineto{\pgfqpoint{11.666318in}{5.031036in}}%
\pgfpathlineto{\pgfqpoint{11.668344in}{4.248052in}}%
\pgfpathlineto{\pgfqpoint{11.670369in}{4.792596in}}%
\pgfpathlineto{\pgfqpoint{11.672394in}{4.486891in}}%
\pgfpathlineto{\pgfqpoint{11.672394in}{4.486891in}}%
\pgfusepath{stroke}%
\end{pgfscope}%
\begin{pgfscope}%
\pgfpathrectangle{\pgfqpoint{3.128011in}{1.247073in}}{\pgfqpoint{8.793249in}{6.674186in}}%
\pgfusepath{clip}%
\pgfsetbuttcap%
\pgfsetroundjoin%
\pgfsetlinewidth{1.003750pt}%
\definecolor{currentstroke}{rgb}{0.888874,0.435649,0.278123}%
\pgfsetstrokecolor{currentstroke}%
\pgfsetdash{}{0pt}%
\pgfpathmoveto{\pgfqpoint{3.376876in}{4.579861in}}%
\pgfpathlineto{\pgfqpoint{11.672394in}{4.579861in}}%
\pgfpathlineto{\pgfqpoint{11.672394in}{4.579861in}}%
\pgfusepath{stroke}%
\end{pgfscope}%
\begin{pgfscope}%
\pgfsetrectcap%
\pgfsetmiterjoin%
\pgfsetlinewidth{1.003750pt}%
\definecolor{currentstroke}{rgb}{0.000000,0.000000,0.000000}%
\pgfsetstrokecolor{currentstroke}%
\pgfsetdash{}{0pt}%
\pgfpathmoveto{\pgfqpoint{3.128011in}{1.247073in}}%
\pgfpathlineto{\pgfqpoint{3.128011in}{7.921260in}}%
\pgfusepath{stroke}%
\end{pgfscope}%
\begin{pgfscope}%
\pgfsetrectcap%
\pgfsetmiterjoin%
\pgfsetlinewidth{1.003750pt}%
\definecolor{currentstroke}{rgb}{0.000000,0.000000,0.000000}%
\pgfsetstrokecolor{currentstroke}%
\pgfsetdash{}{0pt}%
\pgfpathmoveto{\pgfqpoint{3.128011in}{1.247073in}}%
\pgfpathlineto{\pgfqpoint{11.921260in}{1.247073in}}%
\pgfusepath{stroke}%
\end{pgfscope}%
\begin{pgfscope}%
\pgfsetbuttcap%
\pgfsetmiterjoin%
\definecolor{currentfill}{rgb}{1.000000,1.000000,1.000000}%
\pgfsetfillcolor{currentfill}%
\pgfsetlinewidth{1.003750pt}%
\definecolor{currentstroke}{rgb}{0.000000,0.000000,0.000000}%
\pgfsetstrokecolor{currentstroke}%
\pgfsetdash{}{0pt}%
\pgfpathmoveto{\pgfqpoint{10.511589in}{6.787373in}}%
\pgfpathlineto{\pgfqpoint{11.796260in}{6.787373in}}%
\pgfpathlineto{\pgfqpoint{11.796260in}{7.796260in}}%
\pgfpathlineto{\pgfqpoint{10.511589in}{7.796260in}}%
\pgfpathclose%
\pgfusepath{stroke,fill}%
\end{pgfscope}%
\begin{pgfscope}%
\pgfsetbuttcap%
\pgfsetmiterjoin%
\pgfsetlinewidth{2.258437pt}%
\definecolor{currentstroke}{rgb}{0.000000,0.605603,0.978680}%
\pgfsetstrokecolor{currentstroke}%
\pgfsetdash{}{0pt}%
\pgfpathmoveto{\pgfqpoint{10.711589in}{7.493818in}}%
\pgfpathlineto{\pgfqpoint{11.211589in}{7.493818in}}%
\pgfusepath{stroke}%
\end{pgfscope}%
\begin{pgfscope}%
\definecolor{textcolor}{rgb}{0.000000,0.000000,0.000000}%
\pgfsetstrokecolor{textcolor}%
\pgfsetfillcolor{textcolor}%
\pgftext[x=11.411589in,y=7.406318in,left,base]{\color{textcolor}\sffamily\fontsize{18.000000}{21.600000}\selectfont $\displaystyle U$}%
\end{pgfscope}%
\begin{pgfscope}%
\pgfsetbuttcap%
\pgfsetmiterjoin%
\pgfsetlinewidth{2.258437pt}%
\definecolor{currentstroke}{rgb}{0.888874,0.435649,0.278123}%
\pgfsetstrokecolor{currentstroke}%
\pgfsetdash{}{0pt}%
\pgfpathmoveto{\pgfqpoint{10.711589in}{7.126875in}}%
\pgfpathlineto{\pgfqpoint{11.211589in}{7.126875in}}%
\pgfusepath{stroke}%
\end{pgfscope}%
\begin{pgfscope}%
\definecolor{textcolor}{rgb}{0.000000,0.000000,0.000000}%
\pgfsetstrokecolor{textcolor}%
\pgfsetfillcolor{textcolor}%
\pgftext[x=11.411589in,y=7.039375in,left,base]{\color{textcolor}\sffamily\fontsize{18.000000}{21.600000}\selectfont $\displaystyle u$}%
\end{pgfscope}%
\end{pgfpicture}%
\makeatother%
\endgroup%
}
	\caption{迎风格式差分逼近解 $U$ 与真解 $u$}\label{fig:upwind_Uu_noCFL}
\end{figure}

对方波问题, 取 $\nu = 0.5$. $h = 2^{-7}$ 和 $h = 2^{-11}$ 时差分逼近解 $U$ 与真解 $u$ 在 $t = t_{\max }$ 时刻图像如图 \ref{fig:upwind_square_Uu} 所示, 可以看出差分逼近解在间断点附近被磨光.

\begin{figure}[H]\centering\zihao{-5}
	\resizebox{0.4\linewidth}{!}{%% Creator: Matplotlib, PGF backend
%%
%% To include the figure in your LaTeX document, write
%%   \input{<filename>.pgf}
%%
%% Make sure the required packages are loaded in your preamble
%%   \usepackage{pgf}
%%
%% Figures using additional raster images can only be included by \input if
%% they are in the same directory as the main LaTeX file. For loading figures
%% from other directories you can use the `import` package
%%   \usepackage{import}
%%
%% and then include the figures with
%%   \import{<path to file>}{<filename>.pgf}
%%
%% Matplotlib used the following preamble
%%   \usepackage{fontspec}
%%   \setmainfont{DejaVuSerif.ttf}[Path=\detokenize{/Users/quejiahao/.julia/conda/3/lib/python3.9/site-packages/matplotlib/mpl-data/fonts/ttf/}]
%%   \setsansfont{DejaVuSans.ttf}[Path=\detokenize{/Users/quejiahao/.julia/conda/3/lib/python3.9/site-packages/matplotlib/mpl-data/fonts/ttf/}]
%%   \setmonofont{DejaVuSansMono.ttf}[Path=\detokenize{/Users/quejiahao/.julia/conda/3/lib/python3.9/site-packages/matplotlib/mpl-data/fonts/ttf/}]
%%
\begingroup%
\makeatletter%
\begin{pgfpicture}%
\pgfpathrectangle{\pgfpointorigin}{\pgfqpoint{12.000000in}{8.000000in}}%
\pgfusepath{use as bounding box, clip}%
\begin{pgfscope}%
\pgfsetbuttcap%
\pgfsetmiterjoin%
\definecolor{currentfill}{rgb}{1.000000,1.000000,1.000000}%
\pgfsetfillcolor{currentfill}%
\pgfsetlinewidth{0.000000pt}%
\definecolor{currentstroke}{rgb}{1.000000,1.000000,1.000000}%
\pgfsetstrokecolor{currentstroke}%
\pgfsetdash{}{0pt}%
\pgfpathmoveto{\pgfqpoint{0.000000in}{0.000000in}}%
\pgfpathlineto{\pgfqpoint{12.000000in}{0.000000in}}%
\pgfpathlineto{\pgfqpoint{12.000000in}{8.000000in}}%
\pgfpathlineto{\pgfqpoint{0.000000in}{8.000000in}}%
\pgfpathclose%
\pgfusepath{fill}%
\end{pgfscope}%
\begin{pgfscope}%
\pgfsetbuttcap%
\pgfsetmiterjoin%
\definecolor{currentfill}{rgb}{1.000000,1.000000,1.000000}%
\pgfsetfillcolor{currentfill}%
\pgfsetlinewidth{0.000000pt}%
\definecolor{currentstroke}{rgb}{0.000000,0.000000,0.000000}%
\pgfsetstrokecolor{currentstroke}%
\pgfsetstrokeopacity{0.000000}%
\pgfsetdash{}{0pt}%
\pgfpathmoveto{\pgfqpoint{0.978013in}{1.247073in}}%
\pgfpathlineto{\pgfqpoint{11.921260in}{1.247073in}}%
\pgfpathlineto{\pgfqpoint{11.921260in}{7.921260in}}%
\pgfpathlineto{\pgfqpoint{0.978013in}{7.921260in}}%
\pgfpathclose%
\pgfusepath{fill}%
\end{pgfscope}%
\begin{pgfscope}%
\pgfpathrectangle{\pgfqpoint{0.978013in}{1.247073in}}{\pgfqpoint{10.943247in}{6.674186in}}%
\pgfusepath{clip}%
\pgfsetrectcap%
\pgfsetroundjoin%
\pgfsetlinewidth{0.501875pt}%
\definecolor{currentstroke}{rgb}{0.000000,0.000000,0.000000}%
\pgfsetstrokecolor{currentstroke}%
\pgfsetstrokeopacity{0.100000}%
\pgfsetdash{}{0pt}%
\pgfpathmoveto{\pgfqpoint{1.287728in}{1.247073in}}%
\pgfpathlineto{\pgfqpoint{1.287728in}{7.921260in}}%
\pgfusepath{stroke}%
\end{pgfscope}%
\begin{pgfscope}%
\pgfsetbuttcap%
\pgfsetroundjoin%
\definecolor{currentfill}{rgb}{0.000000,0.000000,0.000000}%
\pgfsetfillcolor{currentfill}%
\pgfsetlinewidth{0.501875pt}%
\definecolor{currentstroke}{rgb}{0.000000,0.000000,0.000000}%
\pgfsetstrokecolor{currentstroke}%
\pgfsetdash{}{0pt}%
\pgfsys@defobject{currentmarker}{\pgfqpoint{0.000000in}{0.000000in}}{\pgfqpoint{0.000000in}{0.034722in}}{%
\pgfpathmoveto{\pgfqpoint{0.000000in}{0.000000in}}%
\pgfpathlineto{\pgfqpoint{0.000000in}{0.034722in}}%
\pgfusepath{stroke,fill}%
}%
\begin{pgfscope}%
\pgfsys@transformshift{1.287728in}{1.247073in}%
\pgfsys@useobject{currentmarker}{}%
\end{pgfscope}%
\end{pgfscope}%
\begin{pgfscope}%
\definecolor{textcolor}{rgb}{0.000000,0.000000,0.000000}%
\pgfsetstrokecolor{textcolor}%
\pgfsetfillcolor{textcolor}%
\pgftext[x=1.287728in,y=1.198462in,,top]{\color{textcolor}\sffamily\fontsize{18.000000}{21.600000}\selectfont $\displaystyle 0$}%
\end{pgfscope}%
\begin{pgfscope}%
\pgfpathrectangle{\pgfqpoint{0.978013in}{1.247073in}}{\pgfqpoint{10.943247in}{6.674186in}}%
\pgfusepath{clip}%
\pgfsetrectcap%
\pgfsetroundjoin%
\pgfsetlinewidth{0.501875pt}%
\definecolor{currentstroke}{rgb}{0.000000,0.000000,0.000000}%
\pgfsetstrokecolor{currentstroke}%
\pgfsetstrokeopacity{0.100000}%
\pgfsetdash{}{0pt}%
\pgfpathmoveto{\pgfqpoint{2.930814in}{1.247073in}}%
\pgfpathlineto{\pgfqpoint{2.930814in}{7.921260in}}%
\pgfusepath{stroke}%
\end{pgfscope}%
\begin{pgfscope}%
\pgfsetbuttcap%
\pgfsetroundjoin%
\definecolor{currentfill}{rgb}{0.000000,0.000000,0.000000}%
\pgfsetfillcolor{currentfill}%
\pgfsetlinewidth{0.501875pt}%
\definecolor{currentstroke}{rgb}{0.000000,0.000000,0.000000}%
\pgfsetstrokecolor{currentstroke}%
\pgfsetdash{}{0pt}%
\pgfsys@defobject{currentmarker}{\pgfqpoint{0.000000in}{0.000000in}}{\pgfqpoint{0.000000in}{0.034722in}}{%
\pgfpathmoveto{\pgfqpoint{0.000000in}{0.000000in}}%
\pgfpathlineto{\pgfqpoint{0.000000in}{0.034722in}}%
\pgfusepath{stroke,fill}%
}%
\begin{pgfscope}%
\pgfsys@transformshift{2.930814in}{1.247073in}%
\pgfsys@useobject{currentmarker}{}%
\end{pgfscope}%
\end{pgfscope}%
\begin{pgfscope}%
\definecolor{textcolor}{rgb}{0.000000,0.000000,0.000000}%
\pgfsetstrokecolor{textcolor}%
\pgfsetfillcolor{textcolor}%
\pgftext[x=2.930814in,y=1.198462in,,top]{\color{textcolor}\sffamily\fontsize{18.000000}{21.600000}\selectfont $\displaystyle 1$}%
\end{pgfscope}%
\begin{pgfscope}%
\pgfpathrectangle{\pgfqpoint{0.978013in}{1.247073in}}{\pgfqpoint{10.943247in}{6.674186in}}%
\pgfusepath{clip}%
\pgfsetrectcap%
\pgfsetroundjoin%
\pgfsetlinewidth{0.501875pt}%
\definecolor{currentstroke}{rgb}{0.000000,0.000000,0.000000}%
\pgfsetstrokecolor{currentstroke}%
\pgfsetstrokeopacity{0.100000}%
\pgfsetdash{}{0pt}%
\pgfpathmoveto{\pgfqpoint{4.573901in}{1.247073in}}%
\pgfpathlineto{\pgfqpoint{4.573901in}{7.921260in}}%
\pgfusepath{stroke}%
\end{pgfscope}%
\begin{pgfscope}%
\pgfsetbuttcap%
\pgfsetroundjoin%
\definecolor{currentfill}{rgb}{0.000000,0.000000,0.000000}%
\pgfsetfillcolor{currentfill}%
\pgfsetlinewidth{0.501875pt}%
\definecolor{currentstroke}{rgb}{0.000000,0.000000,0.000000}%
\pgfsetstrokecolor{currentstroke}%
\pgfsetdash{}{0pt}%
\pgfsys@defobject{currentmarker}{\pgfqpoint{0.000000in}{0.000000in}}{\pgfqpoint{0.000000in}{0.034722in}}{%
\pgfpathmoveto{\pgfqpoint{0.000000in}{0.000000in}}%
\pgfpathlineto{\pgfqpoint{0.000000in}{0.034722in}}%
\pgfusepath{stroke,fill}%
}%
\begin{pgfscope}%
\pgfsys@transformshift{4.573901in}{1.247073in}%
\pgfsys@useobject{currentmarker}{}%
\end{pgfscope}%
\end{pgfscope}%
\begin{pgfscope}%
\definecolor{textcolor}{rgb}{0.000000,0.000000,0.000000}%
\pgfsetstrokecolor{textcolor}%
\pgfsetfillcolor{textcolor}%
\pgftext[x=4.573901in,y=1.198462in,,top]{\color{textcolor}\sffamily\fontsize{18.000000}{21.600000}\selectfont $\displaystyle 2$}%
\end{pgfscope}%
\begin{pgfscope}%
\pgfpathrectangle{\pgfqpoint{0.978013in}{1.247073in}}{\pgfqpoint{10.943247in}{6.674186in}}%
\pgfusepath{clip}%
\pgfsetrectcap%
\pgfsetroundjoin%
\pgfsetlinewidth{0.501875pt}%
\definecolor{currentstroke}{rgb}{0.000000,0.000000,0.000000}%
\pgfsetstrokecolor{currentstroke}%
\pgfsetstrokeopacity{0.100000}%
\pgfsetdash{}{0pt}%
\pgfpathmoveto{\pgfqpoint{6.216988in}{1.247073in}}%
\pgfpathlineto{\pgfqpoint{6.216988in}{7.921260in}}%
\pgfusepath{stroke}%
\end{pgfscope}%
\begin{pgfscope}%
\pgfsetbuttcap%
\pgfsetroundjoin%
\definecolor{currentfill}{rgb}{0.000000,0.000000,0.000000}%
\pgfsetfillcolor{currentfill}%
\pgfsetlinewidth{0.501875pt}%
\definecolor{currentstroke}{rgb}{0.000000,0.000000,0.000000}%
\pgfsetstrokecolor{currentstroke}%
\pgfsetdash{}{0pt}%
\pgfsys@defobject{currentmarker}{\pgfqpoint{0.000000in}{0.000000in}}{\pgfqpoint{0.000000in}{0.034722in}}{%
\pgfpathmoveto{\pgfqpoint{0.000000in}{0.000000in}}%
\pgfpathlineto{\pgfqpoint{0.000000in}{0.034722in}}%
\pgfusepath{stroke,fill}%
}%
\begin{pgfscope}%
\pgfsys@transformshift{6.216988in}{1.247073in}%
\pgfsys@useobject{currentmarker}{}%
\end{pgfscope}%
\end{pgfscope}%
\begin{pgfscope}%
\definecolor{textcolor}{rgb}{0.000000,0.000000,0.000000}%
\pgfsetstrokecolor{textcolor}%
\pgfsetfillcolor{textcolor}%
\pgftext[x=6.216988in,y=1.198462in,,top]{\color{textcolor}\sffamily\fontsize{18.000000}{21.600000}\selectfont $\displaystyle 3$}%
\end{pgfscope}%
\begin{pgfscope}%
\pgfpathrectangle{\pgfqpoint{0.978013in}{1.247073in}}{\pgfqpoint{10.943247in}{6.674186in}}%
\pgfusepath{clip}%
\pgfsetrectcap%
\pgfsetroundjoin%
\pgfsetlinewidth{0.501875pt}%
\definecolor{currentstroke}{rgb}{0.000000,0.000000,0.000000}%
\pgfsetstrokecolor{currentstroke}%
\pgfsetstrokeopacity{0.100000}%
\pgfsetdash{}{0pt}%
\pgfpathmoveto{\pgfqpoint{7.860074in}{1.247073in}}%
\pgfpathlineto{\pgfqpoint{7.860074in}{7.921260in}}%
\pgfusepath{stroke}%
\end{pgfscope}%
\begin{pgfscope}%
\pgfsetbuttcap%
\pgfsetroundjoin%
\definecolor{currentfill}{rgb}{0.000000,0.000000,0.000000}%
\pgfsetfillcolor{currentfill}%
\pgfsetlinewidth{0.501875pt}%
\definecolor{currentstroke}{rgb}{0.000000,0.000000,0.000000}%
\pgfsetstrokecolor{currentstroke}%
\pgfsetdash{}{0pt}%
\pgfsys@defobject{currentmarker}{\pgfqpoint{0.000000in}{0.000000in}}{\pgfqpoint{0.000000in}{0.034722in}}{%
\pgfpathmoveto{\pgfqpoint{0.000000in}{0.000000in}}%
\pgfpathlineto{\pgfqpoint{0.000000in}{0.034722in}}%
\pgfusepath{stroke,fill}%
}%
\begin{pgfscope}%
\pgfsys@transformshift{7.860074in}{1.247073in}%
\pgfsys@useobject{currentmarker}{}%
\end{pgfscope}%
\end{pgfscope}%
\begin{pgfscope}%
\definecolor{textcolor}{rgb}{0.000000,0.000000,0.000000}%
\pgfsetstrokecolor{textcolor}%
\pgfsetfillcolor{textcolor}%
\pgftext[x=7.860074in,y=1.198462in,,top]{\color{textcolor}\sffamily\fontsize{18.000000}{21.600000}\selectfont $\displaystyle 4$}%
\end{pgfscope}%
\begin{pgfscope}%
\pgfpathrectangle{\pgfqpoint{0.978013in}{1.247073in}}{\pgfqpoint{10.943247in}{6.674186in}}%
\pgfusepath{clip}%
\pgfsetrectcap%
\pgfsetroundjoin%
\pgfsetlinewidth{0.501875pt}%
\definecolor{currentstroke}{rgb}{0.000000,0.000000,0.000000}%
\pgfsetstrokecolor{currentstroke}%
\pgfsetstrokeopacity{0.100000}%
\pgfsetdash{}{0pt}%
\pgfpathmoveto{\pgfqpoint{9.503161in}{1.247073in}}%
\pgfpathlineto{\pgfqpoint{9.503161in}{7.921260in}}%
\pgfusepath{stroke}%
\end{pgfscope}%
\begin{pgfscope}%
\pgfsetbuttcap%
\pgfsetroundjoin%
\definecolor{currentfill}{rgb}{0.000000,0.000000,0.000000}%
\pgfsetfillcolor{currentfill}%
\pgfsetlinewidth{0.501875pt}%
\definecolor{currentstroke}{rgb}{0.000000,0.000000,0.000000}%
\pgfsetstrokecolor{currentstroke}%
\pgfsetdash{}{0pt}%
\pgfsys@defobject{currentmarker}{\pgfqpoint{0.000000in}{0.000000in}}{\pgfqpoint{0.000000in}{0.034722in}}{%
\pgfpathmoveto{\pgfqpoint{0.000000in}{0.000000in}}%
\pgfpathlineto{\pgfqpoint{0.000000in}{0.034722in}}%
\pgfusepath{stroke,fill}%
}%
\begin{pgfscope}%
\pgfsys@transformshift{9.503161in}{1.247073in}%
\pgfsys@useobject{currentmarker}{}%
\end{pgfscope}%
\end{pgfscope}%
\begin{pgfscope}%
\definecolor{textcolor}{rgb}{0.000000,0.000000,0.000000}%
\pgfsetstrokecolor{textcolor}%
\pgfsetfillcolor{textcolor}%
\pgftext[x=9.503161in,y=1.198462in,,top]{\color{textcolor}\sffamily\fontsize{18.000000}{21.600000}\selectfont $\displaystyle 5$}%
\end{pgfscope}%
\begin{pgfscope}%
\pgfpathrectangle{\pgfqpoint{0.978013in}{1.247073in}}{\pgfqpoint{10.943247in}{6.674186in}}%
\pgfusepath{clip}%
\pgfsetrectcap%
\pgfsetroundjoin%
\pgfsetlinewidth{0.501875pt}%
\definecolor{currentstroke}{rgb}{0.000000,0.000000,0.000000}%
\pgfsetstrokecolor{currentstroke}%
\pgfsetstrokeopacity{0.100000}%
\pgfsetdash{}{0pt}%
\pgfpathmoveto{\pgfqpoint{11.146247in}{1.247073in}}%
\pgfpathlineto{\pgfqpoint{11.146247in}{7.921260in}}%
\pgfusepath{stroke}%
\end{pgfscope}%
\begin{pgfscope}%
\pgfsetbuttcap%
\pgfsetroundjoin%
\definecolor{currentfill}{rgb}{0.000000,0.000000,0.000000}%
\pgfsetfillcolor{currentfill}%
\pgfsetlinewidth{0.501875pt}%
\definecolor{currentstroke}{rgb}{0.000000,0.000000,0.000000}%
\pgfsetstrokecolor{currentstroke}%
\pgfsetdash{}{0pt}%
\pgfsys@defobject{currentmarker}{\pgfqpoint{0.000000in}{0.000000in}}{\pgfqpoint{0.000000in}{0.034722in}}{%
\pgfpathmoveto{\pgfqpoint{0.000000in}{0.000000in}}%
\pgfpathlineto{\pgfqpoint{0.000000in}{0.034722in}}%
\pgfusepath{stroke,fill}%
}%
\begin{pgfscope}%
\pgfsys@transformshift{11.146247in}{1.247073in}%
\pgfsys@useobject{currentmarker}{}%
\end{pgfscope}%
\end{pgfscope}%
\begin{pgfscope}%
\definecolor{textcolor}{rgb}{0.000000,0.000000,0.000000}%
\pgfsetstrokecolor{textcolor}%
\pgfsetfillcolor{textcolor}%
\pgftext[x=11.146247in,y=1.198462in,,top]{\color{textcolor}\sffamily\fontsize{18.000000}{21.600000}\selectfont $\displaystyle 6$}%
\end{pgfscope}%
\begin{pgfscope}%
\definecolor{textcolor}{rgb}{0.000000,0.000000,0.000000}%
\pgfsetstrokecolor{textcolor}%
\pgfsetfillcolor{textcolor}%
\pgftext[x=6.449637in,y=0.900964in,,top]{\color{textcolor}\sffamily\fontsize{18.000000}{21.600000}\selectfont $\displaystyle x$}%
\end{pgfscope}%
\begin{pgfscope}%
\pgfpathrectangle{\pgfqpoint{0.978013in}{1.247073in}}{\pgfqpoint{10.943247in}{6.674186in}}%
\pgfusepath{clip}%
\pgfsetrectcap%
\pgfsetroundjoin%
\pgfsetlinewidth{0.501875pt}%
\definecolor{currentstroke}{rgb}{0.000000,0.000000,0.000000}%
\pgfsetstrokecolor{currentstroke}%
\pgfsetstrokeopacity{0.100000}%
\pgfsetdash{}{0pt}%
\pgfpathmoveto{\pgfqpoint{0.978013in}{1.435966in}}%
\pgfpathlineto{\pgfqpoint{11.921260in}{1.435966in}}%
\pgfusepath{stroke}%
\end{pgfscope}%
\begin{pgfscope}%
\pgfsetbuttcap%
\pgfsetroundjoin%
\definecolor{currentfill}{rgb}{0.000000,0.000000,0.000000}%
\pgfsetfillcolor{currentfill}%
\pgfsetlinewidth{0.501875pt}%
\definecolor{currentstroke}{rgb}{0.000000,0.000000,0.000000}%
\pgfsetstrokecolor{currentstroke}%
\pgfsetdash{}{0pt}%
\pgfsys@defobject{currentmarker}{\pgfqpoint{0.000000in}{0.000000in}}{\pgfqpoint{0.034722in}{0.000000in}}{%
\pgfpathmoveto{\pgfqpoint{0.000000in}{0.000000in}}%
\pgfpathlineto{\pgfqpoint{0.034722in}{0.000000in}}%
\pgfusepath{stroke,fill}%
}%
\begin{pgfscope}%
\pgfsys@transformshift{0.978013in}{1.435966in}%
\pgfsys@useobject{currentmarker}{}%
\end{pgfscope}%
\end{pgfscope}%
\begin{pgfscope}%
\definecolor{textcolor}{rgb}{0.000000,0.000000,0.000000}%
\pgfsetstrokecolor{textcolor}%
\pgfsetfillcolor{textcolor}%
\pgftext[x=0.643989in, y=1.340995in, left, base]{\color{textcolor}\sffamily\fontsize{18.000000}{21.600000}\selectfont $\displaystyle 1.0$}%
\end{pgfscope}%
\begin{pgfscope}%
\pgfpathrectangle{\pgfqpoint{0.978013in}{1.247073in}}{\pgfqpoint{10.943247in}{6.674186in}}%
\pgfusepath{clip}%
\pgfsetrectcap%
\pgfsetroundjoin%
\pgfsetlinewidth{0.501875pt}%
\definecolor{currentstroke}{rgb}{0.000000,0.000000,0.000000}%
\pgfsetstrokecolor{currentstroke}%
\pgfsetstrokeopacity{0.100000}%
\pgfsetdash{}{0pt}%
\pgfpathmoveto{\pgfqpoint{0.978013in}{3.010066in}}%
\pgfpathlineto{\pgfqpoint{11.921260in}{3.010066in}}%
\pgfusepath{stroke}%
\end{pgfscope}%
\begin{pgfscope}%
\pgfsetbuttcap%
\pgfsetroundjoin%
\definecolor{currentfill}{rgb}{0.000000,0.000000,0.000000}%
\pgfsetfillcolor{currentfill}%
\pgfsetlinewidth{0.501875pt}%
\definecolor{currentstroke}{rgb}{0.000000,0.000000,0.000000}%
\pgfsetstrokecolor{currentstroke}%
\pgfsetdash{}{0pt}%
\pgfsys@defobject{currentmarker}{\pgfqpoint{0.000000in}{0.000000in}}{\pgfqpoint{0.034722in}{0.000000in}}{%
\pgfpathmoveto{\pgfqpoint{0.000000in}{0.000000in}}%
\pgfpathlineto{\pgfqpoint{0.034722in}{0.000000in}}%
\pgfusepath{stroke,fill}%
}%
\begin{pgfscope}%
\pgfsys@transformshift{0.978013in}{3.010066in}%
\pgfsys@useobject{currentmarker}{}%
\end{pgfscope}%
\end{pgfscope}%
\begin{pgfscope}%
\definecolor{textcolor}{rgb}{0.000000,0.000000,0.000000}%
\pgfsetstrokecolor{textcolor}%
\pgfsetfillcolor{textcolor}%
\pgftext[x=0.643989in, y=2.915095in, left, base]{\color{textcolor}\sffamily\fontsize{18.000000}{21.600000}\selectfont $\displaystyle 1.5$}%
\end{pgfscope}%
\begin{pgfscope}%
\pgfpathrectangle{\pgfqpoint{0.978013in}{1.247073in}}{\pgfqpoint{10.943247in}{6.674186in}}%
\pgfusepath{clip}%
\pgfsetrectcap%
\pgfsetroundjoin%
\pgfsetlinewidth{0.501875pt}%
\definecolor{currentstroke}{rgb}{0.000000,0.000000,0.000000}%
\pgfsetstrokecolor{currentstroke}%
\pgfsetstrokeopacity{0.100000}%
\pgfsetdash{}{0pt}%
\pgfpathmoveto{\pgfqpoint{0.978013in}{4.584167in}}%
\pgfpathlineto{\pgfqpoint{11.921260in}{4.584167in}}%
\pgfusepath{stroke}%
\end{pgfscope}%
\begin{pgfscope}%
\pgfsetbuttcap%
\pgfsetroundjoin%
\definecolor{currentfill}{rgb}{0.000000,0.000000,0.000000}%
\pgfsetfillcolor{currentfill}%
\pgfsetlinewidth{0.501875pt}%
\definecolor{currentstroke}{rgb}{0.000000,0.000000,0.000000}%
\pgfsetstrokecolor{currentstroke}%
\pgfsetdash{}{0pt}%
\pgfsys@defobject{currentmarker}{\pgfqpoint{0.000000in}{0.000000in}}{\pgfqpoint{0.034722in}{0.000000in}}{%
\pgfpathmoveto{\pgfqpoint{0.000000in}{0.000000in}}%
\pgfpathlineto{\pgfqpoint{0.034722in}{0.000000in}}%
\pgfusepath{stroke,fill}%
}%
\begin{pgfscope}%
\pgfsys@transformshift{0.978013in}{4.584167in}%
\pgfsys@useobject{currentmarker}{}%
\end{pgfscope}%
\end{pgfscope}%
\begin{pgfscope}%
\definecolor{textcolor}{rgb}{0.000000,0.000000,0.000000}%
\pgfsetstrokecolor{textcolor}%
\pgfsetfillcolor{textcolor}%
\pgftext[x=0.643989in, y=4.489196in, left, base]{\color{textcolor}\sffamily\fontsize{18.000000}{21.600000}\selectfont $\displaystyle 2.0$}%
\end{pgfscope}%
\begin{pgfscope}%
\pgfpathrectangle{\pgfqpoint{0.978013in}{1.247073in}}{\pgfqpoint{10.943247in}{6.674186in}}%
\pgfusepath{clip}%
\pgfsetrectcap%
\pgfsetroundjoin%
\pgfsetlinewidth{0.501875pt}%
\definecolor{currentstroke}{rgb}{0.000000,0.000000,0.000000}%
\pgfsetstrokecolor{currentstroke}%
\pgfsetstrokeopacity{0.100000}%
\pgfsetdash{}{0pt}%
\pgfpathmoveto{\pgfqpoint{0.978013in}{6.158267in}}%
\pgfpathlineto{\pgfqpoint{11.921260in}{6.158267in}}%
\pgfusepath{stroke}%
\end{pgfscope}%
\begin{pgfscope}%
\pgfsetbuttcap%
\pgfsetroundjoin%
\definecolor{currentfill}{rgb}{0.000000,0.000000,0.000000}%
\pgfsetfillcolor{currentfill}%
\pgfsetlinewidth{0.501875pt}%
\definecolor{currentstroke}{rgb}{0.000000,0.000000,0.000000}%
\pgfsetstrokecolor{currentstroke}%
\pgfsetdash{}{0pt}%
\pgfsys@defobject{currentmarker}{\pgfqpoint{0.000000in}{0.000000in}}{\pgfqpoint{0.034722in}{0.000000in}}{%
\pgfpathmoveto{\pgfqpoint{0.000000in}{0.000000in}}%
\pgfpathlineto{\pgfqpoint{0.034722in}{0.000000in}}%
\pgfusepath{stroke,fill}%
}%
\begin{pgfscope}%
\pgfsys@transformshift{0.978013in}{6.158267in}%
\pgfsys@useobject{currentmarker}{}%
\end{pgfscope}%
\end{pgfscope}%
\begin{pgfscope}%
\definecolor{textcolor}{rgb}{0.000000,0.000000,0.000000}%
\pgfsetstrokecolor{textcolor}%
\pgfsetfillcolor{textcolor}%
\pgftext[x=0.643989in, y=6.063297in, left, base]{\color{textcolor}\sffamily\fontsize{18.000000}{21.600000}\selectfont $\displaystyle 2.5$}%
\end{pgfscope}%
\begin{pgfscope}%
\pgfpathrectangle{\pgfqpoint{0.978013in}{1.247073in}}{\pgfqpoint{10.943247in}{6.674186in}}%
\pgfusepath{clip}%
\pgfsetrectcap%
\pgfsetroundjoin%
\pgfsetlinewidth{0.501875pt}%
\definecolor{currentstroke}{rgb}{0.000000,0.000000,0.000000}%
\pgfsetstrokecolor{currentstroke}%
\pgfsetstrokeopacity{0.100000}%
\pgfsetdash{}{0pt}%
\pgfpathmoveto{\pgfqpoint{0.978013in}{7.732368in}}%
\pgfpathlineto{\pgfqpoint{11.921260in}{7.732368in}}%
\pgfusepath{stroke}%
\end{pgfscope}%
\begin{pgfscope}%
\pgfsetbuttcap%
\pgfsetroundjoin%
\definecolor{currentfill}{rgb}{0.000000,0.000000,0.000000}%
\pgfsetfillcolor{currentfill}%
\pgfsetlinewidth{0.501875pt}%
\definecolor{currentstroke}{rgb}{0.000000,0.000000,0.000000}%
\pgfsetstrokecolor{currentstroke}%
\pgfsetdash{}{0pt}%
\pgfsys@defobject{currentmarker}{\pgfqpoint{0.000000in}{0.000000in}}{\pgfqpoint{0.034722in}{0.000000in}}{%
\pgfpathmoveto{\pgfqpoint{0.000000in}{0.000000in}}%
\pgfpathlineto{\pgfqpoint{0.034722in}{0.000000in}}%
\pgfusepath{stroke,fill}%
}%
\begin{pgfscope}%
\pgfsys@transformshift{0.978013in}{7.732368in}%
\pgfsys@useobject{currentmarker}{}%
\end{pgfscope}%
\end{pgfscope}%
\begin{pgfscope}%
\definecolor{textcolor}{rgb}{0.000000,0.000000,0.000000}%
\pgfsetstrokecolor{textcolor}%
\pgfsetfillcolor{textcolor}%
\pgftext[x=0.643989in, y=7.637397in, left, base]{\color{textcolor}\sffamily\fontsize{18.000000}{21.600000}\selectfont $\displaystyle 3.0$}%
\end{pgfscope}%
\begin{pgfscope}%
\pgfpathrectangle{\pgfqpoint{0.978013in}{1.247073in}}{\pgfqpoint{10.943247in}{6.674186in}}%
\pgfusepath{clip}%
\pgfsetbuttcap%
\pgfsetroundjoin%
\pgfsetlinewidth{1.003750pt}%
\definecolor{currentstroke}{rgb}{0.000000,0.605603,0.978680}%
\pgfsetstrokecolor{currentstroke}%
\pgfsetdash{}{0pt}%
\pgfpathmoveto{\pgfqpoint{1.287728in}{7.732368in}}%
\pgfpathlineto{\pgfqpoint{5.723743in}{7.731809in}}%
\pgfpathlineto{\pgfqpoint{5.804398in}{7.730602in}}%
\pgfpathlineto{\pgfqpoint{5.885053in}{7.727300in}}%
\pgfpathlineto{\pgfqpoint{5.965708in}{7.719092in}}%
\pgfpathlineto{\pgfqpoint{6.046362in}{7.700479in}}%
\pgfpathlineto{\pgfqpoint{6.127017in}{7.661856in}}%
\pgfpathlineto{\pgfqpoint{6.207672in}{7.588326in}}%
\pgfpathlineto{\pgfqpoint{6.288327in}{7.459610in}}%
\pgfpathlineto{\pgfqpoint{6.368982in}{7.252044in}}%
\pgfpathlineto{\pgfqpoint{6.449637in}{6.943224in}}%
\pgfpathlineto{\pgfqpoint{6.530291in}{6.518790in}}%
\pgfpathlineto{\pgfqpoint{6.610946in}{5.979420in}}%
\pgfpathlineto{\pgfqpoint{6.691601in}{5.345201in}}%
\pgfpathlineto{\pgfqpoint{6.852911in}{3.959064in}}%
\pgfpathlineto{\pgfqpoint{6.933565in}{3.309745in}}%
\pgfpathlineto{\pgfqpoint{7.014220in}{2.748803in}}%
\pgfpathlineto{\pgfqpoint{7.094875in}{2.300415in}}%
\pgfpathlineto{\pgfqpoint{7.175530in}{1.969008in}}%
\pgfpathlineto{\pgfqpoint{7.256185in}{1.742739in}}%
\pgfpathlineto{\pgfqpoint{7.336840in}{1.600207in}}%
\pgfpathlineto{\pgfqpoint{7.417494in}{1.517498in}}%
\pgfpathlineto{\pgfqpoint{7.498149in}{1.473365in}}%
\pgfpathlineto{\pgfqpoint{7.578804in}{1.451761in}}%
\pgfpathlineto{\pgfqpoint{7.659459in}{1.442083in}}%
\pgfpathlineto{\pgfqpoint{7.740114in}{1.438129in}}%
\pgfpathlineto{\pgfqpoint{7.901423in}{1.436167in}}%
\pgfpathlineto{\pgfqpoint{8.546662in}{1.435966in}}%
\pgfpathlineto{\pgfqpoint{11.611545in}{1.435966in}}%
\pgfpathlineto{\pgfqpoint{11.611545in}{1.435966in}}%
\pgfusepath{stroke}%
\end{pgfscope}%
\begin{pgfscope}%
\pgfpathrectangle{\pgfqpoint{0.978013in}{1.247073in}}{\pgfqpoint{10.943247in}{6.674186in}}%
\pgfusepath{clip}%
\pgfsetbuttcap%
\pgfsetroundjoin%
\pgfsetlinewidth{1.003750pt}%
\definecolor{currentstroke}{rgb}{0.888874,0.435649,0.278123}%
\pgfsetstrokecolor{currentstroke}%
\pgfsetdash{}{0pt}%
\pgfpathmoveto{\pgfqpoint{1.287728in}{7.732368in}}%
\pgfpathlineto{\pgfqpoint{6.772256in}{7.732368in}}%
\pgfpathlineto{\pgfqpoint{6.852911in}{1.435966in}}%
\pgfpathlineto{\pgfqpoint{11.611545in}{1.435966in}}%
\pgfpathlineto{\pgfqpoint{11.611545in}{1.435966in}}%
\pgfusepath{stroke}%
\end{pgfscope}%
\begin{pgfscope}%
\pgfsetrectcap%
\pgfsetmiterjoin%
\pgfsetlinewidth{1.003750pt}%
\definecolor{currentstroke}{rgb}{0.000000,0.000000,0.000000}%
\pgfsetstrokecolor{currentstroke}%
\pgfsetdash{}{0pt}%
\pgfpathmoveto{\pgfqpoint{0.978013in}{1.247073in}}%
\pgfpathlineto{\pgfqpoint{0.978013in}{7.921260in}}%
\pgfusepath{stroke}%
\end{pgfscope}%
\begin{pgfscope}%
\pgfsetrectcap%
\pgfsetmiterjoin%
\pgfsetlinewidth{1.003750pt}%
\definecolor{currentstroke}{rgb}{0.000000,0.000000,0.000000}%
\pgfsetstrokecolor{currentstroke}%
\pgfsetdash{}{0pt}%
\pgfpathmoveto{\pgfqpoint{0.978013in}{1.247073in}}%
\pgfpathlineto{\pgfqpoint{11.921260in}{1.247073in}}%
\pgfusepath{stroke}%
\end{pgfscope}%
\begin{pgfscope}%
\pgfsetbuttcap%
\pgfsetmiterjoin%
\definecolor{currentfill}{rgb}{1.000000,1.000000,1.000000}%
\pgfsetfillcolor{currentfill}%
\pgfsetlinewidth{1.003750pt}%
\definecolor{currentstroke}{rgb}{0.000000,0.000000,0.000000}%
\pgfsetstrokecolor{currentstroke}%
\pgfsetdash{}{0pt}%
\pgfpathmoveto{\pgfqpoint{10.511589in}{6.787373in}}%
\pgfpathlineto{\pgfqpoint{11.796260in}{6.787373in}}%
\pgfpathlineto{\pgfqpoint{11.796260in}{7.796260in}}%
\pgfpathlineto{\pgfqpoint{10.511589in}{7.796260in}}%
\pgfpathclose%
\pgfusepath{stroke,fill}%
\end{pgfscope}%
\begin{pgfscope}%
\pgfsetbuttcap%
\pgfsetmiterjoin%
\pgfsetlinewidth{2.258437pt}%
\definecolor{currentstroke}{rgb}{0.000000,0.605603,0.978680}%
\pgfsetstrokecolor{currentstroke}%
\pgfsetdash{}{0pt}%
\pgfpathmoveto{\pgfqpoint{10.711589in}{7.493818in}}%
\pgfpathlineto{\pgfqpoint{11.211589in}{7.493818in}}%
\pgfusepath{stroke}%
\end{pgfscope}%
\begin{pgfscope}%
\definecolor{textcolor}{rgb}{0.000000,0.000000,0.000000}%
\pgfsetstrokecolor{textcolor}%
\pgfsetfillcolor{textcolor}%
\pgftext[x=11.411589in,y=7.406318in,left,base]{\color{textcolor}\sffamily\fontsize{18.000000}{21.600000}\selectfont $\displaystyle U$}%
\end{pgfscope}%
\begin{pgfscope}%
\pgfsetbuttcap%
\pgfsetmiterjoin%
\pgfsetlinewidth{2.258437pt}%
\definecolor{currentstroke}{rgb}{0.888874,0.435649,0.278123}%
\pgfsetstrokecolor{currentstroke}%
\pgfsetdash{}{0pt}%
\pgfpathmoveto{\pgfqpoint{10.711589in}{7.126875in}}%
\pgfpathlineto{\pgfqpoint{11.211589in}{7.126875in}}%
\pgfusepath{stroke}%
\end{pgfscope}%
\begin{pgfscope}%
\definecolor{textcolor}{rgb}{0.000000,0.000000,0.000000}%
\pgfsetstrokecolor{textcolor}%
\pgfsetfillcolor{textcolor}%
\pgftext[x=11.411589in,y=7.039375in,left,base]{\color{textcolor}\sffamily\fontsize{18.000000}{21.600000}\selectfont $\displaystyle u$}%
\end{pgfscope}%
\end{pgfpicture}%
\makeatother%
\endgroup%
}\quad
	\resizebox{0.4\linewidth}{!}{\input{upwind_square_11.pgf}}
	\caption{迎风格式差分逼近解 $U$ 与真解 $u$}\label{fig:upwind_square_Uu}
\end{figure}

取 $\nu = 2$, 结果如图 \ref{fig:upwind_square_Uu_noCFL} 所示, 出现了错误解.

\begin{figure}[H]\centering\zihao{-5}
	\resizebox{0.4\linewidth}{!}{%% Creator: Matplotlib, PGF backend
%%
%% To include the figure in your LaTeX document, write
%%   \input{<filename>.pgf}
%%
%% Make sure the required packages are loaded in your preamble
%%   \usepackage{pgf}
%%
%% Figures using additional raster images can only be included by \input if
%% they are in the same directory as the main LaTeX file. For loading figures
%% from other directories you can use the `import` package
%%   \usepackage{import}
%%
%% and then include the figures with
%%   \import{<path to file>}{<filename>.pgf}
%%
%% Matplotlib used the following preamble
%%   \usepackage{fontspec}
%%   \setmainfont{DejaVuSerif.ttf}[Path=\detokenize{/Users/quejiahao/.julia/conda/3/lib/python3.9/site-packages/matplotlib/mpl-data/fonts/ttf/}]
%%   \setsansfont{DejaVuSans.ttf}[Path=\detokenize{/Users/quejiahao/.julia/conda/3/lib/python3.9/site-packages/matplotlib/mpl-data/fonts/ttf/}]
%%   \setmonofont{DejaVuSansMono.ttf}[Path=\detokenize{/Users/quejiahao/.julia/conda/3/lib/python3.9/site-packages/matplotlib/mpl-data/fonts/ttf/}]
%%
\begingroup%
\makeatletter%
\begin{pgfpicture}%
\pgfpathrectangle{\pgfpointorigin}{\pgfqpoint{12.000000in}{8.000000in}}%
\pgfusepath{use as bounding box, clip}%
\begin{pgfscope}%
\pgfsetbuttcap%
\pgfsetmiterjoin%
\definecolor{currentfill}{rgb}{1.000000,1.000000,1.000000}%
\pgfsetfillcolor{currentfill}%
\pgfsetlinewidth{0.000000pt}%
\definecolor{currentstroke}{rgb}{1.000000,1.000000,1.000000}%
\pgfsetstrokecolor{currentstroke}%
\pgfsetdash{}{0pt}%
\pgfpathmoveto{\pgfqpoint{0.000000in}{0.000000in}}%
\pgfpathlineto{\pgfqpoint{12.000000in}{0.000000in}}%
\pgfpathlineto{\pgfqpoint{12.000000in}{8.000000in}}%
\pgfpathlineto{\pgfqpoint{0.000000in}{8.000000in}}%
\pgfpathclose%
\pgfusepath{fill}%
\end{pgfscope}%
\begin{pgfscope}%
\pgfsetbuttcap%
\pgfsetmiterjoin%
\definecolor{currentfill}{rgb}{1.000000,1.000000,1.000000}%
\pgfsetfillcolor{currentfill}%
\pgfsetlinewidth{0.000000pt}%
\definecolor{currentstroke}{rgb}{0.000000,0.000000,0.000000}%
\pgfsetstrokecolor{currentstroke}%
\pgfsetstrokeopacity{0.000000}%
\pgfsetdash{}{0pt}%
\pgfpathmoveto{\pgfqpoint{2.680774in}{1.247073in}}%
\pgfpathlineto{\pgfqpoint{11.921260in}{1.247073in}}%
\pgfpathlineto{\pgfqpoint{11.921260in}{7.776785in}}%
\pgfpathlineto{\pgfqpoint{2.680774in}{7.776785in}}%
\pgfpathclose%
\pgfusepath{fill}%
\end{pgfscope}%
\begin{pgfscope}%
\pgfpathrectangle{\pgfqpoint{2.680774in}{1.247073in}}{\pgfqpoint{9.240485in}{6.529711in}}%
\pgfusepath{clip}%
\pgfsetrectcap%
\pgfsetroundjoin%
\pgfsetlinewidth{0.501875pt}%
\definecolor{currentstroke}{rgb}{0.000000,0.000000,0.000000}%
\pgfsetstrokecolor{currentstroke}%
\pgfsetstrokeopacity{0.100000}%
\pgfsetdash{}{0pt}%
\pgfpathmoveto{\pgfqpoint{2.942298in}{1.247073in}}%
\pgfpathlineto{\pgfqpoint{2.942298in}{7.776785in}}%
\pgfusepath{stroke}%
\end{pgfscope}%
\begin{pgfscope}%
\pgfsetbuttcap%
\pgfsetroundjoin%
\definecolor{currentfill}{rgb}{0.000000,0.000000,0.000000}%
\pgfsetfillcolor{currentfill}%
\pgfsetlinewidth{0.501875pt}%
\definecolor{currentstroke}{rgb}{0.000000,0.000000,0.000000}%
\pgfsetstrokecolor{currentstroke}%
\pgfsetdash{}{0pt}%
\pgfsys@defobject{currentmarker}{\pgfqpoint{0.000000in}{0.000000in}}{\pgfqpoint{0.000000in}{0.034722in}}{%
\pgfpathmoveto{\pgfqpoint{0.000000in}{0.000000in}}%
\pgfpathlineto{\pgfqpoint{0.000000in}{0.034722in}}%
\pgfusepath{stroke,fill}%
}%
\begin{pgfscope}%
\pgfsys@transformshift{2.942298in}{1.247073in}%
\pgfsys@useobject{currentmarker}{}%
\end{pgfscope}%
\end{pgfscope}%
\begin{pgfscope}%
\definecolor{textcolor}{rgb}{0.000000,0.000000,0.000000}%
\pgfsetstrokecolor{textcolor}%
\pgfsetfillcolor{textcolor}%
\pgftext[x=2.942298in,y=1.198462in,,top]{\color{textcolor}\sffamily\fontsize{18.000000}{21.600000}\selectfont $\displaystyle 0$}%
\end{pgfscope}%
\begin{pgfscope}%
\pgfpathrectangle{\pgfqpoint{2.680774in}{1.247073in}}{\pgfqpoint{9.240485in}{6.529711in}}%
\pgfusepath{clip}%
\pgfsetrectcap%
\pgfsetroundjoin%
\pgfsetlinewidth{0.501875pt}%
\definecolor{currentstroke}{rgb}{0.000000,0.000000,0.000000}%
\pgfsetstrokecolor{currentstroke}%
\pgfsetstrokeopacity{0.100000}%
\pgfsetdash{}{0pt}%
\pgfpathmoveto{\pgfqpoint{4.329721in}{1.247073in}}%
\pgfpathlineto{\pgfqpoint{4.329721in}{7.776785in}}%
\pgfusepath{stroke}%
\end{pgfscope}%
\begin{pgfscope}%
\pgfsetbuttcap%
\pgfsetroundjoin%
\definecolor{currentfill}{rgb}{0.000000,0.000000,0.000000}%
\pgfsetfillcolor{currentfill}%
\pgfsetlinewidth{0.501875pt}%
\definecolor{currentstroke}{rgb}{0.000000,0.000000,0.000000}%
\pgfsetstrokecolor{currentstroke}%
\pgfsetdash{}{0pt}%
\pgfsys@defobject{currentmarker}{\pgfqpoint{0.000000in}{0.000000in}}{\pgfqpoint{0.000000in}{0.034722in}}{%
\pgfpathmoveto{\pgfqpoint{0.000000in}{0.000000in}}%
\pgfpathlineto{\pgfqpoint{0.000000in}{0.034722in}}%
\pgfusepath{stroke,fill}%
}%
\begin{pgfscope}%
\pgfsys@transformshift{4.329721in}{1.247073in}%
\pgfsys@useobject{currentmarker}{}%
\end{pgfscope}%
\end{pgfscope}%
\begin{pgfscope}%
\definecolor{textcolor}{rgb}{0.000000,0.000000,0.000000}%
\pgfsetstrokecolor{textcolor}%
\pgfsetfillcolor{textcolor}%
\pgftext[x=4.329721in,y=1.198462in,,top]{\color{textcolor}\sffamily\fontsize{18.000000}{21.600000}\selectfont $\displaystyle 1$}%
\end{pgfscope}%
\begin{pgfscope}%
\pgfpathrectangle{\pgfqpoint{2.680774in}{1.247073in}}{\pgfqpoint{9.240485in}{6.529711in}}%
\pgfusepath{clip}%
\pgfsetrectcap%
\pgfsetroundjoin%
\pgfsetlinewidth{0.501875pt}%
\definecolor{currentstroke}{rgb}{0.000000,0.000000,0.000000}%
\pgfsetstrokecolor{currentstroke}%
\pgfsetstrokeopacity{0.100000}%
\pgfsetdash{}{0pt}%
\pgfpathmoveto{\pgfqpoint{5.717145in}{1.247073in}}%
\pgfpathlineto{\pgfqpoint{5.717145in}{7.776785in}}%
\pgfusepath{stroke}%
\end{pgfscope}%
\begin{pgfscope}%
\pgfsetbuttcap%
\pgfsetroundjoin%
\definecolor{currentfill}{rgb}{0.000000,0.000000,0.000000}%
\pgfsetfillcolor{currentfill}%
\pgfsetlinewidth{0.501875pt}%
\definecolor{currentstroke}{rgb}{0.000000,0.000000,0.000000}%
\pgfsetstrokecolor{currentstroke}%
\pgfsetdash{}{0pt}%
\pgfsys@defobject{currentmarker}{\pgfqpoint{0.000000in}{0.000000in}}{\pgfqpoint{0.000000in}{0.034722in}}{%
\pgfpathmoveto{\pgfqpoint{0.000000in}{0.000000in}}%
\pgfpathlineto{\pgfqpoint{0.000000in}{0.034722in}}%
\pgfusepath{stroke,fill}%
}%
\begin{pgfscope}%
\pgfsys@transformshift{5.717145in}{1.247073in}%
\pgfsys@useobject{currentmarker}{}%
\end{pgfscope}%
\end{pgfscope}%
\begin{pgfscope}%
\definecolor{textcolor}{rgb}{0.000000,0.000000,0.000000}%
\pgfsetstrokecolor{textcolor}%
\pgfsetfillcolor{textcolor}%
\pgftext[x=5.717145in,y=1.198462in,,top]{\color{textcolor}\sffamily\fontsize{18.000000}{21.600000}\selectfont $\displaystyle 2$}%
\end{pgfscope}%
\begin{pgfscope}%
\pgfpathrectangle{\pgfqpoint{2.680774in}{1.247073in}}{\pgfqpoint{9.240485in}{6.529711in}}%
\pgfusepath{clip}%
\pgfsetrectcap%
\pgfsetroundjoin%
\pgfsetlinewidth{0.501875pt}%
\definecolor{currentstroke}{rgb}{0.000000,0.000000,0.000000}%
\pgfsetstrokecolor{currentstroke}%
\pgfsetstrokeopacity{0.100000}%
\pgfsetdash{}{0pt}%
\pgfpathmoveto{\pgfqpoint{7.104568in}{1.247073in}}%
\pgfpathlineto{\pgfqpoint{7.104568in}{7.776785in}}%
\pgfusepath{stroke}%
\end{pgfscope}%
\begin{pgfscope}%
\pgfsetbuttcap%
\pgfsetroundjoin%
\definecolor{currentfill}{rgb}{0.000000,0.000000,0.000000}%
\pgfsetfillcolor{currentfill}%
\pgfsetlinewidth{0.501875pt}%
\definecolor{currentstroke}{rgb}{0.000000,0.000000,0.000000}%
\pgfsetstrokecolor{currentstroke}%
\pgfsetdash{}{0pt}%
\pgfsys@defobject{currentmarker}{\pgfqpoint{0.000000in}{0.000000in}}{\pgfqpoint{0.000000in}{0.034722in}}{%
\pgfpathmoveto{\pgfqpoint{0.000000in}{0.000000in}}%
\pgfpathlineto{\pgfqpoint{0.000000in}{0.034722in}}%
\pgfusepath{stroke,fill}%
}%
\begin{pgfscope}%
\pgfsys@transformshift{7.104568in}{1.247073in}%
\pgfsys@useobject{currentmarker}{}%
\end{pgfscope}%
\end{pgfscope}%
\begin{pgfscope}%
\definecolor{textcolor}{rgb}{0.000000,0.000000,0.000000}%
\pgfsetstrokecolor{textcolor}%
\pgfsetfillcolor{textcolor}%
\pgftext[x=7.104568in,y=1.198462in,,top]{\color{textcolor}\sffamily\fontsize{18.000000}{21.600000}\selectfont $\displaystyle 3$}%
\end{pgfscope}%
\begin{pgfscope}%
\pgfpathrectangle{\pgfqpoint{2.680774in}{1.247073in}}{\pgfqpoint{9.240485in}{6.529711in}}%
\pgfusepath{clip}%
\pgfsetrectcap%
\pgfsetroundjoin%
\pgfsetlinewidth{0.501875pt}%
\definecolor{currentstroke}{rgb}{0.000000,0.000000,0.000000}%
\pgfsetstrokecolor{currentstroke}%
\pgfsetstrokeopacity{0.100000}%
\pgfsetdash{}{0pt}%
\pgfpathmoveto{\pgfqpoint{8.491992in}{1.247073in}}%
\pgfpathlineto{\pgfqpoint{8.491992in}{7.776785in}}%
\pgfusepath{stroke}%
\end{pgfscope}%
\begin{pgfscope}%
\pgfsetbuttcap%
\pgfsetroundjoin%
\definecolor{currentfill}{rgb}{0.000000,0.000000,0.000000}%
\pgfsetfillcolor{currentfill}%
\pgfsetlinewidth{0.501875pt}%
\definecolor{currentstroke}{rgb}{0.000000,0.000000,0.000000}%
\pgfsetstrokecolor{currentstroke}%
\pgfsetdash{}{0pt}%
\pgfsys@defobject{currentmarker}{\pgfqpoint{0.000000in}{0.000000in}}{\pgfqpoint{0.000000in}{0.034722in}}{%
\pgfpathmoveto{\pgfqpoint{0.000000in}{0.000000in}}%
\pgfpathlineto{\pgfqpoint{0.000000in}{0.034722in}}%
\pgfusepath{stroke,fill}%
}%
\begin{pgfscope}%
\pgfsys@transformshift{8.491992in}{1.247073in}%
\pgfsys@useobject{currentmarker}{}%
\end{pgfscope}%
\end{pgfscope}%
\begin{pgfscope}%
\definecolor{textcolor}{rgb}{0.000000,0.000000,0.000000}%
\pgfsetstrokecolor{textcolor}%
\pgfsetfillcolor{textcolor}%
\pgftext[x=8.491992in,y=1.198462in,,top]{\color{textcolor}\sffamily\fontsize{18.000000}{21.600000}\selectfont $\displaystyle 4$}%
\end{pgfscope}%
\begin{pgfscope}%
\pgfpathrectangle{\pgfqpoint{2.680774in}{1.247073in}}{\pgfqpoint{9.240485in}{6.529711in}}%
\pgfusepath{clip}%
\pgfsetrectcap%
\pgfsetroundjoin%
\pgfsetlinewidth{0.501875pt}%
\definecolor{currentstroke}{rgb}{0.000000,0.000000,0.000000}%
\pgfsetstrokecolor{currentstroke}%
\pgfsetstrokeopacity{0.100000}%
\pgfsetdash{}{0pt}%
\pgfpathmoveto{\pgfqpoint{9.879415in}{1.247073in}}%
\pgfpathlineto{\pgfqpoint{9.879415in}{7.776785in}}%
\pgfusepath{stroke}%
\end{pgfscope}%
\begin{pgfscope}%
\pgfsetbuttcap%
\pgfsetroundjoin%
\definecolor{currentfill}{rgb}{0.000000,0.000000,0.000000}%
\pgfsetfillcolor{currentfill}%
\pgfsetlinewidth{0.501875pt}%
\definecolor{currentstroke}{rgb}{0.000000,0.000000,0.000000}%
\pgfsetstrokecolor{currentstroke}%
\pgfsetdash{}{0pt}%
\pgfsys@defobject{currentmarker}{\pgfqpoint{0.000000in}{0.000000in}}{\pgfqpoint{0.000000in}{0.034722in}}{%
\pgfpathmoveto{\pgfqpoint{0.000000in}{0.000000in}}%
\pgfpathlineto{\pgfqpoint{0.000000in}{0.034722in}}%
\pgfusepath{stroke,fill}%
}%
\begin{pgfscope}%
\pgfsys@transformshift{9.879415in}{1.247073in}%
\pgfsys@useobject{currentmarker}{}%
\end{pgfscope}%
\end{pgfscope}%
\begin{pgfscope}%
\definecolor{textcolor}{rgb}{0.000000,0.000000,0.000000}%
\pgfsetstrokecolor{textcolor}%
\pgfsetfillcolor{textcolor}%
\pgftext[x=9.879415in,y=1.198462in,,top]{\color{textcolor}\sffamily\fontsize{18.000000}{21.600000}\selectfont $\displaystyle 5$}%
\end{pgfscope}%
\begin{pgfscope}%
\pgfpathrectangle{\pgfqpoint{2.680774in}{1.247073in}}{\pgfqpoint{9.240485in}{6.529711in}}%
\pgfusepath{clip}%
\pgfsetrectcap%
\pgfsetroundjoin%
\pgfsetlinewidth{0.501875pt}%
\definecolor{currentstroke}{rgb}{0.000000,0.000000,0.000000}%
\pgfsetstrokecolor{currentstroke}%
\pgfsetstrokeopacity{0.100000}%
\pgfsetdash{}{0pt}%
\pgfpathmoveto{\pgfqpoint{11.266839in}{1.247073in}}%
\pgfpathlineto{\pgfqpoint{11.266839in}{7.776785in}}%
\pgfusepath{stroke}%
\end{pgfscope}%
\begin{pgfscope}%
\pgfsetbuttcap%
\pgfsetroundjoin%
\definecolor{currentfill}{rgb}{0.000000,0.000000,0.000000}%
\pgfsetfillcolor{currentfill}%
\pgfsetlinewidth{0.501875pt}%
\definecolor{currentstroke}{rgb}{0.000000,0.000000,0.000000}%
\pgfsetstrokecolor{currentstroke}%
\pgfsetdash{}{0pt}%
\pgfsys@defobject{currentmarker}{\pgfqpoint{0.000000in}{0.000000in}}{\pgfqpoint{0.000000in}{0.034722in}}{%
\pgfpathmoveto{\pgfqpoint{0.000000in}{0.000000in}}%
\pgfpathlineto{\pgfqpoint{0.000000in}{0.034722in}}%
\pgfusepath{stroke,fill}%
}%
\begin{pgfscope}%
\pgfsys@transformshift{11.266839in}{1.247073in}%
\pgfsys@useobject{currentmarker}{}%
\end{pgfscope}%
\end{pgfscope}%
\begin{pgfscope}%
\definecolor{textcolor}{rgb}{0.000000,0.000000,0.000000}%
\pgfsetstrokecolor{textcolor}%
\pgfsetfillcolor{textcolor}%
\pgftext[x=11.266839in,y=1.198462in,,top]{\color{textcolor}\sffamily\fontsize{18.000000}{21.600000}\selectfont $\displaystyle 6$}%
\end{pgfscope}%
\begin{pgfscope}%
\definecolor{textcolor}{rgb}{0.000000,0.000000,0.000000}%
\pgfsetstrokecolor{textcolor}%
\pgfsetfillcolor{textcolor}%
\pgftext[x=7.301017in,y=0.900964in,,top]{\color{textcolor}\sffamily\fontsize{18.000000}{21.600000}\selectfont $\displaystyle x$}%
\end{pgfscope}%
\begin{pgfscope}%
\pgfpathrectangle{\pgfqpoint{2.680774in}{1.247073in}}{\pgfqpoint{9.240485in}{6.529711in}}%
\pgfusepath{clip}%
\pgfsetrectcap%
\pgfsetroundjoin%
\pgfsetlinewidth{0.501875pt}%
\definecolor{currentstroke}{rgb}{0.000000,0.000000,0.000000}%
\pgfsetstrokecolor{currentstroke}%
\pgfsetstrokeopacity{0.100000}%
\pgfsetdash{}{0pt}%
\pgfpathmoveto{\pgfqpoint{2.680774in}{2.264348in}}%
\pgfpathlineto{\pgfqpoint{11.921260in}{2.264348in}}%
\pgfusepath{stroke}%
\end{pgfscope}%
\begin{pgfscope}%
\pgfsetbuttcap%
\pgfsetroundjoin%
\definecolor{currentfill}{rgb}{0.000000,0.000000,0.000000}%
\pgfsetfillcolor{currentfill}%
\pgfsetlinewidth{0.501875pt}%
\definecolor{currentstroke}{rgb}{0.000000,0.000000,0.000000}%
\pgfsetstrokecolor{currentstroke}%
\pgfsetdash{}{0pt}%
\pgfsys@defobject{currentmarker}{\pgfqpoint{0.000000in}{0.000000in}}{\pgfqpoint{0.034722in}{0.000000in}}{%
\pgfpathmoveto{\pgfqpoint{0.000000in}{0.000000in}}%
\pgfpathlineto{\pgfqpoint{0.034722in}{0.000000in}}%
\pgfusepath{stroke,fill}%
}%
\begin{pgfscope}%
\pgfsys@transformshift{2.680774in}{2.264348in}%
\pgfsys@useobject{currentmarker}{}%
\end{pgfscope}%
\end{pgfscope}%
\begin{pgfscope}%
\definecolor{textcolor}{rgb}{0.000000,0.000000,0.000000}%
\pgfsetstrokecolor{textcolor}%
\pgfsetfillcolor{textcolor}%
\pgftext[x=1.641933in, y=2.169378in, left, base]{\color{textcolor}\sffamily\fontsize{18.000000}{21.600000}\selectfont $\displaystyle -2.0×10^{5}$}%
\end{pgfscope}%
\begin{pgfscope}%
\pgfpathrectangle{\pgfqpoint{2.680774in}{1.247073in}}{\pgfqpoint{9.240485in}{6.529711in}}%
\pgfusepath{clip}%
\pgfsetrectcap%
\pgfsetroundjoin%
\pgfsetlinewidth{0.501875pt}%
\definecolor{currentstroke}{rgb}{0.000000,0.000000,0.000000}%
\pgfsetstrokecolor{currentstroke}%
\pgfsetstrokeopacity{0.100000}%
\pgfsetdash{}{0pt}%
\pgfpathmoveto{\pgfqpoint{2.680774in}{3.360508in}}%
\pgfpathlineto{\pgfqpoint{11.921260in}{3.360508in}}%
\pgfusepath{stroke}%
\end{pgfscope}%
\begin{pgfscope}%
\pgfsetbuttcap%
\pgfsetroundjoin%
\definecolor{currentfill}{rgb}{0.000000,0.000000,0.000000}%
\pgfsetfillcolor{currentfill}%
\pgfsetlinewidth{0.501875pt}%
\definecolor{currentstroke}{rgb}{0.000000,0.000000,0.000000}%
\pgfsetstrokecolor{currentstroke}%
\pgfsetdash{}{0pt}%
\pgfsys@defobject{currentmarker}{\pgfqpoint{0.000000in}{0.000000in}}{\pgfqpoint{0.034722in}{0.000000in}}{%
\pgfpathmoveto{\pgfqpoint{0.000000in}{0.000000in}}%
\pgfpathlineto{\pgfqpoint{0.034722in}{0.000000in}}%
\pgfusepath{stroke,fill}%
}%
\begin{pgfscope}%
\pgfsys@transformshift{2.680774in}{3.360508in}%
\pgfsys@useobject{currentmarker}{}%
\end{pgfscope}%
\end{pgfscope}%
\begin{pgfscope}%
\definecolor{textcolor}{rgb}{0.000000,0.000000,0.000000}%
\pgfsetstrokecolor{textcolor}%
\pgfsetfillcolor{textcolor}%
\pgftext[x=1.641933in, y=3.265537in, left, base]{\color{textcolor}\sffamily\fontsize{18.000000}{21.600000}\selectfont $\displaystyle -1.0×10^{5}$}%
\end{pgfscope}%
\begin{pgfscope}%
\pgfpathrectangle{\pgfqpoint{2.680774in}{1.247073in}}{\pgfqpoint{9.240485in}{6.529711in}}%
\pgfusepath{clip}%
\pgfsetrectcap%
\pgfsetroundjoin%
\pgfsetlinewidth{0.501875pt}%
\definecolor{currentstroke}{rgb}{0.000000,0.000000,0.000000}%
\pgfsetstrokecolor{currentstroke}%
\pgfsetstrokeopacity{0.100000}%
\pgfsetdash{}{0pt}%
\pgfpathmoveto{\pgfqpoint{2.680774in}{4.456668in}}%
\pgfpathlineto{\pgfqpoint{11.921260in}{4.456668in}}%
\pgfusepath{stroke}%
\end{pgfscope}%
\begin{pgfscope}%
\pgfsetbuttcap%
\pgfsetroundjoin%
\definecolor{currentfill}{rgb}{0.000000,0.000000,0.000000}%
\pgfsetfillcolor{currentfill}%
\pgfsetlinewidth{0.501875pt}%
\definecolor{currentstroke}{rgb}{0.000000,0.000000,0.000000}%
\pgfsetstrokecolor{currentstroke}%
\pgfsetdash{}{0pt}%
\pgfsys@defobject{currentmarker}{\pgfqpoint{0.000000in}{0.000000in}}{\pgfqpoint{0.034722in}{0.000000in}}{%
\pgfpathmoveto{\pgfqpoint{0.000000in}{0.000000in}}%
\pgfpathlineto{\pgfqpoint{0.034722in}{0.000000in}}%
\pgfusepath{stroke,fill}%
}%
\begin{pgfscope}%
\pgfsys@transformshift{2.680774in}{4.456668in}%
\pgfsys@useobject{currentmarker}{}%
\end{pgfscope}%
\end{pgfscope}%
\begin{pgfscope}%
\definecolor{textcolor}{rgb}{0.000000,0.000000,0.000000}%
\pgfsetstrokecolor{textcolor}%
\pgfsetfillcolor{textcolor}%
\pgftext[x=2.522095in, y=4.361697in, left, base]{\color{textcolor}\sffamily\fontsize{18.000000}{21.600000}\selectfont $\displaystyle 0$}%
\end{pgfscope}%
\begin{pgfscope}%
\pgfpathrectangle{\pgfqpoint{2.680774in}{1.247073in}}{\pgfqpoint{9.240485in}{6.529711in}}%
\pgfusepath{clip}%
\pgfsetrectcap%
\pgfsetroundjoin%
\pgfsetlinewidth{0.501875pt}%
\definecolor{currentstroke}{rgb}{0.000000,0.000000,0.000000}%
\pgfsetstrokecolor{currentstroke}%
\pgfsetstrokeopacity{0.100000}%
\pgfsetdash{}{0pt}%
\pgfpathmoveto{\pgfqpoint{2.680774in}{5.552827in}}%
\pgfpathlineto{\pgfqpoint{11.921260in}{5.552827in}}%
\pgfusepath{stroke}%
\end{pgfscope}%
\begin{pgfscope}%
\pgfsetbuttcap%
\pgfsetroundjoin%
\definecolor{currentfill}{rgb}{0.000000,0.000000,0.000000}%
\pgfsetfillcolor{currentfill}%
\pgfsetlinewidth{0.501875pt}%
\definecolor{currentstroke}{rgb}{0.000000,0.000000,0.000000}%
\pgfsetstrokecolor{currentstroke}%
\pgfsetdash{}{0pt}%
\pgfsys@defobject{currentmarker}{\pgfqpoint{0.000000in}{0.000000in}}{\pgfqpoint{0.034722in}{0.000000in}}{%
\pgfpathmoveto{\pgfqpoint{0.000000in}{0.000000in}}%
\pgfpathlineto{\pgfqpoint{0.034722in}{0.000000in}}%
\pgfusepath{stroke,fill}%
}%
\begin{pgfscope}%
\pgfsys@transformshift{2.680774in}{5.552827in}%
\pgfsys@useobject{currentmarker}{}%
\end{pgfscope}%
\end{pgfscope}%
\begin{pgfscope}%
\definecolor{textcolor}{rgb}{0.000000,0.000000,0.000000}%
\pgfsetstrokecolor{textcolor}%
\pgfsetfillcolor{textcolor}%
\pgftext[x=1.828600in, y=5.457856in, left, base]{\color{textcolor}\sffamily\fontsize{18.000000}{21.600000}\selectfont $\displaystyle 1.0×10^{5}$}%
\end{pgfscope}%
\begin{pgfscope}%
\pgfpathrectangle{\pgfqpoint{2.680774in}{1.247073in}}{\pgfqpoint{9.240485in}{6.529711in}}%
\pgfusepath{clip}%
\pgfsetrectcap%
\pgfsetroundjoin%
\pgfsetlinewidth{0.501875pt}%
\definecolor{currentstroke}{rgb}{0.000000,0.000000,0.000000}%
\pgfsetstrokecolor{currentstroke}%
\pgfsetstrokeopacity{0.100000}%
\pgfsetdash{}{0pt}%
\pgfpathmoveto{\pgfqpoint{2.680774in}{6.648987in}}%
\pgfpathlineto{\pgfqpoint{11.921260in}{6.648987in}}%
\pgfusepath{stroke}%
\end{pgfscope}%
\begin{pgfscope}%
\pgfsetbuttcap%
\pgfsetroundjoin%
\definecolor{currentfill}{rgb}{0.000000,0.000000,0.000000}%
\pgfsetfillcolor{currentfill}%
\pgfsetlinewidth{0.501875pt}%
\definecolor{currentstroke}{rgb}{0.000000,0.000000,0.000000}%
\pgfsetstrokecolor{currentstroke}%
\pgfsetdash{}{0pt}%
\pgfsys@defobject{currentmarker}{\pgfqpoint{0.000000in}{0.000000in}}{\pgfqpoint{0.034722in}{0.000000in}}{%
\pgfpathmoveto{\pgfqpoint{0.000000in}{0.000000in}}%
\pgfpathlineto{\pgfqpoint{0.034722in}{0.000000in}}%
\pgfusepath{stroke,fill}%
}%
\begin{pgfscope}%
\pgfsys@transformshift{2.680774in}{6.648987in}%
\pgfsys@useobject{currentmarker}{}%
\end{pgfscope}%
\end{pgfscope}%
\begin{pgfscope}%
\definecolor{textcolor}{rgb}{0.000000,0.000000,0.000000}%
\pgfsetstrokecolor{textcolor}%
\pgfsetfillcolor{textcolor}%
\pgftext[x=1.828600in, y=6.554016in, left, base]{\color{textcolor}\sffamily\fontsize{18.000000}{21.600000}\selectfont $\displaystyle 2.0×10^{5}$}%
\end{pgfscope}%
\begin{pgfscope}%
\pgfpathrectangle{\pgfqpoint{2.680774in}{1.247073in}}{\pgfqpoint{9.240485in}{6.529711in}}%
\pgfusepath{clip}%
\pgfsetrectcap%
\pgfsetroundjoin%
\pgfsetlinewidth{0.501875pt}%
\definecolor{currentstroke}{rgb}{0.000000,0.000000,0.000000}%
\pgfsetstrokecolor{currentstroke}%
\pgfsetstrokeopacity{0.100000}%
\pgfsetdash{}{0pt}%
\pgfpathmoveto{\pgfqpoint{2.680774in}{7.745146in}}%
\pgfpathlineto{\pgfqpoint{11.921260in}{7.745146in}}%
\pgfusepath{stroke}%
\end{pgfscope}%
\begin{pgfscope}%
\pgfsetbuttcap%
\pgfsetroundjoin%
\definecolor{currentfill}{rgb}{0.000000,0.000000,0.000000}%
\pgfsetfillcolor{currentfill}%
\pgfsetlinewidth{0.501875pt}%
\definecolor{currentstroke}{rgb}{0.000000,0.000000,0.000000}%
\pgfsetstrokecolor{currentstroke}%
\pgfsetdash{}{0pt}%
\pgfsys@defobject{currentmarker}{\pgfqpoint{0.000000in}{0.000000in}}{\pgfqpoint{0.034722in}{0.000000in}}{%
\pgfpathmoveto{\pgfqpoint{0.000000in}{0.000000in}}%
\pgfpathlineto{\pgfqpoint{0.034722in}{0.000000in}}%
\pgfusepath{stroke,fill}%
}%
\begin{pgfscope}%
\pgfsys@transformshift{2.680774in}{7.745146in}%
\pgfsys@useobject{currentmarker}{}%
\end{pgfscope}%
\end{pgfscope}%
\begin{pgfscope}%
\definecolor{textcolor}{rgb}{0.000000,0.000000,0.000000}%
\pgfsetstrokecolor{textcolor}%
\pgfsetfillcolor{textcolor}%
\pgftext[x=1.828600in, y=7.650176in, left, base]{\color{textcolor}\sffamily\fontsize{18.000000}{21.600000}\selectfont $\displaystyle 3.0×10^{5}$}%
\end{pgfscope}%
\begin{pgfscope}%
\pgfpathrectangle{\pgfqpoint{2.680774in}{1.247073in}}{\pgfqpoint{9.240485in}{6.529711in}}%
\pgfusepath{clip}%
\pgfsetbuttcap%
\pgfsetroundjoin%
\pgfsetlinewidth{1.003750pt}%
\definecolor{currentstroke}{rgb}{0.000000,0.605603,0.978680}%
\pgfsetstrokecolor{currentstroke}%
\pgfsetdash{}{0pt}%
\pgfpathmoveto{\pgfqpoint{2.942298in}{4.456700in}}%
\pgfpathlineto{\pgfqpoint{5.938917in}{4.456329in}}%
\pgfpathlineto{\pgfqpoint{6.007022in}{4.461031in}}%
\pgfpathlineto{\pgfqpoint{6.075127in}{4.426002in}}%
\pgfpathlineto{\pgfqpoint{6.143232in}{4.603929in}}%
\pgfpathlineto{\pgfqpoint{6.211337in}{3.953226in}}%
\pgfpathlineto{\pgfqpoint{6.279442in}{5.715977in}}%
\pgfpathlineto{\pgfqpoint{6.347547in}{2.134514in}}%
\pgfpathlineto{\pgfqpoint{6.415652in}{7.591982in}}%
\pgfpathlineto{\pgfqpoint{6.483757in}{1.431877in}}%
\pgfpathlineto{\pgfqpoint{6.551862in}{6.438184in}}%
\pgfpathlineto{\pgfqpoint{6.619967in}{3.664126in}}%
\pgfpathlineto{\pgfqpoint{6.688072in}{4.603490in}}%
\pgfpathlineto{\pgfqpoint{6.756177in}{4.456679in}}%
\pgfpathlineto{\pgfqpoint{11.659737in}{4.456679in}}%
\pgfpathlineto{\pgfqpoint{11.659737in}{4.456679in}}%
\pgfusepath{stroke}%
\end{pgfscope}%
\begin{pgfscope}%
\pgfpathrectangle{\pgfqpoint{2.680774in}{1.247073in}}{\pgfqpoint{9.240485in}{6.529711in}}%
\pgfusepath{clip}%
\pgfsetbuttcap%
\pgfsetroundjoin%
\pgfsetlinewidth{1.003750pt}%
\definecolor{currentstroke}{rgb}{0.888874,0.435649,0.278123}%
\pgfsetstrokecolor{currentstroke}%
\pgfsetdash{}{0pt}%
\pgfpathmoveto{\pgfqpoint{2.942298in}{4.456700in}}%
\pgfpathlineto{\pgfqpoint{11.659737in}{4.456679in}}%
\pgfpathlineto{\pgfqpoint{11.659737in}{4.456679in}}%
\pgfusepath{stroke}%
\end{pgfscope}%
\begin{pgfscope}%
\pgfsetrectcap%
\pgfsetmiterjoin%
\pgfsetlinewidth{1.003750pt}%
\definecolor{currentstroke}{rgb}{0.000000,0.000000,0.000000}%
\pgfsetstrokecolor{currentstroke}%
\pgfsetdash{}{0pt}%
\pgfpathmoveto{\pgfqpoint{2.680774in}{1.247073in}}%
\pgfpathlineto{\pgfqpoint{2.680774in}{7.776785in}}%
\pgfusepath{stroke}%
\end{pgfscope}%
\begin{pgfscope}%
\pgfsetrectcap%
\pgfsetmiterjoin%
\pgfsetlinewidth{1.003750pt}%
\definecolor{currentstroke}{rgb}{0.000000,0.000000,0.000000}%
\pgfsetstrokecolor{currentstroke}%
\pgfsetdash{}{0pt}%
\pgfpathmoveto{\pgfqpoint{2.680774in}{1.247073in}}%
\pgfpathlineto{\pgfqpoint{11.921260in}{1.247073in}}%
\pgfusepath{stroke}%
\end{pgfscope}%
\begin{pgfscope}%
\pgfsetbuttcap%
\pgfsetmiterjoin%
\definecolor{currentfill}{rgb}{1.000000,1.000000,1.000000}%
\pgfsetfillcolor{currentfill}%
\pgfsetlinewidth{1.003750pt}%
\definecolor{currentstroke}{rgb}{0.000000,0.000000,0.000000}%
\pgfsetstrokecolor{currentstroke}%
\pgfsetdash{}{0pt}%
\pgfpathmoveto{\pgfqpoint{10.511589in}{6.642899in}}%
\pgfpathlineto{\pgfqpoint{11.796260in}{6.642899in}}%
\pgfpathlineto{\pgfqpoint{11.796260in}{7.651785in}}%
\pgfpathlineto{\pgfqpoint{10.511589in}{7.651785in}}%
\pgfpathclose%
\pgfusepath{stroke,fill}%
\end{pgfscope}%
\begin{pgfscope}%
\pgfsetbuttcap%
\pgfsetmiterjoin%
\pgfsetlinewidth{2.258437pt}%
\definecolor{currentstroke}{rgb}{0.000000,0.605603,0.978680}%
\pgfsetstrokecolor{currentstroke}%
\pgfsetdash{}{0pt}%
\pgfpathmoveto{\pgfqpoint{10.711589in}{7.349344in}}%
\pgfpathlineto{\pgfqpoint{11.211589in}{7.349344in}}%
\pgfusepath{stroke}%
\end{pgfscope}%
\begin{pgfscope}%
\definecolor{textcolor}{rgb}{0.000000,0.000000,0.000000}%
\pgfsetstrokecolor{textcolor}%
\pgfsetfillcolor{textcolor}%
\pgftext[x=11.411589in,y=7.261844in,left,base]{\color{textcolor}\sffamily\fontsize{18.000000}{21.600000}\selectfont $\displaystyle U$}%
\end{pgfscope}%
\begin{pgfscope}%
\pgfsetbuttcap%
\pgfsetmiterjoin%
\pgfsetlinewidth{2.258437pt}%
\definecolor{currentstroke}{rgb}{0.888874,0.435649,0.278123}%
\pgfsetstrokecolor{currentstroke}%
\pgfsetdash{}{0pt}%
\pgfpathmoveto{\pgfqpoint{10.711589in}{6.982400in}}%
\pgfpathlineto{\pgfqpoint{11.211589in}{6.982400in}}%
\pgfusepath{stroke}%
\end{pgfscope}%
\begin{pgfscope}%
\definecolor{textcolor}{rgb}{0.000000,0.000000,0.000000}%
\pgfsetstrokecolor{textcolor}%
\pgfsetfillcolor{textcolor}%
\pgftext[x=11.411589in,y=6.894900in,left,base]{\color{textcolor}\sffamily\fontsize{18.000000}{21.600000}\selectfont $\displaystyle u$}%
\end{pgfscope}%
\end{pgfpicture}%
\makeatother%
\endgroup%
}\quad
	\resizebox{0.4\linewidth}{!}{%% Creator: Matplotlib, PGF backend
%%
%% To include the figure in your LaTeX document, write
%%   \input{<filename>.pgf}
%%
%% Make sure the required packages are loaded in your preamble
%%   \usepackage{pgf}
%%
%% Figures using additional raster images can only be included by \input if
%% they are in the same directory as the main LaTeX file. For loading figures
%% from other directories you can use the `import` package
%%   \usepackage{import}
%%
%% and then include the figures with
%%   \import{<path to file>}{<filename>.pgf}
%%
%% Matplotlib used the following preamble
%%   \usepackage{fontspec}
%%   \setmainfont{DejaVuSerif.ttf}[Path=\detokenize{/Users/quejiahao/.julia/conda/3/lib/python3.9/site-packages/matplotlib/mpl-data/fonts/ttf/}]
%%   \setsansfont{DejaVuSans.ttf}[Path=\detokenize{/Users/quejiahao/.julia/conda/3/lib/python3.9/site-packages/matplotlib/mpl-data/fonts/ttf/}]
%%   \setmonofont{DejaVuSansMono.ttf}[Path=\detokenize{/Users/quejiahao/.julia/conda/3/lib/python3.9/site-packages/matplotlib/mpl-data/fonts/ttf/}]
%%
\begingroup%
\makeatletter%
\begin{pgfpicture}%
\pgfpathrectangle{\pgfpointorigin}{\pgfqpoint{12.000000in}{8.000000in}}%
\pgfusepath{use as bounding box, clip}%
\begin{pgfscope}%
\pgfsetbuttcap%
\pgfsetmiterjoin%
\definecolor{currentfill}{rgb}{1.000000,1.000000,1.000000}%
\pgfsetfillcolor{currentfill}%
\pgfsetlinewidth{0.000000pt}%
\definecolor{currentstroke}{rgb}{1.000000,1.000000,1.000000}%
\pgfsetstrokecolor{currentstroke}%
\pgfsetdash{}{0pt}%
\pgfpathmoveto{\pgfqpoint{0.000000in}{0.000000in}}%
\pgfpathlineto{\pgfqpoint{12.000000in}{0.000000in}}%
\pgfpathlineto{\pgfqpoint{12.000000in}{8.000000in}}%
\pgfpathlineto{\pgfqpoint{0.000000in}{8.000000in}}%
\pgfpathclose%
\pgfusepath{fill}%
\end{pgfscope}%
\begin{pgfscope}%
\pgfsetbuttcap%
\pgfsetmiterjoin%
\definecolor{currentfill}{rgb}{1.000000,1.000000,1.000000}%
\pgfsetfillcolor{currentfill}%
\pgfsetlinewidth{0.000000pt}%
\definecolor{currentstroke}{rgb}{0.000000,0.000000,0.000000}%
\pgfsetstrokecolor{currentstroke}%
\pgfsetstrokeopacity{0.000000}%
\pgfsetdash{}{0pt}%
\pgfpathmoveto{\pgfqpoint{2.913455in}{1.247073in}}%
\pgfpathlineto{\pgfqpoint{11.921260in}{1.247073in}}%
\pgfpathlineto{\pgfqpoint{11.921260in}{7.921260in}}%
\pgfpathlineto{\pgfqpoint{2.913455in}{7.921260in}}%
\pgfpathclose%
\pgfusepath{fill}%
\end{pgfscope}%
\begin{pgfscope}%
\pgfpathrectangle{\pgfqpoint{2.913455in}{1.247073in}}{\pgfqpoint{9.007805in}{6.674186in}}%
\pgfusepath{clip}%
\pgfsetrectcap%
\pgfsetroundjoin%
\pgfsetlinewidth{0.501875pt}%
\definecolor{currentstroke}{rgb}{0.000000,0.000000,0.000000}%
\pgfsetstrokecolor{currentstroke}%
\pgfsetstrokeopacity{0.100000}%
\pgfsetdash{}{0pt}%
\pgfpathmoveto{\pgfqpoint{3.168393in}{1.247073in}}%
\pgfpathlineto{\pgfqpoint{3.168393in}{7.921260in}}%
\pgfusepath{stroke}%
\end{pgfscope}%
\begin{pgfscope}%
\pgfsetbuttcap%
\pgfsetroundjoin%
\definecolor{currentfill}{rgb}{0.000000,0.000000,0.000000}%
\pgfsetfillcolor{currentfill}%
\pgfsetlinewidth{0.501875pt}%
\definecolor{currentstroke}{rgb}{0.000000,0.000000,0.000000}%
\pgfsetstrokecolor{currentstroke}%
\pgfsetdash{}{0pt}%
\pgfsys@defobject{currentmarker}{\pgfqpoint{0.000000in}{0.000000in}}{\pgfqpoint{0.000000in}{0.034722in}}{%
\pgfpathmoveto{\pgfqpoint{0.000000in}{0.000000in}}%
\pgfpathlineto{\pgfqpoint{0.000000in}{0.034722in}}%
\pgfusepath{stroke,fill}%
}%
\begin{pgfscope}%
\pgfsys@transformshift{3.168393in}{1.247073in}%
\pgfsys@useobject{currentmarker}{}%
\end{pgfscope}%
\end{pgfscope}%
\begin{pgfscope}%
\definecolor{textcolor}{rgb}{0.000000,0.000000,0.000000}%
\pgfsetstrokecolor{textcolor}%
\pgfsetfillcolor{textcolor}%
\pgftext[x=3.168393in,y=1.198462in,,top]{\color{textcolor}\sffamily\fontsize{18.000000}{21.600000}\selectfont $\displaystyle 0$}%
\end{pgfscope}%
\begin{pgfscope}%
\pgfpathrectangle{\pgfqpoint{2.913455in}{1.247073in}}{\pgfqpoint{9.007805in}{6.674186in}}%
\pgfusepath{clip}%
\pgfsetrectcap%
\pgfsetroundjoin%
\pgfsetlinewidth{0.501875pt}%
\definecolor{currentstroke}{rgb}{0.000000,0.000000,0.000000}%
\pgfsetstrokecolor{currentstroke}%
\pgfsetstrokeopacity{0.100000}%
\pgfsetdash{}{0pt}%
\pgfpathmoveto{\pgfqpoint{4.520880in}{1.247073in}}%
\pgfpathlineto{\pgfqpoint{4.520880in}{7.921260in}}%
\pgfusepath{stroke}%
\end{pgfscope}%
\begin{pgfscope}%
\pgfsetbuttcap%
\pgfsetroundjoin%
\definecolor{currentfill}{rgb}{0.000000,0.000000,0.000000}%
\pgfsetfillcolor{currentfill}%
\pgfsetlinewidth{0.501875pt}%
\definecolor{currentstroke}{rgb}{0.000000,0.000000,0.000000}%
\pgfsetstrokecolor{currentstroke}%
\pgfsetdash{}{0pt}%
\pgfsys@defobject{currentmarker}{\pgfqpoint{0.000000in}{0.000000in}}{\pgfqpoint{0.000000in}{0.034722in}}{%
\pgfpathmoveto{\pgfqpoint{0.000000in}{0.000000in}}%
\pgfpathlineto{\pgfqpoint{0.000000in}{0.034722in}}%
\pgfusepath{stroke,fill}%
}%
\begin{pgfscope}%
\pgfsys@transformshift{4.520880in}{1.247073in}%
\pgfsys@useobject{currentmarker}{}%
\end{pgfscope}%
\end{pgfscope}%
\begin{pgfscope}%
\definecolor{textcolor}{rgb}{0.000000,0.000000,0.000000}%
\pgfsetstrokecolor{textcolor}%
\pgfsetfillcolor{textcolor}%
\pgftext[x=4.520880in,y=1.198462in,,top]{\color{textcolor}\sffamily\fontsize{18.000000}{21.600000}\selectfont $\displaystyle 1$}%
\end{pgfscope}%
\begin{pgfscope}%
\pgfpathrectangle{\pgfqpoint{2.913455in}{1.247073in}}{\pgfqpoint{9.007805in}{6.674186in}}%
\pgfusepath{clip}%
\pgfsetrectcap%
\pgfsetroundjoin%
\pgfsetlinewidth{0.501875pt}%
\definecolor{currentstroke}{rgb}{0.000000,0.000000,0.000000}%
\pgfsetstrokecolor{currentstroke}%
\pgfsetstrokeopacity{0.100000}%
\pgfsetdash{}{0pt}%
\pgfpathmoveto{\pgfqpoint{5.873368in}{1.247073in}}%
\pgfpathlineto{\pgfqpoint{5.873368in}{7.921260in}}%
\pgfusepath{stroke}%
\end{pgfscope}%
\begin{pgfscope}%
\pgfsetbuttcap%
\pgfsetroundjoin%
\definecolor{currentfill}{rgb}{0.000000,0.000000,0.000000}%
\pgfsetfillcolor{currentfill}%
\pgfsetlinewidth{0.501875pt}%
\definecolor{currentstroke}{rgb}{0.000000,0.000000,0.000000}%
\pgfsetstrokecolor{currentstroke}%
\pgfsetdash{}{0pt}%
\pgfsys@defobject{currentmarker}{\pgfqpoint{0.000000in}{0.000000in}}{\pgfqpoint{0.000000in}{0.034722in}}{%
\pgfpathmoveto{\pgfqpoint{0.000000in}{0.000000in}}%
\pgfpathlineto{\pgfqpoint{0.000000in}{0.034722in}}%
\pgfusepath{stroke,fill}%
}%
\begin{pgfscope}%
\pgfsys@transformshift{5.873368in}{1.247073in}%
\pgfsys@useobject{currentmarker}{}%
\end{pgfscope}%
\end{pgfscope}%
\begin{pgfscope}%
\definecolor{textcolor}{rgb}{0.000000,0.000000,0.000000}%
\pgfsetstrokecolor{textcolor}%
\pgfsetfillcolor{textcolor}%
\pgftext[x=5.873368in,y=1.198462in,,top]{\color{textcolor}\sffamily\fontsize{18.000000}{21.600000}\selectfont $\displaystyle 2$}%
\end{pgfscope}%
\begin{pgfscope}%
\pgfpathrectangle{\pgfqpoint{2.913455in}{1.247073in}}{\pgfqpoint{9.007805in}{6.674186in}}%
\pgfusepath{clip}%
\pgfsetrectcap%
\pgfsetroundjoin%
\pgfsetlinewidth{0.501875pt}%
\definecolor{currentstroke}{rgb}{0.000000,0.000000,0.000000}%
\pgfsetstrokecolor{currentstroke}%
\pgfsetstrokeopacity{0.100000}%
\pgfsetdash{}{0pt}%
\pgfpathmoveto{\pgfqpoint{7.225855in}{1.247073in}}%
\pgfpathlineto{\pgfqpoint{7.225855in}{7.921260in}}%
\pgfusepath{stroke}%
\end{pgfscope}%
\begin{pgfscope}%
\pgfsetbuttcap%
\pgfsetroundjoin%
\definecolor{currentfill}{rgb}{0.000000,0.000000,0.000000}%
\pgfsetfillcolor{currentfill}%
\pgfsetlinewidth{0.501875pt}%
\definecolor{currentstroke}{rgb}{0.000000,0.000000,0.000000}%
\pgfsetstrokecolor{currentstroke}%
\pgfsetdash{}{0pt}%
\pgfsys@defobject{currentmarker}{\pgfqpoint{0.000000in}{0.000000in}}{\pgfqpoint{0.000000in}{0.034722in}}{%
\pgfpathmoveto{\pgfqpoint{0.000000in}{0.000000in}}%
\pgfpathlineto{\pgfqpoint{0.000000in}{0.034722in}}%
\pgfusepath{stroke,fill}%
}%
\begin{pgfscope}%
\pgfsys@transformshift{7.225855in}{1.247073in}%
\pgfsys@useobject{currentmarker}{}%
\end{pgfscope}%
\end{pgfscope}%
\begin{pgfscope}%
\definecolor{textcolor}{rgb}{0.000000,0.000000,0.000000}%
\pgfsetstrokecolor{textcolor}%
\pgfsetfillcolor{textcolor}%
\pgftext[x=7.225855in,y=1.198462in,,top]{\color{textcolor}\sffamily\fontsize{18.000000}{21.600000}\selectfont $\displaystyle 3$}%
\end{pgfscope}%
\begin{pgfscope}%
\pgfpathrectangle{\pgfqpoint{2.913455in}{1.247073in}}{\pgfqpoint{9.007805in}{6.674186in}}%
\pgfusepath{clip}%
\pgfsetrectcap%
\pgfsetroundjoin%
\pgfsetlinewidth{0.501875pt}%
\definecolor{currentstroke}{rgb}{0.000000,0.000000,0.000000}%
\pgfsetstrokecolor{currentstroke}%
\pgfsetstrokeopacity{0.100000}%
\pgfsetdash{}{0pt}%
\pgfpathmoveto{\pgfqpoint{8.578343in}{1.247073in}}%
\pgfpathlineto{\pgfqpoint{8.578343in}{7.921260in}}%
\pgfusepath{stroke}%
\end{pgfscope}%
\begin{pgfscope}%
\pgfsetbuttcap%
\pgfsetroundjoin%
\definecolor{currentfill}{rgb}{0.000000,0.000000,0.000000}%
\pgfsetfillcolor{currentfill}%
\pgfsetlinewidth{0.501875pt}%
\definecolor{currentstroke}{rgb}{0.000000,0.000000,0.000000}%
\pgfsetstrokecolor{currentstroke}%
\pgfsetdash{}{0pt}%
\pgfsys@defobject{currentmarker}{\pgfqpoint{0.000000in}{0.000000in}}{\pgfqpoint{0.000000in}{0.034722in}}{%
\pgfpathmoveto{\pgfqpoint{0.000000in}{0.000000in}}%
\pgfpathlineto{\pgfqpoint{0.000000in}{0.034722in}}%
\pgfusepath{stroke,fill}%
}%
\begin{pgfscope}%
\pgfsys@transformshift{8.578343in}{1.247073in}%
\pgfsys@useobject{currentmarker}{}%
\end{pgfscope}%
\end{pgfscope}%
\begin{pgfscope}%
\definecolor{textcolor}{rgb}{0.000000,0.000000,0.000000}%
\pgfsetstrokecolor{textcolor}%
\pgfsetfillcolor{textcolor}%
\pgftext[x=8.578343in,y=1.198462in,,top]{\color{textcolor}\sffamily\fontsize{18.000000}{21.600000}\selectfont $\displaystyle 4$}%
\end{pgfscope}%
\begin{pgfscope}%
\pgfpathrectangle{\pgfqpoint{2.913455in}{1.247073in}}{\pgfqpoint{9.007805in}{6.674186in}}%
\pgfusepath{clip}%
\pgfsetrectcap%
\pgfsetroundjoin%
\pgfsetlinewidth{0.501875pt}%
\definecolor{currentstroke}{rgb}{0.000000,0.000000,0.000000}%
\pgfsetstrokecolor{currentstroke}%
\pgfsetstrokeopacity{0.100000}%
\pgfsetdash{}{0pt}%
\pgfpathmoveto{\pgfqpoint{9.930830in}{1.247073in}}%
\pgfpathlineto{\pgfqpoint{9.930830in}{7.921260in}}%
\pgfusepath{stroke}%
\end{pgfscope}%
\begin{pgfscope}%
\pgfsetbuttcap%
\pgfsetroundjoin%
\definecolor{currentfill}{rgb}{0.000000,0.000000,0.000000}%
\pgfsetfillcolor{currentfill}%
\pgfsetlinewidth{0.501875pt}%
\definecolor{currentstroke}{rgb}{0.000000,0.000000,0.000000}%
\pgfsetstrokecolor{currentstroke}%
\pgfsetdash{}{0pt}%
\pgfsys@defobject{currentmarker}{\pgfqpoint{0.000000in}{0.000000in}}{\pgfqpoint{0.000000in}{0.034722in}}{%
\pgfpathmoveto{\pgfqpoint{0.000000in}{0.000000in}}%
\pgfpathlineto{\pgfqpoint{0.000000in}{0.034722in}}%
\pgfusepath{stroke,fill}%
}%
\begin{pgfscope}%
\pgfsys@transformshift{9.930830in}{1.247073in}%
\pgfsys@useobject{currentmarker}{}%
\end{pgfscope}%
\end{pgfscope}%
\begin{pgfscope}%
\definecolor{textcolor}{rgb}{0.000000,0.000000,0.000000}%
\pgfsetstrokecolor{textcolor}%
\pgfsetfillcolor{textcolor}%
\pgftext[x=9.930830in,y=1.198462in,,top]{\color{textcolor}\sffamily\fontsize{18.000000}{21.600000}\selectfont $\displaystyle 5$}%
\end{pgfscope}%
\begin{pgfscope}%
\pgfpathrectangle{\pgfqpoint{2.913455in}{1.247073in}}{\pgfqpoint{9.007805in}{6.674186in}}%
\pgfusepath{clip}%
\pgfsetrectcap%
\pgfsetroundjoin%
\pgfsetlinewidth{0.501875pt}%
\definecolor{currentstroke}{rgb}{0.000000,0.000000,0.000000}%
\pgfsetstrokecolor{currentstroke}%
\pgfsetstrokeopacity{0.100000}%
\pgfsetdash{}{0pt}%
\pgfpathmoveto{\pgfqpoint{11.283317in}{1.247073in}}%
\pgfpathlineto{\pgfqpoint{11.283317in}{7.921260in}}%
\pgfusepath{stroke}%
\end{pgfscope}%
\begin{pgfscope}%
\pgfsetbuttcap%
\pgfsetroundjoin%
\definecolor{currentfill}{rgb}{0.000000,0.000000,0.000000}%
\pgfsetfillcolor{currentfill}%
\pgfsetlinewidth{0.501875pt}%
\definecolor{currentstroke}{rgb}{0.000000,0.000000,0.000000}%
\pgfsetstrokecolor{currentstroke}%
\pgfsetdash{}{0pt}%
\pgfsys@defobject{currentmarker}{\pgfqpoint{0.000000in}{0.000000in}}{\pgfqpoint{0.000000in}{0.034722in}}{%
\pgfpathmoveto{\pgfqpoint{0.000000in}{0.000000in}}%
\pgfpathlineto{\pgfqpoint{0.000000in}{0.034722in}}%
\pgfusepath{stroke,fill}%
}%
\begin{pgfscope}%
\pgfsys@transformshift{11.283317in}{1.247073in}%
\pgfsys@useobject{currentmarker}{}%
\end{pgfscope}%
\end{pgfscope}%
\begin{pgfscope}%
\definecolor{textcolor}{rgb}{0.000000,0.000000,0.000000}%
\pgfsetstrokecolor{textcolor}%
\pgfsetfillcolor{textcolor}%
\pgftext[x=11.283317in,y=1.198462in,,top]{\color{textcolor}\sffamily\fontsize{18.000000}{21.600000}\selectfont $\displaystyle 6$}%
\end{pgfscope}%
\begin{pgfscope}%
\definecolor{textcolor}{rgb}{0.000000,0.000000,0.000000}%
\pgfsetstrokecolor{textcolor}%
\pgfsetfillcolor{textcolor}%
\pgftext[x=7.417357in,y=0.900964in,,top]{\color{textcolor}\sffamily\fontsize{18.000000}{21.600000}\selectfont $\displaystyle x$}%
\end{pgfscope}%
\begin{pgfscope}%
\pgfpathrectangle{\pgfqpoint{2.913455in}{1.247073in}}{\pgfqpoint{9.007805in}{6.674186in}}%
\pgfusepath{clip}%
\pgfsetrectcap%
\pgfsetroundjoin%
\pgfsetlinewidth{0.501875pt}%
\definecolor{currentstroke}{rgb}{0.000000,0.000000,0.000000}%
\pgfsetstrokecolor{currentstroke}%
\pgfsetstrokeopacity{0.100000}%
\pgfsetdash{}{0pt}%
\pgfpathmoveto{\pgfqpoint{2.913455in}{1.343600in}}%
\pgfpathlineto{\pgfqpoint{11.921260in}{1.343600in}}%
\pgfusepath{stroke}%
\end{pgfscope}%
\begin{pgfscope}%
\pgfsetbuttcap%
\pgfsetroundjoin%
\definecolor{currentfill}{rgb}{0.000000,0.000000,0.000000}%
\pgfsetfillcolor{currentfill}%
\pgfsetlinewidth{0.501875pt}%
\definecolor{currentstroke}{rgb}{0.000000,0.000000,0.000000}%
\pgfsetstrokecolor{currentstroke}%
\pgfsetdash{}{0pt}%
\pgfsys@defobject{currentmarker}{\pgfqpoint{0.000000in}{0.000000in}}{\pgfqpoint{0.034722in}{0.000000in}}{%
\pgfpathmoveto{\pgfqpoint{0.000000in}{0.000000in}}%
\pgfpathlineto{\pgfqpoint{0.034722in}{0.000000in}}%
\pgfusepath{stroke,fill}%
}%
\begin{pgfscope}%
\pgfsys@transformshift{2.913455in}{1.343600in}%
\pgfsys@useobject{currentmarker}{}%
\end{pgfscope}%
\end{pgfscope}%
\begin{pgfscope}%
\definecolor{textcolor}{rgb}{0.000000,0.000000,0.000000}%
\pgfsetstrokecolor{textcolor}%
\pgfsetfillcolor{textcolor}%
\pgftext[x=1.793017in, y=1.248629in, left, base]{\color{textcolor}\sffamily\fontsize{18.000000}{21.600000}\selectfont $\displaystyle -3.0×10^{96}$}%
\end{pgfscope}%
\begin{pgfscope}%
\pgfpathrectangle{\pgfqpoint{2.913455in}{1.247073in}}{\pgfqpoint{9.007805in}{6.674186in}}%
\pgfusepath{clip}%
\pgfsetrectcap%
\pgfsetroundjoin%
\pgfsetlinewidth{0.501875pt}%
\definecolor{currentstroke}{rgb}{0.000000,0.000000,0.000000}%
\pgfsetstrokecolor{currentstroke}%
\pgfsetstrokeopacity{0.100000}%
\pgfsetdash{}{0pt}%
\pgfpathmoveto{\pgfqpoint{2.913455in}{2.422393in}}%
\pgfpathlineto{\pgfqpoint{11.921260in}{2.422393in}}%
\pgfusepath{stroke}%
\end{pgfscope}%
\begin{pgfscope}%
\pgfsetbuttcap%
\pgfsetroundjoin%
\definecolor{currentfill}{rgb}{0.000000,0.000000,0.000000}%
\pgfsetfillcolor{currentfill}%
\pgfsetlinewidth{0.501875pt}%
\definecolor{currentstroke}{rgb}{0.000000,0.000000,0.000000}%
\pgfsetstrokecolor{currentstroke}%
\pgfsetdash{}{0pt}%
\pgfsys@defobject{currentmarker}{\pgfqpoint{0.000000in}{0.000000in}}{\pgfqpoint{0.034722in}{0.000000in}}{%
\pgfpathmoveto{\pgfqpoint{0.000000in}{0.000000in}}%
\pgfpathlineto{\pgfqpoint{0.034722in}{0.000000in}}%
\pgfusepath{stroke,fill}%
}%
\begin{pgfscope}%
\pgfsys@transformshift{2.913455in}{2.422393in}%
\pgfsys@useobject{currentmarker}{}%
\end{pgfscope}%
\end{pgfscope}%
\begin{pgfscope}%
\definecolor{textcolor}{rgb}{0.000000,0.000000,0.000000}%
\pgfsetstrokecolor{textcolor}%
\pgfsetfillcolor{textcolor}%
\pgftext[x=1.793017in, y=2.327422in, left, base]{\color{textcolor}\sffamily\fontsize{18.000000}{21.600000}\selectfont $\displaystyle -2.0×10^{96}$}%
\end{pgfscope}%
\begin{pgfscope}%
\pgfpathrectangle{\pgfqpoint{2.913455in}{1.247073in}}{\pgfqpoint{9.007805in}{6.674186in}}%
\pgfusepath{clip}%
\pgfsetrectcap%
\pgfsetroundjoin%
\pgfsetlinewidth{0.501875pt}%
\definecolor{currentstroke}{rgb}{0.000000,0.000000,0.000000}%
\pgfsetstrokecolor{currentstroke}%
\pgfsetstrokeopacity{0.100000}%
\pgfsetdash{}{0pt}%
\pgfpathmoveto{\pgfqpoint{2.913455in}{3.501186in}}%
\pgfpathlineto{\pgfqpoint{11.921260in}{3.501186in}}%
\pgfusepath{stroke}%
\end{pgfscope}%
\begin{pgfscope}%
\pgfsetbuttcap%
\pgfsetroundjoin%
\definecolor{currentfill}{rgb}{0.000000,0.000000,0.000000}%
\pgfsetfillcolor{currentfill}%
\pgfsetlinewidth{0.501875pt}%
\definecolor{currentstroke}{rgb}{0.000000,0.000000,0.000000}%
\pgfsetstrokecolor{currentstroke}%
\pgfsetdash{}{0pt}%
\pgfsys@defobject{currentmarker}{\pgfqpoint{0.000000in}{0.000000in}}{\pgfqpoint{0.034722in}{0.000000in}}{%
\pgfpathmoveto{\pgfqpoint{0.000000in}{0.000000in}}%
\pgfpathlineto{\pgfqpoint{0.034722in}{0.000000in}}%
\pgfusepath{stroke,fill}%
}%
\begin{pgfscope}%
\pgfsys@transformshift{2.913455in}{3.501186in}%
\pgfsys@useobject{currentmarker}{}%
\end{pgfscope}%
\end{pgfscope}%
\begin{pgfscope}%
\definecolor{textcolor}{rgb}{0.000000,0.000000,0.000000}%
\pgfsetstrokecolor{textcolor}%
\pgfsetfillcolor{textcolor}%
\pgftext[x=1.793017in, y=3.406215in, left, base]{\color{textcolor}\sffamily\fontsize{18.000000}{21.600000}\selectfont $\displaystyle -1.0×10^{96}$}%
\end{pgfscope}%
\begin{pgfscope}%
\pgfpathrectangle{\pgfqpoint{2.913455in}{1.247073in}}{\pgfqpoint{9.007805in}{6.674186in}}%
\pgfusepath{clip}%
\pgfsetrectcap%
\pgfsetroundjoin%
\pgfsetlinewidth{0.501875pt}%
\definecolor{currentstroke}{rgb}{0.000000,0.000000,0.000000}%
\pgfsetstrokecolor{currentstroke}%
\pgfsetstrokeopacity{0.100000}%
\pgfsetdash{}{0pt}%
\pgfpathmoveto{\pgfqpoint{2.913455in}{4.579979in}}%
\pgfpathlineto{\pgfqpoint{11.921260in}{4.579979in}}%
\pgfusepath{stroke}%
\end{pgfscope}%
\begin{pgfscope}%
\pgfsetbuttcap%
\pgfsetroundjoin%
\definecolor{currentfill}{rgb}{0.000000,0.000000,0.000000}%
\pgfsetfillcolor{currentfill}%
\pgfsetlinewidth{0.501875pt}%
\definecolor{currentstroke}{rgb}{0.000000,0.000000,0.000000}%
\pgfsetstrokecolor{currentstroke}%
\pgfsetdash{}{0pt}%
\pgfsys@defobject{currentmarker}{\pgfqpoint{0.000000in}{0.000000in}}{\pgfqpoint{0.034722in}{0.000000in}}{%
\pgfpathmoveto{\pgfqpoint{0.000000in}{0.000000in}}%
\pgfpathlineto{\pgfqpoint{0.034722in}{0.000000in}}%
\pgfusepath{stroke,fill}%
}%
\begin{pgfscope}%
\pgfsys@transformshift{2.913455in}{4.579979in}%
\pgfsys@useobject{currentmarker}{}%
\end{pgfscope}%
\end{pgfscope}%
\begin{pgfscope}%
\definecolor{textcolor}{rgb}{0.000000,0.000000,0.000000}%
\pgfsetstrokecolor{textcolor}%
\pgfsetfillcolor{textcolor}%
\pgftext[x=2.754776in, y=4.485008in, left, base]{\color{textcolor}\sffamily\fontsize{18.000000}{21.600000}\selectfont $\displaystyle 0$}%
\end{pgfscope}%
\begin{pgfscope}%
\pgfpathrectangle{\pgfqpoint{2.913455in}{1.247073in}}{\pgfqpoint{9.007805in}{6.674186in}}%
\pgfusepath{clip}%
\pgfsetrectcap%
\pgfsetroundjoin%
\pgfsetlinewidth{0.501875pt}%
\definecolor{currentstroke}{rgb}{0.000000,0.000000,0.000000}%
\pgfsetstrokecolor{currentstroke}%
\pgfsetstrokeopacity{0.100000}%
\pgfsetdash{}{0pt}%
\pgfpathmoveto{\pgfqpoint{2.913455in}{5.658772in}}%
\pgfpathlineto{\pgfqpoint{11.921260in}{5.658772in}}%
\pgfusepath{stroke}%
\end{pgfscope}%
\begin{pgfscope}%
\pgfsetbuttcap%
\pgfsetroundjoin%
\definecolor{currentfill}{rgb}{0.000000,0.000000,0.000000}%
\pgfsetfillcolor{currentfill}%
\pgfsetlinewidth{0.501875pt}%
\definecolor{currentstroke}{rgb}{0.000000,0.000000,0.000000}%
\pgfsetstrokecolor{currentstroke}%
\pgfsetdash{}{0pt}%
\pgfsys@defobject{currentmarker}{\pgfqpoint{0.000000in}{0.000000in}}{\pgfqpoint{0.034722in}{0.000000in}}{%
\pgfpathmoveto{\pgfqpoint{0.000000in}{0.000000in}}%
\pgfpathlineto{\pgfqpoint{0.034722in}{0.000000in}}%
\pgfusepath{stroke,fill}%
}%
\begin{pgfscope}%
\pgfsys@transformshift{2.913455in}{5.658772in}%
\pgfsys@useobject{currentmarker}{}%
\end{pgfscope}%
\end{pgfscope}%
\begin{pgfscope}%
\definecolor{textcolor}{rgb}{0.000000,0.000000,0.000000}%
\pgfsetstrokecolor{textcolor}%
\pgfsetfillcolor{textcolor}%
\pgftext[x=1.979684in, y=5.563801in, left, base]{\color{textcolor}\sffamily\fontsize{18.000000}{21.600000}\selectfont $\displaystyle 1.0×10^{96}$}%
\end{pgfscope}%
\begin{pgfscope}%
\pgfpathrectangle{\pgfqpoint{2.913455in}{1.247073in}}{\pgfqpoint{9.007805in}{6.674186in}}%
\pgfusepath{clip}%
\pgfsetrectcap%
\pgfsetroundjoin%
\pgfsetlinewidth{0.501875pt}%
\definecolor{currentstroke}{rgb}{0.000000,0.000000,0.000000}%
\pgfsetstrokecolor{currentstroke}%
\pgfsetstrokeopacity{0.100000}%
\pgfsetdash{}{0pt}%
\pgfpathmoveto{\pgfqpoint{2.913455in}{6.737565in}}%
\pgfpathlineto{\pgfqpoint{11.921260in}{6.737565in}}%
\pgfusepath{stroke}%
\end{pgfscope}%
\begin{pgfscope}%
\pgfsetbuttcap%
\pgfsetroundjoin%
\definecolor{currentfill}{rgb}{0.000000,0.000000,0.000000}%
\pgfsetfillcolor{currentfill}%
\pgfsetlinewidth{0.501875pt}%
\definecolor{currentstroke}{rgb}{0.000000,0.000000,0.000000}%
\pgfsetstrokecolor{currentstroke}%
\pgfsetdash{}{0pt}%
\pgfsys@defobject{currentmarker}{\pgfqpoint{0.000000in}{0.000000in}}{\pgfqpoint{0.034722in}{0.000000in}}{%
\pgfpathmoveto{\pgfqpoint{0.000000in}{0.000000in}}%
\pgfpathlineto{\pgfqpoint{0.034722in}{0.000000in}}%
\pgfusepath{stroke,fill}%
}%
\begin{pgfscope}%
\pgfsys@transformshift{2.913455in}{6.737565in}%
\pgfsys@useobject{currentmarker}{}%
\end{pgfscope}%
\end{pgfscope}%
\begin{pgfscope}%
\definecolor{textcolor}{rgb}{0.000000,0.000000,0.000000}%
\pgfsetstrokecolor{textcolor}%
\pgfsetfillcolor{textcolor}%
\pgftext[x=1.979684in, y=6.642595in, left, base]{\color{textcolor}\sffamily\fontsize{18.000000}{21.600000}\selectfont $\displaystyle 2.0×10^{96}$}%
\end{pgfscope}%
\begin{pgfscope}%
\pgfpathrectangle{\pgfqpoint{2.913455in}{1.247073in}}{\pgfqpoint{9.007805in}{6.674186in}}%
\pgfusepath{clip}%
\pgfsetrectcap%
\pgfsetroundjoin%
\pgfsetlinewidth{0.501875pt}%
\definecolor{currentstroke}{rgb}{0.000000,0.000000,0.000000}%
\pgfsetstrokecolor{currentstroke}%
\pgfsetstrokeopacity{0.100000}%
\pgfsetdash{}{0pt}%
\pgfpathmoveto{\pgfqpoint{2.913455in}{7.816358in}}%
\pgfpathlineto{\pgfqpoint{11.921260in}{7.816358in}}%
\pgfusepath{stroke}%
\end{pgfscope}%
\begin{pgfscope}%
\pgfsetbuttcap%
\pgfsetroundjoin%
\definecolor{currentfill}{rgb}{0.000000,0.000000,0.000000}%
\pgfsetfillcolor{currentfill}%
\pgfsetlinewidth{0.501875pt}%
\definecolor{currentstroke}{rgb}{0.000000,0.000000,0.000000}%
\pgfsetstrokecolor{currentstroke}%
\pgfsetdash{}{0pt}%
\pgfsys@defobject{currentmarker}{\pgfqpoint{0.000000in}{0.000000in}}{\pgfqpoint{0.034722in}{0.000000in}}{%
\pgfpathmoveto{\pgfqpoint{0.000000in}{0.000000in}}%
\pgfpathlineto{\pgfqpoint{0.034722in}{0.000000in}}%
\pgfusepath{stroke,fill}%
}%
\begin{pgfscope}%
\pgfsys@transformshift{2.913455in}{7.816358in}%
\pgfsys@useobject{currentmarker}{}%
\end{pgfscope}%
\end{pgfscope}%
\begin{pgfscope}%
\definecolor{textcolor}{rgb}{0.000000,0.000000,0.000000}%
\pgfsetstrokecolor{textcolor}%
\pgfsetfillcolor{textcolor}%
\pgftext[x=1.979684in, y=7.721388in, left, base]{\color{textcolor}\sffamily\fontsize{18.000000}{21.600000}\selectfont $\displaystyle 3.0×10^{96}$}%
\end{pgfscope}%
\begin{pgfscope}%
\pgfpathrectangle{\pgfqpoint{2.913455in}{1.247073in}}{\pgfqpoint{9.007805in}{6.674186in}}%
\pgfusepath{clip}%
\pgfsetbuttcap%
\pgfsetroundjoin%
\pgfsetlinewidth{1.003750pt}%
\definecolor{currentstroke}{rgb}{0.000000,0.605603,0.978680}%
\pgfsetstrokecolor{currentstroke}%
\pgfsetdash{}{0pt}%
\pgfpathmoveto{\pgfqpoint{3.168393in}{4.579979in}}%
\pgfpathlineto{\pgfqpoint{6.450552in}{4.580695in}}%
\pgfpathlineto{\pgfqpoint{6.454701in}{4.578713in}}%
\pgfpathlineto{\pgfqpoint{6.458851in}{4.582175in}}%
\pgfpathlineto{\pgfqpoint{6.463000in}{4.576243in}}%
\pgfpathlineto{\pgfqpoint{6.467150in}{4.586210in}}%
\pgfpathlineto{\pgfqpoint{6.471299in}{4.569788in}}%
\pgfpathlineto{\pgfqpoint{6.475448in}{4.596320in}}%
\pgfpathlineto{\pgfqpoint{6.479598in}{4.554289in}}%
\pgfpathlineto{\pgfqpoint{6.483747in}{4.619573in}}%
\pgfpathlineto{\pgfqpoint{6.487897in}{4.520155in}}%
\pgfpathlineto{\pgfqpoint{6.492046in}{4.668585in}}%
\pgfpathlineto{\pgfqpoint{6.496195in}{4.451341in}}%
\pgfpathlineto{\pgfqpoint{6.500345in}{4.763028in}}%
\pgfpathlineto{\pgfqpoint{6.504494in}{4.324693in}}%
\pgfpathlineto{\pgfqpoint{6.508643in}{4.928892in}}%
\pgfpathlineto{\pgfqpoint{6.512793in}{4.112672in}}%
\pgfpathlineto{\pgfqpoint{6.516942in}{5.193244in}}%
\pgfpathlineto{\pgfqpoint{6.521092in}{3.791454in}}%
\pgfpathlineto{\pgfqpoint{6.525241in}{5.573245in}}%
\pgfpathlineto{\pgfqpoint{6.529390in}{3.354361in}}%
\pgfpathlineto{\pgfqpoint{6.533540in}{6.061269in}}%
\pgfpathlineto{\pgfqpoint{6.537689in}{2.826608in}}%
\pgfpathlineto{\pgfqpoint{6.541838in}{6.612370in}}%
\pgfpathlineto{\pgfqpoint{6.545988in}{2.273291in}}%
\pgfpathlineto{\pgfqpoint{6.550137in}{7.143075in}}%
\pgfpathlineto{\pgfqpoint{6.554287in}{1.792071in}}%
\pgfpathlineto{\pgfqpoint{6.558436in}{7.548025in}}%
\pgfpathlineto{\pgfqpoint{6.562585in}{1.487696in}}%
\pgfpathlineto{\pgfqpoint{6.566735in}{7.732368in}}%
\pgfpathlineto{\pgfqpoint{6.570884in}{1.435966in}}%
\pgfpathlineto{\pgfqpoint{6.575033in}{7.647181in}}%
\pgfpathlineto{\pgfqpoint{6.579183in}{1.653537in}}%
\pgfpathlineto{\pgfqpoint{6.583332in}{7.310211in}}%
\pgfpathlineto{\pgfqpoint{6.587482in}{2.089740in}}%
\pgfpathlineto{\pgfqpoint{6.591631in}{6.800117in}}%
\pgfpathlineto{\pgfqpoint{6.595780in}{2.645658in}}%
\pgfpathlineto{\pgfqpoint{6.599930in}{6.226606in}}%
\pgfpathlineto{\pgfqpoint{6.604079in}{3.210725in}}%
\pgfpathlineto{\pgfqpoint{6.608229in}{5.691954in}}%
\pgfpathlineto{\pgfqpoint{6.612378in}{3.698272in}}%
\pgfpathlineto{\pgfqpoint{6.616527in}{5.262415in}}%
\pgfpathlineto{\pgfqpoint{6.620677in}{4.064519in}}%
\pgfpathlineto{\pgfqpoint{6.624826in}{4.959823in}}%
\pgfpathlineto{\pgfqpoint{6.628975in}{4.306977in}}%
\pgfpathlineto{\pgfqpoint{6.633125in}{4.771294in}}%
\pgfpathlineto{\pgfqpoint{6.637274in}{4.449296in}}%
\pgfpathlineto{\pgfqpoint{6.641424in}{4.666962in}}%
\pgfpathlineto{\pgfqpoint{6.645573in}{4.523584in}}%
\pgfpathlineto{\pgfqpoint{6.649722in}{4.615583in}}%
\pgfpathlineto{\pgfqpoint{6.653872in}{4.558100in}}%
\pgfpathlineto{\pgfqpoint{6.658021in}{4.593060in}}%
\pgfpathlineto{\pgfqpoint{6.662170in}{4.572372in}}%
\pgfpathlineto{\pgfqpoint{6.666320in}{4.584279in}}%
\pgfpathlineto{\pgfqpoint{6.670469in}{4.577617in}}%
\pgfpathlineto{\pgfqpoint{6.674619in}{4.581240in}}%
\pgfpathlineto{\pgfqpoint{6.678768in}{4.579326in}}%
\pgfpathlineto{\pgfqpoint{6.682917in}{4.580307in}}%
\pgfpathlineto{\pgfqpoint{6.695365in}{4.579945in}}%
\pgfpathlineto{\pgfqpoint{6.786652in}{4.579979in}}%
\pgfpathlineto{\pgfqpoint{11.666322in}{4.579979in}}%
\pgfpathlineto{\pgfqpoint{11.666322in}{4.579979in}}%
\pgfusepath{stroke}%
\end{pgfscope}%
\begin{pgfscope}%
\pgfpathrectangle{\pgfqpoint{2.913455in}{1.247073in}}{\pgfqpoint{9.007805in}{6.674186in}}%
\pgfusepath{clip}%
\pgfsetbuttcap%
\pgfsetroundjoin%
\pgfsetlinewidth{1.003750pt}%
\definecolor{currentstroke}{rgb}{0.888874,0.435649,0.278123}%
\pgfsetstrokecolor{currentstroke}%
\pgfsetdash{}{0pt}%
\pgfpathmoveto{\pgfqpoint{3.168393in}{4.579979in}}%
\pgfpathlineto{\pgfqpoint{11.666322in}{4.579979in}}%
\pgfpathlineto{\pgfqpoint{11.666322in}{4.579979in}}%
\pgfusepath{stroke}%
\end{pgfscope}%
\begin{pgfscope}%
\pgfsetrectcap%
\pgfsetmiterjoin%
\pgfsetlinewidth{1.003750pt}%
\definecolor{currentstroke}{rgb}{0.000000,0.000000,0.000000}%
\pgfsetstrokecolor{currentstroke}%
\pgfsetdash{}{0pt}%
\pgfpathmoveto{\pgfqpoint{2.913455in}{1.247073in}}%
\pgfpathlineto{\pgfqpoint{2.913455in}{7.921260in}}%
\pgfusepath{stroke}%
\end{pgfscope}%
\begin{pgfscope}%
\pgfsetrectcap%
\pgfsetmiterjoin%
\pgfsetlinewidth{1.003750pt}%
\definecolor{currentstroke}{rgb}{0.000000,0.000000,0.000000}%
\pgfsetstrokecolor{currentstroke}%
\pgfsetdash{}{0pt}%
\pgfpathmoveto{\pgfqpoint{2.913455in}{1.247073in}}%
\pgfpathlineto{\pgfqpoint{11.921260in}{1.247073in}}%
\pgfusepath{stroke}%
\end{pgfscope}%
\begin{pgfscope}%
\pgfsetbuttcap%
\pgfsetmiterjoin%
\definecolor{currentfill}{rgb}{1.000000,1.000000,1.000000}%
\pgfsetfillcolor{currentfill}%
\pgfsetlinewidth{1.003750pt}%
\definecolor{currentstroke}{rgb}{0.000000,0.000000,0.000000}%
\pgfsetstrokecolor{currentstroke}%
\pgfsetdash{}{0pt}%
\pgfpathmoveto{\pgfqpoint{10.511589in}{6.787373in}}%
\pgfpathlineto{\pgfqpoint{11.796260in}{6.787373in}}%
\pgfpathlineto{\pgfqpoint{11.796260in}{7.796260in}}%
\pgfpathlineto{\pgfqpoint{10.511589in}{7.796260in}}%
\pgfpathclose%
\pgfusepath{stroke,fill}%
\end{pgfscope}%
\begin{pgfscope}%
\pgfsetbuttcap%
\pgfsetmiterjoin%
\pgfsetlinewidth{2.258437pt}%
\definecolor{currentstroke}{rgb}{0.000000,0.605603,0.978680}%
\pgfsetstrokecolor{currentstroke}%
\pgfsetdash{}{0pt}%
\pgfpathmoveto{\pgfqpoint{10.711589in}{7.493818in}}%
\pgfpathlineto{\pgfqpoint{11.211589in}{7.493818in}}%
\pgfusepath{stroke}%
\end{pgfscope}%
\begin{pgfscope}%
\definecolor{textcolor}{rgb}{0.000000,0.000000,0.000000}%
\pgfsetstrokecolor{textcolor}%
\pgfsetfillcolor{textcolor}%
\pgftext[x=11.411589in,y=7.406318in,left,base]{\color{textcolor}\sffamily\fontsize{18.000000}{21.600000}\selectfont $\displaystyle U$}%
\end{pgfscope}%
\begin{pgfscope}%
\pgfsetbuttcap%
\pgfsetmiterjoin%
\pgfsetlinewidth{2.258437pt}%
\definecolor{currentstroke}{rgb}{0.888874,0.435649,0.278123}%
\pgfsetstrokecolor{currentstroke}%
\pgfsetdash{}{0pt}%
\pgfpathmoveto{\pgfqpoint{10.711589in}{7.126875in}}%
\pgfpathlineto{\pgfqpoint{11.211589in}{7.126875in}}%
\pgfusepath{stroke}%
\end{pgfscope}%
\begin{pgfscope}%
\definecolor{textcolor}{rgb}{0.000000,0.000000,0.000000}%
\pgfsetstrokecolor{textcolor}%
\pgfsetfillcolor{textcolor}%
\pgftext[x=11.411589in,y=7.039375in,left,base]{\color{textcolor}\sffamily\fontsize{18.000000}{21.600000}\selectfont $\displaystyle u$}%
\end{pgfscope}%
\end{pgfpicture}%
\makeatother%
\endgroup%
}
	\caption{迎风格式差分逼近解 $U$ 与真解 $u$}\label{fig:upwind_square_Uu_noCFL}
\end{figure}

\subsection{Lax-Wendroff 格式}

取 $\nu = 0.5 \leq 1$, 满足 CFL 条件. 对正弦波问题 \eqref{equ:sine_wave_con} 求解得到误差及收敛阶如表 \ref{tab:lax_wendroff_err} 所示.

\begin{table}[H]\centering\heiti\zihao{-5}
	\caption{Lax-Wendroff 格式不同步长时的 $\mathbb{L}^2$, $\mathbb{L}^\infty$ 误差及收敛阶}\label{tab:lax_wendroff_err}
	\begin{tabular}{|c|c|c|c|c|}\hline
		收敛阶	&	$\mathbb{L}^2$ 误差	&	$h$	&	$\mathbb{L}^\infty$ 误差		&	收敛阶\\\hline
				&	$7.65220 \times 10^{-2}$	&	$2^{-4}$	&	$1.01182 \times 10^{-1}$	&		   \\\hline
		1.97878	&	$1.94140 \times 10^{-2}$	&	$2^{-5}$	&	$2.65973 \times 10^{-2}$	&	1.92760\\\hline
		1.99863	&	$4.85810 \times 10^{-3}$	&	$2^{-6}$	&	$6.76359 \times 10^{-3}$	&	1.97542\\\hline
		2.00126	&	$1.21347 \times 10^{-3}$	&	$2^{-7}$	&	$1.70275 \times 10^{-3}$	&	1.98992\\\hline
		2.00107	&	$3.03142 \times 10^{-4}$	&	$2^{-8}$	&	$4.27037 \times 10^{-4}$	&	1.99543\\\hline
		2.00064	&	$7.57519 \times 10^{-5}$	&	$2^{-9}$	&	$1.06920 \times 10^{-4}$	&	1.99782\\\hline
		2.00035	&	$1.89334 \times 10^{-5}$	&	$2^{-10}$	&	$2.67498 \times 10^{-5}$	&	1.99894\\\hline
		2.00018	&	$4.73277 \times 10^{-6}$	&	$2^{-11}$	&	$6.68988 \times 10^{-6}$	&	1.99948\\\hline
		2.00009	&	$1.18312 \times 10^{-6}$	&	$2^{-12}$	&	$1.67277 \times 10^{-6}$	&	1.99974\\\hline
	\end{tabular}
\end{table}

由数值结果可以看出解序列逐步收敛到模型问题的解, 收敛阶趋于 1, 与理论结果相符. $h = 2^{-7}$ 和 $h = 2^{-10}$ 时差分逼近解 $U$ 与真解 $u$ 在 $t = t_{\max }$ 时刻图像如图 \ref{fig:lax_wendroff_Uu} 所示.

\begin{figure}[H]\centering\zihao{-5}
	\resizebox{0.4\linewidth}{!}{%% Creator: Matplotlib, PGF backend
%%
%% To include the figure in your LaTeX document, write
%%   \input{<filename>.pgf}
%%
%% Make sure the required packages are loaded in your preamble
%%   \usepackage{pgf}
%%
%% Figures using additional raster images can only be included by \input if
%% they are in the same directory as the main LaTeX file. For loading figures
%% from other directories you can use the `import` package
%%   \usepackage{import}
%%
%% and then include the figures with
%%   \import{<path to file>}{<filename>.pgf}
%%
%% Matplotlib used the following preamble
%%   \usepackage{fontspec}
%%   \setmainfont{DejaVuSerif.ttf}[Path=\detokenize{/Users/quejiahao/.julia/conda/3/lib/python3.9/site-packages/matplotlib/mpl-data/fonts/ttf/}]
%%   \setsansfont{DejaVuSans.ttf}[Path=\detokenize{/Users/quejiahao/.julia/conda/3/lib/python3.9/site-packages/matplotlib/mpl-data/fonts/ttf/}]
%%   \setmonofont{DejaVuSansMono.ttf}[Path=\detokenize{/Users/quejiahao/.julia/conda/3/lib/python3.9/site-packages/matplotlib/mpl-data/fonts/ttf/}]
%%
\begingroup%
\makeatletter%
\begin{pgfpicture}%
\pgfpathrectangle{\pgfpointorigin}{\pgfqpoint{12.000000in}{8.000000in}}%
\pgfusepath{use as bounding box, clip}%
\begin{pgfscope}%
\pgfsetbuttcap%
\pgfsetmiterjoin%
\definecolor{currentfill}{rgb}{1.000000,1.000000,1.000000}%
\pgfsetfillcolor{currentfill}%
\pgfsetlinewidth{0.000000pt}%
\definecolor{currentstroke}{rgb}{1.000000,1.000000,1.000000}%
\pgfsetstrokecolor{currentstroke}%
\pgfsetdash{}{0pt}%
\pgfpathmoveto{\pgfqpoint{0.000000in}{0.000000in}}%
\pgfpathlineto{\pgfqpoint{12.000000in}{0.000000in}}%
\pgfpathlineto{\pgfqpoint{12.000000in}{8.000000in}}%
\pgfpathlineto{\pgfqpoint{0.000000in}{8.000000in}}%
\pgfpathclose%
\pgfusepath{fill}%
\end{pgfscope}%
\begin{pgfscope}%
\pgfsetbuttcap%
\pgfsetmiterjoin%
\definecolor{currentfill}{rgb}{1.000000,1.000000,1.000000}%
\pgfsetfillcolor{currentfill}%
\pgfsetlinewidth{0.000000pt}%
\definecolor{currentstroke}{rgb}{0.000000,0.000000,0.000000}%
\pgfsetstrokecolor{currentstroke}%
\pgfsetstrokeopacity{0.000000}%
\pgfsetdash{}{0pt}%
\pgfpathmoveto{\pgfqpoint{1.396958in}{1.247073in}}%
\pgfpathlineto{\pgfqpoint{11.921260in}{1.247073in}}%
\pgfpathlineto{\pgfqpoint{11.921260in}{7.921260in}}%
\pgfpathlineto{\pgfqpoint{1.396958in}{7.921260in}}%
\pgfpathclose%
\pgfusepath{fill}%
\end{pgfscope}%
\begin{pgfscope}%
\pgfpathrectangle{\pgfqpoint{1.396958in}{1.247073in}}{\pgfqpoint{10.524301in}{6.674186in}}%
\pgfusepath{clip}%
\pgfsetrectcap%
\pgfsetroundjoin%
\pgfsetlinewidth{0.501875pt}%
\definecolor{currentstroke}{rgb}{0.000000,0.000000,0.000000}%
\pgfsetstrokecolor{currentstroke}%
\pgfsetstrokeopacity{0.100000}%
\pgfsetdash{}{0pt}%
\pgfpathmoveto{\pgfqpoint{1.694816in}{1.247073in}}%
\pgfpathlineto{\pgfqpoint{1.694816in}{7.921260in}}%
\pgfusepath{stroke}%
\end{pgfscope}%
\begin{pgfscope}%
\pgfsetbuttcap%
\pgfsetroundjoin%
\definecolor{currentfill}{rgb}{0.000000,0.000000,0.000000}%
\pgfsetfillcolor{currentfill}%
\pgfsetlinewidth{0.501875pt}%
\definecolor{currentstroke}{rgb}{0.000000,0.000000,0.000000}%
\pgfsetstrokecolor{currentstroke}%
\pgfsetdash{}{0pt}%
\pgfsys@defobject{currentmarker}{\pgfqpoint{0.000000in}{0.000000in}}{\pgfqpoint{0.000000in}{0.034722in}}{%
\pgfpathmoveto{\pgfqpoint{0.000000in}{0.000000in}}%
\pgfpathlineto{\pgfqpoint{0.000000in}{0.034722in}}%
\pgfusepath{stroke,fill}%
}%
\begin{pgfscope}%
\pgfsys@transformshift{1.694816in}{1.247073in}%
\pgfsys@useobject{currentmarker}{}%
\end{pgfscope}%
\end{pgfscope}%
\begin{pgfscope}%
\definecolor{textcolor}{rgb}{0.000000,0.000000,0.000000}%
\pgfsetstrokecolor{textcolor}%
\pgfsetfillcolor{textcolor}%
\pgftext[x=1.694816in,y=1.198462in,,top]{\color{textcolor}\sffamily\fontsize{18.000000}{21.600000}\selectfont $\displaystyle 0$}%
\end{pgfscope}%
\begin{pgfscope}%
\pgfpathrectangle{\pgfqpoint{1.396958in}{1.247073in}}{\pgfqpoint{10.524301in}{6.674186in}}%
\pgfusepath{clip}%
\pgfsetrectcap%
\pgfsetroundjoin%
\pgfsetlinewidth{0.501875pt}%
\definecolor{currentstroke}{rgb}{0.000000,0.000000,0.000000}%
\pgfsetstrokecolor{currentstroke}%
\pgfsetstrokeopacity{0.100000}%
\pgfsetdash{}{0pt}%
\pgfpathmoveto{\pgfqpoint{3.275000in}{1.247073in}}%
\pgfpathlineto{\pgfqpoint{3.275000in}{7.921260in}}%
\pgfusepath{stroke}%
\end{pgfscope}%
\begin{pgfscope}%
\pgfsetbuttcap%
\pgfsetroundjoin%
\definecolor{currentfill}{rgb}{0.000000,0.000000,0.000000}%
\pgfsetfillcolor{currentfill}%
\pgfsetlinewidth{0.501875pt}%
\definecolor{currentstroke}{rgb}{0.000000,0.000000,0.000000}%
\pgfsetstrokecolor{currentstroke}%
\pgfsetdash{}{0pt}%
\pgfsys@defobject{currentmarker}{\pgfqpoint{0.000000in}{0.000000in}}{\pgfqpoint{0.000000in}{0.034722in}}{%
\pgfpathmoveto{\pgfqpoint{0.000000in}{0.000000in}}%
\pgfpathlineto{\pgfqpoint{0.000000in}{0.034722in}}%
\pgfusepath{stroke,fill}%
}%
\begin{pgfscope}%
\pgfsys@transformshift{3.275000in}{1.247073in}%
\pgfsys@useobject{currentmarker}{}%
\end{pgfscope}%
\end{pgfscope}%
\begin{pgfscope}%
\definecolor{textcolor}{rgb}{0.000000,0.000000,0.000000}%
\pgfsetstrokecolor{textcolor}%
\pgfsetfillcolor{textcolor}%
\pgftext[x=3.275000in,y=1.198462in,,top]{\color{textcolor}\sffamily\fontsize{18.000000}{21.600000}\selectfont $\displaystyle 1$}%
\end{pgfscope}%
\begin{pgfscope}%
\pgfpathrectangle{\pgfqpoint{1.396958in}{1.247073in}}{\pgfqpoint{10.524301in}{6.674186in}}%
\pgfusepath{clip}%
\pgfsetrectcap%
\pgfsetroundjoin%
\pgfsetlinewidth{0.501875pt}%
\definecolor{currentstroke}{rgb}{0.000000,0.000000,0.000000}%
\pgfsetstrokecolor{currentstroke}%
\pgfsetstrokeopacity{0.100000}%
\pgfsetdash{}{0pt}%
\pgfpathmoveto{\pgfqpoint{4.855183in}{1.247073in}}%
\pgfpathlineto{\pgfqpoint{4.855183in}{7.921260in}}%
\pgfusepath{stroke}%
\end{pgfscope}%
\begin{pgfscope}%
\pgfsetbuttcap%
\pgfsetroundjoin%
\definecolor{currentfill}{rgb}{0.000000,0.000000,0.000000}%
\pgfsetfillcolor{currentfill}%
\pgfsetlinewidth{0.501875pt}%
\definecolor{currentstroke}{rgb}{0.000000,0.000000,0.000000}%
\pgfsetstrokecolor{currentstroke}%
\pgfsetdash{}{0pt}%
\pgfsys@defobject{currentmarker}{\pgfqpoint{0.000000in}{0.000000in}}{\pgfqpoint{0.000000in}{0.034722in}}{%
\pgfpathmoveto{\pgfqpoint{0.000000in}{0.000000in}}%
\pgfpathlineto{\pgfqpoint{0.000000in}{0.034722in}}%
\pgfusepath{stroke,fill}%
}%
\begin{pgfscope}%
\pgfsys@transformshift{4.855183in}{1.247073in}%
\pgfsys@useobject{currentmarker}{}%
\end{pgfscope}%
\end{pgfscope}%
\begin{pgfscope}%
\definecolor{textcolor}{rgb}{0.000000,0.000000,0.000000}%
\pgfsetstrokecolor{textcolor}%
\pgfsetfillcolor{textcolor}%
\pgftext[x=4.855183in,y=1.198462in,,top]{\color{textcolor}\sffamily\fontsize{18.000000}{21.600000}\selectfont $\displaystyle 2$}%
\end{pgfscope}%
\begin{pgfscope}%
\pgfpathrectangle{\pgfqpoint{1.396958in}{1.247073in}}{\pgfqpoint{10.524301in}{6.674186in}}%
\pgfusepath{clip}%
\pgfsetrectcap%
\pgfsetroundjoin%
\pgfsetlinewidth{0.501875pt}%
\definecolor{currentstroke}{rgb}{0.000000,0.000000,0.000000}%
\pgfsetstrokecolor{currentstroke}%
\pgfsetstrokeopacity{0.100000}%
\pgfsetdash{}{0pt}%
\pgfpathmoveto{\pgfqpoint{6.435367in}{1.247073in}}%
\pgfpathlineto{\pgfqpoint{6.435367in}{7.921260in}}%
\pgfusepath{stroke}%
\end{pgfscope}%
\begin{pgfscope}%
\pgfsetbuttcap%
\pgfsetroundjoin%
\definecolor{currentfill}{rgb}{0.000000,0.000000,0.000000}%
\pgfsetfillcolor{currentfill}%
\pgfsetlinewidth{0.501875pt}%
\definecolor{currentstroke}{rgb}{0.000000,0.000000,0.000000}%
\pgfsetstrokecolor{currentstroke}%
\pgfsetdash{}{0pt}%
\pgfsys@defobject{currentmarker}{\pgfqpoint{0.000000in}{0.000000in}}{\pgfqpoint{0.000000in}{0.034722in}}{%
\pgfpathmoveto{\pgfqpoint{0.000000in}{0.000000in}}%
\pgfpathlineto{\pgfqpoint{0.000000in}{0.034722in}}%
\pgfusepath{stroke,fill}%
}%
\begin{pgfscope}%
\pgfsys@transformshift{6.435367in}{1.247073in}%
\pgfsys@useobject{currentmarker}{}%
\end{pgfscope}%
\end{pgfscope}%
\begin{pgfscope}%
\definecolor{textcolor}{rgb}{0.000000,0.000000,0.000000}%
\pgfsetstrokecolor{textcolor}%
\pgfsetfillcolor{textcolor}%
\pgftext[x=6.435367in,y=1.198462in,,top]{\color{textcolor}\sffamily\fontsize{18.000000}{21.600000}\selectfont $\displaystyle 3$}%
\end{pgfscope}%
\begin{pgfscope}%
\pgfpathrectangle{\pgfqpoint{1.396958in}{1.247073in}}{\pgfqpoint{10.524301in}{6.674186in}}%
\pgfusepath{clip}%
\pgfsetrectcap%
\pgfsetroundjoin%
\pgfsetlinewidth{0.501875pt}%
\definecolor{currentstroke}{rgb}{0.000000,0.000000,0.000000}%
\pgfsetstrokecolor{currentstroke}%
\pgfsetstrokeopacity{0.100000}%
\pgfsetdash{}{0pt}%
\pgfpathmoveto{\pgfqpoint{8.015550in}{1.247073in}}%
\pgfpathlineto{\pgfqpoint{8.015550in}{7.921260in}}%
\pgfusepath{stroke}%
\end{pgfscope}%
\begin{pgfscope}%
\pgfsetbuttcap%
\pgfsetroundjoin%
\definecolor{currentfill}{rgb}{0.000000,0.000000,0.000000}%
\pgfsetfillcolor{currentfill}%
\pgfsetlinewidth{0.501875pt}%
\definecolor{currentstroke}{rgb}{0.000000,0.000000,0.000000}%
\pgfsetstrokecolor{currentstroke}%
\pgfsetdash{}{0pt}%
\pgfsys@defobject{currentmarker}{\pgfqpoint{0.000000in}{0.000000in}}{\pgfqpoint{0.000000in}{0.034722in}}{%
\pgfpathmoveto{\pgfqpoint{0.000000in}{0.000000in}}%
\pgfpathlineto{\pgfqpoint{0.000000in}{0.034722in}}%
\pgfusepath{stroke,fill}%
}%
\begin{pgfscope}%
\pgfsys@transformshift{8.015550in}{1.247073in}%
\pgfsys@useobject{currentmarker}{}%
\end{pgfscope}%
\end{pgfscope}%
\begin{pgfscope}%
\definecolor{textcolor}{rgb}{0.000000,0.000000,0.000000}%
\pgfsetstrokecolor{textcolor}%
\pgfsetfillcolor{textcolor}%
\pgftext[x=8.015550in,y=1.198462in,,top]{\color{textcolor}\sffamily\fontsize{18.000000}{21.600000}\selectfont $\displaystyle 4$}%
\end{pgfscope}%
\begin{pgfscope}%
\pgfpathrectangle{\pgfqpoint{1.396958in}{1.247073in}}{\pgfqpoint{10.524301in}{6.674186in}}%
\pgfusepath{clip}%
\pgfsetrectcap%
\pgfsetroundjoin%
\pgfsetlinewidth{0.501875pt}%
\definecolor{currentstroke}{rgb}{0.000000,0.000000,0.000000}%
\pgfsetstrokecolor{currentstroke}%
\pgfsetstrokeopacity{0.100000}%
\pgfsetdash{}{0pt}%
\pgfpathmoveto{\pgfqpoint{9.595734in}{1.247073in}}%
\pgfpathlineto{\pgfqpoint{9.595734in}{7.921260in}}%
\pgfusepath{stroke}%
\end{pgfscope}%
\begin{pgfscope}%
\pgfsetbuttcap%
\pgfsetroundjoin%
\definecolor{currentfill}{rgb}{0.000000,0.000000,0.000000}%
\pgfsetfillcolor{currentfill}%
\pgfsetlinewidth{0.501875pt}%
\definecolor{currentstroke}{rgb}{0.000000,0.000000,0.000000}%
\pgfsetstrokecolor{currentstroke}%
\pgfsetdash{}{0pt}%
\pgfsys@defobject{currentmarker}{\pgfqpoint{0.000000in}{0.000000in}}{\pgfqpoint{0.000000in}{0.034722in}}{%
\pgfpathmoveto{\pgfqpoint{0.000000in}{0.000000in}}%
\pgfpathlineto{\pgfqpoint{0.000000in}{0.034722in}}%
\pgfusepath{stroke,fill}%
}%
\begin{pgfscope}%
\pgfsys@transformshift{9.595734in}{1.247073in}%
\pgfsys@useobject{currentmarker}{}%
\end{pgfscope}%
\end{pgfscope}%
\begin{pgfscope}%
\definecolor{textcolor}{rgb}{0.000000,0.000000,0.000000}%
\pgfsetstrokecolor{textcolor}%
\pgfsetfillcolor{textcolor}%
\pgftext[x=9.595734in,y=1.198462in,,top]{\color{textcolor}\sffamily\fontsize{18.000000}{21.600000}\selectfont $\displaystyle 5$}%
\end{pgfscope}%
\begin{pgfscope}%
\pgfpathrectangle{\pgfqpoint{1.396958in}{1.247073in}}{\pgfqpoint{10.524301in}{6.674186in}}%
\pgfusepath{clip}%
\pgfsetrectcap%
\pgfsetroundjoin%
\pgfsetlinewidth{0.501875pt}%
\definecolor{currentstroke}{rgb}{0.000000,0.000000,0.000000}%
\pgfsetstrokecolor{currentstroke}%
\pgfsetstrokeopacity{0.100000}%
\pgfsetdash{}{0pt}%
\pgfpathmoveto{\pgfqpoint{11.175917in}{1.247073in}}%
\pgfpathlineto{\pgfqpoint{11.175917in}{7.921260in}}%
\pgfusepath{stroke}%
\end{pgfscope}%
\begin{pgfscope}%
\pgfsetbuttcap%
\pgfsetroundjoin%
\definecolor{currentfill}{rgb}{0.000000,0.000000,0.000000}%
\pgfsetfillcolor{currentfill}%
\pgfsetlinewidth{0.501875pt}%
\definecolor{currentstroke}{rgb}{0.000000,0.000000,0.000000}%
\pgfsetstrokecolor{currentstroke}%
\pgfsetdash{}{0pt}%
\pgfsys@defobject{currentmarker}{\pgfqpoint{0.000000in}{0.000000in}}{\pgfqpoint{0.000000in}{0.034722in}}{%
\pgfpathmoveto{\pgfqpoint{0.000000in}{0.000000in}}%
\pgfpathlineto{\pgfqpoint{0.000000in}{0.034722in}}%
\pgfusepath{stroke,fill}%
}%
\begin{pgfscope}%
\pgfsys@transformshift{11.175917in}{1.247073in}%
\pgfsys@useobject{currentmarker}{}%
\end{pgfscope}%
\end{pgfscope}%
\begin{pgfscope}%
\definecolor{textcolor}{rgb}{0.000000,0.000000,0.000000}%
\pgfsetstrokecolor{textcolor}%
\pgfsetfillcolor{textcolor}%
\pgftext[x=11.175917in,y=1.198462in,,top]{\color{textcolor}\sffamily\fontsize{18.000000}{21.600000}\selectfont $\displaystyle 6$}%
\end{pgfscope}%
\begin{pgfscope}%
\definecolor{textcolor}{rgb}{0.000000,0.000000,0.000000}%
\pgfsetstrokecolor{textcolor}%
\pgfsetfillcolor{textcolor}%
\pgftext[x=6.659109in,y=0.900964in,,top]{\color{textcolor}\sffamily\fontsize{18.000000}{21.600000}\selectfont $\displaystyle x$}%
\end{pgfscope}%
\begin{pgfscope}%
\pgfpathrectangle{\pgfqpoint{1.396958in}{1.247073in}}{\pgfqpoint{10.524301in}{6.674186in}}%
\pgfusepath{clip}%
\pgfsetrectcap%
\pgfsetroundjoin%
\pgfsetlinewidth{0.501875pt}%
\definecolor{currentstroke}{rgb}{0.000000,0.000000,0.000000}%
\pgfsetstrokecolor{currentstroke}%
\pgfsetstrokeopacity{0.100000}%
\pgfsetdash{}{0pt}%
\pgfpathmoveto{\pgfqpoint{1.396958in}{1.435966in}}%
\pgfpathlineto{\pgfqpoint{11.921260in}{1.435966in}}%
\pgfusepath{stroke}%
\end{pgfscope}%
\begin{pgfscope}%
\pgfsetbuttcap%
\pgfsetroundjoin%
\definecolor{currentfill}{rgb}{0.000000,0.000000,0.000000}%
\pgfsetfillcolor{currentfill}%
\pgfsetlinewidth{0.501875pt}%
\definecolor{currentstroke}{rgb}{0.000000,0.000000,0.000000}%
\pgfsetstrokecolor{currentstroke}%
\pgfsetdash{}{0pt}%
\pgfsys@defobject{currentmarker}{\pgfqpoint{0.000000in}{0.000000in}}{\pgfqpoint{0.034722in}{0.000000in}}{%
\pgfpathmoveto{\pgfqpoint{0.000000in}{0.000000in}}%
\pgfpathlineto{\pgfqpoint{0.034722in}{0.000000in}}%
\pgfusepath{stroke,fill}%
}%
\begin{pgfscope}%
\pgfsys@transformshift{1.396958in}{1.435966in}%
\pgfsys@useobject{currentmarker}{}%
\end{pgfscope}%
\end{pgfscope}%
\begin{pgfscope}%
\definecolor{textcolor}{rgb}{0.000000,0.000000,0.000000}%
\pgfsetstrokecolor{textcolor}%
\pgfsetfillcolor{textcolor}%
\pgftext[x=0.876267in, y=1.340995in, left, base]{\color{textcolor}\sffamily\fontsize{18.000000}{21.600000}\selectfont $\displaystyle -1.0$}%
\end{pgfscope}%
\begin{pgfscope}%
\pgfpathrectangle{\pgfqpoint{1.396958in}{1.247073in}}{\pgfqpoint{10.524301in}{6.674186in}}%
\pgfusepath{clip}%
\pgfsetrectcap%
\pgfsetroundjoin%
\pgfsetlinewidth{0.501875pt}%
\definecolor{currentstroke}{rgb}{0.000000,0.000000,0.000000}%
\pgfsetstrokecolor{currentstroke}%
\pgfsetstrokeopacity{0.100000}%
\pgfsetdash{}{0pt}%
\pgfpathmoveto{\pgfqpoint{1.396958in}{3.010066in}}%
\pgfpathlineto{\pgfqpoint{11.921260in}{3.010066in}}%
\pgfusepath{stroke}%
\end{pgfscope}%
\begin{pgfscope}%
\pgfsetbuttcap%
\pgfsetroundjoin%
\definecolor{currentfill}{rgb}{0.000000,0.000000,0.000000}%
\pgfsetfillcolor{currentfill}%
\pgfsetlinewidth{0.501875pt}%
\definecolor{currentstroke}{rgb}{0.000000,0.000000,0.000000}%
\pgfsetstrokecolor{currentstroke}%
\pgfsetdash{}{0pt}%
\pgfsys@defobject{currentmarker}{\pgfqpoint{0.000000in}{0.000000in}}{\pgfqpoint{0.034722in}{0.000000in}}{%
\pgfpathmoveto{\pgfqpoint{0.000000in}{0.000000in}}%
\pgfpathlineto{\pgfqpoint{0.034722in}{0.000000in}}%
\pgfusepath{stroke,fill}%
}%
\begin{pgfscope}%
\pgfsys@transformshift{1.396958in}{3.010066in}%
\pgfsys@useobject{currentmarker}{}%
\end{pgfscope}%
\end{pgfscope}%
\begin{pgfscope}%
\definecolor{textcolor}{rgb}{0.000000,0.000000,0.000000}%
\pgfsetstrokecolor{textcolor}%
\pgfsetfillcolor{textcolor}%
\pgftext[x=0.876267in, y=2.915095in, left, base]{\color{textcolor}\sffamily\fontsize{18.000000}{21.600000}\selectfont $\displaystyle -0.5$}%
\end{pgfscope}%
\begin{pgfscope}%
\pgfpathrectangle{\pgfqpoint{1.396958in}{1.247073in}}{\pgfqpoint{10.524301in}{6.674186in}}%
\pgfusepath{clip}%
\pgfsetrectcap%
\pgfsetroundjoin%
\pgfsetlinewidth{0.501875pt}%
\definecolor{currentstroke}{rgb}{0.000000,0.000000,0.000000}%
\pgfsetstrokecolor{currentstroke}%
\pgfsetstrokeopacity{0.100000}%
\pgfsetdash{}{0pt}%
\pgfpathmoveto{\pgfqpoint{1.396958in}{4.584167in}}%
\pgfpathlineto{\pgfqpoint{11.921260in}{4.584167in}}%
\pgfusepath{stroke}%
\end{pgfscope}%
\begin{pgfscope}%
\pgfsetbuttcap%
\pgfsetroundjoin%
\definecolor{currentfill}{rgb}{0.000000,0.000000,0.000000}%
\pgfsetfillcolor{currentfill}%
\pgfsetlinewidth{0.501875pt}%
\definecolor{currentstroke}{rgb}{0.000000,0.000000,0.000000}%
\pgfsetstrokecolor{currentstroke}%
\pgfsetdash{}{0pt}%
\pgfsys@defobject{currentmarker}{\pgfqpoint{0.000000in}{0.000000in}}{\pgfqpoint{0.034722in}{0.000000in}}{%
\pgfpathmoveto{\pgfqpoint{0.000000in}{0.000000in}}%
\pgfpathlineto{\pgfqpoint{0.034722in}{0.000000in}}%
\pgfusepath{stroke,fill}%
}%
\begin{pgfscope}%
\pgfsys@transformshift{1.396958in}{4.584167in}%
\pgfsys@useobject{currentmarker}{}%
\end{pgfscope}%
\end{pgfscope}%
\begin{pgfscope}%
\definecolor{textcolor}{rgb}{0.000000,0.000000,0.000000}%
\pgfsetstrokecolor{textcolor}%
\pgfsetfillcolor{textcolor}%
\pgftext[x=1.062934in, y=4.489196in, left, base]{\color{textcolor}\sffamily\fontsize{18.000000}{21.600000}\selectfont $\displaystyle 0.0$}%
\end{pgfscope}%
\begin{pgfscope}%
\pgfpathrectangle{\pgfqpoint{1.396958in}{1.247073in}}{\pgfqpoint{10.524301in}{6.674186in}}%
\pgfusepath{clip}%
\pgfsetrectcap%
\pgfsetroundjoin%
\pgfsetlinewidth{0.501875pt}%
\definecolor{currentstroke}{rgb}{0.000000,0.000000,0.000000}%
\pgfsetstrokecolor{currentstroke}%
\pgfsetstrokeopacity{0.100000}%
\pgfsetdash{}{0pt}%
\pgfpathmoveto{\pgfqpoint{1.396958in}{6.158267in}}%
\pgfpathlineto{\pgfqpoint{11.921260in}{6.158267in}}%
\pgfusepath{stroke}%
\end{pgfscope}%
\begin{pgfscope}%
\pgfsetbuttcap%
\pgfsetroundjoin%
\definecolor{currentfill}{rgb}{0.000000,0.000000,0.000000}%
\pgfsetfillcolor{currentfill}%
\pgfsetlinewidth{0.501875pt}%
\definecolor{currentstroke}{rgb}{0.000000,0.000000,0.000000}%
\pgfsetstrokecolor{currentstroke}%
\pgfsetdash{}{0pt}%
\pgfsys@defobject{currentmarker}{\pgfqpoint{0.000000in}{0.000000in}}{\pgfqpoint{0.034722in}{0.000000in}}{%
\pgfpathmoveto{\pgfqpoint{0.000000in}{0.000000in}}%
\pgfpathlineto{\pgfqpoint{0.034722in}{0.000000in}}%
\pgfusepath{stroke,fill}%
}%
\begin{pgfscope}%
\pgfsys@transformshift{1.396958in}{6.158267in}%
\pgfsys@useobject{currentmarker}{}%
\end{pgfscope}%
\end{pgfscope}%
\begin{pgfscope}%
\definecolor{textcolor}{rgb}{0.000000,0.000000,0.000000}%
\pgfsetstrokecolor{textcolor}%
\pgfsetfillcolor{textcolor}%
\pgftext[x=1.062934in, y=6.063297in, left, base]{\color{textcolor}\sffamily\fontsize{18.000000}{21.600000}\selectfont $\displaystyle 0.5$}%
\end{pgfscope}%
\begin{pgfscope}%
\pgfpathrectangle{\pgfqpoint{1.396958in}{1.247073in}}{\pgfqpoint{10.524301in}{6.674186in}}%
\pgfusepath{clip}%
\pgfsetrectcap%
\pgfsetroundjoin%
\pgfsetlinewidth{0.501875pt}%
\definecolor{currentstroke}{rgb}{0.000000,0.000000,0.000000}%
\pgfsetstrokecolor{currentstroke}%
\pgfsetstrokeopacity{0.100000}%
\pgfsetdash{}{0pt}%
\pgfpathmoveto{\pgfqpoint{1.396958in}{7.732368in}}%
\pgfpathlineto{\pgfqpoint{11.921260in}{7.732368in}}%
\pgfusepath{stroke}%
\end{pgfscope}%
\begin{pgfscope}%
\pgfsetbuttcap%
\pgfsetroundjoin%
\definecolor{currentfill}{rgb}{0.000000,0.000000,0.000000}%
\pgfsetfillcolor{currentfill}%
\pgfsetlinewidth{0.501875pt}%
\definecolor{currentstroke}{rgb}{0.000000,0.000000,0.000000}%
\pgfsetstrokecolor{currentstroke}%
\pgfsetdash{}{0pt}%
\pgfsys@defobject{currentmarker}{\pgfqpoint{0.000000in}{0.000000in}}{\pgfqpoint{0.034722in}{0.000000in}}{%
\pgfpathmoveto{\pgfqpoint{0.000000in}{0.000000in}}%
\pgfpathlineto{\pgfqpoint{0.034722in}{0.000000in}}%
\pgfusepath{stroke,fill}%
}%
\begin{pgfscope}%
\pgfsys@transformshift{1.396958in}{7.732368in}%
\pgfsys@useobject{currentmarker}{}%
\end{pgfscope}%
\end{pgfscope}%
\begin{pgfscope}%
\definecolor{textcolor}{rgb}{0.000000,0.000000,0.000000}%
\pgfsetstrokecolor{textcolor}%
\pgfsetfillcolor{textcolor}%
\pgftext[x=1.062934in, y=7.637397in, left, base]{\color{textcolor}\sffamily\fontsize{18.000000}{21.600000}\selectfont $\displaystyle 1.0$}%
\end{pgfscope}%
\begin{pgfscope}%
\pgfpathrectangle{\pgfqpoint{1.396958in}{1.247073in}}{\pgfqpoint{10.524301in}{6.674186in}}%
\pgfusepath{clip}%
\pgfsetbuttcap%
\pgfsetroundjoin%
\pgfsetlinewidth{1.003750pt}%
\definecolor{currentstroke}{rgb}{0.000000,0.605603,0.978680}%
\pgfsetstrokecolor{currentstroke}%
\pgfsetdash{}{0pt}%
\pgfpathmoveto{\pgfqpoint{1.694816in}{4.578806in}}%
\pgfpathlineto{\pgfqpoint{1.927517in}{4.116936in}}%
\pgfpathlineto{\pgfqpoint{2.082651in}{3.814031in}}%
\pgfpathlineto{\pgfqpoint{2.237786in}{3.518543in}}%
\pgfpathlineto{\pgfqpoint{2.315353in}{3.374473in}}%
\pgfpathlineto{\pgfqpoint{2.392920in}{3.233317in}}%
\pgfpathlineto{\pgfqpoint{2.470487in}{3.095415in}}%
\pgfpathlineto{\pgfqpoint{2.548054in}{2.961101in}}%
\pgfpathlineto{\pgfqpoint{2.625621in}{2.830696in}}%
\pgfpathlineto{\pgfqpoint{2.703188in}{2.704515in}}%
\pgfpathlineto{\pgfqpoint{2.780755in}{2.582863in}}%
\pgfpathlineto{\pgfqpoint{2.858322in}{2.466032in}}%
\pgfpathlineto{\pgfqpoint{2.935889in}{2.354304in}}%
\pgfpathlineto{\pgfqpoint{3.013456in}{2.247948in}}%
\pgfpathlineto{\pgfqpoint{3.091023in}{2.147220in}}%
\pgfpathlineto{\pgfqpoint{3.168591in}{2.052362in}}%
\pgfpathlineto{\pgfqpoint{3.246158in}{1.963605in}}%
\pgfpathlineto{\pgfqpoint{3.323725in}{1.881160in}}%
\pgfpathlineto{\pgfqpoint{3.401292in}{1.805227in}}%
\pgfpathlineto{\pgfqpoint{3.478859in}{1.735989in}}%
\pgfpathlineto{\pgfqpoint{3.556426in}{1.673612in}}%
\pgfpathlineto{\pgfqpoint{3.633993in}{1.618247in}}%
\pgfpathlineto{\pgfqpoint{3.711560in}{1.570027in}}%
\pgfpathlineto{\pgfqpoint{3.789127in}{1.529069in}}%
\pgfpathlineto{\pgfqpoint{3.866694in}{1.495470in}}%
\pgfpathlineto{\pgfqpoint{3.944261in}{1.469313in}}%
\pgfpathlineto{\pgfqpoint{4.021828in}{1.450659in}}%
\pgfpathlineto{\pgfqpoint{4.099396in}{1.439554in}}%
\pgfpathlineto{\pgfqpoint{4.176963in}{1.436025in}}%
\pgfpathlineto{\pgfqpoint{4.254530in}{1.440080in}}%
\pgfpathlineto{\pgfqpoint{4.332097in}{1.451710in}}%
\pgfpathlineto{\pgfqpoint{4.409664in}{1.470886in}}%
\pgfpathlineto{\pgfqpoint{4.487231in}{1.497562in}}%
\pgfpathlineto{\pgfqpoint{4.564798in}{1.531674in}}%
\pgfpathlineto{\pgfqpoint{4.642365in}{1.573139in}}%
\pgfpathlineto{\pgfqpoint{4.719932in}{1.621859in}}%
\pgfpathlineto{\pgfqpoint{4.797499in}{1.677715in}}%
\pgfpathlineto{\pgfqpoint{4.875066in}{1.740572in}}%
\pgfpathlineto{\pgfqpoint{4.952633in}{1.810281in}}%
\pgfpathlineto{\pgfqpoint{5.030200in}{1.886672in}}%
\pgfpathlineto{\pgfqpoint{5.107768in}{1.969561in}}%
\pgfpathlineto{\pgfqpoint{5.185335in}{2.058749in}}%
\pgfpathlineto{\pgfqpoint{5.262902in}{2.154021in}}%
\pgfpathlineto{\pgfqpoint{5.340469in}{2.255148in}}%
\pgfpathlineto{\pgfqpoint{5.418036in}{2.361885in}}%
\pgfpathlineto{\pgfqpoint{5.495603in}{2.473976in}}%
\pgfpathlineto{\pgfqpoint{5.573170in}{2.591151in}}%
\pgfpathlineto{\pgfqpoint{5.650737in}{2.713127in}}%
\pgfpathlineto{\pgfqpoint{5.728304in}{2.839610in}}%
\pgfpathlineto{\pgfqpoint{5.805871in}{2.970296in}}%
\pgfpathlineto{\pgfqpoint{5.883438in}{3.104871in}}%
\pgfpathlineto{\pgfqpoint{5.961005in}{3.243009in}}%
\pgfpathlineto{\pgfqpoint{6.038573in}{3.384378in}}%
\pgfpathlineto{\pgfqpoint{6.116140in}{3.528637in}}%
\pgfpathlineto{\pgfqpoint{6.271274in}{3.824431in}}%
\pgfpathlineto{\pgfqpoint{6.426408in}{4.127541in}}%
\pgfpathlineto{\pgfqpoint{6.659109in}{4.589527in}}%
\pgfpathlineto{\pgfqpoint{6.891810in}{5.051398in}}%
\pgfpathlineto{\pgfqpoint{7.046945in}{5.354303in}}%
\pgfpathlineto{\pgfqpoint{7.202079in}{5.649791in}}%
\pgfpathlineto{\pgfqpoint{7.279646in}{5.793861in}}%
\pgfpathlineto{\pgfqpoint{7.357213in}{5.935017in}}%
\pgfpathlineto{\pgfqpoint{7.434780in}{6.072918in}}%
\pgfpathlineto{\pgfqpoint{7.512347in}{6.207233in}}%
\pgfpathlineto{\pgfqpoint{7.589914in}{6.337638in}}%
\pgfpathlineto{\pgfqpoint{7.667481in}{6.463818in}}%
\pgfpathlineto{\pgfqpoint{7.745048in}{6.585470in}}%
\pgfpathlineto{\pgfqpoint{7.822615in}{6.702301in}}%
\pgfpathlineto{\pgfqpoint{7.900182in}{6.814029in}}%
\pgfpathlineto{\pgfqpoint{7.977750in}{6.920386in}}%
\pgfpathlineto{\pgfqpoint{8.055317in}{7.021114in}}%
\pgfpathlineto{\pgfqpoint{8.132884in}{7.115971in}}%
\pgfpathlineto{\pgfqpoint{8.210451in}{7.204729in}}%
\pgfpathlineto{\pgfqpoint{8.288018in}{7.287174in}}%
\pgfpathlineto{\pgfqpoint{8.365585in}{7.363107in}}%
\pgfpathlineto{\pgfqpoint{8.443152in}{7.432345in}}%
\pgfpathlineto{\pgfqpoint{8.520719in}{7.494722in}}%
\pgfpathlineto{\pgfqpoint{8.598286in}{7.550087in}}%
\pgfpathlineto{\pgfqpoint{8.675853in}{7.598306in}}%
\pgfpathlineto{\pgfqpoint{8.753420in}{7.639265in}}%
\pgfpathlineto{\pgfqpoint{8.830987in}{7.672863in}}%
\pgfpathlineto{\pgfqpoint{8.908554in}{7.699021in}}%
\pgfpathlineto{\pgfqpoint{8.986122in}{7.717674in}}%
\pgfpathlineto{\pgfqpoint{9.063689in}{7.728779in}}%
\pgfpathlineto{\pgfqpoint{9.141256in}{7.732308in}}%
\pgfpathlineto{\pgfqpoint{9.218823in}{7.728253in}}%
\pgfpathlineto{\pgfqpoint{9.296390in}{7.716624in}}%
\pgfpathlineto{\pgfqpoint{9.373957in}{7.697448in}}%
\pgfpathlineto{\pgfqpoint{9.451524in}{7.670772in}}%
\pgfpathlineto{\pgfqpoint{9.529091in}{7.636660in}}%
\pgfpathlineto{\pgfqpoint{9.606658in}{7.595194in}}%
\pgfpathlineto{\pgfqpoint{9.684225in}{7.546475in}}%
\pgfpathlineto{\pgfqpoint{9.761792in}{7.490619in}}%
\pgfpathlineto{\pgfqpoint{9.839359in}{7.427761in}}%
\pgfpathlineto{\pgfqpoint{9.916927in}{7.358053in}}%
\pgfpathlineto{\pgfqpoint{9.994494in}{7.281662in}}%
\pgfpathlineto{\pgfqpoint{10.072061in}{7.198772in}}%
\pgfpathlineto{\pgfqpoint{10.149628in}{7.109584in}}%
\pgfpathlineto{\pgfqpoint{10.227195in}{7.014312in}}%
\pgfpathlineto{\pgfqpoint{10.304762in}{6.913186in}}%
\pgfpathlineto{\pgfqpoint{10.382329in}{6.806448in}}%
\pgfpathlineto{\pgfqpoint{10.459896in}{6.694357in}}%
\pgfpathlineto{\pgfqpoint{10.537463in}{6.577183in}}%
\pgfpathlineto{\pgfqpoint{10.615030in}{6.455207in}}%
\pgfpathlineto{\pgfqpoint{10.692597in}{6.328723in}}%
\pgfpathlineto{\pgfqpoint{10.770164in}{6.198037in}}%
\pgfpathlineto{\pgfqpoint{10.847731in}{6.063463in}}%
\pgfpathlineto{\pgfqpoint{10.925299in}{5.925325in}}%
\pgfpathlineto{\pgfqpoint{11.002866in}{5.783956in}}%
\pgfpathlineto{\pgfqpoint{11.080433in}{5.639696in}}%
\pgfpathlineto{\pgfqpoint{11.235567in}{5.343903in}}%
\pgfpathlineto{\pgfqpoint{11.390701in}{5.040792in}}%
\pgfpathlineto{\pgfqpoint{11.623402in}{4.578806in}}%
\pgfpathlineto{\pgfqpoint{11.623402in}{4.578806in}}%
\pgfusepath{stroke}%
\end{pgfscope}%
\begin{pgfscope}%
\pgfpathrectangle{\pgfqpoint{1.396958in}{1.247073in}}{\pgfqpoint{10.524301in}{6.674186in}}%
\pgfusepath{clip}%
\pgfsetbuttcap%
\pgfsetroundjoin%
\pgfsetlinewidth{1.003750pt}%
\definecolor{currentstroke}{rgb}{0.888874,0.435649,0.278123}%
\pgfsetstrokecolor{currentstroke}%
\pgfsetdash{}{0pt}%
\pgfpathmoveto{\pgfqpoint{1.694816in}{4.584167in}}%
\pgfpathlineto{\pgfqpoint{1.927517in}{4.122230in}}%
\pgfpathlineto{\pgfqpoint{2.082651in}{3.819216in}}%
\pgfpathlineto{\pgfqpoint{2.237786in}{3.523570in}}%
\pgfpathlineto{\pgfqpoint{2.315353in}{3.379402in}}%
\pgfpathlineto{\pgfqpoint{2.392920in}{3.238137in}}%
\pgfpathlineto{\pgfqpoint{2.470487in}{3.100115in}}%
\pgfpathlineto{\pgfqpoint{2.548054in}{2.965668in}}%
\pgfpathlineto{\pgfqpoint{2.625621in}{2.835120in}}%
\pgfpathlineto{\pgfqpoint{2.703188in}{2.708785in}}%
\pgfpathlineto{\pgfqpoint{2.780755in}{2.586969in}}%
\pgfpathlineto{\pgfqpoint{2.858322in}{2.469964in}}%
\pgfpathlineto{\pgfqpoint{2.935889in}{2.358052in}}%
\pgfpathlineto{\pgfqpoint{3.013456in}{2.251504in}}%
\pgfpathlineto{\pgfqpoint{3.091023in}{2.150574in}}%
\pgfpathlineto{\pgfqpoint{3.168591in}{2.055508in}}%
\pgfpathlineto{\pgfqpoint{3.246158in}{1.966533in}}%
\pgfpathlineto{\pgfqpoint{3.323725in}{1.883865in}}%
\pgfpathlineto{\pgfqpoint{3.401292in}{1.807701in}}%
\pgfpathlineto{\pgfqpoint{3.478859in}{1.738227in}}%
\pgfpathlineto{\pgfqpoint{3.556426in}{1.675608in}}%
\pgfpathlineto{\pgfqpoint{3.633993in}{1.619997in}}%
\pgfpathlineto{\pgfqpoint{3.711560in}{1.571526in}}%
\pgfpathlineto{\pgfqpoint{3.789127in}{1.530313in}}%
\pgfpathlineto{\pgfqpoint{3.866694in}{1.496457in}}%
\pgfpathlineto{\pgfqpoint{3.944261in}{1.470040in}}%
\pgfpathlineto{\pgfqpoint{4.021828in}{1.451125in}}%
\pgfpathlineto{\pgfqpoint{4.099396in}{1.439758in}}%
\pgfpathlineto{\pgfqpoint{4.176963in}{1.435966in}}%
\pgfpathlineto{\pgfqpoint{4.254530in}{1.439758in}}%
\pgfpathlineto{\pgfqpoint{4.332097in}{1.451125in}}%
\pgfpathlineto{\pgfqpoint{4.409664in}{1.470040in}}%
\pgfpathlineto{\pgfqpoint{4.487231in}{1.496457in}}%
\pgfpathlineto{\pgfqpoint{4.564798in}{1.530313in}}%
\pgfpathlineto{\pgfqpoint{4.642365in}{1.571526in}}%
\pgfpathlineto{\pgfqpoint{4.719932in}{1.619997in}}%
\pgfpathlineto{\pgfqpoint{4.797499in}{1.675608in}}%
\pgfpathlineto{\pgfqpoint{4.875066in}{1.738227in}}%
\pgfpathlineto{\pgfqpoint{4.952633in}{1.807701in}}%
\pgfpathlineto{\pgfqpoint{5.030200in}{1.883865in}}%
\pgfpathlineto{\pgfqpoint{5.107768in}{1.966533in}}%
\pgfpathlineto{\pgfqpoint{5.185335in}{2.055508in}}%
\pgfpathlineto{\pgfqpoint{5.262902in}{2.150574in}}%
\pgfpathlineto{\pgfqpoint{5.340469in}{2.251504in}}%
\pgfpathlineto{\pgfqpoint{5.418036in}{2.358052in}}%
\pgfpathlineto{\pgfqpoint{5.495603in}{2.469964in}}%
\pgfpathlineto{\pgfqpoint{5.573170in}{2.586969in}}%
\pgfpathlineto{\pgfqpoint{5.650737in}{2.708785in}}%
\pgfpathlineto{\pgfqpoint{5.728304in}{2.835120in}}%
\pgfpathlineto{\pgfqpoint{5.805871in}{2.965668in}}%
\pgfpathlineto{\pgfqpoint{5.883438in}{3.100115in}}%
\pgfpathlineto{\pgfqpoint{5.961005in}{3.238137in}}%
\pgfpathlineto{\pgfqpoint{6.038573in}{3.379402in}}%
\pgfpathlineto{\pgfqpoint{6.116140in}{3.523570in}}%
\pgfpathlineto{\pgfqpoint{6.271274in}{3.819216in}}%
\pgfpathlineto{\pgfqpoint{6.426408in}{4.122230in}}%
\pgfpathlineto{\pgfqpoint{6.659109in}{4.584167in}}%
\pgfpathlineto{\pgfqpoint{6.891810in}{5.046104in}}%
\pgfpathlineto{\pgfqpoint{7.046945in}{5.349117in}}%
\pgfpathlineto{\pgfqpoint{7.202079in}{5.644764in}}%
\pgfpathlineto{\pgfqpoint{7.279646in}{5.788931in}}%
\pgfpathlineto{\pgfqpoint{7.357213in}{5.930196in}}%
\pgfpathlineto{\pgfqpoint{7.434780in}{6.068218in}}%
\pgfpathlineto{\pgfqpoint{7.512347in}{6.202665in}}%
\pgfpathlineto{\pgfqpoint{7.589914in}{6.333213in}}%
\pgfpathlineto{\pgfqpoint{7.667481in}{6.459548in}}%
\pgfpathlineto{\pgfqpoint{7.745048in}{6.581364in}}%
\pgfpathlineto{\pgfqpoint{7.822615in}{6.698369in}}%
\pgfpathlineto{\pgfqpoint{7.900182in}{6.810281in}}%
\pgfpathlineto{\pgfqpoint{7.977750in}{6.916830in}}%
\pgfpathlineto{\pgfqpoint{8.055317in}{7.017759in}}%
\pgfpathlineto{\pgfqpoint{8.132884in}{7.112826in}}%
\pgfpathlineto{\pgfqpoint{8.210451in}{7.201800in}}%
\pgfpathlineto{\pgfqpoint{8.288018in}{7.284469in}}%
\pgfpathlineto{\pgfqpoint{8.365585in}{7.360632in}}%
\pgfpathlineto{\pgfqpoint{8.443152in}{7.430107in}}%
\pgfpathlineto{\pgfqpoint{8.520719in}{7.492725in}}%
\pgfpathlineto{\pgfqpoint{8.598286in}{7.548337in}}%
\pgfpathlineto{\pgfqpoint{8.675853in}{7.596807in}}%
\pgfpathlineto{\pgfqpoint{8.753420in}{7.638020in}}%
\pgfpathlineto{\pgfqpoint{8.830987in}{7.671876in}}%
\pgfpathlineto{\pgfqpoint{8.908554in}{7.698293in}}%
\pgfpathlineto{\pgfqpoint{8.986122in}{7.717208in}}%
\pgfpathlineto{\pgfqpoint{9.063689in}{7.728576in}}%
\pgfpathlineto{\pgfqpoint{9.141256in}{7.732368in}}%
\pgfpathlineto{\pgfqpoint{9.218823in}{7.728576in}}%
\pgfpathlineto{\pgfqpoint{9.296390in}{7.717208in}}%
\pgfpathlineto{\pgfqpoint{9.373957in}{7.698293in}}%
\pgfpathlineto{\pgfqpoint{9.451524in}{7.671876in}}%
\pgfpathlineto{\pgfqpoint{9.529091in}{7.638020in}}%
\pgfpathlineto{\pgfqpoint{9.606658in}{7.596807in}}%
\pgfpathlineto{\pgfqpoint{9.684225in}{7.548337in}}%
\pgfpathlineto{\pgfqpoint{9.761792in}{7.492725in}}%
\pgfpathlineto{\pgfqpoint{9.839359in}{7.430107in}}%
\pgfpathlineto{\pgfqpoint{9.916927in}{7.360632in}}%
\pgfpathlineto{\pgfqpoint{9.994494in}{7.284469in}}%
\pgfpathlineto{\pgfqpoint{10.072061in}{7.201800in}}%
\pgfpathlineto{\pgfqpoint{10.149628in}{7.112826in}}%
\pgfpathlineto{\pgfqpoint{10.227195in}{7.017759in}}%
\pgfpathlineto{\pgfqpoint{10.304762in}{6.916830in}}%
\pgfpathlineto{\pgfqpoint{10.382329in}{6.810281in}}%
\pgfpathlineto{\pgfqpoint{10.459896in}{6.698369in}}%
\pgfpathlineto{\pgfqpoint{10.537463in}{6.581364in}}%
\pgfpathlineto{\pgfqpoint{10.615030in}{6.459548in}}%
\pgfpathlineto{\pgfqpoint{10.692597in}{6.333213in}}%
\pgfpathlineto{\pgfqpoint{10.770164in}{6.202665in}}%
\pgfpathlineto{\pgfqpoint{10.847731in}{6.068218in}}%
\pgfpathlineto{\pgfqpoint{10.925299in}{5.930196in}}%
\pgfpathlineto{\pgfqpoint{11.002866in}{5.788931in}}%
\pgfpathlineto{\pgfqpoint{11.080433in}{5.644764in}}%
\pgfpathlineto{\pgfqpoint{11.235567in}{5.349117in}}%
\pgfpathlineto{\pgfqpoint{11.390701in}{5.046104in}}%
\pgfpathlineto{\pgfqpoint{11.623402in}{4.584167in}}%
\pgfpathlineto{\pgfqpoint{11.623402in}{4.584167in}}%
\pgfusepath{stroke}%
\end{pgfscope}%
\begin{pgfscope}%
\pgfsetrectcap%
\pgfsetmiterjoin%
\pgfsetlinewidth{1.003750pt}%
\definecolor{currentstroke}{rgb}{0.000000,0.000000,0.000000}%
\pgfsetstrokecolor{currentstroke}%
\pgfsetdash{}{0pt}%
\pgfpathmoveto{\pgfqpoint{1.396958in}{1.247073in}}%
\pgfpathlineto{\pgfqpoint{1.396958in}{7.921260in}}%
\pgfusepath{stroke}%
\end{pgfscope}%
\begin{pgfscope}%
\pgfsetrectcap%
\pgfsetmiterjoin%
\pgfsetlinewidth{1.003750pt}%
\definecolor{currentstroke}{rgb}{0.000000,0.000000,0.000000}%
\pgfsetstrokecolor{currentstroke}%
\pgfsetdash{}{0pt}%
\pgfpathmoveto{\pgfqpoint{1.396958in}{1.247073in}}%
\pgfpathlineto{\pgfqpoint{11.921260in}{1.247073in}}%
\pgfusepath{stroke}%
\end{pgfscope}%
\begin{pgfscope}%
\pgfsetbuttcap%
\pgfsetmiterjoin%
\definecolor{currentfill}{rgb}{1.000000,1.000000,1.000000}%
\pgfsetfillcolor{currentfill}%
\pgfsetlinewidth{1.003750pt}%
\definecolor{currentstroke}{rgb}{0.000000,0.000000,0.000000}%
\pgfsetstrokecolor{currentstroke}%
\pgfsetdash{}{0pt}%
\pgfpathmoveto{\pgfqpoint{10.511589in}{6.787373in}}%
\pgfpathlineto{\pgfqpoint{11.796260in}{6.787373in}}%
\pgfpathlineto{\pgfqpoint{11.796260in}{7.796260in}}%
\pgfpathlineto{\pgfqpoint{10.511589in}{7.796260in}}%
\pgfpathclose%
\pgfusepath{stroke,fill}%
\end{pgfscope}%
\begin{pgfscope}%
\pgfsetbuttcap%
\pgfsetmiterjoin%
\pgfsetlinewidth{2.258437pt}%
\definecolor{currentstroke}{rgb}{0.000000,0.605603,0.978680}%
\pgfsetstrokecolor{currentstroke}%
\pgfsetdash{}{0pt}%
\pgfpathmoveto{\pgfqpoint{10.711589in}{7.493818in}}%
\pgfpathlineto{\pgfqpoint{11.211589in}{7.493818in}}%
\pgfusepath{stroke}%
\end{pgfscope}%
\begin{pgfscope}%
\definecolor{textcolor}{rgb}{0.000000,0.000000,0.000000}%
\pgfsetstrokecolor{textcolor}%
\pgfsetfillcolor{textcolor}%
\pgftext[x=11.411589in,y=7.406318in,left,base]{\color{textcolor}\sffamily\fontsize{18.000000}{21.600000}\selectfont $\displaystyle U$}%
\end{pgfscope}%
\begin{pgfscope}%
\pgfsetbuttcap%
\pgfsetmiterjoin%
\pgfsetlinewidth{2.258437pt}%
\definecolor{currentstroke}{rgb}{0.888874,0.435649,0.278123}%
\pgfsetstrokecolor{currentstroke}%
\pgfsetdash{}{0pt}%
\pgfpathmoveto{\pgfqpoint{10.711589in}{7.126875in}}%
\pgfpathlineto{\pgfqpoint{11.211589in}{7.126875in}}%
\pgfusepath{stroke}%
\end{pgfscope}%
\begin{pgfscope}%
\definecolor{textcolor}{rgb}{0.000000,0.000000,0.000000}%
\pgfsetstrokecolor{textcolor}%
\pgfsetfillcolor{textcolor}%
\pgftext[x=11.411589in,y=7.039375in,left,base]{\color{textcolor}\sffamily\fontsize{18.000000}{21.600000}\selectfont $\displaystyle u$}%
\end{pgfscope}%
\end{pgfpicture}%
\makeatother%
\endgroup%
}\quad
	\resizebox{0.4\linewidth}{!}{%% Creator: Matplotlib, PGF backend
%%
%% To include the figure in your LaTeX document, write
%%   \input{<filename>.pgf}
%%
%% Make sure the required packages are loaded in your preamble
%%   \usepackage{pgf}
%%
%% Figures using additional raster images can only be included by \input if
%% they are in the same directory as the main LaTeX file. For loading figures
%% from other directories you can use the `import` package
%%   \usepackage{import}
%%
%% and then include the figures with
%%   \import{<path to file>}{<filename>.pgf}
%%
%% Matplotlib used the following preamble
%%   \usepackage{fontspec}
%%   \setmainfont{DejaVuSerif.ttf}[Path=\detokenize{/Users/quejiahao/.julia/conda/3/lib/python3.9/site-packages/matplotlib/mpl-data/fonts/ttf/}]
%%   \setsansfont{DejaVuSans.ttf}[Path=\detokenize{/Users/quejiahao/.julia/conda/3/lib/python3.9/site-packages/matplotlib/mpl-data/fonts/ttf/}]
%%   \setmonofont{DejaVuSansMono.ttf}[Path=\detokenize{/Users/quejiahao/.julia/conda/3/lib/python3.9/site-packages/matplotlib/mpl-data/fonts/ttf/}]
%%
\begingroup%
\makeatletter%
\begin{pgfpicture}%
\pgfpathrectangle{\pgfpointorigin}{\pgfqpoint{12.000000in}{8.000000in}}%
\pgfusepath{use as bounding box, clip}%
\begin{pgfscope}%
\pgfsetbuttcap%
\pgfsetmiterjoin%
\definecolor{currentfill}{rgb}{1.000000,1.000000,1.000000}%
\pgfsetfillcolor{currentfill}%
\pgfsetlinewidth{0.000000pt}%
\definecolor{currentstroke}{rgb}{1.000000,1.000000,1.000000}%
\pgfsetstrokecolor{currentstroke}%
\pgfsetdash{}{0pt}%
\pgfpathmoveto{\pgfqpoint{0.000000in}{0.000000in}}%
\pgfpathlineto{\pgfqpoint{12.000000in}{0.000000in}}%
\pgfpathlineto{\pgfqpoint{12.000000in}{8.000000in}}%
\pgfpathlineto{\pgfqpoint{0.000000in}{8.000000in}}%
\pgfpathclose%
\pgfusepath{fill}%
\end{pgfscope}%
\begin{pgfscope}%
\pgfsetbuttcap%
\pgfsetmiterjoin%
\definecolor{currentfill}{rgb}{1.000000,1.000000,1.000000}%
\pgfsetfillcolor{currentfill}%
\pgfsetlinewidth{0.000000pt}%
\definecolor{currentstroke}{rgb}{0.000000,0.000000,0.000000}%
\pgfsetstrokecolor{currentstroke}%
\pgfsetstrokeopacity{0.000000}%
\pgfsetdash{}{0pt}%
\pgfpathmoveto{\pgfqpoint{1.396958in}{1.247073in}}%
\pgfpathlineto{\pgfqpoint{11.921260in}{1.247073in}}%
\pgfpathlineto{\pgfqpoint{11.921260in}{7.921260in}}%
\pgfpathlineto{\pgfqpoint{1.396958in}{7.921260in}}%
\pgfpathclose%
\pgfusepath{fill}%
\end{pgfscope}%
\begin{pgfscope}%
\pgfpathrectangle{\pgfqpoint{1.396958in}{1.247073in}}{\pgfqpoint{10.524301in}{6.674186in}}%
\pgfusepath{clip}%
\pgfsetrectcap%
\pgfsetroundjoin%
\pgfsetlinewidth{0.501875pt}%
\definecolor{currentstroke}{rgb}{0.000000,0.000000,0.000000}%
\pgfsetstrokecolor{currentstroke}%
\pgfsetstrokeopacity{0.100000}%
\pgfsetdash{}{0pt}%
\pgfpathmoveto{\pgfqpoint{1.694816in}{1.247073in}}%
\pgfpathlineto{\pgfqpoint{1.694816in}{7.921260in}}%
\pgfusepath{stroke}%
\end{pgfscope}%
\begin{pgfscope}%
\pgfsetbuttcap%
\pgfsetroundjoin%
\definecolor{currentfill}{rgb}{0.000000,0.000000,0.000000}%
\pgfsetfillcolor{currentfill}%
\pgfsetlinewidth{0.501875pt}%
\definecolor{currentstroke}{rgb}{0.000000,0.000000,0.000000}%
\pgfsetstrokecolor{currentstroke}%
\pgfsetdash{}{0pt}%
\pgfsys@defobject{currentmarker}{\pgfqpoint{0.000000in}{0.000000in}}{\pgfqpoint{0.000000in}{0.034722in}}{%
\pgfpathmoveto{\pgfqpoint{0.000000in}{0.000000in}}%
\pgfpathlineto{\pgfqpoint{0.000000in}{0.034722in}}%
\pgfusepath{stroke,fill}%
}%
\begin{pgfscope}%
\pgfsys@transformshift{1.694816in}{1.247073in}%
\pgfsys@useobject{currentmarker}{}%
\end{pgfscope}%
\end{pgfscope}%
\begin{pgfscope}%
\definecolor{textcolor}{rgb}{0.000000,0.000000,0.000000}%
\pgfsetstrokecolor{textcolor}%
\pgfsetfillcolor{textcolor}%
\pgftext[x=1.694816in,y=1.198462in,,top]{\color{textcolor}\sffamily\fontsize{18.000000}{21.600000}\selectfont $\displaystyle 0$}%
\end{pgfscope}%
\begin{pgfscope}%
\pgfpathrectangle{\pgfqpoint{1.396958in}{1.247073in}}{\pgfqpoint{10.524301in}{6.674186in}}%
\pgfusepath{clip}%
\pgfsetrectcap%
\pgfsetroundjoin%
\pgfsetlinewidth{0.501875pt}%
\definecolor{currentstroke}{rgb}{0.000000,0.000000,0.000000}%
\pgfsetstrokecolor{currentstroke}%
\pgfsetstrokeopacity{0.100000}%
\pgfsetdash{}{0pt}%
\pgfpathmoveto{\pgfqpoint{3.275000in}{1.247073in}}%
\pgfpathlineto{\pgfqpoint{3.275000in}{7.921260in}}%
\pgfusepath{stroke}%
\end{pgfscope}%
\begin{pgfscope}%
\pgfsetbuttcap%
\pgfsetroundjoin%
\definecolor{currentfill}{rgb}{0.000000,0.000000,0.000000}%
\pgfsetfillcolor{currentfill}%
\pgfsetlinewidth{0.501875pt}%
\definecolor{currentstroke}{rgb}{0.000000,0.000000,0.000000}%
\pgfsetstrokecolor{currentstroke}%
\pgfsetdash{}{0pt}%
\pgfsys@defobject{currentmarker}{\pgfqpoint{0.000000in}{0.000000in}}{\pgfqpoint{0.000000in}{0.034722in}}{%
\pgfpathmoveto{\pgfqpoint{0.000000in}{0.000000in}}%
\pgfpathlineto{\pgfqpoint{0.000000in}{0.034722in}}%
\pgfusepath{stroke,fill}%
}%
\begin{pgfscope}%
\pgfsys@transformshift{3.275000in}{1.247073in}%
\pgfsys@useobject{currentmarker}{}%
\end{pgfscope}%
\end{pgfscope}%
\begin{pgfscope}%
\definecolor{textcolor}{rgb}{0.000000,0.000000,0.000000}%
\pgfsetstrokecolor{textcolor}%
\pgfsetfillcolor{textcolor}%
\pgftext[x=3.275000in,y=1.198462in,,top]{\color{textcolor}\sffamily\fontsize{18.000000}{21.600000}\selectfont $\displaystyle 1$}%
\end{pgfscope}%
\begin{pgfscope}%
\pgfpathrectangle{\pgfqpoint{1.396958in}{1.247073in}}{\pgfqpoint{10.524301in}{6.674186in}}%
\pgfusepath{clip}%
\pgfsetrectcap%
\pgfsetroundjoin%
\pgfsetlinewidth{0.501875pt}%
\definecolor{currentstroke}{rgb}{0.000000,0.000000,0.000000}%
\pgfsetstrokecolor{currentstroke}%
\pgfsetstrokeopacity{0.100000}%
\pgfsetdash{}{0pt}%
\pgfpathmoveto{\pgfqpoint{4.855183in}{1.247073in}}%
\pgfpathlineto{\pgfqpoint{4.855183in}{7.921260in}}%
\pgfusepath{stroke}%
\end{pgfscope}%
\begin{pgfscope}%
\pgfsetbuttcap%
\pgfsetroundjoin%
\definecolor{currentfill}{rgb}{0.000000,0.000000,0.000000}%
\pgfsetfillcolor{currentfill}%
\pgfsetlinewidth{0.501875pt}%
\definecolor{currentstroke}{rgb}{0.000000,0.000000,0.000000}%
\pgfsetstrokecolor{currentstroke}%
\pgfsetdash{}{0pt}%
\pgfsys@defobject{currentmarker}{\pgfqpoint{0.000000in}{0.000000in}}{\pgfqpoint{0.000000in}{0.034722in}}{%
\pgfpathmoveto{\pgfqpoint{0.000000in}{0.000000in}}%
\pgfpathlineto{\pgfqpoint{0.000000in}{0.034722in}}%
\pgfusepath{stroke,fill}%
}%
\begin{pgfscope}%
\pgfsys@transformshift{4.855183in}{1.247073in}%
\pgfsys@useobject{currentmarker}{}%
\end{pgfscope}%
\end{pgfscope}%
\begin{pgfscope}%
\definecolor{textcolor}{rgb}{0.000000,0.000000,0.000000}%
\pgfsetstrokecolor{textcolor}%
\pgfsetfillcolor{textcolor}%
\pgftext[x=4.855183in,y=1.198462in,,top]{\color{textcolor}\sffamily\fontsize{18.000000}{21.600000}\selectfont $\displaystyle 2$}%
\end{pgfscope}%
\begin{pgfscope}%
\pgfpathrectangle{\pgfqpoint{1.396958in}{1.247073in}}{\pgfqpoint{10.524301in}{6.674186in}}%
\pgfusepath{clip}%
\pgfsetrectcap%
\pgfsetroundjoin%
\pgfsetlinewidth{0.501875pt}%
\definecolor{currentstroke}{rgb}{0.000000,0.000000,0.000000}%
\pgfsetstrokecolor{currentstroke}%
\pgfsetstrokeopacity{0.100000}%
\pgfsetdash{}{0pt}%
\pgfpathmoveto{\pgfqpoint{6.435367in}{1.247073in}}%
\pgfpathlineto{\pgfqpoint{6.435367in}{7.921260in}}%
\pgfusepath{stroke}%
\end{pgfscope}%
\begin{pgfscope}%
\pgfsetbuttcap%
\pgfsetroundjoin%
\definecolor{currentfill}{rgb}{0.000000,0.000000,0.000000}%
\pgfsetfillcolor{currentfill}%
\pgfsetlinewidth{0.501875pt}%
\definecolor{currentstroke}{rgb}{0.000000,0.000000,0.000000}%
\pgfsetstrokecolor{currentstroke}%
\pgfsetdash{}{0pt}%
\pgfsys@defobject{currentmarker}{\pgfqpoint{0.000000in}{0.000000in}}{\pgfqpoint{0.000000in}{0.034722in}}{%
\pgfpathmoveto{\pgfqpoint{0.000000in}{0.000000in}}%
\pgfpathlineto{\pgfqpoint{0.000000in}{0.034722in}}%
\pgfusepath{stroke,fill}%
}%
\begin{pgfscope}%
\pgfsys@transformshift{6.435367in}{1.247073in}%
\pgfsys@useobject{currentmarker}{}%
\end{pgfscope}%
\end{pgfscope}%
\begin{pgfscope}%
\definecolor{textcolor}{rgb}{0.000000,0.000000,0.000000}%
\pgfsetstrokecolor{textcolor}%
\pgfsetfillcolor{textcolor}%
\pgftext[x=6.435367in,y=1.198462in,,top]{\color{textcolor}\sffamily\fontsize{18.000000}{21.600000}\selectfont $\displaystyle 3$}%
\end{pgfscope}%
\begin{pgfscope}%
\pgfpathrectangle{\pgfqpoint{1.396958in}{1.247073in}}{\pgfqpoint{10.524301in}{6.674186in}}%
\pgfusepath{clip}%
\pgfsetrectcap%
\pgfsetroundjoin%
\pgfsetlinewidth{0.501875pt}%
\definecolor{currentstroke}{rgb}{0.000000,0.000000,0.000000}%
\pgfsetstrokecolor{currentstroke}%
\pgfsetstrokeopacity{0.100000}%
\pgfsetdash{}{0pt}%
\pgfpathmoveto{\pgfqpoint{8.015550in}{1.247073in}}%
\pgfpathlineto{\pgfqpoint{8.015550in}{7.921260in}}%
\pgfusepath{stroke}%
\end{pgfscope}%
\begin{pgfscope}%
\pgfsetbuttcap%
\pgfsetroundjoin%
\definecolor{currentfill}{rgb}{0.000000,0.000000,0.000000}%
\pgfsetfillcolor{currentfill}%
\pgfsetlinewidth{0.501875pt}%
\definecolor{currentstroke}{rgb}{0.000000,0.000000,0.000000}%
\pgfsetstrokecolor{currentstroke}%
\pgfsetdash{}{0pt}%
\pgfsys@defobject{currentmarker}{\pgfqpoint{0.000000in}{0.000000in}}{\pgfqpoint{0.000000in}{0.034722in}}{%
\pgfpathmoveto{\pgfqpoint{0.000000in}{0.000000in}}%
\pgfpathlineto{\pgfqpoint{0.000000in}{0.034722in}}%
\pgfusepath{stroke,fill}%
}%
\begin{pgfscope}%
\pgfsys@transformshift{8.015550in}{1.247073in}%
\pgfsys@useobject{currentmarker}{}%
\end{pgfscope}%
\end{pgfscope}%
\begin{pgfscope}%
\definecolor{textcolor}{rgb}{0.000000,0.000000,0.000000}%
\pgfsetstrokecolor{textcolor}%
\pgfsetfillcolor{textcolor}%
\pgftext[x=8.015550in,y=1.198462in,,top]{\color{textcolor}\sffamily\fontsize{18.000000}{21.600000}\selectfont $\displaystyle 4$}%
\end{pgfscope}%
\begin{pgfscope}%
\pgfpathrectangle{\pgfqpoint{1.396958in}{1.247073in}}{\pgfqpoint{10.524301in}{6.674186in}}%
\pgfusepath{clip}%
\pgfsetrectcap%
\pgfsetroundjoin%
\pgfsetlinewidth{0.501875pt}%
\definecolor{currentstroke}{rgb}{0.000000,0.000000,0.000000}%
\pgfsetstrokecolor{currentstroke}%
\pgfsetstrokeopacity{0.100000}%
\pgfsetdash{}{0pt}%
\pgfpathmoveto{\pgfqpoint{9.595734in}{1.247073in}}%
\pgfpathlineto{\pgfqpoint{9.595734in}{7.921260in}}%
\pgfusepath{stroke}%
\end{pgfscope}%
\begin{pgfscope}%
\pgfsetbuttcap%
\pgfsetroundjoin%
\definecolor{currentfill}{rgb}{0.000000,0.000000,0.000000}%
\pgfsetfillcolor{currentfill}%
\pgfsetlinewidth{0.501875pt}%
\definecolor{currentstroke}{rgb}{0.000000,0.000000,0.000000}%
\pgfsetstrokecolor{currentstroke}%
\pgfsetdash{}{0pt}%
\pgfsys@defobject{currentmarker}{\pgfqpoint{0.000000in}{0.000000in}}{\pgfqpoint{0.000000in}{0.034722in}}{%
\pgfpathmoveto{\pgfqpoint{0.000000in}{0.000000in}}%
\pgfpathlineto{\pgfqpoint{0.000000in}{0.034722in}}%
\pgfusepath{stroke,fill}%
}%
\begin{pgfscope}%
\pgfsys@transformshift{9.595734in}{1.247073in}%
\pgfsys@useobject{currentmarker}{}%
\end{pgfscope}%
\end{pgfscope}%
\begin{pgfscope}%
\definecolor{textcolor}{rgb}{0.000000,0.000000,0.000000}%
\pgfsetstrokecolor{textcolor}%
\pgfsetfillcolor{textcolor}%
\pgftext[x=9.595734in,y=1.198462in,,top]{\color{textcolor}\sffamily\fontsize{18.000000}{21.600000}\selectfont $\displaystyle 5$}%
\end{pgfscope}%
\begin{pgfscope}%
\pgfpathrectangle{\pgfqpoint{1.396958in}{1.247073in}}{\pgfqpoint{10.524301in}{6.674186in}}%
\pgfusepath{clip}%
\pgfsetrectcap%
\pgfsetroundjoin%
\pgfsetlinewidth{0.501875pt}%
\definecolor{currentstroke}{rgb}{0.000000,0.000000,0.000000}%
\pgfsetstrokecolor{currentstroke}%
\pgfsetstrokeopacity{0.100000}%
\pgfsetdash{}{0pt}%
\pgfpathmoveto{\pgfqpoint{11.175917in}{1.247073in}}%
\pgfpathlineto{\pgfqpoint{11.175917in}{7.921260in}}%
\pgfusepath{stroke}%
\end{pgfscope}%
\begin{pgfscope}%
\pgfsetbuttcap%
\pgfsetroundjoin%
\definecolor{currentfill}{rgb}{0.000000,0.000000,0.000000}%
\pgfsetfillcolor{currentfill}%
\pgfsetlinewidth{0.501875pt}%
\definecolor{currentstroke}{rgb}{0.000000,0.000000,0.000000}%
\pgfsetstrokecolor{currentstroke}%
\pgfsetdash{}{0pt}%
\pgfsys@defobject{currentmarker}{\pgfqpoint{0.000000in}{0.000000in}}{\pgfqpoint{0.000000in}{0.034722in}}{%
\pgfpathmoveto{\pgfqpoint{0.000000in}{0.000000in}}%
\pgfpathlineto{\pgfqpoint{0.000000in}{0.034722in}}%
\pgfusepath{stroke,fill}%
}%
\begin{pgfscope}%
\pgfsys@transformshift{11.175917in}{1.247073in}%
\pgfsys@useobject{currentmarker}{}%
\end{pgfscope}%
\end{pgfscope}%
\begin{pgfscope}%
\definecolor{textcolor}{rgb}{0.000000,0.000000,0.000000}%
\pgfsetstrokecolor{textcolor}%
\pgfsetfillcolor{textcolor}%
\pgftext[x=11.175917in,y=1.198462in,,top]{\color{textcolor}\sffamily\fontsize{18.000000}{21.600000}\selectfont $\displaystyle 6$}%
\end{pgfscope}%
\begin{pgfscope}%
\definecolor{textcolor}{rgb}{0.000000,0.000000,0.000000}%
\pgfsetstrokecolor{textcolor}%
\pgfsetfillcolor{textcolor}%
\pgftext[x=6.659109in,y=0.900964in,,top]{\color{textcolor}\sffamily\fontsize{18.000000}{21.600000}\selectfont $\displaystyle x$}%
\end{pgfscope}%
\begin{pgfscope}%
\pgfpathrectangle{\pgfqpoint{1.396958in}{1.247073in}}{\pgfqpoint{10.524301in}{6.674186in}}%
\pgfusepath{clip}%
\pgfsetrectcap%
\pgfsetroundjoin%
\pgfsetlinewidth{0.501875pt}%
\definecolor{currentstroke}{rgb}{0.000000,0.000000,0.000000}%
\pgfsetstrokecolor{currentstroke}%
\pgfsetstrokeopacity{0.100000}%
\pgfsetdash{}{0pt}%
\pgfpathmoveto{\pgfqpoint{1.396958in}{1.435966in}}%
\pgfpathlineto{\pgfqpoint{11.921260in}{1.435966in}}%
\pgfusepath{stroke}%
\end{pgfscope}%
\begin{pgfscope}%
\pgfsetbuttcap%
\pgfsetroundjoin%
\definecolor{currentfill}{rgb}{0.000000,0.000000,0.000000}%
\pgfsetfillcolor{currentfill}%
\pgfsetlinewidth{0.501875pt}%
\definecolor{currentstroke}{rgb}{0.000000,0.000000,0.000000}%
\pgfsetstrokecolor{currentstroke}%
\pgfsetdash{}{0pt}%
\pgfsys@defobject{currentmarker}{\pgfqpoint{0.000000in}{0.000000in}}{\pgfqpoint{0.034722in}{0.000000in}}{%
\pgfpathmoveto{\pgfqpoint{0.000000in}{0.000000in}}%
\pgfpathlineto{\pgfqpoint{0.034722in}{0.000000in}}%
\pgfusepath{stroke,fill}%
}%
\begin{pgfscope}%
\pgfsys@transformshift{1.396958in}{1.435966in}%
\pgfsys@useobject{currentmarker}{}%
\end{pgfscope}%
\end{pgfscope}%
\begin{pgfscope}%
\definecolor{textcolor}{rgb}{0.000000,0.000000,0.000000}%
\pgfsetstrokecolor{textcolor}%
\pgfsetfillcolor{textcolor}%
\pgftext[x=0.876267in, y=1.340995in, left, base]{\color{textcolor}\sffamily\fontsize{18.000000}{21.600000}\selectfont $\displaystyle -1.0$}%
\end{pgfscope}%
\begin{pgfscope}%
\pgfpathrectangle{\pgfqpoint{1.396958in}{1.247073in}}{\pgfqpoint{10.524301in}{6.674186in}}%
\pgfusepath{clip}%
\pgfsetrectcap%
\pgfsetroundjoin%
\pgfsetlinewidth{0.501875pt}%
\definecolor{currentstroke}{rgb}{0.000000,0.000000,0.000000}%
\pgfsetstrokecolor{currentstroke}%
\pgfsetstrokeopacity{0.100000}%
\pgfsetdash{}{0pt}%
\pgfpathmoveto{\pgfqpoint{1.396958in}{3.010066in}}%
\pgfpathlineto{\pgfqpoint{11.921260in}{3.010066in}}%
\pgfusepath{stroke}%
\end{pgfscope}%
\begin{pgfscope}%
\pgfsetbuttcap%
\pgfsetroundjoin%
\definecolor{currentfill}{rgb}{0.000000,0.000000,0.000000}%
\pgfsetfillcolor{currentfill}%
\pgfsetlinewidth{0.501875pt}%
\definecolor{currentstroke}{rgb}{0.000000,0.000000,0.000000}%
\pgfsetstrokecolor{currentstroke}%
\pgfsetdash{}{0pt}%
\pgfsys@defobject{currentmarker}{\pgfqpoint{0.000000in}{0.000000in}}{\pgfqpoint{0.034722in}{0.000000in}}{%
\pgfpathmoveto{\pgfqpoint{0.000000in}{0.000000in}}%
\pgfpathlineto{\pgfqpoint{0.034722in}{0.000000in}}%
\pgfusepath{stroke,fill}%
}%
\begin{pgfscope}%
\pgfsys@transformshift{1.396958in}{3.010066in}%
\pgfsys@useobject{currentmarker}{}%
\end{pgfscope}%
\end{pgfscope}%
\begin{pgfscope}%
\definecolor{textcolor}{rgb}{0.000000,0.000000,0.000000}%
\pgfsetstrokecolor{textcolor}%
\pgfsetfillcolor{textcolor}%
\pgftext[x=0.876267in, y=2.915095in, left, base]{\color{textcolor}\sffamily\fontsize{18.000000}{21.600000}\selectfont $\displaystyle -0.5$}%
\end{pgfscope}%
\begin{pgfscope}%
\pgfpathrectangle{\pgfqpoint{1.396958in}{1.247073in}}{\pgfqpoint{10.524301in}{6.674186in}}%
\pgfusepath{clip}%
\pgfsetrectcap%
\pgfsetroundjoin%
\pgfsetlinewidth{0.501875pt}%
\definecolor{currentstroke}{rgb}{0.000000,0.000000,0.000000}%
\pgfsetstrokecolor{currentstroke}%
\pgfsetstrokeopacity{0.100000}%
\pgfsetdash{}{0pt}%
\pgfpathmoveto{\pgfqpoint{1.396958in}{4.584167in}}%
\pgfpathlineto{\pgfqpoint{11.921260in}{4.584167in}}%
\pgfusepath{stroke}%
\end{pgfscope}%
\begin{pgfscope}%
\pgfsetbuttcap%
\pgfsetroundjoin%
\definecolor{currentfill}{rgb}{0.000000,0.000000,0.000000}%
\pgfsetfillcolor{currentfill}%
\pgfsetlinewidth{0.501875pt}%
\definecolor{currentstroke}{rgb}{0.000000,0.000000,0.000000}%
\pgfsetstrokecolor{currentstroke}%
\pgfsetdash{}{0pt}%
\pgfsys@defobject{currentmarker}{\pgfqpoint{0.000000in}{0.000000in}}{\pgfqpoint{0.034722in}{0.000000in}}{%
\pgfpathmoveto{\pgfqpoint{0.000000in}{0.000000in}}%
\pgfpathlineto{\pgfqpoint{0.034722in}{0.000000in}}%
\pgfusepath{stroke,fill}%
}%
\begin{pgfscope}%
\pgfsys@transformshift{1.396958in}{4.584167in}%
\pgfsys@useobject{currentmarker}{}%
\end{pgfscope}%
\end{pgfscope}%
\begin{pgfscope}%
\definecolor{textcolor}{rgb}{0.000000,0.000000,0.000000}%
\pgfsetstrokecolor{textcolor}%
\pgfsetfillcolor{textcolor}%
\pgftext[x=1.062934in, y=4.489196in, left, base]{\color{textcolor}\sffamily\fontsize{18.000000}{21.600000}\selectfont $\displaystyle 0.0$}%
\end{pgfscope}%
\begin{pgfscope}%
\pgfpathrectangle{\pgfqpoint{1.396958in}{1.247073in}}{\pgfqpoint{10.524301in}{6.674186in}}%
\pgfusepath{clip}%
\pgfsetrectcap%
\pgfsetroundjoin%
\pgfsetlinewidth{0.501875pt}%
\definecolor{currentstroke}{rgb}{0.000000,0.000000,0.000000}%
\pgfsetstrokecolor{currentstroke}%
\pgfsetstrokeopacity{0.100000}%
\pgfsetdash{}{0pt}%
\pgfpathmoveto{\pgfqpoint{1.396958in}{6.158267in}}%
\pgfpathlineto{\pgfqpoint{11.921260in}{6.158267in}}%
\pgfusepath{stroke}%
\end{pgfscope}%
\begin{pgfscope}%
\pgfsetbuttcap%
\pgfsetroundjoin%
\definecolor{currentfill}{rgb}{0.000000,0.000000,0.000000}%
\pgfsetfillcolor{currentfill}%
\pgfsetlinewidth{0.501875pt}%
\definecolor{currentstroke}{rgb}{0.000000,0.000000,0.000000}%
\pgfsetstrokecolor{currentstroke}%
\pgfsetdash{}{0pt}%
\pgfsys@defobject{currentmarker}{\pgfqpoint{0.000000in}{0.000000in}}{\pgfqpoint{0.034722in}{0.000000in}}{%
\pgfpathmoveto{\pgfqpoint{0.000000in}{0.000000in}}%
\pgfpathlineto{\pgfqpoint{0.034722in}{0.000000in}}%
\pgfusepath{stroke,fill}%
}%
\begin{pgfscope}%
\pgfsys@transformshift{1.396958in}{6.158267in}%
\pgfsys@useobject{currentmarker}{}%
\end{pgfscope}%
\end{pgfscope}%
\begin{pgfscope}%
\definecolor{textcolor}{rgb}{0.000000,0.000000,0.000000}%
\pgfsetstrokecolor{textcolor}%
\pgfsetfillcolor{textcolor}%
\pgftext[x=1.062934in, y=6.063297in, left, base]{\color{textcolor}\sffamily\fontsize{18.000000}{21.600000}\selectfont $\displaystyle 0.5$}%
\end{pgfscope}%
\begin{pgfscope}%
\pgfpathrectangle{\pgfqpoint{1.396958in}{1.247073in}}{\pgfqpoint{10.524301in}{6.674186in}}%
\pgfusepath{clip}%
\pgfsetrectcap%
\pgfsetroundjoin%
\pgfsetlinewidth{0.501875pt}%
\definecolor{currentstroke}{rgb}{0.000000,0.000000,0.000000}%
\pgfsetstrokecolor{currentstroke}%
\pgfsetstrokeopacity{0.100000}%
\pgfsetdash{}{0pt}%
\pgfpathmoveto{\pgfqpoint{1.396958in}{7.732368in}}%
\pgfpathlineto{\pgfqpoint{11.921260in}{7.732368in}}%
\pgfusepath{stroke}%
\end{pgfscope}%
\begin{pgfscope}%
\pgfsetbuttcap%
\pgfsetroundjoin%
\definecolor{currentfill}{rgb}{0.000000,0.000000,0.000000}%
\pgfsetfillcolor{currentfill}%
\pgfsetlinewidth{0.501875pt}%
\definecolor{currentstroke}{rgb}{0.000000,0.000000,0.000000}%
\pgfsetstrokecolor{currentstroke}%
\pgfsetdash{}{0pt}%
\pgfsys@defobject{currentmarker}{\pgfqpoint{0.000000in}{0.000000in}}{\pgfqpoint{0.034722in}{0.000000in}}{%
\pgfpathmoveto{\pgfqpoint{0.000000in}{0.000000in}}%
\pgfpathlineto{\pgfqpoint{0.034722in}{0.000000in}}%
\pgfusepath{stroke,fill}%
}%
\begin{pgfscope}%
\pgfsys@transformshift{1.396958in}{7.732368in}%
\pgfsys@useobject{currentmarker}{}%
\end{pgfscope}%
\end{pgfscope}%
\begin{pgfscope}%
\definecolor{textcolor}{rgb}{0.000000,0.000000,0.000000}%
\pgfsetstrokecolor{textcolor}%
\pgfsetfillcolor{textcolor}%
\pgftext[x=1.062934in, y=7.637397in, left, base]{\color{textcolor}\sffamily\fontsize{18.000000}{21.600000}\selectfont $\displaystyle 1.0$}%
\end{pgfscope}%
\begin{pgfscope}%
\pgfpathrectangle{\pgfqpoint{1.396958in}{1.247073in}}{\pgfqpoint{10.524301in}{6.674186in}}%
\pgfusepath{clip}%
\pgfsetbuttcap%
\pgfsetroundjoin%
\pgfsetlinewidth{1.003750pt}%
\definecolor{currentstroke}{rgb}{0.000000,0.605603,0.978680}%
\pgfsetstrokecolor{currentstroke}%
\pgfsetdash{}{0pt}%
\pgfpathmoveto{\pgfqpoint{1.694816in}{4.584082in}}%
\pgfpathlineto{\pgfqpoint{1.956605in}{4.064904in}}%
\pgfpathlineto{\pgfqpoint{2.092347in}{3.800411in}}%
\pgfpathlineto{\pgfqpoint{2.208698in}{3.578230in}}%
\pgfpathlineto{\pgfqpoint{2.315353in}{3.379324in}}%
\pgfpathlineto{\pgfqpoint{2.412312in}{3.203239in}}%
\pgfpathlineto{\pgfqpoint{2.499575in}{3.049187in}}%
\pgfpathlineto{\pgfqpoint{2.586837in}{2.899815in}}%
\pgfpathlineto{\pgfqpoint{2.664405in}{2.771338in}}%
\pgfpathlineto{\pgfqpoint{2.741972in}{2.647228in}}%
\pgfpathlineto{\pgfqpoint{2.809843in}{2.542449in}}%
\pgfpathlineto{\pgfqpoint{2.877714in}{2.441436in}}%
\pgfpathlineto{\pgfqpoint{2.945585in}{2.344376in}}%
\pgfpathlineto{\pgfqpoint{3.013456in}{2.251447in}}%
\pgfpathlineto{\pgfqpoint{3.071632in}{2.175212in}}%
\pgfpathlineto{\pgfqpoint{3.129807in}{2.102242in}}%
\pgfpathlineto{\pgfqpoint{3.187982in}{2.032635in}}%
\pgfpathlineto{\pgfqpoint{3.246158in}{1.966486in}}%
\pgfpathlineto{\pgfqpoint{3.304333in}{1.903885in}}%
\pgfpathlineto{\pgfqpoint{3.362508in}{1.844916in}}%
\pgfpathlineto{\pgfqpoint{3.410988in}{1.798608in}}%
\pgfpathlineto{\pgfqpoint{3.459467in}{1.754922in}}%
\pgfpathlineto{\pgfqpoint{3.507947in}{1.713898in}}%
\pgfpathlineto{\pgfqpoint{3.556426in}{1.675576in}}%
\pgfpathlineto{\pgfqpoint{3.604905in}{1.639991in}}%
\pgfpathlineto{\pgfqpoint{3.653385in}{1.607177in}}%
\pgfpathlineto{\pgfqpoint{3.701864in}{1.577165in}}%
\pgfpathlineto{\pgfqpoint{3.750344in}{1.549983in}}%
\pgfpathlineto{\pgfqpoint{3.798823in}{1.525657in}}%
\pgfpathlineto{\pgfqpoint{3.847303in}{1.504209in}}%
\pgfpathlineto{\pgfqpoint{3.895782in}{1.485661in}}%
\pgfpathlineto{\pgfqpoint{3.944261in}{1.470028in}}%
\pgfpathlineto{\pgfqpoint{3.983045in}{1.459631in}}%
\pgfpathlineto{\pgfqpoint{4.021828in}{1.451117in}}%
\pgfpathlineto{\pgfqpoint{4.060612in}{1.444490in}}%
\pgfpathlineto{\pgfqpoint{4.099396in}{1.439754in}}%
\pgfpathlineto{\pgfqpoint{4.138179in}{1.436912in}}%
\pgfpathlineto{\pgfqpoint{4.176963in}{1.435966in}}%
\pgfpathlineto{\pgfqpoint{4.215746in}{1.436916in}}%
\pgfpathlineto{\pgfqpoint{4.254530in}{1.439762in}}%
\pgfpathlineto{\pgfqpoint{4.293313in}{1.444502in}}%
\pgfpathlineto{\pgfqpoint{4.332097in}{1.451133in}}%
\pgfpathlineto{\pgfqpoint{4.370880in}{1.459652in}}%
\pgfpathlineto{\pgfqpoint{4.409664in}{1.470053in}}%
\pgfpathlineto{\pgfqpoint{4.448447in}{1.482329in}}%
\pgfpathlineto{\pgfqpoint{4.496927in}{1.500301in}}%
\pgfpathlineto{\pgfqpoint{4.545406in}{1.521176in}}%
\pgfpathlineto{\pgfqpoint{4.593886in}{1.544933in}}%
\pgfpathlineto{\pgfqpoint{4.642365in}{1.571551in}}%
\pgfpathlineto{\pgfqpoint{4.690845in}{1.601004in}}%
\pgfpathlineto{\pgfqpoint{4.739324in}{1.633264in}}%
\pgfpathlineto{\pgfqpoint{4.787803in}{1.668302in}}%
\pgfpathlineto{\pgfqpoint{4.836283in}{1.706085in}}%
\pgfpathlineto{\pgfqpoint{4.884762in}{1.746576in}}%
\pgfpathlineto{\pgfqpoint{4.933242in}{1.789738in}}%
\pgfpathlineto{\pgfqpoint{4.981721in}{1.835529in}}%
\pgfpathlineto{\pgfqpoint{5.039896in}{1.893890in}}%
\pgfpathlineto{\pgfqpoint{5.098072in}{1.955897in}}%
\pgfpathlineto{\pgfqpoint{5.156247in}{2.021466in}}%
\pgfpathlineto{\pgfqpoint{5.214422in}{2.090507in}}%
\pgfpathlineto{\pgfqpoint{5.272598in}{2.162929in}}%
\pgfpathlineto{\pgfqpoint{5.330773in}{2.238631in}}%
\pgfpathlineto{\pgfqpoint{5.398644in}{2.330961in}}%
\pgfpathlineto{\pgfqpoint{5.466515in}{2.427447in}}%
\pgfpathlineto{\pgfqpoint{5.534387in}{2.527911in}}%
\pgfpathlineto{\pgfqpoint{5.602258in}{2.632168in}}%
\pgfpathlineto{\pgfqpoint{5.679825in}{2.755716in}}%
\pgfpathlineto{\pgfqpoint{5.757392in}{2.883669in}}%
\pgfpathlineto{\pgfqpoint{5.834959in}{3.015719in}}%
\pgfpathlineto{\pgfqpoint{5.922222in}{3.168775in}}%
\pgfpathlineto{\pgfqpoint{6.009485in}{3.326147in}}%
\pgfpathlineto{\pgfqpoint{6.106444in}{3.505481in}}%
\pgfpathlineto{\pgfqpoint{6.213098in}{3.707412in}}%
\pgfpathlineto{\pgfqpoint{6.339145in}{3.951131in}}%
\pgfpathlineto{\pgfqpoint{6.494279in}{4.256455in}}%
\pgfpathlineto{\pgfqpoint{7.037249in}{5.330446in}}%
\pgfpathlineto{\pgfqpoint{7.153599in}{5.553419in}}%
\pgfpathlineto{\pgfqpoint{7.260254in}{5.753226in}}%
\pgfpathlineto{\pgfqpoint{7.357213in}{5.930272in}}%
\pgfpathlineto{\pgfqpoint{7.444476in}{6.085301in}}%
\pgfpathlineto{\pgfqpoint{7.531739in}{6.235752in}}%
\pgfpathlineto{\pgfqpoint{7.609306in}{6.365274in}}%
\pgfpathlineto{\pgfqpoint{7.686873in}{6.490504in}}%
\pgfpathlineto{\pgfqpoint{7.764440in}{6.611142in}}%
\pgfpathlineto{\pgfqpoint{7.832311in}{6.712704in}}%
\pgfpathlineto{\pgfqpoint{7.900182in}{6.810340in}}%
\pgfpathlineto{\pgfqpoint{7.968054in}{6.903870in}}%
\pgfpathlineto{\pgfqpoint{8.026229in}{6.980639in}}%
\pgfpathlineto{\pgfqpoint{8.084404in}{7.054161in}}%
\pgfpathlineto{\pgfqpoint{8.142580in}{7.124335in}}%
\pgfpathlineto{\pgfqpoint{8.200755in}{7.191066in}}%
\pgfpathlineto{\pgfqpoint{8.258930in}{7.254265in}}%
\pgfpathlineto{\pgfqpoint{8.317105in}{7.313844in}}%
\pgfpathlineto{\pgfqpoint{8.375281in}{7.369725in}}%
\pgfpathlineto{\pgfqpoint{8.423760in}{7.413411in}}%
\pgfpathlineto{\pgfqpoint{8.472240in}{7.454435in}}%
\pgfpathlineto{\pgfqpoint{8.520719in}{7.492757in}}%
\pgfpathlineto{\pgfqpoint{8.569198in}{7.528342in}}%
\pgfpathlineto{\pgfqpoint{8.617678in}{7.561156in}}%
\pgfpathlineto{\pgfqpoint{8.666157in}{7.591168in}}%
\pgfpathlineto{\pgfqpoint{8.714637in}{7.618350in}}%
\pgfpathlineto{\pgfqpoint{8.763116in}{7.642676in}}%
\pgfpathlineto{\pgfqpoint{8.811596in}{7.664124in}}%
\pgfpathlineto{\pgfqpoint{8.860075in}{7.682673in}}%
\pgfpathlineto{\pgfqpoint{8.908554in}{7.698306in}}%
\pgfpathlineto{\pgfqpoint{8.947338in}{7.708702in}}%
\pgfpathlineto{\pgfqpoint{8.986122in}{7.717216in}}%
\pgfpathlineto{\pgfqpoint{9.024905in}{7.723844in}}%
\pgfpathlineto{\pgfqpoint{9.063689in}{7.728580in}}%
\pgfpathlineto{\pgfqpoint{9.102472in}{7.731422in}}%
\pgfpathlineto{\pgfqpoint{9.141256in}{7.732368in}}%
\pgfpathlineto{\pgfqpoint{9.180039in}{7.731417in}}%
\pgfpathlineto{\pgfqpoint{9.218823in}{7.728571in}}%
\pgfpathlineto{\pgfqpoint{9.257606in}{7.723831in}}%
\pgfpathlineto{\pgfqpoint{9.296390in}{7.717200in}}%
\pgfpathlineto{\pgfqpoint{9.335173in}{7.708681in}}%
\pgfpathlineto{\pgfqpoint{9.373957in}{7.698281in}}%
\pgfpathlineto{\pgfqpoint{9.412740in}{7.686004in}}%
\pgfpathlineto{\pgfqpoint{9.461220in}{7.668032in}}%
\pgfpathlineto{\pgfqpoint{9.509699in}{7.647158in}}%
\pgfpathlineto{\pgfqpoint{9.558179in}{7.623400in}}%
\pgfpathlineto{\pgfqpoint{9.606658in}{7.596783in}}%
\pgfpathlineto{\pgfqpoint{9.655138in}{7.567330in}}%
\pgfpathlineto{\pgfqpoint{9.703617in}{7.535069in}}%
\pgfpathlineto{\pgfqpoint{9.752096in}{7.500031in}}%
\pgfpathlineto{\pgfqpoint{9.800576in}{7.462249in}}%
\pgfpathlineto{\pgfqpoint{9.849055in}{7.421758in}}%
\pgfpathlineto{\pgfqpoint{9.897535in}{7.378596in}}%
\pgfpathlineto{\pgfqpoint{9.946014in}{7.332804in}}%
\pgfpathlineto{\pgfqpoint{10.004189in}{7.274443in}}%
\pgfpathlineto{\pgfqpoint{10.062365in}{7.212436in}}%
\pgfpathlineto{\pgfqpoint{10.120540in}{7.146868in}}%
\pgfpathlineto{\pgfqpoint{10.178715in}{7.077826in}}%
\pgfpathlineto{\pgfqpoint{10.236891in}{7.005405in}}%
\pgfpathlineto{\pgfqpoint{10.295066in}{6.929702in}}%
\pgfpathlineto{\pgfqpoint{10.362937in}{6.837372in}}%
\pgfpathlineto{\pgfqpoint{10.430808in}{6.740886in}}%
\pgfpathlineto{\pgfqpoint{10.498680in}{6.640422in}}%
\pgfpathlineto{\pgfqpoint{10.566551in}{6.536165in}}%
\pgfpathlineto{\pgfqpoint{10.644118in}{6.412617in}}%
\pgfpathlineto{\pgfqpoint{10.721685in}{6.284664in}}%
\pgfpathlineto{\pgfqpoint{10.799252in}{6.152614in}}%
\pgfpathlineto{\pgfqpoint{10.886515in}{5.999558in}}%
\pgfpathlineto{\pgfqpoint{10.973778in}{5.842187in}}%
\pgfpathlineto{\pgfqpoint{11.070737in}{5.662852in}}%
\pgfpathlineto{\pgfqpoint{11.177392in}{5.460922in}}%
\pgfpathlineto{\pgfqpoint{11.303438in}{5.217202in}}%
\pgfpathlineto{\pgfqpoint{11.458572in}{4.911879in}}%
\pgfpathlineto{\pgfqpoint{11.623402in}{4.584082in}}%
\pgfpathlineto{\pgfqpoint{11.623402in}{4.584082in}}%
\pgfusepath{stroke}%
\end{pgfscope}%
\begin{pgfscope}%
\pgfpathrectangle{\pgfqpoint{1.396958in}{1.247073in}}{\pgfqpoint{10.524301in}{6.674186in}}%
\pgfusepath{clip}%
\pgfsetbuttcap%
\pgfsetroundjoin%
\pgfsetlinewidth{1.003750pt}%
\definecolor{currentstroke}{rgb}{0.888874,0.435649,0.278123}%
\pgfsetstrokecolor{currentstroke}%
\pgfsetdash{}{0pt}%
\pgfpathmoveto{\pgfqpoint{1.694816in}{4.584167in}}%
\pgfpathlineto{\pgfqpoint{1.956605in}{4.064987in}}%
\pgfpathlineto{\pgfqpoint{2.092347in}{3.800493in}}%
\pgfpathlineto{\pgfqpoint{2.208698in}{3.578310in}}%
\pgfpathlineto{\pgfqpoint{2.315353in}{3.379402in}}%
\pgfpathlineto{\pgfqpoint{2.412312in}{3.203315in}}%
\pgfpathlineto{\pgfqpoint{2.499575in}{3.049261in}}%
\pgfpathlineto{\pgfqpoint{2.586837in}{2.899887in}}%
\pgfpathlineto{\pgfqpoint{2.664405in}{2.771407in}}%
\pgfpathlineto{\pgfqpoint{2.741972in}{2.647294in}}%
\pgfpathlineto{\pgfqpoint{2.809843in}{2.542513in}}%
\pgfpathlineto{\pgfqpoint{2.877714in}{2.441498in}}%
\pgfpathlineto{\pgfqpoint{2.945585in}{2.344435in}}%
\pgfpathlineto{\pgfqpoint{3.013456in}{2.251504in}}%
\pgfpathlineto{\pgfqpoint{3.071632in}{2.175266in}}%
\pgfpathlineto{\pgfqpoint{3.129807in}{2.102294in}}%
\pgfpathlineto{\pgfqpoint{3.187982in}{2.032684in}}%
\pgfpathlineto{\pgfqpoint{3.246158in}{1.966533in}}%
\pgfpathlineto{\pgfqpoint{3.304333in}{1.903929in}}%
\pgfpathlineto{\pgfqpoint{3.362508in}{1.844958in}}%
\pgfpathlineto{\pgfqpoint{3.410988in}{1.798647in}}%
\pgfpathlineto{\pgfqpoint{3.459467in}{1.754959in}}%
\pgfpathlineto{\pgfqpoint{3.507947in}{1.713933in}}%
\pgfpathlineto{\pgfqpoint{3.556426in}{1.675608in}}%
\pgfpathlineto{\pgfqpoint{3.604905in}{1.640021in}}%
\pgfpathlineto{\pgfqpoint{3.653385in}{1.607205in}}%
\pgfpathlineto{\pgfqpoint{3.701864in}{1.577190in}}%
\pgfpathlineto{\pgfqpoint{3.750344in}{1.550006in}}%
\pgfpathlineto{\pgfqpoint{3.798823in}{1.525677in}}%
\pgfpathlineto{\pgfqpoint{3.847303in}{1.504227in}}%
\pgfpathlineto{\pgfqpoint{3.895782in}{1.485675in}}%
\pgfpathlineto{\pgfqpoint{3.944261in}{1.470040in}}%
\pgfpathlineto{\pgfqpoint{3.983045in}{1.459641in}}%
\pgfpathlineto{\pgfqpoint{4.021828in}{1.451125in}}%
\pgfpathlineto{\pgfqpoint{4.060612in}{1.444496in}}%
\pgfpathlineto{\pgfqpoint{4.099396in}{1.439758in}}%
\pgfpathlineto{\pgfqpoint{4.138179in}{1.436914in}}%
\pgfpathlineto{\pgfqpoint{4.176963in}{1.435966in}}%
\pgfpathlineto{\pgfqpoint{4.215746in}{1.436914in}}%
\pgfpathlineto{\pgfqpoint{4.254530in}{1.439758in}}%
\pgfpathlineto{\pgfqpoint{4.293313in}{1.444496in}}%
\pgfpathlineto{\pgfqpoint{4.332097in}{1.451125in}}%
\pgfpathlineto{\pgfqpoint{4.370880in}{1.459641in}}%
\pgfpathlineto{\pgfqpoint{4.409664in}{1.470040in}}%
\pgfpathlineto{\pgfqpoint{4.448447in}{1.482314in}}%
\pgfpathlineto{\pgfqpoint{4.496927in}{1.500284in}}%
\pgfpathlineto{\pgfqpoint{4.545406in}{1.521156in}}%
\pgfpathlineto{\pgfqpoint{4.593886in}{1.544911in}}%
\pgfpathlineto{\pgfqpoint{4.642365in}{1.571526in}}%
\pgfpathlineto{\pgfqpoint{4.690845in}{1.600977in}}%
\pgfpathlineto{\pgfqpoint{4.739324in}{1.633235in}}%
\pgfpathlineto{\pgfqpoint{4.787803in}{1.668271in}}%
\pgfpathlineto{\pgfqpoint{4.836283in}{1.706050in}}%
\pgfpathlineto{\pgfqpoint{4.884762in}{1.746539in}}%
\pgfpathlineto{\pgfqpoint{4.933242in}{1.789699in}}%
\pgfpathlineto{\pgfqpoint{4.981721in}{1.835488in}}%
\pgfpathlineto{\pgfqpoint{5.039896in}{1.893846in}}%
\pgfpathlineto{\pgfqpoint{5.098072in}{1.955850in}}%
\pgfpathlineto{\pgfqpoint{5.156247in}{2.021417in}}%
\pgfpathlineto{\pgfqpoint{5.214422in}{2.090456in}}%
\pgfpathlineto{\pgfqpoint{5.272598in}{2.162875in}}%
\pgfpathlineto{\pgfqpoint{5.330773in}{2.238575in}}%
\pgfpathlineto{\pgfqpoint{5.398644in}{2.330902in}}%
\pgfpathlineto{\pgfqpoint{5.466515in}{2.427386in}}%
\pgfpathlineto{\pgfqpoint{5.534387in}{2.527847in}}%
\pgfpathlineto{\pgfqpoint{5.602258in}{2.632102in}}%
\pgfpathlineto{\pgfqpoint{5.679825in}{2.755648in}}%
\pgfpathlineto{\pgfqpoint{5.757392in}{2.883598in}}%
\pgfpathlineto{\pgfqpoint{5.834959in}{3.015646in}}%
\pgfpathlineto{\pgfqpoint{5.922222in}{3.168700in}}%
\pgfpathlineto{\pgfqpoint{6.009485in}{3.326069in}}%
\pgfpathlineto{\pgfqpoint{6.106444in}{3.505402in}}%
\pgfpathlineto{\pgfqpoint{6.213098in}{3.707331in}}%
\pgfpathlineto{\pgfqpoint{6.339145in}{3.951049in}}%
\pgfpathlineto{\pgfqpoint{6.494279in}{4.256371in}}%
\pgfpathlineto{\pgfqpoint{7.037249in}{5.330365in}}%
\pgfpathlineto{\pgfqpoint{7.153599in}{5.553339in}}%
\pgfpathlineto{\pgfqpoint{7.260254in}{5.753148in}}%
\pgfpathlineto{\pgfqpoint{7.357213in}{5.930196in}}%
\pgfpathlineto{\pgfqpoint{7.444476in}{6.085227in}}%
\pgfpathlineto{\pgfqpoint{7.531739in}{6.235680in}}%
\pgfpathlineto{\pgfqpoint{7.609306in}{6.365204in}}%
\pgfpathlineto{\pgfqpoint{7.686873in}{6.490437in}}%
\pgfpathlineto{\pgfqpoint{7.764440in}{6.611078in}}%
\pgfpathlineto{\pgfqpoint{7.832311in}{6.712642in}}%
\pgfpathlineto{\pgfqpoint{7.900182in}{6.810281in}}%
\pgfpathlineto{\pgfqpoint{7.968054in}{6.903813in}}%
\pgfpathlineto{\pgfqpoint{8.026229in}{6.980585in}}%
\pgfpathlineto{\pgfqpoint{8.084404in}{7.054109in}}%
\pgfpathlineto{\pgfqpoint{8.142580in}{7.124285in}}%
\pgfpathlineto{\pgfqpoint{8.200755in}{7.191019in}}%
\pgfpathlineto{\pgfqpoint{8.258930in}{7.254220in}}%
\pgfpathlineto{\pgfqpoint{8.317105in}{7.313803in}}%
\pgfpathlineto{\pgfqpoint{8.375281in}{7.369686in}}%
\pgfpathlineto{\pgfqpoint{8.423760in}{7.413375in}}%
\pgfpathlineto{\pgfqpoint{8.472240in}{7.454401in}}%
\pgfpathlineto{\pgfqpoint{8.520719in}{7.492725in}}%
\pgfpathlineto{\pgfqpoint{8.569198in}{7.528312in}}%
\pgfpathlineto{\pgfqpoint{8.617678in}{7.561129in}}%
\pgfpathlineto{\pgfqpoint{8.666157in}{7.591143in}}%
\pgfpathlineto{\pgfqpoint{8.714637in}{7.618328in}}%
\pgfpathlineto{\pgfqpoint{8.763116in}{7.642656in}}%
\pgfpathlineto{\pgfqpoint{8.811596in}{7.664106in}}%
\pgfpathlineto{\pgfqpoint{8.860075in}{7.682658in}}%
\pgfpathlineto{\pgfqpoint{8.908554in}{7.698293in}}%
\pgfpathlineto{\pgfqpoint{8.947338in}{7.708692in}}%
\pgfpathlineto{\pgfqpoint{8.986122in}{7.717208in}}%
\pgfpathlineto{\pgfqpoint{9.024905in}{7.723838in}}%
\pgfpathlineto{\pgfqpoint{9.063689in}{7.728576in}}%
\pgfpathlineto{\pgfqpoint{9.102472in}{7.731420in}}%
\pgfpathlineto{\pgfqpoint{9.141256in}{7.732368in}}%
\pgfpathlineto{\pgfqpoint{9.180039in}{7.731420in}}%
\pgfpathlineto{\pgfqpoint{9.218823in}{7.728576in}}%
\pgfpathlineto{\pgfqpoint{9.257606in}{7.723838in}}%
\pgfpathlineto{\pgfqpoint{9.296390in}{7.717208in}}%
\pgfpathlineto{\pgfqpoint{9.335173in}{7.708692in}}%
\pgfpathlineto{\pgfqpoint{9.373957in}{7.698293in}}%
\pgfpathlineto{\pgfqpoint{9.412740in}{7.686019in}}%
\pgfpathlineto{\pgfqpoint{9.461220in}{7.668049in}}%
\pgfpathlineto{\pgfqpoint{9.509699in}{7.647177in}}%
\pgfpathlineto{\pgfqpoint{9.558179in}{7.623423in}}%
\pgfpathlineto{\pgfqpoint{9.606658in}{7.596807in}}%
\pgfpathlineto{\pgfqpoint{9.655138in}{7.567357in}}%
\pgfpathlineto{\pgfqpoint{9.703617in}{7.535098in}}%
\pgfpathlineto{\pgfqpoint{9.752096in}{7.500063in}}%
\pgfpathlineto{\pgfqpoint{9.800576in}{7.462283in}}%
\pgfpathlineto{\pgfqpoint{9.849055in}{7.421794in}}%
\pgfpathlineto{\pgfqpoint{9.897535in}{7.378635in}}%
\pgfpathlineto{\pgfqpoint{9.946014in}{7.332845in}}%
\pgfpathlineto{\pgfqpoint{10.004189in}{7.274487in}}%
\pgfpathlineto{\pgfqpoint{10.062365in}{7.212483in}}%
\pgfpathlineto{\pgfqpoint{10.120540in}{7.146917in}}%
\pgfpathlineto{\pgfqpoint{10.178715in}{7.077877in}}%
\pgfpathlineto{\pgfqpoint{10.236891in}{7.005459in}}%
\pgfpathlineto{\pgfqpoint{10.295066in}{6.929758in}}%
\pgfpathlineto{\pgfqpoint{10.362937in}{6.837431in}}%
\pgfpathlineto{\pgfqpoint{10.430808in}{6.740948in}}%
\pgfpathlineto{\pgfqpoint{10.498680in}{6.640486in}}%
\pgfpathlineto{\pgfqpoint{10.566551in}{6.536231in}}%
\pgfpathlineto{\pgfqpoint{10.644118in}{6.412686in}}%
\pgfpathlineto{\pgfqpoint{10.721685in}{6.284735in}}%
\pgfpathlineto{\pgfqpoint{10.799252in}{6.152688in}}%
\pgfpathlineto{\pgfqpoint{10.886515in}{5.999634in}}%
\pgfpathlineto{\pgfqpoint{10.973778in}{5.842264in}}%
\pgfpathlineto{\pgfqpoint{11.070737in}{5.662932in}}%
\pgfpathlineto{\pgfqpoint{11.177392in}{5.461003in}}%
\pgfpathlineto{\pgfqpoint{11.303438in}{5.217285in}}%
\pgfpathlineto{\pgfqpoint{11.458572in}{4.911963in}}%
\pgfpathlineto{\pgfqpoint{11.623402in}{4.584167in}}%
\pgfpathlineto{\pgfqpoint{11.623402in}{4.584167in}}%
\pgfusepath{stroke}%
\end{pgfscope}%
\begin{pgfscope}%
\pgfsetrectcap%
\pgfsetmiterjoin%
\pgfsetlinewidth{1.003750pt}%
\definecolor{currentstroke}{rgb}{0.000000,0.000000,0.000000}%
\pgfsetstrokecolor{currentstroke}%
\pgfsetdash{}{0pt}%
\pgfpathmoveto{\pgfqpoint{1.396958in}{1.247073in}}%
\pgfpathlineto{\pgfqpoint{1.396958in}{7.921260in}}%
\pgfusepath{stroke}%
\end{pgfscope}%
\begin{pgfscope}%
\pgfsetrectcap%
\pgfsetmiterjoin%
\pgfsetlinewidth{1.003750pt}%
\definecolor{currentstroke}{rgb}{0.000000,0.000000,0.000000}%
\pgfsetstrokecolor{currentstroke}%
\pgfsetdash{}{0pt}%
\pgfpathmoveto{\pgfqpoint{1.396958in}{1.247073in}}%
\pgfpathlineto{\pgfqpoint{11.921260in}{1.247073in}}%
\pgfusepath{stroke}%
\end{pgfscope}%
\begin{pgfscope}%
\pgfsetbuttcap%
\pgfsetmiterjoin%
\definecolor{currentfill}{rgb}{1.000000,1.000000,1.000000}%
\pgfsetfillcolor{currentfill}%
\pgfsetlinewidth{1.003750pt}%
\definecolor{currentstroke}{rgb}{0.000000,0.000000,0.000000}%
\pgfsetstrokecolor{currentstroke}%
\pgfsetdash{}{0pt}%
\pgfpathmoveto{\pgfqpoint{10.511589in}{6.787373in}}%
\pgfpathlineto{\pgfqpoint{11.796260in}{6.787373in}}%
\pgfpathlineto{\pgfqpoint{11.796260in}{7.796260in}}%
\pgfpathlineto{\pgfqpoint{10.511589in}{7.796260in}}%
\pgfpathclose%
\pgfusepath{stroke,fill}%
\end{pgfscope}%
\begin{pgfscope}%
\pgfsetbuttcap%
\pgfsetmiterjoin%
\pgfsetlinewidth{2.258437pt}%
\definecolor{currentstroke}{rgb}{0.000000,0.605603,0.978680}%
\pgfsetstrokecolor{currentstroke}%
\pgfsetdash{}{0pt}%
\pgfpathmoveto{\pgfqpoint{10.711589in}{7.493818in}}%
\pgfpathlineto{\pgfqpoint{11.211589in}{7.493818in}}%
\pgfusepath{stroke}%
\end{pgfscope}%
\begin{pgfscope}%
\definecolor{textcolor}{rgb}{0.000000,0.000000,0.000000}%
\pgfsetstrokecolor{textcolor}%
\pgfsetfillcolor{textcolor}%
\pgftext[x=11.411589in,y=7.406318in,left,base]{\color{textcolor}\sffamily\fontsize{18.000000}{21.600000}\selectfont $\displaystyle U$}%
\end{pgfscope}%
\begin{pgfscope}%
\pgfsetbuttcap%
\pgfsetmiterjoin%
\pgfsetlinewidth{2.258437pt}%
\definecolor{currentstroke}{rgb}{0.888874,0.435649,0.278123}%
\pgfsetstrokecolor{currentstroke}%
\pgfsetdash{}{0pt}%
\pgfpathmoveto{\pgfqpoint{10.711589in}{7.126875in}}%
\pgfpathlineto{\pgfqpoint{11.211589in}{7.126875in}}%
\pgfusepath{stroke}%
\end{pgfscope}%
\begin{pgfscope}%
\definecolor{textcolor}{rgb}{0.000000,0.000000,0.000000}%
\pgfsetstrokecolor{textcolor}%
\pgfsetfillcolor{textcolor}%
\pgftext[x=11.411589in,y=7.039375in,left,base]{\color{textcolor}\sffamily\fontsize{18.000000}{21.600000}\selectfont $\displaystyle u$}%
\end{pgfscope}%
\end{pgfpicture}%
\makeatother%
\endgroup%
}
	\caption{Lax-Wendroff 格式差分逼近解 $U$ 与真解 $u$}\label{fig:lax_wendroff_Uu}
\end{figure}

取 $\nu = 2 \geq 1$, 不满足 CFL 条件. $h = 2^{-7}$ 和 $h = 2^{-10}$ 时差分逼近解 $U$ 与真解 $u$ 在 $t = t_{\max }$ 时刻图像如图 \ref{fig:lax_wendroff_Uu_noCFL} 所示. 可以看到出现了错误解.

\begin{figure}[H]\centering\zihao{-5}
	\resizebox{0.4\linewidth}{!}{%% Creator: Matplotlib, PGF backend
%%
%% To include the figure in your LaTeX document, write
%%   \input{<filename>.pgf}
%%
%% Make sure the required packages are loaded in your preamble
%%   \usepackage{pgf}
%%
%% Figures using additional raster images can only be included by \input if
%% they are in the same directory as the main LaTeX file. For loading figures
%% from other directories you can use the `import` package
%%   \usepackage{import}
%%
%% and then include the figures with
%%   \import{<path to file>}{<filename>.pgf}
%%
%% Matplotlib used the following preamble
%%   \usepackage{fontspec}
%%   \setmainfont{DejaVuSerif.ttf}[Path=\detokenize{/Users/quejiahao/.julia/conda/3/lib/python3.9/site-packages/matplotlib/mpl-data/fonts/ttf/}]
%%   \setsansfont{DejaVuSans.ttf}[Path=\detokenize{/Users/quejiahao/.julia/conda/3/lib/python3.9/site-packages/matplotlib/mpl-data/fonts/ttf/}]
%%   \setmonofont{DejaVuSansMono.ttf}[Path=\detokenize{/Users/quejiahao/.julia/conda/3/lib/python3.9/site-packages/matplotlib/mpl-data/fonts/ttf/}]
%%
\begingroup%
\makeatletter%
\begin{pgfpicture}%
\pgfpathrectangle{\pgfpointorigin}{\pgfqpoint{12.000000in}{8.000000in}}%
\pgfusepath{use as bounding box, clip}%
\begin{pgfscope}%
\pgfsetbuttcap%
\pgfsetmiterjoin%
\definecolor{currentfill}{rgb}{1.000000,1.000000,1.000000}%
\pgfsetfillcolor{currentfill}%
\pgfsetlinewidth{0.000000pt}%
\definecolor{currentstroke}{rgb}{1.000000,1.000000,1.000000}%
\pgfsetstrokecolor{currentstroke}%
\pgfsetdash{}{0pt}%
\pgfpathmoveto{\pgfqpoint{0.000000in}{0.000000in}}%
\pgfpathlineto{\pgfqpoint{12.000000in}{0.000000in}}%
\pgfpathlineto{\pgfqpoint{12.000000in}{8.000000in}}%
\pgfpathlineto{\pgfqpoint{0.000000in}{8.000000in}}%
\pgfpathclose%
\pgfusepath{fill}%
\end{pgfscope}%
\begin{pgfscope}%
\pgfsetbuttcap%
\pgfsetmiterjoin%
\definecolor{currentfill}{rgb}{1.000000,1.000000,1.000000}%
\pgfsetfillcolor{currentfill}%
\pgfsetlinewidth{0.000000pt}%
\definecolor{currentstroke}{rgb}{0.000000,0.000000,0.000000}%
\pgfsetstrokecolor{currentstroke}%
\pgfsetstrokeopacity{0.000000}%
\pgfsetdash{}{0pt}%
\pgfpathmoveto{\pgfqpoint{2.902830in}{1.247073in}}%
\pgfpathlineto{\pgfqpoint{11.921260in}{1.247073in}}%
\pgfpathlineto{\pgfqpoint{11.921260in}{7.921260in}}%
\pgfpathlineto{\pgfqpoint{2.902830in}{7.921260in}}%
\pgfpathclose%
\pgfusepath{fill}%
\end{pgfscope}%
\begin{pgfscope}%
\pgfpathrectangle{\pgfqpoint{2.902830in}{1.247073in}}{\pgfqpoint{9.018430in}{6.674186in}}%
\pgfusepath{clip}%
\pgfsetrectcap%
\pgfsetroundjoin%
\pgfsetlinewidth{0.501875pt}%
\definecolor{currentstroke}{rgb}{0.000000,0.000000,0.000000}%
\pgfsetstrokecolor{currentstroke}%
\pgfsetstrokeopacity{0.100000}%
\pgfsetdash{}{0pt}%
\pgfpathmoveto{\pgfqpoint{3.158069in}{1.247073in}}%
\pgfpathlineto{\pgfqpoint{3.158069in}{7.921260in}}%
\pgfusepath{stroke}%
\end{pgfscope}%
\begin{pgfscope}%
\pgfsetbuttcap%
\pgfsetroundjoin%
\definecolor{currentfill}{rgb}{0.000000,0.000000,0.000000}%
\pgfsetfillcolor{currentfill}%
\pgfsetlinewidth{0.501875pt}%
\definecolor{currentstroke}{rgb}{0.000000,0.000000,0.000000}%
\pgfsetstrokecolor{currentstroke}%
\pgfsetdash{}{0pt}%
\pgfsys@defobject{currentmarker}{\pgfqpoint{0.000000in}{0.000000in}}{\pgfqpoint{0.000000in}{0.034722in}}{%
\pgfpathmoveto{\pgfqpoint{0.000000in}{0.000000in}}%
\pgfpathlineto{\pgfqpoint{0.000000in}{0.034722in}}%
\pgfusepath{stroke,fill}%
}%
\begin{pgfscope}%
\pgfsys@transformshift{3.158069in}{1.247073in}%
\pgfsys@useobject{currentmarker}{}%
\end{pgfscope}%
\end{pgfscope}%
\begin{pgfscope}%
\definecolor{textcolor}{rgb}{0.000000,0.000000,0.000000}%
\pgfsetstrokecolor{textcolor}%
\pgfsetfillcolor{textcolor}%
\pgftext[x=3.158069in,y=1.198462in,,top]{\color{textcolor}\sffamily\fontsize{18.000000}{21.600000}\selectfont $\displaystyle 0$}%
\end{pgfscope}%
\begin{pgfscope}%
\pgfpathrectangle{\pgfqpoint{2.902830in}{1.247073in}}{\pgfqpoint{9.018430in}{6.674186in}}%
\pgfusepath{clip}%
\pgfsetrectcap%
\pgfsetroundjoin%
\pgfsetlinewidth{0.501875pt}%
\definecolor{currentstroke}{rgb}{0.000000,0.000000,0.000000}%
\pgfsetstrokecolor{currentstroke}%
\pgfsetstrokeopacity{0.100000}%
\pgfsetdash{}{0pt}%
\pgfpathmoveto{\pgfqpoint{4.512151in}{1.247073in}}%
\pgfpathlineto{\pgfqpoint{4.512151in}{7.921260in}}%
\pgfusepath{stroke}%
\end{pgfscope}%
\begin{pgfscope}%
\pgfsetbuttcap%
\pgfsetroundjoin%
\definecolor{currentfill}{rgb}{0.000000,0.000000,0.000000}%
\pgfsetfillcolor{currentfill}%
\pgfsetlinewidth{0.501875pt}%
\definecolor{currentstroke}{rgb}{0.000000,0.000000,0.000000}%
\pgfsetstrokecolor{currentstroke}%
\pgfsetdash{}{0pt}%
\pgfsys@defobject{currentmarker}{\pgfqpoint{0.000000in}{0.000000in}}{\pgfqpoint{0.000000in}{0.034722in}}{%
\pgfpathmoveto{\pgfqpoint{0.000000in}{0.000000in}}%
\pgfpathlineto{\pgfqpoint{0.000000in}{0.034722in}}%
\pgfusepath{stroke,fill}%
}%
\begin{pgfscope}%
\pgfsys@transformshift{4.512151in}{1.247073in}%
\pgfsys@useobject{currentmarker}{}%
\end{pgfscope}%
\end{pgfscope}%
\begin{pgfscope}%
\definecolor{textcolor}{rgb}{0.000000,0.000000,0.000000}%
\pgfsetstrokecolor{textcolor}%
\pgfsetfillcolor{textcolor}%
\pgftext[x=4.512151in,y=1.198462in,,top]{\color{textcolor}\sffamily\fontsize{18.000000}{21.600000}\selectfont $\displaystyle 1$}%
\end{pgfscope}%
\begin{pgfscope}%
\pgfpathrectangle{\pgfqpoint{2.902830in}{1.247073in}}{\pgfqpoint{9.018430in}{6.674186in}}%
\pgfusepath{clip}%
\pgfsetrectcap%
\pgfsetroundjoin%
\pgfsetlinewidth{0.501875pt}%
\definecolor{currentstroke}{rgb}{0.000000,0.000000,0.000000}%
\pgfsetstrokecolor{currentstroke}%
\pgfsetstrokeopacity{0.100000}%
\pgfsetdash{}{0pt}%
\pgfpathmoveto{\pgfqpoint{5.866234in}{1.247073in}}%
\pgfpathlineto{\pgfqpoint{5.866234in}{7.921260in}}%
\pgfusepath{stroke}%
\end{pgfscope}%
\begin{pgfscope}%
\pgfsetbuttcap%
\pgfsetroundjoin%
\definecolor{currentfill}{rgb}{0.000000,0.000000,0.000000}%
\pgfsetfillcolor{currentfill}%
\pgfsetlinewidth{0.501875pt}%
\definecolor{currentstroke}{rgb}{0.000000,0.000000,0.000000}%
\pgfsetstrokecolor{currentstroke}%
\pgfsetdash{}{0pt}%
\pgfsys@defobject{currentmarker}{\pgfqpoint{0.000000in}{0.000000in}}{\pgfqpoint{0.000000in}{0.034722in}}{%
\pgfpathmoveto{\pgfqpoint{0.000000in}{0.000000in}}%
\pgfpathlineto{\pgfqpoint{0.000000in}{0.034722in}}%
\pgfusepath{stroke,fill}%
}%
\begin{pgfscope}%
\pgfsys@transformshift{5.866234in}{1.247073in}%
\pgfsys@useobject{currentmarker}{}%
\end{pgfscope}%
\end{pgfscope}%
\begin{pgfscope}%
\definecolor{textcolor}{rgb}{0.000000,0.000000,0.000000}%
\pgfsetstrokecolor{textcolor}%
\pgfsetfillcolor{textcolor}%
\pgftext[x=5.866234in,y=1.198462in,,top]{\color{textcolor}\sffamily\fontsize{18.000000}{21.600000}\selectfont $\displaystyle 2$}%
\end{pgfscope}%
\begin{pgfscope}%
\pgfpathrectangle{\pgfqpoint{2.902830in}{1.247073in}}{\pgfqpoint{9.018430in}{6.674186in}}%
\pgfusepath{clip}%
\pgfsetrectcap%
\pgfsetroundjoin%
\pgfsetlinewidth{0.501875pt}%
\definecolor{currentstroke}{rgb}{0.000000,0.000000,0.000000}%
\pgfsetstrokecolor{currentstroke}%
\pgfsetstrokeopacity{0.100000}%
\pgfsetdash{}{0pt}%
\pgfpathmoveto{\pgfqpoint{7.220317in}{1.247073in}}%
\pgfpathlineto{\pgfqpoint{7.220317in}{7.921260in}}%
\pgfusepath{stroke}%
\end{pgfscope}%
\begin{pgfscope}%
\pgfsetbuttcap%
\pgfsetroundjoin%
\definecolor{currentfill}{rgb}{0.000000,0.000000,0.000000}%
\pgfsetfillcolor{currentfill}%
\pgfsetlinewidth{0.501875pt}%
\definecolor{currentstroke}{rgb}{0.000000,0.000000,0.000000}%
\pgfsetstrokecolor{currentstroke}%
\pgfsetdash{}{0pt}%
\pgfsys@defobject{currentmarker}{\pgfqpoint{0.000000in}{0.000000in}}{\pgfqpoint{0.000000in}{0.034722in}}{%
\pgfpathmoveto{\pgfqpoint{0.000000in}{0.000000in}}%
\pgfpathlineto{\pgfqpoint{0.000000in}{0.034722in}}%
\pgfusepath{stroke,fill}%
}%
\begin{pgfscope}%
\pgfsys@transformshift{7.220317in}{1.247073in}%
\pgfsys@useobject{currentmarker}{}%
\end{pgfscope}%
\end{pgfscope}%
\begin{pgfscope}%
\definecolor{textcolor}{rgb}{0.000000,0.000000,0.000000}%
\pgfsetstrokecolor{textcolor}%
\pgfsetfillcolor{textcolor}%
\pgftext[x=7.220317in,y=1.198462in,,top]{\color{textcolor}\sffamily\fontsize{18.000000}{21.600000}\selectfont $\displaystyle 3$}%
\end{pgfscope}%
\begin{pgfscope}%
\pgfpathrectangle{\pgfqpoint{2.902830in}{1.247073in}}{\pgfqpoint{9.018430in}{6.674186in}}%
\pgfusepath{clip}%
\pgfsetrectcap%
\pgfsetroundjoin%
\pgfsetlinewidth{0.501875pt}%
\definecolor{currentstroke}{rgb}{0.000000,0.000000,0.000000}%
\pgfsetstrokecolor{currentstroke}%
\pgfsetstrokeopacity{0.100000}%
\pgfsetdash{}{0pt}%
\pgfpathmoveto{\pgfqpoint{8.574400in}{1.247073in}}%
\pgfpathlineto{\pgfqpoint{8.574400in}{7.921260in}}%
\pgfusepath{stroke}%
\end{pgfscope}%
\begin{pgfscope}%
\pgfsetbuttcap%
\pgfsetroundjoin%
\definecolor{currentfill}{rgb}{0.000000,0.000000,0.000000}%
\pgfsetfillcolor{currentfill}%
\pgfsetlinewidth{0.501875pt}%
\definecolor{currentstroke}{rgb}{0.000000,0.000000,0.000000}%
\pgfsetstrokecolor{currentstroke}%
\pgfsetdash{}{0pt}%
\pgfsys@defobject{currentmarker}{\pgfqpoint{0.000000in}{0.000000in}}{\pgfqpoint{0.000000in}{0.034722in}}{%
\pgfpathmoveto{\pgfqpoint{0.000000in}{0.000000in}}%
\pgfpathlineto{\pgfqpoint{0.000000in}{0.034722in}}%
\pgfusepath{stroke,fill}%
}%
\begin{pgfscope}%
\pgfsys@transformshift{8.574400in}{1.247073in}%
\pgfsys@useobject{currentmarker}{}%
\end{pgfscope}%
\end{pgfscope}%
\begin{pgfscope}%
\definecolor{textcolor}{rgb}{0.000000,0.000000,0.000000}%
\pgfsetstrokecolor{textcolor}%
\pgfsetfillcolor{textcolor}%
\pgftext[x=8.574400in,y=1.198462in,,top]{\color{textcolor}\sffamily\fontsize{18.000000}{21.600000}\selectfont $\displaystyle 4$}%
\end{pgfscope}%
\begin{pgfscope}%
\pgfpathrectangle{\pgfqpoint{2.902830in}{1.247073in}}{\pgfqpoint{9.018430in}{6.674186in}}%
\pgfusepath{clip}%
\pgfsetrectcap%
\pgfsetroundjoin%
\pgfsetlinewidth{0.501875pt}%
\definecolor{currentstroke}{rgb}{0.000000,0.000000,0.000000}%
\pgfsetstrokecolor{currentstroke}%
\pgfsetstrokeopacity{0.100000}%
\pgfsetdash{}{0pt}%
\pgfpathmoveto{\pgfqpoint{9.928482in}{1.247073in}}%
\pgfpathlineto{\pgfqpoint{9.928482in}{7.921260in}}%
\pgfusepath{stroke}%
\end{pgfscope}%
\begin{pgfscope}%
\pgfsetbuttcap%
\pgfsetroundjoin%
\definecolor{currentfill}{rgb}{0.000000,0.000000,0.000000}%
\pgfsetfillcolor{currentfill}%
\pgfsetlinewidth{0.501875pt}%
\definecolor{currentstroke}{rgb}{0.000000,0.000000,0.000000}%
\pgfsetstrokecolor{currentstroke}%
\pgfsetdash{}{0pt}%
\pgfsys@defobject{currentmarker}{\pgfqpoint{0.000000in}{0.000000in}}{\pgfqpoint{0.000000in}{0.034722in}}{%
\pgfpathmoveto{\pgfqpoint{0.000000in}{0.000000in}}%
\pgfpathlineto{\pgfqpoint{0.000000in}{0.034722in}}%
\pgfusepath{stroke,fill}%
}%
\begin{pgfscope}%
\pgfsys@transformshift{9.928482in}{1.247073in}%
\pgfsys@useobject{currentmarker}{}%
\end{pgfscope}%
\end{pgfscope}%
\begin{pgfscope}%
\definecolor{textcolor}{rgb}{0.000000,0.000000,0.000000}%
\pgfsetstrokecolor{textcolor}%
\pgfsetfillcolor{textcolor}%
\pgftext[x=9.928482in,y=1.198462in,,top]{\color{textcolor}\sffamily\fontsize{18.000000}{21.600000}\selectfont $\displaystyle 5$}%
\end{pgfscope}%
\begin{pgfscope}%
\pgfpathrectangle{\pgfqpoint{2.902830in}{1.247073in}}{\pgfqpoint{9.018430in}{6.674186in}}%
\pgfusepath{clip}%
\pgfsetrectcap%
\pgfsetroundjoin%
\pgfsetlinewidth{0.501875pt}%
\definecolor{currentstroke}{rgb}{0.000000,0.000000,0.000000}%
\pgfsetstrokecolor{currentstroke}%
\pgfsetstrokeopacity{0.100000}%
\pgfsetdash{}{0pt}%
\pgfpathmoveto{\pgfqpoint{11.282565in}{1.247073in}}%
\pgfpathlineto{\pgfqpoint{11.282565in}{7.921260in}}%
\pgfusepath{stroke}%
\end{pgfscope}%
\begin{pgfscope}%
\pgfsetbuttcap%
\pgfsetroundjoin%
\definecolor{currentfill}{rgb}{0.000000,0.000000,0.000000}%
\pgfsetfillcolor{currentfill}%
\pgfsetlinewidth{0.501875pt}%
\definecolor{currentstroke}{rgb}{0.000000,0.000000,0.000000}%
\pgfsetstrokecolor{currentstroke}%
\pgfsetdash{}{0pt}%
\pgfsys@defobject{currentmarker}{\pgfqpoint{0.000000in}{0.000000in}}{\pgfqpoint{0.000000in}{0.034722in}}{%
\pgfpathmoveto{\pgfqpoint{0.000000in}{0.000000in}}%
\pgfpathlineto{\pgfqpoint{0.000000in}{0.034722in}}%
\pgfusepath{stroke,fill}%
}%
\begin{pgfscope}%
\pgfsys@transformshift{11.282565in}{1.247073in}%
\pgfsys@useobject{currentmarker}{}%
\end{pgfscope}%
\end{pgfscope}%
\begin{pgfscope}%
\definecolor{textcolor}{rgb}{0.000000,0.000000,0.000000}%
\pgfsetstrokecolor{textcolor}%
\pgfsetfillcolor{textcolor}%
\pgftext[x=11.282565in,y=1.198462in,,top]{\color{textcolor}\sffamily\fontsize{18.000000}{21.600000}\selectfont $\displaystyle 6$}%
\end{pgfscope}%
\begin{pgfscope}%
\definecolor{textcolor}{rgb}{0.000000,0.000000,0.000000}%
\pgfsetstrokecolor{textcolor}%
\pgfsetfillcolor{textcolor}%
\pgftext[x=7.412045in,y=0.900964in,,top]{\color{textcolor}\sffamily\fontsize{18.000000}{21.600000}\selectfont $\displaystyle x$}%
\end{pgfscope}%
\begin{pgfscope}%
\pgfpathrectangle{\pgfqpoint{2.902830in}{1.247073in}}{\pgfqpoint{9.018430in}{6.674186in}}%
\pgfusepath{clip}%
\pgfsetrectcap%
\pgfsetroundjoin%
\pgfsetlinewidth{0.501875pt}%
\definecolor{currentstroke}{rgb}{0.000000,0.000000,0.000000}%
\pgfsetstrokecolor{currentstroke}%
\pgfsetstrokeopacity{0.100000}%
\pgfsetdash{}{0pt}%
\pgfpathmoveto{\pgfqpoint{2.902830in}{1.559415in}}%
\pgfpathlineto{\pgfqpoint{11.921260in}{1.559415in}}%
\pgfusepath{stroke}%
\end{pgfscope}%
\begin{pgfscope}%
\pgfsetbuttcap%
\pgfsetroundjoin%
\definecolor{currentfill}{rgb}{0.000000,0.000000,0.000000}%
\pgfsetfillcolor{currentfill}%
\pgfsetlinewidth{0.501875pt}%
\definecolor{currentstroke}{rgb}{0.000000,0.000000,0.000000}%
\pgfsetstrokecolor{currentstroke}%
\pgfsetdash{}{0pt}%
\pgfsys@defobject{currentmarker}{\pgfqpoint{0.000000in}{0.000000in}}{\pgfqpoint{0.034722in}{0.000000in}}{%
\pgfpathmoveto{\pgfqpoint{0.000000in}{0.000000in}}%
\pgfpathlineto{\pgfqpoint{0.034722in}{0.000000in}}%
\pgfusepath{stroke,fill}%
}%
\begin{pgfscope}%
\pgfsys@transformshift{2.902830in}{1.559415in}%
\pgfsys@useobject{currentmarker}{}%
\end{pgfscope}%
\end{pgfscope}%
\begin{pgfscope}%
\definecolor{textcolor}{rgb}{0.000000,0.000000,0.000000}%
\pgfsetstrokecolor{textcolor}%
\pgfsetfillcolor{textcolor}%
\pgftext[x=1.782392in, y=1.464445in, left, base]{\color{textcolor}\sffamily\fontsize{18.000000}{21.600000}\selectfont $\displaystyle -2.0×10^{11}$}%
\end{pgfscope}%
\begin{pgfscope}%
\pgfpathrectangle{\pgfqpoint{2.902830in}{1.247073in}}{\pgfqpoint{9.018430in}{6.674186in}}%
\pgfusepath{clip}%
\pgfsetrectcap%
\pgfsetroundjoin%
\pgfsetlinewidth{0.501875pt}%
\definecolor{currentstroke}{rgb}{0.000000,0.000000,0.000000}%
\pgfsetstrokecolor{currentstroke}%
\pgfsetstrokeopacity{0.100000}%
\pgfsetdash{}{0pt}%
\pgfpathmoveto{\pgfqpoint{2.902830in}{3.080989in}}%
\pgfpathlineto{\pgfqpoint{11.921260in}{3.080989in}}%
\pgfusepath{stroke}%
\end{pgfscope}%
\begin{pgfscope}%
\pgfsetbuttcap%
\pgfsetroundjoin%
\definecolor{currentfill}{rgb}{0.000000,0.000000,0.000000}%
\pgfsetfillcolor{currentfill}%
\pgfsetlinewidth{0.501875pt}%
\definecolor{currentstroke}{rgb}{0.000000,0.000000,0.000000}%
\pgfsetstrokecolor{currentstroke}%
\pgfsetdash{}{0pt}%
\pgfsys@defobject{currentmarker}{\pgfqpoint{0.000000in}{0.000000in}}{\pgfqpoint{0.034722in}{0.000000in}}{%
\pgfpathmoveto{\pgfqpoint{0.000000in}{0.000000in}}%
\pgfpathlineto{\pgfqpoint{0.034722in}{0.000000in}}%
\pgfusepath{stroke,fill}%
}%
\begin{pgfscope}%
\pgfsys@transformshift{2.902830in}{3.080989in}%
\pgfsys@useobject{currentmarker}{}%
\end{pgfscope}%
\end{pgfscope}%
\begin{pgfscope}%
\definecolor{textcolor}{rgb}{0.000000,0.000000,0.000000}%
\pgfsetstrokecolor{textcolor}%
\pgfsetfillcolor{textcolor}%
\pgftext[x=1.782392in, y=2.986018in, left, base]{\color{textcolor}\sffamily\fontsize{18.000000}{21.600000}\selectfont $\displaystyle -1.0×10^{11}$}%
\end{pgfscope}%
\begin{pgfscope}%
\pgfpathrectangle{\pgfqpoint{2.902830in}{1.247073in}}{\pgfqpoint{9.018430in}{6.674186in}}%
\pgfusepath{clip}%
\pgfsetrectcap%
\pgfsetroundjoin%
\pgfsetlinewidth{0.501875pt}%
\definecolor{currentstroke}{rgb}{0.000000,0.000000,0.000000}%
\pgfsetstrokecolor{currentstroke}%
\pgfsetstrokeopacity{0.100000}%
\pgfsetdash{}{0pt}%
\pgfpathmoveto{\pgfqpoint{2.902830in}{4.602562in}}%
\pgfpathlineto{\pgfqpoint{11.921260in}{4.602562in}}%
\pgfusepath{stroke}%
\end{pgfscope}%
\begin{pgfscope}%
\pgfsetbuttcap%
\pgfsetroundjoin%
\definecolor{currentfill}{rgb}{0.000000,0.000000,0.000000}%
\pgfsetfillcolor{currentfill}%
\pgfsetlinewidth{0.501875pt}%
\definecolor{currentstroke}{rgb}{0.000000,0.000000,0.000000}%
\pgfsetstrokecolor{currentstroke}%
\pgfsetdash{}{0pt}%
\pgfsys@defobject{currentmarker}{\pgfqpoint{0.000000in}{0.000000in}}{\pgfqpoint{0.034722in}{0.000000in}}{%
\pgfpathmoveto{\pgfqpoint{0.000000in}{0.000000in}}%
\pgfpathlineto{\pgfqpoint{0.034722in}{0.000000in}}%
\pgfusepath{stroke,fill}%
}%
\begin{pgfscope}%
\pgfsys@transformshift{2.902830in}{4.602562in}%
\pgfsys@useobject{currentmarker}{}%
\end{pgfscope}%
\end{pgfscope}%
\begin{pgfscope}%
\definecolor{textcolor}{rgb}{0.000000,0.000000,0.000000}%
\pgfsetstrokecolor{textcolor}%
\pgfsetfillcolor{textcolor}%
\pgftext[x=2.744151in, y=4.507591in, left, base]{\color{textcolor}\sffamily\fontsize{18.000000}{21.600000}\selectfont $\displaystyle 0$}%
\end{pgfscope}%
\begin{pgfscope}%
\pgfpathrectangle{\pgfqpoint{2.902830in}{1.247073in}}{\pgfqpoint{9.018430in}{6.674186in}}%
\pgfusepath{clip}%
\pgfsetrectcap%
\pgfsetroundjoin%
\pgfsetlinewidth{0.501875pt}%
\definecolor{currentstroke}{rgb}{0.000000,0.000000,0.000000}%
\pgfsetstrokecolor{currentstroke}%
\pgfsetstrokeopacity{0.100000}%
\pgfsetdash{}{0pt}%
\pgfpathmoveto{\pgfqpoint{2.902830in}{6.124136in}}%
\pgfpathlineto{\pgfqpoint{11.921260in}{6.124136in}}%
\pgfusepath{stroke}%
\end{pgfscope}%
\begin{pgfscope}%
\pgfsetbuttcap%
\pgfsetroundjoin%
\definecolor{currentfill}{rgb}{0.000000,0.000000,0.000000}%
\pgfsetfillcolor{currentfill}%
\pgfsetlinewidth{0.501875pt}%
\definecolor{currentstroke}{rgb}{0.000000,0.000000,0.000000}%
\pgfsetstrokecolor{currentstroke}%
\pgfsetdash{}{0pt}%
\pgfsys@defobject{currentmarker}{\pgfqpoint{0.000000in}{0.000000in}}{\pgfqpoint{0.034722in}{0.000000in}}{%
\pgfpathmoveto{\pgfqpoint{0.000000in}{0.000000in}}%
\pgfpathlineto{\pgfqpoint{0.034722in}{0.000000in}}%
\pgfusepath{stroke,fill}%
}%
\begin{pgfscope}%
\pgfsys@transformshift{2.902830in}{6.124136in}%
\pgfsys@useobject{currentmarker}{}%
\end{pgfscope}%
\end{pgfscope}%
\begin{pgfscope}%
\definecolor{textcolor}{rgb}{0.000000,0.000000,0.000000}%
\pgfsetstrokecolor{textcolor}%
\pgfsetfillcolor{textcolor}%
\pgftext[x=1.969059in, y=6.029165in, left, base]{\color{textcolor}\sffamily\fontsize{18.000000}{21.600000}\selectfont $\displaystyle 1.0×10^{11}$}%
\end{pgfscope}%
\begin{pgfscope}%
\pgfpathrectangle{\pgfqpoint{2.902830in}{1.247073in}}{\pgfqpoint{9.018430in}{6.674186in}}%
\pgfusepath{clip}%
\pgfsetrectcap%
\pgfsetroundjoin%
\pgfsetlinewidth{0.501875pt}%
\definecolor{currentstroke}{rgb}{0.000000,0.000000,0.000000}%
\pgfsetstrokecolor{currentstroke}%
\pgfsetstrokeopacity{0.100000}%
\pgfsetdash{}{0pt}%
\pgfpathmoveto{\pgfqpoint{2.902830in}{7.645709in}}%
\pgfpathlineto{\pgfqpoint{11.921260in}{7.645709in}}%
\pgfusepath{stroke}%
\end{pgfscope}%
\begin{pgfscope}%
\pgfsetbuttcap%
\pgfsetroundjoin%
\definecolor{currentfill}{rgb}{0.000000,0.000000,0.000000}%
\pgfsetfillcolor{currentfill}%
\pgfsetlinewidth{0.501875pt}%
\definecolor{currentstroke}{rgb}{0.000000,0.000000,0.000000}%
\pgfsetstrokecolor{currentstroke}%
\pgfsetdash{}{0pt}%
\pgfsys@defobject{currentmarker}{\pgfqpoint{0.000000in}{0.000000in}}{\pgfqpoint{0.034722in}{0.000000in}}{%
\pgfpathmoveto{\pgfqpoint{0.000000in}{0.000000in}}%
\pgfpathlineto{\pgfqpoint{0.034722in}{0.000000in}}%
\pgfusepath{stroke,fill}%
}%
\begin{pgfscope}%
\pgfsys@transformshift{2.902830in}{7.645709in}%
\pgfsys@useobject{currentmarker}{}%
\end{pgfscope}%
\end{pgfscope}%
\begin{pgfscope}%
\definecolor{textcolor}{rgb}{0.000000,0.000000,0.000000}%
\pgfsetstrokecolor{textcolor}%
\pgfsetfillcolor{textcolor}%
\pgftext[x=1.969059in, y=7.550738in, left, base]{\color{textcolor}\sffamily\fontsize{18.000000}{21.600000}\selectfont $\displaystyle 2.0×10^{11}$}%
\end{pgfscope}%
\begin{pgfscope}%
\pgfpathrectangle{\pgfqpoint{2.902830in}{1.247073in}}{\pgfqpoint{9.018430in}{6.674186in}}%
\pgfusepath{clip}%
\pgfsetbuttcap%
\pgfsetroundjoin%
\pgfsetlinewidth{1.003750pt}%
\definecolor{currentstroke}{rgb}{0.000000,0.605603,0.978680}%
\pgfsetstrokecolor{currentstroke}%
\pgfsetdash{}{0pt}%
\pgfpathmoveto{\pgfqpoint{3.158069in}{2.521527in}}%
\pgfpathlineto{\pgfqpoint{3.224537in}{6.375600in}}%
\pgfpathlineto{\pgfqpoint{3.291005in}{3.304295in}}%
\pgfpathlineto{\pgfqpoint{3.357474in}{5.292920in}}%
\pgfpathlineto{\pgfqpoint{3.423942in}{4.591844in}}%
\pgfpathlineto{\pgfqpoint{3.490411in}{3.945735in}}%
\pgfpathlineto{\pgfqpoint{3.556879in}{5.824820in}}%
\pgfpathlineto{\pgfqpoint{3.623347in}{2.991328in}}%
\pgfpathlineto{\pgfqpoint{3.689816in}{6.386668in}}%
\pgfpathlineto{\pgfqpoint{3.756284in}{2.858055in}}%
\pgfpathlineto{\pgfqpoint{3.822753in}{6.137150in}}%
\pgfpathlineto{\pgfqpoint{3.889221in}{3.382967in}}%
\pgfpathlineto{\pgfqpoint{3.955689in}{5.470914in}}%
\pgfpathlineto{\pgfqpoint{4.022158in}{4.065852in}}%
\pgfpathlineto{\pgfqpoint{4.088626in}{4.861072in}}%
\pgfpathlineto{\pgfqpoint{4.155094in}{4.557704in}}%
\pgfpathlineto{\pgfqpoint{4.221563in}{4.492006in}}%
\pgfpathlineto{\pgfqpoint{4.288031in}{4.826172in}}%
\pgfpathlineto{\pgfqpoint{4.354500in}{4.290130in}}%
\pgfpathlineto{\pgfqpoint{4.420968in}{4.994856in}}%
\pgfpathlineto{\pgfqpoint{4.487436in}{4.129517in}}%
\pgfpathlineto{\pgfqpoint{4.553905in}{5.161224in}}%
\pgfpathlineto{\pgfqpoint{4.620373in}{3.954350in}}%
\pgfpathlineto{\pgfqpoint{4.686841in}{5.340128in}}%
\pgfpathlineto{\pgfqpoint{4.753310in}{3.781263in}}%
\pgfpathlineto{\pgfqpoint{4.819778in}{5.496801in}}%
\pgfpathlineto{\pgfqpoint{4.886247in}{3.650303in}}%
\pgfpathlineto{\pgfqpoint{4.952715in}{5.594831in}}%
\pgfpathlineto{\pgfqpoint{5.019183in}{3.590859in}}%
\pgfpathlineto{\pgfqpoint{5.085652in}{5.610757in}}%
\pgfpathlineto{\pgfqpoint{5.152120in}{3.622560in}}%
\pgfpathlineto{\pgfqpoint{5.218588in}{5.529853in}}%
\pgfpathlineto{\pgfqpoint{5.285057in}{3.748952in}}%
\pgfpathlineto{\pgfqpoint{5.351525in}{5.369044in}}%
\pgfpathlineto{\pgfqpoint{5.417994in}{3.926340in}}%
\pgfpathlineto{\pgfqpoint{5.484462in}{5.195646in}}%
\pgfpathlineto{\pgfqpoint{5.550930in}{4.078675in}}%
\pgfpathlineto{\pgfqpoint{5.617399in}{5.072427in}}%
\pgfpathlineto{\pgfqpoint{5.683867in}{4.175634in}}%
\pgfpathlineto{\pgfqpoint{5.750336in}{4.990828in}}%
\pgfpathlineto{\pgfqpoint{5.816804in}{4.254739in}}%
\pgfpathlineto{\pgfqpoint{5.883272in}{4.905343in}}%
\pgfpathlineto{\pgfqpoint{5.949741in}{4.348777in}}%
\pgfpathlineto{\pgfqpoint{6.016209in}{4.805617in}}%
\pgfpathlineto{\pgfqpoint{6.082677in}{4.450957in}}%
\pgfpathlineto{\pgfqpoint{6.149146in}{4.699696in}}%
\pgfpathlineto{\pgfqpoint{6.215614in}{4.568991in}}%
\pgfpathlineto{\pgfqpoint{6.282083in}{4.554990in}}%
\pgfpathlineto{\pgfqpoint{6.348551in}{4.757032in}}%
\pgfpathlineto{\pgfqpoint{6.415019in}{4.311068in}}%
\pgfpathlineto{\pgfqpoint{6.481488in}{5.058215in}}%
\pgfpathlineto{\pgfqpoint{6.547956in}{3.968599in}}%
\pgfpathlineto{\pgfqpoint{6.614424in}{5.405708in}}%
\pgfpathlineto{\pgfqpoint{6.680893in}{3.669567in}}%
\pgfpathlineto{\pgfqpoint{6.747361in}{5.596256in}}%
\pgfpathlineto{\pgfqpoint{6.813830in}{3.637070in}}%
\pgfpathlineto{\pgfqpoint{6.880298in}{5.449800in}}%
\pgfpathlineto{\pgfqpoint{6.946766in}{3.942785in}}%
\pgfpathlineto{\pgfqpoint{7.013235in}{5.044338in}}%
\pgfpathlineto{\pgfqpoint{7.079703in}{4.363743in}}%
\pgfpathlineto{\pgfqpoint{7.146171in}{4.692505in}}%
\pgfpathlineto{\pgfqpoint{7.212640in}{4.585257in}}%
\pgfpathlineto{\pgfqpoint{7.279108in}{4.625756in}}%
\pgfpathlineto{\pgfqpoint{7.345577in}{4.507737in}}%
\pgfpathlineto{\pgfqpoint{7.412045in}{4.815752in}}%
\pgfpathlineto{\pgfqpoint{7.478513in}{4.242238in}}%
\pgfpathlineto{\pgfqpoint{7.544982in}{5.123393in}}%
\pgfpathlineto{\pgfqpoint{7.611450in}{3.924878in}}%
\pgfpathlineto{\pgfqpoint{7.677919in}{5.409190in}}%
\pgfpathlineto{\pgfqpoint{7.744387in}{3.727531in}}%
\pgfpathlineto{\pgfqpoint{7.810855in}{5.451305in}}%
\pgfpathlineto{\pgfqpoint{7.877324in}{3.897178in}}%
\pgfpathlineto{\pgfqpoint{7.943792in}{5.050381in}}%
\pgfpathlineto{\pgfqpoint{8.010260in}{4.492405in}}%
\pgfpathlineto{\pgfqpoint{8.076729in}{4.354285in}}%
\pgfpathlineto{\pgfqpoint{8.143197in}{5.162213in}}%
\pgfpathlineto{\pgfqpoint{8.209666in}{3.834484in}}%
\pgfpathlineto{\pgfqpoint{8.276134in}{5.447419in}}%
\pgfpathlineto{\pgfqpoint{8.342602in}{3.810483in}}%
\pgfpathlineto{\pgfqpoint{8.409071in}{5.236592in}}%
\pgfpathlineto{\pgfqpoint{8.475539in}{4.200907in}}%
\pgfpathlineto{\pgfqpoint{8.542007in}{4.720849in}}%
\pgfpathlineto{\pgfqpoint{8.608476in}{4.809217in}}%
\pgfpathlineto{\pgfqpoint{8.674944in}{4.028470in}}%
\pgfpathlineto{\pgfqpoint{8.741413in}{5.587796in}}%
\pgfpathlineto{\pgfqpoint{8.807881in}{3.171170in}}%
\pgfpathlineto{\pgfqpoint{8.874349in}{6.491797in}}%
\pgfpathlineto{\pgfqpoint{8.940818in}{2.279061in}}%
\pgfpathlineto{\pgfqpoint{9.007286in}{7.297726in}}%
\pgfpathlineto{\pgfqpoint{9.073754in}{1.631705in}}%
\pgfpathlineto{\pgfqpoint{9.140223in}{7.732368in}}%
\pgfpathlineto{\pgfqpoint{9.206691in}{1.435966in}}%
\pgfpathlineto{\pgfqpoint{9.273160in}{7.692780in}}%
\pgfpathlineto{\pgfqpoint{9.339628in}{1.681842in}}%
\pgfpathlineto{\pgfqpoint{9.406096in}{7.287195in}}%
\pgfpathlineto{\pgfqpoint{9.472565in}{2.192770in}}%
\pgfpathlineto{\pgfqpoint{9.539033in}{6.722838in}}%
\pgfpathlineto{\pgfqpoint{9.605502in}{2.770003in}}%
\pgfpathlineto{\pgfqpoint{9.671970in}{6.156794in}}%
\pgfpathlineto{\pgfqpoint{9.738438in}{3.316581in}}%
\pgfpathlineto{\pgfqpoint{9.804907in}{5.628709in}}%
\pgfpathlineto{\pgfqpoint{9.871375in}{3.826639in}}%
\pgfpathlineto{\pgfqpoint{9.937843in}{5.145427in}}%
\pgfpathlineto{\pgfqpoint{10.004312in}{4.261330in}}%
\pgfpathlineto{\pgfqpoint{10.070780in}{4.791750in}}%
\pgfpathlineto{\pgfqpoint{10.137249in}{4.498355in}}%
\pgfpathlineto{\pgfqpoint{10.203717in}{4.701129in}}%
\pgfpathlineto{\pgfqpoint{10.270185in}{4.426129in}}%
\pgfpathlineto{\pgfqpoint{10.336654in}{4.935732in}}%
\pgfpathlineto{\pgfqpoint{10.403122in}{4.046004in}}%
\pgfpathlineto{\pgfqpoint{10.469590in}{5.431576in}}%
\pgfpathlineto{\pgfqpoint{10.536059in}{3.472877in}}%
\pgfpathlineto{\pgfqpoint{10.602527in}{6.038091in}}%
\pgfpathlineto{\pgfqpoint{10.668996in}{2.881439in}}%
\pgfpathlineto{\pgfqpoint{10.735464in}{6.560335in}}%
\pgfpathlineto{\pgfqpoint{10.801932in}{2.489246in}}%
\pgfpathlineto{\pgfqpoint{10.868401in}{6.756824in}}%
\pgfpathlineto{\pgfqpoint{10.934869in}{2.551346in}}%
\pgfpathlineto{\pgfqpoint{11.001337in}{6.389485in}}%
\pgfpathlineto{\pgfqpoint{11.067806in}{3.238228in}}%
\pgfpathlineto{\pgfqpoint{11.134274in}{5.413594in}}%
\pgfpathlineto{\pgfqpoint{11.200743in}{4.425489in}}%
\pgfpathlineto{\pgfqpoint{11.267211in}{4.128475in}}%
\pgfpathlineto{\pgfqpoint{11.333679in}{5.681095in}}%
\pgfpathlineto{\pgfqpoint{11.400148in}{3.019063in}}%
\pgfpathlineto{\pgfqpoint{11.466616in}{6.555576in}}%
\pgfpathlineto{\pgfqpoint{11.533085in}{2.436332in}}%
\pgfpathlineto{\pgfqpoint{11.599553in}{6.813958in}}%
\pgfpathlineto{\pgfqpoint{11.666021in}{2.521527in}}%
\pgfpathlineto{\pgfqpoint{11.666021in}{2.521527in}}%
\pgfusepath{stroke}%
\end{pgfscope}%
\begin{pgfscope}%
\pgfpathrectangle{\pgfqpoint{2.902830in}{1.247073in}}{\pgfqpoint{9.018430in}{6.674186in}}%
\pgfusepath{clip}%
\pgfsetbuttcap%
\pgfsetroundjoin%
\pgfsetlinewidth{1.003750pt}%
\definecolor{currentstroke}{rgb}{0.888874,0.435649,0.278123}%
\pgfsetstrokecolor{currentstroke}%
\pgfsetdash{}{0pt}%
\pgfpathmoveto{\pgfqpoint{3.158069in}{4.602562in}}%
\pgfpathlineto{\pgfqpoint{11.666021in}{4.602562in}}%
\pgfpathlineto{\pgfqpoint{11.666021in}{4.602562in}}%
\pgfusepath{stroke}%
\end{pgfscope}%
\begin{pgfscope}%
\pgfsetrectcap%
\pgfsetmiterjoin%
\pgfsetlinewidth{1.003750pt}%
\definecolor{currentstroke}{rgb}{0.000000,0.000000,0.000000}%
\pgfsetstrokecolor{currentstroke}%
\pgfsetdash{}{0pt}%
\pgfpathmoveto{\pgfqpoint{2.902830in}{1.247073in}}%
\pgfpathlineto{\pgfqpoint{2.902830in}{7.921260in}}%
\pgfusepath{stroke}%
\end{pgfscope}%
\begin{pgfscope}%
\pgfsetrectcap%
\pgfsetmiterjoin%
\pgfsetlinewidth{1.003750pt}%
\definecolor{currentstroke}{rgb}{0.000000,0.000000,0.000000}%
\pgfsetstrokecolor{currentstroke}%
\pgfsetdash{}{0pt}%
\pgfpathmoveto{\pgfqpoint{2.902830in}{1.247073in}}%
\pgfpathlineto{\pgfqpoint{11.921260in}{1.247073in}}%
\pgfusepath{stroke}%
\end{pgfscope}%
\begin{pgfscope}%
\pgfsetbuttcap%
\pgfsetmiterjoin%
\definecolor{currentfill}{rgb}{1.000000,1.000000,1.000000}%
\pgfsetfillcolor{currentfill}%
\pgfsetlinewidth{1.003750pt}%
\definecolor{currentstroke}{rgb}{0.000000,0.000000,0.000000}%
\pgfsetstrokecolor{currentstroke}%
\pgfsetdash{}{0pt}%
\pgfpathmoveto{\pgfqpoint{10.511589in}{6.787373in}}%
\pgfpathlineto{\pgfqpoint{11.796260in}{6.787373in}}%
\pgfpathlineto{\pgfqpoint{11.796260in}{7.796260in}}%
\pgfpathlineto{\pgfqpoint{10.511589in}{7.796260in}}%
\pgfpathclose%
\pgfusepath{stroke,fill}%
\end{pgfscope}%
\begin{pgfscope}%
\pgfsetbuttcap%
\pgfsetmiterjoin%
\pgfsetlinewidth{2.258437pt}%
\definecolor{currentstroke}{rgb}{0.000000,0.605603,0.978680}%
\pgfsetstrokecolor{currentstroke}%
\pgfsetdash{}{0pt}%
\pgfpathmoveto{\pgfqpoint{10.711589in}{7.493818in}}%
\pgfpathlineto{\pgfqpoint{11.211589in}{7.493818in}}%
\pgfusepath{stroke}%
\end{pgfscope}%
\begin{pgfscope}%
\definecolor{textcolor}{rgb}{0.000000,0.000000,0.000000}%
\pgfsetstrokecolor{textcolor}%
\pgfsetfillcolor{textcolor}%
\pgftext[x=11.411589in,y=7.406318in,left,base]{\color{textcolor}\sffamily\fontsize{18.000000}{21.600000}\selectfont $\displaystyle U$}%
\end{pgfscope}%
\begin{pgfscope}%
\pgfsetbuttcap%
\pgfsetmiterjoin%
\pgfsetlinewidth{2.258437pt}%
\definecolor{currentstroke}{rgb}{0.888874,0.435649,0.278123}%
\pgfsetstrokecolor{currentstroke}%
\pgfsetdash{}{0pt}%
\pgfpathmoveto{\pgfqpoint{10.711589in}{7.126875in}}%
\pgfpathlineto{\pgfqpoint{11.211589in}{7.126875in}}%
\pgfusepath{stroke}%
\end{pgfscope}%
\begin{pgfscope}%
\definecolor{textcolor}{rgb}{0.000000,0.000000,0.000000}%
\pgfsetstrokecolor{textcolor}%
\pgfsetfillcolor{textcolor}%
\pgftext[x=11.411589in,y=7.039375in,left,base]{\color{textcolor}\sffamily\fontsize{18.000000}{21.600000}\selectfont $\displaystyle u$}%
\end{pgfscope}%
\end{pgfpicture}%
\makeatother%
\endgroup%
}\quad
	\resizebox{0.4\linewidth}{!}{%% Creator: Matplotlib, PGF backend
%%
%% To include the figure in your LaTeX document, write
%%   \input{<filename>.pgf}
%%
%% Make sure the required packages are loaded in your preamble
%%   \usepackage{pgf}
%%
%% Figures using additional raster images can only be included by \input if
%% they are in the same directory as the main LaTeX file. For loading figures
%% from other directories you can use the `import` package
%%   \usepackage{import}
%%
%% and then include the figures with
%%   \import{<path to file>}{<filename>.pgf}
%%
%% Matplotlib used the following preamble
%%   \usepackage{fontspec}
%%   \setmainfont{DejaVuSerif.ttf}[Path=\detokenize{/Users/quejiahao/.julia/conda/3/lib/python3.9/site-packages/matplotlib/mpl-data/fonts/ttf/}]
%%   \setsansfont{DejaVuSans.ttf}[Path=\detokenize{/Users/quejiahao/.julia/conda/3/lib/python3.9/site-packages/matplotlib/mpl-data/fonts/ttf/}]
%%   \setmonofont{DejaVuSansMono.ttf}[Path=\detokenize{/Users/quejiahao/.julia/conda/3/lib/python3.9/site-packages/matplotlib/mpl-data/fonts/ttf/}]
%%
\begingroup%
\makeatletter%
\begin{pgfpicture}%
\pgfpathrectangle{\pgfpointorigin}{\pgfqpoint{12.000000in}{8.000000in}}%
\pgfusepath{use as bounding box, clip}%
\begin{pgfscope}%
\pgfsetbuttcap%
\pgfsetmiterjoin%
\definecolor{currentfill}{rgb}{1.000000,1.000000,1.000000}%
\pgfsetfillcolor{currentfill}%
\pgfsetlinewidth{0.000000pt}%
\definecolor{currentstroke}{rgb}{1.000000,1.000000,1.000000}%
\pgfsetstrokecolor{currentstroke}%
\pgfsetdash{}{0pt}%
\pgfpathmoveto{\pgfqpoint{0.000000in}{0.000000in}}%
\pgfpathlineto{\pgfqpoint{12.000000in}{0.000000in}}%
\pgfpathlineto{\pgfqpoint{12.000000in}{8.000000in}}%
\pgfpathlineto{\pgfqpoint{0.000000in}{8.000000in}}%
\pgfpathclose%
\pgfusepath{fill}%
\end{pgfscope}%
\begin{pgfscope}%
\pgfsetbuttcap%
\pgfsetmiterjoin%
\definecolor{currentfill}{rgb}{1.000000,1.000000,1.000000}%
\pgfsetfillcolor{currentfill}%
\pgfsetlinewidth{0.000000pt}%
\definecolor{currentstroke}{rgb}{0.000000,0.000000,0.000000}%
\pgfsetstrokecolor{currentstroke}%
\pgfsetstrokeopacity{0.000000}%
\pgfsetdash{}{0pt}%
\pgfpathmoveto{\pgfqpoint{3.133636in}{1.247073in}}%
\pgfpathlineto{\pgfqpoint{11.921260in}{1.247073in}}%
\pgfpathlineto{\pgfqpoint{11.921260in}{7.921260in}}%
\pgfpathlineto{\pgfqpoint{3.133636in}{7.921260in}}%
\pgfpathclose%
\pgfusepath{fill}%
\end{pgfscope}%
\begin{pgfscope}%
\pgfpathrectangle{\pgfqpoint{3.133636in}{1.247073in}}{\pgfqpoint{8.787624in}{6.674186in}}%
\pgfusepath{clip}%
\pgfsetrectcap%
\pgfsetroundjoin%
\pgfsetlinewidth{0.501875pt}%
\definecolor{currentstroke}{rgb}{0.000000,0.000000,0.000000}%
\pgfsetstrokecolor{currentstroke}%
\pgfsetstrokeopacity{0.100000}%
\pgfsetdash{}{0pt}%
\pgfpathmoveto{\pgfqpoint{3.382342in}{1.247073in}}%
\pgfpathlineto{\pgfqpoint{3.382342in}{7.921260in}}%
\pgfusepath{stroke}%
\end{pgfscope}%
\begin{pgfscope}%
\pgfsetbuttcap%
\pgfsetroundjoin%
\definecolor{currentfill}{rgb}{0.000000,0.000000,0.000000}%
\pgfsetfillcolor{currentfill}%
\pgfsetlinewidth{0.501875pt}%
\definecolor{currentstroke}{rgb}{0.000000,0.000000,0.000000}%
\pgfsetstrokecolor{currentstroke}%
\pgfsetdash{}{0pt}%
\pgfsys@defobject{currentmarker}{\pgfqpoint{0.000000in}{0.000000in}}{\pgfqpoint{0.000000in}{0.034722in}}{%
\pgfpathmoveto{\pgfqpoint{0.000000in}{0.000000in}}%
\pgfpathlineto{\pgfqpoint{0.000000in}{0.034722in}}%
\pgfusepath{stroke,fill}%
}%
\begin{pgfscope}%
\pgfsys@transformshift{3.382342in}{1.247073in}%
\pgfsys@useobject{currentmarker}{}%
\end{pgfscope}%
\end{pgfscope}%
\begin{pgfscope}%
\definecolor{textcolor}{rgb}{0.000000,0.000000,0.000000}%
\pgfsetstrokecolor{textcolor}%
\pgfsetfillcolor{textcolor}%
\pgftext[x=3.382342in,y=1.198462in,,top]{\color{textcolor}\sffamily\fontsize{18.000000}{21.600000}\selectfont $\displaystyle 0$}%
\end{pgfscope}%
\begin{pgfscope}%
\pgfpathrectangle{\pgfqpoint{3.133636in}{1.247073in}}{\pgfqpoint{8.787624in}{6.674186in}}%
\pgfusepath{clip}%
\pgfsetrectcap%
\pgfsetroundjoin%
\pgfsetlinewidth{0.501875pt}%
\definecolor{currentstroke}{rgb}{0.000000,0.000000,0.000000}%
\pgfsetstrokecolor{currentstroke}%
\pgfsetstrokeopacity{0.100000}%
\pgfsetdash{}{0pt}%
\pgfpathmoveto{\pgfqpoint{4.701770in}{1.247073in}}%
\pgfpathlineto{\pgfqpoint{4.701770in}{7.921260in}}%
\pgfusepath{stroke}%
\end{pgfscope}%
\begin{pgfscope}%
\pgfsetbuttcap%
\pgfsetroundjoin%
\definecolor{currentfill}{rgb}{0.000000,0.000000,0.000000}%
\pgfsetfillcolor{currentfill}%
\pgfsetlinewidth{0.501875pt}%
\definecolor{currentstroke}{rgb}{0.000000,0.000000,0.000000}%
\pgfsetstrokecolor{currentstroke}%
\pgfsetdash{}{0pt}%
\pgfsys@defobject{currentmarker}{\pgfqpoint{0.000000in}{0.000000in}}{\pgfqpoint{0.000000in}{0.034722in}}{%
\pgfpathmoveto{\pgfqpoint{0.000000in}{0.000000in}}%
\pgfpathlineto{\pgfqpoint{0.000000in}{0.034722in}}%
\pgfusepath{stroke,fill}%
}%
\begin{pgfscope}%
\pgfsys@transformshift{4.701770in}{1.247073in}%
\pgfsys@useobject{currentmarker}{}%
\end{pgfscope}%
\end{pgfscope}%
\begin{pgfscope}%
\definecolor{textcolor}{rgb}{0.000000,0.000000,0.000000}%
\pgfsetstrokecolor{textcolor}%
\pgfsetfillcolor{textcolor}%
\pgftext[x=4.701770in,y=1.198462in,,top]{\color{textcolor}\sffamily\fontsize{18.000000}{21.600000}\selectfont $\displaystyle 1$}%
\end{pgfscope}%
\begin{pgfscope}%
\pgfpathrectangle{\pgfqpoint{3.133636in}{1.247073in}}{\pgfqpoint{8.787624in}{6.674186in}}%
\pgfusepath{clip}%
\pgfsetrectcap%
\pgfsetroundjoin%
\pgfsetlinewidth{0.501875pt}%
\definecolor{currentstroke}{rgb}{0.000000,0.000000,0.000000}%
\pgfsetstrokecolor{currentstroke}%
\pgfsetstrokeopacity{0.100000}%
\pgfsetdash{}{0pt}%
\pgfpathmoveto{\pgfqpoint{6.021198in}{1.247073in}}%
\pgfpathlineto{\pgfqpoint{6.021198in}{7.921260in}}%
\pgfusepath{stroke}%
\end{pgfscope}%
\begin{pgfscope}%
\pgfsetbuttcap%
\pgfsetroundjoin%
\definecolor{currentfill}{rgb}{0.000000,0.000000,0.000000}%
\pgfsetfillcolor{currentfill}%
\pgfsetlinewidth{0.501875pt}%
\definecolor{currentstroke}{rgb}{0.000000,0.000000,0.000000}%
\pgfsetstrokecolor{currentstroke}%
\pgfsetdash{}{0pt}%
\pgfsys@defobject{currentmarker}{\pgfqpoint{0.000000in}{0.000000in}}{\pgfqpoint{0.000000in}{0.034722in}}{%
\pgfpathmoveto{\pgfqpoint{0.000000in}{0.000000in}}%
\pgfpathlineto{\pgfqpoint{0.000000in}{0.034722in}}%
\pgfusepath{stroke,fill}%
}%
\begin{pgfscope}%
\pgfsys@transformshift{6.021198in}{1.247073in}%
\pgfsys@useobject{currentmarker}{}%
\end{pgfscope}%
\end{pgfscope}%
\begin{pgfscope}%
\definecolor{textcolor}{rgb}{0.000000,0.000000,0.000000}%
\pgfsetstrokecolor{textcolor}%
\pgfsetfillcolor{textcolor}%
\pgftext[x=6.021198in,y=1.198462in,,top]{\color{textcolor}\sffamily\fontsize{18.000000}{21.600000}\selectfont $\displaystyle 2$}%
\end{pgfscope}%
\begin{pgfscope}%
\pgfpathrectangle{\pgfqpoint{3.133636in}{1.247073in}}{\pgfqpoint{8.787624in}{6.674186in}}%
\pgfusepath{clip}%
\pgfsetrectcap%
\pgfsetroundjoin%
\pgfsetlinewidth{0.501875pt}%
\definecolor{currentstroke}{rgb}{0.000000,0.000000,0.000000}%
\pgfsetstrokecolor{currentstroke}%
\pgfsetstrokeopacity{0.100000}%
\pgfsetdash{}{0pt}%
\pgfpathmoveto{\pgfqpoint{7.340627in}{1.247073in}}%
\pgfpathlineto{\pgfqpoint{7.340627in}{7.921260in}}%
\pgfusepath{stroke}%
\end{pgfscope}%
\begin{pgfscope}%
\pgfsetbuttcap%
\pgfsetroundjoin%
\definecolor{currentfill}{rgb}{0.000000,0.000000,0.000000}%
\pgfsetfillcolor{currentfill}%
\pgfsetlinewidth{0.501875pt}%
\definecolor{currentstroke}{rgb}{0.000000,0.000000,0.000000}%
\pgfsetstrokecolor{currentstroke}%
\pgfsetdash{}{0pt}%
\pgfsys@defobject{currentmarker}{\pgfqpoint{0.000000in}{0.000000in}}{\pgfqpoint{0.000000in}{0.034722in}}{%
\pgfpathmoveto{\pgfqpoint{0.000000in}{0.000000in}}%
\pgfpathlineto{\pgfqpoint{0.000000in}{0.034722in}}%
\pgfusepath{stroke,fill}%
}%
\begin{pgfscope}%
\pgfsys@transformshift{7.340627in}{1.247073in}%
\pgfsys@useobject{currentmarker}{}%
\end{pgfscope}%
\end{pgfscope}%
\begin{pgfscope}%
\definecolor{textcolor}{rgb}{0.000000,0.000000,0.000000}%
\pgfsetstrokecolor{textcolor}%
\pgfsetfillcolor{textcolor}%
\pgftext[x=7.340627in,y=1.198462in,,top]{\color{textcolor}\sffamily\fontsize{18.000000}{21.600000}\selectfont $\displaystyle 3$}%
\end{pgfscope}%
\begin{pgfscope}%
\pgfpathrectangle{\pgfqpoint{3.133636in}{1.247073in}}{\pgfqpoint{8.787624in}{6.674186in}}%
\pgfusepath{clip}%
\pgfsetrectcap%
\pgfsetroundjoin%
\pgfsetlinewidth{0.501875pt}%
\definecolor{currentstroke}{rgb}{0.000000,0.000000,0.000000}%
\pgfsetstrokecolor{currentstroke}%
\pgfsetstrokeopacity{0.100000}%
\pgfsetdash{}{0pt}%
\pgfpathmoveto{\pgfqpoint{8.660055in}{1.247073in}}%
\pgfpathlineto{\pgfqpoint{8.660055in}{7.921260in}}%
\pgfusepath{stroke}%
\end{pgfscope}%
\begin{pgfscope}%
\pgfsetbuttcap%
\pgfsetroundjoin%
\definecolor{currentfill}{rgb}{0.000000,0.000000,0.000000}%
\pgfsetfillcolor{currentfill}%
\pgfsetlinewidth{0.501875pt}%
\definecolor{currentstroke}{rgb}{0.000000,0.000000,0.000000}%
\pgfsetstrokecolor{currentstroke}%
\pgfsetdash{}{0pt}%
\pgfsys@defobject{currentmarker}{\pgfqpoint{0.000000in}{0.000000in}}{\pgfqpoint{0.000000in}{0.034722in}}{%
\pgfpathmoveto{\pgfqpoint{0.000000in}{0.000000in}}%
\pgfpathlineto{\pgfqpoint{0.000000in}{0.034722in}}%
\pgfusepath{stroke,fill}%
}%
\begin{pgfscope}%
\pgfsys@transformshift{8.660055in}{1.247073in}%
\pgfsys@useobject{currentmarker}{}%
\end{pgfscope}%
\end{pgfscope}%
\begin{pgfscope}%
\definecolor{textcolor}{rgb}{0.000000,0.000000,0.000000}%
\pgfsetstrokecolor{textcolor}%
\pgfsetfillcolor{textcolor}%
\pgftext[x=8.660055in,y=1.198462in,,top]{\color{textcolor}\sffamily\fontsize{18.000000}{21.600000}\selectfont $\displaystyle 4$}%
\end{pgfscope}%
\begin{pgfscope}%
\pgfpathrectangle{\pgfqpoint{3.133636in}{1.247073in}}{\pgfqpoint{8.787624in}{6.674186in}}%
\pgfusepath{clip}%
\pgfsetrectcap%
\pgfsetroundjoin%
\pgfsetlinewidth{0.501875pt}%
\definecolor{currentstroke}{rgb}{0.000000,0.000000,0.000000}%
\pgfsetstrokecolor{currentstroke}%
\pgfsetstrokeopacity{0.100000}%
\pgfsetdash{}{0pt}%
\pgfpathmoveto{\pgfqpoint{9.979483in}{1.247073in}}%
\pgfpathlineto{\pgfqpoint{9.979483in}{7.921260in}}%
\pgfusepath{stroke}%
\end{pgfscope}%
\begin{pgfscope}%
\pgfsetbuttcap%
\pgfsetroundjoin%
\definecolor{currentfill}{rgb}{0.000000,0.000000,0.000000}%
\pgfsetfillcolor{currentfill}%
\pgfsetlinewidth{0.501875pt}%
\definecolor{currentstroke}{rgb}{0.000000,0.000000,0.000000}%
\pgfsetstrokecolor{currentstroke}%
\pgfsetdash{}{0pt}%
\pgfsys@defobject{currentmarker}{\pgfqpoint{0.000000in}{0.000000in}}{\pgfqpoint{0.000000in}{0.034722in}}{%
\pgfpathmoveto{\pgfqpoint{0.000000in}{0.000000in}}%
\pgfpathlineto{\pgfqpoint{0.000000in}{0.034722in}}%
\pgfusepath{stroke,fill}%
}%
\begin{pgfscope}%
\pgfsys@transformshift{9.979483in}{1.247073in}%
\pgfsys@useobject{currentmarker}{}%
\end{pgfscope}%
\end{pgfscope}%
\begin{pgfscope}%
\definecolor{textcolor}{rgb}{0.000000,0.000000,0.000000}%
\pgfsetstrokecolor{textcolor}%
\pgfsetfillcolor{textcolor}%
\pgftext[x=9.979483in,y=1.198462in,,top]{\color{textcolor}\sffamily\fontsize{18.000000}{21.600000}\selectfont $\displaystyle 5$}%
\end{pgfscope}%
\begin{pgfscope}%
\pgfpathrectangle{\pgfqpoint{3.133636in}{1.247073in}}{\pgfqpoint{8.787624in}{6.674186in}}%
\pgfusepath{clip}%
\pgfsetrectcap%
\pgfsetroundjoin%
\pgfsetlinewidth{0.501875pt}%
\definecolor{currentstroke}{rgb}{0.000000,0.000000,0.000000}%
\pgfsetstrokecolor{currentstroke}%
\pgfsetstrokeopacity{0.100000}%
\pgfsetdash{}{0pt}%
\pgfpathmoveto{\pgfqpoint{11.298911in}{1.247073in}}%
\pgfpathlineto{\pgfqpoint{11.298911in}{7.921260in}}%
\pgfusepath{stroke}%
\end{pgfscope}%
\begin{pgfscope}%
\pgfsetbuttcap%
\pgfsetroundjoin%
\definecolor{currentfill}{rgb}{0.000000,0.000000,0.000000}%
\pgfsetfillcolor{currentfill}%
\pgfsetlinewidth{0.501875pt}%
\definecolor{currentstroke}{rgb}{0.000000,0.000000,0.000000}%
\pgfsetstrokecolor{currentstroke}%
\pgfsetdash{}{0pt}%
\pgfsys@defobject{currentmarker}{\pgfqpoint{0.000000in}{0.000000in}}{\pgfqpoint{0.000000in}{0.034722in}}{%
\pgfpathmoveto{\pgfqpoint{0.000000in}{0.000000in}}%
\pgfpathlineto{\pgfqpoint{0.000000in}{0.034722in}}%
\pgfusepath{stroke,fill}%
}%
\begin{pgfscope}%
\pgfsys@transformshift{11.298911in}{1.247073in}%
\pgfsys@useobject{currentmarker}{}%
\end{pgfscope}%
\end{pgfscope}%
\begin{pgfscope}%
\definecolor{textcolor}{rgb}{0.000000,0.000000,0.000000}%
\pgfsetstrokecolor{textcolor}%
\pgfsetfillcolor{textcolor}%
\pgftext[x=11.298911in,y=1.198462in,,top]{\color{textcolor}\sffamily\fontsize{18.000000}{21.600000}\selectfont $\displaystyle 6$}%
\end{pgfscope}%
\begin{pgfscope}%
\definecolor{textcolor}{rgb}{0.000000,0.000000,0.000000}%
\pgfsetstrokecolor{textcolor}%
\pgfsetfillcolor{textcolor}%
\pgftext[x=7.527448in,y=0.900964in,,top]{\color{textcolor}\sffamily\fontsize{18.000000}{21.600000}\selectfont $\displaystyle x$}%
\end{pgfscope}%
\begin{pgfscope}%
\pgfpathrectangle{\pgfqpoint{3.133636in}{1.247073in}}{\pgfqpoint{8.787624in}{6.674186in}}%
\pgfusepath{clip}%
\pgfsetrectcap%
\pgfsetroundjoin%
\pgfsetlinewidth{0.501875pt}%
\definecolor{currentstroke}{rgb}{0.000000,0.000000,0.000000}%
\pgfsetstrokecolor{currentstroke}%
\pgfsetstrokeopacity{0.100000}%
\pgfsetdash{}{0pt}%
\pgfpathmoveto{\pgfqpoint{3.133636in}{1.556472in}}%
\pgfpathlineto{\pgfqpoint{11.921260in}{1.556472in}}%
\pgfusepath{stroke}%
\end{pgfscope}%
\begin{pgfscope}%
\pgfsetbuttcap%
\pgfsetroundjoin%
\definecolor{currentfill}{rgb}{0.000000,0.000000,0.000000}%
\pgfsetfillcolor{currentfill}%
\pgfsetlinewidth{0.501875pt}%
\definecolor{currentstroke}{rgb}{0.000000,0.000000,0.000000}%
\pgfsetstrokecolor{currentstroke}%
\pgfsetdash{}{0pt}%
\pgfsys@defobject{currentmarker}{\pgfqpoint{0.000000in}{0.000000in}}{\pgfqpoint{0.034722in}{0.000000in}}{%
\pgfpathmoveto{\pgfqpoint{0.000000in}{0.000000in}}%
\pgfpathlineto{\pgfqpoint{0.034722in}{0.000000in}}%
\pgfusepath{stroke,fill}%
}%
\begin{pgfscope}%
\pgfsys@transformshift{3.133636in}{1.556472in}%
\pgfsys@useobject{currentmarker}{}%
\end{pgfscope}%
\end{pgfscope}%
\begin{pgfscope}%
\definecolor{textcolor}{rgb}{0.000000,0.000000,0.000000}%
\pgfsetstrokecolor{textcolor}%
\pgfsetfillcolor{textcolor}%
\pgftext[x=1.931602in, y=1.461501in, left, base]{\color{textcolor}\sffamily\fontsize{18.000000}{21.600000}\selectfont $\displaystyle -2.0×10^{200}$}%
\end{pgfscope}%
\begin{pgfscope}%
\pgfpathrectangle{\pgfqpoint{3.133636in}{1.247073in}}{\pgfqpoint{8.787624in}{6.674186in}}%
\pgfusepath{clip}%
\pgfsetrectcap%
\pgfsetroundjoin%
\pgfsetlinewidth{0.501875pt}%
\definecolor{currentstroke}{rgb}{0.000000,0.000000,0.000000}%
\pgfsetstrokecolor{currentstroke}%
\pgfsetstrokeopacity{0.100000}%
\pgfsetdash{}{0pt}%
\pgfpathmoveto{\pgfqpoint{3.133636in}{3.074076in}}%
\pgfpathlineto{\pgfqpoint{11.921260in}{3.074076in}}%
\pgfusepath{stroke}%
\end{pgfscope}%
\begin{pgfscope}%
\pgfsetbuttcap%
\pgfsetroundjoin%
\definecolor{currentfill}{rgb}{0.000000,0.000000,0.000000}%
\pgfsetfillcolor{currentfill}%
\pgfsetlinewidth{0.501875pt}%
\definecolor{currentstroke}{rgb}{0.000000,0.000000,0.000000}%
\pgfsetstrokecolor{currentstroke}%
\pgfsetdash{}{0pt}%
\pgfsys@defobject{currentmarker}{\pgfqpoint{0.000000in}{0.000000in}}{\pgfqpoint{0.034722in}{0.000000in}}{%
\pgfpathmoveto{\pgfqpoint{0.000000in}{0.000000in}}%
\pgfpathlineto{\pgfqpoint{0.034722in}{0.000000in}}%
\pgfusepath{stroke,fill}%
}%
\begin{pgfscope}%
\pgfsys@transformshift{3.133636in}{3.074076in}%
\pgfsys@useobject{currentmarker}{}%
\end{pgfscope}%
\end{pgfscope}%
\begin{pgfscope}%
\definecolor{textcolor}{rgb}{0.000000,0.000000,0.000000}%
\pgfsetstrokecolor{textcolor}%
\pgfsetfillcolor{textcolor}%
\pgftext[x=1.931602in, y=2.979105in, left, base]{\color{textcolor}\sffamily\fontsize{18.000000}{21.600000}\selectfont $\displaystyle -1.0×10^{200}$}%
\end{pgfscope}%
\begin{pgfscope}%
\pgfpathrectangle{\pgfqpoint{3.133636in}{1.247073in}}{\pgfqpoint{8.787624in}{6.674186in}}%
\pgfusepath{clip}%
\pgfsetrectcap%
\pgfsetroundjoin%
\pgfsetlinewidth{0.501875pt}%
\definecolor{currentstroke}{rgb}{0.000000,0.000000,0.000000}%
\pgfsetstrokecolor{currentstroke}%
\pgfsetstrokeopacity{0.100000}%
\pgfsetdash{}{0pt}%
\pgfpathmoveto{\pgfqpoint{3.133636in}{4.591680in}}%
\pgfpathlineto{\pgfqpoint{11.921260in}{4.591680in}}%
\pgfusepath{stroke}%
\end{pgfscope}%
\begin{pgfscope}%
\pgfsetbuttcap%
\pgfsetroundjoin%
\definecolor{currentfill}{rgb}{0.000000,0.000000,0.000000}%
\pgfsetfillcolor{currentfill}%
\pgfsetlinewidth{0.501875pt}%
\definecolor{currentstroke}{rgb}{0.000000,0.000000,0.000000}%
\pgfsetstrokecolor{currentstroke}%
\pgfsetdash{}{0pt}%
\pgfsys@defobject{currentmarker}{\pgfqpoint{0.000000in}{0.000000in}}{\pgfqpoint{0.034722in}{0.000000in}}{%
\pgfpathmoveto{\pgfqpoint{0.000000in}{0.000000in}}%
\pgfpathlineto{\pgfqpoint{0.034722in}{0.000000in}}%
\pgfusepath{stroke,fill}%
}%
\begin{pgfscope}%
\pgfsys@transformshift{3.133636in}{4.591680in}%
\pgfsys@useobject{currentmarker}{}%
\end{pgfscope}%
\end{pgfscope}%
\begin{pgfscope}%
\definecolor{textcolor}{rgb}{0.000000,0.000000,0.000000}%
\pgfsetstrokecolor{textcolor}%
\pgfsetfillcolor{textcolor}%
\pgftext[x=2.974957in, y=4.496709in, left, base]{\color{textcolor}\sffamily\fontsize{18.000000}{21.600000}\selectfont $\displaystyle 0$}%
\end{pgfscope}%
\begin{pgfscope}%
\pgfpathrectangle{\pgfqpoint{3.133636in}{1.247073in}}{\pgfqpoint{8.787624in}{6.674186in}}%
\pgfusepath{clip}%
\pgfsetrectcap%
\pgfsetroundjoin%
\pgfsetlinewidth{0.501875pt}%
\definecolor{currentstroke}{rgb}{0.000000,0.000000,0.000000}%
\pgfsetstrokecolor{currentstroke}%
\pgfsetstrokeopacity{0.100000}%
\pgfsetdash{}{0pt}%
\pgfpathmoveto{\pgfqpoint{3.133636in}{6.109283in}}%
\pgfpathlineto{\pgfqpoint{11.921260in}{6.109283in}}%
\pgfusepath{stroke}%
\end{pgfscope}%
\begin{pgfscope}%
\pgfsetbuttcap%
\pgfsetroundjoin%
\definecolor{currentfill}{rgb}{0.000000,0.000000,0.000000}%
\pgfsetfillcolor{currentfill}%
\pgfsetlinewidth{0.501875pt}%
\definecolor{currentstroke}{rgb}{0.000000,0.000000,0.000000}%
\pgfsetstrokecolor{currentstroke}%
\pgfsetdash{}{0pt}%
\pgfsys@defobject{currentmarker}{\pgfqpoint{0.000000in}{0.000000in}}{\pgfqpoint{0.034722in}{0.000000in}}{%
\pgfpathmoveto{\pgfqpoint{0.000000in}{0.000000in}}%
\pgfpathlineto{\pgfqpoint{0.034722in}{0.000000in}}%
\pgfusepath{stroke,fill}%
}%
\begin{pgfscope}%
\pgfsys@transformshift{3.133636in}{6.109283in}%
\pgfsys@useobject{currentmarker}{}%
\end{pgfscope}%
\end{pgfscope}%
\begin{pgfscope}%
\definecolor{textcolor}{rgb}{0.000000,0.000000,0.000000}%
\pgfsetstrokecolor{textcolor}%
\pgfsetfillcolor{textcolor}%
\pgftext[x=2.118269in, y=6.014313in, left, base]{\color{textcolor}\sffamily\fontsize{18.000000}{21.600000}\selectfont $\displaystyle 1.0×10^{200}$}%
\end{pgfscope}%
\begin{pgfscope}%
\pgfpathrectangle{\pgfqpoint{3.133636in}{1.247073in}}{\pgfqpoint{8.787624in}{6.674186in}}%
\pgfusepath{clip}%
\pgfsetrectcap%
\pgfsetroundjoin%
\pgfsetlinewidth{0.501875pt}%
\definecolor{currentstroke}{rgb}{0.000000,0.000000,0.000000}%
\pgfsetstrokecolor{currentstroke}%
\pgfsetstrokeopacity{0.100000}%
\pgfsetdash{}{0pt}%
\pgfpathmoveto{\pgfqpoint{3.133636in}{7.626887in}}%
\pgfpathlineto{\pgfqpoint{11.921260in}{7.626887in}}%
\pgfusepath{stroke}%
\end{pgfscope}%
\begin{pgfscope}%
\pgfsetbuttcap%
\pgfsetroundjoin%
\definecolor{currentfill}{rgb}{0.000000,0.000000,0.000000}%
\pgfsetfillcolor{currentfill}%
\pgfsetlinewidth{0.501875pt}%
\definecolor{currentstroke}{rgb}{0.000000,0.000000,0.000000}%
\pgfsetstrokecolor{currentstroke}%
\pgfsetdash{}{0pt}%
\pgfsys@defobject{currentmarker}{\pgfqpoint{0.000000in}{0.000000in}}{\pgfqpoint{0.034722in}{0.000000in}}{%
\pgfpathmoveto{\pgfqpoint{0.000000in}{0.000000in}}%
\pgfpathlineto{\pgfqpoint{0.034722in}{0.000000in}}%
\pgfusepath{stroke,fill}%
}%
\begin{pgfscope}%
\pgfsys@transformshift{3.133636in}{7.626887in}%
\pgfsys@useobject{currentmarker}{}%
\end{pgfscope}%
\end{pgfscope}%
\begin{pgfscope}%
\definecolor{textcolor}{rgb}{0.000000,0.000000,0.000000}%
\pgfsetstrokecolor{textcolor}%
\pgfsetfillcolor{textcolor}%
\pgftext[x=2.118269in, y=7.531916in, left, base]{\color{textcolor}\sffamily\fontsize{18.000000}{21.600000}\selectfont $\displaystyle 2.0×10^{200}$}%
\end{pgfscope}%
\begin{pgfscope}%
\pgfpathrectangle{\pgfqpoint{3.133636in}{1.247073in}}{\pgfqpoint{8.787624in}{6.674186in}}%
\pgfusepath{clip}%
\pgfsetbuttcap%
\pgfsetroundjoin%
\pgfsetlinewidth{1.003750pt}%
\definecolor{currentstroke}{rgb}{0.000000,0.605603,0.978680}%
\pgfsetstrokecolor{currentstroke}%
\pgfsetdash{}{0pt}%
\pgfpathmoveto{\pgfqpoint{3.382342in}{4.496043in}}%
\pgfpathlineto{\pgfqpoint{3.390438in}{4.842406in}}%
\pgfpathlineto{\pgfqpoint{3.398534in}{4.178833in}}%
\pgfpathlineto{\pgfqpoint{3.406630in}{5.170761in}}%
\pgfpathlineto{\pgfqpoint{3.414726in}{3.845261in}}%
\pgfpathlineto{\pgfqpoint{3.422822in}{5.503543in}}%
\pgfpathlineto{\pgfqpoint{3.430918in}{3.519131in}}%
\pgfpathlineto{\pgfqpoint{3.439014in}{5.817494in}}%
\pgfpathlineto{\pgfqpoint{3.447109in}{3.222406in}}%
\pgfpathlineto{\pgfqpoint{3.455205in}{6.092530in}}%
\pgfpathlineto{\pgfqpoint{3.463301in}{2.972885in}}%
\pgfpathlineto{\pgfqpoint{3.471397in}{6.313360in}}%
\pgfpathlineto{\pgfqpoint{3.479493in}{2.783295in}}%
\pgfpathlineto{\pgfqpoint{3.487589in}{6.469742in}}%
\pgfpathlineto{\pgfqpoint{3.495685in}{2.661560in}}%
\pgfpathlineto{\pgfqpoint{3.503781in}{6.555865in}}%
\pgfpathlineto{\pgfqpoint{3.511877in}{2.611577in}}%
\pgfpathlineto{\pgfqpoint{3.519973in}{6.569596in}}%
\pgfpathlineto{\pgfqpoint{3.528069in}{2.633793in}}%
\pgfpathlineto{\pgfqpoint{3.536164in}{6.512191in}}%
\pgfpathlineto{\pgfqpoint{3.544260in}{2.725132in}}%
\pgfpathlineto{\pgfqpoint{3.552356in}{6.388730in}}%
\pgfpathlineto{\pgfqpoint{3.560452in}{2.878280in}}%
\pgfpathlineto{\pgfqpoint{3.568548in}{6.208998in}}%
\pgfpathlineto{\pgfqpoint{3.576644in}{3.080803in}}%
\pgfpathlineto{\pgfqpoint{3.584740in}{5.988156in}}%
\pgfpathlineto{\pgfqpoint{3.592836in}{3.314873in}}%
\pgfpathlineto{\pgfqpoint{3.600932in}{5.746454in}}%
\pgfpathlineto{\pgfqpoint{3.609028in}{3.558265in}}%
\pgfpathlineto{\pgfqpoint{3.617124in}{5.507449in}}%
\pgfpathlineto{\pgfqpoint{3.625219in}{3.786920in}}%
\pgfpathlineto{\pgfqpoint{3.633315in}{5.294733in}}%
\pgfpathlineto{\pgfqpoint{3.641411in}{3.978756in}}%
\pgfpathlineto{\pgfqpoint{3.649507in}{5.127819in}}%
\pgfpathlineto{\pgfqpoint{3.657603in}{4.117817in}}%
\pgfpathlineto{\pgfqpoint{3.665699in}{5.018271in}}%
\pgfpathlineto{\pgfqpoint{3.673795in}{4.197543in}}%
\pgfpathlineto{\pgfqpoint{3.681891in}{4.967322in}}%
\pgfpathlineto{\pgfqpoint{3.689987in}{4.222023in}}%
\pgfpathlineto{\pgfqpoint{3.698083in}{4.965914in}}%
\pgfpathlineto{\pgfqpoint{3.706179in}{4.204590in}}%
\pgfpathlineto{\pgfqpoint{3.714274in}{4.997438in}}%
\pgfpathlineto{\pgfqpoint{3.722370in}{4.163886in}}%
\pgfpathlineto{\pgfqpoint{3.730466in}{5.042625in}}%
\pgfpathlineto{\pgfqpoint{3.738562in}{4.118334in}}%
\pgfpathlineto{\pgfqpoint{3.746658in}{5.085330in}}%
\pgfpathlineto{\pgfqpoint{3.754754in}{4.080517in}}%
\pgfpathlineto{\pgfqpoint{3.762850in}{5.117572in}}%
\pgfpathlineto{\pgfqpoint{3.770946in}{4.053100in}}%
\pgfpathlineto{\pgfqpoint{3.779042in}{5.142328in}}%
\pgfpathlineto{\pgfqpoint{3.787138in}{4.027557in}}%
\pgfpathlineto{\pgfqpoint{3.795234in}{5.173157in}}%
\pgfpathlineto{\pgfqpoint{3.803329in}{3.986210in}}%
\pgfpathlineto{\pgfqpoint{3.811425in}{5.230607in}}%
\pgfpathlineto{\pgfqpoint{3.819521in}{3.907142in}}%
\pgfpathlineto{\pgfqpoint{3.827617in}{5.336308in}}%
\pgfpathlineto{\pgfqpoint{3.835713in}{3.770710in}}%
\pgfpathlineto{\pgfqpoint{3.843809in}{5.506288in}}%
\pgfpathlineto{\pgfqpoint{3.851905in}{3.565935in}}%
\pgfpathlineto{\pgfqpoint{3.860001in}{5.745330in}}%
\pgfpathlineto{\pgfqpoint{3.868097in}{3.295024in}}%
\pgfpathlineto{\pgfqpoint{3.876193in}{6.043864in}}%
\pgfpathlineto{\pgfqpoint{3.884289in}{2.974833in}}%
\pgfpathlineto{\pgfqpoint{3.892384in}{6.378257in}}%
\pgfpathlineto{\pgfqpoint{3.900480in}{2.634860in}}%
\pgfpathlineto{\pgfqpoint{3.908576in}{6.714409in}}%
\pgfpathlineto{\pgfqpoint{3.916672in}{2.312279in}}%
\pgfpathlineto{\pgfqpoint{3.924768in}{7.013770in}}%
\pgfpathlineto{\pgfqpoint{3.932864in}{2.045248in}}%
\pgfpathlineto{\pgfqpoint{3.940960in}{7.240305in}}%
\pgfpathlineto{\pgfqpoint{3.949056in}{1.866083in}}%
\pgfpathlineto{\pgfqpoint{3.957152in}{7.366795in}}%
\pgfpathlineto{\pgfqpoint{3.965248in}{1.795812in}}%
\pgfpathlineto{\pgfqpoint{3.973344in}{7.379161in}}%
\pgfpathlineto{\pgfqpoint{3.981439in}{1.841171in}}%
\pgfpathlineto{\pgfqpoint{3.989535in}{7.278042in}}%
\pgfpathlineto{\pgfqpoint{3.997631in}{1.994459in}}%
\pgfpathlineto{\pgfqpoint{4.005727in}{7.077582in}}%
\pgfpathlineto{\pgfqpoint{4.013823in}{2.235958in}}%
\pgfpathlineto{\pgfqpoint{4.021919in}{6.802010in}}%
\pgfpathlineto{\pgfqpoint{4.030015in}{2.538116in}}%
\pgfpathlineto{\pgfqpoint{4.038111in}{6.480962in}}%
\pgfpathlineto{\pgfqpoint{4.046207in}{2.870439in}}%
\pgfpathlineto{\pgfqpoint{4.054303in}{6.144641in}}%
\pgfpathlineto{\pgfqpoint{4.062399in}{3.204044in}}%
\pgfpathlineto{\pgfqpoint{4.070494in}{5.819732in}}%
\pgfpathlineto{\pgfqpoint{4.078590in}{3.515130in}}%
\pgfpathlineto{\pgfqpoint{4.086686in}{5.526663in}}%
\pgfpathlineto{\pgfqpoint{4.094782in}{3.786946in}}%
\pgfpathlineto{\pgfqpoint{4.102878in}{5.278396in}}%
\pgfpathlineto{\pgfqpoint{4.110974in}{4.010262in}}%
\pgfpathlineto{\pgfqpoint{4.119070in}{5.080619in}}%
\pgfpathlineto{\pgfqpoint{4.127166in}{4.182627in}}%
\pgfpathlineto{\pgfqpoint{4.135262in}{4.932932in}}%
\pgfpathlineto{\pgfqpoint{4.143358in}{4.306857in}}%
\pgfpathlineto{\pgfqpoint{4.151454in}{4.830565in}}%
\pgfpathlineto{\pgfqpoint{4.159549in}{4.389219in}}%
\pgfpathlineto{\pgfqpoint{4.167645in}{4.766185in}}%
\pgfpathlineto{\pgfqpoint{4.175741in}{4.437718in}}%
\pgfpathlineto{\pgfqpoint{4.183837in}{4.731462in}}%
\pgfpathlineto{\pgfqpoint{4.191933in}{4.460722in}}%
\pgfpathlineto{\pgfqpoint{4.200029in}{4.718216in}}%
\pgfpathlineto{\pgfqpoint{4.208125in}{4.466048in}}%
\pgfpathlineto{\pgfqpoint{4.216221in}{4.719111in}}%
\pgfpathlineto{\pgfqpoint{4.224317in}{4.460481in}}%
\pgfpathlineto{\pgfqpoint{4.232413in}{4.727949in}}%
\pgfpathlineto{\pgfqpoint{4.240509in}{4.449628in}}%
\pgfpathlineto{\pgfqpoint{4.248604in}{4.739697in}}%
\pgfpathlineto{\pgfqpoint{4.256700in}{4.437976in}}%
\pgfpathlineto{\pgfqpoint{4.264796in}{4.750382in}}%
\pgfpathlineto{\pgfqpoint{4.272892in}{4.429016in}}%
\pgfpathlineto{\pgfqpoint{4.280988in}{4.756967in}}%
\pgfpathlineto{\pgfqpoint{4.289084in}{4.425350in}}%
\pgfpathlineto{\pgfqpoint{4.297180in}{4.757276in}}%
\pgfpathlineto{\pgfqpoint{4.305276in}{4.428730in}}%
\pgfpathlineto{\pgfqpoint{4.313372in}{4.749988in}}%
\pgfpathlineto{\pgfqpoint{4.321468in}{4.440033in}}%
\pgfpathlineto{\pgfqpoint{4.329564in}{4.734679in}}%
\pgfpathlineto{\pgfqpoint{4.337659in}{4.459223in}}%
\pgfpathlineto{\pgfqpoint{4.345755in}{4.711849in}}%
\pgfpathlineto{\pgfqpoint{4.353851in}{4.485340in}}%
\pgfpathlineto{\pgfqpoint{4.361947in}{4.682897in}}%
\pgfpathlineto{\pgfqpoint{4.370043in}{4.516582in}}%
\pgfpathlineto{\pgfqpoint{4.378139in}{4.649985in}}%
\pgfpathlineto{\pgfqpoint{4.386235in}{4.550487in}}%
\pgfpathlineto{\pgfqpoint{4.394331in}{4.615800in}}%
\pgfpathlineto{\pgfqpoint{4.402427in}{4.584224in}}%
\pgfpathlineto{\pgfqpoint{4.410523in}{4.583232in}}%
\pgfpathlineto{\pgfqpoint{4.418619in}{4.614933in}}%
\pgfpathlineto{\pgfqpoint{4.426714in}{4.555023in}}%
\pgfpathlineto{\pgfqpoint{4.434810in}{4.640072in}}%
\pgfpathlineto{\pgfqpoint{4.442906in}{4.533440in}}%
\pgfpathlineto{\pgfqpoint{4.451002in}{4.657708in}}%
\pgfpathlineto{\pgfqpoint{4.459098in}{4.520036in}}%
\pgfpathlineto{\pgfqpoint{4.467194in}{4.666703in}}%
\pgfpathlineto{\pgfqpoint{4.475290in}{4.515517in}}%
\pgfpathlineto{\pgfqpoint{4.483386in}{4.666784in}}%
\pgfpathlineto{\pgfqpoint{4.491482in}{4.519736in}}%
\pgfpathlineto{\pgfqpoint{4.499578in}{4.658496in}}%
\pgfpathlineto{\pgfqpoint{4.507674in}{4.531784in}}%
\pgfpathlineto{\pgfqpoint{4.515769in}{4.643066in}}%
\pgfpathlineto{\pgfqpoint{4.523865in}{4.550162in}}%
\pgfpathlineto{\pgfqpoint{4.531961in}{4.622213in}}%
\pgfpathlineto{\pgfqpoint{4.540057in}{4.572988in}}%
\pgfpathlineto{\pgfqpoint{4.548153in}{4.597932in}}%
\pgfpathlineto{\pgfqpoint{4.556249in}{4.598204in}}%
\pgfpathlineto{\pgfqpoint{4.564345in}{4.572292in}}%
\pgfpathlineto{\pgfqpoint{4.572441in}{4.623775in}}%
\pgfpathlineto{\pgfqpoint{4.580537in}{4.547258in}}%
\pgfpathlineto{\pgfqpoint{4.588633in}{4.647838in}}%
\pgfpathlineto{\pgfqpoint{4.596729in}{4.524563in}}%
\pgfpathlineto{\pgfqpoint{4.604824in}{4.668812in}}%
\pgfpathlineto{\pgfqpoint{4.612920in}{4.505618in}}%
\pgfpathlineto{\pgfqpoint{4.621016in}{4.685464in}}%
\pgfpathlineto{\pgfqpoint{4.629112in}{4.491475in}}%
\pgfpathlineto{\pgfqpoint{4.637208in}{4.696929in}}%
\pgfpathlineto{\pgfqpoint{4.645304in}{4.482815in}}%
\pgfpathlineto{\pgfqpoint{4.653400in}{4.702700in}}%
\pgfpathlineto{\pgfqpoint{4.661496in}{4.479973in}}%
\pgfpathlineto{\pgfqpoint{4.669592in}{4.702610in}}%
\pgfpathlineto{\pgfqpoint{4.677688in}{4.482959in}}%
\pgfpathlineto{\pgfqpoint{4.685784in}{4.696800in}}%
\pgfpathlineto{\pgfqpoint{4.693879in}{4.491486in}}%
\pgfpathlineto{\pgfqpoint{4.701975in}{4.685697in}}%
\pgfpathlineto{\pgfqpoint{4.710071in}{4.504993in}}%
\pgfpathlineto{\pgfqpoint{4.718167in}{4.669986in}}%
\pgfpathlineto{\pgfqpoint{4.726263in}{4.522683in}}%
\pgfpathlineto{\pgfqpoint{4.734359in}{4.650563in}}%
\pgfpathlineto{\pgfqpoint{4.742455in}{4.543577in}}%
\pgfpathlineto{\pgfqpoint{4.750551in}{4.628467in}}%
\pgfpathlineto{\pgfqpoint{4.758647in}{4.566607in}}%
\pgfpathlineto{\pgfqpoint{4.766743in}{4.604763in}}%
\pgfpathlineto{\pgfqpoint{4.774839in}{4.590743in}}%
\pgfpathlineto{\pgfqpoint{4.782935in}{4.580410in}}%
\pgfpathlineto{\pgfqpoint{4.791030in}{4.615132in}}%
\pgfpathlineto{\pgfqpoint{4.799126in}{4.556125in}}%
\pgfpathlineto{\pgfqpoint{4.807222in}{4.639218in}}%
\pgfpathlineto{\pgfqpoint{4.815318in}{4.532292in}}%
\pgfpathlineto{\pgfqpoint{4.823414in}{4.662787in}}%
\pgfpathlineto{\pgfqpoint{4.831510in}{4.508961in}}%
\pgfpathlineto{\pgfqpoint{4.839606in}{4.685933in}}%
\pgfpathlineto{\pgfqpoint{4.847702in}{4.485930in}}%
\pgfpathlineto{\pgfqpoint{4.855798in}{4.708921in}}%
\pgfpathlineto{\pgfqpoint{4.863894in}{4.462920in}}%
\pgfpathlineto{\pgfqpoint{4.871990in}{4.731999in}}%
\pgfpathlineto{\pgfqpoint{4.880085in}{4.439759in}}%
\pgfpathlineto{\pgfqpoint{4.888181in}{4.755223in}}%
\pgfpathlineto{\pgfqpoint{4.896277in}{4.416531in}}%
\pgfpathlineto{\pgfqpoint{4.904373in}{4.778361in}}%
\pgfpathlineto{\pgfqpoint{4.912469in}{4.393604in}}%
\pgfpathlineto{\pgfqpoint{4.920565in}{4.800937in}}%
\pgfpathlineto{\pgfqpoint{4.928661in}{4.371520in}}%
\pgfpathlineto{\pgfqpoint{4.936757in}{4.822400in}}%
\pgfpathlineto{\pgfqpoint{4.944853in}{4.350778in}}%
\pgfpathlineto{\pgfqpoint{4.952949in}{4.842369in}}%
\pgfpathlineto{\pgfqpoint{4.961045in}{4.331578in}}%
\pgfpathlineto{\pgfqpoint{4.969140in}{4.860870in}}%
\pgfpathlineto{\pgfqpoint{4.977236in}{4.313639in}}%
\pgfpathlineto{\pgfqpoint{4.985332in}{4.878446in}}%
\pgfpathlineto{\pgfqpoint{4.993428in}{4.296174in}}%
\pgfpathlineto{\pgfqpoint{5.001524in}{4.896089in}}%
\pgfpathlineto{\pgfqpoint{5.009620in}{4.278049in}}%
\pgfpathlineto{\pgfqpoint{5.017716in}{4.914992in}}%
\pgfpathlineto{\pgfqpoint{5.025812in}{4.258104in}}%
\pgfpathlineto{\pgfqpoint{5.033908in}{4.936188in}}%
\pgfpathlineto{\pgfqpoint{5.042004in}{4.235523in}}%
\pgfpathlineto{\pgfqpoint{5.050100in}{4.960201in}}%
\pgfpathlineto{\pgfqpoint{5.058195in}{4.210125in}}%
\pgfpathlineto{\pgfqpoint{5.066291in}{4.986844in}}%
\pgfpathlineto{\pgfqpoint{5.074387in}{4.182460in}}%
\pgfpathlineto{\pgfqpoint{5.082483in}{5.015238in}}%
\pgfpathlineto{\pgfqpoint{5.090579in}{4.153674in}}%
\pgfpathlineto{\pgfqpoint{5.098675in}{5.044060in}}%
\pgfpathlineto{\pgfqpoint{5.106771in}{4.125165in}}%
\pgfpathlineto{\pgfqpoint{5.114867in}{5.071946in}}%
\pgfpathlineto{\pgfqpoint{5.122963in}{4.098150in}}%
\pgfpathlineto{\pgfqpoint{5.131059in}{5.097923in}}%
\pgfpathlineto{\pgfqpoint{5.139155in}{4.073284in}}%
\pgfpathlineto{\pgfqpoint{5.147250in}{5.121706in}}%
\pgfpathlineto{\pgfqpoint{5.155346in}{4.050458in}}%
\pgfpathlineto{\pgfqpoint{5.163442in}{5.143788in}}%
\pgfpathlineto{\pgfqpoint{5.171538in}{4.028836in}}%
\pgfpathlineto{\pgfqpoint{5.179634in}{5.165280in}}%
\pgfpathlineto{\pgfqpoint{5.187730in}{4.007122in}}%
\pgfpathlineto{\pgfqpoint{5.195826in}{5.187562in}}%
\pgfpathlineto{\pgfqpoint{5.203922in}{3.983962in}}%
\pgfpathlineto{\pgfqpoint{5.212018in}{5.211855in}}%
\pgfpathlineto{\pgfqpoint{5.220114in}{3.958360in}}%
\pgfpathlineto{\pgfqpoint{5.228210in}{5.238847in}}%
\pgfpathlineto{\pgfqpoint{5.236305in}{3.929996in}}%
\pgfpathlineto{\pgfqpoint{5.244401in}{5.268461in}}%
\pgfpathlineto{\pgfqpoint{5.252497in}{3.899356in}}%
\pgfpathlineto{\pgfqpoint{5.260593in}{5.299815in}}%
\pgfpathlineto{\pgfqpoint{5.268689in}{3.867677in}}%
\pgfpathlineto{\pgfqpoint{5.276785in}{5.331368in}}%
\pgfpathlineto{\pgfqpoint{5.284881in}{3.836735in}}%
\pgfpathlineto{\pgfqpoint{5.292977in}{5.361193in}}%
\pgfpathlineto{\pgfqpoint{5.301073in}{3.808530in}}%
\pgfpathlineto{\pgfqpoint{5.309169in}{5.387301in}}%
\pgfpathlineto{\pgfqpoint{5.317265in}{3.784957in}}%
\pgfpathlineto{\pgfqpoint{5.325360in}{5.407960in}}%
\pgfpathlineto{\pgfqpoint{5.333456in}{3.767517in}}%
\pgfpathlineto{\pgfqpoint{5.341552in}{5.421959in}}%
\pgfpathlineto{\pgfqpoint{5.349648in}{3.757092in}}%
\pgfpathlineto{\pgfqpoint{5.357744in}{5.428773in}}%
\pgfpathlineto{\pgfqpoint{5.365840in}{3.753831in}}%
\pgfpathlineto{\pgfqpoint{5.373936in}{5.428629in}}%
\pgfpathlineto{\pgfqpoint{5.382032in}{3.757146in}}%
\pgfpathlineto{\pgfqpoint{5.390128in}{5.422449in}}%
\pgfpathlineto{\pgfqpoint{5.398224in}{3.765826in}}%
\pgfpathlineto{\pgfqpoint{5.406320in}{5.411674in}}%
\pgfpathlineto{\pgfqpoint{5.414415in}{3.778274in}}%
\pgfpathlineto{\pgfqpoint{5.422511in}{5.397972in}}%
\pgfpathlineto{\pgfqpoint{5.430607in}{3.792841in}}%
\pgfpathlineto{\pgfqpoint{5.438703in}{5.382874in}}%
\pgfpathlineto{\pgfqpoint{5.446799in}{3.808212in}}%
\pgfpathlineto{\pgfqpoint{5.454895in}{5.367388in}}%
\pgfpathlineto{\pgfqpoint{5.462991in}{3.823771in}}%
\pgfpathlineto{\pgfqpoint{5.471087in}{5.351671in}}%
\pgfpathlineto{\pgfqpoint{5.479183in}{3.839862in}}%
\pgfpathlineto{\pgfqpoint{5.487279in}{5.334862in}}%
\pgfpathlineto{\pgfqpoint{5.495375in}{3.857853in}}%
\pgfpathlineto{\pgfqpoint{5.503470in}{5.315127in}}%
\pgfpathlineto{\pgfqpoint{5.511566in}{3.879968in}}%
\pgfpathlineto{\pgfqpoint{5.519662in}{5.289958in}}%
\pgfpathlineto{\pgfqpoint{5.527758in}{3.908865in}}%
\pgfpathlineto{\pgfqpoint{5.535854in}{5.256699in}}%
\pgfpathlineto{\pgfqpoint{5.543950in}{3.947033in}}%
\pgfpathlineto{\pgfqpoint{5.552046in}{5.213203in}}%
\pgfpathlineto{\pgfqpoint{5.560142in}{3.996107in}}%
\pgfpathlineto{\pgfqpoint{5.568238in}{5.158506in}}%
\pgfpathlineto{\pgfqpoint{5.576334in}{4.056242in}}%
\pgfpathlineto{\pgfqpoint{5.584430in}{5.093368in}}%
\pgfpathlineto{\pgfqpoint{5.592525in}{4.125691in}}%
\pgfpathlineto{\pgfqpoint{5.600621in}{5.020552in}}%
\pgfpathlineto{\pgfqpoint{5.608717in}{4.200692in}}%
\pgfpathlineto{\pgfqpoint{5.616813in}{4.944758in}}%
\pgfpathlineto{\pgfqpoint{5.624909in}{4.275718in}}%
\pgfpathlineto{\pgfqpoint{5.633005in}{4.872182in}}%
\pgfpathlineto{\pgfqpoint{5.641101in}{4.344098in}}%
\pgfpathlineto{\pgfqpoint{5.649197in}{4.809742in}}%
\pgfpathlineto{\pgfqpoint{5.657293in}{4.398920in}}%
\pgfpathlineto{\pgfqpoint{5.665389in}{4.764082in}}%
\pgfpathlineto{\pgfqpoint{5.673485in}{4.434078in}}%
\pgfpathlineto{\pgfqpoint{5.681580in}{4.740502in}}%
\pgfpathlineto{\pgfqpoint{5.689676in}{4.445319in}}%
\pgfpathlineto{\pgfqpoint{5.697772in}{4.742002in}}%
\pgfpathlineto{\pgfqpoint{5.705868in}{4.431067in}}%
\pgfpathlineto{\pgfqpoint{5.713964in}{4.768614in}}%
\pgfpathlineto{\pgfqpoint{5.722060in}{4.392889in}}%
\pgfpathlineto{\pgfqpoint{5.730156in}{4.817178in}}%
\pgfpathlineto{\pgfqpoint{5.738252in}{4.335467in}}%
\pgfpathlineto{\pgfqpoint{5.746348in}{4.881633in}}%
\pgfpathlineto{\pgfqpoint{5.754444in}{4.266038in}}%
\pgfpathlineto{\pgfqpoint{5.762540in}{4.953818in}}%
\pgfpathlineto{\pgfqpoint{5.770635in}{4.193387in}}%
\pgfpathlineto{\pgfqpoint{5.778731in}{5.024663in}}%
\pgfpathlineto{\pgfqpoint{5.786827in}{4.126517in}}%
\pgfpathlineto{\pgfqpoint{5.794923in}{5.085581in}}%
\pgfpathlineto{\pgfqpoint{5.803019in}{4.073262in}}%
\pgfpathlineto{\pgfqpoint{5.811115in}{5.129793in}}%
\pgfpathlineto{\pgfqpoint{5.819211in}{4.039089in}}%
\pgfpathlineto{\pgfqpoint{5.827307in}{5.153343in}}%
\pgfpathlineto{\pgfqpoint{5.835403in}{4.026319in}}%
\pgfpathlineto{\pgfqpoint{5.843499in}{5.155596in}}%
\pgfpathlineto{\pgfqpoint{5.851595in}{4.033925in}}%
\pgfpathlineto{\pgfqpoint{5.859690in}{5.139141in}}%
\pgfpathlineto{\pgfqpoint{5.867786in}{4.057926in}}%
\pgfpathlineto{\pgfqpoint{5.875882in}{5.109120in}}%
\pgfpathlineto{\pgfqpoint{5.883978in}{4.092296in}}%
\pgfpathlineto{\pgfqpoint{5.892074in}{5.072135in}}%
\pgfpathlineto{\pgfqpoint{5.900170in}{4.130179in}}%
\pgfpathlineto{\pgfqpoint{5.908266in}{5.034977in}}%
\pgfpathlineto{\pgfqpoint{5.916362in}{4.165154in}}%
\pgfpathlineto{\pgfqpoint{5.924458in}{5.003422in}}%
\pgfpathlineto{\pgfqpoint{5.932554in}{4.192315in}}%
\pgfpathlineto{\pgfqpoint{5.940650in}{4.981338in}}%
\pgfpathlineto{\pgfqpoint{5.948745in}{4.208945in}}%
\pgfpathlineto{\pgfqpoint{5.956841in}{4.970239in}}%
\pgfpathlineto{\pgfqpoint{5.964937in}{4.214714in}}%
\pgfpathlineto{\pgfqpoint{5.973033in}{4.969350in}}%
\pgfpathlineto{\pgfqpoint{5.981129in}{4.211378in}}%
\pgfpathlineto{\pgfqpoint{5.989225in}{4.976106in}}%
\pgfpathlineto{\pgfqpoint{5.997321in}{4.202100in}}%
\pgfpathlineto{\pgfqpoint{6.005417in}{4.986973in}}%
\pgfpathlineto{\pgfqpoint{6.013513in}{4.190553in}}%
\pgfpathlineto{\pgfqpoint{6.021609in}{4.998372in}}%
\pgfpathlineto{\pgfqpoint{6.029705in}{4.180003in}}%
\pgfpathlineto{\pgfqpoint{6.037800in}{5.007540in}}%
\pgfpathlineto{\pgfqpoint{6.045896in}{4.172554in}}%
\pgfpathlineto{\pgfqpoint{6.053992in}{5.013146in}}%
\pgfpathlineto{\pgfqpoint{6.062088in}{4.168693in}}%
\pgfpathlineto{\pgfqpoint{6.070184in}{5.015575in}}%
\pgfpathlineto{\pgfqpoint{6.078280in}{4.167189in}}%
\pgfpathlineto{\pgfqpoint{6.086376in}{5.016830in}}%
\pgfpathlineto{\pgfqpoint{6.094472in}{4.165373in}}%
\pgfpathlineto{\pgfqpoint{6.102568in}{5.020104in}}%
\pgfpathlineto{\pgfqpoint{6.110664in}{4.159700in}}%
\pgfpathlineto{\pgfqpoint{6.118760in}{5.029107in}}%
\pgfpathlineto{\pgfqpoint{6.126855in}{4.146496in}}%
\pgfpathlineto{\pgfqpoint{6.134951in}{5.047281in}}%
\pgfpathlineto{\pgfqpoint{6.143047in}{4.122733in}}%
\pgfpathlineto{\pgfqpoint{6.151143in}{5.077064in}}%
\pgfpathlineto{\pgfqpoint{6.159239in}{4.086709in}}%
\pgfpathlineto{\pgfqpoint{6.167335in}{5.119319in}}%
\pgfpathlineto{\pgfqpoint{6.175431in}{4.038474in}}%
\pgfpathlineto{\pgfqpoint{6.183527in}{5.173046in}}%
\pgfpathlineto{\pgfqpoint{6.191623in}{3.979965in}}%
\pgfpathlineto{\pgfqpoint{6.199719in}{5.235430in}}%
\pgfpathlineto{\pgfqpoint{6.207815in}{3.914779in}}%
\pgfpathlineto{\pgfqpoint{6.215910in}{5.302220in}}%
\pgfpathlineto{\pgfqpoint{6.224006in}{3.847657in}}%
\pgfpathlineto{\pgfqpoint{6.232102in}{5.368374in}}%
\pgfpathlineto{\pgfqpoint{6.240198in}{3.783745in}}%
\pgfpathlineto{\pgfqpoint{6.248294in}{5.428847in}}%
\pgfpathlineto{\pgfqpoint{6.256390in}{3.727789in}}%
\pgfpathlineto{\pgfqpoint{6.264486in}{5.479369in}}%
\pgfpathlineto{\pgfqpoint{6.272582in}{3.683428in}}%
\pgfpathlineto{\pgfqpoint{6.280678in}{5.517049in}}%
\pgfpathlineto{\pgfqpoint{6.288774in}{3.652734in}}%
\pgfpathlineto{\pgfqpoint{6.296870in}{5.540660in}}%
\pgfpathlineto{\pgfqpoint{6.304965in}{3.636116in}}%
\pgfpathlineto{\pgfqpoint{6.313061in}{5.550532in}}%
\pgfpathlineto{\pgfqpoint{6.321157in}{3.632625in}}%
\pgfpathlineto{\pgfqpoint{6.329253in}{5.548076in}}%
\pgfpathlineto{\pgfqpoint{6.337349in}{3.640577in}}%
\pgfpathlineto{\pgfqpoint{6.345445in}{5.535047in}}%
\pgfpathlineto{\pgfqpoint{6.353541in}{3.658349in}}%
\pgfpathlineto{\pgfqpoint{6.361637in}{5.512745in}}%
\pgfpathlineto{\pgfqpoint{6.369733in}{3.685122in}}%
\pgfpathlineto{\pgfqpoint{6.377829in}{5.481389in}}%
\pgfpathlineto{\pgfqpoint{6.385925in}{3.721344in}}%
\pgfpathlineto{\pgfqpoint{6.394020in}{5.439858in}}%
\pgfpathlineto{\pgfqpoint{6.402116in}{3.768759in}}%
\pgfpathlineto{\pgfqpoint{6.410212in}{5.385899in}}%
\pgfpathlineto{\pgfqpoint{6.418308in}{3.829956in}}%
\pgfpathlineto{\pgfqpoint{6.426404in}{5.316793in}}%
\pgfpathlineto{\pgfqpoint{6.434500in}{3.907547in}}%
\pgfpathlineto{\pgfqpoint{6.442596in}{5.230290in}}%
\pgfpathlineto{\pgfqpoint{6.450692in}{4.003180in}}%
\pgfpathlineto{\pgfqpoint{6.458788in}{5.125564in}}%
\pgfpathlineto{\pgfqpoint{6.466884in}{4.116680in}}%
\pgfpathlineto{\pgfqpoint{6.474980in}{5.003912in}}%
\pgfpathlineto{\pgfqpoint{6.483076in}{4.245561in}}%
\pgfpathlineto{\pgfqpoint{6.491171in}{4.869006in}}%
\pgfpathlineto{\pgfqpoint{6.499267in}{4.385041in}}%
\pgfpathlineto{\pgfqpoint{6.507363in}{4.726607in}}%
\pgfpathlineto{\pgfqpoint{6.515459in}{4.528557in}}%
\pgfpathlineto{\pgfqpoint{6.523555in}{4.583866in}}%
\pgfpathlineto{\pgfqpoint{6.531651in}{4.668601in}}%
\pgfpathlineto{\pgfqpoint{6.539747in}{4.448414in}}%
\pgfpathlineto{\pgfqpoint{6.547843in}{4.797640in}}%
\pgfpathlineto{\pgfqpoint{6.555939in}{4.327498in}}%
\pgfpathlineto{\pgfqpoint{6.564035in}{4.908861in}}%
\pgfpathlineto{\pgfqpoint{6.572131in}{4.227388in}}%
\pgfpathlineto{\pgfqpoint{6.580226in}{4.996602in}}%
\pgfpathlineto{\pgfqpoint{6.588322in}{4.153118in}}%
\pgfpathlineto{\pgfqpoint{6.596418in}{5.056450in}}%
\pgfpathlineto{\pgfqpoint{6.604514in}{4.108501in}}%
\pgfpathlineto{\pgfqpoint{6.612610in}{5.085163in}}%
\pgfpathlineto{\pgfqpoint{6.620706in}{4.096229in}}%
\pgfpathlineto{\pgfqpoint{6.628802in}{5.080605in}}%
\pgfpathlineto{\pgfqpoint{6.636898in}{4.117843in}}%
\pgfpathlineto{\pgfqpoint{6.644994in}{5.041901in}}%
\pgfpathlineto{\pgfqpoint{6.653090in}{4.173445in}}%
\pgfpathlineto{\pgfqpoint{6.661186in}{4.969854in}}%
\pgfpathlineto{\pgfqpoint{6.669281in}{4.261179in}}%
\pgfpathlineto{\pgfqpoint{6.677377in}{4.867522in}}%
\pgfpathlineto{\pgfqpoint{6.685473in}{4.376665in}}%
\pgfpathlineto{\pgfqpoint{6.693569in}{4.740686in}}%
\pgfpathlineto{\pgfqpoint{6.701665in}{4.512705in}}%
\pgfpathlineto{\pgfqpoint{6.709761in}{4.597894in}}%
\pgfpathlineto{\pgfqpoint{6.717857in}{4.659556in}}%
\pgfpathlineto{\pgfqpoint{6.725953in}{4.449833in}}%
\pgfpathlineto{\pgfqpoint{6.734049in}{4.805927in}}%
\pgfpathlineto{\pgfqpoint{6.742145in}{4.307985in}}%
\pgfpathlineto{\pgfqpoint{6.750241in}{4.940614in}}%
\pgfpathlineto{\pgfqpoint{6.758336in}{4.182782in}}%
\pgfpathlineto{\pgfqpoint{6.766432in}{5.054433in}}%
\pgfpathlineto{\pgfqpoint{6.774528in}{4.081730in}}%
\pgfpathlineto{\pgfqpoint{6.782624in}{5.141914in}}%
\pgfpathlineto{\pgfqpoint{6.790720in}{4.008022in}}%
\pgfpathlineto{\pgfqpoint{6.798816in}{5.202238in}}%
\pgfpathlineto{\pgfqpoint{6.806912in}{3.960151in}}%
\pgfpathlineto{\pgfqpoint{6.815008in}{5.239038in}}%
\pgfpathlineto{\pgfqpoint{6.823104in}{3.932706in}}%
\pgfpathlineto{\pgfqpoint{6.831200in}{5.259034in}}%
\pgfpathlineto{\pgfqpoint{6.839296in}{3.918217in}}%
\pgfpathlineto{\pgfqpoint{6.847391in}{5.269839in}}%
\pgfpathlineto{\pgfqpoint{6.855487in}{3.909536in}}%
\pgfpathlineto{\pgfqpoint{6.863583in}{5.277572in}}%
\pgfpathlineto{\pgfqpoint{6.871679in}{3.902044in}}%
\pgfpathlineto{\pgfqpoint{6.879775in}{5.285018in}}%
\pgfpathlineto{\pgfqpoint{6.887871in}{3.894965in}}%
\pgfpathlineto{\pgfqpoint{6.895967in}{5.290940in}}%
\pgfpathlineto{\pgfqpoint{6.904063in}{3.891369in}}%
\pgfpathlineto{\pgfqpoint{6.912159in}{5.290787in}}%
\pgfpathlineto{\pgfqpoint{6.920255in}{3.896804in}}%
\pgfpathlineto{\pgfqpoint{6.928351in}{5.278587in}}%
\pgfpathlineto{\pgfqpoint{6.936446in}{3.917040in}}%
\pgfpathlineto{\pgfqpoint{6.944542in}{5.249394in}}%
\pgfpathlineto{\pgfqpoint{6.952638in}{3.955648in}}%
\pgfpathlineto{\pgfqpoint{6.960734in}{5.201447in}}%
\pgfpathlineto{\pgfqpoint{6.968830in}{4.012304in}}%
\pgfpathlineto{\pgfqpoint{6.976926in}{5.137242in}}%
\pgfpathlineto{\pgfqpoint{6.985022in}{4.082449in}}%
\pgfpathlineto{\pgfqpoint{6.993118in}{5.063090in}}%
\pgfpathlineto{\pgfqpoint{7.001214in}{4.158509in}}%
\pgfpathlineto{\pgfqpoint{7.009310in}{4.987206in}}%
\pgfpathlineto{\pgfqpoint{7.017406in}{4.232339in}}%
\pgfpathlineto{\pgfqpoint{7.025501in}{4.916923in}}%
\pgfpathlineto{\pgfqpoint{7.033597in}{4.298119in}}%
\pgfpathlineto{\pgfqpoint{7.041693in}{4.855949in}}%
\pgfpathlineto{\pgfqpoint{7.049789in}{4.354696in}}%
\pgfpathlineto{\pgfqpoint{7.057885in}{4.802643in}}%
\pgfpathlineto{\pgfqpoint{7.065981in}{4.406515in}}%
\pgfpathlineto{\pgfqpoint{7.074077in}{4.749989in}}%
\pgfpathlineto{\pgfqpoint{7.082173in}{4.462695in}}%
\pgfpathlineto{\pgfqpoint{7.090269in}{4.687433in}}%
\pgfpathlineto{\pgfqpoint{7.098365in}{4.534400in}}%
\pgfpathlineto{\pgfqpoint{7.106461in}{4.604123in}}%
\pgfpathlineto{\pgfqpoint{7.114556in}{4.631226in}}%
\pgfpathlineto{\pgfqpoint{7.122652in}{4.492615in}}%
\pgfpathlineto{\pgfqpoint{7.130748in}{4.757690in}}%
\pgfpathlineto{\pgfqpoint{7.138844in}{4.351907in}}%
\pgfpathlineto{\pgfqpoint{7.146940in}{4.910924in}}%
\pgfpathlineto{\pgfqpoint{7.155036in}{4.188821in}}%
\pgfpathlineto{\pgfqpoint{7.163132in}{5.080347in}}%
\pgfpathlineto{\pgfqpoint{7.171228in}{4.017237in}}%
\pgfpathlineto{\pgfqpoint{7.179324in}{5.249482in}}%
\pgfpathlineto{\pgfqpoint{7.187420in}{3.855337in}}%
\pgfpathlineto{\pgfqpoint{7.195516in}{5.399454in}}%
\pgfpathlineto{\pgfqpoint{7.203611in}{3.721630in}}%
\pgfpathlineto{\pgfqpoint{7.211707in}{5.513153in}}%
\pgfpathlineto{\pgfqpoint{7.219803in}{3.630897in}}%
\pgfpathlineto{\pgfqpoint{7.227899in}{5.578883in}}%
\pgfpathlineto{\pgfqpoint{7.235995in}{3.591213in}}%
\pgfpathlineto{\pgfqpoint{7.244091in}{5.592470in}}%
\pgfpathlineto{\pgfqpoint{7.252187in}{3.602845in}}%
\pgfpathlineto{\pgfqpoint{7.260283in}{5.557303in}}%
\pgfpathlineto{\pgfqpoint{7.268379in}{3.659227in}}%
\pgfpathlineto{\pgfqpoint{7.276475in}{5.482464in}}%
\pgfpathlineto{\pgfqpoint{7.284571in}{3.749544in}}%
\pgfpathlineto{\pgfqpoint{7.292666in}{5.379663in}}%
\pgfpathlineto{\pgfqpoint{7.300762in}{3.862001in}}%
\pgfpathlineto{\pgfqpoint{7.308858in}{5.260059in}}%
\pgfpathlineto{\pgfqpoint{7.316954in}{3.986657in}}%
\pgfpathlineto{\pgfqpoint{7.325050in}{5.131989in}}%
\pgfpathlineto{\pgfqpoint{7.333146in}{4.116953in}}%
\pgfpathlineto{\pgfqpoint{7.341242in}{5.000271in}}%
\pgfpathlineto{\pgfqpoint{7.349338in}{4.249569in}}%
\pgfpathlineto{\pgfqpoint{7.357434in}{4.867131in}}%
\pgfpathlineto{\pgfqpoint{7.365530in}{4.382862in}}%
\pgfpathlineto{\pgfqpoint{7.373626in}{4.734198in}}%
\pgfpathlineto{\pgfqpoint{7.381721in}{4.514662in}}%
\pgfpathlineto{\pgfqpoint{7.389817in}{4.604648in}}%
\pgfpathlineto{\pgfqpoint{7.397913in}{4.640459in}}%
\pgfpathlineto{\pgfqpoint{7.406009in}{4.484479in}}%
\pgfpathlineto{\pgfqpoint{7.414105in}{4.752823in}}%
\pgfpathlineto{\pgfqpoint{7.422201in}{4.382280in}}%
\pgfpathlineto{\pgfqpoint{7.430297in}{4.842474in}}%
\pgfpathlineto{\pgfqpoint{7.438393in}{4.307390in}}%
\pgfpathlineto{\pgfqpoint{7.446489in}{4.900761in}}%
\pgfpathlineto{\pgfqpoint{7.454585in}{4.266985in}}%
\pgfpathlineto{\pgfqpoint{7.462681in}{4.922740in}}%
\pgfpathlineto{\pgfqpoint{7.470776in}{4.263115in}}%
\pgfpathlineto{\pgfqpoint{7.478872in}{4.909746in}}%
\pgfpathlineto{\pgfqpoint{7.486968in}{4.290798in}}%
\pgfpathlineto{\pgfqpoint{7.495064in}{4.870410in}}%
\pgfpathlineto{\pgfqpoint{7.503160in}{4.338030in}}%
\pgfpathlineto{\pgfqpoint{7.511256in}{4.819555in}}%
\pgfpathlineto{\pgfqpoint{7.519352in}{4.387969in}}%
\pgfpathlineto{\pgfqpoint{7.527448in}{4.775054in}}%
\pgfpathlineto{\pgfqpoint{7.535544in}{4.422826in}}%
\pgfpathlineto{\pgfqpoint{7.543640in}{4.753452in}}%
\pgfpathlineto{\pgfqpoint{7.551736in}{4.428406in}}%
\pgfpathlineto{\pgfqpoint{7.559831in}{4.765617in}}%
\pgfpathlineto{\pgfqpoint{7.567927in}{4.397940in}}%
\pgfpathlineto{\pgfqpoint{7.576023in}{4.813729in}}%
\pgfpathlineto{\pgfqpoint{7.584119in}{4.334005in}}%
\pgfpathlineto{\pgfqpoint{7.592215in}{4.890622in}}%
\pgfpathlineto{\pgfqpoint{7.600311in}{4.247858in}}%
\pgfpathlineto{\pgfqpoint{7.608407in}{4.981747in}}%
\pgfpathlineto{\pgfqpoint{7.616503in}{4.156297in}}%
\pgfpathlineto{\pgfqpoint{7.624599in}{5.069269in}}%
\pgfpathlineto{\pgfqpoint{7.632695in}{4.076901in}}%
\pgfpathlineto{\pgfqpoint{7.640791in}{5.137133in}}%
\pgfpathlineto{\pgfqpoint{7.648886in}{4.023047in}}%
\pgfpathlineto{\pgfqpoint{7.656982in}{5.175599in}}%
\pgfpathlineto{\pgfqpoint{7.665078in}{4.000155in}}%
\pgfpathlineto{\pgfqpoint{7.673174in}{5.183924in}}%
\pgfpathlineto{\pgfqpoint{7.681270in}{4.004287in}}%
\pgfpathlineto{\pgfqpoint{7.689366in}{5.170380in}}%
\pgfpathlineto{\pgfqpoint{7.697462in}{4.023499in}}%
\pgfpathlineto{\pgfqpoint{7.705558in}{5.149652in}}%
\pgfpathlineto{\pgfqpoint{7.713654in}{4.041501in}}%
\pgfpathlineto{\pgfqpoint{7.721750in}{5.138394in}}%
\pgfpathlineto{\pgfqpoint{7.729846in}{4.042520in}}%
\pgfpathlineto{\pgfqpoint{7.737941in}{5.150332in}}%
\pgfpathlineto{\pgfqpoint{7.746037in}{4.015875in}}%
\pgfpathlineto{\pgfqpoint{7.754133in}{5.192352in}}%
\pgfpathlineto{\pgfqpoint{7.762229in}{3.958923in}}%
\pgfpathlineto{\pgfqpoint{7.770325in}{5.262730in}}%
\pgfpathlineto{\pgfqpoint{7.778421in}{3.877557in}}%
\pgfpathlineto{\pgfqpoint{7.786517in}{5.351912in}}%
\pgfpathlineto{\pgfqpoint{7.794613in}{3.784224in}}%
\pgfpathlineto{\pgfqpoint{7.802709in}{5.445506in}}%
\pgfpathlineto{\pgfqpoint{7.810805in}{3.694208in}}%
\pgfpathlineto{\pgfqpoint{7.818901in}{5.528411in}}%
\pgfpathlineto{\pgfqpoint{7.826996in}{3.621421in}}%
\pgfpathlineto{\pgfqpoint{7.835092in}{5.588760in}}%
\pgfpathlineto{\pgfqpoint{7.843188in}{3.575048in}}%
\pgfpathlineto{\pgfqpoint{7.851284in}{5.620431in}}%
\pgfpathlineto{\pgfqpoint{7.859380in}{3.558040in}}%
\pgfpathlineto{\pgfqpoint{7.867476in}{5.623472in}}%
\pgfpathlineto{\pgfqpoint{7.875572in}{3.567767in}}%
\pgfpathlineto{\pgfqpoint{7.883668in}{5.602485in}}%
\pgfpathlineto{\pgfqpoint{7.891764in}{3.598398in}}%
\pgfpathlineto{\pgfqpoint{7.899860in}{5.563739in}}%
\pgfpathlineto{\pgfqpoint{7.907956in}{3.643991in}}%
\pgfpathlineto{\pgfqpoint{7.916051in}{5.512182in}}%
\pgfpathlineto{\pgfqpoint{7.924147in}{3.701083in}}%
\pgfpathlineto{\pgfqpoint{7.932243in}{5.449516in}}%
\pgfpathlineto{\pgfqpoint{7.940339in}{3.769777in}}%
\pgfpathlineto{\pgfqpoint{7.948435in}{5.374039in}}%
\pgfpathlineto{\pgfqpoint{7.956531in}{3.852942in}}%
\pgfpathlineto{\pgfqpoint{7.964627in}{5.282314in}}%
\pgfpathlineto{\pgfqpoint{7.972723in}{3.953865in}}%
\pgfpathlineto{\pgfqpoint{7.980819in}{5.171970in}}%
\pgfpathlineto{\pgfqpoint{7.988915in}{4.073285in}}%
\pgfpathlineto{\pgfqpoint{7.997011in}{5.044502in}}%
\pgfpathlineto{\pgfqpoint{8.005106in}{4.207049in}}%
\pgfpathlineto{\pgfqpoint{8.013202in}{4.906896in}}%
\pgfpathlineto{\pgfqpoint{8.021298in}{4.345430in}}%
\pgfpathlineto{\pgfqpoint{8.029394in}{4.771257in}}%
\pgfpathlineto{\pgfqpoint{8.037490in}{4.474577in}}%
\pgfpathlineto{\pgfqpoint{8.045586in}{4.652330in}}%
\pgfpathlineto{\pgfqpoint{8.053682in}{4.579850in}}%
\pgfpathlineto{\pgfqpoint{8.061778in}{4.563576in}}%
\pgfpathlineto{\pgfqpoint{8.069874in}{4.650036in}}%
\pgfpathlineto{\pgfqpoint{8.077970in}{4.512999in}}%
\pgfpathlineto{\pgfqpoint{8.086066in}{4.681107in}}%
\pgfpathlineto{\pgfqpoint{8.094161in}{4.500133in}}%
\pgfpathlineto{\pgfqpoint{8.102257in}{4.678230in}}%
\pgfpathlineto{\pgfqpoint{8.110353in}{4.515247in}}%
\pgfpathlineto{\pgfqpoint{8.118449in}{4.655230in}}%
\pgfpathlineto{\pgfqpoint{8.126545in}{4.541202in}}%
\pgfpathlineto{\pgfqpoint{8.134641in}{4.631520in}}%
\pgfpathlineto{\pgfqpoint{8.142737in}{4.557533in}}%
\pgfpathlineto{\pgfqpoint{8.150833in}{4.627301in}}%
\pgfpathlineto{\pgfqpoint{8.158929in}{4.545617in}}%
\pgfpathlineto{\pgfqpoint{8.167025in}{4.658402in}}%
\pgfpathlineto{\pgfqpoint{8.175121in}{4.493445in}}%
\pgfpathlineto{\pgfqpoint{8.183217in}{4.732255in}}%
\pgfpathlineto{\pgfqpoint{8.191312in}{4.398603in}}%
\pgfpathlineto{\pgfqpoint{8.199408in}{4.846160in}}%
\pgfpathlineto{\pgfqpoint{8.207504in}{4.268647in}}%
\pgfpathlineto{\pgfqpoint{8.215600in}{4.988285in}}%
\pgfpathlineto{\pgfqpoint{8.223696in}{4.118836in}}%
\pgfpathlineto{\pgfqpoint{8.231792in}{5.140996in}}%
\pgfpathlineto{\pgfqpoint{8.239888in}{3.968008in}}%
\pgfpathlineto{\pgfqpoint{8.247984in}{5.285444in}}%
\pgfpathlineto{\pgfqpoint{8.256080in}{3.833908in}}%
\pgfpathlineto{\pgfqpoint{8.264176in}{5.405954in}}%
\pgfpathlineto{\pgfqpoint{8.272272in}{3.729385in}}%
\pgfpathlineto{\pgfqpoint{8.280367in}{5.492981in}}%
\pgfpathlineto{\pgfqpoint{8.288463in}{3.660498in}}%
\pgfpathlineto{\pgfqpoint{8.296559in}{5.543854in}}%
\pgfpathlineto{\pgfqpoint{8.304655in}{3.626895in}}%
\pgfpathlineto{\pgfqpoint{8.312751in}{5.561369in}}%
\pgfpathlineto{\pgfqpoint{8.320847in}{3.624054in}}%
\pgfpathlineto{\pgfqpoint{8.328943in}{5.550981in}}%
\pgfpathlineto{\pgfqpoint{8.337039in}{3.646368in}}%
\pgfpathlineto{\pgfqpoint{8.345135in}{5.517762in}}%
\pgfpathlineto{\pgfqpoint{8.353231in}{3.689835in}}%
\pgfpathlineto{\pgfqpoint{8.361327in}{5.464316in}}%
\pgfpathlineto{\pgfqpoint{8.369422in}{3.753344in}}%
\pgfpathlineto{\pgfqpoint{8.377518in}{5.390402in}}%
\pgfpathlineto{\pgfqpoint{8.385614in}{3.838123in}}%
\pgfpathlineto{\pgfqpoint{8.393710in}{5.294349in}}%
\pgfpathlineto{\pgfqpoint{8.401806in}{3.945628in}}%
\pgfpathlineto{\pgfqpoint{8.409902in}{5.175621in}}%
\pgfpathlineto{\pgfqpoint{8.417998in}{4.074796in}}%
\pgfpathlineto{\pgfqpoint{8.426094in}{5.037452in}}%
\pgfpathlineto{\pgfqpoint{8.434190in}{4.219825in}}%
\pgfpathlineto{\pgfqpoint{8.442286in}{4.888383in}}%
\pgfpathlineto{\pgfqpoint{8.450382in}{4.369519in}}%
\pgfpathlineto{\pgfqpoint{8.458477in}{4.741918in}}%
\pgfpathlineto{\pgfqpoint{8.466573in}{4.508670in}}%
\pgfpathlineto{\pgfqpoint{8.474669in}{4.614153in}}%
\pgfpathlineto{\pgfqpoint{8.482765in}{4.621256in}}%
\pgfpathlineto{\pgfqpoint{8.490861in}{4.519997in}}%
\pgfpathlineto{\pgfqpoint{8.498957in}{4.694512in}}%
\pgfpathlineto{\pgfqpoint{8.507053in}{4.469136in}}%
\pgfpathlineto{\pgfqpoint{8.515149in}{4.722587in}}%
\pgfpathlineto{\pgfqpoint{8.523245in}{4.463076in}}%
\pgfpathlineto{\pgfqpoint{8.531341in}{4.708541in}}%
\pgfpathlineto{\pgfqpoint{8.539437in}{4.494294in}}%
\pgfpathlineto{\pgfqpoint{8.547532in}{4.663927in}}%
\pgfpathlineto{\pgfqpoint{8.555628in}{4.547936in}}%
\pgfpathlineto{\pgfqpoint{8.563724in}{4.605927in}}%
\pgfpathlineto{\pgfqpoint{8.571820in}{4.605640in}}%
\pgfpathlineto{\pgfqpoint{8.579916in}{4.552838in}}%
\pgfpathlineto{\pgfqpoint{8.588012in}{4.650417in}}%
\pgfpathlineto{\pgfqpoint{8.596108in}{4.519201in}}%
\pgfpathlineto{\pgfqpoint{8.604204in}{4.671134in}}%
\pgfpathlineto{\pgfqpoint{8.612300in}{4.512041in}}%
\pgfpathlineto{\pgfqpoint{8.620396in}{4.665255in}}%
\pgfpathlineto{\pgfqpoint{8.628492in}{4.529360in}}%
\pgfpathlineto{\pgfqpoint{8.636587in}{4.639015in}}%
\pgfpathlineto{\pgfqpoint{8.644683in}{4.561311in}}%
\pgfpathlineto{\pgfqpoint{8.652779in}{4.604979in}}%
\pgfpathlineto{\pgfqpoint{8.660875in}{4.593696in}}%
\pgfpathlineto{\pgfqpoint{8.668971in}{4.577777in}}%
\pgfpathlineto{\pgfqpoint{8.677067in}{4.612673in}}%
\pgfpathlineto{\pgfqpoint{8.685163in}{4.569327in}}%
\pgfpathlineto{\pgfqpoint{8.693259in}{4.609219in}}%
\pgfpathlineto{\pgfqpoint{8.701355in}{4.585032in}}%
\pgfpathlineto{\pgfqpoint{8.709451in}{4.581975in}}%
\pgfpathlineto{\pgfqpoint{8.717547in}{4.622110in}}%
\pgfpathlineto{\pgfqpoint{8.725642in}{4.537621in}}%
\pgfpathlineto{\pgfqpoint{8.733738in}{4.670535in}}%
\pgfpathlineto{\pgfqpoint{8.741834in}{4.488714in}}%
\pgfpathlineto{\pgfqpoint{8.749930in}{4.716241in}}%
\pgfpathlineto{\pgfqpoint{8.758026in}{4.449695in}}%
\pgfpathlineto{\pgfqpoint{8.766122in}{4.745564in}}%
\pgfpathlineto{\pgfqpoint{8.774218in}{4.432361in}}%
\pgfpathlineto{\pgfqpoint{8.782314in}{4.749511in}}%
\pgfpathlineto{\pgfqpoint{8.790410in}{4.442194in}}%
\pgfpathlineto{\pgfqpoint{8.798506in}{4.726543in}}%
\pgfpathlineto{\pgfqpoint{8.806602in}{4.476663in}}%
\pgfpathlineto{\pgfqpoint{8.814697in}{4.683070in}}%
\pgfpathlineto{\pgfqpoint{8.822793in}{4.525964in}}%
\pgfpathlineto{\pgfqpoint{8.830889in}{4.631565in}}%
\pgfpathlineto{\pgfqpoint{8.838985in}{4.575863in}}%
\pgfpathlineto{\pgfqpoint{8.847081in}{4.587002in}}%
\pgfpathlineto{\pgfqpoint{8.855177in}{4.611689in}}%
\pgfpathlineto{\pgfqpoint{8.863273in}{4.562767in}}%
\pgfpathlineto{\pgfqpoint{8.871369in}{4.622190in}}%
\pgfpathlineto{\pgfqpoint{8.879465in}{4.567328in}}%
\pgfpathlineto{\pgfqpoint{8.887561in}{4.602087in}}%
\pgfpathlineto{\pgfqpoint{8.895657in}{4.602635in}}%
\pgfpathlineto{\pgfqpoint{8.903752in}{4.552640in}}%
\pgfpathlineto{\pgfqpoint{8.911848in}{4.664577in}}%
\pgfpathlineto{\pgfqpoint{8.919944in}{4.480255in}}%
\pgfpathlineto{\pgfqpoint{8.928040in}{4.745139in}}%
\pgfpathlineto{\pgfqpoint{8.936136in}{4.393808in}}%
\pgfpathlineto{\pgfqpoint{8.944232in}{4.835321in}}%
\pgfpathlineto{\pgfqpoint{8.952328in}{4.301773in}}%
\pgfpathlineto{\pgfqpoint{8.960424in}{4.927676in}}%
\pgfpathlineto{\pgfqpoint{8.968520in}{4.210261in}}%
\pgfpathlineto{\pgfqpoint{8.976616in}{5.017514in}}%
\pgfpathlineto{\pgfqpoint{8.984712in}{4.122686in}}%
\pgfpathlineto{\pgfqpoint{8.992807in}{5.102338in}}%
\pgfpathlineto{\pgfqpoint{9.000903in}{4.041164in}}%
\pgfpathlineto{\pgfqpoint{9.008999in}{5.179761in}}%
\pgfpathlineto{\pgfqpoint{9.017095in}{3.969060in}}%
\pgfpathlineto{\pgfqpoint{9.025191in}{5.244753in}}%
\pgfpathlineto{\pgfqpoint{9.033287in}{3.913652in}}%
\pgfpathlineto{\pgfqpoint{9.041383in}{5.287372in}}%
\pgfpathlineto{\pgfqpoint{9.049479in}{3.887741in}}%
\pgfpathlineto{\pgfqpoint{9.057575in}{5.292033in}}%
\pgfpathlineto{\pgfqpoint{9.065671in}{3.909335in}}%
\pgfpathlineto{\pgfqpoint{9.073767in}{5.238948in}}%
\pgfpathlineto{\pgfqpoint{9.081862in}{3.999102in}}%
\pgfpathlineto{\pgfqpoint{9.089958in}{5.107672in}}%
\pgfpathlineto{\pgfqpoint{9.098054in}{4.176016in}}%
\pgfpathlineto{\pgfqpoint{9.106150in}{4.882019in}}%
\pgfpathlineto{\pgfqpoint{9.114246in}{4.452180in}}%
\pgfpathlineto{\pgfqpoint{9.122342in}{4.555153in}}%
\pgfpathlineto{\pgfqpoint{9.130438in}{4.828166in}}%
\pgfpathlineto{\pgfqpoint{9.138534in}{4.133509in}}%
\pgfpathlineto{\pgfqpoint{9.146630in}{5.290105in}}%
\pgfpathlineto{\pgfqpoint{9.154726in}{3.638462in}}%
\pgfpathlineto{\pgfqpoint{9.162822in}{5.809412in}}%
\pgfpathlineto{\pgfqpoint{9.170917in}{3.105156in}}%
\pgfpathlineto{\pgfqpoint{9.179013in}{6.345368in}}%
\pgfpathlineto{\pgfqpoint{9.187109in}{2.578607in}}%
\pgfpathlineto{\pgfqpoint{9.195205in}{6.850168in}}%
\pgfpathlineto{\pgfqpoint{9.203301in}{2.107745in}}%
\pgfpathlineto{\pgfqpoint{9.211397in}{7.275499in}}%
\pgfpathlineto{\pgfqpoint{9.219493in}{1.738523in}}%
\pgfpathlineto{\pgfqpoint{9.227589in}{7.579428in}}%
\pgfpathlineto{\pgfqpoint{9.235685in}{1.507348in}}%
\pgfpathlineto{\pgfqpoint{9.243781in}{7.732368in}}%
\pgfpathlineto{\pgfqpoint{9.251877in}{1.435966in}}%
\pgfpathlineto{\pgfqpoint{9.259972in}{7.721125in}}%
\pgfpathlineto{\pgfqpoint{9.268068in}{1.528639in}}%
\pgfpathlineto{\pgfqpoint{9.276164in}{7.550405in}}%
\pgfpathlineto{\pgfqpoint{9.284260in}{1.771992in}}%
\pgfpathlineto{\pgfqpoint{9.292356in}{7.241633in}}%
\pgfpathlineto{\pgfqpoint{9.300452in}{2.137455in}}%
\pgfpathlineto{\pgfqpoint{9.308548in}{6.829379in}}%
\pgfpathlineto{\pgfqpoint{9.316644in}{2.585803in}}%
\pgfpathlineto{\pgfqpoint{9.324740in}{6.356044in}}%
\pgfpathlineto{\pgfqpoint{9.332836in}{3.073002in}}%
\pgfpathlineto{\pgfqpoint{9.340932in}{5.865741in}}%
\pgfpathlineto{\pgfqpoint{9.349027in}{3.556355in}}%
\pgfpathlineto{\pgfqpoint{9.357123in}{5.398380in}}%
\pgfpathlineto{\pgfqpoint{9.365219in}{3.999938in}}%
\pgfpathlineto{\pgfqpoint{9.373315in}{4.984928in}}%
\pgfpathlineto{\pgfqpoint{9.381411in}{4.378448in}}%
\pgfpathlineto{\pgfqpoint{9.389507in}{4.644596in}}%
\pgfpathlineto{\pgfqpoint{9.397603in}{4.678898in}}%
\pgfpathlineto{\pgfqpoint{9.405699in}{4.384310in}}%
\pgfpathlineto{\pgfqpoint{9.413795in}{4.899989in}}%
\pgfpathlineto{\pgfqpoint{9.421891in}{4.200413in}}%
\pgfpathlineto{\pgfqpoint{9.429987in}{5.049474in}}%
\pgfpathlineto{\pgfqpoint{9.438082in}{4.082049in}}%
\pgfpathlineto{\pgfqpoint{9.446178in}{5.140241in}}%
\pgfpathlineto{\pgfqpoint{9.454274in}{4.015378in}}%
\pgfpathlineto{\pgfqpoint{9.462370in}{5.186063in}}%
\pgfpathlineto{\pgfqpoint{9.470466in}{3.987597in}}%
\pgfpathlineto{\pgfqpoint{9.478562in}{5.198039in}}%
\pgfpathlineto{\pgfqpoint{9.486658in}{3.989846in}}%
\pgfpathlineto{\pgfqpoint{9.494754in}{5.182479in}}%
\pgfpathlineto{\pgfqpoint{9.502850in}{4.018428in}}%
\pgfpathlineto{\pgfqpoint{9.510946in}{5.140624in}}%
\pgfpathlineto{\pgfqpoint{9.519042in}{4.074208in}}%
\pgfpathlineto{\pgfqpoint{9.527137in}{5.070033in}}%
\pgfpathlineto{\pgfqpoint{9.535233in}{4.160558in}}%
\pgfpathlineto{\pgfqpoint{9.543329in}{4.967099in}}%
\pgfpathlineto{\pgfqpoint{9.551425in}{4.280608in}}%
\pgfpathlineto{\pgfqpoint{9.559521in}{4.829833in}}%
\pgfpathlineto{\pgfqpoint{9.567617in}{4.434647in}}%
\pgfpathlineto{\pgfqpoint{9.575713in}{4.660072in}}%
\pgfpathlineto{\pgfqpoint{9.583809in}{4.618451in}}%
\pgfpathlineto{\pgfqpoint{9.591905in}{4.464505in}}%
\pgfpathlineto{\pgfqpoint{9.600001in}{4.822969in}}%
\pgfpathlineto{\pgfqpoint{9.608097in}{4.254277in}}%
\pgfpathlineto{\pgfqpoint{9.616192in}{5.035369in}}%
\pgfpathlineto{\pgfqpoint{9.624288in}{4.043392in}}%
\pgfpathlineto{\pgfqpoint{9.632384in}{5.241056in}}%
\pgfpathlineto{\pgfqpoint{9.640480in}{3.846433in}}%
\pgfpathlineto{\pgfqpoint{9.648576in}{5.426044in}}%
\pgfpathlineto{\pgfqpoint{9.656672in}{3.676265in}}%
\pgfpathlineto{\pgfqpoint{9.664768in}{5.579021in}}%
\pgfpathlineto{\pgfqpoint{9.672864in}{3.542319in}}%
\pgfpathlineto{\pgfqpoint{9.680960in}{5.692644in}}%
\pgfpathlineto{\pgfqpoint{9.689056in}{3.449770in}}%
\pgfpathlineto{\pgfqpoint{9.697152in}{5.763878in}}%
\pgfpathlineto{\pgfqpoint{9.705247in}{3.399635in}}%
\pgfpathlineto{\pgfqpoint{9.713343in}{5.793526in}}%
\pgfpathlineto{\pgfqpoint{9.721439in}{3.389531in}}%
\pgfpathlineto{\pgfqpoint{9.729535in}{5.785291in}}%
\pgfpathlineto{\pgfqpoint{9.737631in}{3.414693in}}%
\pgfpathlineto{\pgfqpoint{9.745727in}{5.744774in}}%
\pgfpathlineto{\pgfqpoint{9.753823in}{3.468873in}}%
\pgfpathlineto{\pgfqpoint{9.761919in}{5.678718in}}%
\pgfpathlineto{\pgfqpoint{9.770015in}{3.544938in}}%
\pgfpathlineto{\pgfqpoint{9.778111in}{5.594584in}}%
\pgfpathlineto{\pgfqpoint{9.786207in}{3.635139in}}%
\pgfpathlineto{\pgfqpoint{9.794302in}{5.500369in}}%
\pgfpathlineto{\pgfqpoint{9.802398in}{3.731274in}}%
\pgfpathlineto{\pgfqpoint{9.810494in}{5.404417in}}%
\pgfpathlineto{\pgfqpoint{9.818590in}{3.824969in}}%
\pgfpathlineto{\pgfqpoint{9.826686in}{5.314979in}}%
\pgfpathlineto{\pgfqpoint{9.834782in}{3.908287in}}%
\pgfpathlineto{\pgfqpoint{9.842878in}{5.239438in}}%
\pgfpathlineto{\pgfqpoint{9.850974in}{3.974669in}}%
\pgfpathlineto{\pgfqpoint{9.859070in}{5.183253in}}%
\pgfpathlineto{\pgfqpoint{9.867166in}{4.020022in}}%
\pgfpathlineto{\pgfqpoint{9.875262in}{5.148917in}}%
\pgfpathlineto{\pgfqpoint{9.883357in}{4.043633in}}%
\pgfpathlineto{\pgfqpoint{9.891453in}{5.135261in}}%
\pgfpathlineto{\pgfqpoint{9.899549in}{4.048557in}}%
\pgfpathlineto{\pgfqpoint{9.907645in}{5.137440in}}%
\pgfpathlineto{\pgfqpoint{9.915741in}{4.041235in}}%
\pgfpathlineto{\pgfqpoint{9.923837in}{5.147713in}}%
\pgfpathlineto{\pgfqpoint{9.931933in}{4.030328in}}%
\pgfpathlineto{\pgfqpoint{9.940029in}{5.156936in}}%
\pgfpathlineto{\pgfqpoint{9.948125in}{4.024988in}}%
\pgfpathlineto{\pgfqpoint{9.956221in}{5.156428in}}%
\pgfpathlineto{\pgfqpoint{9.964317in}{4.032972in}}%
\pgfpathlineto{\pgfqpoint{9.972413in}{5.139756in}}%
\pgfpathlineto{\pgfqpoint{9.980508in}{4.059075in}}%
\pgfpathlineto{\pgfqpoint{9.988604in}{5.103974in}}%
\pgfpathlineto{\pgfqpoint{9.996700in}{4.104290in}}%
\pgfpathlineto{\pgfqpoint{10.004796in}{5.050025in}}%
\pgfpathlineto{\pgfqpoint{10.012892in}{4.165879in}}%
\pgfpathlineto{\pgfqpoint{10.020988in}{4.982204in}}%
\pgfpathlineto{\pgfqpoint{10.029084in}{4.238311in}}%
\pgfpathlineto{\pgfqpoint{10.037180in}{4.906889in}}%
\pgfpathlineto{\pgfqpoint{10.045276in}{4.314784in}}%
\pgfpathlineto{\pgfqpoint{10.053372in}{4.830882in}}%
\pgfpathlineto{\pgfqpoint{10.061468in}{4.388894in}}%
\pgfpathlineto{\pgfqpoint{10.069563in}{4.759832in}}%
\pgfpathlineto{\pgfqpoint{10.077659in}{4.456034in}}%
\pgfpathlineto{\pgfqpoint{10.085755in}{4.697110in}}%
\pgfpathlineto{\pgfqpoint{10.093851in}{4.514175in}}%
\pgfpathlineto{\pgfqpoint{10.101947in}{4.643395in}}%
\pgfpathlineto{\pgfqpoint{10.110043in}{4.563901in}}%
\pgfpathlineto{\pgfqpoint{10.118139in}{4.596997in}}%
\pgfpathlineto{\pgfqpoint{10.126235in}{4.607782in}}%
\pgfpathlineto{\pgfqpoint{10.134331in}{4.554747in}}%
\pgfpathlineto{\pgfqpoint{10.142427in}{4.649287in}}%
\pgfpathlineto{\pgfqpoint{10.150523in}{4.513173in}}%
\pgfpathlineto{\pgfqpoint{10.158618in}{4.691605in}}%
\pgfpathlineto{\pgfqpoint{10.166714in}{4.469624in}}%
\pgfpathlineto{\pgfqpoint{10.174810in}{4.736645in}}%
\pgfpathlineto{\pgfqpoint{10.182906in}{4.423080in}}%
\pgfpathlineto{\pgfqpoint{10.191002in}{4.784458in}}%
\pgfpathlineto{\pgfqpoint{10.199098in}{4.374469in}}%
\pgfpathlineto{\pgfqpoint{10.207194in}{4.833186in}}%
\pgfpathlineto{\pgfqpoint{10.215290in}{4.326478in}}%
\pgfpathlineto{\pgfqpoint{10.223386in}{4.879466in}}%
\pgfpathlineto{\pgfqpoint{10.231482in}{4.282946in}}%
\pgfpathlineto{\pgfqpoint{10.239578in}{4.919205in}}%
\pgfpathlineto{\pgfqpoint{10.247673in}{4.247994in}}%
\pgfpathlineto{\pgfqpoint{10.255769in}{4.948486in}}%
\pgfpathlineto{\pgfqpoint{10.263865in}{4.225108in}}%
\pgfpathlineto{\pgfqpoint{10.271961in}{4.964456in}}%
\pgfpathlineto{\pgfqpoint{10.280057in}{4.216339in}}%
\pgfpathlineto{\pgfqpoint{10.288153in}{4.965998in}}%
\pgfpathlineto{\pgfqpoint{10.296249in}{4.221783in}}%
\pgfpathlineto{\pgfqpoint{10.304345in}{4.954076in}}%
\pgfpathlineto{\pgfqpoint{10.312441in}{4.239423in}}%
\pgfpathlineto{\pgfqpoint{10.320537in}{4.931696in}}%
\pgfpathlineto{\pgfqpoint{10.328633in}{4.265382in}}%
\pgfpathlineto{\pgfqpoint{10.336728in}{4.903451in}}%
\pgfpathlineto{\pgfqpoint{10.344824in}{4.294549in}}%
\pgfpathlineto{\pgfqpoint{10.352920in}{4.874733in}}%
\pgfpathlineto{\pgfqpoint{10.361016in}{4.321510in}}%
\pgfpathlineto{\pgfqpoint{10.369112in}{4.850706in}}%
\pgfpathlineto{\pgfqpoint{10.377208in}{4.341622in}}%
\pgfpathlineto{\pgfqpoint{10.385304in}{4.835235in}}%
\pgfpathlineto{\pgfqpoint{10.393400in}{4.352031in}}%
\pgfpathlineto{\pgfqpoint{10.401496in}{4.829966in}}%
\pgfpathlineto{\pgfqpoint{10.409592in}{4.352441in}}%
\pgfpathlineto{\pgfqpoint{10.417688in}{4.833772in}}%
\pgfpathlineto{\pgfqpoint{10.425783in}{4.345414in}}%
\pgfpathlineto{\pgfqpoint{10.433879in}{4.842712in}}%
\pgfpathlineto{\pgfqpoint{10.441975in}{4.336131in}}%
\pgfpathlineto{\pgfqpoint{10.450071in}{4.850569in}}%
\pgfpathlineto{\pgfqpoint{10.458167in}{4.331590in}}%
\pgfpathlineto{\pgfqpoint{10.466263in}{4.849870in}}%
\pgfpathlineto{\pgfqpoint{10.474359in}{4.339402in}}%
\pgfpathlineto{\pgfqpoint{10.482455in}{4.833212in}}%
\pgfpathlineto{\pgfqpoint{10.490551in}{4.366418in}}%
\pgfpathlineto{\pgfqpoint{10.498647in}{4.794612in}}%
\pgfpathlineto{\pgfqpoint{10.506743in}{4.417483in}}%
\pgfpathlineto{\pgfqpoint{10.514838in}{4.730582in}}%
\pgfpathlineto{\pgfqpoint{10.522934in}{4.494575in}}%
\pgfpathlineto{\pgfqpoint{10.531030in}{4.640736in}}%
\pgfpathlineto{\pgfqpoint{10.539126in}{4.596480in}}%
\pgfpathlineto{\pgfqpoint{10.547222in}{4.527826in}}%
\pgfpathlineto{\pgfqpoint{10.555318in}{4.719023in}}%
\pgfpathlineto{\pgfqpoint{10.563414in}{4.397286in}}%
\pgfpathlineto{\pgfqpoint{10.571510in}{4.855712in}}%
\pgfpathlineto{\pgfqpoint{10.579606in}{4.256456in}}%
\pgfpathlineto{\pgfqpoint{10.587702in}{4.998570in}}%
\pgfpathlineto{\pgfqpoint{10.595798in}{4.113741in}}%
\pgfpathlineto{\pgfqpoint{10.603893in}{5.138951in}}%
\pgfpathlineto{\pgfqpoint{10.611989in}{3.977873in}}%
\pgfpathlineto{\pgfqpoint{10.620085in}{5.268166in}}%
\pgfpathlineto{\pgfqpoint{10.628181in}{3.857389in}}%
\pgfpathlineto{\pgfqpoint{10.636277in}{5.377923in}}%
\pgfpathlineto{\pgfqpoint{10.644373in}{3.760253in}}%
\pgfpathlineto{\pgfqpoint{10.652469in}{5.460675in}}%
\pgfpathlineto{\pgfqpoint{10.660565in}{3.693486in}}%
\pgfpathlineto{\pgfqpoint{10.668661in}{5.510056in}}%
\pgfpathlineto{\pgfqpoint{10.676757in}{3.662639in}}%
\pgfpathlineto{\pgfqpoint{10.684853in}{5.521529in}}%
\pgfpathlineto{\pgfqpoint{10.692948in}{3.671012in}}%
\pgfpathlineto{\pgfqpoint{10.701044in}{5.493270in}}%
\pgfpathlineto{\pgfqpoint{10.709140in}{3.718709in}}%
\pgfpathlineto{\pgfqpoint{10.717236in}{5.427117in}}%
\pgfpathlineto{\pgfqpoint{10.725332in}{3.801771in}}%
\pgfpathlineto{\pgfqpoint{10.733428in}{5.329266in}}%
\pgfpathlineto{\pgfqpoint{10.741524in}{3.911740in}}%
\pgfpathlineto{\pgfqpoint{10.749620in}{5.210350in}}%
\pgfpathlineto{\pgfqpoint{10.757716in}{4.036016in}}%
\pgfpathlineto{\pgfqpoint{10.765812in}{5.084603in}}%
\pgfpathlineto{\pgfqpoint{10.773908in}{4.159186in}}%
\pgfpathlineto{\pgfqpoint{10.782003in}{4.968054in}}%
\pgfpathlineto{\pgfqpoint{10.790099in}{4.265252in}}%
\pgfpathlineto{\pgfqpoint{10.798195in}{4.875973in}}%
\pgfpathlineto{\pgfqpoint{10.806291in}{4.340379in}}%
\pgfpathlineto{\pgfqpoint{10.814387in}{4.820085in}}%
\pgfpathlineto{\pgfqpoint{10.822483in}{4.375549in}}%
\pgfpathlineto{\pgfqpoint{10.830579in}{4.806224in}}%
\pgfpathlineto{\pgfqpoint{10.838675in}{4.368431in}}%
\pgfpathlineto{\pgfqpoint{10.846771in}{4.833081in}}%
\pgfpathlineto{\pgfqpoint{10.854867in}{4.323914in}}%
\pgfpathlineto{\pgfqpoint{10.862963in}{4.892458in}}%
\pgfpathlineto{\pgfqpoint{10.871058in}{4.253041in}}%
\pgfpathlineto{\pgfqpoint{10.879154in}{4.971092in}}%
\pgfpathlineto{\pgfqpoint{10.887250in}{4.170538in}}%
\pgfpathlineto{\pgfqpoint{10.895346in}{5.053637in}}%
\pgfpathlineto{\pgfqpoint{10.903442in}{4.091494in}}%
\pgfpathlineto{\pgfqpoint{10.911538in}{5.126111in}}%
\pgfpathlineto{\pgfqpoint{10.919634in}{4.028022in}}%
\pgfpathlineto{\pgfqpoint{10.927730in}{5.178904in}}%
\pgfpathlineto{\pgfqpoint{10.935826in}{3.986769in}}%
\pgfpathlineto{\pgfqpoint{10.943922in}{5.208584in}}%
\pgfpathlineto{\pgfqpoint{10.952018in}{3.967915in}}%
\pgfpathlineto{\pgfqpoint{10.960113in}{5.218039in}}%
\pgfpathlineto{\pgfqpoint{10.968209in}{3.965890in}}%
\pgfpathlineto{\pgfqpoint{10.976305in}{5.214967in}}%
\pgfpathlineto{\pgfqpoint{10.984401in}{3.971555in}}%
\pgfpathlineto{\pgfqpoint{10.992497in}{5.209190in}}%
\pgfpathlineto{\pgfqpoint{11.000593in}{3.975177in}}%
\pgfpathlineto{\pgfqpoint{11.008689in}{5.209615in}}%
\pgfpathlineto{\pgfqpoint{11.016785in}{3.969312in}}%
\pgfpathlineto{\pgfqpoint{11.024881in}{5.221732in}}%
\pgfpathlineto{\pgfqpoint{11.032977in}{3.950744in}}%
\pgfpathlineto{\pgfqpoint{11.041073in}{5.246362in}}%
\pgfpathlineto{\pgfqpoint{11.049168in}{3.920959in}}%
\pgfpathlineto{\pgfqpoint{11.057264in}{5.279988in}}%
\pgfpathlineto{\pgfqpoint{11.065360in}{3.885067in}}%
\pgfpathlineto{\pgfqpoint{11.073456in}{5.316467in}}%
\pgfpathlineto{\pgfqpoint{11.081552in}{3.849614in}}%
\pgfpathlineto{\pgfqpoint{11.089648in}{5.349504in}}%
\pgfpathlineto{\pgfqpoint{11.097744in}{3.820031in}}%
\pgfpathlineto{\pgfqpoint{11.105840in}{5.375044in}}%
\pgfpathlineto{\pgfqpoint{11.113936in}{3.798620in}}%
\pgfpathlineto{\pgfqpoint{11.122032in}{5.392746in}}%
\pgfpathlineto{\pgfqpoint{11.130128in}{3.783751in}}%
\pgfpathlineto{\pgfqpoint{11.138223in}{5.406029in}}%
\pgfpathlineto{\pgfqpoint{11.146319in}{3.770570in}}%
\pgfpathlineto{\pgfqpoint{11.154415in}{5.420671in}}%
\pgfpathlineto{\pgfqpoint{11.162511in}{3.752990in}}%
\pgfpathlineto{\pgfqpoint{11.170607in}{5.442408in}}%
\pgfpathlineto{\pgfqpoint{11.178703in}{3.726286in}}%
\pgfpathlineto{\pgfqpoint{11.186799in}{5.474360in}}%
\pgfpathlineto{\pgfqpoint{11.194895in}{3.689414in}}%
\pgfpathlineto{\pgfqpoint{11.202991in}{5.515188in}}%
\pgfpathlineto{\pgfqpoint{11.211087in}{3.646202in}}%
\pgfpathlineto{\pgfqpoint{11.219183in}{5.558685in}}%
\pgfpathlineto{\pgfqpoint{11.227278in}{3.604912in}}%
\pgfpathlineto{\pgfqpoint{11.235374in}{5.595055in}}%
\pgfpathlineto{\pgfqpoint{11.243470in}{3.576198in}}%
\pgfpathlineto{\pgfqpoint{11.251566in}{5.613570in}}%
\pgfpathlineto{\pgfqpoint{11.259662in}{3.570024in}}%
\pgfpathlineto{\pgfqpoint{11.267758in}{5.605847in}}%
\pgfpathlineto{\pgfqpoint{11.275854in}{3.592460in}}%
\pgfpathlineto{\pgfqpoint{11.283950in}{5.568721in}}%
\pgfpathlineto{\pgfqpoint{11.292046in}{3.643377in}}%
\pgfpathlineto{\pgfqpoint{11.300142in}{5.505767in}}%
\pgfpathlineto{\pgfqpoint{11.308238in}{3.715845in}}%
\pgfpathlineto{\pgfqpoint{11.316333in}{5.426926in}}%
\pgfpathlineto{\pgfqpoint{11.324429in}{3.797493in}}%
\pgfpathlineto{\pgfqpoint{11.332525in}{5.346230in}}%
\pgfpathlineto{\pgfqpoint{11.340621in}{3.873533in}}%
\pgfpathlineto{\pgfqpoint{11.348717in}{5.278246in}}%
\pgfpathlineto{\pgfqpoint{11.356813in}{3.930591in}}%
\pgfpathlineto{\pgfqpoint{11.364909in}{5.234265in}}%
\pgfpathlineto{\pgfqpoint{11.373005in}{3.960206in}}%
\pgfpathlineto{\pgfqpoint{11.381101in}{5.219375in}}%
\pgfpathlineto{\pgfqpoint{11.389197in}{3.960951in}}%
\pgfpathlineto{\pgfqpoint{11.397293in}{5.231319in}}%
\pgfpathlineto{\pgfqpoint{11.405388in}{3.938518in}}%
\pgfpathlineto{\pgfqpoint{11.413484in}{5.261482in}}%
\pgfpathlineto{\pgfqpoint{11.421580in}{3.903722in}}%
\pgfpathlineto{\pgfqpoint{11.429676in}{5.297723in}}%
\pgfpathlineto{\pgfqpoint{11.437772in}{3.869063in}}%
\pgfpathlineto{\pgfqpoint{11.445868in}{5.328160in}}%
\pgfpathlineto{\pgfqpoint{11.453964in}{3.844906in}}%
\pgfpathlineto{\pgfqpoint{11.462060in}{5.344707in}}%
\pgfpathlineto{\pgfqpoint{11.470156in}{3.836498in}}%
\pgfpathlineto{\pgfqpoint{11.478252in}{5.345252in}}%
\pgfpathlineto{\pgfqpoint{11.486348in}{3.842797in}}%
\pgfpathlineto{\pgfqpoint{11.494443in}{5.333737in}}%
\pgfpathlineto{\pgfqpoint{11.502539in}{3.857483in}}%
\pgfpathlineto{\pgfqpoint{11.510635in}{5.318112in}}%
\pgfpathlineto{\pgfqpoint{11.518731in}{3.871880in}}%
\pgfpathlineto{\pgfqpoint{11.526827in}{5.306789in}}%
\pgfpathlineto{\pgfqpoint{11.534923in}{3.878841in}}%
\pgfpathlineto{\pgfqpoint{11.543019in}{5.304726in}}%
\pgfpathlineto{\pgfqpoint{11.551115in}{3.876361in}}%
\pgfpathlineto{\pgfqpoint{11.559211in}{5.310430in}}%
\pgfpathlineto{\pgfqpoint{11.567307in}{3.869709in}}%
\pgfpathlineto{\pgfqpoint{11.575403in}{5.314878in}}%
\pgfpathlineto{\pgfqpoint{11.583498in}{3.871338in}}%
\pgfpathlineto{\pgfqpoint{11.591594in}{5.302799in}}%
\pgfpathlineto{\pgfqpoint{11.599690in}{3.898481in}}%
\pgfpathlineto{\pgfqpoint{11.607786in}{5.256030in}}%
\pgfpathlineto{\pgfqpoint{11.615882in}{3.969087in}}%
\pgfpathlineto{\pgfqpoint{11.623978in}{5.158005in}}%
\pgfpathlineto{\pgfqpoint{11.632074in}{4.097232in}}%
\pgfpathlineto{\pgfqpoint{11.640170in}{4.998112in}}%
\pgfpathlineto{\pgfqpoint{11.648266in}{4.289296in}}%
\pgfpathlineto{\pgfqpoint{11.656362in}{4.774715in}}%
\pgfpathlineto{\pgfqpoint{11.664458in}{4.541947in}}%
\pgfpathlineto{\pgfqpoint{11.672554in}{4.496043in}}%
\pgfpathlineto{\pgfqpoint{11.672554in}{4.496043in}}%
\pgfusepath{stroke}%
\end{pgfscope}%
\begin{pgfscope}%
\pgfpathrectangle{\pgfqpoint{3.133636in}{1.247073in}}{\pgfqpoint{8.787624in}{6.674186in}}%
\pgfusepath{clip}%
\pgfsetbuttcap%
\pgfsetroundjoin%
\pgfsetlinewidth{1.003750pt}%
\definecolor{currentstroke}{rgb}{0.888874,0.435649,0.278123}%
\pgfsetstrokecolor{currentstroke}%
\pgfsetdash{}{0pt}%
\pgfpathmoveto{\pgfqpoint{3.382342in}{4.591680in}}%
\pgfpathlineto{\pgfqpoint{11.672554in}{4.591680in}}%
\pgfpathlineto{\pgfqpoint{11.672554in}{4.591680in}}%
\pgfusepath{stroke}%
\end{pgfscope}%
\begin{pgfscope}%
\pgfsetrectcap%
\pgfsetmiterjoin%
\pgfsetlinewidth{1.003750pt}%
\definecolor{currentstroke}{rgb}{0.000000,0.000000,0.000000}%
\pgfsetstrokecolor{currentstroke}%
\pgfsetdash{}{0pt}%
\pgfpathmoveto{\pgfqpoint{3.133636in}{1.247073in}}%
\pgfpathlineto{\pgfqpoint{3.133636in}{7.921260in}}%
\pgfusepath{stroke}%
\end{pgfscope}%
\begin{pgfscope}%
\pgfsetrectcap%
\pgfsetmiterjoin%
\pgfsetlinewidth{1.003750pt}%
\definecolor{currentstroke}{rgb}{0.000000,0.000000,0.000000}%
\pgfsetstrokecolor{currentstroke}%
\pgfsetdash{}{0pt}%
\pgfpathmoveto{\pgfqpoint{3.133636in}{1.247073in}}%
\pgfpathlineto{\pgfqpoint{11.921260in}{1.247073in}}%
\pgfusepath{stroke}%
\end{pgfscope}%
\begin{pgfscope}%
\pgfsetbuttcap%
\pgfsetmiterjoin%
\definecolor{currentfill}{rgb}{1.000000,1.000000,1.000000}%
\pgfsetfillcolor{currentfill}%
\pgfsetlinewidth{1.003750pt}%
\definecolor{currentstroke}{rgb}{0.000000,0.000000,0.000000}%
\pgfsetstrokecolor{currentstroke}%
\pgfsetdash{}{0pt}%
\pgfpathmoveto{\pgfqpoint{10.511589in}{6.787373in}}%
\pgfpathlineto{\pgfqpoint{11.796260in}{6.787373in}}%
\pgfpathlineto{\pgfqpoint{11.796260in}{7.796260in}}%
\pgfpathlineto{\pgfqpoint{10.511589in}{7.796260in}}%
\pgfpathclose%
\pgfusepath{stroke,fill}%
\end{pgfscope}%
\begin{pgfscope}%
\pgfsetbuttcap%
\pgfsetmiterjoin%
\pgfsetlinewidth{2.258437pt}%
\definecolor{currentstroke}{rgb}{0.000000,0.605603,0.978680}%
\pgfsetstrokecolor{currentstroke}%
\pgfsetdash{}{0pt}%
\pgfpathmoveto{\pgfqpoint{10.711589in}{7.493818in}}%
\pgfpathlineto{\pgfqpoint{11.211589in}{7.493818in}}%
\pgfusepath{stroke}%
\end{pgfscope}%
\begin{pgfscope}%
\definecolor{textcolor}{rgb}{0.000000,0.000000,0.000000}%
\pgfsetstrokecolor{textcolor}%
\pgfsetfillcolor{textcolor}%
\pgftext[x=11.411589in,y=7.406318in,left,base]{\color{textcolor}\sffamily\fontsize{18.000000}{21.600000}\selectfont $\displaystyle U$}%
\end{pgfscope}%
\begin{pgfscope}%
\pgfsetbuttcap%
\pgfsetmiterjoin%
\pgfsetlinewidth{2.258437pt}%
\definecolor{currentstroke}{rgb}{0.888874,0.435649,0.278123}%
\pgfsetstrokecolor{currentstroke}%
\pgfsetdash{}{0pt}%
\pgfpathmoveto{\pgfqpoint{10.711589in}{7.126875in}}%
\pgfpathlineto{\pgfqpoint{11.211589in}{7.126875in}}%
\pgfusepath{stroke}%
\end{pgfscope}%
\begin{pgfscope}%
\definecolor{textcolor}{rgb}{0.000000,0.000000,0.000000}%
\pgfsetstrokecolor{textcolor}%
\pgfsetfillcolor{textcolor}%
\pgftext[x=11.411589in,y=7.039375in,left,base]{\color{textcolor}\sffamily\fontsize{18.000000}{21.600000}\selectfont $\displaystyle u$}%
\end{pgfscope}%
\end{pgfpicture}%
\makeatother%
\endgroup%
}
	\caption{Lax-Wendroff 差分逼近解 $U$ 与真解 $u$}\label{fig:lax_wendroff_Uu_noCFL}
\end{figure}

对方波问题, 取 $\nu = 0.5$. $h = 2^{-7}$ 和 $h = 2^{-11}$ 时差分逼近解 $U$ 与真解 $u$ 在 $t = t_{\max }$ 时刻图像如图 \ref{fig:lax_wendroff_square_Uu} 所示. 可以看出差分逼近解在间断点左侧出现震荡.

\begin{figure}[H]\centering\zihao{-5}
	\resizebox{0.4\linewidth}{!}{%% Creator: Matplotlib, PGF backend
%%
%% To include the figure in your LaTeX document, write
%%   \input{<filename>.pgf}
%%
%% Make sure the required packages are loaded in your preamble
%%   \usepackage{pgf}
%%
%% Figures using additional raster images can only be included by \input if
%% they are in the same directory as the main LaTeX file. For loading figures
%% from other directories you can use the `import` package
%%   \usepackage{import}
%%
%% and then include the figures with
%%   \import{<path to file>}{<filename>.pgf}
%%
%% Matplotlib used the following preamble
%%   \usepackage{fontspec}
%%   \setmainfont{DejaVuSerif.ttf}[Path=\detokenize{/Users/quejiahao/.julia/conda/3/lib/python3.9/site-packages/matplotlib/mpl-data/fonts/ttf/}]
%%   \setsansfont{DejaVuSans.ttf}[Path=\detokenize{/Users/quejiahao/.julia/conda/3/lib/python3.9/site-packages/matplotlib/mpl-data/fonts/ttf/}]
%%   \setmonofont{DejaVuSansMono.ttf}[Path=\detokenize{/Users/quejiahao/.julia/conda/3/lib/python3.9/site-packages/matplotlib/mpl-data/fonts/ttf/}]
%%
\begingroup%
\makeatletter%
\begin{pgfpicture}%
\pgfpathrectangle{\pgfpointorigin}{\pgfqpoint{12.000000in}{8.000000in}}%
\pgfusepath{use as bounding box, clip}%
\begin{pgfscope}%
\pgfsetbuttcap%
\pgfsetmiterjoin%
\definecolor{currentfill}{rgb}{1.000000,1.000000,1.000000}%
\pgfsetfillcolor{currentfill}%
\pgfsetlinewidth{0.000000pt}%
\definecolor{currentstroke}{rgb}{1.000000,1.000000,1.000000}%
\pgfsetstrokecolor{currentstroke}%
\pgfsetdash{}{0pt}%
\pgfpathmoveto{\pgfqpoint{0.000000in}{0.000000in}}%
\pgfpathlineto{\pgfqpoint{12.000000in}{0.000000in}}%
\pgfpathlineto{\pgfqpoint{12.000000in}{8.000000in}}%
\pgfpathlineto{\pgfqpoint{0.000000in}{8.000000in}}%
\pgfpathclose%
\pgfusepath{fill}%
\end{pgfscope}%
\begin{pgfscope}%
\pgfsetbuttcap%
\pgfsetmiterjoin%
\definecolor{currentfill}{rgb}{1.000000,1.000000,1.000000}%
\pgfsetfillcolor{currentfill}%
\pgfsetlinewidth{0.000000pt}%
\definecolor{currentstroke}{rgb}{0.000000,0.000000,0.000000}%
\pgfsetstrokecolor{currentstroke}%
\pgfsetstrokeopacity{0.000000}%
\pgfsetdash{}{0pt}%
\pgfpathmoveto{\pgfqpoint{0.978013in}{1.247073in}}%
\pgfpathlineto{\pgfqpoint{11.921260in}{1.247073in}}%
\pgfpathlineto{\pgfqpoint{11.921260in}{7.921260in}}%
\pgfpathlineto{\pgfqpoint{0.978013in}{7.921260in}}%
\pgfpathclose%
\pgfusepath{fill}%
\end{pgfscope}%
\begin{pgfscope}%
\pgfpathrectangle{\pgfqpoint{0.978013in}{1.247073in}}{\pgfqpoint{10.943247in}{6.674186in}}%
\pgfusepath{clip}%
\pgfsetrectcap%
\pgfsetroundjoin%
\pgfsetlinewidth{0.501875pt}%
\definecolor{currentstroke}{rgb}{0.000000,0.000000,0.000000}%
\pgfsetstrokecolor{currentstroke}%
\pgfsetstrokeopacity{0.100000}%
\pgfsetdash{}{0pt}%
\pgfpathmoveto{\pgfqpoint{1.287728in}{1.247073in}}%
\pgfpathlineto{\pgfqpoint{1.287728in}{7.921260in}}%
\pgfusepath{stroke}%
\end{pgfscope}%
\begin{pgfscope}%
\pgfsetbuttcap%
\pgfsetroundjoin%
\definecolor{currentfill}{rgb}{0.000000,0.000000,0.000000}%
\pgfsetfillcolor{currentfill}%
\pgfsetlinewidth{0.501875pt}%
\definecolor{currentstroke}{rgb}{0.000000,0.000000,0.000000}%
\pgfsetstrokecolor{currentstroke}%
\pgfsetdash{}{0pt}%
\pgfsys@defobject{currentmarker}{\pgfqpoint{0.000000in}{0.000000in}}{\pgfqpoint{0.000000in}{0.034722in}}{%
\pgfpathmoveto{\pgfqpoint{0.000000in}{0.000000in}}%
\pgfpathlineto{\pgfqpoint{0.000000in}{0.034722in}}%
\pgfusepath{stroke,fill}%
}%
\begin{pgfscope}%
\pgfsys@transformshift{1.287728in}{1.247073in}%
\pgfsys@useobject{currentmarker}{}%
\end{pgfscope}%
\end{pgfscope}%
\begin{pgfscope}%
\definecolor{textcolor}{rgb}{0.000000,0.000000,0.000000}%
\pgfsetstrokecolor{textcolor}%
\pgfsetfillcolor{textcolor}%
\pgftext[x=1.287728in,y=1.198462in,,top]{\color{textcolor}\sffamily\fontsize{18.000000}{21.600000}\selectfont $\displaystyle 0$}%
\end{pgfscope}%
\begin{pgfscope}%
\pgfpathrectangle{\pgfqpoint{0.978013in}{1.247073in}}{\pgfqpoint{10.943247in}{6.674186in}}%
\pgfusepath{clip}%
\pgfsetrectcap%
\pgfsetroundjoin%
\pgfsetlinewidth{0.501875pt}%
\definecolor{currentstroke}{rgb}{0.000000,0.000000,0.000000}%
\pgfsetstrokecolor{currentstroke}%
\pgfsetstrokeopacity{0.100000}%
\pgfsetdash{}{0pt}%
\pgfpathmoveto{\pgfqpoint{2.930814in}{1.247073in}}%
\pgfpathlineto{\pgfqpoint{2.930814in}{7.921260in}}%
\pgfusepath{stroke}%
\end{pgfscope}%
\begin{pgfscope}%
\pgfsetbuttcap%
\pgfsetroundjoin%
\definecolor{currentfill}{rgb}{0.000000,0.000000,0.000000}%
\pgfsetfillcolor{currentfill}%
\pgfsetlinewidth{0.501875pt}%
\definecolor{currentstroke}{rgb}{0.000000,0.000000,0.000000}%
\pgfsetstrokecolor{currentstroke}%
\pgfsetdash{}{0pt}%
\pgfsys@defobject{currentmarker}{\pgfqpoint{0.000000in}{0.000000in}}{\pgfqpoint{0.000000in}{0.034722in}}{%
\pgfpathmoveto{\pgfqpoint{0.000000in}{0.000000in}}%
\pgfpathlineto{\pgfqpoint{0.000000in}{0.034722in}}%
\pgfusepath{stroke,fill}%
}%
\begin{pgfscope}%
\pgfsys@transformshift{2.930814in}{1.247073in}%
\pgfsys@useobject{currentmarker}{}%
\end{pgfscope}%
\end{pgfscope}%
\begin{pgfscope}%
\definecolor{textcolor}{rgb}{0.000000,0.000000,0.000000}%
\pgfsetstrokecolor{textcolor}%
\pgfsetfillcolor{textcolor}%
\pgftext[x=2.930814in,y=1.198462in,,top]{\color{textcolor}\sffamily\fontsize{18.000000}{21.600000}\selectfont $\displaystyle 1$}%
\end{pgfscope}%
\begin{pgfscope}%
\pgfpathrectangle{\pgfqpoint{0.978013in}{1.247073in}}{\pgfqpoint{10.943247in}{6.674186in}}%
\pgfusepath{clip}%
\pgfsetrectcap%
\pgfsetroundjoin%
\pgfsetlinewidth{0.501875pt}%
\definecolor{currentstroke}{rgb}{0.000000,0.000000,0.000000}%
\pgfsetstrokecolor{currentstroke}%
\pgfsetstrokeopacity{0.100000}%
\pgfsetdash{}{0pt}%
\pgfpathmoveto{\pgfqpoint{4.573901in}{1.247073in}}%
\pgfpathlineto{\pgfqpoint{4.573901in}{7.921260in}}%
\pgfusepath{stroke}%
\end{pgfscope}%
\begin{pgfscope}%
\pgfsetbuttcap%
\pgfsetroundjoin%
\definecolor{currentfill}{rgb}{0.000000,0.000000,0.000000}%
\pgfsetfillcolor{currentfill}%
\pgfsetlinewidth{0.501875pt}%
\definecolor{currentstroke}{rgb}{0.000000,0.000000,0.000000}%
\pgfsetstrokecolor{currentstroke}%
\pgfsetdash{}{0pt}%
\pgfsys@defobject{currentmarker}{\pgfqpoint{0.000000in}{0.000000in}}{\pgfqpoint{0.000000in}{0.034722in}}{%
\pgfpathmoveto{\pgfqpoint{0.000000in}{0.000000in}}%
\pgfpathlineto{\pgfqpoint{0.000000in}{0.034722in}}%
\pgfusepath{stroke,fill}%
}%
\begin{pgfscope}%
\pgfsys@transformshift{4.573901in}{1.247073in}%
\pgfsys@useobject{currentmarker}{}%
\end{pgfscope}%
\end{pgfscope}%
\begin{pgfscope}%
\definecolor{textcolor}{rgb}{0.000000,0.000000,0.000000}%
\pgfsetstrokecolor{textcolor}%
\pgfsetfillcolor{textcolor}%
\pgftext[x=4.573901in,y=1.198462in,,top]{\color{textcolor}\sffamily\fontsize{18.000000}{21.600000}\selectfont $\displaystyle 2$}%
\end{pgfscope}%
\begin{pgfscope}%
\pgfpathrectangle{\pgfqpoint{0.978013in}{1.247073in}}{\pgfqpoint{10.943247in}{6.674186in}}%
\pgfusepath{clip}%
\pgfsetrectcap%
\pgfsetroundjoin%
\pgfsetlinewidth{0.501875pt}%
\definecolor{currentstroke}{rgb}{0.000000,0.000000,0.000000}%
\pgfsetstrokecolor{currentstroke}%
\pgfsetstrokeopacity{0.100000}%
\pgfsetdash{}{0pt}%
\pgfpathmoveto{\pgfqpoint{6.216988in}{1.247073in}}%
\pgfpathlineto{\pgfqpoint{6.216988in}{7.921260in}}%
\pgfusepath{stroke}%
\end{pgfscope}%
\begin{pgfscope}%
\pgfsetbuttcap%
\pgfsetroundjoin%
\definecolor{currentfill}{rgb}{0.000000,0.000000,0.000000}%
\pgfsetfillcolor{currentfill}%
\pgfsetlinewidth{0.501875pt}%
\definecolor{currentstroke}{rgb}{0.000000,0.000000,0.000000}%
\pgfsetstrokecolor{currentstroke}%
\pgfsetdash{}{0pt}%
\pgfsys@defobject{currentmarker}{\pgfqpoint{0.000000in}{0.000000in}}{\pgfqpoint{0.000000in}{0.034722in}}{%
\pgfpathmoveto{\pgfqpoint{0.000000in}{0.000000in}}%
\pgfpathlineto{\pgfqpoint{0.000000in}{0.034722in}}%
\pgfusepath{stroke,fill}%
}%
\begin{pgfscope}%
\pgfsys@transformshift{6.216988in}{1.247073in}%
\pgfsys@useobject{currentmarker}{}%
\end{pgfscope}%
\end{pgfscope}%
\begin{pgfscope}%
\definecolor{textcolor}{rgb}{0.000000,0.000000,0.000000}%
\pgfsetstrokecolor{textcolor}%
\pgfsetfillcolor{textcolor}%
\pgftext[x=6.216988in,y=1.198462in,,top]{\color{textcolor}\sffamily\fontsize{18.000000}{21.600000}\selectfont $\displaystyle 3$}%
\end{pgfscope}%
\begin{pgfscope}%
\pgfpathrectangle{\pgfqpoint{0.978013in}{1.247073in}}{\pgfqpoint{10.943247in}{6.674186in}}%
\pgfusepath{clip}%
\pgfsetrectcap%
\pgfsetroundjoin%
\pgfsetlinewidth{0.501875pt}%
\definecolor{currentstroke}{rgb}{0.000000,0.000000,0.000000}%
\pgfsetstrokecolor{currentstroke}%
\pgfsetstrokeopacity{0.100000}%
\pgfsetdash{}{0pt}%
\pgfpathmoveto{\pgfqpoint{7.860074in}{1.247073in}}%
\pgfpathlineto{\pgfqpoint{7.860074in}{7.921260in}}%
\pgfusepath{stroke}%
\end{pgfscope}%
\begin{pgfscope}%
\pgfsetbuttcap%
\pgfsetroundjoin%
\definecolor{currentfill}{rgb}{0.000000,0.000000,0.000000}%
\pgfsetfillcolor{currentfill}%
\pgfsetlinewidth{0.501875pt}%
\definecolor{currentstroke}{rgb}{0.000000,0.000000,0.000000}%
\pgfsetstrokecolor{currentstroke}%
\pgfsetdash{}{0pt}%
\pgfsys@defobject{currentmarker}{\pgfqpoint{0.000000in}{0.000000in}}{\pgfqpoint{0.000000in}{0.034722in}}{%
\pgfpathmoveto{\pgfqpoint{0.000000in}{0.000000in}}%
\pgfpathlineto{\pgfqpoint{0.000000in}{0.034722in}}%
\pgfusepath{stroke,fill}%
}%
\begin{pgfscope}%
\pgfsys@transformshift{7.860074in}{1.247073in}%
\pgfsys@useobject{currentmarker}{}%
\end{pgfscope}%
\end{pgfscope}%
\begin{pgfscope}%
\definecolor{textcolor}{rgb}{0.000000,0.000000,0.000000}%
\pgfsetstrokecolor{textcolor}%
\pgfsetfillcolor{textcolor}%
\pgftext[x=7.860074in,y=1.198462in,,top]{\color{textcolor}\sffamily\fontsize{18.000000}{21.600000}\selectfont $\displaystyle 4$}%
\end{pgfscope}%
\begin{pgfscope}%
\pgfpathrectangle{\pgfqpoint{0.978013in}{1.247073in}}{\pgfqpoint{10.943247in}{6.674186in}}%
\pgfusepath{clip}%
\pgfsetrectcap%
\pgfsetroundjoin%
\pgfsetlinewidth{0.501875pt}%
\definecolor{currentstroke}{rgb}{0.000000,0.000000,0.000000}%
\pgfsetstrokecolor{currentstroke}%
\pgfsetstrokeopacity{0.100000}%
\pgfsetdash{}{0pt}%
\pgfpathmoveto{\pgfqpoint{9.503161in}{1.247073in}}%
\pgfpathlineto{\pgfqpoint{9.503161in}{7.921260in}}%
\pgfusepath{stroke}%
\end{pgfscope}%
\begin{pgfscope}%
\pgfsetbuttcap%
\pgfsetroundjoin%
\definecolor{currentfill}{rgb}{0.000000,0.000000,0.000000}%
\pgfsetfillcolor{currentfill}%
\pgfsetlinewidth{0.501875pt}%
\definecolor{currentstroke}{rgb}{0.000000,0.000000,0.000000}%
\pgfsetstrokecolor{currentstroke}%
\pgfsetdash{}{0pt}%
\pgfsys@defobject{currentmarker}{\pgfqpoint{0.000000in}{0.000000in}}{\pgfqpoint{0.000000in}{0.034722in}}{%
\pgfpathmoveto{\pgfqpoint{0.000000in}{0.000000in}}%
\pgfpathlineto{\pgfqpoint{0.000000in}{0.034722in}}%
\pgfusepath{stroke,fill}%
}%
\begin{pgfscope}%
\pgfsys@transformshift{9.503161in}{1.247073in}%
\pgfsys@useobject{currentmarker}{}%
\end{pgfscope}%
\end{pgfscope}%
\begin{pgfscope}%
\definecolor{textcolor}{rgb}{0.000000,0.000000,0.000000}%
\pgfsetstrokecolor{textcolor}%
\pgfsetfillcolor{textcolor}%
\pgftext[x=9.503161in,y=1.198462in,,top]{\color{textcolor}\sffamily\fontsize{18.000000}{21.600000}\selectfont $\displaystyle 5$}%
\end{pgfscope}%
\begin{pgfscope}%
\pgfpathrectangle{\pgfqpoint{0.978013in}{1.247073in}}{\pgfqpoint{10.943247in}{6.674186in}}%
\pgfusepath{clip}%
\pgfsetrectcap%
\pgfsetroundjoin%
\pgfsetlinewidth{0.501875pt}%
\definecolor{currentstroke}{rgb}{0.000000,0.000000,0.000000}%
\pgfsetstrokecolor{currentstroke}%
\pgfsetstrokeopacity{0.100000}%
\pgfsetdash{}{0pt}%
\pgfpathmoveto{\pgfqpoint{11.146247in}{1.247073in}}%
\pgfpathlineto{\pgfqpoint{11.146247in}{7.921260in}}%
\pgfusepath{stroke}%
\end{pgfscope}%
\begin{pgfscope}%
\pgfsetbuttcap%
\pgfsetroundjoin%
\definecolor{currentfill}{rgb}{0.000000,0.000000,0.000000}%
\pgfsetfillcolor{currentfill}%
\pgfsetlinewidth{0.501875pt}%
\definecolor{currentstroke}{rgb}{0.000000,0.000000,0.000000}%
\pgfsetstrokecolor{currentstroke}%
\pgfsetdash{}{0pt}%
\pgfsys@defobject{currentmarker}{\pgfqpoint{0.000000in}{0.000000in}}{\pgfqpoint{0.000000in}{0.034722in}}{%
\pgfpathmoveto{\pgfqpoint{0.000000in}{0.000000in}}%
\pgfpathlineto{\pgfqpoint{0.000000in}{0.034722in}}%
\pgfusepath{stroke,fill}%
}%
\begin{pgfscope}%
\pgfsys@transformshift{11.146247in}{1.247073in}%
\pgfsys@useobject{currentmarker}{}%
\end{pgfscope}%
\end{pgfscope}%
\begin{pgfscope}%
\definecolor{textcolor}{rgb}{0.000000,0.000000,0.000000}%
\pgfsetstrokecolor{textcolor}%
\pgfsetfillcolor{textcolor}%
\pgftext[x=11.146247in,y=1.198462in,,top]{\color{textcolor}\sffamily\fontsize{18.000000}{21.600000}\selectfont $\displaystyle 6$}%
\end{pgfscope}%
\begin{pgfscope}%
\definecolor{textcolor}{rgb}{0.000000,0.000000,0.000000}%
\pgfsetstrokecolor{textcolor}%
\pgfsetfillcolor{textcolor}%
\pgftext[x=6.449637in,y=0.900964in,,top]{\color{textcolor}\sffamily\fontsize{18.000000}{21.600000}\selectfont $\displaystyle x$}%
\end{pgfscope}%
\begin{pgfscope}%
\pgfpathrectangle{\pgfqpoint{0.978013in}{1.247073in}}{\pgfqpoint{10.943247in}{6.674186in}}%
\pgfusepath{clip}%
\pgfsetrectcap%
\pgfsetroundjoin%
\pgfsetlinewidth{0.501875pt}%
\definecolor{currentstroke}{rgb}{0.000000,0.000000,0.000000}%
\pgfsetstrokecolor{currentstroke}%
\pgfsetstrokeopacity{0.100000}%
\pgfsetdash{}{0pt}%
\pgfpathmoveto{\pgfqpoint{0.978013in}{1.435966in}}%
\pgfpathlineto{\pgfqpoint{11.921260in}{1.435966in}}%
\pgfusepath{stroke}%
\end{pgfscope}%
\begin{pgfscope}%
\pgfsetbuttcap%
\pgfsetroundjoin%
\definecolor{currentfill}{rgb}{0.000000,0.000000,0.000000}%
\pgfsetfillcolor{currentfill}%
\pgfsetlinewidth{0.501875pt}%
\definecolor{currentstroke}{rgb}{0.000000,0.000000,0.000000}%
\pgfsetstrokecolor{currentstroke}%
\pgfsetdash{}{0pt}%
\pgfsys@defobject{currentmarker}{\pgfqpoint{0.000000in}{0.000000in}}{\pgfqpoint{0.034722in}{0.000000in}}{%
\pgfpathmoveto{\pgfqpoint{0.000000in}{0.000000in}}%
\pgfpathlineto{\pgfqpoint{0.034722in}{0.000000in}}%
\pgfusepath{stroke,fill}%
}%
\begin{pgfscope}%
\pgfsys@transformshift{0.978013in}{1.435966in}%
\pgfsys@useobject{currentmarker}{}%
\end{pgfscope}%
\end{pgfscope}%
\begin{pgfscope}%
\definecolor{textcolor}{rgb}{0.000000,0.000000,0.000000}%
\pgfsetstrokecolor{textcolor}%
\pgfsetfillcolor{textcolor}%
\pgftext[x=0.643989in, y=1.340995in, left, base]{\color{textcolor}\sffamily\fontsize{18.000000}{21.600000}\selectfont $\displaystyle 1.0$}%
\end{pgfscope}%
\begin{pgfscope}%
\pgfpathrectangle{\pgfqpoint{0.978013in}{1.247073in}}{\pgfqpoint{10.943247in}{6.674186in}}%
\pgfusepath{clip}%
\pgfsetrectcap%
\pgfsetroundjoin%
\pgfsetlinewidth{0.501875pt}%
\definecolor{currentstroke}{rgb}{0.000000,0.000000,0.000000}%
\pgfsetstrokecolor{currentstroke}%
\pgfsetstrokeopacity{0.100000}%
\pgfsetdash{}{0pt}%
\pgfpathmoveto{\pgfqpoint{0.978013in}{2.751919in}}%
\pgfpathlineto{\pgfqpoint{11.921260in}{2.751919in}}%
\pgfusepath{stroke}%
\end{pgfscope}%
\begin{pgfscope}%
\pgfsetbuttcap%
\pgfsetroundjoin%
\definecolor{currentfill}{rgb}{0.000000,0.000000,0.000000}%
\pgfsetfillcolor{currentfill}%
\pgfsetlinewidth{0.501875pt}%
\definecolor{currentstroke}{rgb}{0.000000,0.000000,0.000000}%
\pgfsetstrokecolor{currentstroke}%
\pgfsetdash{}{0pt}%
\pgfsys@defobject{currentmarker}{\pgfqpoint{0.000000in}{0.000000in}}{\pgfqpoint{0.034722in}{0.000000in}}{%
\pgfpathmoveto{\pgfqpoint{0.000000in}{0.000000in}}%
\pgfpathlineto{\pgfqpoint{0.034722in}{0.000000in}}%
\pgfusepath{stroke,fill}%
}%
\begin{pgfscope}%
\pgfsys@transformshift{0.978013in}{2.751919in}%
\pgfsys@useobject{currentmarker}{}%
\end{pgfscope}%
\end{pgfscope}%
\begin{pgfscope}%
\definecolor{textcolor}{rgb}{0.000000,0.000000,0.000000}%
\pgfsetstrokecolor{textcolor}%
\pgfsetfillcolor{textcolor}%
\pgftext[x=0.643989in, y=2.656948in, left, base]{\color{textcolor}\sffamily\fontsize{18.000000}{21.600000}\selectfont $\displaystyle 1.5$}%
\end{pgfscope}%
\begin{pgfscope}%
\pgfpathrectangle{\pgfqpoint{0.978013in}{1.247073in}}{\pgfqpoint{10.943247in}{6.674186in}}%
\pgfusepath{clip}%
\pgfsetrectcap%
\pgfsetroundjoin%
\pgfsetlinewidth{0.501875pt}%
\definecolor{currentstroke}{rgb}{0.000000,0.000000,0.000000}%
\pgfsetstrokecolor{currentstroke}%
\pgfsetstrokeopacity{0.100000}%
\pgfsetdash{}{0pt}%
\pgfpathmoveto{\pgfqpoint{0.978013in}{4.067873in}}%
\pgfpathlineto{\pgfqpoint{11.921260in}{4.067873in}}%
\pgfusepath{stroke}%
\end{pgfscope}%
\begin{pgfscope}%
\pgfsetbuttcap%
\pgfsetroundjoin%
\definecolor{currentfill}{rgb}{0.000000,0.000000,0.000000}%
\pgfsetfillcolor{currentfill}%
\pgfsetlinewidth{0.501875pt}%
\definecolor{currentstroke}{rgb}{0.000000,0.000000,0.000000}%
\pgfsetstrokecolor{currentstroke}%
\pgfsetdash{}{0pt}%
\pgfsys@defobject{currentmarker}{\pgfqpoint{0.000000in}{0.000000in}}{\pgfqpoint{0.034722in}{0.000000in}}{%
\pgfpathmoveto{\pgfqpoint{0.000000in}{0.000000in}}%
\pgfpathlineto{\pgfqpoint{0.034722in}{0.000000in}}%
\pgfusepath{stroke,fill}%
}%
\begin{pgfscope}%
\pgfsys@transformshift{0.978013in}{4.067873in}%
\pgfsys@useobject{currentmarker}{}%
\end{pgfscope}%
\end{pgfscope}%
\begin{pgfscope}%
\definecolor{textcolor}{rgb}{0.000000,0.000000,0.000000}%
\pgfsetstrokecolor{textcolor}%
\pgfsetfillcolor{textcolor}%
\pgftext[x=0.643989in, y=3.972902in, left, base]{\color{textcolor}\sffamily\fontsize{18.000000}{21.600000}\selectfont $\displaystyle 2.0$}%
\end{pgfscope}%
\begin{pgfscope}%
\pgfpathrectangle{\pgfqpoint{0.978013in}{1.247073in}}{\pgfqpoint{10.943247in}{6.674186in}}%
\pgfusepath{clip}%
\pgfsetrectcap%
\pgfsetroundjoin%
\pgfsetlinewidth{0.501875pt}%
\definecolor{currentstroke}{rgb}{0.000000,0.000000,0.000000}%
\pgfsetstrokecolor{currentstroke}%
\pgfsetstrokeopacity{0.100000}%
\pgfsetdash{}{0pt}%
\pgfpathmoveto{\pgfqpoint{0.978013in}{5.383826in}}%
\pgfpathlineto{\pgfqpoint{11.921260in}{5.383826in}}%
\pgfusepath{stroke}%
\end{pgfscope}%
\begin{pgfscope}%
\pgfsetbuttcap%
\pgfsetroundjoin%
\definecolor{currentfill}{rgb}{0.000000,0.000000,0.000000}%
\pgfsetfillcolor{currentfill}%
\pgfsetlinewidth{0.501875pt}%
\definecolor{currentstroke}{rgb}{0.000000,0.000000,0.000000}%
\pgfsetstrokecolor{currentstroke}%
\pgfsetdash{}{0pt}%
\pgfsys@defobject{currentmarker}{\pgfqpoint{0.000000in}{0.000000in}}{\pgfqpoint{0.034722in}{0.000000in}}{%
\pgfpathmoveto{\pgfqpoint{0.000000in}{0.000000in}}%
\pgfpathlineto{\pgfqpoint{0.034722in}{0.000000in}}%
\pgfusepath{stroke,fill}%
}%
\begin{pgfscope}%
\pgfsys@transformshift{0.978013in}{5.383826in}%
\pgfsys@useobject{currentmarker}{}%
\end{pgfscope}%
\end{pgfscope}%
\begin{pgfscope}%
\definecolor{textcolor}{rgb}{0.000000,0.000000,0.000000}%
\pgfsetstrokecolor{textcolor}%
\pgfsetfillcolor{textcolor}%
\pgftext[x=0.643989in, y=5.288855in, left, base]{\color{textcolor}\sffamily\fontsize{18.000000}{21.600000}\selectfont $\displaystyle 2.5$}%
\end{pgfscope}%
\begin{pgfscope}%
\pgfpathrectangle{\pgfqpoint{0.978013in}{1.247073in}}{\pgfqpoint{10.943247in}{6.674186in}}%
\pgfusepath{clip}%
\pgfsetrectcap%
\pgfsetroundjoin%
\pgfsetlinewidth{0.501875pt}%
\definecolor{currentstroke}{rgb}{0.000000,0.000000,0.000000}%
\pgfsetstrokecolor{currentstroke}%
\pgfsetstrokeopacity{0.100000}%
\pgfsetdash{}{0pt}%
\pgfpathmoveto{\pgfqpoint{0.978013in}{6.699780in}}%
\pgfpathlineto{\pgfqpoint{11.921260in}{6.699780in}}%
\pgfusepath{stroke}%
\end{pgfscope}%
\begin{pgfscope}%
\pgfsetbuttcap%
\pgfsetroundjoin%
\definecolor{currentfill}{rgb}{0.000000,0.000000,0.000000}%
\pgfsetfillcolor{currentfill}%
\pgfsetlinewidth{0.501875pt}%
\definecolor{currentstroke}{rgb}{0.000000,0.000000,0.000000}%
\pgfsetstrokecolor{currentstroke}%
\pgfsetdash{}{0pt}%
\pgfsys@defobject{currentmarker}{\pgfqpoint{0.000000in}{0.000000in}}{\pgfqpoint{0.034722in}{0.000000in}}{%
\pgfpathmoveto{\pgfqpoint{0.000000in}{0.000000in}}%
\pgfpathlineto{\pgfqpoint{0.034722in}{0.000000in}}%
\pgfusepath{stroke,fill}%
}%
\begin{pgfscope}%
\pgfsys@transformshift{0.978013in}{6.699780in}%
\pgfsys@useobject{currentmarker}{}%
\end{pgfscope}%
\end{pgfscope}%
\begin{pgfscope}%
\definecolor{textcolor}{rgb}{0.000000,0.000000,0.000000}%
\pgfsetstrokecolor{textcolor}%
\pgfsetfillcolor{textcolor}%
\pgftext[x=0.643989in, y=6.604809in, left, base]{\color{textcolor}\sffamily\fontsize{18.000000}{21.600000}\selectfont $\displaystyle 3.0$}%
\end{pgfscope}%
\begin{pgfscope}%
\pgfpathrectangle{\pgfqpoint{0.978013in}{1.247073in}}{\pgfqpoint{10.943247in}{6.674186in}}%
\pgfusepath{clip}%
\pgfsetbuttcap%
\pgfsetroundjoin%
\pgfsetlinewidth{1.003750pt}%
\definecolor{currentstroke}{rgb}{0.000000,0.605603,0.978680}%
\pgfsetstrokecolor{currentstroke}%
\pgfsetdash{}{0pt}%
\pgfpathmoveto{\pgfqpoint{1.287728in}{6.699780in}}%
\pgfpathlineto{\pgfqpoint{4.917195in}{6.699519in}}%
\pgfpathlineto{\pgfqpoint{4.997850in}{6.701861in}}%
\pgfpathlineto{\pgfqpoint{5.078504in}{6.701300in}}%
\pgfpathlineto{\pgfqpoint{5.159159in}{6.695004in}}%
\pgfpathlineto{\pgfqpoint{5.239814in}{6.693096in}}%
\pgfpathlineto{\pgfqpoint{5.401124in}{6.721212in}}%
\pgfpathlineto{\pgfqpoint{5.481779in}{6.701890in}}%
\pgfpathlineto{\pgfqpoint{5.562433in}{6.654555in}}%
\pgfpathlineto{\pgfqpoint{5.643088in}{6.648340in}}%
\pgfpathlineto{\pgfqpoint{5.723743in}{6.734748in}}%
\pgfpathlineto{\pgfqpoint{5.804398in}{6.843417in}}%
\pgfpathlineto{\pgfqpoint{5.885053in}{6.817130in}}%
\pgfpathlineto{\pgfqpoint{5.965708in}{6.595075in}}%
\pgfpathlineto{\pgfqpoint{6.046362in}{6.341347in}}%
\pgfpathlineto{\pgfqpoint{6.127017in}{6.338602in}}%
\pgfpathlineto{\pgfqpoint{6.207672in}{6.722690in}}%
\pgfpathlineto{\pgfqpoint{6.288327in}{7.317761in}}%
\pgfpathlineto{\pgfqpoint{6.368982in}{7.732368in}}%
\pgfpathlineto{\pgfqpoint{6.449637in}{7.623986in}}%
\pgfpathlineto{\pgfqpoint{6.530291in}{6.906434in}}%
\pgfpathlineto{\pgfqpoint{6.610946in}{5.762316in}}%
\pgfpathlineto{\pgfqpoint{6.691601in}{4.499352in}}%
\pgfpathlineto{\pgfqpoint{6.772256in}{3.383788in}}%
\pgfpathlineto{\pgfqpoint{6.852911in}{2.551818in}}%
\pgfpathlineto{\pgfqpoint{6.933565in}{2.013630in}}%
\pgfpathlineto{\pgfqpoint{7.014220in}{1.706870in}}%
\pgfpathlineto{\pgfqpoint{7.094875in}{1.551264in}}%
\pgfpathlineto{\pgfqpoint{7.175530in}{1.480554in}}%
\pgfpathlineto{\pgfqpoint{7.256185in}{1.451644in}}%
\pgfpathlineto{\pgfqpoint{7.336840in}{1.440978in}}%
\pgfpathlineto{\pgfqpoint{7.417494in}{1.437422in}}%
\pgfpathlineto{\pgfqpoint{7.578804in}{1.436057in}}%
\pgfpathlineto{\pgfqpoint{8.869281in}{1.435966in}}%
\pgfpathlineto{\pgfqpoint{11.611545in}{1.435966in}}%
\pgfpathlineto{\pgfqpoint{11.611545in}{1.435966in}}%
\pgfusepath{stroke}%
\end{pgfscope}%
\begin{pgfscope}%
\pgfpathrectangle{\pgfqpoint{0.978013in}{1.247073in}}{\pgfqpoint{10.943247in}{6.674186in}}%
\pgfusepath{clip}%
\pgfsetbuttcap%
\pgfsetroundjoin%
\pgfsetlinewidth{1.003750pt}%
\definecolor{currentstroke}{rgb}{0.888874,0.435649,0.278123}%
\pgfsetstrokecolor{currentstroke}%
\pgfsetdash{}{0pt}%
\pgfpathmoveto{\pgfqpoint{1.287728in}{6.699780in}}%
\pgfpathlineto{\pgfqpoint{6.772256in}{6.699780in}}%
\pgfpathlineto{\pgfqpoint{6.852911in}{1.435966in}}%
\pgfpathlineto{\pgfqpoint{11.611545in}{1.435966in}}%
\pgfpathlineto{\pgfqpoint{11.611545in}{1.435966in}}%
\pgfusepath{stroke}%
\end{pgfscope}%
\begin{pgfscope}%
\pgfsetrectcap%
\pgfsetmiterjoin%
\pgfsetlinewidth{1.003750pt}%
\definecolor{currentstroke}{rgb}{0.000000,0.000000,0.000000}%
\pgfsetstrokecolor{currentstroke}%
\pgfsetdash{}{0pt}%
\pgfpathmoveto{\pgfqpoint{0.978013in}{1.247073in}}%
\pgfpathlineto{\pgfqpoint{0.978013in}{7.921260in}}%
\pgfusepath{stroke}%
\end{pgfscope}%
\begin{pgfscope}%
\pgfsetrectcap%
\pgfsetmiterjoin%
\pgfsetlinewidth{1.003750pt}%
\definecolor{currentstroke}{rgb}{0.000000,0.000000,0.000000}%
\pgfsetstrokecolor{currentstroke}%
\pgfsetdash{}{0pt}%
\pgfpathmoveto{\pgfqpoint{0.978013in}{1.247073in}}%
\pgfpathlineto{\pgfqpoint{11.921260in}{1.247073in}}%
\pgfusepath{stroke}%
\end{pgfscope}%
\begin{pgfscope}%
\pgfsetbuttcap%
\pgfsetmiterjoin%
\definecolor{currentfill}{rgb}{1.000000,1.000000,1.000000}%
\pgfsetfillcolor{currentfill}%
\pgfsetlinewidth{1.003750pt}%
\definecolor{currentstroke}{rgb}{0.000000,0.000000,0.000000}%
\pgfsetstrokecolor{currentstroke}%
\pgfsetdash{}{0pt}%
\pgfpathmoveto{\pgfqpoint{10.511589in}{6.787373in}}%
\pgfpathlineto{\pgfqpoint{11.796260in}{6.787373in}}%
\pgfpathlineto{\pgfqpoint{11.796260in}{7.796260in}}%
\pgfpathlineto{\pgfqpoint{10.511589in}{7.796260in}}%
\pgfpathclose%
\pgfusepath{stroke,fill}%
\end{pgfscope}%
\begin{pgfscope}%
\pgfsetbuttcap%
\pgfsetmiterjoin%
\pgfsetlinewidth{2.258437pt}%
\definecolor{currentstroke}{rgb}{0.000000,0.605603,0.978680}%
\pgfsetstrokecolor{currentstroke}%
\pgfsetdash{}{0pt}%
\pgfpathmoveto{\pgfqpoint{10.711589in}{7.493818in}}%
\pgfpathlineto{\pgfqpoint{11.211589in}{7.493818in}}%
\pgfusepath{stroke}%
\end{pgfscope}%
\begin{pgfscope}%
\definecolor{textcolor}{rgb}{0.000000,0.000000,0.000000}%
\pgfsetstrokecolor{textcolor}%
\pgfsetfillcolor{textcolor}%
\pgftext[x=11.411589in,y=7.406318in,left,base]{\color{textcolor}\sffamily\fontsize{18.000000}{21.600000}\selectfont $\displaystyle U$}%
\end{pgfscope}%
\begin{pgfscope}%
\pgfsetbuttcap%
\pgfsetmiterjoin%
\pgfsetlinewidth{2.258437pt}%
\definecolor{currentstroke}{rgb}{0.888874,0.435649,0.278123}%
\pgfsetstrokecolor{currentstroke}%
\pgfsetdash{}{0pt}%
\pgfpathmoveto{\pgfqpoint{10.711589in}{7.126875in}}%
\pgfpathlineto{\pgfqpoint{11.211589in}{7.126875in}}%
\pgfusepath{stroke}%
\end{pgfscope}%
\begin{pgfscope}%
\definecolor{textcolor}{rgb}{0.000000,0.000000,0.000000}%
\pgfsetstrokecolor{textcolor}%
\pgfsetfillcolor{textcolor}%
\pgftext[x=11.411589in,y=7.039375in,left,base]{\color{textcolor}\sffamily\fontsize{18.000000}{21.600000}\selectfont $\displaystyle u$}%
\end{pgfscope}%
\end{pgfpicture}%
\makeatother%
\endgroup%
}\quad
	\resizebox{0.4\linewidth}{!}{%% Creator: Matplotlib, PGF backend
%%
%% To include the figure in your LaTeX document, write
%%   \input{<filename>.pgf}
%%
%% Make sure the required packages are loaded in your preamble
%%   \usepackage{pgf}
%%
%% Figures using additional raster images can only be included by \input if
%% they are in the same directory as the main LaTeX file. For loading figures
%% from other directories you can use the `import` package
%%   \usepackage{import}
%%
%% and then include the figures with
%%   \import{<path to file>}{<filename>.pgf}
%%
%% Matplotlib used the following preamble
%%   \usepackage{fontspec}
%%   \setmainfont{DejaVuSerif.ttf}[Path=\detokenize{/Users/quejiahao/.julia/conda/3/lib/python3.9/site-packages/matplotlib/mpl-data/fonts/ttf/}]
%%   \setsansfont{DejaVuSans.ttf}[Path=\detokenize{/Users/quejiahao/.julia/conda/3/lib/python3.9/site-packages/matplotlib/mpl-data/fonts/ttf/}]
%%   \setmonofont{DejaVuSansMono.ttf}[Path=\detokenize{/Users/quejiahao/.julia/conda/3/lib/python3.9/site-packages/matplotlib/mpl-data/fonts/ttf/}]
%%
\begingroup%
\makeatletter%
\begin{pgfpicture}%
\pgfpathrectangle{\pgfpointorigin}{\pgfqpoint{12.000000in}{8.000000in}}%
\pgfusepath{use as bounding box, clip}%
\begin{pgfscope}%
\pgfsetbuttcap%
\pgfsetmiterjoin%
\definecolor{currentfill}{rgb}{1.000000,1.000000,1.000000}%
\pgfsetfillcolor{currentfill}%
\pgfsetlinewidth{0.000000pt}%
\definecolor{currentstroke}{rgb}{1.000000,1.000000,1.000000}%
\pgfsetstrokecolor{currentstroke}%
\pgfsetdash{}{0pt}%
\pgfpathmoveto{\pgfqpoint{0.000000in}{0.000000in}}%
\pgfpathlineto{\pgfqpoint{12.000000in}{0.000000in}}%
\pgfpathlineto{\pgfqpoint{12.000000in}{8.000000in}}%
\pgfpathlineto{\pgfqpoint{0.000000in}{8.000000in}}%
\pgfpathclose%
\pgfusepath{fill}%
\end{pgfscope}%
\begin{pgfscope}%
\pgfsetbuttcap%
\pgfsetmiterjoin%
\definecolor{currentfill}{rgb}{1.000000,1.000000,1.000000}%
\pgfsetfillcolor{currentfill}%
\pgfsetlinewidth{0.000000pt}%
\definecolor{currentstroke}{rgb}{0.000000,0.000000,0.000000}%
\pgfsetstrokecolor{currentstroke}%
\pgfsetstrokeopacity{0.000000}%
\pgfsetdash{}{0pt}%
\pgfpathmoveto{\pgfqpoint{0.978013in}{1.247073in}}%
\pgfpathlineto{\pgfqpoint{11.921260in}{1.247073in}}%
\pgfpathlineto{\pgfqpoint{11.921260in}{7.921260in}}%
\pgfpathlineto{\pgfqpoint{0.978013in}{7.921260in}}%
\pgfpathclose%
\pgfusepath{fill}%
\end{pgfscope}%
\begin{pgfscope}%
\pgfpathrectangle{\pgfqpoint{0.978013in}{1.247073in}}{\pgfqpoint{10.943247in}{6.674186in}}%
\pgfusepath{clip}%
\pgfsetrectcap%
\pgfsetroundjoin%
\pgfsetlinewidth{0.501875pt}%
\definecolor{currentstroke}{rgb}{0.000000,0.000000,0.000000}%
\pgfsetstrokecolor{currentstroke}%
\pgfsetstrokeopacity{0.100000}%
\pgfsetdash{}{0pt}%
\pgfpathmoveto{\pgfqpoint{1.287728in}{1.247073in}}%
\pgfpathlineto{\pgfqpoint{1.287728in}{7.921260in}}%
\pgfusepath{stroke}%
\end{pgfscope}%
\begin{pgfscope}%
\pgfsetbuttcap%
\pgfsetroundjoin%
\definecolor{currentfill}{rgb}{0.000000,0.000000,0.000000}%
\pgfsetfillcolor{currentfill}%
\pgfsetlinewidth{0.501875pt}%
\definecolor{currentstroke}{rgb}{0.000000,0.000000,0.000000}%
\pgfsetstrokecolor{currentstroke}%
\pgfsetdash{}{0pt}%
\pgfsys@defobject{currentmarker}{\pgfqpoint{0.000000in}{0.000000in}}{\pgfqpoint{0.000000in}{0.034722in}}{%
\pgfpathmoveto{\pgfqpoint{0.000000in}{0.000000in}}%
\pgfpathlineto{\pgfqpoint{0.000000in}{0.034722in}}%
\pgfusepath{stroke,fill}%
}%
\begin{pgfscope}%
\pgfsys@transformshift{1.287728in}{1.247073in}%
\pgfsys@useobject{currentmarker}{}%
\end{pgfscope}%
\end{pgfscope}%
\begin{pgfscope}%
\definecolor{textcolor}{rgb}{0.000000,0.000000,0.000000}%
\pgfsetstrokecolor{textcolor}%
\pgfsetfillcolor{textcolor}%
\pgftext[x=1.287728in,y=1.198462in,,top]{\color{textcolor}\sffamily\fontsize{18.000000}{21.600000}\selectfont $\displaystyle 0$}%
\end{pgfscope}%
\begin{pgfscope}%
\pgfpathrectangle{\pgfqpoint{0.978013in}{1.247073in}}{\pgfqpoint{10.943247in}{6.674186in}}%
\pgfusepath{clip}%
\pgfsetrectcap%
\pgfsetroundjoin%
\pgfsetlinewidth{0.501875pt}%
\definecolor{currentstroke}{rgb}{0.000000,0.000000,0.000000}%
\pgfsetstrokecolor{currentstroke}%
\pgfsetstrokeopacity{0.100000}%
\pgfsetdash{}{0pt}%
\pgfpathmoveto{\pgfqpoint{2.930814in}{1.247073in}}%
\pgfpathlineto{\pgfqpoint{2.930814in}{7.921260in}}%
\pgfusepath{stroke}%
\end{pgfscope}%
\begin{pgfscope}%
\pgfsetbuttcap%
\pgfsetroundjoin%
\definecolor{currentfill}{rgb}{0.000000,0.000000,0.000000}%
\pgfsetfillcolor{currentfill}%
\pgfsetlinewidth{0.501875pt}%
\definecolor{currentstroke}{rgb}{0.000000,0.000000,0.000000}%
\pgfsetstrokecolor{currentstroke}%
\pgfsetdash{}{0pt}%
\pgfsys@defobject{currentmarker}{\pgfqpoint{0.000000in}{0.000000in}}{\pgfqpoint{0.000000in}{0.034722in}}{%
\pgfpathmoveto{\pgfqpoint{0.000000in}{0.000000in}}%
\pgfpathlineto{\pgfqpoint{0.000000in}{0.034722in}}%
\pgfusepath{stroke,fill}%
}%
\begin{pgfscope}%
\pgfsys@transformshift{2.930814in}{1.247073in}%
\pgfsys@useobject{currentmarker}{}%
\end{pgfscope}%
\end{pgfscope}%
\begin{pgfscope}%
\definecolor{textcolor}{rgb}{0.000000,0.000000,0.000000}%
\pgfsetstrokecolor{textcolor}%
\pgfsetfillcolor{textcolor}%
\pgftext[x=2.930814in,y=1.198462in,,top]{\color{textcolor}\sffamily\fontsize{18.000000}{21.600000}\selectfont $\displaystyle 1$}%
\end{pgfscope}%
\begin{pgfscope}%
\pgfpathrectangle{\pgfqpoint{0.978013in}{1.247073in}}{\pgfqpoint{10.943247in}{6.674186in}}%
\pgfusepath{clip}%
\pgfsetrectcap%
\pgfsetroundjoin%
\pgfsetlinewidth{0.501875pt}%
\definecolor{currentstroke}{rgb}{0.000000,0.000000,0.000000}%
\pgfsetstrokecolor{currentstroke}%
\pgfsetstrokeopacity{0.100000}%
\pgfsetdash{}{0pt}%
\pgfpathmoveto{\pgfqpoint{4.573901in}{1.247073in}}%
\pgfpathlineto{\pgfqpoint{4.573901in}{7.921260in}}%
\pgfusepath{stroke}%
\end{pgfscope}%
\begin{pgfscope}%
\pgfsetbuttcap%
\pgfsetroundjoin%
\definecolor{currentfill}{rgb}{0.000000,0.000000,0.000000}%
\pgfsetfillcolor{currentfill}%
\pgfsetlinewidth{0.501875pt}%
\definecolor{currentstroke}{rgb}{0.000000,0.000000,0.000000}%
\pgfsetstrokecolor{currentstroke}%
\pgfsetdash{}{0pt}%
\pgfsys@defobject{currentmarker}{\pgfqpoint{0.000000in}{0.000000in}}{\pgfqpoint{0.000000in}{0.034722in}}{%
\pgfpathmoveto{\pgfqpoint{0.000000in}{0.000000in}}%
\pgfpathlineto{\pgfqpoint{0.000000in}{0.034722in}}%
\pgfusepath{stroke,fill}%
}%
\begin{pgfscope}%
\pgfsys@transformshift{4.573901in}{1.247073in}%
\pgfsys@useobject{currentmarker}{}%
\end{pgfscope}%
\end{pgfscope}%
\begin{pgfscope}%
\definecolor{textcolor}{rgb}{0.000000,0.000000,0.000000}%
\pgfsetstrokecolor{textcolor}%
\pgfsetfillcolor{textcolor}%
\pgftext[x=4.573901in,y=1.198462in,,top]{\color{textcolor}\sffamily\fontsize{18.000000}{21.600000}\selectfont $\displaystyle 2$}%
\end{pgfscope}%
\begin{pgfscope}%
\pgfpathrectangle{\pgfqpoint{0.978013in}{1.247073in}}{\pgfqpoint{10.943247in}{6.674186in}}%
\pgfusepath{clip}%
\pgfsetrectcap%
\pgfsetroundjoin%
\pgfsetlinewidth{0.501875pt}%
\definecolor{currentstroke}{rgb}{0.000000,0.000000,0.000000}%
\pgfsetstrokecolor{currentstroke}%
\pgfsetstrokeopacity{0.100000}%
\pgfsetdash{}{0pt}%
\pgfpathmoveto{\pgfqpoint{6.216988in}{1.247073in}}%
\pgfpathlineto{\pgfqpoint{6.216988in}{7.921260in}}%
\pgfusepath{stroke}%
\end{pgfscope}%
\begin{pgfscope}%
\pgfsetbuttcap%
\pgfsetroundjoin%
\definecolor{currentfill}{rgb}{0.000000,0.000000,0.000000}%
\pgfsetfillcolor{currentfill}%
\pgfsetlinewidth{0.501875pt}%
\definecolor{currentstroke}{rgb}{0.000000,0.000000,0.000000}%
\pgfsetstrokecolor{currentstroke}%
\pgfsetdash{}{0pt}%
\pgfsys@defobject{currentmarker}{\pgfqpoint{0.000000in}{0.000000in}}{\pgfqpoint{0.000000in}{0.034722in}}{%
\pgfpathmoveto{\pgfqpoint{0.000000in}{0.000000in}}%
\pgfpathlineto{\pgfqpoint{0.000000in}{0.034722in}}%
\pgfusepath{stroke,fill}%
}%
\begin{pgfscope}%
\pgfsys@transformshift{6.216988in}{1.247073in}%
\pgfsys@useobject{currentmarker}{}%
\end{pgfscope}%
\end{pgfscope}%
\begin{pgfscope}%
\definecolor{textcolor}{rgb}{0.000000,0.000000,0.000000}%
\pgfsetstrokecolor{textcolor}%
\pgfsetfillcolor{textcolor}%
\pgftext[x=6.216988in,y=1.198462in,,top]{\color{textcolor}\sffamily\fontsize{18.000000}{21.600000}\selectfont $\displaystyle 3$}%
\end{pgfscope}%
\begin{pgfscope}%
\pgfpathrectangle{\pgfqpoint{0.978013in}{1.247073in}}{\pgfqpoint{10.943247in}{6.674186in}}%
\pgfusepath{clip}%
\pgfsetrectcap%
\pgfsetroundjoin%
\pgfsetlinewidth{0.501875pt}%
\definecolor{currentstroke}{rgb}{0.000000,0.000000,0.000000}%
\pgfsetstrokecolor{currentstroke}%
\pgfsetstrokeopacity{0.100000}%
\pgfsetdash{}{0pt}%
\pgfpathmoveto{\pgfqpoint{7.860074in}{1.247073in}}%
\pgfpathlineto{\pgfqpoint{7.860074in}{7.921260in}}%
\pgfusepath{stroke}%
\end{pgfscope}%
\begin{pgfscope}%
\pgfsetbuttcap%
\pgfsetroundjoin%
\definecolor{currentfill}{rgb}{0.000000,0.000000,0.000000}%
\pgfsetfillcolor{currentfill}%
\pgfsetlinewidth{0.501875pt}%
\definecolor{currentstroke}{rgb}{0.000000,0.000000,0.000000}%
\pgfsetstrokecolor{currentstroke}%
\pgfsetdash{}{0pt}%
\pgfsys@defobject{currentmarker}{\pgfqpoint{0.000000in}{0.000000in}}{\pgfqpoint{0.000000in}{0.034722in}}{%
\pgfpathmoveto{\pgfqpoint{0.000000in}{0.000000in}}%
\pgfpathlineto{\pgfqpoint{0.000000in}{0.034722in}}%
\pgfusepath{stroke,fill}%
}%
\begin{pgfscope}%
\pgfsys@transformshift{7.860074in}{1.247073in}%
\pgfsys@useobject{currentmarker}{}%
\end{pgfscope}%
\end{pgfscope}%
\begin{pgfscope}%
\definecolor{textcolor}{rgb}{0.000000,0.000000,0.000000}%
\pgfsetstrokecolor{textcolor}%
\pgfsetfillcolor{textcolor}%
\pgftext[x=7.860074in,y=1.198462in,,top]{\color{textcolor}\sffamily\fontsize{18.000000}{21.600000}\selectfont $\displaystyle 4$}%
\end{pgfscope}%
\begin{pgfscope}%
\pgfpathrectangle{\pgfqpoint{0.978013in}{1.247073in}}{\pgfqpoint{10.943247in}{6.674186in}}%
\pgfusepath{clip}%
\pgfsetrectcap%
\pgfsetroundjoin%
\pgfsetlinewidth{0.501875pt}%
\definecolor{currentstroke}{rgb}{0.000000,0.000000,0.000000}%
\pgfsetstrokecolor{currentstroke}%
\pgfsetstrokeopacity{0.100000}%
\pgfsetdash{}{0pt}%
\pgfpathmoveto{\pgfqpoint{9.503161in}{1.247073in}}%
\pgfpathlineto{\pgfqpoint{9.503161in}{7.921260in}}%
\pgfusepath{stroke}%
\end{pgfscope}%
\begin{pgfscope}%
\pgfsetbuttcap%
\pgfsetroundjoin%
\definecolor{currentfill}{rgb}{0.000000,0.000000,0.000000}%
\pgfsetfillcolor{currentfill}%
\pgfsetlinewidth{0.501875pt}%
\definecolor{currentstroke}{rgb}{0.000000,0.000000,0.000000}%
\pgfsetstrokecolor{currentstroke}%
\pgfsetdash{}{0pt}%
\pgfsys@defobject{currentmarker}{\pgfqpoint{0.000000in}{0.000000in}}{\pgfqpoint{0.000000in}{0.034722in}}{%
\pgfpathmoveto{\pgfqpoint{0.000000in}{0.000000in}}%
\pgfpathlineto{\pgfqpoint{0.000000in}{0.034722in}}%
\pgfusepath{stroke,fill}%
}%
\begin{pgfscope}%
\pgfsys@transformshift{9.503161in}{1.247073in}%
\pgfsys@useobject{currentmarker}{}%
\end{pgfscope}%
\end{pgfscope}%
\begin{pgfscope}%
\definecolor{textcolor}{rgb}{0.000000,0.000000,0.000000}%
\pgfsetstrokecolor{textcolor}%
\pgfsetfillcolor{textcolor}%
\pgftext[x=9.503161in,y=1.198462in,,top]{\color{textcolor}\sffamily\fontsize{18.000000}{21.600000}\selectfont $\displaystyle 5$}%
\end{pgfscope}%
\begin{pgfscope}%
\pgfpathrectangle{\pgfqpoint{0.978013in}{1.247073in}}{\pgfqpoint{10.943247in}{6.674186in}}%
\pgfusepath{clip}%
\pgfsetrectcap%
\pgfsetroundjoin%
\pgfsetlinewidth{0.501875pt}%
\definecolor{currentstroke}{rgb}{0.000000,0.000000,0.000000}%
\pgfsetstrokecolor{currentstroke}%
\pgfsetstrokeopacity{0.100000}%
\pgfsetdash{}{0pt}%
\pgfpathmoveto{\pgfqpoint{11.146247in}{1.247073in}}%
\pgfpathlineto{\pgfqpoint{11.146247in}{7.921260in}}%
\pgfusepath{stroke}%
\end{pgfscope}%
\begin{pgfscope}%
\pgfsetbuttcap%
\pgfsetroundjoin%
\definecolor{currentfill}{rgb}{0.000000,0.000000,0.000000}%
\pgfsetfillcolor{currentfill}%
\pgfsetlinewidth{0.501875pt}%
\definecolor{currentstroke}{rgb}{0.000000,0.000000,0.000000}%
\pgfsetstrokecolor{currentstroke}%
\pgfsetdash{}{0pt}%
\pgfsys@defobject{currentmarker}{\pgfqpoint{0.000000in}{0.000000in}}{\pgfqpoint{0.000000in}{0.034722in}}{%
\pgfpathmoveto{\pgfqpoint{0.000000in}{0.000000in}}%
\pgfpathlineto{\pgfqpoint{0.000000in}{0.034722in}}%
\pgfusepath{stroke,fill}%
}%
\begin{pgfscope}%
\pgfsys@transformshift{11.146247in}{1.247073in}%
\pgfsys@useobject{currentmarker}{}%
\end{pgfscope}%
\end{pgfscope}%
\begin{pgfscope}%
\definecolor{textcolor}{rgb}{0.000000,0.000000,0.000000}%
\pgfsetstrokecolor{textcolor}%
\pgfsetfillcolor{textcolor}%
\pgftext[x=11.146247in,y=1.198462in,,top]{\color{textcolor}\sffamily\fontsize{18.000000}{21.600000}\selectfont $\displaystyle 6$}%
\end{pgfscope}%
\begin{pgfscope}%
\definecolor{textcolor}{rgb}{0.000000,0.000000,0.000000}%
\pgfsetstrokecolor{textcolor}%
\pgfsetfillcolor{textcolor}%
\pgftext[x=6.449637in,y=0.900964in,,top]{\color{textcolor}\sffamily\fontsize{18.000000}{21.600000}\selectfont $\displaystyle x$}%
\end{pgfscope}%
\begin{pgfscope}%
\pgfpathrectangle{\pgfqpoint{0.978013in}{1.247073in}}{\pgfqpoint{10.943247in}{6.674186in}}%
\pgfusepath{clip}%
\pgfsetrectcap%
\pgfsetroundjoin%
\pgfsetlinewidth{0.501875pt}%
\definecolor{currentstroke}{rgb}{0.000000,0.000000,0.000000}%
\pgfsetstrokecolor{currentstroke}%
\pgfsetstrokeopacity{0.100000}%
\pgfsetdash{}{0pt}%
\pgfpathmoveto{\pgfqpoint{0.978013in}{1.435966in}}%
\pgfpathlineto{\pgfqpoint{11.921260in}{1.435966in}}%
\pgfusepath{stroke}%
\end{pgfscope}%
\begin{pgfscope}%
\pgfsetbuttcap%
\pgfsetroundjoin%
\definecolor{currentfill}{rgb}{0.000000,0.000000,0.000000}%
\pgfsetfillcolor{currentfill}%
\pgfsetlinewidth{0.501875pt}%
\definecolor{currentstroke}{rgb}{0.000000,0.000000,0.000000}%
\pgfsetstrokecolor{currentstroke}%
\pgfsetdash{}{0pt}%
\pgfsys@defobject{currentmarker}{\pgfqpoint{0.000000in}{0.000000in}}{\pgfqpoint{0.034722in}{0.000000in}}{%
\pgfpathmoveto{\pgfqpoint{0.000000in}{0.000000in}}%
\pgfpathlineto{\pgfqpoint{0.034722in}{0.000000in}}%
\pgfusepath{stroke,fill}%
}%
\begin{pgfscope}%
\pgfsys@transformshift{0.978013in}{1.435966in}%
\pgfsys@useobject{currentmarker}{}%
\end{pgfscope}%
\end{pgfscope}%
\begin{pgfscope}%
\definecolor{textcolor}{rgb}{0.000000,0.000000,0.000000}%
\pgfsetstrokecolor{textcolor}%
\pgfsetfillcolor{textcolor}%
\pgftext[x=0.643989in, y=1.340995in, left, base]{\color{textcolor}\sffamily\fontsize{18.000000}{21.600000}\selectfont $\displaystyle 1.0$}%
\end{pgfscope}%
\begin{pgfscope}%
\pgfpathrectangle{\pgfqpoint{0.978013in}{1.247073in}}{\pgfqpoint{10.943247in}{6.674186in}}%
\pgfusepath{clip}%
\pgfsetrectcap%
\pgfsetroundjoin%
\pgfsetlinewidth{0.501875pt}%
\definecolor{currentstroke}{rgb}{0.000000,0.000000,0.000000}%
\pgfsetstrokecolor{currentstroke}%
\pgfsetstrokeopacity{0.100000}%
\pgfsetdash{}{0pt}%
\pgfpathmoveto{\pgfqpoint{0.978013in}{2.705573in}}%
\pgfpathlineto{\pgfqpoint{11.921260in}{2.705573in}}%
\pgfusepath{stroke}%
\end{pgfscope}%
\begin{pgfscope}%
\pgfsetbuttcap%
\pgfsetroundjoin%
\definecolor{currentfill}{rgb}{0.000000,0.000000,0.000000}%
\pgfsetfillcolor{currentfill}%
\pgfsetlinewidth{0.501875pt}%
\definecolor{currentstroke}{rgb}{0.000000,0.000000,0.000000}%
\pgfsetstrokecolor{currentstroke}%
\pgfsetdash{}{0pt}%
\pgfsys@defobject{currentmarker}{\pgfqpoint{0.000000in}{0.000000in}}{\pgfqpoint{0.034722in}{0.000000in}}{%
\pgfpathmoveto{\pgfqpoint{0.000000in}{0.000000in}}%
\pgfpathlineto{\pgfqpoint{0.034722in}{0.000000in}}%
\pgfusepath{stroke,fill}%
}%
\begin{pgfscope}%
\pgfsys@transformshift{0.978013in}{2.705573in}%
\pgfsys@useobject{currentmarker}{}%
\end{pgfscope}%
\end{pgfscope}%
\begin{pgfscope}%
\definecolor{textcolor}{rgb}{0.000000,0.000000,0.000000}%
\pgfsetstrokecolor{textcolor}%
\pgfsetfillcolor{textcolor}%
\pgftext[x=0.643989in, y=2.610602in, left, base]{\color{textcolor}\sffamily\fontsize{18.000000}{21.600000}\selectfont $\displaystyle 1.5$}%
\end{pgfscope}%
\begin{pgfscope}%
\pgfpathrectangle{\pgfqpoint{0.978013in}{1.247073in}}{\pgfqpoint{10.943247in}{6.674186in}}%
\pgfusepath{clip}%
\pgfsetrectcap%
\pgfsetroundjoin%
\pgfsetlinewidth{0.501875pt}%
\definecolor{currentstroke}{rgb}{0.000000,0.000000,0.000000}%
\pgfsetstrokecolor{currentstroke}%
\pgfsetstrokeopacity{0.100000}%
\pgfsetdash{}{0pt}%
\pgfpathmoveto{\pgfqpoint{0.978013in}{3.975180in}}%
\pgfpathlineto{\pgfqpoint{11.921260in}{3.975180in}}%
\pgfusepath{stroke}%
\end{pgfscope}%
\begin{pgfscope}%
\pgfsetbuttcap%
\pgfsetroundjoin%
\definecolor{currentfill}{rgb}{0.000000,0.000000,0.000000}%
\pgfsetfillcolor{currentfill}%
\pgfsetlinewidth{0.501875pt}%
\definecolor{currentstroke}{rgb}{0.000000,0.000000,0.000000}%
\pgfsetstrokecolor{currentstroke}%
\pgfsetdash{}{0pt}%
\pgfsys@defobject{currentmarker}{\pgfqpoint{0.000000in}{0.000000in}}{\pgfqpoint{0.034722in}{0.000000in}}{%
\pgfpathmoveto{\pgfqpoint{0.000000in}{0.000000in}}%
\pgfpathlineto{\pgfqpoint{0.034722in}{0.000000in}}%
\pgfusepath{stroke,fill}%
}%
\begin{pgfscope}%
\pgfsys@transformshift{0.978013in}{3.975180in}%
\pgfsys@useobject{currentmarker}{}%
\end{pgfscope}%
\end{pgfscope}%
\begin{pgfscope}%
\definecolor{textcolor}{rgb}{0.000000,0.000000,0.000000}%
\pgfsetstrokecolor{textcolor}%
\pgfsetfillcolor{textcolor}%
\pgftext[x=0.643989in, y=3.880210in, left, base]{\color{textcolor}\sffamily\fontsize{18.000000}{21.600000}\selectfont $\displaystyle 2.0$}%
\end{pgfscope}%
\begin{pgfscope}%
\pgfpathrectangle{\pgfqpoint{0.978013in}{1.247073in}}{\pgfqpoint{10.943247in}{6.674186in}}%
\pgfusepath{clip}%
\pgfsetrectcap%
\pgfsetroundjoin%
\pgfsetlinewidth{0.501875pt}%
\definecolor{currentstroke}{rgb}{0.000000,0.000000,0.000000}%
\pgfsetstrokecolor{currentstroke}%
\pgfsetstrokeopacity{0.100000}%
\pgfsetdash{}{0pt}%
\pgfpathmoveto{\pgfqpoint{0.978013in}{5.244788in}}%
\pgfpathlineto{\pgfqpoint{11.921260in}{5.244788in}}%
\pgfusepath{stroke}%
\end{pgfscope}%
\begin{pgfscope}%
\pgfsetbuttcap%
\pgfsetroundjoin%
\definecolor{currentfill}{rgb}{0.000000,0.000000,0.000000}%
\pgfsetfillcolor{currentfill}%
\pgfsetlinewidth{0.501875pt}%
\definecolor{currentstroke}{rgb}{0.000000,0.000000,0.000000}%
\pgfsetstrokecolor{currentstroke}%
\pgfsetdash{}{0pt}%
\pgfsys@defobject{currentmarker}{\pgfqpoint{0.000000in}{0.000000in}}{\pgfqpoint{0.034722in}{0.000000in}}{%
\pgfpathmoveto{\pgfqpoint{0.000000in}{0.000000in}}%
\pgfpathlineto{\pgfqpoint{0.034722in}{0.000000in}}%
\pgfusepath{stroke,fill}%
}%
\begin{pgfscope}%
\pgfsys@transformshift{0.978013in}{5.244788in}%
\pgfsys@useobject{currentmarker}{}%
\end{pgfscope}%
\end{pgfscope}%
\begin{pgfscope}%
\definecolor{textcolor}{rgb}{0.000000,0.000000,0.000000}%
\pgfsetstrokecolor{textcolor}%
\pgfsetfillcolor{textcolor}%
\pgftext[x=0.643989in, y=5.149817in, left, base]{\color{textcolor}\sffamily\fontsize{18.000000}{21.600000}\selectfont $\displaystyle 2.5$}%
\end{pgfscope}%
\begin{pgfscope}%
\pgfpathrectangle{\pgfqpoint{0.978013in}{1.247073in}}{\pgfqpoint{10.943247in}{6.674186in}}%
\pgfusepath{clip}%
\pgfsetrectcap%
\pgfsetroundjoin%
\pgfsetlinewidth{0.501875pt}%
\definecolor{currentstroke}{rgb}{0.000000,0.000000,0.000000}%
\pgfsetstrokecolor{currentstroke}%
\pgfsetstrokeopacity{0.100000}%
\pgfsetdash{}{0pt}%
\pgfpathmoveto{\pgfqpoint{0.978013in}{6.514395in}}%
\pgfpathlineto{\pgfqpoint{11.921260in}{6.514395in}}%
\pgfusepath{stroke}%
\end{pgfscope}%
\begin{pgfscope}%
\pgfsetbuttcap%
\pgfsetroundjoin%
\definecolor{currentfill}{rgb}{0.000000,0.000000,0.000000}%
\pgfsetfillcolor{currentfill}%
\pgfsetlinewidth{0.501875pt}%
\definecolor{currentstroke}{rgb}{0.000000,0.000000,0.000000}%
\pgfsetstrokecolor{currentstroke}%
\pgfsetdash{}{0pt}%
\pgfsys@defobject{currentmarker}{\pgfqpoint{0.000000in}{0.000000in}}{\pgfqpoint{0.034722in}{0.000000in}}{%
\pgfpathmoveto{\pgfqpoint{0.000000in}{0.000000in}}%
\pgfpathlineto{\pgfqpoint{0.034722in}{0.000000in}}%
\pgfusepath{stroke,fill}%
}%
\begin{pgfscope}%
\pgfsys@transformshift{0.978013in}{6.514395in}%
\pgfsys@useobject{currentmarker}{}%
\end{pgfscope}%
\end{pgfscope}%
\begin{pgfscope}%
\definecolor{textcolor}{rgb}{0.000000,0.000000,0.000000}%
\pgfsetstrokecolor{textcolor}%
\pgfsetfillcolor{textcolor}%
\pgftext[x=0.643989in, y=6.419424in, left, base]{\color{textcolor}\sffamily\fontsize{18.000000}{21.600000}\selectfont $\displaystyle 3.0$}%
\end{pgfscope}%
\begin{pgfscope}%
\pgfpathrectangle{\pgfqpoint{0.978013in}{1.247073in}}{\pgfqpoint{10.943247in}{6.674186in}}%
\pgfusepath{clip}%
\pgfsetrectcap%
\pgfsetroundjoin%
\pgfsetlinewidth{0.501875pt}%
\definecolor{currentstroke}{rgb}{0.000000,0.000000,0.000000}%
\pgfsetstrokecolor{currentstroke}%
\pgfsetstrokeopacity{0.100000}%
\pgfsetdash{}{0pt}%
\pgfpathmoveto{\pgfqpoint{0.978013in}{7.784003in}}%
\pgfpathlineto{\pgfqpoint{11.921260in}{7.784003in}}%
\pgfusepath{stroke}%
\end{pgfscope}%
\begin{pgfscope}%
\pgfsetbuttcap%
\pgfsetroundjoin%
\definecolor{currentfill}{rgb}{0.000000,0.000000,0.000000}%
\pgfsetfillcolor{currentfill}%
\pgfsetlinewidth{0.501875pt}%
\definecolor{currentstroke}{rgb}{0.000000,0.000000,0.000000}%
\pgfsetstrokecolor{currentstroke}%
\pgfsetdash{}{0pt}%
\pgfsys@defobject{currentmarker}{\pgfqpoint{0.000000in}{0.000000in}}{\pgfqpoint{0.034722in}{0.000000in}}{%
\pgfpathmoveto{\pgfqpoint{0.000000in}{0.000000in}}%
\pgfpathlineto{\pgfqpoint{0.034722in}{0.000000in}}%
\pgfusepath{stroke,fill}%
}%
\begin{pgfscope}%
\pgfsys@transformshift{0.978013in}{7.784003in}%
\pgfsys@useobject{currentmarker}{}%
\end{pgfscope}%
\end{pgfscope}%
\begin{pgfscope}%
\definecolor{textcolor}{rgb}{0.000000,0.000000,0.000000}%
\pgfsetstrokecolor{textcolor}%
\pgfsetfillcolor{textcolor}%
\pgftext[x=0.643989in, y=7.689032in, left, base]{\color{textcolor}\sffamily\fontsize{18.000000}{21.600000}\selectfont $\displaystyle 3.5$}%
\end{pgfscope}%
\begin{pgfscope}%
\pgfpathrectangle{\pgfqpoint{0.978013in}{1.247073in}}{\pgfqpoint{10.943247in}{6.674186in}}%
\pgfusepath{clip}%
\pgfsetbuttcap%
\pgfsetroundjoin%
\pgfsetlinewidth{1.003750pt}%
\definecolor{currentstroke}{rgb}{0.000000,0.605603,0.978680}%
\pgfsetstrokecolor{currentstroke}%
\pgfsetdash{}{0pt}%
\pgfpathmoveto{\pgfqpoint{1.287728in}{6.514395in}}%
\pgfpathlineto{\pgfqpoint{6.368982in}{6.515421in}}%
\pgfpathlineto{\pgfqpoint{6.379064in}{6.515971in}}%
\pgfpathlineto{\pgfqpoint{6.399227in}{6.511251in}}%
\pgfpathlineto{\pgfqpoint{6.404268in}{6.512393in}}%
\pgfpathlineto{\pgfqpoint{6.419391in}{6.519867in}}%
\pgfpathlineto{\pgfqpoint{6.424432in}{6.519494in}}%
\pgfpathlineto{\pgfqpoint{6.429473in}{6.516404in}}%
\pgfpathlineto{\pgfqpoint{6.439555in}{6.506693in}}%
\pgfpathlineto{\pgfqpoint{6.444596in}{6.504454in}}%
\pgfpathlineto{\pgfqpoint{6.449637in}{6.506449in}}%
\pgfpathlineto{\pgfqpoint{6.454677in}{6.512737in}}%
\pgfpathlineto{\pgfqpoint{6.464759in}{6.528923in}}%
\pgfpathlineto{\pgfqpoint{6.469800in}{6.531698in}}%
\pgfpathlineto{\pgfqpoint{6.474841in}{6.527397in}}%
\pgfpathlineto{\pgfqpoint{6.479882in}{6.516353in}}%
\pgfpathlineto{\pgfqpoint{6.489964in}{6.489419in}}%
\pgfpathlineto{\pgfqpoint{6.495005in}{6.484586in}}%
\pgfpathlineto{\pgfqpoint{6.500046in}{6.490941in}}%
\pgfpathlineto{\pgfqpoint{6.505087in}{6.508176in}}%
\pgfpathlineto{\pgfqpoint{6.515169in}{6.553339in}}%
\pgfpathlineto{\pgfqpoint{6.520209in}{6.564714in}}%
\pgfpathlineto{\pgfqpoint{6.525250in}{6.559531in}}%
\pgfpathlineto{\pgfqpoint{6.530291in}{6.536651in}}%
\pgfpathlineto{\pgfqpoint{6.545414in}{6.435909in}}%
\pgfpathlineto{\pgfqpoint{6.550455in}{6.429987in}}%
\pgfpathlineto{\pgfqpoint{6.555496in}{6.451096in}}%
\pgfpathlineto{\pgfqpoint{6.560537in}{6.496766in}}%
\pgfpathlineto{\pgfqpoint{6.570619in}{6.613490in}}%
\pgfpathlineto{\pgfqpoint{6.575660in}{6.650381in}}%
\pgfpathlineto{\pgfqpoint{6.580701in}{6.653049in}}%
\pgfpathlineto{\pgfqpoint{6.585742in}{6.615748in}}%
\pgfpathlineto{\pgfqpoint{6.590782in}{6.543386in}}%
\pgfpathlineto{\pgfqpoint{6.600864in}{6.361153in}}%
\pgfpathlineto{\pgfqpoint{6.605905in}{6.297401in}}%
\pgfpathlineto{\pgfqpoint{6.610946in}{6.279524in}}%
\pgfpathlineto{\pgfqpoint{6.615987in}{6.317678in}}%
\pgfpathlineto{\pgfqpoint{6.621028in}{6.409411in}}%
\pgfpathlineto{\pgfqpoint{6.636151in}{6.808447in}}%
\pgfpathlineto{\pgfqpoint{6.641192in}{6.888332in}}%
\pgfpathlineto{\pgfqpoint{6.646233in}{6.900852in}}%
\pgfpathlineto{\pgfqpoint{6.651274in}{6.836734in}}%
\pgfpathlineto{\pgfqpoint{6.656314in}{6.700977in}}%
\pgfpathlineto{\pgfqpoint{6.676478in}{5.942682in}}%
\pgfpathlineto{\pgfqpoint{6.681519in}{5.860966in}}%
\pgfpathlineto{\pgfqpoint{6.686560in}{5.871528in}}%
\pgfpathlineto{\pgfqpoint{6.691601in}{5.979579in}}%
\pgfpathlineto{\pgfqpoint{6.696642in}{6.176838in}}%
\pgfpathlineto{\pgfqpoint{6.706724in}{6.748891in}}%
\pgfpathlineto{\pgfqpoint{6.716806in}{7.342369in}}%
\pgfpathlineto{\pgfqpoint{6.721847in}{7.563688in}}%
\pgfpathlineto{\pgfqpoint{6.726887in}{7.699204in}}%
\pgfpathlineto{\pgfqpoint{6.731928in}{7.732368in}}%
\pgfpathlineto{\pgfqpoint{6.736969in}{7.655914in}}%
\pgfpathlineto{\pgfqpoint{6.742010in}{7.471563in}}%
\pgfpathlineto{\pgfqpoint{6.747051in}{7.188812in}}%
\pgfpathlineto{\pgfqpoint{6.757133in}{6.393844in}}%
\pgfpathlineto{\pgfqpoint{6.787379in}{3.578949in}}%
\pgfpathlineto{\pgfqpoint{6.797460in}{2.874030in}}%
\pgfpathlineto{\pgfqpoint{6.807542in}{2.349837in}}%
\pgfpathlineto{\pgfqpoint{6.817624in}{1.987378in}}%
\pgfpathlineto{\pgfqpoint{6.827706in}{1.752600in}}%
\pgfpathlineto{\pgfqpoint{6.832747in}{1.671641in}}%
\pgfpathlineto{\pgfqpoint{6.837788in}{1.609353in}}%
\pgfpathlineto{\pgfqpoint{6.842829in}{1.562080in}}%
\pgfpathlineto{\pgfqpoint{6.847870in}{1.526673in}}%
\pgfpathlineto{\pgfqpoint{6.852911in}{1.500492in}}%
\pgfpathlineto{\pgfqpoint{6.857952in}{1.481373in}}%
\pgfpathlineto{\pgfqpoint{6.862992in}{1.467581in}}%
\pgfpathlineto{\pgfqpoint{6.868033in}{1.457748in}}%
\pgfpathlineto{\pgfqpoint{6.873074in}{1.450819in}}%
\pgfpathlineto{\pgfqpoint{6.883156in}{1.442666in}}%
\pgfpathlineto{\pgfqpoint{6.893238in}{1.438870in}}%
\pgfpathlineto{\pgfqpoint{6.908361in}{1.436737in}}%
\pgfpathlineto{\pgfqpoint{6.938606in}{1.436008in}}%
\pgfpathlineto{\pgfqpoint{7.276348in}{1.435966in}}%
\pgfpathlineto{\pgfqpoint{11.611545in}{1.435966in}}%
\pgfpathlineto{\pgfqpoint{11.611545in}{1.435966in}}%
\pgfusepath{stroke}%
\end{pgfscope}%
\begin{pgfscope}%
\pgfpathrectangle{\pgfqpoint{0.978013in}{1.247073in}}{\pgfqpoint{10.943247in}{6.674186in}}%
\pgfusepath{clip}%
\pgfsetbuttcap%
\pgfsetroundjoin%
\pgfsetlinewidth{1.003750pt}%
\definecolor{currentstroke}{rgb}{0.888874,0.435649,0.278123}%
\pgfsetstrokecolor{currentstroke}%
\pgfsetdash{}{0pt}%
\pgfpathmoveto{\pgfqpoint{1.287728in}{6.514395in}}%
\pgfpathlineto{\pgfqpoint{6.792420in}{6.514395in}}%
\pgfpathlineto{\pgfqpoint{6.797460in}{1.435966in}}%
\pgfpathlineto{\pgfqpoint{11.611545in}{1.435966in}}%
\pgfpathlineto{\pgfqpoint{11.611545in}{1.435966in}}%
\pgfusepath{stroke}%
\end{pgfscope}%
\begin{pgfscope}%
\pgfsetrectcap%
\pgfsetmiterjoin%
\pgfsetlinewidth{1.003750pt}%
\definecolor{currentstroke}{rgb}{0.000000,0.000000,0.000000}%
\pgfsetstrokecolor{currentstroke}%
\pgfsetdash{}{0pt}%
\pgfpathmoveto{\pgfqpoint{0.978013in}{1.247073in}}%
\pgfpathlineto{\pgfqpoint{0.978013in}{7.921260in}}%
\pgfusepath{stroke}%
\end{pgfscope}%
\begin{pgfscope}%
\pgfsetrectcap%
\pgfsetmiterjoin%
\pgfsetlinewidth{1.003750pt}%
\definecolor{currentstroke}{rgb}{0.000000,0.000000,0.000000}%
\pgfsetstrokecolor{currentstroke}%
\pgfsetdash{}{0pt}%
\pgfpathmoveto{\pgfqpoint{0.978013in}{1.247073in}}%
\pgfpathlineto{\pgfqpoint{11.921260in}{1.247073in}}%
\pgfusepath{stroke}%
\end{pgfscope}%
\begin{pgfscope}%
\pgfsetbuttcap%
\pgfsetmiterjoin%
\definecolor{currentfill}{rgb}{1.000000,1.000000,1.000000}%
\pgfsetfillcolor{currentfill}%
\pgfsetlinewidth{1.003750pt}%
\definecolor{currentstroke}{rgb}{0.000000,0.000000,0.000000}%
\pgfsetstrokecolor{currentstroke}%
\pgfsetdash{}{0pt}%
\pgfpathmoveto{\pgfqpoint{10.511589in}{6.787373in}}%
\pgfpathlineto{\pgfqpoint{11.796260in}{6.787373in}}%
\pgfpathlineto{\pgfqpoint{11.796260in}{7.796260in}}%
\pgfpathlineto{\pgfqpoint{10.511589in}{7.796260in}}%
\pgfpathclose%
\pgfusepath{stroke,fill}%
\end{pgfscope}%
\begin{pgfscope}%
\pgfsetbuttcap%
\pgfsetmiterjoin%
\pgfsetlinewidth{2.258437pt}%
\definecolor{currentstroke}{rgb}{0.000000,0.605603,0.978680}%
\pgfsetstrokecolor{currentstroke}%
\pgfsetdash{}{0pt}%
\pgfpathmoveto{\pgfqpoint{10.711589in}{7.493818in}}%
\pgfpathlineto{\pgfqpoint{11.211589in}{7.493818in}}%
\pgfusepath{stroke}%
\end{pgfscope}%
\begin{pgfscope}%
\definecolor{textcolor}{rgb}{0.000000,0.000000,0.000000}%
\pgfsetstrokecolor{textcolor}%
\pgfsetfillcolor{textcolor}%
\pgftext[x=11.411589in,y=7.406318in,left,base]{\color{textcolor}\sffamily\fontsize{18.000000}{21.600000}\selectfont $\displaystyle U$}%
\end{pgfscope}%
\begin{pgfscope}%
\pgfsetbuttcap%
\pgfsetmiterjoin%
\pgfsetlinewidth{2.258437pt}%
\definecolor{currentstroke}{rgb}{0.888874,0.435649,0.278123}%
\pgfsetstrokecolor{currentstroke}%
\pgfsetdash{}{0pt}%
\pgfpathmoveto{\pgfqpoint{10.711589in}{7.126875in}}%
\pgfpathlineto{\pgfqpoint{11.211589in}{7.126875in}}%
\pgfusepath{stroke}%
\end{pgfscope}%
\begin{pgfscope}%
\definecolor{textcolor}{rgb}{0.000000,0.000000,0.000000}%
\pgfsetstrokecolor{textcolor}%
\pgfsetfillcolor{textcolor}%
\pgftext[x=11.411589in,y=7.039375in,left,base]{\color{textcolor}\sffamily\fontsize{18.000000}{21.600000}\selectfont $\displaystyle u$}%
\end{pgfscope}%
\end{pgfpicture}%
\makeatother%
\endgroup%
}
	\caption{Lax-Wendroff 格式差分逼近解 $U$ 与真解 $u$}\label{fig:lax_wendroff_square_Uu}
\end{figure}

取 $\nu = 2$, 结果如图 \ref{fig:lax_wendroff_square_Uu_noCFL} 所示, 出现了错误解.

\begin{figure}[H]\centering\zihao{-5}
	\resizebox{0.4\linewidth}{!}{%% Creator: Matplotlib, PGF backend
%%
%% To include the figure in your LaTeX document, write
%%   \input{<filename>.pgf}
%%
%% Make sure the required packages are loaded in your preamble
%%   \usepackage{pgf}
%%
%% Figures using additional raster images can only be included by \input if
%% they are in the same directory as the main LaTeX file. For loading figures
%% from other directories you can use the `import` package
%%   \usepackage{import}
%%
%% and then include the figures with
%%   \import{<path to file>}{<filename>.pgf}
%%
%% Matplotlib used the following preamble
%%   \usepackage{fontspec}
%%   \setmainfont{DejaVuSerif.ttf}[Path=\detokenize{/Users/quejiahao/.julia/conda/3/lib/python3.9/site-packages/matplotlib/mpl-data/fonts/ttf/}]
%%   \setsansfont{DejaVuSans.ttf}[Path=\detokenize{/Users/quejiahao/.julia/conda/3/lib/python3.9/site-packages/matplotlib/mpl-data/fonts/ttf/}]
%%   \setmonofont{DejaVuSansMono.ttf}[Path=\detokenize{/Users/quejiahao/.julia/conda/3/lib/python3.9/site-packages/matplotlib/mpl-data/fonts/ttf/}]
%%
\begingroup%
\makeatletter%
\begin{pgfpicture}%
\pgfpathrectangle{\pgfpointorigin}{\pgfqpoint{12.000000in}{8.000000in}}%
\pgfusepath{use as bounding box, clip}%
\begin{pgfscope}%
\pgfsetbuttcap%
\pgfsetmiterjoin%
\definecolor{currentfill}{rgb}{1.000000,1.000000,1.000000}%
\pgfsetfillcolor{currentfill}%
\pgfsetlinewidth{0.000000pt}%
\definecolor{currentstroke}{rgb}{1.000000,1.000000,1.000000}%
\pgfsetstrokecolor{currentstroke}%
\pgfsetdash{}{0pt}%
\pgfpathmoveto{\pgfqpoint{0.000000in}{0.000000in}}%
\pgfpathlineto{\pgfqpoint{12.000000in}{0.000000in}}%
\pgfpathlineto{\pgfqpoint{12.000000in}{8.000000in}}%
\pgfpathlineto{\pgfqpoint{0.000000in}{8.000000in}}%
\pgfpathclose%
\pgfusepath{fill}%
\end{pgfscope}%
\begin{pgfscope}%
\pgfsetbuttcap%
\pgfsetmiterjoin%
\definecolor{currentfill}{rgb}{1.000000,1.000000,1.000000}%
\pgfsetfillcolor{currentfill}%
\pgfsetlinewidth{0.000000pt}%
\definecolor{currentstroke}{rgb}{0.000000,0.000000,0.000000}%
\pgfsetstrokecolor{currentstroke}%
\pgfsetstrokeopacity{0.000000}%
\pgfsetdash{}{0pt}%
\pgfpathmoveto{\pgfqpoint{2.910955in}{1.247073in}}%
\pgfpathlineto{\pgfqpoint{11.921260in}{1.247073in}}%
\pgfpathlineto{\pgfqpoint{11.921260in}{7.784989in}}%
\pgfpathlineto{\pgfqpoint{2.910955in}{7.784989in}}%
\pgfpathclose%
\pgfusepath{fill}%
\end{pgfscope}%
\begin{pgfscope}%
\pgfpathrectangle{\pgfqpoint{2.910955in}{1.247073in}}{\pgfqpoint{9.010305in}{6.537916in}}%
\pgfusepath{clip}%
\pgfsetrectcap%
\pgfsetroundjoin%
\pgfsetlinewidth{0.501875pt}%
\definecolor{currentstroke}{rgb}{0.000000,0.000000,0.000000}%
\pgfsetstrokecolor{currentstroke}%
\pgfsetstrokeopacity{0.100000}%
\pgfsetdash{}{0pt}%
\pgfpathmoveto{\pgfqpoint{3.165964in}{1.247073in}}%
\pgfpathlineto{\pgfqpoint{3.165964in}{7.784989in}}%
\pgfusepath{stroke}%
\end{pgfscope}%
\begin{pgfscope}%
\pgfsetbuttcap%
\pgfsetroundjoin%
\definecolor{currentfill}{rgb}{0.000000,0.000000,0.000000}%
\pgfsetfillcolor{currentfill}%
\pgfsetlinewidth{0.501875pt}%
\definecolor{currentstroke}{rgb}{0.000000,0.000000,0.000000}%
\pgfsetstrokecolor{currentstroke}%
\pgfsetdash{}{0pt}%
\pgfsys@defobject{currentmarker}{\pgfqpoint{0.000000in}{0.000000in}}{\pgfqpoint{0.000000in}{0.034722in}}{%
\pgfpathmoveto{\pgfqpoint{0.000000in}{0.000000in}}%
\pgfpathlineto{\pgfqpoint{0.000000in}{0.034722in}}%
\pgfusepath{stroke,fill}%
}%
\begin{pgfscope}%
\pgfsys@transformshift{3.165964in}{1.247073in}%
\pgfsys@useobject{currentmarker}{}%
\end{pgfscope}%
\end{pgfscope}%
\begin{pgfscope}%
\definecolor{textcolor}{rgb}{0.000000,0.000000,0.000000}%
\pgfsetstrokecolor{textcolor}%
\pgfsetfillcolor{textcolor}%
\pgftext[x=3.165964in,y=1.198462in,,top]{\color{textcolor}\sffamily\fontsize{18.000000}{21.600000}\selectfont $\displaystyle 0$}%
\end{pgfscope}%
\begin{pgfscope}%
\pgfpathrectangle{\pgfqpoint{2.910955in}{1.247073in}}{\pgfqpoint{9.010305in}{6.537916in}}%
\pgfusepath{clip}%
\pgfsetrectcap%
\pgfsetroundjoin%
\pgfsetlinewidth{0.501875pt}%
\definecolor{currentstroke}{rgb}{0.000000,0.000000,0.000000}%
\pgfsetstrokecolor{currentstroke}%
\pgfsetstrokeopacity{0.100000}%
\pgfsetdash{}{0pt}%
\pgfpathmoveto{\pgfqpoint{4.518827in}{1.247073in}}%
\pgfpathlineto{\pgfqpoint{4.518827in}{7.784989in}}%
\pgfusepath{stroke}%
\end{pgfscope}%
\begin{pgfscope}%
\pgfsetbuttcap%
\pgfsetroundjoin%
\definecolor{currentfill}{rgb}{0.000000,0.000000,0.000000}%
\pgfsetfillcolor{currentfill}%
\pgfsetlinewidth{0.501875pt}%
\definecolor{currentstroke}{rgb}{0.000000,0.000000,0.000000}%
\pgfsetstrokecolor{currentstroke}%
\pgfsetdash{}{0pt}%
\pgfsys@defobject{currentmarker}{\pgfqpoint{0.000000in}{0.000000in}}{\pgfqpoint{0.000000in}{0.034722in}}{%
\pgfpathmoveto{\pgfqpoint{0.000000in}{0.000000in}}%
\pgfpathlineto{\pgfqpoint{0.000000in}{0.034722in}}%
\pgfusepath{stroke,fill}%
}%
\begin{pgfscope}%
\pgfsys@transformshift{4.518827in}{1.247073in}%
\pgfsys@useobject{currentmarker}{}%
\end{pgfscope}%
\end{pgfscope}%
\begin{pgfscope}%
\definecolor{textcolor}{rgb}{0.000000,0.000000,0.000000}%
\pgfsetstrokecolor{textcolor}%
\pgfsetfillcolor{textcolor}%
\pgftext[x=4.518827in,y=1.198462in,,top]{\color{textcolor}\sffamily\fontsize{18.000000}{21.600000}\selectfont $\displaystyle 1$}%
\end{pgfscope}%
\begin{pgfscope}%
\pgfpathrectangle{\pgfqpoint{2.910955in}{1.247073in}}{\pgfqpoint{9.010305in}{6.537916in}}%
\pgfusepath{clip}%
\pgfsetrectcap%
\pgfsetroundjoin%
\pgfsetlinewidth{0.501875pt}%
\definecolor{currentstroke}{rgb}{0.000000,0.000000,0.000000}%
\pgfsetstrokecolor{currentstroke}%
\pgfsetstrokeopacity{0.100000}%
\pgfsetdash{}{0pt}%
\pgfpathmoveto{\pgfqpoint{5.871689in}{1.247073in}}%
\pgfpathlineto{\pgfqpoint{5.871689in}{7.784989in}}%
\pgfusepath{stroke}%
\end{pgfscope}%
\begin{pgfscope}%
\pgfsetbuttcap%
\pgfsetroundjoin%
\definecolor{currentfill}{rgb}{0.000000,0.000000,0.000000}%
\pgfsetfillcolor{currentfill}%
\pgfsetlinewidth{0.501875pt}%
\definecolor{currentstroke}{rgb}{0.000000,0.000000,0.000000}%
\pgfsetstrokecolor{currentstroke}%
\pgfsetdash{}{0pt}%
\pgfsys@defobject{currentmarker}{\pgfqpoint{0.000000in}{0.000000in}}{\pgfqpoint{0.000000in}{0.034722in}}{%
\pgfpathmoveto{\pgfqpoint{0.000000in}{0.000000in}}%
\pgfpathlineto{\pgfqpoint{0.000000in}{0.034722in}}%
\pgfusepath{stroke,fill}%
}%
\begin{pgfscope}%
\pgfsys@transformshift{5.871689in}{1.247073in}%
\pgfsys@useobject{currentmarker}{}%
\end{pgfscope}%
\end{pgfscope}%
\begin{pgfscope}%
\definecolor{textcolor}{rgb}{0.000000,0.000000,0.000000}%
\pgfsetstrokecolor{textcolor}%
\pgfsetfillcolor{textcolor}%
\pgftext[x=5.871689in,y=1.198462in,,top]{\color{textcolor}\sffamily\fontsize{18.000000}{21.600000}\selectfont $\displaystyle 2$}%
\end{pgfscope}%
\begin{pgfscope}%
\pgfpathrectangle{\pgfqpoint{2.910955in}{1.247073in}}{\pgfqpoint{9.010305in}{6.537916in}}%
\pgfusepath{clip}%
\pgfsetrectcap%
\pgfsetroundjoin%
\pgfsetlinewidth{0.501875pt}%
\definecolor{currentstroke}{rgb}{0.000000,0.000000,0.000000}%
\pgfsetstrokecolor{currentstroke}%
\pgfsetstrokeopacity{0.100000}%
\pgfsetdash{}{0pt}%
\pgfpathmoveto{\pgfqpoint{7.224552in}{1.247073in}}%
\pgfpathlineto{\pgfqpoint{7.224552in}{7.784989in}}%
\pgfusepath{stroke}%
\end{pgfscope}%
\begin{pgfscope}%
\pgfsetbuttcap%
\pgfsetroundjoin%
\definecolor{currentfill}{rgb}{0.000000,0.000000,0.000000}%
\pgfsetfillcolor{currentfill}%
\pgfsetlinewidth{0.501875pt}%
\definecolor{currentstroke}{rgb}{0.000000,0.000000,0.000000}%
\pgfsetstrokecolor{currentstroke}%
\pgfsetdash{}{0pt}%
\pgfsys@defobject{currentmarker}{\pgfqpoint{0.000000in}{0.000000in}}{\pgfqpoint{0.000000in}{0.034722in}}{%
\pgfpathmoveto{\pgfqpoint{0.000000in}{0.000000in}}%
\pgfpathlineto{\pgfqpoint{0.000000in}{0.034722in}}%
\pgfusepath{stroke,fill}%
}%
\begin{pgfscope}%
\pgfsys@transformshift{7.224552in}{1.247073in}%
\pgfsys@useobject{currentmarker}{}%
\end{pgfscope}%
\end{pgfscope}%
\begin{pgfscope}%
\definecolor{textcolor}{rgb}{0.000000,0.000000,0.000000}%
\pgfsetstrokecolor{textcolor}%
\pgfsetfillcolor{textcolor}%
\pgftext[x=7.224552in,y=1.198462in,,top]{\color{textcolor}\sffamily\fontsize{18.000000}{21.600000}\selectfont $\displaystyle 3$}%
\end{pgfscope}%
\begin{pgfscope}%
\pgfpathrectangle{\pgfqpoint{2.910955in}{1.247073in}}{\pgfqpoint{9.010305in}{6.537916in}}%
\pgfusepath{clip}%
\pgfsetrectcap%
\pgfsetroundjoin%
\pgfsetlinewidth{0.501875pt}%
\definecolor{currentstroke}{rgb}{0.000000,0.000000,0.000000}%
\pgfsetstrokecolor{currentstroke}%
\pgfsetstrokeopacity{0.100000}%
\pgfsetdash{}{0pt}%
\pgfpathmoveto{\pgfqpoint{8.577415in}{1.247073in}}%
\pgfpathlineto{\pgfqpoint{8.577415in}{7.784989in}}%
\pgfusepath{stroke}%
\end{pgfscope}%
\begin{pgfscope}%
\pgfsetbuttcap%
\pgfsetroundjoin%
\definecolor{currentfill}{rgb}{0.000000,0.000000,0.000000}%
\pgfsetfillcolor{currentfill}%
\pgfsetlinewidth{0.501875pt}%
\definecolor{currentstroke}{rgb}{0.000000,0.000000,0.000000}%
\pgfsetstrokecolor{currentstroke}%
\pgfsetdash{}{0pt}%
\pgfsys@defobject{currentmarker}{\pgfqpoint{0.000000in}{0.000000in}}{\pgfqpoint{0.000000in}{0.034722in}}{%
\pgfpathmoveto{\pgfqpoint{0.000000in}{0.000000in}}%
\pgfpathlineto{\pgfqpoint{0.000000in}{0.034722in}}%
\pgfusepath{stroke,fill}%
}%
\begin{pgfscope}%
\pgfsys@transformshift{8.577415in}{1.247073in}%
\pgfsys@useobject{currentmarker}{}%
\end{pgfscope}%
\end{pgfscope}%
\begin{pgfscope}%
\definecolor{textcolor}{rgb}{0.000000,0.000000,0.000000}%
\pgfsetstrokecolor{textcolor}%
\pgfsetfillcolor{textcolor}%
\pgftext[x=8.577415in,y=1.198462in,,top]{\color{textcolor}\sffamily\fontsize{18.000000}{21.600000}\selectfont $\displaystyle 4$}%
\end{pgfscope}%
\begin{pgfscope}%
\pgfpathrectangle{\pgfqpoint{2.910955in}{1.247073in}}{\pgfqpoint{9.010305in}{6.537916in}}%
\pgfusepath{clip}%
\pgfsetrectcap%
\pgfsetroundjoin%
\pgfsetlinewidth{0.501875pt}%
\definecolor{currentstroke}{rgb}{0.000000,0.000000,0.000000}%
\pgfsetstrokecolor{currentstroke}%
\pgfsetstrokeopacity{0.100000}%
\pgfsetdash{}{0pt}%
\pgfpathmoveto{\pgfqpoint{9.930278in}{1.247073in}}%
\pgfpathlineto{\pgfqpoint{9.930278in}{7.784989in}}%
\pgfusepath{stroke}%
\end{pgfscope}%
\begin{pgfscope}%
\pgfsetbuttcap%
\pgfsetroundjoin%
\definecolor{currentfill}{rgb}{0.000000,0.000000,0.000000}%
\pgfsetfillcolor{currentfill}%
\pgfsetlinewidth{0.501875pt}%
\definecolor{currentstroke}{rgb}{0.000000,0.000000,0.000000}%
\pgfsetstrokecolor{currentstroke}%
\pgfsetdash{}{0pt}%
\pgfsys@defobject{currentmarker}{\pgfqpoint{0.000000in}{0.000000in}}{\pgfqpoint{0.000000in}{0.034722in}}{%
\pgfpathmoveto{\pgfqpoint{0.000000in}{0.000000in}}%
\pgfpathlineto{\pgfqpoint{0.000000in}{0.034722in}}%
\pgfusepath{stroke,fill}%
}%
\begin{pgfscope}%
\pgfsys@transformshift{9.930278in}{1.247073in}%
\pgfsys@useobject{currentmarker}{}%
\end{pgfscope}%
\end{pgfscope}%
\begin{pgfscope}%
\definecolor{textcolor}{rgb}{0.000000,0.000000,0.000000}%
\pgfsetstrokecolor{textcolor}%
\pgfsetfillcolor{textcolor}%
\pgftext[x=9.930278in,y=1.198462in,,top]{\color{textcolor}\sffamily\fontsize{18.000000}{21.600000}\selectfont $\displaystyle 5$}%
\end{pgfscope}%
\begin{pgfscope}%
\pgfpathrectangle{\pgfqpoint{2.910955in}{1.247073in}}{\pgfqpoint{9.010305in}{6.537916in}}%
\pgfusepath{clip}%
\pgfsetrectcap%
\pgfsetroundjoin%
\pgfsetlinewidth{0.501875pt}%
\definecolor{currentstroke}{rgb}{0.000000,0.000000,0.000000}%
\pgfsetstrokecolor{currentstroke}%
\pgfsetstrokeopacity{0.100000}%
\pgfsetdash{}{0pt}%
\pgfpathmoveto{\pgfqpoint{11.283140in}{1.247073in}}%
\pgfpathlineto{\pgfqpoint{11.283140in}{7.784989in}}%
\pgfusepath{stroke}%
\end{pgfscope}%
\begin{pgfscope}%
\pgfsetbuttcap%
\pgfsetroundjoin%
\definecolor{currentfill}{rgb}{0.000000,0.000000,0.000000}%
\pgfsetfillcolor{currentfill}%
\pgfsetlinewidth{0.501875pt}%
\definecolor{currentstroke}{rgb}{0.000000,0.000000,0.000000}%
\pgfsetstrokecolor{currentstroke}%
\pgfsetdash{}{0pt}%
\pgfsys@defobject{currentmarker}{\pgfqpoint{0.000000in}{0.000000in}}{\pgfqpoint{0.000000in}{0.034722in}}{%
\pgfpathmoveto{\pgfqpoint{0.000000in}{0.000000in}}%
\pgfpathlineto{\pgfqpoint{0.000000in}{0.034722in}}%
\pgfusepath{stroke,fill}%
}%
\begin{pgfscope}%
\pgfsys@transformshift{11.283140in}{1.247073in}%
\pgfsys@useobject{currentmarker}{}%
\end{pgfscope}%
\end{pgfscope}%
\begin{pgfscope}%
\definecolor{textcolor}{rgb}{0.000000,0.000000,0.000000}%
\pgfsetstrokecolor{textcolor}%
\pgfsetfillcolor{textcolor}%
\pgftext[x=11.283140in,y=1.198462in,,top]{\color{textcolor}\sffamily\fontsize{18.000000}{21.600000}\selectfont $\displaystyle 6$}%
\end{pgfscope}%
\begin{pgfscope}%
\definecolor{textcolor}{rgb}{0.000000,0.000000,0.000000}%
\pgfsetstrokecolor{textcolor}%
\pgfsetfillcolor{textcolor}%
\pgftext[x=7.416107in,y=0.900964in,,top]{\color{textcolor}\sffamily\fontsize{18.000000}{21.600000}\selectfont $\displaystyle x$}%
\end{pgfscope}%
\begin{pgfscope}%
\pgfpathrectangle{\pgfqpoint{2.910955in}{1.247073in}}{\pgfqpoint{9.010305in}{6.537916in}}%
\pgfusepath{clip}%
\pgfsetrectcap%
\pgfsetroundjoin%
\pgfsetlinewidth{0.501875pt}%
\definecolor{currentstroke}{rgb}{0.000000,0.000000,0.000000}%
\pgfsetstrokecolor{currentstroke}%
\pgfsetstrokeopacity{0.100000}%
\pgfsetdash{}{0pt}%
\pgfpathmoveto{\pgfqpoint{2.910955in}{1.300428in}}%
\pgfpathlineto{\pgfqpoint{11.921260in}{1.300428in}}%
\pgfusepath{stroke}%
\end{pgfscope}%
\begin{pgfscope}%
\pgfsetbuttcap%
\pgfsetroundjoin%
\definecolor{currentfill}{rgb}{0.000000,0.000000,0.000000}%
\pgfsetfillcolor{currentfill}%
\pgfsetlinewidth{0.501875pt}%
\definecolor{currentstroke}{rgb}{0.000000,0.000000,0.000000}%
\pgfsetstrokecolor{currentstroke}%
\pgfsetdash{}{0pt}%
\pgfsys@defobject{currentmarker}{\pgfqpoint{0.000000in}{0.000000in}}{\pgfqpoint{0.034722in}{0.000000in}}{%
\pgfpathmoveto{\pgfqpoint{0.000000in}{0.000000in}}%
\pgfpathlineto{\pgfqpoint{0.034722in}{0.000000in}}%
\pgfusepath{stroke,fill}%
}%
\begin{pgfscope}%
\pgfsys@transformshift{2.910955in}{1.300428in}%
\pgfsys@useobject{currentmarker}{}%
\end{pgfscope}%
\end{pgfscope}%
\begin{pgfscope}%
\definecolor{textcolor}{rgb}{0.000000,0.000000,0.000000}%
\pgfsetstrokecolor{textcolor}%
\pgfsetfillcolor{textcolor}%
\pgftext[x=1.790517in, y=1.205458in, left, base]{\color{textcolor}\sffamily\fontsize{18.000000}{21.600000}\selectfont $\displaystyle -1.0×10^{10}$}%
\end{pgfscope}%
\begin{pgfscope}%
\pgfpathrectangle{\pgfqpoint{2.910955in}{1.247073in}}{\pgfqpoint{9.010305in}{6.537916in}}%
\pgfusepath{clip}%
\pgfsetrectcap%
\pgfsetroundjoin%
\pgfsetlinewidth{0.501875pt}%
\definecolor{currentstroke}{rgb}{0.000000,0.000000,0.000000}%
\pgfsetstrokecolor{currentstroke}%
\pgfsetstrokeopacity{0.100000}%
\pgfsetdash{}{0pt}%
\pgfpathmoveto{\pgfqpoint{2.910955in}{2.912811in}}%
\pgfpathlineto{\pgfqpoint{11.921260in}{2.912811in}}%
\pgfusepath{stroke}%
\end{pgfscope}%
\begin{pgfscope}%
\pgfsetbuttcap%
\pgfsetroundjoin%
\definecolor{currentfill}{rgb}{0.000000,0.000000,0.000000}%
\pgfsetfillcolor{currentfill}%
\pgfsetlinewidth{0.501875pt}%
\definecolor{currentstroke}{rgb}{0.000000,0.000000,0.000000}%
\pgfsetstrokecolor{currentstroke}%
\pgfsetdash{}{0pt}%
\pgfsys@defobject{currentmarker}{\pgfqpoint{0.000000in}{0.000000in}}{\pgfqpoint{0.034722in}{0.000000in}}{%
\pgfpathmoveto{\pgfqpoint{0.000000in}{0.000000in}}%
\pgfpathlineto{\pgfqpoint{0.034722in}{0.000000in}}%
\pgfusepath{stroke,fill}%
}%
\begin{pgfscope}%
\pgfsys@transformshift{2.910955in}{2.912811in}%
\pgfsys@useobject{currentmarker}{}%
\end{pgfscope}%
\end{pgfscope}%
\begin{pgfscope}%
\definecolor{textcolor}{rgb}{0.000000,0.000000,0.000000}%
\pgfsetstrokecolor{textcolor}%
\pgfsetfillcolor{textcolor}%
\pgftext[x=1.872114in, y=2.817840in, left, base]{\color{textcolor}\sffamily\fontsize{18.000000}{21.600000}\selectfont $\displaystyle -5.0×10^{9}$}%
\end{pgfscope}%
\begin{pgfscope}%
\pgfpathrectangle{\pgfqpoint{2.910955in}{1.247073in}}{\pgfqpoint{9.010305in}{6.537916in}}%
\pgfusepath{clip}%
\pgfsetrectcap%
\pgfsetroundjoin%
\pgfsetlinewidth{0.501875pt}%
\definecolor{currentstroke}{rgb}{0.000000,0.000000,0.000000}%
\pgfsetstrokecolor{currentstroke}%
\pgfsetstrokeopacity{0.100000}%
\pgfsetdash{}{0pt}%
\pgfpathmoveto{\pgfqpoint{2.910955in}{4.525193in}}%
\pgfpathlineto{\pgfqpoint{11.921260in}{4.525193in}}%
\pgfusepath{stroke}%
\end{pgfscope}%
\begin{pgfscope}%
\pgfsetbuttcap%
\pgfsetroundjoin%
\definecolor{currentfill}{rgb}{0.000000,0.000000,0.000000}%
\pgfsetfillcolor{currentfill}%
\pgfsetlinewidth{0.501875pt}%
\definecolor{currentstroke}{rgb}{0.000000,0.000000,0.000000}%
\pgfsetstrokecolor{currentstroke}%
\pgfsetdash{}{0pt}%
\pgfsys@defobject{currentmarker}{\pgfqpoint{0.000000in}{0.000000in}}{\pgfqpoint{0.034722in}{0.000000in}}{%
\pgfpathmoveto{\pgfqpoint{0.000000in}{0.000000in}}%
\pgfpathlineto{\pgfqpoint{0.034722in}{0.000000in}}%
\pgfusepath{stroke,fill}%
}%
\begin{pgfscope}%
\pgfsys@transformshift{2.910955in}{4.525193in}%
\pgfsys@useobject{currentmarker}{}%
\end{pgfscope}%
\end{pgfscope}%
\begin{pgfscope}%
\definecolor{textcolor}{rgb}{0.000000,0.000000,0.000000}%
\pgfsetstrokecolor{textcolor}%
\pgfsetfillcolor{textcolor}%
\pgftext[x=2.752276in, y=4.430223in, left, base]{\color{textcolor}\sffamily\fontsize{18.000000}{21.600000}\selectfont $\displaystyle 0$}%
\end{pgfscope}%
\begin{pgfscope}%
\pgfpathrectangle{\pgfqpoint{2.910955in}{1.247073in}}{\pgfqpoint{9.010305in}{6.537916in}}%
\pgfusepath{clip}%
\pgfsetrectcap%
\pgfsetroundjoin%
\pgfsetlinewidth{0.501875pt}%
\definecolor{currentstroke}{rgb}{0.000000,0.000000,0.000000}%
\pgfsetstrokecolor{currentstroke}%
\pgfsetstrokeopacity{0.100000}%
\pgfsetdash{}{0pt}%
\pgfpathmoveto{\pgfqpoint{2.910955in}{6.137576in}}%
\pgfpathlineto{\pgfqpoint{11.921260in}{6.137576in}}%
\pgfusepath{stroke}%
\end{pgfscope}%
\begin{pgfscope}%
\pgfsetbuttcap%
\pgfsetroundjoin%
\definecolor{currentfill}{rgb}{0.000000,0.000000,0.000000}%
\pgfsetfillcolor{currentfill}%
\pgfsetlinewidth{0.501875pt}%
\definecolor{currentstroke}{rgb}{0.000000,0.000000,0.000000}%
\pgfsetstrokecolor{currentstroke}%
\pgfsetdash{}{0pt}%
\pgfsys@defobject{currentmarker}{\pgfqpoint{0.000000in}{0.000000in}}{\pgfqpoint{0.034722in}{0.000000in}}{%
\pgfpathmoveto{\pgfqpoint{0.000000in}{0.000000in}}%
\pgfpathlineto{\pgfqpoint{0.034722in}{0.000000in}}%
\pgfusepath{stroke,fill}%
}%
\begin{pgfscope}%
\pgfsys@transformshift{2.910955in}{6.137576in}%
\pgfsys@useobject{currentmarker}{}%
\end{pgfscope}%
\end{pgfscope}%
\begin{pgfscope}%
\definecolor{textcolor}{rgb}{0.000000,0.000000,0.000000}%
\pgfsetstrokecolor{textcolor}%
\pgfsetfillcolor{textcolor}%
\pgftext[x=2.058781in, y=6.042605in, left, base]{\color{textcolor}\sffamily\fontsize{18.000000}{21.600000}\selectfont $\displaystyle 5.0×10^{9}$}%
\end{pgfscope}%
\begin{pgfscope}%
\pgfpathrectangle{\pgfqpoint{2.910955in}{1.247073in}}{\pgfqpoint{9.010305in}{6.537916in}}%
\pgfusepath{clip}%
\pgfsetrectcap%
\pgfsetroundjoin%
\pgfsetlinewidth{0.501875pt}%
\definecolor{currentstroke}{rgb}{0.000000,0.000000,0.000000}%
\pgfsetstrokecolor{currentstroke}%
\pgfsetstrokeopacity{0.100000}%
\pgfsetdash{}{0pt}%
\pgfpathmoveto{\pgfqpoint{2.910955in}{7.749958in}}%
\pgfpathlineto{\pgfqpoint{11.921260in}{7.749958in}}%
\pgfusepath{stroke}%
\end{pgfscope}%
\begin{pgfscope}%
\pgfsetbuttcap%
\pgfsetroundjoin%
\definecolor{currentfill}{rgb}{0.000000,0.000000,0.000000}%
\pgfsetfillcolor{currentfill}%
\pgfsetlinewidth{0.501875pt}%
\definecolor{currentstroke}{rgb}{0.000000,0.000000,0.000000}%
\pgfsetstrokecolor{currentstroke}%
\pgfsetdash{}{0pt}%
\pgfsys@defobject{currentmarker}{\pgfqpoint{0.000000in}{0.000000in}}{\pgfqpoint{0.034722in}{0.000000in}}{%
\pgfpathmoveto{\pgfqpoint{0.000000in}{0.000000in}}%
\pgfpathlineto{\pgfqpoint{0.034722in}{0.000000in}}%
\pgfusepath{stroke,fill}%
}%
\begin{pgfscope}%
\pgfsys@transformshift{2.910955in}{7.749958in}%
\pgfsys@useobject{currentmarker}{}%
\end{pgfscope}%
\end{pgfscope}%
\begin{pgfscope}%
\definecolor{textcolor}{rgb}{0.000000,0.000000,0.000000}%
\pgfsetstrokecolor{textcolor}%
\pgfsetfillcolor{textcolor}%
\pgftext[x=1.977184in, y=7.654988in, left, base]{\color{textcolor}\sffamily\fontsize{18.000000}{21.600000}\selectfont $\displaystyle 1.0×10^{10}$}%
\end{pgfscope}%
\begin{pgfscope}%
\pgfpathrectangle{\pgfqpoint{2.910955in}{1.247073in}}{\pgfqpoint{9.010305in}{6.537916in}}%
\pgfusepath{clip}%
\pgfsetbuttcap%
\pgfsetroundjoin%
\pgfsetlinewidth{1.003750pt}%
\definecolor{currentstroke}{rgb}{0.000000,0.605603,0.978680}%
\pgfsetstrokecolor{currentstroke}%
\pgfsetdash{}{0pt}%
\pgfpathmoveto{\pgfqpoint{3.165964in}{4.525193in}}%
\pgfpathlineto{\pgfqpoint{5.556670in}{4.524215in}}%
\pgfpathlineto{\pgfqpoint{5.623078in}{4.529595in}}%
\pgfpathlineto{\pgfqpoint{5.689487in}{4.508564in}}%
\pgfpathlineto{\pgfqpoint{5.755895in}{4.578591in}}%
\pgfpathlineto{\pgfqpoint{5.822304in}{4.378151in}}%
\pgfpathlineto{\pgfqpoint{5.888712in}{4.874545in}}%
\pgfpathlineto{\pgfqpoint{5.955121in}{3.806325in}}%
\pgfpathlineto{\pgfqpoint{6.021529in}{5.808772in}}%
\pgfpathlineto{\pgfqpoint{6.087938in}{2.536290in}}%
\pgfpathlineto{\pgfqpoint{6.154346in}{7.194935in}}%
\pgfpathlineto{\pgfqpoint{6.220755in}{1.432109in}}%
\pgfpathlineto{\pgfqpoint{6.287163in}{7.599954in}}%
\pgfpathlineto{\pgfqpoint{6.353572in}{1.925282in}}%
\pgfpathlineto{\pgfqpoint{6.419980in}{6.372276in}}%
\pgfpathlineto{\pgfqpoint{6.486389in}{3.441692in}}%
\pgfpathlineto{\pgfqpoint{6.552797in}{5.036954in}}%
\pgfpathlineto{\pgfqpoint{6.619206in}{4.337808in}}%
\pgfpathlineto{\pgfqpoint{6.685614in}{4.575203in}}%
\pgfpathlineto{\pgfqpoint{6.752023in}{4.516521in}}%
\pgfpathlineto{\pgfqpoint{6.818431in}{4.525929in}}%
\pgfpathlineto{\pgfqpoint{6.951248in}{4.525193in}}%
\pgfpathlineto{\pgfqpoint{11.666251in}{4.525193in}}%
\pgfpathlineto{\pgfqpoint{11.666251in}{4.525193in}}%
\pgfusepath{stroke}%
\end{pgfscope}%
\begin{pgfscope}%
\pgfpathrectangle{\pgfqpoint{2.910955in}{1.247073in}}{\pgfqpoint{9.010305in}{6.537916in}}%
\pgfusepath{clip}%
\pgfsetbuttcap%
\pgfsetroundjoin%
\pgfsetlinewidth{1.003750pt}%
\definecolor{currentstroke}{rgb}{0.888874,0.435649,0.278123}%
\pgfsetstrokecolor{currentstroke}%
\pgfsetdash{}{0pt}%
\pgfpathmoveto{\pgfqpoint{3.165964in}{4.525193in}}%
\pgfpathlineto{\pgfqpoint{11.666251in}{4.525193in}}%
\pgfpathlineto{\pgfqpoint{11.666251in}{4.525193in}}%
\pgfusepath{stroke}%
\end{pgfscope}%
\begin{pgfscope}%
\pgfsetrectcap%
\pgfsetmiterjoin%
\pgfsetlinewidth{1.003750pt}%
\definecolor{currentstroke}{rgb}{0.000000,0.000000,0.000000}%
\pgfsetstrokecolor{currentstroke}%
\pgfsetdash{}{0pt}%
\pgfpathmoveto{\pgfqpoint{2.910955in}{1.247073in}}%
\pgfpathlineto{\pgfqpoint{2.910955in}{7.784989in}}%
\pgfusepath{stroke}%
\end{pgfscope}%
\begin{pgfscope}%
\pgfsetrectcap%
\pgfsetmiterjoin%
\pgfsetlinewidth{1.003750pt}%
\definecolor{currentstroke}{rgb}{0.000000,0.000000,0.000000}%
\pgfsetstrokecolor{currentstroke}%
\pgfsetdash{}{0pt}%
\pgfpathmoveto{\pgfqpoint{2.910955in}{1.247073in}}%
\pgfpathlineto{\pgfqpoint{11.921260in}{1.247073in}}%
\pgfusepath{stroke}%
\end{pgfscope}%
\begin{pgfscope}%
\pgfsetbuttcap%
\pgfsetmiterjoin%
\definecolor{currentfill}{rgb}{1.000000,1.000000,1.000000}%
\pgfsetfillcolor{currentfill}%
\pgfsetlinewidth{1.003750pt}%
\definecolor{currentstroke}{rgb}{0.000000,0.000000,0.000000}%
\pgfsetstrokecolor{currentstroke}%
\pgfsetdash{}{0pt}%
\pgfpathmoveto{\pgfqpoint{10.511589in}{6.651103in}}%
\pgfpathlineto{\pgfqpoint{11.796260in}{6.651103in}}%
\pgfpathlineto{\pgfqpoint{11.796260in}{7.659989in}}%
\pgfpathlineto{\pgfqpoint{10.511589in}{7.659989in}}%
\pgfpathclose%
\pgfusepath{stroke,fill}%
\end{pgfscope}%
\begin{pgfscope}%
\pgfsetbuttcap%
\pgfsetmiterjoin%
\pgfsetlinewidth{2.258437pt}%
\definecolor{currentstroke}{rgb}{0.000000,0.605603,0.978680}%
\pgfsetstrokecolor{currentstroke}%
\pgfsetdash{}{0pt}%
\pgfpathmoveto{\pgfqpoint{10.711589in}{7.357548in}}%
\pgfpathlineto{\pgfqpoint{11.211589in}{7.357548in}}%
\pgfusepath{stroke}%
\end{pgfscope}%
\begin{pgfscope}%
\definecolor{textcolor}{rgb}{0.000000,0.000000,0.000000}%
\pgfsetstrokecolor{textcolor}%
\pgfsetfillcolor{textcolor}%
\pgftext[x=11.411589in,y=7.270048in,left,base]{\color{textcolor}\sffamily\fontsize{18.000000}{21.600000}\selectfont $\displaystyle U$}%
\end{pgfscope}%
\begin{pgfscope}%
\pgfsetbuttcap%
\pgfsetmiterjoin%
\pgfsetlinewidth{2.258437pt}%
\definecolor{currentstroke}{rgb}{0.888874,0.435649,0.278123}%
\pgfsetstrokecolor{currentstroke}%
\pgfsetdash{}{0pt}%
\pgfpathmoveto{\pgfqpoint{10.711589in}{6.990605in}}%
\pgfpathlineto{\pgfqpoint{11.211589in}{6.990605in}}%
\pgfusepath{stroke}%
\end{pgfscope}%
\begin{pgfscope}%
\definecolor{textcolor}{rgb}{0.000000,0.000000,0.000000}%
\pgfsetstrokecolor{textcolor}%
\pgfsetfillcolor{textcolor}%
\pgftext[x=11.411589in,y=6.903105in,left,base]{\color{textcolor}\sffamily\fontsize{18.000000}{21.600000}\selectfont $\displaystyle u$}%
\end{pgfscope}%
\end{pgfpicture}%
\makeatother%
\endgroup%
}\quad
	\resizebox{0.4\linewidth}{!}{%% Creator: Matplotlib, PGF backend
%%
%% To include the figure in your LaTeX document, write
%%   \input{<filename>.pgf}
%%
%% Make sure the required packages are loaded in your preamble
%%   \usepackage{pgf}
%%
%% Figures using additional raster images can only be included by \input if
%% they are in the same directory as the main LaTeX file. For loading figures
%% from other directories you can use the `import` package
%%   \usepackage{import}
%%
%% and then include the figures with
%%   \import{<path to file>}{<filename>.pgf}
%%
%% Matplotlib used the following preamble
%%   \usepackage{fontspec}
%%   \setmainfont{DejaVuSerif.ttf}[Path=\detokenize{/Users/quejiahao/.julia/conda/3/lib/python3.9/site-packages/matplotlib/mpl-data/fonts/ttf/}]
%%   \setsansfont{DejaVuSans.ttf}[Path=\detokenize{/Users/quejiahao/.julia/conda/3/lib/python3.9/site-packages/matplotlib/mpl-data/fonts/ttf/}]
%%   \setmonofont{DejaVuSansMono.ttf}[Path=\detokenize{/Users/quejiahao/.julia/conda/3/lib/python3.9/site-packages/matplotlib/mpl-data/fonts/ttf/}]
%%
\begingroup%
\makeatletter%
\begin{pgfpicture}%
\pgfpathrectangle{\pgfpointorigin}{\pgfqpoint{12.000000in}{8.000000in}}%
\pgfusepath{use as bounding box, clip}%
\begin{pgfscope}%
\pgfsetbuttcap%
\pgfsetmiterjoin%
\definecolor{currentfill}{rgb}{1.000000,1.000000,1.000000}%
\pgfsetfillcolor{currentfill}%
\pgfsetlinewidth{0.000000pt}%
\definecolor{currentstroke}{rgb}{1.000000,1.000000,1.000000}%
\pgfsetstrokecolor{currentstroke}%
\pgfsetdash{}{0pt}%
\pgfpathmoveto{\pgfqpoint{0.000000in}{0.000000in}}%
\pgfpathlineto{\pgfqpoint{12.000000in}{0.000000in}}%
\pgfpathlineto{\pgfqpoint{12.000000in}{8.000000in}}%
\pgfpathlineto{\pgfqpoint{0.000000in}{8.000000in}}%
\pgfpathclose%
\pgfusepath{fill}%
\end{pgfscope}%
\begin{pgfscope}%
\pgfsetbuttcap%
\pgfsetmiterjoin%
\definecolor{currentfill}{rgb}{1.000000,1.000000,1.000000}%
\pgfsetfillcolor{currentfill}%
\pgfsetlinewidth{0.000000pt}%
\definecolor{currentstroke}{rgb}{0.000000,0.000000,0.000000}%
\pgfsetstrokecolor{currentstroke}%
\pgfsetstrokeopacity{0.000000}%
\pgfsetdash{}{0pt}%
\pgfpathmoveto{\pgfqpoint{3.125511in}{1.247073in}}%
\pgfpathlineto{\pgfqpoint{11.921260in}{1.247073in}}%
\pgfpathlineto{\pgfqpoint{11.921260in}{7.921260in}}%
\pgfpathlineto{\pgfqpoint{3.125511in}{7.921260in}}%
\pgfpathclose%
\pgfusepath{fill}%
\end{pgfscope}%
\begin{pgfscope}%
\pgfpathrectangle{\pgfqpoint{3.125511in}{1.247073in}}{\pgfqpoint{8.795749in}{6.674186in}}%
\pgfusepath{clip}%
\pgfsetrectcap%
\pgfsetroundjoin%
\pgfsetlinewidth{0.501875pt}%
\definecolor{currentstroke}{rgb}{0.000000,0.000000,0.000000}%
\pgfsetstrokecolor{currentstroke}%
\pgfsetstrokeopacity{0.100000}%
\pgfsetdash{}{0pt}%
\pgfpathmoveto{\pgfqpoint{3.374447in}{1.247073in}}%
\pgfpathlineto{\pgfqpoint{3.374447in}{7.921260in}}%
\pgfusepath{stroke}%
\end{pgfscope}%
\begin{pgfscope}%
\pgfsetbuttcap%
\pgfsetroundjoin%
\definecolor{currentfill}{rgb}{0.000000,0.000000,0.000000}%
\pgfsetfillcolor{currentfill}%
\pgfsetlinewidth{0.501875pt}%
\definecolor{currentstroke}{rgb}{0.000000,0.000000,0.000000}%
\pgfsetstrokecolor{currentstroke}%
\pgfsetdash{}{0pt}%
\pgfsys@defobject{currentmarker}{\pgfqpoint{0.000000in}{0.000000in}}{\pgfqpoint{0.000000in}{0.034722in}}{%
\pgfpathmoveto{\pgfqpoint{0.000000in}{0.000000in}}%
\pgfpathlineto{\pgfqpoint{0.000000in}{0.034722in}}%
\pgfusepath{stroke,fill}%
}%
\begin{pgfscope}%
\pgfsys@transformshift{3.374447in}{1.247073in}%
\pgfsys@useobject{currentmarker}{}%
\end{pgfscope}%
\end{pgfscope}%
\begin{pgfscope}%
\definecolor{textcolor}{rgb}{0.000000,0.000000,0.000000}%
\pgfsetstrokecolor{textcolor}%
\pgfsetfillcolor{textcolor}%
\pgftext[x=3.374447in,y=1.198462in,,top]{\color{textcolor}\sffamily\fontsize{18.000000}{21.600000}\selectfont $\displaystyle 0$}%
\end{pgfscope}%
\begin{pgfscope}%
\pgfpathrectangle{\pgfqpoint{3.125511in}{1.247073in}}{\pgfqpoint{8.795749in}{6.674186in}}%
\pgfusepath{clip}%
\pgfsetrectcap%
\pgfsetroundjoin%
\pgfsetlinewidth{0.501875pt}%
\definecolor{currentstroke}{rgb}{0.000000,0.000000,0.000000}%
\pgfsetstrokecolor{currentstroke}%
\pgfsetstrokeopacity{0.100000}%
\pgfsetdash{}{0pt}%
\pgfpathmoveto{\pgfqpoint{4.695095in}{1.247073in}}%
\pgfpathlineto{\pgfqpoint{4.695095in}{7.921260in}}%
\pgfusepath{stroke}%
\end{pgfscope}%
\begin{pgfscope}%
\pgfsetbuttcap%
\pgfsetroundjoin%
\definecolor{currentfill}{rgb}{0.000000,0.000000,0.000000}%
\pgfsetfillcolor{currentfill}%
\pgfsetlinewidth{0.501875pt}%
\definecolor{currentstroke}{rgb}{0.000000,0.000000,0.000000}%
\pgfsetstrokecolor{currentstroke}%
\pgfsetdash{}{0pt}%
\pgfsys@defobject{currentmarker}{\pgfqpoint{0.000000in}{0.000000in}}{\pgfqpoint{0.000000in}{0.034722in}}{%
\pgfpathmoveto{\pgfqpoint{0.000000in}{0.000000in}}%
\pgfpathlineto{\pgfqpoint{0.000000in}{0.034722in}}%
\pgfusepath{stroke,fill}%
}%
\begin{pgfscope}%
\pgfsys@transformshift{4.695095in}{1.247073in}%
\pgfsys@useobject{currentmarker}{}%
\end{pgfscope}%
\end{pgfscope}%
\begin{pgfscope}%
\definecolor{textcolor}{rgb}{0.000000,0.000000,0.000000}%
\pgfsetstrokecolor{textcolor}%
\pgfsetfillcolor{textcolor}%
\pgftext[x=4.695095in,y=1.198462in,,top]{\color{textcolor}\sffamily\fontsize{18.000000}{21.600000}\selectfont $\displaystyle 1$}%
\end{pgfscope}%
\begin{pgfscope}%
\pgfpathrectangle{\pgfqpoint{3.125511in}{1.247073in}}{\pgfqpoint{8.795749in}{6.674186in}}%
\pgfusepath{clip}%
\pgfsetrectcap%
\pgfsetroundjoin%
\pgfsetlinewidth{0.501875pt}%
\definecolor{currentstroke}{rgb}{0.000000,0.000000,0.000000}%
\pgfsetstrokecolor{currentstroke}%
\pgfsetstrokeopacity{0.100000}%
\pgfsetdash{}{0pt}%
\pgfpathmoveto{\pgfqpoint{6.015743in}{1.247073in}}%
\pgfpathlineto{\pgfqpoint{6.015743in}{7.921260in}}%
\pgfusepath{stroke}%
\end{pgfscope}%
\begin{pgfscope}%
\pgfsetbuttcap%
\pgfsetroundjoin%
\definecolor{currentfill}{rgb}{0.000000,0.000000,0.000000}%
\pgfsetfillcolor{currentfill}%
\pgfsetlinewidth{0.501875pt}%
\definecolor{currentstroke}{rgb}{0.000000,0.000000,0.000000}%
\pgfsetstrokecolor{currentstroke}%
\pgfsetdash{}{0pt}%
\pgfsys@defobject{currentmarker}{\pgfqpoint{0.000000in}{0.000000in}}{\pgfqpoint{0.000000in}{0.034722in}}{%
\pgfpathmoveto{\pgfqpoint{0.000000in}{0.000000in}}%
\pgfpathlineto{\pgfqpoint{0.000000in}{0.034722in}}%
\pgfusepath{stroke,fill}%
}%
\begin{pgfscope}%
\pgfsys@transformshift{6.015743in}{1.247073in}%
\pgfsys@useobject{currentmarker}{}%
\end{pgfscope}%
\end{pgfscope}%
\begin{pgfscope}%
\definecolor{textcolor}{rgb}{0.000000,0.000000,0.000000}%
\pgfsetstrokecolor{textcolor}%
\pgfsetfillcolor{textcolor}%
\pgftext[x=6.015743in,y=1.198462in,,top]{\color{textcolor}\sffamily\fontsize{18.000000}{21.600000}\selectfont $\displaystyle 2$}%
\end{pgfscope}%
\begin{pgfscope}%
\pgfpathrectangle{\pgfqpoint{3.125511in}{1.247073in}}{\pgfqpoint{8.795749in}{6.674186in}}%
\pgfusepath{clip}%
\pgfsetrectcap%
\pgfsetroundjoin%
\pgfsetlinewidth{0.501875pt}%
\definecolor{currentstroke}{rgb}{0.000000,0.000000,0.000000}%
\pgfsetstrokecolor{currentstroke}%
\pgfsetstrokeopacity{0.100000}%
\pgfsetdash{}{0pt}%
\pgfpathmoveto{\pgfqpoint{7.336391in}{1.247073in}}%
\pgfpathlineto{\pgfqpoint{7.336391in}{7.921260in}}%
\pgfusepath{stroke}%
\end{pgfscope}%
\begin{pgfscope}%
\pgfsetbuttcap%
\pgfsetroundjoin%
\definecolor{currentfill}{rgb}{0.000000,0.000000,0.000000}%
\pgfsetfillcolor{currentfill}%
\pgfsetlinewidth{0.501875pt}%
\definecolor{currentstroke}{rgb}{0.000000,0.000000,0.000000}%
\pgfsetstrokecolor{currentstroke}%
\pgfsetdash{}{0pt}%
\pgfsys@defobject{currentmarker}{\pgfqpoint{0.000000in}{0.000000in}}{\pgfqpoint{0.000000in}{0.034722in}}{%
\pgfpathmoveto{\pgfqpoint{0.000000in}{0.000000in}}%
\pgfpathlineto{\pgfqpoint{0.000000in}{0.034722in}}%
\pgfusepath{stroke,fill}%
}%
\begin{pgfscope}%
\pgfsys@transformshift{7.336391in}{1.247073in}%
\pgfsys@useobject{currentmarker}{}%
\end{pgfscope}%
\end{pgfscope}%
\begin{pgfscope}%
\definecolor{textcolor}{rgb}{0.000000,0.000000,0.000000}%
\pgfsetstrokecolor{textcolor}%
\pgfsetfillcolor{textcolor}%
\pgftext[x=7.336391in,y=1.198462in,,top]{\color{textcolor}\sffamily\fontsize{18.000000}{21.600000}\selectfont $\displaystyle 3$}%
\end{pgfscope}%
\begin{pgfscope}%
\pgfpathrectangle{\pgfqpoint{3.125511in}{1.247073in}}{\pgfqpoint{8.795749in}{6.674186in}}%
\pgfusepath{clip}%
\pgfsetrectcap%
\pgfsetroundjoin%
\pgfsetlinewidth{0.501875pt}%
\definecolor{currentstroke}{rgb}{0.000000,0.000000,0.000000}%
\pgfsetstrokecolor{currentstroke}%
\pgfsetstrokeopacity{0.100000}%
\pgfsetdash{}{0pt}%
\pgfpathmoveto{\pgfqpoint{8.657039in}{1.247073in}}%
\pgfpathlineto{\pgfqpoint{8.657039in}{7.921260in}}%
\pgfusepath{stroke}%
\end{pgfscope}%
\begin{pgfscope}%
\pgfsetbuttcap%
\pgfsetroundjoin%
\definecolor{currentfill}{rgb}{0.000000,0.000000,0.000000}%
\pgfsetfillcolor{currentfill}%
\pgfsetlinewidth{0.501875pt}%
\definecolor{currentstroke}{rgb}{0.000000,0.000000,0.000000}%
\pgfsetstrokecolor{currentstroke}%
\pgfsetdash{}{0pt}%
\pgfsys@defobject{currentmarker}{\pgfqpoint{0.000000in}{0.000000in}}{\pgfqpoint{0.000000in}{0.034722in}}{%
\pgfpathmoveto{\pgfqpoint{0.000000in}{0.000000in}}%
\pgfpathlineto{\pgfqpoint{0.000000in}{0.034722in}}%
\pgfusepath{stroke,fill}%
}%
\begin{pgfscope}%
\pgfsys@transformshift{8.657039in}{1.247073in}%
\pgfsys@useobject{currentmarker}{}%
\end{pgfscope}%
\end{pgfscope}%
\begin{pgfscope}%
\definecolor{textcolor}{rgb}{0.000000,0.000000,0.000000}%
\pgfsetstrokecolor{textcolor}%
\pgfsetfillcolor{textcolor}%
\pgftext[x=8.657039in,y=1.198462in,,top]{\color{textcolor}\sffamily\fontsize{18.000000}{21.600000}\selectfont $\displaystyle 4$}%
\end{pgfscope}%
\begin{pgfscope}%
\pgfpathrectangle{\pgfqpoint{3.125511in}{1.247073in}}{\pgfqpoint{8.795749in}{6.674186in}}%
\pgfusepath{clip}%
\pgfsetrectcap%
\pgfsetroundjoin%
\pgfsetlinewidth{0.501875pt}%
\definecolor{currentstroke}{rgb}{0.000000,0.000000,0.000000}%
\pgfsetstrokecolor{currentstroke}%
\pgfsetstrokeopacity{0.100000}%
\pgfsetdash{}{0pt}%
\pgfpathmoveto{\pgfqpoint{9.977687in}{1.247073in}}%
\pgfpathlineto{\pgfqpoint{9.977687in}{7.921260in}}%
\pgfusepath{stroke}%
\end{pgfscope}%
\begin{pgfscope}%
\pgfsetbuttcap%
\pgfsetroundjoin%
\definecolor{currentfill}{rgb}{0.000000,0.000000,0.000000}%
\pgfsetfillcolor{currentfill}%
\pgfsetlinewidth{0.501875pt}%
\definecolor{currentstroke}{rgb}{0.000000,0.000000,0.000000}%
\pgfsetstrokecolor{currentstroke}%
\pgfsetdash{}{0pt}%
\pgfsys@defobject{currentmarker}{\pgfqpoint{0.000000in}{0.000000in}}{\pgfqpoint{0.000000in}{0.034722in}}{%
\pgfpathmoveto{\pgfqpoint{0.000000in}{0.000000in}}%
\pgfpathlineto{\pgfqpoint{0.000000in}{0.034722in}}%
\pgfusepath{stroke,fill}%
}%
\begin{pgfscope}%
\pgfsys@transformshift{9.977687in}{1.247073in}%
\pgfsys@useobject{currentmarker}{}%
\end{pgfscope}%
\end{pgfscope}%
\begin{pgfscope}%
\definecolor{textcolor}{rgb}{0.000000,0.000000,0.000000}%
\pgfsetstrokecolor{textcolor}%
\pgfsetfillcolor{textcolor}%
\pgftext[x=9.977687in,y=1.198462in,,top]{\color{textcolor}\sffamily\fontsize{18.000000}{21.600000}\selectfont $\displaystyle 5$}%
\end{pgfscope}%
\begin{pgfscope}%
\pgfpathrectangle{\pgfqpoint{3.125511in}{1.247073in}}{\pgfqpoint{8.795749in}{6.674186in}}%
\pgfusepath{clip}%
\pgfsetrectcap%
\pgfsetroundjoin%
\pgfsetlinewidth{0.501875pt}%
\definecolor{currentstroke}{rgb}{0.000000,0.000000,0.000000}%
\pgfsetstrokecolor{currentstroke}%
\pgfsetstrokeopacity{0.100000}%
\pgfsetdash{}{0pt}%
\pgfpathmoveto{\pgfqpoint{11.298335in}{1.247073in}}%
\pgfpathlineto{\pgfqpoint{11.298335in}{7.921260in}}%
\pgfusepath{stroke}%
\end{pgfscope}%
\begin{pgfscope}%
\pgfsetbuttcap%
\pgfsetroundjoin%
\definecolor{currentfill}{rgb}{0.000000,0.000000,0.000000}%
\pgfsetfillcolor{currentfill}%
\pgfsetlinewidth{0.501875pt}%
\definecolor{currentstroke}{rgb}{0.000000,0.000000,0.000000}%
\pgfsetstrokecolor{currentstroke}%
\pgfsetdash{}{0pt}%
\pgfsys@defobject{currentmarker}{\pgfqpoint{0.000000in}{0.000000in}}{\pgfqpoint{0.000000in}{0.034722in}}{%
\pgfpathmoveto{\pgfqpoint{0.000000in}{0.000000in}}%
\pgfpathlineto{\pgfqpoint{0.000000in}{0.034722in}}%
\pgfusepath{stroke,fill}%
}%
\begin{pgfscope}%
\pgfsys@transformshift{11.298335in}{1.247073in}%
\pgfsys@useobject{currentmarker}{}%
\end{pgfscope}%
\end{pgfscope}%
\begin{pgfscope}%
\definecolor{textcolor}{rgb}{0.000000,0.000000,0.000000}%
\pgfsetstrokecolor{textcolor}%
\pgfsetfillcolor{textcolor}%
\pgftext[x=11.298335in,y=1.198462in,,top]{\color{textcolor}\sffamily\fontsize{18.000000}{21.600000}\selectfont $\displaystyle 6$}%
\end{pgfscope}%
\begin{pgfscope}%
\definecolor{textcolor}{rgb}{0.000000,0.000000,0.000000}%
\pgfsetstrokecolor{textcolor}%
\pgfsetfillcolor{textcolor}%
\pgftext[x=7.523385in,y=0.900964in,,top]{\color{textcolor}\sffamily\fontsize{18.000000}{21.600000}\selectfont $\displaystyle x$}%
\end{pgfscope}%
\begin{pgfscope}%
\pgfpathrectangle{\pgfqpoint{3.125511in}{1.247073in}}{\pgfqpoint{8.795749in}{6.674186in}}%
\pgfusepath{clip}%
\pgfsetrectcap%
\pgfsetroundjoin%
\pgfsetlinewidth{0.501875pt}%
\definecolor{currentstroke}{rgb}{0.000000,0.000000,0.000000}%
\pgfsetstrokecolor{currentstroke}%
\pgfsetstrokeopacity{0.100000}%
\pgfsetdash{}{0pt}%
\pgfpathmoveto{\pgfqpoint{3.125511in}{1.736714in}}%
\pgfpathlineto{\pgfqpoint{11.921260in}{1.736714in}}%
\pgfusepath{stroke}%
\end{pgfscope}%
\begin{pgfscope}%
\pgfsetbuttcap%
\pgfsetroundjoin%
\definecolor{currentfill}{rgb}{0.000000,0.000000,0.000000}%
\pgfsetfillcolor{currentfill}%
\pgfsetlinewidth{0.501875pt}%
\definecolor{currentstroke}{rgb}{0.000000,0.000000,0.000000}%
\pgfsetstrokecolor{currentstroke}%
\pgfsetdash{}{0pt}%
\pgfsys@defobject{currentmarker}{\pgfqpoint{0.000000in}{0.000000in}}{\pgfqpoint{0.034722in}{0.000000in}}{%
\pgfpathmoveto{\pgfqpoint{0.000000in}{0.000000in}}%
\pgfpathlineto{\pgfqpoint{0.034722in}{0.000000in}}%
\pgfusepath{stroke,fill}%
}%
\begin{pgfscope}%
\pgfsys@transformshift{3.125511in}{1.736714in}%
\pgfsys@useobject{currentmarker}{}%
\end{pgfscope}%
\end{pgfscope}%
\begin{pgfscope}%
\definecolor{textcolor}{rgb}{0.000000,0.000000,0.000000}%
\pgfsetstrokecolor{textcolor}%
\pgfsetfillcolor{textcolor}%
\pgftext[x=1.923477in, y=1.641744in, left, base]{\color{textcolor}\sffamily\fontsize{18.000000}{21.600000}\selectfont $\displaystyle -4.0×10^{171}$}%
\end{pgfscope}%
\begin{pgfscope}%
\pgfpathrectangle{\pgfqpoint{3.125511in}{1.247073in}}{\pgfqpoint{8.795749in}{6.674186in}}%
\pgfusepath{clip}%
\pgfsetrectcap%
\pgfsetroundjoin%
\pgfsetlinewidth{0.501875pt}%
\definecolor{currentstroke}{rgb}{0.000000,0.000000,0.000000}%
\pgfsetstrokecolor{currentstroke}%
\pgfsetstrokeopacity{0.100000}%
\pgfsetdash{}{0pt}%
\pgfpathmoveto{\pgfqpoint{3.125511in}{3.158923in}}%
\pgfpathlineto{\pgfqpoint{11.921260in}{3.158923in}}%
\pgfusepath{stroke}%
\end{pgfscope}%
\begin{pgfscope}%
\pgfsetbuttcap%
\pgfsetroundjoin%
\definecolor{currentfill}{rgb}{0.000000,0.000000,0.000000}%
\pgfsetfillcolor{currentfill}%
\pgfsetlinewidth{0.501875pt}%
\definecolor{currentstroke}{rgb}{0.000000,0.000000,0.000000}%
\pgfsetstrokecolor{currentstroke}%
\pgfsetdash{}{0pt}%
\pgfsys@defobject{currentmarker}{\pgfqpoint{0.000000in}{0.000000in}}{\pgfqpoint{0.034722in}{0.000000in}}{%
\pgfpathmoveto{\pgfqpoint{0.000000in}{0.000000in}}%
\pgfpathlineto{\pgfqpoint{0.034722in}{0.000000in}}%
\pgfusepath{stroke,fill}%
}%
\begin{pgfscope}%
\pgfsys@transformshift{3.125511in}{3.158923in}%
\pgfsys@useobject{currentmarker}{}%
\end{pgfscope}%
\end{pgfscope}%
\begin{pgfscope}%
\definecolor{textcolor}{rgb}{0.000000,0.000000,0.000000}%
\pgfsetstrokecolor{textcolor}%
\pgfsetfillcolor{textcolor}%
\pgftext[x=1.923477in, y=3.063952in, left, base]{\color{textcolor}\sffamily\fontsize{18.000000}{21.600000}\selectfont $\displaystyle -2.0×10^{171}$}%
\end{pgfscope}%
\begin{pgfscope}%
\pgfpathrectangle{\pgfqpoint{3.125511in}{1.247073in}}{\pgfqpoint{8.795749in}{6.674186in}}%
\pgfusepath{clip}%
\pgfsetrectcap%
\pgfsetroundjoin%
\pgfsetlinewidth{0.501875pt}%
\definecolor{currentstroke}{rgb}{0.000000,0.000000,0.000000}%
\pgfsetstrokecolor{currentstroke}%
\pgfsetstrokeopacity{0.100000}%
\pgfsetdash{}{0pt}%
\pgfpathmoveto{\pgfqpoint{3.125511in}{4.581132in}}%
\pgfpathlineto{\pgfqpoint{11.921260in}{4.581132in}}%
\pgfusepath{stroke}%
\end{pgfscope}%
\begin{pgfscope}%
\pgfsetbuttcap%
\pgfsetroundjoin%
\definecolor{currentfill}{rgb}{0.000000,0.000000,0.000000}%
\pgfsetfillcolor{currentfill}%
\pgfsetlinewidth{0.501875pt}%
\definecolor{currentstroke}{rgb}{0.000000,0.000000,0.000000}%
\pgfsetstrokecolor{currentstroke}%
\pgfsetdash{}{0pt}%
\pgfsys@defobject{currentmarker}{\pgfqpoint{0.000000in}{0.000000in}}{\pgfqpoint{0.034722in}{0.000000in}}{%
\pgfpathmoveto{\pgfqpoint{0.000000in}{0.000000in}}%
\pgfpathlineto{\pgfqpoint{0.034722in}{0.000000in}}%
\pgfusepath{stroke,fill}%
}%
\begin{pgfscope}%
\pgfsys@transformshift{3.125511in}{4.581132in}%
\pgfsys@useobject{currentmarker}{}%
\end{pgfscope}%
\end{pgfscope}%
\begin{pgfscope}%
\definecolor{textcolor}{rgb}{0.000000,0.000000,0.000000}%
\pgfsetstrokecolor{textcolor}%
\pgfsetfillcolor{textcolor}%
\pgftext[x=2.966832in, y=4.486161in, left, base]{\color{textcolor}\sffamily\fontsize{18.000000}{21.600000}\selectfont $\displaystyle 0$}%
\end{pgfscope}%
\begin{pgfscope}%
\pgfpathrectangle{\pgfqpoint{3.125511in}{1.247073in}}{\pgfqpoint{8.795749in}{6.674186in}}%
\pgfusepath{clip}%
\pgfsetrectcap%
\pgfsetroundjoin%
\pgfsetlinewidth{0.501875pt}%
\definecolor{currentstroke}{rgb}{0.000000,0.000000,0.000000}%
\pgfsetstrokecolor{currentstroke}%
\pgfsetstrokeopacity{0.100000}%
\pgfsetdash{}{0pt}%
\pgfpathmoveto{\pgfqpoint{3.125511in}{6.003340in}}%
\pgfpathlineto{\pgfqpoint{11.921260in}{6.003340in}}%
\pgfusepath{stroke}%
\end{pgfscope}%
\begin{pgfscope}%
\pgfsetbuttcap%
\pgfsetroundjoin%
\definecolor{currentfill}{rgb}{0.000000,0.000000,0.000000}%
\pgfsetfillcolor{currentfill}%
\pgfsetlinewidth{0.501875pt}%
\definecolor{currentstroke}{rgb}{0.000000,0.000000,0.000000}%
\pgfsetstrokecolor{currentstroke}%
\pgfsetdash{}{0pt}%
\pgfsys@defobject{currentmarker}{\pgfqpoint{0.000000in}{0.000000in}}{\pgfqpoint{0.034722in}{0.000000in}}{%
\pgfpathmoveto{\pgfqpoint{0.000000in}{0.000000in}}%
\pgfpathlineto{\pgfqpoint{0.034722in}{0.000000in}}%
\pgfusepath{stroke,fill}%
}%
\begin{pgfscope}%
\pgfsys@transformshift{3.125511in}{6.003340in}%
\pgfsys@useobject{currentmarker}{}%
\end{pgfscope}%
\end{pgfscope}%
\begin{pgfscope}%
\definecolor{textcolor}{rgb}{0.000000,0.000000,0.000000}%
\pgfsetstrokecolor{textcolor}%
\pgfsetfillcolor{textcolor}%
\pgftext[x=2.110144in, y=5.908370in, left, base]{\color{textcolor}\sffamily\fontsize{18.000000}{21.600000}\selectfont $\displaystyle 2.0×10^{171}$}%
\end{pgfscope}%
\begin{pgfscope}%
\pgfpathrectangle{\pgfqpoint{3.125511in}{1.247073in}}{\pgfqpoint{8.795749in}{6.674186in}}%
\pgfusepath{clip}%
\pgfsetrectcap%
\pgfsetroundjoin%
\pgfsetlinewidth{0.501875pt}%
\definecolor{currentstroke}{rgb}{0.000000,0.000000,0.000000}%
\pgfsetstrokecolor{currentstroke}%
\pgfsetstrokeopacity{0.100000}%
\pgfsetdash{}{0pt}%
\pgfpathmoveto{\pgfqpoint{3.125511in}{7.425549in}}%
\pgfpathlineto{\pgfqpoint{11.921260in}{7.425549in}}%
\pgfusepath{stroke}%
\end{pgfscope}%
\begin{pgfscope}%
\pgfsetbuttcap%
\pgfsetroundjoin%
\definecolor{currentfill}{rgb}{0.000000,0.000000,0.000000}%
\pgfsetfillcolor{currentfill}%
\pgfsetlinewidth{0.501875pt}%
\definecolor{currentstroke}{rgb}{0.000000,0.000000,0.000000}%
\pgfsetstrokecolor{currentstroke}%
\pgfsetdash{}{0pt}%
\pgfsys@defobject{currentmarker}{\pgfqpoint{0.000000in}{0.000000in}}{\pgfqpoint{0.034722in}{0.000000in}}{%
\pgfpathmoveto{\pgfqpoint{0.000000in}{0.000000in}}%
\pgfpathlineto{\pgfqpoint{0.034722in}{0.000000in}}%
\pgfusepath{stroke,fill}%
}%
\begin{pgfscope}%
\pgfsys@transformshift{3.125511in}{7.425549in}%
\pgfsys@useobject{currentmarker}{}%
\end{pgfscope}%
\end{pgfscope}%
\begin{pgfscope}%
\definecolor{textcolor}{rgb}{0.000000,0.000000,0.000000}%
\pgfsetstrokecolor{textcolor}%
\pgfsetfillcolor{textcolor}%
\pgftext[x=2.110144in, y=7.330578in, left, base]{\color{textcolor}\sffamily\fontsize{18.000000}{21.600000}\selectfont $\displaystyle 4.0×10^{171}$}%
\end{pgfscope}%
\begin{pgfscope}%
\pgfpathrectangle{\pgfqpoint{3.125511in}{1.247073in}}{\pgfqpoint{8.795749in}{6.674186in}}%
\pgfusepath{clip}%
\pgfsetbuttcap%
\pgfsetroundjoin%
\pgfsetlinewidth{1.003750pt}%
\definecolor{currentstroke}{rgb}{0.000000,0.605603,0.978680}%
\pgfsetstrokecolor{currentstroke}%
\pgfsetdash{}{0pt}%
\pgfpathmoveto{\pgfqpoint{3.374447in}{4.581132in}}%
\pgfpathlineto{\pgfqpoint{6.210635in}{4.580265in}}%
\pgfpathlineto{\pgfqpoint{6.214687in}{4.582407in}}%
\pgfpathlineto{\pgfqpoint{6.218739in}{4.579270in}}%
\pgfpathlineto{\pgfqpoint{6.222790in}{4.583823in}}%
\pgfpathlineto{\pgfqpoint{6.226842in}{4.577276in}}%
\pgfpathlineto{\pgfqpoint{6.230894in}{4.586604in}}%
\pgfpathlineto{\pgfqpoint{6.234946in}{4.573435in}}%
\pgfpathlineto{\pgfqpoint{6.238997in}{4.591857in}}%
\pgfpathlineto{\pgfqpoint{6.243049in}{4.566323in}}%
\pgfpathlineto{\pgfqpoint{6.247101in}{4.601389in}}%
\pgfpathlineto{\pgfqpoint{6.251152in}{4.553674in}}%
\pgfpathlineto{\pgfqpoint{6.255204in}{4.618004in}}%
\pgfpathlineto{\pgfqpoint{6.259256in}{4.532074in}}%
\pgfpathlineto{\pgfqpoint{6.263307in}{4.645798in}}%
\pgfpathlineto{\pgfqpoint{6.267359in}{4.496680in}}%
\pgfpathlineto{\pgfqpoint{6.271411in}{4.690397in}}%
\pgfpathlineto{\pgfqpoint{6.275463in}{4.441076in}}%
\pgfpathlineto{\pgfqpoint{6.279514in}{4.758981in}}%
\pgfpathlineto{\pgfqpoint{6.283566in}{4.357397in}}%
\pgfpathlineto{\pgfqpoint{6.287618in}{4.859959in}}%
\pgfpathlineto{\pgfqpoint{6.291669in}{4.236900in}}%
\pgfpathlineto{\pgfqpoint{6.295721in}{5.002122in}}%
\pgfpathlineto{\pgfqpoint{6.299773in}{4.071108in}}%
\pgfpathlineto{\pgfqpoint{6.303824in}{5.193197in}}%
\pgfpathlineto{\pgfqpoint{6.307876in}{3.853544in}}%
\pgfpathlineto{\pgfqpoint{6.311928in}{5.437863in}}%
\pgfpathlineto{\pgfqpoint{6.315979in}{3.581899in}}%
\pgfpathlineto{\pgfqpoint{6.320031in}{5.735492in}}%
\pgfpathlineto{\pgfqpoint{6.324083in}{3.260262in}}%
\pgfpathlineto{\pgfqpoint{6.328135in}{6.078097in}}%
\pgfpathlineto{\pgfqpoint{6.332186in}{2.900831in}}%
\pgfpathlineto{\pgfqpoint{6.336238in}{6.449125in}}%
\pgfpathlineto{\pgfqpoint{6.340290in}{2.524455in}}%
\pgfpathlineto{\pgfqpoint{6.344341in}{6.823712in}}%
\pgfpathlineto{\pgfqpoint{6.348393in}{2.159489in}}%
\pgfpathlineto{\pgfqpoint{6.352445in}{7.170780in}}%
\pgfpathlineto{\pgfqpoint{6.356496in}{1.838749in}}%
\pgfpathlineto{\pgfqpoint{6.360548in}{7.456941in}}%
\pgfpathlineto{\pgfqpoint{6.364600in}{1.594895in}}%
\pgfpathlineto{\pgfqpoint{6.368652in}{7.651622in}}%
\pgfpathlineto{\pgfqpoint{6.372703in}{1.455068in}}%
\pgfpathlineto{\pgfqpoint{6.376755in}{7.732368in}}%
\pgfpathlineto{\pgfqpoint{6.380807in}{1.435966in}}%
\pgfpathlineto{\pgfqpoint{6.384858in}{7.689065in}}%
\pgfpathlineto{\pgfqpoint{6.388910in}{1.540590in}}%
\pgfpathlineto{\pgfqpoint{6.392962in}{7.526006in}}%
\pgfpathlineto{\pgfqpoint{6.397013in}{1.757524in}}%
\pgfpathlineto{\pgfqpoint{6.401065in}{7.261213in}}%
\pgfpathlineto{\pgfqpoint{6.405117in}{2.062977in}}%
\pgfpathlineto{\pgfqpoint{6.409169in}{6.923156in}}%
\pgfpathlineto{\pgfqpoint{6.413220in}{2.425082in}}%
\pgfpathlineto{\pgfqpoint{6.417272in}{6.545702in}}%
\pgfpathlineto{\pgfqpoint{6.421324in}{2.809389in}}%
\pgfpathlineto{\pgfqpoint{6.425375in}{6.162528in}}%
\pgfpathlineto{\pgfqpoint{6.429427in}{3.184217in}}%
\pgfpathlineto{\pgfqpoint{6.433479in}{5.802284in}}%
\pgfpathlineto{\pgfqpoint{6.437530in}{3.524747in}}%
\pgfpathlineto{\pgfqpoint{6.441582in}{5.485420in}}%
\pgfpathlineto{\pgfqpoint{6.445634in}{3.815173in}}%
\pgfpathlineto{\pgfqpoint{6.449685in}{5.223076in}}%
\pgfpathlineto{\pgfqpoint{6.453737in}{4.048824in}}%
\pgfpathlineto{\pgfqpoint{6.457789in}{5.017828in}}%
\pgfpathlineto{\pgfqpoint{6.461841in}{4.226703in}}%
\pgfpathlineto{\pgfqpoint{6.465892in}{4.865700in}}%
\pgfpathlineto{\pgfqpoint{6.469944in}{4.355120in}}%
\pgfpathlineto{\pgfqpoint{6.473996in}{4.758690in}}%
\pgfpathlineto{\pgfqpoint{6.478047in}{4.443159in}}%
\pgfpathlineto{\pgfqpoint{6.482099in}{4.687169in}}%
\pgfpathlineto{\pgfqpoint{6.486151in}{4.500534in}}%
\pgfpathlineto{\pgfqpoint{6.490202in}{4.641715in}}%
\pgfpathlineto{\pgfqpoint{6.494254in}{4.536099in}}%
\pgfpathlineto{\pgfqpoint{6.498306in}{4.614231in}}%
\pgfpathlineto{\pgfqpoint{6.502358in}{4.557077in}}%
\pgfpathlineto{\pgfqpoint{6.506409in}{4.598416in}}%
\pgfpathlineto{\pgfqpoint{6.510461in}{4.568854in}}%
\pgfpathlineto{\pgfqpoint{6.514513in}{4.589754in}}%
\pgfpathlineto{\pgfqpoint{6.518564in}{4.575146in}}%
\pgfpathlineto{\pgfqpoint{6.522616in}{4.585238in}}%
\pgfpathlineto{\pgfqpoint{6.526668in}{4.578346in}}%
\pgfpathlineto{\pgfqpoint{6.530719in}{4.582998in}}%
\pgfpathlineto{\pgfqpoint{6.534771in}{4.579895in}}%
\pgfpathlineto{\pgfqpoint{6.538823in}{4.581941in}}%
\pgfpathlineto{\pgfqpoint{6.542875in}{4.580608in}}%
\pgfpathlineto{\pgfqpoint{6.546926in}{4.581466in}}%
\pgfpathlineto{\pgfqpoint{6.555030in}{4.581263in}}%
\pgfpathlineto{\pgfqpoint{6.575288in}{4.581121in}}%
\pgfpathlineto{\pgfqpoint{6.850804in}{4.581132in}}%
\pgfpathlineto{\pgfqpoint{11.672324in}{4.581132in}}%
\pgfpathlineto{\pgfqpoint{11.672324in}{4.581132in}}%
\pgfusepath{stroke}%
\end{pgfscope}%
\begin{pgfscope}%
\pgfpathrectangle{\pgfqpoint{3.125511in}{1.247073in}}{\pgfqpoint{8.795749in}{6.674186in}}%
\pgfusepath{clip}%
\pgfsetbuttcap%
\pgfsetroundjoin%
\pgfsetlinewidth{1.003750pt}%
\definecolor{currentstroke}{rgb}{0.888874,0.435649,0.278123}%
\pgfsetstrokecolor{currentstroke}%
\pgfsetdash{}{0pt}%
\pgfpathmoveto{\pgfqpoint{3.374447in}{4.581132in}}%
\pgfpathlineto{\pgfqpoint{11.672324in}{4.581132in}}%
\pgfpathlineto{\pgfqpoint{11.672324in}{4.581132in}}%
\pgfusepath{stroke}%
\end{pgfscope}%
\begin{pgfscope}%
\pgfsetrectcap%
\pgfsetmiterjoin%
\pgfsetlinewidth{1.003750pt}%
\definecolor{currentstroke}{rgb}{0.000000,0.000000,0.000000}%
\pgfsetstrokecolor{currentstroke}%
\pgfsetdash{}{0pt}%
\pgfpathmoveto{\pgfqpoint{3.125511in}{1.247073in}}%
\pgfpathlineto{\pgfqpoint{3.125511in}{7.921260in}}%
\pgfusepath{stroke}%
\end{pgfscope}%
\begin{pgfscope}%
\pgfsetrectcap%
\pgfsetmiterjoin%
\pgfsetlinewidth{1.003750pt}%
\definecolor{currentstroke}{rgb}{0.000000,0.000000,0.000000}%
\pgfsetstrokecolor{currentstroke}%
\pgfsetdash{}{0pt}%
\pgfpathmoveto{\pgfqpoint{3.125511in}{1.247073in}}%
\pgfpathlineto{\pgfqpoint{11.921260in}{1.247073in}}%
\pgfusepath{stroke}%
\end{pgfscope}%
\begin{pgfscope}%
\pgfsetbuttcap%
\pgfsetmiterjoin%
\definecolor{currentfill}{rgb}{1.000000,1.000000,1.000000}%
\pgfsetfillcolor{currentfill}%
\pgfsetlinewidth{1.003750pt}%
\definecolor{currentstroke}{rgb}{0.000000,0.000000,0.000000}%
\pgfsetstrokecolor{currentstroke}%
\pgfsetdash{}{0pt}%
\pgfpathmoveto{\pgfqpoint{10.511589in}{6.787373in}}%
\pgfpathlineto{\pgfqpoint{11.796260in}{6.787373in}}%
\pgfpathlineto{\pgfqpoint{11.796260in}{7.796260in}}%
\pgfpathlineto{\pgfqpoint{10.511589in}{7.796260in}}%
\pgfpathclose%
\pgfusepath{stroke,fill}%
\end{pgfscope}%
\begin{pgfscope}%
\pgfsetbuttcap%
\pgfsetmiterjoin%
\pgfsetlinewidth{2.258437pt}%
\definecolor{currentstroke}{rgb}{0.000000,0.605603,0.978680}%
\pgfsetstrokecolor{currentstroke}%
\pgfsetdash{}{0pt}%
\pgfpathmoveto{\pgfqpoint{10.711589in}{7.493818in}}%
\pgfpathlineto{\pgfqpoint{11.211589in}{7.493818in}}%
\pgfusepath{stroke}%
\end{pgfscope}%
\begin{pgfscope}%
\definecolor{textcolor}{rgb}{0.000000,0.000000,0.000000}%
\pgfsetstrokecolor{textcolor}%
\pgfsetfillcolor{textcolor}%
\pgftext[x=11.411589in,y=7.406318in,left,base]{\color{textcolor}\sffamily\fontsize{18.000000}{21.600000}\selectfont $\displaystyle U$}%
\end{pgfscope}%
\begin{pgfscope}%
\pgfsetbuttcap%
\pgfsetmiterjoin%
\pgfsetlinewidth{2.258437pt}%
\definecolor{currentstroke}{rgb}{0.888874,0.435649,0.278123}%
\pgfsetstrokecolor{currentstroke}%
\pgfsetdash{}{0pt}%
\pgfpathmoveto{\pgfqpoint{10.711589in}{7.126875in}}%
\pgfpathlineto{\pgfqpoint{11.211589in}{7.126875in}}%
\pgfusepath{stroke}%
\end{pgfscope}%
\begin{pgfscope}%
\definecolor{textcolor}{rgb}{0.000000,0.000000,0.000000}%
\pgfsetstrokecolor{textcolor}%
\pgfsetfillcolor{textcolor}%
\pgftext[x=11.411589in,y=7.039375in,left,base]{\color{textcolor}\sffamily\fontsize{18.000000}{21.600000}\selectfont $\displaystyle u$}%
\end{pgfscope}%
\end{pgfpicture}%
\makeatother%
\endgroup%
}
	\caption{Lax-Wendroff 格式差分逼近解 $U$ 与真解 $u$}\label{fig:lax_wendroff_square_Uu_noCFL}
\end{figure}

\subsection{Beam-Warming 格式}

取 $\nu = 0.5 \leq 1$, 满足 CFL 条件. 对正弦波问题 \eqref{equ:sine_wave_con} 求解得到误差及收敛阶如表 \ref{tab:beam_warming_err} 所示.

\begin{table}[H]\centering\heiti\zihao{-5}
	\caption{Beam-Warming 格式不同步长时的 $\mathbb{L}^2$, $\mathbb{L}^\infty$ 误差及收敛阶}\label{tab:beam_warming_err}
	\begin{tabular}{|c|c|c|c|c|}\hline
		收敛阶	&	$\mathbb{L}^2$ 误差	&	$h$	&	$\mathbb{L}^\infty$ 误差		&	收敛阶\\\hline
				&	$4.42759 \times 10^{-3}$	&	$2^{-4}$	&	$6.26065 \times 10^{-3}$	&		   \\\hline
		1.78045	&	$1.28884 \times 10^{-3}$	&	$2^{-5}$	&	$1.78257 \times 10^{-3}$	&	1.81235\\\hline
		1.66205	&	$4.07259 \times 10^{-4}$	&	$2^{-6}$	&	$5.67289 \times 10^{-4}$	&	1.65180\\\hline
		1.83690	&	$1.14001 \times 10^{-4}$	&	$2^{-7}$	&	$1.59987 \times 10^{-4}$	&	1.82613\\\hline
		1.92219	&	$3.00797 \times 10^{-5}$	&	$2^{-8}$	&	$4.23743 \times 10^{-5}$	&	1.91669\\\hline
		1.96211	&	$7.72001 \times 10^{-6}$	&	$2^{-9}$	&	$1.08965 \times 10^{-5}$	&	1.95932\\\hline
		1.98132	&	$1.95516 \times 10^{-6}$	&	$2^{-10}$	&	$2.76232 \times 10^{-6}$	&	1.97991\\\hline
		1.99072	&	$4.91943 \times 10^{-7}$	&	$2^{-11}$	&	$6.95373 \times 10^{-7}$	&	1.99002\\\hline
		1.99538	&	$1.23380 \times 10^{-7}$	&	$2^{-12}$	&	$1.74444 \times 10^{-7}$	&	1.99503\\\hline
	\end{tabular}
\end{table}

由数值结果可以看出解序列逐步收敛到模型问题的解, 收敛阶趋于 1, 与理论结果相符. $h = 2^{-7}$ 和 $h = 2^{-11}$ 时差分逼近解 $U$ 与真解 $u$ 在 $t = t_{\max }$ 时刻图像如图 \ref{fig:beam_warming_Uu} 所示.

\begin{figure}[H]\centering\zihao{-5}
	\resizebox{0.4\linewidth}{!}{%% Creator: Matplotlib, PGF backend
%%
%% To include the figure in your LaTeX document, write
%%   \input{<filename>.pgf}
%%
%% Make sure the required packages are loaded in your preamble
%%   \usepackage{pgf}
%%
%% Figures using additional raster images can only be included by \input if
%% they are in the same directory as the main LaTeX file. For loading figures
%% from other directories you can use the `import` package
%%   \usepackage{import}
%%
%% and then include the figures with
%%   \import{<path to file>}{<filename>.pgf}
%%
%% Matplotlib used the following preamble
%%   \usepackage{fontspec}
%%   \setmainfont{DejaVuSerif.ttf}[Path=\detokenize{/Users/quejiahao/.julia/conda/3/lib/python3.9/site-packages/matplotlib/mpl-data/fonts/ttf/}]
%%   \setsansfont{DejaVuSans.ttf}[Path=\detokenize{/Users/quejiahao/.julia/conda/3/lib/python3.9/site-packages/matplotlib/mpl-data/fonts/ttf/}]
%%   \setmonofont{DejaVuSansMono.ttf}[Path=\detokenize{/Users/quejiahao/.julia/conda/3/lib/python3.9/site-packages/matplotlib/mpl-data/fonts/ttf/}]
%%
\begingroup%
\makeatletter%
\begin{pgfpicture}%
\pgfpathrectangle{\pgfpointorigin}{\pgfqpoint{12.000000in}{8.000000in}}%
\pgfusepath{use as bounding box, clip}%
\begin{pgfscope}%
\pgfsetbuttcap%
\pgfsetmiterjoin%
\definecolor{currentfill}{rgb}{1.000000,1.000000,1.000000}%
\pgfsetfillcolor{currentfill}%
\pgfsetlinewidth{0.000000pt}%
\definecolor{currentstroke}{rgb}{1.000000,1.000000,1.000000}%
\pgfsetstrokecolor{currentstroke}%
\pgfsetdash{}{0pt}%
\pgfpathmoveto{\pgfqpoint{0.000000in}{0.000000in}}%
\pgfpathlineto{\pgfqpoint{12.000000in}{0.000000in}}%
\pgfpathlineto{\pgfqpoint{12.000000in}{8.000000in}}%
\pgfpathlineto{\pgfqpoint{0.000000in}{8.000000in}}%
\pgfpathclose%
\pgfusepath{fill}%
\end{pgfscope}%
\begin{pgfscope}%
\pgfsetbuttcap%
\pgfsetmiterjoin%
\definecolor{currentfill}{rgb}{1.000000,1.000000,1.000000}%
\pgfsetfillcolor{currentfill}%
\pgfsetlinewidth{0.000000pt}%
\definecolor{currentstroke}{rgb}{0.000000,0.000000,0.000000}%
\pgfsetstrokecolor{currentstroke}%
\pgfsetstrokeopacity{0.000000}%
\pgfsetdash{}{0pt}%
\pgfpathmoveto{\pgfqpoint{1.396958in}{1.247073in}}%
\pgfpathlineto{\pgfqpoint{11.921260in}{1.247073in}}%
\pgfpathlineto{\pgfqpoint{11.921260in}{7.921260in}}%
\pgfpathlineto{\pgfqpoint{1.396958in}{7.921260in}}%
\pgfpathclose%
\pgfusepath{fill}%
\end{pgfscope}%
\begin{pgfscope}%
\pgfpathrectangle{\pgfqpoint{1.396958in}{1.247073in}}{\pgfqpoint{10.524301in}{6.674186in}}%
\pgfusepath{clip}%
\pgfsetrectcap%
\pgfsetroundjoin%
\pgfsetlinewidth{0.501875pt}%
\definecolor{currentstroke}{rgb}{0.000000,0.000000,0.000000}%
\pgfsetstrokecolor{currentstroke}%
\pgfsetstrokeopacity{0.100000}%
\pgfsetdash{}{0pt}%
\pgfpathmoveto{\pgfqpoint{1.694816in}{1.247073in}}%
\pgfpathlineto{\pgfqpoint{1.694816in}{7.921260in}}%
\pgfusepath{stroke}%
\end{pgfscope}%
\begin{pgfscope}%
\pgfsetbuttcap%
\pgfsetroundjoin%
\definecolor{currentfill}{rgb}{0.000000,0.000000,0.000000}%
\pgfsetfillcolor{currentfill}%
\pgfsetlinewidth{0.501875pt}%
\definecolor{currentstroke}{rgb}{0.000000,0.000000,0.000000}%
\pgfsetstrokecolor{currentstroke}%
\pgfsetdash{}{0pt}%
\pgfsys@defobject{currentmarker}{\pgfqpoint{0.000000in}{0.000000in}}{\pgfqpoint{0.000000in}{0.034722in}}{%
\pgfpathmoveto{\pgfqpoint{0.000000in}{0.000000in}}%
\pgfpathlineto{\pgfqpoint{0.000000in}{0.034722in}}%
\pgfusepath{stroke,fill}%
}%
\begin{pgfscope}%
\pgfsys@transformshift{1.694816in}{1.247073in}%
\pgfsys@useobject{currentmarker}{}%
\end{pgfscope}%
\end{pgfscope}%
\begin{pgfscope}%
\definecolor{textcolor}{rgb}{0.000000,0.000000,0.000000}%
\pgfsetstrokecolor{textcolor}%
\pgfsetfillcolor{textcolor}%
\pgftext[x=1.694816in,y=1.198462in,,top]{\color{textcolor}\sffamily\fontsize{18.000000}{21.600000}\selectfont $\displaystyle 0$}%
\end{pgfscope}%
\begin{pgfscope}%
\pgfpathrectangle{\pgfqpoint{1.396958in}{1.247073in}}{\pgfqpoint{10.524301in}{6.674186in}}%
\pgfusepath{clip}%
\pgfsetrectcap%
\pgfsetroundjoin%
\pgfsetlinewidth{0.501875pt}%
\definecolor{currentstroke}{rgb}{0.000000,0.000000,0.000000}%
\pgfsetstrokecolor{currentstroke}%
\pgfsetstrokeopacity{0.100000}%
\pgfsetdash{}{0pt}%
\pgfpathmoveto{\pgfqpoint{3.275000in}{1.247073in}}%
\pgfpathlineto{\pgfqpoint{3.275000in}{7.921260in}}%
\pgfusepath{stroke}%
\end{pgfscope}%
\begin{pgfscope}%
\pgfsetbuttcap%
\pgfsetroundjoin%
\definecolor{currentfill}{rgb}{0.000000,0.000000,0.000000}%
\pgfsetfillcolor{currentfill}%
\pgfsetlinewidth{0.501875pt}%
\definecolor{currentstroke}{rgb}{0.000000,0.000000,0.000000}%
\pgfsetstrokecolor{currentstroke}%
\pgfsetdash{}{0pt}%
\pgfsys@defobject{currentmarker}{\pgfqpoint{0.000000in}{0.000000in}}{\pgfqpoint{0.000000in}{0.034722in}}{%
\pgfpathmoveto{\pgfqpoint{0.000000in}{0.000000in}}%
\pgfpathlineto{\pgfqpoint{0.000000in}{0.034722in}}%
\pgfusepath{stroke,fill}%
}%
\begin{pgfscope}%
\pgfsys@transformshift{3.275000in}{1.247073in}%
\pgfsys@useobject{currentmarker}{}%
\end{pgfscope}%
\end{pgfscope}%
\begin{pgfscope}%
\definecolor{textcolor}{rgb}{0.000000,0.000000,0.000000}%
\pgfsetstrokecolor{textcolor}%
\pgfsetfillcolor{textcolor}%
\pgftext[x=3.275000in,y=1.198462in,,top]{\color{textcolor}\sffamily\fontsize{18.000000}{21.600000}\selectfont $\displaystyle 1$}%
\end{pgfscope}%
\begin{pgfscope}%
\pgfpathrectangle{\pgfqpoint{1.396958in}{1.247073in}}{\pgfqpoint{10.524301in}{6.674186in}}%
\pgfusepath{clip}%
\pgfsetrectcap%
\pgfsetroundjoin%
\pgfsetlinewidth{0.501875pt}%
\definecolor{currentstroke}{rgb}{0.000000,0.000000,0.000000}%
\pgfsetstrokecolor{currentstroke}%
\pgfsetstrokeopacity{0.100000}%
\pgfsetdash{}{0pt}%
\pgfpathmoveto{\pgfqpoint{4.855183in}{1.247073in}}%
\pgfpathlineto{\pgfqpoint{4.855183in}{7.921260in}}%
\pgfusepath{stroke}%
\end{pgfscope}%
\begin{pgfscope}%
\pgfsetbuttcap%
\pgfsetroundjoin%
\definecolor{currentfill}{rgb}{0.000000,0.000000,0.000000}%
\pgfsetfillcolor{currentfill}%
\pgfsetlinewidth{0.501875pt}%
\definecolor{currentstroke}{rgb}{0.000000,0.000000,0.000000}%
\pgfsetstrokecolor{currentstroke}%
\pgfsetdash{}{0pt}%
\pgfsys@defobject{currentmarker}{\pgfqpoint{0.000000in}{0.000000in}}{\pgfqpoint{0.000000in}{0.034722in}}{%
\pgfpathmoveto{\pgfqpoint{0.000000in}{0.000000in}}%
\pgfpathlineto{\pgfqpoint{0.000000in}{0.034722in}}%
\pgfusepath{stroke,fill}%
}%
\begin{pgfscope}%
\pgfsys@transformshift{4.855183in}{1.247073in}%
\pgfsys@useobject{currentmarker}{}%
\end{pgfscope}%
\end{pgfscope}%
\begin{pgfscope}%
\definecolor{textcolor}{rgb}{0.000000,0.000000,0.000000}%
\pgfsetstrokecolor{textcolor}%
\pgfsetfillcolor{textcolor}%
\pgftext[x=4.855183in,y=1.198462in,,top]{\color{textcolor}\sffamily\fontsize{18.000000}{21.600000}\selectfont $\displaystyle 2$}%
\end{pgfscope}%
\begin{pgfscope}%
\pgfpathrectangle{\pgfqpoint{1.396958in}{1.247073in}}{\pgfqpoint{10.524301in}{6.674186in}}%
\pgfusepath{clip}%
\pgfsetrectcap%
\pgfsetroundjoin%
\pgfsetlinewidth{0.501875pt}%
\definecolor{currentstroke}{rgb}{0.000000,0.000000,0.000000}%
\pgfsetstrokecolor{currentstroke}%
\pgfsetstrokeopacity{0.100000}%
\pgfsetdash{}{0pt}%
\pgfpathmoveto{\pgfqpoint{6.435367in}{1.247073in}}%
\pgfpathlineto{\pgfqpoint{6.435367in}{7.921260in}}%
\pgfusepath{stroke}%
\end{pgfscope}%
\begin{pgfscope}%
\pgfsetbuttcap%
\pgfsetroundjoin%
\definecolor{currentfill}{rgb}{0.000000,0.000000,0.000000}%
\pgfsetfillcolor{currentfill}%
\pgfsetlinewidth{0.501875pt}%
\definecolor{currentstroke}{rgb}{0.000000,0.000000,0.000000}%
\pgfsetstrokecolor{currentstroke}%
\pgfsetdash{}{0pt}%
\pgfsys@defobject{currentmarker}{\pgfqpoint{0.000000in}{0.000000in}}{\pgfqpoint{0.000000in}{0.034722in}}{%
\pgfpathmoveto{\pgfqpoint{0.000000in}{0.000000in}}%
\pgfpathlineto{\pgfqpoint{0.000000in}{0.034722in}}%
\pgfusepath{stroke,fill}%
}%
\begin{pgfscope}%
\pgfsys@transformshift{6.435367in}{1.247073in}%
\pgfsys@useobject{currentmarker}{}%
\end{pgfscope}%
\end{pgfscope}%
\begin{pgfscope}%
\definecolor{textcolor}{rgb}{0.000000,0.000000,0.000000}%
\pgfsetstrokecolor{textcolor}%
\pgfsetfillcolor{textcolor}%
\pgftext[x=6.435367in,y=1.198462in,,top]{\color{textcolor}\sffamily\fontsize{18.000000}{21.600000}\selectfont $\displaystyle 3$}%
\end{pgfscope}%
\begin{pgfscope}%
\pgfpathrectangle{\pgfqpoint{1.396958in}{1.247073in}}{\pgfqpoint{10.524301in}{6.674186in}}%
\pgfusepath{clip}%
\pgfsetrectcap%
\pgfsetroundjoin%
\pgfsetlinewidth{0.501875pt}%
\definecolor{currentstroke}{rgb}{0.000000,0.000000,0.000000}%
\pgfsetstrokecolor{currentstroke}%
\pgfsetstrokeopacity{0.100000}%
\pgfsetdash{}{0pt}%
\pgfpathmoveto{\pgfqpoint{8.015550in}{1.247073in}}%
\pgfpathlineto{\pgfqpoint{8.015550in}{7.921260in}}%
\pgfusepath{stroke}%
\end{pgfscope}%
\begin{pgfscope}%
\pgfsetbuttcap%
\pgfsetroundjoin%
\definecolor{currentfill}{rgb}{0.000000,0.000000,0.000000}%
\pgfsetfillcolor{currentfill}%
\pgfsetlinewidth{0.501875pt}%
\definecolor{currentstroke}{rgb}{0.000000,0.000000,0.000000}%
\pgfsetstrokecolor{currentstroke}%
\pgfsetdash{}{0pt}%
\pgfsys@defobject{currentmarker}{\pgfqpoint{0.000000in}{0.000000in}}{\pgfqpoint{0.000000in}{0.034722in}}{%
\pgfpathmoveto{\pgfqpoint{0.000000in}{0.000000in}}%
\pgfpathlineto{\pgfqpoint{0.000000in}{0.034722in}}%
\pgfusepath{stroke,fill}%
}%
\begin{pgfscope}%
\pgfsys@transformshift{8.015550in}{1.247073in}%
\pgfsys@useobject{currentmarker}{}%
\end{pgfscope}%
\end{pgfscope}%
\begin{pgfscope}%
\definecolor{textcolor}{rgb}{0.000000,0.000000,0.000000}%
\pgfsetstrokecolor{textcolor}%
\pgfsetfillcolor{textcolor}%
\pgftext[x=8.015550in,y=1.198462in,,top]{\color{textcolor}\sffamily\fontsize{18.000000}{21.600000}\selectfont $\displaystyle 4$}%
\end{pgfscope}%
\begin{pgfscope}%
\pgfpathrectangle{\pgfqpoint{1.396958in}{1.247073in}}{\pgfqpoint{10.524301in}{6.674186in}}%
\pgfusepath{clip}%
\pgfsetrectcap%
\pgfsetroundjoin%
\pgfsetlinewidth{0.501875pt}%
\definecolor{currentstroke}{rgb}{0.000000,0.000000,0.000000}%
\pgfsetstrokecolor{currentstroke}%
\pgfsetstrokeopacity{0.100000}%
\pgfsetdash{}{0pt}%
\pgfpathmoveto{\pgfqpoint{9.595734in}{1.247073in}}%
\pgfpathlineto{\pgfqpoint{9.595734in}{7.921260in}}%
\pgfusepath{stroke}%
\end{pgfscope}%
\begin{pgfscope}%
\pgfsetbuttcap%
\pgfsetroundjoin%
\definecolor{currentfill}{rgb}{0.000000,0.000000,0.000000}%
\pgfsetfillcolor{currentfill}%
\pgfsetlinewidth{0.501875pt}%
\definecolor{currentstroke}{rgb}{0.000000,0.000000,0.000000}%
\pgfsetstrokecolor{currentstroke}%
\pgfsetdash{}{0pt}%
\pgfsys@defobject{currentmarker}{\pgfqpoint{0.000000in}{0.000000in}}{\pgfqpoint{0.000000in}{0.034722in}}{%
\pgfpathmoveto{\pgfqpoint{0.000000in}{0.000000in}}%
\pgfpathlineto{\pgfqpoint{0.000000in}{0.034722in}}%
\pgfusepath{stroke,fill}%
}%
\begin{pgfscope}%
\pgfsys@transformshift{9.595734in}{1.247073in}%
\pgfsys@useobject{currentmarker}{}%
\end{pgfscope}%
\end{pgfscope}%
\begin{pgfscope}%
\definecolor{textcolor}{rgb}{0.000000,0.000000,0.000000}%
\pgfsetstrokecolor{textcolor}%
\pgfsetfillcolor{textcolor}%
\pgftext[x=9.595734in,y=1.198462in,,top]{\color{textcolor}\sffamily\fontsize{18.000000}{21.600000}\selectfont $\displaystyle 5$}%
\end{pgfscope}%
\begin{pgfscope}%
\pgfpathrectangle{\pgfqpoint{1.396958in}{1.247073in}}{\pgfqpoint{10.524301in}{6.674186in}}%
\pgfusepath{clip}%
\pgfsetrectcap%
\pgfsetroundjoin%
\pgfsetlinewidth{0.501875pt}%
\definecolor{currentstroke}{rgb}{0.000000,0.000000,0.000000}%
\pgfsetstrokecolor{currentstroke}%
\pgfsetstrokeopacity{0.100000}%
\pgfsetdash{}{0pt}%
\pgfpathmoveto{\pgfqpoint{11.175917in}{1.247073in}}%
\pgfpathlineto{\pgfqpoint{11.175917in}{7.921260in}}%
\pgfusepath{stroke}%
\end{pgfscope}%
\begin{pgfscope}%
\pgfsetbuttcap%
\pgfsetroundjoin%
\definecolor{currentfill}{rgb}{0.000000,0.000000,0.000000}%
\pgfsetfillcolor{currentfill}%
\pgfsetlinewidth{0.501875pt}%
\definecolor{currentstroke}{rgb}{0.000000,0.000000,0.000000}%
\pgfsetstrokecolor{currentstroke}%
\pgfsetdash{}{0pt}%
\pgfsys@defobject{currentmarker}{\pgfqpoint{0.000000in}{0.000000in}}{\pgfqpoint{0.000000in}{0.034722in}}{%
\pgfpathmoveto{\pgfqpoint{0.000000in}{0.000000in}}%
\pgfpathlineto{\pgfqpoint{0.000000in}{0.034722in}}%
\pgfusepath{stroke,fill}%
}%
\begin{pgfscope}%
\pgfsys@transformshift{11.175917in}{1.247073in}%
\pgfsys@useobject{currentmarker}{}%
\end{pgfscope}%
\end{pgfscope}%
\begin{pgfscope}%
\definecolor{textcolor}{rgb}{0.000000,0.000000,0.000000}%
\pgfsetstrokecolor{textcolor}%
\pgfsetfillcolor{textcolor}%
\pgftext[x=11.175917in,y=1.198462in,,top]{\color{textcolor}\sffamily\fontsize{18.000000}{21.600000}\selectfont $\displaystyle 6$}%
\end{pgfscope}%
\begin{pgfscope}%
\definecolor{textcolor}{rgb}{0.000000,0.000000,0.000000}%
\pgfsetstrokecolor{textcolor}%
\pgfsetfillcolor{textcolor}%
\pgftext[x=6.659109in,y=0.900964in,,top]{\color{textcolor}\sffamily\fontsize{18.000000}{21.600000}\selectfont $\displaystyle x$}%
\end{pgfscope}%
\begin{pgfscope}%
\pgfpathrectangle{\pgfqpoint{1.396958in}{1.247073in}}{\pgfqpoint{10.524301in}{6.674186in}}%
\pgfusepath{clip}%
\pgfsetrectcap%
\pgfsetroundjoin%
\pgfsetlinewidth{0.501875pt}%
\definecolor{currentstroke}{rgb}{0.000000,0.000000,0.000000}%
\pgfsetstrokecolor{currentstroke}%
\pgfsetstrokeopacity{0.100000}%
\pgfsetdash{}{0pt}%
\pgfpathmoveto{\pgfqpoint{1.396958in}{1.435966in}}%
\pgfpathlineto{\pgfqpoint{11.921260in}{1.435966in}}%
\pgfusepath{stroke}%
\end{pgfscope}%
\begin{pgfscope}%
\pgfsetbuttcap%
\pgfsetroundjoin%
\definecolor{currentfill}{rgb}{0.000000,0.000000,0.000000}%
\pgfsetfillcolor{currentfill}%
\pgfsetlinewidth{0.501875pt}%
\definecolor{currentstroke}{rgb}{0.000000,0.000000,0.000000}%
\pgfsetstrokecolor{currentstroke}%
\pgfsetdash{}{0pt}%
\pgfsys@defobject{currentmarker}{\pgfqpoint{0.000000in}{0.000000in}}{\pgfqpoint{0.034722in}{0.000000in}}{%
\pgfpathmoveto{\pgfqpoint{0.000000in}{0.000000in}}%
\pgfpathlineto{\pgfqpoint{0.034722in}{0.000000in}}%
\pgfusepath{stroke,fill}%
}%
\begin{pgfscope}%
\pgfsys@transformshift{1.396958in}{1.435966in}%
\pgfsys@useobject{currentmarker}{}%
\end{pgfscope}%
\end{pgfscope}%
\begin{pgfscope}%
\definecolor{textcolor}{rgb}{0.000000,0.000000,0.000000}%
\pgfsetstrokecolor{textcolor}%
\pgfsetfillcolor{textcolor}%
\pgftext[x=0.876267in, y=1.340995in, left, base]{\color{textcolor}\sffamily\fontsize{18.000000}{21.600000}\selectfont $\displaystyle -1.0$}%
\end{pgfscope}%
\begin{pgfscope}%
\pgfpathrectangle{\pgfqpoint{1.396958in}{1.247073in}}{\pgfqpoint{10.524301in}{6.674186in}}%
\pgfusepath{clip}%
\pgfsetrectcap%
\pgfsetroundjoin%
\pgfsetlinewidth{0.501875pt}%
\definecolor{currentstroke}{rgb}{0.000000,0.000000,0.000000}%
\pgfsetstrokecolor{currentstroke}%
\pgfsetstrokeopacity{0.100000}%
\pgfsetdash{}{0pt}%
\pgfpathmoveto{\pgfqpoint{1.396958in}{3.010066in}}%
\pgfpathlineto{\pgfqpoint{11.921260in}{3.010066in}}%
\pgfusepath{stroke}%
\end{pgfscope}%
\begin{pgfscope}%
\pgfsetbuttcap%
\pgfsetroundjoin%
\definecolor{currentfill}{rgb}{0.000000,0.000000,0.000000}%
\pgfsetfillcolor{currentfill}%
\pgfsetlinewidth{0.501875pt}%
\definecolor{currentstroke}{rgb}{0.000000,0.000000,0.000000}%
\pgfsetstrokecolor{currentstroke}%
\pgfsetdash{}{0pt}%
\pgfsys@defobject{currentmarker}{\pgfqpoint{0.000000in}{0.000000in}}{\pgfqpoint{0.034722in}{0.000000in}}{%
\pgfpathmoveto{\pgfqpoint{0.000000in}{0.000000in}}%
\pgfpathlineto{\pgfqpoint{0.034722in}{0.000000in}}%
\pgfusepath{stroke,fill}%
}%
\begin{pgfscope}%
\pgfsys@transformshift{1.396958in}{3.010066in}%
\pgfsys@useobject{currentmarker}{}%
\end{pgfscope}%
\end{pgfscope}%
\begin{pgfscope}%
\definecolor{textcolor}{rgb}{0.000000,0.000000,0.000000}%
\pgfsetstrokecolor{textcolor}%
\pgfsetfillcolor{textcolor}%
\pgftext[x=0.876267in, y=2.915095in, left, base]{\color{textcolor}\sffamily\fontsize{18.000000}{21.600000}\selectfont $\displaystyle -0.5$}%
\end{pgfscope}%
\begin{pgfscope}%
\pgfpathrectangle{\pgfqpoint{1.396958in}{1.247073in}}{\pgfqpoint{10.524301in}{6.674186in}}%
\pgfusepath{clip}%
\pgfsetrectcap%
\pgfsetroundjoin%
\pgfsetlinewidth{0.501875pt}%
\definecolor{currentstroke}{rgb}{0.000000,0.000000,0.000000}%
\pgfsetstrokecolor{currentstroke}%
\pgfsetstrokeopacity{0.100000}%
\pgfsetdash{}{0pt}%
\pgfpathmoveto{\pgfqpoint{1.396958in}{4.584167in}}%
\pgfpathlineto{\pgfqpoint{11.921260in}{4.584167in}}%
\pgfusepath{stroke}%
\end{pgfscope}%
\begin{pgfscope}%
\pgfsetbuttcap%
\pgfsetroundjoin%
\definecolor{currentfill}{rgb}{0.000000,0.000000,0.000000}%
\pgfsetfillcolor{currentfill}%
\pgfsetlinewidth{0.501875pt}%
\definecolor{currentstroke}{rgb}{0.000000,0.000000,0.000000}%
\pgfsetstrokecolor{currentstroke}%
\pgfsetdash{}{0pt}%
\pgfsys@defobject{currentmarker}{\pgfqpoint{0.000000in}{0.000000in}}{\pgfqpoint{0.034722in}{0.000000in}}{%
\pgfpathmoveto{\pgfqpoint{0.000000in}{0.000000in}}%
\pgfpathlineto{\pgfqpoint{0.034722in}{0.000000in}}%
\pgfusepath{stroke,fill}%
}%
\begin{pgfscope}%
\pgfsys@transformshift{1.396958in}{4.584167in}%
\pgfsys@useobject{currentmarker}{}%
\end{pgfscope}%
\end{pgfscope}%
\begin{pgfscope}%
\definecolor{textcolor}{rgb}{0.000000,0.000000,0.000000}%
\pgfsetstrokecolor{textcolor}%
\pgfsetfillcolor{textcolor}%
\pgftext[x=1.062934in, y=4.489196in, left, base]{\color{textcolor}\sffamily\fontsize{18.000000}{21.600000}\selectfont $\displaystyle 0.0$}%
\end{pgfscope}%
\begin{pgfscope}%
\pgfpathrectangle{\pgfqpoint{1.396958in}{1.247073in}}{\pgfqpoint{10.524301in}{6.674186in}}%
\pgfusepath{clip}%
\pgfsetrectcap%
\pgfsetroundjoin%
\pgfsetlinewidth{0.501875pt}%
\definecolor{currentstroke}{rgb}{0.000000,0.000000,0.000000}%
\pgfsetstrokecolor{currentstroke}%
\pgfsetstrokeopacity{0.100000}%
\pgfsetdash{}{0pt}%
\pgfpathmoveto{\pgfqpoint{1.396958in}{6.158267in}}%
\pgfpathlineto{\pgfqpoint{11.921260in}{6.158267in}}%
\pgfusepath{stroke}%
\end{pgfscope}%
\begin{pgfscope}%
\pgfsetbuttcap%
\pgfsetroundjoin%
\definecolor{currentfill}{rgb}{0.000000,0.000000,0.000000}%
\pgfsetfillcolor{currentfill}%
\pgfsetlinewidth{0.501875pt}%
\definecolor{currentstroke}{rgb}{0.000000,0.000000,0.000000}%
\pgfsetstrokecolor{currentstroke}%
\pgfsetdash{}{0pt}%
\pgfsys@defobject{currentmarker}{\pgfqpoint{0.000000in}{0.000000in}}{\pgfqpoint{0.034722in}{0.000000in}}{%
\pgfpathmoveto{\pgfqpoint{0.000000in}{0.000000in}}%
\pgfpathlineto{\pgfqpoint{0.034722in}{0.000000in}}%
\pgfusepath{stroke,fill}%
}%
\begin{pgfscope}%
\pgfsys@transformshift{1.396958in}{6.158267in}%
\pgfsys@useobject{currentmarker}{}%
\end{pgfscope}%
\end{pgfscope}%
\begin{pgfscope}%
\definecolor{textcolor}{rgb}{0.000000,0.000000,0.000000}%
\pgfsetstrokecolor{textcolor}%
\pgfsetfillcolor{textcolor}%
\pgftext[x=1.062934in, y=6.063297in, left, base]{\color{textcolor}\sffamily\fontsize{18.000000}{21.600000}\selectfont $\displaystyle 0.5$}%
\end{pgfscope}%
\begin{pgfscope}%
\pgfpathrectangle{\pgfqpoint{1.396958in}{1.247073in}}{\pgfqpoint{10.524301in}{6.674186in}}%
\pgfusepath{clip}%
\pgfsetrectcap%
\pgfsetroundjoin%
\pgfsetlinewidth{0.501875pt}%
\definecolor{currentstroke}{rgb}{0.000000,0.000000,0.000000}%
\pgfsetstrokecolor{currentstroke}%
\pgfsetstrokeopacity{0.100000}%
\pgfsetdash{}{0pt}%
\pgfpathmoveto{\pgfqpoint{1.396958in}{7.732368in}}%
\pgfpathlineto{\pgfqpoint{11.921260in}{7.732368in}}%
\pgfusepath{stroke}%
\end{pgfscope}%
\begin{pgfscope}%
\pgfsetbuttcap%
\pgfsetroundjoin%
\definecolor{currentfill}{rgb}{0.000000,0.000000,0.000000}%
\pgfsetfillcolor{currentfill}%
\pgfsetlinewidth{0.501875pt}%
\definecolor{currentstroke}{rgb}{0.000000,0.000000,0.000000}%
\pgfsetstrokecolor{currentstroke}%
\pgfsetdash{}{0pt}%
\pgfsys@defobject{currentmarker}{\pgfqpoint{0.000000in}{0.000000in}}{\pgfqpoint{0.034722in}{0.000000in}}{%
\pgfpathmoveto{\pgfqpoint{0.000000in}{0.000000in}}%
\pgfpathlineto{\pgfqpoint{0.034722in}{0.000000in}}%
\pgfusepath{stroke,fill}%
}%
\begin{pgfscope}%
\pgfsys@transformshift{1.396958in}{7.732368in}%
\pgfsys@useobject{currentmarker}{}%
\end{pgfscope}%
\end{pgfscope}%
\begin{pgfscope}%
\definecolor{textcolor}{rgb}{0.000000,0.000000,0.000000}%
\pgfsetstrokecolor{textcolor}%
\pgfsetfillcolor{textcolor}%
\pgftext[x=1.062934in, y=7.637397in, left, base]{\color{textcolor}\sffamily\fontsize{18.000000}{21.600000}\selectfont $\displaystyle 1.0$}%
\end{pgfscope}%
\begin{pgfscope}%
\pgfpathrectangle{\pgfqpoint{1.396958in}{1.247073in}}{\pgfqpoint{10.524301in}{6.674186in}}%
\pgfusepath{clip}%
\pgfsetbuttcap%
\pgfsetroundjoin%
\pgfsetlinewidth{1.003750pt}%
\definecolor{currentstroke}{rgb}{0.000000,0.605603,0.978680}%
\pgfsetstrokecolor{currentstroke}%
\pgfsetdash{}{0pt}%
\pgfpathmoveto{\pgfqpoint{1.694816in}{4.584668in}}%
\pgfpathlineto{\pgfqpoint{1.927517in}{4.122733in}}%
\pgfpathlineto{\pgfqpoint{2.082651in}{3.819715in}}%
\pgfpathlineto{\pgfqpoint{2.237786in}{3.524059in}}%
\pgfpathlineto{\pgfqpoint{2.315353in}{3.379885in}}%
\pgfpathlineto{\pgfqpoint{2.392920in}{3.238613in}}%
\pgfpathlineto{\pgfqpoint{2.470487in}{3.100582in}}%
\pgfpathlineto{\pgfqpoint{2.548054in}{2.966125in}}%
\pgfpathlineto{\pgfqpoint{2.625621in}{2.835566in}}%
\pgfpathlineto{\pgfqpoint{2.703188in}{2.709219in}}%
\pgfpathlineto{\pgfqpoint{2.780755in}{2.587390in}}%
\pgfpathlineto{\pgfqpoint{2.858322in}{2.470371in}}%
\pgfpathlineto{\pgfqpoint{2.935889in}{2.358444in}}%
\pgfpathlineto{\pgfqpoint{3.013456in}{2.251879in}}%
\pgfpathlineto{\pgfqpoint{3.091023in}{2.150933in}}%
\pgfpathlineto{\pgfqpoint{3.168591in}{2.055849in}}%
\pgfpathlineto{\pgfqpoint{3.246158in}{1.966855in}}%
\pgfpathlineto{\pgfqpoint{3.323725in}{1.884167in}}%
\pgfpathlineto{\pgfqpoint{3.401292in}{1.807984in}}%
\pgfpathlineto{\pgfqpoint{3.478859in}{1.738489in}}%
\pgfpathlineto{\pgfqpoint{3.556426in}{1.675849in}}%
\pgfpathlineto{\pgfqpoint{3.633993in}{1.620215in}}%
\pgfpathlineto{\pgfqpoint{3.711560in}{1.571722in}}%
\pgfpathlineto{\pgfqpoint{3.789127in}{1.530486in}}%
\pgfpathlineto{\pgfqpoint{3.866694in}{1.496607in}}%
\pgfpathlineto{\pgfqpoint{3.944261in}{1.470166in}}%
\pgfpathlineto{\pgfqpoint{4.021828in}{1.451227in}}%
\pgfpathlineto{\pgfqpoint{4.099396in}{1.439835in}}%
\pgfpathlineto{\pgfqpoint{4.176963in}{1.436018in}}%
\pgfpathlineto{\pgfqpoint{4.254530in}{1.439786in}}%
\pgfpathlineto{\pgfqpoint{4.332097in}{1.451129in}}%
\pgfpathlineto{\pgfqpoint{4.409664in}{1.470019in}}%
\pgfpathlineto{\pgfqpoint{4.487231in}{1.496411in}}%
\pgfpathlineto{\pgfqpoint{4.564798in}{1.530243in}}%
\pgfpathlineto{\pgfqpoint{4.642365in}{1.571431in}}%
\pgfpathlineto{\pgfqpoint{4.719932in}{1.619878in}}%
\pgfpathlineto{\pgfqpoint{4.797499in}{1.675465in}}%
\pgfpathlineto{\pgfqpoint{4.875066in}{1.738060in}}%
\pgfpathlineto{\pgfqpoint{4.952633in}{1.807512in}}%
\pgfpathlineto{\pgfqpoint{5.030200in}{1.883652in}}%
\pgfpathlineto{\pgfqpoint{5.107768in}{1.966299in}}%
\pgfpathlineto{\pgfqpoint{5.185335in}{2.055252in}}%
\pgfpathlineto{\pgfqpoint{5.262902in}{2.150297in}}%
\pgfpathlineto{\pgfqpoint{5.340469in}{2.251206in}}%
\pgfpathlineto{\pgfqpoint{5.418036in}{2.357736in}}%
\pgfpathlineto{\pgfqpoint{5.495603in}{2.469628in}}%
\pgfpathlineto{\pgfqpoint{5.573170in}{2.586615in}}%
\pgfpathlineto{\pgfqpoint{5.650737in}{2.708415in}}%
\pgfpathlineto{\pgfqpoint{5.728304in}{2.834733in}}%
\pgfpathlineto{\pgfqpoint{5.805871in}{2.965265in}}%
\pgfpathlineto{\pgfqpoint{5.883438in}{3.099698in}}%
\pgfpathlineto{\pgfqpoint{5.961005in}{3.237707in}}%
\pgfpathlineto{\pgfqpoint{6.038573in}{3.378960in}}%
\pgfpathlineto{\pgfqpoint{6.116140in}{3.523116in}}%
\pgfpathlineto{\pgfqpoint{6.271274in}{3.818743in}}%
\pgfpathlineto{\pgfqpoint{6.426408in}{4.121742in}}%
\pgfpathlineto{\pgfqpoint{6.659109in}{4.583666in}}%
\pgfpathlineto{\pgfqpoint{6.891810in}{5.045600in}}%
\pgfpathlineto{\pgfqpoint{7.046945in}{5.348618in}}%
\pgfpathlineto{\pgfqpoint{7.202079in}{5.644274in}}%
\pgfpathlineto{\pgfqpoint{7.279646in}{5.788448in}}%
\pgfpathlineto{\pgfqpoint{7.357213in}{5.929721in}}%
\pgfpathlineto{\pgfqpoint{7.434780in}{6.067752in}}%
\pgfpathlineto{\pgfqpoint{7.512347in}{6.202209in}}%
\pgfpathlineto{\pgfqpoint{7.589914in}{6.332768in}}%
\pgfpathlineto{\pgfqpoint{7.667481in}{6.459114in}}%
\pgfpathlineto{\pgfqpoint{7.745048in}{6.580944in}}%
\pgfpathlineto{\pgfqpoint{7.822615in}{6.697963in}}%
\pgfpathlineto{\pgfqpoint{7.900182in}{6.809889in}}%
\pgfpathlineto{\pgfqpoint{7.977750in}{6.916454in}}%
\pgfpathlineto{\pgfqpoint{8.055317in}{7.017400in}}%
\pgfpathlineto{\pgfqpoint{8.132884in}{7.112485in}}%
\pgfpathlineto{\pgfqpoint{8.210451in}{7.201478in}}%
\pgfpathlineto{\pgfqpoint{8.288018in}{7.284166in}}%
\pgfpathlineto{\pgfqpoint{8.365585in}{7.360349in}}%
\pgfpathlineto{\pgfqpoint{8.443152in}{7.429845in}}%
\pgfpathlineto{\pgfqpoint{8.520719in}{7.492485in}}%
\pgfpathlineto{\pgfqpoint{8.598286in}{7.548118in}}%
\pgfpathlineto{\pgfqpoint{8.675853in}{7.596611in}}%
\pgfpathlineto{\pgfqpoint{8.753420in}{7.637847in}}%
\pgfpathlineto{\pgfqpoint{8.830987in}{7.671726in}}%
\pgfpathlineto{\pgfqpoint{8.908554in}{7.698167in}}%
\pgfpathlineto{\pgfqpoint{8.986122in}{7.717107in}}%
\pgfpathlineto{\pgfqpoint{9.063689in}{7.728498in}}%
\pgfpathlineto{\pgfqpoint{9.141256in}{7.732315in}}%
\pgfpathlineto{\pgfqpoint{9.218823in}{7.728547in}}%
\pgfpathlineto{\pgfqpoint{9.296390in}{7.717205in}}%
\pgfpathlineto{\pgfqpoint{9.373957in}{7.698314in}}%
\pgfpathlineto{\pgfqpoint{9.451524in}{7.671922in}}%
\pgfpathlineto{\pgfqpoint{9.529091in}{7.638091in}}%
\pgfpathlineto{\pgfqpoint{9.606658in}{7.596902in}}%
\pgfpathlineto{\pgfqpoint{9.684225in}{7.548456in}}%
\pgfpathlineto{\pgfqpoint{9.761792in}{7.492868in}}%
\pgfpathlineto{\pgfqpoint{9.839359in}{7.430273in}}%
\pgfpathlineto{\pgfqpoint{9.916927in}{7.360822in}}%
\pgfpathlineto{\pgfqpoint{9.994494in}{7.284681in}}%
\pgfpathlineto{\pgfqpoint{10.072061in}{7.202035in}}%
\pgfpathlineto{\pgfqpoint{10.149628in}{7.113081in}}%
\pgfpathlineto{\pgfqpoint{10.227195in}{7.018036in}}%
\pgfpathlineto{\pgfqpoint{10.304762in}{6.917127in}}%
\pgfpathlineto{\pgfqpoint{10.382329in}{6.810598in}}%
\pgfpathlineto{\pgfqpoint{10.459896in}{6.698705in}}%
\pgfpathlineto{\pgfqpoint{10.537463in}{6.581718in}}%
\pgfpathlineto{\pgfqpoint{10.615030in}{6.459919in}}%
\pgfpathlineto{\pgfqpoint{10.692597in}{6.333601in}}%
\pgfpathlineto{\pgfqpoint{10.770164in}{6.203068in}}%
\pgfpathlineto{\pgfqpoint{10.847731in}{6.068635in}}%
\pgfpathlineto{\pgfqpoint{10.925299in}{5.930626in}}%
\pgfpathlineto{\pgfqpoint{11.002866in}{5.789374in}}%
\pgfpathlineto{\pgfqpoint{11.080433in}{5.645217in}}%
\pgfpathlineto{\pgfqpoint{11.235567in}{5.349590in}}%
\pgfpathlineto{\pgfqpoint{11.390701in}{5.046591in}}%
\pgfpathlineto{\pgfqpoint{11.623402in}{4.584668in}}%
\pgfpathlineto{\pgfqpoint{11.623402in}{4.584668in}}%
\pgfusepath{stroke}%
\end{pgfscope}%
\begin{pgfscope}%
\pgfpathrectangle{\pgfqpoint{1.396958in}{1.247073in}}{\pgfqpoint{10.524301in}{6.674186in}}%
\pgfusepath{clip}%
\pgfsetbuttcap%
\pgfsetroundjoin%
\pgfsetlinewidth{1.003750pt}%
\definecolor{currentstroke}{rgb}{0.888874,0.435649,0.278123}%
\pgfsetstrokecolor{currentstroke}%
\pgfsetdash{}{0pt}%
\pgfpathmoveto{\pgfqpoint{1.694816in}{4.584167in}}%
\pgfpathlineto{\pgfqpoint{1.927517in}{4.122230in}}%
\pgfpathlineto{\pgfqpoint{2.082651in}{3.819216in}}%
\pgfpathlineto{\pgfqpoint{2.237786in}{3.523570in}}%
\pgfpathlineto{\pgfqpoint{2.315353in}{3.379402in}}%
\pgfpathlineto{\pgfqpoint{2.392920in}{3.238137in}}%
\pgfpathlineto{\pgfqpoint{2.470487in}{3.100115in}}%
\pgfpathlineto{\pgfqpoint{2.548054in}{2.965668in}}%
\pgfpathlineto{\pgfqpoint{2.625621in}{2.835120in}}%
\pgfpathlineto{\pgfqpoint{2.703188in}{2.708785in}}%
\pgfpathlineto{\pgfqpoint{2.780755in}{2.586969in}}%
\pgfpathlineto{\pgfqpoint{2.858322in}{2.469964in}}%
\pgfpathlineto{\pgfqpoint{2.935889in}{2.358052in}}%
\pgfpathlineto{\pgfqpoint{3.013456in}{2.251504in}}%
\pgfpathlineto{\pgfqpoint{3.091023in}{2.150574in}}%
\pgfpathlineto{\pgfqpoint{3.168591in}{2.055508in}}%
\pgfpathlineto{\pgfqpoint{3.246158in}{1.966533in}}%
\pgfpathlineto{\pgfqpoint{3.323725in}{1.883865in}}%
\pgfpathlineto{\pgfqpoint{3.401292in}{1.807701in}}%
\pgfpathlineto{\pgfqpoint{3.478859in}{1.738227in}}%
\pgfpathlineto{\pgfqpoint{3.556426in}{1.675608in}}%
\pgfpathlineto{\pgfqpoint{3.633993in}{1.619997in}}%
\pgfpathlineto{\pgfqpoint{3.711560in}{1.571526in}}%
\pgfpathlineto{\pgfqpoint{3.789127in}{1.530313in}}%
\pgfpathlineto{\pgfqpoint{3.866694in}{1.496457in}}%
\pgfpathlineto{\pgfqpoint{3.944261in}{1.470040in}}%
\pgfpathlineto{\pgfqpoint{4.021828in}{1.451125in}}%
\pgfpathlineto{\pgfqpoint{4.099396in}{1.439758in}}%
\pgfpathlineto{\pgfqpoint{4.176963in}{1.435966in}}%
\pgfpathlineto{\pgfqpoint{4.254530in}{1.439758in}}%
\pgfpathlineto{\pgfqpoint{4.332097in}{1.451125in}}%
\pgfpathlineto{\pgfqpoint{4.409664in}{1.470040in}}%
\pgfpathlineto{\pgfqpoint{4.487231in}{1.496457in}}%
\pgfpathlineto{\pgfqpoint{4.564798in}{1.530313in}}%
\pgfpathlineto{\pgfqpoint{4.642365in}{1.571526in}}%
\pgfpathlineto{\pgfqpoint{4.719932in}{1.619997in}}%
\pgfpathlineto{\pgfqpoint{4.797499in}{1.675608in}}%
\pgfpathlineto{\pgfqpoint{4.875066in}{1.738227in}}%
\pgfpathlineto{\pgfqpoint{4.952633in}{1.807701in}}%
\pgfpathlineto{\pgfqpoint{5.030200in}{1.883865in}}%
\pgfpathlineto{\pgfqpoint{5.107768in}{1.966533in}}%
\pgfpathlineto{\pgfqpoint{5.185335in}{2.055508in}}%
\pgfpathlineto{\pgfqpoint{5.262902in}{2.150574in}}%
\pgfpathlineto{\pgfqpoint{5.340469in}{2.251504in}}%
\pgfpathlineto{\pgfqpoint{5.418036in}{2.358052in}}%
\pgfpathlineto{\pgfqpoint{5.495603in}{2.469964in}}%
\pgfpathlineto{\pgfqpoint{5.573170in}{2.586969in}}%
\pgfpathlineto{\pgfqpoint{5.650737in}{2.708785in}}%
\pgfpathlineto{\pgfqpoint{5.728304in}{2.835120in}}%
\pgfpathlineto{\pgfqpoint{5.805871in}{2.965668in}}%
\pgfpathlineto{\pgfqpoint{5.883438in}{3.100115in}}%
\pgfpathlineto{\pgfqpoint{5.961005in}{3.238137in}}%
\pgfpathlineto{\pgfqpoint{6.038573in}{3.379402in}}%
\pgfpathlineto{\pgfqpoint{6.116140in}{3.523570in}}%
\pgfpathlineto{\pgfqpoint{6.271274in}{3.819216in}}%
\pgfpathlineto{\pgfqpoint{6.426408in}{4.122230in}}%
\pgfpathlineto{\pgfqpoint{6.659109in}{4.584167in}}%
\pgfpathlineto{\pgfqpoint{6.891810in}{5.046104in}}%
\pgfpathlineto{\pgfqpoint{7.046945in}{5.349117in}}%
\pgfpathlineto{\pgfqpoint{7.202079in}{5.644764in}}%
\pgfpathlineto{\pgfqpoint{7.279646in}{5.788931in}}%
\pgfpathlineto{\pgfqpoint{7.357213in}{5.930196in}}%
\pgfpathlineto{\pgfqpoint{7.434780in}{6.068218in}}%
\pgfpathlineto{\pgfqpoint{7.512347in}{6.202665in}}%
\pgfpathlineto{\pgfqpoint{7.589914in}{6.333213in}}%
\pgfpathlineto{\pgfqpoint{7.667481in}{6.459548in}}%
\pgfpathlineto{\pgfqpoint{7.745048in}{6.581364in}}%
\pgfpathlineto{\pgfqpoint{7.822615in}{6.698369in}}%
\pgfpathlineto{\pgfqpoint{7.900182in}{6.810281in}}%
\pgfpathlineto{\pgfqpoint{7.977750in}{6.916830in}}%
\pgfpathlineto{\pgfqpoint{8.055317in}{7.017759in}}%
\pgfpathlineto{\pgfqpoint{8.132884in}{7.112826in}}%
\pgfpathlineto{\pgfqpoint{8.210451in}{7.201800in}}%
\pgfpathlineto{\pgfqpoint{8.288018in}{7.284469in}}%
\pgfpathlineto{\pgfqpoint{8.365585in}{7.360632in}}%
\pgfpathlineto{\pgfqpoint{8.443152in}{7.430107in}}%
\pgfpathlineto{\pgfqpoint{8.520719in}{7.492725in}}%
\pgfpathlineto{\pgfqpoint{8.598286in}{7.548337in}}%
\pgfpathlineto{\pgfqpoint{8.675853in}{7.596807in}}%
\pgfpathlineto{\pgfqpoint{8.753420in}{7.638020in}}%
\pgfpathlineto{\pgfqpoint{8.830987in}{7.671876in}}%
\pgfpathlineto{\pgfqpoint{8.908554in}{7.698293in}}%
\pgfpathlineto{\pgfqpoint{8.986122in}{7.717208in}}%
\pgfpathlineto{\pgfqpoint{9.063689in}{7.728576in}}%
\pgfpathlineto{\pgfqpoint{9.141256in}{7.732368in}}%
\pgfpathlineto{\pgfqpoint{9.218823in}{7.728576in}}%
\pgfpathlineto{\pgfqpoint{9.296390in}{7.717208in}}%
\pgfpathlineto{\pgfqpoint{9.373957in}{7.698293in}}%
\pgfpathlineto{\pgfqpoint{9.451524in}{7.671876in}}%
\pgfpathlineto{\pgfqpoint{9.529091in}{7.638020in}}%
\pgfpathlineto{\pgfqpoint{9.606658in}{7.596807in}}%
\pgfpathlineto{\pgfqpoint{9.684225in}{7.548337in}}%
\pgfpathlineto{\pgfqpoint{9.761792in}{7.492725in}}%
\pgfpathlineto{\pgfqpoint{9.839359in}{7.430107in}}%
\pgfpathlineto{\pgfqpoint{9.916927in}{7.360632in}}%
\pgfpathlineto{\pgfqpoint{9.994494in}{7.284469in}}%
\pgfpathlineto{\pgfqpoint{10.072061in}{7.201800in}}%
\pgfpathlineto{\pgfqpoint{10.149628in}{7.112826in}}%
\pgfpathlineto{\pgfqpoint{10.227195in}{7.017759in}}%
\pgfpathlineto{\pgfqpoint{10.304762in}{6.916830in}}%
\pgfpathlineto{\pgfqpoint{10.382329in}{6.810281in}}%
\pgfpathlineto{\pgfqpoint{10.459896in}{6.698369in}}%
\pgfpathlineto{\pgfqpoint{10.537463in}{6.581364in}}%
\pgfpathlineto{\pgfqpoint{10.615030in}{6.459548in}}%
\pgfpathlineto{\pgfqpoint{10.692597in}{6.333213in}}%
\pgfpathlineto{\pgfqpoint{10.770164in}{6.202665in}}%
\pgfpathlineto{\pgfqpoint{10.847731in}{6.068218in}}%
\pgfpathlineto{\pgfqpoint{10.925299in}{5.930196in}}%
\pgfpathlineto{\pgfqpoint{11.002866in}{5.788931in}}%
\pgfpathlineto{\pgfqpoint{11.080433in}{5.644764in}}%
\pgfpathlineto{\pgfqpoint{11.235567in}{5.349117in}}%
\pgfpathlineto{\pgfqpoint{11.390701in}{5.046104in}}%
\pgfpathlineto{\pgfqpoint{11.623402in}{4.584167in}}%
\pgfpathlineto{\pgfqpoint{11.623402in}{4.584167in}}%
\pgfusepath{stroke}%
\end{pgfscope}%
\begin{pgfscope}%
\pgfsetrectcap%
\pgfsetmiterjoin%
\pgfsetlinewidth{1.003750pt}%
\definecolor{currentstroke}{rgb}{0.000000,0.000000,0.000000}%
\pgfsetstrokecolor{currentstroke}%
\pgfsetdash{}{0pt}%
\pgfpathmoveto{\pgfqpoint{1.396958in}{1.247073in}}%
\pgfpathlineto{\pgfqpoint{1.396958in}{7.921260in}}%
\pgfusepath{stroke}%
\end{pgfscope}%
\begin{pgfscope}%
\pgfsetrectcap%
\pgfsetmiterjoin%
\pgfsetlinewidth{1.003750pt}%
\definecolor{currentstroke}{rgb}{0.000000,0.000000,0.000000}%
\pgfsetstrokecolor{currentstroke}%
\pgfsetdash{}{0pt}%
\pgfpathmoveto{\pgfqpoint{1.396958in}{1.247073in}}%
\pgfpathlineto{\pgfqpoint{11.921260in}{1.247073in}}%
\pgfusepath{stroke}%
\end{pgfscope}%
\begin{pgfscope}%
\pgfsetbuttcap%
\pgfsetmiterjoin%
\definecolor{currentfill}{rgb}{1.000000,1.000000,1.000000}%
\pgfsetfillcolor{currentfill}%
\pgfsetlinewidth{1.003750pt}%
\definecolor{currentstroke}{rgb}{0.000000,0.000000,0.000000}%
\pgfsetstrokecolor{currentstroke}%
\pgfsetdash{}{0pt}%
\pgfpathmoveto{\pgfqpoint{10.511589in}{6.787373in}}%
\pgfpathlineto{\pgfqpoint{11.796260in}{6.787373in}}%
\pgfpathlineto{\pgfqpoint{11.796260in}{7.796260in}}%
\pgfpathlineto{\pgfqpoint{10.511589in}{7.796260in}}%
\pgfpathclose%
\pgfusepath{stroke,fill}%
\end{pgfscope}%
\begin{pgfscope}%
\pgfsetbuttcap%
\pgfsetmiterjoin%
\pgfsetlinewidth{2.258437pt}%
\definecolor{currentstroke}{rgb}{0.000000,0.605603,0.978680}%
\pgfsetstrokecolor{currentstroke}%
\pgfsetdash{}{0pt}%
\pgfpathmoveto{\pgfqpoint{10.711589in}{7.493818in}}%
\pgfpathlineto{\pgfqpoint{11.211589in}{7.493818in}}%
\pgfusepath{stroke}%
\end{pgfscope}%
\begin{pgfscope}%
\definecolor{textcolor}{rgb}{0.000000,0.000000,0.000000}%
\pgfsetstrokecolor{textcolor}%
\pgfsetfillcolor{textcolor}%
\pgftext[x=11.411589in,y=7.406318in,left,base]{\color{textcolor}\sffamily\fontsize{18.000000}{21.600000}\selectfont $\displaystyle U$}%
\end{pgfscope}%
\begin{pgfscope}%
\pgfsetbuttcap%
\pgfsetmiterjoin%
\pgfsetlinewidth{2.258437pt}%
\definecolor{currentstroke}{rgb}{0.888874,0.435649,0.278123}%
\pgfsetstrokecolor{currentstroke}%
\pgfsetdash{}{0pt}%
\pgfpathmoveto{\pgfqpoint{10.711589in}{7.126875in}}%
\pgfpathlineto{\pgfqpoint{11.211589in}{7.126875in}}%
\pgfusepath{stroke}%
\end{pgfscope}%
\begin{pgfscope}%
\definecolor{textcolor}{rgb}{0.000000,0.000000,0.000000}%
\pgfsetstrokecolor{textcolor}%
\pgfsetfillcolor{textcolor}%
\pgftext[x=11.411589in,y=7.039375in,left,base]{\color{textcolor}\sffamily\fontsize{18.000000}{21.600000}\selectfont $\displaystyle u$}%
\end{pgfscope}%
\end{pgfpicture}%
\makeatother%
\endgroup%
}\quad
	\resizebox{0.4\linewidth}{!}{%% Creator: Matplotlib, PGF backend
%%
%% To include the figure in your LaTeX document, write
%%   \input{<filename>.pgf}
%%
%% Make sure the required packages are loaded in your preamble
%%   \usepackage{pgf}
%%
%% Figures using additional raster images can only be included by \input if
%% they are in the same directory as the main LaTeX file. For loading figures
%% from other directories you can use the `import` package
%%   \usepackage{import}
%%
%% and then include the figures with
%%   \import{<path to file>}{<filename>.pgf}
%%
%% Matplotlib used the following preamble
%%   \usepackage{fontspec}
%%   \setmainfont{DejaVuSerif.ttf}[Path=\detokenize{/Users/quejiahao/.julia/conda/3/lib/python3.9/site-packages/matplotlib/mpl-data/fonts/ttf/}]
%%   \setsansfont{DejaVuSans.ttf}[Path=\detokenize{/Users/quejiahao/.julia/conda/3/lib/python3.9/site-packages/matplotlib/mpl-data/fonts/ttf/}]
%%   \setmonofont{DejaVuSansMono.ttf}[Path=\detokenize{/Users/quejiahao/.julia/conda/3/lib/python3.9/site-packages/matplotlib/mpl-data/fonts/ttf/}]
%%
\begingroup%
\makeatletter%
\begin{pgfpicture}%
\pgfpathrectangle{\pgfpointorigin}{\pgfqpoint{12.000000in}{8.000000in}}%
\pgfusepath{use as bounding box, clip}%
\begin{pgfscope}%
\pgfsetbuttcap%
\pgfsetmiterjoin%
\definecolor{currentfill}{rgb}{1.000000,1.000000,1.000000}%
\pgfsetfillcolor{currentfill}%
\pgfsetlinewidth{0.000000pt}%
\definecolor{currentstroke}{rgb}{1.000000,1.000000,1.000000}%
\pgfsetstrokecolor{currentstroke}%
\pgfsetdash{}{0pt}%
\pgfpathmoveto{\pgfqpoint{0.000000in}{0.000000in}}%
\pgfpathlineto{\pgfqpoint{12.000000in}{0.000000in}}%
\pgfpathlineto{\pgfqpoint{12.000000in}{8.000000in}}%
\pgfpathlineto{\pgfqpoint{0.000000in}{8.000000in}}%
\pgfpathclose%
\pgfusepath{fill}%
\end{pgfscope}%
\begin{pgfscope}%
\pgfsetbuttcap%
\pgfsetmiterjoin%
\definecolor{currentfill}{rgb}{1.000000,1.000000,1.000000}%
\pgfsetfillcolor{currentfill}%
\pgfsetlinewidth{0.000000pt}%
\definecolor{currentstroke}{rgb}{0.000000,0.000000,0.000000}%
\pgfsetstrokecolor{currentstroke}%
\pgfsetstrokeopacity{0.000000}%
\pgfsetdash{}{0pt}%
\pgfpathmoveto{\pgfqpoint{1.396958in}{1.247073in}}%
\pgfpathlineto{\pgfqpoint{11.921260in}{1.247073in}}%
\pgfpathlineto{\pgfqpoint{11.921260in}{7.921260in}}%
\pgfpathlineto{\pgfqpoint{1.396958in}{7.921260in}}%
\pgfpathclose%
\pgfusepath{fill}%
\end{pgfscope}%
\begin{pgfscope}%
\pgfpathrectangle{\pgfqpoint{1.396958in}{1.247073in}}{\pgfqpoint{10.524301in}{6.674186in}}%
\pgfusepath{clip}%
\pgfsetrectcap%
\pgfsetroundjoin%
\pgfsetlinewidth{0.501875pt}%
\definecolor{currentstroke}{rgb}{0.000000,0.000000,0.000000}%
\pgfsetstrokecolor{currentstroke}%
\pgfsetstrokeopacity{0.100000}%
\pgfsetdash{}{0pt}%
\pgfpathmoveto{\pgfqpoint{1.694816in}{1.247073in}}%
\pgfpathlineto{\pgfqpoint{1.694816in}{7.921260in}}%
\pgfusepath{stroke}%
\end{pgfscope}%
\begin{pgfscope}%
\pgfsetbuttcap%
\pgfsetroundjoin%
\definecolor{currentfill}{rgb}{0.000000,0.000000,0.000000}%
\pgfsetfillcolor{currentfill}%
\pgfsetlinewidth{0.501875pt}%
\definecolor{currentstroke}{rgb}{0.000000,0.000000,0.000000}%
\pgfsetstrokecolor{currentstroke}%
\pgfsetdash{}{0pt}%
\pgfsys@defobject{currentmarker}{\pgfqpoint{0.000000in}{0.000000in}}{\pgfqpoint{0.000000in}{0.034722in}}{%
\pgfpathmoveto{\pgfqpoint{0.000000in}{0.000000in}}%
\pgfpathlineto{\pgfqpoint{0.000000in}{0.034722in}}%
\pgfusepath{stroke,fill}%
}%
\begin{pgfscope}%
\pgfsys@transformshift{1.694816in}{1.247073in}%
\pgfsys@useobject{currentmarker}{}%
\end{pgfscope}%
\end{pgfscope}%
\begin{pgfscope}%
\definecolor{textcolor}{rgb}{0.000000,0.000000,0.000000}%
\pgfsetstrokecolor{textcolor}%
\pgfsetfillcolor{textcolor}%
\pgftext[x=1.694816in,y=1.198462in,,top]{\color{textcolor}\sffamily\fontsize{18.000000}{21.600000}\selectfont $\displaystyle 0$}%
\end{pgfscope}%
\begin{pgfscope}%
\pgfpathrectangle{\pgfqpoint{1.396958in}{1.247073in}}{\pgfqpoint{10.524301in}{6.674186in}}%
\pgfusepath{clip}%
\pgfsetrectcap%
\pgfsetroundjoin%
\pgfsetlinewidth{0.501875pt}%
\definecolor{currentstroke}{rgb}{0.000000,0.000000,0.000000}%
\pgfsetstrokecolor{currentstroke}%
\pgfsetstrokeopacity{0.100000}%
\pgfsetdash{}{0pt}%
\pgfpathmoveto{\pgfqpoint{3.275000in}{1.247073in}}%
\pgfpathlineto{\pgfqpoint{3.275000in}{7.921260in}}%
\pgfusepath{stroke}%
\end{pgfscope}%
\begin{pgfscope}%
\pgfsetbuttcap%
\pgfsetroundjoin%
\definecolor{currentfill}{rgb}{0.000000,0.000000,0.000000}%
\pgfsetfillcolor{currentfill}%
\pgfsetlinewidth{0.501875pt}%
\definecolor{currentstroke}{rgb}{0.000000,0.000000,0.000000}%
\pgfsetstrokecolor{currentstroke}%
\pgfsetdash{}{0pt}%
\pgfsys@defobject{currentmarker}{\pgfqpoint{0.000000in}{0.000000in}}{\pgfqpoint{0.000000in}{0.034722in}}{%
\pgfpathmoveto{\pgfqpoint{0.000000in}{0.000000in}}%
\pgfpathlineto{\pgfqpoint{0.000000in}{0.034722in}}%
\pgfusepath{stroke,fill}%
}%
\begin{pgfscope}%
\pgfsys@transformshift{3.275000in}{1.247073in}%
\pgfsys@useobject{currentmarker}{}%
\end{pgfscope}%
\end{pgfscope}%
\begin{pgfscope}%
\definecolor{textcolor}{rgb}{0.000000,0.000000,0.000000}%
\pgfsetstrokecolor{textcolor}%
\pgfsetfillcolor{textcolor}%
\pgftext[x=3.275000in,y=1.198462in,,top]{\color{textcolor}\sffamily\fontsize{18.000000}{21.600000}\selectfont $\displaystyle 1$}%
\end{pgfscope}%
\begin{pgfscope}%
\pgfpathrectangle{\pgfqpoint{1.396958in}{1.247073in}}{\pgfqpoint{10.524301in}{6.674186in}}%
\pgfusepath{clip}%
\pgfsetrectcap%
\pgfsetroundjoin%
\pgfsetlinewidth{0.501875pt}%
\definecolor{currentstroke}{rgb}{0.000000,0.000000,0.000000}%
\pgfsetstrokecolor{currentstroke}%
\pgfsetstrokeopacity{0.100000}%
\pgfsetdash{}{0pt}%
\pgfpathmoveto{\pgfqpoint{4.855183in}{1.247073in}}%
\pgfpathlineto{\pgfqpoint{4.855183in}{7.921260in}}%
\pgfusepath{stroke}%
\end{pgfscope}%
\begin{pgfscope}%
\pgfsetbuttcap%
\pgfsetroundjoin%
\definecolor{currentfill}{rgb}{0.000000,0.000000,0.000000}%
\pgfsetfillcolor{currentfill}%
\pgfsetlinewidth{0.501875pt}%
\definecolor{currentstroke}{rgb}{0.000000,0.000000,0.000000}%
\pgfsetstrokecolor{currentstroke}%
\pgfsetdash{}{0pt}%
\pgfsys@defobject{currentmarker}{\pgfqpoint{0.000000in}{0.000000in}}{\pgfqpoint{0.000000in}{0.034722in}}{%
\pgfpathmoveto{\pgfqpoint{0.000000in}{0.000000in}}%
\pgfpathlineto{\pgfqpoint{0.000000in}{0.034722in}}%
\pgfusepath{stroke,fill}%
}%
\begin{pgfscope}%
\pgfsys@transformshift{4.855183in}{1.247073in}%
\pgfsys@useobject{currentmarker}{}%
\end{pgfscope}%
\end{pgfscope}%
\begin{pgfscope}%
\definecolor{textcolor}{rgb}{0.000000,0.000000,0.000000}%
\pgfsetstrokecolor{textcolor}%
\pgfsetfillcolor{textcolor}%
\pgftext[x=4.855183in,y=1.198462in,,top]{\color{textcolor}\sffamily\fontsize{18.000000}{21.600000}\selectfont $\displaystyle 2$}%
\end{pgfscope}%
\begin{pgfscope}%
\pgfpathrectangle{\pgfqpoint{1.396958in}{1.247073in}}{\pgfqpoint{10.524301in}{6.674186in}}%
\pgfusepath{clip}%
\pgfsetrectcap%
\pgfsetroundjoin%
\pgfsetlinewidth{0.501875pt}%
\definecolor{currentstroke}{rgb}{0.000000,0.000000,0.000000}%
\pgfsetstrokecolor{currentstroke}%
\pgfsetstrokeopacity{0.100000}%
\pgfsetdash{}{0pt}%
\pgfpathmoveto{\pgfqpoint{6.435367in}{1.247073in}}%
\pgfpathlineto{\pgfqpoint{6.435367in}{7.921260in}}%
\pgfusepath{stroke}%
\end{pgfscope}%
\begin{pgfscope}%
\pgfsetbuttcap%
\pgfsetroundjoin%
\definecolor{currentfill}{rgb}{0.000000,0.000000,0.000000}%
\pgfsetfillcolor{currentfill}%
\pgfsetlinewidth{0.501875pt}%
\definecolor{currentstroke}{rgb}{0.000000,0.000000,0.000000}%
\pgfsetstrokecolor{currentstroke}%
\pgfsetdash{}{0pt}%
\pgfsys@defobject{currentmarker}{\pgfqpoint{0.000000in}{0.000000in}}{\pgfqpoint{0.000000in}{0.034722in}}{%
\pgfpathmoveto{\pgfqpoint{0.000000in}{0.000000in}}%
\pgfpathlineto{\pgfqpoint{0.000000in}{0.034722in}}%
\pgfusepath{stroke,fill}%
}%
\begin{pgfscope}%
\pgfsys@transformshift{6.435367in}{1.247073in}%
\pgfsys@useobject{currentmarker}{}%
\end{pgfscope}%
\end{pgfscope}%
\begin{pgfscope}%
\definecolor{textcolor}{rgb}{0.000000,0.000000,0.000000}%
\pgfsetstrokecolor{textcolor}%
\pgfsetfillcolor{textcolor}%
\pgftext[x=6.435367in,y=1.198462in,,top]{\color{textcolor}\sffamily\fontsize{18.000000}{21.600000}\selectfont $\displaystyle 3$}%
\end{pgfscope}%
\begin{pgfscope}%
\pgfpathrectangle{\pgfqpoint{1.396958in}{1.247073in}}{\pgfqpoint{10.524301in}{6.674186in}}%
\pgfusepath{clip}%
\pgfsetrectcap%
\pgfsetroundjoin%
\pgfsetlinewidth{0.501875pt}%
\definecolor{currentstroke}{rgb}{0.000000,0.000000,0.000000}%
\pgfsetstrokecolor{currentstroke}%
\pgfsetstrokeopacity{0.100000}%
\pgfsetdash{}{0pt}%
\pgfpathmoveto{\pgfqpoint{8.015550in}{1.247073in}}%
\pgfpathlineto{\pgfqpoint{8.015550in}{7.921260in}}%
\pgfusepath{stroke}%
\end{pgfscope}%
\begin{pgfscope}%
\pgfsetbuttcap%
\pgfsetroundjoin%
\definecolor{currentfill}{rgb}{0.000000,0.000000,0.000000}%
\pgfsetfillcolor{currentfill}%
\pgfsetlinewidth{0.501875pt}%
\definecolor{currentstroke}{rgb}{0.000000,0.000000,0.000000}%
\pgfsetstrokecolor{currentstroke}%
\pgfsetdash{}{0pt}%
\pgfsys@defobject{currentmarker}{\pgfqpoint{0.000000in}{0.000000in}}{\pgfqpoint{0.000000in}{0.034722in}}{%
\pgfpathmoveto{\pgfqpoint{0.000000in}{0.000000in}}%
\pgfpathlineto{\pgfqpoint{0.000000in}{0.034722in}}%
\pgfusepath{stroke,fill}%
}%
\begin{pgfscope}%
\pgfsys@transformshift{8.015550in}{1.247073in}%
\pgfsys@useobject{currentmarker}{}%
\end{pgfscope}%
\end{pgfscope}%
\begin{pgfscope}%
\definecolor{textcolor}{rgb}{0.000000,0.000000,0.000000}%
\pgfsetstrokecolor{textcolor}%
\pgfsetfillcolor{textcolor}%
\pgftext[x=8.015550in,y=1.198462in,,top]{\color{textcolor}\sffamily\fontsize{18.000000}{21.600000}\selectfont $\displaystyle 4$}%
\end{pgfscope}%
\begin{pgfscope}%
\pgfpathrectangle{\pgfqpoint{1.396958in}{1.247073in}}{\pgfqpoint{10.524301in}{6.674186in}}%
\pgfusepath{clip}%
\pgfsetrectcap%
\pgfsetroundjoin%
\pgfsetlinewidth{0.501875pt}%
\definecolor{currentstroke}{rgb}{0.000000,0.000000,0.000000}%
\pgfsetstrokecolor{currentstroke}%
\pgfsetstrokeopacity{0.100000}%
\pgfsetdash{}{0pt}%
\pgfpathmoveto{\pgfqpoint{9.595734in}{1.247073in}}%
\pgfpathlineto{\pgfqpoint{9.595734in}{7.921260in}}%
\pgfusepath{stroke}%
\end{pgfscope}%
\begin{pgfscope}%
\pgfsetbuttcap%
\pgfsetroundjoin%
\definecolor{currentfill}{rgb}{0.000000,0.000000,0.000000}%
\pgfsetfillcolor{currentfill}%
\pgfsetlinewidth{0.501875pt}%
\definecolor{currentstroke}{rgb}{0.000000,0.000000,0.000000}%
\pgfsetstrokecolor{currentstroke}%
\pgfsetdash{}{0pt}%
\pgfsys@defobject{currentmarker}{\pgfqpoint{0.000000in}{0.000000in}}{\pgfqpoint{0.000000in}{0.034722in}}{%
\pgfpathmoveto{\pgfqpoint{0.000000in}{0.000000in}}%
\pgfpathlineto{\pgfqpoint{0.000000in}{0.034722in}}%
\pgfusepath{stroke,fill}%
}%
\begin{pgfscope}%
\pgfsys@transformshift{9.595734in}{1.247073in}%
\pgfsys@useobject{currentmarker}{}%
\end{pgfscope}%
\end{pgfscope}%
\begin{pgfscope}%
\definecolor{textcolor}{rgb}{0.000000,0.000000,0.000000}%
\pgfsetstrokecolor{textcolor}%
\pgfsetfillcolor{textcolor}%
\pgftext[x=9.595734in,y=1.198462in,,top]{\color{textcolor}\sffamily\fontsize{18.000000}{21.600000}\selectfont $\displaystyle 5$}%
\end{pgfscope}%
\begin{pgfscope}%
\pgfpathrectangle{\pgfqpoint{1.396958in}{1.247073in}}{\pgfqpoint{10.524301in}{6.674186in}}%
\pgfusepath{clip}%
\pgfsetrectcap%
\pgfsetroundjoin%
\pgfsetlinewidth{0.501875pt}%
\definecolor{currentstroke}{rgb}{0.000000,0.000000,0.000000}%
\pgfsetstrokecolor{currentstroke}%
\pgfsetstrokeopacity{0.100000}%
\pgfsetdash{}{0pt}%
\pgfpathmoveto{\pgfqpoint{11.175917in}{1.247073in}}%
\pgfpathlineto{\pgfqpoint{11.175917in}{7.921260in}}%
\pgfusepath{stroke}%
\end{pgfscope}%
\begin{pgfscope}%
\pgfsetbuttcap%
\pgfsetroundjoin%
\definecolor{currentfill}{rgb}{0.000000,0.000000,0.000000}%
\pgfsetfillcolor{currentfill}%
\pgfsetlinewidth{0.501875pt}%
\definecolor{currentstroke}{rgb}{0.000000,0.000000,0.000000}%
\pgfsetstrokecolor{currentstroke}%
\pgfsetdash{}{0pt}%
\pgfsys@defobject{currentmarker}{\pgfqpoint{0.000000in}{0.000000in}}{\pgfqpoint{0.000000in}{0.034722in}}{%
\pgfpathmoveto{\pgfqpoint{0.000000in}{0.000000in}}%
\pgfpathlineto{\pgfqpoint{0.000000in}{0.034722in}}%
\pgfusepath{stroke,fill}%
}%
\begin{pgfscope}%
\pgfsys@transformshift{11.175917in}{1.247073in}%
\pgfsys@useobject{currentmarker}{}%
\end{pgfscope}%
\end{pgfscope}%
\begin{pgfscope}%
\definecolor{textcolor}{rgb}{0.000000,0.000000,0.000000}%
\pgfsetstrokecolor{textcolor}%
\pgfsetfillcolor{textcolor}%
\pgftext[x=11.175917in,y=1.198462in,,top]{\color{textcolor}\sffamily\fontsize{18.000000}{21.600000}\selectfont $\displaystyle 6$}%
\end{pgfscope}%
\begin{pgfscope}%
\definecolor{textcolor}{rgb}{0.000000,0.000000,0.000000}%
\pgfsetstrokecolor{textcolor}%
\pgfsetfillcolor{textcolor}%
\pgftext[x=6.659109in,y=0.900964in,,top]{\color{textcolor}\sffamily\fontsize{18.000000}{21.600000}\selectfont $\displaystyle x$}%
\end{pgfscope}%
\begin{pgfscope}%
\pgfpathrectangle{\pgfqpoint{1.396958in}{1.247073in}}{\pgfqpoint{10.524301in}{6.674186in}}%
\pgfusepath{clip}%
\pgfsetrectcap%
\pgfsetroundjoin%
\pgfsetlinewidth{0.501875pt}%
\definecolor{currentstroke}{rgb}{0.000000,0.000000,0.000000}%
\pgfsetstrokecolor{currentstroke}%
\pgfsetstrokeopacity{0.100000}%
\pgfsetdash{}{0pt}%
\pgfpathmoveto{\pgfqpoint{1.396958in}{1.435966in}}%
\pgfpathlineto{\pgfqpoint{11.921260in}{1.435966in}}%
\pgfusepath{stroke}%
\end{pgfscope}%
\begin{pgfscope}%
\pgfsetbuttcap%
\pgfsetroundjoin%
\definecolor{currentfill}{rgb}{0.000000,0.000000,0.000000}%
\pgfsetfillcolor{currentfill}%
\pgfsetlinewidth{0.501875pt}%
\definecolor{currentstroke}{rgb}{0.000000,0.000000,0.000000}%
\pgfsetstrokecolor{currentstroke}%
\pgfsetdash{}{0pt}%
\pgfsys@defobject{currentmarker}{\pgfqpoint{0.000000in}{0.000000in}}{\pgfqpoint{0.034722in}{0.000000in}}{%
\pgfpathmoveto{\pgfqpoint{0.000000in}{0.000000in}}%
\pgfpathlineto{\pgfqpoint{0.034722in}{0.000000in}}%
\pgfusepath{stroke,fill}%
}%
\begin{pgfscope}%
\pgfsys@transformshift{1.396958in}{1.435966in}%
\pgfsys@useobject{currentmarker}{}%
\end{pgfscope}%
\end{pgfscope}%
\begin{pgfscope}%
\definecolor{textcolor}{rgb}{0.000000,0.000000,0.000000}%
\pgfsetstrokecolor{textcolor}%
\pgfsetfillcolor{textcolor}%
\pgftext[x=0.876267in, y=1.340995in, left, base]{\color{textcolor}\sffamily\fontsize{18.000000}{21.600000}\selectfont $\displaystyle -1.0$}%
\end{pgfscope}%
\begin{pgfscope}%
\pgfpathrectangle{\pgfqpoint{1.396958in}{1.247073in}}{\pgfqpoint{10.524301in}{6.674186in}}%
\pgfusepath{clip}%
\pgfsetrectcap%
\pgfsetroundjoin%
\pgfsetlinewidth{0.501875pt}%
\definecolor{currentstroke}{rgb}{0.000000,0.000000,0.000000}%
\pgfsetstrokecolor{currentstroke}%
\pgfsetstrokeopacity{0.100000}%
\pgfsetdash{}{0pt}%
\pgfpathmoveto{\pgfqpoint{1.396958in}{3.010066in}}%
\pgfpathlineto{\pgfqpoint{11.921260in}{3.010066in}}%
\pgfusepath{stroke}%
\end{pgfscope}%
\begin{pgfscope}%
\pgfsetbuttcap%
\pgfsetroundjoin%
\definecolor{currentfill}{rgb}{0.000000,0.000000,0.000000}%
\pgfsetfillcolor{currentfill}%
\pgfsetlinewidth{0.501875pt}%
\definecolor{currentstroke}{rgb}{0.000000,0.000000,0.000000}%
\pgfsetstrokecolor{currentstroke}%
\pgfsetdash{}{0pt}%
\pgfsys@defobject{currentmarker}{\pgfqpoint{0.000000in}{0.000000in}}{\pgfqpoint{0.034722in}{0.000000in}}{%
\pgfpathmoveto{\pgfqpoint{0.000000in}{0.000000in}}%
\pgfpathlineto{\pgfqpoint{0.034722in}{0.000000in}}%
\pgfusepath{stroke,fill}%
}%
\begin{pgfscope}%
\pgfsys@transformshift{1.396958in}{3.010066in}%
\pgfsys@useobject{currentmarker}{}%
\end{pgfscope}%
\end{pgfscope}%
\begin{pgfscope}%
\definecolor{textcolor}{rgb}{0.000000,0.000000,0.000000}%
\pgfsetstrokecolor{textcolor}%
\pgfsetfillcolor{textcolor}%
\pgftext[x=0.876267in, y=2.915095in, left, base]{\color{textcolor}\sffamily\fontsize{18.000000}{21.600000}\selectfont $\displaystyle -0.5$}%
\end{pgfscope}%
\begin{pgfscope}%
\pgfpathrectangle{\pgfqpoint{1.396958in}{1.247073in}}{\pgfqpoint{10.524301in}{6.674186in}}%
\pgfusepath{clip}%
\pgfsetrectcap%
\pgfsetroundjoin%
\pgfsetlinewidth{0.501875pt}%
\definecolor{currentstroke}{rgb}{0.000000,0.000000,0.000000}%
\pgfsetstrokecolor{currentstroke}%
\pgfsetstrokeopacity{0.100000}%
\pgfsetdash{}{0pt}%
\pgfpathmoveto{\pgfqpoint{1.396958in}{4.584167in}}%
\pgfpathlineto{\pgfqpoint{11.921260in}{4.584167in}}%
\pgfusepath{stroke}%
\end{pgfscope}%
\begin{pgfscope}%
\pgfsetbuttcap%
\pgfsetroundjoin%
\definecolor{currentfill}{rgb}{0.000000,0.000000,0.000000}%
\pgfsetfillcolor{currentfill}%
\pgfsetlinewidth{0.501875pt}%
\definecolor{currentstroke}{rgb}{0.000000,0.000000,0.000000}%
\pgfsetstrokecolor{currentstroke}%
\pgfsetdash{}{0pt}%
\pgfsys@defobject{currentmarker}{\pgfqpoint{0.000000in}{0.000000in}}{\pgfqpoint{0.034722in}{0.000000in}}{%
\pgfpathmoveto{\pgfqpoint{0.000000in}{0.000000in}}%
\pgfpathlineto{\pgfqpoint{0.034722in}{0.000000in}}%
\pgfusepath{stroke,fill}%
}%
\begin{pgfscope}%
\pgfsys@transformshift{1.396958in}{4.584167in}%
\pgfsys@useobject{currentmarker}{}%
\end{pgfscope}%
\end{pgfscope}%
\begin{pgfscope}%
\definecolor{textcolor}{rgb}{0.000000,0.000000,0.000000}%
\pgfsetstrokecolor{textcolor}%
\pgfsetfillcolor{textcolor}%
\pgftext[x=1.062934in, y=4.489196in, left, base]{\color{textcolor}\sffamily\fontsize{18.000000}{21.600000}\selectfont $\displaystyle 0.0$}%
\end{pgfscope}%
\begin{pgfscope}%
\pgfpathrectangle{\pgfqpoint{1.396958in}{1.247073in}}{\pgfqpoint{10.524301in}{6.674186in}}%
\pgfusepath{clip}%
\pgfsetrectcap%
\pgfsetroundjoin%
\pgfsetlinewidth{0.501875pt}%
\definecolor{currentstroke}{rgb}{0.000000,0.000000,0.000000}%
\pgfsetstrokecolor{currentstroke}%
\pgfsetstrokeopacity{0.100000}%
\pgfsetdash{}{0pt}%
\pgfpathmoveto{\pgfqpoint{1.396958in}{6.158267in}}%
\pgfpathlineto{\pgfqpoint{11.921260in}{6.158267in}}%
\pgfusepath{stroke}%
\end{pgfscope}%
\begin{pgfscope}%
\pgfsetbuttcap%
\pgfsetroundjoin%
\definecolor{currentfill}{rgb}{0.000000,0.000000,0.000000}%
\pgfsetfillcolor{currentfill}%
\pgfsetlinewidth{0.501875pt}%
\definecolor{currentstroke}{rgb}{0.000000,0.000000,0.000000}%
\pgfsetstrokecolor{currentstroke}%
\pgfsetdash{}{0pt}%
\pgfsys@defobject{currentmarker}{\pgfqpoint{0.000000in}{0.000000in}}{\pgfqpoint{0.034722in}{0.000000in}}{%
\pgfpathmoveto{\pgfqpoint{0.000000in}{0.000000in}}%
\pgfpathlineto{\pgfqpoint{0.034722in}{0.000000in}}%
\pgfusepath{stroke,fill}%
}%
\begin{pgfscope}%
\pgfsys@transformshift{1.396958in}{6.158267in}%
\pgfsys@useobject{currentmarker}{}%
\end{pgfscope}%
\end{pgfscope}%
\begin{pgfscope}%
\definecolor{textcolor}{rgb}{0.000000,0.000000,0.000000}%
\pgfsetstrokecolor{textcolor}%
\pgfsetfillcolor{textcolor}%
\pgftext[x=1.062934in, y=6.063297in, left, base]{\color{textcolor}\sffamily\fontsize{18.000000}{21.600000}\selectfont $\displaystyle 0.5$}%
\end{pgfscope}%
\begin{pgfscope}%
\pgfpathrectangle{\pgfqpoint{1.396958in}{1.247073in}}{\pgfqpoint{10.524301in}{6.674186in}}%
\pgfusepath{clip}%
\pgfsetrectcap%
\pgfsetroundjoin%
\pgfsetlinewidth{0.501875pt}%
\definecolor{currentstroke}{rgb}{0.000000,0.000000,0.000000}%
\pgfsetstrokecolor{currentstroke}%
\pgfsetstrokeopacity{0.100000}%
\pgfsetdash{}{0pt}%
\pgfpathmoveto{\pgfqpoint{1.396958in}{7.732368in}}%
\pgfpathlineto{\pgfqpoint{11.921260in}{7.732368in}}%
\pgfusepath{stroke}%
\end{pgfscope}%
\begin{pgfscope}%
\pgfsetbuttcap%
\pgfsetroundjoin%
\definecolor{currentfill}{rgb}{0.000000,0.000000,0.000000}%
\pgfsetfillcolor{currentfill}%
\pgfsetlinewidth{0.501875pt}%
\definecolor{currentstroke}{rgb}{0.000000,0.000000,0.000000}%
\pgfsetstrokecolor{currentstroke}%
\pgfsetdash{}{0pt}%
\pgfsys@defobject{currentmarker}{\pgfqpoint{0.000000in}{0.000000in}}{\pgfqpoint{0.034722in}{0.000000in}}{%
\pgfpathmoveto{\pgfqpoint{0.000000in}{0.000000in}}%
\pgfpathlineto{\pgfqpoint{0.034722in}{0.000000in}}%
\pgfusepath{stroke,fill}%
}%
\begin{pgfscope}%
\pgfsys@transformshift{1.396958in}{7.732368in}%
\pgfsys@useobject{currentmarker}{}%
\end{pgfscope}%
\end{pgfscope}%
\begin{pgfscope}%
\definecolor{textcolor}{rgb}{0.000000,0.000000,0.000000}%
\pgfsetstrokecolor{textcolor}%
\pgfsetfillcolor{textcolor}%
\pgftext[x=1.062934in, y=7.637397in, left, base]{\color{textcolor}\sffamily\fontsize{18.000000}{21.600000}\selectfont $\displaystyle 1.0$}%
\end{pgfscope}%
\begin{pgfscope}%
\pgfpathrectangle{\pgfqpoint{1.396958in}{1.247073in}}{\pgfqpoint{10.524301in}{6.674186in}}%
\pgfusepath{clip}%
\pgfsetbuttcap%
\pgfsetroundjoin%
\pgfsetlinewidth{1.003750pt}%
\definecolor{currentstroke}{rgb}{0.000000,0.605603,0.978680}%
\pgfsetstrokecolor{currentstroke}%
\pgfsetdash{}{0pt}%
\pgfpathmoveto{\pgfqpoint{1.694816in}{4.584175in}}%
\pgfpathlineto{\pgfqpoint{1.956605in}{4.064996in}}%
\pgfpathlineto{\pgfqpoint{2.092347in}{3.800501in}}%
\pgfpathlineto{\pgfqpoint{2.208698in}{3.578318in}}%
\pgfpathlineto{\pgfqpoint{2.315353in}{3.379410in}}%
\pgfpathlineto{\pgfqpoint{2.412312in}{3.203322in}}%
\pgfpathlineto{\pgfqpoint{2.499575in}{3.049268in}}%
\pgfpathlineto{\pgfqpoint{2.586837in}{2.899894in}}%
\pgfpathlineto{\pgfqpoint{2.664405in}{2.771414in}}%
\pgfpathlineto{\pgfqpoint{2.741972in}{2.647301in}}%
\pgfpathlineto{\pgfqpoint{2.809843in}{2.542520in}}%
\pgfpathlineto{\pgfqpoint{2.877714in}{2.441504in}}%
\pgfpathlineto{\pgfqpoint{2.945585in}{2.344441in}}%
\pgfpathlineto{\pgfqpoint{3.013456in}{2.251509in}}%
\pgfpathlineto{\pgfqpoint{3.071632in}{2.175272in}}%
\pgfpathlineto{\pgfqpoint{3.129807in}{2.102299in}}%
\pgfpathlineto{\pgfqpoint{3.187982in}{2.032690in}}%
\pgfpathlineto{\pgfqpoint{3.246158in}{1.966538in}}%
\pgfpathlineto{\pgfqpoint{3.304333in}{1.903934in}}%
\pgfpathlineto{\pgfqpoint{3.362508in}{1.844962in}}%
\pgfpathlineto{\pgfqpoint{3.410988in}{1.798652in}}%
\pgfpathlineto{\pgfqpoint{3.459467in}{1.754963in}}%
\pgfpathlineto{\pgfqpoint{3.507947in}{1.713936in}}%
\pgfpathlineto{\pgfqpoint{3.556426in}{1.675612in}}%
\pgfpathlineto{\pgfqpoint{3.604905in}{1.640024in}}%
\pgfpathlineto{\pgfqpoint{3.653385in}{1.607208in}}%
\pgfpathlineto{\pgfqpoint{3.701864in}{1.577193in}}%
\pgfpathlineto{\pgfqpoint{3.750344in}{1.550008in}}%
\pgfpathlineto{\pgfqpoint{3.798823in}{1.525679in}}%
\pgfpathlineto{\pgfqpoint{3.847303in}{1.504229in}}%
\pgfpathlineto{\pgfqpoint{3.895782in}{1.485677in}}%
\pgfpathlineto{\pgfqpoint{3.944261in}{1.470041in}}%
\pgfpathlineto{\pgfqpoint{3.983045in}{1.459643in}}%
\pgfpathlineto{\pgfqpoint{4.021828in}{1.451126in}}%
\pgfpathlineto{\pgfqpoint{4.060612in}{1.444496in}}%
\pgfpathlineto{\pgfqpoint{4.099396in}{1.439758in}}%
\pgfpathlineto{\pgfqpoint{4.138179in}{1.436914in}}%
\pgfpathlineto{\pgfqpoint{4.176963in}{1.435966in}}%
\pgfpathlineto{\pgfqpoint{4.215746in}{1.436914in}}%
\pgfpathlineto{\pgfqpoint{4.254530in}{1.439757in}}%
\pgfpathlineto{\pgfqpoint{4.293313in}{1.444495in}}%
\pgfpathlineto{\pgfqpoint{4.332097in}{1.451124in}}%
\pgfpathlineto{\pgfqpoint{4.370880in}{1.459641in}}%
\pgfpathlineto{\pgfqpoint{4.409664in}{1.470039in}}%
\pgfpathlineto{\pgfqpoint{4.448447in}{1.482313in}}%
\pgfpathlineto{\pgfqpoint{4.496927in}{1.500282in}}%
\pgfpathlineto{\pgfqpoint{4.545406in}{1.521154in}}%
\pgfpathlineto{\pgfqpoint{4.593886in}{1.544909in}}%
\pgfpathlineto{\pgfqpoint{4.642365in}{1.571524in}}%
\pgfpathlineto{\pgfqpoint{4.690845in}{1.600974in}}%
\pgfpathlineto{\pgfqpoint{4.739324in}{1.633232in}}%
\pgfpathlineto{\pgfqpoint{4.787803in}{1.668267in}}%
\pgfpathlineto{\pgfqpoint{4.836283in}{1.706047in}}%
\pgfpathlineto{\pgfqpoint{4.884762in}{1.746536in}}%
\pgfpathlineto{\pgfqpoint{4.933242in}{1.789695in}}%
\pgfpathlineto{\pgfqpoint{4.981721in}{1.835484in}}%
\pgfpathlineto{\pgfqpoint{5.039896in}{1.893842in}}%
\pgfpathlineto{\pgfqpoint{5.098072in}{1.955846in}}%
\pgfpathlineto{\pgfqpoint{5.156247in}{2.021412in}}%
\pgfpathlineto{\pgfqpoint{5.214422in}{2.090451in}}%
\pgfpathlineto{\pgfqpoint{5.272598in}{2.162869in}}%
\pgfpathlineto{\pgfqpoint{5.330773in}{2.238569in}}%
\pgfpathlineto{\pgfqpoint{5.398644in}{2.330896in}}%
\pgfpathlineto{\pgfqpoint{5.466515in}{2.427379in}}%
\pgfpathlineto{\pgfqpoint{5.534387in}{2.527841in}}%
\pgfpathlineto{\pgfqpoint{5.602258in}{2.632095in}}%
\pgfpathlineto{\pgfqpoint{5.679825in}{2.755641in}}%
\pgfpathlineto{\pgfqpoint{5.757392in}{2.883591in}}%
\pgfpathlineto{\pgfqpoint{5.834959in}{3.015638in}}%
\pgfpathlineto{\pgfqpoint{5.922222in}{3.168692in}}%
\pgfpathlineto{\pgfqpoint{6.009485in}{3.326061in}}%
\pgfpathlineto{\pgfqpoint{6.106444in}{3.505394in}}%
\pgfpathlineto{\pgfqpoint{6.213098in}{3.707322in}}%
\pgfpathlineto{\pgfqpoint{6.339145in}{3.951040in}}%
\pgfpathlineto{\pgfqpoint{6.494279in}{4.256362in}}%
\pgfpathlineto{\pgfqpoint{7.037249in}{5.330356in}}%
\pgfpathlineto{\pgfqpoint{7.153599in}{5.553331in}}%
\pgfpathlineto{\pgfqpoint{7.260254in}{5.753140in}}%
\pgfpathlineto{\pgfqpoint{7.357213in}{5.930188in}}%
\pgfpathlineto{\pgfqpoint{7.444476in}{6.085219in}}%
\pgfpathlineto{\pgfqpoint{7.531739in}{6.235673in}}%
\pgfpathlineto{\pgfqpoint{7.609306in}{6.365197in}}%
\pgfpathlineto{\pgfqpoint{7.686873in}{6.490430in}}%
\pgfpathlineto{\pgfqpoint{7.764440in}{6.611071in}}%
\pgfpathlineto{\pgfqpoint{7.832311in}{6.712636in}}%
\pgfpathlineto{\pgfqpoint{7.900182in}{6.810275in}}%
\pgfpathlineto{\pgfqpoint{7.968054in}{6.903807in}}%
\pgfpathlineto{\pgfqpoint{8.026229in}{6.980579in}}%
\pgfpathlineto{\pgfqpoint{8.084404in}{7.054103in}}%
\pgfpathlineto{\pgfqpoint{8.142580in}{7.124280in}}%
\pgfpathlineto{\pgfqpoint{8.200755in}{7.191014in}}%
\pgfpathlineto{\pgfqpoint{8.258930in}{7.254215in}}%
\pgfpathlineto{\pgfqpoint{8.317105in}{7.313798in}}%
\pgfpathlineto{\pgfqpoint{8.375281in}{7.369682in}}%
\pgfpathlineto{\pgfqpoint{8.423760in}{7.413371in}}%
\pgfpathlineto{\pgfqpoint{8.472240in}{7.454397in}}%
\pgfpathlineto{\pgfqpoint{8.520719in}{7.492722in}}%
\pgfpathlineto{\pgfqpoint{8.569198in}{7.528309in}}%
\pgfpathlineto{\pgfqpoint{8.617678in}{7.561126in}}%
\pgfpathlineto{\pgfqpoint{8.666157in}{7.591140in}}%
\pgfpathlineto{\pgfqpoint{8.714637in}{7.618325in}}%
\pgfpathlineto{\pgfqpoint{8.763116in}{7.642654in}}%
\pgfpathlineto{\pgfqpoint{8.811596in}{7.664105in}}%
\pgfpathlineto{\pgfqpoint{8.860075in}{7.682656in}}%
\pgfpathlineto{\pgfqpoint{8.908554in}{7.698292in}}%
\pgfpathlineto{\pgfqpoint{8.947338in}{7.708691in}}%
\pgfpathlineto{\pgfqpoint{8.986122in}{7.717207in}}%
\pgfpathlineto{\pgfqpoint{9.024905in}{7.723837in}}%
\pgfpathlineto{\pgfqpoint{9.063689in}{7.728575in}}%
\pgfpathlineto{\pgfqpoint{9.102472in}{7.731419in}}%
\pgfpathlineto{\pgfqpoint{9.141256in}{7.732368in}}%
\pgfpathlineto{\pgfqpoint{9.180039in}{7.731420in}}%
\pgfpathlineto{\pgfqpoint{9.218823in}{7.728576in}}%
\pgfpathlineto{\pgfqpoint{9.257606in}{7.723838in}}%
\pgfpathlineto{\pgfqpoint{9.296390in}{7.717209in}}%
\pgfpathlineto{\pgfqpoint{9.335173in}{7.708693in}}%
\pgfpathlineto{\pgfqpoint{9.373957in}{7.698294in}}%
\pgfpathlineto{\pgfqpoint{9.412740in}{7.686020in}}%
\pgfpathlineto{\pgfqpoint{9.461220in}{7.668051in}}%
\pgfpathlineto{\pgfqpoint{9.509699in}{7.647179in}}%
\pgfpathlineto{\pgfqpoint{9.558179in}{7.623425in}}%
\pgfpathlineto{\pgfqpoint{9.606658in}{7.596810in}}%
\pgfpathlineto{\pgfqpoint{9.655138in}{7.567359in}}%
\pgfpathlineto{\pgfqpoint{9.703617in}{7.535101in}}%
\pgfpathlineto{\pgfqpoint{9.752096in}{7.500066in}}%
\pgfpathlineto{\pgfqpoint{9.800576in}{7.462286in}}%
\pgfpathlineto{\pgfqpoint{9.849055in}{7.421798in}}%
\pgfpathlineto{\pgfqpoint{9.897535in}{7.378639in}}%
\pgfpathlineto{\pgfqpoint{9.946014in}{7.332849in}}%
\pgfpathlineto{\pgfqpoint{10.004189in}{7.274491in}}%
\pgfpathlineto{\pgfqpoint{10.062365in}{7.212488in}}%
\pgfpathlineto{\pgfqpoint{10.120540in}{7.146922in}}%
\pgfpathlineto{\pgfqpoint{10.178715in}{7.077883in}}%
\pgfpathlineto{\pgfqpoint{10.236891in}{7.005464in}}%
\pgfpathlineto{\pgfqpoint{10.295066in}{6.929764in}}%
\pgfpathlineto{\pgfqpoint{10.362937in}{6.837437in}}%
\pgfpathlineto{\pgfqpoint{10.430808in}{6.740954in}}%
\pgfpathlineto{\pgfqpoint{10.498680in}{6.640493in}}%
\pgfpathlineto{\pgfqpoint{10.566551in}{6.536238in}}%
\pgfpathlineto{\pgfqpoint{10.644118in}{6.412693in}}%
\pgfpathlineto{\pgfqpoint{10.721685in}{6.284742in}}%
\pgfpathlineto{\pgfqpoint{10.799252in}{6.152695in}}%
\pgfpathlineto{\pgfqpoint{10.886515in}{5.999641in}}%
\pgfpathlineto{\pgfqpoint{10.973778in}{5.842272in}}%
\pgfpathlineto{\pgfqpoint{11.070737in}{5.662940in}}%
\pgfpathlineto{\pgfqpoint{11.177392in}{5.461011in}}%
\pgfpathlineto{\pgfqpoint{11.303438in}{5.217293in}}%
\pgfpathlineto{\pgfqpoint{11.458572in}{4.911971in}}%
\pgfpathlineto{\pgfqpoint{11.623402in}{4.584175in}}%
\pgfpathlineto{\pgfqpoint{11.623402in}{4.584175in}}%
\pgfusepath{stroke}%
\end{pgfscope}%
\begin{pgfscope}%
\pgfpathrectangle{\pgfqpoint{1.396958in}{1.247073in}}{\pgfqpoint{10.524301in}{6.674186in}}%
\pgfusepath{clip}%
\pgfsetbuttcap%
\pgfsetroundjoin%
\pgfsetlinewidth{1.003750pt}%
\definecolor{currentstroke}{rgb}{0.888874,0.435649,0.278123}%
\pgfsetstrokecolor{currentstroke}%
\pgfsetdash{}{0pt}%
\pgfpathmoveto{\pgfqpoint{1.694816in}{4.584167in}}%
\pgfpathlineto{\pgfqpoint{1.956605in}{4.064987in}}%
\pgfpathlineto{\pgfqpoint{2.092347in}{3.800493in}}%
\pgfpathlineto{\pgfqpoint{2.208698in}{3.578310in}}%
\pgfpathlineto{\pgfqpoint{2.315353in}{3.379402in}}%
\pgfpathlineto{\pgfqpoint{2.412312in}{3.203315in}}%
\pgfpathlineto{\pgfqpoint{2.499575in}{3.049261in}}%
\pgfpathlineto{\pgfqpoint{2.586837in}{2.899887in}}%
\pgfpathlineto{\pgfqpoint{2.664405in}{2.771407in}}%
\pgfpathlineto{\pgfqpoint{2.741972in}{2.647294in}}%
\pgfpathlineto{\pgfqpoint{2.809843in}{2.542513in}}%
\pgfpathlineto{\pgfqpoint{2.877714in}{2.441498in}}%
\pgfpathlineto{\pgfqpoint{2.945585in}{2.344435in}}%
\pgfpathlineto{\pgfqpoint{3.013456in}{2.251504in}}%
\pgfpathlineto{\pgfqpoint{3.071632in}{2.175266in}}%
\pgfpathlineto{\pgfqpoint{3.129807in}{2.102294in}}%
\pgfpathlineto{\pgfqpoint{3.187982in}{2.032684in}}%
\pgfpathlineto{\pgfqpoint{3.246158in}{1.966533in}}%
\pgfpathlineto{\pgfqpoint{3.304333in}{1.903929in}}%
\pgfpathlineto{\pgfqpoint{3.362508in}{1.844958in}}%
\pgfpathlineto{\pgfqpoint{3.410988in}{1.798647in}}%
\pgfpathlineto{\pgfqpoint{3.459467in}{1.754959in}}%
\pgfpathlineto{\pgfqpoint{3.507947in}{1.713933in}}%
\pgfpathlineto{\pgfqpoint{3.556426in}{1.675608in}}%
\pgfpathlineto{\pgfqpoint{3.604905in}{1.640021in}}%
\pgfpathlineto{\pgfqpoint{3.653385in}{1.607205in}}%
\pgfpathlineto{\pgfqpoint{3.701864in}{1.577190in}}%
\pgfpathlineto{\pgfqpoint{3.750344in}{1.550006in}}%
\pgfpathlineto{\pgfqpoint{3.798823in}{1.525677in}}%
\pgfpathlineto{\pgfqpoint{3.847303in}{1.504227in}}%
\pgfpathlineto{\pgfqpoint{3.895782in}{1.485675in}}%
\pgfpathlineto{\pgfqpoint{3.944261in}{1.470040in}}%
\pgfpathlineto{\pgfqpoint{3.983045in}{1.459641in}}%
\pgfpathlineto{\pgfqpoint{4.021828in}{1.451125in}}%
\pgfpathlineto{\pgfqpoint{4.060612in}{1.444496in}}%
\pgfpathlineto{\pgfqpoint{4.099396in}{1.439758in}}%
\pgfpathlineto{\pgfqpoint{4.138179in}{1.436914in}}%
\pgfpathlineto{\pgfqpoint{4.176963in}{1.435966in}}%
\pgfpathlineto{\pgfqpoint{4.215746in}{1.436914in}}%
\pgfpathlineto{\pgfqpoint{4.254530in}{1.439758in}}%
\pgfpathlineto{\pgfqpoint{4.293313in}{1.444496in}}%
\pgfpathlineto{\pgfqpoint{4.332097in}{1.451125in}}%
\pgfpathlineto{\pgfqpoint{4.370880in}{1.459641in}}%
\pgfpathlineto{\pgfqpoint{4.409664in}{1.470040in}}%
\pgfpathlineto{\pgfqpoint{4.448447in}{1.482314in}}%
\pgfpathlineto{\pgfqpoint{4.496927in}{1.500284in}}%
\pgfpathlineto{\pgfqpoint{4.545406in}{1.521156in}}%
\pgfpathlineto{\pgfqpoint{4.593886in}{1.544911in}}%
\pgfpathlineto{\pgfqpoint{4.642365in}{1.571526in}}%
\pgfpathlineto{\pgfqpoint{4.690845in}{1.600977in}}%
\pgfpathlineto{\pgfqpoint{4.739324in}{1.633235in}}%
\pgfpathlineto{\pgfqpoint{4.787803in}{1.668271in}}%
\pgfpathlineto{\pgfqpoint{4.836283in}{1.706050in}}%
\pgfpathlineto{\pgfqpoint{4.884762in}{1.746539in}}%
\pgfpathlineto{\pgfqpoint{4.933242in}{1.789699in}}%
\pgfpathlineto{\pgfqpoint{4.981721in}{1.835488in}}%
\pgfpathlineto{\pgfqpoint{5.039896in}{1.893846in}}%
\pgfpathlineto{\pgfqpoint{5.098072in}{1.955850in}}%
\pgfpathlineto{\pgfqpoint{5.156247in}{2.021417in}}%
\pgfpathlineto{\pgfqpoint{5.214422in}{2.090456in}}%
\pgfpathlineto{\pgfqpoint{5.272598in}{2.162875in}}%
\pgfpathlineto{\pgfqpoint{5.330773in}{2.238575in}}%
\pgfpathlineto{\pgfqpoint{5.398644in}{2.330902in}}%
\pgfpathlineto{\pgfqpoint{5.466515in}{2.427386in}}%
\pgfpathlineto{\pgfqpoint{5.534387in}{2.527847in}}%
\pgfpathlineto{\pgfqpoint{5.602258in}{2.632102in}}%
\pgfpathlineto{\pgfqpoint{5.679825in}{2.755648in}}%
\pgfpathlineto{\pgfqpoint{5.757392in}{2.883598in}}%
\pgfpathlineto{\pgfqpoint{5.834959in}{3.015646in}}%
\pgfpathlineto{\pgfqpoint{5.922222in}{3.168700in}}%
\pgfpathlineto{\pgfqpoint{6.009485in}{3.326069in}}%
\pgfpathlineto{\pgfqpoint{6.106444in}{3.505402in}}%
\pgfpathlineto{\pgfqpoint{6.213098in}{3.707331in}}%
\pgfpathlineto{\pgfqpoint{6.339145in}{3.951049in}}%
\pgfpathlineto{\pgfqpoint{6.494279in}{4.256371in}}%
\pgfpathlineto{\pgfqpoint{7.037249in}{5.330365in}}%
\pgfpathlineto{\pgfqpoint{7.153599in}{5.553339in}}%
\pgfpathlineto{\pgfqpoint{7.260254in}{5.753148in}}%
\pgfpathlineto{\pgfqpoint{7.357213in}{5.930196in}}%
\pgfpathlineto{\pgfqpoint{7.444476in}{6.085227in}}%
\pgfpathlineto{\pgfqpoint{7.531739in}{6.235680in}}%
\pgfpathlineto{\pgfqpoint{7.609306in}{6.365204in}}%
\pgfpathlineto{\pgfqpoint{7.686873in}{6.490437in}}%
\pgfpathlineto{\pgfqpoint{7.764440in}{6.611078in}}%
\pgfpathlineto{\pgfqpoint{7.832311in}{6.712642in}}%
\pgfpathlineto{\pgfqpoint{7.900182in}{6.810281in}}%
\pgfpathlineto{\pgfqpoint{7.968054in}{6.903813in}}%
\pgfpathlineto{\pgfqpoint{8.026229in}{6.980585in}}%
\pgfpathlineto{\pgfqpoint{8.084404in}{7.054109in}}%
\pgfpathlineto{\pgfqpoint{8.142580in}{7.124285in}}%
\pgfpathlineto{\pgfqpoint{8.200755in}{7.191019in}}%
\pgfpathlineto{\pgfqpoint{8.258930in}{7.254220in}}%
\pgfpathlineto{\pgfqpoint{8.317105in}{7.313803in}}%
\pgfpathlineto{\pgfqpoint{8.375281in}{7.369686in}}%
\pgfpathlineto{\pgfqpoint{8.423760in}{7.413375in}}%
\pgfpathlineto{\pgfqpoint{8.472240in}{7.454401in}}%
\pgfpathlineto{\pgfqpoint{8.520719in}{7.492725in}}%
\pgfpathlineto{\pgfqpoint{8.569198in}{7.528312in}}%
\pgfpathlineto{\pgfqpoint{8.617678in}{7.561129in}}%
\pgfpathlineto{\pgfqpoint{8.666157in}{7.591143in}}%
\pgfpathlineto{\pgfqpoint{8.714637in}{7.618328in}}%
\pgfpathlineto{\pgfqpoint{8.763116in}{7.642656in}}%
\pgfpathlineto{\pgfqpoint{8.811596in}{7.664106in}}%
\pgfpathlineto{\pgfqpoint{8.860075in}{7.682658in}}%
\pgfpathlineto{\pgfqpoint{8.908554in}{7.698293in}}%
\pgfpathlineto{\pgfqpoint{8.947338in}{7.708692in}}%
\pgfpathlineto{\pgfqpoint{8.986122in}{7.717208in}}%
\pgfpathlineto{\pgfqpoint{9.024905in}{7.723838in}}%
\pgfpathlineto{\pgfqpoint{9.063689in}{7.728576in}}%
\pgfpathlineto{\pgfqpoint{9.102472in}{7.731420in}}%
\pgfpathlineto{\pgfqpoint{9.141256in}{7.732368in}}%
\pgfpathlineto{\pgfqpoint{9.180039in}{7.731420in}}%
\pgfpathlineto{\pgfqpoint{9.218823in}{7.728576in}}%
\pgfpathlineto{\pgfqpoint{9.257606in}{7.723838in}}%
\pgfpathlineto{\pgfqpoint{9.296390in}{7.717208in}}%
\pgfpathlineto{\pgfqpoint{9.335173in}{7.708692in}}%
\pgfpathlineto{\pgfqpoint{9.373957in}{7.698293in}}%
\pgfpathlineto{\pgfqpoint{9.412740in}{7.686019in}}%
\pgfpathlineto{\pgfqpoint{9.461220in}{7.668049in}}%
\pgfpathlineto{\pgfqpoint{9.509699in}{7.647177in}}%
\pgfpathlineto{\pgfqpoint{9.558179in}{7.623423in}}%
\pgfpathlineto{\pgfqpoint{9.606658in}{7.596807in}}%
\pgfpathlineto{\pgfqpoint{9.655138in}{7.567357in}}%
\pgfpathlineto{\pgfqpoint{9.703617in}{7.535098in}}%
\pgfpathlineto{\pgfqpoint{9.752096in}{7.500063in}}%
\pgfpathlineto{\pgfqpoint{9.800576in}{7.462283in}}%
\pgfpathlineto{\pgfqpoint{9.849055in}{7.421794in}}%
\pgfpathlineto{\pgfqpoint{9.897535in}{7.378635in}}%
\pgfpathlineto{\pgfqpoint{9.946014in}{7.332845in}}%
\pgfpathlineto{\pgfqpoint{10.004189in}{7.274487in}}%
\pgfpathlineto{\pgfqpoint{10.062365in}{7.212483in}}%
\pgfpathlineto{\pgfqpoint{10.120540in}{7.146917in}}%
\pgfpathlineto{\pgfqpoint{10.178715in}{7.077877in}}%
\pgfpathlineto{\pgfqpoint{10.236891in}{7.005459in}}%
\pgfpathlineto{\pgfqpoint{10.295066in}{6.929758in}}%
\pgfpathlineto{\pgfqpoint{10.362937in}{6.837431in}}%
\pgfpathlineto{\pgfqpoint{10.430808in}{6.740948in}}%
\pgfpathlineto{\pgfqpoint{10.498680in}{6.640486in}}%
\pgfpathlineto{\pgfqpoint{10.566551in}{6.536231in}}%
\pgfpathlineto{\pgfqpoint{10.644118in}{6.412686in}}%
\pgfpathlineto{\pgfqpoint{10.721685in}{6.284735in}}%
\pgfpathlineto{\pgfqpoint{10.799252in}{6.152688in}}%
\pgfpathlineto{\pgfqpoint{10.886515in}{5.999634in}}%
\pgfpathlineto{\pgfqpoint{10.973778in}{5.842264in}}%
\pgfpathlineto{\pgfqpoint{11.070737in}{5.662932in}}%
\pgfpathlineto{\pgfqpoint{11.177392in}{5.461003in}}%
\pgfpathlineto{\pgfqpoint{11.303438in}{5.217285in}}%
\pgfpathlineto{\pgfqpoint{11.458572in}{4.911963in}}%
\pgfpathlineto{\pgfqpoint{11.623402in}{4.584167in}}%
\pgfpathlineto{\pgfqpoint{11.623402in}{4.584167in}}%
\pgfusepath{stroke}%
\end{pgfscope}%
\begin{pgfscope}%
\pgfsetrectcap%
\pgfsetmiterjoin%
\pgfsetlinewidth{1.003750pt}%
\definecolor{currentstroke}{rgb}{0.000000,0.000000,0.000000}%
\pgfsetstrokecolor{currentstroke}%
\pgfsetdash{}{0pt}%
\pgfpathmoveto{\pgfqpoint{1.396958in}{1.247073in}}%
\pgfpathlineto{\pgfqpoint{1.396958in}{7.921260in}}%
\pgfusepath{stroke}%
\end{pgfscope}%
\begin{pgfscope}%
\pgfsetrectcap%
\pgfsetmiterjoin%
\pgfsetlinewidth{1.003750pt}%
\definecolor{currentstroke}{rgb}{0.000000,0.000000,0.000000}%
\pgfsetstrokecolor{currentstroke}%
\pgfsetdash{}{0pt}%
\pgfpathmoveto{\pgfqpoint{1.396958in}{1.247073in}}%
\pgfpathlineto{\pgfqpoint{11.921260in}{1.247073in}}%
\pgfusepath{stroke}%
\end{pgfscope}%
\begin{pgfscope}%
\pgfsetbuttcap%
\pgfsetmiterjoin%
\definecolor{currentfill}{rgb}{1.000000,1.000000,1.000000}%
\pgfsetfillcolor{currentfill}%
\pgfsetlinewidth{1.003750pt}%
\definecolor{currentstroke}{rgb}{0.000000,0.000000,0.000000}%
\pgfsetstrokecolor{currentstroke}%
\pgfsetdash{}{0pt}%
\pgfpathmoveto{\pgfqpoint{10.511589in}{6.787373in}}%
\pgfpathlineto{\pgfqpoint{11.796260in}{6.787373in}}%
\pgfpathlineto{\pgfqpoint{11.796260in}{7.796260in}}%
\pgfpathlineto{\pgfqpoint{10.511589in}{7.796260in}}%
\pgfpathclose%
\pgfusepath{stroke,fill}%
\end{pgfscope}%
\begin{pgfscope}%
\pgfsetbuttcap%
\pgfsetmiterjoin%
\pgfsetlinewidth{2.258437pt}%
\definecolor{currentstroke}{rgb}{0.000000,0.605603,0.978680}%
\pgfsetstrokecolor{currentstroke}%
\pgfsetdash{}{0pt}%
\pgfpathmoveto{\pgfqpoint{10.711589in}{7.493818in}}%
\pgfpathlineto{\pgfqpoint{11.211589in}{7.493818in}}%
\pgfusepath{stroke}%
\end{pgfscope}%
\begin{pgfscope}%
\definecolor{textcolor}{rgb}{0.000000,0.000000,0.000000}%
\pgfsetstrokecolor{textcolor}%
\pgfsetfillcolor{textcolor}%
\pgftext[x=11.411589in,y=7.406318in,left,base]{\color{textcolor}\sffamily\fontsize{18.000000}{21.600000}\selectfont $\displaystyle U$}%
\end{pgfscope}%
\begin{pgfscope}%
\pgfsetbuttcap%
\pgfsetmiterjoin%
\pgfsetlinewidth{2.258437pt}%
\definecolor{currentstroke}{rgb}{0.888874,0.435649,0.278123}%
\pgfsetstrokecolor{currentstroke}%
\pgfsetdash{}{0pt}%
\pgfpathmoveto{\pgfqpoint{10.711589in}{7.126875in}}%
\pgfpathlineto{\pgfqpoint{11.211589in}{7.126875in}}%
\pgfusepath{stroke}%
\end{pgfscope}%
\begin{pgfscope}%
\definecolor{textcolor}{rgb}{0.000000,0.000000,0.000000}%
\pgfsetstrokecolor{textcolor}%
\pgfsetfillcolor{textcolor}%
\pgftext[x=11.411589in,y=7.039375in,left,base]{\color{textcolor}\sffamily\fontsize{18.000000}{21.600000}\selectfont $\displaystyle u$}%
\end{pgfscope}%
\end{pgfpicture}%
\makeatother%
\endgroup%
}
	\caption{Beam-Warming 格式差分逼近解 $U$ 与真解 $u$}\label{fig:beam_warming_Uu}
\end{figure}

取 $\nu = 4 \geq 2$, 不满足 CFL 条件. $h = 2^{-7}$ 和 $h = 2^{-11}$ 时差分逼近解 $U$ 与真解 $u$ 在 $t = t_{\max }$ 时刻图像如图 \ref{fig:beam_warming_Uu_noCFL} 所示. 可以看到出现了错误解.

\begin{figure}[H]\centering\zihao{-5}
	\resizebox{0.4\linewidth}{!}{%% Creator: Matplotlib, PGF backend
%%
%% To include the figure in your LaTeX document, write
%%   \input{<filename>.pgf}
%%
%% Make sure the required packages are loaded in your preamble
%%   \usepackage{pgf}
%%
%% Figures using additional raster images can only be included by \input if
%% they are in the same directory as the main LaTeX file. For loading figures
%% from other directories you can use the `import` package
%%   \usepackage{import}
%%
%% and then include the figures with
%%   \import{<path to file>}{<filename>.pgf}
%%
%% Matplotlib used the following preamble
%%   \usepackage{fontspec}
%%   \setmainfont{DejaVuSerif.ttf}[Path=\detokenize{/Users/quejiahao/.julia/conda/3/lib/python3.9/site-packages/matplotlib/mpl-data/fonts/ttf/}]
%%   \setsansfont{DejaVuSans.ttf}[Path=\detokenize{/Users/quejiahao/.julia/conda/3/lib/python3.9/site-packages/matplotlib/mpl-data/fonts/ttf/}]
%%   \setmonofont{DejaVuSansMono.ttf}[Path=\detokenize{/Users/quejiahao/.julia/conda/3/lib/python3.9/site-packages/matplotlib/mpl-data/fonts/ttf/}]
%%
\begingroup%
\makeatletter%
\begin{pgfpicture}%
\pgfpathrectangle{\pgfpointorigin}{\pgfqpoint{12.000000in}{8.000000in}}%
\pgfusepath{use as bounding box, clip}%
\begin{pgfscope}%
\pgfsetbuttcap%
\pgfsetmiterjoin%
\definecolor{currentfill}{rgb}{1.000000,1.000000,1.000000}%
\pgfsetfillcolor{currentfill}%
\pgfsetlinewidth{0.000000pt}%
\definecolor{currentstroke}{rgb}{1.000000,1.000000,1.000000}%
\pgfsetstrokecolor{currentstroke}%
\pgfsetdash{}{0pt}%
\pgfpathmoveto{\pgfqpoint{0.000000in}{0.000000in}}%
\pgfpathlineto{\pgfqpoint{12.000000in}{0.000000in}}%
\pgfpathlineto{\pgfqpoint{12.000000in}{8.000000in}}%
\pgfpathlineto{\pgfqpoint{0.000000in}{8.000000in}}%
\pgfpathclose%
\pgfusepath{fill}%
\end{pgfscope}%
\begin{pgfscope}%
\pgfsetbuttcap%
\pgfsetmiterjoin%
\definecolor{currentfill}{rgb}{1.000000,1.000000,1.000000}%
\pgfsetfillcolor{currentfill}%
\pgfsetlinewidth{0.000000pt}%
\definecolor{currentstroke}{rgb}{0.000000,0.000000,0.000000}%
\pgfsetstrokecolor{currentstroke}%
\pgfsetstrokeopacity{0.000000}%
\pgfsetdash{}{0pt}%
\pgfpathmoveto{\pgfqpoint{1.874253in}{1.247073in}}%
\pgfpathlineto{\pgfqpoint{11.921260in}{1.247073in}}%
\pgfpathlineto{\pgfqpoint{11.921260in}{7.921260in}}%
\pgfpathlineto{\pgfqpoint{1.874253in}{7.921260in}}%
\pgfpathclose%
\pgfusepath{fill}%
\end{pgfscope}%
\begin{pgfscope}%
\pgfpathrectangle{\pgfqpoint{1.874253in}{1.247073in}}{\pgfqpoint{10.047006in}{6.674186in}}%
\pgfusepath{clip}%
\pgfsetrectcap%
\pgfsetroundjoin%
\pgfsetlinewidth{0.501875pt}%
\definecolor{currentstroke}{rgb}{0.000000,0.000000,0.000000}%
\pgfsetstrokecolor{currentstroke}%
\pgfsetstrokeopacity{0.100000}%
\pgfsetdash{}{0pt}%
\pgfpathmoveto{\pgfqpoint{2.158603in}{1.247073in}}%
\pgfpathlineto{\pgfqpoint{2.158603in}{7.921260in}}%
\pgfusepath{stroke}%
\end{pgfscope}%
\begin{pgfscope}%
\pgfsetbuttcap%
\pgfsetroundjoin%
\definecolor{currentfill}{rgb}{0.000000,0.000000,0.000000}%
\pgfsetfillcolor{currentfill}%
\pgfsetlinewidth{0.501875pt}%
\definecolor{currentstroke}{rgb}{0.000000,0.000000,0.000000}%
\pgfsetstrokecolor{currentstroke}%
\pgfsetdash{}{0pt}%
\pgfsys@defobject{currentmarker}{\pgfqpoint{0.000000in}{0.000000in}}{\pgfqpoint{0.000000in}{0.034722in}}{%
\pgfpathmoveto{\pgfqpoint{0.000000in}{0.000000in}}%
\pgfpathlineto{\pgfqpoint{0.000000in}{0.034722in}}%
\pgfusepath{stroke,fill}%
}%
\begin{pgfscope}%
\pgfsys@transformshift{2.158603in}{1.247073in}%
\pgfsys@useobject{currentmarker}{}%
\end{pgfscope}%
\end{pgfscope}%
\begin{pgfscope}%
\definecolor{textcolor}{rgb}{0.000000,0.000000,0.000000}%
\pgfsetstrokecolor{textcolor}%
\pgfsetfillcolor{textcolor}%
\pgftext[x=2.158603in,y=1.198462in,,top]{\color{textcolor}\sffamily\fontsize{18.000000}{21.600000}\selectfont $\displaystyle 0$}%
\end{pgfscope}%
\begin{pgfscope}%
\pgfpathrectangle{\pgfqpoint{1.874253in}{1.247073in}}{\pgfqpoint{10.047006in}{6.674186in}}%
\pgfusepath{clip}%
\pgfsetrectcap%
\pgfsetroundjoin%
\pgfsetlinewidth{0.501875pt}%
\definecolor{currentstroke}{rgb}{0.000000,0.000000,0.000000}%
\pgfsetstrokecolor{currentstroke}%
\pgfsetstrokeopacity{0.100000}%
\pgfsetdash{}{0pt}%
\pgfpathmoveto{\pgfqpoint{3.667122in}{1.247073in}}%
\pgfpathlineto{\pgfqpoint{3.667122in}{7.921260in}}%
\pgfusepath{stroke}%
\end{pgfscope}%
\begin{pgfscope}%
\pgfsetbuttcap%
\pgfsetroundjoin%
\definecolor{currentfill}{rgb}{0.000000,0.000000,0.000000}%
\pgfsetfillcolor{currentfill}%
\pgfsetlinewidth{0.501875pt}%
\definecolor{currentstroke}{rgb}{0.000000,0.000000,0.000000}%
\pgfsetstrokecolor{currentstroke}%
\pgfsetdash{}{0pt}%
\pgfsys@defobject{currentmarker}{\pgfqpoint{0.000000in}{0.000000in}}{\pgfqpoint{0.000000in}{0.034722in}}{%
\pgfpathmoveto{\pgfqpoint{0.000000in}{0.000000in}}%
\pgfpathlineto{\pgfqpoint{0.000000in}{0.034722in}}%
\pgfusepath{stroke,fill}%
}%
\begin{pgfscope}%
\pgfsys@transformshift{3.667122in}{1.247073in}%
\pgfsys@useobject{currentmarker}{}%
\end{pgfscope}%
\end{pgfscope}%
\begin{pgfscope}%
\definecolor{textcolor}{rgb}{0.000000,0.000000,0.000000}%
\pgfsetstrokecolor{textcolor}%
\pgfsetfillcolor{textcolor}%
\pgftext[x=3.667122in,y=1.198462in,,top]{\color{textcolor}\sffamily\fontsize{18.000000}{21.600000}\selectfont $\displaystyle 1$}%
\end{pgfscope}%
\begin{pgfscope}%
\pgfpathrectangle{\pgfqpoint{1.874253in}{1.247073in}}{\pgfqpoint{10.047006in}{6.674186in}}%
\pgfusepath{clip}%
\pgfsetrectcap%
\pgfsetroundjoin%
\pgfsetlinewidth{0.501875pt}%
\definecolor{currentstroke}{rgb}{0.000000,0.000000,0.000000}%
\pgfsetstrokecolor{currentstroke}%
\pgfsetstrokeopacity{0.100000}%
\pgfsetdash{}{0pt}%
\pgfpathmoveto{\pgfqpoint{5.175642in}{1.247073in}}%
\pgfpathlineto{\pgfqpoint{5.175642in}{7.921260in}}%
\pgfusepath{stroke}%
\end{pgfscope}%
\begin{pgfscope}%
\pgfsetbuttcap%
\pgfsetroundjoin%
\definecolor{currentfill}{rgb}{0.000000,0.000000,0.000000}%
\pgfsetfillcolor{currentfill}%
\pgfsetlinewidth{0.501875pt}%
\definecolor{currentstroke}{rgb}{0.000000,0.000000,0.000000}%
\pgfsetstrokecolor{currentstroke}%
\pgfsetdash{}{0pt}%
\pgfsys@defobject{currentmarker}{\pgfqpoint{0.000000in}{0.000000in}}{\pgfqpoint{0.000000in}{0.034722in}}{%
\pgfpathmoveto{\pgfqpoint{0.000000in}{0.000000in}}%
\pgfpathlineto{\pgfqpoint{0.000000in}{0.034722in}}%
\pgfusepath{stroke,fill}%
}%
\begin{pgfscope}%
\pgfsys@transformshift{5.175642in}{1.247073in}%
\pgfsys@useobject{currentmarker}{}%
\end{pgfscope}%
\end{pgfscope}%
\begin{pgfscope}%
\definecolor{textcolor}{rgb}{0.000000,0.000000,0.000000}%
\pgfsetstrokecolor{textcolor}%
\pgfsetfillcolor{textcolor}%
\pgftext[x=5.175642in,y=1.198462in,,top]{\color{textcolor}\sffamily\fontsize{18.000000}{21.600000}\selectfont $\displaystyle 2$}%
\end{pgfscope}%
\begin{pgfscope}%
\pgfpathrectangle{\pgfqpoint{1.874253in}{1.247073in}}{\pgfqpoint{10.047006in}{6.674186in}}%
\pgfusepath{clip}%
\pgfsetrectcap%
\pgfsetroundjoin%
\pgfsetlinewidth{0.501875pt}%
\definecolor{currentstroke}{rgb}{0.000000,0.000000,0.000000}%
\pgfsetstrokecolor{currentstroke}%
\pgfsetstrokeopacity{0.100000}%
\pgfsetdash{}{0pt}%
\pgfpathmoveto{\pgfqpoint{6.684161in}{1.247073in}}%
\pgfpathlineto{\pgfqpoint{6.684161in}{7.921260in}}%
\pgfusepath{stroke}%
\end{pgfscope}%
\begin{pgfscope}%
\pgfsetbuttcap%
\pgfsetroundjoin%
\definecolor{currentfill}{rgb}{0.000000,0.000000,0.000000}%
\pgfsetfillcolor{currentfill}%
\pgfsetlinewidth{0.501875pt}%
\definecolor{currentstroke}{rgb}{0.000000,0.000000,0.000000}%
\pgfsetstrokecolor{currentstroke}%
\pgfsetdash{}{0pt}%
\pgfsys@defobject{currentmarker}{\pgfqpoint{0.000000in}{0.000000in}}{\pgfqpoint{0.000000in}{0.034722in}}{%
\pgfpathmoveto{\pgfqpoint{0.000000in}{0.000000in}}%
\pgfpathlineto{\pgfqpoint{0.000000in}{0.034722in}}%
\pgfusepath{stroke,fill}%
}%
\begin{pgfscope}%
\pgfsys@transformshift{6.684161in}{1.247073in}%
\pgfsys@useobject{currentmarker}{}%
\end{pgfscope}%
\end{pgfscope}%
\begin{pgfscope}%
\definecolor{textcolor}{rgb}{0.000000,0.000000,0.000000}%
\pgfsetstrokecolor{textcolor}%
\pgfsetfillcolor{textcolor}%
\pgftext[x=6.684161in,y=1.198462in,,top]{\color{textcolor}\sffamily\fontsize{18.000000}{21.600000}\selectfont $\displaystyle 3$}%
\end{pgfscope}%
\begin{pgfscope}%
\pgfpathrectangle{\pgfqpoint{1.874253in}{1.247073in}}{\pgfqpoint{10.047006in}{6.674186in}}%
\pgfusepath{clip}%
\pgfsetrectcap%
\pgfsetroundjoin%
\pgfsetlinewidth{0.501875pt}%
\definecolor{currentstroke}{rgb}{0.000000,0.000000,0.000000}%
\pgfsetstrokecolor{currentstroke}%
\pgfsetstrokeopacity{0.100000}%
\pgfsetdash{}{0pt}%
\pgfpathmoveto{\pgfqpoint{8.192681in}{1.247073in}}%
\pgfpathlineto{\pgfqpoint{8.192681in}{7.921260in}}%
\pgfusepath{stroke}%
\end{pgfscope}%
\begin{pgfscope}%
\pgfsetbuttcap%
\pgfsetroundjoin%
\definecolor{currentfill}{rgb}{0.000000,0.000000,0.000000}%
\pgfsetfillcolor{currentfill}%
\pgfsetlinewidth{0.501875pt}%
\definecolor{currentstroke}{rgb}{0.000000,0.000000,0.000000}%
\pgfsetstrokecolor{currentstroke}%
\pgfsetdash{}{0pt}%
\pgfsys@defobject{currentmarker}{\pgfqpoint{0.000000in}{0.000000in}}{\pgfqpoint{0.000000in}{0.034722in}}{%
\pgfpathmoveto{\pgfqpoint{0.000000in}{0.000000in}}%
\pgfpathlineto{\pgfqpoint{0.000000in}{0.034722in}}%
\pgfusepath{stroke,fill}%
}%
\begin{pgfscope}%
\pgfsys@transformshift{8.192681in}{1.247073in}%
\pgfsys@useobject{currentmarker}{}%
\end{pgfscope}%
\end{pgfscope}%
\begin{pgfscope}%
\definecolor{textcolor}{rgb}{0.000000,0.000000,0.000000}%
\pgfsetstrokecolor{textcolor}%
\pgfsetfillcolor{textcolor}%
\pgftext[x=8.192681in,y=1.198462in,,top]{\color{textcolor}\sffamily\fontsize{18.000000}{21.600000}\selectfont $\displaystyle 4$}%
\end{pgfscope}%
\begin{pgfscope}%
\pgfpathrectangle{\pgfqpoint{1.874253in}{1.247073in}}{\pgfqpoint{10.047006in}{6.674186in}}%
\pgfusepath{clip}%
\pgfsetrectcap%
\pgfsetroundjoin%
\pgfsetlinewidth{0.501875pt}%
\definecolor{currentstroke}{rgb}{0.000000,0.000000,0.000000}%
\pgfsetstrokecolor{currentstroke}%
\pgfsetstrokeopacity{0.100000}%
\pgfsetdash{}{0pt}%
\pgfpathmoveto{\pgfqpoint{9.701200in}{1.247073in}}%
\pgfpathlineto{\pgfqpoint{9.701200in}{7.921260in}}%
\pgfusepath{stroke}%
\end{pgfscope}%
\begin{pgfscope}%
\pgfsetbuttcap%
\pgfsetroundjoin%
\definecolor{currentfill}{rgb}{0.000000,0.000000,0.000000}%
\pgfsetfillcolor{currentfill}%
\pgfsetlinewidth{0.501875pt}%
\definecolor{currentstroke}{rgb}{0.000000,0.000000,0.000000}%
\pgfsetstrokecolor{currentstroke}%
\pgfsetdash{}{0pt}%
\pgfsys@defobject{currentmarker}{\pgfqpoint{0.000000in}{0.000000in}}{\pgfqpoint{0.000000in}{0.034722in}}{%
\pgfpathmoveto{\pgfqpoint{0.000000in}{0.000000in}}%
\pgfpathlineto{\pgfqpoint{0.000000in}{0.034722in}}%
\pgfusepath{stroke,fill}%
}%
\begin{pgfscope}%
\pgfsys@transformshift{9.701200in}{1.247073in}%
\pgfsys@useobject{currentmarker}{}%
\end{pgfscope}%
\end{pgfscope}%
\begin{pgfscope}%
\definecolor{textcolor}{rgb}{0.000000,0.000000,0.000000}%
\pgfsetstrokecolor{textcolor}%
\pgfsetfillcolor{textcolor}%
\pgftext[x=9.701200in,y=1.198462in,,top]{\color{textcolor}\sffamily\fontsize{18.000000}{21.600000}\selectfont $\displaystyle 5$}%
\end{pgfscope}%
\begin{pgfscope}%
\pgfpathrectangle{\pgfqpoint{1.874253in}{1.247073in}}{\pgfqpoint{10.047006in}{6.674186in}}%
\pgfusepath{clip}%
\pgfsetrectcap%
\pgfsetroundjoin%
\pgfsetlinewidth{0.501875pt}%
\definecolor{currentstroke}{rgb}{0.000000,0.000000,0.000000}%
\pgfsetstrokecolor{currentstroke}%
\pgfsetstrokeopacity{0.100000}%
\pgfsetdash{}{0pt}%
\pgfpathmoveto{\pgfqpoint{11.209720in}{1.247073in}}%
\pgfpathlineto{\pgfqpoint{11.209720in}{7.921260in}}%
\pgfusepath{stroke}%
\end{pgfscope}%
\begin{pgfscope}%
\pgfsetbuttcap%
\pgfsetroundjoin%
\definecolor{currentfill}{rgb}{0.000000,0.000000,0.000000}%
\pgfsetfillcolor{currentfill}%
\pgfsetlinewidth{0.501875pt}%
\definecolor{currentstroke}{rgb}{0.000000,0.000000,0.000000}%
\pgfsetstrokecolor{currentstroke}%
\pgfsetdash{}{0pt}%
\pgfsys@defobject{currentmarker}{\pgfqpoint{0.000000in}{0.000000in}}{\pgfqpoint{0.000000in}{0.034722in}}{%
\pgfpathmoveto{\pgfqpoint{0.000000in}{0.000000in}}%
\pgfpathlineto{\pgfqpoint{0.000000in}{0.034722in}}%
\pgfusepath{stroke,fill}%
}%
\begin{pgfscope}%
\pgfsys@transformshift{11.209720in}{1.247073in}%
\pgfsys@useobject{currentmarker}{}%
\end{pgfscope}%
\end{pgfscope}%
\begin{pgfscope}%
\definecolor{textcolor}{rgb}{0.000000,0.000000,0.000000}%
\pgfsetstrokecolor{textcolor}%
\pgfsetfillcolor{textcolor}%
\pgftext[x=11.209720in,y=1.198462in,,top]{\color{textcolor}\sffamily\fontsize{18.000000}{21.600000}\selectfont $\displaystyle 6$}%
\end{pgfscope}%
\begin{pgfscope}%
\definecolor{textcolor}{rgb}{0.000000,0.000000,0.000000}%
\pgfsetstrokecolor{textcolor}%
\pgfsetfillcolor{textcolor}%
\pgftext[x=6.897757in,y=0.900964in,,top]{\color{textcolor}\sffamily\fontsize{18.000000}{21.600000}\selectfont $\displaystyle x$}%
\end{pgfscope}%
\begin{pgfscope}%
\pgfpathrectangle{\pgfqpoint{1.874253in}{1.247073in}}{\pgfqpoint{10.047006in}{6.674186in}}%
\pgfusepath{clip}%
\pgfsetrectcap%
\pgfsetroundjoin%
\pgfsetlinewidth{0.501875pt}%
\definecolor{currentstroke}{rgb}{0.000000,0.000000,0.000000}%
\pgfsetstrokecolor{currentstroke}%
\pgfsetstrokeopacity{0.100000}%
\pgfsetdash{}{0pt}%
\pgfpathmoveto{\pgfqpoint{1.874253in}{2.201795in}}%
\pgfpathlineto{\pgfqpoint{11.921260in}{2.201795in}}%
\pgfusepath{stroke}%
\end{pgfscope}%
\begin{pgfscope}%
\pgfsetbuttcap%
\pgfsetroundjoin%
\definecolor{currentfill}{rgb}{0.000000,0.000000,0.000000}%
\pgfsetfillcolor{currentfill}%
\pgfsetlinewidth{0.501875pt}%
\definecolor{currentstroke}{rgb}{0.000000,0.000000,0.000000}%
\pgfsetstrokecolor{currentstroke}%
\pgfsetdash{}{0pt}%
\pgfsys@defobject{currentmarker}{\pgfqpoint{0.000000in}{0.000000in}}{\pgfqpoint{0.034722in}{0.000000in}}{%
\pgfpathmoveto{\pgfqpoint{0.000000in}{0.000000in}}%
\pgfpathlineto{\pgfqpoint{0.034722in}{0.000000in}}%
\pgfusepath{stroke,fill}%
}%
\begin{pgfscope}%
\pgfsys@transformshift{1.874253in}{2.201795in}%
\pgfsys@useobject{currentmarker}{}%
\end{pgfscope}%
\end{pgfscope}%
\begin{pgfscope}%
\definecolor{textcolor}{rgb}{0.000000,0.000000,0.000000}%
\pgfsetstrokecolor{textcolor}%
\pgfsetfillcolor{textcolor}%
\pgftext[x=1.198702in, y=2.106824in, left, base]{\color{textcolor}\sffamily\fontsize{18.000000}{21.600000}\selectfont $\displaystyle -5000$}%
\end{pgfscope}%
\begin{pgfscope}%
\pgfpathrectangle{\pgfqpoint{1.874253in}{1.247073in}}{\pgfqpoint{10.047006in}{6.674186in}}%
\pgfusepath{clip}%
\pgfsetrectcap%
\pgfsetroundjoin%
\pgfsetlinewidth{0.501875pt}%
\definecolor{currentstroke}{rgb}{0.000000,0.000000,0.000000}%
\pgfsetstrokecolor{currentstroke}%
\pgfsetstrokeopacity{0.100000}%
\pgfsetdash{}{0pt}%
\pgfpathmoveto{\pgfqpoint{1.874253in}{3.441600in}}%
\pgfpathlineto{\pgfqpoint{11.921260in}{3.441600in}}%
\pgfusepath{stroke}%
\end{pgfscope}%
\begin{pgfscope}%
\pgfsetbuttcap%
\pgfsetroundjoin%
\definecolor{currentfill}{rgb}{0.000000,0.000000,0.000000}%
\pgfsetfillcolor{currentfill}%
\pgfsetlinewidth{0.501875pt}%
\definecolor{currentstroke}{rgb}{0.000000,0.000000,0.000000}%
\pgfsetstrokecolor{currentstroke}%
\pgfsetdash{}{0pt}%
\pgfsys@defobject{currentmarker}{\pgfqpoint{0.000000in}{0.000000in}}{\pgfqpoint{0.034722in}{0.000000in}}{%
\pgfpathmoveto{\pgfqpoint{0.000000in}{0.000000in}}%
\pgfpathlineto{\pgfqpoint{0.034722in}{0.000000in}}%
\pgfusepath{stroke,fill}%
}%
\begin{pgfscope}%
\pgfsys@transformshift{1.874253in}{3.441600in}%
\pgfsys@useobject{currentmarker}{}%
\end{pgfscope}%
\end{pgfscope}%
\begin{pgfscope}%
\definecolor{textcolor}{rgb}{0.000000,0.000000,0.000000}%
\pgfsetstrokecolor{textcolor}%
\pgfsetfillcolor{textcolor}%
\pgftext[x=1.198702in, y=3.346629in, left, base]{\color{textcolor}\sffamily\fontsize{18.000000}{21.600000}\selectfont $\displaystyle -2500$}%
\end{pgfscope}%
\begin{pgfscope}%
\pgfpathrectangle{\pgfqpoint{1.874253in}{1.247073in}}{\pgfqpoint{10.047006in}{6.674186in}}%
\pgfusepath{clip}%
\pgfsetrectcap%
\pgfsetroundjoin%
\pgfsetlinewidth{0.501875pt}%
\definecolor{currentstroke}{rgb}{0.000000,0.000000,0.000000}%
\pgfsetstrokecolor{currentstroke}%
\pgfsetstrokeopacity{0.100000}%
\pgfsetdash{}{0pt}%
\pgfpathmoveto{\pgfqpoint{1.874253in}{4.681404in}}%
\pgfpathlineto{\pgfqpoint{11.921260in}{4.681404in}}%
\pgfusepath{stroke}%
\end{pgfscope}%
\begin{pgfscope}%
\pgfsetbuttcap%
\pgfsetroundjoin%
\definecolor{currentfill}{rgb}{0.000000,0.000000,0.000000}%
\pgfsetfillcolor{currentfill}%
\pgfsetlinewidth{0.501875pt}%
\definecolor{currentstroke}{rgb}{0.000000,0.000000,0.000000}%
\pgfsetstrokecolor{currentstroke}%
\pgfsetdash{}{0pt}%
\pgfsys@defobject{currentmarker}{\pgfqpoint{0.000000in}{0.000000in}}{\pgfqpoint{0.034722in}{0.000000in}}{%
\pgfpathmoveto{\pgfqpoint{0.000000in}{0.000000in}}%
\pgfpathlineto{\pgfqpoint{0.034722in}{0.000000in}}%
\pgfusepath{stroke,fill}%
}%
\begin{pgfscope}%
\pgfsys@transformshift{1.874253in}{4.681404in}%
\pgfsys@useobject{currentmarker}{}%
\end{pgfscope}%
\end{pgfscope}%
\begin{pgfscope}%
\definecolor{textcolor}{rgb}{0.000000,0.000000,0.000000}%
\pgfsetstrokecolor{textcolor}%
\pgfsetfillcolor{textcolor}%
\pgftext[x=1.715574in, y=4.586433in, left, base]{\color{textcolor}\sffamily\fontsize{18.000000}{21.600000}\selectfont $\displaystyle 0$}%
\end{pgfscope}%
\begin{pgfscope}%
\pgfpathrectangle{\pgfqpoint{1.874253in}{1.247073in}}{\pgfqpoint{10.047006in}{6.674186in}}%
\pgfusepath{clip}%
\pgfsetrectcap%
\pgfsetroundjoin%
\pgfsetlinewidth{0.501875pt}%
\definecolor{currentstroke}{rgb}{0.000000,0.000000,0.000000}%
\pgfsetstrokecolor{currentstroke}%
\pgfsetstrokeopacity{0.100000}%
\pgfsetdash{}{0pt}%
\pgfpathmoveto{\pgfqpoint{1.874253in}{5.921208in}}%
\pgfpathlineto{\pgfqpoint{11.921260in}{5.921208in}}%
\pgfusepath{stroke}%
\end{pgfscope}%
\begin{pgfscope}%
\pgfsetbuttcap%
\pgfsetroundjoin%
\definecolor{currentfill}{rgb}{0.000000,0.000000,0.000000}%
\pgfsetfillcolor{currentfill}%
\pgfsetlinewidth{0.501875pt}%
\definecolor{currentstroke}{rgb}{0.000000,0.000000,0.000000}%
\pgfsetstrokecolor{currentstroke}%
\pgfsetdash{}{0pt}%
\pgfsys@defobject{currentmarker}{\pgfqpoint{0.000000in}{0.000000in}}{\pgfqpoint{0.034722in}{0.000000in}}{%
\pgfpathmoveto{\pgfqpoint{0.000000in}{0.000000in}}%
\pgfpathlineto{\pgfqpoint{0.034722in}{0.000000in}}%
\pgfusepath{stroke,fill}%
}%
\begin{pgfscope}%
\pgfsys@transformshift{1.874253in}{5.921208in}%
\pgfsys@useobject{currentmarker}{}%
\end{pgfscope}%
\end{pgfscope}%
\begin{pgfscope}%
\definecolor{textcolor}{rgb}{0.000000,0.000000,0.000000}%
\pgfsetstrokecolor{textcolor}%
\pgfsetfillcolor{textcolor}%
\pgftext[x=1.385369in, y=5.826238in, left, base]{\color{textcolor}\sffamily\fontsize{18.000000}{21.600000}\selectfont $\displaystyle 2500$}%
\end{pgfscope}%
\begin{pgfscope}%
\pgfpathrectangle{\pgfqpoint{1.874253in}{1.247073in}}{\pgfqpoint{10.047006in}{6.674186in}}%
\pgfusepath{clip}%
\pgfsetrectcap%
\pgfsetroundjoin%
\pgfsetlinewidth{0.501875pt}%
\definecolor{currentstroke}{rgb}{0.000000,0.000000,0.000000}%
\pgfsetstrokecolor{currentstroke}%
\pgfsetstrokeopacity{0.100000}%
\pgfsetdash{}{0pt}%
\pgfpathmoveto{\pgfqpoint{1.874253in}{7.161013in}}%
\pgfpathlineto{\pgfqpoint{11.921260in}{7.161013in}}%
\pgfusepath{stroke}%
\end{pgfscope}%
\begin{pgfscope}%
\pgfsetbuttcap%
\pgfsetroundjoin%
\definecolor{currentfill}{rgb}{0.000000,0.000000,0.000000}%
\pgfsetfillcolor{currentfill}%
\pgfsetlinewidth{0.501875pt}%
\definecolor{currentstroke}{rgb}{0.000000,0.000000,0.000000}%
\pgfsetstrokecolor{currentstroke}%
\pgfsetdash{}{0pt}%
\pgfsys@defobject{currentmarker}{\pgfqpoint{0.000000in}{0.000000in}}{\pgfqpoint{0.034722in}{0.000000in}}{%
\pgfpathmoveto{\pgfqpoint{0.000000in}{0.000000in}}%
\pgfpathlineto{\pgfqpoint{0.034722in}{0.000000in}}%
\pgfusepath{stroke,fill}%
}%
\begin{pgfscope}%
\pgfsys@transformshift{1.874253in}{7.161013in}%
\pgfsys@useobject{currentmarker}{}%
\end{pgfscope}%
\end{pgfscope}%
\begin{pgfscope}%
\definecolor{textcolor}{rgb}{0.000000,0.000000,0.000000}%
\pgfsetstrokecolor{textcolor}%
\pgfsetfillcolor{textcolor}%
\pgftext[x=1.385369in, y=7.066042in, left, base]{\color{textcolor}\sffamily\fontsize{18.000000}{21.600000}\selectfont $\displaystyle 5000$}%
\end{pgfscope}%
\begin{pgfscope}%
\pgfpathrectangle{\pgfqpoint{1.874253in}{1.247073in}}{\pgfqpoint{10.047006in}{6.674186in}}%
\pgfusepath{clip}%
\pgfsetbuttcap%
\pgfsetroundjoin%
\pgfsetlinewidth{1.003750pt}%
\definecolor{currentstroke}{rgb}{0.000000,0.605603,0.978680}%
\pgfsetstrokecolor{currentstroke}%
\pgfsetdash{}{0pt}%
\pgfpathmoveto{\pgfqpoint{2.158603in}{6.922255in}}%
\pgfpathlineto{\pgfqpoint{2.232652in}{3.221472in}}%
\pgfpathlineto{\pgfqpoint{2.306701in}{5.143460in}}%
\pgfpathlineto{\pgfqpoint{2.380750in}{5.264328in}}%
\pgfpathlineto{\pgfqpoint{2.454800in}{3.211228in}}%
\pgfpathlineto{\pgfqpoint{2.528849in}{6.736664in}}%
\pgfpathlineto{\pgfqpoint{2.602898in}{2.353563in}}%
\pgfpathlineto{\pgfqpoint{2.676948in}{7.086018in}}%
\pgfpathlineto{\pgfqpoint{2.750997in}{2.236620in}}%
\pgfpathlineto{\pgfqpoint{2.825046in}{7.206810in}}%
\pgfpathlineto{\pgfqpoint{2.899095in}{2.107223in}}%
\pgfpathlineto{\pgfqpoint{2.973145in}{7.078016in}}%
\pgfpathlineto{\pgfqpoint{3.047194in}{2.874595in}}%
\pgfpathlineto{\pgfqpoint{3.121243in}{5.440345in}}%
\pgfpathlineto{\pgfqpoint{3.195293in}{5.263235in}}%
\pgfpathlineto{\pgfqpoint{3.269342in}{2.781358in}}%
\pgfpathlineto{\pgfqpoint{3.343391in}{7.538784in}}%
\pgfpathlineto{\pgfqpoint{3.417440in}{1.435966in}}%
\pgfpathlineto{\pgfqpoint{3.491490in}{7.732368in}}%
\pgfpathlineto{\pgfqpoint{3.565539in}{2.237394in}}%
\pgfpathlineto{\pgfqpoint{3.639588in}{6.352261in}}%
\pgfpathlineto{\pgfqpoint{3.713638in}{3.724138in}}%
\pgfpathlineto{\pgfqpoint{3.787687in}{5.110974in}}%
\pgfpathlineto{\pgfqpoint{3.861736in}{4.559890in}}%
\pgfpathlineto{\pgfqpoint{3.935785in}{4.676073in}}%
\pgfpathlineto{\pgfqpoint{4.009835in}{4.689845in}}%
\pgfpathlineto{\pgfqpoint{4.083884in}{4.727645in}}%
\pgfpathlineto{\pgfqpoint{4.157933in}{4.572245in}}%
\pgfpathlineto{\pgfqpoint{4.231983in}{4.819325in}}%
\pgfpathlineto{\pgfqpoint{4.306032in}{4.563962in}}%
\pgfpathlineto{\pgfqpoint{4.380081in}{4.727608in}}%
\pgfpathlineto{\pgfqpoint{4.454130in}{4.735906in}}%
\pgfpathlineto{\pgfqpoint{4.528180in}{4.513449in}}%
\pgfpathlineto{\pgfqpoint{4.602229in}{4.957463in}}%
\pgfpathlineto{\pgfqpoint{4.676278in}{4.303993in}}%
\pgfpathlineto{\pgfqpoint{4.750327in}{5.149676in}}%
\pgfpathlineto{\pgfqpoint{4.824377in}{4.125527in}}%
\pgfpathlineto{\pgfqpoint{4.898426in}{5.323976in}}%
\pgfpathlineto{\pgfqpoint{4.972475in}{3.941843in}}%
\pgfpathlineto{\pgfqpoint{5.046525in}{5.525416in}}%
\pgfpathlineto{\pgfqpoint{5.120574in}{3.734467in}}%
\pgfpathlineto{\pgfqpoint{5.194623in}{5.700341in}}%
\pgfpathlineto{\pgfqpoint{5.268672in}{3.643748in}}%
\pgfpathlineto{\pgfqpoint{5.342722in}{5.670067in}}%
\pgfpathlineto{\pgfqpoint{5.416771in}{3.793283in}}%
\pgfpathlineto{\pgfqpoint{5.490820in}{5.445838in}}%
\pgfpathlineto{\pgfqpoint{5.564870in}{4.022493in}}%
\pgfpathlineto{\pgfqpoint{5.638919in}{5.277811in}}%
\pgfpathlineto{\pgfqpoint{5.712968in}{4.095271in}}%
\pgfpathlineto{\pgfqpoint{5.787017in}{5.287800in}}%
\pgfpathlineto{\pgfqpoint{5.861067in}{4.054566in}}%
\pgfpathlineto{\pgfqpoint{5.935116in}{5.297579in}}%
\pgfpathlineto{\pgfqpoint{6.009165in}{4.110530in}}%
\pgfpathlineto{\pgfqpoint{6.083215in}{5.190892in}}%
\pgfpathlineto{\pgfqpoint{6.157264in}{4.210101in}}%
\pgfpathlineto{\pgfqpoint{6.231313in}{5.167758in}}%
\pgfpathlineto{\pgfqpoint{6.305362in}{4.117364in}}%
\pgfpathlineto{\pgfqpoint{6.379412in}{5.355300in}}%
\pgfpathlineto{\pgfqpoint{6.453461in}{3.926934in}}%
\pgfpathlineto{\pgfqpoint{6.527510in}{5.405080in}}%
\pgfpathlineto{\pgfqpoint{6.601559in}{4.165139in}}%
\pgfpathlineto{\pgfqpoint{6.675609in}{4.798609in}}%
\pgfpathlineto{\pgfqpoint{6.749658in}{5.087674in}}%
\pgfpathlineto{\pgfqpoint{6.823707in}{3.769401in}}%
\pgfpathlineto{\pgfqpoint{6.897757in}{5.906280in}}%
\pgfpathlineto{\pgfqpoint{6.971806in}{3.469348in}}%
\pgfpathlineto{\pgfqpoint{7.045855in}{5.531693in}}%
\pgfpathlineto{\pgfqpoint{7.119904in}{4.428382in}}%
\pgfpathlineto{\pgfqpoint{7.193954in}{4.315585in}}%
\pgfpathlineto{\pgfqpoint{7.268003in}{5.464355in}}%
\pgfpathlineto{\pgfqpoint{7.342052in}{3.814193in}}%
\pgfpathlineto{\pgfqpoint{7.416102in}{5.325042in}}%
\pgfpathlineto{\pgfqpoint{7.490151in}{4.411704in}}%
\pgfpathlineto{\pgfqpoint{7.564200in}{4.622419in}}%
\pgfpathlineto{\pgfqpoint{7.638249in}{4.915579in}}%
\pgfpathlineto{\pgfqpoint{7.712299in}{4.387881in}}%
\pgfpathlineto{\pgfqpoint{7.786348in}{5.067790in}}%
\pgfpathlineto{\pgfqpoint{7.860397in}{4.036464in}}%
\pgfpathlineto{\pgfqpoint{7.934447in}{5.754250in}}%
\pgfpathlineto{\pgfqpoint{8.008496in}{3.175275in}}%
\pgfpathlineto{\pgfqpoint{8.082545in}{6.385173in}}%
\pgfpathlineto{\pgfqpoint{8.156594in}{3.201995in}}%
\pgfpathlineto{\pgfqpoint{8.230644in}{5.505869in}}%
\pgfpathlineto{\pgfqpoint{8.304693in}{4.764564in}}%
\pgfpathlineto{\pgfqpoint{8.378742in}{3.732771in}}%
\pgfpathlineto{\pgfqpoint{8.452792in}{6.189368in}}%
\pgfpathlineto{\pgfqpoint{8.526841in}{3.044706in}}%
\pgfpathlineto{\pgfqpoint{8.600890in}{6.090293in}}%
\pgfpathlineto{\pgfqpoint{8.674939in}{3.650314in}}%
\pgfpathlineto{\pgfqpoint{8.748989in}{5.412964in}}%
\pgfpathlineto{\pgfqpoint{8.823038in}{4.062851in}}%
\pgfpathlineto{\pgfqpoint{8.897087in}{5.320726in}}%
\pgfpathlineto{\pgfqpoint{8.971136in}{4.065300in}}%
\pgfpathlineto{\pgfqpoint{9.045186in}{5.065156in}}%
\pgfpathlineto{\pgfqpoint{9.119235in}{4.795083in}}%
\pgfpathlineto{\pgfqpoint{9.193284in}{3.902359in}}%
\pgfpathlineto{\pgfqpoint{9.267334in}{6.124000in}}%
\pgfpathlineto{\pgfqpoint{9.341383in}{2.732900in}}%
\pgfpathlineto{\pgfqpoint{9.415432in}{6.917823in}}%
\pgfpathlineto{\pgfqpoint{9.489481in}{2.361049in}}%
\pgfpathlineto{\pgfqpoint{9.563531in}{6.934607in}}%
\pgfpathlineto{\pgfqpoint{9.637580in}{2.613100in}}%
\pgfpathlineto{\pgfqpoint{9.711629in}{6.477219in}}%
\pgfpathlineto{\pgfqpoint{9.785679in}{3.204026in}}%
\pgfpathlineto{\pgfqpoint{9.859728in}{5.879292in}}%
\pgfpathlineto{\pgfqpoint{9.933777in}{3.637969in}}%
\pgfpathlineto{\pgfqpoint{10.007826in}{5.743333in}}%
\pgfpathlineto{\pgfqpoint{10.081876in}{3.470015in}}%
\pgfpathlineto{\pgfqpoint{10.155925in}{6.069730in}}%
\pgfpathlineto{\pgfqpoint{10.229974in}{3.210994in}}%
\pgfpathlineto{\pgfqpoint{10.304024in}{6.072619in}}%
\pgfpathlineto{\pgfqpoint{10.378073in}{3.527144in}}%
\pgfpathlineto{\pgfqpoint{10.452122in}{5.511122in}}%
\pgfpathlineto{\pgfqpoint{10.526171in}{4.181823in}}%
\pgfpathlineto{\pgfqpoint{10.600221in}{4.912825in}}%
\pgfpathlineto{\pgfqpoint{10.674270in}{4.627149in}}%
\pgfpathlineto{\pgfqpoint{10.748319in}{4.665004in}}%
\pgfpathlineto{\pgfqpoint{10.822369in}{4.648949in}}%
\pgfpathlineto{\pgfqpoint{10.896418in}{4.907854in}}%
\pgfpathlineto{\pgfqpoint{10.970467in}{4.105301in}}%
\pgfpathlineto{\pgfqpoint{11.044516in}{5.740535in}}%
\pgfpathlineto{\pgfqpoint{11.118566in}{3.076597in}}%
\pgfpathlineto{\pgfqpoint{11.192615in}{6.805645in}}%
\pgfpathlineto{\pgfqpoint{11.266664in}{2.139007in}}%
\pgfpathlineto{\pgfqpoint{11.340713in}{7.508038in}}%
\pgfpathlineto{\pgfqpoint{11.414763in}{1.712554in}}%
\pgfpathlineto{\pgfqpoint{11.488812in}{7.635118in}}%
\pgfpathlineto{\pgfqpoint{11.562861in}{1.949677in}}%
\pgfpathlineto{\pgfqpoint{11.636911in}{6.922255in}}%
\pgfpathlineto{\pgfqpoint{11.636911in}{6.922255in}}%
\pgfusepath{stroke}%
\end{pgfscope}%
\begin{pgfscope}%
\pgfpathrectangle{\pgfqpoint{1.874253in}{1.247073in}}{\pgfqpoint{10.047006in}{6.674186in}}%
\pgfusepath{clip}%
\pgfsetbuttcap%
\pgfsetroundjoin%
\pgfsetlinewidth{1.003750pt}%
\definecolor{currentstroke}{rgb}{0.888874,0.435649,0.278123}%
\pgfsetstrokecolor{currentstroke}%
\pgfsetdash{}{0pt}%
\pgfpathmoveto{\pgfqpoint{2.158603in}{4.681404in}}%
\pgfpathlineto{\pgfqpoint{6.157264in}{4.681170in}}%
\pgfpathlineto{\pgfqpoint{11.192615in}{4.681548in}}%
\pgfpathlineto{\pgfqpoint{11.636911in}{4.681404in}}%
\pgfpathlineto{\pgfqpoint{11.636911in}{4.681404in}}%
\pgfusepath{stroke}%
\end{pgfscope}%
\begin{pgfscope}%
\pgfsetrectcap%
\pgfsetmiterjoin%
\pgfsetlinewidth{1.003750pt}%
\definecolor{currentstroke}{rgb}{0.000000,0.000000,0.000000}%
\pgfsetstrokecolor{currentstroke}%
\pgfsetdash{}{0pt}%
\pgfpathmoveto{\pgfqpoint{1.874253in}{1.247073in}}%
\pgfpathlineto{\pgfqpoint{1.874253in}{7.921260in}}%
\pgfusepath{stroke}%
\end{pgfscope}%
\begin{pgfscope}%
\pgfsetrectcap%
\pgfsetmiterjoin%
\pgfsetlinewidth{1.003750pt}%
\definecolor{currentstroke}{rgb}{0.000000,0.000000,0.000000}%
\pgfsetstrokecolor{currentstroke}%
\pgfsetdash{}{0pt}%
\pgfpathmoveto{\pgfqpoint{1.874253in}{1.247073in}}%
\pgfpathlineto{\pgfqpoint{11.921260in}{1.247073in}}%
\pgfusepath{stroke}%
\end{pgfscope}%
\begin{pgfscope}%
\pgfsetbuttcap%
\pgfsetmiterjoin%
\definecolor{currentfill}{rgb}{1.000000,1.000000,1.000000}%
\pgfsetfillcolor{currentfill}%
\pgfsetlinewidth{1.003750pt}%
\definecolor{currentstroke}{rgb}{0.000000,0.000000,0.000000}%
\pgfsetstrokecolor{currentstroke}%
\pgfsetdash{}{0pt}%
\pgfpathmoveto{\pgfqpoint{10.511589in}{6.787373in}}%
\pgfpathlineto{\pgfqpoint{11.796260in}{6.787373in}}%
\pgfpathlineto{\pgfqpoint{11.796260in}{7.796260in}}%
\pgfpathlineto{\pgfqpoint{10.511589in}{7.796260in}}%
\pgfpathclose%
\pgfusepath{stroke,fill}%
\end{pgfscope}%
\begin{pgfscope}%
\pgfsetbuttcap%
\pgfsetmiterjoin%
\pgfsetlinewidth{2.258437pt}%
\definecolor{currentstroke}{rgb}{0.000000,0.605603,0.978680}%
\pgfsetstrokecolor{currentstroke}%
\pgfsetdash{}{0pt}%
\pgfpathmoveto{\pgfqpoint{10.711589in}{7.493818in}}%
\pgfpathlineto{\pgfqpoint{11.211589in}{7.493818in}}%
\pgfusepath{stroke}%
\end{pgfscope}%
\begin{pgfscope}%
\definecolor{textcolor}{rgb}{0.000000,0.000000,0.000000}%
\pgfsetstrokecolor{textcolor}%
\pgfsetfillcolor{textcolor}%
\pgftext[x=11.411589in,y=7.406318in,left,base]{\color{textcolor}\sffamily\fontsize{18.000000}{21.600000}\selectfont $\displaystyle U$}%
\end{pgfscope}%
\begin{pgfscope}%
\pgfsetbuttcap%
\pgfsetmiterjoin%
\pgfsetlinewidth{2.258437pt}%
\definecolor{currentstroke}{rgb}{0.888874,0.435649,0.278123}%
\pgfsetstrokecolor{currentstroke}%
\pgfsetdash{}{0pt}%
\pgfpathmoveto{\pgfqpoint{10.711589in}{7.126875in}}%
\pgfpathlineto{\pgfqpoint{11.211589in}{7.126875in}}%
\pgfusepath{stroke}%
\end{pgfscope}%
\begin{pgfscope}%
\definecolor{textcolor}{rgb}{0.000000,0.000000,0.000000}%
\pgfsetstrokecolor{textcolor}%
\pgfsetfillcolor{textcolor}%
\pgftext[x=11.411589in,y=7.039375in,left,base]{\color{textcolor}\sffamily\fontsize{18.000000}{21.600000}\selectfont $\displaystyle u$}%
\end{pgfscope}%
\end{pgfpicture}%
\makeatother%
\endgroup%
}\quad
	\resizebox{0.4\linewidth}{!}{%% Creator: Matplotlib, PGF backend
%%
%% To include the figure in your LaTeX document, write
%%   \input{<filename>.pgf}
%%
%% Make sure the required packages are loaded in your preamble
%%   \usepackage{pgf}
%%
%% Figures using additional raster images can only be included by \input if
%% they are in the same directory as the main LaTeX file. For loading figures
%% from other directories you can use the `import` package
%%   \usepackage{import}
%%
%% and then include the figures with
%%   \import{<path to file>}{<filename>.pgf}
%%
%% Matplotlib used the following preamble
%%   \usepackage{fontspec}
%%   \setmainfont{DejaVuSerif.ttf}[Path=\detokenize{/Users/quejiahao/.julia/conda/3/lib/python3.9/site-packages/matplotlib/mpl-data/fonts/ttf/}]
%%   \setsansfont{DejaVuSans.ttf}[Path=\detokenize{/Users/quejiahao/.julia/conda/3/lib/python3.9/site-packages/matplotlib/mpl-data/fonts/ttf/}]
%%   \setmonofont{DejaVuSansMono.ttf}[Path=\detokenize{/Users/quejiahao/.julia/conda/3/lib/python3.9/site-packages/matplotlib/mpl-data/fonts/ttf/}]
%%
\begingroup%
\makeatletter%
\begin{pgfpicture}%
\pgfpathrectangle{\pgfpointorigin}{\pgfqpoint{12.000000in}{8.000000in}}%
\pgfusepath{use as bounding box, clip}%
\begin{pgfscope}%
\pgfsetbuttcap%
\pgfsetmiterjoin%
\definecolor{currentfill}{rgb}{1.000000,1.000000,1.000000}%
\pgfsetfillcolor{currentfill}%
\pgfsetlinewidth{0.000000pt}%
\definecolor{currentstroke}{rgb}{1.000000,1.000000,1.000000}%
\pgfsetstrokecolor{currentstroke}%
\pgfsetdash{}{0pt}%
\pgfpathmoveto{\pgfqpoint{0.000000in}{0.000000in}}%
\pgfpathlineto{\pgfqpoint{12.000000in}{0.000000in}}%
\pgfpathlineto{\pgfqpoint{12.000000in}{8.000000in}}%
\pgfpathlineto{\pgfqpoint{0.000000in}{8.000000in}}%
\pgfpathclose%
\pgfusepath{fill}%
\end{pgfscope}%
\begin{pgfscope}%
\pgfsetbuttcap%
\pgfsetmiterjoin%
\definecolor{currentfill}{rgb}{1.000000,1.000000,1.000000}%
\pgfsetfillcolor{currentfill}%
\pgfsetlinewidth{0.000000pt}%
\definecolor{currentstroke}{rgb}{0.000000,0.000000,0.000000}%
\pgfsetstrokecolor{currentstroke}%
\pgfsetstrokeopacity{0.000000}%
\pgfsetdash{}{0pt}%
\pgfpathmoveto{\pgfqpoint{3.126011in}{1.247073in}}%
\pgfpathlineto{\pgfqpoint{11.921260in}{1.247073in}}%
\pgfpathlineto{\pgfqpoint{11.921260in}{7.921260in}}%
\pgfpathlineto{\pgfqpoint{3.126011in}{7.921260in}}%
\pgfpathclose%
\pgfusepath{fill}%
\end{pgfscope}%
\begin{pgfscope}%
\pgfpathrectangle{\pgfqpoint{3.126011in}{1.247073in}}{\pgfqpoint{8.795249in}{6.674186in}}%
\pgfusepath{clip}%
\pgfsetrectcap%
\pgfsetroundjoin%
\pgfsetlinewidth{0.501875pt}%
\definecolor{currentstroke}{rgb}{0.000000,0.000000,0.000000}%
\pgfsetstrokecolor{currentstroke}%
\pgfsetstrokeopacity{0.100000}%
\pgfsetdash{}{0pt}%
\pgfpathmoveto{\pgfqpoint{3.374933in}{1.247073in}}%
\pgfpathlineto{\pgfqpoint{3.374933in}{7.921260in}}%
\pgfusepath{stroke}%
\end{pgfscope}%
\begin{pgfscope}%
\pgfsetbuttcap%
\pgfsetroundjoin%
\definecolor{currentfill}{rgb}{0.000000,0.000000,0.000000}%
\pgfsetfillcolor{currentfill}%
\pgfsetlinewidth{0.501875pt}%
\definecolor{currentstroke}{rgb}{0.000000,0.000000,0.000000}%
\pgfsetstrokecolor{currentstroke}%
\pgfsetdash{}{0pt}%
\pgfsys@defobject{currentmarker}{\pgfqpoint{0.000000in}{0.000000in}}{\pgfqpoint{0.000000in}{0.034722in}}{%
\pgfpathmoveto{\pgfqpoint{0.000000in}{0.000000in}}%
\pgfpathlineto{\pgfqpoint{0.000000in}{0.034722in}}%
\pgfusepath{stroke,fill}%
}%
\begin{pgfscope}%
\pgfsys@transformshift{3.374933in}{1.247073in}%
\pgfsys@useobject{currentmarker}{}%
\end{pgfscope}%
\end{pgfscope}%
\begin{pgfscope}%
\definecolor{textcolor}{rgb}{0.000000,0.000000,0.000000}%
\pgfsetstrokecolor{textcolor}%
\pgfsetfillcolor{textcolor}%
\pgftext[x=3.374933in,y=1.198462in,,top]{\color{textcolor}\sffamily\fontsize{18.000000}{21.600000}\selectfont $\displaystyle 0$}%
\end{pgfscope}%
\begin{pgfscope}%
\pgfpathrectangle{\pgfqpoint{3.126011in}{1.247073in}}{\pgfqpoint{8.795249in}{6.674186in}}%
\pgfusepath{clip}%
\pgfsetrectcap%
\pgfsetroundjoin%
\pgfsetlinewidth{0.501875pt}%
\definecolor{currentstroke}{rgb}{0.000000,0.000000,0.000000}%
\pgfsetstrokecolor{currentstroke}%
\pgfsetstrokeopacity{0.100000}%
\pgfsetdash{}{0pt}%
\pgfpathmoveto{\pgfqpoint{4.695506in}{1.247073in}}%
\pgfpathlineto{\pgfqpoint{4.695506in}{7.921260in}}%
\pgfusepath{stroke}%
\end{pgfscope}%
\begin{pgfscope}%
\pgfsetbuttcap%
\pgfsetroundjoin%
\definecolor{currentfill}{rgb}{0.000000,0.000000,0.000000}%
\pgfsetfillcolor{currentfill}%
\pgfsetlinewidth{0.501875pt}%
\definecolor{currentstroke}{rgb}{0.000000,0.000000,0.000000}%
\pgfsetstrokecolor{currentstroke}%
\pgfsetdash{}{0pt}%
\pgfsys@defobject{currentmarker}{\pgfqpoint{0.000000in}{0.000000in}}{\pgfqpoint{0.000000in}{0.034722in}}{%
\pgfpathmoveto{\pgfqpoint{0.000000in}{0.000000in}}%
\pgfpathlineto{\pgfqpoint{0.000000in}{0.034722in}}%
\pgfusepath{stroke,fill}%
}%
\begin{pgfscope}%
\pgfsys@transformshift{4.695506in}{1.247073in}%
\pgfsys@useobject{currentmarker}{}%
\end{pgfscope}%
\end{pgfscope}%
\begin{pgfscope}%
\definecolor{textcolor}{rgb}{0.000000,0.000000,0.000000}%
\pgfsetstrokecolor{textcolor}%
\pgfsetfillcolor{textcolor}%
\pgftext[x=4.695506in,y=1.198462in,,top]{\color{textcolor}\sffamily\fontsize{18.000000}{21.600000}\selectfont $\displaystyle 1$}%
\end{pgfscope}%
\begin{pgfscope}%
\pgfpathrectangle{\pgfqpoint{3.126011in}{1.247073in}}{\pgfqpoint{8.795249in}{6.674186in}}%
\pgfusepath{clip}%
\pgfsetrectcap%
\pgfsetroundjoin%
\pgfsetlinewidth{0.501875pt}%
\definecolor{currentstroke}{rgb}{0.000000,0.000000,0.000000}%
\pgfsetstrokecolor{currentstroke}%
\pgfsetstrokeopacity{0.100000}%
\pgfsetdash{}{0pt}%
\pgfpathmoveto{\pgfqpoint{6.016079in}{1.247073in}}%
\pgfpathlineto{\pgfqpoint{6.016079in}{7.921260in}}%
\pgfusepath{stroke}%
\end{pgfscope}%
\begin{pgfscope}%
\pgfsetbuttcap%
\pgfsetroundjoin%
\definecolor{currentfill}{rgb}{0.000000,0.000000,0.000000}%
\pgfsetfillcolor{currentfill}%
\pgfsetlinewidth{0.501875pt}%
\definecolor{currentstroke}{rgb}{0.000000,0.000000,0.000000}%
\pgfsetstrokecolor{currentstroke}%
\pgfsetdash{}{0pt}%
\pgfsys@defobject{currentmarker}{\pgfqpoint{0.000000in}{0.000000in}}{\pgfqpoint{0.000000in}{0.034722in}}{%
\pgfpathmoveto{\pgfqpoint{0.000000in}{0.000000in}}%
\pgfpathlineto{\pgfqpoint{0.000000in}{0.034722in}}%
\pgfusepath{stroke,fill}%
}%
\begin{pgfscope}%
\pgfsys@transformshift{6.016079in}{1.247073in}%
\pgfsys@useobject{currentmarker}{}%
\end{pgfscope}%
\end{pgfscope}%
\begin{pgfscope}%
\definecolor{textcolor}{rgb}{0.000000,0.000000,0.000000}%
\pgfsetstrokecolor{textcolor}%
\pgfsetfillcolor{textcolor}%
\pgftext[x=6.016079in,y=1.198462in,,top]{\color{textcolor}\sffamily\fontsize{18.000000}{21.600000}\selectfont $\displaystyle 2$}%
\end{pgfscope}%
\begin{pgfscope}%
\pgfpathrectangle{\pgfqpoint{3.126011in}{1.247073in}}{\pgfqpoint{8.795249in}{6.674186in}}%
\pgfusepath{clip}%
\pgfsetrectcap%
\pgfsetroundjoin%
\pgfsetlinewidth{0.501875pt}%
\definecolor{currentstroke}{rgb}{0.000000,0.000000,0.000000}%
\pgfsetstrokecolor{currentstroke}%
\pgfsetstrokeopacity{0.100000}%
\pgfsetdash{}{0pt}%
\pgfpathmoveto{\pgfqpoint{7.336652in}{1.247073in}}%
\pgfpathlineto{\pgfqpoint{7.336652in}{7.921260in}}%
\pgfusepath{stroke}%
\end{pgfscope}%
\begin{pgfscope}%
\pgfsetbuttcap%
\pgfsetroundjoin%
\definecolor{currentfill}{rgb}{0.000000,0.000000,0.000000}%
\pgfsetfillcolor{currentfill}%
\pgfsetlinewidth{0.501875pt}%
\definecolor{currentstroke}{rgb}{0.000000,0.000000,0.000000}%
\pgfsetstrokecolor{currentstroke}%
\pgfsetdash{}{0pt}%
\pgfsys@defobject{currentmarker}{\pgfqpoint{0.000000in}{0.000000in}}{\pgfqpoint{0.000000in}{0.034722in}}{%
\pgfpathmoveto{\pgfqpoint{0.000000in}{0.000000in}}%
\pgfpathlineto{\pgfqpoint{0.000000in}{0.034722in}}%
\pgfusepath{stroke,fill}%
}%
\begin{pgfscope}%
\pgfsys@transformshift{7.336652in}{1.247073in}%
\pgfsys@useobject{currentmarker}{}%
\end{pgfscope}%
\end{pgfscope}%
\begin{pgfscope}%
\definecolor{textcolor}{rgb}{0.000000,0.000000,0.000000}%
\pgfsetstrokecolor{textcolor}%
\pgfsetfillcolor{textcolor}%
\pgftext[x=7.336652in,y=1.198462in,,top]{\color{textcolor}\sffamily\fontsize{18.000000}{21.600000}\selectfont $\displaystyle 3$}%
\end{pgfscope}%
\begin{pgfscope}%
\pgfpathrectangle{\pgfqpoint{3.126011in}{1.247073in}}{\pgfqpoint{8.795249in}{6.674186in}}%
\pgfusepath{clip}%
\pgfsetrectcap%
\pgfsetroundjoin%
\pgfsetlinewidth{0.501875pt}%
\definecolor{currentstroke}{rgb}{0.000000,0.000000,0.000000}%
\pgfsetstrokecolor{currentstroke}%
\pgfsetstrokeopacity{0.100000}%
\pgfsetdash{}{0pt}%
\pgfpathmoveto{\pgfqpoint{8.657225in}{1.247073in}}%
\pgfpathlineto{\pgfqpoint{8.657225in}{7.921260in}}%
\pgfusepath{stroke}%
\end{pgfscope}%
\begin{pgfscope}%
\pgfsetbuttcap%
\pgfsetroundjoin%
\definecolor{currentfill}{rgb}{0.000000,0.000000,0.000000}%
\pgfsetfillcolor{currentfill}%
\pgfsetlinewidth{0.501875pt}%
\definecolor{currentstroke}{rgb}{0.000000,0.000000,0.000000}%
\pgfsetstrokecolor{currentstroke}%
\pgfsetdash{}{0pt}%
\pgfsys@defobject{currentmarker}{\pgfqpoint{0.000000in}{0.000000in}}{\pgfqpoint{0.000000in}{0.034722in}}{%
\pgfpathmoveto{\pgfqpoint{0.000000in}{0.000000in}}%
\pgfpathlineto{\pgfqpoint{0.000000in}{0.034722in}}%
\pgfusepath{stroke,fill}%
}%
\begin{pgfscope}%
\pgfsys@transformshift{8.657225in}{1.247073in}%
\pgfsys@useobject{currentmarker}{}%
\end{pgfscope}%
\end{pgfscope}%
\begin{pgfscope}%
\definecolor{textcolor}{rgb}{0.000000,0.000000,0.000000}%
\pgfsetstrokecolor{textcolor}%
\pgfsetfillcolor{textcolor}%
\pgftext[x=8.657225in,y=1.198462in,,top]{\color{textcolor}\sffamily\fontsize{18.000000}{21.600000}\selectfont $\displaystyle 4$}%
\end{pgfscope}%
\begin{pgfscope}%
\pgfpathrectangle{\pgfqpoint{3.126011in}{1.247073in}}{\pgfqpoint{8.795249in}{6.674186in}}%
\pgfusepath{clip}%
\pgfsetrectcap%
\pgfsetroundjoin%
\pgfsetlinewidth{0.501875pt}%
\definecolor{currentstroke}{rgb}{0.000000,0.000000,0.000000}%
\pgfsetstrokecolor{currentstroke}%
\pgfsetstrokeopacity{0.100000}%
\pgfsetdash{}{0pt}%
\pgfpathmoveto{\pgfqpoint{9.977798in}{1.247073in}}%
\pgfpathlineto{\pgfqpoint{9.977798in}{7.921260in}}%
\pgfusepath{stroke}%
\end{pgfscope}%
\begin{pgfscope}%
\pgfsetbuttcap%
\pgfsetroundjoin%
\definecolor{currentfill}{rgb}{0.000000,0.000000,0.000000}%
\pgfsetfillcolor{currentfill}%
\pgfsetlinewidth{0.501875pt}%
\definecolor{currentstroke}{rgb}{0.000000,0.000000,0.000000}%
\pgfsetstrokecolor{currentstroke}%
\pgfsetdash{}{0pt}%
\pgfsys@defobject{currentmarker}{\pgfqpoint{0.000000in}{0.000000in}}{\pgfqpoint{0.000000in}{0.034722in}}{%
\pgfpathmoveto{\pgfqpoint{0.000000in}{0.000000in}}%
\pgfpathlineto{\pgfqpoint{0.000000in}{0.034722in}}%
\pgfusepath{stroke,fill}%
}%
\begin{pgfscope}%
\pgfsys@transformshift{9.977798in}{1.247073in}%
\pgfsys@useobject{currentmarker}{}%
\end{pgfscope}%
\end{pgfscope}%
\begin{pgfscope}%
\definecolor{textcolor}{rgb}{0.000000,0.000000,0.000000}%
\pgfsetstrokecolor{textcolor}%
\pgfsetfillcolor{textcolor}%
\pgftext[x=9.977798in,y=1.198462in,,top]{\color{textcolor}\sffamily\fontsize{18.000000}{21.600000}\selectfont $\displaystyle 5$}%
\end{pgfscope}%
\begin{pgfscope}%
\pgfpathrectangle{\pgfqpoint{3.126011in}{1.247073in}}{\pgfqpoint{8.795249in}{6.674186in}}%
\pgfusepath{clip}%
\pgfsetrectcap%
\pgfsetroundjoin%
\pgfsetlinewidth{0.501875pt}%
\definecolor{currentstroke}{rgb}{0.000000,0.000000,0.000000}%
\pgfsetstrokecolor{currentstroke}%
\pgfsetstrokeopacity{0.100000}%
\pgfsetdash{}{0pt}%
\pgfpathmoveto{\pgfqpoint{11.298371in}{1.247073in}}%
\pgfpathlineto{\pgfqpoint{11.298371in}{7.921260in}}%
\pgfusepath{stroke}%
\end{pgfscope}%
\begin{pgfscope}%
\pgfsetbuttcap%
\pgfsetroundjoin%
\definecolor{currentfill}{rgb}{0.000000,0.000000,0.000000}%
\pgfsetfillcolor{currentfill}%
\pgfsetlinewidth{0.501875pt}%
\definecolor{currentstroke}{rgb}{0.000000,0.000000,0.000000}%
\pgfsetstrokecolor{currentstroke}%
\pgfsetdash{}{0pt}%
\pgfsys@defobject{currentmarker}{\pgfqpoint{0.000000in}{0.000000in}}{\pgfqpoint{0.000000in}{0.034722in}}{%
\pgfpathmoveto{\pgfqpoint{0.000000in}{0.000000in}}%
\pgfpathlineto{\pgfqpoint{0.000000in}{0.034722in}}%
\pgfusepath{stroke,fill}%
}%
\begin{pgfscope}%
\pgfsys@transformshift{11.298371in}{1.247073in}%
\pgfsys@useobject{currentmarker}{}%
\end{pgfscope}%
\end{pgfscope}%
\begin{pgfscope}%
\definecolor{textcolor}{rgb}{0.000000,0.000000,0.000000}%
\pgfsetstrokecolor{textcolor}%
\pgfsetfillcolor{textcolor}%
\pgftext[x=11.298371in,y=1.198462in,,top]{\color{textcolor}\sffamily\fontsize{18.000000}{21.600000}\selectfont $\displaystyle 6$}%
\end{pgfscope}%
\begin{pgfscope}%
\definecolor{textcolor}{rgb}{0.000000,0.000000,0.000000}%
\pgfsetstrokecolor{textcolor}%
\pgfsetfillcolor{textcolor}%
\pgftext[x=7.523635in,y=0.900964in,,top]{\color{textcolor}\sffamily\fontsize{18.000000}{21.600000}\selectfont $\displaystyle x$}%
\end{pgfscope}%
\begin{pgfscope}%
\pgfpathrectangle{\pgfqpoint{3.126011in}{1.247073in}}{\pgfqpoint{8.795249in}{6.674186in}}%
\pgfusepath{clip}%
\pgfsetrectcap%
\pgfsetroundjoin%
\pgfsetlinewidth{0.501875pt}%
\definecolor{currentstroke}{rgb}{0.000000,0.000000,0.000000}%
\pgfsetstrokecolor{currentstroke}%
\pgfsetstrokeopacity{0.100000}%
\pgfsetdash{}{0pt}%
\pgfpathmoveto{\pgfqpoint{3.126011in}{1.510073in}}%
\pgfpathlineto{\pgfqpoint{11.921260in}{1.510073in}}%
\pgfusepath{stroke}%
\end{pgfscope}%
\begin{pgfscope}%
\pgfsetbuttcap%
\pgfsetroundjoin%
\definecolor{currentfill}{rgb}{0.000000,0.000000,0.000000}%
\pgfsetfillcolor{currentfill}%
\pgfsetlinewidth{0.501875pt}%
\definecolor{currentstroke}{rgb}{0.000000,0.000000,0.000000}%
\pgfsetstrokecolor{currentstroke}%
\pgfsetdash{}{0pt}%
\pgfsys@defobject{currentmarker}{\pgfqpoint{0.000000in}{0.000000in}}{\pgfqpoint{0.034722in}{0.000000in}}{%
\pgfpathmoveto{\pgfqpoint{0.000000in}{0.000000in}}%
\pgfpathlineto{\pgfqpoint{0.034722in}{0.000000in}}%
\pgfusepath{stroke,fill}%
}%
\begin{pgfscope}%
\pgfsys@transformshift{3.126011in}{1.510073in}%
\pgfsys@useobject{currentmarker}{}%
\end{pgfscope}%
\end{pgfscope}%
\begin{pgfscope}%
\definecolor{textcolor}{rgb}{0.000000,0.000000,0.000000}%
\pgfsetstrokecolor{textcolor}%
\pgfsetfillcolor{textcolor}%
\pgftext[x=1.923977in, y=1.415102in, left, base]{\color{textcolor}\sffamily\fontsize{18.000000}{21.600000}\selectfont $\displaystyle -3.0×10^{298}$}%
\end{pgfscope}%
\begin{pgfscope}%
\pgfpathrectangle{\pgfqpoint{3.126011in}{1.247073in}}{\pgfqpoint{8.795249in}{6.674186in}}%
\pgfusepath{clip}%
\pgfsetrectcap%
\pgfsetroundjoin%
\pgfsetlinewidth{0.501875pt}%
\definecolor{currentstroke}{rgb}{0.000000,0.000000,0.000000}%
\pgfsetstrokecolor{currentstroke}%
\pgfsetstrokeopacity{0.100000}%
\pgfsetdash{}{0pt}%
\pgfpathmoveto{\pgfqpoint{3.126011in}{2.533183in}}%
\pgfpathlineto{\pgfqpoint{11.921260in}{2.533183in}}%
\pgfusepath{stroke}%
\end{pgfscope}%
\begin{pgfscope}%
\pgfsetbuttcap%
\pgfsetroundjoin%
\definecolor{currentfill}{rgb}{0.000000,0.000000,0.000000}%
\pgfsetfillcolor{currentfill}%
\pgfsetlinewidth{0.501875pt}%
\definecolor{currentstroke}{rgb}{0.000000,0.000000,0.000000}%
\pgfsetstrokecolor{currentstroke}%
\pgfsetdash{}{0pt}%
\pgfsys@defobject{currentmarker}{\pgfqpoint{0.000000in}{0.000000in}}{\pgfqpoint{0.034722in}{0.000000in}}{%
\pgfpathmoveto{\pgfqpoint{0.000000in}{0.000000in}}%
\pgfpathlineto{\pgfqpoint{0.034722in}{0.000000in}}%
\pgfusepath{stroke,fill}%
}%
\begin{pgfscope}%
\pgfsys@transformshift{3.126011in}{2.533183in}%
\pgfsys@useobject{currentmarker}{}%
\end{pgfscope}%
\end{pgfscope}%
\begin{pgfscope}%
\definecolor{textcolor}{rgb}{0.000000,0.000000,0.000000}%
\pgfsetstrokecolor{textcolor}%
\pgfsetfillcolor{textcolor}%
\pgftext[x=1.923977in, y=2.438212in, left, base]{\color{textcolor}\sffamily\fontsize{18.000000}{21.600000}\selectfont $\displaystyle -2.0×10^{298}$}%
\end{pgfscope}%
\begin{pgfscope}%
\pgfpathrectangle{\pgfqpoint{3.126011in}{1.247073in}}{\pgfqpoint{8.795249in}{6.674186in}}%
\pgfusepath{clip}%
\pgfsetrectcap%
\pgfsetroundjoin%
\pgfsetlinewidth{0.501875pt}%
\definecolor{currentstroke}{rgb}{0.000000,0.000000,0.000000}%
\pgfsetstrokecolor{currentstroke}%
\pgfsetstrokeopacity{0.100000}%
\pgfsetdash{}{0pt}%
\pgfpathmoveto{\pgfqpoint{3.126011in}{3.556293in}}%
\pgfpathlineto{\pgfqpoint{11.921260in}{3.556293in}}%
\pgfusepath{stroke}%
\end{pgfscope}%
\begin{pgfscope}%
\pgfsetbuttcap%
\pgfsetroundjoin%
\definecolor{currentfill}{rgb}{0.000000,0.000000,0.000000}%
\pgfsetfillcolor{currentfill}%
\pgfsetlinewidth{0.501875pt}%
\definecolor{currentstroke}{rgb}{0.000000,0.000000,0.000000}%
\pgfsetstrokecolor{currentstroke}%
\pgfsetdash{}{0pt}%
\pgfsys@defobject{currentmarker}{\pgfqpoint{0.000000in}{0.000000in}}{\pgfqpoint{0.034722in}{0.000000in}}{%
\pgfpathmoveto{\pgfqpoint{0.000000in}{0.000000in}}%
\pgfpathlineto{\pgfqpoint{0.034722in}{0.000000in}}%
\pgfusepath{stroke,fill}%
}%
\begin{pgfscope}%
\pgfsys@transformshift{3.126011in}{3.556293in}%
\pgfsys@useobject{currentmarker}{}%
\end{pgfscope}%
\end{pgfscope}%
\begin{pgfscope}%
\definecolor{textcolor}{rgb}{0.000000,0.000000,0.000000}%
\pgfsetstrokecolor{textcolor}%
\pgfsetfillcolor{textcolor}%
\pgftext[x=1.923977in, y=3.461322in, left, base]{\color{textcolor}\sffamily\fontsize{18.000000}{21.600000}\selectfont $\displaystyle -1.0×10^{298}$}%
\end{pgfscope}%
\begin{pgfscope}%
\pgfpathrectangle{\pgfqpoint{3.126011in}{1.247073in}}{\pgfqpoint{8.795249in}{6.674186in}}%
\pgfusepath{clip}%
\pgfsetrectcap%
\pgfsetroundjoin%
\pgfsetlinewidth{0.501875pt}%
\definecolor{currentstroke}{rgb}{0.000000,0.000000,0.000000}%
\pgfsetstrokecolor{currentstroke}%
\pgfsetstrokeopacity{0.100000}%
\pgfsetdash{}{0pt}%
\pgfpathmoveto{\pgfqpoint{3.126011in}{4.579402in}}%
\pgfpathlineto{\pgfqpoint{11.921260in}{4.579402in}}%
\pgfusepath{stroke}%
\end{pgfscope}%
\begin{pgfscope}%
\pgfsetbuttcap%
\pgfsetroundjoin%
\definecolor{currentfill}{rgb}{0.000000,0.000000,0.000000}%
\pgfsetfillcolor{currentfill}%
\pgfsetlinewidth{0.501875pt}%
\definecolor{currentstroke}{rgb}{0.000000,0.000000,0.000000}%
\pgfsetstrokecolor{currentstroke}%
\pgfsetdash{}{0pt}%
\pgfsys@defobject{currentmarker}{\pgfqpoint{0.000000in}{0.000000in}}{\pgfqpoint{0.034722in}{0.000000in}}{%
\pgfpathmoveto{\pgfqpoint{0.000000in}{0.000000in}}%
\pgfpathlineto{\pgfqpoint{0.034722in}{0.000000in}}%
\pgfusepath{stroke,fill}%
}%
\begin{pgfscope}%
\pgfsys@transformshift{3.126011in}{4.579402in}%
\pgfsys@useobject{currentmarker}{}%
\end{pgfscope}%
\end{pgfscope}%
\begin{pgfscope}%
\definecolor{textcolor}{rgb}{0.000000,0.000000,0.000000}%
\pgfsetstrokecolor{textcolor}%
\pgfsetfillcolor{textcolor}%
\pgftext[x=2.967332in, y=4.484432in, left, base]{\color{textcolor}\sffamily\fontsize{18.000000}{21.600000}\selectfont $\displaystyle 0$}%
\end{pgfscope}%
\begin{pgfscope}%
\pgfpathrectangle{\pgfqpoint{3.126011in}{1.247073in}}{\pgfqpoint{8.795249in}{6.674186in}}%
\pgfusepath{clip}%
\pgfsetrectcap%
\pgfsetroundjoin%
\pgfsetlinewidth{0.501875pt}%
\definecolor{currentstroke}{rgb}{0.000000,0.000000,0.000000}%
\pgfsetstrokecolor{currentstroke}%
\pgfsetstrokeopacity{0.100000}%
\pgfsetdash{}{0pt}%
\pgfpathmoveto{\pgfqpoint{3.126011in}{5.602512in}}%
\pgfpathlineto{\pgfqpoint{11.921260in}{5.602512in}}%
\pgfusepath{stroke}%
\end{pgfscope}%
\begin{pgfscope}%
\pgfsetbuttcap%
\pgfsetroundjoin%
\definecolor{currentfill}{rgb}{0.000000,0.000000,0.000000}%
\pgfsetfillcolor{currentfill}%
\pgfsetlinewidth{0.501875pt}%
\definecolor{currentstroke}{rgb}{0.000000,0.000000,0.000000}%
\pgfsetstrokecolor{currentstroke}%
\pgfsetdash{}{0pt}%
\pgfsys@defobject{currentmarker}{\pgfqpoint{0.000000in}{0.000000in}}{\pgfqpoint{0.034722in}{0.000000in}}{%
\pgfpathmoveto{\pgfqpoint{0.000000in}{0.000000in}}%
\pgfpathlineto{\pgfqpoint{0.034722in}{0.000000in}}%
\pgfusepath{stroke,fill}%
}%
\begin{pgfscope}%
\pgfsys@transformshift{3.126011in}{5.602512in}%
\pgfsys@useobject{currentmarker}{}%
\end{pgfscope}%
\end{pgfscope}%
\begin{pgfscope}%
\definecolor{textcolor}{rgb}{0.000000,0.000000,0.000000}%
\pgfsetstrokecolor{textcolor}%
\pgfsetfillcolor{textcolor}%
\pgftext[x=2.110644in, y=5.507542in, left, base]{\color{textcolor}\sffamily\fontsize{18.000000}{21.600000}\selectfont $\displaystyle 1.0×10^{298}$}%
\end{pgfscope}%
\begin{pgfscope}%
\pgfpathrectangle{\pgfqpoint{3.126011in}{1.247073in}}{\pgfqpoint{8.795249in}{6.674186in}}%
\pgfusepath{clip}%
\pgfsetrectcap%
\pgfsetroundjoin%
\pgfsetlinewidth{0.501875pt}%
\definecolor{currentstroke}{rgb}{0.000000,0.000000,0.000000}%
\pgfsetstrokecolor{currentstroke}%
\pgfsetstrokeopacity{0.100000}%
\pgfsetdash{}{0pt}%
\pgfpathmoveto{\pgfqpoint{3.126011in}{6.625622in}}%
\pgfpathlineto{\pgfqpoint{11.921260in}{6.625622in}}%
\pgfusepath{stroke}%
\end{pgfscope}%
\begin{pgfscope}%
\pgfsetbuttcap%
\pgfsetroundjoin%
\definecolor{currentfill}{rgb}{0.000000,0.000000,0.000000}%
\pgfsetfillcolor{currentfill}%
\pgfsetlinewidth{0.501875pt}%
\definecolor{currentstroke}{rgb}{0.000000,0.000000,0.000000}%
\pgfsetstrokecolor{currentstroke}%
\pgfsetdash{}{0pt}%
\pgfsys@defobject{currentmarker}{\pgfqpoint{0.000000in}{0.000000in}}{\pgfqpoint{0.034722in}{0.000000in}}{%
\pgfpathmoveto{\pgfqpoint{0.000000in}{0.000000in}}%
\pgfpathlineto{\pgfqpoint{0.034722in}{0.000000in}}%
\pgfusepath{stroke,fill}%
}%
\begin{pgfscope}%
\pgfsys@transformshift{3.126011in}{6.625622in}%
\pgfsys@useobject{currentmarker}{}%
\end{pgfscope}%
\end{pgfscope}%
\begin{pgfscope}%
\definecolor{textcolor}{rgb}{0.000000,0.000000,0.000000}%
\pgfsetstrokecolor{textcolor}%
\pgfsetfillcolor{textcolor}%
\pgftext[x=2.110644in, y=6.530651in, left, base]{\color{textcolor}\sffamily\fontsize{18.000000}{21.600000}\selectfont $\displaystyle 2.0×10^{298}$}%
\end{pgfscope}%
\begin{pgfscope}%
\pgfpathrectangle{\pgfqpoint{3.126011in}{1.247073in}}{\pgfqpoint{8.795249in}{6.674186in}}%
\pgfusepath{clip}%
\pgfsetrectcap%
\pgfsetroundjoin%
\pgfsetlinewidth{0.501875pt}%
\definecolor{currentstroke}{rgb}{0.000000,0.000000,0.000000}%
\pgfsetstrokecolor{currentstroke}%
\pgfsetstrokeopacity{0.100000}%
\pgfsetdash{}{0pt}%
\pgfpathmoveto{\pgfqpoint{3.126011in}{7.648732in}}%
\pgfpathlineto{\pgfqpoint{11.921260in}{7.648732in}}%
\pgfusepath{stroke}%
\end{pgfscope}%
\begin{pgfscope}%
\pgfsetbuttcap%
\pgfsetroundjoin%
\definecolor{currentfill}{rgb}{0.000000,0.000000,0.000000}%
\pgfsetfillcolor{currentfill}%
\pgfsetlinewidth{0.501875pt}%
\definecolor{currentstroke}{rgb}{0.000000,0.000000,0.000000}%
\pgfsetstrokecolor{currentstroke}%
\pgfsetdash{}{0pt}%
\pgfsys@defobject{currentmarker}{\pgfqpoint{0.000000in}{0.000000in}}{\pgfqpoint{0.034722in}{0.000000in}}{%
\pgfpathmoveto{\pgfqpoint{0.000000in}{0.000000in}}%
\pgfpathlineto{\pgfqpoint{0.034722in}{0.000000in}}%
\pgfusepath{stroke,fill}%
}%
\begin{pgfscope}%
\pgfsys@transformshift{3.126011in}{7.648732in}%
\pgfsys@useobject{currentmarker}{}%
\end{pgfscope}%
\end{pgfscope}%
\begin{pgfscope}%
\definecolor{textcolor}{rgb}{0.000000,0.000000,0.000000}%
\pgfsetstrokecolor{textcolor}%
\pgfsetfillcolor{textcolor}%
\pgftext[x=2.110644in, y=7.553761in, left, base]{\color{textcolor}\sffamily\fontsize{18.000000}{21.600000}\selectfont $\displaystyle 3.0×10^{298}$}%
\end{pgfscope}%
\begin{pgfscope}%
\pgfpathrectangle{\pgfqpoint{3.126011in}{1.247073in}}{\pgfqpoint{8.795249in}{6.674186in}}%
\pgfusepath{clip}%
\pgfsetbuttcap%
\pgfsetroundjoin%
\pgfsetlinewidth{1.003750pt}%
\definecolor{currentstroke}{rgb}{0.000000,0.605603,0.978680}%
\pgfsetstrokecolor{currentstroke}%
\pgfsetdash{}{0pt}%
\pgfpathmoveto{\pgfqpoint{3.374933in}{4.584973in}}%
\pgfpathlineto{\pgfqpoint{3.378984in}{4.618623in}}%
\pgfpathlineto{\pgfqpoint{3.383036in}{4.494557in}}%
\pgfpathlineto{\pgfqpoint{3.387087in}{4.710822in}}%
\pgfpathlineto{\pgfqpoint{3.391139in}{4.400383in}}%
\pgfpathlineto{\pgfqpoint{3.395190in}{4.807014in}}%
\pgfpathlineto{\pgfqpoint{3.399242in}{4.302414in}}%
\pgfpathlineto{\pgfqpoint{3.403293in}{4.906125in}}%
\pgfpathlineto{\pgfqpoint{3.407345in}{4.203265in}}%
\pgfpathlineto{\pgfqpoint{3.411396in}{5.003715in}}%
\pgfpathlineto{\pgfqpoint{3.415448in}{4.109302in}}%
\pgfpathlineto{\pgfqpoint{3.419499in}{5.091584in}}%
\pgfpathlineto{\pgfqpoint{3.423551in}{4.030266in}}%
\pgfpathlineto{\pgfqpoint{3.427602in}{5.158939in}}%
\pgfpathlineto{\pgfqpoint{3.431653in}{3.977361in}}%
\pgfpathlineto{\pgfqpoint{3.435705in}{5.194906in}}%
\pgfpathlineto{\pgfqpoint{3.439756in}{3.960336in}}%
\pgfpathlineto{\pgfqpoint{3.443808in}{5.191651in}}%
\pgfpathlineto{\pgfqpoint{3.447859in}{3.984396in}}%
\pgfpathlineto{\pgfqpoint{3.451911in}{5.147172in}}%
\pgfpathlineto{\pgfqpoint{3.455962in}{4.047955in}}%
\pgfpathlineto{\pgfqpoint{3.460014in}{5.066808in}}%
\pgfpathlineto{\pgfqpoint{3.464065in}{4.141993in}}%
\pgfpathlineto{\pgfqpoint{3.468117in}{4.962939in}}%
\pgfpathlineto{\pgfqpoint{3.472168in}{4.251326in}}%
\pgfpathlineto{\pgfqpoint{3.476220in}{4.852809in}}%
\pgfpathlineto{\pgfqpoint{3.480271in}{4.357535in}}%
\pgfpathlineto{\pgfqpoint{3.484323in}{4.755045in}}%
\pgfpathlineto{\pgfqpoint{3.488374in}{4.442762in}}%
\pgfpathlineto{\pgfqpoint{3.492426in}{4.685813in}}%
\pgfpathlineto{\pgfqpoint{3.496477in}{4.493329in}}%
\pgfpathlineto{\pgfqpoint{3.500528in}{4.655690in}}%
\pgfpathlineto{\pgfqpoint{3.504580in}{4.502156in}}%
\pgfpathlineto{\pgfqpoint{3.508631in}{4.668110in}}%
\pgfpathlineto{\pgfqpoint{3.512683in}{4.469353in}}%
\pgfpathlineto{\pgfqpoint{3.516734in}{4.719757in}}%
\pgfpathlineto{\pgfqpoint{3.520786in}{4.400900in}}%
\pgfpathlineto{\pgfqpoint{3.524837in}{4.802677in}}%
\pgfpathlineto{\pgfqpoint{3.528889in}{4.305945in}}%
\pgfpathlineto{\pgfqpoint{3.532940in}{4.907338in}}%
\pgfpathlineto{\pgfqpoint{3.536992in}{4.193641in}}%
\pgfpathlineto{\pgfqpoint{3.541043in}{5.025612in}}%
\pgfpathlineto{\pgfqpoint{3.545095in}{4.070603in}}%
\pgfpathlineto{\pgfqpoint{3.549146in}{5.152690in}}%
\pgfpathlineto{\pgfqpoint{3.553198in}{3.939766in}}%
\pgfpathlineto{\pgfqpoint{3.557249in}{5.287355in}}%
\pgfpathlineto{\pgfqpoint{3.561300in}{3.800987in}}%
\pgfpathlineto{\pgfqpoint{3.565352in}{5.430589in}}%
\pgfpathlineto{\pgfqpoint{3.569403in}{3.653079in}}%
\pgfpathlineto{\pgfqpoint{3.573455in}{5.583099in}}%
\pgfpathlineto{\pgfqpoint{3.577506in}{3.496481in}}%
\pgfpathlineto{\pgfqpoint{3.581558in}{5.742719in}}%
\pgfpathlineto{\pgfqpoint{3.585609in}{3.335523in}}%
\pgfpathlineto{\pgfqpoint{3.589661in}{5.902708in}}%
\pgfpathlineto{\pgfqpoint{3.593712in}{3.179378in}}%
\pgfpathlineto{\pgfqpoint{3.597764in}{6.051675in}}%
\pgfpathlineto{\pgfqpoint{3.601815in}{3.041218in}}%
\pgfpathlineto{\pgfqpoint{3.605867in}{6.175306in}}%
\pgfpathlineto{\pgfqpoint{3.609918in}{2.935708in}}%
\pgfpathlineto{\pgfqpoint{3.613970in}{6.259460in}}%
\pgfpathlineto{\pgfqpoint{3.618021in}{2.875572in}}%
\pgfpathlineto{\pgfqpoint{3.622072in}{6.293674in}}%
\pgfpathlineto{\pgfqpoint{3.626124in}{2.868294in}}%
\pgfpathlineto{\pgfqpoint{3.630175in}{6.273969in}}%
\pgfpathlineto{\pgfqpoint{3.634227in}{2.914054in}}%
\pgfpathlineto{\pgfqpoint{3.638278in}{6.203993in}}%
\pgfpathlineto{\pgfqpoint{3.642330in}{3.005624in}}%
\pgfpathlineto{\pgfqpoint{3.646381in}{6.094049in}}%
\pgfpathlineto{\pgfqpoint{3.650433in}{3.130351in}}%
\pgfpathlineto{\pgfqpoint{3.654484in}{5.958244in}}%
\pgfpathlineto{\pgfqpoint{3.658536in}{3.273675in}}%
\pgfpathlineto{\pgfqpoint{3.662587in}{5.810568in}}%
\pgfpathlineto{\pgfqpoint{3.666639in}{3.423139in}}%
\pgfpathlineto{\pgfqpoint{3.670690in}{5.661121in}}%
\pgfpathlineto{\pgfqpoint{3.674742in}{3.571620in}}%
\pgfpathlineto{\pgfqpoint{3.678793in}{5.513679in}}%
\pgfpathlineto{\pgfqpoint{3.682844in}{3.718767in}}%
\pgfpathlineto{\pgfqpoint{3.686896in}{5.365394in}}%
\pgfpathlineto{\pgfqpoint{3.690947in}{3.870117in}}%
\pgfpathlineto{\pgfqpoint{3.694999in}{5.208798in}}%
\pgfpathlineto{\pgfqpoint{3.699050in}{4.034123in}}%
\pgfpathlineto{\pgfqpoint{3.703102in}{5.035515in}}%
\pgfpathlineto{\pgfqpoint{3.707153in}{4.217990in}}%
\pgfpathlineto{\pgfqpoint{3.711205in}{4.840548in}}%
\pgfpathlineto{\pgfqpoint{3.715256in}{4.423613in}}%
\pgfpathlineto{\pgfqpoint{3.719308in}{4.625773in}}%
\pgfpathlineto{\pgfqpoint{3.723359in}{4.644961in}}%
\pgfpathlineto{\pgfqpoint{3.727411in}{4.401438in}}%
\pgfpathlineto{\pgfqpoint{3.731462in}{4.867843in}}%
\pgfpathlineto{\pgfqpoint{3.735514in}{4.185070in}}%
\pgfpathlineto{\pgfqpoint{3.739565in}{5.072298in}}%
\pgfpathlineto{\pgfqpoint{3.743616in}{3.997933in}}%
\pgfpathlineto{\pgfqpoint{3.747668in}{5.237048in}}%
\pgfpathlineto{\pgfqpoint{3.751719in}{3.859968in}}%
\pgfpathlineto{\pgfqpoint{3.755771in}{5.344804in}}%
\pgfpathlineto{\pgfqpoint{3.759822in}{3.784630in}}%
\pgfpathlineto{\pgfqpoint{3.763874in}{5.386899in}}%
\pgfpathlineto{\pgfqpoint{3.767925in}{3.775141in}}%
\pgfpathlineto{\pgfqpoint{3.771977in}{5.365864in}}%
\pgfpathlineto{\pgfqpoint{3.776028in}{3.823289in}}%
\pgfpathlineto{\pgfqpoint{3.780080in}{5.295139in}}%
\pgfpathlineto{\pgfqpoint{3.784131in}{3.911206in}}%
\pgfpathlineto{\pgfqpoint{3.788183in}{5.195930in}}%
\pgfpathlineto{\pgfqpoint{3.792234in}{4.015658in}}%
\pgfpathlineto{\pgfqpoint{3.796286in}{5.092068in}}%
\pgfpathlineto{\pgfqpoint{3.800337in}{4.113660in}}%
\pgfpathlineto{\pgfqpoint{3.804388in}{5.004330in}}%
\pgfpathlineto{\pgfqpoint{3.808440in}{4.187821in}}%
\pgfpathlineto{\pgfqpoint{3.812491in}{4.945821in}}%
\pgfpathlineto{\pgfqpoint{3.816543in}{4.229895in}}%
\pgfpathlineto{\pgfqpoint{3.820594in}{4.919719in}}%
\pgfpathlineto{\pgfqpoint{3.824646in}{4.241600in}}%
\pgfpathlineto{\pgfqpoint{3.828697in}{4.919939in}}%
\pgfpathlineto{\pgfqpoint{3.832749in}{4.232553in}}%
\pgfpathlineto{\pgfqpoint{3.836800in}{4.934398in}}%
\pgfpathlineto{\pgfqpoint{3.840852in}{4.216092in}}%
\pgfpathlineto{\pgfqpoint{3.844903in}{4.949760in}}%
\pgfpathlineto{\pgfqpoint{3.848955in}{4.204356in}}%
\pgfpathlineto{\pgfqpoint{3.853006in}{4.956120in}}%
\pgfpathlineto{\pgfqpoint{3.857058in}{4.204216in}}%
\pgfpathlineto{\pgfqpoint{3.861109in}{4.950144in}}%
\pgfpathlineto{\pgfqpoint{3.865160in}{4.215305in}}%
\pgfpathlineto{\pgfqpoint{3.869212in}{4.935716in}}%
\pgfpathlineto{\pgfqpoint{3.873263in}{4.230727in}}%
\pgfpathlineto{\pgfqpoint{3.877315in}{4.921964in}}%
\pgfpathlineto{\pgfqpoint{3.881366in}{4.240108in}}%
\pgfpathlineto{\pgfqpoint{3.885418in}{4.919408in}}%
\pgfpathlineto{\pgfqpoint{3.889469in}{4.233896in}}%
\pgfpathlineto{\pgfqpoint{3.893521in}{4.935600in}}%
\pgfpathlineto{\pgfqpoint{3.897572in}{4.207396in}}%
\pgfpathlineto{\pgfqpoint{3.901624in}{4.971773in}}%
\pgfpathlineto{\pgfqpoint{3.905675in}{4.163138in}}%
\pgfpathlineto{\pgfqpoint{3.909727in}{5.021678in}}%
\pgfpathlineto{\pgfqpoint{3.913778in}{4.110703in}}%
\pgfpathlineto{\pgfqpoint{3.917830in}{5.073088in}}%
\pgfpathlineto{\pgfqpoint{3.921881in}{4.064018in}}%
\pgfpathlineto{\pgfqpoint{3.925932in}{5.111506in}}%
\pgfpathlineto{\pgfqpoint{3.929984in}{4.036948in}}%
\pgfpathlineto{\pgfqpoint{3.934035in}{5.124885in}}%
\pgfpathlineto{\pgfqpoint{3.938087in}{4.038654in}}%
\pgfpathlineto{\pgfqpoint{3.942138in}{5.107790in}}%
\pgfpathlineto{\pgfqpoint{3.946190in}{4.070296in}}%
\pgfpathlineto{\pgfqpoint{3.950241in}{5.063540in}}%
\pgfpathlineto{\pgfqpoint{3.954293in}{4.124255in}}%
\pgfpathlineto{\pgfqpoint{3.958344in}{5.003502in}}%
\pgfpathlineto{\pgfqpoint{3.962396in}{4.186311in}}%
\pgfpathlineto{\pgfqpoint{3.966447in}{4.943574in}}%
\pgfpathlineto{\pgfqpoint{3.970499in}{4.240251in}}%
\pgfpathlineto{\pgfqpoint{3.974550in}{4.898831in}}%
\pgfpathlineto{\pgfqpoint{3.978602in}{4.273576in}}%
\pgfpathlineto{\pgfqpoint{3.982653in}{4.877886in}}%
\pgfpathlineto{\pgfqpoint{3.986705in}{4.282626in}}%
\pgfpathlineto{\pgfqpoint{3.990756in}{4.878703in}}%
\pgfpathlineto{\pgfqpoint{3.994807in}{4.275494in}}%
\pgfpathlineto{\pgfqpoint{3.998859in}{4.887198in}}%
\pgfpathlineto{\pgfqpoint{4.002910in}{4.271759in}}%
\pgfpathlineto{\pgfqpoint{4.006962in}{4.879200in}}%
\pgfpathlineto{\pgfqpoint{4.011013in}{4.298927in}}%
\pgfpathlineto{\pgfqpoint{4.015065in}{4.825395in}}%
\pgfpathlineto{\pgfqpoint{4.019116in}{4.386414in}}%
\pgfpathlineto{\pgfqpoint{4.023168in}{4.698043in}}%
\pgfpathlineto{\pgfqpoint{4.027219in}{4.558548in}}%
\pgfpathlineto{\pgfqpoint{4.031271in}{4.477815in}}%
\pgfpathlineto{\pgfqpoint{4.035322in}{4.828319in}}%
\pgfpathlineto{\pgfqpoint{4.039374in}{4.159082in}}%
\pgfpathlineto{\pgfqpoint{4.043425in}{5.193346in}}%
\pgfpathlineto{\pgfqpoint{4.047477in}{3.752473in}}%
\pgfpathlineto{\pgfqpoint{4.051528in}{5.634921in}}%
\pgfpathlineto{\pgfqpoint{4.055579in}{3.284220in}}%
\pgfpathlineto{\pgfqpoint{4.059631in}{6.120192in}}%
\pgfpathlineto{\pgfqpoint{4.063682in}{2.792604in}}%
\pgfpathlineto{\pgfqpoint{4.067734in}{6.606860in}}%
\pgfpathlineto{\pgfqpoint{4.071785in}{2.322393in}}%
\pgfpathlineto{\pgfqpoint{4.075837in}{7.049297in}}%
\pgfpathlineto{\pgfqpoint{4.079888in}{1.918466in}}%
\pgfpathlineto{\pgfqpoint{4.083940in}{7.404908in}}%
\pgfpathlineto{\pgfqpoint{4.087991in}{1.619735in}}%
\pgfpathlineto{\pgfqpoint{4.092043in}{7.639696in}}%
\pgfpathlineto{\pgfqpoint{4.096094in}{1.454250in}}%
\pgfpathlineto{\pgfqpoint{4.100146in}{7.732368in}}%
\pgfpathlineto{\pgfqpoint{4.104197in}{1.435966in}}%
\pgfpathlineto{\pgfqpoint{4.108249in}{7.676655in}}%
\pgfpathlineto{\pgfqpoint{4.112300in}{1.563351in}}%
\pgfpathlineto{\pgfqpoint{4.116351in}{7.481773in}}%
\pgfpathlineto{\pgfqpoint{4.120403in}{1.819838in}}%
\pgfpathlineto{\pgfqpoint{4.124454in}{7.171098in}}%
\pgfpathlineto{\pgfqpoint{4.128506in}{2.175991in}}%
\pgfpathlineto{\pgfqpoint{4.132557in}{6.779192in}}%
\pgfpathlineto{\pgfqpoint{4.136609in}{2.593225in}}%
\pgfpathlineto{\pgfqpoint{4.140660in}{6.347411in}}%
\pgfpathlineto{\pgfqpoint{4.144712in}{3.028781in}}%
\pgfpathlineto{\pgfqpoint{4.148763in}{5.918475in}}%
\pgfpathlineto{\pgfqpoint{4.152815in}{3.441445in}}%
\pgfpathlineto{\pgfqpoint{4.156866in}{5.530654in}}%
\pgfpathlineto{\pgfqpoint{4.160918in}{3.797234in}}%
\pgfpathlineto{\pgfqpoint{4.164969in}{5.212461in}}%
\pgfpathlineto{\pgfqpoint{4.169021in}{4.074063in}}%
\pgfpathlineto{\pgfqpoint{4.173072in}{4.978878in}}%
\pgfpathlineto{\pgfqpoint{4.177123in}{4.264403in}}%
\pgfpathlineto{\pgfqpoint{4.181175in}{4.829992in}}%
\pgfpathlineto{\pgfqpoint{4.185226in}{4.375222in}}%
\pgfpathlineto{\pgfqpoint{4.189278in}{4.752526in}}%
\pgfpathlineto{\pgfqpoint{4.193329in}{4.425034in}}%
\pgfpathlineto{\pgfqpoint{4.197381in}{4.724070in}}%
\pgfpathlineto{\pgfqpoint{4.201432in}{4.438617in}}%
\pgfpathlineto{\pgfqpoint{4.205484in}{4.719108in}}%
\pgfpathlineto{\pgfqpoint{4.209535in}{4.440598in}}%
\pgfpathlineto{\pgfqpoint{4.213587in}{4.715411in}}%
\pgfpathlineto{\pgfqpoint{4.217638in}{4.449508in}}%
\pgfpathlineto{\pgfqpoint{4.221690in}{4.699153in}}%
\pgfpathlineto{\pgfqpoint{4.225741in}{4.473831in}}%
\pgfpathlineto{\pgfqpoint{4.229793in}{4.667417in}}%
\pgfpathlineto{\pgfqpoint{4.233844in}{4.511107in}}%
\pgfpathlineto{\pgfqpoint{4.237895in}{4.627451in}}%
\pgfpathlineto{\pgfqpoint{4.241947in}{4.550250in}}%
\pgfpathlineto{\pgfqpoint{4.245998in}{4.592948in}}%
\pgfpathlineto{\pgfqpoint{4.250050in}{4.576370in}}%
\pgfpathlineto{\pgfqpoint{4.254101in}{4.578511in}}%
\pgfpathlineto{\pgfqpoint{4.258153in}{4.576595in}}%
\pgfpathlineto{\pgfqpoint{4.262204in}{4.593994in}}%
\pgfpathlineto{\pgfqpoint{4.266256in}{4.545117in}}%
\pgfpathlineto{\pgfqpoint{4.270307in}{4.640466in}}%
\pgfpathlineto{\pgfqpoint{4.274359in}{4.485907in}}%
\pgfpathlineto{\pgfqpoint{4.278410in}{4.709036in}}%
\pgfpathlineto{\pgfqpoint{4.282462in}{4.412252in}}%
\pgfpathlineto{\pgfqpoint{4.286513in}{4.782910in}}%
\pgfpathlineto{\pgfqpoint{4.290565in}{4.343258in}}%
\pgfpathlineto{\pgfqpoint{4.294616in}{4.842067in}}%
\pgfpathlineto{\pgfqpoint{4.298667in}{4.298380in}}%
\pgfpathlineto{\pgfqpoint{4.302719in}{4.869082in}}%
\pgfpathlineto{\pgfqpoint{4.306770in}{4.291685in}}%
\pgfpathlineto{\pgfqpoint{4.310822in}{4.854322in}}%
\pgfpathlineto{\pgfqpoint{4.314873in}{4.327620in}}%
\pgfpathlineto{\pgfqpoint{4.318925in}{4.798886in}}%
\pgfpathlineto{\pgfqpoint{4.322976in}{4.399608in}}%
\pgfpathlineto{\pgfqpoint{4.327028in}{4.714352in}}%
\pgfpathlineto{\pgfqpoint{4.331079in}{4.491923in}}%
\pgfpathlineto{\pgfqpoint{4.335131in}{4.619421in}}%
\pgfpathlineto{\pgfqpoint{4.339182in}{4.584285in}}%
\pgfpathlineto{\pgfqpoint{4.343234in}{4.534448in}}%
\pgfpathlineto{\pgfqpoint{4.347285in}{4.657767in}}%
\pgfpathlineto{\pgfqpoint{4.351337in}{4.475548in}}%
\pgfpathlineto{\pgfqpoint{4.355388in}{4.700222in}}%
\pgfpathlineto{\pgfqpoint{4.359439in}{4.450060in}}%
\pgfpathlineto{\pgfqpoint{4.363491in}{4.709574in}}%
\pgfpathlineto{\pgfqpoint{4.367542in}{4.454752in}}%
\pgfpathlineto{\pgfqpoint{4.371594in}{4.694001in}}%
\pgfpathlineto{\pgfqpoint{4.375645in}{4.477250in}}%
\pgfpathlineto{\pgfqpoint{4.379697in}{4.668996in}}%
\pgfpathlineto{\pgfqpoint{4.383748in}{4.500242in}}%
\pgfpathlineto{\pgfqpoint{4.387800in}{4.652265in}}%
\pgfpathlineto{\pgfqpoint{4.391851in}{4.507076in}}%
\pgfpathlineto{\pgfqpoint{4.395903in}{4.658066in}}%
\pgfpathlineto{\pgfqpoint{4.399954in}{4.487018in}}%
\pgfpathlineto{\pgfqpoint{4.404006in}{4.692770in}}%
\pgfpathlineto{\pgfqpoint{4.408057in}{4.438530in}}%
\pgfpathlineto{\pgfqpoint{4.412109in}{4.753008in}}%
\pgfpathlineto{\pgfqpoint{4.416160in}{4.369565in}}%
\pgfpathlineto{\pgfqpoint{4.420211in}{4.826945in}}%
\pgfpathlineto{\pgfqpoint{4.424263in}{4.294828in}}%
\pgfpathlineto{\pgfqpoint{4.428314in}{4.898247in}}%
\pgfpathlineto{\pgfqpoint{4.432366in}{4.230907in}}%
\pgfpathlineto{\pgfqpoint{4.436417in}{4.951463in}}%
\pgfpathlineto{\pgfqpoint{4.440469in}{4.190820in}}%
\pgfpathlineto{\pgfqpoint{4.444520in}{4.977105in}}%
\pgfpathlineto{\pgfqpoint{4.448572in}{4.179709in}}%
\pgfpathlineto{\pgfqpoint{4.452623in}{4.974844in}}%
\pgfpathlineto{\pgfqpoint{4.456675in}{4.193013in}}%
\pgfpathlineto{\pgfqpoint{4.460726in}{4.953824in}}%
\pgfpathlineto{\pgfqpoint{4.464778in}{4.217675in}}%
\pgfpathlineto{\pgfqpoint{4.468829in}{4.930034in}}%
\pgfpathlineto{\pgfqpoint{4.472881in}{4.235981in}}%
\pgfpathlineto{\pgfqpoint{4.476932in}{4.921577in}}%
\pgfpathlineto{\pgfqpoint{4.480983in}{4.230797in}}%
\pgfpathlineto{\pgfqpoint{4.485035in}{4.943338in}}%
\pgfpathlineto{\pgfqpoint{4.489086in}{4.190590in}}%
\pgfpathlineto{\pgfqpoint{4.493138in}{5.002670in}}%
\pgfpathlineto{\pgfqpoint{4.497189in}{4.112688in}}%
\pgfpathlineto{\pgfqpoint{4.501241in}{5.097413in}}%
\pgfpathlineto{\pgfqpoint{4.505292in}{4.003865in}}%
\pgfpathlineto{\pgfqpoint{4.509344in}{5.216737in}}%
\pgfpathlineto{\pgfqpoint{4.513395in}{3.878160in}}%
\pgfpathlineto{\pgfqpoint{4.517447in}{5.344461in}}%
\pgfpathlineto{\pgfqpoint{4.521498in}{3.752711in}}%
\pgfpathlineto{\pgfqpoint{4.525550in}{5.463705in}}%
\pgfpathlineto{\pgfqpoint{4.529601in}{3.642986in}}%
\pgfpathlineto{\pgfqpoint{4.533653in}{5.561411in}}%
\pgfpathlineto{\pgfqpoint{4.537704in}{3.558861in}}%
\pgfpathlineto{\pgfqpoint{4.541756in}{5.631372in}}%
\pgfpathlineto{\pgfqpoint{4.545807in}{3.502694in}}%
\pgfpathlineto{\pgfqpoint{4.549858in}{5.674956in}}%
\pgfpathlineto{\pgfqpoint{4.553910in}{3.469807in}}%
\pgfpathlineto{\pgfqpoint{4.557961in}{5.699496in}}%
\pgfpathlineto{\pgfqpoint{4.562013in}{3.451043in}}%
\pgfpathlineto{\pgfqpoint{4.566064in}{5.715039in}}%
\pgfpathlineto{\pgfqpoint{4.570116in}{3.436417in}}%
\pgfpathlineto{\pgfqpoint{4.574167in}{5.730605in}}%
\pgfpathlineto{\pgfqpoint{4.578219in}{3.418648in}}%
\pgfpathlineto{\pgfqpoint{4.582270in}{5.751150in}}%
\pgfpathlineto{\pgfqpoint{4.586322in}{3.395477in}}%
\pgfpathlineto{\pgfqpoint{4.590373in}{5.776103in}}%
\pgfpathlineto{\pgfqpoint{4.594425in}{3.370199in}}%
\pgfpathlineto{\pgfqpoint{4.598476in}{5.799764in}}%
\pgfpathlineto{\pgfqpoint{4.602528in}{3.350419in}}%
\pgfpathlineto{\pgfqpoint{4.606579in}{5.813257in}}%
\pgfpathlineto{\pgfqpoint{4.610630in}{3.345578in}}%
\pgfpathlineto{\pgfqpoint{4.614682in}{5.807291in}}%
\pgfpathlineto{\pgfqpoint{4.618733in}{3.364141in}}%
\pgfpathlineto{\pgfqpoint{4.622785in}{5.774825in}}%
\pgfpathlineto{\pgfqpoint{4.626836in}{3.411256in}}%
\pgfpathlineto{\pgfqpoint{4.630888in}{5.712913in}}%
\pgfpathlineto{\pgfqpoint{4.634939in}{3.487516in}}%
\pgfpathlineto{\pgfqpoint{4.638991in}{5.623312in}}%
\pgfpathlineto{\pgfqpoint{4.643042in}{3.588970in}}%
\pgfpathlineto{\pgfqpoint{4.647094in}{5.511880in}}%
\pgfpathlineto{\pgfqpoint{4.651145in}{3.708230in}}%
\pgfpathlineto{\pgfqpoint{4.655197in}{5.387095in}}%
\pgfpathlineto{\pgfqpoint{4.659248in}{3.836196in}}%
\pgfpathlineto{\pgfqpoint{4.663300in}{5.258221in}}%
\pgfpathlineto{\pgfqpoint{4.667351in}{3.963872in}}%
\pgfpathlineto{\pgfqpoint{4.671402in}{5.133609in}}%
\pgfpathlineto{\pgfqpoint{4.675454in}{4.083850in}}%
\pgfpathlineto{\pgfqpoint{4.679505in}{5.019506in}}%
\pgfpathlineto{\pgfqpoint{4.683557in}{4.191178in}}%
\pgfpathlineto{\pgfqpoint{4.687608in}{4.919520in}}%
\pgfpathlineto{\pgfqpoint{4.691660in}{4.283568in}}%
\pgfpathlineto{\pgfqpoint{4.695711in}{4.834705in}}%
\pgfpathlineto{\pgfqpoint{4.699763in}{4.361060in}}%
\pgfpathlineto{\pgfqpoint{4.703814in}{4.764104in}}%
\pgfpathlineto{\pgfqpoint{4.707866in}{4.425329in}}%
\pgfpathlineto{\pgfqpoint{4.711917in}{4.705529in}}%
\pgfpathlineto{\pgfqpoint{4.715969in}{4.478876in}}%
\pgfpathlineto{\pgfqpoint{4.720020in}{4.656356in}}%
\pgfpathlineto{\pgfqpoint{4.724072in}{4.524283in}}%
\pgfpathlineto{\pgfqpoint{4.728123in}{4.614183in}}%
\pgfpathlineto{\pgfqpoint{4.732174in}{4.563660in}}%
\pgfpathlineto{\pgfqpoint{4.736226in}{4.577263in}}%
\pgfpathlineto{\pgfqpoint{4.740277in}{4.598356in}}%
\pgfpathlineto{\pgfqpoint{4.744329in}{4.544658in}}%
\pgfpathlineto{\pgfqpoint{4.748380in}{4.628916in}}%
\pgfpathlineto{\pgfqpoint{4.752432in}{4.516169in}}%
\pgfpathlineto{\pgfqpoint{4.756483in}{4.655257in}}%
\pgfpathlineto{\pgfqpoint{4.760535in}{4.492085in}}%
\pgfpathlineto{\pgfqpoint{4.764586in}{4.676968in}}%
\pgfpathlineto{\pgfqpoint{4.768638in}{4.472850in}}%
\pgfpathlineto{\pgfqpoint{4.772689in}{4.693654in}}%
\pgfpathlineto{\pgfqpoint{4.776741in}{4.458738in}}%
\pgfpathlineto{\pgfqpoint{4.780792in}{4.705226in}}%
\pgfpathlineto{\pgfqpoint{4.784844in}{4.449606in}}%
\pgfpathlineto{\pgfqpoint{4.788895in}{4.712086in}}%
\pgfpathlineto{\pgfqpoint{4.792946in}{4.444785in}}%
\pgfpathlineto{\pgfqpoint{4.796998in}{4.715164in}}%
\pgfpathlineto{\pgfqpoint{4.801049in}{4.443105in}}%
\pgfpathlineto{\pgfqpoint{4.805101in}{4.715826in}}%
\pgfpathlineto{\pgfqpoint{4.809152in}{4.443054in}}%
\pgfpathlineto{\pgfqpoint{4.813204in}{4.715681in}}%
\pgfpathlineto{\pgfqpoint{4.817255in}{4.442988in}}%
\pgfpathlineto{\pgfqpoint{4.821307in}{4.716346in}}%
\pgfpathlineto{\pgfqpoint{4.825358in}{4.441373in}}%
\pgfpathlineto{\pgfqpoint{4.829410in}{4.719215in}}%
\pgfpathlineto{\pgfqpoint{4.833461in}{4.436999in}}%
\pgfpathlineto{\pgfqpoint{4.837513in}{4.725277in}}%
\pgfpathlineto{\pgfqpoint{4.841564in}{4.429140in}}%
\pgfpathlineto{\pgfqpoint{4.845616in}{4.734966in}}%
\pgfpathlineto{\pgfqpoint{4.849667in}{4.417674in}}%
\pgfpathlineto{\pgfqpoint{4.853718in}{4.748070in}}%
\pgfpathlineto{\pgfqpoint{4.857770in}{4.403162in}}%
\pgfpathlineto{\pgfqpoint{4.861821in}{4.763668in}}%
\pgfpathlineto{\pgfqpoint{4.865873in}{4.386891in}}%
\pgfpathlineto{\pgfqpoint{4.869924in}{4.780106in}}%
\pgfpathlineto{\pgfqpoint{4.873976in}{4.370879in}}%
\pgfpathlineto{\pgfqpoint{4.878027in}{4.795020in}}%
\pgfpathlineto{\pgfqpoint{4.882079in}{4.357808in}}%
\pgfpathlineto{\pgfqpoint{4.886130in}{4.805443in}}%
\pgfpathlineto{\pgfqpoint{4.890182in}{4.350879in}}%
\pgfpathlineto{\pgfqpoint{4.894233in}{4.808008in}}%
\pgfpathlineto{\pgfqpoint{4.898285in}{4.353549in}}%
\pgfpathlineto{\pgfqpoint{4.902336in}{4.799255in}}%
\pgfpathlineto{\pgfqpoint{4.906388in}{4.369183in}}%
\pgfpathlineto{\pgfqpoint{4.910439in}{4.776020in}}%
\pgfpathlineto{\pgfqpoint{4.914490in}{4.400636in}}%
\pgfpathlineto{\pgfqpoint{4.918542in}{4.735858in}}%
\pgfpathlineto{\pgfqpoint{4.922593in}{4.449847in}}%
\pgfpathlineto{\pgfqpoint{4.926645in}{4.677427in}}%
\pgfpathlineto{\pgfqpoint{4.930696in}{4.517485in}}%
\pgfpathlineto{\pgfqpoint{4.934748in}{4.600789in}}%
\pgfpathlineto{\pgfqpoint{4.938799in}{4.602718in}}%
\pgfpathlineto{\pgfqpoint{4.942851in}{4.507568in}}%
\pgfpathlineto{\pgfqpoint{4.946902in}{4.703122in}}%
\pgfpathlineto{\pgfqpoint{4.950954in}{4.400976in}}%
\pgfpathlineto{\pgfqpoint{4.955005in}{4.814725in}}%
\pgfpathlineto{\pgfqpoint{4.959057in}{4.285705in}}%
\pgfpathlineto{\pgfqpoint{4.963108in}{4.932172in}}%
\pgfpathlineto{\pgfqpoint{4.967160in}{4.167707in}}%
\pgfpathlineto{\pgfqpoint{4.971211in}{5.048989in}}%
\pgfpathlineto{\pgfqpoint{4.975262in}{4.053886in}}%
\pgfpathlineto{\pgfqpoint{4.979314in}{5.157944in}}%
\pgfpathlineto{\pgfqpoint{4.983365in}{3.951691in}}%
\pgfpathlineto{\pgfqpoint{4.987417in}{5.251497in}}%
\pgfpathlineto{\pgfqpoint{4.991468in}{3.868608in}}%
\pgfpathlineto{\pgfqpoint{4.995520in}{5.322372in}}%
\pgfpathlineto{\pgfqpoint{4.999571in}{3.811543in}}%
\pgfpathlineto{\pgfqpoint{5.003623in}{5.364209in}}%
\pgfpathlineto{\pgfqpoint{5.007674in}{3.786127in}}%
\pgfpathlineto{\pgfqpoint{5.011726in}{5.372280in}}%
\pgfpathlineto{\pgfqpoint{5.015777in}{3.796017in}}%
\pgfpathlineto{\pgfqpoint{5.019829in}{5.344153in}}%
\pgfpathlineto{\pgfqpoint{5.023880in}{3.842294in}}%
\pgfpathlineto{\pgfqpoint{5.027932in}{5.280192in}}%
\pgfpathlineto{\pgfqpoint{5.031983in}{3.923093in}}%
\pgfpathlineto{\pgfqpoint{5.036034in}{5.183767in}}%
\pgfpathlineto{\pgfqpoint{5.040086in}{4.033585in}}%
\pgfpathlineto{\pgfqpoint{5.044137in}{5.061079in}}%
\pgfpathlineto{\pgfqpoint{5.048189in}{4.166337in}}%
\pgfpathlineto{\pgfqpoint{5.052240in}{4.920601in}}%
\pgfpathlineto{\pgfqpoint{5.056292in}{4.312056in}}%
\pgfpathlineto{\pgfqpoint{5.060343in}{4.772203in}}%
\pgfpathlineto{\pgfqpoint{5.064395in}{4.460562in}}%
\pgfpathlineto{\pgfqpoint{5.068446in}{4.626107in}}%
\pgfpathlineto{\pgfqpoint{5.072498in}{4.601847in}}%
\pgfpathlineto{\pgfqpoint{5.076549in}{4.491864in}}%
\pgfpathlineto{\pgfqpoint{5.080601in}{4.727025in}}%
\pgfpathlineto{\pgfqpoint{5.084652in}{4.377527in}}%
\pgfpathlineto{\pgfqpoint{5.088704in}{4.829013in}}%
\pgfpathlineto{\pgfqpoint{5.092755in}{4.289118in}}%
\pgfpathlineto{\pgfqpoint{5.096807in}{4.902892in}}%
\pgfpathlineto{\pgfqpoint{5.100858in}{4.230447in}}%
\pgfpathlineto{\pgfqpoint{5.104909in}{4.945935in}}%
\pgfpathlineto{\pgfqpoint{5.108961in}{4.203213in}}%
\pgfpathlineto{\pgfqpoint{5.113012in}{4.957397in}}%
\pgfpathlineto{\pgfqpoint{5.117064in}{4.207297in}}%
\pgfpathlineto{\pgfqpoint{5.121115in}{4.938164in}}%
\pgfpathlineto{\pgfqpoint{5.125167in}{4.241137in}}%
\pgfpathlineto{\pgfqpoint{5.129218in}{4.890381in}}%
\pgfpathlineto{\pgfqpoint{5.133270in}{4.302094in}}%
\pgfpathlineto{\pgfqpoint{5.137321in}{4.817105in}}%
\pgfpathlineto{\pgfqpoint{5.141373in}{4.386764in}}%
\pgfpathlineto{\pgfqpoint{5.145424in}{4.722019in}}%
\pgfpathlineto{\pgfqpoint{5.149476in}{4.491249in}}%
\pgfpathlineto{\pgfqpoint{5.153527in}{4.609177in}}%
\pgfpathlineto{\pgfqpoint{5.157579in}{4.611398in}}%
\pgfpathlineto{\pgfqpoint{5.161630in}{4.482765in}}%
\pgfpathlineto{\pgfqpoint{5.165681in}{4.743051in}}%
\pgfpathlineto{\pgfqpoint{5.169733in}{4.346853in}}%
\pgfpathlineto{\pgfqpoint{5.173784in}{4.882294in}}%
\pgfpathlineto{\pgfqpoint{5.177836in}{4.205143in}}%
\pgfpathlineto{\pgfqpoint{5.181887in}{5.025687in}}%
\pgfpathlineto{\pgfqpoint{5.185939in}{4.060763in}}%
\pgfpathlineto{\pgfqpoint{5.189990in}{5.170446in}}%
\pgfpathlineto{\pgfqpoint{5.194042in}{3.916138in}}%
\pgfpathlineto{\pgfqpoint{5.198093in}{5.314513in}}%
\pgfpathlineto{\pgfqpoint{5.202145in}{3.772969in}}%
\pgfpathlineto{\pgfqpoint{5.206196in}{5.456519in}}%
\pgfpathlineto{\pgfqpoint{5.210248in}{3.632332in}}%
\pgfpathlineto{\pgfqpoint{5.214299in}{5.595632in}}%
\pgfpathlineto{\pgfqpoint{5.218351in}{3.494868in}}%
\pgfpathlineto{\pgfqpoint{5.222402in}{5.731339in}}%
\pgfpathlineto{\pgfqpoint{5.226453in}{3.361023in}}%
\pgfpathlineto{\pgfqpoint{5.230505in}{5.863203in}}%
\pgfpathlineto{\pgfqpoint{5.234556in}{3.231283in}}%
\pgfpathlineto{\pgfqpoint{5.238608in}{5.990640in}}%
\pgfpathlineto{\pgfqpoint{5.242659in}{3.106370in}}%
\pgfpathlineto{\pgfqpoint{5.246711in}{6.112757in}}%
\pgfpathlineto{\pgfqpoint{5.250762in}{2.987376in}}%
\pgfpathlineto{\pgfqpoint{5.254814in}{6.228242in}}%
\pgfpathlineto{\pgfqpoint{5.258865in}{2.875845in}}%
\pgfpathlineto{\pgfqpoint{5.262917in}{6.335315in}}%
\pgfpathlineto{\pgfqpoint{5.266968in}{2.773793in}}%
\pgfpathlineto{\pgfqpoint{5.271020in}{6.431727in}}%
\pgfpathlineto{\pgfqpoint{5.275071in}{2.683691in}}%
\pgfpathlineto{\pgfqpoint{5.279123in}{6.514805in}}%
\pgfpathlineto{\pgfqpoint{5.283174in}{2.608388in}}%
\pgfpathlineto{\pgfqpoint{5.287225in}{6.581558in}}%
\pgfpathlineto{\pgfqpoint{5.291277in}{2.550972in}}%
\pgfpathlineto{\pgfqpoint{5.295328in}{6.628855in}}%
\pgfpathlineto{\pgfqpoint{5.299380in}{2.514551in}}%
\pgfpathlineto{\pgfqpoint{5.303431in}{6.653689in}}%
\pgfpathlineto{\pgfqpoint{5.307483in}{2.501948in}}%
\pgfpathlineto{\pgfqpoint{5.311534in}{6.653508in}}%
\pgfpathlineto{\pgfqpoint{5.315586in}{2.515349in}}%
\pgfpathlineto{\pgfqpoint{5.319637in}{6.626586in}}%
\pgfpathlineto{\pgfqpoint{5.323689in}{2.555937in}}%
\pgfpathlineto{\pgfqpoint{5.327740in}{6.572361in}}%
\pgfpathlineto{\pgfqpoint{5.331792in}{2.623585in}}%
\pgfpathlineto{\pgfqpoint{5.335843in}{6.491696in}}%
\pgfpathlineto{\pgfqpoint{5.339895in}{2.716669in}}%
\pgfpathlineto{\pgfqpoint{5.343946in}{6.386980in}}%
\pgfpathlineto{\pgfqpoint{5.347997in}{2.832051in}}%
\pgfpathlineto{\pgfqpoint{5.352049in}{6.262062in}}%
\pgfpathlineto{\pgfqpoint{5.356100in}{2.965233in}}%
\pgfpathlineto{\pgfqpoint{5.360152in}{6.122010in}}%
\pgfpathlineto{\pgfqpoint{5.364203in}{3.110668in}}%
\pgfpathlineto{\pgfqpoint{5.368255in}{5.972746in}}%
\pgfpathlineto{\pgfqpoint{5.372306in}{3.262168in}}%
\pgfpathlineto{\pgfqpoint{5.376358in}{5.820610in}}%
\pgfpathlineto{\pgfqpoint{5.380409in}{3.413353in}}%
\pgfpathlineto{\pgfqpoint{5.384461in}{5.671920in}}%
\pgfpathlineto{\pgfqpoint{5.388512in}{3.558070in}}%
\pgfpathlineto{\pgfqpoint{5.392564in}{5.532568in}}%
\pgfpathlineto{\pgfqpoint{5.396615in}{3.690768in}}%
\pgfpathlineto{\pgfqpoint{5.400667in}{5.407694in}}%
\pgfpathlineto{\pgfqpoint{5.404718in}{3.806783in}}%
\pgfpathlineto{\pgfqpoint{5.408769in}{5.301429in}}%
\pgfpathlineto{\pgfqpoint{5.412821in}{3.902562in}}%
\pgfpathlineto{\pgfqpoint{5.416872in}{5.216710in}}%
\pgfpathlineto{\pgfqpoint{5.420924in}{3.975815in}}%
\pgfpathlineto{\pgfqpoint{5.424975in}{5.155158in}}%
\pgfpathlineto{\pgfqpoint{5.429027in}{4.025605in}}%
\pgfpathlineto{\pgfqpoint{5.433078in}{5.117016in}}%
\pgfpathlineto{\pgfqpoint{5.437130in}{4.052384in}}%
\pgfpathlineto{\pgfqpoint{5.441181in}{5.101147in}}%
\pgfpathlineto{\pgfqpoint{5.445233in}{4.057959in}}%
\pgfpathlineto{\pgfqpoint{5.449284in}{5.105094in}}%
\pgfpathlineto{\pgfqpoint{5.453336in}{4.045407in}}%
\pgfpathlineto{\pgfqpoint{5.457387in}{5.125196in}}%
\pgfpathlineto{\pgfqpoint{5.461439in}{4.018936in}}%
\pgfpathlineto{\pgfqpoint{5.465490in}{5.156742in}}%
\pgfpathlineto{\pgfqpoint{5.469541in}{3.983715in}}%
\pgfpathlineto{\pgfqpoint{5.473593in}{5.194148in}}%
\pgfpathlineto{\pgfqpoint{5.477644in}{3.945693in}}%
\pgfpathlineto{\pgfqpoint{5.481696in}{5.231149in}}%
\pgfpathlineto{\pgfqpoint{5.485747in}{3.911404in}}%
\pgfpathlineto{\pgfqpoint{5.489799in}{5.260998in}}%
\pgfpathlineto{\pgfqpoint{5.493850in}{3.887745in}}%
\pgfpathlineto{\pgfqpoint{5.497902in}{5.276714in}}%
\pgfpathlineto{\pgfqpoint{5.501953in}{3.881702in}}%
\pgfpathlineto{\pgfqpoint{5.506005in}{5.271400in}}%
\pgfpathlineto{\pgfqpoint{5.510056in}{3.899978in}}%
\pgfpathlineto{\pgfqpoint{5.514108in}{5.238673in}}%
\pgfpathlineto{\pgfqpoint{5.518159in}{3.948497in}}%
\pgfpathlineto{\pgfqpoint{5.522211in}{5.173210in}}%
\pgfpathlineto{\pgfqpoint{5.526262in}{4.031823in}}%
\pgfpathlineto{\pgfqpoint{5.530313in}{5.071371in}}%
\pgfpathlineto{\pgfqpoint{5.534365in}{4.152521in}}%
\pgfpathlineto{\pgfqpoint{5.538416in}{4.931801in}}%
\pgfpathlineto{\pgfqpoint{5.542468in}{4.310620in}}%
\pgfpathlineto{\pgfqpoint{5.546519in}{4.755882in}}%
\pgfpathlineto{\pgfqpoint{5.550571in}{4.503285in}}%
\pgfpathlineto{\pgfqpoint{5.554622in}{4.547896in}}%
\pgfpathlineto{\pgfqpoint{5.558674in}{4.724847in}}%
\pgfpathlineto{\pgfqpoint{5.562725in}{4.314783in}}%
\pgfpathlineto{\pgfqpoint{5.566777in}{4.967261in}}%
\pgfpathlineto{\pgfqpoint{5.570828in}{4.065478in}}%
\pgfpathlineto{\pgfqpoint{5.574880in}{5.220958in}}%
\pgfpathlineto{\pgfqpoint{5.578931in}{3.809899in}}%
\pgfpathlineto{\pgfqpoint{5.582983in}{5.475969in}}%
\pgfpathlineto{\pgfqpoint{5.587034in}{3.557774in}}%
\pgfpathlineto{\pgfqpoint{5.591085in}{5.723085in}}%
\pgfpathlineto{\pgfqpoint{5.595137in}{3.317552in}}%
\pgfpathlineto{\pgfqpoint{5.599188in}{5.954802in}}%
\pgfpathlineto{\pgfqpoint{5.603240in}{3.095661in}}%
\pgfpathlineto{\pgfqpoint{5.607291in}{6.165831in}}%
\pgfpathlineto{\pgfqpoint{5.611343in}{2.896265in}}%
\pgfpathlineto{\pgfqpoint{5.615394in}{6.353056in}}%
\pgfpathlineto{\pgfqpoint{5.619446in}{2.721562in}}%
\pgfpathlineto{\pgfqpoint{5.623497in}{6.515016in}}%
\pgfpathlineto{\pgfqpoint{5.627549in}{2.572486in}}%
\pgfpathlineto{\pgfqpoint{5.631600in}{6.651095in}}%
\pgfpathlineto{\pgfqpoint{5.635652in}{2.449538in}}%
\pgfpathlineto{\pgfqpoint{5.639703in}{6.760734in}}%
\pgfpathlineto{\pgfqpoint{5.643755in}{2.353444in}}%
\pgfpathlineto{\pgfqpoint{5.647806in}{6.842990in}}%
\pgfpathlineto{\pgfqpoint{5.651858in}{2.285351in}}%
\pgfpathlineto{\pgfqpoint{5.655909in}{6.896609in}}%
\pgfpathlineto{\pgfqpoint{5.659960in}{2.246448in}}%
\pgfpathlineto{\pgfqpoint{5.664012in}{6.920687in}}%
\pgfpathlineto{\pgfqpoint{5.668063in}{2.237095in}}%
\pgfpathlineto{\pgfqpoint{5.672115in}{6.915692in}}%
\pgfpathlineto{\pgfqpoint{5.676166in}{2.255721in}}%
\pgfpathlineto{\pgfqpoint{5.680218in}{6.884540in}}%
\pgfpathlineto{\pgfqpoint{5.684269in}{2.297880in}}%
\pgfpathlineto{\pgfqpoint{5.688321in}{6.833311in}}%
\pgfpathlineto{\pgfqpoint{5.692372in}{2.355847in}}%
\pgfpathlineto{\pgfqpoint{5.696424in}{6.771283in}}%
\pgfpathlineto{\pgfqpoint{5.700475in}{2.418992in}}%
\pgfpathlineto{\pgfqpoint{5.704527in}{6.710136in}}%
\pgfpathlineto{\pgfqpoint{5.708578in}{2.474969in}}%
\pgfpathlineto{\pgfqpoint{5.712630in}{6.662427in}}%
\pgfpathlineto{\pgfqpoint{5.716681in}{2.511511in}}%
\pgfpathlineto{\pgfqpoint{5.720732in}{6.639626in}}%
\pgfpathlineto{\pgfqpoint{5.724784in}{2.518438in}}%
\pgfpathlineto{\pgfqpoint{5.728835in}{6.650173in}}%
\pgfpathlineto{\pgfqpoint{5.732887in}{2.489425in}}%
\pgfpathlineto{\pgfqpoint{5.736938in}{6.697994in}}%
\pgfpathlineto{\pgfqpoint{5.740990in}{2.423122in}}%
\pgfpathlineto{\pgfqpoint{5.745041in}{6.781798in}}%
\pgfpathlineto{\pgfqpoint{5.749093in}{2.323409in}}%
\pgfpathlineto{\pgfqpoint{5.753144in}{6.895281in}}%
\pgfpathlineto{\pgfqpoint{5.757196in}{2.198755in}}%
\pgfpathlineto{\pgfqpoint{5.761247in}{7.028144in}}%
\pgfpathlineto{\pgfqpoint{5.765299in}{2.060897in}}%
\pgfpathlineto{\pgfqpoint{5.769350in}{7.167647in}}%
\pgfpathlineto{\pgfqpoint{5.773402in}{1.923128in}}%
\pgfpathlineto{\pgfqpoint{5.777453in}{7.300378in}}%
\pgfpathlineto{\pgfqpoint{5.781504in}{1.798567in}}%
\pgfpathlineto{\pgfqpoint{5.785556in}{7.413888in}}%
\pgfpathlineto{\pgfqpoint{5.789607in}{1.698672in}}%
\pgfpathlineto{\pgfqpoint{5.793659in}{7.497972in}}%
\pgfpathlineto{\pgfqpoint{5.797710in}{1.632190in}}%
\pgfpathlineto{\pgfqpoint{5.801762in}{7.545485in}}%
\pgfpathlineto{\pgfqpoint{5.805813in}{1.604583in}}%
\pgfpathlineto{\pgfqpoint{5.809865in}{7.552679in}}%
\pgfpathlineto{\pgfqpoint{5.813916in}{1.617899in}}%
\pgfpathlineto{\pgfqpoint{5.817968in}{7.519142in}}%
\pgfpathlineto{\pgfqpoint{5.822019in}{1.671015in}}%
\pgfpathlineto{\pgfqpoint{5.826071in}{7.447400in}}%
\pgfpathlineto{\pgfqpoint{5.830122in}{1.760164in}}%
\pgfpathlineto{\pgfqpoint{5.834174in}{7.342274in}}%
\pgfpathlineto{\pgfqpoint{5.838225in}{1.879680in}}%
\pgfpathlineto{\pgfqpoint{5.842276in}{7.210049in}}%
\pgfpathlineto{\pgfqpoint{5.846328in}{2.022898in}}%
\pgfpathlineto{\pgfqpoint{5.850379in}{7.057530in}}%
\pgfpathlineto{\pgfqpoint{5.854431in}{2.183109in}}%
\pgfpathlineto{\pgfqpoint{5.858482in}{6.891101in}}%
\pgfpathlineto{\pgfqpoint{5.862534in}{2.354459in}}%
\pgfpathlineto{\pgfqpoint{5.866585in}{6.715920in}}%
\pgfpathlineto{\pgfqpoint{5.870637in}{2.532606in}}%
\pgfpathlineto{\pgfqpoint{5.874688in}{6.535447in}}%
\pgfpathlineto{\pgfqpoint{5.878740in}{2.714976in}}%
\pgfpathlineto{\pgfqpoint{5.882791in}{6.351437in}}%
\pgfpathlineto{\pgfqpoint{5.886843in}{2.900491in}}%
\pgfpathlineto{\pgfqpoint{5.890894in}{6.164495in}}%
\pgfpathlineto{\pgfqpoint{5.894946in}{3.088761in}}%
\pgfpathlineto{\pgfqpoint{5.898997in}{5.975096in}}%
\pgfpathlineto{\pgfqpoint{5.903048in}{3.278903in}}%
\pgfpathlineto{\pgfqpoint{5.907100in}{5.784856in}}%
\pgfpathlineto{\pgfqpoint{5.911151in}{3.468274in}}%
\pgfpathlineto{\pgfqpoint{5.915203in}{5.597688in}}%
\pgfpathlineto{\pgfqpoint{5.919254in}{3.651523in}}%
\pgfpathlineto{\pgfqpoint{5.923306in}{5.420446in}}%
\pgfpathlineto{\pgfqpoint{5.927357in}{3.820340in}}%
\pgfpathlineto{\pgfqpoint{5.931409in}{5.262729in}}%
\pgfpathlineto{\pgfqpoint{5.935460in}{3.964126in}}%
\pgfpathlineto{\pgfqpoint{5.939512in}{5.135734in}}%
\pgfpathlineto{\pgfqpoint{5.943563in}{4.071586in}}%
\pgfpathlineto{\pgfqpoint{5.947615in}{5.050285in}}%
\pgfpathlineto{\pgfqpoint{5.951666in}{4.132970in}}%
\pgfpathlineto{\pgfqpoint{5.955718in}{5.014457in}}%
\pgfpathlineto{\pgfqpoint{5.959769in}{4.142439in}}%
\pgfpathlineto{\pgfqpoint{5.963820in}{5.031365in}}%
\pgfpathlineto{\pgfqpoint{5.967872in}{4.099977in}}%
\pgfpathlineto{\pgfqpoint{5.971923in}{5.097695in}}%
\pgfpathlineto{\pgfqpoint{5.975975in}{4.012305in}}%
\pgfpathlineto{\pgfqpoint{5.980026in}{5.203411in}}%
\pgfpathlineto{\pgfqpoint{5.984078in}{3.892513in}}%
\pgfpathlineto{\pgfqpoint{5.988129in}{5.332778in}}%
\pgfpathlineto{\pgfqpoint{5.992181in}{3.758442in}}%
\pgfpathlineto{\pgfqpoint{5.996232in}{5.466487in}}%
\pgfpathlineto{\pgfqpoint{6.000284in}{3.630173in}}%
\pgfpathlineto{\pgfqpoint{6.004335in}{5.584406in}}%
\pgfpathlineto{\pgfqpoint{6.008387in}{3.527189in}}%
\pgfpathlineto{\pgfqpoint{6.012438in}{5.668338in}}%
\pgfpathlineto{\pgfqpoint{6.016490in}{3.465848in}}%
\pgfpathlineto{\pgfqpoint{6.020541in}{5.704203in}}%
\pgfpathlineto{\pgfqpoint{6.024592in}{3.457646in}}%
\pgfpathlineto{\pgfqpoint{6.028644in}{5.683265in}}%
\pgfpathlineto{\pgfqpoint{6.032695in}{3.508507in}}%
\pgfpathlineto{\pgfqpoint{6.036747in}{5.602347in}}%
\pgfpathlineto{\pgfqpoint{6.040798in}{3.619020in}}%
\pgfpathlineto{\pgfqpoint{6.044850in}{5.463228in}}%
\pgfpathlineto{\pgfqpoint{6.048901in}{3.785303in}}%
\pgfpathlineto{\pgfqpoint{6.052953in}{5.271615in}}%
\pgfpathlineto{\pgfqpoint{6.057004in}{4.000077in}}%
\pgfpathlineto{\pgfqpoint{6.061056in}{5.036135in}}%
\pgfpathlineto{\pgfqpoint{6.065107in}{4.253552in}}%
\pgfpathlineto{\pgfqpoint{6.069159in}{4.767609in}}%
\pgfpathlineto{\pgfqpoint{6.073210in}{4.533960in}}%
\pgfpathlineto{\pgfqpoint{6.077262in}{4.478707in}}%
\pgfpathlineto{\pgfqpoint{6.081313in}{4.827758in}}%
\pgfpathlineto{\pgfqpoint{6.085364in}{4.183812in}}%
\pgfpathlineto{\pgfqpoint{6.089416in}{5.119773in}}%
\pgfpathlineto{\pgfqpoint{6.093467in}{3.898795in}}%
\pgfpathlineto{\pgfqpoint{6.097519in}{5.393590in}}%
\pgfpathlineto{\pgfqpoint{6.101570in}{3.640387in}}%
\pgfpathlineto{\pgfqpoint{6.105622in}{5.632467in}}%
\pgfpathlineto{\pgfqpoint{6.109673in}{3.424968in}}%
\pgfpathlineto{\pgfqpoint{6.113725in}{5.820819in}}%
\pgfpathlineto{\pgfqpoint{6.117776in}{3.266850in}}%
\pgfpathlineto{\pgfqpoint{6.121828in}{5.946096in}}%
\pgfpathlineto{\pgfqpoint{6.125879in}{3.176349in}}%
\pgfpathlineto{\pgfqpoint{6.129931in}{6.000648in}}%
\pgfpathlineto{\pgfqpoint{6.133982in}{3.158102in}}%
\pgfpathlineto{\pgfqpoint{6.138034in}{5.983084in}}%
\pgfpathlineto{\pgfqpoint{6.142085in}{3.210141in}}%
\pgfpathlineto{\pgfqpoint{6.146136in}{5.898700in}}%
\pgfpathlineto{\pgfqpoint{6.150188in}{3.324025in}}%
\pgfpathlineto{\pgfqpoint{6.154239in}{5.758758in}}%
\pgfpathlineto{\pgfqpoint{6.158291in}{3.486125in}}%
\pgfpathlineto{\pgfqpoint{6.162342in}{5.578697in}}%
\pgfpathlineto{\pgfqpoint{6.166394in}{3.679823in}}%
\pgfpathlineto{\pgfqpoint{6.170445in}{5.375644in}}%
\pgfpathlineto{\pgfqpoint{6.174497in}{3.888152in}}%
\pgfpathlineto{\pgfqpoint{6.178548in}{5.165774in}}%
\pgfpathlineto{\pgfqpoint{6.182600in}{4.096287in}}%
\pgfpathlineto{\pgfqpoint{6.186651in}{4.962104in}}%
\pgfpathlineto{\pgfqpoint{6.190703in}{4.293358in}}%
\pgfpathlineto{\pgfqpoint{6.194754in}{4.773156in}}%
\pgfpathlineto{\pgfqpoint{6.198806in}{4.473254in}}%
\pgfpathlineto{\pgfqpoint{6.202857in}{4.602698in}}%
\pgfpathlineto{\pgfqpoint{6.206909in}{4.634362in}}%
\pgfpathlineto{\pgfqpoint{6.210960in}{4.450465in}}%
\pgfpathlineto{\pgfqpoint{6.215011in}{4.778480in}}%
\pgfpathlineto{\pgfqpoint{6.219063in}{4.313515in}}%
\pgfpathlineto{\pgfqpoint{6.223114in}{4.909301in}}%
\pgfpathlineto{\pgfqpoint{6.227166in}{4.187774in}}%
\pgfpathlineto{\pgfqpoint{6.231217in}{5.030949in}}%
\pgfpathlineto{\pgfqpoint{6.235269in}{4.069342in}}%
\pgfpathlineto{\pgfqpoint{6.239320in}{5.146904in}}%
\pgfpathlineto{\pgfqpoint{6.243372in}{3.955281in}}%
\pgfpathlineto{\pgfqpoint{6.247423in}{5.259503in}}%
\pgfpathlineto{\pgfqpoint{6.251475in}{3.843851in}}%
\pgfpathlineto{\pgfqpoint{6.255526in}{5.369936in}}%
\pgfpathlineto{\pgfqpoint{6.259578in}{3.734341in}}%
\pgfpathlineto{\pgfqpoint{6.263629in}{5.478510in}}%
\pgfpathlineto{\pgfqpoint{6.267681in}{3.626791in}}%
\pgfpathlineto{\pgfqpoint{6.271732in}{5.584872in}}%
\pgfpathlineto{\pgfqpoint{6.275783in}{3.521874in}}%
\pgfpathlineto{\pgfqpoint{6.279835in}{5.687976in}}%
\pgfpathlineto{\pgfqpoint{6.283886in}{3.421096in}}%
\pgfpathlineto{\pgfqpoint{6.287938in}{5.785736in}}%
\pgfpathlineto{\pgfqpoint{6.291989in}{3.327260in}}%
\pgfpathlineto{\pgfqpoint{6.296041in}{5.874498in}}%
\pgfpathlineto{\pgfqpoint{6.300092in}{3.244982in}}%
\pgfpathlineto{\pgfqpoint{6.304144in}{5.948622in}}%
\pgfpathlineto{\pgfqpoint{6.308195in}{3.180925in}}%
\pgfpathlineto{\pgfqpoint{6.312247in}{6.000496in}}%
\pgfpathlineto{\pgfqpoint{6.316298in}{3.143485in}}%
\pgfpathlineto{\pgfqpoint{6.320350in}{6.021195in}}%
\pgfpathlineto{\pgfqpoint{6.324401in}{3.141790in}}%
\pgfpathlineto{\pgfqpoint{6.328453in}{6.001784in}}%
\pgfpathlineto{\pgfqpoint{6.332504in}{3.184135in}}%
\pgfpathlineto{\pgfqpoint{6.336555in}{5.935062in}}%
\pgfpathlineto{\pgfqpoint{6.340607in}{3.276188in}}%
\pgfpathlineto{\pgfqpoint{6.344658in}{5.817298in}}%
\pgfpathlineto{\pgfqpoint{6.348710in}{3.419410in}}%
\pgfpathlineto{\pgfqpoint{6.352761in}{5.649532in}}%
\pgfpathlineto{\pgfqpoint{6.356813in}{3.610145in}}%
\pgfpathlineto{\pgfqpoint{6.360864in}{5.438030in}}%
\pgfpathlineto{\pgfqpoint{6.364916in}{3.839648in}}%
\pgfpathlineto{\pgfqpoint{6.368967in}{5.193763in}}%
\pgfpathlineto{\pgfqpoint{6.373019in}{4.095081in}}%
\pgfpathlineto{\pgfqpoint{6.377070in}{4.930999in}}%
\pgfpathlineto{\pgfqpoint{6.381122in}{4.361235in}}%
\pgfpathlineto{\pgfqpoint{6.385173in}{4.665371in}}%
\pgfpathlineto{\pgfqpoint{6.389225in}{4.622562in}}%
\pgfpathlineto{\pgfqpoint{6.393276in}{4.411874in}}%
\pgfpathlineto{\pgfqpoint{6.397327in}{4.865033in}}%
\pgfpathlineto{\pgfqpoint{6.401379in}{4.183240in}}%
\pgfpathlineto{\pgfqpoint{6.405430in}{5.077446in}}%
\pgfpathlineto{\pgfqpoint{6.409482in}{3.988996in}}%
\pgfpathlineto{\pgfqpoint{6.413533in}{5.251999in}}%
\pgfpathlineto{\pgfqpoint{6.417585in}{3.835256in}}%
\pgfpathlineto{\pgfqpoint{6.421636in}{5.384162in}}%
\pgfpathlineto{\pgfqpoint{6.425688in}{3.725120in}}%
\pgfpathlineto{\pgfqpoint{6.429739in}{5.472087in}}%
\pgfpathlineto{\pgfqpoint{6.433791in}{3.659370in}}%
\pgfpathlineto{\pgfqpoint{6.437842in}{5.515878in}}%
\pgfpathlineto{\pgfqpoint{6.441894in}{3.637168in}}%
\pgfpathlineto{\pgfqpoint{6.445945in}{5.517001in}}%
\pgfpathlineto{\pgfqpoint{6.449997in}{3.656482in}}%
\pgfpathlineto{\pgfqpoint{6.454048in}{5.478028in}}%
\pgfpathlineto{\pgfqpoint{6.458099in}{3.714183in}}%
\pgfpathlineto{\pgfqpoint{6.462151in}{5.402698in}}%
\pgfpathlineto{\pgfqpoint{6.466202in}{3.805859in}}%
\pgfpathlineto{\pgfqpoint{6.470254in}{5.296159in}}%
\pgfpathlineto{\pgfqpoint{6.474305in}{3.925567in}}%
\pgfpathlineto{\pgfqpoint{6.478357in}{5.165182in}}%
\pgfpathlineto{\pgfqpoint{6.482408in}{4.065723in}}%
\pgfpathlineto{\pgfqpoint{6.486460in}{5.018106in}}%
\pgfpathlineto{\pgfqpoint{6.490511in}{4.217335in}}%
\pgfpathlineto{\pgfqpoint{6.494563in}{4.864416in}}%
\pgfpathlineto{\pgfqpoint{6.498614in}{4.370632in}}%
\pgfpathlineto{\pgfqpoint{6.502666in}{4.713929in}}%
\pgfpathlineto{\pgfqpoint{6.506717in}{4.516017in}}%
\pgfpathlineto{\pgfqpoint{6.514820in}{4.645180in}}%
\pgfpathlineto{\pgfqpoint{6.518871in}{4.457113in}}%
\pgfpathlineto{\pgfqpoint{6.522923in}{4.752129in}}%
\pgfpathlineto{\pgfqpoint{6.526974in}{4.362576in}}%
\pgfpathlineto{\pgfqpoint{6.531026in}{4.833946in}}%
\pgfpathlineto{\pgfqpoint{6.535077in}{4.293352in}}%
\pgfpathlineto{\pgfqpoint{6.539129in}{4.891128in}}%
\pgfpathlineto{\pgfqpoint{6.543180in}{4.247255in}}%
\pgfpathlineto{\pgfqpoint{6.547232in}{4.927469in}}%
\pgfpathlineto{\pgfqpoint{6.551283in}{4.219013in}}%
\pgfpathlineto{\pgfqpoint{6.555335in}{4.949543in}}%
\pgfpathlineto{\pgfqpoint{6.559386in}{4.200963in}}%
\pgfpathlineto{\pgfqpoint{6.563438in}{4.965866in}}%
\pgfpathlineto{\pgfqpoint{6.567489in}{4.183991in}}%
\pgfpathlineto{\pgfqpoint{6.571541in}{4.985872in}}%
\pgfpathlineto{\pgfqpoint{6.575592in}{4.158627in}}%
\pgfpathlineto{\pgfqpoint{6.579643in}{5.018784in}}%
\pgfpathlineto{\pgfqpoint{6.583695in}{4.116184in}}%
\pgfpathlineto{\pgfqpoint{6.587746in}{5.072466in}}%
\pgfpathlineto{\pgfqpoint{6.591798in}{4.049897in}}%
\pgfpathlineto{\pgfqpoint{6.595849in}{5.152321in}}%
\pgfpathlineto{\pgfqpoint{6.599901in}{3.955965in}}%
\pgfpathlineto{\pgfqpoint{6.603952in}{5.260335in}}%
\pgfpathlineto{\pgfqpoint{6.608004in}{3.834400in}}%
\pgfpathlineto{\pgfqpoint{6.612055in}{5.394364in}}%
\pgfpathlineto{\pgfqpoint{6.616107in}{3.689555in}}%
\pgfpathlineto{\pgfqpoint{6.620158in}{5.547829in}}%
\pgfpathlineto{\pgfqpoint{6.624210in}{3.530182in}}%
\pgfpathlineto{\pgfqpoint{6.628261in}{5.709938in}}%
\pgfpathlineto{\pgfqpoint{6.632313in}{3.368898in}}%
\pgfpathlineto{\pgfqpoint{6.636364in}{5.866537in}}%
\pgfpathlineto{\pgfqpoint{6.640415in}{3.221033in}}%
\pgfpathlineto{\pgfqpoint{6.644467in}{6.001555in}}%
\pgfpathlineto{\pgfqpoint{6.648518in}{3.102915in}}%
\pgfpathlineto{\pgfqpoint{6.652570in}{6.098915in}}%
\pgfpathlineto{\pgfqpoint{6.656621in}{3.029843in}}%
\pgfpathlineto{\pgfqpoint{6.660673in}{6.144619in}}%
\pgfpathlineto{\pgfqpoint{6.664724in}{3.014028in}}%
\pgfpathlineto{\pgfqpoint{6.668776in}{6.128669in}}%
\pgfpathlineto{\pgfqpoint{6.672827in}{3.062902in}}%
\pgfpathlineto{\pgfqpoint{6.676879in}{6.046465in}}%
\pgfpathlineto{\pgfqpoint{6.680930in}{3.178077in}}%
\pgfpathlineto{\pgfqpoint{6.684982in}{5.899431in}}%
\pgfpathlineto{\pgfqpoint{6.689033in}{3.355152in}}%
\pgfpathlineto{\pgfqpoint{6.693085in}{5.694773in}}%
\pgfpathlineto{\pgfqpoint{6.697136in}{3.584378in}}%
\pgfpathlineto{\pgfqpoint{6.701187in}{5.444453in}}%
\pgfpathlineto{\pgfqpoint{6.705239in}{3.851970in}}%
\pgfpathlineto{\pgfqpoint{6.709290in}{5.163649in}}%
\pgfpathlineto{\pgfqpoint{6.713342in}{4.141795in}}%
\pgfpathlineto{\pgfqpoint{6.717393in}{4.869025in}}%
\pgfpathlineto{\pgfqpoint{6.721445in}{4.437052in}}%
\pgfpathlineto{\pgfqpoint{6.729548in}{4.721657in}}%
\pgfpathlineto{\pgfqpoint{6.733599in}{4.303432in}}%
\pgfpathlineto{\pgfqpoint{6.737651in}{4.981166in}}%
\pgfpathlineto{\pgfqpoint{6.741702in}{4.061243in}}%
\pgfpathlineto{\pgfqpoint{6.745754in}{5.203221in}}%
\pgfpathlineto{\pgfqpoint{6.749805in}{3.861867in}}%
\pgfpathlineto{\pgfqpoint{6.753857in}{5.377644in}}%
\pgfpathlineto{\pgfqpoint{6.757908in}{3.714408in}}%
\pgfpathlineto{\pgfqpoint{6.761960in}{5.496390in}}%
\pgfpathlineto{\pgfqpoint{6.766011in}{3.625856in}}%
\pgfpathlineto{\pgfqpoint{6.770062in}{5.553543in}}%
\pgfpathlineto{\pgfqpoint{6.774114in}{3.601015in}}%
\pgfpathlineto{\pgfqpoint{6.778165in}{5.545473in}}%
\pgfpathlineto{\pgfqpoint{6.782217in}{3.642255in}}%
\pgfpathlineto{\pgfqpoint{6.786268in}{5.471170in}}%
\pgfpathlineto{\pgfqpoint{6.790320in}{3.749122in}}%
\pgfpathlineto{\pgfqpoint{6.794371in}{5.332647in}}%
\pgfpathlineto{\pgfqpoint{6.798423in}{3.917967in}}%
\pgfpathlineto{\pgfqpoint{6.802474in}{5.135246in}}%
\pgfpathlineto{\pgfqpoint{6.806526in}{4.141737in}}%
\pgfpathlineto{\pgfqpoint{6.810577in}{4.887701in}}%
\pgfpathlineto{\pgfqpoint{6.814629in}{4.410088in}}%
\pgfpathlineto{\pgfqpoint{6.818680in}{4.601845in}}%
\pgfpathlineto{\pgfqpoint{6.822732in}{4.709864in}}%
\pgfpathlineto{\pgfqpoint{6.826783in}{4.291953in}}%
\pgfpathlineto{\pgfqpoint{6.830834in}{5.025914in}}%
\pgfpathlineto{\pgfqpoint{6.834886in}{3.973785in}}%
\pgfpathlineto{\pgfqpoint{6.838937in}{5.342153in}}%
\pgfpathlineto{\pgfqpoint{6.842989in}{3.663456in}}%
\pgfpathlineto{\pgfqpoint{6.847040in}{5.642731in}}%
\pgfpathlineto{\pgfqpoint{6.851092in}{3.376262in}}%
\pgfpathlineto{\pgfqpoint{6.855143in}{5.913183in}}%
\pgfpathlineto{\pgfqpoint{6.859195in}{3.125575in}}%
\pgfpathlineto{\pgfqpoint{6.863246in}{6.141474in}}%
\pgfpathlineto{\pgfqpoint{6.867298in}{2.921871in}}%
\pgfpathlineto{\pgfqpoint{6.871349in}{6.318883in}}%
\pgfpathlineto{\pgfqpoint{6.875401in}{2.771944in}}%
\pgfpathlineto{\pgfqpoint{6.879452in}{6.440685in}}%
\pgfpathlineto{\pgfqpoint{6.883504in}{2.678346in}}%
\pgfpathlineto{\pgfqpoint{6.887555in}{6.506571in}}%
\pgfpathlineto{\pgfqpoint{6.891606in}{2.639119in}}%
\pgfpathlineto{\pgfqpoint{6.895658in}{6.520737in}}%
\pgfpathlineto{\pgfqpoint{6.899709in}{2.647910in}}%
\pgfpathlineto{\pgfqpoint{6.903761in}{6.491547in}}%
\pgfpathlineto{\pgfqpoint{6.907812in}{2.694552in}}%
\pgfpathlineto{\pgfqpoint{6.911864in}{6.430702in}}%
\pgfpathlineto{\pgfqpoint{6.915915in}{2.766148in}}%
\pgfpathlineto{\pgfqpoint{6.919967in}{6.351901in}}%
\pgfpathlineto{\pgfqpoint{6.924018in}{2.848630in}}%
\pgfpathlineto{\pgfqpoint{6.928070in}{6.269116in}}%
\pgfpathlineto{\pgfqpoint{6.932121in}{2.928605in}}%
\pgfpathlineto{\pgfqpoint{6.936173in}{6.194685in}}%
\pgfpathlineto{\pgfqpoint{6.940224in}{2.995236in}}%
\pgfpathlineto{\pgfqpoint{6.944276in}{6.137548in}}%
\pgfpathlineto{\pgfqpoint{6.948327in}{3.041803in}}%
\pgfpathlineto{\pgfqpoint{6.952378in}{6.101974in}}%
\pgfpathlineto{\pgfqpoint{6.956430in}{3.066617in}}%
\pgfpathlineto{\pgfqpoint{6.960481in}{6.087059in}}%
\pgfpathlineto{\pgfqpoint{6.964533in}{3.073069in}}%
\pgfpathlineto{\pgfqpoint{6.968584in}{6.087147in}}%
\pgfpathlineto{\pgfqpoint{6.972636in}{3.068749in}}%
\pgfpathlineto{\pgfqpoint{6.976687in}{6.093129in}}%
\pgfpathlineto{\pgfqpoint{6.980739in}{3.063806in}}%
\pgfpathlineto{\pgfqpoint{6.984790in}{6.094339in}}%
\pgfpathlineto{\pgfqpoint{6.988842in}{3.068885in}}%
\pgfpathlineto{\pgfqpoint{6.992893in}{6.080668in}}%
\pgfpathlineto{\pgfqpoint{6.996945in}{3.093094in}}%
\pgfpathlineto{\pgfqpoint{7.000996in}{6.044406in}}%
\pgfpathlineto{\pgfqpoint{7.005048in}{3.142441in}}%
\pgfpathlineto{\pgfqpoint{7.009099in}{5.981447in}}%
\pgfpathlineto{\pgfqpoint{7.013150in}{3.219044in}}%
\pgfpathlineto{\pgfqpoint{7.017202in}{5.891636in}}%
\pgfpathlineto{\pgfqpoint{7.021253in}{3.321212in}}%
\pgfpathlineto{\pgfqpoint{7.025305in}{5.778307in}}%
\pgfpathlineto{\pgfqpoint{7.029356in}{3.444236in}}%
\pgfpathlineto{\pgfqpoint{7.033408in}{5.647242in}}%
\pgfpathlineto{\pgfqpoint{7.037459in}{3.581576in}}%
\pgfpathlineto{\pgfqpoint{7.041511in}{5.505442in}}%
\pgfpathlineto{\pgfqpoint{7.045562in}{3.726032in}}%
\pgfpathlineto{\pgfqpoint{7.049614in}{5.360086in}}%
\pgfpathlineto{\pgfqpoint{7.053665in}{3.870601in}}%
\pgfpathlineto{\pgfqpoint{7.057717in}{5.217908in}}%
\pgfpathlineto{\pgfqpoint{7.061768in}{4.008863in}}%
\pgfpathlineto{\pgfqpoint{7.065820in}{5.085019in}}%
\pgfpathlineto{\pgfqpoint{7.069871in}{4.134981in}}%
\pgfpathlineto{\pgfqpoint{7.073922in}{4.967016in}}%
\pgfpathlineto{\pgfqpoint{7.077974in}{4.243581in}}%
\pgfpathlineto{\pgfqpoint{7.082025in}{4.869037in}}%
\pgfpathlineto{\pgfqpoint{7.086077in}{4.329817in}}%
\pgfpathlineto{\pgfqpoint{7.090128in}{4.795521in}}%
\pgfpathlineto{\pgfqpoint{7.094180in}{4.389840in}}%
\pgfpathlineto{\pgfqpoint{7.098231in}{4.749493in}}%
\pgfpathlineto{\pgfqpoint{7.102283in}{4.421721in}}%
\pgfpathlineto{\pgfqpoint{7.106334in}{4.731488in}}%
\pgfpathlineto{\pgfqpoint{7.110386in}{4.426612in}}%
\pgfpathlineto{\pgfqpoint{7.114437in}{4.738408in}}%
\pgfpathlineto{\pgfqpoint{7.118489in}{4.409757in}}%
\pgfpathlineto{\pgfqpoint{7.122540in}{4.762745in}}%
\pgfpathlineto{\pgfqpoint{7.126592in}{4.380939in}}%
\pgfpathlineto{\pgfqpoint{7.130643in}{4.792556in}}%
\pgfpathlineto{\pgfqpoint{7.134694in}{4.354013in}}%
\pgfpathlineto{\pgfqpoint{7.138746in}{4.812446in}}%
\pgfpathlineto{\pgfqpoint{7.142797in}{4.345430in}}%
\pgfpathlineto{\pgfqpoint{7.146849in}{4.805497in}}%
\pgfpathlineto{\pgfqpoint{7.150900in}{4.371917in}}%
\pgfpathlineto{\pgfqpoint{7.154952in}{4.755868in}}%
\pgfpathlineto{\pgfqpoint{7.159003in}{4.447724in}}%
\pgfpathlineto{\pgfqpoint{7.163055in}{4.651571in}}%
\pgfpathlineto{\pgfqpoint{7.167106in}{4.581971in}}%
\pgfpathlineto{\pgfqpoint{7.171158in}{4.486866in}}%
\pgfpathlineto{\pgfqpoint{7.175209in}{4.776625in}}%
\pgfpathlineto{\pgfqpoint{7.179261in}{4.263816in}}%
\pgfpathlineto{\pgfqpoint{7.183312in}{5.025487in}}%
\pgfpathlineto{\pgfqpoint{7.187364in}{3.992707in}}%
\pgfpathlineto{\pgfqpoint{7.191415in}{5.314383in}}%
\pgfpathlineto{\pgfqpoint{7.195466in}{3.691263in}}%
\pgfpathlineto{\pgfqpoint{7.199518in}{5.622500in}}%
\pgfpathlineto{\pgfqpoint{7.203569in}{3.382812in}}%
\pgfpathlineto{\pgfqpoint{7.207621in}{5.924656in}}%
\pgfpathlineto{\pgfqpoint{7.211672in}{3.093677in}}%
\pgfpathlineto{\pgfqpoint{7.215724in}{6.194143in}}%
\pgfpathlineto{\pgfqpoint{7.219775in}{2.850178in}}%
\pgfpathlineto{\pgfqpoint{7.223827in}{6.405783in}}%
\pgfpathlineto{\pgfqpoint{7.227878in}{2.675633in}}%
\pgfpathlineto{\pgfqpoint{7.231930in}{6.538783in}}%
\pgfpathlineto{\pgfqpoint{7.235981in}{2.587719in}}%
\pgfpathlineto{\pgfqpoint{7.240033in}{6.579078in}}%
\pgfpathlineto{\pgfqpoint{7.244084in}{2.596500in}}%
\pgfpathlineto{\pgfqpoint{7.248136in}{6.520872in}}%
\pgfpathlineto{\pgfqpoint{7.252187in}{2.703368in}}%
\pgfpathlineto{\pgfqpoint{7.256238in}{6.367184in}}%
\pgfpathlineto{\pgfqpoint{7.260290in}{2.901018in}}%
\pgfpathlineto{\pgfqpoint{7.264341in}{6.129347in}}%
\pgfpathlineto{\pgfqpoint{7.268393in}{3.174470in}}%
\pgfpathlineto{\pgfqpoint{7.272444in}{5.825507in}}%
\pgfpathlineto{\pgfqpoint{7.276496in}{3.502982in}}%
\pgfpathlineto{\pgfqpoint{7.280547in}{5.478362in}}%
\pgfpathlineto{\pgfqpoint{7.284599in}{3.862569in}}%
\pgfpathlineto{\pgfqpoint{7.288650in}{5.112509in}}%
\pgfpathlineto{\pgfqpoint{7.292702in}{4.228686in}}%
\pgfpathlineto{\pgfqpoint{7.296753in}{4.751822in}}%
\pgfpathlineto{\pgfqpoint{7.304856in}{4.417313in}}%
\pgfpathlineto{\pgfqpoint{7.308908in}{4.893499in}}%
\pgfpathlineto{\pgfqpoint{7.312959in}{4.125775in}}%
\pgfpathlineto{\pgfqpoint{7.317011in}{5.158716in}}%
\pgfpathlineto{\pgfqpoint{7.321062in}{3.889341in}}%
\pgfpathlineto{\pgfqpoint{7.325113in}{5.364426in}}%
\pgfpathlineto{\pgfqpoint{7.329165in}{3.715832in}}%
\pgfpathlineto{\pgfqpoint{7.333216in}{5.504662in}}%
\pgfpathlineto{\pgfqpoint{7.337268in}{3.609599in}}%
\pgfpathlineto{\pgfqpoint{7.341319in}{5.576452in}}%
\pgfpathlineto{\pgfqpoint{7.345371in}{3.572445in}}%
\pgfpathlineto{\pgfqpoint{7.349422in}{5.578999in}}%
\pgfpathlineto{\pgfqpoint{7.353474in}{3.604259in}}%
\pgfpathlineto{\pgfqpoint{7.357525in}{5.513293in}}%
\pgfpathlineto{\pgfqpoint{7.361577in}{3.703138in}}%
\pgfpathlineto{\pgfqpoint{7.365628in}{5.382242in}}%
\pgfpathlineto{\pgfqpoint{7.369680in}{3.865038in}}%
\pgfpathlineto{\pgfqpoint{7.373731in}{5.191175in}}%
\pgfpathlineto{\pgfqpoint{7.377783in}{4.083205in}}%
\pgfpathlineto{\pgfqpoint{7.381834in}{4.948374in}}%
\pgfpathlineto{\pgfqpoint{7.385885in}{4.347784in}}%
\pgfpathlineto{\pgfqpoint{7.389937in}{4.665228in}}%
\pgfpathlineto{\pgfqpoint{7.393988in}{4.645990in}}%
\pgfpathlineto{\pgfqpoint{7.398040in}{4.355680in}}%
\pgfpathlineto{\pgfqpoint{7.402091in}{4.963057in}}%
\pgfpathlineto{\pgfqpoint{7.406143in}{4.034904in}}%
\pgfpathlineto{\pgfqpoint{7.410194in}{5.283875in}}%
\pgfpathlineto{\pgfqpoint{7.414246in}{3.717441in}}%
\pgfpathlineto{\pgfqpoint{7.418297in}{5.594965in}}%
\pgfpathlineto{\pgfqpoint{7.422349in}{3.415271in}}%
\pgfpathlineto{\pgfqpoint{7.426400in}{5.886202in}}%
\pgfpathlineto{\pgfqpoint{7.430452in}{3.136421in}}%
\pgfpathlineto{\pgfqpoint{7.434503in}{6.151757in}}%
\pgfpathlineto{\pgfqpoint{7.438555in}{2.884578in}}%
\pgfpathlineto{\pgfqpoint{7.442606in}{6.389876in}}%
\pgfpathlineto{\pgfqpoint{7.446657in}{2.659908in}}%
\pgfpathlineto{\pgfqpoint{7.450709in}{6.601522in}}%
\pgfpathlineto{\pgfqpoint{7.454760in}{2.460855in}}%
\pgfpathlineto{\pgfqpoint{7.458812in}{6.788277in}}%
\pgfpathlineto{\pgfqpoint{7.462863in}{2.286363in}}%
\pgfpathlineto{\pgfqpoint{7.466915in}{6.950185in}}%
\pgfpathlineto{\pgfqpoint{7.470966in}{2.137776in}}%
\pgfpathlineto{\pgfqpoint{7.475018in}{7.084283in}}%
\pgfpathlineto{\pgfqpoint{7.479069in}{2.019738in}}%
\pgfpathlineto{\pgfqpoint{7.483121in}{7.184366in}}%
\pgfpathlineto{\pgfqpoint{7.487172in}{1.939719in}}%
\pgfpathlineto{\pgfqpoint{7.491224in}{7.242139in}}%
\pgfpathlineto{\pgfqpoint{7.495275in}{1.906278in}}%
\pgfpathlineto{\pgfqpoint{7.499327in}{7.249418in}}%
\pgfpathlineto{\pgfqpoint{7.503378in}{1.926576in}}%
\pgfpathlineto{\pgfqpoint{7.507429in}{7.200676in}}%
\pgfpathlineto{\pgfqpoint{7.511481in}{2.003984in}}%
\pgfpathlineto{\pgfqpoint{7.515532in}{7.095082in}}%
\pgfpathlineto{\pgfqpoint{7.519584in}{2.136583in}}%
\pgfpathlineto{\pgfqpoint{7.523635in}{6.937313in}}%
\pgfpathlineto{\pgfqpoint{7.527687in}{2.317128in}}%
\pgfpathlineto{\pgfqpoint{7.531738in}{6.736806in}}%
\pgfpathlineto{\pgfqpoint{7.535790in}{2.534529in}}%
\pgfpathlineto{\pgfqpoint{7.539841in}{6.505650in}}%
\pgfpathlineto{\pgfqpoint{7.543893in}{2.776417in}}%
\pgfpathlineto{\pgfqpoint{7.547944in}{6.255769in}}%
\pgfpathlineto{\pgfqpoint{7.551996in}{3.031976in}}%
\pgfpathlineto{\pgfqpoint{7.556047in}{5.996320in}}%
\pgfpathlineto{\pgfqpoint{7.560099in}{3.294097in}}%
\pgfpathlineto{\pgfqpoint{7.564150in}{5.732183in}}%
\pgfpathlineto{\pgfqpoint{7.568201in}{3.560096in}}%
\pgfpathlineto{\pgfqpoint{7.572253in}{5.464094in}}%
\pgfpathlineto{\pgfqpoint{7.576304in}{3.830727in}}%
\pgfpathlineto{\pgfqpoint{7.580356in}{5.190435in}}%
\pgfpathlineto{\pgfqpoint{7.584407in}{4.107728in}}%
\pgfpathlineto{\pgfqpoint{7.588459in}{4.910148in}}%
\pgfpathlineto{\pgfqpoint{7.592510in}{4.390693in}}%
\pgfpathlineto{\pgfqpoint{7.596562in}{4.625811in}}%
\pgfpathlineto{\pgfqpoint{7.600613in}{4.674308in}}%
\pgfpathlineto{\pgfqpoint{7.604665in}{4.345831in}}%
\pgfpathlineto{\pgfqpoint{7.608716in}{4.946960in}}%
\pgfpathlineto{\pgfqpoint{7.612768in}{4.084875in}}%
\pgfpathlineto{\pgfqpoint{7.616819in}{5.191349in}}%
\pgfpathlineto{\pgfqpoint{7.620871in}{3.862195in}}%
\pgfpathlineto{\pgfqpoint{7.624922in}{5.387171in}}%
\pgfpathlineto{\pgfqpoint{7.628973in}{3.698093in}}%
\pgfpathlineto{\pgfqpoint{7.633025in}{5.515278in}}%
\pgfpathlineto{\pgfqpoint{7.637076in}{3.609393in}}%
\pgfpathlineto{\pgfqpoint{7.641128in}{5.562269in}}%
\pgfpathlineto{\pgfqpoint{7.645179in}{3.605115in}}%
\pgfpathlineto{\pgfqpoint{7.649231in}{5.524248in}}%
\pgfpathlineto{\pgfqpoint{7.653282in}{3.683559in}}%
\pgfpathlineto{\pgfqpoint{7.657334in}{5.408672in}}%
\pgfpathlineto{\pgfqpoint{7.661385in}{3.831679in}}%
\pgfpathlineto{\pgfqpoint{7.665437in}{5.233692in}}%
\pgfpathlineto{\pgfqpoint{7.669488in}{4.026999in}}%
\pgfpathlineto{\pgfqpoint{7.673540in}{5.025083in}}%
\pgfpathlineto{\pgfqpoint{7.677591in}{4.241647in}}%
\pgfpathlineto{\pgfqpoint{7.681643in}{4.811512in}}%
\pgfpathlineto{\pgfqpoint{7.685694in}{4.447479in}}%
\pgfpathlineto{\pgfqpoint{7.689745in}{4.619350in}}%
\pgfpathlineto{\pgfqpoint{7.693797in}{4.621001in}}%
\pgfpathlineto{\pgfqpoint{7.697848in}{4.468321in}}%
\pgfpathlineto{\pgfqpoint{7.701900in}{4.746884in}}%
\pgfpathlineto{\pgfqpoint{7.705951in}{4.369026in}}%
\pgfpathlineto{\pgfqpoint{7.710003in}{4.819294in}}%
\pgfpathlineto{\pgfqpoint{7.714054in}{4.322778in}}%
\pgfpathlineto{\pgfqpoint{7.718106in}{4.840947in}}%
\pgfpathlineto{\pgfqpoint{7.722157in}{4.323520in}}%
\pgfpathlineto{\pgfqpoint{7.726209in}{4.820424in}}%
\pgfpathlineto{\pgfqpoint{7.730260in}{4.361011in}}%
\pgfpathlineto{\pgfqpoint{7.734312in}{4.768777in}}%
\pgfpathlineto{\pgfqpoint{7.738363in}{4.424158in}}%
\pgfpathlineto{\pgfqpoint{7.742415in}{4.696522in}}%
\pgfpathlineto{\pgfqpoint{7.746466in}{4.503460in}}%
\pgfpathlineto{\pgfqpoint{7.750517in}{4.611901in}}%
\pgfpathlineto{\pgfqpoint{7.754569in}{4.591964in}}%
\pgfpathlineto{\pgfqpoint{7.758620in}{4.520730in}}%
\pgfpathlineto{\pgfqpoint{7.762672in}{4.684701in}}%
\pgfpathlineto{\pgfqpoint{7.766723in}{4.427524in}}%
\pgfpathlineto{\pgfqpoint{7.770775in}{4.777185in}}%
\pgfpathlineto{\pgfqpoint{7.774826in}{4.337131in}}%
\pgfpathlineto{\pgfqpoint{7.778878in}{4.863898in}}%
\pgfpathlineto{\pgfqpoint{7.782929in}{4.255903in}}%
\pgfpathlineto{\pgfqpoint{7.786981in}{4.937665in}}%
\pgfpathlineto{\pgfqpoint{7.791032in}{4.191639in}}%
\pgfpathlineto{\pgfqpoint{7.795084in}{4.990460in}}%
\pgfpathlineto{\pgfqpoint{7.799135in}{4.152031in}}%
\pgfpathlineto{\pgfqpoint{7.803187in}{5.015607in}}%
\pgfpathlineto{\pgfqpoint{7.807238in}{4.141987in}}%
\pgfpathlineto{\pgfqpoint{7.811289in}{5.010711in}}%
\pgfpathlineto{\pgfqpoint{7.815341in}{4.160720in}}%
\pgfpathlineto{\pgfqpoint{7.819392in}{4.980262in}}%
\pgfpathlineto{\pgfqpoint{7.823444in}{4.199729in}}%
\pgfpathlineto{\pgfqpoint{7.827495in}{4.936822in}}%
\pgfpathlineto{\pgfqpoint{7.831547in}{4.242635in}}%
\pgfpathlineto{\pgfqpoint{7.835598in}{4.900048in}}%
\pgfpathlineto{\pgfqpoint{7.839650in}{4.267312in}}%
\pgfpathlineto{\pgfqpoint{7.843701in}{4.893487in}}%
\pgfpathlineto{\pgfqpoint{7.847753in}{4.250018in}}%
\pgfpathlineto{\pgfqpoint{7.851804in}{4.939756in}}%
\pgfpathlineto{\pgfqpoint{7.855856in}{4.170587in}}%
\pgfpathlineto{\pgfqpoint{7.859907in}{5.055330in}}%
\pgfpathlineto{\pgfqpoint{7.863959in}{4.017304in}}%
\pgfpathlineto{\pgfqpoint{7.868010in}{5.246341in}}%
\pgfpathlineto{\pgfqpoint{7.872062in}{3.790136in}}%
\pgfpathlineto{\pgfqpoint{7.876113in}{5.506555in}}%
\pgfpathlineto{\pgfqpoint{7.880164in}{3.501396in}}%
\pgfpathlineto{\pgfqpoint{7.884216in}{5.818109in}}%
\pgfpathlineto{\pgfqpoint{7.888267in}{3.173657in}}%
\pgfpathlineto{\pgfqpoint{7.892319in}{6.154803in}}%
\pgfpathlineto{\pgfqpoint{7.896370in}{2.835496in}}%
\pgfpathlineto{\pgfqpoint{7.900422in}{6.487023in}}%
\pgfpathlineto{\pgfqpoint{7.904473in}{2.516232in}}%
\pgfpathlineto{\pgfqpoint{7.908525in}{6.786980in}}%
\pgfpathlineto{\pgfqpoint{7.912576in}{2.241054in}}%
\pgfpathlineto{\pgfqpoint{7.916628in}{7.032932in}}%
\pgfpathlineto{\pgfqpoint{7.920679in}{2.027679in}}%
\pgfpathlineto{\pgfqpoint{7.924731in}{7.211476in}}%
\pgfpathlineto{\pgfqpoint{7.928782in}{1.885181in}}%
\pgfpathlineto{\pgfqpoint{7.932834in}{7.317638in}}%
\pgfpathlineto{\pgfqpoint{7.936885in}{1.814871in}}%
\pgfpathlineto{\pgfqpoint{7.940936in}{7.353186in}}%
\pgfpathlineto{\pgfqpoint{7.944988in}{1.812560in}}%
\pgfpathlineto{\pgfqpoint{7.949039in}{7.324059in}}%
\pgfpathlineto{\pgfqpoint{7.953091in}{1.871182in}}%
\pgfpathlineto{\pgfqpoint{7.957142in}{7.237932in}}%
\pgfpathlineto{\pgfqpoint{7.961194in}{1.982837in}}%
\pgfpathlineto{\pgfqpoint{7.965245in}{7.102703in}}%
\pgfpathlineto{\pgfqpoint{7.969297in}{2.139692in}}%
\pgfpathlineto{\pgfqpoint{7.973348in}{6.926209in}}%
\pgfpathlineto{\pgfqpoint{7.977400in}{2.333732in}}%
\pgfpathlineto{\pgfqpoint{7.981451in}{6.716890in}}%
\pgfpathlineto{\pgfqpoint{7.985503in}{2.555832in}}%
\pgfpathlineto{\pgfqpoint{7.989554in}{6.484775in}}%
\pgfpathlineto{\pgfqpoint{7.993606in}{2.794918in}}%
\pgfpathlineto{\pgfqpoint{7.997657in}{6.242012in}}%
\pgfpathlineto{\pgfqpoint{8.001708in}{3.037880in}}%
\pgfpathlineto{\pgfqpoint{8.005760in}{6.002415in}}%
\pgfpathlineto{\pgfqpoint{8.009811in}{3.270588in}}%
\pgfpathlineto{\pgfqpoint{8.013863in}{5.779936in}}%
\pgfpathlineto{\pgfqpoint{8.017914in}{3.479828in}}%
\pgfpathlineto{\pgfqpoint{8.021966in}{5.586474in}}%
\pgfpathlineto{\pgfqpoint{8.026017in}{3.655556in}}%
\pgfpathlineto{\pgfqpoint{8.030069in}{5.429773in}}%
\pgfpathlineto{\pgfqpoint{8.034120in}{3.792633in}}%
\pgfpathlineto{\pgfqpoint{8.038172in}{5.312241in}}%
\pgfpathlineto{\pgfqpoint{8.042223in}{3.891312in}}%
\pgfpathlineto{\pgfqpoint{8.046275in}{5.231233in}}%
\pgfpathlineto{\pgfqpoint{8.050326in}{3.956165in}}%
\pgfpathlineto{\pgfqpoint{8.054378in}{5.180865in}}%
\pgfpathlineto{\pgfqpoint{8.058429in}{3.993672in}}%
\pgfpathlineto{\pgfqpoint{8.062480in}{5.154828in}}%
\pgfpathlineto{\pgfqpoint{8.066532in}{4.009232in}}%
\pgfpathlineto{\pgfqpoint{8.070583in}{5.149280in}}%
\pgfpathlineto{\pgfqpoint{8.074635in}{4.004634in}}%
\pgfpathlineto{\pgfqpoint{8.078686in}{5.164769in}}%
\pgfpathlineto{\pgfqpoint{8.082738in}{3.976954in}}%
\pgfpathlineto{\pgfqpoint{8.086789in}{5.206378in}}%
\pgfpathlineto{\pgfqpoint{8.090841in}{3.919419in}}%
\pgfpathlineto{\pgfqpoint{8.094892in}{5.281872in}}%
\pgfpathlineto{\pgfqpoint{8.098944in}{3.824145in}}%
\pgfpathlineto{\pgfqpoint{8.102995in}{5.398272in}}%
\pgfpathlineto{\pgfqpoint{8.107047in}{3.685991in}}%
\pgfpathlineto{\pgfqpoint{8.111098in}{5.557880in}}%
\pgfpathlineto{\pgfqpoint{8.115150in}{3.506313in}}%
\pgfpathlineto{\pgfqpoint{8.119201in}{5.755072in}}%
\pgfpathlineto{\pgfqpoint{8.123252in}{3.295339in}}%
\pgfpathlineto{\pgfqpoint{8.127304in}{5.975000in}}%
\pgfpathlineto{\pgfqpoint{8.131355in}{3.072219in}}%
\pgfpathlineto{\pgfqpoint{8.135407in}{6.194854in}}%
\pgfpathlineto{\pgfqpoint{8.139458in}{2.862485in}}%
\pgfpathlineto{\pgfqpoint{8.143510in}{6.387556in}}%
\pgfpathlineto{\pgfqpoint{8.147561in}{2.693433in}}%
\pgfpathlineto{\pgfqpoint{8.151613in}{6.526987in}}%
\pgfpathlineto{\pgfqpoint{8.155664in}{2.588636in}}%
\pgfpathlineto{\pgfqpoint{8.159716in}{6.593356in}}%
\pgfpathlineto{\pgfqpoint{8.163767in}{2.563084in}}%
\pgfpathlineto{\pgfqpoint{8.167819in}{6.577202in}}%
\pgfpathlineto{\pgfqpoint{8.171870in}{2.620325in}}%
\pgfpathlineto{\pgfqpoint{8.175922in}{6.480910in}}%
\pgfpathlineto{\pgfqpoint{8.179973in}{2.752393in}}%
\pgfpathlineto{\pgfqpoint{8.184024in}{6.317331in}}%
\pgfpathlineto{\pgfqpoint{8.188076in}{2.942528in}}%
\pgfpathlineto{\pgfqpoint{8.192127in}{6.105951in}}%
\pgfpathlineto{\pgfqpoint{8.196179in}{3.169824in}}%
\pgfpathlineto{\pgfqpoint{8.200230in}{5.867773in}}%
\pgfpathlineto{\pgfqpoint{8.204282in}{3.414421in}}%
\pgfpathlineto{\pgfqpoint{8.208333in}{5.620437in}}%
\pgfpathlineto{\pgfqpoint{8.212385in}{3.661732in}}%
\pgfpathlineto{\pgfqpoint{8.216436in}{5.374950in}}%
\pgfpathlineto{\pgfqpoint{8.220488in}{3.904526in}}%
\pgfpathlineto{\pgfqpoint{8.224539in}{5.134902in}}%
\pgfpathlineto{\pgfqpoint{8.228591in}{4.142412in}}%
\pgfpathlineto{\pgfqpoint{8.232642in}{4.898186in}}%
\pgfpathlineto{\pgfqpoint{8.236694in}{4.379104in}}%
\pgfpathlineto{\pgfqpoint{8.240745in}{4.660476in}}%
\pgfpathlineto{\pgfqpoint{8.244796in}{4.618530in}}%
\pgfpathlineto{\pgfqpoint{8.248848in}{4.419176in}}%
\pgfpathlineto{\pgfqpoint{8.252899in}{4.861184in}}%
\pgfpathlineto{\pgfqpoint{8.256951in}{4.176436in}}%
\pgfpathlineto{\pgfqpoint{8.261002in}{5.102003in}}%
\pgfpathlineto{\pgfqpoint{8.265054in}{3.940198in}}%
\pgfpathlineto{\pgfqpoint{8.269105in}{5.330495in}}%
\pgfpathlineto{\pgfqpoint{8.273157in}{3.722912in}}%
\pgfpathlineto{\pgfqpoint{8.277208in}{5.533066in}}%
\pgfpathlineto{\pgfqpoint{8.281260in}{3.538354in}}%
\pgfpathlineto{\pgfqpoint{8.285311in}{5.696777in}}%
\pgfpathlineto{\pgfqpoint{8.289363in}{3.397633in}}%
\pgfpathlineto{\pgfqpoint{8.293414in}{5.813232in}}%
\pgfpathlineto{\pgfqpoint{8.297466in}{3.305743in}}%
\pgfpathlineto{\pgfqpoint{8.301517in}{5.881274in}}%
\pgfpathlineto{\pgfqpoint{8.305568in}{3.259862in}}%
\pgfpathlineto{\pgfqpoint{8.309620in}{5.907533in}}%
\pgfpathlineto{\pgfqpoint{8.313671in}{3.250024in}}%
\pgfpathlineto{\pgfqpoint{8.317723in}{5.904571in}}%
\pgfpathlineto{\pgfqpoint{8.321774in}{3.262024in}}%
\pgfpathlineto{\pgfqpoint{8.325826in}{5.887147in}}%
\pgfpathlineto{\pgfqpoint{8.329877in}{3.281682in}}%
\pgfpathlineto{\pgfqpoint{8.333929in}{5.867772in}}%
\pgfpathlineto{\pgfqpoint{8.337980in}{3.299117in}}%
\pgfpathlineto{\pgfqpoint{8.342032in}{5.852957in}}%
\pgfpathlineto{\pgfqpoint{8.346083in}{3.311652in}}%
\pgfpathlineto{\pgfqpoint{8.350135in}{5.841394in}}%
\pgfpathlineto{\pgfqpoint{8.354186in}{3.324389in}}%
\pgfpathlineto{\pgfqpoint{8.358238in}{5.824701in}}%
\pgfpathlineto{\pgfqpoint{8.362289in}{3.348188in}}%
\pgfpathlineto{\pgfqpoint{8.366340in}{5.790577in}}%
\pgfpathlineto{\pgfqpoint{8.370392in}{3.395606in}}%
\pgfpathlineto{\pgfqpoint{8.374443in}{5.727453in}}%
\pgfpathlineto{\pgfqpoint{8.378495in}{3.476016in}}%
\pgfpathlineto{\pgfqpoint{8.382546in}{5.629232in}}%
\pgfpathlineto{\pgfqpoint{8.386598in}{3.591373in}}%
\pgfpathlineto{\pgfqpoint{8.390649in}{5.498675in}}%
\pgfpathlineto{\pgfqpoint{8.394701in}{3.733960in}}%
\pgfpathlineto{\pgfqpoint{8.398752in}{5.348345in}}%
\pgfpathlineto{\pgfqpoint{8.402804in}{3.886830in}}%
\pgfpathlineto{\pgfqpoint{8.406855in}{5.198786in}}%
\pgfpathlineto{\pgfqpoint{8.410907in}{4.026895in}}%
\pgfpathlineto{\pgfqpoint{8.414958in}{5.074382in}}%
\pgfpathlineto{\pgfqpoint{8.419010in}{4.129831in}}%
\pgfpathlineto{\pgfqpoint{8.423061in}{4.998034in}}%
\pgfpathlineto{\pgfqpoint{8.427113in}{4.175444in}}%
\pgfpathlineto{\pgfqpoint{8.431164in}{4.986107in}}%
\pgfpathlineto{\pgfqpoint{8.435215in}{4.152082in}}%
\pgfpathlineto{\pgfqpoint{8.439267in}{5.044948in}}%
\pgfpathlineto{\pgfqpoint{8.443318in}{4.058974in}}%
\pgfpathlineto{\pgfqpoint{8.447370in}{5.169801in}}%
\pgfpathlineto{\pgfqpoint{8.451421in}{3.906050in}}%
\pgfpathlineto{\pgfqpoint{8.455473in}{5.346180in}}%
\pgfpathlineto{\pgfqpoint{8.459524in}{3.711525in}}%
\pgfpathlineto{\pgfqpoint{8.463576in}{5.553109in}}%
\pgfpathlineto{\pgfqpoint{8.467627in}{3.498111in}}%
\pgfpathlineto{\pgfqpoint{8.471679in}{5.767154in}}%
\pgfpathlineto{\pgfqpoint{8.475730in}{3.289020in}}%
\pgfpathlineto{\pgfqpoint{8.479782in}{5.966135in}}%
\pgfpathlineto{\pgfqpoint{8.483833in}{3.104763in}}%
\pgfpathlineto{\pgfqpoint{8.487885in}{6.131662in}}%
\pgfpathlineto{\pgfqpoint{8.491936in}{2.961345in}}%
\pgfpathlineto{\pgfqpoint{8.495987in}{6.250206in}}%
\pgfpathlineto{\pgfqpoint{8.500039in}{2.869866in}}%
\pgfpathlineto{\pgfqpoint{8.504090in}{6.312950in}}%
\pgfpathlineto{\pgfqpoint{8.508142in}{2.837055in}}%
\pgfpathlineto{\pgfqpoint{8.512193in}{6.315055in}}%
\pgfpathlineto{\pgfqpoint{8.516245in}{2.866025in}}%
\pgfpathlineto{\pgfqpoint{8.520296in}{6.255042in}}%
\pgfpathlineto{\pgfqpoint{8.524348in}{2.956628in}}%
\pgfpathlineto{\pgfqpoint{8.528399in}{6.134764in}}%
\pgfpathlineto{\pgfqpoint{8.532451in}{3.105155in}}%
\pgfpathlineto{\pgfqpoint{8.536502in}{5.959979in}}%
\pgfpathlineto{\pgfqpoint{8.540554in}{3.303597in}}%
\pgfpathlineto{\pgfqpoint{8.544605in}{5.741108in}}%
\pgfpathlineto{\pgfqpoint{8.548657in}{3.539047in}}%
\pgfpathlineto{\pgfqpoint{8.552708in}{5.493498in}}%
\pgfpathlineto{\pgfqpoint{8.556759in}{3.793926in}}%
\pgfpathlineto{\pgfqpoint{8.560811in}{5.236573in}}%
\pgfpathlineto{\pgfqpoint{8.564862in}{4.047527in}}%
\pgfpathlineto{\pgfqpoint{8.568914in}{4.991600in}}%
\pgfpathlineto{\pgfqpoint{8.572965in}{4.278869in}}%
\pgfpathlineto{\pgfqpoint{8.577017in}{4.778352in}}%
\pgfpathlineto{\pgfqpoint{8.581068in}{4.470327in}}%
\pgfpathlineto{\pgfqpoint{8.585120in}{4.611424in}}%
\pgfpathlineto{\pgfqpoint{8.589171in}{4.611084in}}%
\pgfpathlineto{\pgfqpoint{8.593223in}{4.497293in}}%
\pgfpathlineto{\pgfqpoint{8.597274in}{4.699327in}}%
\pgfpathlineto{\pgfqpoint{8.601326in}{4.433072in}}%
\pgfpathlineto{\pgfqpoint{8.605377in}{4.742382in}}%
\pgfpathlineto{\pgfqpoint{8.609429in}{4.407540in}}%
\pgfpathlineto{\pgfqpoint{8.613480in}{4.754565in}}%
\pgfpathlineto{\pgfqpoint{8.617531in}{4.404295in}}%
\pgfpathlineto{\pgfqpoint{8.621583in}{4.753215in}}%
\pgfpathlineto{\pgfqpoint{8.625634in}{4.406264in}}%
\pgfpathlineto{\pgfqpoint{8.629686in}{4.753973in}}%
\pgfpathlineto{\pgfqpoint{8.633737in}{4.400284in}}%
\pgfpathlineto{\pgfqpoint{8.637789in}{4.766671in}}%
\pgfpathlineto{\pgfqpoint{8.641840in}{4.380438in}}%
\pgfpathlineto{\pgfqpoint{8.645892in}{4.793039in}}%
\pgfpathlineto{\pgfqpoint{8.649943in}{4.349128in}}%
\pgfpathlineto{\pgfqpoint{8.653995in}{4.826926in}}%
\pgfpathlineto{\pgfqpoint{8.658046in}{4.315581in}}%
\pgfpathlineto{\pgfqpoint{8.662098in}{4.856946in}}%
\pgfpathlineto{\pgfqpoint{8.666149in}{4.292251in}}%
\pgfpathlineto{\pgfqpoint{8.670201in}{4.870741in}}%
\pgfpathlineto{\pgfqpoint{8.674252in}{4.290250in}}%
\pgfpathlineto{\pgfqpoint{8.678303in}{4.859493in}}%
\pgfpathlineto{\pgfqpoint{8.682355in}{4.315246in}}%
\pgfpathlineto{\pgfqpoint{8.686406in}{4.821281in}}%
\pgfpathlineto{\pgfqpoint{8.690458in}{4.365126in}}%
\pgfpathlineto{\pgfqpoint{8.694509in}{4.762203in}}%
\pgfpathlineto{\pgfqpoint{8.698561in}{4.430183in}}%
\pgfpathlineto{\pgfqpoint{8.702612in}{4.694898in}}%
\pgfpathlineto{\pgfqpoint{8.706664in}{4.495770in}}%
\pgfpathlineto{\pgfqpoint{8.710715in}{4.634926in}}%
\pgfpathlineto{\pgfqpoint{8.714767in}{4.546597in}}%
\pgfpathlineto{\pgfqpoint{8.718818in}{4.596137in}}%
\pgfpathlineto{\pgfqpoint{8.722870in}{4.571316in}}%
\pgfpathlineto{\pgfqpoint{8.726921in}{4.586506in}}%
\pgfpathlineto{\pgfqpoint{8.730973in}{4.565933in}}%
\pgfpathlineto{\pgfqpoint{8.735024in}{4.605749in}}%
\pgfpathlineto{\pgfqpoint{8.739075in}{4.534966in}}%
\pgfpathlineto{\pgfqpoint{8.743127in}{4.645494in}}%
\pgfpathlineto{\pgfqpoint{8.747178in}{4.489958in}}%
\pgfpathlineto{\pgfqpoint{8.751230in}{4.691966in}}%
\pgfpathlineto{\pgfqpoint{8.755281in}{4.445800in}}%
\pgfpathlineto{\pgfqpoint{8.759333in}{4.730358in}}%
\pgfpathlineto{\pgfqpoint{8.763384in}{4.416021in}}%
\pgfpathlineto{\pgfqpoint{8.767436in}{4.749513in}}%
\pgfpathlineto{\pgfqpoint{8.771487in}{4.408504in}}%
\pgfpathlineto{\pgfqpoint{8.775539in}{4.745456in}}%
\pgfpathlineto{\pgfqpoint{8.779590in}{4.422996in}}%
\pgfpathlineto{\pgfqpoint{8.783642in}{4.722652in}}%
\pgfpathlineto{\pgfqpoint{8.787693in}{4.451177in}}%
\pgfpathlineto{\pgfqpoint{8.791745in}{4.692607in}}%
\pgfpathlineto{\pgfqpoint{8.795796in}{4.479277in}}%
\pgfpathlineto{\pgfqpoint{8.799847in}{4.670248in}}%
\pgfpathlineto{\pgfqpoint{8.803899in}{4.492412in}}%
\pgfpathlineto{\pgfqpoint{8.807950in}{4.669225in}}%
\pgfpathlineto{\pgfqpoint{8.812002in}{4.479262in}}%
\pgfpathlineto{\pgfqpoint{8.816053in}{4.697620in}}%
\pgfpathlineto{\pgfqpoint{8.820105in}{4.435626in}}%
\pgfpathlineto{\pgfqpoint{8.824156in}{4.755419in}}%
\pgfpathlineto{\pgfqpoint{8.828208in}{4.365726in}}%
\pgfpathlineto{\pgfqpoint{8.832259in}{4.834544in}}%
\pgfpathlineto{\pgfqpoint{8.836311in}{4.280832in}}%
\pgfpathlineto{\pgfqpoint{8.840362in}{4.921452in}}%
\pgfpathlineto{\pgfqpoint{8.844414in}{4.195653in}}%
\pgfpathlineto{\pgfqpoint{8.848465in}{5.001477in}}%
\pgfpathlineto{\pgfqpoint{8.852517in}{4.123603in}}%
\pgfpathlineto{\pgfqpoint{8.856568in}{5.063577in}}%
\pgfpathlineto{\pgfqpoint{8.860619in}{4.072412in}}%
\pgfpathlineto{\pgfqpoint{8.864671in}{5.104011in}}%
\pgfpathlineto{\pgfqpoint{8.868722in}{4.041461in}}%
\pgfpathlineto{\pgfqpoint{8.872774in}{5.127794in}}%
\pgfpathlineto{\pgfqpoint{8.876825in}{4.021648in}}%
\pgfpathlineto{\pgfqpoint{8.880877in}{5.147494in}}%
\pgfpathlineto{\pgfqpoint{8.884928in}{3.997828in}}%
\pgfpathlineto{\pgfqpoint{8.888980in}{5.179738in}}%
\pgfpathlineto{\pgfqpoint{8.893031in}{3.953100in}}%
\pgfpathlineto{\pgfqpoint{8.897083in}{5.240468in}}%
\pgfpathlineto{\pgfqpoint{8.901134in}{3.873649in}}%
\pgfpathlineto{\pgfqpoint{8.905186in}{5.340354in}}%
\pgfpathlineto{\pgfqpoint{8.909237in}{3.752749in}}%
\pgfpathlineto{\pgfqpoint{8.913289in}{5.481659in}}%
\pgfpathlineto{\pgfqpoint{8.917340in}{3.592814in}}%
\pgfpathlineto{\pgfqpoint{8.921391in}{5.657380in}}%
\pgfpathlineto{\pgfqpoint{8.925443in}{3.405055in}}%
\pgfpathlineto{\pgfqpoint{8.929494in}{5.852739in}}%
\pgfpathlineto{\pgfqpoint{8.933546in}{3.206981in}}%
\pgfpathlineto{\pgfqpoint{8.937597in}{6.048452in}}%
\pgfpathlineto{\pgfqpoint{8.941649in}{3.018641in}}%
\pgfpathlineto{\pgfqpoint{8.945700in}{6.224692in}}%
\pgfpathlineto{\pgfqpoint{8.949752in}{2.858749in}}%
\pgfpathlineto{\pgfqpoint{8.953803in}{6.364607in}}%
\pgfpathlineto{\pgfqpoint{8.957855in}{2.741735in}}%
\pgfpathlineto{\pgfqpoint{8.961906in}{6.456544in}}%
\pgfpathlineto{\pgfqpoint{8.965958in}{2.676321in}}%
\pgfpathlineto{\pgfqpoint{8.970009in}{6.494669in}}%
\pgfpathlineto{\pgfqpoint{8.974061in}{2.665648in}}%
\pgfpathlineto{\pgfqpoint{8.978112in}{6.478234in}}%
\pgfpathlineto{\pgfqpoint{8.982164in}{2.708433in}}%
\pgfpathlineto{\pgfqpoint{8.986215in}{6.410184in}}%
\pgfpathlineto{\pgfqpoint{8.990266in}{2.800404in}}%
\pgfpathlineto{\pgfqpoint{8.994318in}{6.295855in}}%
\pgfpathlineto{\pgfqpoint{8.998369in}{2.935323in}}%
\pgfpathlineto{\pgfqpoint{9.002421in}{6.142324in}}%
\pgfpathlineto{\pgfqpoint{9.006472in}{3.105252in}}%
\pgfpathlineto{\pgfqpoint{9.010524in}{5.958486in}}%
\pgfpathlineto{\pgfqpoint{9.014575in}{3.300204in}}%
\pgfpathlineto{\pgfqpoint{9.018627in}{5.755539in}}%
\pgfpathlineto{\pgfqpoint{9.022678in}{3.507706in}}%
\pgfpathlineto{\pgfqpoint{9.026730in}{5.547206in}}%
\pgfpathlineto{\pgfqpoint{9.030781in}{3.712938in}}%
\pgfpathlineto{\pgfqpoint{9.034833in}{5.349102in}}%
\pgfpathlineto{\pgfqpoint{9.038884in}{3.899939in}}%
\pgfpathlineto{\pgfqpoint{9.042936in}{5.176951in}}%
\pgfpathlineto{\pgfqpoint{9.046987in}{4.053916in}}%
\pgfpathlineto{\pgfqpoint{9.051038in}{5.043859in}}%
\pgfpathlineto{\pgfqpoint{9.055090in}{4.164200in}}%
\pgfpathlineto{\pgfqpoint{9.059141in}{4.957367in}}%
\pgfpathlineto{\pgfqpoint{9.063193in}{4.226953in}}%
\pgfpathlineto{\pgfqpoint{9.067244in}{4.917225in}}%
\pgfpathlineto{\pgfqpoint{9.071296in}{4.246664in}}%
\pgfpathlineto{\pgfqpoint{9.075347in}{4.914810in}}%
\pgfpathlineto{\pgfqpoint{9.079399in}{4.235711in}}%
\pgfpathlineto{\pgfqpoint{9.083450in}{4.934632in}}%
\pgfpathlineto{\pgfqpoint{9.087502in}{4.211825in}}%
\pgfpathlineto{\pgfqpoint{9.091553in}{4.957763in}}%
\pgfpathlineto{\pgfqpoint{9.095605in}{4.193981in}}%
\pgfpathlineto{\pgfqpoint{9.099656in}{4.966358in}}%
\pgfpathlineto{\pgfqpoint{9.103708in}{4.197777in}}%
\pgfpathlineto{\pgfqpoint{9.107759in}{4.948037in}}%
\pgfpathlineto{\pgfqpoint{9.111810in}{4.231625in}}%
\pgfpathlineto{\pgfqpoint{9.115862in}{4.898832in}}%
\pgfpathlineto{\pgfqpoint{9.119913in}{4.294890in}}%
\pgfpathlineto{\pgfqpoint{9.123965in}{4.823809in}}%
\pgfpathlineto{\pgfqpoint{9.128016in}{4.378564in}}%
\pgfpathlineto{\pgfqpoint{9.132068in}{4.735143in}}%
\pgfpathlineto{\pgfqpoint{9.136119in}{4.468305in}}%
\pgfpathlineto{\pgfqpoint{9.140171in}{4.648194in}}%
\pgfpathlineto{\pgfqpoint{9.144222in}{4.548946in}}%
\pgfpathlineto{\pgfqpoint{9.152325in}{4.609101in}}%
\pgfpathlineto{\pgfqpoint{9.156377in}{4.528970in}}%
\pgfpathlineto{\pgfqpoint{9.160428in}{4.644489in}}%
\pgfpathlineto{\pgfqpoint{9.164480in}{4.505000in}}%
\pgfpathlineto{\pgfqpoint{9.168531in}{4.658956in}}%
\pgfpathlineto{\pgfqpoint{9.172582in}{4.497370in}}%
\pgfpathlineto{\pgfqpoint{9.176634in}{4.662930in}}%
\pgfpathlineto{\pgfqpoint{9.180685in}{4.493627in}}%
\pgfpathlineto{\pgfqpoint{9.184737in}{4.669815in}}%
\pgfpathlineto{\pgfqpoint{9.188788in}{4.480563in}}%
\pgfpathlineto{\pgfqpoint{9.192840in}{4.691503in}}%
\pgfpathlineto{\pgfqpoint{9.196891in}{4.448603in}}%
\pgfpathlineto{\pgfqpoint{9.200943in}{4.734450in}}%
\pgfpathlineto{\pgfqpoint{9.204994in}{4.394944in}}%
\pgfpathlineto{\pgfqpoint{9.209046in}{4.797584in}}%
\pgfpathlineto{\pgfqpoint{9.213097in}{4.324426in}}%
\pgfpathlineto{\pgfqpoint{9.217149in}{4.872727in}}%
\pgfpathlineto{\pgfqpoint{9.221200in}{4.247841in}}%
\pgfpathlineto{\pgfqpoint{9.225252in}{4.947426in}}%
\pgfpathlineto{\pgfqpoint{9.229303in}{4.178205in}}%
\pgfpathlineto{\pgfqpoint{9.233354in}{5.009265in}}%
\pgfpathlineto{\pgfqpoint{9.237406in}{4.126201in}}%
\pgfpathlineto{\pgfqpoint{9.241457in}{5.050289in}}%
\pgfpathlineto{\pgfqpoint{9.245509in}{4.096279in}}%
\pgfpathlineto{\pgfqpoint{9.249560in}{5.070049in}}%
\pgfpathlineto{\pgfqpoint{9.253612in}{4.084729in}}%
\pgfpathlineto{\pgfqpoint{9.257663in}{5.076217in}}%
\pgfpathlineto{\pgfqpoint{9.261715in}{4.080456in}}%
\pgfpathlineto{\pgfqpoint{9.265766in}{5.082462in}}%
\pgfpathlineto{\pgfqpoint{9.269818in}{4.068313in}}%
\pgfpathlineto{\pgfqpoint{9.273869in}{5.104149in}}%
\pgfpathlineto{\pgfqpoint{9.277921in}{4.034042in}}%
\pgfpathlineto{\pgfqpoint{9.281972in}{5.153144in}}%
\pgfpathlineto{\pgfqpoint{9.286024in}{3.969317in}}%
\pgfpathlineto{\pgfqpoint{9.290075in}{5.233320in}}%
\pgfpathlineto{\pgfqpoint{9.294126in}{3.875311in}}%
\pgfpathlineto{\pgfqpoint{9.298178in}{5.338234in}}%
\pgfpathlineto{\pgfqpoint{9.302229in}{3.763573in}}%
\pgfpathlineto{\pgfqpoint{9.306281in}{5.451784in}}%
\pgfpathlineto{\pgfqpoint{9.310332in}{3.653844in}}%
\pgfpathlineto{\pgfqpoint{9.314384in}{5.551797in}}%
\pgfpathlineto{\pgfqpoint{9.318435in}{3.569316in}}%
\pgfpathlineto{\pgfqpoint{9.322487in}{5.615585in}}%
\pgfpathlineto{\pgfqpoint{9.326538in}{3.530648in}}%
\pgfpathlineto{\pgfqpoint{9.330590in}{5.625932in}}%
\pgfpathlineto{\pgfqpoint{9.334641in}{3.550419in}}%
\pgfpathlineto{\pgfqpoint{9.338693in}{5.575787in}}%
\pgfpathlineto{\pgfqpoint{9.342744in}{3.629619in}}%
\pgfpathlineto{\pgfqpoint{9.346796in}{5.470354in}}%
\pgfpathlineto{\pgfqpoint{9.350847in}{3.757133in}}%
\pgfpathlineto{\pgfqpoint{9.354898in}{5.325990in}}%
\pgfpathlineto{\pgfqpoint{9.358950in}{3.912355in}}%
\pgfpathlineto{\pgfqpoint{9.363001in}{5.166299in}}%
\pgfpathlineto{\pgfqpoint{9.367053in}{4.070105in}}%
\pgfpathlineto{\pgfqpoint{9.371104in}{5.016560in}}%
\pgfpathlineto{\pgfqpoint{9.375156in}{4.206431in}}%
\pgfpathlineto{\pgfqpoint{9.379207in}{4.898114in}}%
\pgfpathlineto{\pgfqpoint{9.383259in}{4.303654in}}%
\pgfpathlineto{\pgfqpoint{9.387310in}{4.824227in}}%
\pgfpathlineto{\pgfqpoint{9.391362in}{4.353334in}}%
\pgfpathlineto{\pgfqpoint{9.395413in}{4.798455in}}%
\pgfpathlineto{\pgfqpoint{9.399465in}{4.356519in}}%
\pgfpathlineto{\pgfqpoint{9.403516in}{4.815724in}}%
\pgfpathlineto{\pgfqpoint{9.407568in}{4.321490in}}%
\pgfpathlineto{\pgfqpoint{9.411619in}{4.865523in}}%
\pgfpathlineto{\pgfqpoint{9.415670in}{4.259944in}}%
\pgfpathlineto{\pgfqpoint{9.419722in}{4.935998in}}%
\pgfpathlineto{\pgfqpoint{9.423773in}{4.182962in}}%
\pgfpathlineto{\pgfqpoint{9.427825in}{5.017598in}}%
\pgfpathlineto{\pgfqpoint{9.431876in}{4.098034in}}%
\pgfpathlineto{\pgfqpoint{9.435928in}{5.105165in}}%
\pgfpathlineto{\pgfqpoint{9.439979in}{4.007977in}}%
\pgfpathlineto{\pgfqpoint{9.444031in}{5.197984in}}%
\pgfpathlineto{\pgfqpoint{9.448082in}{3.911858in}}%
\pgfpathlineto{\pgfqpoint{9.452134in}{5.298028in}}%
\pgfpathlineto{\pgfqpoint{9.456185in}{3.807361in}}%
\pgfpathlineto{\pgfqpoint{9.460237in}{5.407243in}}%
\pgfpathlineto{\pgfqpoint{9.464288in}{3.693564in}}%
\pgfpathlineto{\pgfqpoint{9.468340in}{5.524992in}}%
\pgfpathlineto{\pgfqpoint{9.472391in}{3.573024in}}%
\pgfpathlineto{\pgfqpoint{9.476443in}{5.646647in}}%
\pgfpathlineto{\pgfqpoint{9.480494in}{3.452367in}}%
\pgfpathlineto{\pgfqpoint{9.484545in}{5.763874in}}%
\pgfpathlineto{\pgfqpoint{9.488597in}{3.341161in}}%
\pgfpathlineto{\pgfqpoint{9.492648in}{5.866493in}}%
\pgfpathlineto{\pgfqpoint{9.496700in}{3.249486in}}%
\pgfpathlineto{\pgfqpoint{9.500751in}{5.945254in}}%
\pgfpathlineto{\pgfqpoint{9.504803in}{3.185077in}}%
\pgfpathlineto{\pgfqpoint{9.508854in}{5.994511in}}%
\pgfpathlineto{\pgfqpoint{9.512906in}{3.151079in}}%
\pgfpathlineto{\pgfqpoint{9.516957in}{6.013833in}}%
\pgfpathlineto{\pgfqpoint{9.521009in}{3.145217in}}%
\pgfpathlineto{\pgfqpoint{9.525060in}{6.007976in}}%
\pgfpathlineto{\pgfqpoint{9.529112in}{3.160679in}}%
\pgfpathlineto{\pgfqpoint{9.533163in}{5.985218in}}%
\pgfpathlineto{\pgfqpoint{9.537215in}{3.188418in}}%
\pgfpathlineto{\pgfqpoint{9.541266in}{5.954635in}}%
\pgfpathlineto{\pgfqpoint{9.545317in}{3.220049in}}%
\pgfpathlineto{\pgfqpoint{9.549369in}{5.923281in}}%
\pgfpathlineto{\pgfqpoint{9.553420in}{3.250349in}}%
\pgfpathlineto{\pgfqpoint{9.557472in}{5.894234in}}%
\pgfpathlineto{\pgfqpoint{9.561523in}{3.278497in}}%
\pgfpathlineto{\pgfqpoint{9.565575in}{5.866160in}}%
\pgfpathlineto{\pgfqpoint{9.569626in}{3.307674in}}%
\pgfpathlineto{\pgfqpoint{9.573678in}{5.834501in}}%
\pgfpathlineto{\pgfqpoint{9.577729in}{3.343224in}}%
\pgfpathlineto{\pgfqpoint{9.581781in}{5.793790in}}%
\pgfpathlineto{\pgfqpoint{9.585832in}{3.390079in}}%
\pgfpathlineto{\pgfqpoint{9.589884in}{5.740219in}}%
\pgfpathlineto{\pgfqpoint{9.593935in}{3.450444in}}%
\pgfpathlineto{\pgfqpoint{9.597987in}{5.673502in}}%
\pgfpathlineto{\pgfqpoint{9.602038in}{3.522569in}}%
\pgfpathlineto{\pgfqpoint{9.606089in}{5.597349in}}%
\pgfpathlineto{\pgfqpoint{9.610141in}{3.601049in}}%
\pgfpathlineto{\pgfqpoint{9.614192in}{5.518425in}}%
\pgfpathlineto{\pgfqpoint{9.618244in}{3.678509in}}%
\pgfpathlineto{\pgfqpoint{9.622295in}{5.444203in}}%
\pgfpathlineto{\pgfqpoint{9.626347in}{3.747996in}}%
\pgfpathlineto{\pgfqpoint{9.630398in}{5.380561in}}%
\pgfpathlineto{\pgfqpoint{9.634450in}{3.805140in}}%
\pgfpathlineto{\pgfqpoint{9.638501in}{5.330085in}}%
\pgfpathlineto{\pgfqpoint{9.642553in}{3.849230in}}%
\pgfpathlineto{\pgfqpoint{9.646604in}{5.291726in}}%
\pgfpathlineto{\pgfqpoint{9.650656in}{3.882769in}}%
\pgfpathlineto{\pgfqpoint{9.654707in}{5.261994in}}%
\pgfpathlineto{\pgfqpoint{9.658759in}{3.909643in}}%
\pgfpathlineto{\pgfqpoint{9.662810in}{5.237252in}}%
\pgfpathlineto{\pgfqpoint{9.666861in}{3.932605in}}%
\pgfpathlineto{\pgfqpoint{9.670913in}{5.216203in}}%
\pgfpathlineto{\pgfqpoint{9.674964in}{3.951059in}}%
\pgfpathlineto{\pgfqpoint{9.679016in}{5.201585in}}%
\pgfpathlineto{\pgfqpoint{9.683067in}{3.960091in}}%
\pgfpathlineto{\pgfqpoint{9.687119in}{5.200288in}}%
\pgfpathlineto{\pgfqpoint{9.691170in}{3.951263in}}%
\pgfpathlineto{\pgfqpoint{9.695222in}{5.221674in}}%
\pgfpathlineto{\pgfqpoint{9.699273in}{3.915070in}}%
\pgfpathlineto{\pgfqpoint{9.703325in}{5.274517in}}%
\pgfpathlineto{\pgfqpoint{9.707376in}{3.844352in}}%
\pgfpathlineto{\pgfqpoint{9.711428in}{5.363536in}}%
\pgfpathlineto{\pgfqpoint{9.715479in}{3.737532in}}%
\pgfpathlineto{\pgfqpoint{9.719531in}{5.486670in}}%
\pgfpathlineto{\pgfqpoint{9.723582in}{3.600548in}}%
\pgfpathlineto{\pgfqpoint{9.727633in}{5.634143in}}%
\pgfpathlineto{\pgfqpoint{9.731685in}{3.446690in}}%
\pgfpathlineto{\pgfqpoint{9.735736in}{5.789758in}}%
\pgfpathlineto{\pgfqpoint{9.739788in}{3.294213in}}%
\pgfpathlineto{\pgfqpoint{9.743839in}{5.934228in}}%
\pgfpathlineto{\pgfqpoint{9.747891in}{3.162296in}}%
\pgfpathlineto{\pgfqpoint{9.751942in}{6.049644in}}%
\pgfpathlineto{\pgfqpoint{9.755994in}{3.066481in}}%
\pgfpathlineto{\pgfqpoint{9.760045in}{6.123794in}}%
\pgfpathlineto{\pgfqpoint{9.764097in}{3.014907in}}%
\pgfpathlineto{\pgfqpoint{9.768148in}{6.153075in}}%
\pgfpathlineto{\pgfqpoint{9.772200in}{3.006485in}}%
\pgfpathlineto{\pgfqpoint{9.776251in}{6.143101in}}%
\pgfpathlineto{\pgfqpoint{9.780303in}{3.031554in}}%
\pgfpathlineto{\pgfqpoint{9.784354in}{6.106827in}}%
\pgfpathlineto{\pgfqpoint{9.788405in}{3.074843in}}%
\pgfpathlineto{\pgfqpoint{9.792457in}{6.060715in}}%
\pgfpathlineto{\pgfqpoint{9.796508in}{3.119883in}}%
\pgfpathlineto{\pgfqpoint{9.800560in}{6.020079in}}%
\pgfpathlineto{\pgfqpoint{9.804611in}{3.153565in}}%
\pgfpathlineto{\pgfqpoint{9.808663in}{5.994959in}}%
\pgfpathlineto{\pgfqpoint{9.812714in}{3.169539in}}%
\pgfpathlineto{\pgfqpoint{9.816766in}{5.987688in}}%
\pgfpathlineto{\pgfqpoint{9.820817in}{3.169505in}}%
\pgfpathlineto{\pgfqpoint{9.824869in}{5.992817in}}%
\pgfpathlineto{\pgfqpoint{9.828920in}{3.162098in}}%
\pgfpathlineto{\pgfqpoint{9.832972in}{5.999316in}}%
\pgfpathlineto{\pgfqpoint{9.837023in}{3.159796in}}%
\pgfpathlineto{\pgfqpoint{9.841075in}{5.994296in}}%
\pgfpathlineto{\pgfqpoint{9.845126in}{3.174859in}}%
\pgfpathlineto{\pgfqpoint{9.849177in}{5.967079in}}%
\pgfpathlineto{\pgfqpoint{9.853229in}{3.215584in}}%
\pgfpathlineto{\pgfqpoint{9.857280in}{5.912332in}}%
\pgfpathlineto{\pgfqpoint{9.861332in}{3.284013in}}%
\pgfpathlineto{\pgfqpoint{9.865383in}{5.831350in}}%
\pgfpathlineto{\pgfqpoint{9.869435in}{3.375757in}}%
\pgfpathlineto{\pgfqpoint{9.873486in}{5.731117in}}%
\pgfpathlineto{\pgfqpoint{9.877538in}{3.481944in}}%
\pgfpathlineto{\pgfqpoint{9.881589in}{5.621531in}}%
\pgfpathlineto{\pgfqpoint{9.885641in}{3.592594in}}%
\pgfpathlineto{\pgfqpoint{9.889692in}{5.511707in}}%
\pgfpathlineto{\pgfqpoint{9.893744in}{3.700330in}}%
\pgfpathlineto{\pgfqpoint{9.897795in}{5.406563in}}%
\pgfpathlineto{\pgfqpoint{9.901847in}{3.803206in}}%
\pgfpathlineto{\pgfqpoint{9.905898in}{5.304809in}}%
\pgfpathlineto{\pgfqpoint{9.909949in}{3.905736in}}%
\pgfpathlineto{\pgfqpoint{9.914001in}{5.198989in}}%
\pgfpathlineto{\pgfqpoint{9.918052in}{4.017782in}}%
\pgfpathlineto{\pgfqpoint{9.922104in}{5.077589in}}%
\pgfpathlineto{\pgfqpoint{9.926155in}{4.151607in}}%
\pgfpathlineto{\pgfqpoint{9.930207in}{4.928583in}}%
\pgfpathlineto{\pgfqpoint{9.934258in}{4.317999in}}%
\pgfpathlineto{\pgfqpoint{9.938310in}{4.743349in}}%
\pgfpathlineto{\pgfqpoint{9.942361in}{4.522629in}}%
\pgfpathlineto{\pgfqpoint{9.946413in}{4.519758in}}%
\pgfpathlineto{\pgfqpoint{9.950464in}{4.763737in}}%
\pgfpathlineto{\pgfqpoint{9.954516in}{4.263534in}}%
\pgfpathlineto{\pgfqpoint{9.958567in}{5.031837in}}%
\pgfpathlineto{\pgfqpoint{9.962619in}{3.987469in}}%
\pgfpathlineto{\pgfqpoint{9.966670in}{5.311505in}}%
\pgfpathlineto{\pgfqpoint{9.970721in}{3.708769in}}%
\pgfpathlineto{\pgfqpoint{9.974773in}{5.584708in}}%
\pgfpathlineto{\pgfqpoint{9.978824in}{3.445309in}}%
\pgfpathlineto{\pgfqpoint{9.982876in}{5.834677in}}%
\pgfpathlineto{\pgfqpoint{9.986927in}{3.211904in}}%
\pgfpathlineto{\pgfqpoint{9.990979in}{6.049231in}}%
\pgfpathlineto{\pgfqpoint{9.995030in}{3.017638in}}%
\pgfpathlineto{\pgfqpoint{9.999082in}{6.222616in}}%
\pgfpathlineto{\pgfqpoint{10.003133in}{2.864949in}}%
\pgfpathlineto{\pgfqpoint{10.007185in}{6.355457in}}%
\pgfpathlineto{\pgfqpoint{10.011236in}{2.750607in}}%
\pgfpathlineto{\pgfqpoint{10.015288in}{6.452963in}}%
\pgfpathlineto{\pgfqpoint{10.019339in}{2.668159in}}%
\pgfpathlineto{\pgfqpoint{10.023391in}{6.522053in}}%
\pgfpathlineto{\pgfqpoint{10.027442in}{2.610980in}}%
\pgfpathlineto{\pgfqpoint{10.031494in}{6.568374in}}%
\pgfpathlineto{\pgfqpoint{10.035545in}{2.574948in}}%
\pgfpathlineto{\pgfqpoint{10.039596in}{6.594158in}}%
\pgfpathlineto{\pgfqpoint{10.043648in}{2.559888in}}%
\pgfpathlineto{\pgfqpoint{10.047699in}{6.597573in}}%
\pgfpathlineto{\pgfqpoint{10.051751in}{2.569374in}}%
\pgfpathlineto{\pgfqpoint{10.055802in}{6.573746in}}%
\pgfpathlineto{\pgfqpoint{10.059854in}{2.608987in}}%
\pgfpathlineto{\pgfqpoint{10.063905in}{6.517084in}}%
\pgfpathlineto{\pgfqpoint{10.067957in}{2.683596in}}%
\pgfpathlineto{\pgfqpoint{10.072008in}{6.424158in}}%
\pgfpathlineto{\pgfqpoint{10.076060in}{2.794548in}}%
\pgfpathlineto{\pgfqpoint{10.080111in}{6.296220in}}%
\pgfpathlineto{\pgfqpoint{10.084163in}{2.937644in}}%
\pgfpathlineto{\pgfqpoint{10.088214in}{6.140564in}}%
\pgfpathlineto{\pgfqpoint{10.092266in}{3.102561in}}%
\pgfpathlineto{\pgfqpoint{10.096317in}{5.970271in}}%
\pgfpathlineto{\pgfqpoint{10.100368in}{3.273922in}}%
\pgfpathlineto{\pgfqpoint{10.104420in}{5.802382in}}%
\pgfpathlineto{\pgfqpoint{10.108471in}{3.433773in}}%
\pgfpathlineto{\pgfqpoint{10.112523in}{5.654952in}}%
\pgfpathlineto{\pgfqpoint{10.116574in}{3.564785in}}%
\pgfpathlineto{\pgfqpoint{10.120626in}{5.543792in}}%
\pgfpathlineto{\pgfqpoint{10.124677in}{3.653363in}}%
\pgfpathlineto{\pgfqpoint{10.128729in}{5.479722in}}%
\pgfpathlineto{\pgfqpoint{10.132780in}{3.691856in}}%
\pgfpathlineto{\pgfqpoint{10.136832in}{5.467016in}}%
\pgfpathlineto{\pgfqpoint{10.140883in}{3.679377in}}%
\pgfpathlineto{\pgfqpoint{10.144935in}{5.503352in}}%
\pgfpathlineto{\pgfqpoint{10.148986in}{3.621117in}}%
\pgfpathlineto{\pgfqpoint{10.153038in}{5.581143in}}%
\pgfpathlineto{\pgfqpoint{10.157089in}{3.526489in}}%
\pgfpathlineto{\pgfqpoint{10.161140in}{5.689769in}}%
\pgfpathlineto{\pgfqpoint{10.165192in}{3.406702in}}%
\pgfpathlineto{\pgfqpoint{10.169243in}{5.818007in}}%
\pgfpathlineto{\pgfqpoint{10.173295in}{3.272502in}}%
\pgfpathlineto{\pgfqpoint{10.177346in}{5.955964in}}%
\pgfpathlineto{\pgfqpoint{10.181398in}{3.132685in}}%
\pgfpathlineto{\pgfqpoint{10.185449in}{6.096041in}}%
\pgfpathlineto{\pgfqpoint{10.189501in}{2.993694in}}%
\pgfpathlineto{\pgfqpoint{10.193552in}{6.232785in}}%
\pgfpathlineto{\pgfqpoint{10.197604in}{2.860258in}}%
\pgfpathlineto{\pgfqpoint{10.201655in}{6.361858in}}%
\pgfpathlineto{\pgfqpoint{10.205707in}{2.736689in}}%
\pgfpathlineto{\pgfqpoint{10.209758in}{6.478621in}}%
\pgfpathlineto{\pgfqpoint{10.213810in}{2.628259in}}%
\pgfpathlineto{\pgfqpoint{10.217861in}{6.576932in}}%
\pgfpathlineto{\pgfqpoint{10.221912in}{2.542113in}}%
\pgfpathlineto{\pgfqpoint{10.225964in}{6.648632in}}%
\pgfpathlineto{\pgfqpoint{10.230015in}{2.487322in}}%
\pgfpathlineto{\pgfqpoint{10.234067in}{6.683950in}}%
\pgfpathlineto{\pgfqpoint{10.238118in}{2.474039in}}%
\pgfpathlineto{\pgfqpoint{10.242170in}{6.672745in}}%
\pgfpathlineto{\pgfqpoint{10.246221in}{2.511962in}}%
\pgfpathlineto{\pgfqpoint{10.250273in}{6.606211in}}%
\pgfpathlineto{\pgfqpoint{10.254324in}{2.608567in}}%
\pgfpathlineto{\pgfqpoint{10.258376in}{6.478591in}}%
\pgfpathlineto{\pgfqpoint{10.262427in}{2.767570in}}%
\pgfpathlineto{\pgfqpoint{10.266479in}{6.288450in}}%
\pgfpathlineto{\pgfqpoint{10.270530in}{2.987977in}}%
\pgfpathlineto{\pgfqpoint{10.274582in}{6.039267in}}%
\pgfpathlineto{\pgfqpoint{10.278633in}{3.263855in}}%
\pgfpathlineto{\pgfqpoint{10.282684in}{5.739324in}}%
\pgfpathlineto{\pgfqpoint{10.286736in}{3.584734in}}%
\pgfpathlineto{\pgfqpoint{10.290787in}{5.401073in}}%
\pgfpathlineto{\pgfqpoint{10.294839in}{3.936420in}}%
\pgfpathlineto{\pgfqpoint{10.298890in}{5.040202in}}%
\pgfpathlineto{\pgfqpoint{10.302942in}{4.301979in}}%
\pgfpathlineto{\pgfqpoint{10.306993in}{4.674631in}}%
\pgfpathlineto{\pgfqpoint{10.311045in}{4.662773in}}%
\pgfpathlineto{\pgfqpoint{10.315096in}{4.323450in}}%
\pgfpathlineto{\pgfqpoint{10.319148in}{4.999538in}}%
\pgfpathlineto{\pgfqpoint{10.323199in}{4.005786in}}%
\pgfpathlineto{\pgfqpoint{10.327251in}{5.293627in}}%
\pgfpathlineto{\pgfqpoint{10.331302in}{3.739436in}}%
\pgfpathlineto{\pgfqpoint{10.335354in}{5.528491in}}%
\pgfpathlineto{\pgfqpoint{10.339405in}{3.539274in}}%
\pgfpathlineto{\pgfqpoint{10.343456in}{5.691372in}}%
\pgfpathlineto{\pgfqpoint{10.347508in}{3.415516in}}%
\pgfpathlineto{\pgfqpoint{10.351559in}{5.774986in}}%
\pgfpathlineto{\pgfqpoint{10.355611in}{3.372186in}}%
\pgfpathlineto{\pgfqpoint{10.359662in}{5.778804in}}%
\pgfpathlineto{\pgfqpoint{10.363714in}{3.406196in}}%
\pgfpathlineto{\pgfqpoint{10.367765in}{5.709520in}}%
\pgfpathlineto{\pgfqpoint{10.371817in}{3.507409in}}%
\pgfpathlineto{\pgfqpoint{10.375868in}{5.580401in}}%
\pgfpathlineto{\pgfqpoint{10.379920in}{3.659882in}}%
\pgfpathlineto{\pgfqpoint{10.383971in}{5.409486in}}%
\pgfpathlineto{\pgfqpoint{10.388023in}{3.844151in}}%
\pgfpathlineto{\pgfqpoint{10.392074in}{5.216936in}}%
\pgfpathlineto{\pgfqpoint{10.396126in}{4.040106in}}%
\pgfpathlineto{\pgfqpoint{10.400177in}{5.022092in}}%
\pgfpathlineto{\pgfqpoint{10.404228in}{4.229825in}}%
\pgfpathlineto{\pgfqpoint{10.408280in}{4.840910in}}%
\pgfpathlineto{\pgfqpoint{10.412331in}{4.399720in}}%
\pgfpathlineto{\pgfqpoint{10.416383in}{4.684363in}}%
\pgfpathlineto{\pgfqpoint{10.420434in}{4.541526in}}%
\pgfpathlineto{\pgfqpoint{10.424486in}{4.558080in}}%
\pgfpathlineto{\pgfqpoint{10.428537in}{4.652030in}}%
\pgfpathlineto{\pgfqpoint{10.432589in}{4.463191in}}%
\pgfpathlineto{\pgfqpoint{10.436640in}{4.731773in}}%
\pgfpathlineto{\pgfqpoint{10.440692in}{4.397945in}}%
\pgfpathlineto{\pgfqpoint{10.444743in}{4.783242in}}%
\pgfpathlineto{\pgfqpoint{10.448795in}{4.359554in}}%
\pgfpathlineto{\pgfqpoint{10.452846in}{4.809155in}}%
\pgfpathlineto{\pgfqpoint{10.456898in}{4.345661in}}%
\pgfpathlineto{\pgfqpoint{10.460949in}{4.811326in}}%
\pgfpathlineto{\pgfqpoint{10.465000in}{4.355066in}}%
\pgfpathlineto{\pgfqpoint{10.469052in}{4.790373in}}%
\pgfpathlineto{\pgfqpoint{10.473103in}{4.387606in}}%
\pgfpathlineto{\pgfqpoint{10.477155in}{4.746202in}}%
\pgfpathlineto{\pgfqpoint{10.481206in}{4.443387in}}%
\pgfpathlineto{\pgfqpoint{10.485258in}{4.678962in}}%
\pgfpathlineto{\pgfqpoint{10.489309in}{4.521750in}}%
\pgfpathlineto{\pgfqpoint{10.493361in}{4.590042in}}%
\pgfpathlineto{\pgfqpoint{10.497412in}{4.620405in}}%
\pgfpathlineto{\pgfqpoint{10.501464in}{4.482732in}}%
\pgfpathlineto{\pgfqpoint{10.505515in}{4.735045in}}%
\pgfpathlineto{\pgfqpoint{10.509567in}{4.362297in}}%
\pgfpathlineto{\pgfqpoint{10.513618in}{4.859576in}}%
\pgfpathlineto{\pgfqpoint{10.517670in}{4.235475in}}%
\pgfpathlineto{\pgfqpoint{10.521721in}{4.986845in}}%
\pgfpathlineto{\pgfqpoint{10.525772in}{4.109579in}}%
\pgfpathlineto{\pgfqpoint{10.529824in}{5.109630in}}%
\pgfpathlineto{\pgfqpoint{10.533875in}{3.991512in}}%
\pgfpathlineto{\pgfqpoint{10.537927in}{5.221540in}}%
\pgfpathlineto{\pgfqpoint{10.541978in}{3.887010in}}%
\pgfpathlineto{\pgfqpoint{10.546030in}{5.317580in}}%
\pgfpathlineto{\pgfqpoint{10.550081in}{3.800303in}}%
\pgfpathlineto{\pgfqpoint{10.554133in}{5.394244in}}%
\pgfpathlineto{\pgfqpoint{10.558184in}{3.734262in}}%
\pgfpathlineto{\pgfqpoint{10.562236in}{5.449168in}}%
\pgfpathlineto{\pgfqpoint{10.566287in}{3.690900in}}%
\pgfpathlineto{\pgfqpoint{10.570339in}{5.480534in}}%
\pgfpathlineto{\pgfqpoint{10.574390in}{3.671988in}}%
\pgfpathlineto{\pgfqpoint{10.578442in}{5.486490in}}%
\pgfpathlineto{\pgfqpoint{10.582493in}{3.679546in}}%
\pgfpathlineto{\pgfqpoint{10.586545in}{5.464810in}}%
\pgfpathlineto{\pgfqpoint{10.590596in}{3.715985in}}%
\pgfpathlineto{\pgfqpoint{10.594647in}{5.412972in}}%
\pgfpathlineto{\pgfqpoint{10.598699in}{3.783821in}}%
\pgfpathlineto{\pgfqpoint{10.602750in}{5.328632in}}%
\pgfpathlineto{\pgfqpoint{10.606802in}{3.885024in}}%
\pgfpathlineto{\pgfqpoint{10.610853in}{5.210408in}}%
\pgfpathlineto{\pgfqpoint{10.614905in}{4.020177in}}%
\pgfpathlineto{\pgfqpoint{10.618956in}{5.058715in}}%
\pgfpathlineto{\pgfqpoint{10.623008in}{4.187692in}}%
\pgfpathlineto{\pgfqpoint{10.627059in}{4.876441in}}%
\pgfpathlineto{\pgfqpoint{10.631111in}{4.383309in}}%
\pgfpathlineto{\pgfqpoint{10.635162in}{4.669239in}}%
\pgfpathlineto{\pgfqpoint{10.639214in}{4.600022in}}%
\pgfpathlineto{\pgfqpoint{10.643265in}{4.445365in}}%
\pgfpathlineto{\pgfqpoint{10.647317in}{4.828482in}}%
\pgfpathlineto{\pgfqpoint{10.651368in}{4.215064in}}%
\pgfpathlineto{\pgfqpoint{10.655419in}{5.057776in}}%
\pgfpathlineto{\pgfqpoint{10.659471in}{3.989664in}}%
\pgfpathlineto{\pgfqpoint{10.663522in}{5.276416in}}%
\pgfpathlineto{\pgfqpoint{10.667574in}{3.780570in}}%
\pgfpathlineto{\pgfqpoint{10.671625in}{5.473310in}}%
\pgfpathlineto{\pgfqpoint{10.675677in}{3.598359in}}%
\pgfpathlineto{\pgfqpoint{10.679728in}{5.638554in}}%
\pgfpathlineto{\pgfqpoint{10.683780in}{3.452146in}}%
\pgfpathlineto{\pgfqpoint{10.687831in}{5.763901in}}%
\pgfpathlineto{\pgfqpoint{10.691883in}{3.349271in}}%
\pgfpathlineto{\pgfqpoint{10.695934in}{5.842919in}}%
\pgfpathlineto{\pgfqpoint{10.699986in}{3.295282in}}%
\pgfpathlineto{\pgfqpoint{10.704037in}{5.870906in}}%
\pgfpathlineto{\pgfqpoint{10.708089in}{3.294081in}}%
\pgfpathlineto{\pgfqpoint{10.712140in}{5.844724in}}%
\pgfpathlineto{\pgfqpoint{10.716191in}{3.348062in}}%
\pgfpathlineto{\pgfqpoint{10.720243in}{5.762719in}}%
\pgfpathlineto{\pgfqpoint{10.724294in}{3.458105in}}%
\pgfpathlineto{\pgfqpoint{10.728346in}{5.624857in}}%
\pgfpathlineto{\pgfqpoint{10.732397in}{3.623302in}}%
\pgfpathlineto{\pgfqpoint{10.736449in}{5.433113in}}%
\pgfpathlineto{\pgfqpoint{10.740500in}{3.840459in}}%
\pgfpathlineto{\pgfqpoint{10.744552in}{5.192059in}}%
\pgfpathlineto{\pgfqpoint{10.748603in}{4.103477in}}%
\pgfpathlineto{\pgfqpoint{10.752655in}{4.909456in}}%
\pgfpathlineto{\pgfqpoint{10.756706in}{4.402828in}}%
\pgfpathlineto{\pgfqpoint{10.760758in}{4.596650in}}%
\pgfpathlineto{\pgfqpoint{10.764809in}{4.725363in}}%
\pgfpathlineto{\pgfqpoint{10.768861in}{4.268505in}}%
\pgfpathlineto{\pgfqpoint{10.772912in}{5.054670in}}%
\pgfpathlineto{\pgfqpoint{10.776963in}{3.942728in}}%
\pgfpathlineto{\pgfqpoint{10.781015in}{5.372082in}}%
\pgfpathlineto{\pgfqpoint{10.785066in}{3.638537in}}%
\pgfpathlineto{\pgfqpoint{10.789118in}{5.658306in}}%
\pgfpathlineto{\pgfqpoint{10.793169in}{3.374779in}}%
\pgfpathlineto{\pgfqpoint{10.797221in}{5.895487in}}%
\pgfpathlineto{\pgfqpoint{10.801272in}{3.167766in}}%
\pgfpathlineto{\pgfqpoint{10.805324in}{6.069382in}}%
\pgfpathlineto{\pgfqpoint{10.809375in}{3.029192in}}%
\pgfpathlineto{\pgfqpoint{10.813427in}{6.171257in}}%
\pgfpathlineto{\pgfqpoint{10.817478in}{2.964517in}}%
\pgfpathlineto{\pgfqpoint{10.821530in}{6.199129in}}%
\pgfpathlineto{\pgfqpoint{10.825581in}{2.972166in}}%
\pgfpathlineto{\pgfqpoint{10.829633in}{6.158077in}}%
\pgfpathlineto{\pgfqpoint{10.833684in}{3.043746in}}%
\pgfpathlineto{\pgfqpoint{10.837735in}{6.059496in}}%
\pgfpathlineto{\pgfqpoint{10.841787in}{3.165286in}}%
\pgfpathlineto{\pgfqpoint{10.845838in}{5.919403in}}%
\pgfpathlineto{\pgfqpoint{10.849890in}{3.319324in}}%
\pgfpathlineto{\pgfqpoint{10.853941in}{5.756058in}}%
\pgfpathlineto{\pgfqpoint{10.857993in}{3.487469in}}%
\pgfpathlineto{\pgfqpoint{10.862044in}{5.587334in}}%
\pgfpathlineto{\pgfqpoint{10.866096in}{3.652975in}}%
\pgfpathlineto{\pgfqpoint{10.870147in}{5.428311in}}%
\pgfpathlineto{\pgfqpoint{10.874199in}{3.802864in}}%
\pgfpathlineto{\pgfqpoint{10.878250in}{5.289540in}}%
\pgfpathlineto{\pgfqpoint{10.882302in}{3.929214in}}%
\pgfpathlineto{\pgfqpoint{10.886353in}{5.176251in}}%
\pgfpathlineto{\pgfqpoint{10.890405in}{4.029417in}}%
\pgfpathlineto{\pgfqpoint{10.894456in}{5.088623in}}%
\pgfpathlineto{\pgfqpoint{10.898507in}{4.105419in}}%
\pgfpathlineto{\pgfqpoint{10.902559in}{5.022977in}}%
\pgfpathlineto{\pgfqpoint{10.906610in}{4.162176in}}%
\pgfpathlineto{\pgfqpoint{10.910662in}{4.973572in}}%
\pgfpathlineto{\pgfqpoint{10.914713in}{4.205716in}}%
\pgfpathlineto{\pgfqpoint{10.918765in}{4.934567in}}%
\pgfpathlineto{\pgfqpoint{10.922816in}{4.241274in}}%
\pgfpathlineto{\pgfqpoint{10.926868in}{4.901676in}}%
\pgfpathlineto{\pgfqpoint{10.930919in}{4.271933in}}%
\pgfpathlineto{\pgfqpoint{10.934971in}{4.873163in}}%
\pgfpathlineto{\pgfqpoint{10.939022in}{4.298058in}}%
\pgfpathlineto{\pgfqpoint{10.943074in}{4.849948in}}%
\pgfpathlineto{\pgfqpoint{10.947125in}{4.317628in}}%
\pgfpathlineto{\pgfqpoint{10.951177in}{4.834884in}}%
\pgfpathlineto{\pgfqpoint{10.955228in}{4.327299in}}%
\pgfpathlineto{\pgfqpoint{10.959279in}{4.831424in}}%
\pgfpathlineto{\pgfqpoint{10.963331in}{4.323892in}}%
\pgfpathlineto{\pgfqpoint{10.967382in}{4.842103in}}%
\pgfpathlineto{\pgfqpoint{10.971434in}{4.305846in}}%
\pgfpathlineto{\pgfqpoint{10.975485in}{4.867256in}}%
\pgfpathlineto{\pgfqpoint{10.979537in}{4.274219in}}%
\pgfpathlineto{\pgfqpoint{10.983588in}{4.904364in}}%
\pgfpathlineto{\pgfqpoint{10.987640in}{4.232950in}}%
\pgfpathlineto{\pgfqpoint{10.991691in}{4.948209in}}%
\pgfpathlineto{\pgfqpoint{10.995743in}{4.188302in}}%
\pgfpathlineto{\pgfqpoint{10.999794in}{4.991789in}}%
\pgfpathlineto{\pgfqpoint{11.003846in}{4.147660in}}%
\pgfpathlineto{\pgfqpoint{11.007897in}{5.027721in}}%
\pgfpathlineto{\pgfqpoint{11.011949in}{4.118025in}}%
\pgfpathlineto{\pgfqpoint{11.016000in}{5.049730in}}%
\pgfpathlineto{\pgfqpoint{11.020051in}{4.104658in}}%
\pgfpathlineto{\pgfqpoint{11.024103in}{5.053779in}}%
\pgfpathlineto{\pgfqpoint{11.028154in}{4.110257in}}%
\pgfpathlineto{\pgfqpoint{11.032206in}{5.038526in}}%
\pgfpathlineto{\pgfqpoint{11.036257in}{4.134892in}}%
\pgfpathlineto{\pgfqpoint{11.040309in}{5.004997in}}%
\pgfpathlineto{\pgfqpoint{11.044360in}{4.176690in}}%
\pgfpathlineto{\pgfqpoint{11.048412in}{4.955607in}}%
\pgfpathlineto{\pgfqpoint{11.052463in}{4.233031in}}%
\pgfpathlineto{\pgfqpoint{11.056515in}{4.892842in}}%
\pgfpathlineto{\pgfqpoint{11.060566in}{4.301873in}}%
\pgfpathlineto{\pgfqpoint{11.064618in}{4.818040in}}%
\pgfpathlineto{\pgfqpoint{11.068669in}{4.382772in}}%
\pgfpathlineto{\pgfqpoint{11.072721in}{4.730651in}}%
\pgfpathlineto{\pgfqpoint{11.076772in}{4.477277in}}%
\pgfpathlineto{\pgfqpoint{11.080823in}{4.628217in}}%
\pgfpathlineto{\pgfqpoint{11.084875in}{4.588575in}}%
\pgfpathlineto{\pgfqpoint{11.088926in}{4.507082in}}%
\pgfpathlineto{\pgfqpoint{11.092978in}{4.720471in}}%
\pgfpathlineto{\pgfqpoint{11.097029in}{4.363643in}}%
\pgfpathlineto{\pgfqpoint{11.101081in}{4.876005in}}%
\pgfpathlineto{\pgfqpoint{11.105132in}{4.195771in}}%
\pgfpathlineto{\pgfqpoint{11.109184in}{5.056086in}}%
\pgfpathlineto{\pgfqpoint{11.113235in}{4.004023in}}%
\pgfpathlineto{\pgfqpoint{11.117287in}{5.258526in}}%
\pgfpathlineto{\pgfqpoint{11.121338in}{3.792298in}}%
\pgfpathlineto{\pgfqpoint{11.125390in}{5.477722in}}%
\pgfpathlineto{\pgfqpoint{11.129441in}{3.567806in}}%
\pgfpathlineto{\pgfqpoint{11.133493in}{5.705034in}}%
\pgfpathlineto{\pgfqpoint{11.137544in}{3.340377in}}%
\pgfpathlineto{\pgfqpoint{11.141596in}{5.929726in}}%
\pgfpathlineto{\pgfqpoint{11.145647in}{3.121345in}}%
\pgfpathlineto{\pgfqpoint{11.149698in}{6.140177in}}%
\pgfpathlineto{\pgfqpoint{11.153750in}{2.922333in}}%
\pgfpathlineto{\pgfqpoint{11.157801in}{6.325011in}}%
\pgfpathlineto{\pgfqpoint{11.161853in}{2.754264in}}%
\pgfpathlineto{\pgfqpoint{11.165904in}{6.473896in}}%
\pgfpathlineto{\pgfqpoint{11.169956in}{2.626807in}}%
\pgfpathlineto{\pgfqpoint{11.174007in}{6.577848in}}%
\pgfpathlineto{\pgfqpoint{11.178059in}{2.548289in}}%
\pgfpathlineto{\pgfqpoint{11.182110in}{6.629130in}}%
\pgfpathlineto{\pgfqpoint{11.186162in}{2.525936in}}%
\pgfpathlineto{\pgfqpoint{11.190213in}{6.620957in}}%
\pgfpathlineto{\pgfqpoint{11.194265in}{2.566140in}}%
\pgfpathlineto{\pgfqpoint{11.198316in}{6.547323in}}%
\pgfpathlineto{\pgfqpoint{11.202368in}{2.674468in}}%
\pgfpathlineto{\pgfqpoint{11.206419in}{6.403221in}}%
\pgfpathlineto{\pgfqpoint{11.210470in}{2.855174in}}%
\pgfpathlineto{\pgfqpoint{11.214522in}{6.185404in}}%
\pgfpathlineto{\pgfqpoint{11.218573in}{3.110197in}}%
\pgfpathlineto{\pgfqpoint{11.222625in}{5.893581in}}%
\pgfpathlineto{\pgfqpoint{11.226676in}{3.437835in}}%
\pgfpathlineto{\pgfqpoint{11.230728in}{5.531766in}}%
\pgfpathlineto{\pgfqpoint{11.234779in}{3.831485in}}%
\pgfpathlineto{\pgfqpoint{11.238831in}{5.109353in}}%
\pgfpathlineto{\pgfqpoint{11.242882in}{4.278865in}}%
\pgfpathlineto{\pgfqpoint{11.246934in}{4.641485in}}%
\pgfpathlineto{\pgfqpoint{11.250985in}{4.762124in}}%
\pgfpathlineto{\pgfqpoint{11.255037in}{4.148446in}}%
\pgfpathlineto{\pgfqpoint{11.259088in}{5.258949in}}%
\pgfpathlineto{\pgfqpoint{11.263140in}{3.654063in}}%
\pgfpathlineto{\pgfqpoint{11.267191in}{5.744591in}}%
\pgfpathlineto{\pgfqpoint{11.271242in}{3.183364in}}%
\pgfpathlineto{\pgfqpoint{11.275294in}{6.194407in}}%
\pgfpathlineto{\pgfqpoint{11.279345in}{2.759959in}}%
\pgfpathlineto{\pgfqpoint{11.283397in}{6.586421in}}%
\pgfpathlineto{\pgfqpoint{11.287448in}{2.403660in}}%
\pgfpathlineto{\pgfqpoint{11.291500in}{6.903416in}}%
\pgfpathlineto{\pgfqpoint{11.295551in}{2.128772in}}%
\pgfpathlineto{\pgfqpoint{11.299603in}{7.134203in}}%
\pgfpathlineto{\pgfqpoint{11.303654in}{1.943274in}}%
\pgfpathlineto{\pgfqpoint{11.307706in}{7.274004in}}%
\pgfpathlineto{\pgfqpoint{11.311757in}{1.848843in}}%
\pgfpathlineto{\pgfqpoint{11.315809in}{7.324070in}}%
\pgfpathlineto{\pgfqpoint{11.319860in}{1.841520in}}%
\pgfpathlineto{\pgfqpoint{11.323912in}{7.290825in}}%
\pgfpathlineto{\pgfqpoint{11.327963in}{1.912674in}}%
\pgfpathlineto{\pgfqpoint{11.332014in}{7.184842in}}%
\pgfpathlineto{\pgfqpoint{11.336066in}{2.050043in}}%
\pgfpathlineto{\pgfqpoint{11.340117in}{7.019832in}}%
\pgfpathlineto{\pgfqpoint{11.344169in}{2.238700in}}%
\pgfpathlineto{\pgfqpoint{11.348220in}{6.811707in}}%
\pgfpathlineto{\pgfqpoint{11.352272in}{2.461991in}}%
\pgfpathlineto{\pgfqpoint{11.356323in}{6.577608in}}%
\pgfpathlineto{\pgfqpoint{11.360375in}{2.702560in}}%
\pgfpathlineto{\pgfqpoint{11.364426in}{6.334806in}}%
\pgfpathlineto{\pgfqpoint{11.368478in}{2.943551in}}%
\pgfpathlineto{\pgfqpoint{11.372529in}{6.099392in}}%
\pgfpathlineto{\pgfqpoint{11.376581in}{3.169993in}}%
\pgfpathlineto{\pgfqpoint{11.380632in}{5.884864in}}%
\pgfpathlineto{\pgfqpoint{11.384684in}{3.370188in}}%
\pgfpathlineto{\pgfqpoint{11.388735in}{5.700840in}}%
\pgfpathlineto{\pgfqpoint{11.392786in}{3.536815in}}%
\pgfpathlineto{\pgfqpoint{11.396838in}{5.552216in}}%
\pgfpathlineto{\pgfqpoint{11.400889in}{3.667432in}}%
\pgfpathlineto{\pgfqpoint{11.404941in}{5.439055in}}%
\pgfpathlineto{\pgfqpoint{11.408992in}{3.764168in}}%
\pgfpathlineto{\pgfqpoint{11.413044in}{5.357327in}}%
\pgfpathlineto{\pgfqpoint{11.417095in}{3.832579in}}%
\pgfpathlineto{\pgfqpoint{11.421147in}{5.300386in}}%
\pgfpathlineto{\pgfqpoint{11.425198in}{3.879930in}}%
\pgfpathlineto{\pgfqpoint{11.429250in}{5.260826in}}%
\pgfpathlineto{\pgfqpoint{11.433301in}{3.913323in}}%
\pgfpathlineto{\pgfqpoint{11.437353in}{5.232226in}}%
\pgfpathlineto{\pgfqpoint{11.441404in}{3.938207in}}%
\pgfpathlineto{\pgfqpoint{11.445456in}{5.210288in}}%
\pgfpathlineto{\pgfqpoint{11.449507in}{3.957680in}}%
\pgfpathlineto{\pgfqpoint{11.453558in}{5.193037in}}%
\pgfpathlineto{\pgfqpoint{11.457610in}{3.972786in}}%
\pgfpathlineto{\pgfqpoint{11.461661in}{5.180072in}}%
\pgfpathlineto{\pgfqpoint{11.465713in}{3.983645in}}%
\pgfpathlineto{\pgfqpoint{11.469764in}{5.171157in}}%
\pgfpathlineto{\pgfqpoint{11.473816in}{3.990995in}}%
\pgfpathlineto{\pgfqpoint{11.477867in}{5.164710in}}%
\pgfpathlineto{\pgfqpoint{11.481919in}{3.997521in}}%
\pgfpathlineto{\pgfqpoint{11.485970in}{5.156807in}}%
\pgfpathlineto{\pgfqpoint{11.490022in}{4.008380in}}%
\pgfpathlineto{\pgfqpoint{11.494073in}{5.141209in}}%
\pgfpathlineto{\pgfqpoint{11.498125in}{4.030590in}}%
\pgfpathlineto{\pgfqpoint{11.502176in}{5.110558in}}%
\pgfpathlineto{\pgfqpoint{11.506228in}{4.071309in}}%
\pgfpathlineto{\pgfqpoint{11.510279in}{5.058501in}}%
\pgfpathlineto{\pgfqpoint{11.514330in}{4.135464in}}%
\pgfpathlineto{\pgfqpoint{11.518382in}{4.982124in}}%
\pgfpathlineto{\pgfqpoint{11.522433in}{4.223463in}}%
\pgfpathlineto{\pgfqpoint{11.526485in}{4.883880in}}%
\pgfpathlineto{\pgfqpoint{11.530536in}{4.329808in}}%
\pgfpathlineto{\pgfqpoint{11.534588in}{4.772281in}}%
\pgfpathlineto{\pgfqpoint{11.538639in}{4.443224in}}%
\pgfpathlineto{\pgfqpoint{11.542691in}{4.660911in}}%
\pgfpathlineto{\pgfqpoint{11.546742in}{4.548464in}}%
\pgfpathlineto{\pgfqpoint{11.550794in}{4.565872in}}%
\pgfpathlineto{\pgfqpoint{11.554845in}{4.629484in}}%
\pgfpathlineto{\pgfqpoint{11.558897in}{4.502202in}}%
\pgfpathlineto{\pgfqpoint{11.562948in}{4.673179in}}%
\pgfpathlineto{\pgfqpoint{11.567000in}{4.480229in}}%
\pgfpathlineto{\pgfqpoint{11.571051in}{4.672685in}}%
\pgfpathlineto{\pgfqpoint{11.575102in}{4.502864in}}%
\pgfpathlineto{\pgfqpoint{11.579154in}{4.629299in}}%
\pgfpathlineto{\pgfqpoint{11.583205in}{4.564614in}}%
\pgfpathlineto{\pgfqpoint{11.587257in}{4.552440in}}%
\pgfpathlineto{\pgfqpoint{11.591308in}{4.652641in}}%
\pgfpathlineto{\pgfqpoint{11.595360in}{4.457648in}}%
\pgfpathlineto{\pgfqpoint{11.599411in}{4.749584in}}%
\pgfpathlineto{\pgfqpoint{11.603463in}{4.363128in}}%
\pgfpathlineto{\pgfqpoint{11.607514in}{4.837394in}}%
\pgfpathlineto{\pgfqpoint{11.611566in}{4.285809in}}%
\pgfpathlineto{\pgfqpoint{11.615617in}{4.901126in}}%
\pgfpathlineto{\pgfqpoint{11.619669in}{4.237954in}}%
\pgfpathlineto{\pgfqpoint{11.623720in}{4.931688in}}%
\pgfpathlineto{\pgfqpoint{11.627772in}{4.225218in}}%
\pgfpathlineto{\pgfqpoint{11.631823in}{4.926887in}}%
\pgfpathlineto{\pgfqpoint{11.635874in}{4.246547in}}%
\pgfpathlineto{\pgfqpoint{11.639926in}{4.890612in}}%
\pgfpathlineto{\pgfqpoint{11.643977in}{4.295793in}}%
\pgfpathlineto{\pgfqpoint{11.648029in}{4.830580in}}%
\pgfpathlineto{\pgfqpoint{11.652080in}{4.364404in}}%
\pgfpathlineto{\pgfqpoint{11.656132in}{4.755450in}}%
\pgfpathlineto{\pgfqpoint{11.660183in}{4.444270in}}%
\pgfpathlineto{\pgfqpoint{11.664235in}{4.672258in}}%
\pgfpathlineto{\pgfqpoint{11.668286in}{4.529797in}}%
\pgfpathlineto{\pgfqpoint{11.672338in}{4.584973in}}%
\pgfpathlineto{\pgfqpoint{11.672338in}{4.584973in}}%
\pgfusepath{stroke}%
\end{pgfscope}%
\begin{pgfscope}%
\pgfpathrectangle{\pgfqpoint{3.126011in}{1.247073in}}{\pgfqpoint{8.795249in}{6.674186in}}%
\pgfusepath{clip}%
\pgfsetbuttcap%
\pgfsetroundjoin%
\pgfsetlinewidth{1.003750pt}%
\definecolor{currentstroke}{rgb}{0.888874,0.435649,0.278123}%
\pgfsetstrokecolor{currentstroke}%
\pgfsetdash{}{0pt}%
\pgfpathmoveto{\pgfqpoint{3.374933in}{4.579402in}}%
\pgfpathlineto{\pgfqpoint{11.672338in}{4.579402in}}%
\pgfpathlineto{\pgfqpoint{11.672338in}{4.579402in}}%
\pgfusepath{stroke}%
\end{pgfscope}%
\begin{pgfscope}%
\pgfsetrectcap%
\pgfsetmiterjoin%
\pgfsetlinewidth{1.003750pt}%
\definecolor{currentstroke}{rgb}{0.000000,0.000000,0.000000}%
\pgfsetstrokecolor{currentstroke}%
\pgfsetdash{}{0pt}%
\pgfpathmoveto{\pgfqpoint{3.126011in}{1.247073in}}%
\pgfpathlineto{\pgfqpoint{3.126011in}{7.921260in}}%
\pgfusepath{stroke}%
\end{pgfscope}%
\begin{pgfscope}%
\pgfsetrectcap%
\pgfsetmiterjoin%
\pgfsetlinewidth{1.003750pt}%
\definecolor{currentstroke}{rgb}{0.000000,0.000000,0.000000}%
\pgfsetstrokecolor{currentstroke}%
\pgfsetdash{}{0pt}%
\pgfpathmoveto{\pgfqpoint{3.126011in}{1.247073in}}%
\pgfpathlineto{\pgfqpoint{11.921260in}{1.247073in}}%
\pgfusepath{stroke}%
\end{pgfscope}%
\begin{pgfscope}%
\pgfsetbuttcap%
\pgfsetmiterjoin%
\definecolor{currentfill}{rgb}{1.000000,1.000000,1.000000}%
\pgfsetfillcolor{currentfill}%
\pgfsetlinewidth{1.003750pt}%
\definecolor{currentstroke}{rgb}{0.000000,0.000000,0.000000}%
\pgfsetstrokecolor{currentstroke}%
\pgfsetdash{}{0pt}%
\pgfpathmoveto{\pgfqpoint{10.511589in}{6.787373in}}%
\pgfpathlineto{\pgfqpoint{11.796260in}{6.787373in}}%
\pgfpathlineto{\pgfqpoint{11.796260in}{7.796260in}}%
\pgfpathlineto{\pgfqpoint{10.511589in}{7.796260in}}%
\pgfpathclose%
\pgfusepath{stroke,fill}%
\end{pgfscope}%
\begin{pgfscope}%
\pgfsetbuttcap%
\pgfsetmiterjoin%
\pgfsetlinewidth{2.258437pt}%
\definecolor{currentstroke}{rgb}{0.000000,0.605603,0.978680}%
\pgfsetstrokecolor{currentstroke}%
\pgfsetdash{}{0pt}%
\pgfpathmoveto{\pgfqpoint{10.711589in}{7.493818in}}%
\pgfpathlineto{\pgfqpoint{11.211589in}{7.493818in}}%
\pgfusepath{stroke}%
\end{pgfscope}%
\begin{pgfscope}%
\definecolor{textcolor}{rgb}{0.000000,0.000000,0.000000}%
\pgfsetstrokecolor{textcolor}%
\pgfsetfillcolor{textcolor}%
\pgftext[x=11.411589in,y=7.406318in,left,base]{\color{textcolor}\sffamily\fontsize{18.000000}{21.600000}\selectfont $\displaystyle U$}%
\end{pgfscope}%
\begin{pgfscope}%
\pgfsetbuttcap%
\pgfsetmiterjoin%
\pgfsetlinewidth{2.258437pt}%
\definecolor{currentstroke}{rgb}{0.888874,0.435649,0.278123}%
\pgfsetstrokecolor{currentstroke}%
\pgfsetdash{}{0pt}%
\pgfpathmoveto{\pgfqpoint{10.711589in}{7.126875in}}%
\pgfpathlineto{\pgfqpoint{11.211589in}{7.126875in}}%
\pgfusepath{stroke}%
\end{pgfscope}%
\begin{pgfscope}%
\definecolor{textcolor}{rgb}{0.000000,0.000000,0.000000}%
\pgfsetstrokecolor{textcolor}%
\pgfsetfillcolor{textcolor}%
\pgftext[x=11.411589in,y=7.039375in,left,base]{\color{textcolor}\sffamily\fontsize{18.000000}{21.600000}\selectfont $\displaystyle u$}%
\end{pgfscope}%
\end{pgfpicture}%
\makeatother%
\endgroup%
}
	\caption{Beam-Warming 差分逼近解 $U$ 与真解 $u$}\label{fig:beam_warming_Uu_noCFL}
\end{figure}

对方波问题, 取 $\nu = 0.5$. $h = 2^{-7}$ 和 $h = 2^{-11}$ 时差分逼近解 $U$ 与真解 $u$ 在 $t = t_{\max }$ 时刻图像如图 \ref{fig:beam_warming_square_Uu} 所示. 可以看出差分逼近解在间断点右侧出现震荡.

\begin{figure}[H]\centering\zihao{-5}
	\resizebox{0.4\linewidth}{!}{%% Creator: Matplotlib, PGF backend
%%
%% To include the figure in your LaTeX document, write
%%   \input{<filename>.pgf}
%%
%% Make sure the required packages are loaded in your preamble
%%   \usepackage{pgf}
%%
%% Figures using additional raster images can only be included by \input if
%% they are in the same directory as the main LaTeX file. For loading figures
%% from other directories you can use the `import` package
%%   \usepackage{import}
%%
%% and then include the figures with
%%   \import{<path to file>}{<filename>.pgf}
%%
%% Matplotlib used the following preamble
%%   \usepackage{fontspec}
%%   \setmainfont{DejaVuSerif.ttf}[Path=\detokenize{/Users/quejiahao/.julia/conda/3/lib/python3.9/site-packages/matplotlib/mpl-data/fonts/ttf/}]
%%   \setsansfont{DejaVuSans.ttf}[Path=\detokenize{/Users/quejiahao/.julia/conda/3/lib/python3.9/site-packages/matplotlib/mpl-data/fonts/ttf/}]
%%   \setmonofont{DejaVuSansMono.ttf}[Path=\detokenize{/Users/quejiahao/.julia/conda/3/lib/python3.9/site-packages/matplotlib/mpl-data/fonts/ttf/}]
%%
\begingroup%
\makeatletter%
\begin{pgfpicture}%
\pgfpathrectangle{\pgfpointorigin}{\pgfqpoint{12.000000in}{8.000000in}}%
\pgfusepath{use as bounding box, clip}%
\begin{pgfscope}%
\pgfsetbuttcap%
\pgfsetmiterjoin%
\definecolor{currentfill}{rgb}{1.000000,1.000000,1.000000}%
\pgfsetfillcolor{currentfill}%
\pgfsetlinewidth{0.000000pt}%
\definecolor{currentstroke}{rgb}{1.000000,1.000000,1.000000}%
\pgfsetstrokecolor{currentstroke}%
\pgfsetdash{}{0pt}%
\pgfpathmoveto{\pgfqpoint{0.000000in}{0.000000in}}%
\pgfpathlineto{\pgfqpoint{12.000000in}{0.000000in}}%
\pgfpathlineto{\pgfqpoint{12.000000in}{8.000000in}}%
\pgfpathlineto{\pgfqpoint{0.000000in}{8.000000in}}%
\pgfpathclose%
\pgfusepath{fill}%
\end{pgfscope}%
\begin{pgfscope}%
\pgfsetbuttcap%
\pgfsetmiterjoin%
\definecolor{currentfill}{rgb}{1.000000,1.000000,1.000000}%
\pgfsetfillcolor{currentfill}%
\pgfsetlinewidth{0.000000pt}%
\definecolor{currentstroke}{rgb}{0.000000,0.000000,0.000000}%
\pgfsetstrokecolor{currentstroke}%
\pgfsetstrokeopacity{0.000000}%
\pgfsetdash{}{0pt}%
\pgfpathmoveto{\pgfqpoint{0.978013in}{1.247073in}}%
\pgfpathlineto{\pgfqpoint{11.921260in}{1.247073in}}%
\pgfpathlineto{\pgfqpoint{11.921260in}{7.921260in}}%
\pgfpathlineto{\pgfqpoint{0.978013in}{7.921260in}}%
\pgfpathclose%
\pgfusepath{fill}%
\end{pgfscope}%
\begin{pgfscope}%
\pgfpathrectangle{\pgfqpoint{0.978013in}{1.247073in}}{\pgfqpoint{10.943247in}{6.674186in}}%
\pgfusepath{clip}%
\pgfsetrectcap%
\pgfsetroundjoin%
\pgfsetlinewidth{0.501875pt}%
\definecolor{currentstroke}{rgb}{0.000000,0.000000,0.000000}%
\pgfsetstrokecolor{currentstroke}%
\pgfsetstrokeopacity{0.100000}%
\pgfsetdash{}{0pt}%
\pgfpathmoveto{\pgfqpoint{1.287728in}{1.247073in}}%
\pgfpathlineto{\pgfqpoint{1.287728in}{7.921260in}}%
\pgfusepath{stroke}%
\end{pgfscope}%
\begin{pgfscope}%
\pgfsetbuttcap%
\pgfsetroundjoin%
\definecolor{currentfill}{rgb}{0.000000,0.000000,0.000000}%
\pgfsetfillcolor{currentfill}%
\pgfsetlinewidth{0.501875pt}%
\definecolor{currentstroke}{rgb}{0.000000,0.000000,0.000000}%
\pgfsetstrokecolor{currentstroke}%
\pgfsetdash{}{0pt}%
\pgfsys@defobject{currentmarker}{\pgfqpoint{0.000000in}{0.000000in}}{\pgfqpoint{0.000000in}{0.034722in}}{%
\pgfpathmoveto{\pgfqpoint{0.000000in}{0.000000in}}%
\pgfpathlineto{\pgfqpoint{0.000000in}{0.034722in}}%
\pgfusepath{stroke,fill}%
}%
\begin{pgfscope}%
\pgfsys@transformshift{1.287728in}{1.247073in}%
\pgfsys@useobject{currentmarker}{}%
\end{pgfscope}%
\end{pgfscope}%
\begin{pgfscope}%
\definecolor{textcolor}{rgb}{0.000000,0.000000,0.000000}%
\pgfsetstrokecolor{textcolor}%
\pgfsetfillcolor{textcolor}%
\pgftext[x=1.287728in,y=1.198462in,,top]{\color{textcolor}\sffamily\fontsize{18.000000}{21.600000}\selectfont $\displaystyle 0$}%
\end{pgfscope}%
\begin{pgfscope}%
\pgfpathrectangle{\pgfqpoint{0.978013in}{1.247073in}}{\pgfqpoint{10.943247in}{6.674186in}}%
\pgfusepath{clip}%
\pgfsetrectcap%
\pgfsetroundjoin%
\pgfsetlinewidth{0.501875pt}%
\definecolor{currentstroke}{rgb}{0.000000,0.000000,0.000000}%
\pgfsetstrokecolor{currentstroke}%
\pgfsetstrokeopacity{0.100000}%
\pgfsetdash{}{0pt}%
\pgfpathmoveto{\pgfqpoint{2.930814in}{1.247073in}}%
\pgfpathlineto{\pgfqpoint{2.930814in}{7.921260in}}%
\pgfusepath{stroke}%
\end{pgfscope}%
\begin{pgfscope}%
\pgfsetbuttcap%
\pgfsetroundjoin%
\definecolor{currentfill}{rgb}{0.000000,0.000000,0.000000}%
\pgfsetfillcolor{currentfill}%
\pgfsetlinewidth{0.501875pt}%
\definecolor{currentstroke}{rgb}{0.000000,0.000000,0.000000}%
\pgfsetstrokecolor{currentstroke}%
\pgfsetdash{}{0pt}%
\pgfsys@defobject{currentmarker}{\pgfqpoint{0.000000in}{0.000000in}}{\pgfqpoint{0.000000in}{0.034722in}}{%
\pgfpathmoveto{\pgfqpoint{0.000000in}{0.000000in}}%
\pgfpathlineto{\pgfqpoint{0.000000in}{0.034722in}}%
\pgfusepath{stroke,fill}%
}%
\begin{pgfscope}%
\pgfsys@transformshift{2.930814in}{1.247073in}%
\pgfsys@useobject{currentmarker}{}%
\end{pgfscope}%
\end{pgfscope}%
\begin{pgfscope}%
\definecolor{textcolor}{rgb}{0.000000,0.000000,0.000000}%
\pgfsetstrokecolor{textcolor}%
\pgfsetfillcolor{textcolor}%
\pgftext[x=2.930814in,y=1.198462in,,top]{\color{textcolor}\sffamily\fontsize{18.000000}{21.600000}\selectfont $\displaystyle 1$}%
\end{pgfscope}%
\begin{pgfscope}%
\pgfpathrectangle{\pgfqpoint{0.978013in}{1.247073in}}{\pgfqpoint{10.943247in}{6.674186in}}%
\pgfusepath{clip}%
\pgfsetrectcap%
\pgfsetroundjoin%
\pgfsetlinewidth{0.501875pt}%
\definecolor{currentstroke}{rgb}{0.000000,0.000000,0.000000}%
\pgfsetstrokecolor{currentstroke}%
\pgfsetstrokeopacity{0.100000}%
\pgfsetdash{}{0pt}%
\pgfpathmoveto{\pgfqpoint{4.573901in}{1.247073in}}%
\pgfpathlineto{\pgfqpoint{4.573901in}{7.921260in}}%
\pgfusepath{stroke}%
\end{pgfscope}%
\begin{pgfscope}%
\pgfsetbuttcap%
\pgfsetroundjoin%
\definecolor{currentfill}{rgb}{0.000000,0.000000,0.000000}%
\pgfsetfillcolor{currentfill}%
\pgfsetlinewidth{0.501875pt}%
\definecolor{currentstroke}{rgb}{0.000000,0.000000,0.000000}%
\pgfsetstrokecolor{currentstroke}%
\pgfsetdash{}{0pt}%
\pgfsys@defobject{currentmarker}{\pgfqpoint{0.000000in}{0.000000in}}{\pgfqpoint{0.000000in}{0.034722in}}{%
\pgfpathmoveto{\pgfqpoint{0.000000in}{0.000000in}}%
\pgfpathlineto{\pgfqpoint{0.000000in}{0.034722in}}%
\pgfusepath{stroke,fill}%
}%
\begin{pgfscope}%
\pgfsys@transformshift{4.573901in}{1.247073in}%
\pgfsys@useobject{currentmarker}{}%
\end{pgfscope}%
\end{pgfscope}%
\begin{pgfscope}%
\definecolor{textcolor}{rgb}{0.000000,0.000000,0.000000}%
\pgfsetstrokecolor{textcolor}%
\pgfsetfillcolor{textcolor}%
\pgftext[x=4.573901in,y=1.198462in,,top]{\color{textcolor}\sffamily\fontsize{18.000000}{21.600000}\selectfont $\displaystyle 2$}%
\end{pgfscope}%
\begin{pgfscope}%
\pgfpathrectangle{\pgfqpoint{0.978013in}{1.247073in}}{\pgfqpoint{10.943247in}{6.674186in}}%
\pgfusepath{clip}%
\pgfsetrectcap%
\pgfsetroundjoin%
\pgfsetlinewidth{0.501875pt}%
\definecolor{currentstroke}{rgb}{0.000000,0.000000,0.000000}%
\pgfsetstrokecolor{currentstroke}%
\pgfsetstrokeopacity{0.100000}%
\pgfsetdash{}{0pt}%
\pgfpathmoveto{\pgfqpoint{6.216988in}{1.247073in}}%
\pgfpathlineto{\pgfqpoint{6.216988in}{7.921260in}}%
\pgfusepath{stroke}%
\end{pgfscope}%
\begin{pgfscope}%
\pgfsetbuttcap%
\pgfsetroundjoin%
\definecolor{currentfill}{rgb}{0.000000,0.000000,0.000000}%
\pgfsetfillcolor{currentfill}%
\pgfsetlinewidth{0.501875pt}%
\definecolor{currentstroke}{rgb}{0.000000,0.000000,0.000000}%
\pgfsetstrokecolor{currentstroke}%
\pgfsetdash{}{0pt}%
\pgfsys@defobject{currentmarker}{\pgfqpoint{0.000000in}{0.000000in}}{\pgfqpoint{0.000000in}{0.034722in}}{%
\pgfpathmoveto{\pgfqpoint{0.000000in}{0.000000in}}%
\pgfpathlineto{\pgfqpoint{0.000000in}{0.034722in}}%
\pgfusepath{stroke,fill}%
}%
\begin{pgfscope}%
\pgfsys@transformshift{6.216988in}{1.247073in}%
\pgfsys@useobject{currentmarker}{}%
\end{pgfscope}%
\end{pgfscope}%
\begin{pgfscope}%
\definecolor{textcolor}{rgb}{0.000000,0.000000,0.000000}%
\pgfsetstrokecolor{textcolor}%
\pgfsetfillcolor{textcolor}%
\pgftext[x=6.216988in,y=1.198462in,,top]{\color{textcolor}\sffamily\fontsize{18.000000}{21.600000}\selectfont $\displaystyle 3$}%
\end{pgfscope}%
\begin{pgfscope}%
\pgfpathrectangle{\pgfqpoint{0.978013in}{1.247073in}}{\pgfqpoint{10.943247in}{6.674186in}}%
\pgfusepath{clip}%
\pgfsetrectcap%
\pgfsetroundjoin%
\pgfsetlinewidth{0.501875pt}%
\definecolor{currentstroke}{rgb}{0.000000,0.000000,0.000000}%
\pgfsetstrokecolor{currentstroke}%
\pgfsetstrokeopacity{0.100000}%
\pgfsetdash{}{0pt}%
\pgfpathmoveto{\pgfqpoint{7.860074in}{1.247073in}}%
\pgfpathlineto{\pgfqpoint{7.860074in}{7.921260in}}%
\pgfusepath{stroke}%
\end{pgfscope}%
\begin{pgfscope}%
\pgfsetbuttcap%
\pgfsetroundjoin%
\definecolor{currentfill}{rgb}{0.000000,0.000000,0.000000}%
\pgfsetfillcolor{currentfill}%
\pgfsetlinewidth{0.501875pt}%
\definecolor{currentstroke}{rgb}{0.000000,0.000000,0.000000}%
\pgfsetstrokecolor{currentstroke}%
\pgfsetdash{}{0pt}%
\pgfsys@defobject{currentmarker}{\pgfqpoint{0.000000in}{0.000000in}}{\pgfqpoint{0.000000in}{0.034722in}}{%
\pgfpathmoveto{\pgfqpoint{0.000000in}{0.000000in}}%
\pgfpathlineto{\pgfqpoint{0.000000in}{0.034722in}}%
\pgfusepath{stroke,fill}%
}%
\begin{pgfscope}%
\pgfsys@transformshift{7.860074in}{1.247073in}%
\pgfsys@useobject{currentmarker}{}%
\end{pgfscope}%
\end{pgfscope}%
\begin{pgfscope}%
\definecolor{textcolor}{rgb}{0.000000,0.000000,0.000000}%
\pgfsetstrokecolor{textcolor}%
\pgfsetfillcolor{textcolor}%
\pgftext[x=7.860074in,y=1.198462in,,top]{\color{textcolor}\sffamily\fontsize{18.000000}{21.600000}\selectfont $\displaystyle 4$}%
\end{pgfscope}%
\begin{pgfscope}%
\pgfpathrectangle{\pgfqpoint{0.978013in}{1.247073in}}{\pgfqpoint{10.943247in}{6.674186in}}%
\pgfusepath{clip}%
\pgfsetrectcap%
\pgfsetroundjoin%
\pgfsetlinewidth{0.501875pt}%
\definecolor{currentstroke}{rgb}{0.000000,0.000000,0.000000}%
\pgfsetstrokecolor{currentstroke}%
\pgfsetstrokeopacity{0.100000}%
\pgfsetdash{}{0pt}%
\pgfpathmoveto{\pgfqpoint{9.503161in}{1.247073in}}%
\pgfpathlineto{\pgfqpoint{9.503161in}{7.921260in}}%
\pgfusepath{stroke}%
\end{pgfscope}%
\begin{pgfscope}%
\pgfsetbuttcap%
\pgfsetroundjoin%
\definecolor{currentfill}{rgb}{0.000000,0.000000,0.000000}%
\pgfsetfillcolor{currentfill}%
\pgfsetlinewidth{0.501875pt}%
\definecolor{currentstroke}{rgb}{0.000000,0.000000,0.000000}%
\pgfsetstrokecolor{currentstroke}%
\pgfsetdash{}{0pt}%
\pgfsys@defobject{currentmarker}{\pgfqpoint{0.000000in}{0.000000in}}{\pgfqpoint{0.000000in}{0.034722in}}{%
\pgfpathmoveto{\pgfqpoint{0.000000in}{0.000000in}}%
\pgfpathlineto{\pgfqpoint{0.000000in}{0.034722in}}%
\pgfusepath{stroke,fill}%
}%
\begin{pgfscope}%
\pgfsys@transformshift{9.503161in}{1.247073in}%
\pgfsys@useobject{currentmarker}{}%
\end{pgfscope}%
\end{pgfscope}%
\begin{pgfscope}%
\definecolor{textcolor}{rgb}{0.000000,0.000000,0.000000}%
\pgfsetstrokecolor{textcolor}%
\pgfsetfillcolor{textcolor}%
\pgftext[x=9.503161in,y=1.198462in,,top]{\color{textcolor}\sffamily\fontsize{18.000000}{21.600000}\selectfont $\displaystyle 5$}%
\end{pgfscope}%
\begin{pgfscope}%
\pgfpathrectangle{\pgfqpoint{0.978013in}{1.247073in}}{\pgfqpoint{10.943247in}{6.674186in}}%
\pgfusepath{clip}%
\pgfsetrectcap%
\pgfsetroundjoin%
\pgfsetlinewidth{0.501875pt}%
\definecolor{currentstroke}{rgb}{0.000000,0.000000,0.000000}%
\pgfsetstrokecolor{currentstroke}%
\pgfsetstrokeopacity{0.100000}%
\pgfsetdash{}{0pt}%
\pgfpathmoveto{\pgfqpoint{11.146247in}{1.247073in}}%
\pgfpathlineto{\pgfqpoint{11.146247in}{7.921260in}}%
\pgfusepath{stroke}%
\end{pgfscope}%
\begin{pgfscope}%
\pgfsetbuttcap%
\pgfsetroundjoin%
\definecolor{currentfill}{rgb}{0.000000,0.000000,0.000000}%
\pgfsetfillcolor{currentfill}%
\pgfsetlinewidth{0.501875pt}%
\definecolor{currentstroke}{rgb}{0.000000,0.000000,0.000000}%
\pgfsetstrokecolor{currentstroke}%
\pgfsetdash{}{0pt}%
\pgfsys@defobject{currentmarker}{\pgfqpoint{0.000000in}{0.000000in}}{\pgfqpoint{0.000000in}{0.034722in}}{%
\pgfpathmoveto{\pgfqpoint{0.000000in}{0.000000in}}%
\pgfpathlineto{\pgfqpoint{0.000000in}{0.034722in}}%
\pgfusepath{stroke,fill}%
}%
\begin{pgfscope}%
\pgfsys@transformshift{11.146247in}{1.247073in}%
\pgfsys@useobject{currentmarker}{}%
\end{pgfscope}%
\end{pgfscope}%
\begin{pgfscope}%
\definecolor{textcolor}{rgb}{0.000000,0.000000,0.000000}%
\pgfsetstrokecolor{textcolor}%
\pgfsetfillcolor{textcolor}%
\pgftext[x=11.146247in,y=1.198462in,,top]{\color{textcolor}\sffamily\fontsize{18.000000}{21.600000}\selectfont $\displaystyle 6$}%
\end{pgfscope}%
\begin{pgfscope}%
\definecolor{textcolor}{rgb}{0.000000,0.000000,0.000000}%
\pgfsetstrokecolor{textcolor}%
\pgfsetfillcolor{textcolor}%
\pgftext[x=6.449637in,y=0.900964in,,top]{\color{textcolor}\sffamily\fontsize{18.000000}{21.600000}\selectfont $\displaystyle x$}%
\end{pgfscope}%
\begin{pgfscope}%
\pgfpathrectangle{\pgfqpoint{0.978013in}{1.247073in}}{\pgfqpoint{10.943247in}{6.674186in}}%
\pgfusepath{clip}%
\pgfsetrectcap%
\pgfsetroundjoin%
\pgfsetlinewidth{0.501875pt}%
\definecolor{currentstroke}{rgb}{0.000000,0.000000,0.000000}%
\pgfsetstrokecolor{currentstroke}%
\pgfsetstrokeopacity{0.100000}%
\pgfsetdash{}{0pt}%
\pgfpathmoveto{\pgfqpoint{0.978013in}{2.493267in}}%
\pgfpathlineto{\pgfqpoint{11.921260in}{2.493267in}}%
\pgfusepath{stroke}%
\end{pgfscope}%
\begin{pgfscope}%
\pgfsetbuttcap%
\pgfsetroundjoin%
\definecolor{currentfill}{rgb}{0.000000,0.000000,0.000000}%
\pgfsetfillcolor{currentfill}%
\pgfsetlinewidth{0.501875pt}%
\definecolor{currentstroke}{rgb}{0.000000,0.000000,0.000000}%
\pgfsetstrokecolor{currentstroke}%
\pgfsetdash{}{0pt}%
\pgfsys@defobject{currentmarker}{\pgfqpoint{0.000000in}{0.000000in}}{\pgfqpoint{0.034722in}{0.000000in}}{%
\pgfpathmoveto{\pgfqpoint{0.000000in}{0.000000in}}%
\pgfpathlineto{\pgfqpoint{0.034722in}{0.000000in}}%
\pgfusepath{stroke,fill}%
}%
\begin{pgfscope}%
\pgfsys@transformshift{0.978013in}{2.493267in}%
\pgfsys@useobject{currentmarker}{}%
\end{pgfscope}%
\end{pgfscope}%
\begin{pgfscope}%
\definecolor{textcolor}{rgb}{0.000000,0.000000,0.000000}%
\pgfsetstrokecolor{textcolor}%
\pgfsetfillcolor{textcolor}%
\pgftext[x=0.643989in, y=2.398296in, left, base]{\color{textcolor}\sffamily\fontsize{18.000000}{21.600000}\selectfont $\displaystyle 1.0$}%
\end{pgfscope}%
\begin{pgfscope}%
\pgfpathrectangle{\pgfqpoint{0.978013in}{1.247073in}}{\pgfqpoint{10.943247in}{6.674186in}}%
\pgfusepath{clip}%
\pgfsetrectcap%
\pgfsetroundjoin%
\pgfsetlinewidth{0.501875pt}%
\definecolor{currentstroke}{rgb}{0.000000,0.000000,0.000000}%
\pgfsetstrokecolor{currentstroke}%
\pgfsetstrokeopacity{0.100000}%
\pgfsetdash{}{0pt}%
\pgfpathmoveto{\pgfqpoint{0.978013in}{3.803042in}}%
\pgfpathlineto{\pgfqpoint{11.921260in}{3.803042in}}%
\pgfusepath{stroke}%
\end{pgfscope}%
\begin{pgfscope}%
\pgfsetbuttcap%
\pgfsetroundjoin%
\definecolor{currentfill}{rgb}{0.000000,0.000000,0.000000}%
\pgfsetfillcolor{currentfill}%
\pgfsetlinewidth{0.501875pt}%
\definecolor{currentstroke}{rgb}{0.000000,0.000000,0.000000}%
\pgfsetstrokecolor{currentstroke}%
\pgfsetdash{}{0pt}%
\pgfsys@defobject{currentmarker}{\pgfqpoint{0.000000in}{0.000000in}}{\pgfqpoint{0.034722in}{0.000000in}}{%
\pgfpathmoveto{\pgfqpoint{0.000000in}{0.000000in}}%
\pgfpathlineto{\pgfqpoint{0.034722in}{0.000000in}}%
\pgfusepath{stroke,fill}%
}%
\begin{pgfscope}%
\pgfsys@transformshift{0.978013in}{3.803042in}%
\pgfsys@useobject{currentmarker}{}%
\end{pgfscope}%
\end{pgfscope}%
\begin{pgfscope}%
\definecolor{textcolor}{rgb}{0.000000,0.000000,0.000000}%
\pgfsetstrokecolor{textcolor}%
\pgfsetfillcolor{textcolor}%
\pgftext[x=0.643989in, y=3.708071in, left, base]{\color{textcolor}\sffamily\fontsize{18.000000}{21.600000}\selectfont $\displaystyle 1.5$}%
\end{pgfscope}%
\begin{pgfscope}%
\pgfpathrectangle{\pgfqpoint{0.978013in}{1.247073in}}{\pgfqpoint{10.943247in}{6.674186in}}%
\pgfusepath{clip}%
\pgfsetrectcap%
\pgfsetroundjoin%
\pgfsetlinewidth{0.501875pt}%
\definecolor{currentstroke}{rgb}{0.000000,0.000000,0.000000}%
\pgfsetstrokecolor{currentstroke}%
\pgfsetstrokeopacity{0.100000}%
\pgfsetdash{}{0pt}%
\pgfpathmoveto{\pgfqpoint{0.978013in}{5.112817in}}%
\pgfpathlineto{\pgfqpoint{11.921260in}{5.112817in}}%
\pgfusepath{stroke}%
\end{pgfscope}%
\begin{pgfscope}%
\pgfsetbuttcap%
\pgfsetroundjoin%
\definecolor{currentfill}{rgb}{0.000000,0.000000,0.000000}%
\pgfsetfillcolor{currentfill}%
\pgfsetlinewidth{0.501875pt}%
\definecolor{currentstroke}{rgb}{0.000000,0.000000,0.000000}%
\pgfsetstrokecolor{currentstroke}%
\pgfsetdash{}{0pt}%
\pgfsys@defobject{currentmarker}{\pgfqpoint{0.000000in}{0.000000in}}{\pgfqpoint{0.034722in}{0.000000in}}{%
\pgfpathmoveto{\pgfqpoint{0.000000in}{0.000000in}}%
\pgfpathlineto{\pgfqpoint{0.034722in}{0.000000in}}%
\pgfusepath{stroke,fill}%
}%
\begin{pgfscope}%
\pgfsys@transformshift{0.978013in}{5.112817in}%
\pgfsys@useobject{currentmarker}{}%
\end{pgfscope}%
\end{pgfscope}%
\begin{pgfscope}%
\definecolor{textcolor}{rgb}{0.000000,0.000000,0.000000}%
\pgfsetstrokecolor{textcolor}%
\pgfsetfillcolor{textcolor}%
\pgftext[x=0.643989in, y=5.017847in, left, base]{\color{textcolor}\sffamily\fontsize{18.000000}{21.600000}\selectfont $\displaystyle 2.0$}%
\end{pgfscope}%
\begin{pgfscope}%
\pgfpathrectangle{\pgfqpoint{0.978013in}{1.247073in}}{\pgfqpoint{10.943247in}{6.674186in}}%
\pgfusepath{clip}%
\pgfsetrectcap%
\pgfsetroundjoin%
\pgfsetlinewidth{0.501875pt}%
\definecolor{currentstroke}{rgb}{0.000000,0.000000,0.000000}%
\pgfsetstrokecolor{currentstroke}%
\pgfsetstrokeopacity{0.100000}%
\pgfsetdash{}{0pt}%
\pgfpathmoveto{\pgfqpoint{0.978013in}{6.422593in}}%
\pgfpathlineto{\pgfqpoint{11.921260in}{6.422593in}}%
\pgfusepath{stroke}%
\end{pgfscope}%
\begin{pgfscope}%
\pgfsetbuttcap%
\pgfsetroundjoin%
\definecolor{currentfill}{rgb}{0.000000,0.000000,0.000000}%
\pgfsetfillcolor{currentfill}%
\pgfsetlinewidth{0.501875pt}%
\definecolor{currentstroke}{rgb}{0.000000,0.000000,0.000000}%
\pgfsetstrokecolor{currentstroke}%
\pgfsetdash{}{0pt}%
\pgfsys@defobject{currentmarker}{\pgfqpoint{0.000000in}{0.000000in}}{\pgfqpoint{0.034722in}{0.000000in}}{%
\pgfpathmoveto{\pgfqpoint{0.000000in}{0.000000in}}%
\pgfpathlineto{\pgfqpoint{0.034722in}{0.000000in}}%
\pgfusepath{stroke,fill}%
}%
\begin{pgfscope}%
\pgfsys@transformshift{0.978013in}{6.422593in}%
\pgfsys@useobject{currentmarker}{}%
\end{pgfscope}%
\end{pgfscope}%
\begin{pgfscope}%
\definecolor{textcolor}{rgb}{0.000000,0.000000,0.000000}%
\pgfsetstrokecolor{textcolor}%
\pgfsetfillcolor{textcolor}%
\pgftext[x=0.643989in, y=6.327622in, left, base]{\color{textcolor}\sffamily\fontsize{18.000000}{21.600000}\selectfont $\displaystyle 2.5$}%
\end{pgfscope}%
\begin{pgfscope}%
\pgfpathrectangle{\pgfqpoint{0.978013in}{1.247073in}}{\pgfqpoint{10.943247in}{6.674186in}}%
\pgfusepath{clip}%
\pgfsetrectcap%
\pgfsetroundjoin%
\pgfsetlinewidth{0.501875pt}%
\definecolor{currentstroke}{rgb}{0.000000,0.000000,0.000000}%
\pgfsetstrokecolor{currentstroke}%
\pgfsetstrokeopacity{0.100000}%
\pgfsetdash{}{0pt}%
\pgfpathmoveto{\pgfqpoint{0.978013in}{7.732368in}}%
\pgfpathlineto{\pgfqpoint{11.921260in}{7.732368in}}%
\pgfusepath{stroke}%
\end{pgfscope}%
\begin{pgfscope}%
\pgfsetbuttcap%
\pgfsetroundjoin%
\definecolor{currentfill}{rgb}{0.000000,0.000000,0.000000}%
\pgfsetfillcolor{currentfill}%
\pgfsetlinewidth{0.501875pt}%
\definecolor{currentstroke}{rgb}{0.000000,0.000000,0.000000}%
\pgfsetstrokecolor{currentstroke}%
\pgfsetdash{}{0pt}%
\pgfsys@defobject{currentmarker}{\pgfqpoint{0.000000in}{0.000000in}}{\pgfqpoint{0.034722in}{0.000000in}}{%
\pgfpathmoveto{\pgfqpoint{0.000000in}{0.000000in}}%
\pgfpathlineto{\pgfqpoint{0.034722in}{0.000000in}}%
\pgfusepath{stroke,fill}%
}%
\begin{pgfscope}%
\pgfsys@transformshift{0.978013in}{7.732368in}%
\pgfsys@useobject{currentmarker}{}%
\end{pgfscope}%
\end{pgfscope}%
\begin{pgfscope}%
\definecolor{textcolor}{rgb}{0.000000,0.000000,0.000000}%
\pgfsetstrokecolor{textcolor}%
\pgfsetfillcolor{textcolor}%
\pgftext[x=0.643989in, y=7.637397in, left, base]{\color{textcolor}\sffamily\fontsize{18.000000}{21.600000}\selectfont $\displaystyle 3.0$}%
\end{pgfscope}%
\begin{pgfscope}%
\pgfpathrectangle{\pgfqpoint{0.978013in}{1.247073in}}{\pgfqpoint{10.943247in}{6.674186in}}%
\pgfusepath{clip}%
\pgfsetbuttcap%
\pgfsetroundjoin%
\pgfsetlinewidth{1.003750pt}%
\definecolor{currentstroke}{rgb}{0.000000,0.605603,0.978680}%
\pgfsetstrokecolor{currentstroke}%
\pgfsetdash{}{0pt}%
\pgfpathmoveto{\pgfqpoint{1.287728in}{7.732368in}}%
\pgfpathlineto{\pgfqpoint{6.046362in}{7.732077in}}%
\pgfpathlineto{\pgfqpoint{6.127017in}{7.731246in}}%
\pgfpathlineto{\pgfqpoint{6.207672in}{7.728439in}}%
\pgfpathlineto{\pgfqpoint{6.288327in}{7.719855in}}%
\pgfpathlineto{\pgfqpoint{6.368982in}{7.696122in}}%
\pgfpathlineto{\pgfqpoint{6.449637in}{7.636870in}}%
\pgfpathlineto{\pgfqpoint{6.530291in}{7.503641in}}%
\pgfpathlineto{\pgfqpoint{6.610946in}{7.234947in}}%
\pgfpathlineto{\pgfqpoint{6.691601in}{6.751867in}}%
\pgfpathlineto{\pgfqpoint{6.772256in}{5.984703in}}%
\pgfpathlineto{\pgfqpoint{6.852911in}{4.923903in}}%
\pgfpathlineto{\pgfqpoint{6.933565in}{3.677089in}}%
\pgfpathlineto{\pgfqpoint{7.014220in}{2.487659in}}%
\pgfpathlineto{\pgfqpoint{7.094875in}{1.666588in}}%
\pgfpathlineto{\pgfqpoint{7.175530in}{1.435966in}}%
\pgfpathlineto{\pgfqpoint{7.256185in}{1.768919in}}%
\pgfpathlineto{\pgfqpoint{7.336840in}{2.362261in}}%
\pgfpathlineto{\pgfqpoint{7.417494in}{2.810848in}}%
\pgfpathlineto{\pgfqpoint{7.498149in}{2.880235in}}%
\pgfpathlineto{\pgfqpoint{7.578804in}{2.651841in}}%
\pgfpathlineto{\pgfqpoint{7.659459in}{2.403309in}}%
\pgfpathlineto{\pgfqpoint{7.740114in}{2.338791in}}%
\pgfpathlineto{\pgfqpoint{7.820769in}{2.435517in}}%
\pgfpathlineto{\pgfqpoint{7.901423in}{2.536540in}}%
\pgfpathlineto{\pgfqpoint{7.982078in}{2.545774in}}%
\pgfpathlineto{\pgfqpoint{8.062733in}{2.498753in}}%
\pgfpathlineto{\pgfqpoint{8.143388in}{2.471619in}}%
\pgfpathlineto{\pgfqpoint{8.224043in}{2.482522in}}%
\pgfpathlineto{\pgfqpoint{8.304697in}{2.498916in}}%
\pgfpathlineto{\pgfqpoint{8.385352in}{2.499267in}}%
\pgfpathlineto{\pgfqpoint{8.466007in}{2.492432in}}%
\pgfpathlineto{\pgfqpoint{8.546662in}{2.490824in}}%
\pgfpathlineto{\pgfqpoint{8.627317in}{2.493226in}}%
\pgfpathlineto{\pgfqpoint{8.707972in}{2.494127in}}%
\pgfpathlineto{\pgfqpoint{8.869281in}{2.492985in}}%
\pgfpathlineto{\pgfqpoint{9.272555in}{2.493272in}}%
\pgfpathlineto{\pgfqpoint{11.611545in}{2.493267in}}%
\pgfpathlineto{\pgfqpoint{11.611545in}{2.493267in}}%
\pgfusepath{stroke}%
\end{pgfscope}%
\begin{pgfscope}%
\pgfpathrectangle{\pgfqpoint{0.978013in}{1.247073in}}{\pgfqpoint{10.943247in}{6.674186in}}%
\pgfusepath{clip}%
\pgfsetbuttcap%
\pgfsetroundjoin%
\pgfsetlinewidth{1.003750pt}%
\definecolor{currentstroke}{rgb}{0.888874,0.435649,0.278123}%
\pgfsetstrokecolor{currentstroke}%
\pgfsetdash{}{0pt}%
\pgfpathmoveto{\pgfqpoint{1.287728in}{7.732368in}}%
\pgfpathlineto{\pgfqpoint{6.772256in}{7.732368in}}%
\pgfpathlineto{\pgfqpoint{6.852911in}{2.493267in}}%
\pgfpathlineto{\pgfqpoint{11.611545in}{2.493267in}}%
\pgfpathlineto{\pgfqpoint{11.611545in}{2.493267in}}%
\pgfusepath{stroke}%
\end{pgfscope}%
\begin{pgfscope}%
\pgfsetrectcap%
\pgfsetmiterjoin%
\pgfsetlinewidth{1.003750pt}%
\definecolor{currentstroke}{rgb}{0.000000,0.000000,0.000000}%
\pgfsetstrokecolor{currentstroke}%
\pgfsetdash{}{0pt}%
\pgfpathmoveto{\pgfqpoint{0.978013in}{1.247073in}}%
\pgfpathlineto{\pgfqpoint{0.978013in}{7.921260in}}%
\pgfusepath{stroke}%
\end{pgfscope}%
\begin{pgfscope}%
\pgfsetrectcap%
\pgfsetmiterjoin%
\pgfsetlinewidth{1.003750pt}%
\definecolor{currentstroke}{rgb}{0.000000,0.000000,0.000000}%
\pgfsetstrokecolor{currentstroke}%
\pgfsetdash{}{0pt}%
\pgfpathmoveto{\pgfqpoint{0.978013in}{1.247073in}}%
\pgfpathlineto{\pgfqpoint{11.921260in}{1.247073in}}%
\pgfusepath{stroke}%
\end{pgfscope}%
\begin{pgfscope}%
\pgfsetbuttcap%
\pgfsetmiterjoin%
\definecolor{currentfill}{rgb}{1.000000,1.000000,1.000000}%
\pgfsetfillcolor{currentfill}%
\pgfsetlinewidth{1.003750pt}%
\definecolor{currentstroke}{rgb}{0.000000,0.000000,0.000000}%
\pgfsetstrokecolor{currentstroke}%
\pgfsetdash{}{0pt}%
\pgfpathmoveto{\pgfqpoint{10.511589in}{6.787373in}}%
\pgfpathlineto{\pgfqpoint{11.796260in}{6.787373in}}%
\pgfpathlineto{\pgfqpoint{11.796260in}{7.796260in}}%
\pgfpathlineto{\pgfqpoint{10.511589in}{7.796260in}}%
\pgfpathclose%
\pgfusepath{stroke,fill}%
\end{pgfscope}%
\begin{pgfscope}%
\pgfsetbuttcap%
\pgfsetmiterjoin%
\pgfsetlinewidth{2.258437pt}%
\definecolor{currentstroke}{rgb}{0.000000,0.605603,0.978680}%
\pgfsetstrokecolor{currentstroke}%
\pgfsetdash{}{0pt}%
\pgfpathmoveto{\pgfqpoint{10.711589in}{7.493818in}}%
\pgfpathlineto{\pgfqpoint{11.211589in}{7.493818in}}%
\pgfusepath{stroke}%
\end{pgfscope}%
\begin{pgfscope}%
\definecolor{textcolor}{rgb}{0.000000,0.000000,0.000000}%
\pgfsetstrokecolor{textcolor}%
\pgfsetfillcolor{textcolor}%
\pgftext[x=11.411589in,y=7.406318in,left,base]{\color{textcolor}\sffamily\fontsize{18.000000}{21.600000}\selectfont $\displaystyle U$}%
\end{pgfscope}%
\begin{pgfscope}%
\pgfsetbuttcap%
\pgfsetmiterjoin%
\pgfsetlinewidth{2.258437pt}%
\definecolor{currentstroke}{rgb}{0.888874,0.435649,0.278123}%
\pgfsetstrokecolor{currentstroke}%
\pgfsetdash{}{0pt}%
\pgfpathmoveto{\pgfqpoint{10.711589in}{7.126875in}}%
\pgfpathlineto{\pgfqpoint{11.211589in}{7.126875in}}%
\pgfusepath{stroke}%
\end{pgfscope}%
\begin{pgfscope}%
\definecolor{textcolor}{rgb}{0.000000,0.000000,0.000000}%
\pgfsetstrokecolor{textcolor}%
\pgfsetfillcolor{textcolor}%
\pgftext[x=11.411589in,y=7.039375in,left,base]{\color{textcolor}\sffamily\fontsize{18.000000}{21.600000}\selectfont $\displaystyle u$}%
\end{pgfscope}%
\end{pgfpicture}%
\makeatother%
\endgroup%
}\quad
	\resizebox{0.4\linewidth}{!}{%% Creator: Matplotlib, PGF backend
%%
%% To include the figure in your LaTeX document, write
%%   \input{<filename>.pgf}
%%
%% Make sure the required packages are loaded in your preamble
%%   \usepackage{pgf}
%%
%% Figures using additional raster images can only be included by \input if
%% they are in the same directory as the main LaTeX file. For loading figures
%% from other directories you can use the `import` package
%%   \usepackage{import}
%%
%% and then include the figures with
%%   \import{<path to file>}{<filename>.pgf}
%%
%% Matplotlib used the following preamble
%%   \usepackage{fontspec}
%%   \setmainfont{DejaVuSerif.ttf}[Path=\detokenize{/Users/quejiahao/.julia/conda/3/lib/python3.9/site-packages/matplotlib/mpl-data/fonts/ttf/}]
%%   \setsansfont{DejaVuSans.ttf}[Path=\detokenize{/Users/quejiahao/.julia/conda/3/lib/python3.9/site-packages/matplotlib/mpl-data/fonts/ttf/}]
%%   \setmonofont{DejaVuSansMono.ttf}[Path=\detokenize{/Users/quejiahao/.julia/conda/3/lib/python3.9/site-packages/matplotlib/mpl-data/fonts/ttf/}]
%%
\begingroup%
\makeatletter%
\begin{pgfpicture}%
\pgfpathrectangle{\pgfpointorigin}{\pgfqpoint{12.000000in}{8.000000in}}%
\pgfusepath{use as bounding box, clip}%
\begin{pgfscope}%
\pgfsetbuttcap%
\pgfsetmiterjoin%
\definecolor{currentfill}{rgb}{1.000000,1.000000,1.000000}%
\pgfsetfillcolor{currentfill}%
\pgfsetlinewidth{0.000000pt}%
\definecolor{currentstroke}{rgb}{1.000000,1.000000,1.000000}%
\pgfsetstrokecolor{currentstroke}%
\pgfsetdash{}{0pt}%
\pgfpathmoveto{\pgfqpoint{0.000000in}{0.000000in}}%
\pgfpathlineto{\pgfqpoint{12.000000in}{0.000000in}}%
\pgfpathlineto{\pgfqpoint{12.000000in}{8.000000in}}%
\pgfpathlineto{\pgfqpoint{0.000000in}{8.000000in}}%
\pgfpathclose%
\pgfusepath{fill}%
\end{pgfscope}%
\begin{pgfscope}%
\pgfsetbuttcap%
\pgfsetmiterjoin%
\definecolor{currentfill}{rgb}{1.000000,1.000000,1.000000}%
\pgfsetfillcolor{currentfill}%
\pgfsetlinewidth{0.000000pt}%
\definecolor{currentstroke}{rgb}{0.000000,0.000000,0.000000}%
\pgfsetstrokecolor{currentstroke}%
\pgfsetstrokeopacity{0.000000}%
\pgfsetdash{}{0pt}%
\pgfpathmoveto{\pgfqpoint{0.978013in}{1.247073in}}%
\pgfpathlineto{\pgfqpoint{11.921260in}{1.247073in}}%
\pgfpathlineto{\pgfqpoint{11.921260in}{7.921260in}}%
\pgfpathlineto{\pgfqpoint{0.978013in}{7.921260in}}%
\pgfpathclose%
\pgfusepath{fill}%
\end{pgfscope}%
\begin{pgfscope}%
\pgfpathrectangle{\pgfqpoint{0.978013in}{1.247073in}}{\pgfqpoint{10.943247in}{6.674186in}}%
\pgfusepath{clip}%
\pgfsetrectcap%
\pgfsetroundjoin%
\pgfsetlinewidth{0.501875pt}%
\definecolor{currentstroke}{rgb}{0.000000,0.000000,0.000000}%
\pgfsetstrokecolor{currentstroke}%
\pgfsetstrokeopacity{0.100000}%
\pgfsetdash{}{0pt}%
\pgfpathmoveto{\pgfqpoint{1.287728in}{1.247073in}}%
\pgfpathlineto{\pgfqpoint{1.287728in}{7.921260in}}%
\pgfusepath{stroke}%
\end{pgfscope}%
\begin{pgfscope}%
\pgfsetbuttcap%
\pgfsetroundjoin%
\definecolor{currentfill}{rgb}{0.000000,0.000000,0.000000}%
\pgfsetfillcolor{currentfill}%
\pgfsetlinewidth{0.501875pt}%
\definecolor{currentstroke}{rgb}{0.000000,0.000000,0.000000}%
\pgfsetstrokecolor{currentstroke}%
\pgfsetdash{}{0pt}%
\pgfsys@defobject{currentmarker}{\pgfqpoint{0.000000in}{0.000000in}}{\pgfqpoint{0.000000in}{0.034722in}}{%
\pgfpathmoveto{\pgfqpoint{0.000000in}{0.000000in}}%
\pgfpathlineto{\pgfqpoint{0.000000in}{0.034722in}}%
\pgfusepath{stroke,fill}%
}%
\begin{pgfscope}%
\pgfsys@transformshift{1.287728in}{1.247073in}%
\pgfsys@useobject{currentmarker}{}%
\end{pgfscope}%
\end{pgfscope}%
\begin{pgfscope}%
\definecolor{textcolor}{rgb}{0.000000,0.000000,0.000000}%
\pgfsetstrokecolor{textcolor}%
\pgfsetfillcolor{textcolor}%
\pgftext[x=1.287728in,y=1.198462in,,top]{\color{textcolor}\sffamily\fontsize{18.000000}{21.600000}\selectfont $\displaystyle 0$}%
\end{pgfscope}%
\begin{pgfscope}%
\pgfpathrectangle{\pgfqpoint{0.978013in}{1.247073in}}{\pgfqpoint{10.943247in}{6.674186in}}%
\pgfusepath{clip}%
\pgfsetrectcap%
\pgfsetroundjoin%
\pgfsetlinewidth{0.501875pt}%
\definecolor{currentstroke}{rgb}{0.000000,0.000000,0.000000}%
\pgfsetstrokecolor{currentstroke}%
\pgfsetstrokeopacity{0.100000}%
\pgfsetdash{}{0pt}%
\pgfpathmoveto{\pgfqpoint{2.930814in}{1.247073in}}%
\pgfpathlineto{\pgfqpoint{2.930814in}{7.921260in}}%
\pgfusepath{stroke}%
\end{pgfscope}%
\begin{pgfscope}%
\pgfsetbuttcap%
\pgfsetroundjoin%
\definecolor{currentfill}{rgb}{0.000000,0.000000,0.000000}%
\pgfsetfillcolor{currentfill}%
\pgfsetlinewidth{0.501875pt}%
\definecolor{currentstroke}{rgb}{0.000000,0.000000,0.000000}%
\pgfsetstrokecolor{currentstroke}%
\pgfsetdash{}{0pt}%
\pgfsys@defobject{currentmarker}{\pgfqpoint{0.000000in}{0.000000in}}{\pgfqpoint{0.000000in}{0.034722in}}{%
\pgfpathmoveto{\pgfqpoint{0.000000in}{0.000000in}}%
\pgfpathlineto{\pgfqpoint{0.000000in}{0.034722in}}%
\pgfusepath{stroke,fill}%
}%
\begin{pgfscope}%
\pgfsys@transformshift{2.930814in}{1.247073in}%
\pgfsys@useobject{currentmarker}{}%
\end{pgfscope}%
\end{pgfscope}%
\begin{pgfscope}%
\definecolor{textcolor}{rgb}{0.000000,0.000000,0.000000}%
\pgfsetstrokecolor{textcolor}%
\pgfsetfillcolor{textcolor}%
\pgftext[x=2.930814in,y=1.198462in,,top]{\color{textcolor}\sffamily\fontsize{18.000000}{21.600000}\selectfont $\displaystyle 1$}%
\end{pgfscope}%
\begin{pgfscope}%
\pgfpathrectangle{\pgfqpoint{0.978013in}{1.247073in}}{\pgfqpoint{10.943247in}{6.674186in}}%
\pgfusepath{clip}%
\pgfsetrectcap%
\pgfsetroundjoin%
\pgfsetlinewidth{0.501875pt}%
\definecolor{currentstroke}{rgb}{0.000000,0.000000,0.000000}%
\pgfsetstrokecolor{currentstroke}%
\pgfsetstrokeopacity{0.100000}%
\pgfsetdash{}{0pt}%
\pgfpathmoveto{\pgfqpoint{4.573901in}{1.247073in}}%
\pgfpathlineto{\pgfqpoint{4.573901in}{7.921260in}}%
\pgfusepath{stroke}%
\end{pgfscope}%
\begin{pgfscope}%
\pgfsetbuttcap%
\pgfsetroundjoin%
\definecolor{currentfill}{rgb}{0.000000,0.000000,0.000000}%
\pgfsetfillcolor{currentfill}%
\pgfsetlinewidth{0.501875pt}%
\definecolor{currentstroke}{rgb}{0.000000,0.000000,0.000000}%
\pgfsetstrokecolor{currentstroke}%
\pgfsetdash{}{0pt}%
\pgfsys@defobject{currentmarker}{\pgfqpoint{0.000000in}{0.000000in}}{\pgfqpoint{0.000000in}{0.034722in}}{%
\pgfpathmoveto{\pgfqpoint{0.000000in}{0.000000in}}%
\pgfpathlineto{\pgfqpoint{0.000000in}{0.034722in}}%
\pgfusepath{stroke,fill}%
}%
\begin{pgfscope}%
\pgfsys@transformshift{4.573901in}{1.247073in}%
\pgfsys@useobject{currentmarker}{}%
\end{pgfscope}%
\end{pgfscope}%
\begin{pgfscope}%
\definecolor{textcolor}{rgb}{0.000000,0.000000,0.000000}%
\pgfsetstrokecolor{textcolor}%
\pgfsetfillcolor{textcolor}%
\pgftext[x=4.573901in,y=1.198462in,,top]{\color{textcolor}\sffamily\fontsize{18.000000}{21.600000}\selectfont $\displaystyle 2$}%
\end{pgfscope}%
\begin{pgfscope}%
\pgfpathrectangle{\pgfqpoint{0.978013in}{1.247073in}}{\pgfqpoint{10.943247in}{6.674186in}}%
\pgfusepath{clip}%
\pgfsetrectcap%
\pgfsetroundjoin%
\pgfsetlinewidth{0.501875pt}%
\definecolor{currentstroke}{rgb}{0.000000,0.000000,0.000000}%
\pgfsetstrokecolor{currentstroke}%
\pgfsetstrokeopacity{0.100000}%
\pgfsetdash{}{0pt}%
\pgfpathmoveto{\pgfqpoint{6.216988in}{1.247073in}}%
\pgfpathlineto{\pgfqpoint{6.216988in}{7.921260in}}%
\pgfusepath{stroke}%
\end{pgfscope}%
\begin{pgfscope}%
\pgfsetbuttcap%
\pgfsetroundjoin%
\definecolor{currentfill}{rgb}{0.000000,0.000000,0.000000}%
\pgfsetfillcolor{currentfill}%
\pgfsetlinewidth{0.501875pt}%
\definecolor{currentstroke}{rgb}{0.000000,0.000000,0.000000}%
\pgfsetstrokecolor{currentstroke}%
\pgfsetdash{}{0pt}%
\pgfsys@defobject{currentmarker}{\pgfqpoint{0.000000in}{0.000000in}}{\pgfqpoint{0.000000in}{0.034722in}}{%
\pgfpathmoveto{\pgfqpoint{0.000000in}{0.000000in}}%
\pgfpathlineto{\pgfqpoint{0.000000in}{0.034722in}}%
\pgfusepath{stroke,fill}%
}%
\begin{pgfscope}%
\pgfsys@transformshift{6.216988in}{1.247073in}%
\pgfsys@useobject{currentmarker}{}%
\end{pgfscope}%
\end{pgfscope}%
\begin{pgfscope}%
\definecolor{textcolor}{rgb}{0.000000,0.000000,0.000000}%
\pgfsetstrokecolor{textcolor}%
\pgfsetfillcolor{textcolor}%
\pgftext[x=6.216988in,y=1.198462in,,top]{\color{textcolor}\sffamily\fontsize{18.000000}{21.600000}\selectfont $\displaystyle 3$}%
\end{pgfscope}%
\begin{pgfscope}%
\pgfpathrectangle{\pgfqpoint{0.978013in}{1.247073in}}{\pgfqpoint{10.943247in}{6.674186in}}%
\pgfusepath{clip}%
\pgfsetrectcap%
\pgfsetroundjoin%
\pgfsetlinewidth{0.501875pt}%
\definecolor{currentstroke}{rgb}{0.000000,0.000000,0.000000}%
\pgfsetstrokecolor{currentstroke}%
\pgfsetstrokeopacity{0.100000}%
\pgfsetdash{}{0pt}%
\pgfpathmoveto{\pgfqpoint{7.860074in}{1.247073in}}%
\pgfpathlineto{\pgfqpoint{7.860074in}{7.921260in}}%
\pgfusepath{stroke}%
\end{pgfscope}%
\begin{pgfscope}%
\pgfsetbuttcap%
\pgfsetroundjoin%
\definecolor{currentfill}{rgb}{0.000000,0.000000,0.000000}%
\pgfsetfillcolor{currentfill}%
\pgfsetlinewidth{0.501875pt}%
\definecolor{currentstroke}{rgb}{0.000000,0.000000,0.000000}%
\pgfsetstrokecolor{currentstroke}%
\pgfsetdash{}{0pt}%
\pgfsys@defobject{currentmarker}{\pgfqpoint{0.000000in}{0.000000in}}{\pgfqpoint{0.000000in}{0.034722in}}{%
\pgfpathmoveto{\pgfqpoint{0.000000in}{0.000000in}}%
\pgfpathlineto{\pgfqpoint{0.000000in}{0.034722in}}%
\pgfusepath{stroke,fill}%
}%
\begin{pgfscope}%
\pgfsys@transformshift{7.860074in}{1.247073in}%
\pgfsys@useobject{currentmarker}{}%
\end{pgfscope}%
\end{pgfscope}%
\begin{pgfscope}%
\definecolor{textcolor}{rgb}{0.000000,0.000000,0.000000}%
\pgfsetstrokecolor{textcolor}%
\pgfsetfillcolor{textcolor}%
\pgftext[x=7.860074in,y=1.198462in,,top]{\color{textcolor}\sffamily\fontsize{18.000000}{21.600000}\selectfont $\displaystyle 4$}%
\end{pgfscope}%
\begin{pgfscope}%
\pgfpathrectangle{\pgfqpoint{0.978013in}{1.247073in}}{\pgfqpoint{10.943247in}{6.674186in}}%
\pgfusepath{clip}%
\pgfsetrectcap%
\pgfsetroundjoin%
\pgfsetlinewidth{0.501875pt}%
\definecolor{currentstroke}{rgb}{0.000000,0.000000,0.000000}%
\pgfsetstrokecolor{currentstroke}%
\pgfsetstrokeopacity{0.100000}%
\pgfsetdash{}{0pt}%
\pgfpathmoveto{\pgfqpoint{9.503161in}{1.247073in}}%
\pgfpathlineto{\pgfqpoint{9.503161in}{7.921260in}}%
\pgfusepath{stroke}%
\end{pgfscope}%
\begin{pgfscope}%
\pgfsetbuttcap%
\pgfsetroundjoin%
\definecolor{currentfill}{rgb}{0.000000,0.000000,0.000000}%
\pgfsetfillcolor{currentfill}%
\pgfsetlinewidth{0.501875pt}%
\definecolor{currentstroke}{rgb}{0.000000,0.000000,0.000000}%
\pgfsetstrokecolor{currentstroke}%
\pgfsetdash{}{0pt}%
\pgfsys@defobject{currentmarker}{\pgfqpoint{0.000000in}{0.000000in}}{\pgfqpoint{0.000000in}{0.034722in}}{%
\pgfpathmoveto{\pgfqpoint{0.000000in}{0.000000in}}%
\pgfpathlineto{\pgfqpoint{0.000000in}{0.034722in}}%
\pgfusepath{stroke,fill}%
}%
\begin{pgfscope}%
\pgfsys@transformshift{9.503161in}{1.247073in}%
\pgfsys@useobject{currentmarker}{}%
\end{pgfscope}%
\end{pgfscope}%
\begin{pgfscope}%
\definecolor{textcolor}{rgb}{0.000000,0.000000,0.000000}%
\pgfsetstrokecolor{textcolor}%
\pgfsetfillcolor{textcolor}%
\pgftext[x=9.503161in,y=1.198462in,,top]{\color{textcolor}\sffamily\fontsize{18.000000}{21.600000}\selectfont $\displaystyle 5$}%
\end{pgfscope}%
\begin{pgfscope}%
\pgfpathrectangle{\pgfqpoint{0.978013in}{1.247073in}}{\pgfqpoint{10.943247in}{6.674186in}}%
\pgfusepath{clip}%
\pgfsetrectcap%
\pgfsetroundjoin%
\pgfsetlinewidth{0.501875pt}%
\definecolor{currentstroke}{rgb}{0.000000,0.000000,0.000000}%
\pgfsetstrokecolor{currentstroke}%
\pgfsetstrokeopacity{0.100000}%
\pgfsetdash{}{0pt}%
\pgfpathmoveto{\pgfqpoint{11.146247in}{1.247073in}}%
\pgfpathlineto{\pgfqpoint{11.146247in}{7.921260in}}%
\pgfusepath{stroke}%
\end{pgfscope}%
\begin{pgfscope}%
\pgfsetbuttcap%
\pgfsetroundjoin%
\definecolor{currentfill}{rgb}{0.000000,0.000000,0.000000}%
\pgfsetfillcolor{currentfill}%
\pgfsetlinewidth{0.501875pt}%
\definecolor{currentstroke}{rgb}{0.000000,0.000000,0.000000}%
\pgfsetstrokecolor{currentstroke}%
\pgfsetdash{}{0pt}%
\pgfsys@defobject{currentmarker}{\pgfqpoint{0.000000in}{0.000000in}}{\pgfqpoint{0.000000in}{0.034722in}}{%
\pgfpathmoveto{\pgfqpoint{0.000000in}{0.000000in}}%
\pgfpathlineto{\pgfqpoint{0.000000in}{0.034722in}}%
\pgfusepath{stroke,fill}%
}%
\begin{pgfscope}%
\pgfsys@transformshift{11.146247in}{1.247073in}%
\pgfsys@useobject{currentmarker}{}%
\end{pgfscope}%
\end{pgfscope}%
\begin{pgfscope}%
\definecolor{textcolor}{rgb}{0.000000,0.000000,0.000000}%
\pgfsetstrokecolor{textcolor}%
\pgfsetfillcolor{textcolor}%
\pgftext[x=11.146247in,y=1.198462in,,top]{\color{textcolor}\sffamily\fontsize{18.000000}{21.600000}\selectfont $\displaystyle 6$}%
\end{pgfscope}%
\begin{pgfscope}%
\definecolor{textcolor}{rgb}{0.000000,0.000000,0.000000}%
\pgfsetstrokecolor{textcolor}%
\pgfsetfillcolor{textcolor}%
\pgftext[x=6.449637in,y=0.900964in,,top]{\color{textcolor}\sffamily\fontsize{18.000000}{21.600000}\selectfont $\displaystyle x$}%
\end{pgfscope}%
\begin{pgfscope}%
\pgfpathrectangle{\pgfqpoint{0.978013in}{1.247073in}}{\pgfqpoint{10.943247in}{6.674186in}}%
\pgfusepath{clip}%
\pgfsetrectcap%
\pgfsetroundjoin%
\pgfsetlinewidth{0.501875pt}%
\definecolor{currentstroke}{rgb}{0.000000,0.000000,0.000000}%
\pgfsetstrokecolor{currentstroke}%
\pgfsetstrokeopacity{0.100000}%
\pgfsetdash{}{0pt}%
\pgfpathmoveto{\pgfqpoint{0.978013in}{1.377919in}}%
\pgfpathlineto{\pgfqpoint{11.921260in}{1.377919in}}%
\pgfusepath{stroke}%
\end{pgfscope}%
\begin{pgfscope}%
\pgfsetbuttcap%
\pgfsetroundjoin%
\definecolor{currentfill}{rgb}{0.000000,0.000000,0.000000}%
\pgfsetfillcolor{currentfill}%
\pgfsetlinewidth{0.501875pt}%
\definecolor{currentstroke}{rgb}{0.000000,0.000000,0.000000}%
\pgfsetstrokecolor{currentstroke}%
\pgfsetdash{}{0pt}%
\pgfsys@defobject{currentmarker}{\pgfqpoint{0.000000in}{0.000000in}}{\pgfqpoint{0.034722in}{0.000000in}}{%
\pgfpathmoveto{\pgfqpoint{0.000000in}{0.000000in}}%
\pgfpathlineto{\pgfqpoint{0.034722in}{0.000000in}}%
\pgfusepath{stroke,fill}%
}%
\begin{pgfscope}%
\pgfsys@transformshift{0.978013in}{1.377919in}%
\pgfsys@useobject{currentmarker}{}%
\end{pgfscope}%
\end{pgfscope}%
\begin{pgfscope}%
\definecolor{textcolor}{rgb}{0.000000,0.000000,0.000000}%
\pgfsetstrokecolor{textcolor}%
\pgfsetfillcolor{textcolor}%
\pgftext[x=0.643989in, y=1.282948in, left, base]{\color{textcolor}\sffamily\fontsize{18.000000}{21.600000}\selectfont $\displaystyle 0.5$}%
\end{pgfscope}%
\begin{pgfscope}%
\pgfpathrectangle{\pgfqpoint{0.978013in}{1.247073in}}{\pgfqpoint{10.943247in}{6.674186in}}%
\pgfusepath{clip}%
\pgfsetrectcap%
\pgfsetroundjoin%
\pgfsetlinewidth{0.501875pt}%
\definecolor{currentstroke}{rgb}{0.000000,0.000000,0.000000}%
\pgfsetstrokecolor{currentstroke}%
\pgfsetstrokeopacity{0.100000}%
\pgfsetdash{}{0pt}%
\pgfpathmoveto{\pgfqpoint{0.978013in}{2.648809in}}%
\pgfpathlineto{\pgfqpoint{11.921260in}{2.648809in}}%
\pgfusepath{stroke}%
\end{pgfscope}%
\begin{pgfscope}%
\pgfsetbuttcap%
\pgfsetroundjoin%
\definecolor{currentfill}{rgb}{0.000000,0.000000,0.000000}%
\pgfsetfillcolor{currentfill}%
\pgfsetlinewidth{0.501875pt}%
\definecolor{currentstroke}{rgb}{0.000000,0.000000,0.000000}%
\pgfsetstrokecolor{currentstroke}%
\pgfsetdash{}{0pt}%
\pgfsys@defobject{currentmarker}{\pgfqpoint{0.000000in}{0.000000in}}{\pgfqpoint{0.034722in}{0.000000in}}{%
\pgfpathmoveto{\pgfqpoint{0.000000in}{0.000000in}}%
\pgfpathlineto{\pgfqpoint{0.034722in}{0.000000in}}%
\pgfusepath{stroke,fill}%
}%
\begin{pgfscope}%
\pgfsys@transformshift{0.978013in}{2.648809in}%
\pgfsys@useobject{currentmarker}{}%
\end{pgfscope}%
\end{pgfscope}%
\begin{pgfscope}%
\definecolor{textcolor}{rgb}{0.000000,0.000000,0.000000}%
\pgfsetstrokecolor{textcolor}%
\pgfsetfillcolor{textcolor}%
\pgftext[x=0.643989in, y=2.553838in, left, base]{\color{textcolor}\sffamily\fontsize{18.000000}{21.600000}\selectfont $\displaystyle 1.0$}%
\end{pgfscope}%
\begin{pgfscope}%
\pgfpathrectangle{\pgfqpoint{0.978013in}{1.247073in}}{\pgfqpoint{10.943247in}{6.674186in}}%
\pgfusepath{clip}%
\pgfsetrectcap%
\pgfsetroundjoin%
\pgfsetlinewidth{0.501875pt}%
\definecolor{currentstroke}{rgb}{0.000000,0.000000,0.000000}%
\pgfsetstrokecolor{currentstroke}%
\pgfsetstrokeopacity{0.100000}%
\pgfsetdash{}{0pt}%
\pgfpathmoveto{\pgfqpoint{0.978013in}{3.919699in}}%
\pgfpathlineto{\pgfqpoint{11.921260in}{3.919699in}}%
\pgfusepath{stroke}%
\end{pgfscope}%
\begin{pgfscope}%
\pgfsetbuttcap%
\pgfsetroundjoin%
\definecolor{currentfill}{rgb}{0.000000,0.000000,0.000000}%
\pgfsetfillcolor{currentfill}%
\pgfsetlinewidth{0.501875pt}%
\definecolor{currentstroke}{rgb}{0.000000,0.000000,0.000000}%
\pgfsetstrokecolor{currentstroke}%
\pgfsetdash{}{0pt}%
\pgfsys@defobject{currentmarker}{\pgfqpoint{0.000000in}{0.000000in}}{\pgfqpoint{0.034722in}{0.000000in}}{%
\pgfpathmoveto{\pgfqpoint{0.000000in}{0.000000in}}%
\pgfpathlineto{\pgfqpoint{0.034722in}{0.000000in}}%
\pgfusepath{stroke,fill}%
}%
\begin{pgfscope}%
\pgfsys@transformshift{0.978013in}{3.919699in}%
\pgfsys@useobject{currentmarker}{}%
\end{pgfscope}%
\end{pgfscope}%
\begin{pgfscope}%
\definecolor{textcolor}{rgb}{0.000000,0.000000,0.000000}%
\pgfsetstrokecolor{textcolor}%
\pgfsetfillcolor{textcolor}%
\pgftext[x=0.643989in, y=3.824728in, left, base]{\color{textcolor}\sffamily\fontsize{18.000000}{21.600000}\selectfont $\displaystyle 1.5$}%
\end{pgfscope}%
\begin{pgfscope}%
\pgfpathrectangle{\pgfqpoint{0.978013in}{1.247073in}}{\pgfqpoint{10.943247in}{6.674186in}}%
\pgfusepath{clip}%
\pgfsetrectcap%
\pgfsetroundjoin%
\pgfsetlinewidth{0.501875pt}%
\definecolor{currentstroke}{rgb}{0.000000,0.000000,0.000000}%
\pgfsetstrokecolor{currentstroke}%
\pgfsetstrokeopacity{0.100000}%
\pgfsetdash{}{0pt}%
\pgfpathmoveto{\pgfqpoint{0.978013in}{5.190588in}}%
\pgfpathlineto{\pgfqpoint{11.921260in}{5.190588in}}%
\pgfusepath{stroke}%
\end{pgfscope}%
\begin{pgfscope}%
\pgfsetbuttcap%
\pgfsetroundjoin%
\definecolor{currentfill}{rgb}{0.000000,0.000000,0.000000}%
\pgfsetfillcolor{currentfill}%
\pgfsetlinewidth{0.501875pt}%
\definecolor{currentstroke}{rgb}{0.000000,0.000000,0.000000}%
\pgfsetstrokecolor{currentstroke}%
\pgfsetdash{}{0pt}%
\pgfsys@defobject{currentmarker}{\pgfqpoint{0.000000in}{0.000000in}}{\pgfqpoint{0.034722in}{0.000000in}}{%
\pgfpathmoveto{\pgfqpoint{0.000000in}{0.000000in}}%
\pgfpathlineto{\pgfqpoint{0.034722in}{0.000000in}}%
\pgfusepath{stroke,fill}%
}%
\begin{pgfscope}%
\pgfsys@transformshift{0.978013in}{5.190588in}%
\pgfsys@useobject{currentmarker}{}%
\end{pgfscope}%
\end{pgfscope}%
\begin{pgfscope}%
\definecolor{textcolor}{rgb}{0.000000,0.000000,0.000000}%
\pgfsetstrokecolor{textcolor}%
\pgfsetfillcolor{textcolor}%
\pgftext[x=0.643989in, y=5.095618in, left, base]{\color{textcolor}\sffamily\fontsize{18.000000}{21.600000}\selectfont $\displaystyle 2.0$}%
\end{pgfscope}%
\begin{pgfscope}%
\pgfpathrectangle{\pgfqpoint{0.978013in}{1.247073in}}{\pgfqpoint{10.943247in}{6.674186in}}%
\pgfusepath{clip}%
\pgfsetrectcap%
\pgfsetroundjoin%
\pgfsetlinewidth{0.501875pt}%
\definecolor{currentstroke}{rgb}{0.000000,0.000000,0.000000}%
\pgfsetstrokecolor{currentstroke}%
\pgfsetstrokeopacity{0.100000}%
\pgfsetdash{}{0pt}%
\pgfpathmoveto{\pgfqpoint{0.978013in}{6.461478in}}%
\pgfpathlineto{\pgfqpoint{11.921260in}{6.461478in}}%
\pgfusepath{stroke}%
\end{pgfscope}%
\begin{pgfscope}%
\pgfsetbuttcap%
\pgfsetroundjoin%
\definecolor{currentfill}{rgb}{0.000000,0.000000,0.000000}%
\pgfsetfillcolor{currentfill}%
\pgfsetlinewidth{0.501875pt}%
\definecolor{currentstroke}{rgb}{0.000000,0.000000,0.000000}%
\pgfsetstrokecolor{currentstroke}%
\pgfsetdash{}{0pt}%
\pgfsys@defobject{currentmarker}{\pgfqpoint{0.000000in}{0.000000in}}{\pgfqpoint{0.034722in}{0.000000in}}{%
\pgfpathmoveto{\pgfqpoint{0.000000in}{0.000000in}}%
\pgfpathlineto{\pgfqpoint{0.034722in}{0.000000in}}%
\pgfusepath{stroke,fill}%
}%
\begin{pgfscope}%
\pgfsys@transformshift{0.978013in}{6.461478in}%
\pgfsys@useobject{currentmarker}{}%
\end{pgfscope}%
\end{pgfscope}%
\begin{pgfscope}%
\definecolor{textcolor}{rgb}{0.000000,0.000000,0.000000}%
\pgfsetstrokecolor{textcolor}%
\pgfsetfillcolor{textcolor}%
\pgftext[x=0.643989in, y=6.366507in, left, base]{\color{textcolor}\sffamily\fontsize{18.000000}{21.600000}\selectfont $\displaystyle 2.5$}%
\end{pgfscope}%
\begin{pgfscope}%
\pgfpathrectangle{\pgfqpoint{0.978013in}{1.247073in}}{\pgfqpoint{10.943247in}{6.674186in}}%
\pgfusepath{clip}%
\pgfsetrectcap%
\pgfsetroundjoin%
\pgfsetlinewidth{0.501875pt}%
\definecolor{currentstroke}{rgb}{0.000000,0.000000,0.000000}%
\pgfsetstrokecolor{currentstroke}%
\pgfsetstrokeopacity{0.100000}%
\pgfsetdash{}{0pt}%
\pgfpathmoveto{\pgfqpoint{0.978013in}{7.732368in}}%
\pgfpathlineto{\pgfqpoint{11.921260in}{7.732368in}}%
\pgfusepath{stroke}%
\end{pgfscope}%
\begin{pgfscope}%
\pgfsetbuttcap%
\pgfsetroundjoin%
\definecolor{currentfill}{rgb}{0.000000,0.000000,0.000000}%
\pgfsetfillcolor{currentfill}%
\pgfsetlinewidth{0.501875pt}%
\definecolor{currentstroke}{rgb}{0.000000,0.000000,0.000000}%
\pgfsetstrokecolor{currentstroke}%
\pgfsetdash{}{0pt}%
\pgfsys@defobject{currentmarker}{\pgfqpoint{0.000000in}{0.000000in}}{\pgfqpoint{0.034722in}{0.000000in}}{%
\pgfpathmoveto{\pgfqpoint{0.000000in}{0.000000in}}%
\pgfpathlineto{\pgfqpoint{0.034722in}{0.000000in}}%
\pgfusepath{stroke,fill}%
}%
\begin{pgfscope}%
\pgfsys@transformshift{0.978013in}{7.732368in}%
\pgfsys@useobject{currentmarker}{}%
\end{pgfscope}%
\end{pgfscope}%
\begin{pgfscope}%
\definecolor{textcolor}{rgb}{0.000000,0.000000,0.000000}%
\pgfsetstrokecolor{textcolor}%
\pgfsetfillcolor{textcolor}%
\pgftext[x=0.643989in, y=7.637397in, left, base]{\color{textcolor}\sffamily\fontsize{18.000000}{21.600000}\selectfont $\displaystyle 3.0$}%
\end{pgfscope}%
\begin{pgfscope}%
\pgfpathrectangle{\pgfqpoint{0.978013in}{1.247073in}}{\pgfqpoint{10.943247in}{6.674186in}}%
\pgfusepath{clip}%
\pgfsetbuttcap%
\pgfsetroundjoin%
\pgfsetlinewidth{1.003750pt}%
\definecolor{currentstroke}{rgb}{0.000000,0.605603,0.978680}%
\pgfsetstrokecolor{currentstroke}%
\pgfsetdash{}{0pt}%
\pgfpathmoveto{\pgfqpoint{1.287728in}{7.732368in}}%
\pgfpathlineto{\pgfqpoint{6.681519in}{7.731259in}}%
\pgfpathlineto{\pgfqpoint{6.696642in}{7.728288in}}%
\pgfpathlineto{\pgfqpoint{6.706724in}{7.723104in}}%
\pgfpathlineto{\pgfqpoint{6.711765in}{7.718615in}}%
\pgfpathlineto{\pgfqpoint{6.716806in}{7.712158in}}%
\pgfpathlineto{\pgfqpoint{6.721847in}{7.702974in}}%
\pgfpathlineto{\pgfqpoint{6.726887in}{7.690061in}}%
\pgfpathlineto{\pgfqpoint{6.731928in}{7.672118in}}%
\pgfpathlineto{\pgfqpoint{6.736969in}{7.647487in}}%
\pgfpathlineto{\pgfqpoint{6.742010in}{7.614090in}}%
\pgfpathlineto{\pgfqpoint{6.747051in}{7.569385in}}%
\pgfpathlineto{\pgfqpoint{6.752092in}{7.510321in}}%
\pgfpathlineto{\pgfqpoint{6.757133in}{7.433339in}}%
\pgfpathlineto{\pgfqpoint{6.762174in}{7.334399in}}%
\pgfpathlineto{\pgfqpoint{6.767215in}{7.209076in}}%
\pgfpathlineto{\pgfqpoint{6.772256in}{7.052715in}}%
\pgfpathlineto{\pgfqpoint{6.782338in}{6.628726in}}%
\pgfpathlineto{\pgfqpoint{6.792420in}{6.032503in}}%
\pgfpathlineto{\pgfqpoint{6.802501in}{5.255795in}}%
\pgfpathlineto{\pgfqpoint{6.817624in}{3.835204in}}%
\pgfpathlineto{\pgfqpoint{6.832747in}{2.421094in}}%
\pgfpathlineto{\pgfqpoint{6.842829in}{1.739646in}}%
\pgfpathlineto{\pgfqpoint{6.847870in}{1.534458in}}%
\pgfpathlineto{\pgfqpoint{6.852911in}{1.435966in}}%
\pgfpathlineto{\pgfqpoint{6.857952in}{1.447567in}}%
\pgfpathlineto{\pgfqpoint{6.862992in}{1.563840in}}%
\pgfpathlineto{\pgfqpoint{6.868033in}{1.770043in}}%
\pgfpathlineto{\pgfqpoint{6.878115in}{2.351320in}}%
\pgfpathlineto{\pgfqpoint{6.888197in}{2.938418in}}%
\pgfpathlineto{\pgfqpoint{6.893238in}{3.151601in}}%
\pgfpathlineto{\pgfqpoint{6.898279in}{3.278625in}}%
\pgfpathlineto{\pgfqpoint{6.903320in}{3.308630in}}%
\pgfpathlineto{\pgfqpoint{6.908361in}{3.244117in}}%
\pgfpathlineto{\pgfqpoint{6.913402in}{3.100852in}}%
\pgfpathlineto{\pgfqpoint{6.933565in}{2.347720in}}%
\pgfpathlineto{\pgfqpoint{6.938606in}{2.268424in}}%
\pgfpathlineto{\pgfqpoint{6.943647in}{2.266040in}}%
\pgfpathlineto{\pgfqpoint{6.948688in}{2.334031in}}%
\pgfpathlineto{\pgfqpoint{6.958770in}{2.595445in}}%
\pgfpathlineto{\pgfqpoint{6.963811in}{2.730526in}}%
\pgfpathlineto{\pgfqpoint{6.968852in}{2.831583in}}%
\pgfpathlineto{\pgfqpoint{6.973893in}{2.881033in}}%
\pgfpathlineto{\pgfqpoint{6.978934in}{2.873796in}}%
\pgfpathlineto{\pgfqpoint{6.983975in}{2.817706in}}%
\pgfpathlineto{\pgfqpoint{6.999097in}{2.559166in}}%
\pgfpathlineto{\pgfqpoint{7.004138in}{2.514128in}}%
\pgfpathlineto{\pgfqpoint{7.009179in}{2.509034in}}%
\pgfpathlineto{\pgfqpoint{7.014220in}{2.540126in}}%
\pgfpathlineto{\pgfqpoint{7.029343in}{2.704927in}}%
\pgfpathlineto{\pgfqpoint{7.034384in}{2.731392in}}%
\pgfpathlineto{\pgfqpoint{7.039425in}{2.730585in}}%
\pgfpathlineto{\pgfqpoint{7.044466in}{2.706609in}}%
\pgfpathlineto{\pgfqpoint{7.054548in}{2.632770in}}%
\pgfpathlineto{\pgfqpoint{7.059589in}{2.606759in}}%
\pgfpathlineto{\pgfqpoint{7.064630in}{2.597978in}}%
\pgfpathlineto{\pgfqpoint{7.069670in}{2.606478in}}%
\pgfpathlineto{\pgfqpoint{7.084793in}{2.669699in}}%
\pgfpathlineto{\pgfqpoint{7.089834in}{2.678372in}}%
\pgfpathlineto{\pgfqpoint{7.094875in}{2.675549in}}%
\pgfpathlineto{\pgfqpoint{7.099916in}{2.664142in}}%
\pgfpathlineto{\pgfqpoint{7.109998in}{2.637475in}}%
\pgfpathlineto{\pgfqpoint{7.115039in}{2.631717in}}%
\pgfpathlineto{\pgfqpoint{7.120080in}{2.633223in}}%
\pgfpathlineto{\pgfqpoint{7.130162in}{2.648871in}}%
\pgfpathlineto{\pgfqpoint{7.135203in}{2.655849in}}%
\pgfpathlineto{\pgfqpoint{7.140243in}{2.658749in}}%
\pgfpathlineto{\pgfqpoint{7.145284in}{2.657240in}}%
\pgfpathlineto{\pgfqpoint{7.160407in}{2.644094in}}%
\pgfpathlineto{\pgfqpoint{7.165448in}{2.643183in}}%
\pgfpathlineto{\pgfqpoint{7.170489in}{2.644761in}}%
\pgfpathlineto{\pgfqpoint{7.180571in}{2.650409in}}%
\pgfpathlineto{\pgfqpoint{7.185612in}{2.651872in}}%
\pgfpathlineto{\pgfqpoint{7.190653in}{2.651702in}}%
\pgfpathlineto{\pgfqpoint{7.210816in}{2.647058in}}%
\pgfpathlineto{\pgfqpoint{7.236021in}{2.649597in}}%
\pgfpathlineto{\pgfqpoint{7.256185in}{2.648340in}}%
\pgfpathlineto{\pgfqpoint{7.281389in}{2.648938in}}%
\pgfpathlineto{\pgfqpoint{7.316676in}{2.648883in}}%
\pgfpathlineto{\pgfqpoint{7.467904in}{2.648809in}}%
\pgfpathlineto{\pgfqpoint{11.611545in}{2.648809in}}%
\pgfpathlineto{\pgfqpoint{11.611545in}{2.648809in}}%
\pgfusepath{stroke}%
\end{pgfscope}%
\begin{pgfscope}%
\pgfpathrectangle{\pgfqpoint{0.978013in}{1.247073in}}{\pgfqpoint{10.943247in}{6.674186in}}%
\pgfusepath{clip}%
\pgfsetbuttcap%
\pgfsetroundjoin%
\pgfsetlinewidth{1.003750pt}%
\definecolor{currentstroke}{rgb}{0.888874,0.435649,0.278123}%
\pgfsetstrokecolor{currentstroke}%
\pgfsetdash{}{0pt}%
\pgfpathmoveto{\pgfqpoint{1.287728in}{7.732368in}}%
\pgfpathlineto{\pgfqpoint{6.792420in}{7.732368in}}%
\pgfpathlineto{\pgfqpoint{6.797460in}{2.648809in}}%
\pgfpathlineto{\pgfqpoint{11.611545in}{2.648809in}}%
\pgfpathlineto{\pgfqpoint{11.611545in}{2.648809in}}%
\pgfusepath{stroke}%
\end{pgfscope}%
\begin{pgfscope}%
\pgfsetrectcap%
\pgfsetmiterjoin%
\pgfsetlinewidth{1.003750pt}%
\definecolor{currentstroke}{rgb}{0.000000,0.000000,0.000000}%
\pgfsetstrokecolor{currentstroke}%
\pgfsetdash{}{0pt}%
\pgfpathmoveto{\pgfqpoint{0.978013in}{1.247073in}}%
\pgfpathlineto{\pgfqpoint{0.978013in}{7.921260in}}%
\pgfusepath{stroke}%
\end{pgfscope}%
\begin{pgfscope}%
\pgfsetrectcap%
\pgfsetmiterjoin%
\pgfsetlinewidth{1.003750pt}%
\definecolor{currentstroke}{rgb}{0.000000,0.000000,0.000000}%
\pgfsetstrokecolor{currentstroke}%
\pgfsetdash{}{0pt}%
\pgfpathmoveto{\pgfqpoint{0.978013in}{1.247073in}}%
\pgfpathlineto{\pgfqpoint{11.921260in}{1.247073in}}%
\pgfusepath{stroke}%
\end{pgfscope}%
\begin{pgfscope}%
\pgfsetbuttcap%
\pgfsetmiterjoin%
\definecolor{currentfill}{rgb}{1.000000,1.000000,1.000000}%
\pgfsetfillcolor{currentfill}%
\pgfsetlinewidth{1.003750pt}%
\definecolor{currentstroke}{rgb}{0.000000,0.000000,0.000000}%
\pgfsetstrokecolor{currentstroke}%
\pgfsetdash{}{0pt}%
\pgfpathmoveto{\pgfqpoint{10.511589in}{6.787373in}}%
\pgfpathlineto{\pgfqpoint{11.796260in}{6.787373in}}%
\pgfpathlineto{\pgfqpoint{11.796260in}{7.796260in}}%
\pgfpathlineto{\pgfqpoint{10.511589in}{7.796260in}}%
\pgfpathclose%
\pgfusepath{stroke,fill}%
\end{pgfscope}%
\begin{pgfscope}%
\pgfsetbuttcap%
\pgfsetmiterjoin%
\pgfsetlinewidth{2.258437pt}%
\definecolor{currentstroke}{rgb}{0.000000,0.605603,0.978680}%
\pgfsetstrokecolor{currentstroke}%
\pgfsetdash{}{0pt}%
\pgfpathmoveto{\pgfqpoint{10.711589in}{7.493818in}}%
\pgfpathlineto{\pgfqpoint{11.211589in}{7.493818in}}%
\pgfusepath{stroke}%
\end{pgfscope}%
\begin{pgfscope}%
\definecolor{textcolor}{rgb}{0.000000,0.000000,0.000000}%
\pgfsetstrokecolor{textcolor}%
\pgfsetfillcolor{textcolor}%
\pgftext[x=11.411589in,y=7.406318in,left,base]{\color{textcolor}\sffamily\fontsize{18.000000}{21.600000}\selectfont $\displaystyle U$}%
\end{pgfscope}%
\begin{pgfscope}%
\pgfsetbuttcap%
\pgfsetmiterjoin%
\pgfsetlinewidth{2.258437pt}%
\definecolor{currentstroke}{rgb}{0.888874,0.435649,0.278123}%
\pgfsetstrokecolor{currentstroke}%
\pgfsetdash{}{0pt}%
\pgfpathmoveto{\pgfqpoint{10.711589in}{7.126875in}}%
\pgfpathlineto{\pgfqpoint{11.211589in}{7.126875in}}%
\pgfusepath{stroke}%
\end{pgfscope}%
\begin{pgfscope}%
\definecolor{textcolor}{rgb}{0.000000,0.000000,0.000000}%
\pgfsetstrokecolor{textcolor}%
\pgfsetfillcolor{textcolor}%
\pgftext[x=11.411589in,y=7.039375in,left,base]{\color{textcolor}\sffamily\fontsize{18.000000}{21.600000}\selectfont $\displaystyle u$}%
\end{pgfscope}%
\end{pgfpicture}%
\makeatother%
\endgroup%
}
	\caption{Beam-Warming 差分逼近解 $U$ 与真解 $u$}\label{fig:beam_warming_square_Uu}
\end{figure}

取 $\nu = 4$, 结果如图 \ref{fig:beam_warming_square_Uu_noCFL} 所示, 出现了错误解.

\begin{figure}[H]\centering\zihao{-5}
	\resizebox{0.4\linewidth}{!}{%% Creator: Matplotlib, PGF backend
%%
%% To include the figure in your LaTeX document, write
%%   \input{<filename>.pgf}
%%
%% Make sure the required packages are loaded in your preamble
%%   \usepackage{pgf}
%%
%% Figures using additional raster images can only be included by \input if
%% they are in the same directory as the main LaTeX file. For loading figures
%% from other directories you can use the `import` package
%%   \usepackage{import}
%%
%% and then include the figures with
%%   \import{<path to file>}{<filename>.pgf}
%%
%% Matplotlib used the following preamble
%%   \usepackage{fontspec}
%%   \setmainfont{DejaVuSerif.ttf}[Path=\detokenize{/Users/quejiahao/.julia/conda/3/lib/python3.9/site-packages/matplotlib/mpl-data/fonts/ttf/}]
%%   \setsansfont{DejaVuSans.ttf}[Path=\detokenize{/Users/quejiahao/.julia/conda/3/lib/python3.9/site-packages/matplotlib/mpl-data/fonts/ttf/}]
%%   \setmonofont{DejaVuSansMono.ttf}[Path=\detokenize{/Users/quejiahao/.julia/conda/3/lib/python3.9/site-packages/matplotlib/mpl-data/fonts/ttf/}]
%%
\begingroup%
\makeatletter%
\begin{pgfpicture}%
\pgfpathrectangle{\pgfpointorigin}{\pgfqpoint{12.000000in}{8.000000in}}%
\pgfusepath{use as bounding box, clip}%
\begin{pgfscope}%
\pgfsetbuttcap%
\pgfsetmiterjoin%
\definecolor{currentfill}{rgb}{1.000000,1.000000,1.000000}%
\pgfsetfillcolor{currentfill}%
\pgfsetlinewidth{0.000000pt}%
\definecolor{currentstroke}{rgb}{1.000000,1.000000,1.000000}%
\pgfsetstrokecolor{currentstroke}%
\pgfsetdash{}{0pt}%
\pgfpathmoveto{\pgfqpoint{0.000000in}{0.000000in}}%
\pgfpathlineto{\pgfqpoint{12.000000in}{0.000000in}}%
\pgfpathlineto{\pgfqpoint{12.000000in}{8.000000in}}%
\pgfpathlineto{\pgfqpoint{0.000000in}{8.000000in}}%
\pgfpathclose%
\pgfusepath{fill}%
\end{pgfscope}%
\begin{pgfscope}%
\pgfsetbuttcap%
\pgfsetmiterjoin%
\definecolor{currentfill}{rgb}{1.000000,1.000000,1.000000}%
\pgfsetfillcolor{currentfill}%
\pgfsetlinewidth{0.000000pt}%
\definecolor{currentstroke}{rgb}{0.000000,0.000000,0.000000}%
\pgfsetstrokecolor{currentstroke}%
\pgfsetstrokeopacity{0.000000}%
\pgfsetdash{}{0pt}%
\pgfpathmoveto{\pgfqpoint{3.008890in}{1.247073in}}%
\pgfpathlineto{\pgfqpoint{11.921260in}{1.247073in}}%
\pgfpathlineto{\pgfqpoint{11.921260in}{7.921260in}}%
\pgfpathlineto{\pgfqpoint{3.008890in}{7.921260in}}%
\pgfpathclose%
\pgfusepath{fill}%
\end{pgfscope}%
\begin{pgfscope}%
\pgfpathrectangle{\pgfqpoint{3.008890in}{1.247073in}}{\pgfqpoint{8.912370in}{6.674186in}}%
\pgfusepath{clip}%
\pgfsetrectcap%
\pgfsetroundjoin%
\pgfsetlinewidth{0.501875pt}%
\definecolor{currentstroke}{rgb}{0.000000,0.000000,0.000000}%
\pgfsetstrokecolor{currentstroke}%
\pgfsetstrokeopacity{0.100000}%
\pgfsetdash{}{0pt}%
\pgfpathmoveto{\pgfqpoint{3.261127in}{1.247073in}}%
\pgfpathlineto{\pgfqpoint{3.261127in}{7.921260in}}%
\pgfusepath{stroke}%
\end{pgfscope}%
\begin{pgfscope}%
\pgfsetbuttcap%
\pgfsetroundjoin%
\definecolor{currentfill}{rgb}{0.000000,0.000000,0.000000}%
\pgfsetfillcolor{currentfill}%
\pgfsetlinewidth{0.501875pt}%
\definecolor{currentstroke}{rgb}{0.000000,0.000000,0.000000}%
\pgfsetstrokecolor{currentstroke}%
\pgfsetdash{}{0pt}%
\pgfsys@defobject{currentmarker}{\pgfqpoint{0.000000in}{0.000000in}}{\pgfqpoint{0.000000in}{0.034722in}}{%
\pgfpathmoveto{\pgfqpoint{0.000000in}{0.000000in}}%
\pgfpathlineto{\pgfqpoint{0.000000in}{0.034722in}}%
\pgfusepath{stroke,fill}%
}%
\begin{pgfscope}%
\pgfsys@transformshift{3.261127in}{1.247073in}%
\pgfsys@useobject{currentmarker}{}%
\end{pgfscope}%
\end{pgfscope}%
\begin{pgfscope}%
\definecolor{textcolor}{rgb}{0.000000,0.000000,0.000000}%
\pgfsetstrokecolor{textcolor}%
\pgfsetfillcolor{textcolor}%
\pgftext[x=3.261127in,y=1.198462in,,top]{\color{textcolor}\sffamily\fontsize{18.000000}{21.600000}\selectfont $\displaystyle 0$}%
\end{pgfscope}%
\begin{pgfscope}%
\pgfpathrectangle{\pgfqpoint{3.008890in}{1.247073in}}{\pgfqpoint{8.912370in}{6.674186in}}%
\pgfusepath{clip}%
\pgfsetrectcap%
\pgfsetroundjoin%
\pgfsetlinewidth{0.501875pt}%
\definecolor{currentstroke}{rgb}{0.000000,0.000000,0.000000}%
\pgfsetstrokecolor{currentstroke}%
\pgfsetstrokeopacity{0.100000}%
\pgfsetdash{}{0pt}%
\pgfpathmoveto{\pgfqpoint{4.599285in}{1.247073in}}%
\pgfpathlineto{\pgfqpoint{4.599285in}{7.921260in}}%
\pgfusepath{stroke}%
\end{pgfscope}%
\begin{pgfscope}%
\pgfsetbuttcap%
\pgfsetroundjoin%
\definecolor{currentfill}{rgb}{0.000000,0.000000,0.000000}%
\pgfsetfillcolor{currentfill}%
\pgfsetlinewidth{0.501875pt}%
\definecolor{currentstroke}{rgb}{0.000000,0.000000,0.000000}%
\pgfsetstrokecolor{currentstroke}%
\pgfsetdash{}{0pt}%
\pgfsys@defobject{currentmarker}{\pgfqpoint{0.000000in}{0.000000in}}{\pgfqpoint{0.000000in}{0.034722in}}{%
\pgfpathmoveto{\pgfqpoint{0.000000in}{0.000000in}}%
\pgfpathlineto{\pgfqpoint{0.000000in}{0.034722in}}%
\pgfusepath{stroke,fill}%
}%
\begin{pgfscope}%
\pgfsys@transformshift{4.599285in}{1.247073in}%
\pgfsys@useobject{currentmarker}{}%
\end{pgfscope}%
\end{pgfscope}%
\begin{pgfscope}%
\definecolor{textcolor}{rgb}{0.000000,0.000000,0.000000}%
\pgfsetstrokecolor{textcolor}%
\pgfsetfillcolor{textcolor}%
\pgftext[x=4.599285in,y=1.198462in,,top]{\color{textcolor}\sffamily\fontsize{18.000000}{21.600000}\selectfont $\displaystyle 1$}%
\end{pgfscope}%
\begin{pgfscope}%
\pgfpathrectangle{\pgfqpoint{3.008890in}{1.247073in}}{\pgfqpoint{8.912370in}{6.674186in}}%
\pgfusepath{clip}%
\pgfsetrectcap%
\pgfsetroundjoin%
\pgfsetlinewidth{0.501875pt}%
\definecolor{currentstroke}{rgb}{0.000000,0.000000,0.000000}%
\pgfsetstrokecolor{currentstroke}%
\pgfsetstrokeopacity{0.100000}%
\pgfsetdash{}{0pt}%
\pgfpathmoveto{\pgfqpoint{5.937443in}{1.247073in}}%
\pgfpathlineto{\pgfqpoint{5.937443in}{7.921260in}}%
\pgfusepath{stroke}%
\end{pgfscope}%
\begin{pgfscope}%
\pgfsetbuttcap%
\pgfsetroundjoin%
\definecolor{currentfill}{rgb}{0.000000,0.000000,0.000000}%
\pgfsetfillcolor{currentfill}%
\pgfsetlinewidth{0.501875pt}%
\definecolor{currentstroke}{rgb}{0.000000,0.000000,0.000000}%
\pgfsetstrokecolor{currentstroke}%
\pgfsetdash{}{0pt}%
\pgfsys@defobject{currentmarker}{\pgfqpoint{0.000000in}{0.000000in}}{\pgfqpoint{0.000000in}{0.034722in}}{%
\pgfpathmoveto{\pgfqpoint{0.000000in}{0.000000in}}%
\pgfpathlineto{\pgfqpoint{0.000000in}{0.034722in}}%
\pgfusepath{stroke,fill}%
}%
\begin{pgfscope}%
\pgfsys@transformshift{5.937443in}{1.247073in}%
\pgfsys@useobject{currentmarker}{}%
\end{pgfscope}%
\end{pgfscope}%
\begin{pgfscope}%
\definecolor{textcolor}{rgb}{0.000000,0.000000,0.000000}%
\pgfsetstrokecolor{textcolor}%
\pgfsetfillcolor{textcolor}%
\pgftext[x=5.937443in,y=1.198462in,,top]{\color{textcolor}\sffamily\fontsize{18.000000}{21.600000}\selectfont $\displaystyle 2$}%
\end{pgfscope}%
\begin{pgfscope}%
\pgfpathrectangle{\pgfqpoint{3.008890in}{1.247073in}}{\pgfqpoint{8.912370in}{6.674186in}}%
\pgfusepath{clip}%
\pgfsetrectcap%
\pgfsetroundjoin%
\pgfsetlinewidth{0.501875pt}%
\definecolor{currentstroke}{rgb}{0.000000,0.000000,0.000000}%
\pgfsetstrokecolor{currentstroke}%
\pgfsetstrokeopacity{0.100000}%
\pgfsetdash{}{0pt}%
\pgfpathmoveto{\pgfqpoint{7.275601in}{1.247073in}}%
\pgfpathlineto{\pgfqpoint{7.275601in}{7.921260in}}%
\pgfusepath{stroke}%
\end{pgfscope}%
\begin{pgfscope}%
\pgfsetbuttcap%
\pgfsetroundjoin%
\definecolor{currentfill}{rgb}{0.000000,0.000000,0.000000}%
\pgfsetfillcolor{currentfill}%
\pgfsetlinewidth{0.501875pt}%
\definecolor{currentstroke}{rgb}{0.000000,0.000000,0.000000}%
\pgfsetstrokecolor{currentstroke}%
\pgfsetdash{}{0pt}%
\pgfsys@defobject{currentmarker}{\pgfqpoint{0.000000in}{0.000000in}}{\pgfqpoint{0.000000in}{0.034722in}}{%
\pgfpathmoveto{\pgfqpoint{0.000000in}{0.000000in}}%
\pgfpathlineto{\pgfqpoint{0.000000in}{0.034722in}}%
\pgfusepath{stroke,fill}%
}%
\begin{pgfscope}%
\pgfsys@transformshift{7.275601in}{1.247073in}%
\pgfsys@useobject{currentmarker}{}%
\end{pgfscope}%
\end{pgfscope}%
\begin{pgfscope}%
\definecolor{textcolor}{rgb}{0.000000,0.000000,0.000000}%
\pgfsetstrokecolor{textcolor}%
\pgfsetfillcolor{textcolor}%
\pgftext[x=7.275601in,y=1.198462in,,top]{\color{textcolor}\sffamily\fontsize{18.000000}{21.600000}\selectfont $\displaystyle 3$}%
\end{pgfscope}%
\begin{pgfscope}%
\pgfpathrectangle{\pgfqpoint{3.008890in}{1.247073in}}{\pgfqpoint{8.912370in}{6.674186in}}%
\pgfusepath{clip}%
\pgfsetrectcap%
\pgfsetroundjoin%
\pgfsetlinewidth{0.501875pt}%
\definecolor{currentstroke}{rgb}{0.000000,0.000000,0.000000}%
\pgfsetstrokecolor{currentstroke}%
\pgfsetstrokeopacity{0.100000}%
\pgfsetdash{}{0pt}%
\pgfpathmoveto{\pgfqpoint{8.613760in}{1.247073in}}%
\pgfpathlineto{\pgfqpoint{8.613760in}{7.921260in}}%
\pgfusepath{stroke}%
\end{pgfscope}%
\begin{pgfscope}%
\pgfsetbuttcap%
\pgfsetroundjoin%
\definecolor{currentfill}{rgb}{0.000000,0.000000,0.000000}%
\pgfsetfillcolor{currentfill}%
\pgfsetlinewidth{0.501875pt}%
\definecolor{currentstroke}{rgb}{0.000000,0.000000,0.000000}%
\pgfsetstrokecolor{currentstroke}%
\pgfsetdash{}{0pt}%
\pgfsys@defobject{currentmarker}{\pgfqpoint{0.000000in}{0.000000in}}{\pgfqpoint{0.000000in}{0.034722in}}{%
\pgfpathmoveto{\pgfqpoint{0.000000in}{0.000000in}}%
\pgfpathlineto{\pgfqpoint{0.000000in}{0.034722in}}%
\pgfusepath{stroke,fill}%
}%
\begin{pgfscope}%
\pgfsys@transformshift{8.613760in}{1.247073in}%
\pgfsys@useobject{currentmarker}{}%
\end{pgfscope}%
\end{pgfscope}%
\begin{pgfscope}%
\definecolor{textcolor}{rgb}{0.000000,0.000000,0.000000}%
\pgfsetstrokecolor{textcolor}%
\pgfsetfillcolor{textcolor}%
\pgftext[x=8.613760in,y=1.198462in,,top]{\color{textcolor}\sffamily\fontsize{18.000000}{21.600000}\selectfont $\displaystyle 4$}%
\end{pgfscope}%
\begin{pgfscope}%
\pgfpathrectangle{\pgfqpoint{3.008890in}{1.247073in}}{\pgfqpoint{8.912370in}{6.674186in}}%
\pgfusepath{clip}%
\pgfsetrectcap%
\pgfsetroundjoin%
\pgfsetlinewidth{0.501875pt}%
\definecolor{currentstroke}{rgb}{0.000000,0.000000,0.000000}%
\pgfsetstrokecolor{currentstroke}%
\pgfsetstrokeopacity{0.100000}%
\pgfsetdash{}{0pt}%
\pgfpathmoveto{\pgfqpoint{9.951918in}{1.247073in}}%
\pgfpathlineto{\pgfqpoint{9.951918in}{7.921260in}}%
\pgfusepath{stroke}%
\end{pgfscope}%
\begin{pgfscope}%
\pgfsetbuttcap%
\pgfsetroundjoin%
\definecolor{currentfill}{rgb}{0.000000,0.000000,0.000000}%
\pgfsetfillcolor{currentfill}%
\pgfsetlinewidth{0.501875pt}%
\definecolor{currentstroke}{rgb}{0.000000,0.000000,0.000000}%
\pgfsetstrokecolor{currentstroke}%
\pgfsetdash{}{0pt}%
\pgfsys@defobject{currentmarker}{\pgfqpoint{0.000000in}{0.000000in}}{\pgfqpoint{0.000000in}{0.034722in}}{%
\pgfpathmoveto{\pgfqpoint{0.000000in}{0.000000in}}%
\pgfpathlineto{\pgfqpoint{0.000000in}{0.034722in}}%
\pgfusepath{stroke,fill}%
}%
\begin{pgfscope}%
\pgfsys@transformshift{9.951918in}{1.247073in}%
\pgfsys@useobject{currentmarker}{}%
\end{pgfscope}%
\end{pgfscope}%
\begin{pgfscope}%
\definecolor{textcolor}{rgb}{0.000000,0.000000,0.000000}%
\pgfsetstrokecolor{textcolor}%
\pgfsetfillcolor{textcolor}%
\pgftext[x=9.951918in,y=1.198462in,,top]{\color{textcolor}\sffamily\fontsize{18.000000}{21.600000}\selectfont $\displaystyle 5$}%
\end{pgfscope}%
\begin{pgfscope}%
\pgfpathrectangle{\pgfqpoint{3.008890in}{1.247073in}}{\pgfqpoint{8.912370in}{6.674186in}}%
\pgfusepath{clip}%
\pgfsetrectcap%
\pgfsetroundjoin%
\pgfsetlinewidth{0.501875pt}%
\definecolor{currentstroke}{rgb}{0.000000,0.000000,0.000000}%
\pgfsetstrokecolor{currentstroke}%
\pgfsetstrokeopacity{0.100000}%
\pgfsetdash{}{0pt}%
\pgfpathmoveto{\pgfqpoint{11.290076in}{1.247073in}}%
\pgfpathlineto{\pgfqpoint{11.290076in}{7.921260in}}%
\pgfusepath{stroke}%
\end{pgfscope}%
\begin{pgfscope}%
\pgfsetbuttcap%
\pgfsetroundjoin%
\definecolor{currentfill}{rgb}{0.000000,0.000000,0.000000}%
\pgfsetfillcolor{currentfill}%
\pgfsetlinewidth{0.501875pt}%
\definecolor{currentstroke}{rgb}{0.000000,0.000000,0.000000}%
\pgfsetstrokecolor{currentstroke}%
\pgfsetdash{}{0pt}%
\pgfsys@defobject{currentmarker}{\pgfqpoint{0.000000in}{0.000000in}}{\pgfqpoint{0.000000in}{0.034722in}}{%
\pgfpathmoveto{\pgfqpoint{0.000000in}{0.000000in}}%
\pgfpathlineto{\pgfqpoint{0.000000in}{0.034722in}}%
\pgfusepath{stroke,fill}%
}%
\begin{pgfscope}%
\pgfsys@transformshift{11.290076in}{1.247073in}%
\pgfsys@useobject{currentmarker}{}%
\end{pgfscope}%
\end{pgfscope}%
\begin{pgfscope}%
\definecolor{textcolor}{rgb}{0.000000,0.000000,0.000000}%
\pgfsetstrokecolor{textcolor}%
\pgfsetfillcolor{textcolor}%
\pgftext[x=11.290076in,y=1.198462in,,top]{\color{textcolor}\sffamily\fontsize{18.000000}{21.600000}\selectfont $\displaystyle 6$}%
\end{pgfscope}%
\begin{pgfscope}%
\definecolor{textcolor}{rgb}{0.000000,0.000000,0.000000}%
\pgfsetstrokecolor{textcolor}%
\pgfsetfillcolor{textcolor}%
\pgftext[x=7.465075in,y=0.900964in,,top]{\color{textcolor}\sffamily\fontsize{18.000000}{21.600000}\selectfont $\displaystyle x$}%
\end{pgfscope}%
\begin{pgfscope}%
\pgfpathrectangle{\pgfqpoint{3.008890in}{1.247073in}}{\pgfqpoint{8.912370in}{6.674186in}}%
\pgfusepath{clip}%
\pgfsetrectcap%
\pgfsetroundjoin%
\pgfsetlinewidth{0.501875pt}%
\definecolor{currentstroke}{rgb}{0.000000,0.000000,0.000000}%
\pgfsetstrokecolor{currentstroke}%
\pgfsetstrokeopacity{0.100000}%
\pgfsetdash{}{0pt}%
\pgfpathmoveto{\pgfqpoint{3.008890in}{1.601215in}}%
\pgfpathlineto{\pgfqpoint{11.921260in}{1.601215in}}%
\pgfusepath{stroke}%
\end{pgfscope}%
\begin{pgfscope}%
\pgfsetbuttcap%
\pgfsetroundjoin%
\definecolor{currentfill}{rgb}{0.000000,0.000000,0.000000}%
\pgfsetfillcolor{currentfill}%
\pgfsetlinewidth{0.501875pt}%
\definecolor{currentstroke}{rgb}{0.000000,0.000000,0.000000}%
\pgfsetstrokecolor{currentstroke}%
\pgfsetdash{}{0pt}%
\pgfsys@defobject{currentmarker}{\pgfqpoint{0.000000in}{0.000000in}}{\pgfqpoint{0.034722in}{0.000000in}}{%
\pgfpathmoveto{\pgfqpoint{0.000000in}{0.000000in}}%
\pgfpathlineto{\pgfqpoint{0.034722in}{0.000000in}}%
\pgfusepath{stroke,fill}%
}%
\begin{pgfscope}%
\pgfsys@transformshift{3.008890in}{1.601215in}%
\pgfsys@useobject{currentmarker}{}%
\end{pgfscope}%
\end{pgfscope}%
\begin{pgfscope}%
\definecolor{textcolor}{rgb}{0.000000,0.000000,0.000000}%
\pgfsetstrokecolor{textcolor}%
\pgfsetfillcolor{textcolor}%
\pgftext[x=1.859980in, y=1.506244in, left, base]{\color{textcolor}\sffamily\fontsize{18.000000}{21.600000}\selectfont $\displaystyle -1.50×10^{7}$}%
\end{pgfscope}%
\begin{pgfscope}%
\pgfpathrectangle{\pgfqpoint{3.008890in}{1.247073in}}{\pgfqpoint{8.912370in}{6.674186in}}%
\pgfusepath{clip}%
\pgfsetrectcap%
\pgfsetroundjoin%
\pgfsetlinewidth{0.501875pt}%
\definecolor{currentstroke}{rgb}{0.000000,0.000000,0.000000}%
\pgfsetstrokecolor{currentstroke}%
\pgfsetstrokeopacity{0.100000}%
\pgfsetdash{}{0pt}%
\pgfpathmoveto{\pgfqpoint{3.008890in}{2.585796in}}%
\pgfpathlineto{\pgfqpoint{11.921260in}{2.585796in}}%
\pgfusepath{stroke}%
\end{pgfscope}%
\begin{pgfscope}%
\pgfsetbuttcap%
\pgfsetroundjoin%
\definecolor{currentfill}{rgb}{0.000000,0.000000,0.000000}%
\pgfsetfillcolor{currentfill}%
\pgfsetlinewidth{0.501875pt}%
\definecolor{currentstroke}{rgb}{0.000000,0.000000,0.000000}%
\pgfsetstrokecolor{currentstroke}%
\pgfsetdash{}{0pt}%
\pgfsys@defobject{currentmarker}{\pgfqpoint{0.000000in}{0.000000in}}{\pgfqpoint{0.034722in}{0.000000in}}{%
\pgfpathmoveto{\pgfqpoint{0.000000in}{0.000000in}}%
\pgfpathlineto{\pgfqpoint{0.034722in}{0.000000in}}%
\pgfusepath{stroke,fill}%
}%
\begin{pgfscope}%
\pgfsys@transformshift{3.008890in}{2.585796in}%
\pgfsys@useobject{currentmarker}{}%
\end{pgfscope}%
\end{pgfscope}%
\begin{pgfscope}%
\definecolor{textcolor}{rgb}{0.000000,0.000000,0.000000}%
\pgfsetstrokecolor{textcolor}%
\pgfsetfillcolor{textcolor}%
\pgftext[x=1.859980in, y=2.490825in, left, base]{\color{textcolor}\sffamily\fontsize{18.000000}{21.600000}\selectfont $\displaystyle -1.00×10^{7}$}%
\end{pgfscope}%
\begin{pgfscope}%
\pgfpathrectangle{\pgfqpoint{3.008890in}{1.247073in}}{\pgfqpoint{8.912370in}{6.674186in}}%
\pgfusepath{clip}%
\pgfsetrectcap%
\pgfsetroundjoin%
\pgfsetlinewidth{0.501875pt}%
\definecolor{currentstroke}{rgb}{0.000000,0.000000,0.000000}%
\pgfsetstrokecolor{currentstroke}%
\pgfsetstrokeopacity{0.100000}%
\pgfsetdash{}{0pt}%
\pgfpathmoveto{\pgfqpoint{3.008890in}{3.570378in}}%
\pgfpathlineto{\pgfqpoint{11.921260in}{3.570378in}}%
\pgfusepath{stroke}%
\end{pgfscope}%
\begin{pgfscope}%
\pgfsetbuttcap%
\pgfsetroundjoin%
\definecolor{currentfill}{rgb}{0.000000,0.000000,0.000000}%
\pgfsetfillcolor{currentfill}%
\pgfsetlinewidth{0.501875pt}%
\definecolor{currentstroke}{rgb}{0.000000,0.000000,0.000000}%
\pgfsetstrokecolor{currentstroke}%
\pgfsetdash{}{0pt}%
\pgfsys@defobject{currentmarker}{\pgfqpoint{0.000000in}{0.000000in}}{\pgfqpoint{0.034722in}{0.000000in}}{%
\pgfpathmoveto{\pgfqpoint{0.000000in}{0.000000in}}%
\pgfpathlineto{\pgfqpoint{0.034722in}{0.000000in}}%
\pgfusepath{stroke,fill}%
}%
\begin{pgfscope}%
\pgfsys@transformshift{3.008890in}{3.570378in}%
\pgfsys@useobject{currentmarker}{}%
\end{pgfscope}%
\end{pgfscope}%
\begin{pgfscope}%
\definecolor{textcolor}{rgb}{0.000000,0.000000,0.000000}%
\pgfsetstrokecolor{textcolor}%
\pgfsetfillcolor{textcolor}%
\pgftext[x=1.859980in, y=3.475407in, left, base]{\color{textcolor}\sffamily\fontsize{18.000000}{21.600000}\selectfont $\displaystyle -5.00×10^{6}$}%
\end{pgfscope}%
\begin{pgfscope}%
\pgfpathrectangle{\pgfqpoint{3.008890in}{1.247073in}}{\pgfqpoint{8.912370in}{6.674186in}}%
\pgfusepath{clip}%
\pgfsetrectcap%
\pgfsetroundjoin%
\pgfsetlinewidth{0.501875pt}%
\definecolor{currentstroke}{rgb}{0.000000,0.000000,0.000000}%
\pgfsetstrokecolor{currentstroke}%
\pgfsetstrokeopacity{0.100000}%
\pgfsetdash{}{0pt}%
\pgfpathmoveto{\pgfqpoint{3.008890in}{4.554959in}}%
\pgfpathlineto{\pgfqpoint{11.921260in}{4.554959in}}%
\pgfusepath{stroke}%
\end{pgfscope}%
\begin{pgfscope}%
\pgfsetbuttcap%
\pgfsetroundjoin%
\definecolor{currentfill}{rgb}{0.000000,0.000000,0.000000}%
\pgfsetfillcolor{currentfill}%
\pgfsetlinewidth{0.501875pt}%
\definecolor{currentstroke}{rgb}{0.000000,0.000000,0.000000}%
\pgfsetstrokecolor{currentstroke}%
\pgfsetdash{}{0pt}%
\pgfsys@defobject{currentmarker}{\pgfqpoint{0.000000in}{0.000000in}}{\pgfqpoint{0.034722in}{0.000000in}}{%
\pgfpathmoveto{\pgfqpoint{0.000000in}{0.000000in}}%
\pgfpathlineto{\pgfqpoint{0.034722in}{0.000000in}}%
\pgfusepath{stroke,fill}%
}%
\begin{pgfscope}%
\pgfsys@transformshift{3.008890in}{4.554959in}%
\pgfsys@useobject{currentmarker}{}%
\end{pgfscope}%
\end{pgfscope}%
\begin{pgfscope}%
\definecolor{textcolor}{rgb}{0.000000,0.000000,0.000000}%
\pgfsetstrokecolor{textcolor}%
\pgfsetfillcolor{textcolor}%
\pgftext[x=2.850210in, y=4.459988in, left, base]{\color{textcolor}\sffamily\fontsize{18.000000}{21.600000}\selectfont $\displaystyle 0$}%
\end{pgfscope}%
\begin{pgfscope}%
\pgfpathrectangle{\pgfqpoint{3.008890in}{1.247073in}}{\pgfqpoint{8.912370in}{6.674186in}}%
\pgfusepath{clip}%
\pgfsetrectcap%
\pgfsetroundjoin%
\pgfsetlinewidth{0.501875pt}%
\definecolor{currentstroke}{rgb}{0.000000,0.000000,0.000000}%
\pgfsetstrokecolor{currentstroke}%
\pgfsetstrokeopacity{0.100000}%
\pgfsetdash{}{0pt}%
\pgfpathmoveto{\pgfqpoint{3.008890in}{5.539541in}}%
\pgfpathlineto{\pgfqpoint{11.921260in}{5.539541in}}%
\pgfusepath{stroke}%
\end{pgfscope}%
\begin{pgfscope}%
\pgfsetbuttcap%
\pgfsetroundjoin%
\definecolor{currentfill}{rgb}{0.000000,0.000000,0.000000}%
\pgfsetfillcolor{currentfill}%
\pgfsetlinewidth{0.501875pt}%
\definecolor{currentstroke}{rgb}{0.000000,0.000000,0.000000}%
\pgfsetstrokecolor{currentstroke}%
\pgfsetdash{}{0pt}%
\pgfsys@defobject{currentmarker}{\pgfqpoint{0.000000in}{0.000000in}}{\pgfqpoint{0.034722in}{0.000000in}}{%
\pgfpathmoveto{\pgfqpoint{0.000000in}{0.000000in}}%
\pgfpathlineto{\pgfqpoint{0.034722in}{0.000000in}}%
\pgfusepath{stroke,fill}%
}%
\begin{pgfscope}%
\pgfsys@transformshift{3.008890in}{5.539541in}%
\pgfsys@useobject{currentmarker}{}%
\end{pgfscope}%
\end{pgfscope}%
\begin{pgfscope}%
\definecolor{textcolor}{rgb}{0.000000,0.000000,0.000000}%
\pgfsetstrokecolor{textcolor}%
\pgfsetfillcolor{textcolor}%
\pgftext[x=2.046647in, y=5.444570in, left, base]{\color{textcolor}\sffamily\fontsize{18.000000}{21.600000}\selectfont $\displaystyle 5.00×10^{6}$}%
\end{pgfscope}%
\begin{pgfscope}%
\pgfpathrectangle{\pgfqpoint{3.008890in}{1.247073in}}{\pgfqpoint{8.912370in}{6.674186in}}%
\pgfusepath{clip}%
\pgfsetrectcap%
\pgfsetroundjoin%
\pgfsetlinewidth{0.501875pt}%
\definecolor{currentstroke}{rgb}{0.000000,0.000000,0.000000}%
\pgfsetstrokecolor{currentstroke}%
\pgfsetstrokeopacity{0.100000}%
\pgfsetdash{}{0pt}%
\pgfpathmoveto{\pgfqpoint{3.008890in}{6.524122in}}%
\pgfpathlineto{\pgfqpoint{11.921260in}{6.524122in}}%
\pgfusepath{stroke}%
\end{pgfscope}%
\begin{pgfscope}%
\pgfsetbuttcap%
\pgfsetroundjoin%
\definecolor{currentfill}{rgb}{0.000000,0.000000,0.000000}%
\pgfsetfillcolor{currentfill}%
\pgfsetlinewidth{0.501875pt}%
\definecolor{currentstroke}{rgb}{0.000000,0.000000,0.000000}%
\pgfsetstrokecolor{currentstroke}%
\pgfsetdash{}{0pt}%
\pgfsys@defobject{currentmarker}{\pgfqpoint{0.000000in}{0.000000in}}{\pgfqpoint{0.034722in}{0.000000in}}{%
\pgfpathmoveto{\pgfqpoint{0.000000in}{0.000000in}}%
\pgfpathlineto{\pgfqpoint{0.034722in}{0.000000in}}%
\pgfusepath{stroke,fill}%
}%
\begin{pgfscope}%
\pgfsys@transformshift{3.008890in}{6.524122in}%
\pgfsys@useobject{currentmarker}{}%
\end{pgfscope}%
\end{pgfscope}%
\begin{pgfscope}%
\definecolor{textcolor}{rgb}{0.000000,0.000000,0.000000}%
\pgfsetstrokecolor{textcolor}%
\pgfsetfillcolor{textcolor}%
\pgftext[x=2.046647in, y=6.429151in, left, base]{\color{textcolor}\sffamily\fontsize{18.000000}{21.600000}\selectfont $\displaystyle 1.00×10^{7}$}%
\end{pgfscope}%
\begin{pgfscope}%
\pgfpathrectangle{\pgfqpoint{3.008890in}{1.247073in}}{\pgfqpoint{8.912370in}{6.674186in}}%
\pgfusepath{clip}%
\pgfsetrectcap%
\pgfsetroundjoin%
\pgfsetlinewidth{0.501875pt}%
\definecolor{currentstroke}{rgb}{0.000000,0.000000,0.000000}%
\pgfsetstrokecolor{currentstroke}%
\pgfsetstrokeopacity{0.100000}%
\pgfsetdash{}{0pt}%
\pgfpathmoveto{\pgfqpoint{3.008890in}{7.508704in}}%
\pgfpathlineto{\pgfqpoint{11.921260in}{7.508704in}}%
\pgfusepath{stroke}%
\end{pgfscope}%
\begin{pgfscope}%
\pgfsetbuttcap%
\pgfsetroundjoin%
\definecolor{currentfill}{rgb}{0.000000,0.000000,0.000000}%
\pgfsetfillcolor{currentfill}%
\pgfsetlinewidth{0.501875pt}%
\definecolor{currentstroke}{rgb}{0.000000,0.000000,0.000000}%
\pgfsetstrokecolor{currentstroke}%
\pgfsetdash{}{0pt}%
\pgfsys@defobject{currentmarker}{\pgfqpoint{0.000000in}{0.000000in}}{\pgfqpoint{0.034722in}{0.000000in}}{%
\pgfpathmoveto{\pgfqpoint{0.000000in}{0.000000in}}%
\pgfpathlineto{\pgfqpoint{0.034722in}{0.000000in}}%
\pgfusepath{stroke,fill}%
}%
\begin{pgfscope}%
\pgfsys@transformshift{3.008890in}{7.508704in}%
\pgfsys@useobject{currentmarker}{}%
\end{pgfscope}%
\end{pgfscope}%
\begin{pgfscope}%
\definecolor{textcolor}{rgb}{0.000000,0.000000,0.000000}%
\pgfsetstrokecolor{textcolor}%
\pgfsetfillcolor{textcolor}%
\pgftext[x=2.046647in, y=7.413733in, left, base]{\color{textcolor}\sffamily\fontsize{18.000000}{21.600000}\selectfont $\displaystyle 1.50×10^{7}$}%
\end{pgfscope}%
\begin{pgfscope}%
\pgfpathrectangle{\pgfqpoint{3.008890in}{1.247073in}}{\pgfqpoint{8.912370in}{6.674186in}}%
\pgfusepath{clip}%
\pgfsetbuttcap%
\pgfsetroundjoin%
\pgfsetlinewidth{1.003750pt}%
\definecolor{currentstroke}{rgb}{0.000000,0.605603,0.978680}%
\pgfsetstrokecolor{currentstroke}%
\pgfsetdash{}{0pt}%
\pgfpathmoveto{\pgfqpoint{3.261127in}{4.554960in}}%
\pgfpathlineto{\pgfqpoint{6.085654in}{4.553946in}}%
\pgfpathlineto{\pgfqpoint{6.151341in}{4.569840in}}%
\pgfpathlineto{\pgfqpoint{6.217028in}{4.454585in}}%
\pgfpathlineto{\pgfqpoint{6.282714in}{4.965723in}}%
\pgfpathlineto{\pgfqpoint{6.348401in}{3.421930in}}%
\pgfpathlineto{\pgfqpoint{6.414088in}{6.766894in}}%
\pgfpathlineto{\pgfqpoint{6.479774in}{1.435966in}}%
\pgfpathlineto{\pgfqpoint{6.545461in}{7.732368in}}%
\pgfpathlineto{\pgfqpoint{6.611148in}{2.262319in}}%
\pgfpathlineto{\pgfqpoint{6.676835in}{5.671435in}}%
\pgfpathlineto{\pgfqpoint{6.742521in}{4.224444in}}%
\pgfpathlineto{\pgfqpoint{6.808208in}{4.600071in}}%
\pgfpathlineto{\pgfqpoint{6.873895in}{4.554959in}}%
\pgfpathlineto{\pgfqpoint{11.669023in}{4.554959in}}%
\pgfpathlineto{\pgfqpoint{11.669023in}{4.554959in}}%
\pgfusepath{stroke}%
\end{pgfscope}%
\begin{pgfscope}%
\pgfpathrectangle{\pgfqpoint{3.008890in}{1.247073in}}{\pgfqpoint{8.912370in}{6.674186in}}%
\pgfusepath{clip}%
\pgfsetbuttcap%
\pgfsetroundjoin%
\pgfsetlinewidth{1.003750pt}%
\definecolor{currentstroke}{rgb}{0.888874,0.435649,0.278123}%
\pgfsetstrokecolor{currentstroke}%
\pgfsetdash{}{0pt}%
\pgfpathmoveto{\pgfqpoint{3.261127in}{4.554960in}}%
\pgfpathlineto{\pgfqpoint{11.669023in}{4.554959in}}%
\pgfpathlineto{\pgfqpoint{11.669023in}{4.554959in}}%
\pgfusepath{stroke}%
\end{pgfscope}%
\begin{pgfscope}%
\pgfsetrectcap%
\pgfsetmiterjoin%
\pgfsetlinewidth{1.003750pt}%
\definecolor{currentstroke}{rgb}{0.000000,0.000000,0.000000}%
\pgfsetstrokecolor{currentstroke}%
\pgfsetdash{}{0pt}%
\pgfpathmoveto{\pgfqpoint{3.008890in}{1.247073in}}%
\pgfpathlineto{\pgfqpoint{3.008890in}{7.921260in}}%
\pgfusepath{stroke}%
\end{pgfscope}%
\begin{pgfscope}%
\pgfsetrectcap%
\pgfsetmiterjoin%
\pgfsetlinewidth{1.003750pt}%
\definecolor{currentstroke}{rgb}{0.000000,0.000000,0.000000}%
\pgfsetstrokecolor{currentstroke}%
\pgfsetdash{}{0pt}%
\pgfpathmoveto{\pgfqpoint{3.008890in}{1.247073in}}%
\pgfpathlineto{\pgfqpoint{11.921260in}{1.247073in}}%
\pgfusepath{stroke}%
\end{pgfscope}%
\begin{pgfscope}%
\pgfsetbuttcap%
\pgfsetmiterjoin%
\definecolor{currentfill}{rgb}{1.000000,1.000000,1.000000}%
\pgfsetfillcolor{currentfill}%
\pgfsetlinewidth{1.003750pt}%
\definecolor{currentstroke}{rgb}{0.000000,0.000000,0.000000}%
\pgfsetstrokecolor{currentstroke}%
\pgfsetdash{}{0pt}%
\pgfpathmoveto{\pgfqpoint{10.511589in}{6.787373in}}%
\pgfpathlineto{\pgfqpoint{11.796260in}{6.787373in}}%
\pgfpathlineto{\pgfqpoint{11.796260in}{7.796260in}}%
\pgfpathlineto{\pgfqpoint{10.511589in}{7.796260in}}%
\pgfpathclose%
\pgfusepath{stroke,fill}%
\end{pgfscope}%
\begin{pgfscope}%
\pgfsetbuttcap%
\pgfsetmiterjoin%
\pgfsetlinewidth{2.258437pt}%
\definecolor{currentstroke}{rgb}{0.000000,0.605603,0.978680}%
\pgfsetstrokecolor{currentstroke}%
\pgfsetdash{}{0pt}%
\pgfpathmoveto{\pgfqpoint{10.711589in}{7.493818in}}%
\pgfpathlineto{\pgfqpoint{11.211589in}{7.493818in}}%
\pgfusepath{stroke}%
\end{pgfscope}%
\begin{pgfscope}%
\definecolor{textcolor}{rgb}{0.000000,0.000000,0.000000}%
\pgfsetstrokecolor{textcolor}%
\pgfsetfillcolor{textcolor}%
\pgftext[x=11.411589in,y=7.406318in,left,base]{\color{textcolor}\sffamily\fontsize{18.000000}{21.600000}\selectfont $\displaystyle U$}%
\end{pgfscope}%
\begin{pgfscope}%
\pgfsetbuttcap%
\pgfsetmiterjoin%
\pgfsetlinewidth{2.258437pt}%
\definecolor{currentstroke}{rgb}{0.888874,0.435649,0.278123}%
\pgfsetstrokecolor{currentstroke}%
\pgfsetdash{}{0pt}%
\pgfpathmoveto{\pgfqpoint{10.711589in}{7.126875in}}%
\pgfpathlineto{\pgfqpoint{11.211589in}{7.126875in}}%
\pgfusepath{stroke}%
\end{pgfscope}%
\begin{pgfscope}%
\definecolor{textcolor}{rgb}{0.000000,0.000000,0.000000}%
\pgfsetstrokecolor{textcolor}%
\pgfsetfillcolor{textcolor}%
\pgftext[x=11.411589in,y=7.039375in,left,base]{\color{textcolor}\sffamily\fontsize{18.000000}{21.600000}\selectfont $\displaystyle u$}%
\end{pgfscope}%
\end{pgfpicture}%
\makeatother%
\endgroup%
}\quad
	\resizebox{0.4\linewidth}{!}{%% Creator: Matplotlib, PGF backend
%%
%% To include the figure in your LaTeX document, write
%%   \input{<filename>.pgf}
%%
%% Make sure the required packages are loaded in your preamble
%%   \usepackage{pgf}
%%
%% Figures using additional raster images can only be included by \input if
%% they are in the same directory as the main LaTeX file. For loading figures
%% from other directories you can use the `import` package
%%   \usepackage{import}
%%
%% and then include the figures with
%%   \import{<path to file>}{<filename>.pgf}
%%
%% Matplotlib used the following preamble
%%   \usepackage{fontspec}
%%   \setmainfont{DejaVuSerif.ttf}[Path=\detokenize{/Users/quejiahao/.julia/conda/3/lib/python3.9/site-packages/matplotlib/mpl-data/fonts/ttf/}]
%%   \setsansfont{DejaVuSans.ttf}[Path=\detokenize{/Users/quejiahao/.julia/conda/3/lib/python3.9/site-packages/matplotlib/mpl-data/fonts/ttf/}]
%%   \setmonofont{DejaVuSansMono.ttf}[Path=\detokenize{/Users/quejiahao/.julia/conda/3/lib/python3.9/site-packages/matplotlib/mpl-data/fonts/ttf/}]
%%
\begingroup%
\makeatletter%
\begin{pgfpicture}%
\pgfpathrectangle{\pgfpointorigin}{\pgfqpoint{12.000000in}{8.000000in}}%
\pgfusepath{use as bounding box, clip}%
\begin{pgfscope}%
\pgfsetbuttcap%
\pgfsetmiterjoin%
\definecolor{currentfill}{rgb}{1.000000,1.000000,1.000000}%
\pgfsetfillcolor{currentfill}%
\pgfsetlinewidth{0.000000pt}%
\definecolor{currentstroke}{rgb}{1.000000,1.000000,1.000000}%
\pgfsetstrokecolor{currentstroke}%
\pgfsetdash{}{0pt}%
\pgfpathmoveto{\pgfqpoint{0.000000in}{0.000000in}}%
\pgfpathlineto{\pgfqpoint{12.000000in}{0.000000in}}%
\pgfpathlineto{\pgfqpoint{12.000000in}{8.000000in}}%
\pgfpathlineto{\pgfqpoint{0.000000in}{8.000000in}}%
\pgfpathclose%
\pgfusepath{fill}%
\end{pgfscope}%
\begin{pgfscope}%
\pgfsetbuttcap%
\pgfsetmiterjoin%
\definecolor{currentfill}{rgb}{1.000000,1.000000,1.000000}%
\pgfsetfillcolor{currentfill}%
\pgfsetlinewidth{0.000000pt}%
\definecolor{currentstroke}{rgb}{0.000000,0.000000,0.000000}%
\pgfsetstrokecolor{currentstroke}%
\pgfsetstrokeopacity{0.000000}%
\pgfsetdash{}{0pt}%
\pgfpathmoveto{\pgfqpoint{3.137698in}{1.247073in}}%
\pgfpathlineto{\pgfqpoint{11.921260in}{1.247073in}}%
\pgfpathlineto{\pgfqpoint{11.921260in}{7.921260in}}%
\pgfpathlineto{\pgfqpoint{3.137698in}{7.921260in}}%
\pgfpathclose%
\pgfusepath{fill}%
\end{pgfscope}%
\begin{pgfscope}%
\pgfpathrectangle{\pgfqpoint{3.137698in}{1.247073in}}{\pgfqpoint{8.783562in}{6.674186in}}%
\pgfusepath{clip}%
\pgfsetrectcap%
\pgfsetroundjoin%
\pgfsetlinewidth{0.501875pt}%
\definecolor{currentstroke}{rgb}{0.000000,0.000000,0.000000}%
\pgfsetstrokecolor{currentstroke}%
\pgfsetstrokeopacity{0.100000}%
\pgfsetdash{}{0pt}%
\pgfpathmoveto{\pgfqpoint{3.386290in}{1.247073in}}%
\pgfpathlineto{\pgfqpoint{3.386290in}{7.921260in}}%
\pgfusepath{stroke}%
\end{pgfscope}%
\begin{pgfscope}%
\pgfsetbuttcap%
\pgfsetroundjoin%
\definecolor{currentfill}{rgb}{0.000000,0.000000,0.000000}%
\pgfsetfillcolor{currentfill}%
\pgfsetlinewidth{0.501875pt}%
\definecolor{currentstroke}{rgb}{0.000000,0.000000,0.000000}%
\pgfsetstrokecolor{currentstroke}%
\pgfsetdash{}{0pt}%
\pgfsys@defobject{currentmarker}{\pgfqpoint{0.000000in}{0.000000in}}{\pgfqpoint{0.000000in}{0.034722in}}{%
\pgfpathmoveto{\pgfqpoint{0.000000in}{0.000000in}}%
\pgfpathlineto{\pgfqpoint{0.000000in}{0.034722in}}%
\pgfusepath{stroke,fill}%
}%
\begin{pgfscope}%
\pgfsys@transformshift{3.386290in}{1.247073in}%
\pgfsys@useobject{currentmarker}{}%
\end{pgfscope}%
\end{pgfscope}%
\begin{pgfscope}%
\definecolor{textcolor}{rgb}{0.000000,0.000000,0.000000}%
\pgfsetstrokecolor{textcolor}%
\pgfsetfillcolor{textcolor}%
\pgftext[x=3.386290in,y=1.198462in,,top]{\color{textcolor}\sffamily\fontsize{18.000000}{21.600000}\selectfont $\displaystyle 0$}%
\end{pgfscope}%
\begin{pgfscope}%
\pgfpathrectangle{\pgfqpoint{3.137698in}{1.247073in}}{\pgfqpoint{8.783562in}{6.674186in}}%
\pgfusepath{clip}%
\pgfsetrectcap%
\pgfsetroundjoin%
\pgfsetlinewidth{0.501875pt}%
\definecolor{currentstroke}{rgb}{0.000000,0.000000,0.000000}%
\pgfsetstrokecolor{currentstroke}%
\pgfsetstrokeopacity{0.100000}%
\pgfsetdash{}{0pt}%
\pgfpathmoveto{\pgfqpoint{4.705108in}{1.247073in}}%
\pgfpathlineto{\pgfqpoint{4.705108in}{7.921260in}}%
\pgfusepath{stroke}%
\end{pgfscope}%
\begin{pgfscope}%
\pgfsetbuttcap%
\pgfsetroundjoin%
\definecolor{currentfill}{rgb}{0.000000,0.000000,0.000000}%
\pgfsetfillcolor{currentfill}%
\pgfsetlinewidth{0.501875pt}%
\definecolor{currentstroke}{rgb}{0.000000,0.000000,0.000000}%
\pgfsetstrokecolor{currentstroke}%
\pgfsetdash{}{0pt}%
\pgfsys@defobject{currentmarker}{\pgfqpoint{0.000000in}{0.000000in}}{\pgfqpoint{0.000000in}{0.034722in}}{%
\pgfpathmoveto{\pgfqpoint{0.000000in}{0.000000in}}%
\pgfpathlineto{\pgfqpoint{0.000000in}{0.034722in}}%
\pgfusepath{stroke,fill}%
}%
\begin{pgfscope}%
\pgfsys@transformshift{4.705108in}{1.247073in}%
\pgfsys@useobject{currentmarker}{}%
\end{pgfscope}%
\end{pgfscope}%
\begin{pgfscope}%
\definecolor{textcolor}{rgb}{0.000000,0.000000,0.000000}%
\pgfsetstrokecolor{textcolor}%
\pgfsetfillcolor{textcolor}%
\pgftext[x=4.705108in,y=1.198462in,,top]{\color{textcolor}\sffamily\fontsize{18.000000}{21.600000}\selectfont $\displaystyle 1$}%
\end{pgfscope}%
\begin{pgfscope}%
\pgfpathrectangle{\pgfqpoint{3.137698in}{1.247073in}}{\pgfqpoint{8.783562in}{6.674186in}}%
\pgfusepath{clip}%
\pgfsetrectcap%
\pgfsetroundjoin%
\pgfsetlinewidth{0.501875pt}%
\definecolor{currentstroke}{rgb}{0.000000,0.000000,0.000000}%
\pgfsetstrokecolor{currentstroke}%
\pgfsetstrokeopacity{0.100000}%
\pgfsetdash{}{0pt}%
\pgfpathmoveto{\pgfqpoint{6.023926in}{1.247073in}}%
\pgfpathlineto{\pgfqpoint{6.023926in}{7.921260in}}%
\pgfusepath{stroke}%
\end{pgfscope}%
\begin{pgfscope}%
\pgfsetbuttcap%
\pgfsetroundjoin%
\definecolor{currentfill}{rgb}{0.000000,0.000000,0.000000}%
\pgfsetfillcolor{currentfill}%
\pgfsetlinewidth{0.501875pt}%
\definecolor{currentstroke}{rgb}{0.000000,0.000000,0.000000}%
\pgfsetstrokecolor{currentstroke}%
\pgfsetdash{}{0pt}%
\pgfsys@defobject{currentmarker}{\pgfqpoint{0.000000in}{0.000000in}}{\pgfqpoint{0.000000in}{0.034722in}}{%
\pgfpathmoveto{\pgfqpoint{0.000000in}{0.000000in}}%
\pgfpathlineto{\pgfqpoint{0.000000in}{0.034722in}}%
\pgfusepath{stroke,fill}%
}%
\begin{pgfscope}%
\pgfsys@transformshift{6.023926in}{1.247073in}%
\pgfsys@useobject{currentmarker}{}%
\end{pgfscope}%
\end{pgfscope}%
\begin{pgfscope}%
\definecolor{textcolor}{rgb}{0.000000,0.000000,0.000000}%
\pgfsetstrokecolor{textcolor}%
\pgfsetfillcolor{textcolor}%
\pgftext[x=6.023926in,y=1.198462in,,top]{\color{textcolor}\sffamily\fontsize{18.000000}{21.600000}\selectfont $\displaystyle 2$}%
\end{pgfscope}%
\begin{pgfscope}%
\pgfpathrectangle{\pgfqpoint{3.137698in}{1.247073in}}{\pgfqpoint{8.783562in}{6.674186in}}%
\pgfusepath{clip}%
\pgfsetrectcap%
\pgfsetroundjoin%
\pgfsetlinewidth{0.501875pt}%
\definecolor{currentstroke}{rgb}{0.000000,0.000000,0.000000}%
\pgfsetstrokecolor{currentstroke}%
\pgfsetstrokeopacity{0.100000}%
\pgfsetdash{}{0pt}%
\pgfpathmoveto{\pgfqpoint{7.342744in}{1.247073in}}%
\pgfpathlineto{\pgfqpoint{7.342744in}{7.921260in}}%
\pgfusepath{stroke}%
\end{pgfscope}%
\begin{pgfscope}%
\pgfsetbuttcap%
\pgfsetroundjoin%
\definecolor{currentfill}{rgb}{0.000000,0.000000,0.000000}%
\pgfsetfillcolor{currentfill}%
\pgfsetlinewidth{0.501875pt}%
\definecolor{currentstroke}{rgb}{0.000000,0.000000,0.000000}%
\pgfsetstrokecolor{currentstroke}%
\pgfsetdash{}{0pt}%
\pgfsys@defobject{currentmarker}{\pgfqpoint{0.000000in}{0.000000in}}{\pgfqpoint{0.000000in}{0.034722in}}{%
\pgfpathmoveto{\pgfqpoint{0.000000in}{0.000000in}}%
\pgfpathlineto{\pgfqpoint{0.000000in}{0.034722in}}%
\pgfusepath{stroke,fill}%
}%
\begin{pgfscope}%
\pgfsys@transformshift{7.342744in}{1.247073in}%
\pgfsys@useobject{currentmarker}{}%
\end{pgfscope}%
\end{pgfscope}%
\begin{pgfscope}%
\definecolor{textcolor}{rgb}{0.000000,0.000000,0.000000}%
\pgfsetstrokecolor{textcolor}%
\pgfsetfillcolor{textcolor}%
\pgftext[x=7.342744in,y=1.198462in,,top]{\color{textcolor}\sffamily\fontsize{18.000000}{21.600000}\selectfont $\displaystyle 3$}%
\end{pgfscope}%
\begin{pgfscope}%
\pgfpathrectangle{\pgfqpoint{3.137698in}{1.247073in}}{\pgfqpoint{8.783562in}{6.674186in}}%
\pgfusepath{clip}%
\pgfsetrectcap%
\pgfsetroundjoin%
\pgfsetlinewidth{0.501875pt}%
\definecolor{currentstroke}{rgb}{0.000000,0.000000,0.000000}%
\pgfsetstrokecolor{currentstroke}%
\pgfsetstrokeopacity{0.100000}%
\pgfsetdash{}{0pt}%
\pgfpathmoveto{\pgfqpoint{8.661562in}{1.247073in}}%
\pgfpathlineto{\pgfqpoint{8.661562in}{7.921260in}}%
\pgfusepath{stroke}%
\end{pgfscope}%
\begin{pgfscope}%
\pgfsetbuttcap%
\pgfsetroundjoin%
\definecolor{currentfill}{rgb}{0.000000,0.000000,0.000000}%
\pgfsetfillcolor{currentfill}%
\pgfsetlinewidth{0.501875pt}%
\definecolor{currentstroke}{rgb}{0.000000,0.000000,0.000000}%
\pgfsetstrokecolor{currentstroke}%
\pgfsetdash{}{0pt}%
\pgfsys@defobject{currentmarker}{\pgfqpoint{0.000000in}{0.000000in}}{\pgfqpoint{0.000000in}{0.034722in}}{%
\pgfpathmoveto{\pgfqpoint{0.000000in}{0.000000in}}%
\pgfpathlineto{\pgfqpoint{0.000000in}{0.034722in}}%
\pgfusepath{stroke,fill}%
}%
\begin{pgfscope}%
\pgfsys@transformshift{8.661562in}{1.247073in}%
\pgfsys@useobject{currentmarker}{}%
\end{pgfscope}%
\end{pgfscope}%
\begin{pgfscope}%
\definecolor{textcolor}{rgb}{0.000000,0.000000,0.000000}%
\pgfsetstrokecolor{textcolor}%
\pgfsetfillcolor{textcolor}%
\pgftext[x=8.661562in,y=1.198462in,,top]{\color{textcolor}\sffamily\fontsize{18.000000}{21.600000}\selectfont $\displaystyle 4$}%
\end{pgfscope}%
\begin{pgfscope}%
\pgfpathrectangle{\pgfqpoint{3.137698in}{1.247073in}}{\pgfqpoint{8.783562in}{6.674186in}}%
\pgfusepath{clip}%
\pgfsetrectcap%
\pgfsetroundjoin%
\pgfsetlinewidth{0.501875pt}%
\definecolor{currentstroke}{rgb}{0.000000,0.000000,0.000000}%
\pgfsetstrokecolor{currentstroke}%
\pgfsetstrokeopacity{0.100000}%
\pgfsetdash{}{0pt}%
\pgfpathmoveto{\pgfqpoint{9.980380in}{1.247073in}}%
\pgfpathlineto{\pgfqpoint{9.980380in}{7.921260in}}%
\pgfusepath{stroke}%
\end{pgfscope}%
\begin{pgfscope}%
\pgfsetbuttcap%
\pgfsetroundjoin%
\definecolor{currentfill}{rgb}{0.000000,0.000000,0.000000}%
\pgfsetfillcolor{currentfill}%
\pgfsetlinewidth{0.501875pt}%
\definecolor{currentstroke}{rgb}{0.000000,0.000000,0.000000}%
\pgfsetstrokecolor{currentstroke}%
\pgfsetdash{}{0pt}%
\pgfsys@defobject{currentmarker}{\pgfqpoint{0.000000in}{0.000000in}}{\pgfqpoint{0.000000in}{0.034722in}}{%
\pgfpathmoveto{\pgfqpoint{0.000000in}{0.000000in}}%
\pgfpathlineto{\pgfqpoint{0.000000in}{0.034722in}}%
\pgfusepath{stroke,fill}%
}%
\begin{pgfscope}%
\pgfsys@transformshift{9.980380in}{1.247073in}%
\pgfsys@useobject{currentmarker}{}%
\end{pgfscope}%
\end{pgfscope}%
\begin{pgfscope}%
\definecolor{textcolor}{rgb}{0.000000,0.000000,0.000000}%
\pgfsetstrokecolor{textcolor}%
\pgfsetfillcolor{textcolor}%
\pgftext[x=9.980380in,y=1.198462in,,top]{\color{textcolor}\sffamily\fontsize{18.000000}{21.600000}\selectfont $\displaystyle 5$}%
\end{pgfscope}%
\begin{pgfscope}%
\pgfpathrectangle{\pgfqpoint{3.137698in}{1.247073in}}{\pgfqpoint{8.783562in}{6.674186in}}%
\pgfusepath{clip}%
\pgfsetrectcap%
\pgfsetroundjoin%
\pgfsetlinewidth{0.501875pt}%
\definecolor{currentstroke}{rgb}{0.000000,0.000000,0.000000}%
\pgfsetstrokecolor{currentstroke}%
\pgfsetstrokeopacity{0.100000}%
\pgfsetdash{}{0pt}%
\pgfpathmoveto{\pgfqpoint{11.299199in}{1.247073in}}%
\pgfpathlineto{\pgfqpoint{11.299199in}{7.921260in}}%
\pgfusepath{stroke}%
\end{pgfscope}%
\begin{pgfscope}%
\pgfsetbuttcap%
\pgfsetroundjoin%
\definecolor{currentfill}{rgb}{0.000000,0.000000,0.000000}%
\pgfsetfillcolor{currentfill}%
\pgfsetlinewidth{0.501875pt}%
\definecolor{currentstroke}{rgb}{0.000000,0.000000,0.000000}%
\pgfsetstrokecolor{currentstroke}%
\pgfsetdash{}{0pt}%
\pgfsys@defobject{currentmarker}{\pgfqpoint{0.000000in}{0.000000in}}{\pgfqpoint{0.000000in}{0.034722in}}{%
\pgfpathmoveto{\pgfqpoint{0.000000in}{0.000000in}}%
\pgfpathlineto{\pgfqpoint{0.000000in}{0.034722in}}%
\pgfusepath{stroke,fill}%
}%
\begin{pgfscope}%
\pgfsys@transformshift{11.299199in}{1.247073in}%
\pgfsys@useobject{currentmarker}{}%
\end{pgfscope}%
\end{pgfscope}%
\begin{pgfscope}%
\definecolor{textcolor}{rgb}{0.000000,0.000000,0.000000}%
\pgfsetstrokecolor{textcolor}%
\pgfsetfillcolor{textcolor}%
\pgftext[x=11.299199in,y=1.198462in,,top]{\color{textcolor}\sffamily\fontsize{18.000000}{21.600000}\selectfont $\displaystyle 6$}%
\end{pgfscope}%
\begin{pgfscope}%
\definecolor{textcolor}{rgb}{0.000000,0.000000,0.000000}%
\pgfsetstrokecolor{textcolor}%
\pgfsetfillcolor{textcolor}%
\pgftext[x=7.529479in,y=0.900964in,,top]{\color{textcolor}\sffamily\fontsize{18.000000}{21.600000}\selectfont $\displaystyle x$}%
\end{pgfscope}%
\begin{pgfscope}%
\pgfpathrectangle{\pgfqpoint{3.137698in}{1.247073in}}{\pgfqpoint{8.783562in}{6.674186in}}%
\pgfusepath{clip}%
\pgfsetrectcap%
\pgfsetroundjoin%
\pgfsetlinewidth{0.501875pt}%
\definecolor{currentstroke}{rgb}{0.000000,0.000000,0.000000}%
\pgfsetstrokecolor{currentstroke}%
\pgfsetstrokeopacity{0.100000}%
\pgfsetdash{}{0pt}%
\pgfpathmoveto{\pgfqpoint{3.137698in}{2.303732in}}%
\pgfpathlineto{\pgfqpoint{11.921260in}{2.303732in}}%
\pgfusepath{stroke}%
\end{pgfscope}%
\begin{pgfscope}%
\pgfsetbuttcap%
\pgfsetroundjoin%
\definecolor{currentfill}{rgb}{0.000000,0.000000,0.000000}%
\pgfsetfillcolor{currentfill}%
\pgfsetlinewidth{0.501875pt}%
\definecolor{currentstroke}{rgb}{0.000000,0.000000,0.000000}%
\pgfsetstrokecolor{currentstroke}%
\pgfsetdash{}{0pt}%
\pgfsys@defobject{currentmarker}{\pgfqpoint{0.000000in}{0.000000in}}{\pgfqpoint{0.034722in}{0.000000in}}{%
\pgfpathmoveto{\pgfqpoint{0.000000in}{0.000000in}}%
\pgfpathlineto{\pgfqpoint{0.034722in}{0.000000in}}%
\pgfusepath{stroke,fill}%
}%
\begin{pgfscope}%
\pgfsys@transformshift{3.137698in}{2.303732in}%
\pgfsys@useobject{currentmarker}{}%
\end{pgfscope}%
\end{pgfscope}%
\begin{pgfscope}%
\definecolor{textcolor}{rgb}{0.000000,0.000000,0.000000}%
\pgfsetstrokecolor{textcolor}%
\pgfsetfillcolor{textcolor}%
\pgftext[x=1.935664in, y=2.208762in, left, base]{\color{textcolor}\sffamily\fontsize{18.000000}{21.600000}\selectfont $\displaystyle -4.0×10^{124}$}%
\end{pgfscope}%
\begin{pgfscope}%
\pgfpathrectangle{\pgfqpoint{3.137698in}{1.247073in}}{\pgfqpoint{8.783562in}{6.674186in}}%
\pgfusepath{clip}%
\pgfsetrectcap%
\pgfsetroundjoin%
\pgfsetlinewidth{0.501875pt}%
\definecolor{currentstroke}{rgb}{0.000000,0.000000,0.000000}%
\pgfsetstrokecolor{currentstroke}%
\pgfsetstrokeopacity{0.100000}%
\pgfsetdash{}{0pt}%
\pgfpathmoveto{\pgfqpoint{3.137698in}{3.443772in}}%
\pgfpathlineto{\pgfqpoint{11.921260in}{3.443772in}}%
\pgfusepath{stroke}%
\end{pgfscope}%
\begin{pgfscope}%
\pgfsetbuttcap%
\pgfsetroundjoin%
\definecolor{currentfill}{rgb}{0.000000,0.000000,0.000000}%
\pgfsetfillcolor{currentfill}%
\pgfsetlinewidth{0.501875pt}%
\definecolor{currentstroke}{rgb}{0.000000,0.000000,0.000000}%
\pgfsetstrokecolor{currentstroke}%
\pgfsetdash{}{0pt}%
\pgfsys@defobject{currentmarker}{\pgfqpoint{0.000000in}{0.000000in}}{\pgfqpoint{0.034722in}{0.000000in}}{%
\pgfpathmoveto{\pgfqpoint{0.000000in}{0.000000in}}%
\pgfpathlineto{\pgfqpoint{0.034722in}{0.000000in}}%
\pgfusepath{stroke,fill}%
}%
\begin{pgfscope}%
\pgfsys@transformshift{3.137698in}{3.443772in}%
\pgfsys@useobject{currentmarker}{}%
\end{pgfscope}%
\end{pgfscope}%
\begin{pgfscope}%
\definecolor{textcolor}{rgb}{0.000000,0.000000,0.000000}%
\pgfsetstrokecolor{textcolor}%
\pgfsetfillcolor{textcolor}%
\pgftext[x=1.935664in, y=3.348801in, left, base]{\color{textcolor}\sffamily\fontsize{18.000000}{21.600000}\selectfont $\displaystyle -2.0×10^{124}$}%
\end{pgfscope}%
\begin{pgfscope}%
\pgfpathrectangle{\pgfqpoint{3.137698in}{1.247073in}}{\pgfqpoint{8.783562in}{6.674186in}}%
\pgfusepath{clip}%
\pgfsetrectcap%
\pgfsetroundjoin%
\pgfsetlinewidth{0.501875pt}%
\definecolor{currentstroke}{rgb}{0.000000,0.000000,0.000000}%
\pgfsetstrokecolor{currentstroke}%
\pgfsetstrokeopacity{0.100000}%
\pgfsetdash{}{0pt}%
\pgfpathmoveto{\pgfqpoint{3.137698in}{4.583811in}}%
\pgfpathlineto{\pgfqpoint{11.921260in}{4.583811in}}%
\pgfusepath{stroke}%
\end{pgfscope}%
\begin{pgfscope}%
\pgfsetbuttcap%
\pgfsetroundjoin%
\definecolor{currentfill}{rgb}{0.000000,0.000000,0.000000}%
\pgfsetfillcolor{currentfill}%
\pgfsetlinewidth{0.501875pt}%
\definecolor{currentstroke}{rgb}{0.000000,0.000000,0.000000}%
\pgfsetstrokecolor{currentstroke}%
\pgfsetdash{}{0pt}%
\pgfsys@defobject{currentmarker}{\pgfqpoint{0.000000in}{0.000000in}}{\pgfqpoint{0.034722in}{0.000000in}}{%
\pgfpathmoveto{\pgfqpoint{0.000000in}{0.000000in}}%
\pgfpathlineto{\pgfqpoint{0.034722in}{0.000000in}}%
\pgfusepath{stroke,fill}%
}%
\begin{pgfscope}%
\pgfsys@transformshift{3.137698in}{4.583811in}%
\pgfsys@useobject{currentmarker}{}%
\end{pgfscope}%
\end{pgfscope}%
\begin{pgfscope}%
\definecolor{textcolor}{rgb}{0.000000,0.000000,0.000000}%
\pgfsetstrokecolor{textcolor}%
\pgfsetfillcolor{textcolor}%
\pgftext[x=2.979019in, y=4.488840in, left, base]{\color{textcolor}\sffamily\fontsize{18.000000}{21.600000}\selectfont $\displaystyle 0$}%
\end{pgfscope}%
\begin{pgfscope}%
\pgfpathrectangle{\pgfqpoint{3.137698in}{1.247073in}}{\pgfqpoint{8.783562in}{6.674186in}}%
\pgfusepath{clip}%
\pgfsetrectcap%
\pgfsetroundjoin%
\pgfsetlinewidth{0.501875pt}%
\definecolor{currentstroke}{rgb}{0.000000,0.000000,0.000000}%
\pgfsetstrokecolor{currentstroke}%
\pgfsetstrokeopacity{0.100000}%
\pgfsetdash{}{0pt}%
\pgfpathmoveto{\pgfqpoint{3.137698in}{5.723850in}}%
\pgfpathlineto{\pgfqpoint{11.921260in}{5.723850in}}%
\pgfusepath{stroke}%
\end{pgfscope}%
\begin{pgfscope}%
\pgfsetbuttcap%
\pgfsetroundjoin%
\definecolor{currentfill}{rgb}{0.000000,0.000000,0.000000}%
\pgfsetfillcolor{currentfill}%
\pgfsetlinewidth{0.501875pt}%
\definecolor{currentstroke}{rgb}{0.000000,0.000000,0.000000}%
\pgfsetstrokecolor{currentstroke}%
\pgfsetdash{}{0pt}%
\pgfsys@defobject{currentmarker}{\pgfqpoint{0.000000in}{0.000000in}}{\pgfqpoint{0.034722in}{0.000000in}}{%
\pgfpathmoveto{\pgfqpoint{0.000000in}{0.000000in}}%
\pgfpathlineto{\pgfqpoint{0.034722in}{0.000000in}}%
\pgfusepath{stroke,fill}%
}%
\begin{pgfscope}%
\pgfsys@transformshift{3.137698in}{5.723850in}%
\pgfsys@useobject{currentmarker}{}%
\end{pgfscope}%
\end{pgfscope}%
\begin{pgfscope}%
\definecolor{textcolor}{rgb}{0.000000,0.000000,0.000000}%
\pgfsetstrokecolor{textcolor}%
\pgfsetfillcolor{textcolor}%
\pgftext[x=2.122331in, y=5.628879in, left, base]{\color{textcolor}\sffamily\fontsize{18.000000}{21.600000}\selectfont $\displaystyle 2.0×10^{124}$}%
\end{pgfscope}%
\begin{pgfscope}%
\pgfpathrectangle{\pgfqpoint{3.137698in}{1.247073in}}{\pgfqpoint{8.783562in}{6.674186in}}%
\pgfusepath{clip}%
\pgfsetrectcap%
\pgfsetroundjoin%
\pgfsetlinewidth{0.501875pt}%
\definecolor{currentstroke}{rgb}{0.000000,0.000000,0.000000}%
\pgfsetstrokecolor{currentstroke}%
\pgfsetstrokeopacity{0.100000}%
\pgfsetdash{}{0pt}%
\pgfpathmoveto{\pgfqpoint{3.137698in}{6.863889in}}%
\pgfpathlineto{\pgfqpoint{11.921260in}{6.863889in}}%
\pgfusepath{stroke}%
\end{pgfscope}%
\begin{pgfscope}%
\pgfsetbuttcap%
\pgfsetroundjoin%
\definecolor{currentfill}{rgb}{0.000000,0.000000,0.000000}%
\pgfsetfillcolor{currentfill}%
\pgfsetlinewidth{0.501875pt}%
\definecolor{currentstroke}{rgb}{0.000000,0.000000,0.000000}%
\pgfsetstrokecolor{currentstroke}%
\pgfsetdash{}{0pt}%
\pgfsys@defobject{currentmarker}{\pgfqpoint{0.000000in}{0.000000in}}{\pgfqpoint{0.034722in}{0.000000in}}{%
\pgfpathmoveto{\pgfqpoint{0.000000in}{0.000000in}}%
\pgfpathlineto{\pgfqpoint{0.034722in}{0.000000in}}%
\pgfusepath{stroke,fill}%
}%
\begin{pgfscope}%
\pgfsys@transformshift{3.137698in}{6.863889in}%
\pgfsys@useobject{currentmarker}{}%
\end{pgfscope}%
\end{pgfscope}%
\begin{pgfscope}%
\definecolor{textcolor}{rgb}{0.000000,0.000000,0.000000}%
\pgfsetstrokecolor{textcolor}%
\pgfsetfillcolor{textcolor}%
\pgftext[x=2.122331in, y=6.768918in, left, base]{\color{textcolor}\sffamily\fontsize{18.000000}{21.600000}\selectfont $\displaystyle 4.0×10^{124}$}%
\end{pgfscope}%
\begin{pgfscope}%
\pgfpathrectangle{\pgfqpoint{3.137698in}{1.247073in}}{\pgfqpoint{8.783562in}{6.674186in}}%
\pgfusepath{clip}%
\pgfsetbuttcap%
\pgfsetroundjoin%
\pgfsetlinewidth{1.003750pt}%
\definecolor{currentstroke}{rgb}{0.000000,0.605603,0.978680}%
\pgfsetstrokecolor{currentstroke}%
\pgfsetdash{}{0pt}%
\pgfpathmoveto{\pgfqpoint{3.386290in}{4.583811in}}%
\pgfpathlineto{\pgfqpoint{6.513912in}{4.583083in}}%
\pgfpathlineto{\pgfqpoint{6.517958in}{4.585080in}}%
\pgfpathlineto{\pgfqpoint{6.522004in}{4.581638in}}%
\pgfpathlineto{\pgfqpoint{6.526050in}{4.587458in}}%
\pgfpathlineto{\pgfqpoint{6.530096in}{4.577804in}}%
\pgfpathlineto{\pgfqpoint{6.534143in}{4.593516in}}%
\pgfpathlineto{\pgfqpoint{6.538189in}{4.568427in}}%
\pgfpathlineto{\pgfqpoint{6.542235in}{4.607734in}}%
\pgfpathlineto{\pgfqpoint{6.546281in}{4.547309in}}%
\pgfpathlineto{\pgfqpoint{6.550327in}{4.638449in}}%
\pgfpathlineto{\pgfqpoint{6.554373in}{4.503565in}}%
\pgfpathlineto{\pgfqpoint{6.558419in}{4.699440in}}%
\pgfpathlineto{\pgfqpoint{6.562465in}{4.420338in}}%
\pgfpathlineto{\pgfqpoint{6.566511in}{4.810563in}}%
\pgfpathlineto{\pgfqpoint{6.570557in}{4.275217in}}%
\pgfpathlineto{\pgfqpoint{6.574603in}{4.995858in}}%
\pgfpathlineto{\pgfqpoint{6.578649in}{4.044018in}}%
\pgfpathlineto{\pgfqpoint{6.582696in}{5.277588in}}%
\pgfpathlineto{\pgfqpoint{6.586742in}{3.708990in}}%
\pgfpathlineto{\pgfqpoint{6.590788in}{5.666024in}}%
\pgfpathlineto{\pgfqpoint{6.594834in}{3.270431in}}%
\pgfpathlineto{\pgfqpoint{6.598880in}{6.147457in}}%
\pgfpathlineto{\pgfqpoint{6.602926in}{2.757634in}}%
\pgfpathlineto{\pgfqpoint{6.606972in}{6.675927in}}%
\pgfpathlineto{\pgfqpoint{6.611018in}{2.232827in}}%
\pgfpathlineto{\pgfqpoint{6.615064in}{7.175089in}}%
\pgfpathlineto{\pgfqpoint{6.619110in}{1.782527in}}%
\pgfpathlineto{\pgfqpoint{6.623156in}{7.553801in}}%
\pgfpathlineto{\pgfqpoint{6.627202in}{1.495774in}}%
\pgfpathlineto{\pgfqpoint{6.631249in}{7.732368in}}%
\pgfpathlineto{\pgfqpoint{6.635295in}{1.435966in}}%
\pgfpathlineto{\pgfqpoint{6.639341in}{7.669529in}}%
\pgfpathlineto{\pgfqpoint{6.643387in}{1.618247in}}%
\pgfpathlineto{\pgfqpoint{6.647433in}{7.377839in}}%
\pgfpathlineto{\pgfqpoint{6.651479in}{2.003395in}}%
\pgfpathlineto{\pgfqpoint{6.655525in}{6.919656in}}%
\pgfpathlineto{\pgfqpoint{6.659571in}{2.511510in}}%
\pgfpathlineto{\pgfqpoint{6.663617in}{6.385465in}}%
\pgfpathlineto{\pgfqpoint{6.667663in}{3.049002in}}%
\pgfpathlineto{\pgfqpoint{6.671709in}{5.864820in}}%
\pgfpathlineto{\pgfqpoint{6.675755in}{3.536406in}}%
\pgfpathlineto{\pgfqpoint{6.679802in}{5.422668in}}%
\pgfpathlineto{\pgfqpoint{6.683848in}{3.925825in}}%
\pgfpathlineto{\pgfqpoint{6.687894in}{5.089217in}}%
\pgfpathlineto{\pgfqpoint{6.691940in}{4.203709in}}%
\pgfpathlineto{\pgfqpoint{6.695986in}{4.863663in}}%
\pgfpathlineto{\pgfqpoint{6.700032in}{4.382131in}}%
\pgfpathlineto{\pgfqpoint{6.704078in}{4.726052in}}%
\pgfpathlineto{\pgfqpoint{6.708124in}{4.485647in}}%
\pgfpathlineto{\pgfqpoint{6.712170in}{4.650087in}}%
\pgfpathlineto{\pgfqpoint{6.716216in}{4.540041in}}%
\pgfpathlineto{\pgfqpoint{6.720262in}{4.612079in}}%
\pgfpathlineto{\pgfqpoint{6.724308in}{4.565960in}}%
\pgfpathlineto{\pgfqpoint{6.728355in}{4.594830in}}%
\pgfpathlineto{\pgfqpoint{6.732401in}{4.577162in}}%
\pgfpathlineto{\pgfqpoint{6.736447in}{4.587730in}}%
\pgfpathlineto{\pgfqpoint{6.740493in}{4.581554in}}%
\pgfpathlineto{\pgfqpoint{6.744539in}{4.585080in}}%
\pgfpathlineto{\pgfqpoint{6.748585in}{4.583114in}}%
\pgfpathlineto{\pgfqpoint{6.752631in}{4.584184in}}%
\pgfpathlineto{\pgfqpoint{6.760723in}{4.583910in}}%
\pgfpathlineto{\pgfqpoint{6.793092in}{4.583811in}}%
\pgfpathlineto{\pgfqpoint{11.672668in}{4.583811in}}%
\pgfpathlineto{\pgfqpoint{11.672668in}{4.583811in}}%
\pgfusepath{stroke}%
\end{pgfscope}%
\begin{pgfscope}%
\pgfpathrectangle{\pgfqpoint{3.137698in}{1.247073in}}{\pgfqpoint{8.783562in}{6.674186in}}%
\pgfusepath{clip}%
\pgfsetbuttcap%
\pgfsetroundjoin%
\pgfsetlinewidth{1.003750pt}%
\definecolor{currentstroke}{rgb}{0.888874,0.435649,0.278123}%
\pgfsetstrokecolor{currentstroke}%
\pgfsetdash{}{0pt}%
\pgfpathmoveto{\pgfqpoint{3.386290in}{4.583811in}}%
\pgfpathlineto{\pgfqpoint{11.672668in}{4.583811in}}%
\pgfpathlineto{\pgfqpoint{11.672668in}{4.583811in}}%
\pgfusepath{stroke}%
\end{pgfscope}%
\begin{pgfscope}%
\pgfsetrectcap%
\pgfsetmiterjoin%
\pgfsetlinewidth{1.003750pt}%
\definecolor{currentstroke}{rgb}{0.000000,0.000000,0.000000}%
\pgfsetstrokecolor{currentstroke}%
\pgfsetdash{}{0pt}%
\pgfpathmoveto{\pgfqpoint{3.137698in}{1.247073in}}%
\pgfpathlineto{\pgfqpoint{3.137698in}{7.921260in}}%
\pgfusepath{stroke}%
\end{pgfscope}%
\begin{pgfscope}%
\pgfsetrectcap%
\pgfsetmiterjoin%
\pgfsetlinewidth{1.003750pt}%
\definecolor{currentstroke}{rgb}{0.000000,0.000000,0.000000}%
\pgfsetstrokecolor{currentstroke}%
\pgfsetdash{}{0pt}%
\pgfpathmoveto{\pgfqpoint{3.137698in}{1.247073in}}%
\pgfpathlineto{\pgfqpoint{11.921260in}{1.247073in}}%
\pgfusepath{stroke}%
\end{pgfscope}%
\begin{pgfscope}%
\pgfsetbuttcap%
\pgfsetmiterjoin%
\definecolor{currentfill}{rgb}{1.000000,1.000000,1.000000}%
\pgfsetfillcolor{currentfill}%
\pgfsetlinewidth{1.003750pt}%
\definecolor{currentstroke}{rgb}{0.000000,0.000000,0.000000}%
\pgfsetstrokecolor{currentstroke}%
\pgfsetdash{}{0pt}%
\pgfpathmoveto{\pgfqpoint{10.511589in}{6.787373in}}%
\pgfpathlineto{\pgfqpoint{11.796260in}{6.787373in}}%
\pgfpathlineto{\pgfqpoint{11.796260in}{7.796260in}}%
\pgfpathlineto{\pgfqpoint{10.511589in}{7.796260in}}%
\pgfpathclose%
\pgfusepath{stroke,fill}%
\end{pgfscope}%
\begin{pgfscope}%
\pgfsetbuttcap%
\pgfsetmiterjoin%
\pgfsetlinewidth{2.258437pt}%
\definecolor{currentstroke}{rgb}{0.000000,0.605603,0.978680}%
\pgfsetstrokecolor{currentstroke}%
\pgfsetdash{}{0pt}%
\pgfpathmoveto{\pgfqpoint{10.711589in}{7.493818in}}%
\pgfpathlineto{\pgfqpoint{11.211589in}{7.493818in}}%
\pgfusepath{stroke}%
\end{pgfscope}%
\begin{pgfscope}%
\definecolor{textcolor}{rgb}{0.000000,0.000000,0.000000}%
\pgfsetstrokecolor{textcolor}%
\pgfsetfillcolor{textcolor}%
\pgftext[x=11.411589in,y=7.406318in,left,base]{\color{textcolor}\sffamily\fontsize{18.000000}{21.600000}\selectfont $\displaystyle U$}%
\end{pgfscope}%
\begin{pgfscope}%
\pgfsetbuttcap%
\pgfsetmiterjoin%
\pgfsetlinewidth{2.258437pt}%
\definecolor{currentstroke}{rgb}{0.888874,0.435649,0.278123}%
\pgfsetstrokecolor{currentstroke}%
\pgfsetdash{}{0pt}%
\pgfpathmoveto{\pgfqpoint{10.711589in}{7.126875in}}%
\pgfpathlineto{\pgfqpoint{11.211589in}{7.126875in}}%
\pgfusepath{stroke}%
\end{pgfscope}%
\begin{pgfscope}%
\definecolor{textcolor}{rgb}{0.000000,0.000000,0.000000}%
\pgfsetstrokecolor{textcolor}%
\pgfsetfillcolor{textcolor}%
\pgftext[x=11.411589in,y=7.039375in,left,base]{\color{textcolor}\sffamily\fontsize{18.000000}{21.600000}\selectfont $\displaystyle u$}%
\end{pgfscope}%
\end{pgfpicture}%
\makeatother%
\endgroup%
}
	\caption{Beam-Warming 差分逼近解 $U$ 与真解 $u$}\label{fig:beam_warming_square_Uu_noCFL}
\end{figure}

\section{Burgers 方程的 Riemann 问题}

\subsection{模型问题}

\begin{equation}\label{equ:burgers_riemann}
	\begin{aligned}
		u_t + \(f(u)\)_x &= 0,	&	x &\in I, t > 0,\\
		f(u) &= \dfrac{1}{2} u^2,	&&\\
		u(x, 0) &= \lb\begin{array}{ll}
			u_l,	&	x < x_0,\\
			u_r,	&	x > x_0,
		\end{array} \rd	&	x &\in I,
	\end{aligned}
\end{equation}
其中 $I \subset \R$ 为开区间. 守恒律方程加上单个间断的分片常值初值称为 Riemann 问题. 因非线性问题的特征线会相交, 故在积分形式下求弱解.

$u_l > u_r$ 时, 有唯一激波弱解
$$u(x, t) = \lb\begin{array}{ll}
	u_l,	&	x - x_0 < st,\\
	u_r,	&	x - x_0 > st,
\end{array}\rd$$
其中, 激波速度 $s = \dfrac{u_l + u_r}{2}$.

$u_l < u_r$ 时, 有稀疏波弱解
$$u(x, t) = \lb\begin{array}{ll}
	u_l,	&	x - x_0 < st,\\
	\dfrac{x - x_0}{t},	&	u_l t \leq x u_r t,\\
	u_r,	&	x - x_0 > st.
\end{array}\rd$$
上述两种弱解均为该情况下的粘性消失广义解.

\subsection{有限体积格式}

在 $\l[x_l, x_r\r] \times [t_a, t_b]$ 上对方程 \eqref{equ:burgers_riemann} 积分, 可得积分形式的守恒律方程:
$$\dint_{x_l}^{x_r} u\(x, t_b\) \di x = \dint_{x_l}^{x_r} u\(x, t_a\) \di x - \( \dint_{t_a}^{t_b} f\( u\( x_r, t\) \) \di t - \dint_{t_a}^{t_b} f\( u\( x_l, t\) \) \di t \).$$
在控制体 $I_j := \l[ x_{j - \frac{1}{2}}, x_{j + \frac{1}{2}} \r]$ 中, 取 $t_a = t_m$, $t_b = t_{m + 1}$, 空间单元积分平均值
$$U_j^m = \dfrac{1}{\D x} \dint_{I_j} u\(x, t^m\) \di x,$$
$x_{j + \frac{1}{2}}$ 上的数值通量为
$$F_{j + \frac{1}{2}}^{m + \frac{1}{2}} = \dfrac{1}{\D t}\dint_{t_m}^{t^{m + 1}} f\( u\( x_{j + \frac{1}{2}}, t \) \) \di t.$$
于是得到守恒差分格式
$$U_j^{m + 1} = U_j^m - \dfrac{\D t}{\D x} \( F_{j + \frac{1}{2}}^{m + \frac{1}{2}} - F_{j - \frac{1}{2}}^{m + \frac{1}{2}} \).$$
其中, $F_{j + \frac{1}{2}}^{m + \frac{1}{2}} := F\( U_{j + 1}^m, U_j^m \)$ 在数值方法中可以有不同的求解, 可以通过求解如下的界面两侧的 Riemann 问题确定.
$$\lb\begin{aligned}
	u_t + \(\dfrac{1}{2} u^2 \)_x &= 0,\\
	u(x, 0) &= \lb\begin{array}{ll}
		u_l,	&	x < 0,\\
		u_r,	&	x > 0,
	\end{array}\rd
\end{aligned} \rd$$
其中 $x = 0$ 代表 $x_{j + \frac{1}{2}}$ 界面, $u_l = U_j$, $u_r = U_{j + 1}$.

\subsection{非守恒形式}

对于问题 \eqref{equ:burgers_riemann}, 从非守恒形式 $u_t + uu_x = 0$ 出发构造迎风格式
\begin{equation}\label{equ:upwind_rie}
	U_j^{m + 1}\lb \begin{aligned}
		U_j^m &- \dfrac{\tau }{h} U_j^m \(U_j^m - U_{j - 1}^m\),	&	U_j^m &\geq 0,\\
		U_j^m &- \dfrac{\tau }{h} U_j^m \(U_{j + 1}^m - U_j^m\),	&	U_j^m &< 0.
	\end{aligned} \rd
\end{equation}
取初值为
$$u(x, 0) = \lb \begin{array}{ll}
	1,	&	x < 0,\\
	0,	&	x \geq 0,
\end{array} \rd$$
此时 $1 =: u_l > u_r := 0$. $\tau \to 0$, $h \to 0$ 时, 数值解
$$u(x, t) = \lb \begin{array}{ll}
	1,	&	x < 0,\\
	0,	&	x \geq 0
\end{array} \rd$$
与 Rankine-Hugoniot 间断跳跃条件下的激波解
$$u(x, t) = \lb \begin{array}{ll}
	1,	&	x < \dfrac{t}{2},\\
	0,	&	x \geq \dfrac{t}{2}
\end{array} \rd$$
矛盾.

\subsection{数值实验}

取 $I = [-4, 4]$, $x_0 = 0$, $t_{\max } = 2^{-5}$, $\D t = 2^{-6}$, 取不同的 $u_l$, $u_r$ 进行数值实验.

\subsubsection{激波}

取 $u_l = 2$, $u_r = 1$, 此时激波速度 $s = 1.5$, 结果如图 \ref{fig:burgers1} 所示, 可以看到是向前传播的.

取 $u_l = -1$, $u_r = -2$, 此时激波速度 $s = -2.5$, 结果如图 \ref{fig:burgers2} 所示, 可以看到是向后传播的.

\begin{figure}[H]\centering\zihao{-5}
	\begin{minipage}[t]{.48\linewidth}
		\resizebox{0.99\linewidth}{!}{\input{burgers1.pgf}}
		\caption{$u_l = 2$, $u_r = 1$}\label{fig:burgers1}
	\end{minipage}
	\begin{minipage}[t]{.48\linewidth}
		\resizebox{0.99\linewidth}{!}{\input{burgers2.pgf}}
		\caption{$u_l = -1$, $u_r = -2$}\label{fig:burgers2}
	\end{minipage}
\end{figure}

\subsubsection{稀疏波}

取 $u_l = 1$, $u_r = 2$, 结果如图 \ref{fig:burgers3} 所示.

取 $u_l = -2$, $u_r = -1$, 结果如图 \ref{fig:burgers4} 所示.

取 $u_l = -1$, $u_r = 2$, 结果如图 \ref{fig:burgers5} 所示, 可以看到一个明显的转折.

\begin{figure}[H]\centering\zihao{-5}
	\begin{minipage}[t]{.32\linewidth}
		\resizebox{0.99\linewidth}{!}{%% Creator: Matplotlib, PGF backend
%%
%% To include the figure in your LaTeX document, write
%%   \input{<filename>.pgf}
%%
%% Make sure the required packages are loaded in your preamble
%%   \usepackage{pgf}
%%
%% Figures using additional raster images can only be included by \input if
%% they are in the same directory as the main LaTeX file. For loading figures
%% from other directories you can use the `import` package
%%   \usepackage{import}
%%
%% and then include the figures with
%%   \import{<path to file>}{<filename>.pgf}
%%
%% Matplotlib used the following preamble
%%   \usepackage{fontspec}
%%   \setmainfont{DejaVuSerif.ttf}[Path=\detokenize{/Users/quejiahao/.julia/conda/3/lib/python3.9/site-packages/matplotlib/mpl-data/fonts/ttf/}]
%%   \setsansfont{DejaVuSans.ttf}[Path=\detokenize{/Users/quejiahao/.julia/conda/3/lib/python3.9/site-packages/matplotlib/mpl-data/fonts/ttf/}]
%%   \setmonofont{DejaVuSansMono.ttf}[Path=\detokenize{/Users/quejiahao/.julia/conda/3/lib/python3.9/site-packages/matplotlib/mpl-data/fonts/ttf/}]
%%
\begingroup%
\makeatletter%
\begin{pgfpicture}%
\pgfpathrectangle{\pgfpointorigin}{\pgfqpoint{12.000000in}{8.000000in}}%
\pgfusepath{use as bounding box, clip}%
\begin{pgfscope}%
\pgfsetbuttcap%
\pgfsetmiterjoin%
\definecolor{currentfill}{rgb}{1.000000,1.000000,1.000000}%
\pgfsetfillcolor{currentfill}%
\pgfsetlinewidth{0.000000pt}%
\definecolor{currentstroke}{rgb}{1.000000,1.000000,1.000000}%
\pgfsetstrokecolor{currentstroke}%
\pgfsetdash{}{0pt}%
\pgfpathmoveto{\pgfqpoint{0.000000in}{0.000000in}}%
\pgfpathlineto{\pgfqpoint{12.000000in}{0.000000in}}%
\pgfpathlineto{\pgfqpoint{12.000000in}{8.000000in}}%
\pgfpathlineto{\pgfqpoint{0.000000in}{8.000000in}}%
\pgfpathclose%
\pgfusepath{fill}%
\end{pgfscope}%
\begin{pgfscope}%
\pgfsetbuttcap%
\pgfsetmiterjoin%
\definecolor{currentfill}{rgb}{1.000000,1.000000,1.000000}%
\pgfsetfillcolor{currentfill}%
\pgfsetlinewidth{0.000000pt}%
\definecolor{currentstroke}{rgb}{0.000000,0.000000,0.000000}%
\pgfsetstrokecolor{currentstroke}%
\pgfsetstrokeopacity{0.000000}%
\pgfsetdash{}{0pt}%
\pgfpathmoveto{\pgfqpoint{1.296128in}{1.247073in}}%
\pgfpathlineto{\pgfqpoint{11.921260in}{1.247073in}}%
\pgfpathlineto{\pgfqpoint{11.921260in}{7.921260in}}%
\pgfpathlineto{\pgfqpoint{1.296128in}{7.921260in}}%
\pgfpathclose%
\pgfusepath{fill}%
\end{pgfscope}%
\begin{pgfscope}%
\pgfpathrectangle{\pgfqpoint{1.296128in}{1.247073in}}{\pgfqpoint{10.625131in}{6.674186in}}%
\pgfusepath{clip}%
\pgfsetrectcap%
\pgfsetroundjoin%
\pgfsetlinewidth{0.501875pt}%
\definecolor{currentstroke}{rgb}{0.000000,0.000000,0.000000}%
\pgfsetstrokecolor{currentstroke}%
\pgfsetstrokeopacity{0.100000}%
\pgfsetdash{}{0pt}%
\pgfpathmoveto{\pgfqpoint{1.596840in}{1.247073in}}%
\pgfpathlineto{\pgfqpoint{1.596840in}{7.921260in}}%
\pgfusepath{stroke}%
\end{pgfscope}%
\begin{pgfscope}%
\pgfsetbuttcap%
\pgfsetroundjoin%
\definecolor{currentfill}{rgb}{0.000000,0.000000,0.000000}%
\pgfsetfillcolor{currentfill}%
\pgfsetlinewidth{0.501875pt}%
\definecolor{currentstroke}{rgb}{0.000000,0.000000,0.000000}%
\pgfsetstrokecolor{currentstroke}%
\pgfsetdash{}{0pt}%
\pgfsys@defobject{currentmarker}{\pgfqpoint{0.000000in}{0.000000in}}{\pgfqpoint{0.000000in}{0.034722in}}{%
\pgfpathmoveto{\pgfqpoint{0.000000in}{0.000000in}}%
\pgfpathlineto{\pgfqpoint{0.000000in}{0.034722in}}%
\pgfusepath{stroke,fill}%
}%
\begin{pgfscope}%
\pgfsys@transformshift{1.596840in}{1.247073in}%
\pgfsys@useobject{currentmarker}{}%
\end{pgfscope}%
\end{pgfscope}%
\begin{pgfscope}%
\definecolor{textcolor}{rgb}{0.000000,0.000000,0.000000}%
\pgfsetstrokecolor{textcolor}%
\pgfsetfillcolor{textcolor}%
\pgftext[x=1.596840in,y=1.198462in,,top]{\color{textcolor}\sffamily\fontsize{18.000000}{21.600000}\selectfont $\displaystyle -4$}%
\end{pgfscope}%
\begin{pgfscope}%
\pgfpathrectangle{\pgfqpoint{1.296128in}{1.247073in}}{\pgfqpoint{10.625131in}{6.674186in}}%
\pgfusepath{clip}%
\pgfsetrectcap%
\pgfsetroundjoin%
\pgfsetlinewidth{0.501875pt}%
\definecolor{currentstroke}{rgb}{0.000000,0.000000,0.000000}%
\pgfsetstrokecolor{currentstroke}%
\pgfsetstrokeopacity{0.100000}%
\pgfsetdash{}{0pt}%
\pgfpathmoveto{\pgfqpoint{4.102767in}{1.247073in}}%
\pgfpathlineto{\pgfqpoint{4.102767in}{7.921260in}}%
\pgfusepath{stroke}%
\end{pgfscope}%
\begin{pgfscope}%
\pgfsetbuttcap%
\pgfsetroundjoin%
\definecolor{currentfill}{rgb}{0.000000,0.000000,0.000000}%
\pgfsetfillcolor{currentfill}%
\pgfsetlinewidth{0.501875pt}%
\definecolor{currentstroke}{rgb}{0.000000,0.000000,0.000000}%
\pgfsetstrokecolor{currentstroke}%
\pgfsetdash{}{0pt}%
\pgfsys@defobject{currentmarker}{\pgfqpoint{0.000000in}{0.000000in}}{\pgfqpoint{0.000000in}{0.034722in}}{%
\pgfpathmoveto{\pgfqpoint{0.000000in}{0.000000in}}%
\pgfpathlineto{\pgfqpoint{0.000000in}{0.034722in}}%
\pgfusepath{stroke,fill}%
}%
\begin{pgfscope}%
\pgfsys@transformshift{4.102767in}{1.247073in}%
\pgfsys@useobject{currentmarker}{}%
\end{pgfscope}%
\end{pgfscope}%
\begin{pgfscope}%
\definecolor{textcolor}{rgb}{0.000000,0.000000,0.000000}%
\pgfsetstrokecolor{textcolor}%
\pgfsetfillcolor{textcolor}%
\pgftext[x=4.102767in,y=1.198462in,,top]{\color{textcolor}\sffamily\fontsize{18.000000}{21.600000}\selectfont $\displaystyle -2$}%
\end{pgfscope}%
\begin{pgfscope}%
\pgfpathrectangle{\pgfqpoint{1.296128in}{1.247073in}}{\pgfqpoint{10.625131in}{6.674186in}}%
\pgfusepath{clip}%
\pgfsetrectcap%
\pgfsetroundjoin%
\pgfsetlinewidth{0.501875pt}%
\definecolor{currentstroke}{rgb}{0.000000,0.000000,0.000000}%
\pgfsetstrokecolor{currentstroke}%
\pgfsetstrokeopacity{0.100000}%
\pgfsetdash{}{0pt}%
\pgfpathmoveto{\pgfqpoint{6.608694in}{1.247073in}}%
\pgfpathlineto{\pgfqpoint{6.608694in}{7.921260in}}%
\pgfusepath{stroke}%
\end{pgfscope}%
\begin{pgfscope}%
\pgfsetbuttcap%
\pgfsetroundjoin%
\definecolor{currentfill}{rgb}{0.000000,0.000000,0.000000}%
\pgfsetfillcolor{currentfill}%
\pgfsetlinewidth{0.501875pt}%
\definecolor{currentstroke}{rgb}{0.000000,0.000000,0.000000}%
\pgfsetstrokecolor{currentstroke}%
\pgfsetdash{}{0pt}%
\pgfsys@defobject{currentmarker}{\pgfqpoint{0.000000in}{0.000000in}}{\pgfqpoint{0.000000in}{0.034722in}}{%
\pgfpathmoveto{\pgfqpoint{0.000000in}{0.000000in}}%
\pgfpathlineto{\pgfqpoint{0.000000in}{0.034722in}}%
\pgfusepath{stroke,fill}%
}%
\begin{pgfscope}%
\pgfsys@transformshift{6.608694in}{1.247073in}%
\pgfsys@useobject{currentmarker}{}%
\end{pgfscope}%
\end{pgfscope}%
\begin{pgfscope}%
\definecolor{textcolor}{rgb}{0.000000,0.000000,0.000000}%
\pgfsetstrokecolor{textcolor}%
\pgfsetfillcolor{textcolor}%
\pgftext[x=6.608694in,y=1.198462in,,top]{\color{textcolor}\sffamily\fontsize{18.000000}{21.600000}\selectfont $\displaystyle 0$}%
\end{pgfscope}%
\begin{pgfscope}%
\pgfpathrectangle{\pgfqpoint{1.296128in}{1.247073in}}{\pgfqpoint{10.625131in}{6.674186in}}%
\pgfusepath{clip}%
\pgfsetrectcap%
\pgfsetroundjoin%
\pgfsetlinewidth{0.501875pt}%
\definecolor{currentstroke}{rgb}{0.000000,0.000000,0.000000}%
\pgfsetstrokecolor{currentstroke}%
\pgfsetstrokeopacity{0.100000}%
\pgfsetdash{}{0pt}%
\pgfpathmoveto{\pgfqpoint{9.114621in}{1.247073in}}%
\pgfpathlineto{\pgfqpoint{9.114621in}{7.921260in}}%
\pgfusepath{stroke}%
\end{pgfscope}%
\begin{pgfscope}%
\pgfsetbuttcap%
\pgfsetroundjoin%
\definecolor{currentfill}{rgb}{0.000000,0.000000,0.000000}%
\pgfsetfillcolor{currentfill}%
\pgfsetlinewidth{0.501875pt}%
\definecolor{currentstroke}{rgb}{0.000000,0.000000,0.000000}%
\pgfsetstrokecolor{currentstroke}%
\pgfsetdash{}{0pt}%
\pgfsys@defobject{currentmarker}{\pgfqpoint{0.000000in}{0.000000in}}{\pgfqpoint{0.000000in}{0.034722in}}{%
\pgfpathmoveto{\pgfqpoint{0.000000in}{0.000000in}}%
\pgfpathlineto{\pgfqpoint{0.000000in}{0.034722in}}%
\pgfusepath{stroke,fill}%
}%
\begin{pgfscope}%
\pgfsys@transformshift{9.114621in}{1.247073in}%
\pgfsys@useobject{currentmarker}{}%
\end{pgfscope}%
\end{pgfscope}%
\begin{pgfscope}%
\definecolor{textcolor}{rgb}{0.000000,0.000000,0.000000}%
\pgfsetstrokecolor{textcolor}%
\pgfsetfillcolor{textcolor}%
\pgftext[x=9.114621in,y=1.198462in,,top]{\color{textcolor}\sffamily\fontsize{18.000000}{21.600000}\selectfont $\displaystyle 2$}%
\end{pgfscope}%
\begin{pgfscope}%
\pgfpathrectangle{\pgfqpoint{1.296128in}{1.247073in}}{\pgfqpoint{10.625131in}{6.674186in}}%
\pgfusepath{clip}%
\pgfsetrectcap%
\pgfsetroundjoin%
\pgfsetlinewidth{0.501875pt}%
\definecolor{currentstroke}{rgb}{0.000000,0.000000,0.000000}%
\pgfsetstrokecolor{currentstroke}%
\pgfsetstrokeopacity{0.100000}%
\pgfsetdash{}{0pt}%
\pgfpathmoveto{\pgfqpoint{11.620549in}{1.247073in}}%
\pgfpathlineto{\pgfqpoint{11.620549in}{7.921260in}}%
\pgfusepath{stroke}%
\end{pgfscope}%
\begin{pgfscope}%
\pgfsetbuttcap%
\pgfsetroundjoin%
\definecolor{currentfill}{rgb}{0.000000,0.000000,0.000000}%
\pgfsetfillcolor{currentfill}%
\pgfsetlinewidth{0.501875pt}%
\definecolor{currentstroke}{rgb}{0.000000,0.000000,0.000000}%
\pgfsetstrokecolor{currentstroke}%
\pgfsetdash{}{0pt}%
\pgfsys@defobject{currentmarker}{\pgfqpoint{0.000000in}{0.000000in}}{\pgfqpoint{0.000000in}{0.034722in}}{%
\pgfpathmoveto{\pgfqpoint{0.000000in}{0.000000in}}%
\pgfpathlineto{\pgfqpoint{0.000000in}{0.034722in}}%
\pgfusepath{stroke,fill}%
}%
\begin{pgfscope}%
\pgfsys@transformshift{11.620549in}{1.247073in}%
\pgfsys@useobject{currentmarker}{}%
\end{pgfscope}%
\end{pgfscope}%
\begin{pgfscope}%
\definecolor{textcolor}{rgb}{0.000000,0.000000,0.000000}%
\pgfsetstrokecolor{textcolor}%
\pgfsetfillcolor{textcolor}%
\pgftext[x=11.620549in,y=1.198462in,,top]{\color{textcolor}\sffamily\fontsize{18.000000}{21.600000}\selectfont $\displaystyle 4$}%
\end{pgfscope}%
\begin{pgfscope}%
\definecolor{textcolor}{rgb}{0.000000,0.000000,0.000000}%
\pgfsetstrokecolor{textcolor}%
\pgfsetfillcolor{textcolor}%
\pgftext[x=6.608694in,y=0.900964in,,top]{\color{textcolor}\sffamily\fontsize{18.000000}{21.600000}\selectfont $\displaystyle x$}%
\end{pgfscope}%
\begin{pgfscope}%
\pgfpathrectangle{\pgfqpoint{1.296128in}{1.247073in}}{\pgfqpoint{10.625131in}{6.674186in}}%
\pgfusepath{clip}%
\pgfsetrectcap%
\pgfsetroundjoin%
\pgfsetlinewidth{0.501875pt}%
\definecolor{currentstroke}{rgb}{0.000000,0.000000,0.000000}%
\pgfsetstrokecolor{currentstroke}%
\pgfsetstrokeopacity{0.100000}%
\pgfsetdash{}{0pt}%
\pgfpathmoveto{\pgfqpoint{1.296128in}{1.435966in}}%
\pgfpathlineto{\pgfqpoint{11.921260in}{1.435966in}}%
\pgfusepath{stroke}%
\end{pgfscope}%
\begin{pgfscope}%
\pgfsetbuttcap%
\pgfsetroundjoin%
\definecolor{currentfill}{rgb}{0.000000,0.000000,0.000000}%
\pgfsetfillcolor{currentfill}%
\pgfsetlinewidth{0.501875pt}%
\definecolor{currentstroke}{rgb}{0.000000,0.000000,0.000000}%
\pgfsetstrokecolor{currentstroke}%
\pgfsetdash{}{0pt}%
\pgfsys@defobject{currentmarker}{\pgfqpoint{0.000000in}{0.000000in}}{\pgfqpoint{0.034722in}{0.000000in}}{%
\pgfpathmoveto{\pgfqpoint{0.000000in}{0.000000in}}%
\pgfpathlineto{\pgfqpoint{0.034722in}{0.000000in}}%
\pgfusepath{stroke,fill}%
}%
\begin{pgfscope}%
\pgfsys@transformshift{1.296128in}{1.435966in}%
\pgfsys@useobject{currentmarker}{}%
\end{pgfscope}%
\end{pgfscope}%
\begin{pgfscope}%
\definecolor{textcolor}{rgb}{0.000000,0.000000,0.000000}%
\pgfsetstrokecolor{textcolor}%
\pgfsetfillcolor{textcolor}%
\pgftext[x=0.852036in, y=1.340995in, left, base]{\color{textcolor}\sffamily\fontsize{18.000000}{21.600000}\selectfont $\displaystyle 1.00$}%
\end{pgfscope}%
\begin{pgfscope}%
\pgfpathrectangle{\pgfqpoint{1.296128in}{1.247073in}}{\pgfqpoint{10.625131in}{6.674186in}}%
\pgfusepath{clip}%
\pgfsetrectcap%
\pgfsetroundjoin%
\pgfsetlinewidth{0.501875pt}%
\definecolor{currentstroke}{rgb}{0.000000,0.000000,0.000000}%
\pgfsetstrokecolor{currentstroke}%
\pgfsetstrokeopacity{0.100000}%
\pgfsetdash{}{0pt}%
\pgfpathmoveto{\pgfqpoint{1.296128in}{3.010066in}}%
\pgfpathlineto{\pgfqpoint{11.921260in}{3.010066in}}%
\pgfusepath{stroke}%
\end{pgfscope}%
\begin{pgfscope}%
\pgfsetbuttcap%
\pgfsetroundjoin%
\definecolor{currentfill}{rgb}{0.000000,0.000000,0.000000}%
\pgfsetfillcolor{currentfill}%
\pgfsetlinewidth{0.501875pt}%
\definecolor{currentstroke}{rgb}{0.000000,0.000000,0.000000}%
\pgfsetstrokecolor{currentstroke}%
\pgfsetdash{}{0pt}%
\pgfsys@defobject{currentmarker}{\pgfqpoint{0.000000in}{0.000000in}}{\pgfqpoint{0.034722in}{0.000000in}}{%
\pgfpathmoveto{\pgfqpoint{0.000000in}{0.000000in}}%
\pgfpathlineto{\pgfqpoint{0.034722in}{0.000000in}}%
\pgfusepath{stroke,fill}%
}%
\begin{pgfscope}%
\pgfsys@transformshift{1.296128in}{3.010066in}%
\pgfsys@useobject{currentmarker}{}%
\end{pgfscope}%
\end{pgfscope}%
\begin{pgfscope}%
\definecolor{textcolor}{rgb}{0.000000,0.000000,0.000000}%
\pgfsetstrokecolor{textcolor}%
\pgfsetfillcolor{textcolor}%
\pgftext[x=0.852036in, y=2.915095in, left, base]{\color{textcolor}\sffamily\fontsize{18.000000}{21.600000}\selectfont $\displaystyle 1.25$}%
\end{pgfscope}%
\begin{pgfscope}%
\pgfpathrectangle{\pgfqpoint{1.296128in}{1.247073in}}{\pgfqpoint{10.625131in}{6.674186in}}%
\pgfusepath{clip}%
\pgfsetrectcap%
\pgfsetroundjoin%
\pgfsetlinewidth{0.501875pt}%
\definecolor{currentstroke}{rgb}{0.000000,0.000000,0.000000}%
\pgfsetstrokecolor{currentstroke}%
\pgfsetstrokeopacity{0.100000}%
\pgfsetdash{}{0pt}%
\pgfpathmoveto{\pgfqpoint{1.296128in}{4.584167in}}%
\pgfpathlineto{\pgfqpoint{11.921260in}{4.584167in}}%
\pgfusepath{stroke}%
\end{pgfscope}%
\begin{pgfscope}%
\pgfsetbuttcap%
\pgfsetroundjoin%
\definecolor{currentfill}{rgb}{0.000000,0.000000,0.000000}%
\pgfsetfillcolor{currentfill}%
\pgfsetlinewidth{0.501875pt}%
\definecolor{currentstroke}{rgb}{0.000000,0.000000,0.000000}%
\pgfsetstrokecolor{currentstroke}%
\pgfsetdash{}{0pt}%
\pgfsys@defobject{currentmarker}{\pgfqpoint{0.000000in}{0.000000in}}{\pgfqpoint{0.034722in}{0.000000in}}{%
\pgfpathmoveto{\pgfqpoint{0.000000in}{0.000000in}}%
\pgfpathlineto{\pgfqpoint{0.034722in}{0.000000in}}%
\pgfusepath{stroke,fill}%
}%
\begin{pgfscope}%
\pgfsys@transformshift{1.296128in}{4.584167in}%
\pgfsys@useobject{currentmarker}{}%
\end{pgfscope}%
\end{pgfscope}%
\begin{pgfscope}%
\definecolor{textcolor}{rgb}{0.000000,0.000000,0.000000}%
\pgfsetstrokecolor{textcolor}%
\pgfsetfillcolor{textcolor}%
\pgftext[x=0.852036in, y=4.489196in, left, base]{\color{textcolor}\sffamily\fontsize{18.000000}{21.600000}\selectfont $\displaystyle 1.50$}%
\end{pgfscope}%
\begin{pgfscope}%
\pgfpathrectangle{\pgfqpoint{1.296128in}{1.247073in}}{\pgfqpoint{10.625131in}{6.674186in}}%
\pgfusepath{clip}%
\pgfsetrectcap%
\pgfsetroundjoin%
\pgfsetlinewidth{0.501875pt}%
\definecolor{currentstroke}{rgb}{0.000000,0.000000,0.000000}%
\pgfsetstrokecolor{currentstroke}%
\pgfsetstrokeopacity{0.100000}%
\pgfsetdash{}{0pt}%
\pgfpathmoveto{\pgfqpoint{1.296128in}{6.158267in}}%
\pgfpathlineto{\pgfqpoint{11.921260in}{6.158267in}}%
\pgfusepath{stroke}%
\end{pgfscope}%
\begin{pgfscope}%
\pgfsetbuttcap%
\pgfsetroundjoin%
\definecolor{currentfill}{rgb}{0.000000,0.000000,0.000000}%
\pgfsetfillcolor{currentfill}%
\pgfsetlinewidth{0.501875pt}%
\definecolor{currentstroke}{rgb}{0.000000,0.000000,0.000000}%
\pgfsetstrokecolor{currentstroke}%
\pgfsetdash{}{0pt}%
\pgfsys@defobject{currentmarker}{\pgfqpoint{0.000000in}{0.000000in}}{\pgfqpoint{0.034722in}{0.000000in}}{%
\pgfpathmoveto{\pgfqpoint{0.000000in}{0.000000in}}%
\pgfpathlineto{\pgfqpoint{0.034722in}{0.000000in}}%
\pgfusepath{stroke,fill}%
}%
\begin{pgfscope}%
\pgfsys@transformshift{1.296128in}{6.158267in}%
\pgfsys@useobject{currentmarker}{}%
\end{pgfscope}%
\end{pgfscope}%
\begin{pgfscope}%
\definecolor{textcolor}{rgb}{0.000000,0.000000,0.000000}%
\pgfsetstrokecolor{textcolor}%
\pgfsetfillcolor{textcolor}%
\pgftext[x=0.852036in, y=6.063297in, left, base]{\color{textcolor}\sffamily\fontsize{18.000000}{21.600000}\selectfont $\displaystyle 1.75$}%
\end{pgfscope}%
\begin{pgfscope}%
\pgfpathrectangle{\pgfqpoint{1.296128in}{1.247073in}}{\pgfqpoint{10.625131in}{6.674186in}}%
\pgfusepath{clip}%
\pgfsetrectcap%
\pgfsetroundjoin%
\pgfsetlinewidth{0.501875pt}%
\definecolor{currentstroke}{rgb}{0.000000,0.000000,0.000000}%
\pgfsetstrokecolor{currentstroke}%
\pgfsetstrokeopacity{0.100000}%
\pgfsetdash{}{0pt}%
\pgfpathmoveto{\pgfqpoint{1.296128in}{7.732368in}}%
\pgfpathlineto{\pgfqpoint{11.921260in}{7.732368in}}%
\pgfusepath{stroke}%
\end{pgfscope}%
\begin{pgfscope}%
\pgfsetbuttcap%
\pgfsetroundjoin%
\definecolor{currentfill}{rgb}{0.000000,0.000000,0.000000}%
\pgfsetfillcolor{currentfill}%
\pgfsetlinewidth{0.501875pt}%
\definecolor{currentstroke}{rgb}{0.000000,0.000000,0.000000}%
\pgfsetstrokecolor{currentstroke}%
\pgfsetdash{}{0pt}%
\pgfsys@defobject{currentmarker}{\pgfqpoint{0.000000in}{0.000000in}}{\pgfqpoint{0.034722in}{0.000000in}}{%
\pgfpathmoveto{\pgfqpoint{0.000000in}{0.000000in}}%
\pgfpathlineto{\pgfqpoint{0.034722in}{0.000000in}}%
\pgfusepath{stroke,fill}%
}%
\begin{pgfscope}%
\pgfsys@transformshift{1.296128in}{7.732368in}%
\pgfsys@useobject{currentmarker}{}%
\end{pgfscope}%
\end{pgfscope}%
\begin{pgfscope}%
\definecolor{textcolor}{rgb}{0.000000,0.000000,0.000000}%
\pgfsetstrokecolor{textcolor}%
\pgfsetfillcolor{textcolor}%
\pgftext[x=0.852036in, y=7.637397in, left, base]{\color{textcolor}\sffamily\fontsize{18.000000}{21.600000}\selectfont $\displaystyle 2.00$}%
\end{pgfscope}%
\begin{pgfscope}%
\pgfpathrectangle{\pgfqpoint{1.296128in}{1.247073in}}{\pgfqpoint{10.625131in}{6.674186in}}%
\pgfusepath{clip}%
\pgfsetbuttcap%
\pgfsetroundjoin%
\pgfsetlinewidth{1.003750pt}%
\definecolor{currentstroke}{rgb}{0.000000,0.605603,0.978680}%
\pgfsetstrokecolor{currentstroke}%
\pgfsetdash{}{0pt}%
\pgfpathmoveto{\pgfqpoint{1.596840in}{1.435966in}}%
\pgfpathlineto{\pgfqpoint{7.313486in}{1.436489in}}%
\pgfpathlineto{\pgfqpoint{7.391796in}{1.438860in}}%
\pgfpathlineto{\pgfqpoint{7.430951in}{1.442083in}}%
\pgfpathlineto{\pgfqpoint{7.470107in}{1.448052in}}%
\pgfpathlineto{\pgfqpoint{7.509262in}{1.458341in}}%
\pgfpathlineto{\pgfqpoint{7.548417in}{1.474902in}}%
\pgfpathlineto{\pgfqpoint{7.587572in}{1.499882in}}%
\pgfpathlineto{\pgfqpoint{7.626727in}{1.535336in}}%
\pgfpathlineto{\pgfqpoint{7.665882in}{1.582943in}}%
\pgfpathlineto{\pgfqpoint{7.705037in}{1.643773in}}%
\pgfpathlineto{\pgfqpoint{7.744192in}{1.718201in}}%
\pgfpathlineto{\pgfqpoint{7.783348in}{1.805949in}}%
\pgfpathlineto{\pgfqpoint{7.822503in}{1.906231in}}%
\pgfpathlineto{\pgfqpoint{7.861658in}{2.017929in}}%
\pgfpathlineto{\pgfqpoint{7.900813in}{2.139767in}}%
\pgfpathlineto{\pgfqpoint{7.939968in}{2.270441in}}%
\pgfpathlineto{\pgfqpoint{7.979123in}{2.408712in}}%
\pgfpathlineto{\pgfqpoint{8.018278in}{2.553458in}}%
\pgfpathlineto{\pgfqpoint{8.057433in}{2.703692in}}%
\pgfpathlineto{\pgfqpoint{8.135744in}{3.017372in}}%
\pgfpathlineto{\pgfqpoint{8.214054in}{3.344471in}}%
\pgfpathlineto{\pgfqpoint{8.292364in}{3.681339in}}%
\pgfpathlineto{\pgfqpoint{8.409829in}{4.199795in}}%
\pgfpathlineto{\pgfqpoint{8.527295in}{4.729790in}}%
\pgfpathlineto{\pgfqpoint{8.644760in}{5.268903in}}%
\pgfpathlineto{\pgfqpoint{8.762225in}{5.816749in}}%
\pgfpathlineto{\pgfqpoint{8.879691in}{6.375574in}}%
\pgfpathlineto{\pgfqpoint{8.958001in}{6.758149in}}%
\pgfpathlineto{\pgfqpoint{8.997156in}{6.954967in}}%
\pgfpathlineto{\pgfqpoint{9.036311in}{7.158753in}}%
\pgfpathlineto{\pgfqpoint{9.075466in}{7.377760in}}%
\pgfpathlineto{\pgfqpoint{9.114621in}{7.732368in}}%
\pgfpathlineto{\pgfqpoint{11.620549in}{7.732368in}}%
\pgfpathlineto{\pgfqpoint{11.620549in}{7.732368in}}%
\pgfusepath{stroke}%
\end{pgfscope}%
\begin{pgfscope}%
\pgfpathrectangle{\pgfqpoint{1.296128in}{1.247073in}}{\pgfqpoint{10.625131in}{6.674186in}}%
\pgfusepath{clip}%
\pgfsetbuttcap%
\pgfsetroundjoin%
\pgfsetlinewidth{1.003750pt}%
\definecolor{currentstroke}{rgb}{0.888874,0.435649,0.278123}%
\pgfsetstrokecolor{currentstroke}%
\pgfsetdash{}{0pt}%
\pgfpathmoveto{\pgfqpoint{1.596840in}{1.435966in}}%
\pgfpathlineto{\pgfqpoint{7.861658in}{1.435966in}}%
\pgfpathlineto{\pgfqpoint{9.114621in}{7.732368in}}%
\pgfpathlineto{\pgfqpoint{11.620549in}{7.732368in}}%
\pgfpathlineto{\pgfqpoint{11.620549in}{7.732368in}}%
\pgfusepath{stroke}%
\end{pgfscope}%
\begin{pgfscope}%
\pgfsetrectcap%
\pgfsetmiterjoin%
\pgfsetlinewidth{1.003750pt}%
\definecolor{currentstroke}{rgb}{0.000000,0.000000,0.000000}%
\pgfsetstrokecolor{currentstroke}%
\pgfsetdash{}{0pt}%
\pgfpathmoveto{\pgfqpoint{1.296128in}{1.247073in}}%
\pgfpathlineto{\pgfqpoint{1.296128in}{7.921260in}}%
\pgfusepath{stroke}%
\end{pgfscope}%
\begin{pgfscope}%
\pgfsetrectcap%
\pgfsetmiterjoin%
\pgfsetlinewidth{1.003750pt}%
\definecolor{currentstroke}{rgb}{0.000000,0.000000,0.000000}%
\pgfsetstrokecolor{currentstroke}%
\pgfsetdash{}{0pt}%
\pgfpathmoveto{\pgfqpoint{1.296128in}{1.247073in}}%
\pgfpathlineto{\pgfqpoint{11.921260in}{1.247073in}}%
\pgfusepath{stroke}%
\end{pgfscope}%
\begin{pgfscope}%
\pgfsetbuttcap%
\pgfsetmiterjoin%
\definecolor{currentfill}{rgb}{1.000000,1.000000,1.000000}%
\pgfsetfillcolor{currentfill}%
\pgfsetlinewidth{1.003750pt}%
\definecolor{currentstroke}{rgb}{0.000000,0.000000,0.000000}%
\pgfsetstrokecolor{currentstroke}%
\pgfsetdash{}{0pt}%
\pgfpathmoveto{\pgfqpoint{10.511589in}{6.787373in}}%
\pgfpathlineto{\pgfqpoint{11.796260in}{6.787373in}}%
\pgfpathlineto{\pgfqpoint{11.796260in}{7.796260in}}%
\pgfpathlineto{\pgfqpoint{10.511589in}{7.796260in}}%
\pgfpathclose%
\pgfusepath{stroke,fill}%
\end{pgfscope}%
\begin{pgfscope}%
\pgfsetbuttcap%
\pgfsetmiterjoin%
\pgfsetlinewidth{2.258437pt}%
\definecolor{currentstroke}{rgb}{0.000000,0.605603,0.978680}%
\pgfsetstrokecolor{currentstroke}%
\pgfsetdash{}{0pt}%
\pgfpathmoveto{\pgfqpoint{10.711589in}{7.493818in}}%
\pgfpathlineto{\pgfqpoint{11.211589in}{7.493818in}}%
\pgfusepath{stroke}%
\end{pgfscope}%
\begin{pgfscope}%
\definecolor{textcolor}{rgb}{0.000000,0.000000,0.000000}%
\pgfsetstrokecolor{textcolor}%
\pgfsetfillcolor{textcolor}%
\pgftext[x=11.411589in,y=7.406318in,left,base]{\color{textcolor}\sffamily\fontsize{18.000000}{21.600000}\selectfont $\displaystyle U$}%
\end{pgfscope}%
\begin{pgfscope}%
\pgfsetbuttcap%
\pgfsetmiterjoin%
\pgfsetlinewidth{2.258437pt}%
\definecolor{currentstroke}{rgb}{0.888874,0.435649,0.278123}%
\pgfsetstrokecolor{currentstroke}%
\pgfsetdash{}{0pt}%
\pgfpathmoveto{\pgfqpoint{10.711589in}{7.126875in}}%
\pgfpathlineto{\pgfqpoint{11.211589in}{7.126875in}}%
\pgfusepath{stroke}%
\end{pgfscope}%
\begin{pgfscope}%
\definecolor{textcolor}{rgb}{0.000000,0.000000,0.000000}%
\pgfsetstrokecolor{textcolor}%
\pgfsetfillcolor{textcolor}%
\pgftext[x=11.411589in,y=7.039375in,left,base]{\color{textcolor}\sffamily\fontsize{18.000000}{21.600000}\selectfont $\displaystyle u$}%
\end{pgfscope}%
\end{pgfpicture}%
\makeatother%
\endgroup%
}
		\caption{$u_l = 1$, $u_r = 2$}\label{fig:burgers3}
	\end{minipage}
	\begin{minipage}[t]{.32\linewidth}
		\resizebox{0.99\linewidth}{!}{%% Creator: Matplotlib, PGF backend
%%
%% To include the figure in your LaTeX document, write
%%   \input{<filename>.pgf}
%%
%% Make sure the required packages are loaded in your preamble
%%   \usepackage{pgf}
%%
%% Figures using additional raster images can only be included by \input if
%% they are in the same directory as the main LaTeX file. For loading figures
%% from other directories you can use the `import` package
%%   \usepackage{import}
%%
%% and then include the figures with
%%   \import{<path to file>}{<filename>.pgf}
%%
%% Matplotlib used the following preamble
%%   \usepackage{fontspec}
%%   \setmainfont{DejaVuSerif.ttf}[Path=\detokenize{/Users/quejiahao/.julia/conda/3/lib/python3.9/site-packages/matplotlib/mpl-data/fonts/ttf/}]
%%   \setsansfont{DejaVuSans.ttf}[Path=\detokenize{/Users/quejiahao/.julia/conda/3/lib/python3.9/site-packages/matplotlib/mpl-data/fonts/ttf/}]
%%   \setmonofont{DejaVuSansMono.ttf}[Path=\detokenize{/Users/quejiahao/.julia/conda/3/lib/python3.9/site-packages/matplotlib/mpl-data/fonts/ttf/}]
%%
\begingroup%
\makeatletter%
\begin{pgfpicture}%
\pgfpathrectangle{\pgfpointorigin}{\pgfqpoint{12.000000in}{8.000000in}}%
\pgfusepath{use as bounding box, clip}%
\begin{pgfscope}%
\pgfsetbuttcap%
\pgfsetmiterjoin%
\definecolor{currentfill}{rgb}{1.000000,1.000000,1.000000}%
\pgfsetfillcolor{currentfill}%
\pgfsetlinewidth{0.000000pt}%
\definecolor{currentstroke}{rgb}{1.000000,1.000000,1.000000}%
\pgfsetstrokecolor{currentstroke}%
\pgfsetdash{}{0pt}%
\pgfpathmoveto{\pgfqpoint{0.000000in}{0.000000in}}%
\pgfpathlineto{\pgfqpoint{12.000000in}{0.000000in}}%
\pgfpathlineto{\pgfqpoint{12.000000in}{8.000000in}}%
\pgfpathlineto{\pgfqpoint{0.000000in}{8.000000in}}%
\pgfpathclose%
\pgfusepath{fill}%
\end{pgfscope}%
\begin{pgfscope}%
\pgfsetbuttcap%
\pgfsetmiterjoin%
\definecolor{currentfill}{rgb}{1.000000,1.000000,1.000000}%
\pgfsetfillcolor{currentfill}%
\pgfsetlinewidth{0.000000pt}%
\definecolor{currentstroke}{rgb}{0.000000,0.000000,0.000000}%
\pgfsetstrokecolor{currentstroke}%
\pgfsetstrokeopacity{0.000000}%
\pgfsetdash{}{0pt}%
\pgfpathmoveto{\pgfqpoint{1.715074in}{1.247073in}}%
\pgfpathlineto{\pgfqpoint{11.921260in}{1.247073in}}%
\pgfpathlineto{\pgfqpoint{11.921260in}{7.921260in}}%
\pgfpathlineto{\pgfqpoint{1.715074in}{7.921260in}}%
\pgfpathclose%
\pgfusepath{fill}%
\end{pgfscope}%
\begin{pgfscope}%
\pgfpathrectangle{\pgfqpoint{1.715074in}{1.247073in}}{\pgfqpoint{10.206186in}{6.674186in}}%
\pgfusepath{clip}%
\pgfsetrectcap%
\pgfsetroundjoin%
\pgfsetlinewidth{0.501875pt}%
\definecolor{currentstroke}{rgb}{0.000000,0.000000,0.000000}%
\pgfsetstrokecolor{currentstroke}%
\pgfsetstrokeopacity{0.100000}%
\pgfsetdash{}{0pt}%
\pgfpathmoveto{\pgfqpoint{2.003928in}{1.247073in}}%
\pgfpathlineto{\pgfqpoint{2.003928in}{7.921260in}}%
\pgfusepath{stroke}%
\end{pgfscope}%
\begin{pgfscope}%
\pgfsetbuttcap%
\pgfsetroundjoin%
\definecolor{currentfill}{rgb}{0.000000,0.000000,0.000000}%
\pgfsetfillcolor{currentfill}%
\pgfsetlinewidth{0.501875pt}%
\definecolor{currentstroke}{rgb}{0.000000,0.000000,0.000000}%
\pgfsetstrokecolor{currentstroke}%
\pgfsetdash{}{0pt}%
\pgfsys@defobject{currentmarker}{\pgfqpoint{0.000000in}{0.000000in}}{\pgfqpoint{0.000000in}{0.034722in}}{%
\pgfpathmoveto{\pgfqpoint{0.000000in}{0.000000in}}%
\pgfpathlineto{\pgfqpoint{0.000000in}{0.034722in}}%
\pgfusepath{stroke,fill}%
}%
\begin{pgfscope}%
\pgfsys@transformshift{2.003928in}{1.247073in}%
\pgfsys@useobject{currentmarker}{}%
\end{pgfscope}%
\end{pgfscope}%
\begin{pgfscope}%
\definecolor{textcolor}{rgb}{0.000000,0.000000,0.000000}%
\pgfsetstrokecolor{textcolor}%
\pgfsetfillcolor{textcolor}%
\pgftext[x=2.003928in,y=1.198462in,,top]{\color{textcolor}\sffamily\fontsize{18.000000}{21.600000}\selectfont $\displaystyle -4$}%
\end{pgfscope}%
\begin{pgfscope}%
\pgfpathrectangle{\pgfqpoint{1.715074in}{1.247073in}}{\pgfqpoint{10.206186in}{6.674186in}}%
\pgfusepath{clip}%
\pgfsetrectcap%
\pgfsetroundjoin%
\pgfsetlinewidth{0.501875pt}%
\definecolor{currentstroke}{rgb}{0.000000,0.000000,0.000000}%
\pgfsetstrokecolor{currentstroke}%
\pgfsetstrokeopacity{0.100000}%
\pgfsetdash{}{0pt}%
\pgfpathmoveto{\pgfqpoint{4.411047in}{1.247073in}}%
\pgfpathlineto{\pgfqpoint{4.411047in}{7.921260in}}%
\pgfusepath{stroke}%
\end{pgfscope}%
\begin{pgfscope}%
\pgfsetbuttcap%
\pgfsetroundjoin%
\definecolor{currentfill}{rgb}{0.000000,0.000000,0.000000}%
\pgfsetfillcolor{currentfill}%
\pgfsetlinewidth{0.501875pt}%
\definecolor{currentstroke}{rgb}{0.000000,0.000000,0.000000}%
\pgfsetstrokecolor{currentstroke}%
\pgfsetdash{}{0pt}%
\pgfsys@defobject{currentmarker}{\pgfqpoint{0.000000in}{0.000000in}}{\pgfqpoint{0.000000in}{0.034722in}}{%
\pgfpathmoveto{\pgfqpoint{0.000000in}{0.000000in}}%
\pgfpathlineto{\pgfqpoint{0.000000in}{0.034722in}}%
\pgfusepath{stroke,fill}%
}%
\begin{pgfscope}%
\pgfsys@transformshift{4.411047in}{1.247073in}%
\pgfsys@useobject{currentmarker}{}%
\end{pgfscope}%
\end{pgfscope}%
\begin{pgfscope}%
\definecolor{textcolor}{rgb}{0.000000,0.000000,0.000000}%
\pgfsetstrokecolor{textcolor}%
\pgfsetfillcolor{textcolor}%
\pgftext[x=4.411047in,y=1.198462in,,top]{\color{textcolor}\sffamily\fontsize{18.000000}{21.600000}\selectfont $\displaystyle -2$}%
\end{pgfscope}%
\begin{pgfscope}%
\pgfpathrectangle{\pgfqpoint{1.715074in}{1.247073in}}{\pgfqpoint{10.206186in}{6.674186in}}%
\pgfusepath{clip}%
\pgfsetrectcap%
\pgfsetroundjoin%
\pgfsetlinewidth{0.501875pt}%
\definecolor{currentstroke}{rgb}{0.000000,0.000000,0.000000}%
\pgfsetstrokecolor{currentstroke}%
\pgfsetstrokeopacity{0.100000}%
\pgfsetdash{}{0pt}%
\pgfpathmoveto{\pgfqpoint{6.818167in}{1.247073in}}%
\pgfpathlineto{\pgfqpoint{6.818167in}{7.921260in}}%
\pgfusepath{stroke}%
\end{pgfscope}%
\begin{pgfscope}%
\pgfsetbuttcap%
\pgfsetroundjoin%
\definecolor{currentfill}{rgb}{0.000000,0.000000,0.000000}%
\pgfsetfillcolor{currentfill}%
\pgfsetlinewidth{0.501875pt}%
\definecolor{currentstroke}{rgb}{0.000000,0.000000,0.000000}%
\pgfsetstrokecolor{currentstroke}%
\pgfsetdash{}{0pt}%
\pgfsys@defobject{currentmarker}{\pgfqpoint{0.000000in}{0.000000in}}{\pgfqpoint{0.000000in}{0.034722in}}{%
\pgfpathmoveto{\pgfqpoint{0.000000in}{0.000000in}}%
\pgfpathlineto{\pgfqpoint{0.000000in}{0.034722in}}%
\pgfusepath{stroke,fill}%
}%
\begin{pgfscope}%
\pgfsys@transformshift{6.818167in}{1.247073in}%
\pgfsys@useobject{currentmarker}{}%
\end{pgfscope}%
\end{pgfscope}%
\begin{pgfscope}%
\definecolor{textcolor}{rgb}{0.000000,0.000000,0.000000}%
\pgfsetstrokecolor{textcolor}%
\pgfsetfillcolor{textcolor}%
\pgftext[x=6.818167in,y=1.198462in,,top]{\color{textcolor}\sffamily\fontsize{18.000000}{21.600000}\selectfont $\displaystyle 0$}%
\end{pgfscope}%
\begin{pgfscope}%
\pgfpathrectangle{\pgfqpoint{1.715074in}{1.247073in}}{\pgfqpoint{10.206186in}{6.674186in}}%
\pgfusepath{clip}%
\pgfsetrectcap%
\pgfsetroundjoin%
\pgfsetlinewidth{0.501875pt}%
\definecolor{currentstroke}{rgb}{0.000000,0.000000,0.000000}%
\pgfsetstrokecolor{currentstroke}%
\pgfsetstrokeopacity{0.100000}%
\pgfsetdash{}{0pt}%
\pgfpathmoveto{\pgfqpoint{9.225286in}{1.247073in}}%
\pgfpathlineto{\pgfqpoint{9.225286in}{7.921260in}}%
\pgfusepath{stroke}%
\end{pgfscope}%
\begin{pgfscope}%
\pgfsetbuttcap%
\pgfsetroundjoin%
\definecolor{currentfill}{rgb}{0.000000,0.000000,0.000000}%
\pgfsetfillcolor{currentfill}%
\pgfsetlinewidth{0.501875pt}%
\definecolor{currentstroke}{rgb}{0.000000,0.000000,0.000000}%
\pgfsetstrokecolor{currentstroke}%
\pgfsetdash{}{0pt}%
\pgfsys@defobject{currentmarker}{\pgfqpoint{0.000000in}{0.000000in}}{\pgfqpoint{0.000000in}{0.034722in}}{%
\pgfpathmoveto{\pgfqpoint{0.000000in}{0.000000in}}%
\pgfpathlineto{\pgfqpoint{0.000000in}{0.034722in}}%
\pgfusepath{stroke,fill}%
}%
\begin{pgfscope}%
\pgfsys@transformshift{9.225286in}{1.247073in}%
\pgfsys@useobject{currentmarker}{}%
\end{pgfscope}%
\end{pgfscope}%
\begin{pgfscope}%
\definecolor{textcolor}{rgb}{0.000000,0.000000,0.000000}%
\pgfsetstrokecolor{textcolor}%
\pgfsetfillcolor{textcolor}%
\pgftext[x=9.225286in,y=1.198462in,,top]{\color{textcolor}\sffamily\fontsize{18.000000}{21.600000}\selectfont $\displaystyle 2$}%
\end{pgfscope}%
\begin{pgfscope}%
\pgfpathrectangle{\pgfqpoint{1.715074in}{1.247073in}}{\pgfqpoint{10.206186in}{6.674186in}}%
\pgfusepath{clip}%
\pgfsetrectcap%
\pgfsetroundjoin%
\pgfsetlinewidth{0.501875pt}%
\definecolor{currentstroke}{rgb}{0.000000,0.000000,0.000000}%
\pgfsetstrokecolor{currentstroke}%
\pgfsetstrokeopacity{0.100000}%
\pgfsetdash{}{0pt}%
\pgfpathmoveto{\pgfqpoint{11.632406in}{1.247073in}}%
\pgfpathlineto{\pgfqpoint{11.632406in}{7.921260in}}%
\pgfusepath{stroke}%
\end{pgfscope}%
\begin{pgfscope}%
\pgfsetbuttcap%
\pgfsetroundjoin%
\definecolor{currentfill}{rgb}{0.000000,0.000000,0.000000}%
\pgfsetfillcolor{currentfill}%
\pgfsetlinewidth{0.501875pt}%
\definecolor{currentstroke}{rgb}{0.000000,0.000000,0.000000}%
\pgfsetstrokecolor{currentstroke}%
\pgfsetdash{}{0pt}%
\pgfsys@defobject{currentmarker}{\pgfqpoint{0.000000in}{0.000000in}}{\pgfqpoint{0.000000in}{0.034722in}}{%
\pgfpathmoveto{\pgfqpoint{0.000000in}{0.000000in}}%
\pgfpathlineto{\pgfqpoint{0.000000in}{0.034722in}}%
\pgfusepath{stroke,fill}%
}%
\begin{pgfscope}%
\pgfsys@transformshift{11.632406in}{1.247073in}%
\pgfsys@useobject{currentmarker}{}%
\end{pgfscope}%
\end{pgfscope}%
\begin{pgfscope}%
\definecolor{textcolor}{rgb}{0.000000,0.000000,0.000000}%
\pgfsetstrokecolor{textcolor}%
\pgfsetfillcolor{textcolor}%
\pgftext[x=11.632406in,y=1.198462in,,top]{\color{textcolor}\sffamily\fontsize{18.000000}{21.600000}\selectfont $\displaystyle 4$}%
\end{pgfscope}%
\begin{pgfscope}%
\definecolor{textcolor}{rgb}{0.000000,0.000000,0.000000}%
\pgfsetstrokecolor{textcolor}%
\pgfsetfillcolor{textcolor}%
\pgftext[x=6.818167in,y=0.900964in,,top]{\color{textcolor}\sffamily\fontsize{18.000000}{21.600000}\selectfont $\displaystyle x$}%
\end{pgfscope}%
\begin{pgfscope}%
\pgfpathrectangle{\pgfqpoint{1.715074in}{1.247073in}}{\pgfqpoint{10.206186in}{6.674186in}}%
\pgfusepath{clip}%
\pgfsetrectcap%
\pgfsetroundjoin%
\pgfsetlinewidth{0.501875pt}%
\definecolor{currentstroke}{rgb}{0.000000,0.000000,0.000000}%
\pgfsetstrokecolor{currentstroke}%
\pgfsetstrokeopacity{0.100000}%
\pgfsetdash{}{0pt}%
\pgfpathmoveto{\pgfqpoint{1.715074in}{1.435966in}}%
\pgfpathlineto{\pgfqpoint{11.921260in}{1.435966in}}%
\pgfusepath{stroke}%
\end{pgfscope}%
\begin{pgfscope}%
\pgfsetbuttcap%
\pgfsetroundjoin%
\definecolor{currentfill}{rgb}{0.000000,0.000000,0.000000}%
\pgfsetfillcolor{currentfill}%
\pgfsetlinewidth{0.501875pt}%
\definecolor{currentstroke}{rgb}{0.000000,0.000000,0.000000}%
\pgfsetstrokecolor{currentstroke}%
\pgfsetdash{}{0pt}%
\pgfsys@defobject{currentmarker}{\pgfqpoint{0.000000in}{0.000000in}}{\pgfqpoint{0.034722in}{0.000000in}}{%
\pgfpathmoveto{\pgfqpoint{0.000000in}{0.000000in}}%
\pgfpathlineto{\pgfqpoint{0.034722in}{0.000000in}}%
\pgfusepath{stroke,fill}%
}%
\begin{pgfscope}%
\pgfsys@transformshift{1.715074in}{1.435966in}%
\pgfsys@useobject{currentmarker}{}%
\end{pgfscope}%
\end{pgfscope}%
\begin{pgfscope}%
\definecolor{textcolor}{rgb}{0.000000,0.000000,0.000000}%
\pgfsetstrokecolor{textcolor}%
\pgfsetfillcolor{textcolor}%
\pgftext[x=1.084314in, y=1.340995in, left, base]{\color{textcolor}\sffamily\fontsize{18.000000}{21.600000}\selectfont $\displaystyle -2.00$}%
\end{pgfscope}%
\begin{pgfscope}%
\pgfpathrectangle{\pgfqpoint{1.715074in}{1.247073in}}{\pgfqpoint{10.206186in}{6.674186in}}%
\pgfusepath{clip}%
\pgfsetrectcap%
\pgfsetroundjoin%
\pgfsetlinewidth{0.501875pt}%
\definecolor{currentstroke}{rgb}{0.000000,0.000000,0.000000}%
\pgfsetstrokecolor{currentstroke}%
\pgfsetstrokeopacity{0.100000}%
\pgfsetdash{}{0pt}%
\pgfpathmoveto{\pgfqpoint{1.715074in}{3.010066in}}%
\pgfpathlineto{\pgfqpoint{11.921260in}{3.010066in}}%
\pgfusepath{stroke}%
\end{pgfscope}%
\begin{pgfscope}%
\pgfsetbuttcap%
\pgfsetroundjoin%
\definecolor{currentfill}{rgb}{0.000000,0.000000,0.000000}%
\pgfsetfillcolor{currentfill}%
\pgfsetlinewidth{0.501875pt}%
\definecolor{currentstroke}{rgb}{0.000000,0.000000,0.000000}%
\pgfsetstrokecolor{currentstroke}%
\pgfsetdash{}{0pt}%
\pgfsys@defobject{currentmarker}{\pgfqpoint{0.000000in}{0.000000in}}{\pgfqpoint{0.034722in}{0.000000in}}{%
\pgfpathmoveto{\pgfqpoint{0.000000in}{0.000000in}}%
\pgfpathlineto{\pgfqpoint{0.034722in}{0.000000in}}%
\pgfusepath{stroke,fill}%
}%
\begin{pgfscope}%
\pgfsys@transformshift{1.715074in}{3.010066in}%
\pgfsys@useobject{currentmarker}{}%
\end{pgfscope}%
\end{pgfscope}%
\begin{pgfscope}%
\definecolor{textcolor}{rgb}{0.000000,0.000000,0.000000}%
\pgfsetstrokecolor{textcolor}%
\pgfsetfillcolor{textcolor}%
\pgftext[x=1.084314in, y=2.915095in, left, base]{\color{textcolor}\sffamily\fontsize{18.000000}{21.600000}\selectfont $\displaystyle -1.75$}%
\end{pgfscope}%
\begin{pgfscope}%
\pgfpathrectangle{\pgfqpoint{1.715074in}{1.247073in}}{\pgfqpoint{10.206186in}{6.674186in}}%
\pgfusepath{clip}%
\pgfsetrectcap%
\pgfsetroundjoin%
\pgfsetlinewidth{0.501875pt}%
\definecolor{currentstroke}{rgb}{0.000000,0.000000,0.000000}%
\pgfsetstrokecolor{currentstroke}%
\pgfsetstrokeopacity{0.100000}%
\pgfsetdash{}{0pt}%
\pgfpathmoveto{\pgfqpoint{1.715074in}{4.584167in}}%
\pgfpathlineto{\pgfqpoint{11.921260in}{4.584167in}}%
\pgfusepath{stroke}%
\end{pgfscope}%
\begin{pgfscope}%
\pgfsetbuttcap%
\pgfsetroundjoin%
\definecolor{currentfill}{rgb}{0.000000,0.000000,0.000000}%
\pgfsetfillcolor{currentfill}%
\pgfsetlinewidth{0.501875pt}%
\definecolor{currentstroke}{rgb}{0.000000,0.000000,0.000000}%
\pgfsetstrokecolor{currentstroke}%
\pgfsetdash{}{0pt}%
\pgfsys@defobject{currentmarker}{\pgfqpoint{0.000000in}{0.000000in}}{\pgfqpoint{0.034722in}{0.000000in}}{%
\pgfpathmoveto{\pgfqpoint{0.000000in}{0.000000in}}%
\pgfpathlineto{\pgfqpoint{0.034722in}{0.000000in}}%
\pgfusepath{stroke,fill}%
}%
\begin{pgfscope}%
\pgfsys@transformshift{1.715074in}{4.584167in}%
\pgfsys@useobject{currentmarker}{}%
\end{pgfscope}%
\end{pgfscope}%
\begin{pgfscope}%
\definecolor{textcolor}{rgb}{0.000000,0.000000,0.000000}%
\pgfsetstrokecolor{textcolor}%
\pgfsetfillcolor{textcolor}%
\pgftext[x=1.084314in, y=4.489196in, left, base]{\color{textcolor}\sffamily\fontsize{18.000000}{21.600000}\selectfont $\displaystyle -1.50$}%
\end{pgfscope}%
\begin{pgfscope}%
\pgfpathrectangle{\pgfqpoint{1.715074in}{1.247073in}}{\pgfqpoint{10.206186in}{6.674186in}}%
\pgfusepath{clip}%
\pgfsetrectcap%
\pgfsetroundjoin%
\pgfsetlinewidth{0.501875pt}%
\definecolor{currentstroke}{rgb}{0.000000,0.000000,0.000000}%
\pgfsetstrokecolor{currentstroke}%
\pgfsetstrokeopacity{0.100000}%
\pgfsetdash{}{0pt}%
\pgfpathmoveto{\pgfqpoint{1.715074in}{6.158267in}}%
\pgfpathlineto{\pgfqpoint{11.921260in}{6.158267in}}%
\pgfusepath{stroke}%
\end{pgfscope}%
\begin{pgfscope}%
\pgfsetbuttcap%
\pgfsetroundjoin%
\definecolor{currentfill}{rgb}{0.000000,0.000000,0.000000}%
\pgfsetfillcolor{currentfill}%
\pgfsetlinewidth{0.501875pt}%
\definecolor{currentstroke}{rgb}{0.000000,0.000000,0.000000}%
\pgfsetstrokecolor{currentstroke}%
\pgfsetdash{}{0pt}%
\pgfsys@defobject{currentmarker}{\pgfqpoint{0.000000in}{0.000000in}}{\pgfqpoint{0.034722in}{0.000000in}}{%
\pgfpathmoveto{\pgfqpoint{0.000000in}{0.000000in}}%
\pgfpathlineto{\pgfqpoint{0.034722in}{0.000000in}}%
\pgfusepath{stroke,fill}%
}%
\begin{pgfscope}%
\pgfsys@transformshift{1.715074in}{6.158267in}%
\pgfsys@useobject{currentmarker}{}%
\end{pgfscope}%
\end{pgfscope}%
\begin{pgfscope}%
\definecolor{textcolor}{rgb}{0.000000,0.000000,0.000000}%
\pgfsetstrokecolor{textcolor}%
\pgfsetfillcolor{textcolor}%
\pgftext[x=1.084314in, y=6.063297in, left, base]{\color{textcolor}\sffamily\fontsize{18.000000}{21.600000}\selectfont $\displaystyle -1.25$}%
\end{pgfscope}%
\begin{pgfscope}%
\pgfpathrectangle{\pgfqpoint{1.715074in}{1.247073in}}{\pgfqpoint{10.206186in}{6.674186in}}%
\pgfusepath{clip}%
\pgfsetrectcap%
\pgfsetroundjoin%
\pgfsetlinewidth{0.501875pt}%
\definecolor{currentstroke}{rgb}{0.000000,0.000000,0.000000}%
\pgfsetstrokecolor{currentstroke}%
\pgfsetstrokeopacity{0.100000}%
\pgfsetdash{}{0pt}%
\pgfpathmoveto{\pgfqpoint{1.715074in}{7.732368in}}%
\pgfpathlineto{\pgfqpoint{11.921260in}{7.732368in}}%
\pgfusepath{stroke}%
\end{pgfscope}%
\begin{pgfscope}%
\pgfsetbuttcap%
\pgfsetroundjoin%
\definecolor{currentfill}{rgb}{0.000000,0.000000,0.000000}%
\pgfsetfillcolor{currentfill}%
\pgfsetlinewidth{0.501875pt}%
\definecolor{currentstroke}{rgb}{0.000000,0.000000,0.000000}%
\pgfsetstrokecolor{currentstroke}%
\pgfsetdash{}{0pt}%
\pgfsys@defobject{currentmarker}{\pgfqpoint{0.000000in}{0.000000in}}{\pgfqpoint{0.034722in}{0.000000in}}{%
\pgfpathmoveto{\pgfqpoint{0.000000in}{0.000000in}}%
\pgfpathlineto{\pgfqpoint{0.034722in}{0.000000in}}%
\pgfusepath{stroke,fill}%
}%
\begin{pgfscope}%
\pgfsys@transformshift{1.715074in}{7.732368in}%
\pgfsys@useobject{currentmarker}{}%
\end{pgfscope}%
\end{pgfscope}%
\begin{pgfscope}%
\definecolor{textcolor}{rgb}{0.000000,0.000000,0.000000}%
\pgfsetstrokecolor{textcolor}%
\pgfsetfillcolor{textcolor}%
\pgftext[x=1.084314in, y=7.637397in, left, base]{\color{textcolor}\sffamily\fontsize{18.000000}{21.600000}\selectfont $\displaystyle -1.00$}%
\end{pgfscope}%
\begin{pgfscope}%
\pgfpathrectangle{\pgfqpoint{1.715074in}{1.247073in}}{\pgfqpoint{10.206186in}{6.674186in}}%
\pgfusepath{clip}%
\pgfsetbuttcap%
\pgfsetroundjoin%
\pgfsetlinewidth{1.003750pt}%
\definecolor{currentstroke}{rgb}{0.000000,0.605603,0.978680}%
\pgfsetstrokecolor{currentstroke}%
\pgfsetdash{}{0pt}%
\pgfpathmoveto{\pgfqpoint{2.003928in}{1.435966in}}%
\pgfpathlineto{\pgfqpoint{4.373436in}{1.435966in}}%
\pgfpathlineto{\pgfqpoint{4.411047in}{1.790573in}}%
\pgfpathlineto{\pgfqpoint{4.448659in}{2.009580in}}%
\pgfpathlineto{\pgfqpoint{4.486270in}{2.213367in}}%
\pgfpathlineto{\pgfqpoint{4.561492in}{2.602872in}}%
\pgfpathlineto{\pgfqpoint{4.636715in}{2.980575in}}%
\pgfpathlineto{\pgfqpoint{4.749549in}{3.535235in}}%
\pgfpathlineto{\pgfqpoint{4.862382in}{4.080082in}}%
\pgfpathlineto{\pgfqpoint{4.975216in}{4.616297in}}%
\pgfpathlineto{\pgfqpoint{5.088050in}{5.142824in}}%
\pgfpathlineto{\pgfqpoint{5.200883in}{5.656473in}}%
\pgfpathlineto{\pgfqpoint{5.276106in}{5.988829in}}%
\pgfpathlineto{\pgfqpoint{5.351328in}{6.309765in}}%
\pgfpathlineto{\pgfqpoint{5.426551in}{6.614875in}}%
\pgfpathlineto{\pgfqpoint{5.464162in}{6.759621in}}%
\pgfpathlineto{\pgfqpoint{5.501773in}{6.897893in}}%
\pgfpathlineto{\pgfqpoint{5.539385in}{7.028567in}}%
\pgfpathlineto{\pgfqpoint{5.576996in}{7.150404in}}%
\pgfpathlineto{\pgfqpoint{5.614607in}{7.262102in}}%
\pgfpathlineto{\pgfqpoint{5.652218in}{7.362384in}}%
\pgfpathlineto{\pgfqpoint{5.689830in}{7.450133in}}%
\pgfpathlineto{\pgfqpoint{5.727441in}{7.524560in}}%
\pgfpathlineto{\pgfqpoint{5.765052in}{7.585390in}}%
\pgfpathlineto{\pgfqpoint{5.802663in}{7.632997in}}%
\pgfpathlineto{\pgfqpoint{5.840275in}{7.668452in}}%
\pgfpathlineto{\pgfqpoint{5.877886in}{7.693431in}}%
\pgfpathlineto{\pgfqpoint{5.915497in}{7.709993in}}%
\pgfpathlineto{\pgfqpoint{5.953108in}{7.720282in}}%
\pgfpathlineto{\pgfqpoint{5.990719in}{7.726250in}}%
\pgfpathlineto{\pgfqpoint{6.028331in}{7.729473in}}%
\pgfpathlineto{\pgfqpoint{6.103553in}{7.731844in}}%
\pgfpathlineto{\pgfqpoint{6.291609in}{7.732366in}}%
\pgfpathlineto{\pgfqpoint{11.632406in}{7.732368in}}%
\pgfpathlineto{\pgfqpoint{11.632406in}{7.732368in}}%
\pgfusepath{stroke}%
\end{pgfscope}%
\begin{pgfscope}%
\pgfpathrectangle{\pgfqpoint{1.715074in}{1.247073in}}{\pgfqpoint{10.206186in}{6.674186in}}%
\pgfusepath{clip}%
\pgfsetbuttcap%
\pgfsetroundjoin%
\pgfsetlinewidth{1.003750pt}%
\definecolor{currentstroke}{rgb}{0.888874,0.435649,0.278123}%
\pgfsetstrokecolor{currentstroke}%
\pgfsetdash{}{0pt}%
\pgfpathmoveto{\pgfqpoint{2.003928in}{1.435966in}}%
\pgfpathlineto{\pgfqpoint{4.411047in}{1.435966in}}%
\pgfpathlineto{\pgfqpoint{5.614607in}{7.732368in}}%
\pgfpathlineto{\pgfqpoint{11.632406in}{7.732368in}}%
\pgfpathlineto{\pgfqpoint{11.632406in}{7.732368in}}%
\pgfusepath{stroke}%
\end{pgfscope}%
\begin{pgfscope}%
\pgfsetrectcap%
\pgfsetmiterjoin%
\pgfsetlinewidth{1.003750pt}%
\definecolor{currentstroke}{rgb}{0.000000,0.000000,0.000000}%
\pgfsetstrokecolor{currentstroke}%
\pgfsetdash{}{0pt}%
\pgfpathmoveto{\pgfqpoint{1.715074in}{1.247073in}}%
\pgfpathlineto{\pgfqpoint{1.715074in}{7.921260in}}%
\pgfusepath{stroke}%
\end{pgfscope}%
\begin{pgfscope}%
\pgfsetrectcap%
\pgfsetmiterjoin%
\pgfsetlinewidth{1.003750pt}%
\definecolor{currentstroke}{rgb}{0.000000,0.000000,0.000000}%
\pgfsetstrokecolor{currentstroke}%
\pgfsetdash{}{0pt}%
\pgfpathmoveto{\pgfqpoint{1.715074in}{1.247073in}}%
\pgfpathlineto{\pgfqpoint{11.921260in}{1.247073in}}%
\pgfusepath{stroke}%
\end{pgfscope}%
\begin{pgfscope}%
\pgfsetbuttcap%
\pgfsetmiterjoin%
\definecolor{currentfill}{rgb}{1.000000,1.000000,1.000000}%
\pgfsetfillcolor{currentfill}%
\pgfsetlinewidth{1.003750pt}%
\definecolor{currentstroke}{rgb}{0.000000,0.000000,0.000000}%
\pgfsetstrokecolor{currentstroke}%
\pgfsetdash{}{0pt}%
\pgfpathmoveto{\pgfqpoint{10.511589in}{6.787373in}}%
\pgfpathlineto{\pgfqpoint{11.796260in}{6.787373in}}%
\pgfpathlineto{\pgfqpoint{11.796260in}{7.796260in}}%
\pgfpathlineto{\pgfqpoint{10.511589in}{7.796260in}}%
\pgfpathclose%
\pgfusepath{stroke,fill}%
\end{pgfscope}%
\begin{pgfscope}%
\pgfsetbuttcap%
\pgfsetmiterjoin%
\pgfsetlinewidth{2.258437pt}%
\definecolor{currentstroke}{rgb}{0.000000,0.605603,0.978680}%
\pgfsetstrokecolor{currentstroke}%
\pgfsetdash{}{0pt}%
\pgfpathmoveto{\pgfqpoint{10.711589in}{7.493818in}}%
\pgfpathlineto{\pgfqpoint{11.211589in}{7.493818in}}%
\pgfusepath{stroke}%
\end{pgfscope}%
\begin{pgfscope}%
\definecolor{textcolor}{rgb}{0.000000,0.000000,0.000000}%
\pgfsetstrokecolor{textcolor}%
\pgfsetfillcolor{textcolor}%
\pgftext[x=11.411589in,y=7.406318in,left,base]{\color{textcolor}\sffamily\fontsize{18.000000}{21.600000}\selectfont $\displaystyle U$}%
\end{pgfscope}%
\begin{pgfscope}%
\pgfsetbuttcap%
\pgfsetmiterjoin%
\pgfsetlinewidth{2.258437pt}%
\definecolor{currentstroke}{rgb}{0.888874,0.435649,0.278123}%
\pgfsetstrokecolor{currentstroke}%
\pgfsetdash{}{0pt}%
\pgfpathmoveto{\pgfqpoint{10.711589in}{7.126875in}}%
\pgfpathlineto{\pgfqpoint{11.211589in}{7.126875in}}%
\pgfusepath{stroke}%
\end{pgfscope}%
\begin{pgfscope}%
\definecolor{textcolor}{rgb}{0.000000,0.000000,0.000000}%
\pgfsetstrokecolor{textcolor}%
\pgfsetfillcolor{textcolor}%
\pgftext[x=11.411589in,y=7.039375in,left,base]{\color{textcolor}\sffamily\fontsize{18.000000}{21.600000}\selectfont $\displaystyle u$}%
\end{pgfscope}%
\end{pgfpicture}%
\makeatother%
\endgroup%
}
		\caption{$u_l = -2$, $u_r = -1$}\label{fig:burgers4}
	\end{minipage}
	\begin{minipage}[t]{.32\linewidth}
		\resizebox{0.99\linewidth}{!}{%% Creator: Matplotlib, PGF backend
%%
%% To include the figure in your LaTeX document, write
%%   \input{<filename>.pgf}
%%
%% Make sure the required packages are loaded in your preamble
%%   \usepackage{pgf}
%%
%% Figures using additional raster images can only be included by \input if
%% they are in the same directory as the main LaTeX file. For loading figures
%% from other directories you can use the `import` package
%%   \usepackage{import}
%%
%% and then include the figures with
%%   \import{<path to file>}{<filename>.pgf}
%%
%% Matplotlib used the following preamble
%%   \usepackage{fontspec}
%%   \setmainfont{DejaVuSerif.ttf}[Path=\detokenize{/Users/quejiahao/.julia/conda/3/lib/python3.9/site-packages/matplotlib/mpl-data/fonts/ttf/}]
%%   \setsansfont{DejaVuSans.ttf}[Path=\detokenize{/Users/quejiahao/.julia/conda/3/lib/python3.9/site-packages/matplotlib/mpl-data/fonts/ttf/}]
%%   \setmonofont{DejaVuSansMono.ttf}[Path=\detokenize{/Users/quejiahao/.julia/conda/3/lib/python3.9/site-packages/matplotlib/mpl-data/fonts/ttf/}]
%%
\begingroup%
\makeatletter%
\begin{pgfpicture}%
\pgfpathrectangle{\pgfpointorigin}{\pgfqpoint{12.000000in}{8.000000in}}%
\pgfusepath{use as bounding box, clip}%
\begin{pgfscope}%
\pgfsetbuttcap%
\pgfsetmiterjoin%
\definecolor{currentfill}{rgb}{1.000000,1.000000,1.000000}%
\pgfsetfillcolor{currentfill}%
\pgfsetlinewidth{0.000000pt}%
\definecolor{currentstroke}{rgb}{1.000000,1.000000,1.000000}%
\pgfsetstrokecolor{currentstroke}%
\pgfsetdash{}{0pt}%
\pgfpathmoveto{\pgfqpoint{0.000000in}{0.000000in}}%
\pgfpathlineto{\pgfqpoint{12.000000in}{0.000000in}}%
\pgfpathlineto{\pgfqpoint{12.000000in}{8.000000in}}%
\pgfpathlineto{\pgfqpoint{0.000000in}{8.000000in}}%
\pgfpathclose%
\pgfusepath{fill}%
\end{pgfscope}%
\begin{pgfscope}%
\pgfsetbuttcap%
\pgfsetmiterjoin%
\definecolor{currentfill}{rgb}{1.000000,1.000000,1.000000}%
\pgfsetfillcolor{currentfill}%
\pgfsetlinewidth{0.000000pt}%
\definecolor{currentstroke}{rgb}{0.000000,0.000000,0.000000}%
\pgfsetstrokecolor{currentstroke}%
\pgfsetstrokeopacity{0.000000}%
\pgfsetdash{}{0pt}%
\pgfpathmoveto{\pgfqpoint{0.907408in}{1.247073in}}%
\pgfpathlineto{\pgfqpoint{11.921260in}{1.247073in}}%
\pgfpathlineto{\pgfqpoint{11.921260in}{7.921260in}}%
\pgfpathlineto{\pgfqpoint{0.907408in}{7.921260in}}%
\pgfpathclose%
\pgfusepath{fill}%
\end{pgfscope}%
\begin{pgfscope}%
\pgfpathrectangle{\pgfqpoint{0.907408in}{1.247073in}}{\pgfqpoint{11.013852in}{6.674186in}}%
\pgfusepath{clip}%
\pgfsetrectcap%
\pgfsetroundjoin%
\pgfsetlinewidth{0.501875pt}%
\definecolor{currentstroke}{rgb}{0.000000,0.000000,0.000000}%
\pgfsetstrokecolor{currentstroke}%
\pgfsetstrokeopacity{0.100000}%
\pgfsetdash{}{0pt}%
\pgfpathmoveto{\pgfqpoint{1.219120in}{1.247073in}}%
\pgfpathlineto{\pgfqpoint{1.219120in}{7.921260in}}%
\pgfusepath{stroke}%
\end{pgfscope}%
\begin{pgfscope}%
\pgfsetbuttcap%
\pgfsetroundjoin%
\definecolor{currentfill}{rgb}{0.000000,0.000000,0.000000}%
\pgfsetfillcolor{currentfill}%
\pgfsetlinewidth{0.501875pt}%
\definecolor{currentstroke}{rgb}{0.000000,0.000000,0.000000}%
\pgfsetstrokecolor{currentstroke}%
\pgfsetdash{}{0pt}%
\pgfsys@defobject{currentmarker}{\pgfqpoint{0.000000in}{0.000000in}}{\pgfqpoint{0.000000in}{0.034722in}}{%
\pgfpathmoveto{\pgfqpoint{0.000000in}{0.000000in}}%
\pgfpathlineto{\pgfqpoint{0.000000in}{0.034722in}}%
\pgfusepath{stroke,fill}%
}%
\begin{pgfscope}%
\pgfsys@transformshift{1.219120in}{1.247073in}%
\pgfsys@useobject{currentmarker}{}%
\end{pgfscope}%
\end{pgfscope}%
\begin{pgfscope}%
\definecolor{textcolor}{rgb}{0.000000,0.000000,0.000000}%
\pgfsetstrokecolor{textcolor}%
\pgfsetfillcolor{textcolor}%
\pgftext[x=1.219120in,y=1.198462in,,top]{\color{textcolor}\sffamily\fontsize{18.000000}{21.600000}\selectfont $\displaystyle -4$}%
\end{pgfscope}%
\begin{pgfscope}%
\pgfpathrectangle{\pgfqpoint{0.907408in}{1.247073in}}{\pgfqpoint{11.013852in}{6.674186in}}%
\pgfusepath{clip}%
\pgfsetrectcap%
\pgfsetroundjoin%
\pgfsetlinewidth{0.501875pt}%
\definecolor{currentstroke}{rgb}{0.000000,0.000000,0.000000}%
\pgfsetstrokecolor{currentstroke}%
\pgfsetstrokeopacity{0.100000}%
\pgfsetdash{}{0pt}%
\pgfpathmoveto{\pgfqpoint{3.816727in}{1.247073in}}%
\pgfpathlineto{\pgfqpoint{3.816727in}{7.921260in}}%
\pgfusepath{stroke}%
\end{pgfscope}%
\begin{pgfscope}%
\pgfsetbuttcap%
\pgfsetroundjoin%
\definecolor{currentfill}{rgb}{0.000000,0.000000,0.000000}%
\pgfsetfillcolor{currentfill}%
\pgfsetlinewidth{0.501875pt}%
\definecolor{currentstroke}{rgb}{0.000000,0.000000,0.000000}%
\pgfsetstrokecolor{currentstroke}%
\pgfsetdash{}{0pt}%
\pgfsys@defobject{currentmarker}{\pgfqpoint{0.000000in}{0.000000in}}{\pgfqpoint{0.000000in}{0.034722in}}{%
\pgfpathmoveto{\pgfqpoint{0.000000in}{0.000000in}}%
\pgfpathlineto{\pgfqpoint{0.000000in}{0.034722in}}%
\pgfusepath{stroke,fill}%
}%
\begin{pgfscope}%
\pgfsys@transformshift{3.816727in}{1.247073in}%
\pgfsys@useobject{currentmarker}{}%
\end{pgfscope}%
\end{pgfscope}%
\begin{pgfscope}%
\definecolor{textcolor}{rgb}{0.000000,0.000000,0.000000}%
\pgfsetstrokecolor{textcolor}%
\pgfsetfillcolor{textcolor}%
\pgftext[x=3.816727in,y=1.198462in,,top]{\color{textcolor}\sffamily\fontsize{18.000000}{21.600000}\selectfont $\displaystyle -2$}%
\end{pgfscope}%
\begin{pgfscope}%
\pgfpathrectangle{\pgfqpoint{0.907408in}{1.247073in}}{\pgfqpoint{11.013852in}{6.674186in}}%
\pgfusepath{clip}%
\pgfsetrectcap%
\pgfsetroundjoin%
\pgfsetlinewidth{0.501875pt}%
\definecolor{currentstroke}{rgb}{0.000000,0.000000,0.000000}%
\pgfsetstrokecolor{currentstroke}%
\pgfsetstrokeopacity{0.100000}%
\pgfsetdash{}{0pt}%
\pgfpathmoveto{\pgfqpoint{6.414334in}{1.247073in}}%
\pgfpathlineto{\pgfqpoint{6.414334in}{7.921260in}}%
\pgfusepath{stroke}%
\end{pgfscope}%
\begin{pgfscope}%
\pgfsetbuttcap%
\pgfsetroundjoin%
\definecolor{currentfill}{rgb}{0.000000,0.000000,0.000000}%
\pgfsetfillcolor{currentfill}%
\pgfsetlinewidth{0.501875pt}%
\definecolor{currentstroke}{rgb}{0.000000,0.000000,0.000000}%
\pgfsetstrokecolor{currentstroke}%
\pgfsetdash{}{0pt}%
\pgfsys@defobject{currentmarker}{\pgfqpoint{0.000000in}{0.000000in}}{\pgfqpoint{0.000000in}{0.034722in}}{%
\pgfpathmoveto{\pgfqpoint{0.000000in}{0.000000in}}%
\pgfpathlineto{\pgfqpoint{0.000000in}{0.034722in}}%
\pgfusepath{stroke,fill}%
}%
\begin{pgfscope}%
\pgfsys@transformshift{6.414334in}{1.247073in}%
\pgfsys@useobject{currentmarker}{}%
\end{pgfscope}%
\end{pgfscope}%
\begin{pgfscope}%
\definecolor{textcolor}{rgb}{0.000000,0.000000,0.000000}%
\pgfsetstrokecolor{textcolor}%
\pgfsetfillcolor{textcolor}%
\pgftext[x=6.414334in,y=1.198462in,,top]{\color{textcolor}\sffamily\fontsize{18.000000}{21.600000}\selectfont $\displaystyle 0$}%
\end{pgfscope}%
\begin{pgfscope}%
\pgfpathrectangle{\pgfqpoint{0.907408in}{1.247073in}}{\pgfqpoint{11.013852in}{6.674186in}}%
\pgfusepath{clip}%
\pgfsetrectcap%
\pgfsetroundjoin%
\pgfsetlinewidth{0.501875pt}%
\definecolor{currentstroke}{rgb}{0.000000,0.000000,0.000000}%
\pgfsetstrokecolor{currentstroke}%
\pgfsetstrokeopacity{0.100000}%
\pgfsetdash{}{0pt}%
\pgfpathmoveto{\pgfqpoint{9.011940in}{1.247073in}}%
\pgfpathlineto{\pgfqpoint{9.011940in}{7.921260in}}%
\pgfusepath{stroke}%
\end{pgfscope}%
\begin{pgfscope}%
\pgfsetbuttcap%
\pgfsetroundjoin%
\definecolor{currentfill}{rgb}{0.000000,0.000000,0.000000}%
\pgfsetfillcolor{currentfill}%
\pgfsetlinewidth{0.501875pt}%
\definecolor{currentstroke}{rgb}{0.000000,0.000000,0.000000}%
\pgfsetstrokecolor{currentstroke}%
\pgfsetdash{}{0pt}%
\pgfsys@defobject{currentmarker}{\pgfqpoint{0.000000in}{0.000000in}}{\pgfqpoint{0.000000in}{0.034722in}}{%
\pgfpathmoveto{\pgfqpoint{0.000000in}{0.000000in}}%
\pgfpathlineto{\pgfqpoint{0.000000in}{0.034722in}}%
\pgfusepath{stroke,fill}%
}%
\begin{pgfscope}%
\pgfsys@transformshift{9.011940in}{1.247073in}%
\pgfsys@useobject{currentmarker}{}%
\end{pgfscope}%
\end{pgfscope}%
\begin{pgfscope}%
\definecolor{textcolor}{rgb}{0.000000,0.000000,0.000000}%
\pgfsetstrokecolor{textcolor}%
\pgfsetfillcolor{textcolor}%
\pgftext[x=9.011940in,y=1.198462in,,top]{\color{textcolor}\sffamily\fontsize{18.000000}{21.600000}\selectfont $\displaystyle 2$}%
\end{pgfscope}%
\begin{pgfscope}%
\pgfpathrectangle{\pgfqpoint{0.907408in}{1.247073in}}{\pgfqpoint{11.013852in}{6.674186in}}%
\pgfusepath{clip}%
\pgfsetrectcap%
\pgfsetroundjoin%
\pgfsetlinewidth{0.501875pt}%
\definecolor{currentstroke}{rgb}{0.000000,0.000000,0.000000}%
\pgfsetstrokecolor{currentstroke}%
\pgfsetstrokeopacity{0.100000}%
\pgfsetdash{}{0pt}%
\pgfpathmoveto{\pgfqpoint{11.609547in}{1.247073in}}%
\pgfpathlineto{\pgfqpoint{11.609547in}{7.921260in}}%
\pgfusepath{stroke}%
\end{pgfscope}%
\begin{pgfscope}%
\pgfsetbuttcap%
\pgfsetroundjoin%
\definecolor{currentfill}{rgb}{0.000000,0.000000,0.000000}%
\pgfsetfillcolor{currentfill}%
\pgfsetlinewidth{0.501875pt}%
\definecolor{currentstroke}{rgb}{0.000000,0.000000,0.000000}%
\pgfsetstrokecolor{currentstroke}%
\pgfsetdash{}{0pt}%
\pgfsys@defobject{currentmarker}{\pgfqpoint{0.000000in}{0.000000in}}{\pgfqpoint{0.000000in}{0.034722in}}{%
\pgfpathmoveto{\pgfqpoint{0.000000in}{0.000000in}}%
\pgfpathlineto{\pgfqpoint{0.000000in}{0.034722in}}%
\pgfusepath{stroke,fill}%
}%
\begin{pgfscope}%
\pgfsys@transformshift{11.609547in}{1.247073in}%
\pgfsys@useobject{currentmarker}{}%
\end{pgfscope}%
\end{pgfscope}%
\begin{pgfscope}%
\definecolor{textcolor}{rgb}{0.000000,0.000000,0.000000}%
\pgfsetstrokecolor{textcolor}%
\pgfsetfillcolor{textcolor}%
\pgftext[x=11.609547in,y=1.198462in,,top]{\color{textcolor}\sffamily\fontsize{18.000000}{21.600000}\selectfont $\displaystyle 4$}%
\end{pgfscope}%
\begin{pgfscope}%
\definecolor{textcolor}{rgb}{0.000000,0.000000,0.000000}%
\pgfsetstrokecolor{textcolor}%
\pgfsetfillcolor{textcolor}%
\pgftext[x=6.414334in,y=0.900964in,,top]{\color{textcolor}\sffamily\fontsize{18.000000}{21.600000}\selectfont $\displaystyle x$}%
\end{pgfscope}%
\begin{pgfscope}%
\pgfpathrectangle{\pgfqpoint{0.907408in}{1.247073in}}{\pgfqpoint{11.013852in}{6.674186in}}%
\pgfusepath{clip}%
\pgfsetrectcap%
\pgfsetroundjoin%
\pgfsetlinewidth{0.501875pt}%
\definecolor{currentstroke}{rgb}{0.000000,0.000000,0.000000}%
\pgfsetstrokecolor{currentstroke}%
\pgfsetstrokeopacity{0.100000}%
\pgfsetdash{}{0pt}%
\pgfpathmoveto{\pgfqpoint{0.907408in}{1.435966in}}%
\pgfpathlineto{\pgfqpoint{11.921260in}{1.435966in}}%
\pgfusepath{stroke}%
\end{pgfscope}%
\begin{pgfscope}%
\pgfsetbuttcap%
\pgfsetroundjoin%
\definecolor{currentfill}{rgb}{0.000000,0.000000,0.000000}%
\pgfsetfillcolor{currentfill}%
\pgfsetlinewidth{0.501875pt}%
\definecolor{currentstroke}{rgb}{0.000000,0.000000,0.000000}%
\pgfsetstrokecolor{currentstroke}%
\pgfsetdash{}{0pt}%
\pgfsys@defobject{currentmarker}{\pgfqpoint{0.000000in}{0.000000in}}{\pgfqpoint{0.034722in}{0.000000in}}{%
\pgfpathmoveto{\pgfqpoint{0.000000in}{0.000000in}}%
\pgfpathlineto{\pgfqpoint{0.034722in}{0.000000in}}%
\pgfusepath{stroke,fill}%
}%
\begin{pgfscope}%
\pgfsys@transformshift{0.907408in}{1.435966in}%
\pgfsys@useobject{currentmarker}{}%
\end{pgfscope}%
\end{pgfscope}%
\begin{pgfscope}%
\definecolor{textcolor}{rgb}{0.000000,0.000000,0.000000}%
\pgfsetstrokecolor{textcolor}%
\pgfsetfillcolor{textcolor}%
\pgftext[x=0.562061in, y=1.340995in, left, base]{\color{textcolor}\sffamily\fontsize{18.000000}{21.600000}\selectfont $\displaystyle -1$}%
\end{pgfscope}%
\begin{pgfscope}%
\pgfpathrectangle{\pgfqpoint{0.907408in}{1.247073in}}{\pgfqpoint{11.013852in}{6.674186in}}%
\pgfusepath{clip}%
\pgfsetrectcap%
\pgfsetroundjoin%
\pgfsetlinewidth{0.501875pt}%
\definecolor{currentstroke}{rgb}{0.000000,0.000000,0.000000}%
\pgfsetstrokecolor{currentstroke}%
\pgfsetstrokeopacity{0.100000}%
\pgfsetdash{}{0pt}%
\pgfpathmoveto{\pgfqpoint{0.907408in}{3.534766in}}%
\pgfpathlineto{\pgfqpoint{11.921260in}{3.534766in}}%
\pgfusepath{stroke}%
\end{pgfscope}%
\begin{pgfscope}%
\pgfsetbuttcap%
\pgfsetroundjoin%
\definecolor{currentfill}{rgb}{0.000000,0.000000,0.000000}%
\pgfsetfillcolor{currentfill}%
\pgfsetlinewidth{0.501875pt}%
\definecolor{currentstroke}{rgb}{0.000000,0.000000,0.000000}%
\pgfsetstrokecolor{currentstroke}%
\pgfsetdash{}{0pt}%
\pgfsys@defobject{currentmarker}{\pgfqpoint{0.000000in}{0.000000in}}{\pgfqpoint{0.034722in}{0.000000in}}{%
\pgfpathmoveto{\pgfqpoint{0.000000in}{0.000000in}}%
\pgfpathlineto{\pgfqpoint{0.034722in}{0.000000in}}%
\pgfusepath{stroke,fill}%
}%
\begin{pgfscope}%
\pgfsys@transformshift{0.907408in}{3.534766in}%
\pgfsys@useobject{currentmarker}{}%
\end{pgfscope}%
\end{pgfscope}%
\begin{pgfscope}%
\definecolor{textcolor}{rgb}{0.000000,0.000000,0.000000}%
\pgfsetstrokecolor{textcolor}%
\pgfsetfillcolor{textcolor}%
\pgftext[x=0.748728in, y=3.439796in, left, base]{\color{textcolor}\sffamily\fontsize{18.000000}{21.600000}\selectfont $\displaystyle 0$}%
\end{pgfscope}%
\begin{pgfscope}%
\pgfpathrectangle{\pgfqpoint{0.907408in}{1.247073in}}{\pgfqpoint{11.013852in}{6.674186in}}%
\pgfusepath{clip}%
\pgfsetrectcap%
\pgfsetroundjoin%
\pgfsetlinewidth{0.501875pt}%
\definecolor{currentstroke}{rgb}{0.000000,0.000000,0.000000}%
\pgfsetstrokecolor{currentstroke}%
\pgfsetstrokeopacity{0.100000}%
\pgfsetdash{}{0pt}%
\pgfpathmoveto{\pgfqpoint{0.907408in}{5.633567in}}%
\pgfpathlineto{\pgfqpoint{11.921260in}{5.633567in}}%
\pgfusepath{stroke}%
\end{pgfscope}%
\begin{pgfscope}%
\pgfsetbuttcap%
\pgfsetroundjoin%
\definecolor{currentfill}{rgb}{0.000000,0.000000,0.000000}%
\pgfsetfillcolor{currentfill}%
\pgfsetlinewidth{0.501875pt}%
\definecolor{currentstroke}{rgb}{0.000000,0.000000,0.000000}%
\pgfsetstrokecolor{currentstroke}%
\pgfsetdash{}{0pt}%
\pgfsys@defobject{currentmarker}{\pgfqpoint{0.000000in}{0.000000in}}{\pgfqpoint{0.034722in}{0.000000in}}{%
\pgfpathmoveto{\pgfqpoint{0.000000in}{0.000000in}}%
\pgfpathlineto{\pgfqpoint{0.034722in}{0.000000in}}%
\pgfusepath{stroke,fill}%
}%
\begin{pgfscope}%
\pgfsys@transformshift{0.907408in}{5.633567in}%
\pgfsys@useobject{currentmarker}{}%
\end{pgfscope}%
\end{pgfscope}%
\begin{pgfscope}%
\definecolor{textcolor}{rgb}{0.000000,0.000000,0.000000}%
\pgfsetstrokecolor{textcolor}%
\pgfsetfillcolor{textcolor}%
\pgftext[x=0.748728in, y=5.538596in, left, base]{\color{textcolor}\sffamily\fontsize{18.000000}{21.600000}\selectfont $\displaystyle 1$}%
\end{pgfscope}%
\begin{pgfscope}%
\pgfpathrectangle{\pgfqpoint{0.907408in}{1.247073in}}{\pgfqpoint{11.013852in}{6.674186in}}%
\pgfusepath{clip}%
\pgfsetrectcap%
\pgfsetroundjoin%
\pgfsetlinewidth{0.501875pt}%
\definecolor{currentstroke}{rgb}{0.000000,0.000000,0.000000}%
\pgfsetstrokecolor{currentstroke}%
\pgfsetstrokeopacity{0.100000}%
\pgfsetdash{}{0pt}%
\pgfpathmoveto{\pgfqpoint{0.907408in}{7.732368in}}%
\pgfpathlineto{\pgfqpoint{11.921260in}{7.732368in}}%
\pgfusepath{stroke}%
\end{pgfscope}%
\begin{pgfscope}%
\pgfsetbuttcap%
\pgfsetroundjoin%
\definecolor{currentfill}{rgb}{0.000000,0.000000,0.000000}%
\pgfsetfillcolor{currentfill}%
\pgfsetlinewidth{0.501875pt}%
\definecolor{currentstroke}{rgb}{0.000000,0.000000,0.000000}%
\pgfsetstrokecolor{currentstroke}%
\pgfsetdash{}{0pt}%
\pgfsys@defobject{currentmarker}{\pgfqpoint{0.000000in}{0.000000in}}{\pgfqpoint{0.034722in}{0.000000in}}{%
\pgfpathmoveto{\pgfqpoint{0.000000in}{0.000000in}}%
\pgfpathlineto{\pgfqpoint{0.034722in}{0.000000in}}%
\pgfusepath{stroke,fill}%
}%
\begin{pgfscope}%
\pgfsys@transformshift{0.907408in}{7.732368in}%
\pgfsys@useobject{currentmarker}{}%
\end{pgfscope}%
\end{pgfscope}%
\begin{pgfscope}%
\definecolor{textcolor}{rgb}{0.000000,0.000000,0.000000}%
\pgfsetstrokecolor{textcolor}%
\pgfsetfillcolor{textcolor}%
\pgftext[x=0.748728in, y=7.637397in, left, base]{\color{textcolor}\sffamily\fontsize{18.000000}{21.600000}\selectfont $\displaystyle 2$}%
\end{pgfscope}%
\begin{pgfscope}%
\pgfpathrectangle{\pgfqpoint{0.907408in}{1.247073in}}{\pgfqpoint{11.013852in}{6.674186in}}%
\pgfusepath{clip}%
\pgfsetbuttcap%
\pgfsetroundjoin%
\pgfsetlinewidth{1.003750pt}%
\definecolor{currentstroke}{rgb}{0.000000,0.605603,0.978680}%
\pgfsetstrokecolor{currentstroke}%
\pgfsetdash{}{0pt}%
\pgfpathmoveto{\pgfqpoint{1.219120in}{1.435966in}}%
\pgfpathlineto{\pgfqpoint{4.628479in}{1.436930in}}%
\pgfpathlineto{\pgfqpoint{4.709654in}{1.439994in}}%
\pgfpathlineto{\pgfqpoint{4.750242in}{1.443424in}}%
\pgfpathlineto{\pgfqpoint{4.790830in}{1.448945in}}%
\pgfpathlineto{\pgfqpoint{4.831417in}{1.457271in}}%
\pgfpathlineto{\pgfqpoint{4.872005in}{1.469089in}}%
\pgfpathlineto{\pgfqpoint{4.912592in}{1.484958in}}%
\pgfpathlineto{\pgfqpoint{4.953180in}{1.505235in}}%
\pgfpathlineto{\pgfqpoint{4.993768in}{1.530044in}}%
\pgfpathlineto{\pgfqpoint{5.034355in}{1.559293in}}%
\pgfpathlineto{\pgfqpoint{5.074943in}{1.592721in}}%
\pgfpathlineto{\pgfqpoint{5.115530in}{1.629953in}}%
\pgfpathlineto{\pgfqpoint{5.156118in}{1.670566in}}%
\pgfpathlineto{\pgfqpoint{5.196706in}{1.714124in}}%
\pgfpathlineto{\pgfqpoint{5.237293in}{1.760214in}}%
\pgfpathlineto{\pgfqpoint{5.277881in}{1.808463in}}%
\pgfpathlineto{\pgfqpoint{5.359056in}{1.910167in}}%
\pgfpathlineto{\pgfqpoint{5.440231in}{2.017145in}}%
\pgfpathlineto{\pgfqpoint{5.521406in}{2.127931in}}%
\pgfpathlineto{\pgfqpoint{5.643169in}{2.299147in}}%
\pgfpathlineto{\pgfqpoint{5.764932in}{2.474656in}}%
\pgfpathlineto{\pgfqpoint{5.927283in}{2.713611in}}%
\pgfpathlineto{\pgfqpoint{6.049045in}{2.896227in}}%
\pgfpathlineto{\pgfqpoint{6.170808in}{3.082502in}}%
\pgfpathlineto{\pgfqpoint{6.251983in}{3.210027in}}%
\pgfpathlineto{\pgfqpoint{6.292571in}{3.275633in}}%
\pgfpathlineto{\pgfqpoint{6.333159in}{3.343561in}}%
\pgfpathlineto{\pgfqpoint{6.373746in}{3.416564in}}%
\pgfpathlineto{\pgfqpoint{6.414334in}{3.654682in}}%
\pgfpathlineto{\pgfqpoint{6.454921in}{3.728800in}}%
\pgfpathlineto{\pgfqpoint{6.495509in}{3.797823in}}%
\pgfpathlineto{\pgfqpoint{6.576684in}{3.929936in}}%
\pgfpathlineto{\pgfqpoint{6.698447in}{4.121766in}}%
\pgfpathlineto{\pgfqpoint{6.860797in}{4.372511in}}%
\pgfpathlineto{\pgfqpoint{7.104323in}{4.743926in}}%
\pgfpathlineto{\pgfqpoint{7.510199in}{5.357972in}}%
\pgfpathlineto{\pgfqpoint{8.321951in}{6.584873in}}%
\pgfpathlineto{\pgfqpoint{8.565477in}{6.956976in}}%
\pgfpathlineto{\pgfqpoint{8.727827in}{7.208794in}}%
\pgfpathlineto{\pgfqpoint{8.809002in}{7.337198in}}%
\pgfpathlineto{\pgfqpoint{8.890178in}{7.469311in}}%
\pgfpathlineto{\pgfqpoint{8.930765in}{7.538334in}}%
\pgfpathlineto{\pgfqpoint{8.971353in}{7.612452in}}%
\pgfpathlineto{\pgfqpoint{9.011940in}{7.732368in}}%
\pgfpathlineto{\pgfqpoint{11.609547in}{7.732368in}}%
\pgfpathlineto{\pgfqpoint{11.609547in}{7.732368in}}%
\pgfusepath{stroke}%
\end{pgfscope}%
\begin{pgfscope}%
\pgfpathrectangle{\pgfqpoint{0.907408in}{1.247073in}}{\pgfqpoint{11.013852in}{6.674186in}}%
\pgfusepath{clip}%
\pgfsetbuttcap%
\pgfsetroundjoin%
\pgfsetlinewidth{1.003750pt}%
\definecolor{currentstroke}{rgb}{0.888874,0.435649,0.278123}%
\pgfsetstrokecolor{currentstroke}%
\pgfsetdash{}{0pt}%
\pgfpathmoveto{\pgfqpoint{1.219120in}{1.435966in}}%
\pgfpathlineto{\pgfqpoint{5.115530in}{1.435966in}}%
\pgfpathlineto{\pgfqpoint{9.011940in}{7.732368in}}%
\pgfpathlineto{\pgfqpoint{11.609547in}{7.732368in}}%
\pgfpathlineto{\pgfqpoint{11.609547in}{7.732368in}}%
\pgfusepath{stroke}%
\end{pgfscope}%
\begin{pgfscope}%
\pgfsetrectcap%
\pgfsetmiterjoin%
\pgfsetlinewidth{1.003750pt}%
\definecolor{currentstroke}{rgb}{0.000000,0.000000,0.000000}%
\pgfsetstrokecolor{currentstroke}%
\pgfsetdash{}{0pt}%
\pgfpathmoveto{\pgfqpoint{0.907408in}{1.247073in}}%
\pgfpathlineto{\pgfqpoint{0.907408in}{7.921260in}}%
\pgfusepath{stroke}%
\end{pgfscope}%
\begin{pgfscope}%
\pgfsetrectcap%
\pgfsetmiterjoin%
\pgfsetlinewidth{1.003750pt}%
\definecolor{currentstroke}{rgb}{0.000000,0.000000,0.000000}%
\pgfsetstrokecolor{currentstroke}%
\pgfsetdash{}{0pt}%
\pgfpathmoveto{\pgfqpoint{0.907408in}{1.247073in}}%
\pgfpathlineto{\pgfqpoint{11.921260in}{1.247073in}}%
\pgfusepath{stroke}%
\end{pgfscope}%
\begin{pgfscope}%
\pgfsetbuttcap%
\pgfsetmiterjoin%
\definecolor{currentfill}{rgb}{1.000000,1.000000,1.000000}%
\pgfsetfillcolor{currentfill}%
\pgfsetlinewidth{1.003750pt}%
\definecolor{currentstroke}{rgb}{0.000000,0.000000,0.000000}%
\pgfsetstrokecolor{currentstroke}%
\pgfsetdash{}{0pt}%
\pgfpathmoveto{\pgfqpoint{10.511589in}{6.787373in}}%
\pgfpathlineto{\pgfqpoint{11.796260in}{6.787373in}}%
\pgfpathlineto{\pgfqpoint{11.796260in}{7.796260in}}%
\pgfpathlineto{\pgfqpoint{10.511589in}{7.796260in}}%
\pgfpathclose%
\pgfusepath{stroke,fill}%
\end{pgfscope}%
\begin{pgfscope}%
\pgfsetbuttcap%
\pgfsetmiterjoin%
\pgfsetlinewidth{2.258437pt}%
\definecolor{currentstroke}{rgb}{0.000000,0.605603,0.978680}%
\pgfsetstrokecolor{currentstroke}%
\pgfsetdash{}{0pt}%
\pgfpathmoveto{\pgfqpoint{10.711589in}{7.493818in}}%
\pgfpathlineto{\pgfqpoint{11.211589in}{7.493818in}}%
\pgfusepath{stroke}%
\end{pgfscope}%
\begin{pgfscope}%
\definecolor{textcolor}{rgb}{0.000000,0.000000,0.000000}%
\pgfsetstrokecolor{textcolor}%
\pgfsetfillcolor{textcolor}%
\pgftext[x=11.411589in,y=7.406318in,left,base]{\color{textcolor}\sffamily\fontsize{18.000000}{21.600000}\selectfont $\displaystyle U$}%
\end{pgfscope}%
\begin{pgfscope}%
\pgfsetbuttcap%
\pgfsetmiterjoin%
\pgfsetlinewidth{2.258437pt}%
\definecolor{currentstroke}{rgb}{0.888874,0.435649,0.278123}%
\pgfsetstrokecolor{currentstroke}%
\pgfsetdash{}{0pt}%
\pgfpathmoveto{\pgfqpoint{10.711589in}{7.126875in}}%
\pgfpathlineto{\pgfqpoint{11.211589in}{7.126875in}}%
\pgfusepath{stroke}%
\end{pgfscope}%
\begin{pgfscope}%
\definecolor{textcolor}{rgb}{0.000000,0.000000,0.000000}%
\pgfsetstrokecolor{textcolor}%
\pgfsetfillcolor{textcolor}%
\pgftext[x=11.411589in,y=7.039375in,left,base]{\color{textcolor}\sffamily\fontsize{18.000000}{21.600000}\selectfont $\displaystyle u$}%
\end{pgfscope}%
\end{pgfpicture}%
\makeatother%
\endgroup%
}
		\caption{$u_l = -1$, $u_r = 2$}\label{fig:burgers5}
	\end{minipage}
\end{figure}

\subsubsection{例 3.4 迎风格式与有限体积格式对比}

利用迎风格式进行数值实验, 显然满足 CFL 条件, 区间及步长取法同上, 初值为
$$u(x, 0) = \lb \begin{aligned}
	1,	&	x < 0,\\
	0,	&	x \geq 0.
\end{aligned} \rd$$
结果如图 \ref{fig:burgers6} 所示, 迎风格式得到了一个错误的解.

\begin{figure}[H]\centering\zihao{-5}
	\resizebox{0.6\linewidth}{!}{\input{burgers6.pgf}}
	\caption{$u_l = 2$, $u_r = 1$}\label{fig:burgers6}
\end{figure}

% 0.0       0.188703      -4.0  0.266867     0.0
% 0.898863  0.101203      -5.0  0.143123     0.898863
% 0.947041  0.0524937     -6.0  0.0742372    0.947041
% 0.972873  0.026745      -7.0  0.0378232    0.972873
% 0.986268  0.0135004     -8.0  0.0190924    0.986268
% 0.993091  0.0067826     -9.0  0.00959205   0.993091
% 0.996535  0.00339946   -10.0  0.00480756   0.996535
% 0.998265  0.00170177   -11.0  0.00240667   0.998265
% 0.999132  0.000851399  -12.0  0.00120406   0.999132
% 0.0       0.0447726     -4.0  0.0588846    0.0
% 2.02465   0.0110036     -5.0  0.0150502    1.9681
% 2.01832   0.00271618    -6.0  0.00377986   1.99338
% 2.01033   0.000674201   -7.0  0.000945935  1.99852
% 2.00541   0.000167919   -8.0  0.000236541  1.99965
% 2.00276   4.18994e-5    -9.0  5.91387e-5   1.99992
% 2.0014    1.04647e-5   -10.0  1.47849e-5   1.99998
% 2.0007    2.61491e-6   -11.0  3.69623e-6   1.99999
% 2.00035   6.53568e-7   -12.0  9.24059e-7   2.0
% 0.0       0.0447726     -4.0  0.0588846    0.0
% 2.02465   0.0110036     -5.0  0.0150502    1.9681
% 2.01832   0.00271618    -6.0  0.00377986   1.99338
% 2.01033   0.000674201   -7.0  0.000945935  1.99852
% 2.00541   0.000167919   -8.0  0.000236541  1.99965
% 2.00276   4.18994e-5    -9.0  5.91387e-5   1.99992
% 2.0014    1.04647e-5   -10.0  1.47849e-5   1.99998
% 2.0007    2.61491e-6   -11.0  3.69623e-6   1.99999
% 2.00035   6.53568e-7   -12.0  9.24059e-7   2.0

\end{document}
