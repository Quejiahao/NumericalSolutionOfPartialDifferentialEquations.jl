% !TeX program	= xelatex
% !TeX encoding	= UTF-8

%-------------------- 文类 --------------------
\documentclass[UTF8, a4paper, 12pt, oneside, onecolumn]{article}

%-------------------- 宏包 --------------------
\input{../../template/usepackage}

%-------------------- 杂项 --------------------
\input{../../template/misc}
\def\homeworkName{Poisson 方程 Dirichlet 边值问题的五点差分法}
\hypersetup
{
	% 颜色
	colorlinks	= true,
	linkcolor	= black,
	urlcolor	= red,
	citecolor	= black,
	anchorcolor	= blue,
}

%-------------------- 字体设置 --------------------
\input{../../template/fonts}

%-------------------- 标题样式 --------------------
\input{../title}

%-------------------- 自定义符号 --------------------
\input{../../template/symbols}

%-------------------- 自定义环境 --------------------
\input{../../template/environments}

%-------------------- \item 编号 --------------------
\input{../../template/itemstyle}

%-------------------- 代码样式设置 --------------------
\input{../../template/codestyle}

%-------------------- PDF 元信息, 建模时注释掉 --------------------
\input{../pdfinfo}

%-------------------- 页眉页脚 --------------------
\input{../headersfooters}

%-------------------- 正文 --------------------
\begin{document}

\thispagestyle{plain}

%\columnseprule = 1pt	% 栏线
\begin{center}
	{\zihao{-2}\heiti \homeworkName} \\
	\vspace{1.5ex}
	{\zihao{-4}\fangsong 阙嘉豪\textsuperscript{\hyperref[auth:1]{1}}} \\
	{\zihao{6}\songti \label{auth:1}(1. 北京师范大学 数学科学学院, 北京~~100875)}
\end{center}

\zihao{5}

%\watermark{60}{10}{\currenttime}

\section{模型问题}

考虑 $\Omega := (0, 1) \times (0, 1)$ 上的 Poisson 方程的 Dirichlet 边值问题:
\begin{equation}\label{equ:DirichletProblem}
	\lb\begin{array}{lr@{~}l}
		-\D u(x, y) = f(x, y),	&	\forall (x, y) &\in \Omega, \\
		u(0, y) = u_1(y),\quad u(1, y) = u_3(y),	&	\forall y &\in [0, 1], \\
		u(x, 0) = u_2(x),\quad u(x, 1) = u_4(x),	&	\forall x &\in [0, 1].
	\end{array}\rd
\end{equation}

\section{差分逼近}

\subsection{差分格式}

取空间步长 $\D x = \D y = h = \dfrac{1}{N}$, 则由网格线
$$x_i = i\D x,\quad y_j = j\D y,\quad i, j = 0, 1, \cdots, N$$
定义的指标集为 
$$J = \lrb{(i, j) : (x_i, y_j) \in \ol{\Omega}}.$$
记 Dirichlet 边界节点指标集为
$$J_D = \lrb{(i, j) : (x_i, y_j) \in \pa \Omega}.$$
将函数 $u$, $f$ 和网格函数 $U$ 在节点 $(x_i, y_j)$ 上的取值分别记为 $u_{i,j}$, $f_{i,j}$ 和 $U_{i,j}$.

由 Poisson 方程的五点差分格式和边值问题 \eqref{equ:DirichletProblem} 给出的 Dirichlet 边界条件得到线性方程组:
\begin{equation*}%\label{equ:LinearEquation}
	\lb\begin{array}{ll}
		-L_h U_{i, j} := \dfrac{4U_{i, j} - U_{i - 1, j} - U_{i, j - 1} - U_{i + 1, j} - U_{i, j + 1}}{h^2} = f_{i, j},	&	1 \leq i, j \leq N - 1, \\
		u_{0, j} = u_1(jh),\quad u_{N, j} = u_3(jh),	&	0 \leq j \leq N, \\
		u_{i, 0} = u_2(ih),\quad u_{i, N} = u_4(ih),	&	0 \leq i \leq N.
	\end{array}\rd
\end{equation*}
整理写成矩阵形式 $L_2 U = g$, 其中
$$L_2 = \begin{pmatrix}
		A	&	-I		&			&			&	\\
		-I	&	A		&	-I		&			&	\\
			&	\ddots	&	\ddots	&	\ddots	&	\\
			&			&	-I		&	A		&	-I	\\
			&			&			&	-I		&	A	\\
	\end{pmatrix},\quad A = \begin{pmatrix}
	4	&	-1	&		&		&	\\
	-1	&	4	&	-1	&		&	\\
		&	-1	&	\ddots	&	\ddots	&	\\
		&		&	\ddots	&	4	&	-1\\
		&		&		&	-1	&	4
	\end{pmatrix}_{(N - 1) \times (N - 1)},$$
$$U = \begin{pmatrix}
	U_{1, 1}\\
	U_{2, 1}\\
	\vdots\\
	U_{N - 1, 1}\\
	U_{1, 2}\\
	U_{2, 2}\\
	\vdots\\
	U_{N - 1, N - 1}
\end{pmatrix},\quad g = \begin{pmatrix}
	g_1\\
	\vdots\\
	g_{N - 1}
\end{pmatrix},\quad g_i = \begin{pmatrix}
	h^2 f_{1, i} + u_2(ih)\\
	h^2 f_{2, i}\\
	\vdots\\
	h^2 f_{N - 2, i}\\
	h^2 f_{N - 1, i} + u_4(ih)
\end{pmatrix},\quad i = 2, \cdots, N - 2,$$
$$g_1 = \begin{pmatrix}
	h^2 f_{1, 1} + u_1(h)\\
	h^2 f_{2, 1} + u_1(2h)\\
	\vdots\\
	h^2 f_{N - 1, 1} + u_1((N - 1)h)
\end{pmatrix},\quad g_{N - 1} = \begin{pmatrix}
	h^2 f_{1, N - 1} + u_3(h)\\
	h^2 f_{2, N - 1} + u_3(2h)\\
	\vdots\\
	h^2 f_{N - 1, N - 1} + u_3((N - 1)h)
\end{pmatrix}.$$

\subsection{相容性}

假设边值问题 \eqref{equ:DirichletProblem} 的真解 $u$ 充分光滑, 则由 $u$ 在 $(x_i, y_j)$ 的 Taylor 展式可得
\begin{align*}
	T_{i, j} :&= L_hu_{i, j} - (\D u)_{i, j} = L_hu_{i, j} + f_{i, j}\\
	&= \dfrac{1}{12}h^2\(\pa_x^4 u + \pa_y^4 u\)_{i, j} + \dfrac{1}{360}h^4\(\pa_x^6 u + \pa_y^6 u\)_{i, j} + O\(h^6\),\quad \forall (i, j) \in J_\Omega.
\end{align*}
则对于截断误差 $T_h := \lrb{T_{i, j}}_{(i, j) \in \Omega}$ 有
$$\lim_{h \to 0} \lrvv{T_h}_\infty = \lim_{h \to 0} \max_{(i, j) \in J_\Omega} \lrv{T_{i, j}} = 0.$$
由此知该差分格式具有相容性, 且 $\lrvv{T_h}_\infty = O\(h^2\)$, 即差分格式具有二阶精度.

\subsection{稳定性}

记 $F := \dmax_{(i, j) \in J_\Omega} \lrv{f_{i, j}}$, 令 $\Phi(x, y) := \(x - \dfrac{1}{2}\)^2 + \(y - \dfrac{1}{2}\)^2$, 取比较函数
\begin{equation}\label{equ:compareFunction}
	\Psi_{i, j}^\pm := \pm U_{i, j} + \dfrac{1}{4} F\Phi_{i, j},\quad \forall (i, j) \in J,
\end{equation}
于是
$$L_h \Phi_{i, j}^\pm = \pm L_h U_{i, j} + \dfrac{1}{4} F L_h \Phi_{i, j} = \pm f_{i, j} + F \geq 0.$$
由此, 最大值原理, $\Phi$ 非负且 $U$ 满足边界条件 $U_{i, j} = u_D\(x_i, y_j\)$, $(i, j) \in J_D$ 可得
\begin{equation*}
	\pm U_{i, j} \leq \pm U_{i, j} + \dfrac{1}{4} F\Phi_{i, j} \leq \max_{(i, j) \in J_D} \lrv{\(u_D\)_{i, j}} + \dfrac{1}{8}F,\quad \forall (i, j) \in J_\Omega.
\end{equation*}
进而
\begin{equation*}
	\max_{(i, j) \in J} \lrv{U_{i, j}} \leq \max_{(i, j) \in J_D} \lrv{\(u_D\)_{i, j}} + \dfrac{1}{8}F,
\end{equation*}
即差分格式具有稳定性.

\subsection{收敛性}

在 \eqref{equ:compareFunction} 式中分别将 $U_{i, j}$ 和 $F$ 替换为误差 $e_{i, j} := U_{i, j} - u_{i, j}$ 和 $\lrvv{T_h}_\infty$, 则有
$$\lrvv{e}_\infty \leq \max_{(i, j) \in J_D} \lrv{e_{i, j}} + \dfrac{1}{8} \lrvv{T_h}_\infty \leq \dfrac{1}{96} \(\max_{(x, y) \in J_{\ol\Omega}} \lrv{\pa_x^4 u} + \max_{(x, y) \in J_{\ol\Omega}} \lrv{\pa_y^4 u}\),$$
进而 $\dlim_{h \to 0} \lrvv{e}_\infty = 0$ 且 $\lrvv{e}_\infty = O\(h^2\)$, 即差分逼近解在 $\mathbb{L}^\infty$ 意义下是二阶收敛的.

\subsection{误差分析}

假设误差主项的阶数为 $\a$, 有
\begin{equation}\label{equ:error1}
	U_{h, \bm j} = u_{\bm j} + C_{\bm j} h^\a + o\(h^\a\).
\end{equation}
对于已知真解的情况, 只需计算 $\ln \lrvv{e}_\infty$ 关于 $\ln h$ 的斜率即可. 对于未知真解的情况, 需在 \ref{equ:error1} 式中取不同的 $h$, 然后做差再取对数. 如
\begin{equation*}
	U_{h, \bm j} - U_{h/2, \bm j} = \(1 - 2^{-\a}\)C_{\bm j} h^\a + o\(h^\a\),
\end{equation*}
两边取范数并取对数得
$$\ln \lrvv{U_{h} - U_{h/2}}_\infty \approx \ln \(\(1 - 2^{-\a}\)C\) + \a \ln h.$$
于是计算 $\ln \lrvv{U_{h} - U_{h/2}}_\infty$ 关于 $\ln h$ 的斜率可得收敛阶.

\section{数值实验}

分别求解下述边值问题对应的差分方程 $L_2U = g$, 得到数值解 $U_{i, j}$.

\subsection{已知真解算例}

取
\begin{equation*}
	\lb\begin{aligned}
		f(x, y) &= \(\pi^2 - 1\)\e^x \sin (\pi y),\quad (x, y) 
		\in \Omega,\\
		u_1(y) &= \sin (\pi y),\quad u_3(y) = \e \sin (\pi y),\\
		u_2(x) &= u_4(x) = 0,
	\end{aligned}\rd
\end{equation*}
此时边值问题 \eqref{equ:DirichletProblem} 的解析解为 $u(x, y) = \e^x\sin (\pi y)$.

$h = 2^{-5}$ 时求得的差分逼近解如图 \ref{fig:known1} 所示.

\begin{figure}[H]\centering
	\resizebox{0.9\linewidth}{!}{%% Creator: Matplotlib, PGF backend
%%
%% To include the figure in your LaTeX document, write
%%   \input{<filename>.pgf}
%%
%% Make sure the required packages are loaded in your preamble
%%   \usepackage{pgf}
%%
%% Figures using additional raster images can only be included by \input if
%% they are in the same directory as the main LaTeX file. For loading figures
%% from other directories you can use the `import` package
%%   \usepackage{import}
%%
%% and then include the figures with
%%   \import{<path to file>}{<filename>.pgf}
%%
%% Matplotlib used the following preamble
%%   \usepackage{fontspec}
%%   \setmainfont{DejaVuSerif.ttf}[Path=\detokenize{/Users/quejiahao/.julia/conda/3/lib/python3.9/site-packages/matplotlib/mpl-data/fonts/ttf/}]
%%   \setsansfont{DejaVuSans.ttf}[Path=\detokenize{/Users/quejiahao/.julia/conda/3/lib/python3.9/site-packages/matplotlib/mpl-data/fonts/ttf/}]
%%   \setmonofont{DejaVuSansMono.ttf}[Path=\detokenize{/Users/quejiahao/.julia/conda/3/lib/python3.9/site-packages/matplotlib/mpl-data/fonts/ttf/}]
%%
\begingroup%
\makeatletter%
\begin{pgfpicture}%
\pgfpathrectangle{\pgfpointorigin}{\pgfqpoint{12.000000in}{8.000000in}}%
\pgfusepath{use as bounding box, clip}%
\begin{pgfscope}%
\pgfsetbuttcap%
\pgfsetmiterjoin%
\definecolor{currentfill}{rgb}{0.152941,0.098039,0.141176}%
\pgfsetfillcolor{currentfill}%
\pgfsetlinewidth{0.000000pt}%
\definecolor{currentstroke}{rgb}{1.000000,1.000000,1.000000}%
\pgfsetstrokecolor{currentstroke}%
\pgfsetdash{}{0pt}%
\pgfpathmoveto{\pgfqpoint{0.000000in}{0.000000in}}%
\pgfpathlineto{\pgfqpoint{12.000000in}{0.000000in}}%
\pgfpathlineto{\pgfqpoint{12.000000in}{8.000000in}}%
\pgfpathlineto{\pgfqpoint{0.000000in}{8.000000in}}%
\pgfpathclose%
\pgfusepath{fill}%
\end{pgfscope}%
\begin{pgfscope}%
\pgfsetbuttcap%
\pgfsetmiterjoin%
\definecolor{currentfill}{rgb}{0.152941,0.098039,0.141176}%
\pgfsetfillcolor{currentfill}%
\pgfsetlinewidth{0.000000pt}%
\definecolor{currentstroke}{rgb}{0.000000,0.000000,0.000000}%
\pgfsetstrokecolor{currentstroke}%
\pgfsetstrokeopacity{0.000000}%
\pgfsetdash{}{0pt}%
\pgfpathmoveto{\pgfqpoint{0.680860in}{0.078740in}}%
\pgfpathlineto{\pgfqpoint{8.523380in}{0.078740in}}%
\pgfpathlineto{\pgfqpoint{8.523380in}{7.921260in}}%
\pgfpathlineto{\pgfqpoint{0.680860in}{7.921260in}}%
\pgfpathclose%
\pgfusepath{fill}%
\end{pgfscope}%
\begin{pgfscope}%
\pgfsetbuttcap%
\pgfsetmiterjoin%
\definecolor{currentfill}{rgb}{0.950000,0.950000,0.950000}%
\pgfsetfillcolor{currentfill}%
\pgfsetfillopacity{0.500000}%
\pgfsetlinewidth{1.003750pt}%
\definecolor{currentstroke}{rgb}{0.950000,0.950000,0.950000}%
\pgfsetstrokecolor{currentstroke}%
\pgfsetstrokeopacity{0.500000}%
\pgfsetdash{}{0pt}%
\pgfpathmoveto{\pgfqpoint{1.273025in}{2.012454in}}%
\pgfpathlineto{\pgfqpoint{3.862881in}{4.183323in}}%
\pgfpathlineto{\pgfqpoint{3.826879in}{7.314104in}}%
\pgfpathlineto{\pgfqpoint{1.113085in}{5.333700in}}%
\pgfusepath{stroke,fill}%
\end{pgfscope}%
\begin{pgfscope}%
\pgfsetbuttcap%
\pgfsetmiterjoin%
\definecolor{currentfill}{rgb}{0.900000,0.900000,0.900000}%
\pgfsetfillcolor{currentfill}%
\pgfsetfillopacity{0.500000}%
\pgfsetlinewidth{1.003750pt}%
\definecolor{currentstroke}{rgb}{0.900000,0.900000,0.900000}%
\pgfsetstrokecolor{currentstroke}%
\pgfsetstrokeopacity{0.500000}%
\pgfsetdash{}{0pt}%
\pgfpathmoveto{\pgfqpoint{3.862881in}{4.183323in}}%
\pgfpathlineto{\pgfqpoint{8.018676in}{2.975397in}}%
\pgfpathlineto{\pgfqpoint{8.166981in}{6.214014in}}%
\pgfpathlineto{\pgfqpoint{3.826879in}{7.314104in}}%
\pgfusepath{stroke,fill}%
\end{pgfscope}%
\begin{pgfscope}%
\pgfsetbuttcap%
\pgfsetmiterjoin%
\definecolor{currentfill}{rgb}{0.925000,0.925000,0.925000}%
\pgfsetfillcolor{currentfill}%
\pgfsetfillopacity{0.500000}%
\pgfsetlinewidth{1.003750pt}%
\definecolor{currentstroke}{rgb}{0.925000,0.925000,0.925000}%
\pgfsetstrokecolor{currentstroke}%
\pgfsetstrokeopacity{0.500000}%
\pgfsetdash{}{0pt}%
\pgfpathmoveto{\pgfqpoint{1.273025in}{2.012454in}}%
\pgfpathlineto{\pgfqpoint{5.678370in}{0.573668in}}%
\pgfpathlineto{\pgfqpoint{8.018676in}{2.975397in}}%
\pgfpathlineto{\pgfqpoint{3.862881in}{4.183323in}}%
\pgfusepath{stroke,fill}%
\end{pgfscope}%
\begin{pgfscope}%
\pgfsetrectcap%
\pgfsetroundjoin%
\pgfsetlinewidth{0.803000pt}%
\definecolor{currentstroke}{rgb}{0.000000,0.000000,0.000000}%
\pgfsetstrokecolor{currentstroke}%
\pgfsetdash{}{0pt}%
\pgfpathmoveto{\pgfqpoint{1.273025in}{2.012454in}}%
\pgfpathlineto{\pgfqpoint{5.678370in}{0.573668in}}%
\pgfusepath{stroke}%
\end{pgfscope}%
\begin{pgfscope}%
\definecolor{textcolor}{rgb}{0.525490,0.694118,0.356863}%
\pgfsetstrokecolor{textcolor}%
\pgfsetfillcolor{textcolor}%
\pgftext[x=3.192347in,y=0.824279in,,]{\color{textcolor}\sffamily\fontsize{24.000000}{13.200000}\bfseries\selectfont $x$}%
\end{pgfscope}%
\begin{pgfscope}%
\pgfsetbuttcap%
\pgfsetroundjoin%
\pgfsetlinewidth{0.803000pt}%
\definecolor{currentstroke}{rgb}{0.690196,0.690196,0.690196}%
\pgfsetstrokecolor{currentstroke}%
\pgfsetdash{}{0pt}%
\pgfpathmoveto{\pgfqpoint{2.086803in}{1.746674in}}%
\pgfpathlineto{\pgfqpoint{4.633328in}{3.959384in}}%
\pgfpathlineto{\pgfqpoint{4.630112in}{7.110508in}}%
\pgfusepath{stroke}%
\end{pgfscope}%
\begin{pgfscope}%
\pgfsetbuttcap%
\pgfsetroundjoin%
\pgfsetlinewidth{0.803000pt}%
\definecolor{currentstroke}{rgb}{0.690196,0.690196,0.690196}%
\pgfsetstrokecolor{currentstroke}%
\pgfsetdash{}{0pt}%
\pgfpathmoveto{\pgfqpoint{2.969255in}{1.458465in}}%
\pgfpathlineto{\pgfqpoint{5.467371in}{3.716961in}}%
\pgfpathlineto{\pgfqpoint{5.500357in}{6.889926in}}%
\pgfusepath{stroke}%
\end{pgfscope}%
\begin{pgfscope}%
\pgfsetbuttcap%
\pgfsetroundjoin%
\pgfsetlinewidth{0.803000pt}%
\definecolor{currentstroke}{rgb}{0.690196,0.690196,0.690196}%
\pgfsetstrokecolor{currentstroke}%
\pgfsetdash{}{0pt}%
\pgfpathmoveto{\pgfqpoint{3.870869in}{1.163998in}}%
\pgfpathlineto{\pgfqpoint{6.318002in}{3.469716in}}%
\pgfpathlineto{\pgfqpoint{6.388669in}{6.664765in}}%
\pgfusepath{stroke}%
\end{pgfscope}%
\begin{pgfscope}%
\pgfsetbuttcap%
\pgfsetroundjoin%
\pgfsetlinewidth{0.803000pt}%
\definecolor{currentstroke}{rgb}{0.690196,0.690196,0.690196}%
\pgfsetstrokecolor{currentstroke}%
\pgfsetdash{}{0pt}%
\pgfpathmoveto{\pgfqpoint{4.792276in}{0.863067in}}%
\pgfpathlineto{\pgfqpoint{7.185720in}{3.217505in}}%
\pgfpathlineto{\pgfqpoint{7.295616in}{6.434880in}}%
\pgfusepath{stroke}%
\end{pgfscope}%
\begin{pgfscope}%
\pgfpathrectangle{\pgfqpoint{0.680860in}{0.078740in}}{\pgfqpoint{7.842520in}{7.842520in}}%
\pgfusepath{clip}%
\pgfsetrectcap%
\pgfsetroundjoin%
\pgfsetlinewidth{0.501875pt}%
\definecolor{currentstroke}{rgb}{0.980392,0.811765,0.352941}%
\pgfsetstrokecolor{currentstroke}%
\pgfsetstrokeopacity{0.100000}%
\pgfsetdash{}{0pt}%
\pgfpathmoveto{\pgfqpoint{4.708100in}{4.105980in}}%
\pgfusepath{stroke}%
\end{pgfscope}%
\begin{pgfscope}%
\pgfsetrectcap%
\pgfsetroundjoin%
\pgfsetlinewidth{0.803000pt}%
\definecolor{currentstroke}{rgb}{0.980392,0.811765,0.352941}%
\pgfsetstrokecolor{currentstroke}%
\pgfsetdash{}{0pt}%
\pgfpathmoveto{\pgfqpoint{2.108990in}{1.765953in}}%
\pgfpathlineto{\pgfqpoint{2.042333in}{1.708034in}}%
\pgfusepath{stroke}%
\end{pgfscope}%
\begin{pgfscope}%
\definecolor{textcolor}{rgb}{0.525490,0.694118,0.356863}%
\pgfsetstrokecolor{textcolor}%
\pgfsetfillcolor{textcolor}%
\pgftext[x=1.982700in,y=1.527121in,,top]{\color{textcolor}\sffamily\fontsize{18.000000}{9.600000}\selectfont $\displaystyle 0.2$}%
\end{pgfscope}%
\begin{pgfscope}%
\pgfpathrectangle{\pgfqpoint{0.680860in}{0.078740in}}{\pgfqpoint{7.842520in}{7.842520in}}%
\pgfusepath{clip}%
\pgfsetrectcap%
\pgfsetroundjoin%
\pgfsetlinewidth{0.501875pt}%
\definecolor{currentstroke}{rgb}{0.980392,0.811765,0.352941}%
\pgfsetstrokecolor{currentstroke}%
\pgfsetstrokeopacity{0.100000}%
\pgfsetdash{}{0pt}%
\pgfpathmoveto{\pgfqpoint{4.708100in}{4.105980in}}%
\pgfusepath{stroke}%
\end{pgfscope}%
\begin{pgfscope}%
\pgfsetrectcap%
\pgfsetroundjoin%
\pgfsetlinewidth{0.803000pt}%
\definecolor{currentstroke}{rgb}{0.980392,0.811765,0.352941}%
\pgfsetstrokecolor{currentstroke}%
\pgfsetdash{}{0pt}%
\pgfpathmoveto{\pgfqpoint{2.991040in}{1.478160in}}%
\pgfpathlineto{\pgfqpoint{2.925591in}{1.418990in}}%
\pgfusepath{stroke}%
\end{pgfscope}%
\begin{pgfscope}%
\definecolor{textcolor}{rgb}{0.525490,0.694118,0.356863}%
\pgfsetstrokecolor{textcolor}%
\pgfsetfillcolor{textcolor}%
\pgftext[x=2.865602in,y=1.236037in,,top]{\color{textcolor}\sffamily\fontsize{18.000000}{9.600000}\selectfont $\displaystyle 0.4$}%
\end{pgfscope}%
\begin{pgfscope}%
\pgfpathrectangle{\pgfqpoint{0.680860in}{0.078740in}}{\pgfqpoint{7.842520in}{7.842520in}}%
\pgfusepath{clip}%
\pgfsetrectcap%
\pgfsetroundjoin%
\pgfsetlinewidth{0.501875pt}%
\definecolor{currentstroke}{rgb}{0.980392,0.811765,0.352941}%
\pgfsetstrokecolor{currentstroke}%
\pgfsetstrokeopacity{0.100000}%
\pgfsetdash{}{0pt}%
\pgfpathmoveto{\pgfqpoint{4.708100in}{4.105980in}}%
\pgfusepath{stroke}%
\end{pgfscope}%
\begin{pgfscope}%
\pgfsetrectcap%
\pgfsetroundjoin%
\pgfsetlinewidth{0.803000pt}%
\definecolor{currentstroke}{rgb}{0.980392,0.811765,0.352941}%
\pgfsetstrokecolor{currentstroke}%
\pgfsetdash{}{0pt}%
\pgfpathmoveto{\pgfqpoint{3.892228in}{1.184123in}}%
\pgfpathlineto{\pgfqpoint{3.828057in}{1.123660in}}%
\pgfusepath{stroke}%
\end{pgfscope}%
\begin{pgfscope}%
\definecolor{textcolor}{rgb}{0.525490,0.694118,0.356863}%
\pgfsetstrokecolor{textcolor}%
\pgfsetfillcolor{textcolor}%
\pgftext[x=3.767710in,y=0.938621in,,top]{\color{textcolor}\sffamily\fontsize{18.000000}{9.600000}\selectfont $\displaystyle 0.6$}%
\end{pgfscope}%
\begin{pgfscope}%
\pgfpathrectangle{\pgfqpoint{0.680860in}{0.078740in}}{\pgfqpoint{7.842520in}{7.842520in}}%
\pgfusepath{clip}%
\pgfsetrectcap%
\pgfsetroundjoin%
\pgfsetlinewidth{0.501875pt}%
\definecolor{currentstroke}{rgb}{0.980392,0.811765,0.352941}%
\pgfsetstrokecolor{currentstroke}%
\pgfsetstrokeopacity{0.100000}%
\pgfsetdash{}{0pt}%
\pgfpathmoveto{\pgfqpoint{4.708100in}{4.105980in}}%
\pgfusepath{stroke}%
\end{pgfscope}%
\begin{pgfscope}%
\pgfsetrectcap%
\pgfsetroundjoin%
\pgfsetlinewidth{0.803000pt}%
\definecolor{currentstroke}{rgb}{0.980392,0.811765,0.352941}%
\pgfsetstrokecolor{currentstroke}%
\pgfsetdash{}{0pt}%
\pgfpathmoveto{\pgfqpoint{4.813185in}{0.883635in}}%
\pgfpathlineto{\pgfqpoint{4.750363in}{0.821838in}}%
\pgfusepath{stroke}%
\end{pgfscope}%
\begin{pgfscope}%
\definecolor{textcolor}{rgb}{0.525490,0.694118,0.356863}%
\pgfsetstrokecolor{textcolor}%
\pgfsetfillcolor{textcolor}%
\pgftext[x=4.689658in,y=0.634665in,,top]{\color{textcolor}\sffamily\fontsize{18.000000}{9.600000}\selectfont $\displaystyle 0.8$}%
\end{pgfscope}%
\begin{pgfscope}%
\pgfsetrectcap%
\pgfsetroundjoin%
\pgfsetlinewidth{0.803000pt}%
\definecolor{currentstroke}{rgb}{0.000000,0.000000,0.000000}%
\pgfsetstrokecolor{currentstroke}%
\pgfsetdash{}{0pt}%
\pgfpathmoveto{\pgfqpoint{8.018676in}{2.975397in}}%
\pgfpathlineto{\pgfqpoint{5.678370in}{0.573668in}}%
\pgfusepath{stroke}%
\end{pgfscope}%
\begin{pgfscope}%
\definecolor{textcolor}{rgb}{0.525490,0.694118,0.356863}%
\pgfsetstrokecolor{textcolor}%
\pgfsetfillcolor{textcolor}%
\pgftext[x=7.275268in,y=1.439527in,,]{\color{textcolor}\sffamily\fontsize{24.000000}{13.200000}\bfseries\selectfont $y$}%
\end{pgfscope}%
\begin{pgfscope}%
\pgfsetbuttcap%
\pgfsetroundjoin%
\pgfsetlinewidth{0.803000pt}%
\definecolor{currentstroke}{rgb}{0.690196,0.690196,0.690196}%
\pgfsetstrokecolor{currentstroke}%
\pgfsetdash{}{0pt}%
\pgfpathmoveto{\pgfqpoint{1.674940in}{5.743717in}}%
\pgfpathlineto{\pgfqpoint{1.807607in}{2.460552in}}%
\pgfpathlineto{\pgfqpoint{6.163137in}{1.071158in}}%
\pgfusepath{stroke}%
\end{pgfscope}%
\begin{pgfscope}%
\pgfsetbuttcap%
\pgfsetroundjoin%
\pgfsetlinewidth{0.803000pt}%
\definecolor{currentstroke}{rgb}{0.690196,0.690196,0.690196}%
\pgfsetstrokecolor{currentstroke}%
\pgfsetdash{}{0pt}%
\pgfpathmoveto{\pgfqpoint{2.250457in}{6.163702in}}%
\pgfpathlineto{\pgfqpoint{2.356059in}{2.920275in}}%
\pgfpathlineto{\pgfqpoint{6.659562in}{1.580612in}}%
\pgfusepath{stroke}%
\end{pgfscope}%
\begin{pgfscope}%
\pgfsetbuttcap%
\pgfsetroundjoin%
\pgfsetlinewidth{0.803000pt}%
\definecolor{currentstroke}{rgb}{0.690196,0.690196,0.690196}%
\pgfsetstrokecolor{currentstroke}%
\pgfsetdash{}{0pt}%
\pgfpathmoveto{\pgfqpoint{2.805188in}{6.568520in}}%
\pgfpathlineto{\pgfqpoint{2.885540in}{3.364097in}}%
\pgfpathlineto{\pgfqpoint{7.137935in}{2.071541in}}%
\pgfusepath{stroke}%
\end{pgfscope}%
\begin{pgfscope}%
\pgfsetbuttcap%
\pgfsetroundjoin%
\pgfsetlinewidth{0.803000pt}%
\definecolor{currentstroke}{rgb}{0.690196,0.690196,0.690196}%
\pgfsetstrokecolor{currentstroke}%
\pgfsetdash{}{0pt}%
\pgfpathmoveto{\pgfqpoint{3.340239in}{6.958976in}}%
\pgfpathlineto{\pgfqpoint{3.397016in}{3.792826in}}%
\pgfpathlineto{\pgfqpoint{7.599223in}{2.544936in}}%
\pgfusepath{stroke}%
\end{pgfscope}%
\begin{pgfscope}%
\pgfpathrectangle{\pgfqpoint{0.680860in}{0.078740in}}{\pgfqpoint{7.842520in}{7.842520in}}%
\pgfusepath{clip}%
\pgfsetrectcap%
\pgfsetroundjoin%
\pgfsetlinewidth{0.501875pt}%
\definecolor{currentstroke}{rgb}{0.980392,0.811765,0.352941}%
\pgfsetstrokecolor{currentstroke}%
\pgfsetstrokeopacity{0.100000}%
\pgfsetdash{}{0pt}%
\pgfpathmoveto{\pgfqpoint{4.708100in}{4.105980in}}%
\pgfusepath{stroke}%
\end{pgfscope}%
\begin{pgfscope}%
\pgfsetrectcap%
\pgfsetroundjoin%
\pgfsetlinewidth{0.803000pt}%
\definecolor{currentstroke}{rgb}{0.980392,0.811765,0.352941}%
\pgfsetstrokecolor{currentstroke}%
\pgfsetdash{}{0pt}%
\pgfpathmoveto{\pgfqpoint{6.126455in}{1.082859in}}%
\pgfpathlineto{\pgfqpoint{6.236595in}{1.047725in}}%
\pgfusepath{stroke}%
\end{pgfscope}%
\begin{pgfscope}%
\definecolor{textcolor}{rgb}{0.525490,0.694118,0.356863}%
\pgfsetstrokecolor{textcolor}%
\pgfsetfillcolor{textcolor}%
\pgftext[x=6.339939in,y=0.887093in,,top]{\color{textcolor}\sffamily\fontsize{18.000000}{9.600000}\selectfont $\displaystyle 0.2$}%
\end{pgfscope}%
\begin{pgfscope}%
\pgfpathrectangle{\pgfqpoint{0.680860in}{0.078740in}}{\pgfqpoint{7.842520in}{7.842520in}}%
\pgfusepath{clip}%
\pgfsetrectcap%
\pgfsetroundjoin%
\pgfsetlinewidth{0.501875pt}%
\definecolor{currentstroke}{rgb}{0.980392,0.811765,0.352941}%
\pgfsetstrokecolor{currentstroke}%
\pgfsetstrokeopacity{0.100000}%
\pgfsetdash{}{0pt}%
\pgfpathmoveto{\pgfqpoint{4.708100in}{4.105980in}}%
\pgfusepath{stroke}%
\end{pgfscope}%
\begin{pgfscope}%
\pgfsetrectcap%
\pgfsetroundjoin%
\pgfsetlinewidth{0.803000pt}%
\definecolor{currentstroke}{rgb}{0.980392,0.811765,0.352941}%
\pgfsetstrokecolor{currentstroke}%
\pgfsetdash{}{0pt}%
\pgfpathmoveto{\pgfqpoint{6.623352in}{1.591885in}}%
\pgfpathlineto{\pgfqpoint{6.732073in}{1.558040in}}%
\pgfusepath{stroke}%
\end{pgfscope}%
\begin{pgfscope}%
\definecolor{textcolor}{rgb}{0.525490,0.694118,0.356863}%
\pgfsetstrokecolor{textcolor}%
\pgfsetfillcolor{textcolor}%
\pgftext[x=6.833225in,y=1.400097in,,top]{\color{textcolor}\sffamily\fontsize{18.000000}{9.600000}\selectfont $\displaystyle 0.4$}%
\end{pgfscope}%
\begin{pgfscope}%
\pgfpathrectangle{\pgfqpoint{0.680860in}{0.078740in}}{\pgfqpoint{7.842520in}{7.842520in}}%
\pgfusepath{clip}%
\pgfsetrectcap%
\pgfsetroundjoin%
\pgfsetlinewidth{0.501875pt}%
\definecolor{currentstroke}{rgb}{0.980392,0.811765,0.352941}%
\pgfsetstrokecolor{currentstroke}%
\pgfsetstrokeopacity{0.100000}%
\pgfsetdash{}{0pt}%
\pgfpathmoveto{\pgfqpoint{4.708100in}{4.105980in}}%
\pgfusepath{stroke}%
\end{pgfscope}%
\begin{pgfscope}%
\pgfsetrectcap%
\pgfsetroundjoin%
\pgfsetlinewidth{0.803000pt}%
\definecolor{currentstroke}{rgb}{0.980392,0.811765,0.352941}%
\pgfsetstrokecolor{currentstroke}%
\pgfsetdash{}{0pt}%
\pgfpathmoveto{\pgfqpoint{7.102187in}{2.082407in}}%
\pgfpathlineto{\pgfqpoint{7.209520in}{2.049782in}}%
\pgfusepath{stroke}%
\end{pgfscope}%
\begin{pgfscope}%
\definecolor{textcolor}{rgb}{0.525490,0.694118,0.356863}%
\pgfsetstrokecolor{textcolor}%
\pgfsetfillcolor{textcolor}%
\pgftext[x=7.308568in,y=1.894440in,,top]{\color{textcolor}\sffamily\fontsize{18.000000}{9.600000}\selectfont $\displaystyle 0.6$}%
\end{pgfscope}%
\begin{pgfscope}%
\pgfpathrectangle{\pgfqpoint{0.680860in}{0.078740in}}{\pgfqpoint{7.842520in}{7.842520in}}%
\pgfusepath{clip}%
\pgfsetrectcap%
\pgfsetroundjoin%
\pgfsetlinewidth{0.501875pt}%
\definecolor{currentstroke}{rgb}{0.980392,0.811765,0.352941}%
\pgfsetstrokecolor{currentstroke}%
\pgfsetstrokeopacity{0.100000}%
\pgfsetdash{}{0pt}%
\pgfpathmoveto{\pgfqpoint{4.708100in}{4.105980in}}%
\pgfusepath{stroke}%
\end{pgfscope}%
\begin{pgfscope}%
\pgfsetrectcap%
\pgfsetroundjoin%
\pgfsetlinewidth{0.803000pt}%
\definecolor{currentstroke}{rgb}{0.980392,0.811765,0.352941}%
\pgfsetstrokecolor{currentstroke}%
\pgfsetdash{}{0pt}%
\pgfpathmoveto{\pgfqpoint{7.563927in}{2.555417in}}%
\pgfpathlineto{\pgfqpoint{7.669900in}{2.523947in}}%
\pgfusepath{stroke}%
\end{pgfscope}%
\begin{pgfscope}%
\definecolor{textcolor}{rgb}{0.525490,0.694118,0.356863}%
\pgfsetstrokecolor{textcolor}%
\pgfsetfillcolor{textcolor}%
\pgftext[x=7.766930in,y=2.371123in,,top]{\color{textcolor}\sffamily\fontsize{18.000000}{9.600000}\selectfont $\displaystyle 0.8$}%
\end{pgfscope}%
\begin{pgfscope}%
\pgfsetrectcap%
\pgfsetroundjoin%
\pgfsetlinewidth{0.803000pt}%
\definecolor{currentstroke}{rgb}{0.000000,0.000000,0.000000}%
\pgfsetstrokecolor{currentstroke}%
\pgfsetdash{}{0pt}%
\pgfpathmoveto{\pgfqpoint{8.018676in}{2.975397in}}%
\pgfpathlineto{\pgfqpoint{8.166981in}{6.214014in}}%
\pgfusepath{stroke}%
\end{pgfscope}%
\begin{pgfscope}%
\definecolor{textcolor}{rgb}{0.525490,0.694118,0.356863}%
\pgfsetstrokecolor{textcolor}%
\pgfsetfillcolor{textcolor}%
\pgftext[x=8.653954in,y=4.634624in,,]{\color{textcolor}\sffamily\fontsize{24.000000}{13.200000}\bfseries\selectfont $U$}%
\end{pgfscope}%
\begin{pgfscope}%
\pgfsetbuttcap%
\pgfsetroundjoin%
\pgfsetlinewidth{0.803000pt}%
\definecolor{currentstroke}{rgb}{0.690196,0.690196,0.690196}%
\pgfsetstrokecolor{currentstroke}%
\pgfsetdash{}{0pt}%
\pgfpathmoveto{\pgfqpoint{8.043146in}{3.509775in}}%
\pgfpathlineto{\pgfqpoint{3.856930in}{4.700818in}}%
\pgfpathlineto{\pgfqpoint{1.246671in}{2.559694in}}%
\pgfusepath{stroke}%
\end{pgfscope}%
\begin{pgfscope}%
\pgfsetbuttcap%
\pgfsetroundjoin%
\pgfsetlinewidth{0.803000pt}%
\definecolor{currentstroke}{rgb}{0.690196,0.690196,0.690196}%
\pgfsetstrokecolor{currentstroke}%
\pgfsetdash{}{0pt}%
\pgfpathmoveto{\pgfqpoint{8.070662in}{4.110652in}}%
\pgfpathlineto{\pgfqpoint{3.850244in}{5.282282in}}%
\pgfpathlineto{\pgfqpoint{1.217021in}{3.175399in}}%
\pgfusepath{stroke}%
\end{pgfscope}%
\begin{pgfscope}%
\pgfsetbuttcap%
\pgfsetroundjoin%
\pgfsetlinewidth{0.803000pt}%
\definecolor{currentstroke}{rgb}{0.690196,0.690196,0.690196}%
\pgfsetstrokecolor{currentstroke}%
\pgfsetdash{}{0pt}%
\pgfpathmoveto{\pgfqpoint{8.098636in}{4.721527in}}%
\pgfpathlineto{\pgfqpoint{3.843451in}{5.872954in}}%
\pgfpathlineto{\pgfqpoint{1.186859in}{3.801742in}}%
\pgfusepath{stroke}%
\end{pgfscope}%
\begin{pgfscope}%
\pgfsetbuttcap%
\pgfsetroundjoin%
\pgfsetlinewidth{0.803000pt}%
\definecolor{currentstroke}{rgb}{0.690196,0.690196,0.690196}%
\pgfsetstrokecolor{currentstroke}%
\pgfsetdash{}{0pt}%
\pgfpathmoveto{\pgfqpoint{8.127079in}{5.342651in}}%
\pgfpathlineto{\pgfqpoint{3.836551in}{6.473053in}}%
\pgfpathlineto{\pgfqpoint{1.156171in}{4.439003in}}%
\pgfusepath{stroke}%
\end{pgfscope}%
\begin{pgfscope}%
\pgfsetbuttcap%
\pgfsetroundjoin%
\pgfsetlinewidth{0.803000pt}%
\definecolor{currentstroke}{rgb}{0.690196,0.690196,0.690196}%
\pgfsetstrokecolor{currentstroke}%
\pgfsetdash{}{0pt}%
\pgfpathmoveto{\pgfqpoint{8.156003in}{5.974284in}}%
\pgfpathlineto{\pgfqpoint{3.829539in}{7.082809in}}%
\pgfpathlineto{\pgfqpoint{1.124943in}{5.087470in}}%
\pgfusepath{stroke}%
\end{pgfscope}%
\begin{pgfscope}%
\pgfpathrectangle{\pgfqpoint{0.680860in}{0.078740in}}{\pgfqpoint{7.842520in}{7.842520in}}%
\pgfusepath{clip}%
\pgfsetrectcap%
\pgfsetroundjoin%
\pgfsetlinewidth{0.501875pt}%
\definecolor{currentstroke}{rgb}{0.980392,0.811765,0.352941}%
\pgfsetstrokecolor{currentstroke}%
\pgfsetstrokeopacity{0.100000}%
\pgfsetdash{}{0pt}%
\pgfpathmoveto{\pgfqpoint{4.708100in}{4.105980in}}%
\pgfusepath{stroke}%
\end{pgfscope}%
\begin{pgfscope}%
\pgfsetrectcap%
\pgfsetroundjoin%
\pgfsetlinewidth{0.803000pt}%
\definecolor{currentstroke}{rgb}{0.980392,0.811765,0.352941}%
\pgfsetstrokecolor{currentstroke}%
\pgfsetdash{}{0pt}%
\pgfpathmoveto{\pgfqpoint{8.008000in}{3.519774in}}%
\pgfpathlineto{\pgfqpoint{8.113523in}{3.489752in}}%
\pgfusepath{stroke}%
\end{pgfscope}%
\begin{pgfscope}%
\definecolor{textcolor}{rgb}{0.525490,0.694118,0.356863}%
\pgfsetstrokecolor{textcolor}%
\pgfsetfillcolor{textcolor}%
\pgftext[x=8.298841in,y=3.545518in,,top]{\color{textcolor}\sffamily\fontsize{18.000000}{9.600000}\selectfont $\displaystyle 0.5$}%
\end{pgfscope}%
\begin{pgfscope}%
\pgfpathrectangle{\pgfqpoint{0.680860in}{0.078740in}}{\pgfqpoint{7.842520in}{7.842520in}}%
\pgfusepath{clip}%
\pgfsetrectcap%
\pgfsetroundjoin%
\pgfsetlinewidth{0.501875pt}%
\definecolor{currentstroke}{rgb}{0.980392,0.811765,0.352941}%
\pgfsetstrokecolor{currentstroke}%
\pgfsetstrokeopacity{0.100000}%
\pgfsetdash{}{0pt}%
\pgfpathmoveto{\pgfqpoint{4.708100in}{4.105980in}}%
\pgfusepath{stroke}%
\end{pgfscope}%
\begin{pgfscope}%
\pgfsetrectcap%
\pgfsetroundjoin%
\pgfsetlinewidth{0.803000pt}%
\definecolor{currentstroke}{rgb}{0.980392,0.811765,0.352941}%
\pgfsetstrokecolor{currentstroke}%
\pgfsetdash{}{0pt}%
\pgfpathmoveto{\pgfqpoint{8.035215in}{4.120492in}}%
\pgfpathlineto{\pgfqpoint{8.141642in}{4.090947in}}%
\pgfusepath{stroke}%
\end{pgfscope}%
\begin{pgfscope}%
\definecolor{textcolor}{rgb}{0.525490,0.694118,0.356863}%
\pgfsetstrokecolor{textcolor}%
\pgfsetfillcolor{textcolor}%
\pgftext[x=8.328424in,y=4.145826in,,top]{\color{textcolor}\sffamily\fontsize{18.000000}{9.600000}\selectfont $\displaystyle 1.0$}%
\end{pgfscope}%
\begin{pgfscope}%
\pgfpathrectangle{\pgfqpoint{0.680860in}{0.078740in}}{\pgfqpoint{7.842520in}{7.842520in}}%
\pgfusepath{clip}%
\pgfsetrectcap%
\pgfsetroundjoin%
\pgfsetlinewidth{0.501875pt}%
\definecolor{currentstroke}{rgb}{0.980392,0.811765,0.352941}%
\pgfsetstrokecolor{currentstroke}%
\pgfsetstrokeopacity{0.100000}%
\pgfsetdash{}{0pt}%
\pgfpathmoveto{\pgfqpoint{4.708100in}{4.105980in}}%
\pgfusepath{stroke}%
\end{pgfscope}%
\begin{pgfscope}%
\pgfsetrectcap%
\pgfsetroundjoin%
\pgfsetlinewidth{0.803000pt}%
\definecolor{currentstroke}{rgb}{0.980392,0.811765,0.352941}%
\pgfsetstrokecolor{currentstroke}%
\pgfsetdash{}{0pt}%
\pgfpathmoveto{\pgfqpoint{8.062883in}{4.731201in}}%
\pgfpathlineto{\pgfqpoint{8.170229in}{4.702154in}}%
\pgfusepath{stroke}%
\end{pgfscope}%
\begin{pgfscope}%
\definecolor{textcolor}{rgb}{0.525490,0.694118,0.356863}%
\pgfsetstrokecolor{textcolor}%
\pgfsetfillcolor{textcolor}%
\pgftext[x=8.358498in,y=4.756108in,,top]{\color{textcolor}\sffamily\fontsize{18.000000}{9.600000}\selectfont $\displaystyle 1.5$}%
\end{pgfscope}%
\begin{pgfscope}%
\pgfpathrectangle{\pgfqpoint{0.680860in}{0.078740in}}{\pgfqpoint{7.842520in}{7.842520in}}%
\pgfusepath{clip}%
\pgfsetrectcap%
\pgfsetroundjoin%
\pgfsetlinewidth{0.501875pt}%
\definecolor{currentstroke}{rgb}{0.980392,0.811765,0.352941}%
\pgfsetstrokecolor{currentstroke}%
\pgfsetstrokeopacity{0.100000}%
\pgfsetdash{}{0pt}%
\pgfpathmoveto{\pgfqpoint{4.708100in}{4.105980in}}%
\pgfusepath{stroke}%
\end{pgfscope}%
\begin{pgfscope}%
\pgfsetrectcap%
\pgfsetroundjoin%
\pgfsetlinewidth{0.803000pt}%
\definecolor{currentstroke}{rgb}{0.980392,0.811765,0.352941}%
\pgfsetstrokecolor{currentstroke}%
\pgfsetdash{}{0pt}%
\pgfpathmoveto{\pgfqpoint{8.091015in}{5.352153in}}%
\pgfpathlineto{\pgfqpoint{8.199297in}{5.323624in}}%
\pgfusepath{stroke}%
\end{pgfscope}%
\begin{pgfscope}%
\definecolor{textcolor}{rgb}{0.525490,0.694118,0.356863}%
\pgfsetstrokecolor{textcolor}%
\pgfsetfillcolor{textcolor}%
\pgftext[x=8.389077in,y=5.376614in,,top]{\color{textcolor}\sffamily\fontsize{18.000000}{9.600000}\selectfont $\displaystyle 2.0$}%
\end{pgfscope}%
\begin{pgfscope}%
\pgfpathrectangle{\pgfqpoint{0.680860in}{0.078740in}}{\pgfqpoint{7.842520in}{7.842520in}}%
\pgfusepath{clip}%
\pgfsetrectcap%
\pgfsetroundjoin%
\pgfsetlinewidth{0.501875pt}%
\definecolor{currentstroke}{rgb}{0.980392,0.811765,0.352941}%
\pgfsetstrokecolor{currentstroke}%
\pgfsetstrokeopacity{0.100000}%
\pgfsetdash{}{0pt}%
\pgfpathmoveto{\pgfqpoint{4.708100in}{4.105980in}}%
\pgfusepath{stroke}%
\end{pgfscope}%
\begin{pgfscope}%
\pgfsetrectcap%
\pgfsetroundjoin%
\pgfsetlinewidth{0.803000pt}%
\definecolor{currentstroke}{rgb}{0.980392,0.811765,0.352941}%
\pgfsetstrokecolor{currentstroke}%
\pgfsetdash{}{0pt}%
\pgfpathmoveto{\pgfqpoint{8.119622in}{5.983606in}}%
\pgfpathlineto{\pgfqpoint{8.228856in}{5.955618in}}%
\pgfusepath{stroke}%
\end{pgfscope}%
\begin{pgfscope}%
\definecolor{textcolor}{rgb}{0.525490,0.694118,0.356863}%
\pgfsetstrokecolor{textcolor}%
\pgfsetfillcolor{textcolor}%
\pgftext[x=8.420172in,y=6.007603in,,top]{\color{textcolor}\sffamily\fontsize{18.000000}{9.600000}\selectfont $\displaystyle 2.5$}%
\end{pgfscope}%
\begin{pgfscope}%
\pgfpathrectangle{\pgfqpoint{0.680860in}{0.078740in}}{\pgfqpoint{7.842520in}{7.842520in}}%
\pgfusepath{clip}%
\pgfsetbuttcap%
\pgfsetroundjoin%
\definecolor{currentfill}{rgb}{0.271305,0.019942,0.347269}%
\pgfsetfillcolor{currentfill}%
\pgfsetlinewidth{0.000000pt}%
\definecolor{currentstroke}{rgb}{0.267004,0.004874,0.329415}%
\pgfsetstrokecolor{currentstroke}%
\pgfsetdash{}{0pt}%
\pgfpathmoveto{\pgfqpoint{3.943776in}{4.201288in}}%
\pgfpathlineto{\pgfqpoint{4.021275in}{4.147144in}}%
\pgfpathlineto{\pgfqpoint{3.893926in}{4.180482in}}%
\pgfpathclose%
\pgfusepath{fill}%
\end{pgfscope}%
\begin{pgfscope}%
\pgfpathrectangle{\pgfqpoint{0.680860in}{0.078740in}}{\pgfqpoint{7.842520in}{7.842520in}}%
\pgfusepath{clip}%
\pgfsetbuttcap%
\pgfsetroundjoin%
\definecolor{currentfill}{rgb}{0.274952,0.037752,0.364543}%
\pgfsetfillcolor{currentfill}%
\pgfsetlinewidth{0.000000pt}%
\definecolor{currentstroke}{rgb}{0.268510,0.009605,0.335427}%
\pgfsetstrokecolor{currentstroke}%
\pgfsetdash{}{0pt}%
\pgfpathmoveto{\pgfqpoint{3.893926in}{4.180482in}}%
\pgfpathlineto{\pgfqpoint{3.816052in}{4.231025in}}%
\pgfpathlineto{\pgfqpoint{3.943776in}{4.201288in}}%
\pgfpathclose%
\pgfusepath{fill}%
\end{pgfscope}%
\begin{pgfscope}%
\pgfpathrectangle{\pgfqpoint{0.680860in}{0.078740in}}{\pgfqpoint{7.842520in}{7.842520in}}%
\pgfusepath{clip}%
\pgfsetbuttcap%
\pgfsetroundjoin%
\definecolor{currentfill}{rgb}{0.271305,0.019942,0.347269}%
\pgfsetfillcolor{currentfill}%
\pgfsetlinewidth{0.000000pt}%
\definecolor{currentstroke}{rgb}{0.269944,0.014625,0.341379}%
\pgfsetstrokecolor{currentstroke}%
\pgfsetdash{}{0pt}%
\pgfpathmoveto{\pgfqpoint{4.149023in}{4.113818in}}%
\pgfpathlineto{\pgfqpoint{4.021275in}{4.147144in}}%
\pgfpathlineto{\pgfqpoint{3.943776in}{4.201288in}}%
\pgfpathclose%
\pgfusepath{fill}%
\end{pgfscope}%
\begin{pgfscope}%
\pgfpathrectangle{\pgfqpoint{0.680860in}{0.078740in}}{\pgfqpoint{7.842520in}{7.842520in}}%
\pgfusepath{clip}%
\pgfsetbuttcap%
\pgfsetroundjoin%
\definecolor{currentfill}{rgb}{0.276022,0.044167,0.370164}%
\pgfsetfillcolor{currentfill}%
\pgfsetlinewidth{0.000000pt}%
\definecolor{currentstroke}{rgb}{0.271305,0.019942,0.347269}%
\pgfsetstrokecolor{currentstroke}%
\pgfsetdash{}{0pt}%
\pgfpathmoveto{\pgfqpoint{4.071913in}{4.171686in}}%
\pgfpathlineto{\pgfqpoint{4.149023in}{4.113818in}}%
\pgfpathlineto{\pgfqpoint{3.943776in}{4.201288in}}%
\pgfpathclose%
\pgfusepath{fill}%
\end{pgfscope}%
\begin{pgfscope}%
\pgfpathrectangle{\pgfqpoint{0.680860in}{0.078740in}}{\pgfqpoint{7.842520in}{7.842520in}}%
\pgfusepath{clip}%
\pgfsetbuttcap%
\pgfsetroundjoin%
\definecolor{currentfill}{rgb}{0.272594,0.025563,0.353093}%
\pgfsetfillcolor{currentfill}%
\pgfsetlinewidth{0.000000pt}%
\definecolor{currentstroke}{rgb}{0.272594,0.025563,0.353093}%
\pgfsetstrokecolor{currentstroke}%
\pgfsetdash{}{0pt}%
\pgfpathmoveto{\pgfqpoint{4.200467in}{4.142226in}}%
\pgfpathlineto{\pgfqpoint{4.277172in}{4.080506in}}%
\pgfpathlineto{\pgfqpoint{4.149023in}{4.113818in}}%
\pgfpathclose%
\pgfusepath{fill}%
\end{pgfscope}%
\begin{pgfscope}%
\pgfpathrectangle{\pgfqpoint{0.680860in}{0.078740in}}{\pgfqpoint{7.842520in}{7.842520in}}%
\pgfusepath{clip}%
\pgfsetbuttcap%
\pgfsetroundjoin%
\definecolor{currentfill}{rgb}{0.277018,0.050344,0.375715}%
\pgfsetfillcolor{currentfill}%
\pgfsetlinewidth{0.000000pt}%
\definecolor{currentstroke}{rgb}{0.273809,0.031497,0.358853}%
\pgfsetstrokecolor{currentstroke}%
\pgfsetdash{}{0pt}%
\pgfpathmoveto{\pgfqpoint{4.200467in}{4.142226in}}%
\pgfpathlineto{\pgfqpoint{4.149023in}{4.113818in}}%
\pgfpathlineto{\pgfqpoint{4.071913in}{4.171686in}}%
\pgfpathclose%
\pgfusepath{fill}%
\end{pgfscope}%
\begin{pgfscope}%
\pgfpathrectangle{\pgfqpoint{0.680860in}{0.078740in}}{\pgfqpoint{7.842520in}{7.842520in}}%
\pgfusepath{clip}%
\pgfsetbuttcap%
\pgfsetroundjoin%
\definecolor{currentfill}{rgb}{0.280894,0.078907,0.402329}%
\pgfsetfillcolor{currentfill}%
\pgfsetlinewidth{0.000000pt}%
\definecolor{currentstroke}{rgb}{0.274952,0.037752,0.364543}%
\pgfsetstrokecolor{currentstroke}%
\pgfsetdash{}{0pt}%
\pgfpathmoveto{\pgfqpoint{3.865641in}{4.253585in}}%
\pgfpathlineto{\pgfqpoint{3.943776in}{4.201288in}}%
\pgfpathlineto{\pgfqpoint{3.816052in}{4.231025in}}%
\pgfpathclose%
\pgfusepath{fill}%
\end{pgfscope}%
\begin{pgfscope}%
\pgfpathrectangle{\pgfqpoint{0.680860in}{0.078740in}}{\pgfqpoint{7.842520in}{7.842520in}}%
\pgfusepath{clip}%
\pgfsetbuttcap%
\pgfsetroundjoin%
\definecolor{currentfill}{rgb}{0.272594,0.025563,0.353093}%
\pgfsetfillcolor{currentfill}%
\pgfsetlinewidth{0.000000pt}%
\definecolor{currentstroke}{rgb}{0.276022,0.044167,0.370164}%
\pgfsetstrokecolor{currentstroke}%
\pgfsetdash{}{0pt}%
\pgfpathmoveto{\pgfqpoint{4.405725in}{4.047213in}}%
\pgfpathlineto{\pgfqpoint{4.277172in}{4.080506in}}%
\pgfpathlineto{\pgfqpoint{4.200467in}{4.142226in}}%
\pgfpathclose%
\pgfusepath{fill}%
\end{pgfscope}%
\begin{pgfscope}%
\pgfpathrectangle{\pgfqpoint{0.680860in}{0.078740in}}{\pgfqpoint{7.842520in}{7.842520in}}%
\pgfusepath{clip}%
\pgfsetbuttcap%
\pgfsetroundjoin%
\definecolor{currentfill}{rgb}{0.282656,0.100196,0.422160}%
\pgfsetfillcolor{currentfill}%
\pgfsetlinewidth{0.000000pt}%
\definecolor{currentstroke}{rgb}{0.277018,0.050344,0.375715}%
\pgfsetstrokecolor{currentstroke}%
\pgfsetdash{}{0pt}%
\pgfpathmoveto{\pgfqpoint{3.865641in}{4.253585in}}%
\pgfpathlineto{\pgfqpoint{3.816052in}{4.231025in}}%
\pgfpathlineto{\pgfqpoint{3.737551in}{4.279760in}}%
\pgfpathclose%
\pgfusepath{fill}%
\end{pgfscope}%
\begin{pgfscope}%
\pgfpathrectangle{\pgfqpoint{0.680860in}{0.078740in}}{\pgfqpoint{7.842520in}{7.842520in}}%
\pgfusepath{clip}%
\pgfsetbuttcap%
\pgfsetroundjoin%
\definecolor{currentfill}{rgb}{0.281446,0.084320,0.407414}%
\pgfsetfillcolor{currentfill}%
\pgfsetlinewidth{0.000000pt}%
\definecolor{currentstroke}{rgb}{0.277941,0.056324,0.381191}%
\pgfsetstrokecolor{currentstroke}%
\pgfsetdash{}{0pt}%
\pgfpathmoveto{\pgfqpoint{4.071913in}{4.171686in}}%
\pgfpathlineto{\pgfqpoint{3.943776in}{4.201288in}}%
\pgfpathlineto{\pgfqpoint{3.865641in}{4.253585in}}%
\pgfpathclose%
\pgfusepath{fill}%
\end{pgfscope}%
\begin{pgfscope}%
\pgfpathrectangle{\pgfqpoint{0.680860in}{0.078740in}}{\pgfqpoint{7.842520in}{7.842520in}}%
\pgfusepath{clip}%
\pgfsetbuttcap%
\pgfsetroundjoin%
\definecolor{currentfill}{rgb}{0.273809,0.031497,0.358853}%
\pgfsetfillcolor{currentfill}%
\pgfsetlinewidth{0.000000pt}%
\definecolor{currentstroke}{rgb}{0.278791,0.062145,0.386592}%
\pgfsetstrokecolor{currentstroke}%
\pgfsetdash{}{0pt}%
\pgfpathmoveto{\pgfqpoint{4.329441in}{4.112918in}}%
\pgfpathlineto{\pgfqpoint{4.534685in}{4.013942in}}%
\pgfpathlineto{\pgfqpoint{4.405725in}{4.047213in}}%
\pgfpathclose%
\pgfusepath{fill}%
\end{pgfscope}%
\begin{pgfscope}%
\pgfpathrectangle{\pgfqpoint{0.680860in}{0.078740in}}{\pgfqpoint{7.842520in}{7.842520in}}%
\pgfusepath{clip}%
\pgfsetbuttcap%
\pgfsetroundjoin%
\definecolor{currentfill}{rgb}{0.277018,0.050344,0.375715}%
\pgfsetfillcolor{currentfill}%
\pgfsetlinewidth{0.000000pt}%
\definecolor{currentstroke}{rgb}{0.279566,0.067836,0.391917}%
\pgfsetstrokecolor{currentstroke}%
\pgfsetdash{}{0pt}%
\pgfpathmoveto{\pgfqpoint{4.200467in}{4.142226in}}%
\pgfpathlineto{\pgfqpoint{4.329441in}{4.112918in}}%
\pgfpathlineto{\pgfqpoint{4.405725in}{4.047213in}}%
\pgfpathclose%
\pgfusepath{fill}%
\end{pgfscope}%
\begin{pgfscope}%
\pgfpathrectangle{\pgfqpoint{0.680860in}{0.078740in}}{\pgfqpoint{7.842520in}{7.842520in}}%
\pgfusepath{clip}%
\pgfsetbuttcap%
\pgfsetroundjoin%
\definecolor{currentfill}{rgb}{0.281924,0.089666,0.412415}%
\pgfsetfillcolor{currentfill}%
\pgfsetlinewidth{0.000000pt}%
\definecolor{currentstroke}{rgb}{0.280267,0.073417,0.397163}%
\pgfsetstrokecolor{currentstroke}%
\pgfsetdash{}{0pt}%
\pgfpathmoveto{\pgfqpoint{4.071913in}{4.171686in}}%
\pgfpathlineto{\pgfqpoint{3.994158in}{4.227669in}}%
\pgfpathlineto{\pgfqpoint{4.200467in}{4.142226in}}%
\pgfpathclose%
\pgfusepath{fill}%
\end{pgfscope}%
\begin{pgfscope}%
\pgfpathrectangle{\pgfqpoint{0.680860in}{0.078740in}}{\pgfqpoint{7.842520in}{7.842520in}}%
\pgfusepath{clip}%
\pgfsetbuttcap%
\pgfsetroundjoin%
\definecolor{currentfill}{rgb}{0.282910,0.105393,0.426902}%
\pgfsetfillcolor{currentfill}%
\pgfsetlinewidth{0.000000pt}%
\definecolor{currentstroke}{rgb}{0.280894,0.078907,0.402329}%
\pgfsetstrokecolor{currentstroke}%
\pgfsetdash{}{0pt}%
\pgfpathmoveto{\pgfqpoint{3.865641in}{4.253585in}}%
\pgfpathlineto{\pgfqpoint{3.994158in}{4.227669in}}%
\pgfpathlineto{\pgfqpoint{4.071913in}{4.171686in}}%
\pgfpathclose%
\pgfusepath{fill}%
\end{pgfscope}%
\begin{pgfscope}%
\pgfpathrectangle{\pgfqpoint{0.680860in}{0.078740in}}{\pgfqpoint{7.842520in}{7.842520in}}%
\pgfusepath{clip}%
\pgfsetbuttcap%
\pgfsetroundjoin%
\definecolor{currentfill}{rgb}{0.283072,0.130895,0.449241}%
\pgfsetfillcolor{currentfill}%
\pgfsetlinewidth{0.000000pt}%
\definecolor{currentstroke}{rgb}{0.281446,0.084320,0.407414}%
\pgfsetstrokecolor{currentstroke}%
\pgfsetdash{}{0pt}%
\pgfpathmoveto{\pgfqpoint{3.737551in}{4.279760in}}%
\pgfpathlineto{\pgfqpoint{3.658439in}{4.325563in}}%
\pgfpathlineto{\pgfqpoint{3.865641in}{4.253585in}}%
\pgfpathclose%
\pgfusepath{fill}%
\end{pgfscope}%
\begin{pgfscope}%
\pgfpathrectangle{\pgfqpoint{0.680860in}{0.078740in}}{\pgfqpoint{7.842520in}{7.842520in}}%
\pgfusepath{clip}%
\pgfsetbuttcap%
\pgfsetroundjoin%
\definecolor{currentfill}{rgb}{0.277941,0.056324,0.381191}%
\pgfsetfillcolor{currentfill}%
\pgfsetlinewidth{0.000000pt}%
\definecolor{currentstroke}{rgb}{0.281924,0.089666,0.412415}%
\pgfsetstrokecolor{currentstroke}%
\pgfsetdash{}{0pt}%
\pgfpathmoveto{\pgfqpoint{4.458838in}{4.083769in}}%
\pgfpathlineto{\pgfqpoint{4.534685in}{4.013942in}}%
\pgfpathlineto{\pgfqpoint{4.329441in}{4.112918in}}%
\pgfpathclose%
\pgfusepath{fill}%
\end{pgfscope}%
\begin{pgfscope}%
\pgfpathrectangle{\pgfqpoint{0.680860in}{0.078740in}}{\pgfqpoint{7.842520in}{7.842520in}}%
\pgfusepath{clip}%
\pgfsetbuttcap%
\pgfsetroundjoin%
\definecolor{currentfill}{rgb}{0.274952,0.037752,0.364543}%
\pgfsetfillcolor{currentfill}%
\pgfsetlinewidth{0.000000pt}%
\definecolor{currentstroke}{rgb}{0.282327,0.094955,0.417331}%
\pgfsetstrokecolor{currentstroke}%
\pgfsetdash{}{0pt}%
\pgfpathmoveto{\pgfqpoint{4.664054in}{3.980699in}}%
\pgfpathlineto{\pgfqpoint{4.534685in}{4.013942in}}%
\pgfpathlineto{\pgfqpoint{4.588661in}{4.054789in}}%
\pgfpathclose%
\pgfusepath{fill}%
\end{pgfscope}%
\begin{pgfscope}%
\pgfpathrectangle{\pgfqpoint{0.680860in}{0.078740in}}{\pgfqpoint{7.842520in}{7.842520in}}%
\pgfusepath{clip}%
\pgfsetbuttcap%
\pgfsetroundjoin%
\definecolor{currentfill}{rgb}{0.283091,0.110553,0.431554}%
\pgfsetfillcolor{currentfill}%
\pgfsetlinewidth{0.000000pt}%
\definecolor{currentstroke}{rgb}{0.282656,0.100196,0.422160}%
\pgfsetstrokecolor{currentstroke}%
\pgfsetdash{}{0pt}%
\pgfpathmoveto{\pgfqpoint{4.200467in}{4.142226in}}%
\pgfpathlineto{\pgfqpoint{3.994158in}{4.227669in}}%
\pgfpathlineto{\pgfqpoint{4.123107in}{4.202024in}}%
\pgfpathclose%
\pgfusepath{fill}%
\end{pgfscope}%
\begin{pgfscope}%
\pgfpathrectangle{\pgfqpoint{0.680860in}{0.078740in}}{\pgfqpoint{7.842520in}{7.842520in}}%
\pgfusepath{clip}%
\pgfsetbuttcap%
\pgfsetroundjoin%
\definecolor{currentfill}{rgb}{0.277941,0.056324,0.381191}%
\pgfsetfillcolor{currentfill}%
\pgfsetlinewidth{0.000000pt}%
\definecolor{currentstroke}{rgb}{0.282910,0.105393,0.426902}%
\pgfsetstrokecolor{currentstroke}%
\pgfsetdash{}{0pt}%
\pgfpathmoveto{\pgfqpoint{4.588661in}{4.054789in}}%
\pgfpathlineto{\pgfqpoint{4.534685in}{4.013942in}}%
\pgfpathlineto{\pgfqpoint{4.458838in}{4.083769in}}%
\pgfpathclose%
\pgfusepath{fill}%
\end{pgfscope}%
\begin{pgfscope}%
\pgfpathrectangle{\pgfqpoint{0.680860in}{0.078740in}}{\pgfqpoint{7.842520in}{7.842520in}}%
\pgfusepath{clip}%
\pgfsetbuttcap%
\pgfsetroundjoin%
\definecolor{currentfill}{rgb}{0.274952,0.037752,0.364543}%
\pgfsetfillcolor{currentfill}%
\pgfsetlinewidth{0.000000pt}%
\definecolor{currentstroke}{rgb}{0.283091,0.110553,0.431554}%
\pgfsetstrokecolor{currentstroke}%
\pgfsetdash{}{0pt}%
\pgfpathmoveto{\pgfqpoint{4.793834in}{3.947485in}}%
\pgfpathlineto{\pgfqpoint{4.664054in}{3.980699in}}%
\pgfpathlineto{\pgfqpoint{4.588661in}{4.054789in}}%
\pgfpathclose%
\pgfusepath{fill}%
\end{pgfscope}%
\begin{pgfscope}%
\pgfpathrectangle{\pgfqpoint{0.680860in}{0.078740in}}{\pgfqpoint{7.842520in}{7.842520in}}%
\pgfusepath{clip}%
\pgfsetbuttcap%
\pgfsetroundjoin%
\definecolor{currentfill}{rgb}{0.282327,0.094955,0.417331}%
\pgfsetfillcolor{currentfill}%
\pgfsetlinewidth{0.000000pt}%
\definecolor{currentstroke}{rgb}{0.283197,0.115680,0.436115}%
\pgfsetstrokecolor{currentstroke}%
\pgfsetdash{}{0pt}%
\pgfpathmoveto{\pgfqpoint{4.252492in}{4.176662in}}%
\pgfpathlineto{\pgfqpoint{4.329441in}{4.112918in}}%
\pgfpathlineto{\pgfqpoint{4.200467in}{4.142226in}}%
\pgfpathclose%
\pgfusepath{fill}%
\end{pgfscope}%
\begin{pgfscope}%
\pgfpathrectangle{\pgfqpoint{0.680860in}{0.078740in}}{\pgfqpoint{7.842520in}{7.842520in}}%
\pgfusepath{clip}%
\pgfsetbuttcap%
\pgfsetroundjoin%
\definecolor{currentfill}{rgb}{0.281887,0.150881,0.465405}%
\pgfsetfillcolor{currentfill}%
\pgfsetlinewidth{0.000000pt}%
\definecolor{currentstroke}{rgb}{0.283229,0.120777,0.440584}%
\pgfsetstrokecolor{currentstroke}%
\pgfsetdash{}{0pt}%
\pgfpathmoveto{\pgfqpoint{3.658439in}{4.325563in}}%
\pgfpathlineto{\pgfqpoint{3.786884in}{4.302875in}}%
\pgfpathlineto{\pgfqpoint{3.865641in}{4.253585in}}%
\pgfpathclose%
\pgfusepath{fill}%
\end{pgfscope}%
\begin{pgfscope}%
\pgfpathrectangle{\pgfqpoint{0.680860in}{0.078740in}}{\pgfqpoint{7.842520in}{7.842520in}}%
\pgfusepath{clip}%
\pgfsetbuttcap%
\pgfsetroundjoin%
\definecolor{currentfill}{rgb}{0.283091,0.110553,0.431554}%
\pgfsetfillcolor{currentfill}%
\pgfsetlinewidth{0.000000pt}%
\definecolor{currentstroke}{rgb}{0.283187,0.125848,0.444960}%
\pgfsetstrokecolor{currentstroke}%
\pgfsetdash{}{0pt}%
\pgfpathmoveto{\pgfqpoint{4.123107in}{4.202024in}}%
\pgfpathlineto{\pgfqpoint{4.252492in}{4.176662in}}%
\pgfpathlineto{\pgfqpoint{4.200467in}{4.142226in}}%
\pgfpathclose%
\pgfusepath{fill}%
\end{pgfscope}%
\begin{pgfscope}%
\pgfpathrectangle{\pgfqpoint{0.680860in}{0.078740in}}{\pgfqpoint{7.842520in}{7.842520in}}%
\pgfusepath{clip}%
\pgfsetbuttcap%
\pgfsetroundjoin%
\definecolor{currentfill}{rgb}{0.282623,0.140926,0.457517}%
\pgfsetfillcolor{currentfill}%
\pgfsetlinewidth{0.000000pt}%
\definecolor{currentstroke}{rgb}{0.283072,0.130895,0.449241}%
\pgfsetstrokecolor{currentstroke}%
\pgfsetdash{}{0pt}%
\pgfpathmoveto{\pgfqpoint{3.915771in}{4.280568in}}%
\pgfpathlineto{\pgfqpoint{3.994158in}{4.227669in}}%
\pgfpathlineto{\pgfqpoint{3.865641in}{4.253585in}}%
\pgfpathclose%
\pgfusepath{fill}%
\end{pgfscope}%
\begin{pgfscope}%
\pgfpathrectangle{\pgfqpoint{0.680860in}{0.078740in}}{\pgfqpoint{7.842520in}{7.842520in}}%
\pgfusepath{clip}%
\pgfsetbuttcap%
\pgfsetroundjoin%
\definecolor{currentfill}{rgb}{0.282656,0.100196,0.422160}%
\pgfsetfillcolor{currentfill}%
\pgfsetlinewidth{0.000000pt}%
\definecolor{currentstroke}{rgb}{0.282884,0.135920,0.453427}%
\pgfsetstrokecolor{currentstroke}%
\pgfsetdash{}{0pt}%
\pgfpathmoveto{\pgfqpoint{4.329441in}{4.112918in}}%
\pgfpathlineto{\pgfqpoint{4.382316in}{4.151596in}}%
\pgfpathlineto{\pgfqpoint{4.458838in}{4.083769in}}%
\pgfpathclose%
\pgfusepath{fill}%
\end{pgfscope}%
\begin{pgfscope}%
\pgfpathrectangle{\pgfqpoint{0.680860in}{0.078740in}}{\pgfqpoint{7.842520in}{7.842520in}}%
\pgfusepath{clip}%
\pgfsetbuttcap%
\pgfsetroundjoin%
\definecolor{currentfill}{rgb}{0.281412,0.155834,0.469201}%
\pgfsetfillcolor{currentfill}%
\pgfsetlinewidth{0.000000pt}%
\definecolor{currentstroke}{rgb}{0.282623,0.140926,0.457517}%
\pgfsetstrokecolor{currentstroke}%
\pgfsetdash{}{0pt}%
\pgfpathmoveto{\pgfqpoint{3.865641in}{4.253585in}}%
\pgfpathlineto{\pgfqpoint{3.786884in}{4.302875in}}%
\pgfpathlineto{\pgfqpoint{3.915771in}{4.280568in}}%
\pgfpathclose%
\pgfusepath{fill}%
\end{pgfscope}%
\begin{pgfscope}%
\pgfpathrectangle{\pgfqpoint{0.680860in}{0.078740in}}{\pgfqpoint{7.842520in}{7.842520in}}%
\pgfusepath{clip}%
\pgfsetbuttcap%
\pgfsetroundjoin%
\definecolor{currentfill}{rgb}{0.278791,0.062145,0.386592}%
\pgfsetfillcolor{currentfill}%
\pgfsetlinewidth{0.000000pt}%
\definecolor{currentstroke}{rgb}{0.282290,0.145912,0.461510}%
\pgfsetstrokecolor{currentstroke}%
\pgfsetdash{}{0pt}%
\pgfpathmoveto{\pgfqpoint{4.718915in}{4.025986in}}%
\pgfpathlineto{\pgfqpoint{4.793834in}{3.947485in}}%
\pgfpathlineto{\pgfqpoint{4.588661in}{4.054789in}}%
\pgfpathclose%
\pgfusepath{fill}%
\end{pgfscope}%
\begin{pgfscope}%
\pgfpathrectangle{\pgfqpoint{0.680860in}{0.078740in}}{\pgfqpoint{7.842520in}{7.842520in}}%
\pgfusepath{clip}%
\pgfsetbuttcap%
\pgfsetroundjoin%
\definecolor{currentfill}{rgb}{0.276022,0.044167,0.370164}%
\pgfsetfillcolor{currentfill}%
\pgfsetlinewidth{0.000000pt}%
\definecolor{currentstroke}{rgb}{0.281887,0.150881,0.465405}%
\pgfsetstrokecolor{currentstroke}%
\pgfsetdash{}{0pt}%
\pgfpathmoveto{\pgfqpoint{4.849603in}{3.997369in}}%
\pgfpathlineto{\pgfqpoint{4.924029in}{3.914307in}}%
\pgfpathlineto{\pgfqpoint{4.793834in}{3.947485in}}%
\pgfpathclose%
\pgfusepath{fill}%
\end{pgfscope}%
\begin{pgfscope}%
\pgfpathrectangle{\pgfqpoint{0.680860in}{0.078740in}}{\pgfqpoint{7.842520in}{7.842520in}}%
\pgfusepath{clip}%
\pgfsetbuttcap%
\pgfsetroundjoin%
\definecolor{currentfill}{rgb}{0.283197,0.115680,0.436115}%
\pgfsetfillcolor{currentfill}%
\pgfsetlinewidth{0.000000pt}%
\definecolor{currentstroke}{rgb}{0.281412,0.155834,0.469201}%
\pgfsetstrokecolor{currentstroke}%
\pgfsetdash{}{0pt}%
\pgfpathmoveto{\pgfqpoint{4.252492in}{4.176662in}}%
\pgfpathlineto{\pgfqpoint{4.382316in}{4.151596in}}%
\pgfpathlineto{\pgfqpoint{4.329441in}{4.112918in}}%
\pgfpathclose%
\pgfusepath{fill}%
\end{pgfscope}%
\begin{pgfscope}%
\pgfpathrectangle{\pgfqpoint{0.680860in}{0.078740in}}{\pgfqpoint{7.842520in}{7.842520in}}%
\pgfusepath{clip}%
\pgfsetbuttcap%
\pgfsetroundjoin%
\definecolor{currentfill}{rgb}{0.282290,0.145912,0.461510}%
\pgfsetfillcolor{currentfill}%
\pgfsetlinewidth{0.000000pt}%
\definecolor{currentstroke}{rgb}{0.280868,0.160771,0.472899}%
\pgfsetstrokecolor{currentstroke}%
\pgfsetdash{}{0pt}%
\pgfpathmoveto{\pgfqpoint{4.123107in}{4.202024in}}%
\pgfpathlineto{\pgfqpoint{3.994158in}{4.227669in}}%
\pgfpathlineto{\pgfqpoint{4.045104in}{4.258658in}}%
\pgfpathclose%
\pgfusepath{fill}%
\end{pgfscope}%
\begin{pgfscope}%
\pgfpathrectangle{\pgfqpoint{0.680860in}{0.078740in}}{\pgfqpoint{7.842520in}{7.842520in}}%
\pgfusepath{clip}%
\pgfsetbuttcap%
\pgfsetroundjoin%
\definecolor{currentfill}{rgb}{0.282656,0.100196,0.422160}%
\pgfsetfillcolor{currentfill}%
\pgfsetlinewidth{0.000000pt}%
\definecolor{currentstroke}{rgb}{0.280255,0.165693,0.476498}%
\pgfsetstrokecolor{currentstroke}%
\pgfsetdash{}{0pt}%
\pgfpathmoveto{\pgfqpoint{4.382316in}{4.151596in}}%
\pgfpathlineto{\pgfqpoint{4.588661in}{4.054789in}}%
\pgfpathlineto{\pgfqpoint{4.458838in}{4.083769in}}%
\pgfpathclose%
\pgfusepath{fill}%
\end{pgfscope}%
\begin{pgfscope}%
\pgfpathrectangle{\pgfqpoint{0.680860in}{0.078740in}}{\pgfqpoint{7.842520in}{7.842520in}}%
\pgfusepath{clip}%
\pgfsetbuttcap%
\pgfsetroundjoin%
\definecolor{currentfill}{rgb}{0.278012,0.180367,0.486697}%
\pgfsetfillcolor{currentfill}%
\pgfsetlinewidth{0.000000pt}%
\definecolor{currentstroke}{rgb}{0.279574,0.170599,0.479997}%
\pgfsetstrokecolor{currentstroke}%
\pgfsetdash{}{0pt}%
\pgfpathmoveto{\pgfqpoint{3.786884in}{4.302875in}}%
\pgfpathlineto{\pgfqpoint{3.658439in}{4.325563in}}%
\pgfpathlineto{\pgfqpoint{3.578733in}{4.367322in}}%
\pgfpathclose%
\pgfusepath{fill}%
\end{pgfscope}%
\begin{pgfscope}%
\pgfpathrectangle{\pgfqpoint{0.680860in}{0.078740in}}{\pgfqpoint{7.842520in}{7.842520in}}%
\pgfusepath{clip}%
\pgfsetbuttcap%
\pgfsetroundjoin%
\definecolor{currentfill}{rgb}{0.279566,0.067836,0.391917}%
\pgfsetfillcolor{currentfill}%
\pgfsetlinewidth{0.000000pt}%
\definecolor{currentstroke}{rgb}{0.278826,0.175490,0.483397}%
\pgfsetstrokecolor{currentstroke}%
\pgfsetdash{}{0pt}%
\pgfpathmoveto{\pgfqpoint{4.849603in}{3.997369in}}%
\pgfpathlineto{\pgfqpoint{4.793834in}{3.947485in}}%
\pgfpathlineto{\pgfqpoint{4.718915in}{4.025986in}}%
\pgfpathclose%
\pgfusepath{fill}%
\end{pgfscope}%
\begin{pgfscope}%
\pgfpathrectangle{\pgfqpoint{0.680860in}{0.078740in}}{\pgfqpoint{7.842520in}{7.842520in}}%
\pgfusepath{clip}%
\pgfsetbuttcap%
\pgfsetroundjoin%
\definecolor{currentfill}{rgb}{0.276022,0.044167,0.370164}%
\pgfsetfillcolor{currentfill}%
\pgfsetlinewidth{0.000000pt}%
\definecolor{currentstroke}{rgb}{0.278012,0.180367,0.486697}%
\pgfsetstrokecolor{currentstroke}%
\pgfsetdash{}{0pt}%
\pgfpathmoveto{\pgfqpoint{5.054641in}{3.881169in}}%
\pgfpathlineto{\pgfqpoint{4.924029in}{3.914307in}}%
\pgfpathlineto{\pgfqpoint{4.849603in}{3.997369in}}%
\pgfpathclose%
\pgfusepath{fill}%
\end{pgfscope}%
\begin{pgfscope}%
\pgfpathrectangle{\pgfqpoint{0.680860in}{0.078740in}}{\pgfqpoint{7.842520in}{7.842520in}}%
\pgfusepath{clip}%
\pgfsetbuttcap%
\pgfsetroundjoin%
\definecolor{currentfill}{rgb}{0.280868,0.160771,0.472899}%
\pgfsetfillcolor{currentfill}%
\pgfsetlinewidth{0.000000pt}%
\definecolor{currentstroke}{rgb}{0.277134,0.185228,0.489898}%
\pgfsetstrokecolor{currentstroke}%
\pgfsetdash{}{0pt}%
\pgfpathmoveto{\pgfqpoint{4.045104in}{4.258658in}}%
\pgfpathlineto{\pgfqpoint{3.994158in}{4.227669in}}%
\pgfpathlineto{\pgfqpoint{3.915771in}{4.280568in}}%
\pgfpathclose%
\pgfusepath{fill}%
\end{pgfscope}%
\begin{pgfscope}%
\pgfpathrectangle{\pgfqpoint{0.680860in}{0.078740in}}{\pgfqpoint{7.842520in}{7.842520in}}%
\pgfusepath{clip}%
\pgfsetbuttcap%
\pgfsetroundjoin%
\definecolor{currentfill}{rgb}{0.282290,0.145912,0.461510}%
\pgfsetfillcolor{currentfill}%
\pgfsetlinewidth{0.000000pt}%
\definecolor{currentstroke}{rgb}{0.276194,0.190074,0.493001}%
\pgfsetstrokecolor{currentstroke}%
\pgfsetdash{}{0pt}%
\pgfpathmoveto{\pgfqpoint{4.045104in}{4.258658in}}%
\pgfpathlineto{\pgfqpoint{4.252492in}{4.176662in}}%
\pgfpathlineto{\pgfqpoint{4.123107in}{4.202024in}}%
\pgfpathclose%
\pgfusepath{fill}%
\end{pgfscope}%
\begin{pgfscope}%
\pgfpathrectangle{\pgfqpoint{0.680860in}{0.078740in}}{\pgfqpoint{7.842520in}{7.842520in}}%
\pgfusepath{clip}%
\pgfsetbuttcap%
\pgfsetroundjoin%
\definecolor{currentfill}{rgb}{0.276022,0.044167,0.370164}%
\pgfsetfillcolor{currentfill}%
\pgfsetlinewidth{0.000000pt}%
\definecolor{currentstroke}{rgb}{0.275191,0.194905,0.496005}%
\pgfsetstrokecolor{currentstroke}%
\pgfsetdash{}{0pt}%
\pgfpathmoveto{\pgfqpoint{4.980728in}{3.968949in}}%
\pgfpathlineto{\pgfqpoint{5.185673in}{3.848075in}}%
\pgfpathlineto{\pgfqpoint{5.054641in}{3.881169in}}%
\pgfpathclose%
\pgfusepath{fill}%
\end{pgfscope}%
\begin{pgfscope}%
\pgfpathrectangle{\pgfqpoint{0.680860in}{0.078740in}}{\pgfqpoint{7.842520in}{7.842520in}}%
\pgfusepath{clip}%
\pgfsetbuttcap%
\pgfsetroundjoin%
\definecolor{currentfill}{rgb}{0.275191,0.194905,0.496005}%
\pgfsetfillcolor{currentfill}%
\pgfsetlinewidth{0.000000pt}%
\definecolor{currentstroke}{rgb}{0.274128,0.199721,0.498911}%
\pgfsetstrokecolor{currentstroke}%
\pgfsetdash{}{0pt}%
\pgfpathmoveto{\pgfqpoint{3.578733in}{4.367322in}}%
\pgfpathlineto{\pgfqpoint{3.707523in}{4.348009in}}%
\pgfpathlineto{\pgfqpoint{3.786884in}{4.302875in}}%
\pgfpathclose%
\pgfusepath{fill}%
\end{pgfscope}%
\begin{pgfscope}%
\pgfpathrectangle{\pgfqpoint{0.680860in}{0.078740in}}{\pgfqpoint{7.842520in}{7.842520in}}%
\pgfusepath{clip}%
\pgfsetbuttcap%
\pgfsetroundjoin%
\definecolor{currentfill}{rgb}{0.283229,0.120777,0.440584}%
\pgfsetfillcolor{currentfill}%
\pgfsetlinewidth{0.000000pt}%
\definecolor{currentstroke}{rgb}{0.273006,0.204520,0.501721}%
\pgfsetstrokecolor{currentstroke}%
\pgfsetdash{}{0pt}%
\pgfpathmoveto{\pgfqpoint{4.512584in}{4.126840in}}%
\pgfpathlineto{\pgfqpoint{4.588661in}{4.054789in}}%
\pgfpathlineto{\pgfqpoint{4.382316in}{4.151596in}}%
\pgfpathclose%
\pgfusepath{fill}%
\end{pgfscope}%
\begin{pgfscope}%
\pgfpathrectangle{\pgfqpoint{0.680860in}{0.078740in}}{\pgfqpoint{7.842520in}{7.842520in}}%
\pgfusepath{clip}%
\pgfsetbuttcap%
\pgfsetroundjoin%
\definecolor{currentfill}{rgb}{0.282910,0.105393,0.426902}%
\pgfsetfillcolor{currentfill}%
\pgfsetlinewidth{0.000000pt}%
\definecolor{currentstroke}{rgb}{0.271828,0.209303,0.504434}%
\pgfsetstrokecolor{currentstroke}%
\pgfsetdash{}{0pt}%
\pgfpathmoveto{\pgfqpoint{4.718915in}{4.025986in}}%
\pgfpathlineto{\pgfqpoint{4.588661in}{4.054789in}}%
\pgfpathlineto{\pgfqpoint{4.643301in}{4.102408in}}%
\pgfpathclose%
\pgfusepath{fill}%
\end{pgfscope}%
\begin{pgfscope}%
\pgfpathrectangle{\pgfqpoint{0.680860in}{0.078740in}}{\pgfqpoint{7.842520in}{7.842520in}}%
\pgfusepath{clip}%
\pgfsetbuttcap%
\pgfsetroundjoin%
\definecolor{currentfill}{rgb}{0.276194,0.190074,0.493001}%
\pgfsetfillcolor{currentfill}%
\pgfsetlinewidth{0.000000pt}%
\definecolor{currentstroke}{rgb}{0.270595,0.214069,0.507052}%
\pgfsetstrokecolor{currentstroke}%
\pgfsetdash{}{0pt}%
\pgfpathmoveto{\pgfqpoint{3.786884in}{4.302875in}}%
\pgfpathlineto{\pgfqpoint{3.836768in}{4.329195in}}%
\pgfpathlineto{\pgfqpoint{3.915771in}{4.280568in}}%
\pgfpathclose%
\pgfusepath{fill}%
\end{pgfscope}%
\begin{pgfscope}%
\pgfpathrectangle{\pgfqpoint{0.680860in}{0.078740in}}{\pgfqpoint{7.842520in}{7.842520in}}%
\pgfusepath{clip}%
\pgfsetbuttcap%
\pgfsetroundjoin%
\definecolor{currentfill}{rgb}{0.279566,0.067836,0.391917}%
\pgfsetfillcolor{currentfill}%
\pgfsetlinewidth{0.000000pt}%
\definecolor{currentstroke}{rgb}{0.269308,0.218818,0.509577}%
\pgfsetstrokecolor{currentstroke}%
\pgfsetdash{}{0pt}%
\pgfpathmoveto{\pgfqpoint{4.849603in}{3.997369in}}%
\pgfpathlineto{\pgfqpoint{4.980728in}{3.968949in}}%
\pgfpathlineto{\pgfqpoint{5.054641in}{3.881169in}}%
\pgfpathclose%
\pgfusepath{fill}%
\end{pgfscope}%
\begin{pgfscope}%
\pgfpathrectangle{\pgfqpoint{0.680860in}{0.078740in}}{\pgfqpoint{7.842520in}{7.842520in}}%
\pgfusepath{clip}%
\pgfsetbuttcap%
\pgfsetroundjoin%
\definecolor{currentfill}{rgb}{0.280255,0.165693,0.476498}%
\pgfsetfillcolor{currentfill}%
\pgfsetlinewidth{0.000000pt}%
\definecolor{currentstroke}{rgb}{0.267968,0.223549,0.512008}%
\pgfsetstrokecolor{currentstroke}%
\pgfsetdash{}{0pt}%
\pgfpathmoveto{\pgfqpoint{4.045104in}{4.258658in}}%
\pgfpathlineto{\pgfqpoint{4.174888in}{4.237162in}}%
\pgfpathlineto{\pgfqpoint{4.252492in}{4.176662in}}%
\pgfpathclose%
\pgfusepath{fill}%
\end{pgfscope}%
\begin{pgfscope}%
\pgfpathrectangle{\pgfqpoint{0.680860in}{0.078740in}}{\pgfqpoint{7.842520in}{7.842520in}}%
\pgfusepath{clip}%
\pgfsetbuttcap%
\pgfsetroundjoin%
\definecolor{currentfill}{rgb}{0.283187,0.125848,0.444960}%
\pgfsetfillcolor{currentfill}%
\pgfsetlinewidth{0.000000pt}%
\definecolor{currentstroke}{rgb}{0.266580,0.228262,0.514349}%
\pgfsetstrokecolor{currentstroke}%
\pgfsetdash{}{0pt}%
\pgfpathmoveto{\pgfqpoint{4.643301in}{4.102408in}}%
\pgfpathlineto{\pgfqpoint{4.588661in}{4.054789in}}%
\pgfpathlineto{\pgfqpoint{4.512584in}{4.126840in}}%
\pgfpathclose%
\pgfusepath{fill}%
\end{pgfscope}%
\begin{pgfscope}%
\pgfpathrectangle{\pgfqpoint{0.680860in}{0.078740in}}{\pgfqpoint{7.842520in}{7.842520in}}%
\pgfusepath{clip}%
\pgfsetbuttcap%
\pgfsetroundjoin%
\definecolor{currentfill}{rgb}{0.274128,0.199721,0.498911}%
\pgfsetfillcolor{currentfill}%
\pgfsetlinewidth{0.000000pt}%
\definecolor{currentstroke}{rgb}{0.265145,0.232956,0.516599}%
\pgfsetstrokecolor{currentstroke}%
\pgfsetdash{}{0pt}%
\pgfpathmoveto{\pgfqpoint{3.707523in}{4.348009in}}%
\pgfpathlineto{\pgfqpoint{3.836768in}{4.329195in}}%
\pgfpathlineto{\pgfqpoint{3.786884in}{4.302875in}}%
\pgfpathclose%
\pgfusepath{fill}%
\end{pgfscope}%
\begin{pgfscope}%
\pgfpathrectangle{\pgfqpoint{0.680860in}{0.078740in}}{\pgfqpoint{7.842520in}{7.842520in}}%
\pgfusepath{clip}%
\pgfsetbuttcap%
\pgfsetroundjoin%
\definecolor{currentfill}{rgb}{0.281412,0.155834,0.469201}%
\pgfsetfillcolor{currentfill}%
\pgfsetlinewidth{0.000000pt}%
\definecolor{currentstroke}{rgb}{0.263663,0.237631,0.518762}%
\pgfsetstrokecolor{currentstroke}%
\pgfsetdash{}{0pt}%
\pgfpathmoveto{\pgfqpoint{4.305127in}{4.216098in}}%
\pgfpathlineto{\pgfqpoint{4.382316in}{4.151596in}}%
\pgfpathlineto{\pgfqpoint{4.252492in}{4.176662in}}%
\pgfpathclose%
\pgfusepath{fill}%
\end{pgfscope}%
\begin{pgfscope}%
\pgfpathrectangle{\pgfqpoint{0.680860in}{0.078740in}}{\pgfqpoint{7.842520in}{7.842520in}}%
\pgfusepath{clip}%
\pgfsetbuttcap%
\pgfsetroundjoin%
\definecolor{currentfill}{rgb}{0.283091,0.110553,0.431554}%
\pgfsetfillcolor{currentfill}%
\pgfsetlinewidth{0.000000pt}%
\definecolor{currentstroke}{rgb}{0.262138,0.242286,0.520837}%
\pgfsetstrokecolor{currentstroke}%
\pgfsetdash{}{0pt}%
\pgfpathmoveto{\pgfqpoint{4.718915in}{4.025986in}}%
\pgfpathlineto{\pgfqpoint{4.643301in}{4.102408in}}%
\pgfpathlineto{\pgfqpoint{4.849603in}{3.997369in}}%
\pgfpathclose%
\pgfusepath{fill}%
\end{pgfscope}%
\begin{pgfscope}%
\pgfpathrectangle{\pgfqpoint{0.680860in}{0.078740in}}{\pgfqpoint{7.842520in}{7.842520in}}%
\pgfusepath{clip}%
\pgfsetbuttcap%
\pgfsetroundjoin%
\definecolor{currentfill}{rgb}{0.276194,0.190074,0.493001}%
\pgfsetfillcolor{currentfill}%
\pgfsetlinewidth{0.000000pt}%
\definecolor{currentstroke}{rgb}{0.260571,0.246922,0.522828}%
\pgfsetstrokecolor{currentstroke}%
\pgfsetdash{}{0pt}%
\pgfpathmoveto{\pgfqpoint{3.915771in}{4.280568in}}%
\pgfpathlineto{\pgfqpoint{3.836768in}{4.329195in}}%
\pgfpathlineto{\pgfqpoint{4.045104in}{4.258658in}}%
\pgfpathclose%
\pgfusepath{fill}%
\end{pgfscope}%
\begin{pgfscope}%
\pgfpathrectangle{\pgfqpoint{0.680860in}{0.078740in}}{\pgfqpoint{7.842520in}{7.842520in}}%
\pgfusepath{clip}%
\pgfsetbuttcap%
\pgfsetroundjoin%
\definecolor{currentfill}{rgb}{0.280267,0.073417,0.397163}%
\pgfsetfillcolor{currentfill}%
\pgfsetlinewidth{0.000000pt}%
\definecolor{currentstroke}{rgb}{0.258965,0.251537,0.524736}%
\pgfsetstrokecolor{currentstroke}%
\pgfsetdash{}{0pt}%
\pgfpathmoveto{\pgfqpoint{5.112294in}{3.940735in}}%
\pgfpathlineto{\pgfqpoint{5.185673in}{3.848075in}}%
\pgfpathlineto{\pgfqpoint{4.980728in}{3.968949in}}%
\pgfpathclose%
\pgfusepath{fill}%
\end{pgfscope}%
\begin{pgfscope}%
\pgfpathrectangle{\pgfqpoint{0.680860in}{0.078740in}}{\pgfqpoint{7.842520in}{7.842520in}}%
\pgfusepath{clip}%
\pgfsetbuttcap%
\pgfsetroundjoin%
\definecolor{currentfill}{rgb}{0.267968,0.223549,0.512008}%
\pgfsetfillcolor{currentfill}%
\pgfsetlinewidth{0.000000pt}%
\definecolor{currentstroke}{rgb}{0.257322,0.256130,0.526563}%
\pgfsetstrokecolor{currentstroke}%
\pgfsetdash{}{0pt}%
\pgfpathmoveto{\pgfqpoint{3.578733in}{4.367322in}}%
\pgfpathlineto{\pgfqpoint{3.498460in}{4.403950in}}%
\pgfpathlineto{\pgfqpoint{3.707523in}{4.348009in}}%
\pgfpathclose%
\pgfusepath{fill}%
\end{pgfscope}%
\begin{pgfscope}%
\pgfpathrectangle{\pgfqpoint{0.680860in}{0.078740in}}{\pgfqpoint{7.842520in}{7.842520in}}%
\pgfusepath{clip}%
\pgfsetbuttcap%
\pgfsetroundjoin%
\definecolor{currentfill}{rgb}{0.277018,0.050344,0.375715}%
\pgfsetfillcolor{currentfill}%
\pgfsetlinewidth{0.000000pt}%
\definecolor{currentstroke}{rgb}{0.255645,0.260703,0.528312}%
\pgfsetstrokecolor{currentstroke}%
\pgfsetdash{}{0pt}%
\pgfpathmoveto{\pgfqpoint{5.317128in}{3.815030in}}%
\pgfpathlineto{\pgfqpoint{5.185673in}{3.848075in}}%
\pgfpathlineto{\pgfqpoint{5.244306in}{3.912738in}}%
\pgfpathclose%
\pgfusepath{fill}%
\end{pgfscope}%
\begin{pgfscope}%
\pgfpathrectangle{\pgfqpoint{0.680860in}{0.078740in}}{\pgfqpoint{7.842520in}{7.842520in}}%
\pgfusepath{clip}%
\pgfsetbuttcap%
\pgfsetroundjoin%
\definecolor{currentfill}{rgb}{0.279574,0.170599,0.479997}%
\pgfsetfillcolor{currentfill}%
\pgfsetlinewidth{0.000000pt}%
\definecolor{currentstroke}{rgb}{0.253935,0.265254,0.529983}%
\pgfsetstrokecolor{currentstroke}%
\pgfsetdash{}{0pt}%
\pgfpathmoveto{\pgfqpoint{4.252492in}{4.176662in}}%
\pgfpathlineto{\pgfqpoint{4.174888in}{4.237162in}}%
\pgfpathlineto{\pgfqpoint{4.305127in}{4.216098in}}%
\pgfpathclose%
\pgfusepath{fill}%
\end{pgfscope}%
\begin{pgfscope}%
\pgfpathrectangle{\pgfqpoint{0.680860in}{0.078740in}}{\pgfqpoint{7.842520in}{7.842520in}}%
\pgfusepath{clip}%
\pgfsetbuttcap%
\pgfsetroundjoin%
\definecolor{currentfill}{rgb}{0.280868,0.160771,0.472899}%
\pgfsetfillcolor{currentfill}%
\pgfsetlinewidth{0.000000pt}%
\definecolor{currentstroke}{rgb}{0.252194,0.269783,0.531579}%
\pgfsetstrokecolor{currentstroke}%
\pgfsetdash{}{0pt}%
\pgfpathmoveto{\pgfqpoint{4.512584in}{4.126840in}}%
\pgfpathlineto{\pgfqpoint{4.382316in}{4.151596in}}%
\pgfpathlineto{\pgfqpoint{4.435827in}{4.195483in}}%
\pgfpathclose%
\pgfusepath{fill}%
\end{pgfscope}%
\begin{pgfscope}%
\pgfpathrectangle{\pgfqpoint{0.680860in}{0.078740in}}{\pgfqpoint{7.842520in}{7.842520in}}%
\pgfusepath{clip}%
\pgfsetbuttcap%
\pgfsetroundjoin%
\definecolor{currentfill}{rgb}{0.277941,0.056324,0.381191}%
\pgfsetfillcolor{currentfill}%
\pgfsetlinewidth{0.000000pt}%
\definecolor{currentstroke}{rgb}{0.250425,0.274290,0.533103}%
\pgfsetstrokecolor{currentstroke}%
\pgfsetdash{}{0pt}%
\pgfpathmoveto{\pgfqpoint{5.244306in}{3.912738in}}%
\pgfpathlineto{\pgfqpoint{5.449009in}{3.782039in}}%
\pgfpathlineto{\pgfqpoint{5.317128in}{3.815030in}}%
\pgfpathclose%
\pgfusepath{fill}%
\end{pgfscope}%
\begin{pgfscope}%
\pgfpathrectangle{\pgfqpoint{0.680860in}{0.078740in}}{\pgfqpoint{7.842520in}{7.842520in}}%
\pgfusepath{clip}%
\pgfsetbuttcap%
\pgfsetroundjoin%
\definecolor{currentfill}{rgb}{0.280894,0.078907,0.402329}%
\pgfsetfillcolor{currentfill}%
\pgfsetlinewidth{0.000000pt}%
\definecolor{currentstroke}{rgb}{0.248629,0.278775,0.534556}%
\pgfsetstrokecolor{currentstroke}%
\pgfsetdash{}{0pt}%
\pgfpathmoveto{\pgfqpoint{5.244306in}{3.912738in}}%
\pgfpathlineto{\pgfqpoint{5.185673in}{3.848075in}}%
\pgfpathlineto{\pgfqpoint{5.112294in}{3.940735in}}%
\pgfpathclose%
\pgfusepath{fill}%
\end{pgfscope}%
\begin{pgfscope}%
\pgfpathrectangle{\pgfqpoint{0.680860in}{0.078740in}}{\pgfqpoint{7.842520in}{7.842520in}}%
\pgfusepath{clip}%
\pgfsetbuttcap%
\pgfsetroundjoin%
\definecolor{currentfill}{rgb}{0.282884,0.135920,0.453427}%
\pgfsetfillcolor{currentfill}%
\pgfsetlinewidth{0.000000pt}%
\definecolor{currentstroke}{rgb}{0.246811,0.283237,0.535941}%
\pgfsetstrokecolor{currentstroke}%
\pgfsetdash{}{0pt}%
\pgfpathmoveto{\pgfqpoint{4.643301in}{4.102408in}}%
\pgfpathlineto{\pgfqpoint{4.774470in}{4.078314in}}%
\pgfpathlineto{\pgfqpoint{4.849603in}{3.997369in}}%
\pgfpathclose%
\pgfusepath{fill}%
\end{pgfscope}%
\begin{pgfscope}%
\pgfpathrectangle{\pgfqpoint{0.680860in}{0.078740in}}{\pgfqpoint{7.842520in}{7.842520in}}%
\pgfusepath{clip}%
\pgfsetbuttcap%
\pgfsetroundjoin%
\definecolor{currentfill}{rgb}{0.271828,0.209303,0.504434}%
\pgfsetfillcolor{currentfill}%
\pgfsetlinewidth{0.000000pt}%
\definecolor{currentstroke}{rgb}{0.244972,0.287675,0.537260}%
\pgfsetstrokecolor{currentstroke}%
\pgfsetdash{}{0pt}%
\pgfpathmoveto{\pgfqpoint{3.836768in}{4.329195in}}%
\pgfpathlineto{\pgfqpoint{3.966472in}{4.310901in}}%
\pgfpathlineto{\pgfqpoint{4.045104in}{4.258658in}}%
\pgfpathclose%
\pgfusepath{fill}%
\end{pgfscope}%
\begin{pgfscope}%
\pgfpathrectangle{\pgfqpoint{0.680860in}{0.078740in}}{\pgfqpoint{7.842520in}{7.842520in}}%
\pgfusepath{clip}%
\pgfsetbuttcap%
\pgfsetroundjoin%
\definecolor{currentfill}{rgb}{0.283197,0.115680,0.436115}%
\pgfsetfillcolor{currentfill}%
\pgfsetlinewidth{0.000000pt}%
\definecolor{currentstroke}{rgb}{0.243113,0.292092,0.538516}%
\pgfsetstrokecolor{currentstroke}%
\pgfsetdash{}{0pt}%
\pgfpathmoveto{\pgfqpoint{4.980728in}{3.968949in}}%
\pgfpathlineto{\pgfqpoint{4.849603in}{3.997369in}}%
\pgfpathlineto{\pgfqpoint{4.906097in}{4.054573in}}%
\pgfpathclose%
\pgfusepath{fill}%
\end{pgfscope}%
\begin{pgfscope}%
\pgfpathrectangle{\pgfqpoint{0.680860in}{0.078740in}}{\pgfqpoint{7.842520in}{7.842520in}}%
\pgfusepath{clip}%
\pgfsetbuttcap%
\pgfsetroundjoin%
\definecolor{currentfill}{rgb}{0.273006,0.204520,0.501721}%
\pgfsetfillcolor{currentfill}%
\pgfsetlinewidth{0.000000pt}%
\definecolor{currentstroke}{rgb}{0.241237,0.296485,0.539709}%
\pgfsetstrokecolor{currentstroke}%
\pgfsetdash{}{0pt}%
\pgfpathmoveto{\pgfqpoint{4.096642in}{4.293149in}}%
\pgfpathlineto{\pgfqpoint{4.174888in}{4.237162in}}%
\pgfpathlineto{\pgfqpoint{4.045104in}{4.258658in}}%
\pgfpathclose%
\pgfusepath{fill}%
\end{pgfscope}%
\begin{pgfscope}%
\pgfpathrectangle{\pgfqpoint{0.680860in}{0.078740in}}{\pgfqpoint{7.842520in}{7.842520in}}%
\pgfusepath{clip}%
\pgfsetbuttcap%
\pgfsetroundjoin%
\definecolor{currentfill}{rgb}{0.278012,0.180367,0.486697}%
\pgfsetfillcolor{currentfill}%
\pgfsetlinewidth{0.000000pt}%
\definecolor{currentstroke}{rgb}{0.239346,0.300855,0.540844}%
\pgfsetstrokecolor{currentstroke}%
\pgfsetdash{}{0pt}%
\pgfpathmoveto{\pgfqpoint{4.382316in}{4.151596in}}%
\pgfpathlineto{\pgfqpoint{4.305127in}{4.216098in}}%
\pgfpathlineto{\pgfqpoint{4.435827in}{4.195483in}}%
\pgfpathclose%
\pgfusepath{fill}%
\end{pgfscope}%
\begin{pgfscope}%
\pgfpathrectangle{\pgfqpoint{0.680860in}{0.078740in}}{\pgfqpoint{7.842520in}{7.842520in}}%
\pgfusepath{clip}%
\pgfsetbuttcap%
\pgfsetroundjoin%
\definecolor{currentfill}{rgb}{0.263663,0.237631,0.518762}%
\pgfsetfillcolor{currentfill}%
\pgfsetlinewidth{0.000000pt}%
\definecolor{currentstroke}{rgb}{0.237441,0.305202,0.541921}%
\pgfsetstrokecolor{currentstroke}%
\pgfsetdash{}{0pt}%
\pgfpathmoveto{\pgfqpoint{3.707523in}{4.348009in}}%
\pgfpathlineto{\pgfqpoint{3.498460in}{4.403950in}}%
\pgfpathlineto{\pgfqpoint{3.627581in}{4.387860in}}%
\pgfpathclose%
\pgfusepath{fill}%
\end{pgfscope}%
\begin{pgfscope}%
\pgfpathrectangle{\pgfqpoint{0.680860in}{0.078740in}}{\pgfqpoint{7.842520in}{7.842520in}}%
\pgfusepath{clip}%
\pgfsetbuttcap%
\pgfsetroundjoin%
\definecolor{currentfill}{rgb}{0.280255,0.165693,0.476498}%
\pgfsetfillcolor{currentfill}%
\pgfsetlinewidth{0.000000pt}%
\definecolor{currentstroke}{rgb}{0.235526,0.309527,0.542944}%
\pgfsetstrokecolor{currentstroke}%
\pgfsetdash{}{0pt}%
\pgfpathmoveto{\pgfqpoint{4.435827in}{4.195483in}}%
\pgfpathlineto{\pgfqpoint{4.643301in}{4.102408in}}%
\pgfpathlineto{\pgfqpoint{4.512584in}{4.126840in}}%
\pgfpathclose%
\pgfusepath{fill}%
\end{pgfscope}%
\begin{pgfscope}%
\pgfpathrectangle{\pgfqpoint{0.680860in}{0.078740in}}{\pgfqpoint{7.842520in}{7.842520in}}%
\pgfusepath{clip}%
\pgfsetbuttcap%
\pgfsetroundjoin%
\definecolor{currentfill}{rgb}{0.265145,0.232956,0.516599}%
\pgfsetfillcolor{currentfill}%
\pgfsetlinewidth{0.000000pt}%
\definecolor{currentstroke}{rgb}{0.233603,0.313828,0.543914}%
\pgfsetstrokecolor{currentstroke}%
\pgfsetdash{}{0pt}%
\pgfpathmoveto{\pgfqpoint{3.757170in}{4.372384in}}%
\pgfpathlineto{\pgfqpoint{3.836768in}{4.329195in}}%
\pgfpathlineto{\pgfqpoint{3.707523in}{4.348009in}}%
\pgfpathclose%
\pgfusepath{fill}%
\end{pgfscope}%
\begin{pgfscope}%
\pgfpathrectangle{\pgfqpoint{0.680860in}{0.078740in}}{\pgfqpoint{7.842520in}{7.842520in}}%
\pgfusepath{clip}%
\pgfsetbuttcap%
\pgfsetroundjoin%
\definecolor{currentfill}{rgb}{0.282884,0.135920,0.453427}%
\pgfsetfillcolor{currentfill}%
\pgfsetlinewidth{0.000000pt}%
\definecolor{currentstroke}{rgb}{0.231674,0.318106,0.544834}%
\pgfsetstrokecolor{currentstroke}%
\pgfsetdash{}{0pt}%
\pgfpathmoveto{\pgfqpoint{4.906097in}{4.054573in}}%
\pgfpathlineto{\pgfqpoint{4.849603in}{3.997369in}}%
\pgfpathlineto{\pgfqpoint{4.774470in}{4.078314in}}%
\pgfpathclose%
\pgfusepath{fill}%
\end{pgfscope}%
\begin{pgfscope}%
\pgfpathrectangle{\pgfqpoint{0.680860in}{0.078740in}}{\pgfqpoint{7.842520in}{7.842520in}}%
\pgfusepath{clip}%
\pgfsetbuttcap%
\pgfsetroundjoin%
\definecolor{currentfill}{rgb}{0.283229,0.120777,0.440584}%
\pgfsetfillcolor{currentfill}%
\pgfsetlinewidth{0.000000pt}%
\definecolor{currentstroke}{rgb}{0.229739,0.322361,0.545706}%
\pgfsetstrokecolor{currentstroke}%
\pgfsetdash{}{0pt}%
\pgfpathmoveto{\pgfqpoint{5.112294in}{3.940735in}}%
\pgfpathlineto{\pgfqpoint{4.980728in}{3.968949in}}%
\pgfpathlineto{\pgfqpoint{4.906097in}{4.054573in}}%
\pgfpathclose%
\pgfusepath{fill}%
\end{pgfscope}%
\begin{pgfscope}%
\pgfpathrectangle{\pgfqpoint{0.680860in}{0.078740in}}{\pgfqpoint{7.842520in}{7.842520in}}%
\pgfusepath{clip}%
\pgfsetbuttcap%
\pgfsetroundjoin%
\definecolor{currentfill}{rgb}{0.270595,0.214069,0.507052}%
\pgfsetfillcolor{currentfill}%
\pgfsetlinewidth{0.000000pt}%
\definecolor{currentstroke}{rgb}{0.227802,0.326594,0.546532}%
\pgfsetstrokecolor{currentstroke}%
\pgfsetdash{}{0pt}%
\pgfpathmoveto{\pgfqpoint{4.045104in}{4.258658in}}%
\pgfpathlineto{\pgfqpoint{3.966472in}{4.310901in}}%
\pgfpathlineto{\pgfqpoint{4.096642in}{4.293149in}}%
\pgfpathclose%
\pgfusepath{fill}%
\end{pgfscope}%
\begin{pgfscope}%
\pgfpathrectangle{\pgfqpoint{0.680860in}{0.078740in}}{\pgfqpoint{7.842520in}{7.842520in}}%
\pgfusepath{clip}%
\pgfsetbuttcap%
\pgfsetroundjoin%
\definecolor{currentfill}{rgb}{0.260571,0.246922,0.522828}%
\pgfsetfillcolor{currentfill}%
\pgfsetlinewidth{0.000000pt}%
\definecolor{currentstroke}{rgb}{0.225863,0.330805,0.547314}%
\pgfsetstrokecolor{currentstroke}%
\pgfsetdash{}{0pt}%
\pgfpathmoveto{\pgfqpoint{3.757170in}{4.372384in}}%
\pgfpathlineto{\pgfqpoint{3.707523in}{4.348009in}}%
\pgfpathlineto{\pgfqpoint{3.627581in}{4.387860in}}%
\pgfpathclose%
\pgfusepath{fill}%
\end{pgfscope}%
\begin{pgfscope}%
\pgfpathrectangle{\pgfqpoint{0.680860in}{0.078740in}}{\pgfqpoint{7.842520in}{7.842520in}}%
\pgfusepath{clip}%
\pgfsetbuttcap%
\pgfsetroundjoin%
\definecolor{currentfill}{rgb}{0.281446,0.084320,0.407414}%
\pgfsetfillcolor{currentfill}%
\pgfsetlinewidth{0.000000pt}%
\definecolor{currentstroke}{rgb}{0.223925,0.334994,0.548053}%
\pgfsetstrokecolor{currentstroke}%
\pgfsetdash{}{0pt}%
\pgfpathmoveto{\pgfqpoint{5.376767in}{3.884968in}}%
\pgfpathlineto{\pgfqpoint{5.449009in}{3.782039in}}%
\pgfpathlineto{\pgfqpoint{5.244306in}{3.912738in}}%
\pgfpathclose%
\pgfusepath{fill}%
\end{pgfscope}%
\begin{pgfscope}%
\pgfpathrectangle{\pgfqpoint{0.680860in}{0.078740in}}{\pgfqpoint{7.842520in}{7.842520in}}%
\pgfusepath{clip}%
\pgfsetbuttcap%
\pgfsetroundjoin%
\definecolor{currentfill}{rgb}{0.278791,0.062145,0.386592}%
\pgfsetfillcolor{currentfill}%
\pgfsetlinewidth{0.000000pt}%
\definecolor{currentstroke}{rgb}{0.221989,0.339161,0.548752}%
\pgfsetstrokecolor{currentstroke}%
\pgfsetdash{}{0pt}%
\pgfpathmoveto{\pgfqpoint{5.581319in}{3.749107in}}%
\pgfpathlineto{\pgfqpoint{5.449009in}{3.782039in}}%
\pgfpathlineto{\pgfqpoint{5.509682in}{3.857436in}}%
\pgfpathclose%
\pgfusepath{fill}%
\end{pgfscope}%
\begin{pgfscope}%
\pgfpathrectangle{\pgfqpoint{0.680860in}{0.078740in}}{\pgfqpoint{7.842520in}{7.842520in}}%
\pgfusepath{clip}%
\pgfsetbuttcap%
\pgfsetroundjoin%
\definecolor{currentfill}{rgb}{0.271828,0.209303,0.504434}%
\pgfsetfillcolor{currentfill}%
\pgfsetlinewidth{0.000000pt}%
\definecolor{currentstroke}{rgb}{0.220057,0.343307,0.549413}%
\pgfsetstrokecolor{currentstroke}%
\pgfsetdash{}{0pt}%
\pgfpathmoveto{\pgfqpoint{4.227282in}{4.275960in}}%
\pgfpathlineto{\pgfqpoint{4.305127in}{4.216098in}}%
\pgfpathlineto{\pgfqpoint{4.174888in}{4.237162in}}%
\pgfpathclose%
\pgfusepath{fill}%
\end{pgfscope}%
\begin{pgfscope}%
\pgfpathrectangle{\pgfqpoint{0.680860in}{0.078740in}}{\pgfqpoint{7.842520in}{7.842520in}}%
\pgfusepath{clip}%
\pgfsetbuttcap%
\pgfsetroundjoin%
\definecolor{currentfill}{rgb}{0.263663,0.237631,0.518762}%
\pgfsetfillcolor{currentfill}%
\pgfsetlinewidth{0.000000pt}%
\definecolor{currentstroke}{rgb}{0.218130,0.347432,0.550038}%
\pgfsetstrokecolor{currentstroke}%
\pgfsetdash{}{0pt}%
\pgfpathmoveto{\pgfqpoint{3.757170in}{4.372384in}}%
\pgfpathlineto{\pgfqpoint{3.966472in}{4.310901in}}%
\pgfpathlineto{\pgfqpoint{3.836768in}{4.329195in}}%
\pgfpathclose%
\pgfusepath{fill}%
\end{pgfscope}%
\begin{pgfscope}%
\pgfpathrectangle{\pgfqpoint{0.680860in}{0.078740in}}{\pgfqpoint{7.842520in}{7.842520in}}%
\pgfusepath{clip}%
\pgfsetbuttcap%
\pgfsetroundjoin%
\definecolor{currentfill}{rgb}{0.279574,0.170599,0.479997}%
\pgfsetfillcolor{currentfill}%
\pgfsetlinewidth{0.000000pt}%
\definecolor{currentstroke}{rgb}{0.216210,0.351535,0.550627}%
\pgfsetstrokecolor{currentstroke}%
\pgfsetdash{}{0pt}%
\pgfpathmoveto{\pgfqpoint{4.643301in}{4.102408in}}%
\pgfpathlineto{\pgfqpoint{4.566993in}{4.175336in}}%
\pgfpathlineto{\pgfqpoint{4.774470in}{4.078314in}}%
\pgfpathclose%
\pgfusepath{fill}%
\end{pgfscope}%
\begin{pgfscope}%
\pgfpathrectangle{\pgfqpoint{0.680860in}{0.078740in}}{\pgfqpoint{7.842520in}{7.842520in}}%
\pgfusepath{clip}%
\pgfsetbuttcap%
\pgfsetroundjoin%
\definecolor{currentfill}{rgb}{0.253935,0.265254,0.529983}%
\pgfsetfillcolor{currentfill}%
\pgfsetlinewidth{0.000000pt}%
\definecolor{currentstroke}{rgb}{0.214298,0.355619,0.551184}%
\pgfsetstrokecolor{currentstroke}%
\pgfsetdash{}{0pt}%
\pgfpathmoveto{\pgfqpoint{3.627581in}{4.387860in}}%
\pgfpathlineto{\pgfqpoint{3.498460in}{4.403950in}}%
\pgfpathlineto{\pgfqpoint{3.417646in}{4.434394in}}%
\pgfpathclose%
\pgfusepath{fill}%
\end{pgfscope}%
\begin{pgfscope}%
\pgfpathrectangle{\pgfqpoint{0.680860in}{0.078740in}}{\pgfqpoint{7.842520in}{7.842520in}}%
\pgfusepath{clip}%
\pgfsetbuttcap%
\pgfsetroundjoin%
\definecolor{currentfill}{rgb}{0.277134,0.185228,0.489898}%
\pgfsetfillcolor{currentfill}%
\pgfsetlinewidth{0.000000pt}%
\definecolor{currentstroke}{rgb}{0.212395,0.359683,0.551710}%
\pgfsetstrokecolor{currentstroke}%
\pgfsetdash{}{0pt}%
\pgfpathmoveto{\pgfqpoint{4.435827in}{4.195483in}}%
\pgfpathlineto{\pgfqpoint{4.566993in}{4.175336in}}%
\pgfpathlineto{\pgfqpoint{4.643301in}{4.102408in}}%
\pgfpathclose%
\pgfusepath{fill}%
\end{pgfscope}%
\begin{pgfscope}%
\pgfpathrectangle{\pgfqpoint{0.680860in}{0.078740in}}{\pgfqpoint{7.842520in}{7.842520in}}%
\pgfusepath{clip}%
\pgfsetbuttcap%
\pgfsetroundjoin%
\definecolor{currentfill}{rgb}{0.278791,0.062145,0.386592}%
\pgfsetfillcolor{currentfill}%
\pgfsetlinewidth{0.000000pt}%
\definecolor{currentstroke}{rgb}{0.210503,0.363727,0.552206}%
\pgfsetstrokecolor{currentstroke}%
\pgfsetdash{}{0pt}%
\pgfpathmoveto{\pgfqpoint{5.509682in}{3.857436in}}%
\pgfpathlineto{\pgfqpoint{5.714060in}{3.716240in}}%
\pgfpathlineto{\pgfqpoint{5.581319in}{3.749107in}}%
\pgfpathclose%
\pgfusepath{fill}%
\end{pgfscope}%
\begin{pgfscope}%
\pgfpathrectangle{\pgfqpoint{0.680860in}{0.078740in}}{\pgfqpoint{7.842520in}{7.842520in}}%
\pgfusepath{clip}%
\pgfsetbuttcap%
\pgfsetroundjoin%
\definecolor{currentfill}{rgb}{0.281446,0.084320,0.407414}%
\pgfsetfillcolor{currentfill}%
\pgfsetlinewidth{0.000000pt}%
\definecolor{currentstroke}{rgb}{0.208623,0.367752,0.552675}%
\pgfsetstrokecolor{currentstroke}%
\pgfsetdash{}{0pt}%
\pgfpathmoveto{\pgfqpoint{5.509682in}{3.857436in}}%
\pgfpathlineto{\pgfqpoint{5.449009in}{3.782039in}}%
\pgfpathlineto{\pgfqpoint{5.376767in}{3.884968in}}%
\pgfpathclose%
\pgfusepath{fill}%
\end{pgfscope}%
\begin{pgfscope}%
\pgfpathrectangle{\pgfqpoint{0.680860in}{0.078740in}}{\pgfqpoint{7.842520in}{7.842520in}}%
\pgfusepath{clip}%
\pgfsetbuttcap%
\pgfsetroundjoin%
\definecolor{currentfill}{rgb}{0.267968,0.223549,0.512008}%
\pgfsetfillcolor{currentfill}%
\pgfsetlinewidth{0.000000pt}%
\definecolor{currentstroke}{rgb}{0.206756,0.371758,0.553117}%
\pgfsetstrokecolor{currentstroke}%
\pgfsetdash{}{0pt}%
\pgfpathmoveto{\pgfqpoint{4.227282in}{4.275960in}}%
\pgfpathlineto{\pgfqpoint{4.174888in}{4.237162in}}%
\pgfpathlineto{\pgfqpoint{4.096642in}{4.293149in}}%
\pgfpathclose%
\pgfusepath{fill}%
\end{pgfscope}%
\begin{pgfscope}%
\pgfpathrectangle{\pgfqpoint{0.680860in}{0.078740in}}{\pgfqpoint{7.842520in}{7.842520in}}%
\pgfusepath{clip}%
\pgfsetbuttcap%
\pgfsetroundjoin%
\definecolor{currentfill}{rgb}{0.282290,0.145912,0.461510}%
\pgfsetfillcolor{currentfill}%
\pgfsetlinewidth{0.000000pt}%
\definecolor{currentstroke}{rgb}{0.204903,0.375746,0.553533}%
\pgfsetstrokecolor{currentstroke}%
\pgfsetdash{}{0pt}%
\pgfpathmoveto{\pgfqpoint{5.038186in}{4.031200in}}%
\pgfpathlineto{\pgfqpoint{5.112294in}{3.940735in}}%
\pgfpathlineto{\pgfqpoint{4.906097in}{4.054573in}}%
\pgfpathclose%
\pgfusepath{fill}%
\end{pgfscope}%
\begin{pgfscope}%
\pgfpathrectangle{\pgfqpoint{0.680860in}{0.078740in}}{\pgfqpoint{7.842520in}{7.842520in}}%
\pgfusepath{clip}%
\pgfsetbuttcap%
\pgfsetroundjoin%
\definecolor{currentfill}{rgb}{0.270595,0.214069,0.507052}%
\pgfsetfillcolor{currentfill}%
\pgfsetlinewidth{0.000000pt}%
\definecolor{currentstroke}{rgb}{0.203063,0.379716,0.553925}%
\pgfsetstrokecolor{currentstroke}%
\pgfsetdash{}{0pt}%
\pgfpathmoveto{\pgfqpoint{4.227282in}{4.275960in}}%
\pgfpathlineto{\pgfqpoint{4.435827in}{4.195483in}}%
\pgfpathlineto{\pgfqpoint{4.305127in}{4.216098in}}%
\pgfpathclose%
\pgfusepath{fill}%
\end{pgfscope}%
\begin{pgfscope}%
\pgfpathrectangle{\pgfqpoint{0.680860in}{0.078740in}}{\pgfqpoint{7.842520in}{7.842520in}}%
\pgfusepath{clip}%
\pgfsetbuttcap%
\pgfsetroundjoin%
\definecolor{currentfill}{rgb}{0.283187,0.125848,0.444960}%
\pgfsetfillcolor{currentfill}%
\pgfsetlinewidth{0.000000pt}%
\definecolor{currentstroke}{rgb}{0.201239,0.383670,0.554294}%
\pgfsetstrokecolor{currentstroke}%
\pgfsetdash{}{0pt}%
\pgfpathmoveto{\pgfqpoint{5.170743in}{4.008212in}}%
\pgfpathlineto{\pgfqpoint{5.244306in}{3.912738in}}%
\pgfpathlineto{\pgfqpoint{5.112294in}{3.940735in}}%
\pgfpathclose%
\pgfusepath{fill}%
\end{pgfscope}%
\begin{pgfscope}%
\pgfpathrectangle{\pgfqpoint{0.680860in}{0.078740in}}{\pgfqpoint{7.842520in}{7.842520in}}%
\pgfusepath{clip}%
\pgfsetbuttcap%
\pgfsetroundjoin%
\definecolor{currentfill}{rgb}{0.278012,0.180367,0.486697}%
\pgfsetfillcolor{currentfill}%
\pgfsetlinewidth{0.000000pt}%
\definecolor{currentstroke}{rgb}{0.199430,0.387607,0.554642}%
\pgfsetstrokecolor{currentstroke}%
\pgfsetdash{}{0pt}%
\pgfpathmoveto{\pgfqpoint{4.906097in}{4.054573in}}%
\pgfpathlineto{\pgfqpoint{4.774470in}{4.078314in}}%
\pgfpathlineto{\pgfqpoint{4.698631in}{4.155678in}}%
\pgfpathclose%
\pgfusepath{fill}%
\end{pgfscope}%
\begin{pgfscope}%
\pgfpathrectangle{\pgfqpoint{0.680860in}{0.078740in}}{\pgfqpoint{7.842520in}{7.842520in}}%
\pgfusepath{clip}%
\pgfsetbuttcap%
\pgfsetroundjoin%
\definecolor{currentfill}{rgb}{0.257322,0.256130,0.526563}%
\pgfsetfillcolor{currentfill}%
\pgfsetlinewidth{0.000000pt}%
\definecolor{currentstroke}{rgb}{0.197636,0.391528,0.554969}%
\pgfsetstrokecolor{currentstroke}%
\pgfsetdash{}{0pt}%
\pgfpathmoveto{\pgfqpoint{3.887232in}{4.357547in}}%
\pgfpathlineto{\pgfqpoint{3.966472in}{4.310901in}}%
\pgfpathlineto{\pgfqpoint{3.757170in}{4.372384in}}%
\pgfpathclose%
\pgfusepath{fill}%
\end{pgfscope}%
\begin{pgfscope}%
\pgfpathrectangle{\pgfqpoint{0.680860in}{0.078740in}}{\pgfqpoint{7.842520in}{7.842520in}}%
\pgfusepath{clip}%
\pgfsetbuttcap%
\pgfsetroundjoin%
\definecolor{currentfill}{rgb}{0.250425,0.274290,0.533103}%
\pgfsetfillcolor{currentfill}%
\pgfsetlinewidth{0.000000pt}%
\definecolor{currentstroke}{rgb}{0.195860,0.395433,0.555276}%
\pgfsetstrokecolor{currentstroke}%
\pgfsetdash{}{0pt}%
\pgfpathmoveto{\pgfqpoint{3.627581in}{4.387860in}}%
\pgfpathlineto{\pgfqpoint{3.547085in}{4.421340in}}%
\pgfpathlineto{\pgfqpoint{3.757170in}{4.372384in}}%
\pgfpathclose%
\pgfusepath{fill}%
\end{pgfscope}%
\begin{pgfscope}%
\pgfpathrectangle{\pgfqpoint{0.680860in}{0.078740in}}{\pgfqpoint{7.842520in}{7.842520in}}%
\pgfusepath{clip}%
\pgfsetbuttcap%
\pgfsetroundjoin%
\definecolor{currentfill}{rgb}{0.281887,0.150881,0.465405}%
\pgfsetfillcolor{currentfill}%
\pgfsetlinewidth{0.000000pt}%
\definecolor{currentstroke}{rgb}{0.194100,0.399323,0.555565}%
\pgfsetstrokecolor{currentstroke}%
\pgfsetdash{}{0pt}%
\pgfpathmoveto{\pgfqpoint{5.170743in}{4.008212in}}%
\pgfpathlineto{\pgfqpoint{5.112294in}{3.940735in}}%
\pgfpathlineto{\pgfqpoint{5.038186in}{4.031200in}}%
\pgfpathclose%
\pgfusepath{fill}%
\end{pgfscope}%
\begin{pgfscope}%
\pgfpathrectangle{\pgfqpoint{0.680860in}{0.078740in}}{\pgfqpoint{7.842520in}{7.842520in}}%
\pgfusepath{clip}%
\pgfsetbuttcap%
\pgfsetroundjoin%
\definecolor{currentfill}{rgb}{0.275191,0.194905,0.496005}%
\pgfsetfillcolor{currentfill}%
\pgfsetlinewidth{0.000000pt}%
\definecolor{currentstroke}{rgb}{0.192357,0.403199,0.555836}%
\pgfsetstrokecolor{currentstroke}%
\pgfsetdash{}{0pt}%
\pgfpathmoveto{\pgfqpoint{4.774470in}{4.078314in}}%
\pgfpathlineto{\pgfqpoint{4.566993in}{4.175336in}}%
\pgfpathlineto{\pgfqpoint{4.698631in}{4.155678in}}%
\pgfpathclose%
\pgfusepath{fill}%
\end{pgfscope}%
\begin{pgfscope}%
\pgfpathrectangle{\pgfqpoint{0.680860in}{0.078740in}}{\pgfqpoint{7.842520in}{7.842520in}}%
\pgfusepath{clip}%
\pgfsetbuttcap%
\pgfsetroundjoin%
\definecolor{currentfill}{rgb}{0.248629,0.278775,0.534556}%
\pgfsetfillcolor{currentfill}%
\pgfsetlinewidth{0.000000pt}%
\definecolor{currentstroke}{rgb}{0.190631,0.407061,0.556089}%
\pgfsetstrokecolor{currentstroke}%
\pgfsetdash{}{0pt}%
\pgfpathmoveto{\pgfqpoint{3.417646in}{4.434394in}}%
\pgfpathlineto{\pgfqpoint{3.547085in}{4.421340in}}%
\pgfpathlineto{\pgfqpoint{3.627581in}{4.387860in}}%
\pgfpathclose%
\pgfusepath{fill}%
\end{pgfscope}%
\begin{pgfscope}%
\pgfpathrectangle{\pgfqpoint{0.680860in}{0.078740in}}{\pgfqpoint{7.842520in}{7.842520in}}%
\pgfusepath{clip}%
\pgfsetbuttcap%
\pgfsetroundjoin%
\definecolor{currentfill}{rgb}{0.283072,0.130895,0.449241}%
\pgfsetfillcolor{currentfill}%
\pgfsetlinewidth{0.000000pt}%
\definecolor{currentstroke}{rgb}{0.188923,0.410910,0.556326}%
\pgfsetstrokecolor{currentstroke}%
\pgfsetdash{}{0pt}%
\pgfpathmoveto{\pgfqpoint{5.376767in}{3.884968in}}%
\pgfpathlineto{\pgfqpoint{5.244306in}{3.912738in}}%
\pgfpathlineto{\pgfqpoint{5.170743in}{4.008212in}}%
\pgfpathclose%
\pgfusepath{fill}%
\end{pgfscope}%
\begin{pgfscope}%
\pgfpathrectangle{\pgfqpoint{0.680860in}{0.078740in}}{\pgfqpoint{7.842520in}{7.842520in}}%
\pgfusepath{clip}%
\pgfsetbuttcap%
\pgfsetroundjoin%
\definecolor{currentfill}{rgb}{0.258965,0.251537,0.524736}%
\pgfsetfillcolor{currentfill}%
\pgfsetlinewidth{0.000000pt}%
\definecolor{currentstroke}{rgb}{0.187231,0.414746,0.556547}%
\pgfsetstrokecolor{currentstroke}%
\pgfsetdash{}{0pt}%
\pgfpathmoveto{\pgfqpoint{4.096642in}{4.293149in}}%
\pgfpathlineto{\pgfqpoint{3.966472in}{4.310901in}}%
\pgfpathlineto{\pgfqpoint{4.017773in}{4.343374in}}%
\pgfpathclose%
\pgfusepath{fill}%
\end{pgfscope}%
\begin{pgfscope}%
\pgfpathrectangle{\pgfqpoint{0.680860in}{0.078740in}}{\pgfqpoint{7.842520in}{7.842520in}}%
\pgfusepath{clip}%
\pgfsetbuttcap%
\pgfsetroundjoin%
\definecolor{currentfill}{rgb}{0.279566,0.067836,0.391917}%
\pgfsetfillcolor{currentfill}%
\pgfsetlinewidth{0.000000pt}%
\definecolor{currentstroke}{rgb}{0.185556,0.418570,0.556753}%
\pgfsetstrokecolor{currentstroke}%
\pgfsetdash{}{0pt}%
\pgfpathmoveto{\pgfqpoint{5.847236in}{3.683443in}}%
\pgfpathlineto{\pgfqpoint{5.714060in}{3.716240in}}%
\pgfpathlineto{\pgfqpoint{5.776889in}{3.803133in}}%
\pgfpathclose%
\pgfusepath{fill}%
\end{pgfscope}%
\begin{pgfscope}%
\pgfpathrectangle{\pgfqpoint{0.680860in}{0.078740in}}{\pgfqpoint{7.842520in}{7.842520in}}%
\pgfusepath{clip}%
\pgfsetbuttcap%
\pgfsetroundjoin%
\definecolor{currentfill}{rgb}{0.281924,0.089666,0.412415}%
\pgfsetfillcolor{currentfill}%
\pgfsetlinewidth{0.000000pt}%
\definecolor{currentstroke}{rgb}{0.183898,0.422383,0.556944}%
\pgfsetstrokecolor{currentstroke}%
\pgfsetdash{}{0pt}%
\pgfpathmoveto{\pgfqpoint{5.643054in}{3.830153in}}%
\pgfpathlineto{\pgfqpoint{5.714060in}{3.716240in}}%
\pgfpathlineto{\pgfqpoint{5.509682in}{3.857436in}}%
\pgfpathclose%
\pgfusepath{fill}%
\end{pgfscope}%
\begin{pgfscope}%
\pgfpathrectangle{\pgfqpoint{0.680860in}{0.078740in}}{\pgfqpoint{7.842520in}{7.842520in}}%
\pgfusepath{clip}%
\pgfsetbuttcap%
\pgfsetroundjoin%
\definecolor{currentfill}{rgb}{0.265145,0.232956,0.516599}%
\pgfsetfillcolor{currentfill}%
\pgfsetlinewidth{0.000000pt}%
\definecolor{currentstroke}{rgb}{0.182256,0.426184,0.557120}%
\pgfsetstrokecolor{currentstroke}%
\pgfsetdash{}{0pt}%
\pgfpathmoveto{\pgfqpoint{4.358400in}{4.259357in}}%
\pgfpathlineto{\pgfqpoint{4.435827in}{4.195483in}}%
\pgfpathlineto{\pgfqpoint{4.227282in}{4.275960in}}%
\pgfpathclose%
\pgfusepath{fill}%
\end{pgfscope}%
\begin{pgfscope}%
\pgfpathrectangle{\pgfqpoint{0.680860in}{0.078740in}}{\pgfqpoint{7.842520in}{7.842520in}}%
\pgfusepath{clip}%
\pgfsetbuttcap%
\pgfsetroundjoin%
\definecolor{currentfill}{rgb}{0.255645,0.260703,0.528312}%
\pgfsetfillcolor{currentfill}%
\pgfsetlinewidth{0.000000pt}%
\definecolor{currentstroke}{rgb}{0.180629,0.429975,0.557282}%
\pgfsetstrokecolor{currentstroke}%
\pgfsetdash{}{0pt}%
\pgfpathmoveto{\pgfqpoint{4.017773in}{4.343374in}}%
\pgfpathlineto{\pgfqpoint{3.966472in}{4.310901in}}%
\pgfpathlineto{\pgfqpoint{3.887232in}{4.357547in}}%
\pgfpathclose%
\pgfusepath{fill}%
\end{pgfscope}%
\begin{pgfscope}%
\pgfpathrectangle{\pgfqpoint{0.680860in}{0.078740in}}{\pgfqpoint{7.842520in}{7.842520in}}%
\pgfusepath{clip}%
\pgfsetbuttcap%
\pgfsetroundjoin%
\definecolor{currentfill}{rgb}{0.267968,0.223549,0.512008}%
\pgfsetfillcolor{currentfill}%
\pgfsetlinewidth{0.000000pt}%
\definecolor{currentstroke}{rgb}{0.179019,0.433756,0.557430}%
\pgfsetstrokecolor{currentstroke}%
\pgfsetdash{}{0pt}%
\pgfpathmoveto{\pgfqpoint{4.490000in}{4.243364in}}%
\pgfpathlineto{\pgfqpoint{4.566993in}{4.175336in}}%
\pgfpathlineto{\pgfqpoint{4.435827in}{4.195483in}}%
\pgfpathclose%
\pgfusepath{fill}%
\end{pgfscope}%
\begin{pgfscope}%
\pgfpathrectangle{\pgfqpoint{0.680860in}{0.078740in}}{\pgfqpoint{7.842520in}{7.842520in}}%
\pgfusepath{clip}%
\pgfsetbuttcap%
\pgfsetroundjoin%
\definecolor{currentfill}{rgb}{0.257322,0.256130,0.526563}%
\pgfsetfillcolor{currentfill}%
\pgfsetlinewidth{0.000000pt}%
\definecolor{currentstroke}{rgb}{0.177423,0.437527,0.557565}%
\pgfsetstrokecolor{currentstroke}%
\pgfsetdash{}{0pt}%
\pgfpathmoveto{\pgfqpoint{4.096642in}{4.293149in}}%
\pgfpathlineto{\pgfqpoint{4.017773in}{4.343374in}}%
\pgfpathlineto{\pgfqpoint{4.227282in}{4.275960in}}%
\pgfpathclose%
\pgfusepath{fill}%
\end{pgfscope}%
\begin{pgfscope}%
\pgfpathrectangle{\pgfqpoint{0.680860in}{0.078740in}}{\pgfqpoint{7.842520in}{7.842520in}}%
\pgfusepath{clip}%
\pgfsetbuttcap%
\pgfsetroundjoin%
\definecolor{currentfill}{rgb}{0.274128,0.199721,0.498911}%
\pgfsetfillcolor{currentfill}%
\pgfsetlinewidth{0.000000pt}%
\definecolor{currentstroke}{rgb}{0.175841,0.441290,0.557685}%
\pgfsetstrokecolor{currentstroke}%
\pgfsetdash{}{0pt}%
\pgfpathmoveto{\pgfqpoint{4.698631in}{4.155678in}}%
\pgfpathlineto{\pgfqpoint{4.830745in}{4.136528in}}%
\pgfpathlineto{\pgfqpoint{4.906097in}{4.054573in}}%
\pgfpathclose%
\pgfusepath{fill}%
\end{pgfscope}%
\begin{pgfscope}%
\pgfpathrectangle{\pgfqpoint{0.680860in}{0.078740in}}{\pgfqpoint{7.842520in}{7.842520in}}%
\pgfusepath{clip}%
\pgfsetbuttcap%
\pgfsetroundjoin%
\definecolor{currentfill}{rgb}{0.282327,0.094955,0.417331}%
\pgfsetfillcolor{currentfill}%
\pgfsetlinewidth{0.000000pt}%
\definecolor{currentstroke}{rgb}{0.174274,0.445044,0.557792}%
\pgfsetstrokecolor{currentstroke}%
\pgfsetdash{}{0pt}%
\pgfpathmoveto{\pgfqpoint{5.776889in}{3.803133in}}%
\pgfpathlineto{\pgfqpoint{5.714060in}{3.716240in}}%
\pgfpathlineto{\pgfqpoint{5.643054in}{3.830153in}}%
\pgfpathclose%
\pgfusepath{fill}%
\end{pgfscope}%
\begin{pgfscope}%
\pgfpathrectangle{\pgfqpoint{0.680860in}{0.078740in}}{\pgfqpoint{7.842520in}{7.842520in}}%
\pgfusepath{clip}%
\pgfsetbuttcap%
\pgfsetroundjoin%
\definecolor{currentfill}{rgb}{0.244972,0.287675,0.537260}%
\pgfsetfillcolor{currentfill}%
\pgfsetlinewidth{0.000000pt}%
\definecolor{currentstroke}{rgb}{0.172719,0.448791,0.557885}%
\pgfsetstrokecolor{currentstroke}%
\pgfsetdash{}{0pt}%
\pgfpathmoveto{\pgfqpoint{3.547085in}{4.421340in}}%
\pgfpathlineto{\pgfqpoint{3.677004in}{4.409009in}}%
\pgfpathlineto{\pgfqpoint{3.757170in}{4.372384in}}%
\pgfpathclose%
\pgfusepath{fill}%
\end{pgfscope}%
\begin{pgfscope}%
\pgfpathrectangle{\pgfqpoint{0.680860in}{0.078740in}}{\pgfqpoint{7.842520in}{7.842520in}}%
\pgfusepath{clip}%
\pgfsetbuttcap%
\pgfsetroundjoin%
\definecolor{currentfill}{rgb}{0.276194,0.190074,0.493001}%
\pgfsetfillcolor{currentfill}%
\pgfsetlinewidth{0.000000pt}%
\definecolor{currentstroke}{rgb}{0.171176,0.452530,0.557965}%
\pgfsetstrokecolor{currentstroke}%
\pgfsetdash{}{0pt}%
\pgfpathmoveto{\pgfqpoint{4.906097in}{4.054573in}}%
\pgfpathlineto{\pgfqpoint{4.963342in}{4.117906in}}%
\pgfpathlineto{\pgfqpoint{5.038186in}{4.031200in}}%
\pgfpathclose%
\pgfusepath{fill}%
\end{pgfscope}%
\begin{pgfscope}%
\pgfpathrectangle{\pgfqpoint{0.680860in}{0.078740in}}{\pgfqpoint{7.842520in}{7.842520in}}%
\pgfusepath{clip}%
\pgfsetbuttcap%
\pgfsetroundjoin%
\definecolor{currentfill}{rgb}{0.263663,0.237631,0.518762}%
\pgfsetfillcolor{currentfill}%
\pgfsetlinewidth{0.000000pt}%
\definecolor{currentstroke}{rgb}{0.169646,0.456262,0.558030}%
\pgfsetstrokecolor{currentstroke}%
\pgfsetdash{}{0pt}%
\pgfpathmoveto{\pgfqpoint{4.435827in}{4.195483in}}%
\pgfpathlineto{\pgfqpoint{4.358400in}{4.259357in}}%
\pgfpathlineto{\pgfqpoint{4.490000in}{4.243364in}}%
\pgfpathclose%
\pgfusepath{fill}%
\end{pgfscope}%
\begin{pgfscope}%
\pgfpathrectangle{\pgfqpoint{0.680860in}{0.078740in}}{\pgfqpoint{7.842520in}{7.842520in}}%
\pgfusepath{clip}%
\pgfsetbuttcap%
\pgfsetroundjoin%
\definecolor{currentfill}{rgb}{0.281412,0.155834,0.469201}%
\pgfsetfillcolor{currentfill}%
\pgfsetlinewidth{0.000000pt}%
\definecolor{currentstroke}{rgb}{0.168126,0.459988,0.558082}%
\pgfsetstrokecolor{currentstroke}%
\pgfsetdash{}{0pt}%
\pgfpathmoveto{\pgfqpoint{5.170743in}{4.008212in}}%
\pgfpathlineto{\pgfqpoint{5.303772in}{3.985624in}}%
\pgfpathlineto{\pgfqpoint{5.376767in}{3.884968in}}%
\pgfpathclose%
\pgfusepath{fill}%
\end{pgfscope}%
\begin{pgfscope}%
\pgfpathrectangle{\pgfqpoint{0.680860in}{0.078740in}}{\pgfqpoint{7.842520in}{7.842520in}}%
\pgfusepath{clip}%
\pgfsetbuttcap%
\pgfsetroundjoin%
\definecolor{currentfill}{rgb}{0.280267,0.073417,0.397163}%
\pgfsetfillcolor{currentfill}%
\pgfsetlinewidth{0.000000pt}%
\definecolor{currentstroke}{rgb}{0.166617,0.463708,0.558119}%
\pgfsetstrokecolor{currentstroke}%
\pgfsetdash{}{0pt}%
\pgfpathmoveto{\pgfqpoint{5.911190in}{3.776386in}}%
\pgfpathlineto{\pgfqpoint{5.980851in}{3.650721in}}%
\pgfpathlineto{\pgfqpoint{5.847236in}{3.683443in}}%
\pgfpathclose%
\pgfusepath{fill}%
\end{pgfscope}%
\begin{pgfscope}%
\pgfpathrectangle{\pgfqpoint{0.680860in}{0.078740in}}{\pgfqpoint{7.842520in}{7.842520in}}%
\pgfusepath{clip}%
\pgfsetbuttcap%
\pgfsetroundjoin%
\definecolor{currentfill}{rgb}{0.282884,0.135920,0.453427}%
\pgfsetfillcolor{currentfill}%
\pgfsetlinewidth{0.000000pt}%
\definecolor{currentstroke}{rgb}{0.165117,0.467423,0.558141}%
\pgfsetstrokecolor{currentstroke}%
\pgfsetdash{}{0pt}%
\pgfpathmoveto{\pgfqpoint{5.437278in}{3.963455in}}%
\pgfpathlineto{\pgfqpoint{5.509682in}{3.857436in}}%
\pgfpathlineto{\pgfqpoint{5.376767in}{3.884968in}}%
\pgfpathclose%
\pgfusepath{fill}%
\end{pgfscope}%
\begin{pgfscope}%
\pgfpathrectangle{\pgfqpoint{0.680860in}{0.078740in}}{\pgfqpoint{7.842520in}{7.842520in}}%
\pgfusepath{clip}%
\pgfsetbuttcap%
\pgfsetroundjoin%
\definecolor{currentfill}{rgb}{0.246811,0.283237,0.535941}%
\pgfsetfillcolor{currentfill}%
\pgfsetlinewidth{0.000000pt}%
\definecolor{currentstroke}{rgb}{0.163625,0.471133,0.558148}%
\pgfsetstrokecolor{currentstroke}%
\pgfsetdash{}{0pt}%
\pgfpathmoveto{\pgfqpoint{3.757170in}{4.372384in}}%
\pgfpathlineto{\pgfqpoint{3.807408in}{4.397430in}}%
\pgfpathlineto{\pgfqpoint{3.887232in}{4.357547in}}%
\pgfpathclose%
\pgfusepath{fill}%
\end{pgfscope}%
\begin{pgfscope}%
\pgfpathrectangle{\pgfqpoint{0.680860in}{0.078740in}}{\pgfqpoint{7.842520in}{7.842520in}}%
\pgfusepath{clip}%
\pgfsetbuttcap%
\pgfsetroundjoin%
\definecolor{currentfill}{rgb}{0.239346,0.300855,0.540844}%
\pgfsetfillcolor{currentfill}%
\pgfsetlinewidth{0.000000pt}%
\definecolor{currentstroke}{rgb}{0.162142,0.474838,0.558140}%
\pgfsetstrokecolor{currentstroke}%
\pgfsetdash{}{0pt}%
\pgfpathmoveto{\pgfqpoint{3.466067in}{4.447407in}}%
\pgfpathlineto{\pgfqpoint{3.547085in}{4.421340in}}%
\pgfpathlineto{\pgfqpoint{3.417646in}{4.434394in}}%
\pgfpathclose%
\pgfusepath{fill}%
\end{pgfscope}%
\begin{pgfscope}%
\pgfpathrectangle{\pgfqpoint{0.680860in}{0.078740in}}{\pgfqpoint{7.842520in}{7.842520in}}%
\pgfusepath{clip}%
\pgfsetbuttcap%
\pgfsetroundjoin%
\definecolor{currentfill}{rgb}{0.273006,0.204520,0.501721}%
\pgfsetfillcolor{currentfill}%
\pgfsetlinewidth{0.000000pt}%
\definecolor{currentstroke}{rgb}{0.160665,0.478540,0.558115}%
\pgfsetstrokecolor{currentstroke}%
\pgfsetdash{}{0pt}%
\pgfpathmoveto{\pgfqpoint{4.830745in}{4.136528in}}%
\pgfpathlineto{\pgfqpoint{4.963342in}{4.117906in}}%
\pgfpathlineto{\pgfqpoint{4.906097in}{4.054573in}}%
\pgfpathclose%
\pgfusepath{fill}%
\end{pgfscope}%
\begin{pgfscope}%
\pgfpathrectangle{\pgfqpoint{0.680860in}{0.078740in}}{\pgfqpoint{7.842520in}{7.842520in}}%
\pgfusepath{clip}%
\pgfsetbuttcap%
\pgfsetroundjoin%
\definecolor{currentfill}{rgb}{0.265145,0.232956,0.516599}%
\pgfsetfillcolor{currentfill}%
\pgfsetlinewidth{0.000000pt}%
\definecolor{currentstroke}{rgb}{0.159194,0.482237,0.558073}%
\pgfsetstrokecolor{currentstroke}%
\pgfsetdash{}{0pt}%
\pgfpathmoveto{\pgfqpoint{4.698631in}{4.155678in}}%
\pgfpathlineto{\pgfqpoint{4.566993in}{4.175336in}}%
\pgfpathlineto{\pgfqpoint{4.622090in}{4.228007in}}%
\pgfpathclose%
\pgfusepath{fill}%
\end{pgfscope}%
\begin{pgfscope}%
\pgfpathrectangle{\pgfqpoint{0.680860in}{0.078740in}}{\pgfqpoint{7.842520in}{7.842520in}}%
\pgfusepath{clip}%
\pgfsetbuttcap%
\pgfsetroundjoin%
\definecolor{currentfill}{rgb}{0.276194,0.190074,0.493001}%
\pgfsetfillcolor{currentfill}%
\pgfsetlinewidth{0.000000pt}%
\definecolor{currentstroke}{rgb}{0.157729,0.485932,0.558013}%
\pgfsetstrokecolor{currentstroke}%
\pgfsetdash{}{0pt}%
\pgfpathmoveto{\pgfqpoint{4.963342in}{4.117906in}}%
\pgfpathlineto{\pgfqpoint{5.170743in}{4.008212in}}%
\pgfpathlineto{\pgfqpoint{5.038186in}{4.031200in}}%
\pgfpathclose%
\pgfusepath{fill}%
\end{pgfscope}%
\begin{pgfscope}%
\pgfpathrectangle{\pgfqpoint{0.680860in}{0.078740in}}{\pgfqpoint{7.842520in}{7.842520in}}%
\pgfusepath{clip}%
\pgfsetbuttcap%
\pgfsetroundjoin%
\definecolor{currentfill}{rgb}{0.241237,0.296485,0.539709}%
\pgfsetfillcolor{currentfill}%
\pgfsetlinewidth{0.000000pt}%
\definecolor{currentstroke}{rgb}{0.156270,0.489624,0.557936}%
\pgfsetstrokecolor{currentstroke}%
\pgfsetdash{}{0pt}%
\pgfpathmoveto{\pgfqpoint{3.757170in}{4.372384in}}%
\pgfpathlineto{\pgfqpoint{3.677004in}{4.409009in}}%
\pgfpathlineto{\pgfqpoint{3.807408in}{4.397430in}}%
\pgfpathclose%
\pgfusepath{fill}%
\end{pgfscope}%
\begin{pgfscope}%
\pgfpathrectangle{\pgfqpoint{0.680860in}{0.078740in}}{\pgfqpoint{7.842520in}{7.842520in}}%
\pgfusepath{clip}%
\pgfsetbuttcap%
\pgfsetroundjoin%
\definecolor{currentfill}{rgb}{0.235526,0.309527,0.542944}%
\pgfsetfillcolor{currentfill}%
\pgfsetlinewidth{0.000000pt}%
\definecolor{currentstroke}{rgb}{0.154815,0.493313,0.557840}%
\pgfsetstrokecolor{currentstroke}%
\pgfsetdash{}{0pt}%
\pgfpathmoveto{\pgfqpoint{3.417646in}{4.434394in}}%
\pgfpathlineto{\pgfqpoint{3.336323in}{4.457647in}}%
\pgfpathlineto{\pgfqpoint{3.466067in}{4.447407in}}%
\pgfpathclose%
\pgfusepath{fill}%
\end{pgfscope}%
\begin{pgfscope}%
\pgfpathrectangle{\pgfqpoint{0.680860in}{0.078740in}}{\pgfqpoint{7.842520in}{7.842520in}}%
\pgfusepath{clip}%
\pgfsetbuttcap%
\pgfsetroundjoin%
\definecolor{currentfill}{rgb}{0.280267,0.073417,0.397163}%
\pgfsetfillcolor{currentfill}%
\pgfsetlinewidth{0.000000pt}%
\definecolor{currentstroke}{rgb}{0.153364,0.497000,0.557724}%
\pgfsetstrokecolor{currentstroke}%
\pgfsetdash{}{0pt}%
\pgfpathmoveto{\pgfqpoint{6.114907in}{3.618082in}}%
\pgfpathlineto{\pgfqpoint{5.980851in}{3.650721in}}%
\pgfpathlineto{\pgfqpoint{5.911190in}{3.776386in}}%
\pgfpathclose%
\pgfusepath{fill}%
\end{pgfscope}%
\begin{pgfscope}%
\pgfpathrectangle{\pgfqpoint{0.680860in}{0.078740in}}{\pgfqpoint{7.842520in}{7.842520in}}%
\pgfusepath{clip}%
\pgfsetbuttcap%
\pgfsetroundjoin%
\definecolor{currentfill}{rgb}{0.250425,0.274290,0.533103}%
\pgfsetfillcolor{currentfill}%
\pgfsetlinewidth{0.000000pt}%
\definecolor{currentstroke}{rgb}{0.151918,0.500685,0.557587}%
\pgfsetstrokecolor{currentstroke}%
\pgfsetdash{}{0pt}%
\pgfpathmoveto{\pgfqpoint{4.227282in}{4.275960in}}%
\pgfpathlineto{\pgfqpoint{4.017773in}{4.343374in}}%
\pgfpathlineto{\pgfqpoint{4.148799in}{4.329892in}}%
\pgfpathclose%
\pgfusepath{fill}%
\end{pgfscope}%
\begin{pgfscope}%
\pgfpathrectangle{\pgfqpoint{0.680860in}{0.078740in}}{\pgfqpoint{7.842520in}{7.842520in}}%
\pgfusepath{clip}%
\pgfsetbuttcap%
\pgfsetroundjoin%
\definecolor{currentfill}{rgb}{0.280868,0.160771,0.472899}%
\pgfsetfillcolor{currentfill}%
\pgfsetlinewidth{0.000000pt}%
\definecolor{currentstroke}{rgb}{0.150476,0.504369,0.557430}%
\pgfsetstrokecolor{currentstroke}%
\pgfsetdash{}{0pt}%
\pgfpathmoveto{\pgfqpoint{5.376767in}{3.884968in}}%
\pgfpathlineto{\pgfqpoint{5.303772in}{3.985624in}}%
\pgfpathlineto{\pgfqpoint{5.437278in}{3.963455in}}%
\pgfpathclose%
\pgfusepath{fill}%
\end{pgfscope}%
\begin{pgfscope}%
\pgfpathrectangle{\pgfqpoint{0.680860in}{0.078740in}}{\pgfqpoint{7.842520in}{7.842520in}}%
\pgfusepath{clip}%
\pgfsetbuttcap%
\pgfsetroundjoin%
\definecolor{currentfill}{rgb}{0.282656,0.100196,0.422160}%
\pgfsetfillcolor{currentfill}%
\pgfsetlinewidth{0.000000pt}%
\definecolor{currentstroke}{rgb}{0.149039,0.508051,0.557250}%
\pgfsetstrokecolor{currentstroke}%
\pgfsetdash{}{0pt}%
\pgfpathmoveto{\pgfqpoint{5.847236in}{3.683443in}}%
\pgfpathlineto{\pgfqpoint{5.776889in}{3.803133in}}%
\pgfpathlineto{\pgfqpoint{5.911190in}{3.776386in}}%
\pgfpathclose%
\pgfusepath{fill}%
\end{pgfscope}%
\begin{pgfscope}%
\pgfpathrectangle{\pgfqpoint{0.680860in}{0.078740in}}{\pgfqpoint{7.842520in}{7.842520in}}%
\pgfusepath{clip}%
\pgfsetbuttcap%
\pgfsetroundjoin%
\definecolor{currentfill}{rgb}{0.252194,0.269783,0.531579}%
\pgfsetfillcolor{currentfill}%
\pgfsetlinewidth{0.000000pt}%
\definecolor{currentstroke}{rgb}{0.147607,0.511733,0.557049}%
\pgfsetstrokecolor{currentstroke}%
\pgfsetdash{}{0pt}%
\pgfpathmoveto{\pgfqpoint{4.280318in}{4.317129in}}%
\pgfpathlineto{\pgfqpoint{4.358400in}{4.259357in}}%
\pgfpathlineto{\pgfqpoint{4.227282in}{4.275960in}}%
\pgfpathclose%
\pgfusepath{fill}%
\end{pgfscope}%
\begin{pgfscope}%
\pgfpathrectangle{\pgfqpoint{0.680860in}{0.078740in}}{\pgfqpoint{7.842520in}{7.842520in}}%
\pgfusepath{clip}%
\pgfsetbuttcap%
\pgfsetroundjoin%
\definecolor{currentfill}{rgb}{0.235526,0.309527,0.542944}%
\pgfsetfillcolor{currentfill}%
\pgfsetlinewidth{0.000000pt}%
\definecolor{currentstroke}{rgb}{0.146180,0.515413,0.556823}%
\pgfsetstrokecolor{currentstroke}%
\pgfsetdash{}{0pt}%
\pgfpathmoveto{\pgfqpoint{3.677004in}{4.409009in}}%
\pgfpathlineto{\pgfqpoint{3.547085in}{4.421340in}}%
\pgfpathlineto{\pgfqpoint{3.466067in}{4.447407in}}%
\pgfpathclose%
\pgfusepath{fill}%
\end{pgfscope}%
\begin{pgfscope}%
\pgfpathrectangle{\pgfqpoint{0.680860in}{0.078740in}}{\pgfqpoint{7.842520in}{7.842520in}}%
\pgfusepath{clip}%
\pgfsetbuttcap%
\pgfsetroundjoin%
\definecolor{currentfill}{rgb}{0.260571,0.246922,0.522828}%
\pgfsetfillcolor{currentfill}%
\pgfsetlinewidth{0.000000pt}%
\definecolor{currentstroke}{rgb}{0.144759,0.519093,0.556572}%
\pgfsetstrokecolor{currentstroke}%
\pgfsetdash{}{0pt}%
\pgfpathmoveto{\pgfqpoint{4.622090in}{4.228007in}}%
\pgfpathlineto{\pgfqpoint{4.566993in}{4.175336in}}%
\pgfpathlineto{\pgfqpoint{4.490000in}{4.243364in}}%
\pgfpathclose%
\pgfusepath{fill}%
\end{pgfscope}%
\begin{pgfscope}%
\pgfpathrectangle{\pgfqpoint{0.680860in}{0.078740in}}{\pgfqpoint{7.842520in}{7.842520in}}%
\pgfusepath{clip}%
\pgfsetbuttcap%
\pgfsetroundjoin%
\definecolor{currentfill}{rgb}{0.282290,0.145912,0.461510}%
\pgfsetfillcolor{currentfill}%
\pgfsetlinewidth{0.000000pt}%
\definecolor{currentstroke}{rgb}{0.143343,0.522773,0.556295}%
\pgfsetstrokecolor{currentstroke}%
\pgfsetdash{}{0pt}%
\pgfpathmoveto{\pgfqpoint{5.643054in}{3.830153in}}%
\pgfpathlineto{\pgfqpoint{5.509682in}{3.857436in}}%
\pgfpathlineto{\pgfqpoint{5.571268in}{3.941720in}}%
\pgfpathclose%
\pgfusepath{fill}%
\end{pgfscope}%
\begin{pgfscope}%
\pgfpathrectangle{\pgfqpoint{0.680860in}{0.078740in}}{\pgfqpoint{7.842520in}{7.842520in}}%
\pgfusepath{clip}%
\pgfsetbuttcap%
\pgfsetroundjoin%
\definecolor{currentfill}{rgb}{0.263663,0.237631,0.518762}%
\pgfsetfillcolor{currentfill}%
\pgfsetlinewidth{0.000000pt}%
\definecolor{currentstroke}{rgb}{0.141935,0.526453,0.555991}%
\pgfsetstrokecolor{currentstroke}%
\pgfsetdash{}{0pt}%
\pgfpathmoveto{\pgfqpoint{4.622090in}{4.228007in}}%
\pgfpathlineto{\pgfqpoint{4.830745in}{4.136528in}}%
\pgfpathlineto{\pgfqpoint{4.698631in}{4.155678in}}%
\pgfpathclose%
\pgfusepath{fill}%
\end{pgfscope}%
\begin{pgfscope}%
\pgfpathrectangle{\pgfqpoint{0.680860in}{0.078740in}}{\pgfqpoint{7.842520in}{7.842520in}}%
\pgfusepath{clip}%
\pgfsetbuttcap%
\pgfsetroundjoin%
\definecolor{currentfill}{rgb}{0.243113,0.292092,0.538516}%
\pgfsetfillcolor{currentfill}%
\pgfsetlinewidth{0.000000pt}%
\definecolor{currentstroke}{rgb}{0.140536,0.530132,0.555659}%
\pgfsetstrokecolor{currentstroke}%
\pgfsetdash{}{0pt}%
\pgfpathmoveto{\pgfqpoint{3.938304in}{4.386632in}}%
\pgfpathlineto{\pgfqpoint{4.017773in}{4.343374in}}%
\pgfpathlineto{\pgfqpoint{3.887232in}{4.357547in}}%
\pgfpathclose%
\pgfusepath{fill}%
\end{pgfscope}%
\begin{pgfscope}%
\pgfpathrectangle{\pgfqpoint{0.680860in}{0.078740in}}{\pgfqpoint{7.842520in}{7.842520in}}%
\pgfusepath{clip}%
\pgfsetbuttcap%
\pgfsetroundjoin%
\definecolor{currentfill}{rgb}{0.248629,0.278775,0.534556}%
\pgfsetfillcolor{currentfill}%
\pgfsetlinewidth{0.000000pt}%
\definecolor{currentstroke}{rgb}{0.139147,0.533812,0.555298}%
\pgfsetstrokecolor{currentstroke}%
\pgfsetdash{}{0pt}%
\pgfpathmoveto{\pgfqpoint{4.227282in}{4.275960in}}%
\pgfpathlineto{\pgfqpoint{4.148799in}{4.329892in}}%
\pgfpathlineto{\pgfqpoint{4.280318in}{4.317129in}}%
\pgfpathclose%
\pgfusepath{fill}%
\end{pgfscope}%
\begin{pgfscope}%
\pgfpathrectangle{\pgfqpoint{0.680860in}{0.078740in}}{\pgfqpoint{7.842520in}{7.842520in}}%
\pgfusepath{clip}%
\pgfsetbuttcap%
\pgfsetroundjoin%
\definecolor{currentfill}{rgb}{0.280894,0.078907,0.402329}%
\pgfsetfillcolor{currentfill}%
\pgfsetlinewidth{0.000000pt}%
\definecolor{currentstroke}{rgb}{0.137770,0.537492,0.554906}%
\pgfsetstrokecolor{currentstroke}%
\pgfsetdash{}{0pt}%
\pgfpathmoveto{\pgfqpoint{6.249408in}{3.585530in}}%
\pgfpathlineto{\pgfqpoint{6.114907in}{3.618082in}}%
\pgfpathlineto{\pgfqpoint{6.045962in}{3.749925in}}%
\pgfpathclose%
\pgfusepath{fill}%
\end{pgfscope}%
\begin{pgfscope}%
\pgfpathrectangle{\pgfqpoint{0.680860in}{0.078740in}}{\pgfqpoint{7.842520in}{7.842520in}}%
\pgfusepath{clip}%
\pgfsetbuttcap%
\pgfsetroundjoin%
\definecolor{currentfill}{rgb}{0.237441,0.305202,0.541921}%
\pgfsetfillcolor{currentfill}%
\pgfsetlinewidth{0.000000pt}%
\definecolor{currentstroke}{rgb}{0.136408,0.541173,0.554483}%
\pgfsetstrokecolor{currentstroke}%
\pgfsetdash{}{0pt}%
\pgfpathmoveto{\pgfqpoint{3.887232in}{4.357547in}}%
\pgfpathlineto{\pgfqpoint{3.807408in}{4.397430in}}%
\pgfpathlineto{\pgfqpoint{3.938304in}{4.386632in}}%
\pgfpathclose%
\pgfusepath{fill}%
\end{pgfscope}%
\begin{pgfscope}%
\pgfpathrectangle{\pgfqpoint{0.680860in}{0.078740in}}{\pgfqpoint{7.842520in}{7.842520in}}%
\pgfusepath{clip}%
\pgfsetbuttcap%
\pgfsetroundjoin%
\definecolor{currentfill}{rgb}{0.280255,0.165693,0.476498}%
\pgfsetfillcolor{currentfill}%
\pgfsetlinewidth{0.000000pt}%
\definecolor{currentstroke}{rgb}{0.135066,0.544853,0.554029}%
\pgfsetstrokecolor{currentstroke}%
\pgfsetdash{}{0pt}%
\pgfpathmoveto{\pgfqpoint{5.509682in}{3.857436in}}%
\pgfpathlineto{\pgfqpoint{5.437278in}{3.963455in}}%
\pgfpathlineto{\pgfqpoint{5.571268in}{3.941720in}}%
\pgfpathclose%
\pgfusepath{fill}%
\end{pgfscope}%
\begin{pgfscope}%
\pgfpathrectangle{\pgfqpoint{0.680860in}{0.078740in}}{\pgfqpoint{7.842520in}{7.842520in}}%
\pgfusepath{clip}%
\pgfsetbuttcap%
\pgfsetroundjoin%
\definecolor{currentfill}{rgb}{0.282290,0.145912,0.461510}%
\pgfsetfillcolor{currentfill}%
\pgfsetlinewidth{0.000000pt}%
\definecolor{currentstroke}{rgb}{0.133743,0.548535,0.553541}%
\pgfsetstrokecolor{currentstroke}%
\pgfsetdash{}{0pt}%
\pgfpathmoveto{\pgfqpoint{5.571268in}{3.941720in}}%
\pgfpathlineto{\pgfqpoint{5.776889in}{3.803133in}}%
\pgfpathlineto{\pgfqpoint{5.643054in}{3.830153in}}%
\pgfpathclose%
\pgfusepath{fill}%
\end{pgfscope}%
\begin{pgfscope}%
\pgfpathrectangle{\pgfqpoint{0.680860in}{0.078740in}}{\pgfqpoint{7.842520in}{7.842520in}}%
\pgfusepath{clip}%
\pgfsetbuttcap%
\pgfsetroundjoin%
\definecolor{currentfill}{rgb}{0.270595,0.214069,0.507052}%
\pgfsetfillcolor{currentfill}%
\pgfsetlinewidth{0.000000pt}%
\definecolor{currentstroke}{rgb}{0.132444,0.552216,0.553018}%
\pgfsetstrokecolor{currentstroke}%
\pgfsetdash{}{0pt}%
\pgfpathmoveto{\pgfqpoint{5.096428in}{4.099835in}}%
\pgfpathlineto{\pgfqpoint{5.170743in}{4.008212in}}%
\pgfpathlineto{\pgfqpoint{4.963342in}{4.117906in}}%
\pgfpathclose%
\pgfusepath{fill}%
\end{pgfscope}%
\begin{pgfscope}%
\pgfpathrectangle{\pgfqpoint{0.680860in}{0.078740in}}{\pgfqpoint{7.842520in}{7.842520in}}%
\pgfusepath{clip}%
\pgfsetbuttcap%
\pgfsetroundjoin%
\definecolor{currentfill}{rgb}{0.273006,0.204520,0.501721}%
\pgfsetfillcolor{currentfill}%
\pgfsetlinewidth{0.000000pt}%
\definecolor{currentstroke}{rgb}{0.131172,0.555899,0.552459}%
\pgfsetstrokecolor{currentstroke}%
\pgfsetdash{}{0pt}%
\pgfpathmoveto{\pgfqpoint{5.230009in}{4.082337in}}%
\pgfpathlineto{\pgfqpoint{5.303772in}{3.985624in}}%
\pgfpathlineto{\pgfqpoint{5.170743in}{4.008212in}}%
\pgfpathclose%
\pgfusepath{fill}%
\end{pgfscope}%
\begin{pgfscope}%
\pgfpathrectangle{\pgfqpoint{0.680860in}{0.078740in}}{\pgfqpoint{7.842520in}{7.842520in}}%
\pgfusepath{clip}%
\pgfsetbuttcap%
\pgfsetroundjoin%
\definecolor{currentfill}{rgb}{0.239346,0.300855,0.540844}%
\pgfsetfillcolor{currentfill}%
\pgfsetlinewidth{0.000000pt}%
\definecolor{currentstroke}{rgb}{0.129933,0.559582,0.551864}%
\pgfsetstrokecolor{currentstroke}%
\pgfsetdash{}{0pt}%
\pgfpathmoveto{\pgfqpoint{4.148799in}{4.329892in}}%
\pgfpathlineto{\pgfqpoint{4.017773in}{4.343374in}}%
\pgfpathlineto{\pgfqpoint{3.938304in}{4.386632in}}%
\pgfpathclose%
\pgfusepath{fill}%
\end{pgfscope}%
\begin{pgfscope}%
\pgfpathrectangle{\pgfqpoint{0.680860in}{0.078740in}}{\pgfqpoint{7.842520in}{7.842520in}}%
\pgfusepath{clip}%
\pgfsetbuttcap%
\pgfsetroundjoin%
\definecolor{currentfill}{rgb}{0.250425,0.274290,0.533103}%
\pgfsetfillcolor{currentfill}%
\pgfsetlinewidth{0.000000pt}%
\definecolor{currentstroke}{rgb}{0.128729,0.563265,0.551229}%
\pgfsetstrokecolor{currentstroke}%
\pgfsetdash{}{0pt}%
\pgfpathmoveto{\pgfqpoint{4.358400in}{4.259357in}}%
\pgfpathlineto{\pgfqpoint{4.412335in}{4.305112in}}%
\pgfpathlineto{\pgfqpoint{4.490000in}{4.243364in}}%
\pgfpathclose%
\pgfusepath{fill}%
\end{pgfscope}%
\begin{pgfscope}%
\pgfpathrectangle{\pgfqpoint{0.680860in}{0.078740in}}{\pgfqpoint{7.842520in}{7.842520in}}%
\pgfusepath{clip}%
\pgfsetbuttcap%
\pgfsetroundjoin%
\definecolor{currentfill}{rgb}{0.282910,0.105393,0.426902}%
\pgfsetfillcolor{currentfill}%
\pgfsetlinewidth{0.000000pt}%
\definecolor{currentstroke}{rgb}{0.127568,0.566949,0.550556}%
\pgfsetstrokecolor{currentstroke}%
\pgfsetdash{}{0pt}%
\pgfpathmoveto{\pgfqpoint{6.045962in}{3.749925in}}%
\pgfpathlineto{\pgfqpoint{6.114907in}{3.618082in}}%
\pgfpathlineto{\pgfqpoint{5.911190in}{3.776386in}}%
\pgfpathclose%
\pgfusepath{fill}%
\end{pgfscope}%
\begin{pgfscope}%
\pgfpathrectangle{\pgfqpoint{0.680860in}{0.078740in}}{\pgfqpoint{7.842520in}{7.842520in}}%
\pgfusepath{clip}%
\pgfsetbuttcap%
\pgfsetroundjoin%
\definecolor{currentfill}{rgb}{0.229739,0.322361,0.545706}%
\pgfsetfillcolor{currentfill}%
\pgfsetlinewidth{0.000000pt}%
\definecolor{currentstroke}{rgb}{0.126453,0.570633,0.549841}%
\pgfsetstrokecolor{currentstroke}%
\pgfsetdash{}{0pt}%
\pgfpathmoveto{\pgfqpoint{3.466067in}{4.447407in}}%
\pgfpathlineto{\pgfqpoint{3.596301in}{4.437991in}}%
\pgfpathlineto{\pgfqpoint{3.677004in}{4.409009in}}%
\pgfpathclose%
\pgfusepath{fill}%
\end{pgfscope}%
\begin{pgfscope}%
\pgfpathrectangle{\pgfqpoint{0.680860in}{0.078740in}}{\pgfqpoint{7.842520in}{7.842520in}}%
\pgfusepath{clip}%
\pgfsetbuttcap%
\pgfsetroundjoin%
\definecolor{currentfill}{rgb}{0.260571,0.246922,0.522828}%
\pgfsetfillcolor{currentfill}%
\pgfsetlinewidth{0.000000pt}%
\definecolor{currentstroke}{rgb}{0.125394,0.574318,0.549086}%
\pgfsetstrokecolor{currentstroke}%
\pgfsetdash{}{0pt}%
\pgfpathmoveto{\pgfqpoint{4.963342in}{4.117906in}}%
\pgfpathlineto{\pgfqpoint{4.830745in}{4.136528in}}%
\pgfpathlineto{\pgfqpoint{4.754675in}{4.213309in}}%
\pgfpathclose%
\pgfusepath{fill}%
\end{pgfscope}%
\begin{pgfscope}%
\pgfpathrectangle{\pgfqpoint{0.680860in}{0.078740in}}{\pgfqpoint{7.842520in}{7.842520in}}%
\pgfusepath{clip}%
\pgfsetbuttcap%
\pgfsetroundjoin%
\definecolor{currentfill}{rgb}{0.255645,0.260703,0.528312}%
\pgfsetfillcolor{currentfill}%
\pgfsetlinewidth{0.000000pt}%
\definecolor{currentstroke}{rgb}{0.124395,0.578002,0.548287}%
\pgfsetstrokecolor{currentstroke}%
\pgfsetdash{}{0pt}%
\pgfpathmoveto{\pgfqpoint{4.754675in}{4.213309in}}%
\pgfpathlineto{\pgfqpoint{4.830745in}{4.136528in}}%
\pgfpathlineto{\pgfqpoint{4.622090in}{4.228007in}}%
\pgfpathclose%
\pgfusepath{fill}%
\end{pgfscope}%
\begin{pgfscope}%
\pgfpathrectangle{\pgfqpoint{0.680860in}{0.078740in}}{\pgfqpoint{7.842520in}{7.842520in}}%
\pgfusepath{clip}%
\pgfsetbuttcap%
\pgfsetroundjoin%
\definecolor{currentfill}{rgb}{0.229739,0.322361,0.545706}%
\pgfsetfillcolor{currentfill}%
\pgfsetlinewidth{0.000000pt}%
\definecolor{currentstroke}{rgb}{0.123463,0.581687,0.547445}%
\pgfsetstrokecolor{currentstroke}%
\pgfsetdash{}{0pt}%
\pgfpathmoveto{\pgfqpoint{3.727031in}{4.429432in}}%
\pgfpathlineto{\pgfqpoint{3.807408in}{4.397430in}}%
\pgfpathlineto{\pgfqpoint{3.677004in}{4.409009in}}%
\pgfpathclose%
\pgfusepath{fill}%
\end{pgfscope}%
\begin{pgfscope}%
\pgfpathrectangle{\pgfqpoint{0.680860in}{0.078740in}}{\pgfqpoint{7.842520in}{7.842520in}}%
\pgfusepath{clip}%
\pgfsetbuttcap%
\pgfsetroundjoin%
\definecolor{currentfill}{rgb}{0.244972,0.287675,0.537260}%
\pgfsetfillcolor{currentfill}%
\pgfsetlinewidth{0.000000pt}%
\definecolor{currentstroke}{rgb}{0.122606,0.585371,0.546557}%
\pgfsetstrokecolor{currentstroke}%
\pgfsetdash{}{0pt}%
\pgfpathmoveto{\pgfqpoint{4.280318in}{4.317129in}}%
\pgfpathlineto{\pgfqpoint{4.412335in}{4.305112in}}%
\pgfpathlineto{\pgfqpoint{4.358400in}{4.259357in}}%
\pgfpathclose%
\pgfusepath{fill}%
\end{pgfscope}%
\begin{pgfscope}%
\pgfpathrectangle{\pgfqpoint{0.680860in}{0.078740in}}{\pgfqpoint{7.842520in}{7.842520in}}%
\pgfusepath{clip}%
\pgfsetbuttcap%
\pgfsetroundjoin%
\definecolor{currentfill}{rgb}{0.223925,0.334994,0.548053}%
\pgfsetfillcolor{currentfill}%
\pgfsetlinewidth{0.000000pt}%
\definecolor{currentstroke}{rgb}{0.121831,0.589055,0.545623}%
\pgfsetstrokecolor{currentstroke}%
\pgfsetdash{}{0pt}%
\pgfpathmoveto{\pgfqpoint{3.466067in}{4.447407in}}%
\pgfpathlineto{\pgfqpoint{3.336323in}{4.457647in}}%
\pgfpathlineto{\pgfqpoint{3.384561in}{4.465080in}}%
\pgfpathclose%
\pgfusepath{fill}%
\end{pgfscope}%
\begin{pgfscope}%
\pgfpathrectangle{\pgfqpoint{0.680860in}{0.078740in}}{\pgfqpoint{7.842520in}{7.842520in}}%
\pgfusepath{clip}%
\pgfsetbuttcap%
\pgfsetroundjoin%
\definecolor{currentfill}{rgb}{0.267968,0.223549,0.512008}%
\pgfsetfillcolor{currentfill}%
\pgfsetlinewidth{0.000000pt}%
\definecolor{currentstroke}{rgb}{0.121148,0.592739,0.544641}%
\pgfsetstrokecolor{currentstroke}%
\pgfsetdash{}{0pt}%
\pgfpathmoveto{\pgfqpoint{5.170743in}{4.008212in}}%
\pgfpathlineto{\pgfqpoint{5.096428in}{4.099835in}}%
\pgfpathlineto{\pgfqpoint{5.230009in}{4.082337in}}%
\pgfpathclose%
\pgfusepath{fill}%
\end{pgfscope}%
\begin{pgfscope}%
\pgfpathrectangle{\pgfqpoint{0.680860in}{0.078740in}}{\pgfqpoint{7.842520in}{7.842520in}}%
\pgfusepath{clip}%
\pgfsetbuttcap%
\pgfsetroundjoin%
\definecolor{currentfill}{rgb}{0.246811,0.283237,0.535941}%
\pgfsetfillcolor{currentfill}%
\pgfsetlinewidth{0.000000pt}%
\definecolor{currentstroke}{rgb}{0.120565,0.596422,0.543611}%
\pgfsetstrokecolor{currentstroke}%
\pgfsetdash{}{0pt}%
\pgfpathmoveto{\pgfqpoint{4.490000in}{4.243364in}}%
\pgfpathlineto{\pgfqpoint{4.412335in}{4.305112in}}%
\pgfpathlineto{\pgfqpoint{4.622090in}{4.228007in}}%
\pgfpathclose%
\pgfusepath{fill}%
\end{pgfscope}%
\begin{pgfscope}%
\pgfpathrectangle{\pgfqpoint{0.680860in}{0.078740in}}{\pgfqpoint{7.842520in}{7.842520in}}%
\pgfusepath{clip}%
\pgfsetbuttcap%
\pgfsetroundjoin%
\definecolor{currentfill}{rgb}{0.281887,0.150881,0.465405}%
\pgfsetfillcolor{currentfill}%
\pgfsetlinewidth{0.000000pt}%
\definecolor{currentstroke}{rgb}{0.120092,0.600104,0.542530}%
\pgfsetstrokecolor{currentstroke}%
\pgfsetdash{}{0pt}%
\pgfpathmoveto{\pgfqpoint{5.705747in}{3.920439in}}%
\pgfpathlineto{\pgfqpoint{5.911190in}{3.776386in}}%
\pgfpathlineto{\pgfqpoint{5.776889in}{3.803133in}}%
\pgfpathclose%
\pgfusepath{fill}%
\end{pgfscope}%
\begin{pgfscope}%
\pgfpathrectangle{\pgfqpoint{0.680860in}{0.078740in}}{\pgfqpoint{7.842520in}{7.842520in}}%
\pgfusepath{clip}%
\pgfsetbuttcap%
\pgfsetroundjoin%
\definecolor{currentfill}{rgb}{0.221989,0.339161,0.548752}%
\pgfsetfillcolor{currentfill}%
\pgfsetlinewidth{0.000000pt}%
\definecolor{currentstroke}{rgb}{0.119738,0.603785,0.541400}%
\pgfsetstrokecolor{currentstroke}%
\pgfsetdash{}{0pt}%
\pgfpathmoveto{\pgfqpoint{3.336323in}{4.457647in}}%
\pgfpathlineto{\pgfqpoint{3.254528in}{4.472763in}}%
\pgfpathlineto{\pgfqpoint{3.384561in}{4.465080in}}%
\pgfpathclose%
\pgfusepath{fill}%
\end{pgfscope}%
\begin{pgfscope}%
\pgfpathrectangle{\pgfqpoint{0.680860in}{0.078740in}}{\pgfqpoint{7.842520in}{7.842520in}}%
\pgfusepath{clip}%
\pgfsetbuttcap%
\pgfsetroundjoin%
\definecolor{currentfill}{rgb}{0.225863,0.330805,0.547314}%
\pgfsetfillcolor{currentfill}%
\pgfsetlinewidth{0.000000pt}%
\definecolor{currentstroke}{rgb}{0.119512,0.607464,0.540218}%
\pgfsetstrokecolor{currentstroke}%
\pgfsetdash{}{0pt}%
\pgfpathmoveto{\pgfqpoint{3.677004in}{4.409009in}}%
\pgfpathlineto{\pgfqpoint{3.596301in}{4.437991in}}%
\pgfpathlineto{\pgfqpoint{3.727031in}{4.429432in}}%
\pgfpathclose%
\pgfusepath{fill}%
\end{pgfscope}%
\begin{pgfscope}%
\pgfpathrectangle{\pgfqpoint{0.680860in}{0.078740in}}{\pgfqpoint{7.842520in}{7.842520in}}%
\pgfusepath{clip}%
\pgfsetbuttcap%
\pgfsetroundjoin%
\definecolor{currentfill}{rgb}{0.281446,0.084320,0.407414}%
\pgfsetfillcolor{currentfill}%
\pgfsetlinewidth{0.000000pt}%
\definecolor{currentstroke}{rgb}{0.119423,0.611141,0.538982}%
\pgfsetstrokecolor{currentstroke}%
\pgfsetdash{}{0pt}%
\pgfpathmoveto{\pgfqpoint{6.249408in}{3.585530in}}%
\pgfpathlineto{\pgfqpoint{6.316940in}{3.697915in}}%
\pgfpathlineto{\pgfqpoint{6.384357in}{3.553073in}}%
\pgfpathclose%
\pgfusepath{fill}%
\end{pgfscope}%
\begin{pgfscope}%
\pgfpathrectangle{\pgfqpoint{0.680860in}{0.078740in}}{\pgfqpoint{7.842520in}{7.842520in}}%
\pgfusepath{clip}%
\pgfsetbuttcap%
\pgfsetroundjoin%
\definecolor{currentfill}{rgb}{0.271828,0.209303,0.504434}%
\pgfsetfillcolor{currentfill}%
\pgfsetlinewidth{0.000000pt}%
\definecolor{currentstroke}{rgb}{0.119483,0.614817,0.537692}%
\pgfsetstrokecolor{currentstroke}%
\pgfsetdash{}{0pt}%
\pgfpathmoveto{\pgfqpoint{5.364091in}{4.065435in}}%
\pgfpathlineto{\pgfqpoint{5.437278in}{3.963455in}}%
\pgfpathlineto{\pgfqpoint{5.303772in}{3.985624in}}%
\pgfpathclose%
\pgfusepath{fill}%
\end{pgfscope}%
\begin{pgfscope}%
\pgfpathrectangle{\pgfqpoint{0.680860in}{0.078740in}}{\pgfqpoint{7.842520in}{7.842520in}}%
\pgfusepath{clip}%
\pgfsetbuttcap%
\pgfsetroundjoin%
\definecolor{currentfill}{rgb}{0.283091,0.110553,0.431554}%
\pgfsetfillcolor{currentfill}%
\pgfsetlinewidth{0.000000pt}%
\definecolor{currentstroke}{rgb}{0.119699,0.618490,0.536347}%
\pgfsetstrokecolor{currentstroke}%
\pgfsetdash{}{0pt}%
\pgfpathmoveto{\pgfqpoint{6.045962in}{3.749925in}}%
\pgfpathlineto{\pgfqpoint{6.181211in}{3.723764in}}%
\pgfpathlineto{\pgfqpoint{6.249408in}{3.585530in}}%
\pgfpathclose%
\pgfusepath{fill}%
\end{pgfscope}%
\begin{pgfscope}%
\pgfpathrectangle{\pgfqpoint{0.680860in}{0.078740in}}{\pgfqpoint{7.842520in}{7.842520in}}%
\pgfusepath{clip}%
\pgfsetbuttcap%
\pgfsetroundjoin%
\definecolor{currentfill}{rgb}{0.278826,0.175490,0.483397}%
\pgfsetfillcolor{currentfill}%
\pgfsetlinewidth{0.000000pt}%
\definecolor{currentstroke}{rgb}{0.120081,0.622161,0.534946}%
\pgfsetstrokecolor{currentstroke}%
\pgfsetdash{}{0pt}%
\pgfpathmoveto{\pgfqpoint{5.571268in}{3.941720in}}%
\pgfpathlineto{\pgfqpoint{5.705747in}{3.920439in}}%
\pgfpathlineto{\pgfqpoint{5.776889in}{3.803133in}}%
\pgfpathclose%
\pgfusepath{fill}%
\end{pgfscope}%
\begin{pgfscope}%
\pgfpathrectangle{\pgfqpoint{0.680860in}{0.078740in}}{\pgfqpoint{7.842520in}{7.842520in}}%
\pgfusepath{clip}%
\pgfsetbuttcap%
\pgfsetroundjoin%
\definecolor{currentfill}{rgb}{0.231674,0.318106,0.544834}%
\pgfsetfillcolor{currentfill}%
\pgfsetlinewidth{0.000000pt}%
\definecolor{currentstroke}{rgb}{0.120638,0.625828,0.533488}%
\pgfsetstrokecolor{currentstroke}%
\pgfsetdash{}{0pt}%
\pgfpathmoveto{\pgfqpoint{4.069699in}{4.376646in}}%
\pgfpathlineto{\pgfqpoint{4.148799in}{4.329892in}}%
\pgfpathlineto{\pgfqpoint{3.938304in}{4.386632in}}%
\pgfpathclose%
\pgfusepath{fill}%
\end{pgfscope}%
\begin{pgfscope}%
\pgfpathrectangle{\pgfqpoint{0.680860in}{0.078740in}}{\pgfqpoint{7.842520in}{7.842520in}}%
\pgfusepath{clip}%
\pgfsetbuttcap%
\pgfsetroundjoin%
\definecolor{currentfill}{rgb}{0.227802,0.326594,0.546532}%
\pgfsetfillcolor{currentfill}%
\pgfsetlinewidth{0.000000pt}%
\definecolor{currentstroke}{rgb}{0.121380,0.629492,0.531973}%
\pgfsetstrokecolor{currentstroke}%
\pgfsetdash{}{0pt}%
\pgfpathmoveto{\pgfqpoint{3.938304in}{4.386632in}}%
\pgfpathlineto{\pgfqpoint{3.807408in}{4.397430in}}%
\pgfpathlineto{\pgfqpoint{3.727031in}{4.429432in}}%
\pgfpathclose%
\pgfusepath{fill}%
\end{pgfscope}%
\begin{pgfscope}%
\pgfpathrectangle{\pgfqpoint{0.680860in}{0.078740in}}{\pgfqpoint{7.842520in}{7.842520in}}%
\pgfusepath{clip}%
\pgfsetbuttcap%
\pgfsetroundjoin%
\definecolor{currentfill}{rgb}{0.220057,0.343307,0.549413}%
\pgfsetfillcolor{currentfill}%
\pgfsetlinewidth{0.000000pt}%
\definecolor{currentstroke}{rgb}{0.122312,0.633153,0.530398}%
\pgfsetstrokecolor{currentstroke}%
\pgfsetdash{}{0pt}%
\pgfpathmoveto{\pgfqpoint{3.384561in}{4.465080in}}%
\pgfpathlineto{\pgfqpoint{3.596301in}{4.437991in}}%
\pgfpathlineto{\pgfqpoint{3.466067in}{4.447407in}}%
\pgfpathclose%
\pgfusepath{fill}%
\end{pgfscope}%
\begin{pgfscope}%
\pgfpathrectangle{\pgfqpoint{0.680860in}{0.078740in}}{\pgfqpoint{7.842520in}{7.842520in}}%
\pgfusepath{clip}%
\pgfsetbuttcap%
\pgfsetroundjoin%
\definecolor{currentfill}{rgb}{0.233603,0.313828,0.543914}%
\pgfsetfillcolor{currentfill}%
\pgfsetlinewidth{0.000000pt}%
\definecolor{currentstroke}{rgb}{0.123444,0.636809,0.528763}%
\pgfsetstrokecolor{currentstroke}%
\pgfsetdash{}{0pt}%
\pgfpathmoveto{\pgfqpoint{4.148799in}{4.329892in}}%
\pgfpathlineto{\pgfqpoint{4.201601in}{4.367504in}}%
\pgfpathlineto{\pgfqpoint{4.280318in}{4.317129in}}%
\pgfpathclose%
\pgfusepath{fill}%
\end{pgfscope}%
\begin{pgfscope}%
\pgfpathrectangle{\pgfqpoint{0.680860in}{0.078740in}}{\pgfqpoint{7.842520in}{7.842520in}}%
\pgfusepath{clip}%
\pgfsetbuttcap%
\pgfsetroundjoin%
\definecolor{currentfill}{rgb}{0.252194,0.269783,0.531579}%
\pgfsetfillcolor{currentfill}%
\pgfsetlinewidth{0.000000pt}%
\definecolor{currentstroke}{rgb}{0.124780,0.640461,0.527068}%
\pgfsetstrokecolor{currentstroke}%
\pgfsetdash{}{0pt}%
\pgfpathmoveto{\pgfqpoint{4.754675in}{4.213309in}}%
\pgfpathlineto{\pgfqpoint{4.887762in}{4.199297in}}%
\pgfpathlineto{\pgfqpoint{4.963342in}{4.117906in}}%
\pgfpathclose%
\pgfusepath{fill}%
\end{pgfscope}%
\begin{pgfscope}%
\pgfpathrectangle{\pgfqpoint{0.680860in}{0.078740in}}{\pgfqpoint{7.842520in}{7.842520in}}%
\pgfusepath{clip}%
\pgfsetbuttcap%
\pgfsetroundjoin%
\definecolor{currentfill}{rgb}{0.281446,0.084320,0.407414}%
\pgfsetfillcolor{currentfill}%
\pgfsetlinewidth{0.000000pt}%
\definecolor{currentstroke}{rgb}{0.126326,0.644107,0.525311}%
\pgfsetstrokecolor{currentstroke}%
\pgfsetdash{}{0pt}%
\pgfpathmoveto{\pgfqpoint{6.384357in}{3.553073in}}%
\pgfpathlineto{\pgfqpoint{6.316940in}{3.697915in}}%
\pgfpathlineto{\pgfqpoint{6.519758in}{3.520716in}}%
\pgfpathclose%
\pgfusepath{fill}%
\end{pgfscope}%
\begin{pgfscope}%
\pgfpathrectangle{\pgfqpoint{0.680860in}{0.078740in}}{\pgfqpoint{7.842520in}{7.842520in}}%
\pgfusepath{clip}%
\pgfsetbuttcap%
\pgfsetroundjoin%
\definecolor{currentfill}{rgb}{0.266580,0.228262,0.514349}%
\pgfsetfillcolor{currentfill}%
\pgfsetlinewidth{0.000000pt}%
\definecolor{currentstroke}{rgb}{0.128087,0.647749,0.523491}%
\pgfsetstrokecolor{currentstroke}%
\pgfsetdash{}{0pt}%
\pgfpathmoveto{\pgfqpoint{5.364091in}{4.065435in}}%
\pgfpathlineto{\pgfqpoint{5.303772in}{3.985624in}}%
\pgfpathlineto{\pgfqpoint{5.230009in}{4.082337in}}%
\pgfpathclose%
\pgfusepath{fill}%
\end{pgfscope}%
\begin{pgfscope}%
\pgfpathrectangle{\pgfqpoint{0.680860in}{0.078740in}}{\pgfqpoint{7.842520in}{7.842520in}}%
\pgfusepath{clip}%
\pgfsetbuttcap%
\pgfsetroundjoin%
\definecolor{currentfill}{rgb}{0.270595,0.214069,0.507052}%
\pgfsetfillcolor{currentfill}%
\pgfsetlinewidth{0.000000pt}%
\definecolor{currentstroke}{rgb}{0.130067,0.651384,0.521608}%
\pgfsetstrokecolor{currentstroke}%
\pgfsetdash{}{0pt}%
\pgfpathmoveto{\pgfqpoint{5.364091in}{4.065435in}}%
\pgfpathlineto{\pgfqpoint{5.571268in}{3.941720in}}%
\pgfpathlineto{\pgfqpoint{5.437278in}{3.963455in}}%
\pgfpathclose%
\pgfusepath{fill}%
\end{pgfscope}%
\begin{pgfscope}%
\pgfpathrectangle{\pgfqpoint{0.680860in}{0.078740in}}{\pgfqpoint{7.842520in}{7.842520in}}%
\pgfusepath{clip}%
\pgfsetbuttcap%
\pgfsetroundjoin%
\definecolor{currentfill}{rgb}{0.257322,0.256130,0.526563}%
\pgfsetfillcolor{currentfill}%
\pgfsetlinewidth{0.000000pt}%
\definecolor{currentstroke}{rgb}{0.132268,0.655014,0.519661}%
\pgfsetstrokecolor{currentstroke}%
\pgfsetdash{}{0pt}%
\pgfpathmoveto{\pgfqpoint{5.021359in}{4.185999in}}%
\pgfpathlineto{\pgfqpoint{5.096428in}{4.099835in}}%
\pgfpathlineto{\pgfqpoint{4.963342in}{4.117906in}}%
\pgfpathclose%
\pgfusepath{fill}%
\end{pgfscope}%
\begin{pgfscope}%
\pgfpathrectangle{\pgfqpoint{0.680860in}{0.078740in}}{\pgfqpoint{7.842520in}{7.842520in}}%
\pgfusepath{clip}%
\pgfsetbuttcap%
\pgfsetroundjoin%
\definecolor{currentfill}{rgb}{0.283091,0.110553,0.431554}%
\pgfsetfillcolor{currentfill}%
\pgfsetlinewidth{0.000000pt}%
\definecolor{currentstroke}{rgb}{0.134692,0.658636,0.517649}%
\pgfsetstrokecolor{currentstroke}%
\pgfsetdash{}{0pt}%
\pgfpathmoveto{\pgfqpoint{6.181211in}{3.723764in}}%
\pgfpathlineto{\pgfqpoint{6.316940in}{3.697915in}}%
\pgfpathlineto{\pgfqpoint{6.249408in}{3.585530in}}%
\pgfpathclose%
\pgfusepath{fill}%
\end{pgfscope}%
\begin{pgfscope}%
\pgfpathrectangle{\pgfqpoint{0.680860in}{0.078740in}}{\pgfqpoint{7.842520in}{7.842520in}}%
\pgfusepath{clip}%
\pgfsetbuttcap%
\pgfsetroundjoin%
\definecolor{currentfill}{rgb}{0.229739,0.322361,0.545706}%
\pgfsetfillcolor{currentfill}%
\pgfsetlinewidth{0.000000pt}%
\definecolor{currentstroke}{rgb}{0.137339,0.662252,0.515571}%
\pgfsetstrokecolor{currentstroke}%
\pgfsetdash{}{0pt}%
\pgfpathmoveto{\pgfqpoint{4.069699in}{4.376646in}}%
\pgfpathlineto{\pgfqpoint{4.201601in}{4.367504in}}%
\pgfpathlineto{\pgfqpoint{4.148799in}{4.329892in}}%
\pgfpathclose%
\pgfusepath{fill}%
\end{pgfscope}%
\begin{pgfscope}%
\pgfpathrectangle{\pgfqpoint{0.680860in}{0.078740in}}{\pgfqpoint{7.842520in}{7.842520in}}%
\pgfusepath{clip}%
\pgfsetbuttcap%
\pgfsetroundjoin%
\definecolor{currentfill}{rgb}{0.239346,0.300855,0.540844}%
\pgfsetfillcolor{currentfill}%
\pgfsetlinewidth{0.000000pt}%
\definecolor{currentstroke}{rgb}{0.140210,0.665859,0.513427}%
\pgfsetstrokecolor{currentstroke}%
\pgfsetdash{}{0pt}%
\pgfpathmoveto{\pgfqpoint{4.622090in}{4.228007in}}%
\pgfpathlineto{\pgfqpoint{4.412335in}{4.305112in}}%
\pgfpathlineto{\pgfqpoint{4.544859in}{4.293872in}}%
\pgfpathclose%
\pgfusepath{fill}%
\end{pgfscope}%
\begin{pgfscope}%
\pgfpathrectangle{\pgfqpoint{0.680860in}{0.078740in}}{\pgfqpoint{7.842520in}{7.842520in}}%
\pgfusepath{clip}%
\pgfsetbuttcap%
\pgfsetroundjoin%
\definecolor{currentfill}{rgb}{0.278012,0.180367,0.486697}%
\pgfsetfillcolor{currentfill}%
\pgfsetlinewidth{0.000000pt}%
\definecolor{currentstroke}{rgb}{0.143303,0.669459,0.511215}%
\pgfsetstrokecolor{currentstroke}%
\pgfsetdash{}{0pt}%
\pgfpathmoveto{\pgfqpoint{5.840720in}{3.899631in}}%
\pgfpathlineto{\pgfqpoint{5.911190in}{3.776386in}}%
\pgfpathlineto{\pgfqpoint{5.705747in}{3.920439in}}%
\pgfpathclose%
\pgfusepath{fill}%
\end{pgfscope}%
\begin{pgfscope}%
\pgfpathrectangle{\pgfqpoint{0.680860in}{0.078740in}}{\pgfqpoint{7.842520in}{7.842520in}}%
\pgfusepath{clip}%
\pgfsetbuttcap%
\pgfsetroundjoin%
\definecolor{currentfill}{rgb}{0.280868,0.160771,0.472899}%
\pgfsetfillcolor{currentfill}%
\pgfsetlinewidth{0.000000pt}%
\definecolor{currentstroke}{rgb}{0.146616,0.673050,0.508936}%
\pgfsetstrokecolor{currentstroke}%
\pgfsetdash{}{0pt}%
\pgfpathmoveto{\pgfqpoint{5.911190in}{3.776386in}}%
\pgfpathlineto{\pgfqpoint{5.976195in}{3.879315in}}%
\pgfpathlineto{\pgfqpoint{6.045962in}{3.749925in}}%
\pgfpathclose%
\pgfusepath{fill}%
\end{pgfscope}%
\begin{pgfscope}%
\pgfpathrectangle{\pgfqpoint{0.680860in}{0.078740in}}{\pgfqpoint{7.842520in}{7.842520in}}%
\pgfusepath{clip}%
\pgfsetbuttcap%
\pgfsetroundjoin%
\definecolor{currentfill}{rgb}{0.231674,0.318106,0.544834}%
\pgfsetfillcolor{currentfill}%
\pgfsetlinewidth{0.000000pt}%
\definecolor{currentstroke}{rgb}{0.150148,0.676631,0.506589}%
\pgfsetstrokecolor{currentstroke}%
\pgfsetdash{}{0pt}%
\pgfpathmoveto{\pgfqpoint{4.280318in}{4.317129in}}%
\pgfpathlineto{\pgfqpoint{4.201601in}{4.367504in}}%
\pgfpathlineto{\pgfqpoint{4.412335in}{4.305112in}}%
\pgfpathclose%
\pgfusepath{fill}%
\end{pgfscope}%
\begin{pgfscope}%
\pgfpathrectangle{\pgfqpoint{0.680860in}{0.078740in}}{\pgfqpoint{7.842520in}{7.842520in}}%
\pgfusepath{clip}%
\pgfsetbuttcap%
\pgfsetroundjoin%
\definecolor{currentfill}{rgb}{0.241237,0.296485,0.539709}%
\pgfsetfillcolor{currentfill}%
\pgfsetlinewidth{0.000000pt}%
\definecolor{currentstroke}{rgb}{0.153894,0.680203,0.504172}%
\pgfsetstrokecolor{currentstroke}%
\pgfsetdash{}{0pt}%
\pgfpathmoveto{\pgfqpoint{4.677895in}{4.283438in}}%
\pgfpathlineto{\pgfqpoint{4.754675in}{4.213309in}}%
\pgfpathlineto{\pgfqpoint{4.622090in}{4.228007in}}%
\pgfpathclose%
\pgfusepath{fill}%
\end{pgfscope}%
\begin{pgfscope}%
\pgfpathrectangle{\pgfqpoint{0.680860in}{0.078740in}}{\pgfqpoint{7.842520in}{7.842520in}}%
\pgfusepath{clip}%
\pgfsetbuttcap%
\pgfsetroundjoin%
\definecolor{currentfill}{rgb}{0.250425,0.274290,0.533103}%
\pgfsetfillcolor{currentfill}%
\pgfsetlinewidth{0.000000pt}%
\definecolor{currentstroke}{rgb}{0.157851,0.683765,0.501686}%
\pgfsetstrokecolor{currentstroke}%
\pgfsetdash{}{0pt}%
\pgfpathmoveto{\pgfqpoint{4.963342in}{4.117906in}}%
\pgfpathlineto{\pgfqpoint{4.887762in}{4.199297in}}%
\pgfpathlineto{\pgfqpoint{5.021359in}{4.185999in}}%
\pgfpathclose%
\pgfusepath{fill}%
\end{pgfscope}%
\begin{pgfscope}%
\pgfpathrectangle{\pgfqpoint{0.680860in}{0.078740in}}{\pgfqpoint{7.842520in}{7.842520in}}%
\pgfusepath{clip}%
\pgfsetbuttcap%
\pgfsetroundjoin%
\definecolor{currentfill}{rgb}{0.212395,0.359683,0.551710}%
\pgfsetfillcolor{currentfill}%
\pgfsetlinewidth{0.000000pt}%
\definecolor{currentstroke}{rgb}{0.162016,0.687316,0.499129}%
\pgfsetstrokecolor{currentstroke}%
\pgfsetdash{}{0pt}%
\pgfpathmoveto{\pgfqpoint{3.384561in}{4.465080in}}%
\pgfpathlineto{\pgfqpoint{3.254528in}{4.472763in}}%
\pgfpathlineto{\pgfqpoint{3.172299in}{4.478866in}}%
\pgfpathclose%
\pgfusepath{fill}%
\end{pgfscope}%
\begin{pgfscope}%
\pgfpathrectangle{\pgfqpoint{0.680860in}{0.078740in}}{\pgfqpoint{7.842520in}{7.842520in}}%
\pgfusepath{clip}%
\pgfsetbuttcap%
\pgfsetroundjoin%
\definecolor{currentfill}{rgb}{0.220057,0.343307,0.549413}%
\pgfsetfillcolor{currentfill}%
\pgfsetlinewidth{0.000000pt}%
\definecolor{currentstroke}{rgb}{0.166383,0.690856,0.496502}%
\pgfsetstrokecolor{currentstroke}%
\pgfsetdash{}{0pt}%
\pgfpathmoveto{\pgfqpoint{3.727031in}{4.429432in}}%
\pgfpathlineto{\pgfqpoint{3.858265in}{4.421765in}}%
\pgfpathlineto{\pgfqpoint{3.938304in}{4.386632in}}%
\pgfpathclose%
\pgfusepath{fill}%
\end{pgfscope}%
\begin{pgfscope}%
\pgfpathrectangle{\pgfqpoint{0.680860in}{0.078740in}}{\pgfqpoint{7.842520in}{7.842520in}}%
\pgfusepath{clip}%
\pgfsetbuttcap%
\pgfsetroundjoin%
\definecolor{currentfill}{rgb}{0.214298,0.355619,0.551184}%
\pgfsetfillcolor{currentfill}%
\pgfsetlinewidth{0.000000pt}%
\definecolor{currentstroke}{rgb}{0.170948,0.694384,0.493803}%
\pgfsetstrokecolor{currentstroke}%
\pgfsetdash{}{0pt}%
\pgfpathmoveto{\pgfqpoint{3.515094in}{4.458315in}}%
\pgfpathlineto{\pgfqpoint{3.596301in}{4.437991in}}%
\pgfpathlineto{\pgfqpoint{3.384561in}{4.465080in}}%
\pgfpathclose%
\pgfusepath{fill}%
\end{pgfscope}%
\begin{pgfscope}%
\pgfpathrectangle{\pgfqpoint{0.680860in}{0.078740in}}{\pgfqpoint{7.842520in}{7.842520in}}%
\pgfusepath{clip}%
\pgfsetbuttcap%
\pgfsetroundjoin%
\definecolor{currentfill}{rgb}{0.235526,0.309527,0.542944}%
\pgfsetfillcolor{currentfill}%
\pgfsetlinewidth{0.000000pt}%
\definecolor{currentstroke}{rgb}{0.175707,0.697900,0.491033}%
\pgfsetstrokecolor{currentstroke}%
\pgfsetdash{}{0pt}%
\pgfpathmoveto{\pgfqpoint{4.622090in}{4.228007in}}%
\pgfpathlineto{\pgfqpoint{4.544859in}{4.293872in}}%
\pgfpathlineto{\pgfqpoint{4.677895in}{4.283438in}}%
\pgfpathclose%
\pgfusepath{fill}%
\end{pgfscope}%
\begin{pgfscope}%
\pgfpathrectangle{\pgfqpoint{0.680860in}{0.078740in}}{\pgfqpoint{7.842520in}{7.842520in}}%
\pgfusepath{clip}%
\pgfsetbuttcap%
\pgfsetroundjoin%
\definecolor{currentfill}{rgb}{0.220057,0.343307,0.549413}%
\pgfsetfillcolor{currentfill}%
\pgfsetlinewidth{0.000000pt}%
\definecolor{currentstroke}{rgb}{0.180653,0.701402,0.488189}%
\pgfsetstrokecolor{currentstroke}%
\pgfsetdash{}{0pt}%
\pgfpathmoveto{\pgfqpoint{3.938304in}{4.386632in}}%
\pgfpathlineto{\pgfqpoint{3.990011in}{4.415024in}}%
\pgfpathlineto{\pgfqpoint{4.069699in}{4.376646in}}%
\pgfpathclose%
\pgfusepath{fill}%
\end{pgfscope}%
\begin{pgfscope}%
\pgfpathrectangle{\pgfqpoint{0.680860in}{0.078740in}}{\pgfqpoint{7.842520in}{7.842520in}}%
\pgfusepath{clip}%
\pgfsetbuttcap%
\pgfsetroundjoin%
\definecolor{currentfill}{rgb}{0.253935,0.265254,0.529983}%
\pgfsetfillcolor{currentfill}%
\pgfsetlinewidth{0.000000pt}%
\definecolor{currentstroke}{rgb}{0.185783,0.704891,0.485273}%
\pgfsetstrokecolor{currentstroke}%
\pgfsetdash{}{0pt}%
\pgfpathmoveto{\pgfqpoint{5.096428in}{4.099835in}}%
\pgfpathlineto{\pgfqpoint{5.155473in}{4.173442in}}%
\pgfpathlineto{\pgfqpoint{5.230009in}{4.082337in}}%
\pgfpathclose%
\pgfusepath{fill}%
\end{pgfscope}%
\begin{pgfscope}%
\pgfpathrectangle{\pgfqpoint{0.680860in}{0.078740in}}{\pgfqpoint{7.842520in}{7.842520in}}%
\pgfusepath{clip}%
\pgfsetbuttcap%
\pgfsetroundjoin%
\definecolor{currentfill}{rgb}{0.214298,0.355619,0.551184}%
\pgfsetfillcolor{currentfill}%
\pgfsetlinewidth{0.000000pt}%
\definecolor{currentstroke}{rgb}{0.191090,0.708366,0.482284}%
\pgfsetstrokecolor{currentstroke}%
\pgfsetdash{}{0pt}%
\pgfpathmoveto{\pgfqpoint{3.727031in}{4.429432in}}%
\pgfpathlineto{\pgfqpoint{3.596301in}{4.437991in}}%
\pgfpathlineto{\pgfqpoint{3.646134in}{4.452504in}}%
\pgfpathclose%
\pgfusepath{fill}%
\end{pgfscope}%
\begin{pgfscope}%
\pgfpathrectangle{\pgfqpoint{0.680860in}{0.078740in}}{\pgfqpoint{7.842520in}{7.842520in}}%
\pgfusepath{clip}%
\pgfsetbuttcap%
\pgfsetroundjoin%
\definecolor{currentfill}{rgb}{0.277134,0.185228,0.489898}%
\pgfsetfillcolor{currentfill}%
\pgfsetlinewidth{0.000000pt}%
\definecolor{currentstroke}{rgb}{0.196571,0.711827,0.479221}%
\pgfsetstrokecolor{currentstroke}%
\pgfsetdash{}{0pt}%
\pgfpathmoveto{\pgfqpoint{5.840720in}{3.899631in}}%
\pgfpathlineto{\pgfqpoint{5.976195in}{3.879315in}}%
\pgfpathlineto{\pgfqpoint{5.911190in}{3.776386in}}%
\pgfpathclose%
\pgfusepath{fill}%
\end{pgfscope}%
\begin{pgfscope}%
\pgfpathrectangle{\pgfqpoint{0.680860in}{0.078740in}}{\pgfqpoint{7.842520in}{7.842520in}}%
\pgfusepath{clip}%
\pgfsetbuttcap%
\pgfsetroundjoin%
\definecolor{currentfill}{rgb}{0.281924,0.089666,0.412415}%
\pgfsetfillcolor{currentfill}%
\pgfsetlinewidth{0.000000pt}%
\definecolor{currentstroke}{rgb}{0.202219,0.715272,0.476084}%
\pgfsetstrokecolor{currentstroke}%
\pgfsetdash{}{0pt}%
\pgfpathmoveto{\pgfqpoint{6.519758in}{3.520716in}}%
\pgfpathlineto{\pgfqpoint{6.589863in}{3.647214in}}%
\pgfpathlineto{\pgfqpoint{6.655614in}{3.488467in}}%
\pgfpathclose%
\pgfusepath{fill}%
\end{pgfscope}%
\begin{pgfscope}%
\pgfpathrectangle{\pgfqpoint{0.680860in}{0.078740in}}{\pgfqpoint{7.842520in}{7.842520in}}%
\pgfusepath{clip}%
\pgfsetbuttcap%
\pgfsetroundjoin%
\definecolor{currentfill}{rgb}{0.263663,0.237631,0.518762}%
\pgfsetfillcolor{currentfill}%
\pgfsetlinewidth{0.000000pt}%
\definecolor{currentstroke}{rgb}{0.208030,0.718701,0.472873}%
\pgfsetstrokecolor{currentstroke}%
\pgfsetdash{}{0pt}%
\pgfpathmoveto{\pgfqpoint{5.498681in}{4.049153in}}%
\pgfpathlineto{\pgfqpoint{5.571268in}{3.941720in}}%
\pgfpathlineto{\pgfqpoint{5.364091in}{4.065435in}}%
\pgfpathclose%
\pgfusepath{fill}%
\end{pgfscope}%
\begin{pgfscope}%
\pgfpathrectangle{\pgfqpoint{0.680860in}{0.078740in}}{\pgfqpoint{7.842520in}{7.842520in}}%
\pgfusepath{clip}%
\pgfsetbuttcap%
\pgfsetroundjoin%
\definecolor{currentfill}{rgb}{0.283229,0.120777,0.440584}%
\pgfsetfillcolor{currentfill}%
\pgfsetlinewidth{0.000000pt}%
\definecolor{currentstroke}{rgb}{0.214000,0.722114,0.469588}%
\pgfsetstrokecolor{currentstroke}%
\pgfsetdash{}{0pt}%
\pgfpathmoveto{\pgfqpoint{6.316940in}{3.697915in}}%
\pgfpathlineto{\pgfqpoint{6.453156in}{3.672394in}}%
\pgfpathlineto{\pgfqpoint{6.519758in}{3.520716in}}%
\pgfpathclose%
\pgfusepath{fill}%
\end{pgfscope}%
\begin{pgfscope}%
\pgfpathrectangle{\pgfqpoint{0.680860in}{0.078740in}}{\pgfqpoint{7.842520in}{7.842520in}}%
\pgfusepath{clip}%
\pgfsetbuttcap%
\pgfsetroundjoin%
\definecolor{currentfill}{rgb}{0.267968,0.223549,0.512008}%
\pgfsetfillcolor{currentfill}%
\pgfsetlinewidth{0.000000pt}%
\definecolor{currentstroke}{rgb}{0.220124,0.725509,0.466226}%
\pgfsetstrokecolor{currentstroke}%
\pgfsetdash{}{0pt}%
\pgfpathmoveto{\pgfqpoint{5.633787in}{4.033516in}}%
\pgfpathlineto{\pgfqpoint{5.705747in}{3.920439in}}%
\pgfpathlineto{\pgfqpoint{5.571268in}{3.941720in}}%
\pgfpathclose%
\pgfusepath{fill}%
\end{pgfscope}%
\begin{pgfscope}%
\pgfpathrectangle{\pgfqpoint{0.680860in}{0.078740in}}{\pgfqpoint{7.842520in}{7.842520in}}%
\pgfusepath{clip}%
\pgfsetbuttcap%
\pgfsetroundjoin%
\definecolor{currentfill}{rgb}{0.280255,0.165693,0.476498}%
\pgfsetfillcolor{currentfill}%
\pgfsetlinewidth{0.000000pt}%
\definecolor{currentstroke}{rgb}{0.226397,0.728888,0.462789}%
\pgfsetstrokecolor{currentstroke}%
\pgfsetdash{}{0pt}%
\pgfpathmoveto{\pgfqpoint{6.112176in}{3.859511in}}%
\pgfpathlineto{\pgfqpoint{6.181211in}{3.723764in}}%
\pgfpathlineto{\pgfqpoint{6.045962in}{3.749925in}}%
\pgfpathclose%
\pgfusepath{fill}%
\end{pgfscope}%
\begin{pgfscope}%
\pgfpathrectangle{\pgfqpoint{0.680860in}{0.078740in}}{\pgfqpoint{7.842520in}{7.842520in}}%
\pgfusepath{clip}%
\pgfsetbuttcap%
\pgfsetroundjoin%
\definecolor{currentfill}{rgb}{0.216210,0.351535,0.550627}%
\pgfsetfillcolor{currentfill}%
\pgfsetlinewidth{0.000000pt}%
\definecolor{currentstroke}{rgb}{0.232815,0.732247,0.459277}%
\pgfsetstrokecolor{currentstroke}%
\pgfsetdash{}{0pt}%
\pgfpathmoveto{\pgfqpoint{3.858265in}{4.421765in}}%
\pgfpathlineto{\pgfqpoint{3.990011in}{4.415024in}}%
\pgfpathlineto{\pgfqpoint{3.938304in}{4.386632in}}%
\pgfpathclose%
\pgfusepath{fill}%
\end{pgfscope}%
\begin{pgfscope}%
\pgfpathrectangle{\pgfqpoint{0.680860in}{0.078740in}}{\pgfqpoint{7.842520in}{7.842520in}}%
\pgfusepath{clip}%
\pgfsetbuttcap%
\pgfsetroundjoin%
\definecolor{currentfill}{rgb}{0.210503,0.363727,0.552206}%
\pgfsetfillcolor{currentfill}%
\pgfsetlinewidth{0.000000pt}%
\definecolor{currentstroke}{rgb}{0.239374,0.735588,0.455688}%
\pgfsetstrokecolor{currentstroke}%
\pgfsetdash{}{0pt}%
\pgfpathmoveto{\pgfqpoint{3.646134in}{4.452504in}}%
\pgfpathlineto{\pgfqpoint{3.596301in}{4.437991in}}%
\pgfpathlineto{\pgfqpoint{3.515094in}{4.458315in}}%
\pgfpathclose%
\pgfusepath{fill}%
\end{pgfscope}%
\begin{pgfscope}%
\pgfpathrectangle{\pgfqpoint{0.680860in}{0.078740in}}{\pgfqpoint{7.842520in}{7.842520in}}%
\pgfusepath{clip}%
\pgfsetbuttcap%
\pgfsetroundjoin%
\definecolor{currentfill}{rgb}{0.237441,0.305202,0.541921}%
\pgfsetfillcolor{currentfill}%
\pgfsetlinewidth{0.000000pt}%
\definecolor{currentstroke}{rgb}{0.246070,0.738910,0.452024}%
\pgfsetstrokecolor{currentstroke}%
\pgfsetdash{}{0pt}%
\pgfpathmoveto{\pgfqpoint{4.811454in}{4.273843in}}%
\pgfpathlineto{\pgfqpoint{4.887762in}{4.199297in}}%
\pgfpathlineto{\pgfqpoint{4.754675in}{4.213309in}}%
\pgfpathclose%
\pgfusepath{fill}%
\end{pgfscope}%
\begin{pgfscope}%
\pgfpathrectangle{\pgfqpoint{0.680860in}{0.078740in}}{\pgfqpoint{7.842520in}{7.842520in}}%
\pgfusepath{clip}%
\pgfsetbuttcap%
\pgfsetroundjoin%
\definecolor{currentfill}{rgb}{0.246811,0.283237,0.535941}%
\pgfsetfillcolor{currentfill}%
\pgfsetlinewidth{0.000000pt}%
\definecolor{currentstroke}{rgb}{0.252899,0.742211,0.448284}%
\pgfsetstrokecolor{currentstroke}%
\pgfsetdash{}{0pt}%
\pgfpathmoveto{\pgfqpoint{5.021359in}{4.185999in}}%
\pgfpathlineto{\pgfqpoint{5.155473in}{4.173442in}}%
\pgfpathlineto{\pgfqpoint{5.096428in}{4.099835in}}%
\pgfpathclose%
\pgfusepath{fill}%
\end{pgfscope}%
\begin{pgfscope}%
\pgfpathrectangle{\pgfqpoint{0.680860in}{0.078740in}}{\pgfqpoint{7.842520in}{7.842520in}}%
\pgfusepath{clip}%
\pgfsetbuttcap%
\pgfsetroundjoin%
\definecolor{currentfill}{rgb}{0.223925,0.334994,0.548053}%
\pgfsetfillcolor{currentfill}%
\pgfsetlinewidth{0.000000pt}%
\definecolor{currentstroke}{rgb}{0.259857,0.745492,0.444467}%
\pgfsetstrokecolor{currentstroke}%
\pgfsetdash{}{0pt}%
\pgfpathmoveto{\pgfqpoint{4.412335in}{4.305112in}}%
\pgfpathlineto{\pgfqpoint{4.201601in}{4.367504in}}%
\pgfpathlineto{\pgfqpoint{4.334017in}{4.359240in}}%
\pgfpathclose%
\pgfusepath{fill}%
\end{pgfscope}%
\begin{pgfscope}%
\pgfpathrectangle{\pgfqpoint{0.680860in}{0.078740in}}{\pgfqpoint{7.842520in}{7.842520in}}%
\pgfusepath{clip}%
\pgfsetbuttcap%
\pgfsetroundjoin%
\definecolor{currentfill}{rgb}{0.206756,0.371758,0.553117}%
\pgfsetfillcolor{currentfill}%
\pgfsetlinewidth{0.000000pt}%
\definecolor{currentstroke}{rgb}{0.266941,0.748751,0.440573}%
\pgfsetstrokecolor{currentstroke}%
\pgfsetdash{}{0pt}%
\pgfpathmoveto{\pgfqpoint{3.172299in}{4.478866in}}%
\pgfpathlineto{\pgfqpoint{3.302607in}{4.473451in}}%
\pgfpathlineto{\pgfqpoint{3.384561in}{4.465080in}}%
\pgfpathclose%
\pgfusepath{fill}%
\end{pgfscope}%
\begin{pgfscope}%
\pgfpathrectangle{\pgfqpoint{0.680860in}{0.078740in}}{\pgfqpoint{7.842520in}{7.842520in}}%
\pgfusepath{clip}%
\pgfsetbuttcap%
\pgfsetroundjoin%
\definecolor{currentfill}{rgb}{0.252194,0.269783,0.531579}%
\pgfsetfillcolor{currentfill}%
\pgfsetlinewidth{0.000000pt}%
\definecolor{currentstroke}{rgb}{0.274149,0.751988,0.436601}%
\pgfsetstrokecolor{currentstroke}%
\pgfsetdash{}{0pt}%
\pgfpathmoveto{\pgfqpoint{5.230009in}{4.082337in}}%
\pgfpathlineto{\pgfqpoint{5.155473in}{4.173442in}}%
\pgfpathlineto{\pgfqpoint{5.364091in}{4.065435in}}%
\pgfpathclose%
\pgfusepath{fill}%
\end{pgfscope}%
\begin{pgfscope}%
\pgfpathrectangle{\pgfqpoint{0.680860in}{0.078740in}}{\pgfqpoint{7.842520in}{7.842520in}}%
\pgfusepath{clip}%
\pgfsetbuttcap%
\pgfsetroundjoin%
\definecolor{currentfill}{rgb}{0.210503,0.363727,0.552206}%
\pgfsetfillcolor{currentfill}%
\pgfsetlinewidth{0.000000pt}%
\definecolor{currentstroke}{rgb}{0.281477,0.755203,0.432552}%
\pgfsetstrokecolor{currentstroke}%
\pgfsetdash{}{0pt}%
\pgfpathmoveto{\pgfqpoint{3.646134in}{4.452504in}}%
\pgfpathlineto{\pgfqpoint{3.858265in}{4.421765in}}%
\pgfpathlineto{\pgfqpoint{3.727031in}{4.429432in}}%
\pgfpathclose%
\pgfusepath{fill}%
\end{pgfscope}%
\begin{pgfscope}%
\pgfpathrectangle{\pgfqpoint{0.680860in}{0.078740in}}{\pgfqpoint{7.842520in}{7.842520in}}%
\pgfusepath{clip}%
\pgfsetbuttcap%
\pgfsetroundjoin%
\definecolor{currentfill}{rgb}{0.225863,0.330805,0.547314}%
\pgfsetfillcolor{currentfill}%
\pgfsetlinewidth{0.000000pt}%
\definecolor{currentstroke}{rgb}{0.288921,0.758394,0.428426}%
\pgfsetstrokecolor{currentstroke}%
\pgfsetdash{}{0pt}%
\pgfpathmoveto{\pgfqpoint{4.544859in}{4.293872in}}%
\pgfpathlineto{\pgfqpoint{4.412335in}{4.305112in}}%
\pgfpathlineto{\pgfqpoint{4.466954in}{4.351887in}}%
\pgfpathclose%
\pgfusepath{fill}%
\end{pgfscope}%
\begin{pgfscope}%
\pgfpathrectangle{\pgfqpoint{0.680860in}{0.078740in}}{\pgfqpoint{7.842520in}{7.842520in}}%
\pgfusepath{clip}%
\pgfsetbuttcap%
\pgfsetroundjoin%
\definecolor{currentfill}{rgb}{0.283229,0.120777,0.440584}%
\pgfsetfillcolor{currentfill}%
\pgfsetlinewidth{0.000000pt}%
\definecolor{currentstroke}{rgb}{0.296479,0.761561,0.424223}%
\pgfsetstrokecolor{currentstroke}%
\pgfsetdash{}{0pt}%
\pgfpathmoveto{\pgfqpoint{6.519758in}{3.520716in}}%
\pgfpathlineto{\pgfqpoint{6.453156in}{3.672394in}}%
\pgfpathlineto{\pgfqpoint{6.589863in}{3.647214in}}%
\pgfpathclose%
\pgfusepath{fill}%
\end{pgfscope}%
\begin{pgfscope}%
\pgfpathrectangle{\pgfqpoint{0.680860in}{0.078740in}}{\pgfqpoint{7.842520in}{7.842520in}}%
\pgfusepath{clip}%
\pgfsetbuttcap%
\pgfsetroundjoin%
\definecolor{currentfill}{rgb}{0.260571,0.246922,0.522828}%
\pgfsetfillcolor{currentfill}%
\pgfsetlinewidth{0.000000pt}%
\definecolor{currentstroke}{rgb}{0.304148,0.764704,0.419943}%
\pgfsetstrokecolor{currentstroke}%
\pgfsetdash{}{0pt}%
\pgfpathmoveto{\pgfqpoint{5.498681in}{4.049153in}}%
\pgfpathlineto{\pgfqpoint{5.633787in}{4.033516in}}%
\pgfpathlineto{\pgfqpoint{5.571268in}{3.941720in}}%
\pgfpathclose%
\pgfusepath{fill}%
\end{pgfscope}%
\begin{pgfscope}%
\pgfpathrectangle{\pgfqpoint{0.680860in}{0.078740in}}{\pgfqpoint{7.842520in}{7.842520in}}%
\pgfusepath{clip}%
\pgfsetbuttcap%
\pgfsetroundjoin%
\definecolor{currentfill}{rgb}{0.279574,0.170599,0.479997}%
\pgfsetfillcolor{currentfill}%
\pgfsetlinewidth{0.000000pt}%
\definecolor{currentstroke}{rgb}{0.311925,0.767822,0.415586}%
\pgfsetstrokecolor{currentstroke}%
\pgfsetdash{}{0pt}%
\pgfpathmoveto{\pgfqpoint{6.112176in}{3.859511in}}%
\pgfpathlineto{\pgfqpoint{6.316940in}{3.697915in}}%
\pgfpathlineto{\pgfqpoint{6.181211in}{3.723764in}}%
\pgfpathclose%
\pgfusepath{fill}%
\end{pgfscope}%
\begin{pgfscope}%
\pgfpathrectangle{\pgfqpoint{0.680860in}{0.078740in}}{\pgfqpoint{7.842520in}{7.842520in}}%
\pgfusepath{clip}%
\pgfsetbuttcap%
\pgfsetroundjoin%
\definecolor{currentfill}{rgb}{0.231674,0.318106,0.544834}%
\pgfsetfillcolor{currentfill}%
\pgfsetlinewidth{0.000000pt}%
\definecolor{currentstroke}{rgb}{0.319809,0.770914,0.411152}%
\pgfsetstrokecolor{currentstroke}%
\pgfsetdash{}{0pt}%
\pgfpathmoveto{\pgfqpoint{4.811454in}{4.273843in}}%
\pgfpathlineto{\pgfqpoint{4.754675in}{4.213309in}}%
\pgfpathlineto{\pgfqpoint{4.677895in}{4.283438in}}%
\pgfpathclose%
\pgfusepath{fill}%
\end{pgfscope}%
\begin{pgfscope}%
\pgfpathrectangle{\pgfqpoint{0.680860in}{0.078740in}}{\pgfqpoint{7.842520in}{7.842520in}}%
\pgfusepath{clip}%
\pgfsetbuttcap%
\pgfsetroundjoin%
\definecolor{currentfill}{rgb}{0.275191,0.194905,0.496005}%
\pgfsetfillcolor{currentfill}%
\pgfsetlinewidth{0.000000pt}%
\definecolor{currentstroke}{rgb}{0.327796,0.773980,0.406640}%
\pgfsetstrokecolor{currentstroke}%
\pgfsetdash{}{0pt}%
\pgfpathmoveto{\pgfqpoint{6.045962in}{3.749925in}}%
\pgfpathlineto{\pgfqpoint{5.976195in}{3.879315in}}%
\pgfpathlineto{\pgfqpoint{6.112176in}{3.859511in}}%
\pgfpathclose%
\pgfusepath{fill}%
\end{pgfscope}%
\begin{pgfscope}%
\pgfpathrectangle{\pgfqpoint{0.680860in}{0.078740in}}{\pgfqpoint{7.842520in}{7.842520in}}%
\pgfusepath{clip}%
\pgfsetbuttcap%
\pgfsetroundjoin%
\definecolor{currentfill}{rgb}{0.206756,0.371758,0.553117}%
\pgfsetfillcolor{currentfill}%
\pgfsetlinewidth{0.000000pt}%
\definecolor{currentstroke}{rgb}{0.335885,0.777018,0.402049}%
\pgfsetstrokecolor{currentstroke}%
\pgfsetdash{}{0pt}%
\pgfpathmoveto{\pgfqpoint{3.433422in}{4.469040in}}%
\pgfpathlineto{\pgfqpoint{3.515094in}{4.458315in}}%
\pgfpathlineto{\pgfqpoint{3.384561in}{4.465080in}}%
\pgfpathclose%
\pgfusepath{fill}%
\end{pgfscope}%
\begin{pgfscope}%
\pgfpathrectangle{\pgfqpoint{0.680860in}{0.078740in}}{\pgfqpoint{7.842520in}{7.842520in}}%
\pgfusepath{clip}%
\pgfsetbuttcap%
\pgfsetroundjoin%
\definecolor{currentfill}{rgb}{0.216210,0.351535,0.550627}%
\pgfsetfillcolor{currentfill}%
\pgfsetlinewidth{0.000000pt}%
\definecolor{currentstroke}{rgb}{0.344074,0.780029,0.397381}%
\pgfsetstrokecolor{currentstroke}%
\pgfsetdash{}{0pt}%
\pgfpathmoveto{\pgfqpoint{4.122277in}{4.409244in}}%
\pgfpathlineto{\pgfqpoint{4.201601in}{4.367504in}}%
\pgfpathlineto{\pgfqpoint{4.069699in}{4.376646in}}%
\pgfpathclose%
\pgfusepath{fill}%
\end{pgfscope}%
\begin{pgfscope}%
\pgfpathrectangle{\pgfqpoint{0.680860in}{0.078740in}}{\pgfqpoint{7.842520in}{7.842520in}}%
\pgfusepath{clip}%
\pgfsetbuttcap%
\pgfsetroundjoin%
\definecolor{currentfill}{rgb}{0.282327,0.094955,0.417331}%
\pgfsetfillcolor{currentfill}%
\pgfsetlinewidth{0.000000pt}%
\definecolor{currentstroke}{rgb}{0.352360,0.783011,0.392636}%
\pgfsetstrokecolor{currentstroke}%
\pgfsetdash{}{0pt}%
\pgfpathmoveto{\pgfqpoint{6.727067in}{3.622390in}}%
\pgfpathlineto{\pgfqpoint{6.791930in}{3.456333in}}%
\pgfpathlineto{\pgfqpoint{6.655614in}{3.488467in}}%
\pgfpathclose%
\pgfusepath{fill}%
\end{pgfscope}%
\begin{pgfscope}%
\pgfpathrectangle{\pgfqpoint{0.680860in}{0.078740in}}{\pgfqpoint{7.842520in}{7.842520in}}%
\pgfusepath{clip}%
\pgfsetbuttcap%
\pgfsetroundjoin%
\definecolor{currentfill}{rgb}{0.235526,0.309527,0.542944}%
\pgfsetfillcolor{currentfill}%
\pgfsetlinewidth{0.000000pt}%
\definecolor{currentstroke}{rgb}{0.360741,0.785964,0.387814}%
\pgfsetstrokecolor{currentstroke}%
\pgfsetdash{}{0pt}%
\pgfpathmoveto{\pgfqpoint{5.021359in}{4.185999in}}%
\pgfpathlineto{\pgfqpoint{4.887762in}{4.199297in}}%
\pgfpathlineto{\pgfqpoint{4.811454in}{4.273843in}}%
\pgfpathclose%
\pgfusepath{fill}%
\end{pgfscope}%
\begin{pgfscope}%
\pgfpathrectangle{\pgfqpoint{0.680860in}{0.078740in}}{\pgfqpoint{7.842520in}{7.842520in}}%
\pgfusepath{clip}%
\pgfsetbuttcap%
\pgfsetroundjoin%
\definecolor{currentfill}{rgb}{0.220057,0.343307,0.549413}%
\pgfsetfillcolor{currentfill}%
\pgfsetlinewidth{0.000000pt}%
\definecolor{currentstroke}{rgb}{0.369214,0.788888,0.382914}%
\pgfsetstrokecolor{currentstroke}%
\pgfsetdash{}{0pt}%
\pgfpathmoveto{\pgfqpoint{4.466954in}{4.351887in}}%
\pgfpathlineto{\pgfqpoint{4.412335in}{4.305112in}}%
\pgfpathlineto{\pgfqpoint{4.334017in}{4.359240in}}%
\pgfpathclose%
\pgfusepath{fill}%
\end{pgfscope}%
\begin{pgfscope}%
\pgfpathrectangle{\pgfqpoint{0.680860in}{0.078740in}}{\pgfqpoint{7.842520in}{7.842520in}}%
\pgfusepath{clip}%
\pgfsetbuttcap%
\pgfsetroundjoin%
\definecolor{currentfill}{rgb}{0.265145,0.232956,0.516599}%
\pgfsetfillcolor{currentfill}%
\pgfsetlinewidth{0.000000pt}%
\definecolor{currentstroke}{rgb}{0.377779,0.791781,0.377939}%
\pgfsetstrokecolor{currentstroke}%
\pgfsetdash{}{0pt}%
\pgfpathmoveto{\pgfqpoint{5.705747in}{3.920439in}}%
\pgfpathlineto{\pgfqpoint{5.769414in}{4.018548in}}%
\pgfpathlineto{\pgfqpoint{5.840720in}{3.899631in}}%
\pgfpathclose%
\pgfusepath{fill}%
\end{pgfscope}%
\begin{pgfscope}%
\pgfpathrectangle{\pgfqpoint{0.680860in}{0.078740in}}{\pgfqpoint{7.842520in}{7.842520in}}%
\pgfusepath{clip}%
\pgfsetbuttcap%
\pgfsetroundjoin%
\definecolor{currentfill}{rgb}{0.203063,0.379716,0.553925}%
\pgfsetfillcolor{currentfill}%
\pgfsetlinewidth{0.000000pt}%
\definecolor{currentstroke}{rgb}{0.386433,0.794644,0.372886}%
\pgfsetstrokecolor{currentstroke}%
\pgfsetdash{}{0pt}%
\pgfpathmoveto{\pgfqpoint{3.384561in}{4.465080in}}%
\pgfpathlineto{\pgfqpoint{3.302607in}{4.473451in}}%
\pgfpathlineto{\pgfqpoint{3.433422in}{4.469040in}}%
\pgfpathclose%
\pgfusepath{fill}%
\end{pgfscope}%
\begin{pgfscope}%
\pgfpathrectangle{\pgfqpoint{0.680860in}{0.078740in}}{\pgfqpoint{7.842520in}{7.842520in}}%
\pgfusepath{clip}%
\pgfsetbuttcap%
\pgfsetroundjoin%
\definecolor{currentfill}{rgb}{0.212395,0.359683,0.551710}%
\pgfsetfillcolor{currentfill}%
\pgfsetlinewidth{0.000000pt}%
\definecolor{currentstroke}{rgb}{0.395174,0.797475,0.367757}%
\pgfsetstrokecolor{currentstroke}%
\pgfsetdash{}{0pt}%
\pgfpathmoveto{\pgfqpoint{4.069699in}{4.376646in}}%
\pgfpathlineto{\pgfqpoint{3.990011in}{4.415024in}}%
\pgfpathlineto{\pgfqpoint{4.122277in}{4.409244in}}%
\pgfpathclose%
\pgfusepath{fill}%
\end{pgfscope}%
\begin{pgfscope}%
\pgfpathrectangle{\pgfqpoint{0.680860in}{0.078740in}}{\pgfqpoint{7.842520in}{7.842520in}}%
\pgfusepath{clip}%
\pgfsetbuttcap%
\pgfsetroundjoin%
\definecolor{currentfill}{rgb}{0.221989,0.339161,0.548752}%
\pgfsetfillcolor{currentfill}%
\pgfsetlinewidth{0.000000pt}%
\definecolor{currentstroke}{rgb}{0.404001,0.800275,0.362552}%
\pgfsetstrokecolor{currentstroke}%
\pgfsetdash{}{0pt}%
\pgfpathmoveto{\pgfqpoint{4.677895in}{4.283438in}}%
\pgfpathlineto{\pgfqpoint{4.544859in}{4.293872in}}%
\pgfpathlineto{\pgfqpoint{4.466954in}{4.351887in}}%
\pgfpathclose%
\pgfusepath{fill}%
\end{pgfscope}%
\begin{pgfscope}%
\pgfpathrectangle{\pgfqpoint{0.680860in}{0.078740in}}{\pgfqpoint{7.842520in}{7.842520in}}%
\pgfusepath{clip}%
\pgfsetbuttcap%
\pgfsetroundjoin%
\definecolor{currentfill}{rgb}{0.201239,0.383670,0.554294}%
\pgfsetfillcolor{currentfill}%
\pgfsetlinewidth{0.000000pt}%
\definecolor{currentstroke}{rgb}{0.412913,0.803041,0.357269}%
\pgfsetstrokecolor{currentstroke}%
\pgfsetdash{}{0pt}%
\pgfpathmoveto{\pgfqpoint{3.089677in}{4.475160in}}%
\pgfpathlineto{\pgfqpoint{3.302607in}{4.473451in}}%
\pgfpathlineto{\pgfqpoint{3.172299in}{4.478866in}}%
\pgfpathclose%
\pgfusepath{fill}%
\end{pgfscope}%
\begin{pgfscope}%
\pgfpathrectangle{\pgfqpoint{0.680860in}{0.078740in}}{\pgfqpoint{7.842520in}{7.842520in}}%
\pgfusepath{clip}%
\pgfsetbuttcap%
\pgfsetroundjoin%
\definecolor{currentfill}{rgb}{0.282656,0.100196,0.422160}%
\pgfsetfillcolor{currentfill}%
\pgfsetlinewidth{0.000000pt}%
\definecolor{currentstroke}{rgb}{0.421908,0.805774,0.351910}%
\pgfsetstrokecolor{currentstroke}%
\pgfsetdash{}{0pt}%
\pgfpathmoveto{\pgfqpoint{6.928708in}{3.424320in}}%
\pgfpathlineto{\pgfqpoint{6.791930in}{3.456333in}}%
\pgfpathlineto{\pgfqpoint{6.727067in}{3.622390in}}%
\pgfpathclose%
\pgfusepath{fill}%
\end{pgfscope}%
\begin{pgfscope}%
\pgfpathrectangle{\pgfqpoint{0.680860in}{0.078740in}}{\pgfqpoint{7.842520in}{7.842520in}}%
\pgfusepath{clip}%
\pgfsetbuttcap%
\pgfsetroundjoin%
\definecolor{currentfill}{rgb}{0.212395,0.359683,0.551710}%
\pgfsetfillcolor{currentfill}%
\pgfsetlinewidth{0.000000pt}%
\definecolor{currentstroke}{rgb}{0.430983,0.808473,0.346476}%
\pgfsetstrokecolor{currentstroke}%
\pgfsetdash{}{0pt}%
\pgfpathmoveto{\pgfqpoint{4.334017in}{4.359240in}}%
\pgfpathlineto{\pgfqpoint{4.201601in}{4.367504in}}%
\pgfpathlineto{\pgfqpoint{4.122277in}{4.409244in}}%
\pgfpathclose%
\pgfusepath{fill}%
\end{pgfscope}%
\begin{pgfscope}%
\pgfpathrectangle{\pgfqpoint{0.680860in}{0.078740in}}{\pgfqpoint{7.842520in}{7.842520in}}%
\pgfusepath{clip}%
\pgfsetbuttcap%
\pgfsetroundjoin%
\definecolor{currentfill}{rgb}{0.283187,0.125848,0.444960}%
\pgfsetfillcolor{currentfill}%
\pgfsetlinewidth{0.000000pt}%
\definecolor{currentstroke}{rgb}{0.440137,0.811138,0.340967}%
\pgfsetstrokecolor{currentstroke}%
\pgfsetdash{}{0pt}%
\pgfpathmoveto{\pgfqpoint{6.655614in}{3.488467in}}%
\pgfpathlineto{\pgfqpoint{6.589863in}{3.647214in}}%
\pgfpathlineto{\pgfqpoint{6.727067in}{3.622390in}}%
\pgfpathclose%
\pgfusepath{fill}%
\end{pgfscope}%
\begin{pgfscope}%
\pgfpathrectangle{\pgfqpoint{0.680860in}{0.078740in}}{\pgfqpoint{7.842520in}{7.842520in}}%
\pgfusepath{clip}%
\pgfsetbuttcap%
\pgfsetroundjoin%
\definecolor{currentfill}{rgb}{0.204903,0.375746,0.553533}%
\pgfsetfillcolor{currentfill}%
\pgfsetlinewidth{0.000000pt}%
\definecolor{currentstroke}{rgb}{0.449368,0.813768,0.335384}%
\pgfsetstrokecolor{currentstroke}%
\pgfsetdash{}{0pt}%
\pgfpathmoveto{\pgfqpoint{3.777688in}{4.447685in}}%
\pgfpathlineto{\pgfqpoint{3.858265in}{4.421765in}}%
\pgfpathlineto{\pgfqpoint{3.646134in}{4.452504in}}%
\pgfpathclose%
\pgfusepath{fill}%
\end{pgfscope}%
\begin{pgfscope}%
\pgfpathrectangle{\pgfqpoint{0.680860in}{0.078740in}}{\pgfqpoint{7.842520in}{7.842520in}}%
\pgfusepath{clip}%
\pgfsetbuttcap%
\pgfsetroundjoin%
\definecolor{currentfill}{rgb}{0.206756,0.371758,0.553117}%
\pgfsetfillcolor{currentfill}%
\pgfsetlinewidth{0.000000pt}%
\definecolor{currentstroke}{rgb}{0.458674,0.816363,0.329727}%
\pgfsetstrokecolor{currentstroke}%
\pgfsetdash{}{0pt}%
\pgfpathmoveto{\pgfqpoint{3.990011in}{4.415024in}}%
\pgfpathlineto{\pgfqpoint{3.858265in}{4.421765in}}%
\pgfpathlineto{\pgfqpoint{3.777688in}{4.447685in}}%
\pgfpathclose%
\pgfusepath{fill}%
\end{pgfscope}%
\begin{pgfscope}%
\pgfpathrectangle{\pgfqpoint{0.680860in}{0.078740in}}{\pgfqpoint{7.842520in}{7.842520in}}%
\pgfusepath{clip}%
\pgfsetbuttcap%
\pgfsetroundjoin%
\definecolor{currentfill}{rgb}{0.241237,0.296485,0.539709}%
\pgfsetfillcolor{currentfill}%
\pgfsetlinewidth{0.000000pt}%
\definecolor{currentstroke}{rgb}{0.468053,0.818921,0.323998}%
\pgfsetstrokecolor{currentstroke}%
\pgfsetdash{}{0pt}%
\pgfpathmoveto{\pgfqpoint{5.155473in}{4.173442in}}%
\pgfpathlineto{\pgfqpoint{5.290111in}{4.161656in}}%
\pgfpathlineto{\pgfqpoint{5.364091in}{4.065435in}}%
\pgfpathclose%
\pgfusepath{fill}%
\end{pgfscope}%
\begin{pgfscope}%
\pgfpathrectangle{\pgfqpoint{0.680860in}{0.078740in}}{\pgfqpoint{7.842520in}{7.842520in}}%
\pgfusepath{clip}%
\pgfsetbuttcap%
\pgfsetroundjoin%
\definecolor{currentfill}{rgb}{0.257322,0.256130,0.526563}%
\pgfsetfillcolor{currentfill}%
\pgfsetlinewidth{0.000000pt}%
\definecolor{currentstroke}{rgb}{0.477504,0.821444,0.318195}%
\pgfsetstrokecolor{currentstroke}%
\pgfsetdash{}{0pt}%
\pgfpathmoveto{\pgfqpoint{5.705747in}{3.920439in}}%
\pgfpathlineto{\pgfqpoint{5.633787in}{4.033516in}}%
\pgfpathlineto{\pgfqpoint{5.769414in}{4.018548in}}%
\pgfpathclose%
\pgfusepath{fill}%
\end{pgfscope}%
\begin{pgfscope}%
\pgfpathrectangle{\pgfqpoint{0.680860in}{0.078740in}}{\pgfqpoint{7.842520in}{7.842520in}}%
\pgfusepath{clip}%
\pgfsetbuttcap%
\pgfsetroundjoin%
\definecolor{currentfill}{rgb}{0.278012,0.180367,0.486697}%
\pgfsetfillcolor{currentfill}%
\pgfsetlinewidth{0.000000pt}%
\definecolor{currentstroke}{rgb}{0.487026,0.823929,0.312321}%
\pgfsetstrokecolor{currentstroke}%
\pgfsetdash{}{0pt}%
\pgfpathmoveto{\pgfqpoint{6.453156in}{3.672394in}}%
\pgfpathlineto{\pgfqpoint{6.316940in}{3.697915in}}%
\pgfpathlineto{\pgfqpoint{6.248671in}{3.840241in}}%
\pgfpathclose%
\pgfusepath{fill}%
\end{pgfscope}%
\begin{pgfscope}%
\pgfpathrectangle{\pgfqpoint{0.680860in}{0.078740in}}{\pgfqpoint{7.842520in}{7.842520in}}%
\pgfusepath{clip}%
\pgfsetbuttcap%
\pgfsetroundjoin%
\definecolor{currentfill}{rgb}{0.263663,0.237631,0.518762}%
\pgfsetfillcolor{currentfill}%
\pgfsetlinewidth{0.000000pt}%
\definecolor{currentstroke}{rgb}{0.496615,0.826376,0.306377}%
\pgfsetstrokecolor{currentstroke}%
\pgfsetdash{}{0pt}%
\pgfpathmoveto{\pgfqpoint{5.769414in}{4.018548in}}%
\pgfpathlineto{\pgfqpoint{5.976195in}{3.879315in}}%
\pgfpathlineto{\pgfqpoint{5.840720in}{3.899631in}}%
\pgfpathclose%
\pgfusepath{fill}%
\end{pgfscope}%
\begin{pgfscope}%
\pgfpathrectangle{\pgfqpoint{0.680860in}{0.078740in}}{\pgfqpoint{7.842520in}{7.842520in}}%
\pgfusepath{clip}%
\pgfsetbuttcap%
\pgfsetroundjoin%
\definecolor{currentfill}{rgb}{0.246811,0.283237,0.535941}%
\pgfsetfillcolor{currentfill}%
\pgfsetlinewidth{0.000000pt}%
\definecolor{currentstroke}{rgb}{0.506271,0.828786,0.300362}%
\pgfsetstrokecolor{currentstroke}%
\pgfsetdash{}{0pt}%
\pgfpathmoveto{\pgfqpoint{5.364091in}{4.065435in}}%
\pgfpathlineto{\pgfqpoint{5.425281in}{4.150671in}}%
\pgfpathlineto{\pgfqpoint{5.498681in}{4.049153in}}%
\pgfpathclose%
\pgfusepath{fill}%
\end{pgfscope}%
\begin{pgfscope}%
\pgfpathrectangle{\pgfqpoint{0.680860in}{0.078740in}}{\pgfqpoint{7.842520in}{7.842520in}}%
\pgfusepath{clip}%
\pgfsetbuttcap%
\pgfsetroundjoin%
\definecolor{currentfill}{rgb}{0.201239,0.383670,0.554294}%
\pgfsetfillcolor{currentfill}%
\pgfsetlinewidth{0.000000pt}%
\definecolor{currentstroke}{rgb}{0.515992,0.831158,0.294279}%
\pgfsetstrokecolor{currentstroke}%
\pgfsetdash{}{0pt}%
\pgfpathmoveto{\pgfqpoint{3.564754in}{4.465670in}}%
\pgfpathlineto{\pgfqpoint{3.646134in}{4.452504in}}%
\pgfpathlineto{\pgfqpoint{3.515094in}{4.458315in}}%
\pgfpathclose%
\pgfusepath{fill}%
\end{pgfscope}%
\begin{pgfscope}%
\pgfpathrectangle{\pgfqpoint{0.680860in}{0.078740in}}{\pgfqpoint{7.842520in}{7.842520in}}%
\pgfusepath{clip}%
\pgfsetbuttcap%
\pgfsetroundjoin%
\definecolor{currentfill}{rgb}{0.231674,0.318106,0.544834}%
\pgfsetfillcolor{currentfill}%
\pgfsetlinewidth{0.000000pt}%
\definecolor{currentstroke}{rgb}{0.525776,0.833491,0.288127}%
\pgfsetstrokecolor{currentstroke}%
\pgfsetdash{}{0pt}%
\pgfpathmoveto{\pgfqpoint{4.945541in}{4.265119in}}%
\pgfpathlineto{\pgfqpoint{5.155473in}{4.173442in}}%
\pgfpathlineto{\pgfqpoint{5.021359in}{4.185999in}}%
\pgfpathclose%
\pgfusepath{fill}%
\end{pgfscope}%
\begin{pgfscope}%
\pgfpathrectangle{\pgfqpoint{0.680860in}{0.078740in}}{\pgfqpoint{7.842520in}{7.842520in}}%
\pgfusepath{clip}%
\pgfsetbuttcap%
\pgfsetroundjoin%
\definecolor{currentfill}{rgb}{0.273006,0.204520,0.501721}%
\pgfsetfillcolor{currentfill}%
\pgfsetlinewidth{0.000000pt}%
\definecolor{currentstroke}{rgb}{0.535621,0.835785,0.281908}%
\pgfsetstrokecolor{currentstroke}%
\pgfsetdash{}{0pt}%
\pgfpathmoveto{\pgfqpoint{6.248671in}{3.840241in}}%
\pgfpathlineto{\pgfqpoint{6.316940in}{3.697915in}}%
\pgfpathlineto{\pgfqpoint{6.112176in}{3.859511in}}%
\pgfpathclose%
\pgfusepath{fill}%
\end{pgfscope}%
\begin{pgfscope}%
\pgfpathrectangle{\pgfqpoint{0.680860in}{0.078740in}}{\pgfqpoint{7.842520in}{7.842520in}}%
\pgfusepath{clip}%
\pgfsetbuttcap%
\pgfsetroundjoin%
\definecolor{currentfill}{rgb}{0.225863,0.330805,0.547314}%
\pgfsetfillcolor{currentfill}%
\pgfsetlinewidth{0.000000pt}%
\definecolor{currentstroke}{rgb}{0.545524,0.838039,0.275626}%
\pgfsetstrokecolor{currentstroke}%
\pgfsetdash{}{0pt}%
\pgfpathmoveto{\pgfqpoint{4.811454in}{4.273843in}}%
\pgfpathlineto{\pgfqpoint{4.945541in}{4.265119in}}%
\pgfpathlineto{\pgfqpoint{5.021359in}{4.185999in}}%
\pgfpathclose%
\pgfusepath{fill}%
\end{pgfscope}%
\begin{pgfscope}%
\pgfpathrectangle{\pgfqpoint{0.680860in}{0.078740in}}{\pgfqpoint{7.842520in}{7.842520in}}%
\pgfusepath{clip}%
\pgfsetbuttcap%
\pgfsetroundjoin%
\definecolor{currentfill}{rgb}{0.199430,0.387607,0.554642}%
\pgfsetfillcolor{currentfill}%
\pgfsetlinewidth{0.000000pt}%
\definecolor{currentstroke}{rgb}{0.555484,0.840254,0.269281}%
\pgfsetstrokecolor{currentstroke}%
\pgfsetdash{}{0pt}%
\pgfpathmoveto{\pgfqpoint{3.564754in}{4.465670in}}%
\pgfpathlineto{\pgfqpoint{3.515094in}{4.458315in}}%
\pgfpathlineto{\pgfqpoint{3.433422in}{4.469040in}}%
\pgfpathclose%
\pgfusepath{fill}%
\end{pgfscope}%
\begin{pgfscope}%
\pgfpathrectangle{\pgfqpoint{0.680860in}{0.078740in}}{\pgfqpoint{7.842520in}{7.842520in}}%
\pgfusepath{clip}%
\pgfsetbuttcap%
\pgfsetroundjoin%
\definecolor{currentfill}{rgb}{0.197636,0.391528,0.554969}%
\pgfsetfillcolor{currentfill}%
\pgfsetlinewidth{0.000000pt}%
\definecolor{currentstroke}{rgb}{0.565498,0.842430,0.262877}%
\pgfsetstrokecolor{currentstroke}%
\pgfsetdash{}{0pt}%
\pgfpathmoveto{\pgfqpoint{3.089677in}{4.475160in}}%
\pgfpathlineto{\pgfqpoint{3.220244in}{4.471698in}}%
\pgfpathlineto{\pgfqpoint{3.302607in}{4.473451in}}%
\pgfpathclose%
\pgfusepath{fill}%
\end{pgfscope}%
\begin{pgfscope}%
\pgfpathrectangle{\pgfqpoint{0.680860in}{0.078740in}}{\pgfqpoint{7.842520in}{7.842520in}}%
\pgfusepath{clip}%
\pgfsetbuttcap%
\pgfsetroundjoin%
\definecolor{currentfill}{rgb}{0.282656,0.100196,0.422160}%
\pgfsetfillcolor{currentfill}%
\pgfsetlinewidth{0.000000pt}%
\definecolor{currentstroke}{rgb}{0.575563,0.844566,0.256415}%
\pgfsetstrokecolor{currentstroke}%
\pgfsetdash{}{0pt}%
\pgfpathmoveto{\pgfqpoint{7.065952in}{3.392437in}}%
\pgfpathlineto{\pgfqpoint{6.928708in}{3.424320in}}%
\pgfpathlineto{\pgfqpoint{6.864774in}{3.597939in}}%
\pgfpathclose%
\pgfusepath{fill}%
\end{pgfscope}%
\begin{pgfscope}%
\pgfpathrectangle{\pgfqpoint{0.680860in}{0.078740in}}{\pgfqpoint{7.842520in}{7.842520in}}%
\pgfusepath{clip}%
\pgfsetbuttcap%
\pgfsetroundjoin%
\definecolor{currentfill}{rgb}{0.239346,0.300855,0.540844}%
\pgfsetfillcolor{currentfill}%
\pgfsetlinewidth{0.000000pt}%
\definecolor{currentstroke}{rgb}{0.585678,0.846661,0.249897}%
\pgfsetstrokecolor{currentstroke}%
\pgfsetdash{}{0pt}%
\pgfpathmoveto{\pgfqpoint{5.364091in}{4.065435in}}%
\pgfpathlineto{\pgfqpoint{5.290111in}{4.161656in}}%
\pgfpathlineto{\pgfqpoint{5.425281in}{4.150671in}}%
\pgfpathclose%
\pgfusepath{fill}%
\end{pgfscope}%
\begin{pgfscope}%
\pgfpathrectangle{\pgfqpoint{0.680860in}{0.078740in}}{\pgfqpoint{7.842520in}{7.842520in}}%
\pgfusepath{clip}%
\pgfsetbuttcap%
\pgfsetroundjoin%
\definecolor{currentfill}{rgb}{0.218130,0.347432,0.550038}%
\pgfsetfillcolor{currentfill}%
\pgfsetlinewidth{0.000000pt}%
\definecolor{currentstroke}{rgb}{0.595839,0.848717,0.243329}%
\pgfsetstrokecolor{currentstroke}%
\pgfsetdash{}{0pt}%
\pgfpathmoveto{\pgfqpoint{4.677895in}{4.283438in}}%
\pgfpathlineto{\pgfqpoint{4.600422in}{4.345480in}}%
\pgfpathlineto{\pgfqpoint{4.811454in}{4.273843in}}%
\pgfpathclose%
\pgfusepath{fill}%
\end{pgfscope}%
\begin{pgfscope}%
\pgfpathrectangle{\pgfqpoint{0.680860in}{0.078740in}}{\pgfqpoint{7.842520in}{7.842520in}}%
\pgfusepath{clip}%
\pgfsetbuttcap%
\pgfsetroundjoin%
\definecolor{currentfill}{rgb}{0.197636,0.391528,0.554969}%
\pgfsetfillcolor{currentfill}%
\pgfsetlinewidth{0.000000pt}%
\definecolor{currentstroke}{rgb}{0.606045,0.850733,0.236712}%
\pgfsetstrokecolor{currentstroke}%
\pgfsetdash{}{0pt}%
\pgfpathmoveto{\pgfqpoint{3.433422in}{4.469040in}}%
\pgfpathlineto{\pgfqpoint{3.302607in}{4.473451in}}%
\pgfpathlineto{\pgfqpoint{3.220244in}{4.471698in}}%
\pgfpathclose%
\pgfusepath{fill}%
\end{pgfscope}%
\begin{pgfscope}%
\pgfpathrectangle{\pgfqpoint{0.680860in}{0.078740in}}{\pgfqpoint{7.842520in}{7.842520in}}%
\pgfusepath{clip}%
\pgfsetbuttcap%
\pgfsetroundjoin%
\definecolor{currentfill}{rgb}{0.214298,0.355619,0.551184}%
\pgfsetfillcolor{currentfill}%
\pgfsetlinewidth{0.000000pt}%
\definecolor{currentstroke}{rgb}{0.616293,0.852709,0.230052}%
\pgfsetstrokecolor{currentstroke}%
\pgfsetdash{}{0pt}%
\pgfpathmoveto{\pgfqpoint{4.466954in}{4.351887in}}%
\pgfpathlineto{\pgfqpoint{4.600422in}{4.345480in}}%
\pgfpathlineto{\pgfqpoint{4.677895in}{4.283438in}}%
\pgfpathclose%
\pgfusepath{fill}%
\end{pgfscope}%
\begin{pgfscope}%
\pgfpathrectangle{\pgfqpoint{0.680860in}{0.078740in}}{\pgfqpoint{7.842520in}{7.842520in}}%
\pgfusepath{clip}%
\pgfsetbuttcap%
\pgfsetroundjoin%
\definecolor{currentfill}{rgb}{0.243113,0.292092,0.538516}%
\pgfsetfillcolor{currentfill}%
\pgfsetlinewidth{0.000000pt}%
\definecolor{currentstroke}{rgb}{0.626579,0.854645,0.223353}%
\pgfsetstrokecolor{currentstroke}%
\pgfsetdash{}{0pt}%
\pgfpathmoveto{\pgfqpoint{5.425281in}{4.150671in}}%
\pgfpathlineto{\pgfqpoint{5.633787in}{4.033516in}}%
\pgfpathlineto{\pgfqpoint{5.498681in}{4.049153in}}%
\pgfpathclose%
\pgfusepath{fill}%
\end{pgfscope}%
\begin{pgfscope}%
\pgfpathrectangle{\pgfqpoint{0.680860in}{0.078740in}}{\pgfqpoint{7.842520in}{7.842520in}}%
\pgfusepath{clip}%
\pgfsetbuttcap%
\pgfsetroundjoin%
\definecolor{currentfill}{rgb}{0.206756,0.371758,0.553117}%
\pgfsetfillcolor{currentfill}%
\pgfsetlinewidth{0.000000pt}%
\definecolor{currentstroke}{rgb}{0.636902,0.856542,0.216620}%
\pgfsetstrokecolor{currentstroke}%
\pgfsetdash{}{0pt}%
\pgfpathmoveto{\pgfqpoint{4.334017in}{4.359240in}}%
\pgfpathlineto{\pgfqpoint{4.255070in}{4.404464in}}%
\pgfpathlineto{\pgfqpoint{4.466954in}{4.351887in}}%
\pgfpathclose%
\pgfusepath{fill}%
\end{pgfscope}%
\begin{pgfscope}%
\pgfpathrectangle{\pgfqpoint{0.680860in}{0.078740in}}{\pgfqpoint{7.842520in}{7.842520in}}%
\pgfusepath{clip}%
\pgfsetbuttcap%
\pgfsetroundjoin%
\definecolor{currentfill}{rgb}{0.197636,0.391528,0.554969}%
\pgfsetfillcolor{currentfill}%
\pgfsetlinewidth{0.000000pt}%
\definecolor{currentstroke}{rgb}{0.647257,0.858400,0.209861}%
\pgfsetstrokecolor{currentstroke}%
\pgfsetdash{}{0pt}%
\pgfpathmoveto{\pgfqpoint{3.777688in}{4.447685in}}%
\pgfpathlineto{\pgfqpoint{3.646134in}{4.452504in}}%
\pgfpathlineto{\pgfqpoint{3.564754in}{4.465670in}}%
\pgfpathclose%
\pgfusepath{fill}%
\end{pgfscope}%
\begin{pgfscope}%
\pgfpathrectangle{\pgfqpoint{0.680860in}{0.078740in}}{\pgfqpoint{7.842520in}{7.842520in}}%
\pgfusepath{clip}%
\pgfsetbuttcap%
\pgfsetroundjoin%
\definecolor{currentfill}{rgb}{0.204903,0.375746,0.553533}%
\pgfsetfillcolor{currentfill}%
\pgfsetlinewidth{0.000000pt}%
\definecolor{currentstroke}{rgb}{0.657642,0.860219,0.203082}%
\pgfsetstrokecolor{currentstroke}%
\pgfsetdash{}{0pt}%
\pgfpathmoveto{\pgfqpoint{4.122277in}{4.409244in}}%
\pgfpathlineto{\pgfqpoint{4.255070in}{4.404464in}}%
\pgfpathlineto{\pgfqpoint{4.334017in}{4.359240in}}%
\pgfpathclose%
\pgfusepath{fill}%
\end{pgfscope}%
\begin{pgfscope}%
\pgfpathrectangle{\pgfqpoint{0.680860in}{0.078740in}}{\pgfqpoint{7.842520in}{7.842520in}}%
\pgfusepath{clip}%
\pgfsetbuttcap%
\pgfsetroundjoin%
\definecolor{currentfill}{rgb}{0.199430,0.387607,0.554642}%
\pgfsetfillcolor{currentfill}%
\pgfsetlinewidth{0.000000pt}%
\definecolor{currentstroke}{rgb}{0.668054,0.861999,0.196293}%
\pgfsetstrokecolor{currentstroke}%
\pgfsetdash{}{0pt}%
\pgfpathmoveto{\pgfqpoint{3.777688in}{4.447685in}}%
\pgfpathlineto{\pgfqpoint{3.909766in}{4.443897in}}%
\pgfpathlineto{\pgfqpoint{3.990011in}{4.415024in}}%
\pgfpathclose%
\pgfusepath{fill}%
\end{pgfscope}%
\begin{pgfscope}%
\pgfpathrectangle{\pgfqpoint{0.680860in}{0.078740in}}{\pgfqpoint{7.842520in}{7.842520in}}%
\pgfusepath{clip}%
\pgfsetbuttcap%
\pgfsetroundjoin%
\definecolor{currentfill}{rgb}{0.201239,0.383670,0.554294}%
\pgfsetfillcolor{currentfill}%
\pgfsetlinewidth{0.000000pt}%
\definecolor{currentstroke}{rgb}{0.678489,0.863742,0.189503}%
\pgfsetstrokecolor{currentstroke}%
\pgfsetdash{}{0pt}%
\pgfpathmoveto{\pgfqpoint{4.122277in}{4.409244in}}%
\pgfpathlineto{\pgfqpoint{3.990011in}{4.415024in}}%
\pgfpathlineto{\pgfqpoint{3.909766in}{4.443897in}}%
\pgfpathclose%
\pgfusepath{fill}%
\end{pgfscope}%
\begin{pgfscope}%
\pgfpathrectangle{\pgfqpoint{0.680860in}{0.078740in}}{\pgfqpoint{7.842520in}{7.842520in}}%
\pgfusepath{clip}%
\pgfsetbuttcap%
\pgfsetroundjoin%
\definecolor{currentfill}{rgb}{0.260571,0.246922,0.522828}%
\pgfsetfillcolor{currentfill}%
\pgfsetlinewidth{0.000000pt}%
\definecolor{currentstroke}{rgb}{0.688944,0.865448,0.182725}%
\pgfsetstrokecolor{currentstroke}%
\pgfsetdash{}{0pt}%
\pgfpathmoveto{\pgfqpoint{6.112176in}{3.859511in}}%
\pgfpathlineto{\pgfqpoint{5.976195in}{3.879315in}}%
\pgfpathlineto{\pgfqpoint{5.905571in}{4.004277in}}%
\pgfpathclose%
\pgfusepath{fill}%
\end{pgfscope}%
\begin{pgfscope}%
\pgfpathrectangle{\pgfqpoint{0.680860in}{0.078740in}}{\pgfqpoint{7.842520in}{7.842520in}}%
\pgfusepath{clip}%
\pgfsetbuttcap%
\pgfsetroundjoin%
\definecolor{currentfill}{rgb}{0.282884,0.135920,0.453427}%
\pgfsetfillcolor{currentfill}%
\pgfsetlinewidth{0.000000pt}%
\definecolor{currentstroke}{rgb}{0.699415,0.867117,0.175971}%
\pgfsetstrokecolor{currentstroke}%
\pgfsetdash{}{0pt}%
\pgfpathmoveto{\pgfqpoint{6.864774in}{3.597939in}}%
\pgfpathlineto{\pgfqpoint{6.928708in}{3.424320in}}%
\pgfpathlineto{\pgfqpoint{6.727067in}{3.622390in}}%
\pgfpathclose%
\pgfusepath{fill}%
\end{pgfscope}%
\begin{pgfscope}%
\pgfpathrectangle{\pgfqpoint{0.680860in}{0.078740in}}{\pgfqpoint{7.842520in}{7.842520in}}%
\pgfusepath{clip}%
\pgfsetbuttcap%
\pgfsetroundjoin%
\definecolor{currentfill}{rgb}{0.253935,0.265254,0.529983}%
\pgfsetfillcolor{currentfill}%
\pgfsetlinewidth{0.000000pt}%
\definecolor{currentstroke}{rgb}{0.709898,0.868751,0.169257}%
\pgfsetstrokecolor{currentstroke}%
\pgfsetdash{}{0pt}%
\pgfpathmoveto{\pgfqpoint{5.905571in}{4.004277in}}%
\pgfpathlineto{\pgfqpoint{5.976195in}{3.879315in}}%
\pgfpathlineto{\pgfqpoint{5.769414in}{4.018548in}}%
\pgfpathclose%
\pgfusepath{fill}%
\end{pgfscope}%
\begin{pgfscope}%
\pgfpathrectangle{\pgfqpoint{0.680860in}{0.078740in}}{\pgfqpoint{7.842520in}{7.842520in}}%
\pgfusepath{clip}%
\pgfsetbuttcap%
\pgfsetroundjoin%
\definecolor{currentfill}{rgb}{0.277134,0.185228,0.489898}%
\pgfsetfillcolor{currentfill}%
\pgfsetlinewidth{0.000000pt}%
\definecolor{currentstroke}{rgb}{0.720391,0.870350,0.162603}%
\pgfsetstrokecolor{currentstroke}%
\pgfsetdash{}{0pt}%
\pgfpathmoveto{\pgfqpoint{6.523228in}{3.803386in}}%
\pgfpathlineto{\pgfqpoint{6.589863in}{3.647214in}}%
\pgfpathlineto{\pgfqpoint{6.453156in}{3.672394in}}%
\pgfpathclose%
\pgfusepath{fill}%
\end{pgfscope}%
\begin{pgfscope}%
\pgfpathrectangle{\pgfqpoint{0.680860in}{0.078740in}}{\pgfqpoint{7.842520in}{7.842520in}}%
\pgfusepath{clip}%
\pgfsetbuttcap%
\pgfsetroundjoin%
\definecolor{currentfill}{rgb}{0.225863,0.330805,0.547314}%
\pgfsetfillcolor{currentfill}%
\pgfsetlinewidth{0.000000pt}%
\definecolor{currentstroke}{rgb}{0.730889,0.871916,0.156029}%
\pgfsetstrokecolor{currentstroke}%
\pgfsetdash{}{0pt}%
\pgfpathmoveto{\pgfqpoint{5.155473in}{4.173442in}}%
\pgfpathlineto{\pgfqpoint{5.080165in}{4.257300in}}%
\pgfpathlineto{\pgfqpoint{5.290111in}{4.161656in}}%
\pgfpathclose%
\pgfusepath{fill}%
\end{pgfscope}%
\begin{pgfscope}%
\pgfpathrectangle{\pgfqpoint{0.680860in}{0.078740in}}{\pgfqpoint{7.842520in}{7.842520in}}%
\pgfusepath{clip}%
\pgfsetbuttcap%
\pgfsetroundjoin%
\definecolor{currentfill}{rgb}{0.271828,0.209303,0.504434}%
\pgfsetfillcolor{currentfill}%
\pgfsetlinewidth{0.000000pt}%
\definecolor{currentstroke}{rgb}{0.741388,0.873449,0.149561}%
\pgfsetstrokecolor{currentstroke}%
\pgfsetdash{}{0pt}%
\pgfpathmoveto{\pgfqpoint{6.248671in}{3.840241in}}%
\pgfpathlineto{\pgfqpoint{6.385686in}{3.821525in}}%
\pgfpathlineto{\pgfqpoint{6.453156in}{3.672394in}}%
\pgfpathclose%
\pgfusepath{fill}%
\end{pgfscope}%
\begin{pgfscope}%
\pgfpathrectangle{\pgfqpoint{0.680860in}{0.078740in}}{\pgfqpoint{7.842520in}{7.842520in}}%
\pgfusepath{clip}%
\pgfsetbuttcap%
\pgfsetroundjoin%
\definecolor{currentfill}{rgb}{0.221989,0.339161,0.548752}%
\pgfsetfillcolor{currentfill}%
\pgfsetlinewidth{0.000000pt}%
\definecolor{currentstroke}{rgb}{0.751884,0.874951,0.143228}%
\pgfsetstrokecolor{currentstroke}%
\pgfsetdash{}{0pt}%
\pgfpathmoveto{\pgfqpoint{4.945541in}{4.265119in}}%
\pgfpathlineto{\pgfqpoint{5.080165in}{4.257300in}}%
\pgfpathlineto{\pgfqpoint{5.155473in}{4.173442in}}%
\pgfpathclose%
\pgfusepath{fill}%
\end{pgfscope}%
\begin{pgfscope}%
\pgfpathrectangle{\pgfqpoint{0.680860in}{0.078740in}}{\pgfqpoint{7.842520in}{7.842520in}}%
\pgfusepath{clip}%
\pgfsetbuttcap%
\pgfsetroundjoin%
\definecolor{currentfill}{rgb}{0.192357,0.403199,0.555836}%
\pgfsetfillcolor{currentfill}%
\pgfsetlinewidth{0.000000pt}%
\definecolor{currentstroke}{rgb}{0.762373,0.876424,0.137064}%
\pgfsetstrokecolor{currentstroke}%
\pgfsetdash{}{0pt}%
\pgfpathmoveto{\pgfqpoint{3.220244in}{4.471698in}}%
\pgfpathlineto{\pgfqpoint{3.351326in}{4.469312in}}%
\pgfpathlineto{\pgfqpoint{3.433422in}{4.469040in}}%
\pgfpathclose%
\pgfusepath{fill}%
\end{pgfscope}%
\begin{pgfscope}%
\pgfpathrectangle{\pgfqpoint{0.680860in}{0.078740in}}{\pgfqpoint{7.842520in}{7.842520in}}%
\pgfusepath{clip}%
\pgfsetbuttcap%
\pgfsetroundjoin%
\definecolor{currentfill}{rgb}{0.208623,0.367752,0.552675}%
\pgfsetfillcolor{currentfill}%
\pgfsetlinewidth{0.000000pt}%
\definecolor{currentstroke}{rgb}{0.772852,0.877868,0.131109}%
\pgfsetstrokecolor{currentstroke}%
\pgfsetdash{}{0pt}%
\pgfpathmoveto{\pgfqpoint{4.600422in}{4.345480in}}%
\pgfpathlineto{\pgfqpoint{4.734429in}{4.340057in}}%
\pgfpathlineto{\pgfqpoint{4.811454in}{4.273843in}}%
\pgfpathclose%
\pgfusepath{fill}%
\end{pgfscope}%
\begin{pgfscope}%
\pgfpathrectangle{\pgfqpoint{0.680860in}{0.078740in}}{\pgfqpoint{7.842520in}{7.842520in}}%
\pgfusepath{clip}%
\pgfsetbuttcap%
\pgfsetroundjoin%
\definecolor{currentfill}{rgb}{0.283091,0.110553,0.431554}%
\pgfsetfillcolor{currentfill}%
\pgfsetlinewidth{0.000000pt}%
\definecolor{currentstroke}{rgb}{0.783315,0.879285,0.125405}%
\pgfsetstrokecolor{currentstroke}%
\pgfsetdash{}{0pt}%
\pgfpathmoveto{\pgfqpoint{7.141717in}{3.550216in}}%
\pgfpathlineto{\pgfqpoint{7.203668in}{3.360691in}}%
\pgfpathlineto{\pgfqpoint{7.065952in}{3.392437in}}%
\pgfpathclose%
\pgfusepath{fill}%
\end{pgfscope}%
\begin{pgfscope}%
\pgfpathrectangle{\pgfqpoint{0.680860in}{0.078740in}}{\pgfqpoint{7.842520in}{7.842520in}}%
\pgfusepath{clip}%
\pgfsetbuttcap%
\pgfsetroundjoin%
\definecolor{currentfill}{rgb}{0.190631,0.407061,0.556089}%
\pgfsetfillcolor{currentfill}%
\pgfsetlinewidth{0.000000pt}%
\definecolor{currentstroke}{rgb}{0.793760,0.880678,0.120005}%
\pgfsetstrokecolor{currentstroke}%
\pgfsetdash{}{0pt}%
\pgfpathmoveto{\pgfqpoint{3.137515in}{4.459090in}}%
\pgfpathlineto{\pgfqpoint{3.220244in}{4.471698in}}%
\pgfpathlineto{\pgfqpoint{3.089677in}{4.475160in}}%
\pgfpathclose%
\pgfusepath{fill}%
\end{pgfscope}%
\begin{pgfscope}%
\pgfpathrectangle{\pgfqpoint{0.680860in}{0.078740in}}{\pgfqpoint{7.842520in}{7.842520in}}%
\pgfusepath{clip}%
\pgfsetbuttcap%
\pgfsetroundjoin%
\definecolor{currentfill}{rgb}{0.192357,0.403199,0.555836}%
\pgfsetfillcolor{currentfill}%
\pgfsetlinewidth{0.000000pt}%
\definecolor{currentstroke}{rgb}{0.804182,0.882046,0.114965}%
\pgfsetstrokecolor{currentstroke}%
\pgfsetdash{}{0pt}%
\pgfpathmoveto{\pgfqpoint{3.696611in}{4.463383in}}%
\pgfpathlineto{\pgfqpoint{3.777688in}{4.447685in}}%
\pgfpathlineto{\pgfqpoint{3.564754in}{4.465670in}}%
\pgfpathclose%
\pgfusepath{fill}%
\end{pgfscope}%
\begin{pgfscope}%
\pgfpathrectangle{\pgfqpoint{0.680860in}{0.078740in}}{\pgfqpoint{7.842520in}{7.842520in}}%
\pgfusepath{clip}%
\pgfsetbuttcap%
\pgfsetroundjoin%
\definecolor{currentfill}{rgb}{0.276194,0.190074,0.493001}%
\pgfsetfillcolor{currentfill}%
\pgfsetlinewidth{0.000000pt}%
\definecolor{currentstroke}{rgb}{0.814576,0.883393,0.110347}%
\pgfsetstrokecolor{currentstroke}%
\pgfsetdash{}{0pt}%
\pgfpathmoveto{\pgfqpoint{6.727067in}{3.622390in}}%
\pgfpathlineto{\pgfqpoint{6.589863in}{3.647214in}}%
\pgfpathlineto{\pgfqpoint{6.523228in}{3.803386in}}%
\pgfpathclose%
\pgfusepath{fill}%
\end{pgfscope}%
\begin{pgfscope}%
\pgfpathrectangle{\pgfqpoint{0.680860in}{0.078740in}}{\pgfqpoint{7.842520in}{7.842520in}}%
\pgfusepath{clip}%
\pgfsetbuttcap%
\pgfsetroundjoin%
\definecolor{currentfill}{rgb}{0.199430,0.387607,0.554642}%
\pgfsetfillcolor{currentfill}%
\pgfsetlinewidth{0.000000pt}%
\definecolor{currentstroke}{rgb}{0.824940,0.884720,0.106217}%
\pgfsetstrokecolor{currentstroke}%
\pgfsetdash{}{0pt}%
\pgfpathmoveto{\pgfqpoint{4.466954in}{4.351887in}}%
\pgfpathlineto{\pgfqpoint{4.255070in}{4.404464in}}%
\pgfpathlineto{\pgfqpoint{4.388401in}{4.400722in}}%
\pgfpathclose%
\pgfusepath{fill}%
\end{pgfscope}%
\begin{pgfscope}%
\pgfpathrectangle{\pgfqpoint{0.680860in}{0.078740in}}{\pgfqpoint{7.842520in}{7.842520in}}%
\pgfusepath{clip}%
\pgfsetbuttcap%
\pgfsetroundjoin%
\definecolor{currentfill}{rgb}{0.270595,0.214069,0.507052}%
\pgfsetfillcolor{currentfill}%
\pgfsetlinewidth{0.000000pt}%
\definecolor{currentstroke}{rgb}{0.835270,0.886029,0.102646}%
\pgfsetstrokecolor{currentstroke}%
\pgfsetdash{}{0pt}%
\pgfpathmoveto{\pgfqpoint{6.453156in}{3.672394in}}%
\pgfpathlineto{\pgfqpoint{6.385686in}{3.821525in}}%
\pgfpathlineto{\pgfqpoint{6.523228in}{3.803386in}}%
\pgfpathclose%
\pgfusepath{fill}%
\end{pgfscope}%
\begin{pgfscope}%
\pgfpathrectangle{\pgfqpoint{0.680860in}{0.078740in}}{\pgfqpoint{7.842520in}{7.842520in}}%
\pgfusepath{clip}%
\pgfsetbuttcap%
\pgfsetroundjoin%
\definecolor{currentfill}{rgb}{0.233603,0.313828,0.543914}%
\pgfsetfillcolor{currentfill}%
\pgfsetlinewidth{0.000000pt}%
\definecolor{currentstroke}{rgb}{0.845561,0.887322,0.099702}%
\pgfsetstrokecolor{currentstroke}%
\pgfsetdash{}{0pt}%
\pgfpathmoveto{\pgfqpoint{5.560991in}{4.140518in}}%
\pgfpathlineto{\pgfqpoint{5.633787in}{4.033516in}}%
\pgfpathlineto{\pgfqpoint{5.425281in}{4.150671in}}%
\pgfpathclose%
\pgfusepath{fill}%
\end{pgfscope}%
\begin{pgfscope}%
\pgfpathrectangle{\pgfqpoint{0.680860in}{0.078740in}}{\pgfqpoint{7.842520in}{7.842520in}}%
\pgfusepath{clip}%
\pgfsetbuttcap%
\pgfsetroundjoin%
\definecolor{currentfill}{rgb}{0.282623,0.140926,0.457517}%
\pgfsetfillcolor{currentfill}%
\pgfsetlinewidth{0.000000pt}%
\definecolor{currentstroke}{rgb}{0.855810,0.888601,0.097452}%
\pgfsetstrokecolor{currentstroke}%
\pgfsetdash{}{0pt}%
\pgfpathmoveto{\pgfqpoint{6.864774in}{3.597939in}}%
\pgfpathlineto{\pgfqpoint{7.002988in}{3.573875in}}%
\pgfpathlineto{\pgfqpoint{7.065952in}{3.392437in}}%
\pgfpathclose%
\pgfusepath{fill}%
\end{pgfscope}%
\begin{pgfscope}%
\pgfpathrectangle{\pgfqpoint{0.680860in}{0.078740in}}{\pgfqpoint{7.842520in}{7.842520in}}%
\pgfusepath{clip}%
\pgfsetbuttcap%
\pgfsetroundjoin%
\definecolor{currentfill}{rgb}{0.194100,0.399323,0.555565}%
\pgfsetfillcolor{currentfill}%
\pgfsetlinewidth{0.000000pt}%
\definecolor{currentstroke}{rgb}{0.866013,0.889868,0.095953}%
\pgfsetstrokecolor{currentstroke}%
\pgfsetdash{}{0pt}%
\pgfpathmoveto{\pgfqpoint{3.909766in}{4.443897in}}%
\pgfpathlineto{\pgfqpoint{4.042375in}{4.441179in}}%
\pgfpathlineto{\pgfqpoint{4.122277in}{4.409244in}}%
\pgfpathclose%
\pgfusepath{fill}%
\end{pgfscope}%
\begin{pgfscope}%
\pgfpathrectangle{\pgfqpoint{0.680860in}{0.078740in}}{\pgfqpoint{7.842520in}{7.842520in}}%
\pgfusepath{clip}%
\pgfsetbuttcap%
\pgfsetroundjoin%
\definecolor{currentfill}{rgb}{0.210503,0.363727,0.552206}%
\pgfsetfillcolor{currentfill}%
\pgfsetlinewidth{0.000000pt}%
\definecolor{currentstroke}{rgb}{0.876168,0.891125,0.095250}%
\pgfsetstrokecolor{currentstroke}%
\pgfsetdash{}{0pt}%
\pgfpathmoveto{\pgfqpoint{4.868983in}{4.335656in}}%
\pgfpathlineto{\pgfqpoint{4.945541in}{4.265119in}}%
\pgfpathlineto{\pgfqpoint{4.811454in}{4.273843in}}%
\pgfpathclose%
\pgfusepath{fill}%
\end{pgfscope}%
\begin{pgfscope}%
\pgfpathrectangle{\pgfqpoint{0.680860in}{0.078740in}}{\pgfqpoint{7.842520in}{7.842520in}}%
\pgfusepath{clip}%
\pgfsetbuttcap%
\pgfsetroundjoin%
\definecolor{currentfill}{rgb}{0.190631,0.407061,0.556089}%
\pgfsetfillcolor{currentfill}%
\pgfsetlinewidth{0.000000pt}%
\definecolor{currentstroke}{rgb}{0.886271,0.892374,0.095374}%
\pgfsetstrokecolor{currentstroke}%
\pgfsetdash{}{0pt}%
\pgfpathmoveto{\pgfqpoint{3.006704in}{4.460941in}}%
\pgfpathlineto{\pgfqpoint{3.137515in}{4.459090in}}%
\pgfpathlineto{\pgfqpoint{3.089677in}{4.475160in}}%
\pgfpathclose%
\pgfusepath{fill}%
\end{pgfscope}%
\begin{pgfscope}%
\pgfpathrectangle{\pgfqpoint{0.680860in}{0.078740in}}{\pgfqpoint{7.842520in}{7.842520in}}%
\pgfusepath{clip}%
\pgfsetbuttcap%
\pgfsetroundjoin%
\definecolor{currentfill}{rgb}{0.239346,0.300855,0.540844}%
\pgfsetfillcolor{currentfill}%
\pgfsetlinewidth{0.000000pt}%
\definecolor{currentstroke}{rgb}{0.896320,0.893616,0.096335}%
\pgfsetstrokecolor{currentstroke}%
\pgfsetdash{}{0pt}%
\pgfpathmoveto{\pgfqpoint{5.633787in}{4.033516in}}%
\pgfpathlineto{\pgfqpoint{5.697250in}{4.131229in}}%
\pgfpathlineto{\pgfqpoint{5.769414in}{4.018548in}}%
\pgfpathclose%
\pgfusepath{fill}%
\end{pgfscope}%
\begin{pgfscope}%
\pgfpathrectangle{\pgfqpoint{0.680860in}{0.078740in}}{\pgfqpoint{7.842520in}{7.842520in}}%
\pgfusepath{clip}%
\pgfsetbuttcap%
\pgfsetroundjoin%
\definecolor{currentfill}{rgb}{0.190631,0.407061,0.556089}%
\pgfsetfillcolor{currentfill}%
\pgfsetlinewidth{0.000000pt}%
\definecolor{currentstroke}{rgb}{0.906311,0.894855,0.098125}%
\pgfsetstrokecolor{currentstroke}%
\pgfsetdash{}{0pt}%
\pgfpathmoveto{\pgfqpoint{3.482933in}{4.468046in}}%
\pgfpathlineto{\pgfqpoint{3.564754in}{4.465670in}}%
\pgfpathlineto{\pgfqpoint{3.433422in}{4.469040in}}%
\pgfpathclose%
\pgfusepath{fill}%
\end{pgfscope}%
\begin{pgfscope}%
\pgfpathrectangle{\pgfqpoint{0.680860in}{0.078740in}}{\pgfqpoint{7.842520in}{7.842520in}}%
\pgfusepath{clip}%
\pgfsetbuttcap%
\pgfsetroundjoin%
\definecolor{currentfill}{rgb}{0.250425,0.274290,0.533103}%
\pgfsetfillcolor{currentfill}%
\pgfsetlinewidth{0.000000pt}%
\definecolor{currentstroke}{rgb}{0.916242,0.896091,0.100717}%
\pgfsetstrokecolor{currentstroke}%
\pgfsetdash{}{0pt}%
\pgfpathmoveto{\pgfqpoint{5.905571in}{4.004277in}}%
\pgfpathlineto{\pgfqpoint{6.042265in}{3.990731in}}%
\pgfpathlineto{\pgfqpoint{6.112176in}{3.859511in}}%
\pgfpathclose%
\pgfusepath{fill}%
\end{pgfscope}%
\begin{pgfscope}%
\pgfpathrectangle{\pgfqpoint{0.680860in}{0.078740in}}{\pgfqpoint{7.842520in}{7.842520in}}%
\pgfusepath{clip}%
\pgfsetbuttcap%
\pgfsetroundjoin%
\definecolor{currentfill}{rgb}{0.201239,0.383670,0.554294}%
\pgfsetfillcolor{currentfill}%
\pgfsetlinewidth{0.000000pt}%
\definecolor{currentstroke}{rgb}{0.926106,0.897330,0.104071}%
\pgfsetstrokecolor{currentstroke}%
\pgfsetdash{}{0pt}%
\pgfpathmoveto{\pgfqpoint{4.466954in}{4.351887in}}%
\pgfpathlineto{\pgfqpoint{4.522276in}{4.398058in}}%
\pgfpathlineto{\pgfqpoint{4.600422in}{4.345480in}}%
\pgfpathclose%
\pgfusepath{fill}%
\end{pgfscope}%
\begin{pgfscope}%
\pgfpathrectangle{\pgfqpoint{0.680860in}{0.078740in}}{\pgfqpoint{7.842520in}{7.842520in}}%
\pgfusepath{clip}%
\pgfsetbuttcap%
\pgfsetroundjoin%
\definecolor{currentfill}{rgb}{0.257322,0.256130,0.526563}%
\pgfsetfillcolor{currentfill}%
\pgfsetlinewidth{0.000000pt}%
\definecolor{currentstroke}{rgb}{0.935904,0.898570,0.108131}%
\pgfsetstrokecolor{currentstroke}%
\pgfsetdash{}{0pt}%
\pgfpathmoveto{\pgfqpoint{6.248671in}{3.840241in}}%
\pgfpathlineto{\pgfqpoint{6.112176in}{3.859511in}}%
\pgfpathlineto{\pgfqpoint{6.179505in}{3.977936in}}%
\pgfpathclose%
\pgfusepath{fill}%
\end{pgfscope}%
\begin{pgfscope}%
\pgfpathrectangle{\pgfqpoint{0.680860in}{0.078740in}}{\pgfqpoint{7.842520in}{7.842520in}}%
\pgfusepath{clip}%
\pgfsetbuttcap%
\pgfsetroundjoin%
\definecolor{currentfill}{rgb}{0.190631,0.407061,0.556089}%
\pgfsetfillcolor{currentfill}%
\pgfsetlinewidth{0.000000pt}%
\definecolor{currentstroke}{rgb}{0.945636,0.899815,0.112838}%
\pgfsetstrokecolor{currentstroke}%
\pgfsetdash{}{0pt}%
\pgfpathmoveto{\pgfqpoint{3.829001in}{4.462220in}}%
\pgfpathlineto{\pgfqpoint{3.909766in}{4.443897in}}%
\pgfpathlineto{\pgfqpoint{3.777688in}{4.447685in}}%
\pgfpathclose%
\pgfusepath{fill}%
\end{pgfscope}%
\begin{pgfscope}%
\pgfpathrectangle{\pgfqpoint{0.680860in}{0.078740in}}{\pgfqpoint{7.842520in}{7.842520in}}%
\pgfusepath{clip}%
\pgfsetbuttcap%
\pgfsetroundjoin%
\definecolor{currentfill}{rgb}{0.194100,0.399323,0.555565}%
\pgfsetfillcolor{currentfill}%
\pgfsetlinewidth{0.000000pt}%
\definecolor{currentstroke}{rgb}{0.955300,0.901065,0.118128}%
\pgfsetstrokecolor{currentstroke}%
\pgfsetdash{}{0pt}%
\pgfpathmoveto{\pgfqpoint{4.122277in}{4.409244in}}%
\pgfpathlineto{\pgfqpoint{4.175526in}{4.439574in}}%
\pgfpathlineto{\pgfqpoint{4.255070in}{4.404464in}}%
\pgfpathclose%
\pgfusepath{fill}%
\end{pgfscope}%
\begin{pgfscope}%
\pgfpathrectangle{\pgfqpoint{0.680860in}{0.078740in}}{\pgfqpoint{7.842520in}{7.842520in}}%
\pgfusepath{clip}%
\pgfsetbuttcap%
\pgfsetroundjoin%
\definecolor{currentfill}{rgb}{0.188923,0.410910,0.556326}%
\pgfsetfillcolor{currentfill}%
\pgfsetlinewidth{0.000000pt}%
\definecolor{currentstroke}{rgb}{0.964894,0.902323,0.123941}%
\pgfsetstrokecolor{currentstroke}%
\pgfsetdash{}{0pt}%
\pgfpathmoveto{\pgfqpoint{3.433422in}{4.469040in}}%
\pgfpathlineto{\pgfqpoint{3.351326in}{4.469312in}}%
\pgfpathlineto{\pgfqpoint{3.482933in}{4.468046in}}%
\pgfpathclose%
\pgfusepath{fill}%
\end{pgfscope}%
\begin{pgfscope}%
\pgfpathrectangle{\pgfqpoint{0.680860in}{0.078740in}}{\pgfqpoint{7.842520in}{7.842520in}}%
\pgfusepath{clip}%
\pgfsetbuttcap%
\pgfsetroundjoin%
\definecolor{currentfill}{rgb}{0.216210,0.351535,0.550627}%
\pgfsetfillcolor{currentfill}%
\pgfsetlinewidth{0.000000pt}%
\definecolor{currentstroke}{rgb}{0.974417,0.903590,0.130215}%
\pgfsetstrokecolor{currentstroke}%
\pgfsetdash{}{0pt}%
\pgfpathmoveto{\pgfqpoint{5.290111in}{4.161656in}}%
\pgfpathlineto{\pgfqpoint{5.080165in}{4.257300in}}%
\pgfpathlineto{\pgfqpoint{5.215336in}{4.250420in}}%
\pgfpathclose%
\pgfusepath{fill}%
\end{pgfscope}%
\begin{pgfscope}%
\pgfpathrectangle{\pgfqpoint{0.680860in}{0.078740in}}{\pgfqpoint{7.842520in}{7.842520in}}%
\pgfusepath{clip}%
\pgfsetbuttcap%
\pgfsetroundjoin%
\definecolor{currentfill}{rgb}{0.204903,0.375746,0.553533}%
\pgfsetfillcolor{currentfill}%
\pgfsetlinewidth{0.000000pt}%
\definecolor{currentstroke}{rgb}{0.983868,0.904867,0.136897}%
\pgfsetstrokecolor{currentstroke}%
\pgfsetdash{}{0pt}%
\pgfpathmoveto{\pgfqpoint{4.811454in}{4.273843in}}%
\pgfpathlineto{\pgfqpoint{4.734429in}{4.340057in}}%
\pgfpathlineto{\pgfqpoint{4.868983in}{4.335656in}}%
\pgfpathclose%
\pgfusepath{fill}%
\end{pgfscope}%
\begin{pgfscope}%
\pgfpathrectangle{\pgfqpoint{0.680860in}{0.078740in}}{\pgfqpoint{7.842520in}{7.842520in}}%
\pgfusepath{clip}%
\pgfsetbuttcap%
\pgfsetroundjoin%
\definecolor{currentfill}{rgb}{0.282290,0.145912,0.461510}%
\pgfsetfillcolor{currentfill}%
\pgfsetlinewidth{0.000000pt}%
\definecolor{currentstroke}{rgb}{0.993248,0.906157,0.143936}%
\pgfsetstrokecolor{currentstroke}%
\pgfsetdash{}{0pt}%
\pgfpathmoveto{\pgfqpoint{7.065952in}{3.392437in}}%
\pgfpathlineto{\pgfqpoint{7.002988in}{3.573875in}}%
\pgfpathlineto{\pgfqpoint{7.141717in}{3.550216in}}%
\pgfpathclose%
\pgfusepath{fill}%
\end{pgfscope}%
\begin{pgfscope}%
\pgfpathrectangle{\pgfqpoint{0.680860in}{0.078740in}}{\pgfqpoint{7.842520in}{7.842520in}}%
\pgfusepath{clip}%
\pgfsetbuttcap%
\pgfsetroundjoin%
\definecolor{currentfill}{rgb}{0.188923,0.410910,0.556326}%
\pgfsetfillcolor{currentfill}%
\pgfsetlinewidth{0.000000pt}%
\definecolor{currentstroke}{rgb}{0.267004,0.004874,0.329415}%
\pgfsetstrokecolor{currentstroke}%
\pgfsetdash{}{0pt}%
\pgfpathmoveto{\pgfqpoint{3.829001in}{4.462220in}}%
\pgfpathlineto{\pgfqpoint{3.777688in}{4.447685in}}%
\pgfpathlineto{\pgfqpoint{3.696611in}{4.463383in}}%
\pgfpathclose%
\pgfusepath{fill}%
\end{pgfscope}%
\begin{pgfscope}%
\pgfpathrectangle{\pgfqpoint{0.680860in}{0.078740in}}{\pgfqpoint{7.842520in}{7.842520in}}%
\pgfusepath{clip}%
\pgfsetbuttcap%
\pgfsetroundjoin%
\definecolor{currentfill}{rgb}{0.229739,0.322361,0.545706}%
\pgfsetfillcolor{currentfill}%
\pgfsetlinewidth{0.000000pt}%
\definecolor{currentstroke}{rgb}{0.268510,0.009605,0.335427}%
\pgfsetstrokecolor{currentstroke}%
\pgfsetdash{}{0pt}%
\pgfpathmoveto{\pgfqpoint{5.560991in}{4.140518in}}%
\pgfpathlineto{\pgfqpoint{5.697250in}{4.131229in}}%
\pgfpathlineto{\pgfqpoint{5.633787in}{4.033516in}}%
\pgfpathclose%
\pgfusepath{fill}%
\end{pgfscope}%
\begin{pgfscope}%
\pgfpathrectangle{\pgfqpoint{0.680860in}{0.078740in}}{\pgfqpoint{7.842520in}{7.842520in}}%
\pgfusepath{clip}%
\pgfsetbuttcap%
\pgfsetroundjoin%
\definecolor{currentfill}{rgb}{0.195860,0.395433,0.555276}%
\pgfsetfillcolor{currentfill}%
\pgfsetlinewidth{0.000000pt}%
\definecolor{currentstroke}{rgb}{0.269944,0.014625,0.341379}%
\pgfsetstrokecolor{currentstroke}%
\pgfsetdash{}{0pt}%
\pgfpathmoveto{\pgfqpoint{4.388401in}{4.400722in}}%
\pgfpathlineto{\pgfqpoint{4.522276in}{4.398058in}}%
\pgfpathlineto{\pgfqpoint{4.466954in}{4.351887in}}%
\pgfpathclose%
\pgfusepath{fill}%
\end{pgfscope}%
\begin{pgfscope}%
\pgfpathrectangle{\pgfqpoint{0.680860in}{0.078740in}}{\pgfqpoint{7.842520in}{7.842520in}}%
\pgfusepath{clip}%
\pgfsetbuttcap%
\pgfsetroundjoin%
\definecolor{currentfill}{rgb}{0.220057,0.343307,0.549413}%
\pgfsetfillcolor{currentfill}%
\pgfsetlinewidth{0.000000pt}%
\definecolor{currentstroke}{rgb}{0.271305,0.019942,0.347269}%
\pgfsetstrokecolor{currentstroke}%
\pgfsetdash{}{0pt}%
\pgfpathmoveto{\pgfqpoint{5.351061in}{4.244517in}}%
\pgfpathlineto{\pgfqpoint{5.425281in}{4.150671in}}%
\pgfpathlineto{\pgfqpoint{5.290111in}{4.161656in}}%
\pgfpathclose%
\pgfusepath{fill}%
\end{pgfscope}%
\begin{pgfscope}%
\pgfpathrectangle{\pgfqpoint{0.680860in}{0.078740in}}{\pgfqpoint{7.842520in}{7.842520in}}%
\pgfusepath{clip}%
\pgfsetbuttcap%
\pgfsetroundjoin%
\definecolor{currentfill}{rgb}{0.190631,0.407061,0.556089}%
\pgfsetfillcolor{currentfill}%
\pgfsetlinewidth{0.000000pt}%
\definecolor{currentstroke}{rgb}{0.272594,0.025563,0.353093}%
\pgfsetstrokecolor{currentstroke}%
\pgfsetdash{}{0pt}%
\pgfpathmoveto{\pgfqpoint{4.122277in}{4.409244in}}%
\pgfpathlineto{\pgfqpoint{4.042375in}{4.441179in}}%
\pgfpathlineto{\pgfqpoint{4.175526in}{4.439574in}}%
\pgfpathclose%
\pgfusepath{fill}%
\end{pgfscope}%
\begin{pgfscope}%
\pgfpathrectangle{\pgfqpoint{0.680860in}{0.078740in}}{\pgfqpoint{7.842520in}{7.842520in}}%
\pgfusepath{clip}%
\pgfsetbuttcap%
\pgfsetroundjoin%
\definecolor{currentfill}{rgb}{0.283197,0.115680,0.436115}%
\pgfsetfillcolor{currentfill}%
\pgfsetlinewidth{0.000000pt}%
\definecolor{currentstroke}{rgb}{0.273809,0.031497,0.358853}%
\pgfsetstrokecolor{currentstroke}%
\pgfsetdash{}{0pt}%
\pgfpathmoveto{\pgfqpoint{7.280966in}{3.526978in}}%
\pgfpathlineto{\pgfqpoint{7.341858in}{3.329090in}}%
\pgfpathlineto{\pgfqpoint{7.203668in}{3.360691in}}%
\pgfpathclose%
\pgfusepath{fill}%
\end{pgfscope}%
\begin{pgfscope}%
\pgfpathrectangle{\pgfqpoint{0.680860in}{0.078740in}}{\pgfqpoint{7.842520in}{7.842520in}}%
\pgfusepath{clip}%
\pgfsetbuttcap%
\pgfsetroundjoin%
\definecolor{currentfill}{rgb}{0.187231,0.414746,0.556547}%
\pgfsetfillcolor{currentfill}%
\pgfsetlinewidth{0.000000pt}%
\definecolor{currentstroke}{rgb}{0.274952,0.037752,0.364543}%
\pgfsetstrokecolor{currentstroke}%
\pgfsetdash{}{0pt}%
\pgfpathmoveto{\pgfqpoint{3.696611in}{4.463383in}}%
\pgfpathlineto{\pgfqpoint{3.564754in}{4.465670in}}%
\pgfpathlineto{\pgfqpoint{3.482933in}{4.468046in}}%
\pgfpathclose%
\pgfusepath{fill}%
\end{pgfscope}%
\begin{pgfscope}%
\pgfpathrectangle{\pgfqpoint{0.680860in}{0.078740in}}{\pgfqpoint{7.842520in}{7.842520in}}%
\pgfusepath{clip}%
\pgfsetbuttcap%
\pgfsetroundjoin%
\definecolor{currentfill}{rgb}{0.185556,0.418570,0.556753}%
\pgfsetfillcolor{currentfill}%
\pgfsetlinewidth{0.000000pt}%
\definecolor{currentstroke}{rgb}{0.276022,0.044167,0.370164}%
\pgfsetstrokecolor{currentstroke}%
\pgfsetdash{}{0pt}%
\pgfpathmoveto{\pgfqpoint{3.268849in}{4.458377in}}%
\pgfpathlineto{\pgfqpoint{3.351326in}{4.469312in}}%
\pgfpathlineto{\pgfqpoint{3.220244in}{4.471698in}}%
\pgfpathclose%
\pgfusepath{fill}%
\end{pgfscope}%
\begin{pgfscope}%
\pgfpathrectangle{\pgfqpoint{0.680860in}{0.078740in}}{\pgfqpoint{7.842520in}{7.842520in}}%
\pgfusepath{clip}%
\pgfsetbuttcap%
\pgfsetroundjoin%
\definecolor{currentfill}{rgb}{0.248629,0.278775,0.534556}%
\pgfsetfillcolor{currentfill}%
\pgfsetlinewidth{0.000000pt}%
\definecolor{currentstroke}{rgb}{0.277018,0.050344,0.375715}%
\pgfsetstrokecolor{currentstroke}%
\pgfsetdash{}{0pt}%
\pgfpathmoveto{\pgfqpoint{6.179505in}{3.977936in}}%
\pgfpathlineto{\pgfqpoint{6.112176in}{3.859511in}}%
\pgfpathlineto{\pgfqpoint{6.042265in}{3.990731in}}%
\pgfpathclose%
\pgfusepath{fill}%
\end{pgfscope}%
\begin{pgfscope}%
\pgfpathrectangle{\pgfqpoint{0.680860in}{0.078740in}}{\pgfqpoint{7.842520in}{7.842520in}}%
\pgfusepath{clip}%
\pgfsetbuttcap%
\pgfsetroundjoin%
\definecolor{currentfill}{rgb}{0.197636,0.391528,0.554969}%
\pgfsetfillcolor{currentfill}%
\pgfsetlinewidth{0.000000pt}%
\definecolor{currentstroke}{rgb}{0.277941,0.056324,0.381191}%
\pgfsetstrokecolor{currentstroke}%
\pgfsetdash{}{0pt}%
\pgfpathmoveto{\pgfqpoint{4.522276in}{4.398058in}}%
\pgfpathlineto{\pgfqpoint{4.734429in}{4.340057in}}%
\pgfpathlineto{\pgfqpoint{4.600422in}{4.345480in}}%
\pgfpathclose%
\pgfusepath{fill}%
\end{pgfscope}%
\begin{pgfscope}%
\pgfpathrectangle{\pgfqpoint{0.680860in}{0.078740in}}{\pgfqpoint{7.842520in}{7.842520in}}%
\pgfusepath{clip}%
\pgfsetbuttcap%
\pgfsetroundjoin%
\definecolor{currentfill}{rgb}{0.187231,0.414746,0.556547}%
\pgfsetfillcolor{currentfill}%
\pgfsetlinewidth{0.000000pt}%
\definecolor{currentstroke}{rgb}{0.278791,0.062145,0.386592}%
\pgfsetstrokecolor{currentstroke}%
\pgfsetdash{}{0pt}%
\pgfpathmoveto{\pgfqpoint{3.909766in}{4.443897in}}%
\pgfpathlineto{\pgfqpoint{3.829001in}{4.462220in}}%
\pgfpathlineto{\pgfqpoint{4.042375in}{4.441179in}}%
\pgfpathclose%
\pgfusepath{fill}%
\end{pgfscope}%
\begin{pgfscope}%
\pgfpathrectangle{\pgfqpoint{0.680860in}{0.078740in}}{\pgfqpoint{7.842520in}{7.842520in}}%
\pgfusepath{clip}%
\pgfsetbuttcap%
\pgfsetroundjoin%
\definecolor{currentfill}{rgb}{0.190631,0.407061,0.556089}%
\pgfsetfillcolor{currentfill}%
\pgfsetlinewidth{0.000000pt}%
\definecolor{currentstroke}{rgb}{0.279566,0.067836,0.391917}%
\pgfsetstrokecolor{currentstroke}%
\pgfsetdash{}{0pt}%
\pgfpathmoveto{\pgfqpoint{4.388401in}{4.400722in}}%
\pgfpathlineto{\pgfqpoint{4.255070in}{4.404464in}}%
\pgfpathlineto{\pgfqpoint{4.175526in}{4.439574in}}%
\pgfpathclose%
\pgfusepath{fill}%
\end{pgfscope}%
\begin{pgfscope}%
\pgfpathrectangle{\pgfqpoint{0.680860in}{0.078740in}}{\pgfqpoint{7.842520in}{7.842520in}}%
\pgfusepath{clip}%
\pgfsetbuttcap%
\pgfsetroundjoin%
\definecolor{currentfill}{rgb}{0.204903,0.375746,0.553533}%
\pgfsetfillcolor{currentfill}%
\pgfsetlinewidth{0.000000pt}%
\definecolor{currentstroke}{rgb}{0.280267,0.073417,0.397163}%
\pgfsetstrokecolor{currentstroke}%
\pgfsetdash{}{0pt}%
\pgfpathmoveto{\pgfqpoint{4.945541in}{4.265119in}}%
\pgfpathlineto{\pgfqpoint{5.004093in}{4.332316in}}%
\pgfpathlineto{\pgfqpoint{5.080165in}{4.257300in}}%
\pgfpathclose%
\pgfusepath{fill}%
\end{pgfscope}%
\begin{pgfscope}%
\pgfpathrectangle{\pgfqpoint{0.680860in}{0.078740in}}{\pgfqpoint{7.842520in}{7.842520in}}%
\pgfusepath{clip}%
\pgfsetbuttcap%
\pgfsetroundjoin%
\definecolor{currentfill}{rgb}{0.233603,0.313828,0.543914}%
\pgfsetfillcolor{currentfill}%
\pgfsetlinewidth{0.000000pt}%
\definecolor{currentstroke}{rgb}{0.280894,0.078907,0.402329}%
\pgfsetstrokecolor{currentstroke}%
\pgfsetdash{}{0pt}%
\pgfpathmoveto{\pgfqpoint{5.905571in}{4.004277in}}%
\pgfpathlineto{\pgfqpoint{5.769414in}{4.018548in}}%
\pgfpathlineto{\pgfqpoint{5.834066in}{4.122838in}}%
\pgfpathclose%
\pgfusepath{fill}%
\end{pgfscope}%
\begin{pgfscope}%
\pgfpathrectangle{\pgfqpoint{0.680860in}{0.078740in}}{\pgfqpoint{7.842520in}{7.842520in}}%
\pgfusepath{clip}%
\pgfsetbuttcap%
\pgfsetroundjoin%
\definecolor{currentfill}{rgb}{0.185556,0.418570,0.556753}%
\pgfsetfillcolor{currentfill}%
\pgfsetlinewidth{0.000000pt}%
\definecolor{currentstroke}{rgb}{0.281446,0.084320,0.407414}%
\pgfsetstrokecolor{currentstroke}%
\pgfsetdash{}{0pt}%
\pgfpathmoveto{\pgfqpoint{3.220244in}{4.471698in}}%
\pgfpathlineto{\pgfqpoint{3.137515in}{4.459090in}}%
\pgfpathlineto{\pgfqpoint{3.268849in}{4.458377in}}%
\pgfpathclose%
\pgfusepath{fill}%
\end{pgfscope}%
\begin{pgfscope}%
\pgfpathrectangle{\pgfqpoint{0.680860in}{0.078740in}}{\pgfqpoint{7.842520in}{7.842520in}}%
\pgfusepath{clip}%
\pgfsetbuttcap%
\pgfsetroundjoin%
\definecolor{currentfill}{rgb}{0.274128,0.199721,0.498911}%
\pgfsetfillcolor{currentfill}%
\pgfsetlinewidth{0.000000pt}%
\definecolor{currentstroke}{rgb}{0.281924,0.089666,0.412415}%
\pgfsetstrokecolor{currentstroke}%
\pgfsetdash{}{0pt}%
\pgfpathmoveto{\pgfqpoint{6.727067in}{3.622390in}}%
\pgfpathlineto{\pgfqpoint{6.799922in}{3.768934in}}%
\pgfpathlineto{\pgfqpoint{6.864774in}{3.597939in}}%
\pgfpathclose%
\pgfusepath{fill}%
\end{pgfscope}%
\begin{pgfscope}%
\pgfpathrectangle{\pgfqpoint{0.680860in}{0.078740in}}{\pgfqpoint{7.842520in}{7.842520in}}%
\pgfusepath{clip}%
\pgfsetbuttcap%
\pgfsetroundjoin%
\definecolor{currentfill}{rgb}{0.212395,0.359683,0.551710}%
\pgfsetfillcolor{currentfill}%
\pgfsetlinewidth{0.000000pt}%
\definecolor{currentstroke}{rgb}{0.282327,0.094955,0.417331}%
\pgfsetstrokecolor{currentstroke}%
\pgfsetdash{}{0pt}%
\pgfpathmoveto{\pgfqpoint{5.215336in}{4.250420in}}%
\pgfpathlineto{\pgfqpoint{5.351061in}{4.244517in}}%
\pgfpathlineto{\pgfqpoint{5.290111in}{4.161656in}}%
\pgfpathclose%
\pgfusepath{fill}%
\end{pgfscope}%
\begin{pgfscope}%
\pgfpathrectangle{\pgfqpoint{0.680860in}{0.078740in}}{\pgfqpoint{7.842520in}{7.842520in}}%
\pgfusepath{clip}%
\pgfsetbuttcap%
\pgfsetroundjoin%
\definecolor{currentfill}{rgb}{0.267968,0.223549,0.512008}%
\pgfsetfillcolor{currentfill}%
\pgfsetlinewidth{0.000000pt}%
\definecolor{currentstroke}{rgb}{0.282656,0.100196,0.422160}%
\pgfsetstrokecolor{currentstroke}%
\pgfsetdash{}{0pt}%
\pgfpathmoveto{\pgfqpoint{6.523228in}{3.803386in}}%
\pgfpathlineto{\pgfqpoint{6.661304in}{3.785848in}}%
\pgfpathlineto{\pgfqpoint{6.727067in}{3.622390in}}%
\pgfpathclose%
\pgfusepath{fill}%
\end{pgfscope}%
\begin{pgfscope}%
\pgfpathrectangle{\pgfqpoint{0.680860in}{0.078740in}}{\pgfqpoint{7.842520in}{7.842520in}}%
\pgfusepath{clip}%
\pgfsetbuttcap%
\pgfsetroundjoin%
\definecolor{currentfill}{rgb}{0.185556,0.418570,0.556753}%
\pgfsetfillcolor{currentfill}%
\pgfsetlinewidth{0.000000pt}%
\definecolor{currentstroke}{rgb}{0.282910,0.105393,0.426902}%
\pgfsetstrokecolor{currentstroke}%
\pgfsetdash{}{0pt}%
\pgfpathmoveto{\pgfqpoint{2.923425in}{4.435607in}}%
\pgfpathlineto{\pgfqpoint{3.137515in}{4.459090in}}%
\pgfpathlineto{\pgfqpoint{3.006704in}{4.460941in}}%
\pgfpathclose%
\pgfusepath{fill}%
\end{pgfscope}%
\begin{pgfscope}%
\pgfpathrectangle{\pgfqpoint{0.680860in}{0.078740in}}{\pgfqpoint{7.842520in}{7.842520in}}%
\pgfusepath{clip}%
\pgfsetbuttcap%
\pgfsetroundjoin%
\definecolor{currentfill}{rgb}{0.253935,0.265254,0.529983}%
\pgfsetfillcolor{currentfill}%
\pgfsetlinewidth{0.000000pt}%
\definecolor{currentstroke}{rgb}{0.283091,0.110553,0.431554}%
\pgfsetstrokecolor{currentstroke}%
\pgfsetdash{}{0pt}%
\pgfpathmoveto{\pgfqpoint{6.317298in}{3.965924in}}%
\pgfpathlineto{\pgfqpoint{6.385686in}{3.821525in}}%
\pgfpathlineto{\pgfqpoint{6.248671in}{3.840241in}}%
\pgfpathclose%
\pgfusepath{fill}%
\end{pgfscope}%
\begin{pgfscope}%
\pgfpathrectangle{\pgfqpoint{0.680860in}{0.078740in}}{\pgfqpoint{7.842520in}{7.842520in}}%
\pgfusepath{clip}%
\pgfsetbuttcap%
\pgfsetroundjoin%
\definecolor{currentfill}{rgb}{0.283197,0.115680,0.436115}%
\pgfsetfillcolor{currentfill}%
\pgfsetlinewidth{0.000000pt}%
\definecolor{currentstroke}{rgb}{0.283197,0.115680,0.436115}%
\pgfsetstrokecolor{currentstroke}%
\pgfsetdash{}{0pt}%
\pgfpathmoveto{\pgfqpoint{7.480526in}{3.297643in}}%
\pgfpathlineto{\pgfqpoint{7.341858in}{3.329090in}}%
\pgfpathlineto{\pgfqpoint{7.280966in}{3.526978in}}%
\pgfpathclose%
\pgfusepath{fill}%
\end{pgfscope}%
\begin{pgfscope}%
\pgfpathrectangle{\pgfqpoint{0.680860in}{0.078740in}}{\pgfqpoint{7.842520in}{7.842520in}}%
\pgfusepath{clip}%
\pgfsetbuttcap%
\pgfsetroundjoin%
\definecolor{currentfill}{rgb}{0.281887,0.150881,0.465405}%
\pgfsetfillcolor{currentfill}%
\pgfsetlinewidth{0.000000pt}%
\definecolor{currentstroke}{rgb}{0.283229,0.120777,0.440584}%
\pgfsetstrokecolor{currentstroke}%
\pgfsetdash{}{0pt}%
\pgfpathmoveto{\pgfqpoint{7.280966in}{3.526978in}}%
\pgfpathlineto{\pgfqpoint{7.203668in}{3.360691in}}%
\pgfpathlineto{\pgfqpoint{7.141717in}{3.550216in}}%
\pgfpathclose%
\pgfusepath{fill}%
\end{pgfscope}%
\begin{pgfscope}%
\pgfpathrectangle{\pgfqpoint{0.680860in}{0.078740in}}{\pgfqpoint{7.842520in}{7.842520in}}%
\pgfusepath{clip}%
\pgfsetbuttcap%
\pgfsetroundjoin%
\definecolor{currentfill}{rgb}{0.199430,0.387607,0.554642}%
\pgfsetfillcolor{currentfill}%
\pgfsetlinewidth{0.000000pt}%
\definecolor{currentstroke}{rgb}{0.283187,0.125848,0.444960}%
\pgfsetstrokecolor{currentstroke}%
\pgfsetdash{}{0pt}%
\pgfpathmoveto{\pgfqpoint{4.868983in}{4.335656in}}%
\pgfpathlineto{\pgfqpoint{5.004093in}{4.332316in}}%
\pgfpathlineto{\pgfqpoint{4.945541in}{4.265119in}}%
\pgfpathclose%
\pgfusepath{fill}%
\end{pgfscope}%
\begin{pgfscope}%
\pgfpathrectangle{\pgfqpoint{0.680860in}{0.078740in}}{\pgfqpoint{7.842520in}{7.842520in}}%
\pgfusepath{clip}%
\pgfsetbuttcap%
\pgfsetroundjoin%
\definecolor{currentfill}{rgb}{0.182256,0.426184,0.557120}%
\pgfsetfillcolor{currentfill}%
\pgfsetlinewidth{0.000000pt}%
\definecolor{currentstroke}{rgb}{0.283072,0.130895,0.449241}%
\pgfsetstrokecolor{currentstroke}%
\pgfsetdash{}{0pt}%
\pgfpathmoveto{\pgfqpoint{3.482933in}{4.468046in}}%
\pgfpathlineto{\pgfqpoint{3.351326in}{4.469312in}}%
\pgfpathlineto{\pgfqpoint{3.268849in}{4.458377in}}%
\pgfpathclose%
\pgfusepath{fill}%
\end{pgfscope}%
\begin{pgfscope}%
\pgfpathrectangle{\pgfqpoint{0.680860in}{0.078740in}}{\pgfqpoint{7.842520in}{7.842520in}}%
\pgfusepath{clip}%
\pgfsetbuttcap%
\pgfsetroundjoin%
\definecolor{currentfill}{rgb}{0.225863,0.330805,0.547314}%
\pgfsetfillcolor{currentfill}%
\pgfsetlinewidth{0.000000pt}%
\definecolor{currentstroke}{rgb}{0.282884,0.135920,0.453427}%
\pgfsetstrokecolor{currentstroke}%
\pgfsetdash{}{0pt}%
\pgfpathmoveto{\pgfqpoint{5.769414in}{4.018548in}}%
\pgfpathlineto{\pgfqpoint{5.697250in}{4.131229in}}%
\pgfpathlineto{\pgfqpoint{5.834066in}{4.122838in}}%
\pgfpathclose%
\pgfusepath{fill}%
\end{pgfscope}%
\begin{pgfscope}%
\pgfpathrectangle{\pgfqpoint{0.680860in}{0.078740in}}{\pgfqpoint{7.842520in}{7.842520in}}%
\pgfusepath{clip}%
\pgfsetbuttcap%
\pgfsetroundjoin%
\definecolor{currentfill}{rgb}{0.214298,0.355619,0.551184}%
\pgfsetfillcolor{currentfill}%
\pgfsetlinewidth{0.000000pt}%
\definecolor{currentstroke}{rgb}{0.282623,0.140926,0.457517}%
\pgfsetstrokecolor{currentstroke}%
\pgfsetdash{}{0pt}%
\pgfpathmoveto{\pgfqpoint{5.560991in}{4.140518in}}%
\pgfpathlineto{\pgfqpoint{5.425281in}{4.150671in}}%
\pgfpathlineto{\pgfqpoint{5.487351in}{4.239628in}}%
\pgfpathclose%
\pgfusepath{fill}%
\end{pgfscope}%
\begin{pgfscope}%
\pgfpathrectangle{\pgfqpoint{0.680860in}{0.078740in}}{\pgfqpoint{7.842520in}{7.842520in}}%
\pgfusepath{clip}%
\pgfsetbuttcap%
\pgfsetroundjoin%
\definecolor{currentfill}{rgb}{0.203063,0.379716,0.553925}%
\pgfsetfillcolor{currentfill}%
\pgfsetlinewidth{0.000000pt}%
\definecolor{currentstroke}{rgb}{0.282290,0.145912,0.461510}%
\pgfsetstrokecolor{currentstroke}%
\pgfsetdash{}{0pt}%
\pgfpathmoveto{\pgfqpoint{5.215336in}{4.250420in}}%
\pgfpathlineto{\pgfqpoint{5.080165in}{4.257300in}}%
\pgfpathlineto{\pgfqpoint{5.004093in}{4.332316in}}%
\pgfpathclose%
\pgfusepath{fill}%
\end{pgfscope}%
\begin{pgfscope}%
\pgfpathrectangle{\pgfqpoint{0.680860in}{0.078740in}}{\pgfqpoint{7.842520in}{7.842520in}}%
\pgfusepath{clip}%
\pgfsetbuttcap%
\pgfsetroundjoin%
\definecolor{currentfill}{rgb}{0.182256,0.426184,0.557120}%
\pgfsetfillcolor{currentfill}%
\pgfsetlinewidth{0.000000pt}%
\definecolor{currentstroke}{rgb}{0.281887,0.150881,0.465405}%
\pgfsetstrokecolor{currentstroke}%
\pgfsetdash{}{0pt}%
\pgfpathmoveto{\pgfqpoint{3.482933in}{4.468046in}}%
\pgfpathlineto{\pgfqpoint{3.615073in}{4.467941in}}%
\pgfpathlineto{\pgfqpoint{3.696611in}{4.463383in}}%
\pgfpathclose%
\pgfusepath{fill}%
\end{pgfscope}%
\begin{pgfscope}%
\pgfpathrectangle{\pgfqpoint{0.680860in}{0.078740in}}{\pgfqpoint{7.842520in}{7.842520in}}%
\pgfusepath{clip}%
\pgfsetbuttcap%
\pgfsetroundjoin%
\definecolor{currentfill}{rgb}{0.265145,0.232956,0.516599}%
\pgfsetfillcolor{currentfill}%
\pgfsetlinewidth{0.000000pt}%
\definecolor{currentstroke}{rgb}{0.281412,0.155834,0.469201}%
\pgfsetstrokecolor{currentstroke}%
\pgfsetdash{}{0pt}%
\pgfpathmoveto{\pgfqpoint{6.661304in}{3.785848in}}%
\pgfpathlineto{\pgfqpoint{6.799922in}{3.768934in}}%
\pgfpathlineto{\pgfqpoint{6.727067in}{3.622390in}}%
\pgfpathclose%
\pgfusepath{fill}%
\end{pgfscope}%
\begin{pgfscope}%
\pgfpathrectangle{\pgfqpoint{0.680860in}{0.078740in}}{\pgfqpoint{7.842520in}{7.842520in}}%
\pgfusepath{clip}%
\pgfsetbuttcap%
\pgfsetroundjoin%
\definecolor{currentfill}{rgb}{0.180629,0.429975,0.557282}%
\pgfsetfillcolor{currentfill}%
\pgfsetlinewidth{0.000000pt}%
\definecolor{currentstroke}{rgb}{0.280868,0.160771,0.472899}%
\pgfsetstrokecolor{currentstroke}%
\pgfsetdash{}{0pt}%
\pgfpathmoveto{\pgfqpoint{3.696611in}{4.463383in}}%
\pgfpathlineto{\pgfqpoint{3.615073in}{4.467941in}}%
\pgfpathlineto{\pgfqpoint{3.829001in}{4.462220in}}%
\pgfpathclose%
\pgfusepath{fill}%
\end{pgfscope}%
\begin{pgfscope}%
\pgfpathrectangle{\pgfqpoint{0.680860in}{0.078740in}}{\pgfqpoint{7.842520in}{7.842520in}}%
\pgfusepath{clip}%
\pgfsetbuttcap%
\pgfsetroundjoin%
\definecolor{currentfill}{rgb}{0.244972,0.287675,0.537260}%
\pgfsetfillcolor{currentfill}%
\pgfsetlinewidth{0.000000pt}%
\definecolor{currentstroke}{rgb}{0.280255,0.165693,0.476498}%
\pgfsetstrokecolor{currentstroke}%
\pgfsetdash{}{0pt}%
\pgfpathmoveto{\pgfqpoint{6.248671in}{3.840241in}}%
\pgfpathlineto{\pgfqpoint{6.179505in}{3.977936in}}%
\pgfpathlineto{\pgfqpoint{6.317298in}{3.965924in}}%
\pgfpathclose%
\pgfusepath{fill}%
\end{pgfscope}%
\begin{pgfscope}%
\pgfpathrectangle{\pgfqpoint{0.680860in}{0.078740in}}{\pgfqpoint{7.842520in}{7.842520in}}%
\pgfusepath{clip}%
\pgfsetbuttcap%
\pgfsetroundjoin%
\definecolor{currentfill}{rgb}{0.252194,0.269783,0.531579}%
\pgfsetfillcolor{currentfill}%
\pgfsetlinewidth{0.000000pt}%
\definecolor{currentstroke}{rgb}{0.279574,0.170599,0.479997}%
\pgfsetstrokecolor{currentstroke}%
\pgfsetdash{}{0pt}%
\pgfpathmoveto{\pgfqpoint{6.523228in}{3.803386in}}%
\pgfpathlineto{\pgfqpoint{6.385686in}{3.821525in}}%
\pgfpathlineto{\pgfqpoint{6.317298in}{3.965924in}}%
\pgfpathclose%
\pgfusepath{fill}%
\end{pgfscope}%
\begin{pgfscope}%
\pgfpathrectangle{\pgfqpoint{0.680860in}{0.078740in}}{\pgfqpoint{7.842520in}{7.842520in}}%
\pgfusepath{clip}%
\pgfsetbuttcap%
\pgfsetroundjoin%
\definecolor{currentfill}{rgb}{0.182256,0.426184,0.557120}%
\pgfsetfillcolor{currentfill}%
\pgfsetlinewidth{0.000000pt}%
\definecolor{currentstroke}{rgb}{0.278826,0.175490,0.483397}%
\pgfsetstrokecolor{currentstroke}%
\pgfsetdash{}{0pt}%
\pgfpathmoveto{\pgfqpoint{4.042375in}{4.441179in}}%
\pgfpathlineto{\pgfqpoint{3.829001in}{4.462220in}}%
\pgfpathlineto{\pgfqpoint{3.961933in}{4.462226in}}%
\pgfpathclose%
\pgfusepath{fill}%
\end{pgfscope}%
\begin{pgfscope}%
\pgfpathrectangle{\pgfqpoint{0.680860in}{0.078740in}}{\pgfqpoint{7.842520in}{7.842520in}}%
\pgfusepath{clip}%
\pgfsetbuttcap%
\pgfsetroundjoin%
\definecolor{currentfill}{rgb}{0.188923,0.410910,0.556326}%
\pgfsetfillcolor{currentfill}%
\pgfsetlinewidth{0.000000pt}%
\definecolor{currentstroke}{rgb}{0.278012,0.180367,0.486697}%
\pgfsetstrokecolor{currentstroke}%
\pgfsetdash{}{0pt}%
\pgfpathmoveto{\pgfqpoint{4.656707in}{4.396514in}}%
\pgfpathlineto{\pgfqpoint{4.734429in}{4.340057in}}%
\pgfpathlineto{\pgfqpoint{4.522276in}{4.398058in}}%
\pgfpathclose%
\pgfusepath{fill}%
\end{pgfscope}%
\begin{pgfscope}%
\pgfpathrectangle{\pgfqpoint{0.680860in}{0.078740in}}{\pgfqpoint{7.842520in}{7.842520in}}%
\pgfusepath{clip}%
\pgfsetbuttcap%
\pgfsetroundjoin%
\definecolor{currentfill}{rgb}{0.271828,0.209303,0.504434}%
\pgfsetfillcolor{currentfill}%
\pgfsetlinewidth{0.000000pt}%
\definecolor{currentstroke}{rgb}{0.277134,0.185228,0.489898}%
\pgfsetstrokecolor{currentstroke}%
\pgfsetdash{}{0pt}%
\pgfpathmoveto{\pgfqpoint{6.939090in}{3.752670in}}%
\pgfpathlineto{\pgfqpoint{7.002988in}{3.573875in}}%
\pgfpathlineto{\pgfqpoint{6.864774in}{3.597939in}}%
\pgfpathclose%
\pgfusepath{fill}%
\end{pgfscope}%
\begin{pgfscope}%
\pgfpathrectangle{\pgfqpoint{0.680860in}{0.078740in}}{\pgfqpoint{7.842520in}{7.842520in}}%
\pgfusepath{clip}%
\pgfsetbuttcap%
\pgfsetroundjoin%
\definecolor{currentfill}{rgb}{0.229739,0.322361,0.545706}%
\pgfsetfillcolor{currentfill}%
\pgfsetlinewidth{0.000000pt}%
\definecolor{currentstroke}{rgb}{0.276194,0.190074,0.493001}%
\pgfsetstrokecolor{currentstroke}%
\pgfsetdash{}{0pt}%
\pgfpathmoveto{\pgfqpoint{5.971448in}{4.115380in}}%
\pgfpathlineto{\pgfqpoint{6.042265in}{3.990731in}}%
\pgfpathlineto{\pgfqpoint{5.905571in}{4.004277in}}%
\pgfpathclose%
\pgfusepath{fill}%
\end{pgfscope}%
\begin{pgfscope}%
\pgfpathrectangle{\pgfqpoint{0.680860in}{0.078740in}}{\pgfqpoint{7.842520in}{7.842520in}}%
\pgfusepath{clip}%
\pgfsetbuttcap%
\pgfsetroundjoin%
\definecolor{currentfill}{rgb}{0.183898,0.422383,0.556944}%
\pgfsetfillcolor{currentfill}%
\pgfsetlinewidth{0.000000pt}%
\definecolor{currentstroke}{rgb}{0.275191,0.194905,0.496005}%
\pgfsetstrokecolor{currentstroke}%
\pgfsetdash{}{0pt}%
\pgfpathmoveto{\pgfqpoint{4.309226in}{4.439123in}}%
\pgfpathlineto{\pgfqpoint{4.388401in}{4.400722in}}%
\pgfpathlineto{\pgfqpoint{4.175526in}{4.439574in}}%
\pgfpathclose%
\pgfusepath{fill}%
\end{pgfscope}%
\begin{pgfscope}%
\pgfpathrectangle{\pgfqpoint{0.680860in}{0.078740in}}{\pgfqpoint{7.842520in}{7.842520in}}%
\pgfusepath{clip}%
\pgfsetbuttcap%
\pgfsetroundjoin%
\definecolor{currentfill}{rgb}{0.185556,0.418570,0.556753}%
\pgfsetfillcolor{currentfill}%
\pgfsetlinewidth{0.000000pt}%
\definecolor{currentstroke}{rgb}{0.274128,0.199721,0.498911}%
\pgfsetstrokecolor{currentstroke}%
\pgfsetdash{}{0pt}%
\pgfpathmoveto{\pgfqpoint{4.522276in}{4.398058in}}%
\pgfpathlineto{\pgfqpoint{4.388401in}{4.400722in}}%
\pgfpathlineto{\pgfqpoint{4.309226in}{4.439123in}}%
\pgfpathclose%
\pgfusepath{fill}%
\end{pgfscope}%
\begin{pgfscope}%
\pgfpathrectangle{\pgfqpoint{0.680860in}{0.078740in}}{\pgfqpoint{7.842520in}{7.842520in}}%
\pgfusepath{clip}%
\pgfsetbuttcap%
\pgfsetroundjoin%
\definecolor{currentfill}{rgb}{0.182256,0.426184,0.557120}%
\pgfsetfillcolor{currentfill}%
\pgfsetlinewidth{0.000000pt}%
\definecolor{currentstroke}{rgb}{0.273006,0.204520,0.501721}%
\pgfsetstrokecolor{currentstroke}%
\pgfsetdash{}{0pt}%
\pgfpathmoveto{\pgfqpoint{3.054466in}{4.435002in}}%
\pgfpathlineto{\pgfqpoint{3.137515in}{4.459090in}}%
\pgfpathlineto{\pgfqpoint{2.923425in}{4.435607in}}%
\pgfpathclose%
\pgfusepath{fill}%
\end{pgfscope}%
\begin{pgfscope}%
\pgfpathrectangle{\pgfqpoint{0.680860in}{0.078740in}}{\pgfqpoint{7.842520in}{7.842520in}}%
\pgfusepath{clip}%
\pgfsetbuttcap%
\pgfsetroundjoin%
\definecolor{currentfill}{rgb}{0.283229,0.120777,0.440584}%
\pgfsetfillcolor{currentfill}%
\pgfsetlinewidth{0.000000pt}%
\definecolor{currentstroke}{rgb}{0.271828,0.209303,0.504434}%
\pgfsetstrokecolor{currentstroke}%
\pgfsetdash{}{0pt}%
\pgfpathmoveto{\pgfqpoint{7.420741in}{3.504180in}}%
\pgfpathlineto{\pgfqpoint{7.619678in}{3.266359in}}%
\pgfpathlineto{\pgfqpoint{7.480526in}{3.297643in}}%
\pgfpathclose%
\pgfusepath{fill}%
\end{pgfscope}%
\begin{pgfscope}%
\pgfpathrectangle{\pgfqpoint{0.680860in}{0.078740in}}{\pgfqpoint{7.842520in}{7.842520in}}%
\pgfusepath{clip}%
\pgfsetbuttcap%
\pgfsetroundjoin%
\definecolor{currentfill}{rgb}{0.208623,0.367752,0.552675}%
\pgfsetfillcolor{currentfill}%
\pgfsetlinewidth{0.000000pt}%
\definecolor{currentstroke}{rgb}{0.270595,0.214069,0.507052}%
\pgfsetstrokecolor{currentstroke}%
\pgfsetdash{}{0pt}%
\pgfpathmoveto{\pgfqpoint{5.487351in}{4.239628in}}%
\pgfpathlineto{\pgfqpoint{5.425281in}{4.150671in}}%
\pgfpathlineto{\pgfqpoint{5.351061in}{4.244517in}}%
\pgfpathclose%
\pgfusepath{fill}%
\end{pgfscope}%
\begin{pgfscope}%
\pgfpathrectangle{\pgfqpoint{0.680860in}{0.078740in}}{\pgfqpoint{7.842520in}{7.842520in}}%
\pgfusepath{clip}%
\pgfsetbuttcap%
\pgfsetroundjoin%
\definecolor{currentfill}{rgb}{0.212395,0.359683,0.551710}%
\pgfsetfillcolor{currentfill}%
\pgfsetlinewidth{0.000000pt}%
\definecolor{currentstroke}{rgb}{0.269308,0.218818,0.509577}%
\pgfsetstrokecolor{currentstroke}%
\pgfsetdash{}{0pt}%
\pgfpathmoveto{\pgfqpoint{5.487351in}{4.239628in}}%
\pgfpathlineto{\pgfqpoint{5.697250in}{4.131229in}}%
\pgfpathlineto{\pgfqpoint{5.560991in}{4.140518in}}%
\pgfpathclose%
\pgfusepath{fill}%
\end{pgfscope}%
\begin{pgfscope}%
\pgfpathrectangle{\pgfqpoint{0.680860in}{0.078740in}}{\pgfqpoint{7.842520in}{7.842520in}}%
\pgfusepath{clip}%
\pgfsetbuttcap%
\pgfsetroundjoin%
\definecolor{currentfill}{rgb}{0.190631,0.407061,0.556089}%
\pgfsetfillcolor{currentfill}%
\pgfsetlinewidth{0.000000pt}%
\definecolor{currentstroke}{rgb}{0.267968,0.223549,0.512008}%
\pgfsetstrokecolor{currentstroke}%
\pgfsetdash{}{0pt}%
\pgfpathmoveto{\pgfqpoint{4.734429in}{4.340057in}}%
\pgfpathlineto{\pgfqpoint{4.791703in}{4.396133in}}%
\pgfpathlineto{\pgfqpoint{4.868983in}{4.335656in}}%
\pgfpathclose%
\pgfusepath{fill}%
\end{pgfscope}%
\begin{pgfscope}%
\pgfpathrectangle{\pgfqpoint{0.680860in}{0.078740in}}{\pgfqpoint{7.842520in}{7.842520in}}%
\pgfusepath{clip}%
\pgfsetbuttcap%
\pgfsetroundjoin%
\definecolor{currentfill}{rgb}{0.180629,0.429975,0.557282}%
\pgfsetfillcolor{currentfill}%
\pgfsetlinewidth{0.000000pt}%
\definecolor{currentstroke}{rgb}{0.266580,0.228262,0.514349}%
\pgfsetstrokecolor{currentstroke}%
\pgfsetdash{}{0pt}%
\pgfpathmoveto{\pgfqpoint{4.095418in}{4.463445in}}%
\pgfpathlineto{\pgfqpoint{4.175526in}{4.439574in}}%
\pgfpathlineto{\pgfqpoint{4.042375in}{4.441179in}}%
\pgfpathclose%
\pgfusepath{fill}%
\end{pgfscope}%
\begin{pgfscope}%
\pgfpathrectangle{\pgfqpoint{0.680860in}{0.078740in}}{\pgfqpoint{7.842520in}{7.842520in}}%
\pgfusepath{clip}%
\pgfsetbuttcap%
\pgfsetroundjoin%
\definecolor{currentfill}{rgb}{0.179019,0.433756,0.557430}%
\pgfsetfillcolor{currentfill}%
\pgfsetlinewidth{0.000000pt}%
\definecolor{currentstroke}{rgb}{0.265145,0.232956,0.516599}%
\pgfsetstrokecolor{currentstroke}%
\pgfsetdash{}{0pt}%
\pgfpathmoveto{\pgfqpoint{3.268849in}{4.458377in}}%
\pgfpathlineto{\pgfqpoint{3.400714in}{4.458847in}}%
\pgfpathlineto{\pgfqpoint{3.482933in}{4.468046in}}%
\pgfpathclose%
\pgfusepath{fill}%
\end{pgfscope}%
\begin{pgfscope}%
\pgfpathrectangle{\pgfqpoint{0.680860in}{0.078740in}}{\pgfqpoint{7.842520in}{7.842520in}}%
\pgfusepath{clip}%
\pgfsetbuttcap%
\pgfsetroundjoin%
\definecolor{currentfill}{rgb}{0.179019,0.433756,0.557430}%
\pgfsetfillcolor{currentfill}%
\pgfsetlinewidth{0.000000pt}%
\definecolor{currentstroke}{rgb}{0.263663,0.237631,0.518762}%
\pgfsetstrokecolor{currentstroke}%
\pgfsetdash{}{0pt}%
\pgfpathmoveto{\pgfqpoint{3.268849in}{4.458377in}}%
\pgfpathlineto{\pgfqpoint{3.137515in}{4.459090in}}%
\pgfpathlineto{\pgfqpoint{3.186035in}{4.435586in}}%
\pgfpathclose%
\pgfusepath{fill}%
\end{pgfscope}%
\begin{pgfscope}%
\pgfpathrectangle{\pgfqpoint{0.680860in}{0.078740in}}{\pgfqpoint{7.842520in}{7.842520in}}%
\pgfusepath{clip}%
\pgfsetbuttcap%
\pgfsetroundjoin%
\definecolor{currentfill}{rgb}{0.220057,0.343307,0.549413}%
\pgfsetfillcolor{currentfill}%
\pgfsetlinewidth{0.000000pt}%
\definecolor{currentstroke}{rgb}{0.262138,0.242286,0.520837}%
\pgfsetstrokecolor{currentstroke}%
\pgfsetdash{}{0pt}%
\pgfpathmoveto{\pgfqpoint{5.905571in}{4.004277in}}%
\pgfpathlineto{\pgfqpoint{5.834066in}{4.122838in}}%
\pgfpathlineto{\pgfqpoint{5.971448in}{4.115380in}}%
\pgfpathclose%
\pgfusepath{fill}%
\end{pgfscope}%
\begin{pgfscope}%
\pgfpathrectangle{\pgfqpoint{0.680860in}{0.078740in}}{\pgfqpoint{7.842520in}{7.842520in}}%
\pgfusepath{clip}%
\pgfsetbuttcap%
\pgfsetroundjoin%
\definecolor{currentfill}{rgb}{0.270595,0.214069,0.507052}%
\pgfsetfillcolor{currentfill}%
\pgfsetlinewidth{0.000000pt}%
\definecolor{currentstroke}{rgb}{0.260571,0.246922,0.522828}%
\pgfsetstrokecolor{currentstroke}%
\pgfsetdash{}{0pt}%
\pgfpathmoveto{\pgfqpoint{7.141717in}{3.550216in}}%
\pgfpathlineto{\pgfqpoint{7.002988in}{3.573875in}}%
\pgfpathlineto{\pgfqpoint{6.939090in}{3.752670in}}%
\pgfpathclose%
\pgfusepath{fill}%
\end{pgfscope}%
\begin{pgfscope}%
\pgfpathrectangle{\pgfqpoint{0.680860in}{0.078740in}}{\pgfqpoint{7.842520in}{7.842520in}}%
\pgfusepath{clip}%
\pgfsetbuttcap%
\pgfsetroundjoin%
\definecolor{currentfill}{rgb}{0.281412,0.155834,0.469201}%
\pgfsetfillcolor{currentfill}%
\pgfsetlinewidth{0.000000pt}%
\definecolor{currentstroke}{rgb}{0.258965,0.251537,0.524736}%
\pgfsetstrokecolor{currentstroke}%
\pgfsetdash{}{0pt}%
\pgfpathmoveto{\pgfqpoint{7.280966in}{3.526978in}}%
\pgfpathlineto{\pgfqpoint{7.420741in}{3.504180in}}%
\pgfpathlineto{\pgfqpoint{7.480526in}{3.297643in}}%
\pgfpathclose%
\pgfusepath{fill}%
\end{pgfscope}%
\begin{pgfscope}%
\pgfpathrectangle{\pgfqpoint{0.680860in}{0.078740in}}{\pgfqpoint{7.842520in}{7.842520in}}%
\pgfusepath{clip}%
\pgfsetbuttcap%
\pgfsetroundjoin%
\definecolor{currentfill}{rgb}{0.177423,0.437527,0.557565}%
\pgfsetfillcolor{currentfill}%
\pgfsetlinewidth{0.000000pt}%
\definecolor{currentstroke}{rgb}{0.257322,0.256130,0.526563}%
\pgfsetstrokecolor{currentstroke}%
\pgfsetdash{}{0pt}%
\pgfpathmoveto{\pgfqpoint{3.961933in}{4.462226in}}%
\pgfpathlineto{\pgfqpoint{4.095418in}{4.463445in}}%
\pgfpathlineto{\pgfqpoint{4.042375in}{4.441179in}}%
\pgfpathclose%
\pgfusepath{fill}%
\end{pgfscope}%
\begin{pgfscope}%
\pgfpathrectangle{\pgfqpoint{0.680860in}{0.078740in}}{\pgfqpoint{7.842520in}{7.842520in}}%
\pgfusepath{clip}%
\pgfsetbuttcap%
\pgfsetroundjoin%
\definecolor{currentfill}{rgb}{0.227802,0.326594,0.546532}%
\pgfsetfillcolor{currentfill}%
\pgfsetlinewidth{0.000000pt}%
\definecolor{currentstroke}{rgb}{0.255645,0.260703,0.528312}%
\pgfsetstrokecolor{currentstroke}%
\pgfsetdash{}{0pt}%
\pgfpathmoveto{\pgfqpoint{5.971448in}{4.115380in}}%
\pgfpathlineto{\pgfqpoint{6.179505in}{3.977936in}}%
\pgfpathlineto{\pgfqpoint{6.042265in}{3.990731in}}%
\pgfpathclose%
\pgfusepath{fill}%
\end{pgfscope}%
\begin{pgfscope}%
\pgfpathrectangle{\pgfqpoint{0.680860in}{0.078740in}}{\pgfqpoint{7.842520in}{7.842520in}}%
\pgfusepath{clip}%
\pgfsetbuttcap%
\pgfsetroundjoin%
\definecolor{currentfill}{rgb}{0.185556,0.418570,0.556753}%
\pgfsetfillcolor{currentfill}%
\pgfsetlinewidth{0.000000pt}%
\definecolor{currentstroke}{rgb}{0.253935,0.265254,0.529983}%
\pgfsetstrokecolor{currentstroke}%
\pgfsetdash{}{0pt}%
\pgfpathmoveto{\pgfqpoint{4.656707in}{4.396514in}}%
\pgfpathlineto{\pgfqpoint{4.791703in}{4.396133in}}%
\pgfpathlineto{\pgfqpoint{4.734429in}{4.340057in}}%
\pgfpathclose%
\pgfusepath{fill}%
\end{pgfscope}%
\begin{pgfscope}%
\pgfpathrectangle{\pgfqpoint{0.680860in}{0.078740in}}{\pgfqpoint{7.842520in}{7.842520in}}%
\pgfusepath{clip}%
\pgfsetbuttcap%
\pgfsetroundjoin%
\definecolor{currentfill}{rgb}{0.263663,0.237631,0.518762}%
\pgfsetfillcolor{currentfill}%
\pgfsetlinewidth{0.000000pt}%
\definecolor{currentstroke}{rgb}{0.252194,0.269783,0.531579}%
\pgfsetstrokecolor{currentstroke}%
\pgfsetdash{}{0pt}%
\pgfpathmoveto{\pgfqpoint{6.864774in}{3.597939in}}%
\pgfpathlineto{\pgfqpoint{6.799922in}{3.768934in}}%
\pgfpathlineto{\pgfqpoint{6.939090in}{3.752670in}}%
\pgfpathclose%
\pgfusepath{fill}%
\end{pgfscope}%
\begin{pgfscope}%
\pgfpathrectangle{\pgfqpoint{0.680860in}{0.078740in}}{\pgfqpoint{7.842520in}{7.842520in}}%
\pgfusepath{clip}%
\pgfsetbuttcap%
\pgfsetroundjoin%
\definecolor{currentfill}{rgb}{0.179019,0.433756,0.557430}%
\pgfsetfillcolor{currentfill}%
\pgfsetlinewidth{0.000000pt}%
\definecolor{currentstroke}{rgb}{0.250425,0.274290,0.533103}%
\pgfsetstrokecolor{currentstroke}%
\pgfsetdash{}{0pt}%
\pgfpathmoveto{\pgfqpoint{3.186035in}{4.435586in}}%
\pgfpathlineto{\pgfqpoint{3.137515in}{4.459090in}}%
\pgfpathlineto{\pgfqpoint{3.054466in}{4.435002in}}%
\pgfpathclose%
\pgfusepath{fill}%
\end{pgfscope}%
\begin{pgfscope}%
\pgfpathrectangle{\pgfqpoint{0.680860in}{0.078740in}}{\pgfqpoint{7.842520in}{7.842520in}}%
\pgfusepath{clip}%
\pgfsetbuttcap%
\pgfsetroundjoin%
\definecolor{currentfill}{rgb}{0.248629,0.278775,0.534556}%
\pgfsetfillcolor{currentfill}%
\pgfsetlinewidth{0.000000pt}%
\definecolor{currentstroke}{rgb}{0.248629,0.278775,0.534556}%
\pgfsetstrokecolor{currentstroke}%
\pgfsetdash{}{0pt}%
\pgfpathmoveto{\pgfqpoint{6.455653in}{3.954725in}}%
\pgfpathlineto{\pgfqpoint{6.661304in}{3.785848in}}%
\pgfpathlineto{\pgfqpoint{6.523228in}{3.803386in}}%
\pgfpathclose%
\pgfusepath{fill}%
\end{pgfscope}%
\begin{pgfscope}%
\pgfpathrectangle{\pgfqpoint{0.680860in}{0.078740in}}{\pgfqpoint{7.842520in}{7.842520in}}%
\pgfusepath{clip}%
\pgfsetbuttcap%
\pgfsetroundjoin%
\definecolor{currentfill}{rgb}{0.175841,0.441290,0.557685}%
\pgfsetfillcolor{currentfill}%
\pgfsetlinewidth{0.000000pt}%
\definecolor{currentstroke}{rgb}{0.246811,0.283237,0.535941}%
\pgfsetstrokecolor{currentstroke}%
\pgfsetdash{}{0pt}%
\pgfpathmoveto{\pgfqpoint{3.829001in}{4.462220in}}%
\pgfpathlineto{\pgfqpoint{3.615073in}{4.467941in}}%
\pgfpathlineto{\pgfqpoint{3.747755in}{4.469043in}}%
\pgfpathclose%
\pgfusepath{fill}%
\end{pgfscope}%
\begin{pgfscope}%
\pgfpathrectangle{\pgfqpoint{0.680860in}{0.078740in}}{\pgfqpoint{7.842520in}{7.842520in}}%
\pgfusepath{clip}%
\pgfsetbuttcap%
\pgfsetroundjoin%
\definecolor{currentfill}{rgb}{0.194100,0.399323,0.555565}%
\pgfsetfillcolor{currentfill}%
\pgfsetlinewidth{0.000000pt}%
\definecolor{currentstroke}{rgb}{0.244972,0.287675,0.537260}%
\pgfsetstrokecolor{currentstroke}%
\pgfsetdash{}{0pt}%
\pgfpathmoveto{\pgfqpoint{5.139771in}{4.330077in}}%
\pgfpathlineto{\pgfqpoint{5.215336in}{4.250420in}}%
\pgfpathlineto{\pgfqpoint{5.004093in}{4.332316in}}%
\pgfpathclose%
\pgfusepath{fill}%
\end{pgfscope}%
\begin{pgfscope}%
\pgfpathrectangle{\pgfqpoint{0.680860in}{0.078740in}}{\pgfqpoint{7.842520in}{7.842520in}}%
\pgfusepath{clip}%
\pgfsetbuttcap%
\pgfsetroundjoin%
\definecolor{currentfill}{rgb}{0.175841,0.441290,0.557685}%
\pgfsetfillcolor{currentfill}%
\pgfsetlinewidth{0.000000pt}%
\definecolor{currentstroke}{rgb}{0.243113,0.292092,0.538516}%
\pgfsetstrokecolor{currentstroke}%
\pgfsetdash{}{0pt}%
\pgfpathmoveto{\pgfqpoint{3.615073in}{4.467941in}}%
\pgfpathlineto{\pgfqpoint{3.482933in}{4.468046in}}%
\pgfpathlineto{\pgfqpoint{3.533119in}{4.460547in}}%
\pgfpathclose%
\pgfusepath{fill}%
\end{pgfscope}%
\begin{pgfscope}%
\pgfpathrectangle{\pgfqpoint{0.680860in}{0.078740in}}{\pgfqpoint{7.842520in}{7.842520in}}%
\pgfusepath{clip}%
\pgfsetbuttcap%
\pgfsetroundjoin%
\definecolor{currentfill}{rgb}{0.187231,0.414746,0.556547}%
\pgfsetfillcolor{currentfill}%
\pgfsetlinewidth{0.000000pt}%
\definecolor{currentstroke}{rgb}{0.241237,0.296485,0.539709}%
\pgfsetstrokecolor{currentstroke}%
\pgfsetdash{}{0pt}%
\pgfpathmoveto{\pgfqpoint{4.868983in}{4.335656in}}%
\pgfpathlineto{\pgfqpoint{4.791703in}{4.396133in}}%
\pgfpathlineto{\pgfqpoint{5.004093in}{4.332316in}}%
\pgfpathclose%
\pgfusepath{fill}%
\end{pgfscope}%
\begin{pgfscope}%
\pgfpathrectangle{\pgfqpoint{0.680860in}{0.078740in}}{\pgfqpoint{7.842520in}{7.842520in}}%
\pgfusepath{clip}%
\pgfsetbuttcap%
\pgfsetroundjoin%
\definecolor{currentfill}{rgb}{0.195860,0.395433,0.555276}%
\pgfsetfillcolor{currentfill}%
\pgfsetlinewidth{0.000000pt}%
\definecolor{currentstroke}{rgb}{0.239346,0.300855,0.540844}%
\pgfsetstrokecolor{currentstroke}%
\pgfsetdash{}{0pt}%
\pgfpathmoveto{\pgfqpoint{5.276024in}{4.328982in}}%
\pgfpathlineto{\pgfqpoint{5.351061in}{4.244517in}}%
\pgfpathlineto{\pgfqpoint{5.215336in}{4.250420in}}%
\pgfpathclose%
\pgfusepath{fill}%
\end{pgfscope}%
\begin{pgfscope}%
\pgfpathrectangle{\pgfqpoint{0.680860in}{0.078740in}}{\pgfqpoint{7.842520in}{7.842520in}}%
\pgfusepath{clip}%
\pgfsetbuttcap%
\pgfsetroundjoin%
\definecolor{currentfill}{rgb}{0.179019,0.433756,0.557430}%
\pgfsetfillcolor{currentfill}%
\pgfsetlinewidth{0.000000pt}%
\definecolor{currentstroke}{rgb}{0.237441,0.305202,0.541921}%
\pgfsetstrokecolor{currentstroke}%
\pgfsetdash{}{0pt}%
\pgfpathmoveto{\pgfqpoint{4.309226in}{4.439123in}}%
\pgfpathlineto{\pgfqpoint{4.443486in}{4.439871in}}%
\pgfpathlineto{\pgfqpoint{4.522276in}{4.398058in}}%
\pgfpathclose%
\pgfusepath{fill}%
\end{pgfscope}%
\begin{pgfscope}%
\pgfpathrectangle{\pgfqpoint{0.680860in}{0.078740in}}{\pgfqpoint{7.842520in}{7.842520in}}%
\pgfusepath{clip}%
\pgfsetbuttcap%
\pgfsetroundjoin%
\definecolor{currentfill}{rgb}{0.239346,0.300855,0.540844}%
\pgfsetfillcolor{currentfill}%
\pgfsetlinewidth{0.000000pt}%
\definecolor{currentstroke}{rgb}{0.235526,0.309527,0.542944}%
\pgfsetstrokecolor{currentstroke}%
\pgfsetdash{}{0pt}%
\pgfpathmoveto{\pgfqpoint{6.317298in}{3.965924in}}%
\pgfpathlineto{\pgfqpoint{6.455653in}{3.954725in}}%
\pgfpathlineto{\pgfqpoint{6.523228in}{3.803386in}}%
\pgfpathclose%
\pgfusepath{fill}%
\end{pgfscope}%
\begin{pgfscope}%
\pgfpathrectangle{\pgfqpoint{0.680860in}{0.078740in}}{\pgfqpoint{7.842520in}{7.842520in}}%
\pgfusepath{clip}%
\pgfsetbuttcap%
\pgfsetroundjoin%
\definecolor{currentfill}{rgb}{0.179019,0.433756,0.557430}%
\pgfsetfillcolor{currentfill}%
\pgfsetlinewidth{0.000000pt}%
\definecolor{currentstroke}{rgb}{0.233603,0.313828,0.543914}%
\pgfsetstrokecolor{currentstroke}%
\pgfsetdash{}{0pt}%
\pgfpathmoveto{\pgfqpoint{2.971142in}{4.398922in}}%
\pgfpathlineto{\pgfqpoint{3.054466in}{4.435002in}}%
\pgfpathlineto{\pgfqpoint{2.923425in}{4.435607in}}%
\pgfpathclose%
\pgfusepath{fill}%
\end{pgfscope}%
\begin{pgfscope}%
\pgfpathrectangle{\pgfqpoint{0.680860in}{0.078740in}}{\pgfqpoint{7.842520in}{7.842520in}}%
\pgfusepath{clip}%
\pgfsetbuttcap%
\pgfsetroundjoin%
\definecolor{currentfill}{rgb}{0.174274,0.445044,0.557792}%
\pgfsetfillcolor{currentfill}%
\pgfsetlinewidth{0.000000pt}%
\definecolor{currentstroke}{rgb}{0.231674,0.318106,0.544834}%
\pgfsetstrokecolor{currentstroke}%
\pgfsetdash{}{0pt}%
\pgfpathmoveto{\pgfqpoint{3.533119in}{4.460547in}}%
\pgfpathlineto{\pgfqpoint{3.482933in}{4.468046in}}%
\pgfpathlineto{\pgfqpoint{3.400714in}{4.458847in}}%
\pgfpathclose%
\pgfusepath{fill}%
\end{pgfscope}%
\begin{pgfscope}%
\pgfpathrectangle{\pgfqpoint{0.680860in}{0.078740in}}{\pgfqpoint{7.842520in}{7.842520in}}%
\pgfusepath{clip}%
\pgfsetbuttcap%
\pgfsetroundjoin%
\definecolor{currentfill}{rgb}{0.174274,0.445044,0.557792}%
\pgfsetfillcolor{currentfill}%
\pgfsetlinewidth{0.000000pt}%
\definecolor{currentstroke}{rgb}{0.229739,0.322361,0.545706}%
\pgfsetstrokecolor{currentstroke}%
\pgfsetdash{}{0pt}%
\pgfpathmoveto{\pgfqpoint{3.829001in}{4.462220in}}%
\pgfpathlineto{\pgfqpoint{3.880989in}{4.471399in}}%
\pgfpathlineto{\pgfqpoint{3.961933in}{4.462226in}}%
\pgfpathclose%
\pgfusepath{fill}%
\end{pgfscope}%
\begin{pgfscope}%
\pgfpathrectangle{\pgfqpoint{0.680860in}{0.078740in}}{\pgfqpoint{7.842520in}{7.842520in}}%
\pgfusepath{clip}%
\pgfsetbuttcap%
\pgfsetroundjoin%
\definecolor{currentfill}{rgb}{0.175841,0.441290,0.557685}%
\pgfsetfillcolor{currentfill}%
\pgfsetlinewidth{0.000000pt}%
\definecolor{currentstroke}{rgb}{0.227802,0.326594,0.546532}%
\pgfsetstrokecolor{currentstroke}%
\pgfsetdash{}{0pt}%
\pgfpathmoveto{\pgfqpoint{3.268849in}{4.458377in}}%
\pgfpathlineto{\pgfqpoint{3.186035in}{4.435586in}}%
\pgfpathlineto{\pgfqpoint{3.400714in}{4.458847in}}%
\pgfpathclose%
\pgfusepath{fill}%
\end{pgfscope}%
\begin{pgfscope}%
\pgfpathrectangle{\pgfqpoint{0.680860in}{0.078740in}}{\pgfqpoint{7.842520in}{7.842520in}}%
\pgfusepath{clip}%
\pgfsetbuttcap%
\pgfsetroundjoin%
\definecolor{currentfill}{rgb}{0.283072,0.130895,0.449241}%
\pgfsetfillcolor{currentfill}%
\pgfsetlinewidth{0.000000pt}%
\definecolor{currentstroke}{rgb}{0.225863,0.330805,0.547314}%
\pgfsetstrokecolor{currentstroke}%
\pgfsetdash{}{0pt}%
\pgfpathmoveto{\pgfqpoint{7.701899in}{3.459977in}}%
\pgfpathlineto{\pgfqpoint{7.759317in}{3.235245in}}%
\pgfpathlineto{\pgfqpoint{7.619678in}{3.266359in}}%
\pgfpathclose%
\pgfusepath{fill}%
\end{pgfscope}%
\begin{pgfscope}%
\pgfpathrectangle{\pgfqpoint{0.680860in}{0.078740in}}{\pgfqpoint{7.842520in}{7.842520in}}%
\pgfusepath{clip}%
\pgfsetbuttcap%
\pgfsetroundjoin%
\definecolor{currentfill}{rgb}{0.206756,0.371758,0.553117}%
\pgfsetfillcolor{currentfill}%
\pgfsetlinewidth{0.000000pt}%
\definecolor{currentstroke}{rgb}{0.223925,0.334994,0.548053}%
\pgfsetstrokecolor{currentstroke}%
\pgfsetdash{}{0pt}%
\pgfpathmoveto{\pgfqpoint{5.834066in}{4.122838in}}%
\pgfpathlineto{\pgfqpoint{5.697250in}{4.131229in}}%
\pgfpathlineto{\pgfqpoint{5.624214in}{4.235792in}}%
\pgfpathclose%
\pgfusepath{fill}%
\end{pgfscope}%
\begin{pgfscope}%
\pgfpathrectangle{\pgfqpoint{0.680860in}{0.078740in}}{\pgfqpoint{7.842520in}{7.842520in}}%
\pgfusepath{clip}%
\pgfsetbuttcap%
\pgfsetroundjoin%
\definecolor{currentfill}{rgb}{0.179019,0.433756,0.557430}%
\pgfsetfillcolor{currentfill}%
\pgfsetlinewidth{0.000000pt}%
\definecolor{currentstroke}{rgb}{0.221989,0.339161,0.548752}%
\pgfsetstrokecolor{currentstroke}%
\pgfsetdash{}{0pt}%
\pgfpathmoveto{\pgfqpoint{2.923425in}{4.435607in}}%
\pgfpathlineto{\pgfqpoint{2.839884in}{4.398663in}}%
\pgfpathlineto{\pgfqpoint{2.971142in}{4.398922in}}%
\pgfpathclose%
\pgfusepath{fill}%
\end{pgfscope}%
\begin{pgfscope}%
\pgfpathrectangle{\pgfqpoint{0.680860in}{0.078740in}}{\pgfqpoint{7.842520in}{7.842520in}}%
\pgfusepath{clip}%
\pgfsetbuttcap%
\pgfsetroundjoin%
\definecolor{currentfill}{rgb}{0.175841,0.441290,0.557685}%
\pgfsetfillcolor{currentfill}%
\pgfsetlinewidth{0.000000pt}%
\definecolor{currentstroke}{rgb}{0.220057,0.343307,0.549413}%
\pgfsetstrokecolor{currentstroke}%
\pgfsetdash{}{0pt}%
\pgfpathmoveto{\pgfqpoint{4.309226in}{4.439123in}}%
\pgfpathlineto{\pgfqpoint{4.175526in}{4.439574in}}%
\pgfpathlineto{\pgfqpoint{4.229464in}{4.465925in}}%
\pgfpathclose%
\pgfusepath{fill}%
\end{pgfscope}%
\begin{pgfscope}%
\pgfpathrectangle{\pgfqpoint{0.680860in}{0.078740in}}{\pgfqpoint{7.842520in}{7.842520in}}%
\pgfusepath{clip}%
\pgfsetbuttcap%
\pgfsetroundjoin%
\definecolor{currentfill}{rgb}{0.179019,0.433756,0.557430}%
\pgfsetfillcolor{currentfill}%
\pgfsetlinewidth{0.000000pt}%
\definecolor{currentstroke}{rgb}{0.218130,0.347432,0.550038}%
\pgfsetstrokecolor{currentstroke}%
\pgfsetdash{}{0pt}%
\pgfpathmoveto{\pgfqpoint{4.522276in}{4.398058in}}%
\pgfpathlineto{\pgfqpoint{4.578315in}{4.441865in}}%
\pgfpathlineto{\pgfqpoint{4.656707in}{4.396514in}}%
\pgfpathclose%
\pgfusepath{fill}%
\end{pgfscope}%
\begin{pgfscope}%
\pgfpathrectangle{\pgfqpoint{0.680860in}{0.078740in}}{\pgfqpoint{7.842520in}{7.842520in}}%
\pgfusepath{clip}%
\pgfsetbuttcap%
\pgfsetroundjoin%
\definecolor{currentfill}{rgb}{0.201239,0.383670,0.554294}%
\pgfsetfillcolor{currentfill}%
\pgfsetlinewidth{0.000000pt}%
\definecolor{currentstroke}{rgb}{0.216210,0.351535,0.550627}%
\pgfsetstrokecolor{currentstroke}%
\pgfsetdash{}{0pt}%
\pgfpathmoveto{\pgfqpoint{5.624214in}{4.235792in}}%
\pgfpathlineto{\pgfqpoint{5.697250in}{4.131229in}}%
\pgfpathlineto{\pgfqpoint{5.487351in}{4.239628in}}%
\pgfpathclose%
\pgfusepath{fill}%
\end{pgfscope}%
\begin{pgfscope}%
\pgfpathrectangle{\pgfqpoint{0.680860in}{0.078740in}}{\pgfqpoint{7.842520in}{7.842520in}}%
\pgfusepath{clip}%
\pgfsetbuttcap%
\pgfsetroundjoin%
\definecolor{currentfill}{rgb}{0.172719,0.448791,0.557885}%
\pgfsetfillcolor{currentfill}%
\pgfsetlinewidth{0.000000pt}%
\definecolor{currentstroke}{rgb}{0.214298,0.355619,0.551184}%
\pgfsetstrokecolor{currentstroke}%
\pgfsetdash{}{0pt}%
\pgfpathmoveto{\pgfqpoint{3.747755in}{4.469043in}}%
\pgfpathlineto{\pgfqpoint{3.880989in}{4.471399in}}%
\pgfpathlineto{\pgfqpoint{3.829001in}{4.462220in}}%
\pgfpathclose%
\pgfusepath{fill}%
\end{pgfscope}%
\begin{pgfscope}%
\pgfpathrectangle{\pgfqpoint{0.680860in}{0.078740in}}{\pgfqpoint{7.842520in}{7.842520in}}%
\pgfusepath{clip}%
\pgfsetbuttcap%
\pgfsetroundjoin%
\definecolor{currentfill}{rgb}{0.280868,0.160771,0.472899}%
\pgfsetfillcolor{currentfill}%
\pgfsetlinewidth{0.000000pt}%
\definecolor{currentstroke}{rgb}{0.212395,0.359683,0.551710}%
\pgfsetstrokecolor{currentstroke}%
\pgfsetdash{}{0pt}%
\pgfpathmoveto{\pgfqpoint{7.561050in}{3.481840in}}%
\pgfpathlineto{\pgfqpoint{7.619678in}{3.266359in}}%
\pgfpathlineto{\pgfqpoint{7.420741in}{3.504180in}}%
\pgfpathclose%
\pgfusepath{fill}%
\end{pgfscope}%
\begin{pgfscope}%
\pgfpathrectangle{\pgfqpoint{0.680860in}{0.078740in}}{\pgfqpoint{7.842520in}{7.842520in}}%
\pgfusepath{clip}%
\pgfsetbuttcap%
\pgfsetroundjoin%
\definecolor{currentfill}{rgb}{0.188923,0.410910,0.556326}%
\pgfsetfillcolor{currentfill}%
\pgfsetlinewidth{0.000000pt}%
\definecolor{currentstroke}{rgb}{0.210503,0.363727,0.552206}%
\pgfsetstrokecolor{currentstroke}%
\pgfsetdash{}{0pt}%
\pgfpathmoveto{\pgfqpoint{5.276024in}{4.328982in}}%
\pgfpathlineto{\pgfqpoint{5.215336in}{4.250420in}}%
\pgfpathlineto{\pgfqpoint{5.139771in}{4.330077in}}%
\pgfpathclose%
\pgfusepath{fill}%
\end{pgfscope}%
\begin{pgfscope}%
\pgfpathrectangle{\pgfqpoint{0.680860in}{0.078740in}}{\pgfqpoint{7.842520in}{7.842520in}}%
\pgfusepath{clip}%
\pgfsetbuttcap%
\pgfsetroundjoin%
\definecolor{currentfill}{rgb}{0.172719,0.448791,0.557885}%
\pgfsetfillcolor{currentfill}%
\pgfsetlinewidth{0.000000pt}%
\definecolor{currentstroke}{rgb}{0.208623,0.367752,0.552675}%
\pgfsetstrokecolor{currentstroke}%
\pgfsetdash{}{0pt}%
\pgfpathmoveto{\pgfqpoint{3.747755in}{4.469043in}}%
\pgfpathlineto{\pgfqpoint{3.615073in}{4.467941in}}%
\pgfpathlineto{\pgfqpoint{3.533119in}{4.460547in}}%
\pgfpathclose%
\pgfusepath{fill}%
\end{pgfscope}%
\begin{pgfscope}%
\pgfpathrectangle{\pgfqpoint{0.680860in}{0.078740in}}{\pgfqpoint{7.842520in}{7.842520in}}%
\pgfusepath{clip}%
\pgfsetbuttcap%
\pgfsetroundjoin%
\definecolor{currentfill}{rgb}{0.172719,0.448791,0.557885}%
\pgfsetfillcolor{currentfill}%
\pgfsetlinewidth{0.000000pt}%
\definecolor{currentstroke}{rgb}{0.206756,0.371758,0.553117}%
\pgfsetstrokecolor{currentstroke}%
\pgfsetdash{}{0pt}%
\pgfpathmoveto{\pgfqpoint{4.229464in}{4.465925in}}%
\pgfpathlineto{\pgfqpoint{4.175526in}{4.439574in}}%
\pgfpathlineto{\pgfqpoint{4.095418in}{4.463445in}}%
\pgfpathclose%
\pgfusepath{fill}%
\end{pgfscope}%
\begin{pgfscope}%
\pgfpathrectangle{\pgfqpoint{0.680860in}{0.078740in}}{\pgfqpoint{7.842520in}{7.842520in}}%
\pgfusepath{clip}%
\pgfsetbuttcap%
\pgfsetroundjoin%
\definecolor{currentfill}{rgb}{0.192357,0.403199,0.555836}%
\pgfsetfillcolor{currentfill}%
\pgfsetlinewidth{0.000000pt}%
\definecolor{currentstroke}{rgb}{0.204903,0.375746,0.553533}%
\pgfsetstrokecolor{currentstroke}%
\pgfsetdash{}{0pt}%
\pgfpathmoveto{\pgfqpoint{5.276024in}{4.328982in}}%
\pgfpathlineto{\pgfqpoint{5.487351in}{4.239628in}}%
\pgfpathlineto{\pgfqpoint{5.351061in}{4.244517in}}%
\pgfpathclose%
\pgfusepath{fill}%
\end{pgfscope}%
\begin{pgfscope}%
\pgfpathrectangle{\pgfqpoint{0.680860in}{0.078740in}}{\pgfqpoint{7.842520in}{7.842520in}}%
\pgfusepath{clip}%
\pgfsetbuttcap%
\pgfsetroundjoin%
\definecolor{currentfill}{rgb}{0.175841,0.441290,0.557685}%
\pgfsetfillcolor{currentfill}%
\pgfsetlinewidth{0.000000pt}%
\definecolor{currentstroke}{rgb}{0.203063,0.379716,0.553925}%
\pgfsetstrokecolor{currentstroke}%
\pgfsetdash{}{0pt}%
\pgfpathmoveto{\pgfqpoint{4.522276in}{4.398058in}}%
\pgfpathlineto{\pgfqpoint{4.443486in}{4.439871in}}%
\pgfpathlineto{\pgfqpoint{4.578315in}{4.441865in}}%
\pgfpathclose%
\pgfusepath{fill}%
\end{pgfscope}%
\begin{pgfscope}%
\pgfpathrectangle{\pgfqpoint{0.680860in}{0.078740in}}{\pgfqpoint{7.842520in}{7.842520in}}%
\pgfusepath{clip}%
\pgfsetbuttcap%
\pgfsetroundjoin%
\definecolor{currentfill}{rgb}{0.267968,0.223549,0.512008}%
\pgfsetfillcolor{currentfill}%
\pgfsetlinewidth{0.000000pt}%
\definecolor{currentstroke}{rgb}{0.201239,0.383670,0.554294}%
\pgfsetstrokecolor{currentstroke}%
\pgfsetdash{}{0pt}%
\pgfpathmoveto{\pgfqpoint{7.219105in}{3.722192in}}%
\pgfpathlineto{\pgfqpoint{7.280966in}{3.526978in}}%
\pgfpathlineto{\pgfqpoint{7.141717in}{3.550216in}}%
\pgfpathclose%
\pgfusepath{fill}%
\end{pgfscope}%
\begin{pgfscope}%
\pgfpathrectangle{\pgfqpoint{0.680860in}{0.078740in}}{\pgfqpoint{7.842520in}{7.842520in}}%
\pgfusepath{clip}%
\pgfsetbuttcap%
\pgfsetroundjoin%
\definecolor{currentfill}{rgb}{0.283072,0.130895,0.449241}%
\pgfsetfillcolor{currentfill}%
\pgfsetlinewidth{0.000000pt}%
\definecolor{currentstroke}{rgb}{0.199430,0.387607,0.554642}%
\pgfsetstrokecolor{currentstroke}%
\pgfsetdash{}{0pt}%
\pgfpathmoveto{\pgfqpoint{7.899449in}{3.204312in}}%
\pgfpathlineto{\pgfqpoint{7.759317in}{3.235245in}}%
\pgfpathlineto{\pgfqpoint{7.701899in}{3.459977in}}%
\pgfpathclose%
\pgfusepath{fill}%
\end{pgfscope}%
\begin{pgfscope}%
\pgfpathrectangle{\pgfqpoint{0.680860in}{0.078740in}}{\pgfqpoint{7.842520in}{7.842520in}}%
\pgfusepath{clip}%
\pgfsetbuttcap%
\pgfsetroundjoin%
\definecolor{currentfill}{rgb}{0.214298,0.355619,0.551184}%
\pgfsetfillcolor{currentfill}%
\pgfsetlinewidth{0.000000pt}%
\definecolor{currentstroke}{rgb}{0.197636,0.391528,0.554969}%
\pgfsetstrokecolor{currentstroke}%
\pgfsetdash{}{0pt}%
\pgfpathmoveto{\pgfqpoint{6.109407in}{4.108890in}}%
\pgfpathlineto{\pgfqpoint{6.179505in}{3.977936in}}%
\pgfpathlineto{\pgfqpoint{5.971448in}{4.115380in}}%
\pgfpathclose%
\pgfusepath{fill}%
\end{pgfscope}%
\begin{pgfscope}%
\pgfpathrectangle{\pgfqpoint{0.680860in}{0.078740in}}{\pgfqpoint{7.842520in}{7.842520in}}%
\pgfusepath{clip}%
\pgfsetbuttcap%
\pgfsetroundjoin%
\definecolor{currentfill}{rgb}{0.260571,0.246922,0.522828}%
\pgfsetfillcolor{currentfill}%
\pgfsetlinewidth{0.000000pt}%
\definecolor{currentstroke}{rgb}{0.195860,0.395433,0.555276}%
\pgfsetstrokecolor{currentstroke}%
\pgfsetdash{}{0pt}%
\pgfpathmoveto{\pgfqpoint{6.939090in}{3.752670in}}%
\pgfpathlineto{\pgfqpoint{7.078815in}{3.737080in}}%
\pgfpathlineto{\pgfqpoint{7.141717in}{3.550216in}}%
\pgfpathclose%
\pgfusepath{fill}%
\end{pgfscope}%
\begin{pgfscope}%
\pgfpathrectangle{\pgfqpoint{0.680860in}{0.078740in}}{\pgfqpoint{7.842520in}{7.842520in}}%
\pgfusepath{clip}%
\pgfsetbuttcap%
\pgfsetroundjoin%
\definecolor{currentfill}{rgb}{0.220057,0.343307,0.549413}%
\pgfsetfillcolor{currentfill}%
\pgfsetlinewidth{0.000000pt}%
\definecolor{currentstroke}{rgb}{0.194100,0.399323,0.555565}%
\pgfsetstrokecolor{currentstroke}%
\pgfsetdash{}{0pt}%
\pgfpathmoveto{\pgfqpoint{6.247951in}{4.103406in}}%
\pgfpathlineto{\pgfqpoint{6.317298in}{3.965924in}}%
\pgfpathlineto{\pgfqpoint{6.179505in}{3.977936in}}%
\pgfpathclose%
\pgfusepath{fill}%
\end{pgfscope}%
\begin{pgfscope}%
\pgfpathrectangle{\pgfqpoint{0.680860in}{0.078740in}}{\pgfqpoint{7.842520in}{7.842520in}}%
\pgfusepath{clip}%
\pgfsetbuttcap%
\pgfsetroundjoin%
\definecolor{currentfill}{rgb}{0.235526,0.309527,0.542944}%
\pgfsetfillcolor{currentfill}%
\pgfsetlinewidth{0.000000pt}%
\definecolor{currentstroke}{rgb}{0.192357,0.403199,0.555836}%
\pgfsetstrokecolor{currentstroke}%
\pgfsetdash{}{0pt}%
\pgfpathmoveto{\pgfqpoint{6.455653in}{3.954725in}}%
\pgfpathlineto{\pgfqpoint{6.594579in}{3.944369in}}%
\pgfpathlineto{\pgfqpoint{6.661304in}{3.785848in}}%
\pgfpathclose%
\pgfusepath{fill}%
\end{pgfscope}%
\begin{pgfscope}%
\pgfpathrectangle{\pgfqpoint{0.680860in}{0.078740in}}{\pgfqpoint{7.842520in}{7.842520in}}%
\pgfusepath{clip}%
\pgfsetbuttcap%
\pgfsetroundjoin%
\definecolor{currentfill}{rgb}{0.179019,0.433756,0.557430}%
\pgfsetfillcolor{currentfill}%
\pgfsetlinewidth{0.000000pt}%
\definecolor{currentstroke}{rgb}{0.190631,0.407061,0.556089}%
\pgfsetstrokecolor{currentstroke}%
\pgfsetdash{}{0pt}%
\pgfpathmoveto{\pgfqpoint{5.004093in}{4.332316in}}%
\pgfpathlineto{\pgfqpoint{4.791703in}{4.396133in}}%
\pgfpathlineto{\pgfqpoint{4.927273in}{4.396959in}}%
\pgfpathclose%
\pgfusepath{fill}%
\end{pgfscope}%
\begin{pgfscope}%
\pgfpathrectangle{\pgfqpoint{0.680860in}{0.078740in}}{\pgfqpoint{7.842520in}{7.842520in}}%
\pgfusepath{clip}%
\pgfsetbuttcap%
\pgfsetroundjoin%
\definecolor{currentfill}{rgb}{0.243113,0.292092,0.538516}%
\pgfsetfillcolor{currentfill}%
\pgfsetlinewidth{0.000000pt}%
\definecolor{currentstroke}{rgb}{0.188923,0.410910,0.556326}%
\pgfsetstrokecolor{currentstroke}%
\pgfsetdash{}{0pt}%
\pgfpathmoveto{\pgfqpoint{6.734085in}{3.934890in}}%
\pgfpathlineto{\pgfqpoint{6.799922in}{3.768934in}}%
\pgfpathlineto{\pgfqpoint{6.661304in}{3.785848in}}%
\pgfpathclose%
\pgfusepath{fill}%
\end{pgfscope}%
\begin{pgfscope}%
\pgfpathrectangle{\pgfqpoint{0.680860in}{0.078740in}}{\pgfqpoint{7.842520in}{7.842520in}}%
\pgfusepath{clip}%
\pgfsetbuttcap%
\pgfsetroundjoin%
\definecolor{currentfill}{rgb}{0.171176,0.452530,0.557965}%
\pgfsetfillcolor{currentfill}%
\pgfsetlinewidth{0.000000pt}%
\definecolor{currentstroke}{rgb}{0.187231,0.414746,0.556547}%
\pgfsetstrokecolor{currentstroke}%
\pgfsetdash{}{0pt}%
\pgfpathmoveto{\pgfqpoint{4.229464in}{4.465925in}}%
\pgfpathlineto{\pgfqpoint{4.443486in}{4.439871in}}%
\pgfpathlineto{\pgfqpoint{4.309226in}{4.439123in}}%
\pgfpathclose%
\pgfusepath{fill}%
\end{pgfscope}%
\begin{pgfscope}%
\pgfpathrectangle{\pgfqpoint{0.680860in}{0.078740in}}{\pgfqpoint{7.842520in}{7.842520in}}%
\pgfusepath{clip}%
\pgfsetbuttcap%
\pgfsetroundjoin%
\definecolor{currentfill}{rgb}{0.280255,0.165693,0.476498}%
\pgfsetfillcolor{currentfill}%
\pgfsetlinewidth{0.000000pt}%
\definecolor{currentstroke}{rgb}{0.185556,0.418570,0.556753}%
\pgfsetstrokecolor{currentstroke}%
\pgfsetdash{}{0pt}%
\pgfpathmoveto{\pgfqpoint{7.701899in}{3.459977in}}%
\pgfpathlineto{\pgfqpoint{7.619678in}{3.266359in}}%
\pgfpathlineto{\pgfqpoint{7.561050in}{3.481840in}}%
\pgfpathclose%
\pgfusepath{fill}%
\end{pgfscope}%
\begin{pgfscope}%
\pgfpathrectangle{\pgfqpoint{0.680860in}{0.078740in}}{\pgfqpoint{7.842520in}{7.842520in}}%
\pgfusepath{clip}%
\pgfsetbuttcap%
\pgfsetroundjoin%
\definecolor{currentfill}{rgb}{0.175841,0.441290,0.557685}%
\pgfsetfillcolor{currentfill}%
\pgfsetlinewidth{0.000000pt}%
\definecolor{currentstroke}{rgb}{0.183898,0.422383,0.556944}%
\pgfsetstrokecolor{currentstroke}%
\pgfsetdash{}{0pt}%
\pgfpathmoveto{\pgfqpoint{4.656707in}{4.396514in}}%
\pgfpathlineto{\pgfqpoint{4.578315in}{4.441865in}}%
\pgfpathlineto{\pgfqpoint{4.791703in}{4.396133in}}%
\pgfpathclose%
\pgfusepath{fill}%
\end{pgfscope}%
\begin{pgfscope}%
\pgfpathrectangle{\pgfqpoint{0.680860in}{0.078740in}}{\pgfqpoint{7.842520in}{7.842520in}}%
\pgfusepath{clip}%
\pgfsetbuttcap%
\pgfsetroundjoin%
\definecolor{currentfill}{rgb}{0.172719,0.448791,0.557885}%
\pgfsetfillcolor{currentfill}%
\pgfsetlinewidth{0.000000pt}%
\definecolor{currentstroke}{rgb}{0.182256,0.426184,0.557120}%
\pgfsetstrokecolor{currentstroke}%
\pgfsetdash{}{0pt}%
\pgfpathmoveto{\pgfqpoint{3.400714in}{4.458847in}}%
\pgfpathlineto{\pgfqpoint{3.186035in}{4.435586in}}%
\pgfpathlineto{\pgfqpoint{3.318141in}{4.437404in}}%
\pgfpathclose%
\pgfusepath{fill}%
\end{pgfscope}%
\begin{pgfscope}%
\pgfpathrectangle{\pgfqpoint{0.680860in}{0.078740in}}{\pgfqpoint{7.842520in}{7.842520in}}%
\pgfusepath{clip}%
\pgfsetbuttcap%
\pgfsetroundjoin%
\definecolor{currentfill}{rgb}{0.174274,0.445044,0.557792}%
\pgfsetfillcolor{currentfill}%
\pgfsetlinewidth{0.000000pt}%
\definecolor{currentstroke}{rgb}{0.180629,0.429975,0.557282}%
\pgfsetstrokecolor{currentstroke}%
\pgfsetdash{}{0pt}%
\pgfpathmoveto{\pgfqpoint{3.102931in}{4.400407in}}%
\pgfpathlineto{\pgfqpoint{3.186035in}{4.435586in}}%
\pgfpathlineto{\pgfqpoint{3.054466in}{4.435002in}}%
\pgfpathclose%
\pgfusepath{fill}%
\end{pgfscope}%
\begin{pgfscope}%
\pgfpathrectangle{\pgfqpoint{0.680860in}{0.078740in}}{\pgfqpoint{7.842520in}{7.842520in}}%
\pgfusepath{clip}%
\pgfsetbuttcap%
\pgfsetroundjoin%
\definecolor{currentfill}{rgb}{0.169646,0.456262,0.558030}%
\pgfsetfillcolor{currentfill}%
\pgfsetlinewidth{0.000000pt}%
\definecolor{currentstroke}{rgb}{0.179019,0.433756,0.557430}%
\pgfsetstrokecolor{currentstroke}%
\pgfsetdash{}{0pt}%
\pgfpathmoveto{\pgfqpoint{4.014785in}{4.475058in}}%
\pgfpathlineto{\pgfqpoint{4.095418in}{4.463445in}}%
\pgfpathlineto{\pgfqpoint{3.961933in}{4.462226in}}%
\pgfpathclose%
\pgfusepath{fill}%
\end{pgfscope}%
\begin{pgfscope}%
\pgfpathrectangle{\pgfqpoint{0.680860in}{0.078740in}}{\pgfqpoint{7.842520in}{7.842520in}}%
\pgfusepath{clip}%
\pgfsetbuttcap%
\pgfsetroundjoin%
\definecolor{currentfill}{rgb}{0.201239,0.383670,0.554294}%
\pgfsetfillcolor{currentfill}%
\pgfsetlinewidth{0.000000pt}%
\definecolor{currentstroke}{rgb}{0.177423,0.437527,0.557565}%
\pgfsetstrokecolor{currentstroke}%
\pgfsetdash{}{0pt}%
\pgfpathmoveto{\pgfqpoint{5.971448in}{4.115380in}}%
\pgfpathlineto{\pgfqpoint{5.834066in}{4.122838in}}%
\pgfpathlineto{\pgfqpoint{5.761662in}{4.233049in}}%
\pgfpathclose%
\pgfusepath{fill}%
\end{pgfscope}%
\begin{pgfscope}%
\pgfpathrectangle{\pgfqpoint{0.680860in}{0.078740in}}{\pgfqpoint{7.842520in}{7.842520in}}%
\pgfusepath{clip}%
\pgfsetbuttcap%
\pgfsetroundjoin%
\definecolor{currentfill}{rgb}{0.171176,0.452530,0.557965}%
\pgfsetfillcolor{currentfill}%
\pgfsetlinewidth{0.000000pt}%
\definecolor{currentstroke}{rgb}{0.175841,0.441290,0.557685}%
\pgfsetstrokecolor{currentstroke}%
\pgfsetdash{}{0pt}%
\pgfpathmoveto{\pgfqpoint{3.533119in}{4.460547in}}%
\pgfpathlineto{\pgfqpoint{3.400714in}{4.458847in}}%
\pgfpathlineto{\pgfqpoint{3.318141in}{4.437404in}}%
\pgfpathclose%
\pgfusepath{fill}%
\end{pgfscope}%
\begin{pgfscope}%
\pgfpathrectangle{\pgfqpoint{0.680860in}{0.078740in}}{\pgfqpoint{7.842520in}{7.842520in}}%
\pgfusepath{clip}%
\pgfsetbuttcap%
\pgfsetroundjoin%
\definecolor{currentfill}{rgb}{0.180629,0.429975,0.557282}%
\pgfsetfillcolor{currentfill}%
\pgfsetlinewidth{0.000000pt}%
\definecolor{currentstroke}{rgb}{0.174274,0.445044,0.557792}%
\pgfsetstrokecolor{currentstroke}%
\pgfsetdash{}{0pt}%
\pgfpathmoveto{\pgfqpoint{5.004093in}{4.332316in}}%
\pgfpathlineto{\pgfqpoint{5.063428in}{4.399039in}}%
\pgfpathlineto{\pgfqpoint{5.139771in}{4.330077in}}%
\pgfpathclose%
\pgfusepath{fill}%
\end{pgfscope}%
\begin{pgfscope}%
\pgfpathrectangle{\pgfqpoint{0.680860in}{0.078740in}}{\pgfqpoint{7.842520in}{7.842520in}}%
\pgfusepath{clip}%
\pgfsetbuttcap%
\pgfsetroundjoin%
\definecolor{currentfill}{rgb}{0.174274,0.445044,0.557792}%
\pgfsetfillcolor{currentfill}%
\pgfsetlinewidth{0.000000pt}%
\definecolor{currentstroke}{rgb}{0.172719,0.448791,0.557885}%
\pgfsetstrokecolor{currentstroke}%
\pgfsetdash{}{0pt}%
\pgfpathmoveto{\pgfqpoint{3.102931in}{4.400407in}}%
\pgfpathlineto{\pgfqpoint{3.054466in}{4.435002in}}%
\pgfpathlineto{\pgfqpoint{2.971142in}{4.398922in}}%
\pgfpathclose%
\pgfusepath{fill}%
\end{pgfscope}%
\begin{pgfscope}%
\pgfpathrectangle{\pgfqpoint{0.680860in}{0.078740in}}{\pgfqpoint{7.842520in}{7.842520in}}%
\pgfusepath{clip}%
\pgfsetbuttcap%
\pgfsetroundjoin%
\definecolor{currentfill}{rgb}{0.266580,0.228262,0.514349}%
\pgfsetfillcolor{currentfill}%
\pgfsetlinewidth{0.000000pt}%
\definecolor{currentstroke}{rgb}{0.171176,0.452530,0.557965}%
\pgfsetstrokecolor{currentstroke}%
\pgfsetdash{}{0pt}%
\pgfpathmoveto{\pgfqpoint{7.420741in}{3.504180in}}%
\pgfpathlineto{\pgfqpoint{7.280966in}{3.526978in}}%
\pgfpathlineto{\pgfqpoint{7.219105in}{3.722192in}}%
\pgfpathclose%
\pgfusepath{fill}%
\end{pgfscope}%
\begin{pgfscope}%
\pgfpathrectangle{\pgfqpoint{0.680860in}{0.078740in}}{\pgfqpoint{7.842520in}{7.842520in}}%
\pgfusepath{clip}%
\pgfsetbuttcap%
\pgfsetroundjoin%
\definecolor{currentfill}{rgb}{0.195860,0.395433,0.555276}%
\pgfsetfillcolor{currentfill}%
\pgfsetlinewidth{0.000000pt}%
\definecolor{currentstroke}{rgb}{0.169646,0.456262,0.558030}%
\pgfsetstrokecolor{currentstroke}%
\pgfsetdash{}{0pt}%
\pgfpathmoveto{\pgfqpoint{5.624214in}{4.235792in}}%
\pgfpathlineto{\pgfqpoint{5.761662in}{4.233049in}}%
\pgfpathlineto{\pgfqpoint{5.834066in}{4.122838in}}%
\pgfpathclose%
\pgfusepath{fill}%
\end{pgfscope}%
\begin{pgfscope}%
\pgfpathrectangle{\pgfqpoint{0.680860in}{0.078740in}}{\pgfqpoint{7.842520in}{7.842520in}}%
\pgfusepath{clip}%
\pgfsetbuttcap%
\pgfsetroundjoin%
\definecolor{currentfill}{rgb}{0.168126,0.459988,0.558082}%
\pgfsetfillcolor{currentfill}%
\pgfsetlinewidth{0.000000pt}%
\definecolor{currentstroke}{rgb}{0.168126,0.459988,0.558082}%
\pgfsetstrokecolor{currentstroke}%
\pgfsetdash{}{0pt}%
\pgfpathmoveto{\pgfqpoint{3.961933in}{4.462226in}}%
\pgfpathlineto{\pgfqpoint{3.880989in}{4.471399in}}%
\pgfpathlineto{\pgfqpoint{4.014785in}{4.475058in}}%
\pgfpathclose%
\pgfusepath{fill}%
\end{pgfscope}%
\begin{pgfscope}%
\pgfpathrectangle{\pgfqpoint{0.680860in}{0.078740in}}{\pgfqpoint{7.842520in}{7.842520in}}%
\pgfusepath{clip}%
\pgfsetbuttcap%
\pgfsetroundjoin%
\definecolor{currentfill}{rgb}{0.210503,0.363727,0.552206}%
\pgfsetfillcolor{currentfill}%
\pgfsetlinewidth{0.000000pt}%
\definecolor{currentstroke}{rgb}{0.166617,0.463708,0.558119}%
\pgfsetstrokecolor{currentstroke}%
\pgfsetdash{}{0pt}%
\pgfpathmoveto{\pgfqpoint{6.247951in}{4.103406in}}%
\pgfpathlineto{\pgfqpoint{6.179505in}{3.977936in}}%
\pgfpathlineto{\pgfqpoint{6.109407in}{4.108890in}}%
\pgfpathclose%
\pgfusepath{fill}%
\end{pgfscope}%
\begin{pgfscope}%
\pgfpathrectangle{\pgfqpoint{0.680860in}{0.078740in}}{\pgfqpoint{7.842520in}{7.842520in}}%
\pgfusepath{clip}%
\pgfsetbuttcap%
\pgfsetroundjoin%
\definecolor{currentfill}{rgb}{0.257322,0.256130,0.526563}%
\pgfsetfillcolor{currentfill}%
\pgfsetlinewidth{0.000000pt}%
\definecolor{currentstroke}{rgb}{0.165117,0.467423,0.558141}%
\pgfsetstrokecolor{currentstroke}%
\pgfsetdash{}{0pt}%
\pgfpathmoveto{\pgfqpoint{7.141717in}{3.550216in}}%
\pgfpathlineto{\pgfqpoint{7.078815in}{3.737080in}}%
\pgfpathlineto{\pgfqpoint{7.219105in}{3.722192in}}%
\pgfpathclose%
\pgfusepath{fill}%
\end{pgfscope}%
\begin{pgfscope}%
\pgfpathrectangle{\pgfqpoint{0.680860in}{0.078740in}}{\pgfqpoint{7.842520in}{7.842520in}}%
\pgfusepath{clip}%
\pgfsetbuttcap%
\pgfsetroundjoin%
\definecolor{currentfill}{rgb}{0.168126,0.459988,0.558082}%
\pgfsetfillcolor{currentfill}%
\pgfsetlinewidth{0.000000pt}%
\definecolor{currentstroke}{rgb}{0.163625,0.471133,0.558148}%
\pgfsetstrokecolor{currentstroke}%
\pgfsetdash{}{0pt}%
\pgfpathmoveto{\pgfqpoint{3.533119in}{4.460547in}}%
\pgfpathlineto{\pgfqpoint{3.666073in}{4.463523in}}%
\pgfpathlineto{\pgfqpoint{3.747755in}{4.469043in}}%
\pgfpathclose%
\pgfusepath{fill}%
\end{pgfscope}%
\begin{pgfscope}%
\pgfpathrectangle{\pgfqpoint{0.680860in}{0.078740in}}{\pgfqpoint{7.842520in}{7.842520in}}%
\pgfusepath{clip}%
\pgfsetbuttcap%
\pgfsetroundjoin%
\definecolor{currentfill}{rgb}{0.218130,0.347432,0.550038}%
\pgfsetfillcolor{currentfill}%
\pgfsetlinewidth{0.000000pt}%
\definecolor{currentstroke}{rgb}{0.162142,0.474838,0.558140}%
\pgfsetstrokecolor{currentstroke}%
\pgfsetdash{}{0pt}%
\pgfpathmoveto{\pgfqpoint{6.455653in}{3.954725in}}%
\pgfpathlineto{\pgfqpoint{6.317298in}{3.965924in}}%
\pgfpathlineto{\pgfqpoint{6.247951in}{4.103406in}}%
\pgfpathclose%
\pgfusepath{fill}%
\end{pgfscope}%
\begin{pgfscope}%
\pgfpathrectangle{\pgfqpoint{0.680860in}{0.078740in}}{\pgfqpoint{7.842520in}{7.842520in}}%
\pgfusepath{clip}%
\pgfsetbuttcap%
\pgfsetroundjoin%
\definecolor{currentfill}{rgb}{0.175841,0.441290,0.557685}%
\pgfsetfillcolor{currentfill}%
\pgfsetlinewidth{0.000000pt}%
\definecolor{currentstroke}{rgb}{0.160665,0.478540,0.558115}%
\pgfsetstrokecolor{currentstroke}%
\pgfsetdash{}{0pt}%
\pgfpathmoveto{\pgfqpoint{2.971142in}{4.398922in}}%
\pgfpathlineto{\pgfqpoint{2.839884in}{4.398663in}}%
\pgfpathlineto{\pgfqpoint{2.887586in}{4.350458in}}%
\pgfpathclose%
\pgfusepath{fill}%
\end{pgfscope}%
\begin{pgfscope}%
\pgfpathrectangle{\pgfqpoint{0.680860in}{0.078740in}}{\pgfqpoint{7.842520in}{7.842520in}}%
\pgfusepath{clip}%
\pgfsetbuttcap%
\pgfsetroundjoin%
\definecolor{currentfill}{rgb}{0.239346,0.300855,0.540844}%
\pgfsetfillcolor{currentfill}%
\pgfsetlinewidth{0.000000pt}%
\definecolor{currentstroke}{rgb}{0.159194,0.482237,0.558073}%
\pgfsetstrokecolor{currentstroke}%
\pgfsetdash{}{0pt}%
\pgfpathmoveto{\pgfqpoint{6.734085in}{3.934890in}}%
\pgfpathlineto{\pgfqpoint{6.939090in}{3.752670in}}%
\pgfpathlineto{\pgfqpoint{6.799922in}{3.768934in}}%
\pgfpathclose%
\pgfusepath{fill}%
\end{pgfscope}%
\begin{pgfscope}%
\pgfpathrectangle{\pgfqpoint{0.680860in}{0.078740in}}{\pgfqpoint{7.842520in}{7.842520in}}%
\pgfusepath{clip}%
\pgfsetbuttcap%
\pgfsetroundjoin%
\definecolor{currentfill}{rgb}{0.166617,0.463708,0.558119}%
\pgfsetfillcolor{currentfill}%
\pgfsetlinewidth{0.000000pt}%
\definecolor{currentstroke}{rgb}{0.157729,0.485932,0.558013}%
\pgfsetstrokecolor{currentstroke}%
\pgfsetdash{}{0pt}%
\pgfpathmoveto{\pgfqpoint{3.666073in}{4.463523in}}%
\pgfpathlineto{\pgfqpoint{3.880989in}{4.471399in}}%
\pgfpathlineto{\pgfqpoint{3.747755in}{4.469043in}}%
\pgfpathclose%
\pgfusepath{fill}%
\end{pgfscope}%
\begin{pgfscope}%
\pgfpathrectangle{\pgfqpoint{0.680860in}{0.078740in}}{\pgfqpoint{7.842520in}{7.842520in}}%
\pgfusepath{clip}%
\pgfsetbuttcap%
\pgfsetroundjoin%
\definecolor{currentfill}{rgb}{0.231674,0.318106,0.544834}%
\pgfsetfillcolor{currentfill}%
\pgfsetlinewidth{0.000000pt}%
\definecolor{currentstroke}{rgb}{0.156270,0.489624,0.557936}%
\pgfsetstrokecolor{currentstroke}%
\pgfsetdash{}{0pt}%
\pgfpathmoveto{\pgfqpoint{6.661304in}{3.785848in}}%
\pgfpathlineto{\pgfqpoint{6.594579in}{3.944369in}}%
\pgfpathlineto{\pgfqpoint{6.734085in}{3.934890in}}%
\pgfpathclose%
\pgfusepath{fill}%
\end{pgfscope}%
\begin{pgfscope}%
\pgfpathrectangle{\pgfqpoint{0.680860in}{0.078740in}}{\pgfqpoint{7.842520in}{7.842520in}}%
\pgfusepath{clip}%
\pgfsetbuttcap%
\pgfsetroundjoin%
\definecolor{currentfill}{rgb}{0.175841,0.441290,0.557685}%
\pgfsetfillcolor{currentfill}%
\pgfsetlinewidth{0.000000pt}%
\definecolor{currentstroke}{rgb}{0.154815,0.493313,0.557840}%
\pgfsetstrokecolor{currentstroke}%
\pgfsetdash{}{0pt}%
\pgfpathmoveto{\pgfqpoint{4.927273in}{4.396959in}}%
\pgfpathlineto{\pgfqpoint{5.063428in}{4.399039in}}%
\pgfpathlineto{\pgfqpoint{5.004093in}{4.332316in}}%
\pgfpathclose%
\pgfusepath{fill}%
\end{pgfscope}%
\begin{pgfscope}%
\pgfpathrectangle{\pgfqpoint{0.680860in}{0.078740in}}{\pgfqpoint{7.842520in}{7.842520in}}%
\pgfusepath{clip}%
\pgfsetbuttcap%
\pgfsetroundjoin%
\definecolor{currentfill}{rgb}{0.175841,0.441290,0.557685}%
\pgfsetfillcolor{currentfill}%
\pgfsetlinewidth{0.000000pt}%
\definecolor{currentstroke}{rgb}{0.153364,0.497000,0.557724}%
\pgfsetstrokecolor{currentstroke}%
\pgfsetdash{}{0pt}%
\pgfpathmoveto{\pgfqpoint{2.839884in}{4.398663in}}%
\pgfpathlineto{\pgfqpoint{2.756125in}{4.349731in}}%
\pgfpathlineto{\pgfqpoint{2.887586in}{4.350458in}}%
\pgfpathclose%
\pgfusepath{fill}%
\end{pgfscope}%
\begin{pgfscope}%
\pgfpathrectangle{\pgfqpoint{0.680860in}{0.078740in}}{\pgfqpoint{7.842520in}{7.842520in}}%
\pgfusepath{clip}%
\pgfsetbuttcap%
\pgfsetroundjoin%
\definecolor{currentfill}{rgb}{0.182256,0.426184,0.557120}%
\pgfsetfillcolor{currentfill}%
\pgfsetlinewidth{0.000000pt}%
\definecolor{currentstroke}{rgb}{0.151918,0.500685,0.557587}%
\pgfsetstrokecolor{currentstroke}%
\pgfsetdash{}{0pt}%
\pgfpathmoveto{\pgfqpoint{5.412865in}{4.329075in}}%
\pgfpathlineto{\pgfqpoint{5.487351in}{4.239628in}}%
\pgfpathlineto{\pgfqpoint{5.276024in}{4.328982in}}%
\pgfpathclose%
\pgfusepath{fill}%
\end{pgfscope}%
\begin{pgfscope}%
\pgfpathrectangle{\pgfqpoint{0.680860in}{0.078740in}}{\pgfqpoint{7.842520in}{7.842520in}}%
\pgfusepath{clip}%
\pgfsetbuttcap%
\pgfsetroundjoin%
\definecolor{currentfill}{rgb}{0.169646,0.456262,0.558030}%
\pgfsetfillcolor{currentfill}%
\pgfsetlinewidth{0.000000pt}%
\definecolor{currentstroke}{rgb}{0.150476,0.504369,0.557430}%
\pgfsetstrokecolor{currentstroke}%
\pgfsetdash{}{0pt}%
\pgfpathmoveto{\pgfqpoint{3.318141in}{4.437404in}}%
\pgfpathlineto{\pgfqpoint{3.186035in}{4.435586in}}%
\pgfpathlineto{\pgfqpoint{3.102931in}{4.400407in}}%
\pgfpathclose%
\pgfusepath{fill}%
\end{pgfscope}%
\begin{pgfscope}%
\pgfpathrectangle{\pgfqpoint{0.680860in}{0.078740in}}{\pgfqpoint{7.842520in}{7.842520in}}%
\pgfusepath{clip}%
\pgfsetbuttcap%
\pgfsetroundjoin%
\definecolor{currentfill}{rgb}{0.165117,0.467423,0.558141}%
\pgfsetfillcolor{currentfill}%
\pgfsetlinewidth{0.000000pt}%
\definecolor{currentstroke}{rgb}{0.149039,0.508051,0.557250}%
\pgfsetstrokecolor{currentstroke}%
\pgfsetdash{}{0pt}%
\pgfpathmoveto{\pgfqpoint{4.014785in}{4.475058in}}%
\pgfpathlineto{\pgfqpoint{4.229464in}{4.465925in}}%
\pgfpathlineto{\pgfqpoint{4.095418in}{4.463445in}}%
\pgfpathclose%
\pgfusepath{fill}%
\end{pgfscope}%
\begin{pgfscope}%
\pgfpathrectangle{\pgfqpoint{0.680860in}{0.078740in}}{\pgfqpoint{7.842520in}{7.842520in}}%
\pgfusepath{clip}%
\pgfsetbuttcap%
\pgfsetroundjoin%
\definecolor{currentfill}{rgb}{0.177423,0.437527,0.557565}%
\pgfsetfillcolor{currentfill}%
\pgfsetlinewidth{0.000000pt}%
\definecolor{currentstroke}{rgb}{0.147607,0.511733,0.557049}%
\pgfsetstrokecolor{currentstroke}%
\pgfsetdash{}{0pt}%
\pgfpathmoveto{\pgfqpoint{5.139771in}{4.330077in}}%
\pgfpathlineto{\pgfqpoint{5.063428in}{4.399039in}}%
\pgfpathlineto{\pgfqpoint{5.276024in}{4.328982in}}%
\pgfpathclose%
\pgfusepath{fill}%
\end{pgfscope}%
\begin{pgfscope}%
\pgfpathrectangle{\pgfqpoint{0.680860in}{0.078740in}}{\pgfqpoint{7.842520in}{7.842520in}}%
\pgfusepath{clip}%
\pgfsetbuttcap%
\pgfsetroundjoin%
\definecolor{currentfill}{rgb}{0.166617,0.463708,0.558119}%
\pgfsetfillcolor{currentfill}%
\pgfsetlinewidth{0.000000pt}%
\definecolor{currentstroke}{rgb}{0.146180,0.515413,0.556823}%
\pgfsetstrokecolor{currentstroke}%
\pgfsetdash{}{0pt}%
\pgfpathmoveto{\pgfqpoint{4.364082in}{4.469713in}}%
\pgfpathlineto{\pgfqpoint{4.443486in}{4.439871in}}%
\pgfpathlineto{\pgfqpoint{4.229464in}{4.465925in}}%
\pgfpathclose%
\pgfusepath{fill}%
\end{pgfscope}%
\begin{pgfscope}%
\pgfpathrectangle{\pgfqpoint{0.680860in}{0.078740in}}{\pgfqpoint{7.842520in}{7.842520in}}%
\pgfusepath{clip}%
\pgfsetbuttcap%
\pgfsetroundjoin%
\definecolor{currentfill}{rgb}{0.185556,0.418570,0.556753}%
\pgfsetfillcolor{currentfill}%
\pgfsetlinewidth{0.000000pt}%
\definecolor{currentstroke}{rgb}{0.144759,0.519093,0.556572}%
\pgfsetstrokecolor{currentstroke}%
\pgfsetdash{}{0pt}%
\pgfpathmoveto{\pgfqpoint{5.487351in}{4.239628in}}%
\pgfpathlineto{\pgfqpoint{5.550303in}{4.330401in}}%
\pgfpathlineto{\pgfqpoint{5.624214in}{4.235792in}}%
\pgfpathclose%
\pgfusepath{fill}%
\end{pgfscope}%
\begin{pgfscope}%
\pgfpathrectangle{\pgfqpoint{0.680860in}{0.078740in}}{\pgfqpoint{7.842520in}{7.842520in}}%
\pgfusepath{clip}%
\pgfsetbuttcap%
\pgfsetroundjoin%
\definecolor{currentfill}{rgb}{0.168126,0.459988,0.558082}%
\pgfsetfillcolor{currentfill}%
\pgfsetlinewidth{0.000000pt}%
\definecolor{currentstroke}{rgb}{0.143343,0.522773,0.556295}%
\pgfsetstrokecolor{currentstroke}%
\pgfsetdash{}{0pt}%
\pgfpathmoveto{\pgfqpoint{4.578315in}{4.441865in}}%
\pgfpathlineto{\pgfqpoint{4.713725in}{4.445152in}}%
\pgfpathlineto{\pgfqpoint{4.791703in}{4.396133in}}%
\pgfpathclose%
\pgfusepath{fill}%
\end{pgfscope}%
\begin{pgfscope}%
\pgfpathrectangle{\pgfqpoint{0.680860in}{0.078740in}}{\pgfqpoint{7.842520in}{7.842520in}}%
\pgfusepath{clip}%
\pgfsetbuttcap%
\pgfsetroundjoin%
\definecolor{currentfill}{rgb}{0.166617,0.463708,0.558119}%
\pgfsetfillcolor{currentfill}%
\pgfsetlinewidth{0.000000pt}%
\definecolor{currentstroke}{rgb}{0.141935,0.526453,0.555991}%
\pgfsetstrokecolor{currentstroke}%
\pgfsetdash{}{0pt}%
\pgfpathmoveto{\pgfqpoint{3.318141in}{4.437404in}}%
\pgfpathlineto{\pgfqpoint{3.450792in}{4.440505in}}%
\pgfpathlineto{\pgfqpoint{3.533119in}{4.460547in}}%
\pgfpathclose%
\pgfusepath{fill}%
\end{pgfscope}%
\begin{pgfscope}%
\pgfpathrectangle{\pgfqpoint{0.680860in}{0.078740in}}{\pgfqpoint{7.842520in}{7.842520in}}%
\pgfusepath{clip}%
\pgfsetbuttcap%
\pgfsetroundjoin%
\definecolor{currentfill}{rgb}{0.278826,0.175490,0.483397}%
\pgfsetfillcolor{currentfill}%
\pgfsetlinewidth{0.000000pt}%
\definecolor{currentstroke}{rgb}{0.140536,0.530132,0.555659}%
\pgfsetstrokecolor{currentstroke}%
\pgfsetdash{}{0pt}%
\pgfpathmoveto{\pgfqpoint{7.701899in}{3.459977in}}%
\pgfpathlineto{\pgfqpoint{7.843295in}{3.438610in}}%
\pgfpathlineto{\pgfqpoint{7.899449in}{3.204312in}}%
\pgfpathclose%
\pgfusepath{fill}%
\end{pgfscope}%
\begin{pgfscope}%
\pgfpathrectangle{\pgfqpoint{0.680860in}{0.078740in}}{\pgfqpoint{7.842520in}{7.842520in}}%
\pgfusepath{clip}%
\pgfsetbuttcap%
\pgfsetroundjoin%
\definecolor{currentfill}{rgb}{0.172719,0.448791,0.557885}%
\pgfsetfillcolor{currentfill}%
\pgfsetlinewidth{0.000000pt}%
\definecolor{currentstroke}{rgb}{0.139147,0.533812,0.555298}%
\pgfsetstrokecolor{currentstroke}%
\pgfsetdash{}{0pt}%
\pgfpathmoveto{\pgfqpoint{2.971142in}{4.398922in}}%
\pgfpathlineto{\pgfqpoint{2.887586in}{4.350458in}}%
\pgfpathlineto{\pgfqpoint{3.102931in}{4.400407in}}%
\pgfpathclose%
\pgfusepath{fill}%
\end{pgfscope}%
\begin{pgfscope}%
\pgfpathrectangle{\pgfqpoint{0.680860in}{0.078740in}}{\pgfqpoint{7.842520in}{7.842520in}}%
\pgfusepath{clip}%
\pgfsetbuttcap%
\pgfsetroundjoin%
\definecolor{currentfill}{rgb}{0.190631,0.407061,0.556089}%
\pgfsetfillcolor{currentfill}%
\pgfsetlinewidth{0.000000pt}%
\definecolor{currentstroke}{rgb}{0.137770,0.537492,0.554906}%
\pgfsetstrokecolor{currentstroke}%
\pgfsetdash{}{0pt}%
\pgfpathmoveto{\pgfqpoint{5.761662in}{4.233049in}}%
\pgfpathlineto{\pgfqpoint{5.899703in}{4.231441in}}%
\pgfpathlineto{\pgfqpoint{5.971448in}{4.115380in}}%
\pgfpathclose%
\pgfusepath{fill}%
\end{pgfscope}%
\begin{pgfscope}%
\pgfpathrectangle{\pgfqpoint{0.680860in}{0.078740in}}{\pgfqpoint{7.842520in}{7.842520in}}%
\pgfusepath{clip}%
\pgfsetbuttcap%
\pgfsetroundjoin%
\definecolor{currentfill}{rgb}{0.165117,0.467423,0.558141}%
\pgfsetfillcolor{currentfill}%
\pgfsetlinewidth{0.000000pt}%
\definecolor{currentstroke}{rgb}{0.136408,0.541173,0.554483}%
\pgfsetstrokecolor{currentstroke}%
\pgfsetdash{}{0pt}%
\pgfpathmoveto{\pgfqpoint{4.443486in}{4.439871in}}%
\pgfpathlineto{\pgfqpoint{4.499284in}{4.474859in}}%
\pgfpathlineto{\pgfqpoint{4.578315in}{4.441865in}}%
\pgfpathclose%
\pgfusepath{fill}%
\end{pgfscope}%
\begin{pgfscope}%
\pgfpathrectangle{\pgfqpoint{0.680860in}{0.078740in}}{\pgfqpoint{7.842520in}{7.842520in}}%
\pgfusepath{clip}%
\pgfsetbuttcap%
\pgfsetroundjoin%
\definecolor{currentfill}{rgb}{0.168126,0.459988,0.558082}%
\pgfsetfillcolor{currentfill}%
\pgfsetlinewidth{0.000000pt}%
\definecolor{currentstroke}{rgb}{0.135066,0.544853,0.554029}%
\pgfsetstrokecolor{currentstroke}%
\pgfsetdash{}{0pt}%
\pgfpathmoveto{\pgfqpoint{4.927273in}{4.396959in}}%
\pgfpathlineto{\pgfqpoint{4.791703in}{4.396133in}}%
\pgfpathlineto{\pgfqpoint{4.849726in}{4.449783in}}%
\pgfpathclose%
\pgfusepath{fill}%
\end{pgfscope}%
\begin{pgfscope}%
\pgfpathrectangle{\pgfqpoint{0.680860in}{0.078740in}}{\pgfqpoint{7.842520in}{7.842520in}}%
\pgfusepath{clip}%
\pgfsetbuttcap%
\pgfsetroundjoin%
\definecolor{currentfill}{rgb}{0.179019,0.433756,0.557430}%
\pgfsetfillcolor{currentfill}%
\pgfsetlinewidth{0.000000pt}%
\definecolor{currentstroke}{rgb}{0.133743,0.548535,0.553541}%
\pgfsetstrokecolor{currentstroke}%
\pgfsetdash{}{0pt}%
\pgfpathmoveto{\pgfqpoint{5.412865in}{4.329075in}}%
\pgfpathlineto{\pgfqpoint{5.550303in}{4.330401in}}%
\pgfpathlineto{\pgfqpoint{5.487351in}{4.239628in}}%
\pgfpathclose%
\pgfusepath{fill}%
\end{pgfscope}%
\begin{pgfscope}%
\pgfpathrectangle{\pgfqpoint{0.680860in}{0.078740in}}{\pgfqpoint{7.842520in}{7.842520in}}%
\pgfusepath{clip}%
\pgfsetbuttcap%
\pgfsetroundjoin%
\definecolor{currentfill}{rgb}{0.162142,0.474838,0.558140}%
\pgfsetfillcolor{currentfill}%
\pgfsetlinewidth{0.000000pt}%
\definecolor{currentstroke}{rgb}{0.132444,0.552216,0.553018}%
\pgfsetstrokecolor{currentstroke}%
\pgfsetdash{}{0pt}%
\pgfpathmoveto{\pgfqpoint{3.799587in}{4.467826in}}%
\pgfpathlineto{\pgfqpoint{3.880989in}{4.471399in}}%
\pgfpathlineto{\pgfqpoint{3.666073in}{4.463523in}}%
\pgfpathclose%
\pgfusepath{fill}%
\end{pgfscope}%
\begin{pgfscope}%
\pgfpathrectangle{\pgfqpoint{0.680860in}{0.078740in}}{\pgfqpoint{7.842520in}{7.842520in}}%
\pgfusepath{clip}%
\pgfsetbuttcap%
\pgfsetroundjoin%
\definecolor{currentfill}{rgb}{0.194100,0.399323,0.555565}%
\pgfsetfillcolor{currentfill}%
\pgfsetlinewidth{0.000000pt}%
\definecolor{currentstroke}{rgb}{0.131172,0.555899,0.552459}%
\pgfsetstrokecolor{currentstroke}%
\pgfsetdash{}{0pt}%
\pgfpathmoveto{\pgfqpoint{5.971448in}{4.115380in}}%
\pgfpathlineto{\pgfqpoint{6.038351in}{4.231012in}}%
\pgfpathlineto{\pgfqpoint{6.109407in}{4.108890in}}%
\pgfpathclose%
\pgfusepath{fill}%
\end{pgfscope}%
\begin{pgfscope}%
\pgfpathrectangle{\pgfqpoint{0.680860in}{0.078740in}}{\pgfqpoint{7.842520in}{7.842520in}}%
\pgfusepath{clip}%
\pgfsetbuttcap%
\pgfsetroundjoin%
\definecolor{currentfill}{rgb}{0.262138,0.242286,0.520837}%
\pgfsetfillcolor{currentfill}%
\pgfsetlinewidth{0.000000pt}%
\definecolor{currentstroke}{rgb}{0.129933,0.559582,0.551864}%
\pgfsetstrokecolor{currentstroke}%
\pgfsetdash{}{0pt}%
\pgfpathmoveto{\pgfqpoint{7.420741in}{3.504180in}}%
\pgfpathlineto{\pgfqpoint{7.501418in}{3.694632in}}%
\pgfpathlineto{\pgfqpoint{7.561050in}{3.481840in}}%
\pgfpathclose%
\pgfusepath{fill}%
\end{pgfscope}%
\begin{pgfscope}%
\pgfpathrectangle{\pgfqpoint{0.680860in}{0.078740in}}{\pgfqpoint{7.842520in}{7.842520in}}%
\pgfusepath{clip}%
\pgfsetbuttcap%
\pgfsetroundjoin%
\definecolor{currentfill}{rgb}{0.163625,0.471133,0.558148}%
\pgfsetfillcolor{currentfill}%
\pgfsetlinewidth{0.000000pt}%
\definecolor{currentstroke}{rgb}{0.128729,0.563265,0.551229}%
\pgfsetstrokecolor{currentstroke}%
\pgfsetdash{}{0pt}%
\pgfpathmoveto{\pgfqpoint{3.533119in}{4.460547in}}%
\pgfpathlineto{\pgfqpoint{3.583999in}{4.444939in}}%
\pgfpathlineto{\pgfqpoint{3.666073in}{4.463523in}}%
\pgfpathclose%
\pgfusepath{fill}%
\end{pgfscope}%
\begin{pgfscope}%
\pgfpathrectangle{\pgfqpoint{0.680860in}{0.078740in}}{\pgfqpoint{7.842520in}{7.842520in}}%
\pgfusepath{clip}%
\pgfsetbuttcap%
\pgfsetroundjoin%
\definecolor{currentfill}{rgb}{0.166617,0.463708,0.558119}%
\pgfsetfillcolor{currentfill}%
\pgfsetlinewidth{0.000000pt}%
\definecolor{currentstroke}{rgb}{0.127568,0.566949,0.550556}%
\pgfsetstrokecolor{currentstroke}%
\pgfsetdash{}{0pt}%
\pgfpathmoveto{\pgfqpoint{3.235261in}{4.403164in}}%
\pgfpathlineto{\pgfqpoint{3.318141in}{4.437404in}}%
\pgfpathlineto{\pgfqpoint{3.102931in}{4.400407in}}%
\pgfpathclose%
\pgfusepath{fill}%
\end{pgfscope}%
\begin{pgfscope}%
\pgfpathrectangle{\pgfqpoint{0.680860in}{0.078740in}}{\pgfqpoint{7.842520in}{7.842520in}}%
\pgfusepath{clip}%
\pgfsetbuttcap%
\pgfsetroundjoin%
\definecolor{currentfill}{rgb}{0.253935,0.265254,0.529983}%
\pgfsetfillcolor{currentfill}%
\pgfsetlinewidth{0.000000pt}%
\definecolor{currentstroke}{rgb}{0.126453,0.570633,0.549841}%
\pgfsetstrokecolor{currentstroke}%
\pgfsetdash{}{0pt}%
\pgfpathmoveto{\pgfqpoint{7.359970in}{3.708033in}}%
\pgfpathlineto{\pgfqpoint{7.420741in}{3.504180in}}%
\pgfpathlineto{\pgfqpoint{7.219105in}{3.722192in}}%
\pgfpathclose%
\pgfusepath{fill}%
\end{pgfscope}%
\begin{pgfscope}%
\pgfpathrectangle{\pgfqpoint{0.680860in}{0.078740in}}{\pgfqpoint{7.842520in}{7.842520in}}%
\pgfusepath{clip}%
\pgfsetbuttcap%
\pgfsetroundjoin%
\definecolor{currentfill}{rgb}{0.162142,0.474838,0.558140}%
\pgfsetfillcolor{currentfill}%
\pgfsetlinewidth{0.000000pt}%
\definecolor{currentstroke}{rgb}{0.125394,0.574318,0.549086}%
\pgfsetstrokecolor{currentstroke}%
\pgfsetdash{}{0pt}%
\pgfpathmoveto{\pgfqpoint{4.364082in}{4.469713in}}%
\pgfpathlineto{\pgfqpoint{4.499284in}{4.474859in}}%
\pgfpathlineto{\pgfqpoint{4.443486in}{4.439871in}}%
\pgfpathclose%
\pgfusepath{fill}%
\end{pgfscope}%
\begin{pgfscope}%
\pgfpathrectangle{\pgfqpoint{0.680860in}{0.078740in}}{\pgfqpoint{7.842520in}{7.842520in}}%
\pgfusepath{clip}%
\pgfsetbuttcap%
\pgfsetroundjoin%
\definecolor{currentfill}{rgb}{0.204903,0.375746,0.553533}%
\pgfsetfillcolor{currentfill}%
\pgfsetlinewidth{0.000000pt}%
\definecolor{currentstroke}{rgb}{0.124395,0.578002,0.548287}%
\pgfsetstrokecolor{currentstroke}%
\pgfsetdash{}{0pt}%
\pgfpathmoveto{\pgfqpoint{6.387092in}{4.098967in}}%
\pgfpathlineto{\pgfqpoint{6.455653in}{3.954725in}}%
\pgfpathlineto{\pgfqpoint{6.247951in}{4.103406in}}%
\pgfpathclose%
\pgfusepath{fill}%
\end{pgfscope}%
\begin{pgfscope}%
\pgfpathrectangle{\pgfqpoint{0.680860in}{0.078740in}}{\pgfqpoint{7.842520in}{7.842520in}}%
\pgfusepath{clip}%
\pgfsetbuttcap%
\pgfsetroundjoin%
\definecolor{currentfill}{rgb}{0.165117,0.467423,0.558141}%
\pgfsetfillcolor{currentfill}%
\pgfsetlinewidth{0.000000pt}%
\definecolor{currentstroke}{rgb}{0.123463,0.581687,0.547445}%
\pgfsetstrokecolor{currentstroke}%
\pgfsetdash{}{0pt}%
\pgfpathmoveto{\pgfqpoint{4.849726in}{4.449783in}}%
\pgfpathlineto{\pgfqpoint{4.791703in}{4.396133in}}%
\pgfpathlineto{\pgfqpoint{4.713725in}{4.445152in}}%
\pgfpathclose%
\pgfusepath{fill}%
\end{pgfscope}%
\begin{pgfscope}%
\pgfpathrectangle{\pgfqpoint{0.680860in}{0.078740in}}{\pgfqpoint{7.842520in}{7.842520in}}%
\pgfusepath{clip}%
\pgfsetbuttcap%
\pgfsetroundjoin%
\definecolor{currentfill}{rgb}{0.163625,0.471133,0.558148}%
\pgfsetfillcolor{currentfill}%
\pgfsetlinewidth{0.000000pt}%
\definecolor{currentstroke}{rgb}{0.122606,0.585371,0.546557}%
\pgfsetstrokecolor{currentstroke}%
\pgfsetdash{}{0pt}%
\pgfpathmoveto{\pgfqpoint{3.450792in}{4.440505in}}%
\pgfpathlineto{\pgfqpoint{3.583999in}{4.444939in}}%
\pgfpathlineto{\pgfqpoint{3.533119in}{4.460547in}}%
\pgfpathclose%
\pgfusepath{fill}%
\end{pgfscope}%
\begin{pgfscope}%
\pgfpathrectangle{\pgfqpoint{0.680860in}{0.078740in}}{\pgfqpoint{7.842520in}{7.842520in}}%
\pgfusepath{clip}%
\pgfsetbuttcap%
\pgfsetroundjoin%
\definecolor{currentfill}{rgb}{0.165117,0.467423,0.558141}%
\pgfsetfillcolor{currentfill}%
\pgfsetlinewidth{0.000000pt}%
\definecolor{currentstroke}{rgb}{0.121831,0.589055,0.545623}%
\pgfsetstrokecolor{currentstroke}%
\pgfsetdash{}{0pt}%
\pgfpathmoveto{\pgfqpoint{3.450792in}{4.440505in}}%
\pgfpathlineto{\pgfqpoint{3.318141in}{4.437404in}}%
\pgfpathlineto{\pgfqpoint{3.235261in}{4.403164in}}%
\pgfpathclose%
\pgfusepath{fill}%
\end{pgfscope}%
\begin{pgfscope}%
\pgfpathrectangle{\pgfqpoint{0.680860in}{0.078740in}}{\pgfqpoint{7.842520in}{7.842520in}}%
\pgfusepath{clip}%
\pgfsetbuttcap%
\pgfsetroundjoin%
\definecolor{currentfill}{rgb}{0.210503,0.363727,0.552206}%
\pgfsetfillcolor{currentfill}%
\pgfsetlinewidth{0.000000pt}%
\definecolor{currentstroke}{rgb}{0.121148,0.592739,0.544641}%
\pgfsetstrokecolor{currentstroke}%
\pgfsetdash{}{0pt}%
\pgfpathmoveto{\pgfqpoint{6.594579in}{3.944369in}}%
\pgfpathlineto{\pgfqpoint{6.455653in}{3.954725in}}%
\pgfpathlineto{\pgfqpoint{6.526838in}{4.095612in}}%
\pgfpathclose%
\pgfusepath{fill}%
\end{pgfscope}%
\begin{pgfscope}%
\pgfpathrectangle{\pgfqpoint{0.680860in}{0.078740in}}{\pgfqpoint{7.842520in}{7.842520in}}%
\pgfusepath{clip}%
\pgfsetbuttcap%
\pgfsetroundjoin%
\definecolor{currentfill}{rgb}{0.160665,0.478540,0.558115}%
\pgfsetfillcolor{currentfill}%
\pgfsetlinewidth{0.000000pt}%
\definecolor{currentstroke}{rgb}{0.120565,0.596422,0.543611}%
\pgfsetstrokecolor{currentstroke}%
\pgfsetdash{}{0pt}%
\pgfpathmoveto{\pgfqpoint{4.149153in}{4.480069in}}%
\pgfpathlineto{\pgfqpoint{4.229464in}{4.465925in}}%
\pgfpathlineto{\pgfqpoint{4.014785in}{4.475058in}}%
\pgfpathclose%
\pgfusepath{fill}%
\end{pgfscope}%
\begin{pgfscope}%
\pgfpathrectangle{\pgfqpoint{0.680860in}{0.078740in}}{\pgfqpoint{7.842520in}{7.842520in}}%
\pgfusepath{clip}%
\pgfsetbuttcap%
\pgfsetroundjoin%
\definecolor{currentfill}{rgb}{0.233603,0.313828,0.543914}%
\pgfsetfillcolor{currentfill}%
\pgfsetlinewidth{0.000000pt}%
\definecolor{currentstroke}{rgb}{0.120092,0.600104,0.542530}%
\pgfsetstrokecolor{currentstroke}%
\pgfsetdash{}{0pt}%
\pgfpathmoveto{\pgfqpoint{6.939090in}{3.752670in}}%
\pgfpathlineto{\pgfqpoint{7.014878in}{3.918699in}}%
\pgfpathlineto{\pgfqpoint{7.078815in}{3.737080in}}%
\pgfpathclose%
\pgfusepath{fill}%
\end{pgfscope}%
\begin{pgfscope}%
\pgfpathrectangle{\pgfqpoint{0.680860in}{0.078740in}}{\pgfqpoint{7.842520in}{7.842520in}}%
\pgfusepath{clip}%
\pgfsetbuttcap%
\pgfsetroundjoin%
\definecolor{currentfill}{rgb}{0.225863,0.330805,0.547314}%
\pgfsetfillcolor{currentfill}%
\pgfsetlinewidth{0.000000pt}%
\definecolor{currentstroke}{rgb}{0.119738,0.603785,0.541400}%
\pgfsetstrokecolor{currentstroke}%
\pgfsetdash{}{0pt}%
\pgfpathmoveto{\pgfqpoint{6.874182in}{3.926322in}}%
\pgfpathlineto{\pgfqpoint{6.939090in}{3.752670in}}%
\pgfpathlineto{\pgfqpoint{6.734085in}{3.934890in}}%
\pgfpathclose%
\pgfusepath{fill}%
\end{pgfscope}%
\begin{pgfscope}%
\pgfpathrectangle{\pgfqpoint{0.680860in}{0.078740in}}{\pgfqpoint{7.842520in}{7.842520in}}%
\pgfusepath{clip}%
\pgfsetbuttcap%
\pgfsetroundjoin%
\definecolor{currentfill}{rgb}{0.160665,0.478540,0.558115}%
\pgfsetfillcolor{currentfill}%
\pgfsetlinewidth{0.000000pt}%
\definecolor{currentstroke}{rgb}{0.119512,0.607464,0.540218}%
\pgfsetstrokecolor{currentstroke}%
\pgfsetdash{}{0pt}%
\pgfpathmoveto{\pgfqpoint{3.933672in}{4.473506in}}%
\pgfpathlineto{\pgfqpoint{4.014785in}{4.475058in}}%
\pgfpathlineto{\pgfqpoint{3.880989in}{4.471399in}}%
\pgfpathclose%
\pgfusepath{fill}%
\end{pgfscope}%
\begin{pgfscope}%
\pgfpathrectangle{\pgfqpoint{0.680860in}{0.078740in}}{\pgfqpoint{7.842520in}{7.842520in}}%
\pgfusepath{clip}%
\pgfsetbuttcap%
\pgfsetroundjoin%
\definecolor{currentfill}{rgb}{0.174274,0.445044,0.557792}%
\pgfsetfillcolor{currentfill}%
\pgfsetlinewidth{0.000000pt}%
\definecolor{currentstroke}{rgb}{0.119423,0.611141,0.538982}%
\pgfsetstrokecolor{currentstroke}%
\pgfsetdash{}{0pt}%
\pgfpathmoveto{\pgfqpoint{2.803844in}{4.289341in}}%
\pgfpathlineto{\pgfqpoint{2.887586in}{4.350458in}}%
\pgfpathlineto{\pgfqpoint{2.756125in}{4.349731in}}%
\pgfpathclose%
\pgfusepath{fill}%
\end{pgfscope}%
\begin{pgfscope}%
\pgfpathrectangle{\pgfqpoint{0.680860in}{0.078740in}}{\pgfqpoint{7.842520in}{7.842520in}}%
\pgfusepath{clip}%
\pgfsetbuttcap%
\pgfsetroundjoin%
\definecolor{currentfill}{rgb}{0.180629,0.429975,0.557282}%
\pgfsetfillcolor{currentfill}%
\pgfsetlinewidth{0.000000pt}%
\definecolor{currentstroke}{rgb}{0.119483,0.614817,0.537692}%
\pgfsetstrokecolor{currentstroke}%
\pgfsetdash{}{0pt}%
\pgfpathmoveto{\pgfqpoint{5.688350in}{4.333008in}}%
\pgfpathlineto{\pgfqpoint{5.761662in}{4.233049in}}%
\pgfpathlineto{\pgfqpoint{5.624214in}{4.235792in}}%
\pgfpathclose%
\pgfusepath{fill}%
\end{pgfscope}%
\begin{pgfscope}%
\pgfpathrectangle{\pgfqpoint{0.680860in}{0.078740in}}{\pgfqpoint{7.842520in}{7.842520in}}%
\pgfusepath{clip}%
\pgfsetbuttcap%
\pgfsetroundjoin%
\definecolor{currentfill}{rgb}{0.169646,0.456262,0.558030}%
\pgfsetfillcolor{currentfill}%
\pgfsetlinewidth{0.000000pt}%
\definecolor{currentstroke}{rgb}{0.119699,0.618490,0.536347}%
\pgfsetstrokecolor{currentstroke}%
\pgfsetdash{}{0pt}%
\pgfpathmoveto{\pgfqpoint{5.276024in}{4.328982in}}%
\pgfpathlineto{\pgfqpoint{5.063428in}{4.399039in}}%
\pgfpathlineto{\pgfqpoint{5.200180in}{4.402421in}}%
\pgfpathclose%
\pgfusepath{fill}%
\end{pgfscope}%
\begin{pgfscope}%
\pgfpathrectangle{\pgfqpoint{0.680860in}{0.078740in}}{\pgfqpoint{7.842520in}{7.842520in}}%
\pgfusepath{clip}%
\pgfsetbuttcap%
\pgfsetroundjoin%
\definecolor{currentfill}{rgb}{0.187231,0.414746,0.556547}%
\pgfsetfillcolor{currentfill}%
\pgfsetlinewidth{0.000000pt}%
\definecolor{currentstroke}{rgb}{0.120081,0.622161,0.534946}%
\pgfsetstrokecolor{currentstroke}%
\pgfsetdash{}{0pt}%
\pgfpathmoveto{\pgfqpoint{5.971448in}{4.115380in}}%
\pgfpathlineto{\pgfqpoint{5.899703in}{4.231441in}}%
\pgfpathlineto{\pgfqpoint{6.038351in}{4.231012in}}%
\pgfpathclose%
\pgfusepath{fill}%
\end{pgfscope}%
\begin{pgfscope}%
\pgfpathrectangle{\pgfqpoint{0.680860in}{0.078740in}}{\pgfqpoint{7.842520in}{7.842520in}}%
\pgfusepath{clip}%
\pgfsetbuttcap%
\pgfsetroundjoin%
\definecolor{currentfill}{rgb}{0.169646,0.456262,0.558030}%
\pgfsetfillcolor{currentfill}%
\pgfsetlinewidth{0.000000pt}%
\definecolor{currentstroke}{rgb}{0.120638,0.625828,0.533488}%
\pgfsetstrokecolor{currentstroke}%
\pgfsetdash{}{0pt}%
\pgfpathmoveto{\pgfqpoint{3.102931in}{4.400407in}}%
\pgfpathlineto{\pgfqpoint{2.887586in}{4.350458in}}%
\pgfpathlineto{\pgfqpoint{3.019581in}{4.352433in}}%
\pgfpathclose%
\pgfusepath{fill}%
\end{pgfscope}%
\begin{pgfscope}%
\pgfpathrectangle{\pgfqpoint{0.680860in}{0.078740in}}{\pgfqpoint{7.842520in}{7.842520in}}%
\pgfusepath{clip}%
\pgfsetbuttcap%
\pgfsetroundjoin%
\definecolor{currentfill}{rgb}{0.165117,0.467423,0.558141}%
\pgfsetfillcolor{currentfill}%
\pgfsetlinewidth{0.000000pt}%
\definecolor{currentstroke}{rgb}{0.121380,0.629492,0.531973}%
\pgfsetstrokecolor{currentstroke}%
\pgfsetdash{}{0pt}%
\pgfpathmoveto{\pgfqpoint{4.849726in}{4.449783in}}%
\pgfpathlineto{\pgfqpoint{5.063428in}{4.399039in}}%
\pgfpathlineto{\pgfqpoint{4.927273in}{4.396959in}}%
\pgfpathclose%
\pgfusepath{fill}%
\end{pgfscope}%
\begin{pgfscope}%
\pgfpathrectangle{\pgfqpoint{0.680860in}{0.078740in}}{\pgfqpoint{7.842520in}{7.842520in}}%
\pgfusepath{clip}%
\pgfsetbuttcap%
\pgfsetroundjoin%
\definecolor{currentfill}{rgb}{0.175841,0.441290,0.557685}%
\pgfsetfillcolor{currentfill}%
\pgfsetlinewidth{0.000000pt}%
\definecolor{currentstroke}{rgb}{0.122312,0.633153,0.530398}%
\pgfsetstrokecolor{currentstroke}%
\pgfsetdash{}{0pt}%
\pgfpathmoveto{\pgfqpoint{2.803844in}{4.289341in}}%
\pgfpathlineto{\pgfqpoint{2.756125in}{4.349731in}}%
\pgfpathlineto{\pgfqpoint{2.672191in}{4.288552in}}%
\pgfpathclose%
\pgfusepath{fill}%
\end{pgfscope}%
\begin{pgfscope}%
\pgfpathrectangle{\pgfqpoint{0.680860in}{0.078740in}}{\pgfqpoint{7.842520in}{7.842520in}}%
\pgfusepath{clip}%
\pgfsetbuttcap%
\pgfsetroundjoin%
\definecolor{currentfill}{rgb}{0.159194,0.482237,0.558073}%
\pgfsetfillcolor{currentfill}%
\pgfsetlinewidth{0.000000pt}%
\definecolor{currentstroke}{rgb}{0.123444,0.636809,0.528763}%
\pgfsetstrokecolor{currentstroke}%
\pgfsetdash{}{0pt}%
\pgfpathmoveto{\pgfqpoint{3.933672in}{4.473506in}}%
\pgfpathlineto{\pgfqpoint{3.880989in}{4.471399in}}%
\pgfpathlineto{\pgfqpoint{3.799587in}{4.467826in}}%
\pgfpathclose%
\pgfusepath{fill}%
\end{pgfscope}%
\begin{pgfscope}%
\pgfpathrectangle{\pgfqpoint{0.680860in}{0.078740in}}{\pgfqpoint{7.842520in}{7.842520in}}%
\pgfusepath{clip}%
\pgfsetbuttcap%
\pgfsetroundjoin%
\definecolor{currentfill}{rgb}{0.159194,0.482237,0.558073}%
\pgfsetfillcolor{currentfill}%
\pgfsetlinewidth{0.000000pt}%
\definecolor{currentstroke}{rgb}{0.124780,0.640461,0.527068}%
\pgfsetstrokecolor{currentstroke}%
\pgfsetdash{}{0pt}%
\pgfpathmoveto{\pgfqpoint{4.229464in}{4.465925in}}%
\pgfpathlineto{\pgfqpoint{4.284104in}{4.486484in}}%
\pgfpathlineto{\pgfqpoint{4.364082in}{4.469713in}}%
\pgfpathclose%
\pgfusepath{fill}%
\end{pgfscope}%
\begin{pgfscope}%
\pgfpathrectangle{\pgfqpoint{0.680860in}{0.078740in}}{\pgfqpoint{7.842520in}{7.842520in}}%
\pgfusepath{clip}%
\pgfsetbuttcap%
\pgfsetroundjoin%
\definecolor{currentfill}{rgb}{0.160665,0.478540,0.558115}%
\pgfsetfillcolor{currentfill}%
\pgfsetlinewidth{0.000000pt}%
\definecolor{currentstroke}{rgb}{0.126326,0.644107,0.525311}%
\pgfsetstrokecolor{currentstroke}%
\pgfsetdash{}{0pt}%
\pgfpathmoveto{\pgfqpoint{3.799587in}{4.467826in}}%
\pgfpathlineto{\pgfqpoint{3.666073in}{4.463523in}}%
\pgfpathlineto{\pgfqpoint{3.583999in}{4.444939in}}%
\pgfpathclose%
\pgfusepath{fill}%
\end{pgfscope}%
\begin{pgfscope}%
\pgfpathrectangle{\pgfqpoint{0.680860in}{0.078740in}}{\pgfqpoint{7.842520in}{7.842520in}}%
\pgfusepath{clip}%
\pgfsetbuttcap%
\pgfsetroundjoin%
\definecolor{currentfill}{rgb}{0.250425,0.274290,0.533103}%
\pgfsetfillcolor{currentfill}%
\pgfsetlinewidth{0.000000pt}%
\definecolor{currentstroke}{rgb}{0.128087,0.647749,0.523491}%
\pgfsetstrokecolor{currentstroke}%
\pgfsetdash{}{0pt}%
\pgfpathmoveto{\pgfqpoint{7.359970in}{3.708033in}}%
\pgfpathlineto{\pgfqpoint{7.501418in}{3.694632in}}%
\pgfpathlineto{\pgfqpoint{7.420741in}{3.504180in}}%
\pgfpathclose%
\pgfusepath{fill}%
\end{pgfscope}%
\begin{pgfscope}%
\pgfpathrectangle{\pgfqpoint{0.680860in}{0.078740in}}{\pgfqpoint{7.842520in}{7.842520in}}%
\pgfusepath{clip}%
\pgfsetbuttcap%
\pgfsetroundjoin%
\definecolor{currentfill}{rgb}{0.169646,0.456262,0.558030}%
\pgfsetfillcolor{currentfill}%
\pgfsetlinewidth{0.000000pt}%
\definecolor{currentstroke}{rgb}{0.130067,0.651384,0.521608}%
\pgfsetstrokecolor{currentstroke}%
\pgfsetdash{}{0pt}%
\pgfpathmoveto{\pgfqpoint{5.337540in}{4.407155in}}%
\pgfpathlineto{\pgfqpoint{5.412865in}{4.329075in}}%
\pgfpathlineto{\pgfqpoint{5.276024in}{4.328982in}}%
\pgfpathclose%
\pgfusepath{fill}%
\end{pgfscope}%
\begin{pgfscope}%
\pgfpathrectangle{\pgfqpoint{0.680860in}{0.078740in}}{\pgfqpoint{7.842520in}{7.842520in}}%
\pgfusepath{clip}%
\pgfsetbuttcap%
\pgfsetroundjoin%
\definecolor{currentfill}{rgb}{0.201239,0.383670,0.554294}%
\pgfsetfillcolor{currentfill}%
\pgfsetlinewidth{0.000000pt}%
\definecolor{currentstroke}{rgb}{0.132268,0.655014,0.519661}%
\pgfsetstrokecolor{currentstroke}%
\pgfsetdash{}{0pt}%
\pgfpathmoveto{\pgfqpoint{6.526838in}{4.095612in}}%
\pgfpathlineto{\pgfqpoint{6.455653in}{3.954725in}}%
\pgfpathlineto{\pgfqpoint{6.387092in}{4.098967in}}%
\pgfpathclose%
\pgfusepath{fill}%
\end{pgfscope}%
\begin{pgfscope}%
\pgfpathrectangle{\pgfqpoint{0.680860in}{0.078740in}}{\pgfqpoint{7.842520in}{7.842520in}}%
\pgfusepath{clip}%
\pgfsetbuttcap%
\pgfsetroundjoin%
\definecolor{currentfill}{rgb}{0.159194,0.482237,0.558073}%
\pgfsetfillcolor{currentfill}%
\pgfsetlinewidth{0.000000pt}%
\definecolor{currentstroke}{rgb}{0.134692,0.658636,0.517649}%
\pgfsetstrokecolor{currentstroke}%
\pgfsetdash{}{0pt}%
\pgfpathmoveto{\pgfqpoint{4.635079in}{4.481417in}}%
\pgfpathlineto{\pgfqpoint{4.713725in}{4.445152in}}%
\pgfpathlineto{\pgfqpoint{4.578315in}{4.441865in}}%
\pgfpathclose%
\pgfusepath{fill}%
\end{pgfscope}%
\begin{pgfscope}%
\pgfpathrectangle{\pgfqpoint{0.680860in}{0.078740in}}{\pgfqpoint{7.842520in}{7.842520in}}%
\pgfusepath{clip}%
\pgfsetbuttcap%
\pgfsetroundjoin%
\definecolor{currentfill}{rgb}{0.231674,0.318106,0.544834}%
\pgfsetfillcolor{currentfill}%
\pgfsetlinewidth{0.000000pt}%
\definecolor{currentstroke}{rgb}{0.137339,0.662252,0.515571}%
\pgfsetstrokecolor{currentstroke}%
\pgfsetdash{}{0pt}%
\pgfpathmoveto{\pgfqpoint{7.078815in}{3.737080in}}%
\pgfpathlineto{\pgfqpoint{7.014878in}{3.918699in}}%
\pgfpathlineto{\pgfqpoint{7.219105in}{3.722192in}}%
\pgfpathclose%
\pgfusepath{fill}%
\end{pgfscope}%
\begin{pgfscope}%
\pgfpathrectangle{\pgfqpoint{0.680860in}{0.078740in}}{\pgfqpoint{7.842520in}{7.842520in}}%
\pgfusepath{clip}%
\pgfsetbuttcap%
\pgfsetroundjoin%
\definecolor{currentfill}{rgb}{0.260571,0.246922,0.522828}%
\pgfsetfillcolor{currentfill}%
\pgfsetlinewidth{0.000000pt}%
\definecolor{currentstroke}{rgb}{0.140210,0.665859,0.513427}%
\pgfsetstrokecolor{currentstroke}%
\pgfsetdash{}{0pt}%
\pgfpathmoveto{\pgfqpoint{7.643458in}{3.682018in}}%
\pgfpathlineto{\pgfqpoint{7.701899in}{3.459977in}}%
\pgfpathlineto{\pgfqpoint{7.561050in}{3.481840in}}%
\pgfpathclose%
\pgfusepath{fill}%
\end{pgfscope}%
\begin{pgfscope}%
\pgfpathrectangle{\pgfqpoint{0.680860in}{0.078740in}}{\pgfqpoint{7.842520in}{7.842520in}}%
\pgfusepath{clip}%
\pgfsetbuttcap%
\pgfsetroundjoin%
\definecolor{currentfill}{rgb}{0.174274,0.445044,0.557792}%
\pgfsetfillcolor{currentfill}%
\pgfsetlinewidth{0.000000pt}%
\definecolor{currentstroke}{rgb}{0.143303,0.669459,0.511215}%
\pgfsetstrokecolor{currentstroke}%
\pgfsetdash{}{0pt}%
\pgfpathmoveto{\pgfqpoint{5.624214in}{4.235792in}}%
\pgfpathlineto{\pgfqpoint{5.550303in}{4.330401in}}%
\pgfpathlineto{\pgfqpoint{5.688350in}{4.333008in}}%
\pgfpathclose%
\pgfusepath{fill}%
\end{pgfscope}%
\begin{pgfscope}%
\pgfpathrectangle{\pgfqpoint{0.680860in}{0.078740in}}{\pgfqpoint{7.842520in}{7.842520in}}%
\pgfusepath{clip}%
\pgfsetbuttcap%
\pgfsetroundjoin%
\definecolor{currentfill}{rgb}{0.221989,0.339161,0.548752}%
\pgfsetfillcolor{currentfill}%
\pgfsetlinewidth{0.000000pt}%
\definecolor{currentstroke}{rgb}{0.146616,0.673050,0.508936}%
\pgfsetstrokecolor{currentstroke}%
\pgfsetdash{}{0pt}%
\pgfpathmoveto{\pgfqpoint{6.874182in}{3.926322in}}%
\pgfpathlineto{\pgfqpoint{7.014878in}{3.918699in}}%
\pgfpathlineto{\pgfqpoint{6.939090in}{3.752670in}}%
\pgfpathclose%
\pgfusepath{fill}%
\end{pgfscope}%
\begin{pgfscope}%
\pgfpathrectangle{\pgfqpoint{0.680860in}{0.078740in}}{\pgfqpoint{7.842520in}{7.842520in}}%
\pgfusepath{clip}%
\pgfsetbuttcap%
\pgfsetroundjoin%
\definecolor{currentfill}{rgb}{0.188923,0.410910,0.556326}%
\pgfsetfillcolor{currentfill}%
\pgfsetlinewidth{0.000000pt}%
\definecolor{currentstroke}{rgb}{0.150148,0.676631,0.506589}%
\pgfsetstrokecolor{currentstroke}%
\pgfsetdash{}{0pt}%
\pgfpathmoveto{\pgfqpoint{6.177614in}{4.231807in}}%
\pgfpathlineto{\pgfqpoint{6.247951in}{4.103406in}}%
\pgfpathlineto{\pgfqpoint{6.109407in}{4.108890in}}%
\pgfpathclose%
\pgfusepath{fill}%
\end{pgfscope}%
\begin{pgfscope}%
\pgfpathrectangle{\pgfqpoint{0.680860in}{0.078740in}}{\pgfqpoint{7.842520in}{7.842520in}}%
\pgfusepath{clip}%
\pgfsetbuttcap%
\pgfsetroundjoin%
\definecolor{currentfill}{rgb}{0.157729,0.485932,0.558013}%
\pgfsetfillcolor{currentfill}%
\pgfsetlinewidth{0.000000pt}%
\definecolor{currentstroke}{rgb}{0.153894,0.680203,0.504172}%
\pgfsetstrokecolor{currentstroke}%
\pgfsetdash{}{0pt}%
\pgfpathmoveto{\pgfqpoint{4.149153in}{4.480069in}}%
\pgfpathlineto{\pgfqpoint{4.284104in}{4.486484in}}%
\pgfpathlineto{\pgfqpoint{4.229464in}{4.465925in}}%
\pgfpathclose%
\pgfusepath{fill}%
\end{pgfscope}%
\begin{pgfscope}%
\pgfpathrectangle{\pgfqpoint{0.680860in}{0.078740in}}{\pgfqpoint{7.842520in}{7.842520in}}%
\pgfusepath{clip}%
\pgfsetbuttcap%
\pgfsetroundjoin%
\definecolor{currentfill}{rgb}{0.165117,0.467423,0.558141}%
\pgfsetfillcolor{currentfill}%
\pgfsetlinewidth{0.000000pt}%
\definecolor{currentstroke}{rgb}{0.157851,0.683765,0.501686}%
\pgfsetstrokecolor{currentstroke}%
\pgfsetdash{}{0pt}%
\pgfpathmoveto{\pgfqpoint{3.102931in}{4.400407in}}%
\pgfpathlineto{\pgfqpoint{3.152120in}{4.355705in}}%
\pgfpathlineto{\pgfqpoint{3.235261in}{4.403164in}}%
\pgfpathclose%
\pgfusepath{fill}%
\end{pgfscope}%
\begin{pgfscope}%
\pgfpathrectangle{\pgfqpoint{0.680860in}{0.078740in}}{\pgfqpoint{7.842520in}{7.842520in}}%
\pgfusepath{clip}%
\pgfsetbuttcap%
\pgfsetroundjoin%
\definecolor{currentfill}{rgb}{0.162142,0.474838,0.558140}%
\pgfsetfillcolor{currentfill}%
\pgfsetlinewidth{0.000000pt}%
\definecolor{currentstroke}{rgb}{0.162016,0.687316,0.499129}%
\pgfsetstrokecolor{currentstroke}%
\pgfsetdash{}{0pt}%
\pgfpathmoveto{\pgfqpoint{3.235261in}{4.403164in}}%
\pgfpathlineto{\pgfqpoint{3.368141in}{4.407243in}}%
\pgfpathlineto{\pgfqpoint{3.450792in}{4.440505in}}%
\pgfpathclose%
\pgfusepath{fill}%
\end{pgfscope}%
\begin{pgfscope}%
\pgfpathrectangle{\pgfqpoint{0.680860in}{0.078740in}}{\pgfqpoint{7.842520in}{7.842520in}}%
\pgfusepath{clip}%
\pgfsetbuttcap%
\pgfsetroundjoin%
\definecolor{currentfill}{rgb}{0.175841,0.441290,0.557685}%
\pgfsetfillcolor{currentfill}%
\pgfsetlinewidth{0.000000pt}%
\definecolor{currentstroke}{rgb}{0.166383,0.690856,0.496502}%
\pgfsetstrokecolor{currentstroke}%
\pgfsetdash{}{0pt}%
\pgfpathmoveto{\pgfqpoint{5.688350in}{4.333008in}}%
\pgfpathlineto{\pgfqpoint{5.899703in}{4.231441in}}%
\pgfpathlineto{\pgfqpoint{5.761662in}{4.233049in}}%
\pgfpathclose%
\pgfusepath{fill}%
\end{pgfscope}%
\begin{pgfscope}%
\pgfpathrectangle{\pgfqpoint{0.680860in}{0.078740in}}{\pgfqpoint{7.842520in}{7.842520in}}%
\pgfusepath{clip}%
\pgfsetbuttcap%
\pgfsetroundjoin%
\definecolor{currentfill}{rgb}{0.165117,0.467423,0.558141}%
\pgfsetfillcolor{currentfill}%
\pgfsetlinewidth{0.000000pt}%
\definecolor{currentstroke}{rgb}{0.170948,0.694384,0.493803}%
\pgfsetstrokecolor{currentstroke}%
\pgfsetdash{}{0pt}%
\pgfpathmoveto{\pgfqpoint{5.276024in}{4.328982in}}%
\pgfpathlineto{\pgfqpoint{5.200180in}{4.402421in}}%
\pgfpathlineto{\pgfqpoint{5.337540in}{4.407155in}}%
\pgfpathclose%
\pgfusepath{fill}%
\end{pgfscope}%
\begin{pgfscope}%
\pgfpathrectangle{\pgfqpoint{0.680860in}{0.078740in}}{\pgfqpoint{7.842520in}{7.842520in}}%
\pgfusepath{clip}%
\pgfsetbuttcap%
\pgfsetroundjoin%
\definecolor{currentfill}{rgb}{0.157729,0.485932,0.558013}%
\pgfsetfillcolor{currentfill}%
\pgfsetlinewidth{0.000000pt}%
\definecolor{currentstroke}{rgb}{0.175707,0.697900,0.491033}%
\pgfsetstrokecolor{currentstroke}%
\pgfsetdash{}{0pt}%
\pgfpathmoveto{\pgfqpoint{4.578315in}{4.441865in}}%
\pgfpathlineto{\pgfqpoint{4.499284in}{4.474859in}}%
\pgfpathlineto{\pgfqpoint{4.635079in}{4.481417in}}%
\pgfpathclose%
\pgfusepath{fill}%
\end{pgfscope}%
\begin{pgfscope}%
\pgfpathrectangle{\pgfqpoint{0.680860in}{0.078740in}}{\pgfqpoint{7.842520in}{7.842520in}}%
\pgfusepath{clip}%
\pgfsetbuttcap%
\pgfsetroundjoin%
\definecolor{currentfill}{rgb}{0.166617,0.463708,0.558119}%
\pgfsetfillcolor{currentfill}%
\pgfsetlinewidth{0.000000pt}%
\definecolor{currentstroke}{rgb}{0.180653,0.701402,0.488189}%
\pgfsetstrokecolor{currentstroke}%
\pgfsetdash{}{0pt}%
\pgfpathmoveto{\pgfqpoint{3.019581in}{4.352433in}}%
\pgfpathlineto{\pgfqpoint{3.152120in}{4.355705in}}%
\pgfpathlineto{\pgfqpoint{3.102931in}{4.400407in}}%
\pgfpathclose%
\pgfusepath{fill}%
\end{pgfscope}%
\begin{pgfscope}%
\pgfpathrectangle{\pgfqpoint{0.680860in}{0.078740in}}{\pgfqpoint{7.842520in}{7.842520in}}%
\pgfusepath{clip}%
\pgfsetbuttcap%
\pgfsetroundjoin%
\definecolor{currentfill}{rgb}{0.204903,0.375746,0.553533}%
\pgfsetfillcolor{currentfill}%
\pgfsetlinewidth{0.000000pt}%
\definecolor{currentstroke}{rgb}{0.185783,0.704891,0.485273}%
\pgfsetstrokecolor{currentstroke}%
\pgfsetdash{}{0pt}%
\pgfpathmoveto{\pgfqpoint{6.667203in}{4.093383in}}%
\pgfpathlineto{\pgfqpoint{6.734085in}{3.934890in}}%
\pgfpathlineto{\pgfqpoint{6.594579in}{3.944369in}}%
\pgfpathclose%
\pgfusepath{fill}%
\end{pgfscope}%
\begin{pgfscope}%
\pgfpathrectangle{\pgfqpoint{0.680860in}{0.078740in}}{\pgfqpoint{7.842520in}{7.842520in}}%
\pgfusepath{clip}%
\pgfsetbuttcap%
\pgfsetroundjoin%
\definecolor{currentfill}{rgb}{0.160665,0.478540,0.558115}%
\pgfsetfillcolor{currentfill}%
\pgfsetlinewidth{0.000000pt}%
\definecolor{currentstroke}{rgb}{0.191090,0.708366,0.482284}%
\pgfsetstrokecolor{currentstroke}%
\pgfsetdash{}{0pt}%
\pgfpathmoveto{\pgfqpoint{3.583999in}{4.444939in}}%
\pgfpathlineto{\pgfqpoint{3.450792in}{4.440505in}}%
\pgfpathlineto{\pgfqpoint{3.368141in}{4.407243in}}%
\pgfpathclose%
\pgfusepath{fill}%
\end{pgfscope}%
\begin{pgfscope}%
\pgfpathrectangle{\pgfqpoint{0.680860in}{0.078740in}}{\pgfqpoint{7.842520in}{7.842520in}}%
\pgfusepath{clip}%
\pgfsetbuttcap%
\pgfsetroundjoin%
\definecolor{currentfill}{rgb}{0.154815,0.493313,0.557840}%
\pgfsetfillcolor{currentfill}%
\pgfsetlinewidth{0.000000pt}%
\definecolor{currentstroke}{rgb}{0.196571,0.711827,0.479221}%
\pgfsetstrokecolor{currentstroke}%
\pgfsetdash{}{0pt}%
\pgfpathmoveto{\pgfqpoint{4.284104in}{4.486484in}}%
\pgfpathlineto{\pgfqpoint{4.499284in}{4.474859in}}%
\pgfpathlineto{\pgfqpoint{4.364082in}{4.469713in}}%
\pgfpathclose%
\pgfusepath{fill}%
\end{pgfscope}%
\begin{pgfscope}%
\pgfpathrectangle{\pgfqpoint{0.680860in}{0.078740in}}{\pgfqpoint{7.842520in}{7.842520in}}%
\pgfusepath{clip}%
\pgfsetbuttcap%
\pgfsetroundjoin%
\definecolor{currentfill}{rgb}{0.166617,0.463708,0.558119}%
\pgfsetfillcolor{currentfill}%
\pgfsetlinewidth{0.000000pt}%
\definecolor{currentstroke}{rgb}{0.202219,0.715272,0.476084}%
\pgfsetstrokecolor{currentstroke}%
\pgfsetdash{}{0pt}%
\pgfpathmoveto{\pgfqpoint{5.550303in}{4.330401in}}%
\pgfpathlineto{\pgfqpoint{5.412865in}{4.329075in}}%
\pgfpathlineto{\pgfqpoint{5.337540in}{4.407155in}}%
\pgfpathclose%
\pgfusepath{fill}%
\end{pgfscope}%
\begin{pgfscope}%
\pgfpathrectangle{\pgfqpoint{0.680860in}{0.078740in}}{\pgfqpoint{7.842520in}{7.842520in}}%
\pgfusepath{clip}%
\pgfsetbuttcap%
\pgfsetroundjoin%
\definecolor{currentfill}{rgb}{0.175841,0.441290,0.557685}%
\pgfsetfillcolor{currentfill}%
\pgfsetlinewidth{0.000000pt}%
\definecolor{currentstroke}{rgb}{0.208030,0.718701,0.472873}%
\pgfsetstrokecolor{currentstroke}%
\pgfsetdash{}{0pt}%
\pgfpathmoveto{\pgfqpoint{2.672191in}{4.288552in}}%
\pgfpathlineto{\pgfqpoint{2.588124in}{4.214993in}}%
\pgfpathlineto{\pgfqpoint{2.803844in}{4.289341in}}%
\pgfpathclose%
\pgfusepath{fill}%
\end{pgfscope}%
\begin{pgfscope}%
\pgfpathrectangle{\pgfqpoint{0.680860in}{0.078740in}}{\pgfqpoint{7.842520in}{7.842520in}}%
\pgfusepath{clip}%
\pgfsetbuttcap%
\pgfsetroundjoin%
\definecolor{currentfill}{rgb}{0.168126,0.459988,0.558082}%
\pgfsetfillcolor{currentfill}%
\pgfsetlinewidth{0.000000pt}%
\definecolor{currentstroke}{rgb}{0.214000,0.722114,0.469588}%
\pgfsetstrokecolor{currentstroke}%
\pgfsetdash{}{0pt}%
\pgfpathmoveto{\pgfqpoint{3.019581in}{4.352433in}}%
\pgfpathlineto{\pgfqpoint{2.887586in}{4.350458in}}%
\pgfpathlineto{\pgfqpoint{2.936032in}{4.291388in}}%
\pgfpathclose%
\pgfusepath{fill}%
\end{pgfscope}%
\begin{pgfscope}%
\pgfpathrectangle{\pgfqpoint{0.680860in}{0.078740in}}{\pgfqpoint{7.842520in}{7.842520in}}%
\pgfusepath{clip}%
\pgfsetbuttcap%
\pgfsetroundjoin%
\definecolor{currentfill}{rgb}{0.154815,0.493313,0.557840}%
\pgfsetfillcolor{currentfill}%
\pgfsetlinewidth{0.000000pt}%
\definecolor{currentstroke}{rgb}{0.220124,0.725509,0.466226}%
\pgfsetstrokecolor{currentstroke}%
\pgfsetdash{}{0pt}%
\pgfpathmoveto{\pgfqpoint{4.149153in}{4.480069in}}%
\pgfpathlineto{\pgfqpoint{4.014785in}{4.475058in}}%
\pgfpathlineto{\pgfqpoint{4.068337in}{4.480616in}}%
\pgfpathclose%
\pgfusepath{fill}%
\end{pgfscope}%
\begin{pgfscope}%
\pgfpathrectangle{\pgfqpoint{0.680860in}{0.078740in}}{\pgfqpoint{7.842520in}{7.842520in}}%
\pgfusepath{clip}%
\pgfsetbuttcap%
\pgfsetroundjoin%
\definecolor{currentfill}{rgb}{0.258965,0.251537,0.524736}%
\pgfsetfillcolor{currentfill}%
\pgfsetlinewidth{0.000000pt}%
\definecolor{currentstroke}{rgb}{0.226397,0.728888,0.462789}%
\pgfsetstrokecolor{currentstroke}%
\pgfsetdash{}{0pt}%
\pgfpathmoveto{\pgfqpoint{7.843295in}{3.438610in}}%
\pgfpathlineto{\pgfqpoint{7.701899in}{3.459977in}}%
\pgfpathlineto{\pgfqpoint{7.643458in}{3.682018in}}%
\pgfpathclose%
\pgfusepath{fill}%
\end{pgfscope}%
\begin{pgfscope}%
\pgfpathrectangle{\pgfqpoint{0.680860in}{0.078740in}}{\pgfqpoint{7.842520in}{7.842520in}}%
\pgfusepath{clip}%
\pgfsetbuttcap%
\pgfsetroundjoin%
\definecolor{currentfill}{rgb}{0.182256,0.426184,0.557120}%
\pgfsetfillcolor{currentfill}%
\pgfsetlinewidth{0.000000pt}%
\definecolor{currentstroke}{rgb}{0.232815,0.732247,0.459277}%
\pgfsetstrokecolor{currentstroke}%
\pgfsetdash{}{0pt}%
\pgfpathmoveto{\pgfqpoint{6.109407in}{4.108890in}}%
\pgfpathlineto{\pgfqpoint{6.038351in}{4.231012in}}%
\pgfpathlineto{\pgfqpoint{6.177614in}{4.231807in}}%
\pgfpathclose%
\pgfusepath{fill}%
\end{pgfscope}%
\begin{pgfscope}%
\pgfpathrectangle{\pgfqpoint{0.680860in}{0.078740in}}{\pgfqpoint{7.842520in}{7.842520in}}%
\pgfusepath{clip}%
\pgfsetbuttcap%
\pgfsetroundjoin%
\definecolor{currentfill}{rgb}{0.156270,0.489624,0.557936}%
\pgfsetfillcolor{currentfill}%
\pgfsetlinewidth{0.000000pt}%
\definecolor{currentstroke}{rgb}{0.239374,0.735588,0.455688}%
\pgfsetstrokecolor{currentstroke}%
\pgfsetdash{}{0pt}%
\pgfpathmoveto{\pgfqpoint{4.713725in}{4.445152in}}%
\pgfpathlineto{\pgfqpoint{4.635079in}{4.481417in}}%
\pgfpathlineto{\pgfqpoint{4.849726in}{4.449783in}}%
\pgfpathclose%
\pgfusepath{fill}%
\end{pgfscope}%
\begin{pgfscope}%
\pgfpathrectangle{\pgfqpoint{0.680860in}{0.078740in}}{\pgfqpoint{7.842520in}{7.842520in}}%
\pgfusepath{clip}%
\pgfsetbuttcap%
\pgfsetroundjoin%
\definecolor{currentfill}{rgb}{0.169646,0.456262,0.558030}%
\pgfsetfillcolor{currentfill}%
\pgfsetlinewidth{0.000000pt}%
\definecolor{currentstroke}{rgb}{0.246070,0.738910,0.452024}%
\pgfsetstrokecolor{currentstroke}%
\pgfsetdash{}{0pt}%
\pgfpathmoveto{\pgfqpoint{2.887586in}{4.350458in}}%
\pgfpathlineto{\pgfqpoint{2.803844in}{4.289341in}}%
\pgfpathlineto{\pgfqpoint{2.936032in}{4.291388in}}%
\pgfpathclose%
\pgfusepath{fill}%
\end{pgfscope}%
\begin{pgfscope}%
\pgfpathrectangle{\pgfqpoint{0.680860in}{0.078740in}}{\pgfqpoint{7.842520in}{7.842520in}}%
\pgfusepath{clip}%
\pgfsetbuttcap%
\pgfsetroundjoin%
\definecolor{currentfill}{rgb}{0.157729,0.485932,0.558013}%
\pgfsetfillcolor{currentfill}%
\pgfsetlinewidth{0.000000pt}%
\definecolor{currentstroke}{rgb}{0.252899,0.742211,0.448284}%
\pgfsetstrokecolor{currentstroke}%
\pgfsetdash{}{0pt}%
\pgfpathmoveto{\pgfqpoint{4.986329in}{4.455807in}}%
\pgfpathlineto{\pgfqpoint{5.063428in}{4.399039in}}%
\pgfpathlineto{\pgfqpoint{4.849726in}{4.449783in}}%
\pgfpathclose%
\pgfusepath{fill}%
\end{pgfscope}%
\begin{pgfscope}%
\pgfpathrectangle{\pgfqpoint{0.680860in}{0.078740in}}{\pgfqpoint{7.842520in}{7.842520in}}%
\pgfusepath{clip}%
\pgfsetbuttcap%
\pgfsetroundjoin%
\definecolor{currentfill}{rgb}{0.156270,0.489624,0.557936}%
\pgfsetfillcolor{currentfill}%
\pgfsetlinewidth{0.000000pt}%
\definecolor{currentstroke}{rgb}{0.259857,0.745492,0.444467}%
\pgfsetstrokecolor{currentstroke}%
\pgfsetdash{}{0pt}%
\pgfpathmoveto{\pgfqpoint{3.583999in}{4.444939in}}%
\pgfpathlineto{\pgfqpoint{3.717773in}{4.450757in}}%
\pgfpathlineto{\pgfqpoint{3.799587in}{4.467826in}}%
\pgfpathclose%
\pgfusepath{fill}%
\end{pgfscope}%
\begin{pgfscope}%
\pgfpathrectangle{\pgfqpoint{0.680860in}{0.078740in}}{\pgfqpoint{7.842520in}{7.842520in}}%
\pgfusepath{clip}%
\pgfsetbuttcap%
\pgfsetroundjoin%
\definecolor{currentfill}{rgb}{0.246811,0.283237,0.535941}%
\pgfsetfillcolor{currentfill}%
\pgfsetlinewidth{0.000000pt}%
\definecolor{currentstroke}{rgb}{0.266941,0.748751,0.440573}%
\pgfsetstrokecolor{currentstroke}%
\pgfsetdash{}{0pt}%
\pgfpathmoveto{\pgfqpoint{7.561050in}{3.481840in}}%
\pgfpathlineto{\pgfqpoint{7.501418in}{3.694632in}}%
\pgfpathlineto{\pgfqpoint{7.643458in}{3.682018in}}%
\pgfpathclose%
\pgfusepath{fill}%
\end{pgfscope}%
\begin{pgfscope}%
\pgfpathrectangle{\pgfqpoint{0.680860in}{0.078740in}}{\pgfqpoint{7.842520in}{7.842520in}}%
\pgfusepath{clip}%
\pgfsetbuttcap%
\pgfsetroundjoin%
\definecolor{currentfill}{rgb}{0.153364,0.497000,0.557724}%
\pgfsetfillcolor{currentfill}%
\pgfsetlinewidth{0.000000pt}%
\definecolor{currentstroke}{rgb}{0.274149,0.751988,0.436601}%
\pgfsetstrokecolor{currentstroke}%
\pgfsetdash{}{0pt}%
\pgfpathmoveto{\pgfqpoint{4.014785in}{4.475058in}}%
\pgfpathlineto{\pgfqpoint{3.933672in}{4.473506in}}%
\pgfpathlineto{\pgfqpoint{4.068337in}{4.480616in}}%
\pgfpathclose%
\pgfusepath{fill}%
\end{pgfscope}%
\begin{pgfscope}%
\pgfpathrectangle{\pgfqpoint{0.680860in}{0.078740in}}{\pgfqpoint{7.842520in}{7.842520in}}%
\pgfusepath{clip}%
\pgfsetbuttcap%
\pgfsetroundjoin%
\definecolor{currentfill}{rgb}{0.162142,0.474838,0.558140}%
\pgfsetfillcolor{currentfill}%
\pgfsetlinewidth{0.000000pt}%
\definecolor{currentstroke}{rgb}{0.281477,0.755203,0.432552}%
\pgfsetstrokecolor{currentstroke}%
\pgfsetdash{}{0pt}%
\pgfpathmoveto{\pgfqpoint{3.152120in}{4.355705in}}%
\pgfpathlineto{\pgfqpoint{3.368141in}{4.407243in}}%
\pgfpathlineto{\pgfqpoint{3.235261in}{4.403164in}}%
\pgfpathclose%
\pgfusepath{fill}%
\end{pgfscope}%
\begin{pgfscope}%
\pgfpathrectangle{\pgfqpoint{0.680860in}{0.078740in}}{\pgfqpoint{7.842520in}{7.842520in}}%
\pgfusepath{clip}%
\pgfsetbuttcap%
\pgfsetroundjoin%
\definecolor{currentfill}{rgb}{0.195860,0.395433,0.555276}%
\pgfsetfillcolor{currentfill}%
\pgfsetlinewidth{0.000000pt}%
\definecolor{currentstroke}{rgb}{0.288921,0.758394,0.428426}%
\pgfsetstrokecolor{currentstroke}%
\pgfsetdash{}{0pt}%
\pgfpathmoveto{\pgfqpoint{6.594579in}{3.944369in}}%
\pgfpathlineto{\pgfqpoint{6.526838in}{4.095612in}}%
\pgfpathlineto{\pgfqpoint{6.667203in}{4.093383in}}%
\pgfpathclose%
\pgfusepath{fill}%
\end{pgfscope}%
\begin{pgfscope}%
\pgfpathrectangle{\pgfqpoint{0.680860in}{0.078740in}}{\pgfqpoint{7.842520in}{7.842520in}}%
\pgfusepath{clip}%
\pgfsetbuttcap%
\pgfsetroundjoin%
\definecolor{currentfill}{rgb}{0.203063,0.379716,0.553925}%
\pgfsetfillcolor{currentfill}%
\pgfsetlinewidth{0.000000pt}%
\definecolor{currentstroke}{rgb}{0.296479,0.761561,0.424223}%
\pgfsetstrokecolor{currentstroke}%
\pgfsetdash{}{0pt}%
\pgfpathmoveto{\pgfqpoint{6.874182in}{3.926322in}}%
\pgfpathlineto{\pgfqpoint{6.734085in}{3.934890in}}%
\pgfpathlineto{\pgfqpoint{6.667203in}{4.093383in}}%
\pgfpathclose%
\pgfusepath{fill}%
\end{pgfscope}%
\begin{pgfscope}%
\pgfpathrectangle{\pgfqpoint{0.680860in}{0.078740in}}{\pgfqpoint{7.842520in}{7.842520in}}%
\pgfusepath{clip}%
\pgfsetbuttcap%
\pgfsetroundjoin%
\definecolor{currentfill}{rgb}{0.157729,0.485932,0.558013}%
\pgfsetfillcolor{currentfill}%
\pgfsetlinewidth{0.000000pt}%
\definecolor{currentstroke}{rgb}{0.304148,0.764704,0.419943}%
\pgfsetstrokecolor{currentstroke}%
\pgfsetdash{}{0pt}%
\pgfpathmoveto{\pgfqpoint{5.200180in}{4.402421in}}%
\pgfpathlineto{\pgfqpoint{5.063428in}{4.399039in}}%
\pgfpathlineto{\pgfqpoint{5.123547in}{4.463279in}}%
\pgfpathclose%
\pgfusepath{fill}%
\end{pgfscope}%
\begin{pgfscope}%
\pgfpathrectangle{\pgfqpoint{0.680860in}{0.078740in}}{\pgfqpoint{7.842520in}{7.842520in}}%
\pgfusepath{clip}%
\pgfsetbuttcap%
\pgfsetroundjoin%
\definecolor{currentfill}{rgb}{0.183898,0.422383,0.556944}%
\pgfsetfillcolor{currentfill}%
\pgfsetlinewidth{0.000000pt}%
\definecolor{currentstroke}{rgb}{0.311925,0.767822,0.415586}%
\pgfsetstrokecolor{currentstroke}%
\pgfsetdash{}{0pt}%
\pgfpathmoveto{\pgfqpoint{6.387092in}{4.098967in}}%
\pgfpathlineto{\pgfqpoint{6.247951in}{4.103406in}}%
\pgfpathlineto{\pgfqpoint{6.317507in}{4.233872in}}%
\pgfpathclose%
\pgfusepath{fill}%
\end{pgfscope}%
\begin{pgfscope}%
\pgfpathrectangle{\pgfqpoint{0.680860in}{0.078740in}}{\pgfqpoint{7.842520in}{7.842520in}}%
\pgfusepath{clip}%
\pgfsetbuttcap%
\pgfsetroundjoin%
\definecolor{currentfill}{rgb}{0.153364,0.497000,0.557724}%
\pgfsetfillcolor{currentfill}%
\pgfsetlinewidth{0.000000pt}%
\definecolor{currentstroke}{rgb}{0.319809,0.770914,0.411152}%
\pgfsetstrokecolor{currentstroke}%
\pgfsetdash{}{0pt}%
\pgfpathmoveto{\pgfqpoint{3.852122in}{4.458012in}}%
\pgfpathlineto{\pgfqpoint{3.933672in}{4.473506in}}%
\pgfpathlineto{\pgfqpoint{3.799587in}{4.467826in}}%
\pgfpathclose%
\pgfusepath{fill}%
\end{pgfscope}%
\begin{pgfscope}%
\pgfpathrectangle{\pgfqpoint{0.680860in}{0.078740in}}{\pgfqpoint{7.842520in}{7.842520in}}%
\pgfusepath{clip}%
\pgfsetbuttcap%
\pgfsetroundjoin%
\definecolor{currentfill}{rgb}{0.165117,0.467423,0.558141}%
\pgfsetfillcolor{currentfill}%
\pgfsetlinewidth{0.000000pt}%
\definecolor{currentstroke}{rgb}{0.327796,0.773980,0.406640}%
\pgfsetstrokecolor{currentstroke}%
\pgfsetdash{}{0pt}%
\pgfpathmoveto{\pgfqpoint{2.936032in}{4.291388in}}%
\pgfpathlineto{\pgfqpoint{3.152120in}{4.355705in}}%
\pgfpathlineto{\pgfqpoint{3.019581in}{4.352433in}}%
\pgfpathclose%
\pgfusepath{fill}%
\end{pgfscope}%
\begin{pgfscope}%
\pgfpathrectangle{\pgfqpoint{0.680860in}{0.078740in}}{\pgfqpoint{7.842520in}{7.842520in}}%
\pgfusepath{clip}%
\pgfsetbuttcap%
\pgfsetroundjoin%
\definecolor{currentfill}{rgb}{0.156270,0.489624,0.557936}%
\pgfsetfillcolor{currentfill}%
\pgfsetlinewidth{0.000000pt}%
\definecolor{currentstroke}{rgb}{0.335885,0.777018,0.402049}%
\pgfsetstrokecolor{currentstroke}%
\pgfsetdash{}{0pt}%
\pgfpathmoveto{\pgfqpoint{3.583999in}{4.444939in}}%
\pgfpathlineto{\pgfqpoint{3.368141in}{4.407243in}}%
\pgfpathlineto{\pgfqpoint{3.501582in}{4.412697in}}%
\pgfpathclose%
\pgfusepath{fill}%
\end{pgfscope}%
\begin{pgfscope}%
\pgfpathrectangle{\pgfqpoint{0.680860in}{0.078740in}}{\pgfqpoint{7.842520in}{7.842520in}}%
\pgfusepath{clip}%
\pgfsetbuttcap%
\pgfsetroundjoin%
\definecolor{currentfill}{rgb}{0.150476,0.504369,0.557430}%
\pgfsetfillcolor{currentfill}%
\pgfsetlinewidth{0.000000pt}%
\definecolor{currentstroke}{rgb}{0.344074,0.780029,0.397381}%
\pgfsetstrokecolor{currentstroke}%
\pgfsetdash{}{0pt}%
\pgfpathmoveto{\pgfqpoint{4.068337in}{4.480616in}}%
\pgfpathlineto{\pgfqpoint{4.284104in}{4.486484in}}%
\pgfpathlineto{\pgfqpoint{4.149153in}{4.480069in}}%
\pgfpathclose%
\pgfusepath{fill}%
\end{pgfscope}%
\begin{pgfscope}%
\pgfpathrectangle{\pgfqpoint{0.680860in}{0.078740in}}{\pgfqpoint{7.842520in}{7.842520in}}%
\pgfusepath{clip}%
\pgfsetbuttcap%
\pgfsetroundjoin%
\definecolor{currentfill}{rgb}{0.225863,0.330805,0.547314}%
\pgfsetfillcolor{currentfill}%
\pgfsetlinewidth{0.000000pt}%
\definecolor{currentstroke}{rgb}{0.352360,0.783011,0.392636}%
\pgfsetstrokecolor{currentstroke}%
\pgfsetdash{}{0pt}%
\pgfpathmoveto{\pgfqpoint{7.359970in}{3.708033in}}%
\pgfpathlineto{\pgfqpoint{7.219105in}{3.722192in}}%
\pgfpathlineto{\pgfqpoint{7.298112in}{3.906438in}}%
\pgfpathclose%
\pgfusepath{fill}%
\end{pgfscope}%
\begin{pgfscope}%
\pgfpathrectangle{\pgfqpoint{0.680860in}{0.078740in}}{\pgfqpoint{7.842520in}{7.842520in}}%
\pgfusepath{clip}%
\pgfsetbuttcap%
\pgfsetroundjoin%
\definecolor{currentfill}{rgb}{0.174274,0.445044,0.557792}%
\pgfsetfillcolor{currentfill}%
\pgfsetlinewidth{0.000000pt}%
\definecolor{currentstroke}{rgb}{0.360741,0.785964,0.387814}%
\pgfsetstrokecolor{currentstroke}%
\pgfsetdash{}{0pt}%
\pgfpathmoveto{\pgfqpoint{2.588124in}{4.214993in}}%
\pgfpathlineto{\pgfqpoint{2.719957in}{4.215434in}}%
\pgfpathlineto{\pgfqpoint{2.803844in}{4.289341in}}%
\pgfpathclose%
\pgfusepath{fill}%
\end{pgfscope}%
\begin{pgfscope}%
\pgfpathrectangle{\pgfqpoint{0.680860in}{0.078740in}}{\pgfqpoint{7.842520in}{7.842520in}}%
\pgfusepath{clip}%
\pgfsetbuttcap%
\pgfsetroundjoin%
\definecolor{currentfill}{rgb}{0.216210,0.351535,0.550627}%
\pgfsetfillcolor{currentfill}%
\pgfsetlinewidth{0.000000pt}%
\definecolor{currentstroke}{rgb}{0.369214,0.788888,0.382914}%
\pgfsetstrokecolor{currentstroke}%
\pgfsetdash{}{0pt}%
\pgfpathmoveto{\pgfqpoint{7.014878in}{3.918699in}}%
\pgfpathlineto{\pgfqpoint{7.156184in}{3.912059in}}%
\pgfpathlineto{\pgfqpoint{7.219105in}{3.722192in}}%
\pgfpathclose%
\pgfusepath{fill}%
\end{pgfscope}%
\begin{pgfscope}%
\pgfpathrectangle{\pgfqpoint{0.680860in}{0.078740in}}{\pgfqpoint{7.842520in}{7.842520in}}%
\pgfusepath{clip}%
\pgfsetbuttcap%
\pgfsetroundjoin%
\definecolor{currentfill}{rgb}{0.168126,0.459988,0.558082}%
\pgfsetfillcolor{currentfill}%
\pgfsetlinewidth{0.000000pt}%
\definecolor{currentstroke}{rgb}{0.377779,0.791781,0.377939}%
\pgfsetstrokecolor{currentstroke}%
\pgfsetdash{}{0pt}%
\pgfpathmoveto{\pgfqpoint{5.827018in}{4.336945in}}%
\pgfpathlineto{\pgfqpoint{5.899703in}{4.231441in}}%
\pgfpathlineto{\pgfqpoint{5.688350in}{4.333008in}}%
\pgfpathclose%
\pgfusepath{fill}%
\end{pgfscope}%
\begin{pgfscope}%
\pgfpathrectangle{\pgfqpoint{0.680860in}{0.078740in}}{\pgfqpoint{7.842520in}{7.842520in}}%
\pgfusepath{clip}%
\pgfsetbuttcap%
\pgfsetroundjoin%
\definecolor{currentfill}{rgb}{0.154815,0.493313,0.557840}%
\pgfsetfillcolor{currentfill}%
\pgfsetlinewidth{0.000000pt}%
\definecolor{currentstroke}{rgb}{0.386433,0.794644,0.372886}%
\pgfsetstrokecolor{currentstroke}%
\pgfsetdash{}{0pt}%
\pgfpathmoveto{\pgfqpoint{5.123547in}{4.463279in}}%
\pgfpathlineto{\pgfqpoint{5.063428in}{4.399039in}}%
\pgfpathlineto{\pgfqpoint{4.986329in}{4.455807in}}%
\pgfpathclose%
\pgfusepath{fill}%
\end{pgfscope}%
\begin{pgfscope}%
\pgfpathrectangle{\pgfqpoint{0.680860in}{0.078740in}}{\pgfqpoint{7.842520in}{7.842520in}}%
\pgfusepath{clip}%
\pgfsetbuttcap%
\pgfsetroundjoin%
\definecolor{currentfill}{rgb}{0.151918,0.500685,0.557587}%
\pgfsetfillcolor{currentfill}%
\pgfsetlinewidth{0.000000pt}%
\definecolor{currentstroke}{rgb}{0.395174,0.797475,0.367757}%
\pgfsetstrokecolor{currentstroke}%
\pgfsetdash{}{0pt}%
\pgfpathmoveto{\pgfqpoint{3.799587in}{4.467826in}}%
\pgfpathlineto{\pgfqpoint{3.717773in}{4.450757in}}%
\pgfpathlineto{\pgfqpoint{3.852122in}{4.458012in}}%
\pgfpathclose%
\pgfusepath{fill}%
\end{pgfscope}%
\begin{pgfscope}%
\pgfpathrectangle{\pgfqpoint{0.680860in}{0.078740in}}{\pgfqpoint{7.842520in}{7.842520in}}%
\pgfusepath{clip}%
\pgfsetbuttcap%
\pgfsetroundjoin%
\definecolor{currentfill}{rgb}{0.150476,0.504369,0.557430}%
\pgfsetfillcolor{currentfill}%
\pgfsetlinewidth{0.000000pt}%
\definecolor{currentstroke}{rgb}{0.404001,0.800275,0.362552}%
\pgfsetstrokecolor{currentstroke}%
\pgfsetdash{}{0pt}%
\pgfpathmoveto{\pgfqpoint{4.419650in}{4.494358in}}%
\pgfpathlineto{\pgfqpoint{4.499284in}{4.474859in}}%
\pgfpathlineto{\pgfqpoint{4.284104in}{4.486484in}}%
\pgfpathclose%
\pgfusepath{fill}%
\end{pgfscope}%
\begin{pgfscope}%
\pgfpathrectangle{\pgfqpoint{0.680860in}{0.078740in}}{\pgfqpoint{7.842520in}{7.842520in}}%
\pgfusepath{clip}%
\pgfsetbuttcap%
\pgfsetroundjoin%
\definecolor{currentfill}{rgb}{0.150476,0.504369,0.557430}%
\pgfsetfillcolor{currentfill}%
\pgfsetlinewidth{0.000000pt}%
\definecolor{currentstroke}{rgb}{0.412913,0.803041,0.357269}%
\pgfsetstrokecolor{currentstroke}%
\pgfsetdash{}{0pt}%
\pgfpathmoveto{\pgfqpoint{4.635079in}{4.481417in}}%
\pgfpathlineto{\pgfqpoint{4.499284in}{4.474859in}}%
\pgfpathlineto{\pgfqpoint{4.419650in}{4.494358in}}%
\pgfpathclose%
\pgfusepath{fill}%
\end{pgfscope}%
\begin{pgfscope}%
\pgfpathrectangle{\pgfqpoint{0.680860in}{0.078740in}}{\pgfqpoint{7.842520in}{7.842520in}}%
\pgfusepath{clip}%
\pgfsetbuttcap%
\pgfsetroundjoin%
\definecolor{currentfill}{rgb}{0.169646,0.456262,0.558030}%
\pgfsetfillcolor{currentfill}%
\pgfsetlinewidth{0.000000pt}%
\definecolor{currentstroke}{rgb}{0.421908,0.805774,0.351910}%
\pgfsetstrokecolor{currentstroke}%
\pgfsetdash{}{0pt}%
\pgfpathmoveto{\pgfqpoint{2.936032in}{4.291388in}}%
\pgfpathlineto{\pgfqpoint{2.803844in}{4.289341in}}%
\pgfpathlineto{\pgfqpoint{2.719957in}{4.215434in}}%
\pgfpathclose%
\pgfusepath{fill}%
\end{pgfscope}%
\begin{pgfscope}%
\pgfpathrectangle{\pgfqpoint{0.680860in}{0.078740in}}{\pgfqpoint{7.842520in}{7.842520in}}%
\pgfusepath{clip}%
\pgfsetbuttcap%
\pgfsetroundjoin%
\definecolor{currentfill}{rgb}{0.160665,0.478540,0.558115}%
\pgfsetfillcolor{currentfill}%
\pgfsetlinewidth{0.000000pt}%
\definecolor{currentstroke}{rgb}{0.430983,0.808473,0.346476}%
\pgfsetstrokecolor{currentstroke}%
\pgfsetdash{}{0pt}%
\pgfpathmoveto{\pgfqpoint{5.688350in}{4.333008in}}%
\pgfpathlineto{\pgfqpoint{5.550303in}{4.330401in}}%
\pgfpathlineto{\pgfqpoint{5.475520in}{4.413292in}}%
\pgfpathclose%
\pgfusepath{fill}%
\end{pgfscope}%
\begin{pgfscope}%
\pgfpathrectangle{\pgfqpoint{0.680860in}{0.078740in}}{\pgfqpoint{7.842520in}{7.842520in}}%
\pgfusepath{clip}%
\pgfsetbuttcap%
\pgfsetroundjoin%
\definecolor{currentfill}{rgb}{0.169646,0.456262,0.558030}%
\pgfsetfillcolor{currentfill}%
\pgfsetlinewidth{0.000000pt}%
\definecolor{currentstroke}{rgb}{0.440137,0.811138,0.340967}%
\pgfsetstrokecolor{currentstroke}%
\pgfsetdash{}{0pt}%
\pgfpathmoveto{\pgfqpoint{5.966319in}{4.342261in}}%
\pgfpathlineto{\pgfqpoint{6.038351in}{4.231012in}}%
\pgfpathlineto{\pgfqpoint{5.899703in}{4.231441in}}%
\pgfpathclose%
\pgfusepath{fill}%
\end{pgfscope}%
\begin{pgfscope}%
\pgfpathrectangle{\pgfqpoint{0.680860in}{0.078740in}}{\pgfqpoint{7.842520in}{7.842520in}}%
\pgfusepath{clip}%
\pgfsetbuttcap%
\pgfsetroundjoin%
\definecolor{currentfill}{rgb}{0.177423,0.437527,0.557565}%
\pgfsetfillcolor{currentfill}%
\pgfsetlinewidth{0.000000pt}%
\definecolor{currentstroke}{rgb}{0.449368,0.813768,0.335384}%
\pgfsetstrokecolor{currentstroke}%
\pgfsetdash{}{0pt}%
\pgfpathmoveto{\pgfqpoint{6.247951in}{4.103406in}}%
\pgfpathlineto{\pgfqpoint{6.177614in}{4.231807in}}%
\pgfpathlineto{\pgfqpoint{6.317507in}{4.233872in}}%
\pgfpathclose%
\pgfusepath{fill}%
\end{pgfscope}%
\begin{pgfscope}%
\pgfpathrectangle{\pgfqpoint{0.680860in}{0.078740in}}{\pgfqpoint{7.842520in}{7.842520in}}%
\pgfusepath{clip}%
\pgfsetbuttcap%
\pgfsetroundjoin%
\definecolor{currentfill}{rgb}{0.159194,0.482237,0.558073}%
\pgfsetfillcolor{currentfill}%
\pgfsetlinewidth{0.000000pt}%
\definecolor{currentstroke}{rgb}{0.458674,0.816363,0.329727}%
\pgfsetstrokecolor{currentstroke}%
\pgfsetdash{}{0pt}%
\pgfpathmoveto{\pgfqpoint{5.337540in}{4.407155in}}%
\pgfpathlineto{\pgfqpoint{5.475520in}{4.413292in}}%
\pgfpathlineto{\pgfqpoint{5.550303in}{4.330401in}}%
\pgfpathclose%
\pgfusepath{fill}%
\end{pgfscope}%
\begin{pgfscope}%
\pgfpathrectangle{\pgfqpoint{0.680860in}{0.078740in}}{\pgfqpoint{7.842520in}{7.842520in}}%
\pgfusepath{clip}%
\pgfsetbuttcap%
\pgfsetroundjoin%
\definecolor{currentfill}{rgb}{0.150476,0.504369,0.557430}%
\pgfsetfillcolor{currentfill}%
\pgfsetlinewidth{0.000000pt}%
\definecolor{currentstroke}{rgb}{0.468053,0.818921,0.323998}%
\pgfsetstrokecolor{currentstroke}%
\pgfsetdash{}{0pt}%
\pgfpathmoveto{\pgfqpoint{4.849726in}{4.449783in}}%
\pgfpathlineto{\pgfqpoint{4.635079in}{4.481417in}}%
\pgfpathlineto{\pgfqpoint{4.771480in}{4.489440in}}%
\pgfpathclose%
\pgfusepath{fill}%
\end{pgfscope}%
\begin{pgfscope}%
\pgfpathrectangle{\pgfqpoint{0.680860in}{0.078740in}}{\pgfqpoint{7.842520in}{7.842520in}}%
\pgfusepath{clip}%
\pgfsetbuttcap%
\pgfsetroundjoin%
\definecolor{currentfill}{rgb}{0.159194,0.482237,0.558073}%
\pgfsetfillcolor{currentfill}%
\pgfsetlinewidth{0.000000pt}%
\definecolor{currentstroke}{rgb}{0.477504,0.821444,0.318195}%
\pgfsetstrokecolor{currentstroke}%
\pgfsetdash{}{0pt}%
\pgfpathmoveto{\pgfqpoint{3.285212in}{4.360324in}}%
\pgfpathlineto{\pgfqpoint{3.368141in}{4.407243in}}%
\pgfpathlineto{\pgfqpoint{3.152120in}{4.355705in}}%
\pgfpathclose%
\pgfusepath{fill}%
\end{pgfscope}%
\begin{pgfscope}%
\pgfpathrectangle{\pgfqpoint{0.680860in}{0.078740in}}{\pgfqpoint{7.842520in}{7.842520in}}%
\pgfusepath{clip}%
\pgfsetbuttcap%
\pgfsetroundjoin%
\definecolor{currentfill}{rgb}{0.180629,0.429975,0.557282}%
\pgfsetfillcolor{currentfill}%
\pgfsetlinewidth{0.000000pt}%
\definecolor{currentstroke}{rgb}{0.487026,0.823929,0.312321}%
\pgfsetstrokecolor{currentstroke}%
\pgfsetdash{}{0pt}%
\pgfpathmoveto{\pgfqpoint{6.317507in}{4.233872in}}%
\pgfpathlineto{\pgfqpoint{6.526838in}{4.095612in}}%
\pgfpathlineto{\pgfqpoint{6.387092in}{4.098967in}}%
\pgfpathclose%
\pgfusepath{fill}%
\end{pgfscope}%
\begin{pgfscope}%
\pgfpathrectangle{\pgfqpoint{0.680860in}{0.078740in}}{\pgfqpoint{7.842520in}{7.842520in}}%
\pgfusepath{clip}%
\pgfsetbuttcap%
\pgfsetroundjoin%
\definecolor{currentfill}{rgb}{0.175841,0.441290,0.557685}%
\pgfsetfillcolor{currentfill}%
\pgfsetlinewidth{0.000000pt}%
\definecolor{currentstroke}{rgb}{0.496615,0.826376,0.306377}%
\pgfsetstrokecolor{currentstroke}%
\pgfsetdash{}{0pt}%
\pgfpathmoveto{\pgfqpoint{2.719957in}{4.215434in}}%
\pgfpathlineto{\pgfqpoint{2.588124in}{4.214993in}}%
\pgfpathlineto{\pgfqpoint{2.503961in}{4.129048in}}%
\pgfpathclose%
\pgfusepath{fill}%
\end{pgfscope}%
\begin{pgfscope}%
\pgfpathrectangle{\pgfqpoint{0.680860in}{0.078740in}}{\pgfqpoint{7.842520in}{7.842520in}}%
\pgfusepath{clip}%
\pgfsetbuttcap%
\pgfsetroundjoin%
\definecolor{currentfill}{rgb}{0.153364,0.497000,0.557724}%
\pgfsetfillcolor{currentfill}%
\pgfsetlinewidth{0.000000pt}%
\definecolor{currentstroke}{rgb}{0.506271,0.828786,0.300362}%
\pgfsetstrokecolor{currentstroke}%
\pgfsetdash{}{0pt}%
\pgfpathmoveto{\pgfqpoint{3.583999in}{4.444939in}}%
\pgfpathlineto{\pgfqpoint{3.635593in}{4.419577in}}%
\pgfpathlineto{\pgfqpoint{3.717773in}{4.450757in}}%
\pgfpathclose%
\pgfusepath{fill}%
\end{pgfscope}%
\begin{pgfscope}%
\pgfpathrectangle{\pgfqpoint{0.680860in}{0.078740in}}{\pgfqpoint{7.842520in}{7.842520in}}%
\pgfusepath{clip}%
\pgfsetbuttcap%
\pgfsetroundjoin%
\definecolor{currentfill}{rgb}{0.221989,0.339161,0.548752}%
\pgfsetfillcolor{currentfill}%
\pgfsetlinewidth{0.000000pt}%
\definecolor{currentstroke}{rgb}{0.515992,0.831158,0.294279}%
\pgfsetstrokecolor{currentstroke}%
\pgfsetdash{}{0pt}%
\pgfpathmoveto{\pgfqpoint{7.298112in}{3.906438in}}%
\pgfpathlineto{\pgfqpoint{7.501418in}{3.694632in}}%
\pgfpathlineto{\pgfqpoint{7.359970in}{3.708033in}}%
\pgfpathclose%
\pgfusepath{fill}%
\end{pgfscope}%
\begin{pgfscope}%
\pgfpathrectangle{\pgfqpoint{0.680860in}{0.078740in}}{\pgfqpoint{7.842520in}{7.842520in}}%
\pgfusepath{clip}%
\pgfsetbuttcap%
\pgfsetroundjoin%
\definecolor{currentfill}{rgb}{0.149039,0.508051,0.557250}%
\pgfsetfillcolor{currentfill}%
\pgfsetlinewidth{0.000000pt}%
\definecolor{currentstroke}{rgb}{0.525776,0.833491,0.288127}%
\pgfsetstrokecolor{currentstroke}%
\pgfsetdash{}{0pt}%
\pgfpathmoveto{\pgfqpoint{3.852122in}{4.458012in}}%
\pgfpathlineto{\pgfqpoint{4.068337in}{4.480616in}}%
\pgfpathlineto{\pgfqpoint{3.933672in}{4.473506in}}%
\pgfpathclose%
\pgfusepath{fill}%
\end{pgfscope}%
\begin{pgfscope}%
\pgfpathrectangle{\pgfqpoint{0.680860in}{0.078740in}}{\pgfqpoint{7.842520in}{7.842520in}}%
\pgfusepath{clip}%
\pgfsetbuttcap%
\pgfsetroundjoin%
\definecolor{currentfill}{rgb}{0.197636,0.391528,0.554969}%
\pgfsetfillcolor{currentfill}%
\pgfsetlinewidth{0.000000pt}%
\definecolor{currentstroke}{rgb}{0.535621,0.835785,0.281908}%
\pgfsetstrokecolor{currentstroke}%
\pgfsetdash{}{0pt}%
\pgfpathmoveto{\pgfqpoint{6.874182in}{3.926322in}}%
\pgfpathlineto{\pgfqpoint{6.808197in}{4.092323in}}%
\pgfpathlineto{\pgfqpoint{7.014878in}{3.918699in}}%
\pgfpathclose%
\pgfusepath{fill}%
\end{pgfscope}%
\begin{pgfscope}%
\pgfpathrectangle{\pgfqpoint{0.680860in}{0.078740in}}{\pgfqpoint{7.842520in}{7.842520in}}%
\pgfusepath{clip}%
\pgfsetbuttcap%
\pgfsetroundjoin%
\definecolor{currentfill}{rgb}{0.153364,0.497000,0.557724}%
\pgfsetfillcolor{currentfill}%
\pgfsetlinewidth{0.000000pt}%
\definecolor{currentstroke}{rgb}{0.545524,0.838039,0.275626}%
\pgfsetstrokecolor{currentstroke}%
\pgfsetdash{}{0pt}%
\pgfpathmoveto{\pgfqpoint{3.501582in}{4.412697in}}%
\pgfpathlineto{\pgfqpoint{3.635593in}{4.419577in}}%
\pgfpathlineto{\pgfqpoint{3.583999in}{4.444939in}}%
\pgfpathclose%
\pgfusepath{fill}%
\end{pgfscope}%
\begin{pgfscope}%
\pgfpathrectangle{\pgfqpoint{0.680860in}{0.078740in}}{\pgfqpoint{7.842520in}{7.842520in}}%
\pgfusepath{clip}%
\pgfsetbuttcap%
\pgfsetroundjoin%
\definecolor{currentfill}{rgb}{0.212395,0.359683,0.551710}%
\pgfsetfillcolor{currentfill}%
\pgfsetlinewidth{0.000000pt}%
\definecolor{currentstroke}{rgb}{0.555484,0.840254,0.269281}%
\pgfsetstrokecolor{currentstroke}%
\pgfsetdash{}{0pt}%
\pgfpathmoveto{\pgfqpoint{7.298112in}{3.906438in}}%
\pgfpathlineto{\pgfqpoint{7.219105in}{3.722192in}}%
\pgfpathlineto{\pgfqpoint{7.156184in}{3.912059in}}%
\pgfpathclose%
\pgfusepath{fill}%
\end{pgfscope}%
\begin{pgfscope}%
\pgfpathrectangle{\pgfqpoint{0.680860in}{0.078740in}}{\pgfqpoint{7.842520in}{7.842520in}}%
\pgfusepath{clip}%
\pgfsetbuttcap%
\pgfsetroundjoin%
\definecolor{currentfill}{rgb}{0.243113,0.292092,0.538516}%
\pgfsetfillcolor{currentfill}%
\pgfsetlinewidth{0.000000pt}%
\definecolor{currentstroke}{rgb}{0.565498,0.842430,0.262877}%
\pgfsetstrokecolor{currentstroke}%
\pgfsetdash{}{0pt}%
\pgfpathmoveto{\pgfqpoint{7.786100in}{3.670222in}}%
\pgfpathlineto{\pgfqpoint{7.843295in}{3.438610in}}%
\pgfpathlineto{\pgfqpoint{7.643458in}{3.682018in}}%
\pgfpathclose%
\pgfusepath{fill}%
\end{pgfscope}%
\begin{pgfscope}%
\pgfpathrectangle{\pgfqpoint{0.680860in}{0.078740in}}{\pgfqpoint{7.842520in}{7.842520in}}%
\pgfusepath{clip}%
\pgfsetbuttcap%
\pgfsetroundjoin%
\definecolor{currentfill}{rgb}{0.163625,0.471133,0.558148}%
\pgfsetfillcolor{currentfill}%
\pgfsetlinewidth{0.000000pt}%
\definecolor{currentstroke}{rgb}{0.575563,0.844566,0.256415}%
\pgfsetstrokecolor{currentstroke}%
\pgfsetdash{}{0pt}%
\pgfpathmoveto{\pgfqpoint{3.068765in}{4.294740in}}%
\pgfpathlineto{\pgfqpoint{3.152120in}{4.355705in}}%
\pgfpathlineto{\pgfqpoint{2.936032in}{4.291388in}}%
\pgfpathclose%
\pgfusepath{fill}%
\end{pgfscope}%
\begin{pgfscope}%
\pgfpathrectangle{\pgfqpoint{0.680860in}{0.078740in}}{\pgfqpoint{7.842520in}{7.842520in}}%
\pgfusepath{clip}%
\pgfsetbuttcap%
\pgfsetroundjoin%
\definecolor{currentfill}{rgb}{0.149039,0.508051,0.557250}%
\pgfsetfillcolor{currentfill}%
\pgfsetlinewidth{0.000000pt}%
\definecolor{currentstroke}{rgb}{0.585678,0.846661,0.249897}%
\pgfsetstrokecolor{currentstroke}%
\pgfsetdash{}{0pt}%
\pgfpathmoveto{\pgfqpoint{4.849726in}{4.449783in}}%
\pgfpathlineto{\pgfqpoint{4.908499in}{4.498984in}}%
\pgfpathlineto{\pgfqpoint{4.986329in}{4.455807in}}%
\pgfpathclose%
\pgfusepath{fill}%
\end{pgfscope}%
\begin{pgfscope}%
\pgfpathrectangle{\pgfqpoint{0.680860in}{0.078740in}}{\pgfqpoint{7.842520in}{7.842520in}}%
\pgfusepath{clip}%
\pgfsetbuttcap%
\pgfsetroundjoin%
\definecolor{currentfill}{rgb}{0.163625,0.471133,0.558148}%
\pgfsetfillcolor{currentfill}%
\pgfsetlinewidth{0.000000pt}%
\definecolor{currentstroke}{rgb}{0.595839,0.848717,0.243329}%
\pgfsetstrokecolor{currentstroke}%
\pgfsetdash{}{0pt}%
\pgfpathmoveto{\pgfqpoint{5.827018in}{4.336945in}}%
\pgfpathlineto{\pgfqpoint{5.966319in}{4.342261in}}%
\pgfpathlineto{\pgfqpoint{5.899703in}{4.231441in}}%
\pgfpathclose%
\pgfusepath{fill}%
\end{pgfscope}%
\begin{pgfscope}%
\pgfpathrectangle{\pgfqpoint{0.680860in}{0.078740in}}{\pgfqpoint{7.842520in}{7.842520in}}%
\pgfusepath{clip}%
\pgfsetbuttcap%
\pgfsetroundjoin%
\definecolor{currentfill}{rgb}{0.151918,0.500685,0.557587}%
\pgfsetfillcolor{currentfill}%
\pgfsetlinewidth{0.000000pt}%
\definecolor{currentstroke}{rgb}{0.606045,0.850733,0.236712}%
\pgfsetstrokecolor{currentstroke}%
\pgfsetdash{}{0pt}%
\pgfpathmoveto{\pgfqpoint{5.337540in}{4.407155in}}%
\pgfpathlineto{\pgfqpoint{5.200180in}{4.402421in}}%
\pgfpathlineto{\pgfqpoint{5.261392in}{4.472255in}}%
\pgfpathclose%
\pgfusepath{fill}%
\end{pgfscope}%
\begin{pgfscope}%
\pgfpathrectangle{\pgfqpoint{0.680860in}{0.078740in}}{\pgfqpoint{7.842520in}{7.842520in}}%
\pgfusepath{clip}%
\pgfsetbuttcap%
\pgfsetroundjoin%
\definecolor{currentfill}{rgb}{0.146180,0.515413,0.556823}%
\pgfsetfillcolor{currentfill}%
\pgfsetlinewidth{0.000000pt}%
\definecolor{currentstroke}{rgb}{0.616293,0.852709,0.230052}%
\pgfsetstrokecolor{currentstroke}%
\pgfsetdash{}{0pt}%
\pgfpathmoveto{\pgfqpoint{4.203594in}{4.489212in}}%
\pgfpathlineto{\pgfqpoint{4.284104in}{4.486484in}}%
\pgfpathlineto{\pgfqpoint{4.068337in}{4.480616in}}%
\pgfpathclose%
\pgfusepath{fill}%
\end{pgfscope}%
\begin{pgfscope}%
\pgfpathrectangle{\pgfqpoint{0.680860in}{0.078740in}}{\pgfqpoint{7.842520in}{7.842520in}}%
\pgfusepath{clip}%
\pgfsetbuttcap%
\pgfsetroundjoin%
\definecolor{currentfill}{rgb}{0.188923,0.410910,0.556326}%
\pgfsetfillcolor{currentfill}%
\pgfsetlinewidth{0.000000pt}%
\definecolor{currentstroke}{rgb}{0.626579,0.854645,0.223353}%
\pgfsetstrokecolor{currentstroke}%
\pgfsetdash{}{0pt}%
\pgfpathmoveto{\pgfqpoint{6.667203in}{4.093383in}}%
\pgfpathlineto{\pgfqpoint{6.808197in}{4.092323in}}%
\pgfpathlineto{\pgfqpoint{6.874182in}{3.926322in}}%
\pgfpathclose%
\pgfusepath{fill}%
\end{pgfscope}%
\begin{pgfscope}%
\pgfpathrectangle{\pgfqpoint{0.680860in}{0.078740in}}{\pgfqpoint{7.842520in}{7.842520in}}%
\pgfusepath{clip}%
\pgfsetbuttcap%
\pgfsetroundjoin%
\definecolor{currentfill}{rgb}{0.154815,0.493313,0.557840}%
\pgfsetfillcolor{currentfill}%
\pgfsetlinewidth{0.000000pt}%
\definecolor{currentstroke}{rgb}{0.636902,0.856542,0.216620}%
\pgfsetstrokecolor{currentstroke}%
\pgfsetdash{}{0pt}%
\pgfpathmoveto{\pgfqpoint{3.501582in}{4.412697in}}%
\pgfpathlineto{\pgfqpoint{3.368141in}{4.407243in}}%
\pgfpathlineto{\pgfqpoint{3.418868in}{4.366343in}}%
\pgfpathclose%
\pgfusepath{fill}%
\end{pgfscope}%
\begin{pgfscope}%
\pgfpathrectangle{\pgfqpoint{0.680860in}{0.078740in}}{\pgfqpoint{7.842520in}{7.842520in}}%
\pgfusepath{clip}%
\pgfsetbuttcap%
\pgfsetroundjoin%
\definecolor{currentfill}{rgb}{0.168126,0.459988,0.558082}%
\pgfsetfillcolor{currentfill}%
\pgfsetlinewidth{0.000000pt}%
\definecolor{currentstroke}{rgb}{0.647257,0.858400,0.209861}%
\pgfsetstrokecolor{currentstroke}%
\pgfsetdash{}{0pt}%
\pgfpathmoveto{\pgfqpoint{2.719957in}{4.215434in}}%
\pgfpathlineto{\pgfqpoint{2.852326in}{4.217128in}}%
\pgfpathlineto{\pgfqpoint{2.936032in}{4.291388in}}%
\pgfpathclose%
\pgfusepath{fill}%
\end{pgfscope}%
\begin{pgfscope}%
\pgfpathrectangle{\pgfqpoint{0.680860in}{0.078740in}}{\pgfqpoint{7.842520in}{7.842520in}}%
\pgfusepath{clip}%
\pgfsetbuttcap%
\pgfsetroundjoin%
\definecolor{currentfill}{rgb}{0.146180,0.515413,0.556823}%
\pgfsetfillcolor{currentfill}%
\pgfsetlinewidth{0.000000pt}%
\definecolor{currentstroke}{rgb}{0.657642,0.860219,0.203082}%
\pgfsetstrokecolor{currentstroke}%
\pgfsetdash{}{0pt}%
\pgfpathmoveto{\pgfqpoint{4.771480in}{4.489440in}}%
\pgfpathlineto{\pgfqpoint{4.908499in}{4.498984in}}%
\pgfpathlineto{\pgfqpoint{4.849726in}{4.449783in}}%
\pgfpathclose%
\pgfusepath{fill}%
\end{pgfscope}%
\begin{pgfscope}%
\pgfpathrectangle{\pgfqpoint{0.680860in}{0.078740in}}{\pgfqpoint{7.842520in}{7.842520in}}%
\pgfusepath{clip}%
\pgfsetbuttcap%
\pgfsetroundjoin%
\definecolor{currentfill}{rgb}{0.154815,0.493313,0.557840}%
\pgfsetfillcolor{currentfill}%
\pgfsetlinewidth{0.000000pt}%
\definecolor{currentstroke}{rgb}{0.668054,0.861999,0.196293}%
\pgfsetstrokecolor{currentstroke}%
\pgfsetdash{}{0pt}%
\pgfpathmoveto{\pgfqpoint{3.418868in}{4.366343in}}%
\pgfpathlineto{\pgfqpoint{3.368141in}{4.407243in}}%
\pgfpathlineto{\pgfqpoint{3.285212in}{4.360324in}}%
\pgfpathclose%
\pgfusepath{fill}%
\end{pgfscope}%
\begin{pgfscope}%
\pgfpathrectangle{\pgfqpoint{0.680860in}{0.078740in}}{\pgfqpoint{7.842520in}{7.842520in}}%
\pgfusepath{clip}%
\pgfsetbuttcap%
\pgfsetroundjoin%
\definecolor{currentfill}{rgb}{0.149039,0.508051,0.557250}%
\pgfsetfillcolor{currentfill}%
\pgfsetlinewidth{0.000000pt}%
\definecolor{currentstroke}{rgb}{0.678489,0.863742,0.189503}%
\pgfsetstrokecolor{currentstroke}%
\pgfsetdash{}{0pt}%
\pgfpathmoveto{\pgfqpoint{5.200180in}{4.402421in}}%
\pgfpathlineto{\pgfqpoint{5.123547in}{4.463279in}}%
\pgfpathlineto{\pgfqpoint{5.261392in}{4.472255in}}%
\pgfpathclose%
\pgfusepath{fill}%
\end{pgfscope}%
\begin{pgfscope}%
\pgfpathrectangle{\pgfqpoint{0.680860in}{0.078740in}}{\pgfqpoint{7.842520in}{7.842520in}}%
\pgfusepath{clip}%
\pgfsetbuttcap%
\pgfsetroundjoin%
\definecolor{currentfill}{rgb}{0.149039,0.508051,0.557250}%
\pgfsetfillcolor{currentfill}%
\pgfsetlinewidth{0.000000pt}%
\definecolor{currentstroke}{rgb}{0.688944,0.865448,0.182725}%
\pgfsetstrokecolor{currentstroke}%
\pgfsetdash{}{0pt}%
\pgfpathmoveto{\pgfqpoint{3.717773in}{4.450757in}}%
\pgfpathlineto{\pgfqpoint{3.635593in}{4.419577in}}%
\pgfpathlineto{\pgfqpoint{3.852122in}{4.458012in}}%
\pgfpathclose%
\pgfusepath{fill}%
\end{pgfscope}%
\begin{pgfscope}%
\pgfpathrectangle{\pgfqpoint{0.680860in}{0.078740in}}{\pgfqpoint{7.842520in}{7.842520in}}%
\pgfusepath{clip}%
\pgfsetbuttcap%
\pgfsetroundjoin%
\definecolor{currentfill}{rgb}{0.175841,0.441290,0.557685}%
\pgfsetfillcolor{currentfill}%
\pgfsetlinewidth{0.000000pt}%
\definecolor{currentstroke}{rgb}{0.699415,0.867117,0.175971}%
\pgfsetstrokecolor{currentstroke}%
\pgfsetdash{}{0pt}%
\pgfpathmoveto{\pgfqpoint{2.503961in}{4.129048in}}%
\pgfpathlineto{\pgfqpoint{2.635965in}{4.128731in}}%
\pgfpathlineto{\pgfqpoint{2.719957in}{4.215434in}}%
\pgfpathclose%
\pgfusepath{fill}%
\end{pgfscope}%
\begin{pgfscope}%
\pgfpathrectangle{\pgfqpoint{0.680860in}{0.078740in}}{\pgfqpoint{7.842520in}{7.842520in}}%
\pgfusepath{clip}%
\pgfsetbuttcap%
\pgfsetroundjoin%
\definecolor{currentfill}{rgb}{0.144759,0.519093,0.556572}%
\pgfsetfillcolor{currentfill}%
\pgfsetlinewidth{0.000000pt}%
\definecolor{currentstroke}{rgb}{0.709898,0.868751,0.169257}%
\pgfsetstrokecolor{currentstroke}%
\pgfsetdash{}{0pt}%
\pgfpathmoveto{\pgfqpoint{4.419650in}{4.494358in}}%
\pgfpathlineto{\pgfqpoint{4.555801in}{4.503746in}}%
\pgfpathlineto{\pgfqpoint{4.635079in}{4.481417in}}%
\pgfpathclose%
\pgfusepath{fill}%
\end{pgfscope}%
\begin{pgfscope}%
\pgfpathrectangle{\pgfqpoint{0.680860in}{0.078740in}}{\pgfqpoint{7.842520in}{7.842520in}}%
\pgfusepath{clip}%
\pgfsetbuttcap%
\pgfsetroundjoin%
\definecolor{currentfill}{rgb}{0.163625,0.471133,0.558148}%
\pgfsetfillcolor{currentfill}%
\pgfsetlinewidth{0.000000pt}%
\definecolor{currentstroke}{rgb}{0.720391,0.870350,0.162603}%
\pgfsetstrokecolor{currentstroke}%
\pgfsetdash{}{0pt}%
\pgfpathmoveto{\pgfqpoint{6.106265in}{4.349010in}}%
\pgfpathlineto{\pgfqpoint{6.177614in}{4.231807in}}%
\pgfpathlineto{\pgfqpoint{6.038351in}{4.231012in}}%
\pgfpathclose%
\pgfusepath{fill}%
\end{pgfscope}%
\begin{pgfscope}%
\pgfpathrectangle{\pgfqpoint{0.680860in}{0.078740in}}{\pgfqpoint{7.842520in}{7.842520in}}%
\pgfusepath{clip}%
\pgfsetbuttcap%
\pgfsetroundjoin%
\definecolor{currentfill}{rgb}{0.157729,0.485932,0.558013}%
\pgfsetfillcolor{currentfill}%
\pgfsetlinewidth{0.000000pt}%
\definecolor{currentstroke}{rgb}{0.730889,0.871916,0.156029}%
\pgfsetstrokecolor{currentstroke}%
\pgfsetdash{}{0pt}%
\pgfpathmoveto{\pgfqpoint{3.152120in}{4.355705in}}%
\pgfpathlineto{\pgfqpoint{3.202053in}{4.299450in}}%
\pgfpathlineto{\pgfqpoint{3.285212in}{4.360324in}}%
\pgfpathclose%
\pgfusepath{fill}%
\end{pgfscope}%
\begin{pgfscope}%
\pgfpathrectangle{\pgfqpoint{0.680860in}{0.078740in}}{\pgfqpoint{7.842520in}{7.842520in}}%
\pgfusepath{clip}%
\pgfsetbuttcap%
\pgfsetroundjoin%
\definecolor{currentfill}{rgb}{0.153364,0.497000,0.557724}%
\pgfsetfillcolor{currentfill}%
\pgfsetlinewidth{0.000000pt}%
\definecolor{currentstroke}{rgb}{0.741388,0.873449,0.149561}%
\pgfsetstrokecolor{currentstroke}%
\pgfsetdash{}{0pt}%
\pgfpathmoveto{\pgfqpoint{5.475520in}{4.413292in}}%
\pgfpathlineto{\pgfqpoint{5.614131in}{4.420886in}}%
\pgfpathlineto{\pgfqpoint{5.688350in}{4.333008in}}%
\pgfpathclose%
\pgfusepath{fill}%
\end{pgfscope}%
\begin{pgfscope}%
\pgfpathrectangle{\pgfqpoint{0.680860in}{0.078740in}}{\pgfqpoint{7.842520in}{7.842520in}}%
\pgfusepath{clip}%
\pgfsetbuttcap%
\pgfsetroundjoin%
\definecolor{currentfill}{rgb}{0.143343,0.522773,0.556295}%
\pgfsetfillcolor{currentfill}%
\pgfsetlinewidth{0.000000pt}%
\definecolor{currentstroke}{rgb}{0.751884,0.874951,0.143228}%
\pgfsetstrokecolor{currentstroke}%
\pgfsetdash{}{0pt}%
\pgfpathmoveto{\pgfqpoint{4.284104in}{4.486484in}}%
\pgfpathlineto{\pgfqpoint{4.339456in}{4.499350in}}%
\pgfpathlineto{\pgfqpoint{4.419650in}{4.494358in}}%
\pgfpathclose%
\pgfusepath{fill}%
\end{pgfscope}%
\begin{pgfscope}%
\pgfpathrectangle{\pgfqpoint{0.680860in}{0.078740in}}{\pgfqpoint{7.842520in}{7.842520in}}%
\pgfusepath{clip}%
\pgfsetbuttcap%
\pgfsetroundjoin%
\definecolor{currentfill}{rgb}{0.175841,0.441290,0.557685}%
\pgfsetfillcolor{currentfill}%
\pgfsetlinewidth{0.000000pt}%
\definecolor{currentstroke}{rgb}{0.762373,0.876424,0.137064}%
\pgfsetstrokecolor{currentstroke}%
\pgfsetdash{}{0pt}%
\pgfpathmoveto{\pgfqpoint{6.667203in}{4.093383in}}%
\pgfpathlineto{\pgfqpoint{6.526838in}{4.095612in}}%
\pgfpathlineto{\pgfqpoint{6.458040in}{4.237255in}}%
\pgfpathclose%
\pgfusepath{fill}%
\end{pgfscope}%
\begin{pgfscope}%
\pgfpathrectangle{\pgfqpoint{0.680860in}{0.078740in}}{\pgfqpoint{7.842520in}{7.842520in}}%
\pgfusepath{clip}%
\pgfsetbuttcap%
\pgfsetroundjoin%
\definecolor{currentfill}{rgb}{0.159194,0.482237,0.558073}%
\pgfsetfillcolor{currentfill}%
\pgfsetlinewidth{0.000000pt}%
\definecolor{currentstroke}{rgb}{0.772852,0.877868,0.131109}%
\pgfsetstrokecolor{currentstroke}%
\pgfsetdash{}{0pt}%
\pgfpathmoveto{\pgfqpoint{3.068765in}{4.294740in}}%
\pgfpathlineto{\pgfqpoint{3.202053in}{4.299450in}}%
\pgfpathlineto{\pgfqpoint{3.152120in}{4.355705in}}%
\pgfpathclose%
\pgfusepath{fill}%
\end{pgfscope}%
\begin{pgfscope}%
\pgfpathrectangle{\pgfqpoint{0.680860in}{0.078740in}}{\pgfqpoint{7.842520in}{7.842520in}}%
\pgfusepath{clip}%
\pgfsetbuttcap%
\pgfsetroundjoin%
\definecolor{currentfill}{rgb}{0.144759,0.519093,0.556572}%
\pgfsetfillcolor{currentfill}%
\pgfsetlinewidth{0.000000pt}%
\definecolor{currentstroke}{rgb}{0.783315,0.879285,0.125405}%
\pgfsetstrokecolor{currentstroke}%
\pgfsetdash{}{0pt}%
\pgfpathmoveto{\pgfqpoint{4.908499in}{4.498984in}}%
\pgfpathlineto{\pgfqpoint{5.123547in}{4.463279in}}%
\pgfpathlineto{\pgfqpoint{4.986329in}{4.455807in}}%
\pgfpathclose%
\pgfusepath{fill}%
\end{pgfscope}%
\begin{pgfscope}%
\pgfpathrectangle{\pgfqpoint{0.680860in}{0.078740in}}{\pgfqpoint{7.842520in}{7.842520in}}%
\pgfusepath{clip}%
\pgfsetbuttcap%
\pgfsetroundjoin%
\definecolor{currentfill}{rgb}{0.171176,0.452530,0.557965}%
\pgfsetfillcolor{currentfill}%
\pgfsetlinewidth{0.000000pt}%
\definecolor{currentstroke}{rgb}{0.793760,0.880678,0.120005}%
\pgfsetstrokecolor{currentstroke}%
\pgfsetdash{}{0pt}%
\pgfpathmoveto{\pgfqpoint{2.852326in}{4.217128in}}%
\pgfpathlineto{\pgfqpoint{2.719957in}{4.215434in}}%
\pgfpathlineto{\pgfqpoint{2.635965in}{4.128731in}}%
\pgfpathclose%
\pgfusepath{fill}%
\end{pgfscope}%
\begin{pgfscope}%
\pgfpathrectangle{\pgfqpoint{0.680860in}{0.078740in}}{\pgfqpoint{7.842520in}{7.842520in}}%
\pgfusepath{clip}%
\pgfsetbuttcap%
\pgfsetroundjoin%
\definecolor{currentfill}{rgb}{0.144759,0.519093,0.556572}%
\pgfsetfillcolor{currentfill}%
\pgfsetlinewidth{0.000000pt}%
\definecolor{currentstroke}{rgb}{0.804182,0.882046,0.114965}%
\pgfsetstrokecolor{currentstroke}%
\pgfsetdash{}{0pt}%
\pgfpathmoveto{\pgfqpoint{3.852122in}{4.458012in}}%
\pgfpathlineto{\pgfqpoint{3.987060in}{4.466760in}}%
\pgfpathlineto{\pgfqpoint{4.068337in}{4.480616in}}%
\pgfpathclose%
\pgfusepath{fill}%
\end{pgfscope}%
\begin{pgfscope}%
\pgfpathrectangle{\pgfqpoint{0.680860in}{0.078740in}}{\pgfqpoint{7.842520in}{7.842520in}}%
\pgfusepath{clip}%
\pgfsetbuttcap%
\pgfsetroundjoin%
\definecolor{currentfill}{rgb}{0.218130,0.347432,0.550038}%
\pgfsetfillcolor{currentfill}%
\pgfsetlinewidth{0.000000pt}%
\definecolor{currentstroke}{rgb}{0.814576,0.883393,0.110347}%
\pgfsetstrokecolor{currentstroke}%
\pgfsetdash{}{0pt}%
\pgfpathmoveto{\pgfqpoint{7.643458in}{3.682018in}}%
\pgfpathlineto{\pgfqpoint{7.501418in}{3.694632in}}%
\pgfpathlineto{\pgfqpoint{7.440672in}{3.901876in}}%
\pgfpathclose%
\pgfusepath{fill}%
\end{pgfscope}%
\begin{pgfscope}%
\pgfpathrectangle{\pgfqpoint{0.680860in}{0.078740in}}{\pgfqpoint{7.842520in}{7.842520in}}%
\pgfusepath{clip}%
\pgfsetbuttcap%
\pgfsetroundjoin%
\definecolor{currentfill}{rgb}{0.143343,0.522773,0.556295}%
\pgfsetfillcolor{currentfill}%
\pgfsetlinewidth{0.000000pt}%
\definecolor{currentstroke}{rgb}{0.824940,0.884720,0.106217}%
\pgfsetstrokecolor{currentstroke}%
\pgfsetdash{}{0pt}%
\pgfpathmoveto{\pgfqpoint{4.284104in}{4.486484in}}%
\pgfpathlineto{\pgfqpoint{4.203594in}{4.489212in}}%
\pgfpathlineto{\pgfqpoint{4.339456in}{4.499350in}}%
\pgfpathclose%
\pgfusepath{fill}%
\end{pgfscope}%
\begin{pgfscope}%
\pgfpathrectangle{\pgfqpoint{0.680860in}{0.078740in}}{\pgfqpoint{7.842520in}{7.842520in}}%
\pgfusepath{clip}%
\pgfsetbuttcap%
\pgfsetroundjoin%
\definecolor{currentfill}{rgb}{0.163625,0.471133,0.558148}%
\pgfsetfillcolor{currentfill}%
\pgfsetlinewidth{0.000000pt}%
\definecolor{currentstroke}{rgb}{0.835270,0.886029,0.102646}%
\pgfsetstrokecolor{currentstroke}%
\pgfsetdash{}{0pt}%
\pgfpathmoveto{\pgfqpoint{3.068765in}{4.294740in}}%
\pgfpathlineto{\pgfqpoint{2.936032in}{4.291388in}}%
\pgfpathlineto{\pgfqpoint{2.985240in}{4.220121in}}%
\pgfpathclose%
\pgfusepath{fill}%
\end{pgfscope}%
\begin{pgfscope}%
\pgfpathrectangle{\pgfqpoint{0.680860in}{0.078740in}}{\pgfqpoint{7.842520in}{7.842520in}}%
\pgfusepath{clip}%
\pgfsetbuttcap%
\pgfsetroundjoin%
\definecolor{currentfill}{rgb}{0.192357,0.403199,0.555836}%
\pgfsetfillcolor{currentfill}%
\pgfsetlinewidth{0.000000pt}%
\definecolor{currentstroke}{rgb}{0.845561,0.887322,0.099702}%
\pgfsetstrokecolor{currentstroke}%
\pgfsetdash{}{0pt}%
\pgfpathmoveto{\pgfqpoint{7.156184in}{3.912059in}}%
\pgfpathlineto{\pgfqpoint{7.014878in}{3.918699in}}%
\pgfpathlineto{\pgfqpoint{6.949832in}{4.092476in}}%
\pgfpathclose%
\pgfusepath{fill}%
\end{pgfscope}%
\begin{pgfscope}%
\pgfpathrectangle{\pgfqpoint{0.680860in}{0.078740in}}{\pgfqpoint{7.842520in}{7.842520in}}%
\pgfusepath{clip}%
\pgfsetbuttcap%
\pgfsetroundjoin%
\definecolor{currentfill}{rgb}{0.169646,0.456262,0.558030}%
\pgfsetfillcolor{currentfill}%
\pgfsetlinewidth{0.000000pt}%
\definecolor{currentstroke}{rgb}{0.855810,0.888601,0.097452}%
\pgfsetstrokecolor{currentstroke}%
\pgfsetdash{}{0pt}%
\pgfpathmoveto{\pgfqpoint{6.458040in}{4.237255in}}%
\pgfpathlineto{\pgfqpoint{6.526838in}{4.095612in}}%
\pgfpathlineto{\pgfqpoint{6.317507in}{4.233872in}}%
\pgfpathclose%
\pgfusepath{fill}%
\end{pgfscope}%
\begin{pgfscope}%
\pgfpathrectangle{\pgfqpoint{0.680860in}{0.078740in}}{\pgfqpoint{7.842520in}{7.842520in}}%
\pgfusepath{clip}%
\pgfsetbuttcap%
\pgfsetroundjoin%
\definecolor{currentfill}{rgb}{0.150476,0.504369,0.557430}%
\pgfsetfillcolor{currentfill}%
\pgfsetlinewidth{0.000000pt}%
\definecolor{currentstroke}{rgb}{0.866013,0.889868,0.095953}%
\pgfsetstrokecolor{currentstroke}%
\pgfsetdash{}{0pt}%
\pgfpathmoveto{\pgfqpoint{3.501582in}{4.412697in}}%
\pgfpathlineto{\pgfqpoint{3.418868in}{4.366343in}}%
\pgfpathlineto{\pgfqpoint{3.635593in}{4.419577in}}%
\pgfpathclose%
\pgfusepath{fill}%
\end{pgfscope}%
\begin{pgfscope}%
\pgfpathrectangle{\pgfqpoint{0.680860in}{0.078740in}}{\pgfqpoint{7.842520in}{7.842520in}}%
\pgfusepath{clip}%
\pgfsetbuttcap%
\pgfsetroundjoin%
\definecolor{currentfill}{rgb}{0.153364,0.497000,0.557724}%
\pgfsetfillcolor{currentfill}%
\pgfsetlinewidth{0.000000pt}%
\definecolor{currentstroke}{rgb}{0.876168,0.891125,0.095250}%
\pgfsetstrokecolor{currentstroke}%
\pgfsetdash{}{0pt}%
\pgfpathmoveto{\pgfqpoint{5.688350in}{4.333008in}}%
\pgfpathlineto{\pgfqpoint{5.753388in}{4.429993in}}%
\pgfpathlineto{\pgfqpoint{5.827018in}{4.336945in}}%
\pgfpathclose%
\pgfusepath{fill}%
\end{pgfscope}%
\begin{pgfscope}%
\pgfpathrectangle{\pgfqpoint{0.680860in}{0.078740in}}{\pgfqpoint{7.842520in}{7.842520in}}%
\pgfusepath{clip}%
\pgfsetbuttcap%
\pgfsetroundjoin%
\definecolor{currentfill}{rgb}{0.179019,0.433756,0.557430}%
\pgfsetfillcolor{currentfill}%
\pgfsetlinewidth{0.000000pt}%
\definecolor{currentstroke}{rgb}{0.886271,0.892374,0.095374}%
\pgfsetstrokecolor{currentstroke}%
\pgfsetdash{}{0pt}%
\pgfpathmoveto{\pgfqpoint{2.503961in}{4.129048in}}%
\pgfpathlineto{\pgfqpoint{2.419739in}{4.030838in}}%
\pgfpathlineto{\pgfqpoint{2.635965in}{4.128731in}}%
\pgfpathclose%
\pgfusepath{fill}%
\end{pgfscope}%
\begin{pgfscope}%
\pgfpathrectangle{\pgfqpoint{0.680860in}{0.078740in}}{\pgfqpoint{7.842520in}{7.842520in}}%
\pgfusepath{clip}%
\pgfsetbuttcap%
\pgfsetroundjoin%
\definecolor{currentfill}{rgb}{0.141935,0.526453,0.555991}%
\pgfsetfillcolor{currentfill}%
\pgfsetlinewidth{0.000000pt}%
\definecolor{currentstroke}{rgb}{0.896320,0.893616,0.096335}%
\pgfsetstrokecolor{currentstroke}%
\pgfsetdash{}{0pt}%
\pgfpathmoveto{\pgfqpoint{4.771480in}{4.489440in}}%
\pgfpathlineto{\pgfqpoint{4.635079in}{4.481417in}}%
\pgfpathlineto{\pgfqpoint{4.692572in}{4.514706in}}%
\pgfpathclose%
\pgfusepath{fill}%
\end{pgfscope}%
\begin{pgfscope}%
\pgfpathrectangle{\pgfqpoint{0.680860in}{0.078740in}}{\pgfqpoint{7.842520in}{7.842520in}}%
\pgfusepath{clip}%
\pgfsetbuttcap%
\pgfsetroundjoin%
\definecolor{currentfill}{rgb}{0.143343,0.522773,0.556295}%
\pgfsetfillcolor{currentfill}%
\pgfsetlinewidth{0.000000pt}%
\definecolor{currentstroke}{rgb}{0.906311,0.894855,0.098125}%
\pgfsetstrokecolor{currentstroke}%
\pgfsetdash{}{0pt}%
\pgfpathmoveto{\pgfqpoint{4.203594in}{4.489212in}}%
\pgfpathlineto{\pgfqpoint{4.068337in}{4.480616in}}%
\pgfpathlineto{\pgfqpoint{3.987060in}{4.466760in}}%
\pgfpathclose%
\pgfusepath{fill}%
\end{pgfscope}%
\begin{pgfscope}%
\pgfpathrectangle{\pgfqpoint{0.680860in}{0.078740in}}{\pgfqpoint{7.842520in}{7.842520in}}%
\pgfusepath{clip}%
\pgfsetbuttcap%
\pgfsetroundjoin%
\definecolor{currentfill}{rgb}{0.165117,0.467423,0.558141}%
\pgfsetfillcolor{currentfill}%
\pgfsetlinewidth{0.000000pt}%
\definecolor{currentstroke}{rgb}{0.916242,0.896091,0.100717}%
\pgfsetstrokecolor{currentstroke}%
\pgfsetdash{}{0pt}%
\pgfpathmoveto{\pgfqpoint{2.985240in}{4.220121in}}%
\pgfpathlineto{\pgfqpoint{2.936032in}{4.291388in}}%
\pgfpathlineto{\pgfqpoint{2.852326in}{4.217128in}}%
\pgfpathclose%
\pgfusepath{fill}%
\end{pgfscope}%
\begin{pgfscope}%
\pgfpathrectangle{\pgfqpoint{0.680860in}{0.078740in}}{\pgfqpoint{7.842520in}{7.842520in}}%
\pgfusepath{clip}%
\pgfsetbuttcap%
\pgfsetroundjoin%
\definecolor{currentfill}{rgb}{0.159194,0.482237,0.558073}%
\pgfsetfillcolor{currentfill}%
\pgfsetlinewidth{0.000000pt}%
\definecolor{currentstroke}{rgb}{0.926106,0.897330,0.104071}%
\pgfsetstrokecolor{currentstroke}%
\pgfsetdash{}{0pt}%
\pgfpathmoveto{\pgfqpoint{6.106265in}{4.349010in}}%
\pgfpathlineto{\pgfqpoint{6.038351in}{4.231012in}}%
\pgfpathlineto{\pgfqpoint{5.966319in}{4.342261in}}%
\pgfpathclose%
\pgfusepath{fill}%
\end{pgfscope}%
\begin{pgfscope}%
\pgfpathrectangle{\pgfqpoint{0.680860in}{0.078740in}}{\pgfqpoint{7.842520in}{7.842520in}}%
\pgfusepath{clip}%
\pgfsetbuttcap%
\pgfsetroundjoin%
\definecolor{currentfill}{rgb}{0.147607,0.511733,0.557049}%
\pgfsetfillcolor{currentfill}%
\pgfsetlinewidth{0.000000pt}%
\definecolor{currentstroke}{rgb}{0.935904,0.898570,0.108131}%
\pgfsetstrokecolor{currentstroke}%
\pgfsetdash{}{0pt}%
\pgfpathmoveto{\pgfqpoint{5.399877in}{4.482791in}}%
\pgfpathlineto{\pgfqpoint{5.475520in}{4.413292in}}%
\pgfpathlineto{\pgfqpoint{5.337540in}{4.407155in}}%
\pgfpathclose%
\pgfusepath{fill}%
\end{pgfscope}%
\begin{pgfscope}%
\pgfpathrectangle{\pgfqpoint{0.680860in}{0.078740in}}{\pgfqpoint{7.842520in}{7.842520in}}%
\pgfusepath{clip}%
\pgfsetbuttcap%
\pgfsetroundjoin%
\definecolor{currentfill}{rgb}{0.183898,0.422383,0.556944}%
\pgfsetfillcolor{currentfill}%
\pgfsetlinewidth{0.000000pt}%
\definecolor{currentstroke}{rgb}{0.945636,0.899815,0.112838}%
\pgfsetstrokecolor{currentstroke}%
\pgfsetdash{}{0pt}%
\pgfpathmoveto{\pgfqpoint{7.014878in}{3.918699in}}%
\pgfpathlineto{\pgfqpoint{6.808197in}{4.092323in}}%
\pgfpathlineto{\pgfqpoint{6.949832in}{4.092476in}}%
\pgfpathclose%
\pgfusepath{fill}%
\end{pgfscope}%
\begin{pgfscope}%
\pgfpathrectangle{\pgfqpoint{0.680860in}{0.078740in}}{\pgfqpoint{7.842520in}{7.842520in}}%
\pgfusepath{clip}%
\pgfsetbuttcap%
\pgfsetroundjoin%
\definecolor{currentfill}{rgb}{0.206756,0.371758,0.553117}%
\pgfsetfillcolor{currentfill}%
\pgfsetlinewidth{0.000000pt}%
\definecolor{currentstroke}{rgb}{0.955300,0.901065,0.118128}%
\pgfsetstrokecolor{currentstroke}%
\pgfsetdash{}{0pt}%
\pgfpathmoveto{\pgfqpoint{7.440672in}{3.901876in}}%
\pgfpathlineto{\pgfqpoint{7.501418in}{3.694632in}}%
\pgfpathlineto{\pgfqpoint{7.298112in}{3.906438in}}%
\pgfpathclose%
\pgfusepath{fill}%
\end{pgfscope}%
\begin{pgfscope}%
\pgfpathrectangle{\pgfqpoint{0.680860in}{0.078740in}}{\pgfqpoint{7.842520in}{7.842520in}}%
\pgfusepath{clip}%
\pgfsetbuttcap%
\pgfsetroundjoin%
\definecolor{currentfill}{rgb}{0.140536,0.530132,0.555659}%
\pgfsetfillcolor{currentfill}%
\pgfsetlinewidth{0.000000pt}%
\definecolor{currentstroke}{rgb}{0.964894,0.902323,0.123941}%
\pgfsetstrokecolor{currentstroke}%
\pgfsetdash{}{0pt}%
\pgfpathmoveto{\pgfqpoint{4.692572in}{4.514706in}}%
\pgfpathlineto{\pgfqpoint{4.635079in}{4.481417in}}%
\pgfpathlineto{\pgfqpoint{4.555801in}{4.503746in}}%
\pgfpathclose%
\pgfusepath{fill}%
\end{pgfscope}%
\begin{pgfscope}%
\pgfpathrectangle{\pgfqpoint{0.680860in}{0.078740in}}{\pgfqpoint{7.842520in}{7.842520in}}%
\pgfusepath{clip}%
\pgfsetbuttcap%
\pgfsetroundjoin%
\definecolor{currentfill}{rgb}{0.160665,0.478540,0.558115}%
\pgfsetfillcolor{currentfill}%
\pgfsetlinewidth{0.000000pt}%
\definecolor{currentstroke}{rgb}{0.974417,0.903590,0.130215}%
\pgfsetstrokecolor{currentstroke}%
\pgfsetdash{}{0pt}%
\pgfpathmoveto{\pgfqpoint{6.106265in}{4.349010in}}%
\pgfpathlineto{\pgfqpoint{6.317507in}{4.233872in}}%
\pgfpathlineto{\pgfqpoint{6.177614in}{4.231807in}}%
\pgfpathclose%
\pgfusepath{fill}%
\end{pgfscope}%
\begin{pgfscope}%
\pgfpathrectangle{\pgfqpoint{0.680860in}{0.078740in}}{\pgfqpoint{7.842520in}{7.842520in}}%
\pgfusepath{clip}%
\pgfsetbuttcap%
\pgfsetroundjoin%
\definecolor{currentfill}{rgb}{0.154815,0.493313,0.557840}%
\pgfsetfillcolor{currentfill}%
\pgfsetlinewidth{0.000000pt}%
\definecolor{currentstroke}{rgb}{0.983868,0.904867,0.136897}%
\pgfsetstrokecolor{currentstroke}%
\pgfsetdash{}{0pt}%
\pgfpathmoveto{\pgfqpoint{3.285212in}{4.360324in}}%
\pgfpathlineto{\pgfqpoint{3.202053in}{4.299450in}}%
\pgfpathlineto{\pgfqpoint{3.418868in}{4.366343in}}%
\pgfpathclose%
\pgfusepath{fill}%
\end{pgfscope}%
\begin{pgfscope}%
\pgfpathrectangle{\pgfqpoint{0.680860in}{0.078740in}}{\pgfqpoint{7.842520in}{7.842520in}}%
\pgfusepath{clip}%
\pgfsetbuttcap%
\pgfsetroundjoin%
\definecolor{currentfill}{rgb}{0.146180,0.515413,0.556823}%
\pgfsetfillcolor{currentfill}%
\pgfsetlinewidth{0.000000pt}%
\definecolor{currentstroke}{rgb}{0.993248,0.906157,0.143936}%
\pgfsetstrokecolor{currentstroke}%
\pgfsetdash{}{0pt}%
\pgfpathmoveto{\pgfqpoint{3.852122in}{4.458012in}}%
\pgfpathlineto{\pgfqpoint{3.635593in}{4.419577in}}%
\pgfpathlineto{\pgfqpoint{3.770186in}{4.427939in}}%
\pgfpathclose%
\pgfusepath{fill}%
\end{pgfscope}%
\begin{pgfscope}%
\pgfpathrectangle{\pgfqpoint{0.680860in}{0.078740in}}{\pgfqpoint{7.842520in}{7.842520in}}%
\pgfusepath{clip}%
\pgfsetbuttcap%
\pgfsetroundjoin%
\definecolor{currentfill}{rgb}{0.149039,0.508051,0.557250}%
\pgfsetfillcolor{currentfill}%
\pgfsetlinewidth{0.000000pt}%
\definecolor{currentstroke}{rgb}{0.267004,0.004874,0.329415}%
\pgfsetstrokecolor{currentstroke}%
\pgfsetdash{}{0pt}%
\pgfpathmoveto{\pgfqpoint{5.614131in}{4.420886in}}%
\pgfpathlineto{\pgfqpoint{5.753388in}{4.429993in}}%
\pgfpathlineto{\pgfqpoint{5.688350in}{4.333008in}}%
\pgfpathclose%
\pgfusepath{fill}%
\end{pgfscope}%
\begin{pgfscope}%
\pgfpathrectangle{\pgfqpoint{0.680860in}{0.078740in}}{\pgfqpoint{7.842520in}{7.842520in}}%
\pgfusepath{clip}%
\pgfsetbuttcap%
\pgfsetroundjoin%
\definecolor{currentfill}{rgb}{0.143343,0.522773,0.556295}%
\pgfsetfillcolor{currentfill}%
\pgfsetlinewidth{0.000000pt}%
\definecolor{currentstroke}{rgb}{0.268510,0.009605,0.335427}%
\pgfsetstrokecolor{currentstroke}%
\pgfsetdash{}{0pt}%
\pgfpathmoveto{\pgfqpoint{5.337540in}{4.407155in}}%
\pgfpathlineto{\pgfqpoint{5.261392in}{4.472255in}}%
\pgfpathlineto{\pgfqpoint{5.399877in}{4.482791in}}%
\pgfpathclose%
\pgfusepath{fill}%
\end{pgfscope}%
\begin{pgfscope}%
\pgfpathrectangle{\pgfqpoint{0.680860in}{0.078740in}}{\pgfqpoint{7.842520in}{7.842520in}}%
\pgfusepath{clip}%
\pgfsetbuttcap%
\pgfsetroundjoin%
\definecolor{currentfill}{rgb}{0.169646,0.456262,0.558030}%
\pgfsetfillcolor{currentfill}%
\pgfsetlinewidth{0.000000pt}%
\definecolor{currentstroke}{rgb}{0.269944,0.014625,0.341379}%
\pgfsetstrokecolor{currentstroke}%
\pgfsetdash{}{0pt}%
\pgfpathmoveto{\pgfqpoint{2.852326in}{4.217128in}}%
\pgfpathlineto{\pgfqpoint{2.635965in}{4.128731in}}%
\pgfpathlineto{\pgfqpoint{2.768504in}{4.129646in}}%
\pgfpathclose%
\pgfusepath{fill}%
\end{pgfscope}%
\begin{pgfscope}%
\pgfpathrectangle{\pgfqpoint{0.680860in}{0.078740in}}{\pgfqpoint{7.842520in}{7.842520in}}%
\pgfusepath{clip}%
\pgfsetbuttcap%
\pgfsetroundjoin%
\definecolor{currentfill}{rgb}{0.159194,0.482237,0.558073}%
\pgfsetfillcolor{currentfill}%
\pgfsetlinewidth{0.000000pt}%
\definecolor{currentstroke}{rgb}{0.271305,0.019942,0.347269}%
\pgfsetstrokecolor{currentstroke}%
\pgfsetdash{}{0pt}%
\pgfpathmoveto{\pgfqpoint{2.985240in}{4.220121in}}%
\pgfpathlineto{\pgfqpoint{3.202053in}{4.299450in}}%
\pgfpathlineto{\pgfqpoint{3.068765in}{4.294740in}}%
\pgfpathclose%
\pgfusepath{fill}%
\end{pgfscope}%
\begin{pgfscope}%
\pgfpathrectangle{\pgfqpoint{0.680860in}{0.078740in}}{\pgfqpoint{7.842520in}{7.842520in}}%
\pgfusepath{clip}%
\pgfsetbuttcap%
\pgfsetroundjoin%
\definecolor{currentfill}{rgb}{0.179019,0.433756,0.557430}%
\pgfsetfillcolor{currentfill}%
\pgfsetlinewidth{0.000000pt}%
\definecolor{currentstroke}{rgb}{0.272594,0.025563,0.353093}%
\pgfsetstrokecolor{currentstroke}%
\pgfsetdash{}{0pt}%
\pgfpathmoveto{\pgfqpoint{2.635965in}{4.128731in}}%
\pgfpathlineto{\pgfqpoint{2.419739in}{4.030838in}}%
\pgfpathlineto{\pgfqpoint{2.551905in}{4.029357in}}%
\pgfpathclose%
\pgfusepath{fill}%
\end{pgfscope}%
\begin{pgfscope}%
\pgfpathrectangle{\pgfqpoint{0.680860in}{0.078740in}}{\pgfqpoint{7.842520in}{7.842520in}}%
\pgfusepath{clip}%
\pgfsetbuttcap%
\pgfsetroundjoin%
\definecolor{currentfill}{rgb}{0.139147,0.533812,0.555298}%
\pgfsetfillcolor{currentfill}%
\pgfsetlinewidth{0.000000pt}%
\definecolor{currentstroke}{rgb}{0.273809,0.031497,0.358853}%
\pgfsetstrokecolor{currentstroke}%
\pgfsetdash{}{0pt}%
\pgfpathmoveto{\pgfqpoint{4.692572in}{4.514706in}}%
\pgfpathlineto{\pgfqpoint{4.908499in}{4.498984in}}%
\pgfpathlineto{\pgfqpoint{4.771480in}{4.489440in}}%
\pgfpathclose%
\pgfusepath{fill}%
\end{pgfscope}%
\begin{pgfscope}%
\pgfpathrectangle{\pgfqpoint{0.680860in}{0.078740in}}{\pgfqpoint{7.842520in}{7.842520in}}%
\pgfusepath{clip}%
\pgfsetbuttcap%
\pgfsetroundjoin%
\definecolor{currentfill}{rgb}{0.137770,0.537492,0.554906}%
\pgfsetfillcolor{currentfill}%
\pgfsetlinewidth{0.000000pt}%
\definecolor{currentstroke}{rgb}{0.274952,0.037752,0.364543}%
\pgfsetstrokecolor{currentstroke}%
\pgfsetdash{}{0pt}%
\pgfpathmoveto{\pgfqpoint{4.475933in}{4.511090in}}%
\pgfpathlineto{\pgfqpoint{4.555801in}{4.503746in}}%
\pgfpathlineto{\pgfqpoint{4.419650in}{4.494358in}}%
\pgfpathclose%
\pgfusepath{fill}%
\end{pgfscope}%
\begin{pgfscope}%
\pgfpathrectangle{\pgfqpoint{0.680860in}{0.078740in}}{\pgfqpoint{7.842520in}{7.842520in}}%
\pgfusepath{clip}%
\pgfsetbuttcap%
\pgfsetroundjoin%
\definecolor{currentfill}{rgb}{0.139147,0.533812,0.555298}%
\pgfsetfillcolor{currentfill}%
\pgfsetlinewidth{0.000000pt}%
\definecolor{currentstroke}{rgb}{0.276022,0.044167,0.370164}%
\pgfsetstrokecolor{currentstroke}%
\pgfsetdash{}{0pt}%
\pgfpathmoveto{\pgfqpoint{4.908499in}{4.498984in}}%
\pgfpathlineto{\pgfqpoint{5.046149in}{4.510108in}}%
\pgfpathlineto{\pgfqpoint{5.123547in}{4.463279in}}%
\pgfpathclose%
\pgfusepath{fill}%
\end{pgfscope}%
\begin{pgfscope}%
\pgfpathrectangle{\pgfqpoint{0.680860in}{0.078740in}}{\pgfqpoint{7.842520in}{7.842520in}}%
\pgfusepath{clip}%
\pgfsetbuttcap%
\pgfsetroundjoin%
\definecolor{currentfill}{rgb}{0.147607,0.511733,0.557049}%
\pgfsetfillcolor{currentfill}%
\pgfsetlinewidth{0.000000pt}%
\definecolor{currentstroke}{rgb}{0.277018,0.050344,0.375715}%
\pgfsetstrokecolor{currentstroke}%
\pgfsetdash{}{0pt}%
\pgfpathmoveto{\pgfqpoint{3.635593in}{4.419577in}}%
\pgfpathlineto{\pgfqpoint{3.418868in}{4.366343in}}%
\pgfpathlineto{\pgfqpoint{3.553097in}{4.373815in}}%
\pgfpathclose%
\pgfusepath{fill}%
\end{pgfscope}%
\begin{pgfscope}%
\pgfpathrectangle{\pgfqpoint{0.680860in}{0.078740in}}{\pgfqpoint{7.842520in}{7.842520in}}%
\pgfusepath{clip}%
\pgfsetbuttcap%
\pgfsetroundjoin%
\definecolor{currentfill}{rgb}{0.139147,0.533812,0.555298}%
\pgfsetfillcolor{currentfill}%
\pgfsetlinewidth{0.000000pt}%
\definecolor{currentstroke}{rgb}{0.277941,0.056324,0.381191}%
\pgfsetstrokecolor{currentstroke}%
\pgfsetdash{}{0pt}%
\pgfpathmoveto{\pgfqpoint{5.123547in}{4.463279in}}%
\pgfpathlineto{\pgfqpoint{5.046149in}{4.510108in}}%
\pgfpathlineto{\pgfqpoint{5.261392in}{4.472255in}}%
\pgfpathclose%
\pgfusepath{fill}%
\end{pgfscope}%
\begin{pgfscope}%
\pgfpathrectangle{\pgfqpoint{0.680860in}{0.078740in}}{\pgfqpoint{7.842520in}{7.842520in}}%
\pgfusepath{clip}%
\pgfsetbuttcap%
\pgfsetroundjoin%
\definecolor{currentfill}{rgb}{0.166617,0.463708,0.558119}%
\pgfsetfillcolor{currentfill}%
\pgfsetlinewidth{0.000000pt}%
\definecolor{currentstroke}{rgb}{0.278791,0.062145,0.386592}%
\pgfsetstrokecolor{currentstroke}%
\pgfsetdash{}{0pt}%
\pgfpathmoveto{\pgfqpoint{2.985240in}{4.220121in}}%
\pgfpathlineto{\pgfqpoint{2.852326in}{4.217128in}}%
\pgfpathlineto{\pgfqpoint{2.768504in}{4.129646in}}%
\pgfpathclose%
\pgfusepath{fill}%
\end{pgfscope}%
\begin{pgfscope}%
\pgfpathrectangle{\pgfqpoint{0.680860in}{0.078740in}}{\pgfqpoint{7.842520in}{7.842520in}}%
\pgfusepath{clip}%
\pgfsetbuttcap%
\pgfsetroundjoin%
\definecolor{currentfill}{rgb}{0.139147,0.533812,0.555298}%
\pgfsetfillcolor{currentfill}%
\pgfsetlinewidth{0.000000pt}%
\definecolor{currentstroke}{rgb}{0.279566,0.067836,0.391917}%
\pgfsetstrokecolor{currentstroke}%
\pgfsetdash{}{0pt}%
\pgfpathmoveto{\pgfqpoint{4.122598in}{4.477059in}}%
\pgfpathlineto{\pgfqpoint{4.203594in}{4.489212in}}%
\pgfpathlineto{\pgfqpoint{3.987060in}{4.466760in}}%
\pgfpathclose%
\pgfusepath{fill}%
\end{pgfscope}%
\begin{pgfscope}%
\pgfpathrectangle{\pgfqpoint{0.680860in}{0.078740in}}{\pgfqpoint{7.842520in}{7.842520in}}%
\pgfusepath{clip}%
\pgfsetbuttcap%
\pgfsetroundjoin%
\definecolor{currentfill}{rgb}{0.141935,0.526453,0.555991}%
\pgfsetfillcolor{currentfill}%
\pgfsetlinewidth{0.000000pt}%
\definecolor{currentstroke}{rgb}{0.280267,0.073417,0.397163}%
\pgfsetstrokecolor{currentstroke}%
\pgfsetdash{}{0pt}%
\pgfpathmoveto{\pgfqpoint{3.905372in}{4.437840in}}%
\pgfpathlineto{\pgfqpoint{3.987060in}{4.466760in}}%
\pgfpathlineto{\pgfqpoint{3.852122in}{4.458012in}}%
\pgfpathclose%
\pgfusepath{fill}%
\end{pgfscope}%
\begin{pgfscope}%
\pgfpathrectangle{\pgfqpoint{0.680860in}{0.078740in}}{\pgfqpoint{7.842520in}{7.842520in}}%
\pgfusepath{clip}%
\pgfsetbuttcap%
\pgfsetroundjoin%
\definecolor{currentfill}{rgb}{0.187231,0.414746,0.556547}%
\pgfsetfillcolor{currentfill}%
\pgfsetlinewidth{0.000000pt}%
\definecolor{currentstroke}{rgb}{0.280894,0.078907,0.402329}%
\pgfsetstrokecolor{currentstroke}%
\pgfsetdash{}{0pt}%
\pgfpathmoveto{\pgfqpoint{7.298112in}{3.906438in}}%
\pgfpathlineto{\pgfqpoint{7.156184in}{3.912059in}}%
\pgfpathlineto{\pgfqpoint{7.092121in}{4.093890in}}%
\pgfpathclose%
\pgfusepath{fill}%
\end{pgfscope}%
\begin{pgfscope}%
\pgfpathrectangle{\pgfqpoint{0.680860in}{0.078740in}}{\pgfqpoint{7.842520in}{7.842520in}}%
\pgfusepath{clip}%
\pgfsetbuttcap%
\pgfsetroundjoin%
\definecolor{currentfill}{rgb}{0.165117,0.467423,0.558141}%
\pgfsetfillcolor{currentfill}%
\pgfsetlinewidth{0.000000pt}%
\definecolor{currentstroke}{rgb}{0.281446,0.084320,0.407414}%
\pgfsetstrokecolor{currentstroke}%
\pgfsetdash{}{0pt}%
\pgfpathmoveto{\pgfqpoint{6.667203in}{4.093383in}}%
\pgfpathlineto{\pgfqpoint{6.458040in}{4.237255in}}%
\pgfpathlineto{\pgfqpoint{6.599226in}{4.242008in}}%
\pgfpathclose%
\pgfusepath{fill}%
\end{pgfscope}%
\begin{pgfscope}%
\pgfpathrectangle{\pgfqpoint{0.680860in}{0.078740in}}{\pgfqpoint{7.842520in}{7.842520in}}%
\pgfusepath{clip}%
\pgfsetbuttcap%
\pgfsetroundjoin%
\definecolor{currentfill}{rgb}{0.137770,0.537492,0.554906}%
\pgfsetfillcolor{currentfill}%
\pgfsetlinewidth{0.000000pt}%
\definecolor{currentstroke}{rgb}{0.281924,0.089666,0.412415}%
\pgfsetstrokecolor{currentstroke}%
\pgfsetdash{}{0pt}%
\pgfpathmoveto{\pgfqpoint{4.419650in}{4.494358in}}%
\pgfpathlineto{\pgfqpoint{4.339456in}{4.499350in}}%
\pgfpathlineto{\pgfqpoint{4.475933in}{4.511090in}}%
\pgfpathclose%
\pgfusepath{fill}%
\end{pgfscope}%
\begin{pgfscope}%
\pgfpathrectangle{\pgfqpoint{0.680860in}{0.078740in}}{\pgfqpoint{7.842520in}{7.842520in}}%
\pgfusepath{clip}%
\pgfsetbuttcap%
\pgfsetroundjoin%
\definecolor{currentfill}{rgb}{0.149039,0.508051,0.557250}%
\pgfsetfillcolor{currentfill}%
\pgfsetlinewidth{0.000000pt}%
\definecolor{currentstroke}{rgb}{0.282327,0.094955,0.417331}%
\pgfsetstrokecolor{currentstroke}%
\pgfsetdash{}{0pt}%
\pgfpathmoveto{\pgfqpoint{5.893304in}{4.440669in}}%
\pgfpathlineto{\pgfqpoint{5.966319in}{4.342261in}}%
\pgfpathlineto{\pgfqpoint{5.827018in}{4.336945in}}%
\pgfpathclose%
\pgfusepath{fill}%
\end{pgfscope}%
\begin{pgfscope}%
\pgfpathrectangle{\pgfqpoint{0.680860in}{0.078740in}}{\pgfqpoint{7.842520in}{7.842520in}}%
\pgfusepath{clip}%
\pgfsetbuttcap%
\pgfsetroundjoin%
\definecolor{currentfill}{rgb}{0.141935,0.526453,0.555991}%
\pgfsetfillcolor{currentfill}%
\pgfsetlinewidth{0.000000pt}%
\definecolor{currentstroke}{rgb}{0.282656,0.100196,0.422160}%
\pgfsetstrokecolor{currentstroke}%
\pgfsetdash{}{0pt}%
\pgfpathmoveto{\pgfqpoint{3.770186in}{4.427939in}}%
\pgfpathlineto{\pgfqpoint{3.905372in}{4.437840in}}%
\pgfpathlineto{\pgfqpoint{3.852122in}{4.458012in}}%
\pgfpathclose%
\pgfusepath{fill}%
\end{pgfscope}%
\begin{pgfscope}%
\pgfpathrectangle{\pgfqpoint{0.680860in}{0.078740in}}{\pgfqpoint{7.842520in}{7.842520in}}%
\pgfusepath{clip}%
\pgfsetbuttcap%
\pgfsetroundjoin%
\definecolor{currentfill}{rgb}{0.169646,0.456262,0.558030}%
\pgfsetfillcolor{currentfill}%
\pgfsetlinewidth{0.000000pt}%
\definecolor{currentstroke}{rgb}{0.282910,0.105393,0.426902}%
\pgfsetstrokecolor{currentstroke}%
\pgfsetdash{}{0pt}%
\pgfpathmoveto{\pgfqpoint{6.741080in}{4.248182in}}%
\pgfpathlineto{\pgfqpoint{6.808197in}{4.092323in}}%
\pgfpathlineto{\pgfqpoint{6.667203in}{4.093383in}}%
\pgfpathclose%
\pgfusepath{fill}%
\end{pgfscope}%
\begin{pgfscope}%
\pgfpathrectangle{\pgfqpoint{0.680860in}{0.078740in}}{\pgfqpoint{7.842520in}{7.842520in}}%
\pgfusepath{clip}%
\pgfsetbuttcap%
\pgfsetroundjoin%
\definecolor{currentfill}{rgb}{0.151918,0.500685,0.557587}%
\pgfsetfillcolor{currentfill}%
\pgfsetlinewidth{0.000000pt}%
\definecolor{currentstroke}{rgb}{0.283091,0.110553,0.431554}%
\pgfsetstrokecolor{currentstroke}%
\pgfsetdash{}{0pt}%
\pgfpathmoveto{\pgfqpoint{3.418868in}{4.366343in}}%
\pgfpathlineto{\pgfqpoint{3.202053in}{4.299450in}}%
\pgfpathlineto{\pgfqpoint{3.335906in}{4.305569in}}%
\pgfpathclose%
\pgfusepath{fill}%
\end{pgfscope}%
\begin{pgfscope}%
\pgfpathrectangle{\pgfqpoint{0.680860in}{0.078740in}}{\pgfqpoint{7.842520in}{7.842520in}}%
\pgfusepath{clip}%
\pgfsetbuttcap%
\pgfsetroundjoin%
\definecolor{currentfill}{rgb}{0.210503,0.363727,0.552206}%
\pgfsetfillcolor{currentfill}%
\pgfsetlinewidth{0.000000pt}%
\definecolor{currentstroke}{rgb}{0.283197,0.115680,0.436115}%
\pgfsetstrokecolor{currentstroke}%
\pgfsetdash{}{0pt}%
\pgfpathmoveto{\pgfqpoint{7.643458in}{3.682018in}}%
\pgfpathlineto{\pgfqpoint{7.727735in}{3.896091in}}%
\pgfpathlineto{\pgfqpoint{7.786100in}{3.670222in}}%
\pgfpathclose%
\pgfusepath{fill}%
\end{pgfscope}%
\begin{pgfscope}%
\pgfpathrectangle{\pgfqpoint{0.680860in}{0.078740in}}{\pgfqpoint{7.842520in}{7.842520in}}%
\pgfusepath{clip}%
\pgfsetbuttcap%
\pgfsetroundjoin%
\definecolor{currentfill}{rgb}{0.141935,0.526453,0.555991}%
\pgfsetfillcolor{currentfill}%
\pgfsetlinewidth{0.000000pt}%
\definecolor{currentstroke}{rgb}{0.283229,0.120777,0.440584}%
\pgfsetstrokecolor{currentstroke}%
\pgfsetdash{}{0pt}%
\pgfpathmoveto{\pgfqpoint{5.539015in}{4.494947in}}%
\pgfpathlineto{\pgfqpoint{5.614131in}{4.420886in}}%
\pgfpathlineto{\pgfqpoint{5.475520in}{4.413292in}}%
\pgfpathclose%
\pgfusepath{fill}%
\end{pgfscope}%
\begin{pgfscope}%
\pgfpathrectangle{\pgfqpoint{0.680860in}{0.078740in}}{\pgfqpoint{7.842520in}{7.842520in}}%
\pgfusepath{clip}%
\pgfsetbuttcap%
\pgfsetroundjoin%
\definecolor{currentfill}{rgb}{0.201239,0.383670,0.554294}%
\pgfsetfillcolor{currentfill}%
\pgfsetlinewidth{0.000000pt}%
\definecolor{currentstroke}{rgb}{0.283187,0.125848,0.444960}%
\pgfsetstrokecolor{currentstroke}%
\pgfsetdash{}{0pt}%
\pgfpathmoveto{\pgfqpoint{7.440672in}{3.901876in}}%
\pgfpathlineto{\pgfqpoint{7.583875in}{3.898413in}}%
\pgfpathlineto{\pgfqpoint{7.643458in}{3.682018in}}%
\pgfpathclose%
\pgfusepath{fill}%
\end{pgfscope}%
\begin{pgfscope}%
\pgfpathrectangle{\pgfqpoint{0.680860in}{0.078740in}}{\pgfqpoint{7.842520in}{7.842520in}}%
\pgfusepath{clip}%
\pgfsetbuttcap%
\pgfsetroundjoin%
\definecolor{currentfill}{rgb}{0.179019,0.433756,0.557430}%
\pgfsetfillcolor{currentfill}%
\pgfsetlinewidth{0.000000pt}%
\definecolor{currentstroke}{rgb}{0.283072,0.130895,0.449241}%
\pgfsetstrokecolor{currentstroke}%
\pgfsetdash{}{0pt}%
\pgfpathmoveto{\pgfqpoint{6.949832in}{4.092476in}}%
\pgfpathlineto{\pgfqpoint{7.092121in}{4.093890in}}%
\pgfpathlineto{\pgfqpoint{7.156184in}{3.912059in}}%
\pgfpathclose%
\pgfusepath{fill}%
\end{pgfscope}%
\begin{pgfscope}%
\pgfpathrectangle{\pgfqpoint{0.680860in}{0.078740in}}{\pgfqpoint{7.842520in}{7.842520in}}%
\pgfusepath{clip}%
\pgfsetbuttcap%
\pgfsetroundjoin%
\definecolor{currentfill}{rgb}{0.172719,0.448791,0.557885}%
\pgfsetfillcolor{currentfill}%
\pgfsetlinewidth{0.000000pt}%
\definecolor{currentstroke}{rgb}{0.282884,0.135920,0.453427}%
\pgfsetstrokecolor{currentstroke}%
\pgfsetdash{}{0pt}%
\pgfpathmoveto{\pgfqpoint{2.768504in}{4.129646in}}%
\pgfpathlineto{\pgfqpoint{2.635965in}{4.128731in}}%
\pgfpathlineto{\pgfqpoint{2.684604in}{4.029075in}}%
\pgfpathclose%
\pgfusepath{fill}%
\end{pgfscope}%
\begin{pgfscope}%
\pgfpathrectangle{\pgfqpoint{0.680860in}{0.078740in}}{\pgfqpoint{7.842520in}{7.842520in}}%
\pgfusepath{clip}%
\pgfsetbuttcap%
\pgfsetroundjoin%
\definecolor{currentfill}{rgb}{0.175841,0.441290,0.557685}%
\pgfsetfillcolor{currentfill}%
\pgfsetlinewidth{0.000000pt}%
\definecolor{currentstroke}{rgb}{0.282623,0.140926,0.457517}%
\pgfsetstrokecolor{currentstroke}%
\pgfsetdash{}{0pt}%
\pgfpathmoveto{\pgfqpoint{2.684604in}{4.029075in}}%
\pgfpathlineto{\pgfqpoint{2.635965in}{4.128731in}}%
\pgfpathlineto{\pgfqpoint{2.551905in}{4.029357in}}%
\pgfpathclose%
\pgfusepath{fill}%
\end{pgfscope}%
\begin{pgfscope}%
\pgfpathrectangle{\pgfqpoint{0.680860in}{0.078740in}}{\pgfqpoint{7.842520in}{7.842520in}}%
\pgfusepath{clip}%
\pgfsetbuttcap%
\pgfsetroundjoin%
\definecolor{currentfill}{rgb}{0.143343,0.522773,0.556295}%
\pgfsetfillcolor{currentfill}%
\pgfsetlinewidth{0.000000pt}%
\definecolor{currentstroke}{rgb}{0.282290,0.145912,0.461510}%
\pgfsetstrokecolor{currentstroke}%
\pgfsetdash{}{0pt}%
\pgfpathmoveto{\pgfqpoint{3.635593in}{4.419577in}}%
\pgfpathlineto{\pgfqpoint{3.687912in}{4.382797in}}%
\pgfpathlineto{\pgfqpoint{3.770186in}{4.427939in}}%
\pgfpathclose%
\pgfusepath{fill}%
\end{pgfscope}%
\begin{pgfscope}%
\pgfpathrectangle{\pgfqpoint{0.680860in}{0.078740in}}{\pgfqpoint{7.842520in}{7.842520in}}%
\pgfusepath{clip}%
\pgfsetbuttcap%
\pgfsetroundjoin%
\definecolor{currentfill}{rgb}{0.157729,0.485932,0.558013}%
\pgfsetfillcolor{currentfill}%
\pgfsetlinewidth{0.000000pt}%
\definecolor{currentstroke}{rgb}{0.281887,0.150881,0.465405}%
\pgfsetstrokecolor{currentstroke}%
\pgfsetdash{}{0pt}%
\pgfpathmoveto{\pgfqpoint{2.985240in}{4.220121in}}%
\pgfpathlineto{\pgfqpoint{3.118708in}{4.224466in}}%
\pgfpathlineto{\pgfqpoint{3.202053in}{4.299450in}}%
\pgfpathclose%
\pgfusepath{fill}%
\end{pgfscope}%
\begin{pgfscope}%
\pgfpathrectangle{\pgfqpoint{0.680860in}{0.078740in}}{\pgfqpoint{7.842520in}{7.842520in}}%
\pgfusepath{clip}%
\pgfsetbuttcap%
\pgfsetroundjoin%
\definecolor{currentfill}{rgb}{0.182256,0.426184,0.557120}%
\pgfsetfillcolor{currentfill}%
\pgfsetlinewidth{0.000000pt}%
\definecolor{currentstroke}{rgb}{0.281412,0.155834,0.469201}%
\pgfsetstrokecolor{currentstroke}%
\pgfsetdash{}{0pt}%
\pgfpathmoveto{\pgfqpoint{2.551905in}{4.029357in}}%
\pgfpathlineto{\pgfqpoint{2.419739in}{4.030838in}}%
\pgfpathlineto{\pgfqpoint{2.467809in}{3.917569in}}%
\pgfpathclose%
\pgfusepath{fill}%
\end{pgfscope}%
\begin{pgfscope}%
\pgfpathrectangle{\pgfqpoint{0.680860in}{0.078740in}}{\pgfqpoint{7.842520in}{7.842520in}}%
\pgfusepath{clip}%
\pgfsetbuttcap%
\pgfsetroundjoin%
\definecolor{currentfill}{rgb}{0.185556,0.418570,0.556753}%
\pgfsetfillcolor{currentfill}%
\pgfsetlinewidth{0.000000pt}%
\definecolor{currentstroke}{rgb}{0.280868,0.160771,0.472899}%
\pgfsetstrokecolor{currentstroke}%
\pgfsetdash{}{0pt}%
\pgfpathmoveto{\pgfqpoint{2.467809in}{3.917569in}}%
\pgfpathlineto{\pgfqpoint{2.419739in}{4.030838in}}%
\pgfpathlineto{\pgfqpoint{2.335487in}{3.920610in}}%
\pgfpathclose%
\pgfusepath{fill}%
\end{pgfscope}%
\begin{pgfscope}%
\pgfpathrectangle{\pgfqpoint{0.680860in}{0.078740in}}{\pgfqpoint{7.842520in}{7.842520in}}%
\pgfusepath{clip}%
\pgfsetbuttcap%
\pgfsetroundjoin%
\definecolor{currentfill}{rgb}{0.136408,0.541173,0.554483}%
\pgfsetfillcolor{currentfill}%
\pgfsetlinewidth{0.000000pt}%
\definecolor{currentstroke}{rgb}{0.280255,0.165693,0.476498}%
\pgfsetstrokecolor{currentstroke}%
\pgfsetdash{}{0pt}%
\pgfpathmoveto{\pgfqpoint{4.339456in}{4.499350in}}%
\pgfpathlineto{\pgfqpoint{4.203594in}{4.489212in}}%
\pgfpathlineto{\pgfqpoint{4.258747in}{4.488968in}}%
\pgfpathclose%
\pgfusepath{fill}%
\end{pgfscope}%
\begin{pgfscope}%
\pgfpathrectangle{\pgfqpoint{0.680860in}{0.078740in}}{\pgfqpoint{7.842520in}{7.842520in}}%
\pgfusepath{clip}%
\pgfsetbuttcap%
\pgfsetroundjoin%
\definecolor{currentfill}{rgb}{0.143343,0.522773,0.556295}%
\pgfsetfillcolor{currentfill}%
\pgfsetlinewidth{0.000000pt}%
\definecolor{currentstroke}{rgb}{0.279574,0.170599,0.479997}%
\pgfsetstrokecolor{currentstroke}%
\pgfsetdash{}{0pt}%
\pgfpathmoveto{\pgfqpoint{5.827018in}{4.336945in}}%
\pgfpathlineto{\pgfqpoint{5.753388in}{4.429993in}}%
\pgfpathlineto{\pgfqpoint{5.893304in}{4.440669in}}%
\pgfpathclose%
\pgfusepath{fill}%
\end{pgfscope}%
\begin{pgfscope}%
\pgfpathrectangle{\pgfqpoint{0.680860in}{0.078740in}}{\pgfqpoint{7.842520in}{7.842520in}}%
\pgfusepath{clip}%
\pgfsetbuttcap%
\pgfsetroundjoin%
\definecolor{currentfill}{rgb}{0.135066,0.544853,0.554029}%
\pgfsetfillcolor{currentfill}%
\pgfsetlinewidth{0.000000pt}%
\definecolor{currentstroke}{rgb}{0.278826,0.175490,0.483397}%
\pgfsetstrokecolor{currentstroke}%
\pgfsetdash{}{0pt}%
\pgfpathmoveto{\pgfqpoint{4.475933in}{4.511090in}}%
\pgfpathlineto{\pgfqpoint{4.692572in}{4.514706in}}%
\pgfpathlineto{\pgfqpoint{4.555801in}{4.503746in}}%
\pgfpathclose%
\pgfusepath{fill}%
\end{pgfscope}%
\begin{pgfscope}%
\pgfpathrectangle{\pgfqpoint{0.680860in}{0.078740in}}{\pgfqpoint{7.842520in}{7.842520in}}%
\pgfusepath{clip}%
\pgfsetbuttcap%
\pgfsetroundjoin%
\definecolor{currentfill}{rgb}{0.151918,0.500685,0.557587}%
\pgfsetfillcolor{currentfill}%
\pgfsetlinewidth{0.000000pt}%
\definecolor{currentstroke}{rgb}{0.278012,0.180367,0.486697}%
\pgfsetstrokecolor{currentstroke}%
\pgfsetdash{}{0pt}%
\pgfpathmoveto{\pgfqpoint{6.246871in}{4.357246in}}%
\pgfpathlineto{\pgfqpoint{6.317507in}{4.233872in}}%
\pgfpathlineto{\pgfqpoint{6.106265in}{4.349010in}}%
\pgfpathclose%
\pgfusepath{fill}%
\end{pgfscope}%
\begin{pgfscope}%
\pgfpathrectangle{\pgfqpoint{0.680860in}{0.078740in}}{\pgfqpoint{7.842520in}{7.842520in}}%
\pgfusepath{clip}%
\pgfsetbuttcap%
\pgfsetroundjoin%
\definecolor{currentfill}{rgb}{0.144759,0.519093,0.556572}%
\pgfsetfillcolor{currentfill}%
\pgfsetlinewidth{0.000000pt}%
\definecolor{currentstroke}{rgb}{0.277134,0.185228,0.489898}%
\pgfsetstrokecolor{currentstroke}%
\pgfsetdash{}{0pt}%
\pgfpathmoveto{\pgfqpoint{3.553097in}{4.373815in}}%
\pgfpathlineto{\pgfqpoint{3.687912in}{4.382797in}}%
\pgfpathlineto{\pgfqpoint{3.635593in}{4.419577in}}%
\pgfpathclose%
\pgfusepath{fill}%
\end{pgfscope}%
\begin{pgfscope}%
\pgfpathrectangle{\pgfqpoint{0.680860in}{0.078740in}}{\pgfqpoint{7.842520in}{7.842520in}}%
\pgfusepath{clip}%
\pgfsetbuttcap%
\pgfsetroundjoin%
\definecolor{currentfill}{rgb}{0.137770,0.537492,0.554906}%
\pgfsetfillcolor{currentfill}%
\pgfsetlinewidth{0.000000pt}%
\definecolor{currentstroke}{rgb}{0.276194,0.190074,0.493001}%
\pgfsetstrokecolor{currentstroke}%
\pgfsetdash{}{0pt}%
\pgfpathmoveto{\pgfqpoint{5.475520in}{4.413292in}}%
\pgfpathlineto{\pgfqpoint{5.399877in}{4.482791in}}%
\pgfpathlineto{\pgfqpoint{5.539015in}{4.494947in}}%
\pgfpathclose%
\pgfusepath{fill}%
\end{pgfscope}%
\begin{pgfscope}%
\pgfpathrectangle{\pgfqpoint{0.680860in}{0.078740in}}{\pgfqpoint{7.842520in}{7.842520in}}%
\pgfusepath{clip}%
\pgfsetbuttcap%
\pgfsetroundjoin%
\definecolor{currentfill}{rgb}{0.165117,0.467423,0.558141}%
\pgfsetfillcolor{currentfill}%
\pgfsetlinewidth{0.000000pt}%
\definecolor{currentstroke}{rgb}{0.275191,0.194905,0.496005}%
\pgfsetstrokecolor{currentstroke}%
\pgfsetdash{}{0pt}%
\pgfpathmoveto{\pgfqpoint{2.768504in}{4.129646in}}%
\pgfpathlineto{\pgfqpoint{2.901587in}{4.131842in}}%
\pgfpathlineto{\pgfqpoint{2.985240in}{4.220121in}}%
\pgfpathclose%
\pgfusepath{fill}%
\end{pgfscope}%
\begin{pgfscope}%
\pgfpathrectangle{\pgfqpoint{0.680860in}{0.078740in}}{\pgfqpoint{7.842520in}{7.842520in}}%
\pgfusepath{clip}%
\pgfsetbuttcap%
\pgfsetroundjoin%
\definecolor{currentfill}{rgb}{0.136408,0.541173,0.554483}%
\pgfsetfillcolor{currentfill}%
\pgfsetlinewidth{0.000000pt}%
\definecolor{currentstroke}{rgb}{0.274128,0.199721,0.498911}%
\pgfsetstrokecolor{currentstroke}%
\pgfsetdash{}{0pt}%
\pgfpathmoveto{\pgfqpoint{4.258747in}{4.488968in}}%
\pgfpathlineto{\pgfqpoint{4.203594in}{4.489212in}}%
\pgfpathlineto{\pgfqpoint{4.122598in}{4.477059in}}%
\pgfpathclose%
\pgfusepath{fill}%
\end{pgfscope}%
\begin{pgfscope}%
\pgfpathrectangle{\pgfqpoint{0.680860in}{0.078740in}}{\pgfqpoint{7.842520in}{7.842520in}}%
\pgfusepath{clip}%
\pgfsetbuttcap%
\pgfsetroundjoin%
\definecolor{currentfill}{rgb}{0.153364,0.497000,0.557724}%
\pgfsetfillcolor{currentfill}%
\pgfsetlinewidth{0.000000pt}%
\definecolor{currentstroke}{rgb}{0.273006,0.204520,0.501721}%
\pgfsetstrokecolor{currentstroke}%
\pgfsetdash{}{0pt}%
\pgfpathmoveto{\pgfqpoint{3.335906in}{4.305569in}}%
\pgfpathlineto{\pgfqpoint{3.202053in}{4.299450in}}%
\pgfpathlineto{\pgfqpoint{3.118708in}{4.224466in}}%
\pgfpathclose%
\pgfusepath{fill}%
\end{pgfscope}%
\begin{pgfscope}%
\pgfpathrectangle{\pgfqpoint{0.680860in}{0.078740in}}{\pgfqpoint{7.842520in}{7.842520in}}%
\pgfusepath{clip}%
\pgfsetbuttcap%
\pgfsetroundjoin%
\definecolor{currentfill}{rgb}{0.162142,0.474838,0.558140}%
\pgfsetfillcolor{currentfill}%
\pgfsetlinewidth{0.000000pt}%
\definecolor{currentstroke}{rgb}{0.271828,0.209303,0.504434}%
\pgfsetstrokecolor{currentstroke}%
\pgfsetdash{}{0pt}%
\pgfpathmoveto{\pgfqpoint{6.667203in}{4.093383in}}%
\pgfpathlineto{\pgfqpoint{6.599226in}{4.242008in}}%
\pgfpathlineto{\pgfqpoint{6.741080in}{4.248182in}}%
\pgfpathclose%
\pgfusepath{fill}%
\end{pgfscope}%
\begin{pgfscope}%
\pgfpathrectangle{\pgfqpoint{0.680860in}{0.078740in}}{\pgfqpoint{7.842520in}{7.842520in}}%
\pgfusepath{clip}%
\pgfsetbuttcap%
\pgfsetroundjoin%
\definecolor{currentfill}{rgb}{0.133743,0.548535,0.553541}%
\pgfsetfillcolor{currentfill}%
\pgfsetlinewidth{0.000000pt}%
\definecolor{currentstroke}{rgb}{0.270595,0.214069,0.507052}%
\pgfsetstrokecolor{currentstroke}%
\pgfsetdash{}{0pt}%
\pgfpathmoveto{\pgfqpoint{4.829973in}{4.527300in}}%
\pgfpathlineto{\pgfqpoint{4.908499in}{4.498984in}}%
\pgfpathlineto{\pgfqpoint{4.692572in}{4.514706in}}%
\pgfpathclose%
\pgfusepath{fill}%
\end{pgfscope}%
\begin{pgfscope}%
\pgfpathrectangle{\pgfqpoint{0.680860in}{0.078740in}}{\pgfqpoint{7.842520in}{7.842520in}}%
\pgfusepath{clip}%
\pgfsetbuttcap%
\pgfsetroundjoin%
\definecolor{currentfill}{rgb}{0.153364,0.497000,0.557724}%
\pgfsetfillcolor{currentfill}%
\pgfsetlinewidth{0.000000pt}%
\definecolor{currentstroke}{rgb}{0.269308,0.218818,0.509577}%
\pgfsetstrokecolor{currentstroke}%
\pgfsetdash{}{0pt}%
\pgfpathmoveto{\pgfqpoint{6.388150in}{4.367027in}}%
\pgfpathlineto{\pgfqpoint{6.458040in}{4.237255in}}%
\pgfpathlineto{\pgfqpoint{6.317507in}{4.233872in}}%
\pgfpathclose%
\pgfusepath{fill}%
\end{pgfscope}%
\begin{pgfscope}%
\pgfpathrectangle{\pgfqpoint{0.680860in}{0.078740in}}{\pgfqpoint{7.842520in}{7.842520in}}%
\pgfusepath{clip}%
\pgfsetbuttcap%
\pgfsetroundjoin%
\definecolor{currentfill}{rgb}{0.144759,0.519093,0.556572}%
\pgfsetfillcolor{currentfill}%
\pgfsetlinewidth{0.000000pt}%
\definecolor{currentstroke}{rgb}{0.267968,0.223549,0.512008}%
\pgfsetstrokecolor{currentstroke}%
\pgfsetdash{}{0pt}%
\pgfpathmoveto{\pgfqpoint{5.966319in}{4.342261in}}%
\pgfpathlineto{\pgfqpoint{5.893304in}{4.440669in}}%
\pgfpathlineto{\pgfqpoint{6.106265in}{4.349010in}}%
\pgfpathclose%
\pgfusepath{fill}%
\end{pgfscope}%
\begin{pgfscope}%
\pgfpathrectangle{\pgfqpoint{0.680860in}{0.078740in}}{\pgfqpoint{7.842520in}{7.842520in}}%
\pgfusepath{clip}%
\pgfsetbuttcap%
\pgfsetroundjoin%
\definecolor{currentfill}{rgb}{0.147607,0.511733,0.557049}%
\pgfsetfillcolor{currentfill}%
\pgfsetlinewidth{0.000000pt}%
\definecolor{currentstroke}{rgb}{0.266580,0.228262,0.514349}%
\pgfsetstrokecolor{currentstroke}%
\pgfsetdash{}{0pt}%
\pgfpathmoveto{\pgfqpoint{3.470334in}{4.313152in}}%
\pgfpathlineto{\pgfqpoint{3.553097in}{4.373815in}}%
\pgfpathlineto{\pgfqpoint{3.418868in}{4.366343in}}%
\pgfpathclose%
\pgfusepath{fill}%
\end{pgfscope}%
\begin{pgfscope}%
\pgfpathrectangle{\pgfqpoint{0.680860in}{0.078740in}}{\pgfqpoint{7.842520in}{7.842520in}}%
\pgfusepath{clip}%
\pgfsetbuttcap%
\pgfsetroundjoin%
\definecolor{currentfill}{rgb}{0.149039,0.508051,0.557250}%
\pgfsetfillcolor{currentfill}%
\pgfsetlinewidth{0.000000pt}%
\definecolor{currentstroke}{rgb}{0.265145,0.232956,0.516599}%
\pgfsetstrokecolor{currentstroke}%
\pgfsetdash{}{0pt}%
\pgfpathmoveto{\pgfqpoint{3.335906in}{4.305569in}}%
\pgfpathlineto{\pgfqpoint{3.470334in}{4.313152in}}%
\pgfpathlineto{\pgfqpoint{3.418868in}{4.366343in}}%
\pgfpathclose%
\pgfusepath{fill}%
\end{pgfscope}%
\begin{pgfscope}%
\pgfpathrectangle{\pgfqpoint{0.680860in}{0.078740in}}{\pgfqpoint{7.842520in}{7.842520in}}%
\pgfusepath{clip}%
\pgfsetbuttcap%
\pgfsetroundjoin%
\definecolor{currentfill}{rgb}{0.133743,0.548535,0.553541}%
\pgfsetfillcolor{currentfill}%
\pgfsetlinewidth{0.000000pt}%
\definecolor{currentstroke}{rgb}{0.263663,0.237631,0.518762}%
\pgfsetstrokecolor{currentstroke}%
\pgfsetdash{}{0pt}%
\pgfpathmoveto{\pgfqpoint{5.261392in}{4.472255in}}%
\pgfpathlineto{\pgfqpoint{5.046149in}{4.510108in}}%
\pgfpathlineto{\pgfqpoint{5.184443in}{4.522872in}}%
\pgfpathclose%
\pgfusepath{fill}%
\end{pgfscope}%
\begin{pgfscope}%
\pgfpathrectangle{\pgfqpoint{0.680860in}{0.078740in}}{\pgfqpoint{7.842520in}{7.842520in}}%
\pgfusepath{clip}%
\pgfsetbuttcap%
\pgfsetroundjoin%
\definecolor{currentfill}{rgb}{0.197636,0.391528,0.554969}%
\pgfsetfillcolor{currentfill}%
\pgfsetlinewidth{0.000000pt}%
\definecolor{currentstroke}{rgb}{0.262138,0.242286,0.520837}%
\pgfsetstrokecolor{currentstroke}%
\pgfsetdash{}{0pt}%
\pgfpathmoveto{\pgfqpoint{7.643458in}{3.682018in}}%
\pgfpathlineto{\pgfqpoint{7.583875in}{3.898413in}}%
\pgfpathlineto{\pgfqpoint{7.727735in}{3.896091in}}%
\pgfpathclose%
\pgfusepath{fill}%
\end{pgfscope}%
\begin{pgfscope}%
\pgfpathrectangle{\pgfqpoint{0.680860in}{0.078740in}}{\pgfqpoint{7.842520in}{7.842520in}}%
\pgfusepath{clip}%
\pgfsetbuttcap%
\pgfsetroundjoin%
\definecolor{currentfill}{rgb}{0.137770,0.537492,0.554906}%
\pgfsetfillcolor{currentfill}%
\pgfsetlinewidth{0.000000pt}%
\definecolor{currentstroke}{rgb}{0.260571,0.246922,0.522828}%
\pgfsetstrokecolor{currentstroke}%
\pgfsetdash{}{0pt}%
\pgfpathmoveto{\pgfqpoint{5.539015in}{4.494947in}}%
\pgfpathlineto{\pgfqpoint{5.753388in}{4.429993in}}%
\pgfpathlineto{\pgfqpoint{5.614131in}{4.420886in}}%
\pgfpathclose%
\pgfusepath{fill}%
\end{pgfscope}%
\begin{pgfscope}%
\pgfpathrectangle{\pgfqpoint{0.680860in}{0.078740in}}{\pgfqpoint{7.842520in}{7.842520in}}%
\pgfusepath{clip}%
\pgfsetbuttcap%
\pgfsetroundjoin%
\definecolor{currentfill}{rgb}{0.136408,0.541173,0.554483}%
\pgfsetfillcolor{currentfill}%
\pgfsetlinewidth{0.000000pt}%
\definecolor{currentstroke}{rgb}{0.258965,0.251537,0.524736}%
\pgfsetstrokecolor{currentstroke}%
\pgfsetdash{}{0pt}%
\pgfpathmoveto{\pgfqpoint{3.987060in}{4.466760in}}%
\pgfpathlineto{\pgfqpoint{4.041163in}{4.449339in}}%
\pgfpathlineto{\pgfqpoint{4.122598in}{4.477059in}}%
\pgfpathclose%
\pgfusepath{fill}%
\end{pgfscope}%
\begin{pgfscope}%
\pgfpathrectangle{\pgfqpoint{0.680860in}{0.078740in}}{\pgfqpoint{7.842520in}{7.842520in}}%
\pgfusepath{clip}%
\pgfsetbuttcap%
\pgfsetroundjoin%
\definecolor{currentfill}{rgb}{0.169646,0.456262,0.558030}%
\pgfsetfillcolor{currentfill}%
\pgfsetlinewidth{0.000000pt}%
\definecolor{currentstroke}{rgb}{0.257322,0.256130,0.526563}%
\pgfsetstrokecolor{currentstroke}%
\pgfsetdash{}{0pt}%
\pgfpathmoveto{\pgfqpoint{2.768504in}{4.129646in}}%
\pgfpathlineto{\pgfqpoint{2.684604in}{4.029075in}}%
\pgfpathlineto{\pgfqpoint{2.901587in}{4.131842in}}%
\pgfpathclose%
\pgfusepath{fill}%
\end{pgfscope}%
\begin{pgfscope}%
\pgfpathrectangle{\pgfqpoint{0.680860in}{0.078740in}}{\pgfqpoint{7.842520in}{7.842520in}}%
\pgfusepath{clip}%
\pgfsetbuttcap%
\pgfsetroundjoin%
\definecolor{currentfill}{rgb}{0.139147,0.533812,0.555298}%
\pgfsetfillcolor{currentfill}%
\pgfsetlinewidth{0.000000pt}%
\definecolor{currentstroke}{rgb}{0.255645,0.260703,0.528312}%
\pgfsetstrokecolor{currentstroke}%
\pgfsetdash{}{0pt}%
\pgfpathmoveto{\pgfqpoint{3.770186in}{4.427939in}}%
\pgfpathlineto{\pgfqpoint{3.687912in}{4.382797in}}%
\pgfpathlineto{\pgfqpoint{3.905372in}{4.437840in}}%
\pgfpathclose%
\pgfusepath{fill}%
\end{pgfscope}%
\begin{pgfscope}%
\pgfpathrectangle{\pgfqpoint{0.680860in}{0.078740in}}{\pgfqpoint{7.842520in}{7.842520in}}%
\pgfusepath{clip}%
\pgfsetbuttcap%
\pgfsetroundjoin%
\definecolor{currentfill}{rgb}{0.131172,0.555899,0.552459}%
\pgfsetfillcolor{currentfill}%
\pgfsetlinewidth{0.000000pt}%
\definecolor{currentstroke}{rgb}{0.253935,0.265254,0.529983}%
\pgfsetstrokecolor{currentstroke}%
\pgfsetdash{}{0pt}%
\pgfpathmoveto{\pgfqpoint{4.908499in}{4.498984in}}%
\pgfpathlineto{\pgfqpoint{4.968019in}{4.541588in}}%
\pgfpathlineto{\pgfqpoint{5.046149in}{4.510108in}}%
\pgfpathclose%
\pgfusepath{fill}%
\end{pgfscope}%
\begin{pgfscope}%
\pgfpathrectangle{\pgfqpoint{0.680860in}{0.078740in}}{\pgfqpoint{7.842520in}{7.842520in}}%
\pgfusepath{clip}%
\pgfsetbuttcap%
\pgfsetroundjoin%
\definecolor{currentfill}{rgb}{0.163625,0.471133,0.558148}%
\pgfsetfillcolor{currentfill}%
\pgfsetlinewidth{0.000000pt}%
\definecolor{currentstroke}{rgb}{0.252194,0.269783,0.531579}%
\pgfsetstrokecolor{currentstroke}%
\pgfsetdash{}{0pt}%
\pgfpathmoveto{\pgfqpoint{6.949832in}{4.092476in}}%
\pgfpathlineto{\pgfqpoint{6.808197in}{4.092323in}}%
\pgfpathlineto{\pgfqpoint{6.883614in}{4.255831in}}%
\pgfpathclose%
\pgfusepath{fill}%
\end{pgfscope}%
\begin{pgfscope}%
\pgfpathrectangle{\pgfqpoint{0.680860in}{0.078740in}}{\pgfqpoint{7.842520in}{7.842520in}}%
\pgfusepath{clip}%
\pgfsetbuttcap%
\pgfsetroundjoin%
\definecolor{currentfill}{rgb}{0.136408,0.541173,0.554483}%
\pgfsetfillcolor{currentfill}%
\pgfsetlinewidth{0.000000pt}%
\definecolor{currentstroke}{rgb}{0.250425,0.274290,0.533103}%
\pgfsetstrokecolor{currentstroke}%
\pgfsetdash{}{0pt}%
\pgfpathmoveto{\pgfqpoint{3.987060in}{4.466760in}}%
\pgfpathlineto{\pgfqpoint{3.905372in}{4.437840in}}%
\pgfpathlineto{\pgfqpoint{4.041163in}{4.449339in}}%
\pgfpathclose%
\pgfusepath{fill}%
\end{pgfscope}%
\begin{pgfscope}%
\pgfpathrectangle{\pgfqpoint{0.680860in}{0.078740in}}{\pgfqpoint{7.842520in}{7.842520in}}%
\pgfusepath{clip}%
\pgfsetbuttcap%
\pgfsetroundjoin%
\definecolor{currentfill}{rgb}{0.147607,0.511733,0.557049}%
\pgfsetfillcolor{currentfill}%
\pgfsetlinewidth{0.000000pt}%
\definecolor{currentstroke}{rgb}{0.248629,0.278775,0.534556}%
\pgfsetstrokecolor{currentstroke}%
\pgfsetdash{}{0pt}%
\pgfpathmoveto{\pgfqpoint{6.317507in}{4.233872in}}%
\pgfpathlineto{\pgfqpoint{6.246871in}{4.357246in}}%
\pgfpathlineto{\pgfqpoint{6.388150in}{4.367027in}}%
\pgfpathclose%
\pgfusepath{fill}%
\end{pgfscope}%
\begin{pgfscope}%
\pgfpathrectangle{\pgfqpoint{0.680860in}{0.078740in}}{\pgfqpoint{7.842520in}{7.842520in}}%
\pgfusepath{clip}%
\pgfsetbuttcap%
\pgfsetroundjoin%
\definecolor{currentfill}{rgb}{0.159194,0.482237,0.558073}%
\pgfsetfillcolor{currentfill}%
\pgfsetlinewidth{0.000000pt}%
\definecolor{currentstroke}{rgb}{0.246811,0.283237,0.535941}%
\pgfsetstrokecolor{currentstroke}%
\pgfsetdash{}{0pt}%
\pgfpathmoveto{\pgfqpoint{3.035223in}{4.135367in}}%
\pgfpathlineto{\pgfqpoint{3.118708in}{4.224466in}}%
\pgfpathlineto{\pgfqpoint{2.985240in}{4.220121in}}%
\pgfpathclose%
\pgfusepath{fill}%
\end{pgfscope}%
\begin{pgfscope}%
\pgfpathrectangle{\pgfqpoint{0.680860in}{0.078740in}}{\pgfqpoint{7.842520in}{7.842520in}}%
\pgfusepath{clip}%
\pgfsetbuttcap%
\pgfsetroundjoin%
\definecolor{currentfill}{rgb}{0.174274,0.445044,0.557792}%
\pgfsetfillcolor{currentfill}%
\pgfsetlinewidth{0.000000pt}%
\definecolor{currentstroke}{rgb}{0.244972,0.287675,0.537260}%
\pgfsetstrokecolor{currentstroke}%
\pgfsetdash{}{0pt}%
\pgfpathmoveto{\pgfqpoint{7.092121in}{4.093890in}}%
\pgfpathlineto{\pgfqpoint{7.235076in}{4.096611in}}%
\pgfpathlineto{\pgfqpoint{7.298112in}{3.906438in}}%
\pgfpathclose%
\pgfusepath{fill}%
\end{pgfscope}%
\begin{pgfscope}%
\pgfpathrectangle{\pgfqpoint{0.680860in}{0.078740in}}{\pgfqpoint{7.842520in}{7.842520in}}%
\pgfusepath{clip}%
\pgfsetbuttcap%
\pgfsetroundjoin%
\definecolor{currentfill}{rgb}{0.162142,0.474838,0.558140}%
\pgfsetfillcolor{currentfill}%
\pgfsetlinewidth{0.000000pt}%
\definecolor{currentstroke}{rgb}{0.243113,0.292092,0.538516}%
\pgfsetstrokecolor{currentstroke}%
\pgfsetdash{}{0pt}%
\pgfpathmoveto{\pgfqpoint{2.901587in}{4.131842in}}%
\pgfpathlineto{\pgfqpoint{3.035223in}{4.135367in}}%
\pgfpathlineto{\pgfqpoint{2.985240in}{4.220121in}}%
\pgfpathclose%
\pgfusepath{fill}%
\end{pgfscope}%
\begin{pgfscope}%
\pgfpathrectangle{\pgfqpoint{0.680860in}{0.078740in}}{\pgfqpoint{7.842520in}{7.842520in}}%
\pgfusepath{clip}%
\pgfsetbuttcap%
\pgfsetroundjoin%
\definecolor{currentfill}{rgb}{0.132444,0.552216,0.553018}%
\pgfsetfillcolor{currentfill}%
\pgfsetlinewidth{0.000000pt}%
\definecolor{currentstroke}{rgb}{0.241237,0.296485,0.539709}%
\pgfsetstrokecolor{currentstroke}%
\pgfsetdash{}{0pt}%
\pgfpathmoveto{\pgfqpoint{5.323395in}{4.537338in}}%
\pgfpathlineto{\pgfqpoint{5.399877in}{4.482791in}}%
\pgfpathlineto{\pgfqpoint{5.261392in}{4.472255in}}%
\pgfpathclose%
\pgfusepath{fill}%
\end{pgfscope}%
\begin{pgfscope}%
\pgfpathrectangle{\pgfqpoint{0.680860in}{0.078740in}}{\pgfqpoint{7.842520in}{7.842520in}}%
\pgfusepath{clip}%
\pgfsetbuttcap%
\pgfsetroundjoin%
\definecolor{currentfill}{rgb}{0.180629,0.429975,0.557282}%
\pgfsetfillcolor{currentfill}%
\pgfsetlinewidth{0.000000pt}%
\definecolor{currentstroke}{rgb}{0.239346,0.300855,0.540844}%
\pgfsetstrokecolor{currentstroke}%
\pgfsetdash{}{0pt}%
\pgfpathmoveto{\pgfqpoint{7.298112in}{3.906438in}}%
\pgfpathlineto{\pgfqpoint{7.378712in}{4.100689in}}%
\pgfpathlineto{\pgfqpoint{7.440672in}{3.901876in}}%
\pgfpathclose%
\pgfusepath{fill}%
\end{pgfscope}%
\begin{pgfscope}%
\pgfpathrectangle{\pgfqpoint{0.680860in}{0.078740in}}{\pgfqpoint{7.842520in}{7.842520in}}%
\pgfusepath{clip}%
\pgfsetbuttcap%
\pgfsetroundjoin%
\definecolor{currentfill}{rgb}{0.177423,0.437527,0.557565}%
\pgfsetfillcolor{currentfill}%
\pgfsetlinewidth{0.000000pt}%
\definecolor{currentstroke}{rgb}{0.237441,0.305202,0.541921}%
\pgfsetstrokecolor{currentstroke}%
\pgfsetdash{}{0pt}%
\pgfpathmoveto{\pgfqpoint{2.600661in}{3.915679in}}%
\pgfpathlineto{\pgfqpoint{2.684604in}{4.029075in}}%
\pgfpathlineto{\pgfqpoint{2.551905in}{4.029357in}}%
\pgfpathclose%
\pgfusepath{fill}%
\end{pgfscope}%
\begin{pgfscope}%
\pgfpathrectangle{\pgfqpoint{0.680860in}{0.078740in}}{\pgfqpoint{7.842520in}{7.842520in}}%
\pgfusepath{clip}%
\pgfsetbuttcap%
\pgfsetroundjoin%
\definecolor{currentfill}{rgb}{0.129933,0.559582,0.551864}%
\pgfsetfillcolor{currentfill}%
\pgfsetlinewidth{0.000000pt}%
\definecolor{currentstroke}{rgb}{0.235526,0.309527,0.542944}%
\pgfsetstrokecolor{currentstroke}%
\pgfsetdash{}{0pt}%
\pgfpathmoveto{\pgfqpoint{4.829973in}{4.527300in}}%
\pgfpathlineto{\pgfqpoint{4.968019in}{4.541588in}}%
\pgfpathlineto{\pgfqpoint{4.908499in}{4.498984in}}%
\pgfpathclose%
\pgfusepath{fill}%
\end{pgfscope}%
\begin{pgfscope}%
\pgfpathrectangle{\pgfqpoint{0.680860in}{0.078740in}}{\pgfqpoint{7.842520in}{7.842520in}}%
\pgfusepath{clip}%
\pgfsetbuttcap%
\pgfsetroundjoin%
\definecolor{currentfill}{rgb}{0.192357,0.403199,0.555836}%
\pgfsetfillcolor{currentfill}%
\pgfsetlinewidth{0.000000pt}%
\definecolor{currentstroke}{rgb}{0.233603,0.313828,0.543914}%
\pgfsetstrokecolor{currentstroke}%
\pgfsetdash{}{0pt}%
\pgfpathmoveto{\pgfqpoint{2.335487in}{3.920610in}}%
\pgfpathlineto{\pgfqpoint{2.251233in}{3.798735in}}%
\pgfpathlineto{\pgfqpoint{2.383706in}{3.793752in}}%
\pgfpathclose%
\pgfusepath{fill}%
\end{pgfscope}%
\begin{pgfscope}%
\pgfpathrectangle{\pgfqpoint{0.680860in}{0.078740in}}{\pgfqpoint{7.842520in}{7.842520in}}%
\pgfusepath{clip}%
\pgfsetbuttcap%
\pgfsetroundjoin%
\definecolor{currentfill}{rgb}{0.180629,0.429975,0.557282}%
\pgfsetfillcolor{currentfill}%
\pgfsetlinewidth{0.000000pt}%
\definecolor{currentstroke}{rgb}{0.231674,0.318106,0.544834}%
\pgfsetstrokecolor{currentstroke}%
\pgfsetdash{}{0pt}%
\pgfpathmoveto{\pgfqpoint{2.551905in}{4.029357in}}%
\pgfpathlineto{\pgfqpoint{2.467809in}{3.917569in}}%
\pgfpathlineto{\pgfqpoint{2.600661in}{3.915679in}}%
\pgfpathclose%
\pgfusepath{fill}%
\end{pgfscope}%
\begin{pgfscope}%
\pgfpathrectangle{\pgfqpoint{0.680860in}{0.078740in}}{\pgfqpoint{7.842520in}{7.842520in}}%
\pgfusepath{clip}%
\pgfsetbuttcap%
\pgfsetroundjoin%
\definecolor{currentfill}{rgb}{0.188923,0.410910,0.556326}%
\pgfsetfillcolor{currentfill}%
\pgfsetlinewidth{0.000000pt}%
\definecolor{currentstroke}{rgb}{0.229739,0.322361,0.545706}%
\pgfsetstrokecolor{currentstroke}%
\pgfsetdash{}{0pt}%
\pgfpathmoveto{\pgfqpoint{2.383706in}{3.793752in}}%
\pgfpathlineto{\pgfqpoint{2.467809in}{3.917569in}}%
\pgfpathlineto{\pgfqpoint{2.335487in}{3.920610in}}%
\pgfpathclose%
\pgfusepath{fill}%
\end{pgfscope}%
\begin{pgfscope}%
\pgfpathrectangle{\pgfqpoint{0.680860in}{0.078740in}}{\pgfqpoint{7.842520in}{7.842520in}}%
\pgfusepath{clip}%
\pgfsetbuttcap%
\pgfsetroundjoin%
\definecolor{currentfill}{rgb}{0.151918,0.500685,0.557587}%
\pgfsetfillcolor{currentfill}%
\pgfsetlinewidth{0.000000pt}%
\definecolor{currentstroke}{rgb}{0.227802,0.326594,0.546532}%
\pgfsetstrokecolor{currentstroke}%
\pgfsetdash{}{0pt}%
\pgfpathmoveto{\pgfqpoint{3.118708in}{4.224466in}}%
\pgfpathlineto{\pgfqpoint{3.252743in}{4.230215in}}%
\pgfpathlineto{\pgfqpoint{3.335906in}{4.305569in}}%
\pgfpathclose%
\pgfusepath{fill}%
\end{pgfscope}%
\begin{pgfscope}%
\pgfpathrectangle{\pgfqpoint{0.680860in}{0.078740in}}{\pgfqpoint{7.842520in}{7.842520in}}%
\pgfusepath{clip}%
\pgfsetbuttcap%
\pgfsetroundjoin%
\definecolor{currentfill}{rgb}{0.129933,0.559582,0.551864}%
\pgfsetfillcolor{currentfill}%
\pgfsetlinewidth{0.000000pt}%
\definecolor{currentstroke}{rgb}{0.225863,0.330805,0.547314}%
\pgfsetstrokecolor{currentstroke}%
\pgfsetdash{}{0pt}%
\pgfpathmoveto{\pgfqpoint{4.613040in}{4.524493in}}%
\pgfpathlineto{\pgfqpoint{4.692572in}{4.514706in}}%
\pgfpathlineto{\pgfqpoint{4.475933in}{4.511090in}}%
\pgfpathclose%
\pgfusepath{fill}%
\end{pgfscope}%
\begin{pgfscope}%
\pgfpathrectangle{\pgfqpoint{0.680860in}{0.078740in}}{\pgfqpoint{7.842520in}{7.842520in}}%
\pgfusepath{clip}%
\pgfsetbuttcap%
\pgfsetroundjoin%
\definecolor{currentfill}{rgb}{0.143343,0.522773,0.556295}%
\pgfsetfillcolor{currentfill}%
\pgfsetlinewidth{0.000000pt}%
\definecolor{currentstroke}{rgb}{0.223925,0.334994,0.548053}%
\pgfsetstrokecolor{currentstroke}%
\pgfsetdash{}{0pt}%
\pgfpathmoveto{\pgfqpoint{3.687912in}{4.382797in}}%
\pgfpathlineto{\pgfqpoint{3.553097in}{4.373815in}}%
\pgfpathlineto{\pgfqpoint{3.470334in}{4.313152in}}%
\pgfpathclose%
\pgfusepath{fill}%
\end{pgfscope}%
\begin{pgfscope}%
\pgfpathrectangle{\pgfqpoint{0.680860in}{0.078740in}}{\pgfqpoint{7.842520in}{7.842520in}}%
\pgfusepath{clip}%
\pgfsetbuttcap%
\pgfsetroundjoin%
\definecolor{currentfill}{rgb}{0.131172,0.555899,0.552459}%
\pgfsetfillcolor{currentfill}%
\pgfsetlinewidth{0.000000pt}%
\definecolor{currentstroke}{rgb}{0.221989,0.339161,0.548752}%
\pgfsetstrokecolor{currentstroke}%
\pgfsetdash{}{0pt}%
\pgfpathmoveto{\pgfqpoint{4.395520in}{4.502548in}}%
\pgfpathlineto{\pgfqpoint{4.475933in}{4.511090in}}%
\pgfpathlineto{\pgfqpoint{4.339456in}{4.499350in}}%
\pgfpathclose%
\pgfusepath{fill}%
\end{pgfscope}%
\begin{pgfscope}%
\pgfpathrectangle{\pgfqpoint{0.680860in}{0.078740in}}{\pgfqpoint{7.842520in}{7.842520in}}%
\pgfusepath{clip}%
\pgfsetbuttcap%
\pgfsetroundjoin%
\definecolor{currentfill}{rgb}{0.129933,0.559582,0.551864}%
\pgfsetfillcolor{currentfill}%
\pgfsetlinewidth{0.000000pt}%
\definecolor{currentstroke}{rgb}{0.220057,0.343307,0.549413}%
\pgfsetstrokecolor{currentstroke}%
\pgfsetdash{}{0pt}%
\pgfpathmoveto{\pgfqpoint{5.323395in}{4.537338in}}%
\pgfpathlineto{\pgfqpoint{5.261392in}{4.472255in}}%
\pgfpathlineto{\pgfqpoint{5.184443in}{4.522872in}}%
\pgfpathclose%
\pgfusepath{fill}%
\end{pgfscope}%
\begin{pgfscope}%
\pgfpathrectangle{\pgfqpoint{0.680860in}{0.078740in}}{\pgfqpoint{7.842520in}{7.842520in}}%
\pgfusepath{clip}%
\pgfsetbuttcap%
\pgfsetroundjoin%
\definecolor{currentfill}{rgb}{0.168126,0.459988,0.558082}%
\pgfsetfillcolor{currentfill}%
\pgfsetlinewidth{0.000000pt}%
\definecolor{currentstroke}{rgb}{0.218130,0.347432,0.550038}%
\pgfsetstrokecolor{currentstroke}%
\pgfsetdash{}{0pt}%
\pgfpathmoveto{\pgfqpoint{2.901587in}{4.131842in}}%
\pgfpathlineto{\pgfqpoint{2.684604in}{4.029075in}}%
\pgfpathlineto{\pgfqpoint{2.817845in}{4.030038in}}%
\pgfpathclose%
\pgfusepath{fill}%
\end{pgfscope}%
\begin{pgfscope}%
\pgfpathrectangle{\pgfqpoint{0.680860in}{0.078740in}}{\pgfqpoint{7.842520in}{7.842520in}}%
\pgfusepath{clip}%
\pgfsetbuttcap%
\pgfsetroundjoin%
\definecolor{currentfill}{rgb}{0.132444,0.552216,0.553018}%
\pgfsetfillcolor{currentfill}%
\pgfsetlinewidth{0.000000pt}%
\definecolor{currentstroke}{rgb}{0.216210,0.351535,0.550627}%
\pgfsetstrokecolor{currentstroke}%
\pgfsetdash{}{0pt}%
\pgfpathmoveto{\pgfqpoint{4.122598in}{4.477059in}}%
\pgfpathlineto{\pgfqpoint{4.041163in}{4.449339in}}%
\pgfpathlineto{\pgfqpoint{4.258747in}{4.488968in}}%
\pgfpathclose%
\pgfusepath{fill}%
\end{pgfscope}%
\begin{pgfscope}%
\pgfpathrectangle{\pgfqpoint{0.680860in}{0.078740in}}{\pgfqpoint{7.842520in}{7.842520in}}%
\pgfusepath{clip}%
\pgfsetbuttcap%
\pgfsetroundjoin%
\definecolor{currentfill}{rgb}{0.156270,0.489624,0.557936}%
\pgfsetfillcolor{currentfill}%
\pgfsetlinewidth{0.000000pt}%
\definecolor{currentstroke}{rgb}{0.214298,0.355619,0.551184}%
\pgfsetstrokecolor{currentstroke}%
\pgfsetdash{}{0pt}%
\pgfpathmoveto{\pgfqpoint{6.883614in}{4.255831in}}%
\pgfpathlineto{\pgfqpoint{6.808197in}{4.092323in}}%
\pgfpathlineto{\pgfqpoint{6.741080in}{4.248182in}}%
\pgfpathclose%
\pgfusepath{fill}%
\end{pgfscope}%
\begin{pgfscope}%
\pgfpathrectangle{\pgfqpoint{0.680860in}{0.078740in}}{\pgfqpoint{7.842520in}{7.842520in}}%
\pgfusepath{clip}%
\pgfsetbuttcap%
\pgfsetroundjoin%
\definecolor{currentfill}{rgb}{0.129933,0.559582,0.551864}%
\pgfsetfillcolor{currentfill}%
\pgfsetlinewidth{0.000000pt}%
\definecolor{currentstroke}{rgb}{0.212395,0.359683,0.551710}%
\pgfsetstrokecolor{currentstroke}%
\pgfsetdash{}{0pt}%
\pgfpathmoveto{\pgfqpoint{4.339456in}{4.499350in}}%
\pgfpathlineto{\pgfqpoint{4.258747in}{4.488968in}}%
\pgfpathlineto{\pgfqpoint{4.395520in}{4.502548in}}%
\pgfpathclose%
\pgfusepath{fill}%
\end{pgfscope}%
\begin{pgfscope}%
\pgfpathrectangle{\pgfqpoint{0.680860in}{0.078740in}}{\pgfqpoint{7.842520in}{7.842520in}}%
\pgfusepath{clip}%
\pgfsetbuttcap%
\pgfsetroundjoin%
\definecolor{currentfill}{rgb}{0.149039,0.508051,0.557250}%
\pgfsetfillcolor{currentfill}%
\pgfsetlinewidth{0.000000pt}%
\definecolor{currentstroke}{rgb}{0.210503,0.363727,0.552206}%
\pgfsetstrokecolor{currentstroke}%
\pgfsetdash{}{0pt}%
\pgfpathmoveto{\pgfqpoint{3.252743in}{4.230215in}}%
\pgfpathlineto{\pgfqpoint{3.470334in}{4.313152in}}%
\pgfpathlineto{\pgfqpoint{3.335906in}{4.305569in}}%
\pgfpathclose%
\pgfusepath{fill}%
\end{pgfscope}%
\begin{pgfscope}%
\pgfpathrectangle{\pgfqpoint{0.680860in}{0.078740in}}{\pgfqpoint{7.842520in}{7.842520in}}%
\pgfusepath{clip}%
\pgfsetbuttcap%
\pgfsetroundjoin%
\definecolor{currentfill}{rgb}{0.147607,0.511733,0.557049}%
\pgfsetfillcolor{currentfill}%
\pgfsetlinewidth{0.000000pt}%
\definecolor{currentstroke}{rgb}{0.208623,0.367752,0.552675}%
\pgfsetstrokecolor{currentstroke}%
\pgfsetdash{}{0pt}%
\pgfpathmoveto{\pgfqpoint{6.599226in}{4.242008in}}%
\pgfpathlineto{\pgfqpoint{6.458040in}{4.237255in}}%
\pgfpathlineto{\pgfqpoint{6.530116in}{4.378410in}}%
\pgfpathclose%
\pgfusepath{fill}%
\end{pgfscope}%
\begin{pgfscope}%
\pgfpathrectangle{\pgfqpoint{0.680860in}{0.078740in}}{\pgfqpoint{7.842520in}{7.842520in}}%
\pgfusepath{clip}%
\pgfsetbuttcap%
\pgfsetroundjoin%
\definecolor{currentfill}{rgb}{0.127568,0.566949,0.550556}%
\pgfsetfillcolor{currentfill}%
\pgfsetlinewidth{0.000000pt}%
\definecolor{currentstroke}{rgb}{0.206756,0.371758,0.553117}%
\pgfsetstrokecolor{currentstroke}%
\pgfsetdash{}{0pt}%
\pgfpathmoveto{\pgfqpoint{5.184443in}{4.522872in}}%
\pgfpathlineto{\pgfqpoint{5.046149in}{4.510108in}}%
\pgfpathlineto{\pgfqpoint{4.968019in}{4.541588in}}%
\pgfpathclose%
\pgfusepath{fill}%
\end{pgfscope}%
\begin{pgfscope}%
\pgfpathrectangle{\pgfqpoint{0.680860in}{0.078740in}}{\pgfqpoint{7.842520in}{7.842520in}}%
\pgfusepath{clip}%
\pgfsetbuttcap%
\pgfsetroundjoin%
\definecolor{currentfill}{rgb}{0.136408,0.541173,0.554483}%
\pgfsetfillcolor{currentfill}%
\pgfsetlinewidth{0.000000pt}%
\definecolor{currentstroke}{rgb}{0.204903,0.375746,0.553533}%
\pgfsetstrokecolor{currentstroke}%
\pgfsetdash{}{0pt}%
\pgfpathmoveto{\pgfqpoint{6.106265in}{4.349010in}}%
\pgfpathlineto{\pgfqpoint{5.893304in}{4.440669in}}%
\pgfpathlineto{\pgfqpoint{6.033892in}{4.452976in}}%
\pgfpathclose%
\pgfusepath{fill}%
\end{pgfscope}%
\begin{pgfscope}%
\pgfpathrectangle{\pgfqpoint{0.680860in}{0.078740in}}{\pgfqpoint{7.842520in}{7.842520in}}%
\pgfusepath{clip}%
\pgfsetbuttcap%
\pgfsetroundjoin%
\definecolor{currentfill}{rgb}{0.136408,0.541173,0.554483}%
\pgfsetfillcolor{currentfill}%
\pgfsetlinewidth{0.000000pt}%
\definecolor{currentstroke}{rgb}{0.203063,0.379716,0.553925}%
\pgfsetstrokecolor{currentstroke}%
\pgfsetdash{}{0pt}%
\pgfpathmoveto{\pgfqpoint{3.905372in}{4.437840in}}%
\pgfpathlineto{\pgfqpoint{3.687912in}{4.382797in}}%
\pgfpathlineto{\pgfqpoint{3.823323in}{4.393347in}}%
\pgfpathclose%
\pgfusepath{fill}%
\end{pgfscope}%
\begin{pgfscope}%
\pgfpathrectangle{\pgfqpoint{0.680860in}{0.078740in}}{\pgfqpoint{7.842520in}{7.842520in}}%
\pgfusepath{clip}%
\pgfsetbuttcap%
\pgfsetroundjoin%
\definecolor{currentfill}{rgb}{0.127568,0.566949,0.550556}%
\pgfsetfillcolor{currentfill}%
\pgfsetlinewidth{0.000000pt}%
\definecolor{currentstroke}{rgb}{0.201239,0.383670,0.554294}%
\pgfsetstrokecolor{currentstroke}%
\pgfsetdash{}{0pt}%
\pgfpathmoveto{\pgfqpoint{4.829973in}{4.527300in}}%
\pgfpathlineto{\pgfqpoint{4.692572in}{4.514706in}}%
\pgfpathlineto{\pgfqpoint{4.750789in}{4.539623in}}%
\pgfpathclose%
\pgfusepath{fill}%
\end{pgfscope}%
\begin{pgfscope}%
\pgfpathrectangle{\pgfqpoint{0.680860in}{0.078740in}}{\pgfqpoint{7.842520in}{7.842520in}}%
\pgfusepath{clip}%
\pgfsetbuttcap%
\pgfsetroundjoin%
\definecolor{currentfill}{rgb}{0.177423,0.437527,0.557565}%
\pgfsetfillcolor{currentfill}%
\pgfsetlinewidth{0.000000pt}%
\definecolor{currentstroke}{rgb}{0.199430,0.387607,0.554642}%
\pgfsetstrokecolor{currentstroke}%
\pgfsetdash{}{0pt}%
\pgfpathmoveto{\pgfqpoint{7.440672in}{3.901876in}}%
\pgfpathlineto{\pgfqpoint{7.378712in}{4.100689in}}%
\pgfpathlineto{\pgfqpoint{7.583875in}{3.898413in}}%
\pgfpathclose%
\pgfusepath{fill}%
\end{pgfscope}%
\begin{pgfscope}%
\pgfpathrectangle{\pgfqpoint{0.680860in}{0.078740in}}{\pgfqpoint{7.842520in}{7.842520in}}%
\pgfusepath{clip}%
\pgfsetbuttcap%
\pgfsetroundjoin%
\definecolor{currentfill}{rgb}{0.169646,0.456262,0.558030}%
\pgfsetfillcolor{currentfill}%
\pgfsetlinewidth{0.000000pt}%
\definecolor{currentstroke}{rgb}{0.197636,0.391528,0.554969}%
\pgfsetstrokecolor{currentstroke}%
\pgfsetdash{}{0pt}%
\pgfpathmoveto{\pgfqpoint{7.298112in}{3.906438in}}%
\pgfpathlineto{\pgfqpoint{7.235076in}{4.096611in}}%
\pgfpathlineto{\pgfqpoint{7.378712in}{4.100689in}}%
\pgfpathclose%
\pgfusepath{fill}%
\end{pgfscope}%
\begin{pgfscope}%
\pgfpathrectangle{\pgfqpoint{0.680860in}{0.078740in}}{\pgfqpoint{7.842520in}{7.842520in}}%
\pgfusepath{clip}%
\pgfsetbuttcap%
\pgfsetroundjoin%
\definecolor{currentfill}{rgb}{0.132444,0.552216,0.553018}%
\pgfsetfillcolor{currentfill}%
\pgfsetlinewidth{0.000000pt}%
\definecolor{currentstroke}{rgb}{0.195860,0.395433,0.555276}%
\pgfsetstrokecolor{currentstroke}%
\pgfsetdash{}{0pt}%
\pgfpathmoveto{\pgfqpoint{5.893304in}{4.440669in}}%
\pgfpathlineto{\pgfqpoint{5.753388in}{4.429993in}}%
\pgfpathlineto{\pgfqpoint{5.678820in}{4.508785in}}%
\pgfpathclose%
\pgfusepath{fill}%
\end{pgfscope}%
\begin{pgfscope}%
\pgfpathrectangle{\pgfqpoint{0.680860in}{0.078740in}}{\pgfqpoint{7.842520in}{7.842520in}}%
\pgfusepath{clip}%
\pgfsetbuttcap%
\pgfsetroundjoin%
\definecolor{currentfill}{rgb}{0.128729,0.563265,0.551229}%
\pgfsetfillcolor{currentfill}%
\pgfsetlinewidth{0.000000pt}%
\definecolor{currentstroke}{rgb}{0.194100,0.399323,0.555565}%
\pgfsetstrokecolor{currentstroke}%
\pgfsetdash{}{0pt}%
\pgfpathmoveto{\pgfqpoint{5.399877in}{4.482791in}}%
\pgfpathlineto{\pgfqpoint{5.323395in}{4.537338in}}%
\pgfpathlineto{\pgfqpoint{5.539015in}{4.494947in}}%
\pgfpathclose%
\pgfusepath{fill}%
\end{pgfscope}%
\begin{pgfscope}%
\pgfpathrectangle{\pgfqpoint{0.680860in}{0.078740in}}{\pgfqpoint{7.842520in}{7.842520in}}%
\pgfusepath{clip}%
\pgfsetbuttcap%
\pgfsetroundjoin%
\definecolor{currentfill}{rgb}{0.131172,0.555899,0.552459}%
\pgfsetfillcolor{currentfill}%
\pgfsetlinewidth{0.000000pt}%
\definecolor{currentstroke}{rgb}{0.192357,0.403199,0.555836}%
\pgfsetstrokecolor{currentstroke}%
\pgfsetdash{}{0pt}%
\pgfpathmoveto{\pgfqpoint{5.678820in}{4.508785in}}%
\pgfpathlineto{\pgfqpoint{5.753388in}{4.429993in}}%
\pgfpathlineto{\pgfqpoint{5.539015in}{4.494947in}}%
\pgfpathclose%
\pgfusepath{fill}%
\end{pgfscope}%
\begin{pgfscope}%
\pgfpathrectangle{\pgfqpoint{0.680860in}{0.078740in}}{\pgfqpoint{7.842520in}{7.842520in}}%
\pgfusepath{clip}%
\pgfsetbuttcap%
\pgfsetroundjoin%
\definecolor{currentfill}{rgb}{0.163625,0.471133,0.558148}%
\pgfsetfillcolor{currentfill}%
\pgfsetlinewidth{0.000000pt}%
\definecolor{currentstroke}{rgb}{0.190631,0.407061,0.556089}%
\pgfsetstrokecolor{currentstroke}%
\pgfsetdash{}{0pt}%
\pgfpathmoveto{\pgfqpoint{3.035223in}{4.135367in}}%
\pgfpathlineto{\pgfqpoint{2.901587in}{4.131842in}}%
\pgfpathlineto{\pgfqpoint{2.817845in}{4.030038in}}%
\pgfpathclose%
\pgfusepath{fill}%
\end{pgfscope}%
\begin{pgfscope}%
\pgfpathrectangle{\pgfqpoint{0.680860in}{0.078740in}}{\pgfqpoint{7.842520in}{7.842520in}}%
\pgfusepath{clip}%
\pgfsetbuttcap%
\pgfsetroundjoin%
\definecolor{currentfill}{rgb}{0.199430,0.387607,0.554642}%
\pgfsetfillcolor{currentfill}%
\pgfsetlinewidth{0.000000pt}%
\definecolor{currentstroke}{rgb}{0.188923,0.410910,0.556326}%
\pgfsetstrokecolor{currentstroke}%
\pgfsetdash{}{0pt}%
\pgfpathmoveto{\pgfqpoint{2.383706in}{3.793752in}}%
\pgfpathlineto{\pgfqpoint{2.251233in}{3.798735in}}%
\pgfpathlineto{\pgfqpoint{2.166997in}{3.665702in}}%
\pgfpathclose%
\pgfusepath{fill}%
\end{pgfscope}%
\begin{pgfscope}%
\pgfpathrectangle{\pgfqpoint{0.680860in}{0.078740in}}{\pgfqpoint{7.842520in}{7.842520in}}%
\pgfusepath{clip}%
\pgfsetbuttcap%
\pgfsetroundjoin%
\definecolor{currentfill}{rgb}{0.174274,0.445044,0.557792}%
\pgfsetfillcolor{currentfill}%
\pgfsetlinewidth{0.000000pt}%
\definecolor{currentstroke}{rgb}{0.187231,0.414746,0.556547}%
\pgfsetstrokecolor{currentstroke}%
\pgfsetdash{}{0pt}%
\pgfpathmoveto{\pgfqpoint{2.817845in}{4.030038in}}%
\pgfpathlineto{\pgfqpoint{2.684604in}{4.029075in}}%
\pgfpathlineto{\pgfqpoint{2.600661in}{3.915679in}}%
\pgfpathclose%
\pgfusepath{fill}%
\end{pgfscope}%
\begin{pgfscope}%
\pgfpathrectangle{\pgfqpoint{0.680860in}{0.078740in}}{\pgfqpoint{7.842520in}{7.842520in}}%
\pgfusepath{clip}%
\pgfsetbuttcap%
\pgfsetroundjoin%
\definecolor{currentfill}{rgb}{0.185556,0.418570,0.556753}%
\pgfsetfillcolor{currentfill}%
\pgfsetlinewidth{0.000000pt}%
\definecolor{currentstroke}{rgb}{0.185556,0.418570,0.556753}%
\pgfsetstrokecolor{currentstroke}%
\pgfsetdash{}{0pt}%
\pgfpathmoveto{\pgfqpoint{2.467809in}{3.917569in}}%
\pgfpathlineto{\pgfqpoint{2.383706in}{3.793752in}}%
\pgfpathlineto{\pgfqpoint{2.600661in}{3.915679in}}%
\pgfpathclose%
\pgfusepath{fill}%
\end{pgfscope}%
\begin{pgfscope}%
\pgfpathrectangle{\pgfqpoint{0.680860in}{0.078740in}}{\pgfqpoint{7.842520in}{7.842520in}}%
\pgfusepath{clip}%
\pgfsetbuttcap%
\pgfsetroundjoin%
\definecolor{currentfill}{rgb}{0.137770,0.537492,0.554906}%
\pgfsetfillcolor{currentfill}%
\pgfsetlinewidth{0.000000pt}%
\definecolor{currentstroke}{rgb}{0.183898,0.422383,0.556944}%
\pgfsetstrokecolor{currentstroke}%
\pgfsetdash{}{0pt}%
\pgfpathmoveto{\pgfqpoint{6.175169in}{4.466975in}}%
\pgfpathlineto{\pgfqpoint{6.246871in}{4.357246in}}%
\pgfpathlineto{\pgfqpoint{6.106265in}{4.349010in}}%
\pgfpathclose%
\pgfusepath{fill}%
\end{pgfscope}%
\begin{pgfscope}%
\pgfpathrectangle{\pgfqpoint{0.680860in}{0.078740in}}{\pgfqpoint{7.842520in}{7.842520in}}%
\pgfusepath{clip}%
\pgfsetbuttcap%
\pgfsetroundjoin%
\definecolor{currentfill}{rgb}{0.157729,0.485932,0.558013}%
\pgfsetfillcolor{currentfill}%
\pgfsetlinewidth{0.000000pt}%
\definecolor{currentstroke}{rgb}{0.182256,0.426184,0.557120}%
\pgfsetstrokecolor{currentstroke}%
\pgfsetdash{}{0pt}%
\pgfpathmoveto{\pgfqpoint{7.026844in}{4.265011in}}%
\pgfpathlineto{\pgfqpoint{7.092121in}{4.093890in}}%
\pgfpathlineto{\pgfqpoint{6.949832in}{4.092476in}}%
\pgfpathclose%
\pgfusepath{fill}%
\end{pgfscope}%
\begin{pgfscope}%
\pgfpathrectangle{\pgfqpoint{0.680860in}{0.078740in}}{\pgfqpoint{7.842520in}{7.842520in}}%
\pgfusepath{clip}%
\pgfsetbuttcap%
\pgfsetroundjoin%
\definecolor{currentfill}{rgb}{0.126453,0.570633,0.549841}%
\pgfsetfillcolor{currentfill}%
\pgfsetlinewidth{0.000000pt}%
\definecolor{currentstroke}{rgb}{0.180629,0.429975,0.557282}%
\pgfsetstrokecolor{currentstroke}%
\pgfsetdash{}{0pt}%
\pgfpathmoveto{\pgfqpoint{4.750789in}{4.539623in}}%
\pgfpathlineto{\pgfqpoint{4.692572in}{4.514706in}}%
\pgfpathlineto{\pgfqpoint{4.613040in}{4.524493in}}%
\pgfpathclose%
\pgfusepath{fill}%
\end{pgfscope}%
\begin{pgfscope}%
\pgfpathrectangle{\pgfqpoint{0.680860in}{0.078740in}}{\pgfqpoint{7.842520in}{7.842520in}}%
\pgfusepath{clip}%
\pgfsetbuttcap%
\pgfsetroundjoin%
\definecolor{currentfill}{rgb}{0.140536,0.530132,0.555659}%
\pgfsetfillcolor{currentfill}%
\pgfsetlinewidth{0.000000pt}%
\definecolor{currentstroke}{rgb}{0.179019,0.433756,0.557430}%
\pgfsetstrokecolor{currentstroke}%
\pgfsetdash{}{0pt}%
\pgfpathmoveto{\pgfqpoint{3.687912in}{4.382797in}}%
\pgfpathlineto{\pgfqpoint{3.470334in}{4.313152in}}%
\pgfpathlineto{\pgfqpoint{3.605349in}{4.322255in}}%
\pgfpathclose%
\pgfusepath{fill}%
\end{pgfscope}%
\begin{pgfscope}%
\pgfpathrectangle{\pgfqpoint{0.680860in}{0.078740in}}{\pgfqpoint{7.842520in}{7.842520in}}%
\pgfusepath{clip}%
\pgfsetbuttcap%
\pgfsetroundjoin%
\definecolor{currentfill}{rgb}{0.153364,0.497000,0.557724}%
\pgfsetfillcolor{currentfill}%
\pgfsetlinewidth{0.000000pt}%
\definecolor{currentstroke}{rgb}{0.177423,0.437527,0.557565}%
\pgfsetstrokecolor{currentstroke}%
\pgfsetdash{}{0pt}%
\pgfpathmoveto{\pgfqpoint{3.169424in}{4.140274in}}%
\pgfpathlineto{\pgfqpoint{3.252743in}{4.230215in}}%
\pgfpathlineto{\pgfqpoint{3.118708in}{4.224466in}}%
\pgfpathclose%
\pgfusepath{fill}%
\end{pgfscope}%
\begin{pgfscope}%
\pgfpathrectangle{\pgfqpoint{0.680860in}{0.078740in}}{\pgfqpoint{7.842520in}{7.842520in}}%
\pgfusepath{clip}%
\pgfsetbuttcap%
\pgfsetroundjoin%
\definecolor{currentfill}{rgb}{0.141935,0.526453,0.555991}%
\pgfsetfillcolor{currentfill}%
\pgfsetlinewidth{0.000000pt}%
\definecolor{currentstroke}{rgb}{0.175841,0.441290,0.557685}%
\pgfsetstrokecolor{currentstroke}%
\pgfsetdash{}{0pt}%
\pgfpathmoveto{\pgfqpoint{6.530116in}{4.378410in}}%
\pgfpathlineto{\pgfqpoint{6.458040in}{4.237255in}}%
\pgfpathlineto{\pgfqpoint{6.388150in}{4.367027in}}%
\pgfpathclose%
\pgfusepath{fill}%
\end{pgfscope}%
\begin{pgfscope}%
\pgfpathrectangle{\pgfqpoint{0.680860in}{0.078740in}}{\pgfqpoint{7.842520in}{7.842520in}}%
\pgfusepath{clip}%
\pgfsetbuttcap%
\pgfsetroundjoin%
\definecolor{currentfill}{rgb}{0.156270,0.489624,0.557936}%
\pgfsetfillcolor{currentfill}%
\pgfsetlinewidth{0.000000pt}%
\definecolor{currentstroke}{rgb}{0.174274,0.445044,0.557792}%
\pgfsetstrokecolor{currentstroke}%
\pgfsetdash{}{0pt}%
\pgfpathmoveto{\pgfqpoint{3.118708in}{4.224466in}}%
\pgfpathlineto{\pgfqpoint{3.035223in}{4.135367in}}%
\pgfpathlineto{\pgfqpoint{3.169424in}{4.140274in}}%
\pgfpathclose%
\pgfusepath{fill}%
\end{pgfscope}%
\begin{pgfscope}%
\pgfpathrectangle{\pgfqpoint{0.680860in}{0.078740in}}{\pgfqpoint{7.842520in}{7.842520in}}%
\pgfusepath{clip}%
\pgfsetbuttcap%
\pgfsetroundjoin%
\definecolor{currentfill}{rgb}{0.144759,0.519093,0.556572}%
\pgfsetfillcolor{currentfill}%
\pgfsetlinewidth{0.000000pt}%
\definecolor{currentstroke}{rgb}{0.172719,0.448791,0.557885}%
\pgfsetstrokecolor{currentstroke}%
\pgfsetdash{}{0pt}%
\pgfpathmoveto{\pgfqpoint{6.741080in}{4.248182in}}%
\pgfpathlineto{\pgfqpoint{6.599226in}{4.242008in}}%
\pgfpathlineto{\pgfqpoint{6.530116in}{4.378410in}}%
\pgfpathclose%
\pgfusepath{fill}%
\end{pgfscope}%
\begin{pgfscope}%
\pgfpathrectangle{\pgfqpoint{0.680860in}{0.078740in}}{\pgfqpoint{7.842520in}{7.842520in}}%
\pgfusepath{clip}%
\pgfsetbuttcap%
\pgfsetroundjoin%
\definecolor{currentfill}{rgb}{0.147607,0.511733,0.557049}%
\pgfsetfillcolor{currentfill}%
\pgfsetlinewidth{0.000000pt}%
\definecolor{currentstroke}{rgb}{0.171176,0.452530,0.557965}%
\pgfsetstrokecolor{currentstroke}%
\pgfsetdash{}{0pt}%
\pgfpathmoveto{\pgfqpoint{3.387352in}{4.237421in}}%
\pgfpathlineto{\pgfqpoint{3.470334in}{4.313152in}}%
\pgfpathlineto{\pgfqpoint{3.252743in}{4.230215in}}%
\pgfpathclose%
\pgfusepath{fill}%
\end{pgfscope}%
\begin{pgfscope}%
\pgfpathrectangle{\pgfqpoint{0.680860in}{0.078740in}}{\pgfqpoint{7.842520in}{7.842520in}}%
\pgfusepath{clip}%
\pgfsetbuttcap%
\pgfsetroundjoin%
\definecolor{currentfill}{rgb}{0.132444,0.552216,0.553018}%
\pgfsetfillcolor{currentfill}%
\pgfsetlinewidth{0.000000pt}%
\definecolor{currentstroke}{rgb}{0.169646,0.456262,0.558030}%
\pgfsetstrokecolor{currentstroke}%
\pgfsetdash{}{0pt}%
\pgfpathmoveto{\pgfqpoint{3.959342in}{4.405526in}}%
\pgfpathlineto{\pgfqpoint{4.041163in}{4.449339in}}%
\pgfpathlineto{\pgfqpoint{3.905372in}{4.437840in}}%
\pgfpathclose%
\pgfusepath{fill}%
\end{pgfscope}%
\begin{pgfscope}%
\pgfpathrectangle{\pgfqpoint{0.680860in}{0.078740in}}{\pgfqpoint{7.842520in}{7.842520in}}%
\pgfusepath{clip}%
\pgfsetbuttcap%
\pgfsetroundjoin%
\definecolor{currentfill}{rgb}{0.128729,0.563265,0.551229}%
\pgfsetfillcolor{currentfill}%
\pgfsetlinewidth{0.000000pt}%
\definecolor{currentstroke}{rgb}{0.168126,0.459988,0.558082}%
\pgfsetstrokecolor{currentstroke}%
\pgfsetdash{}{0pt}%
\pgfpathmoveto{\pgfqpoint{4.258747in}{4.488968in}}%
\pgfpathlineto{\pgfqpoint{4.041163in}{4.449339in}}%
\pgfpathlineto{\pgfqpoint{4.177572in}{4.462498in}}%
\pgfpathclose%
\pgfusepath{fill}%
\end{pgfscope}%
\begin{pgfscope}%
\pgfpathrectangle{\pgfqpoint{0.680860in}{0.078740in}}{\pgfqpoint{7.842520in}{7.842520in}}%
\pgfusepath{clip}%
\pgfsetbuttcap%
\pgfsetroundjoin%
\definecolor{currentfill}{rgb}{0.132444,0.552216,0.553018}%
\pgfsetfillcolor{currentfill}%
\pgfsetlinewidth{0.000000pt}%
\definecolor{currentstroke}{rgb}{0.166617,0.463708,0.558119}%
\pgfsetstrokecolor{currentstroke}%
\pgfsetdash{}{0pt}%
\pgfpathmoveto{\pgfqpoint{6.106265in}{4.349010in}}%
\pgfpathlineto{\pgfqpoint{6.033892in}{4.452976in}}%
\pgfpathlineto{\pgfqpoint{6.175169in}{4.466975in}}%
\pgfpathclose%
\pgfusepath{fill}%
\end{pgfscope}%
\begin{pgfscope}%
\pgfpathrectangle{\pgfqpoint{0.680860in}{0.078740in}}{\pgfqpoint{7.842520in}{7.842520in}}%
\pgfusepath{clip}%
\pgfsetbuttcap%
\pgfsetroundjoin%
\definecolor{currentfill}{rgb}{0.125394,0.574318,0.549086}%
\pgfsetfillcolor{currentfill}%
\pgfsetlinewidth{0.000000pt}%
\definecolor{currentstroke}{rgb}{0.165117,0.467423,0.558141}%
\pgfsetstrokecolor{currentstroke}%
\pgfsetdash{}{0pt}%
\pgfpathmoveto{\pgfqpoint{4.613040in}{4.524493in}}%
\pgfpathlineto{\pgfqpoint{4.475933in}{4.511090in}}%
\pgfpathlineto{\pgfqpoint{4.532931in}{4.517865in}}%
\pgfpathclose%
\pgfusepath{fill}%
\end{pgfscope}%
\begin{pgfscope}%
\pgfpathrectangle{\pgfqpoint{0.680860in}{0.078740in}}{\pgfqpoint{7.842520in}{7.842520in}}%
\pgfusepath{clip}%
\pgfsetbuttcap%
\pgfsetroundjoin%
\definecolor{currentfill}{rgb}{0.133743,0.548535,0.553541}%
\pgfsetfillcolor{currentfill}%
\pgfsetlinewidth{0.000000pt}%
\definecolor{currentstroke}{rgb}{0.163625,0.471133,0.558148}%
\pgfsetstrokecolor{currentstroke}%
\pgfsetdash{}{0pt}%
\pgfpathmoveto{\pgfqpoint{3.905372in}{4.437840in}}%
\pgfpathlineto{\pgfqpoint{3.823323in}{4.393347in}}%
\pgfpathlineto{\pgfqpoint{3.959342in}{4.405526in}}%
\pgfpathclose%
\pgfusepath{fill}%
\end{pgfscope}%
\begin{pgfscope}%
\pgfpathrectangle{\pgfqpoint{0.680860in}{0.078740in}}{\pgfqpoint{7.842520in}{7.842520in}}%
\pgfusepath{clip}%
\pgfsetbuttcap%
\pgfsetroundjoin%
\definecolor{currentfill}{rgb}{0.124395,0.578002,0.548287}%
\pgfsetfillcolor{currentfill}%
\pgfsetlinewidth{0.000000pt}%
\definecolor{currentstroke}{rgb}{0.162142,0.474838,0.558140}%
\pgfsetstrokecolor{currentstroke}%
\pgfsetdash{}{0pt}%
\pgfpathmoveto{\pgfqpoint{4.750789in}{4.539623in}}%
\pgfpathlineto{\pgfqpoint{4.968019in}{4.541588in}}%
\pgfpathlineto{\pgfqpoint{4.829973in}{4.527300in}}%
\pgfpathclose%
\pgfusepath{fill}%
\end{pgfscope}%
\begin{pgfscope}%
\pgfpathrectangle{\pgfqpoint{0.680860in}{0.078740in}}{\pgfqpoint{7.842520in}{7.842520in}}%
\pgfusepath{clip}%
\pgfsetbuttcap%
\pgfsetroundjoin%
\definecolor{currentfill}{rgb}{0.199430,0.387607,0.554642}%
\pgfsetfillcolor{currentfill}%
\pgfsetlinewidth{0.000000pt}%
\definecolor{currentstroke}{rgb}{0.160665,0.478540,0.558115}%
\pgfsetstrokecolor{currentstroke}%
\pgfsetdash{}{0pt}%
\pgfpathmoveto{\pgfqpoint{2.166997in}{3.665702in}}%
\pgfpathlineto{\pgfqpoint{2.299617in}{3.658412in}}%
\pgfpathlineto{\pgfqpoint{2.383706in}{3.793752in}}%
\pgfpathclose%
\pgfusepath{fill}%
\end{pgfscope}%
\begin{pgfscope}%
\pgfpathrectangle{\pgfqpoint{0.680860in}{0.078740in}}{\pgfqpoint{7.842520in}{7.842520in}}%
\pgfusepath{clip}%
\pgfsetbuttcap%
\pgfsetroundjoin%
\definecolor{currentfill}{rgb}{0.163625,0.471133,0.558148}%
\pgfsetfillcolor{currentfill}%
\pgfsetlinewidth{0.000000pt}%
\definecolor{currentstroke}{rgb}{0.159194,0.482237,0.558073}%
\pgfsetstrokecolor{currentstroke}%
\pgfsetdash{}{0pt}%
\pgfpathmoveto{\pgfqpoint{3.035223in}{4.135367in}}%
\pgfpathlineto{\pgfqpoint{2.817845in}{4.030038in}}%
\pgfpathlineto{\pgfqpoint{2.951638in}{4.032294in}}%
\pgfpathclose%
\pgfusepath{fill}%
\end{pgfscope}%
\begin{pgfscope}%
\pgfpathrectangle{\pgfqpoint{0.680860in}{0.078740in}}{\pgfqpoint{7.842520in}{7.842520in}}%
\pgfusepath{clip}%
\pgfsetbuttcap%
\pgfsetroundjoin%
\definecolor{currentfill}{rgb}{0.185556,0.418570,0.556753}%
\pgfsetfillcolor{currentfill}%
\pgfsetlinewidth{0.000000pt}%
\definecolor{currentstroke}{rgb}{0.157729,0.485932,0.558013}%
\pgfsetstrokecolor{currentstroke}%
\pgfsetdash{}{0pt}%
\pgfpathmoveto{\pgfqpoint{2.600661in}{3.915679in}}%
\pgfpathlineto{\pgfqpoint{2.383706in}{3.793752in}}%
\pgfpathlineto{\pgfqpoint{2.516704in}{3.789858in}}%
\pgfpathclose%
\pgfusepath{fill}%
\end{pgfscope}%
\begin{pgfscope}%
\pgfpathrectangle{\pgfqpoint{0.680860in}{0.078740in}}{\pgfqpoint{7.842520in}{7.842520in}}%
\pgfusepath{clip}%
\pgfsetbuttcap%
\pgfsetroundjoin%
\definecolor{currentfill}{rgb}{0.150476,0.504369,0.557430}%
\pgfsetfillcolor{currentfill}%
\pgfsetlinewidth{0.000000pt}%
\definecolor{currentstroke}{rgb}{0.156270,0.489624,0.557936}%
\pgfsetstrokecolor{currentstroke}%
\pgfsetdash{}{0pt}%
\pgfpathmoveto{\pgfqpoint{6.949832in}{4.092476in}}%
\pgfpathlineto{\pgfqpoint{6.883614in}{4.255831in}}%
\pgfpathlineto{\pgfqpoint{7.026844in}{4.265011in}}%
\pgfpathclose%
\pgfusepath{fill}%
\end{pgfscope}%
\begin{pgfscope}%
\pgfpathrectangle{\pgfqpoint{0.680860in}{0.078740in}}{\pgfqpoint{7.842520in}{7.842520in}}%
\pgfusepath{clip}%
\pgfsetbuttcap%
\pgfsetroundjoin%
\definecolor{currentfill}{rgb}{0.174274,0.445044,0.557792}%
\pgfsetfillcolor{currentfill}%
\pgfsetlinewidth{0.000000pt}%
\definecolor{currentstroke}{rgb}{0.154815,0.493313,0.557840}%
\pgfsetstrokecolor{currentstroke}%
\pgfsetdash{}{0pt}%
\pgfpathmoveto{\pgfqpoint{2.600661in}{3.915679in}}%
\pgfpathlineto{\pgfqpoint{2.734052in}{3.914985in}}%
\pgfpathlineto{\pgfqpoint{2.817845in}{4.030038in}}%
\pgfpathclose%
\pgfusepath{fill}%
\end{pgfscope}%
\begin{pgfscope}%
\pgfpathrectangle{\pgfqpoint{0.680860in}{0.078740in}}{\pgfqpoint{7.842520in}{7.842520in}}%
\pgfusepath{clip}%
\pgfsetbuttcap%
\pgfsetroundjoin%
\definecolor{currentfill}{rgb}{0.127568,0.566949,0.550556}%
\pgfsetfillcolor{currentfill}%
\pgfsetlinewidth{0.000000pt}%
\definecolor{currentstroke}{rgb}{0.153364,0.497000,0.557724}%
\pgfsetstrokecolor{currentstroke}%
\pgfsetdash{}{0pt}%
\pgfpathmoveto{\pgfqpoint{4.395520in}{4.502548in}}%
\pgfpathlineto{\pgfqpoint{4.258747in}{4.488968in}}%
\pgfpathlineto{\pgfqpoint{4.177572in}{4.462498in}}%
\pgfpathclose%
\pgfusepath{fill}%
\end{pgfscope}%
\begin{pgfscope}%
\pgfpathrectangle{\pgfqpoint{0.680860in}{0.078740in}}{\pgfqpoint{7.842520in}{7.842520in}}%
\pgfusepath{clip}%
\pgfsetbuttcap%
\pgfsetroundjoin%
\definecolor{currentfill}{rgb}{0.125394,0.574318,0.549086}%
\pgfsetfillcolor{currentfill}%
\pgfsetlinewidth{0.000000pt}%
\definecolor{currentstroke}{rgb}{0.151918,0.500685,0.557587}%
\pgfsetstrokecolor{currentstroke}%
\pgfsetdash{}{0pt}%
\pgfpathmoveto{\pgfqpoint{4.475933in}{4.511090in}}%
\pgfpathlineto{\pgfqpoint{4.395520in}{4.502548in}}%
\pgfpathlineto{\pgfqpoint{4.532931in}{4.517865in}}%
\pgfpathclose%
\pgfusepath{fill}%
\end{pgfscope}%
\begin{pgfscope}%
\pgfpathrectangle{\pgfqpoint{0.680860in}{0.078740in}}{\pgfqpoint{7.842520in}{7.842520in}}%
\pgfusepath{clip}%
\pgfsetbuttcap%
\pgfsetroundjoin%
\definecolor{currentfill}{rgb}{0.154815,0.493313,0.557840}%
\pgfsetfillcolor{currentfill}%
\pgfsetlinewidth{0.000000pt}%
\definecolor{currentstroke}{rgb}{0.150476,0.504369,0.557430}%
\pgfsetstrokecolor{currentstroke}%
\pgfsetdash{}{0pt}%
\pgfpathmoveto{\pgfqpoint{7.026844in}{4.265011in}}%
\pgfpathlineto{\pgfqpoint{7.235076in}{4.096611in}}%
\pgfpathlineto{\pgfqpoint{7.092121in}{4.093890in}}%
\pgfpathclose%
\pgfusepath{fill}%
\end{pgfscope}%
\begin{pgfscope}%
\pgfpathrectangle{\pgfqpoint{0.680860in}{0.078740in}}{\pgfqpoint{7.842520in}{7.842520in}}%
\pgfusepath{clip}%
\pgfsetbuttcap%
\pgfsetroundjoin%
\definecolor{currentfill}{rgb}{0.123463,0.581687,0.547445}%
\pgfsetfillcolor{currentfill}%
\pgfsetlinewidth{0.000000pt}%
\definecolor{currentstroke}{rgb}{0.149039,0.508051,0.557250}%
\pgfsetstrokecolor{currentstroke}%
\pgfsetdash{}{0pt}%
\pgfpathmoveto{\pgfqpoint{5.106724in}{4.557637in}}%
\pgfpathlineto{\pgfqpoint{5.184443in}{4.522872in}}%
\pgfpathlineto{\pgfqpoint{4.968019in}{4.541588in}}%
\pgfpathclose%
\pgfusepath{fill}%
\end{pgfscope}%
\begin{pgfscope}%
\pgfpathrectangle{\pgfqpoint{0.680860in}{0.078740in}}{\pgfqpoint{7.842520in}{7.842520in}}%
\pgfusepath{clip}%
\pgfsetbuttcap%
\pgfsetroundjoin%
\definecolor{currentfill}{rgb}{0.136408,0.541173,0.554483}%
\pgfsetfillcolor{currentfill}%
\pgfsetlinewidth{0.000000pt}%
\definecolor{currentstroke}{rgb}{0.147607,0.511733,0.557049}%
\pgfsetstrokecolor{currentstroke}%
\pgfsetdash{}{0pt}%
\pgfpathmoveto{\pgfqpoint{3.823323in}{4.393347in}}%
\pgfpathlineto{\pgfqpoint{3.687912in}{4.382797in}}%
\pgfpathlineto{\pgfqpoint{3.740963in}{4.332939in}}%
\pgfpathclose%
\pgfusepath{fill}%
\end{pgfscope}%
\begin{pgfscope}%
\pgfpathrectangle{\pgfqpoint{0.680860in}{0.078740in}}{\pgfqpoint{7.842520in}{7.842520in}}%
\pgfusepath{clip}%
\pgfsetbuttcap%
\pgfsetroundjoin%
\definecolor{currentfill}{rgb}{0.208623,0.367752,0.552675}%
\pgfsetfillcolor{currentfill}%
\pgfsetlinewidth{0.000000pt}%
\definecolor{currentstroke}{rgb}{0.146180,0.515413,0.556823}%
\pgfsetstrokecolor{currentstroke}%
\pgfsetdash{}{0pt}%
\pgfpathmoveto{\pgfqpoint{2.166997in}{3.665702in}}%
\pgfpathlineto{\pgfqpoint{2.082795in}{3.522114in}}%
\pgfpathlineto{\pgfqpoint{2.299617in}{3.658412in}}%
\pgfpathclose%
\pgfusepath{fill}%
\end{pgfscope}%
\begin{pgfscope}%
\pgfpathrectangle{\pgfqpoint{0.680860in}{0.078740in}}{\pgfqpoint{7.842520in}{7.842520in}}%
\pgfusepath{clip}%
\pgfsetbuttcap%
\pgfsetroundjoin%
\definecolor{currentfill}{rgb}{0.123463,0.581687,0.547445}%
\pgfsetfillcolor{currentfill}%
\pgfsetlinewidth{0.000000pt}%
\definecolor{currentstroke}{rgb}{0.144759,0.519093,0.556572}%
\pgfsetstrokecolor{currentstroke}%
\pgfsetdash{}{0pt}%
\pgfpathmoveto{\pgfqpoint{5.323395in}{4.537338in}}%
\pgfpathlineto{\pgfqpoint{5.184443in}{4.522872in}}%
\pgfpathlineto{\pgfqpoint{5.106724in}{4.557637in}}%
\pgfpathclose%
\pgfusepath{fill}%
\end{pgfscope}%
\begin{pgfscope}%
\pgfpathrectangle{\pgfqpoint{0.680860in}{0.078740in}}{\pgfqpoint{7.842520in}{7.842520in}}%
\pgfusepath{clip}%
\pgfsetbuttcap%
\pgfsetroundjoin%
\definecolor{currentfill}{rgb}{0.137770,0.537492,0.554906}%
\pgfsetfillcolor{currentfill}%
\pgfsetlinewidth{0.000000pt}%
\definecolor{currentstroke}{rgb}{0.143343,0.522773,0.556295}%
\pgfsetstrokecolor{currentstroke}%
\pgfsetdash{}{0pt}%
\pgfpathmoveto{\pgfqpoint{3.740963in}{4.332939in}}%
\pgfpathlineto{\pgfqpoint{3.687912in}{4.382797in}}%
\pgfpathlineto{\pgfqpoint{3.605349in}{4.322255in}}%
\pgfpathclose%
\pgfusepath{fill}%
\end{pgfscope}%
\begin{pgfscope}%
\pgfpathrectangle{\pgfqpoint{0.680860in}{0.078740in}}{\pgfqpoint{7.842520in}{7.842520in}}%
\pgfusepath{clip}%
\pgfsetbuttcap%
\pgfsetroundjoin%
\definecolor{currentfill}{rgb}{0.123463,0.581687,0.547445}%
\pgfsetfillcolor{currentfill}%
\pgfsetlinewidth{0.000000pt}%
\definecolor{currentstroke}{rgb}{0.141935,0.526453,0.555991}%
\pgfsetstrokecolor{currentstroke}%
\pgfsetdash{}{0pt}%
\pgfpathmoveto{\pgfqpoint{5.539015in}{4.494947in}}%
\pgfpathlineto{\pgfqpoint{5.323395in}{4.537338in}}%
\pgfpathlineto{\pgfqpoint{5.463019in}{4.553572in}}%
\pgfpathclose%
\pgfusepath{fill}%
\end{pgfscope}%
\begin{pgfscope}%
\pgfpathrectangle{\pgfqpoint{0.680860in}{0.078740in}}{\pgfqpoint{7.842520in}{7.842520in}}%
\pgfusepath{clip}%
\pgfsetbuttcap%
\pgfsetroundjoin%
\definecolor{currentfill}{rgb}{0.125394,0.574318,0.549086}%
\pgfsetfillcolor{currentfill}%
\pgfsetlinewidth{0.000000pt}%
\definecolor{currentstroke}{rgb}{0.140536,0.530132,0.555659}%
\pgfsetstrokecolor{currentstroke}%
\pgfsetdash{}{0pt}%
\pgfpathmoveto{\pgfqpoint{5.678820in}{4.508785in}}%
\pgfpathlineto{\pgfqpoint{5.819308in}{4.524369in}}%
\pgfpathlineto{\pgfqpoint{5.893304in}{4.440669in}}%
\pgfpathclose%
\pgfusepath{fill}%
\end{pgfscope}%
\begin{pgfscope}%
\pgfpathrectangle{\pgfqpoint{0.680860in}{0.078740in}}{\pgfqpoint{7.842520in}{7.842520in}}%
\pgfusepath{clip}%
\pgfsetbuttcap%
\pgfsetroundjoin%
\definecolor{currentfill}{rgb}{0.140536,0.530132,0.555659}%
\pgfsetfillcolor{currentfill}%
\pgfsetlinewidth{0.000000pt}%
\definecolor{currentstroke}{rgb}{0.139147,0.533812,0.555298}%
\pgfsetstrokecolor{currentstroke}%
\pgfsetdash{}{0pt}%
\pgfpathmoveto{\pgfqpoint{3.605349in}{4.322255in}}%
\pgfpathlineto{\pgfqpoint{3.470334in}{4.313152in}}%
\pgfpathlineto{\pgfqpoint{3.522549in}{4.246142in}}%
\pgfpathclose%
\pgfusepath{fill}%
\end{pgfscope}%
\begin{pgfscope}%
\pgfpathrectangle{\pgfqpoint{0.680860in}{0.078740in}}{\pgfqpoint{7.842520in}{7.842520in}}%
\pgfusepath{clip}%
\pgfsetbuttcap%
\pgfsetroundjoin%
\definecolor{currentfill}{rgb}{0.143343,0.522773,0.556295}%
\pgfsetfillcolor{currentfill}%
\pgfsetlinewidth{0.000000pt}%
\definecolor{currentstroke}{rgb}{0.137770,0.537492,0.554906}%
\pgfsetstrokecolor{currentstroke}%
\pgfsetdash{}{0pt}%
\pgfpathmoveto{\pgfqpoint{3.522549in}{4.246142in}}%
\pgfpathlineto{\pgfqpoint{3.470334in}{4.313152in}}%
\pgfpathlineto{\pgfqpoint{3.387352in}{4.237421in}}%
\pgfpathclose%
\pgfusepath{fill}%
\end{pgfscope}%
\begin{pgfscope}%
\pgfpathrectangle{\pgfqpoint{0.680860in}{0.078740in}}{\pgfqpoint{7.842520in}{7.842520in}}%
\pgfusepath{clip}%
\pgfsetbuttcap%
\pgfsetroundjoin%
\definecolor{currentfill}{rgb}{0.131172,0.555899,0.552459}%
\pgfsetfillcolor{currentfill}%
\pgfsetlinewidth{0.000000pt}%
\definecolor{currentstroke}{rgb}{0.136408,0.541173,0.554483}%
\pgfsetstrokecolor{currentstroke}%
\pgfsetdash{}{0pt}%
\pgfpathmoveto{\pgfqpoint{6.246871in}{4.357246in}}%
\pgfpathlineto{\pgfqpoint{6.317148in}{4.482731in}}%
\pgfpathlineto{\pgfqpoint{6.388150in}{4.367027in}}%
\pgfpathclose%
\pgfusepath{fill}%
\end{pgfscope}%
\begin{pgfscope}%
\pgfpathrectangle{\pgfqpoint{0.680860in}{0.078740in}}{\pgfqpoint{7.842520in}{7.842520in}}%
\pgfusepath{clip}%
\pgfsetbuttcap%
\pgfsetroundjoin%
\definecolor{currentfill}{rgb}{0.163625,0.471133,0.558148}%
\pgfsetfillcolor{currentfill}%
\pgfsetlinewidth{0.000000pt}%
\definecolor{currentstroke}{rgb}{0.135066,0.544853,0.554029}%
\pgfsetstrokecolor{currentstroke}%
\pgfsetdash{}{0pt}%
\pgfpathmoveto{\pgfqpoint{7.583875in}{3.898413in}}%
\pgfpathlineto{\pgfqpoint{7.378712in}{4.100689in}}%
\pgfpathlineto{\pgfqpoint{7.523042in}{4.106176in}}%
\pgfpathclose%
\pgfusepath{fill}%
\end{pgfscope}%
\begin{pgfscope}%
\pgfpathrectangle{\pgfqpoint{0.680860in}{0.078740in}}{\pgfqpoint{7.842520in}{7.842520in}}%
\pgfusepath{clip}%
\pgfsetbuttcap%
\pgfsetroundjoin%
\definecolor{currentfill}{rgb}{0.147607,0.511733,0.557049}%
\pgfsetfillcolor{currentfill}%
\pgfsetlinewidth{0.000000pt}%
\definecolor{currentstroke}{rgb}{0.133743,0.548535,0.553541}%
\pgfsetstrokecolor{currentstroke}%
\pgfsetdash{}{0pt}%
\pgfpathmoveto{\pgfqpoint{3.304198in}{4.146617in}}%
\pgfpathlineto{\pgfqpoint{3.387352in}{4.237421in}}%
\pgfpathlineto{\pgfqpoint{3.252743in}{4.230215in}}%
\pgfpathclose%
\pgfusepath{fill}%
\end{pgfscope}%
\begin{pgfscope}%
\pgfpathrectangle{\pgfqpoint{0.680860in}{0.078740in}}{\pgfqpoint{7.842520in}{7.842520in}}%
\pgfusepath{clip}%
\pgfsetbuttcap%
\pgfsetroundjoin%
\definecolor{currentfill}{rgb}{0.169646,0.456262,0.558030}%
\pgfsetfillcolor{currentfill}%
\pgfsetlinewidth{0.000000pt}%
\definecolor{currentstroke}{rgb}{0.132444,0.552216,0.553018}%
\pgfsetstrokecolor{currentstroke}%
\pgfsetdash{}{0pt}%
\pgfpathmoveto{\pgfqpoint{7.668081in}{4.113125in}}%
\pgfpathlineto{\pgfqpoint{7.727735in}{3.896091in}}%
\pgfpathlineto{\pgfqpoint{7.583875in}{3.898413in}}%
\pgfpathclose%
\pgfusepath{fill}%
\end{pgfscope}%
\begin{pgfscope}%
\pgfpathrectangle{\pgfqpoint{0.680860in}{0.078740in}}{\pgfqpoint{7.842520in}{7.842520in}}%
\pgfusepath{clip}%
\pgfsetbuttcap%
\pgfsetroundjoin%
\definecolor{currentfill}{rgb}{0.197636,0.391528,0.554969}%
\pgfsetfillcolor{currentfill}%
\pgfsetlinewidth{0.000000pt}%
\definecolor{currentstroke}{rgb}{0.131172,0.555899,0.552459}%
\pgfsetstrokecolor{currentstroke}%
\pgfsetdash{}{0pt}%
\pgfpathmoveto{\pgfqpoint{2.383706in}{3.793752in}}%
\pgfpathlineto{\pgfqpoint{2.299617in}{3.658412in}}%
\pgfpathlineto{\pgfqpoint{2.432758in}{3.652139in}}%
\pgfpathclose%
\pgfusepath{fill}%
\end{pgfscope}%
\begin{pgfscope}%
\pgfpathrectangle{\pgfqpoint{0.680860in}{0.078740in}}{\pgfqpoint{7.842520in}{7.842520in}}%
\pgfusepath{clip}%
\pgfsetbuttcap%
\pgfsetroundjoin%
\definecolor{currentfill}{rgb}{0.122606,0.585371,0.546557}%
\pgfsetfillcolor{currentfill}%
\pgfsetlinewidth{0.000000pt}%
\definecolor{currentstroke}{rgb}{0.129933,0.559582,0.551864}%
\pgfsetstrokecolor{currentstroke}%
\pgfsetdash{}{0pt}%
\pgfpathmoveto{\pgfqpoint{4.532931in}{4.517865in}}%
\pgfpathlineto{\pgfqpoint{4.750789in}{4.539623in}}%
\pgfpathlineto{\pgfqpoint{4.613040in}{4.524493in}}%
\pgfpathclose%
\pgfusepath{fill}%
\end{pgfscope}%
\begin{pgfscope}%
\pgfpathrectangle{\pgfqpoint{0.680860in}{0.078740in}}{\pgfqpoint{7.842520in}{7.842520in}}%
\pgfusepath{clip}%
\pgfsetbuttcap%
\pgfsetroundjoin%
\definecolor{currentfill}{rgb}{0.150476,0.504369,0.557430}%
\pgfsetfillcolor{currentfill}%
\pgfsetlinewidth{0.000000pt}%
\definecolor{currentstroke}{rgb}{0.128729,0.563265,0.551229}%
\pgfsetstrokecolor{currentstroke}%
\pgfsetdash{}{0pt}%
\pgfpathmoveto{\pgfqpoint{3.252743in}{4.230215in}}%
\pgfpathlineto{\pgfqpoint{3.169424in}{4.140274in}}%
\pgfpathlineto{\pgfqpoint{3.304198in}{4.146617in}}%
\pgfpathclose%
\pgfusepath{fill}%
\end{pgfscope}%
\begin{pgfscope}%
\pgfpathrectangle{\pgfqpoint{0.680860in}{0.078740in}}{\pgfqpoint{7.842520in}{7.842520in}}%
\pgfusepath{clip}%
\pgfsetbuttcap%
\pgfsetroundjoin%
\definecolor{currentfill}{rgb}{0.192357,0.403199,0.555836}%
\pgfsetfillcolor{currentfill}%
\pgfsetlinewidth{0.000000pt}%
\definecolor{currentstroke}{rgb}{0.127568,0.566949,0.550556}%
\pgfsetstrokecolor{currentstroke}%
\pgfsetdash{}{0pt}%
\pgfpathmoveto{\pgfqpoint{2.432758in}{3.652139in}}%
\pgfpathlineto{\pgfqpoint{2.516704in}{3.789858in}}%
\pgfpathlineto{\pgfqpoint{2.383706in}{3.793752in}}%
\pgfpathclose%
\pgfusepath{fill}%
\end{pgfscope}%
\begin{pgfscope}%
\pgfpathrectangle{\pgfqpoint{0.680860in}{0.078740in}}{\pgfqpoint{7.842520in}{7.842520in}}%
\pgfusepath{clip}%
\pgfsetbuttcap%
\pgfsetroundjoin%
\definecolor{currentfill}{rgb}{0.182256,0.426184,0.557120}%
\pgfsetfillcolor{currentfill}%
\pgfsetlinewidth{0.000000pt}%
\definecolor{currentstroke}{rgb}{0.126453,0.570633,0.549841}%
\pgfsetstrokecolor{currentstroke}%
\pgfsetdash{}{0pt}%
\pgfpathmoveto{\pgfqpoint{2.600661in}{3.915679in}}%
\pgfpathlineto{\pgfqpoint{2.516704in}{3.789858in}}%
\pgfpathlineto{\pgfqpoint{2.650237in}{3.787098in}}%
\pgfpathclose%
\pgfusepath{fill}%
\end{pgfscope}%
\begin{pgfscope}%
\pgfpathrectangle{\pgfqpoint{0.680860in}{0.078740in}}{\pgfqpoint{7.842520in}{7.842520in}}%
\pgfusepath{clip}%
\pgfsetbuttcap%
\pgfsetroundjoin%
\definecolor{currentfill}{rgb}{0.156270,0.489624,0.557936}%
\pgfsetfillcolor{currentfill}%
\pgfsetlinewidth{0.000000pt}%
\definecolor{currentstroke}{rgb}{0.125394,0.574318,0.549086}%
\pgfsetstrokecolor{currentstroke}%
\pgfsetdash{}{0pt}%
\pgfpathmoveto{\pgfqpoint{3.169424in}{4.140274in}}%
\pgfpathlineto{\pgfqpoint{3.035223in}{4.135367in}}%
\pgfpathlineto{\pgfqpoint{3.085992in}{4.035894in}}%
\pgfpathclose%
\pgfusepath{fill}%
\end{pgfscope}%
\begin{pgfscope}%
\pgfpathrectangle{\pgfqpoint{0.680860in}{0.078740in}}{\pgfqpoint{7.842520in}{7.842520in}}%
\pgfusepath{clip}%
\pgfsetbuttcap%
\pgfsetroundjoin%
\definecolor{currentfill}{rgb}{0.179019,0.433756,0.557430}%
\pgfsetfillcolor{currentfill}%
\pgfsetlinewidth{0.000000pt}%
\definecolor{currentstroke}{rgb}{0.124395,0.578002,0.548287}%
\pgfsetstrokecolor{currentstroke}%
\pgfsetdash{}{0pt}%
\pgfpathmoveto{\pgfqpoint{2.650237in}{3.787098in}}%
\pgfpathlineto{\pgfqpoint{2.734052in}{3.914985in}}%
\pgfpathlineto{\pgfqpoint{2.600661in}{3.915679in}}%
\pgfpathclose%
\pgfusepath{fill}%
\end{pgfscope}%
\begin{pgfscope}%
\pgfpathrectangle{\pgfqpoint{0.680860in}{0.078740in}}{\pgfqpoint{7.842520in}{7.842520in}}%
\pgfusepath{clip}%
\pgfsetbuttcap%
\pgfsetroundjoin%
\definecolor{currentfill}{rgb}{0.160665,0.478540,0.558115}%
\pgfsetfillcolor{currentfill}%
\pgfsetlinewidth{0.000000pt}%
\definecolor{currentstroke}{rgb}{0.123463,0.581687,0.547445}%
\pgfsetstrokecolor{currentstroke}%
\pgfsetdash{}{0pt}%
\pgfpathmoveto{\pgfqpoint{3.085992in}{4.035894in}}%
\pgfpathlineto{\pgfqpoint{3.035223in}{4.135367in}}%
\pgfpathlineto{\pgfqpoint{2.951638in}{4.032294in}}%
\pgfpathclose%
\pgfusepath{fill}%
\end{pgfscope}%
\begin{pgfscope}%
\pgfpathrectangle{\pgfqpoint{0.680860in}{0.078740in}}{\pgfqpoint{7.842520in}{7.842520in}}%
\pgfusepath{clip}%
\pgfsetbuttcap%
\pgfsetroundjoin%
\definecolor{currentfill}{rgb}{0.171176,0.452530,0.557965}%
\pgfsetfillcolor{currentfill}%
\pgfsetlinewidth{0.000000pt}%
\definecolor{currentstroke}{rgb}{0.122606,0.585371,0.546557}%
\pgfsetstrokecolor{currentstroke}%
\pgfsetdash{}{0pt}%
\pgfpathmoveto{\pgfqpoint{2.817845in}{4.030038in}}%
\pgfpathlineto{\pgfqpoint{2.734052in}{3.914985in}}%
\pgfpathlineto{\pgfqpoint{2.867990in}{3.915533in}}%
\pgfpathclose%
\pgfusepath{fill}%
\end{pgfscope}%
\begin{pgfscope}%
\pgfpathrectangle{\pgfqpoint{0.680860in}{0.078740in}}{\pgfqpoint{7.842520in}{7.842520in}}%
\pgfusepath{clip}%
\pgfsetbuttcap%
\pgfsetroundjoin%
\definecolor{currentfill}{rgb}{0.125394,0.574318,0.549086}%
\pgfsetfillcolor{currentfill}%
\pgfsetlinewidth{0.000000pt}%
\definecolor{currentstroke}{rgb}{0.121831,0.589055,0.545623}%
\pgfsetstrokecolor{currentstroke}%
\pgfsetdash{}{0pt}%
\pgfpathmoveto{\pgfqpoint{5.960494in}{4.541766in}}%
\pgfpathlineto{\pgfqpoint{6.033892in}{4.452976in}}%
\pgfpathlineto{\pgfqpoint{5.893304in}{4.440669in}}%
\pgfpathclose%
\pgfusepath{fill}%
\end{pgfscope}%
\begin{pgfscope}%
\pgfpathrectangle{\pgfqpoint{0.680860in}{0.078740in}}{\pgfqpoint{7.842520in}{7.842520in}}%
\pgfusepath{clip}%
\pgfsetbuttcap%
\pgfsetroundjoin%
\definecolor{currentfill}{rgb}{0.166617,0.463708,0.558119}%
\pgfsetfillcolor{currentfill}%
\pgfsetlinewidth{0.000000pt}%
\definecolor{currentstroke}{rgb}{0.121148,0.592739,0.544641}%
\pgfsetstrokecolor{currentstroke}%
\pgfsetdash{}{0pt}%
\pgfpathmoveto{\pgfqpoint{2.951638in}{4.032294in}}%
\pgfpathlineto{\pgfqpoint{2.817845in}{4.030038in}}%
\pgfpathlineto{\pgfqpoint{2.867990in}{3.915533in}}%
\pgfpathclose%
\pgfusepath{fill}%
\end{pgfscope}%
\begin{pgfscope}%
\pgfpathrectangle{\pgfqpoint{0.680860in}{0.078740in}}{\pgfqpoint{7.842520in}{7.842520in}}%
\pgfusepath{clip}%
\pgfsetbuttcap%
\pgfsetroundjoin%
\definecolor{currentfill}{rgb}{0.122606,0.585371,0.546557}%
\pgfsetfillcolor{currentfill}%
\pgfsetlinewidth{0.000000pt}%
\definecolor{currentstroke}{rgb}{0.120565,0.596422,0.543611}%
\pgfsetstrokecolor{currentstroke}%
\pgfsetdash{}{0pt}%
\pgfpathmoveto{\pgfqpoint{5.539015in}{4.494947in}}%
\pgfpathlineto{\pgfqpoint{5.603330in}{4.571642in}}%
\pgfpathlineto{\pgfqpoint{5.678820in}{4.508785in}}%
\pgfpathclose%
\pgfusepath{fill}%
\end{pgfscope}%
\begin{pgfscope}%
\pgfpathrectangle{\pgfqpoint{0.680860in}{0.078740in}}{\pgfqpoint{7.842520in}{7.842520in}}%
\pgfusepath{clip}%
\pgfsetbuttcap%
\pgfsetroundjoin%
\definecolor{currentfill}{rgb}{0.139147,0.533812,0.555298}%
\pgfsetfillcolor{currentfill}%
\pgfsetlinewidth{0.000000pt}%
\definecolor{currentstroke}{rgb}{0.120092,0.600104,0.542530}%
\pgfsetstrokecolor{currentstroke}%
\pgfsetdash{}{0pt}%
\pgfpathmoveto{\pgfqpoint{6.741080in}{4.248182in}}%
\pgfpathlineto{\pgfqpoint{6.672785in}{4.391456in}}%
\pgfpathlineto{\pgfqpoint{6.883614in}{4.255831in}}%
\pgfpathclose%
\pgfusepath{fill}%
\end{pgfscope}%
\begin{pgfscope}%
\pgfpathrectangle{\pgfqpoint{0.680860in}{0.078740in}}{\pgfqpoint{7.842520in}{7.842520in}}%
\pgfusepath{clip}%
\pgfsetbuttcap%
\pgfsetroundjoin%
\definecolor{currentfill}{rgb}{0.210503,0.363727,0.552206}%
\pgfsetfillcolor{currentfill}%
\pgfsetlinewidth{0.000000pt}%
\definecolor{currentstroke}{rgb}{0.119738,0.603785,0.541400}%
\pgfsetstrokecolor{currentstroke}%
\pgfsetdash{}{0pt}%
\pgfpathmoveto{\pgfqpoint{2.082795in}{3.522114in}}%
\pgfpathlineto{\pgfqpoint{2.215559in}{3.512175in}}%
\pgfpathlineto{\pgfqpoint{2.299617in}{3.658412in}}%
\pgfpathclose%
\pgfusepath{fill}%
\end{pgfscope}%
\begin{pgfscope}%
\pgfpathrectangle{\pgfqpoint{0.680860in}{0.078740in}}{\pgfqpoint{7.842520in}{7.842520in}}%
\pgfusepath{clip}%
\pgfsetbuttcap%
\pgfsetroundjoin%
\definecolor{currentfill}{rgb}{0.132444,0.552216,0.553018}%
\pgfsetfillcolor{currentfill}%
\pgfsetlinewidth{0.000000pt}%
\definecolor{currentstroke}{rgb}{0.119512,0.607464,0.540218}%
\pgfsetstrokecolor{currentstroke}%
\pgfsetdash{}{0pt}%
\pgfpathmoveto{\pgfqpoint{3.959342in}{4.405526in}}%
\pgfpathlineto{\pgfqpoint{3.823323in}{4.393347in}}%
\pgfpathlineto{\pgfqpoint{3.740963in}{4.332939in}}%
\pgfpathclose%
\pgfusepath{fill}%
\end{pgfscope}%
\begin{pgfscope}%
\pgfpathrectangle{\pgfqpoint{0.680860in}{0.078740in}}{\pgfqpoint{7.842520in}{7.842520in}}%
\pgfusepath{clip}%
\pgfsetbuttcap%
\pgfsetroundjoin%
\definecolor{currentfill}{rgb}{0.127568,0.566949,0.550556}%
\pgfsetfillcolor{currentfill}%
\pgfsetlinewidth{0.000000pt}%
\definecolor{currentstroke}{rgb}{0.119423,0.611141,0.538982}%
\pgfsetstrokecolor{currentstroke}%
\pgfsetdash{}{0pt}%
\pgfpathmoveto{\pgfqpoint{4.177572in}{4.462498in}}%
\pgfpathlineto{\pgfqpoint{4.041163in}{4.449339in}}%
\pgfpathlineto{\pgfqpoint{4.095983in}{4.419396in}}%
\pgfpathclose%
\pgfusepath{fill}%
\end{pgfscope}%
\begin{pgfscope}%
\pgfpathrectangle{\pgfqpoint{0.680860in}{0.078740in}}{\pgfqpoint{7.842520in}{7.842520in}}%
\pgfusepath{clip}%
\pgfsetbuttcap%
\pgfsetroundjoin%
\definecolor{currentfill}{rgb}{0.121148,0.592739,0.544641}%
\pgfsetfillcolor{currentfill}%
\pgfsetlinewidth{0.000000pt}%
\definecolor{currentstroke}{rgb}{0.119483,0.614817,0.537692}%
\pgfsetstrokecolor{currentstroke}%
\pgfsetdash{}{0pt}%
\pgfpathmoveto{\pgfqpoint{4.889195in}{4.556547in}}%
\pgfpathlineto{\pgfqpoint{4.968019in}{4.541588in}}%
\pgfpathlineto{\pgfqpoint{4.750789in}{4.539623in}}%
\pgfpathclose%
\pgfusepath{fill}%
\end{pgfscope}%
\begin{pgfscope}%
\pgfpathrectangle{\pgfqpoint{0.680860in}{0.078740in}}{\pgfqpoint{7.842520in}{7.842520in}}%
\pgfusepath{clip}%
\pgfsetbuttcap%
\pgfsetroundjoin%
\definecolor{currentfill}{rgb}{0.135066,0.544853,0.554029}%
\pgfsetfillcolor{currentfill}%
\pgfsetlinewidth{0.000000pt}%
\definecolor{currentstroke}{rgb}{0.119699,0.618490,0.536347}%
\pgfsetstrokecolor{currentstroke}%
\pgfsetdash{}{0pt}%
\pgfpathmoveto{\pgfqpoint{6.530116in}{4.378410in}}%
\pgfpathlineto{\pgfqpoint{6.672785in}{4.391456in}}%
\pgfpathlineto{\pgfqpoint{6.741080in}{4.248182in}}%
\pgfpathclose%
\pgfusepath{fill}%
\end{pgfscope}%
\begin{pgfscope}%
\pgfpathrectangle{\pgfqpoint{0.680860in}{0.078740in}}{\pgfqpoint{7.842520in}{7.842520in}}%
\pgfusepath{clip}%
\pgfsetbuttcap%
\pgfsetroundjoin%
\definecolor{currentfill}{rgb}{0.127568,0.566949,0.550556}%
\pgfsetfillcolor{currentfill}%
\pgfsetlinewidth{0.000000pt}%
\definecolor{currentstroke}{rgb}{0.120081,0.622161,0.534946}%
\pgfsetstrokecolor{currentstroke}%
\pgfsetdash{}{0pt}%
\pgfpathmoveto{\pgfqpoint{6.175169in}{4.466975in}}%
\pgfpathlineto{\pgfqpoint{6.317148in}{4.482731in}}%
\pgfpathlineto{\pgfqpoint{6.246871in}{4.357246in}}%
\pgfpathclose%
\pgfusepath{fill}%
\end{pgfscope}%
\begin{pgfscope}%
\pgfpathrectangle{\pgfqpoint{0.680860in}{0.078740in}}{\pgfqpoint{7.842520in}{7.842520in}}%
\pgfusepath{clip}%
\pgfsetbuttcap%
\pgfsetroundjoin%
\definecolor{currentfill}{rgb}{0.124395,0.578002,0.548287}%
\pgfsetfillcolor{currentfill}%
\pgfsetlinewidth{0.000000pt}%
\definecolor{currentstroke}{rgb}{0.120638,0.625828,0.533488}%
\pgfsetstrokecolor{currentstroke}%
\pgfsetdash{}{0pt}%
\pgfpathmoveto{\pgfqpoint{4.177572in}{4.462498in}}%
\pgfpathlineto{\pgfqpoint{4.314611in}{4.477381in}}%
\pgfpathlineto{\pgfqpoint{4.395520in}{4.502548in}}%
\pgfpathclose%
\pgfusepath{fill}%
\end{pgfscope}%
\begin{pgfscope}%
\pgfpathrectangle{\pgfqpoint{0.680860in}{0.078740in}}{\pgfqpoint{7.842520in}{7.842520in}}%
\pgfusepath{clip}%
\pgfsetbuttcap%
\pgfsetroundjoin%
\definecolor{currentfill}{rgb}{0.127568,0.566949,0.550556}%
\pgfsetfillcolor{currentfill}%
\pgfsetlinewidth{0.000000pt}%
\definecolor{currentstroke}{rgb}{0.121380,0.629492,0.531973}%
\pgfsetstrokecolor{currentstroke}%
\pgfsetdash{}{0pt}%
\pgfpathmoveto{\pgfqpoint{4.095983in}{4.419396in}}%
\pgfpathlineto{\pgfqpoint{4.041163in}{4.449339in}}%
\pgfpathlineto{\pgfqpoint{3.959342in}{4.405526in}}%
\pgfpathclose%
\pgfusepath{fill}%
\end{pgfscope}%
\begin{pgfscope}%
\pgfpathrectangle{\pgfqpoint{0.680860in}{0.078740in}}{\pgfqpoint{7.842520in}{7.842520in}}%
\pgfusepath{clip}%
\pgfsetbuttcap%
\pgfsetroundjoin%
\definecolor{currentfill}{rgb}{0.221989,0.339161,0.548752}%
\pgfsetfillcolor{currentfill}%
\pgfsetlinewidth{0.000000pt}%
\definecolor{currentstroke}{rgb}{0.122312,0.633153,0.530398}%
\pgfsetstrokecolor{currentstroke}%
\pgfsetdash{}{0pt}%
\pgfpathmoveto{\pgfqpoint{2.215559in}{3.512175in}}%
\pgfpathlineto{\pgfqpoint{2.082795in}{3.522114in}}%
\pgfpathlineto{\pgfqpoint{1.998634in}{3.368677in}}%
\pgfpathclose%
\pgfusepath{fill}%
\end{pgfscope}%
\begin{pgfscope}%
\pgfpathrectangle{\pgfqpoint{0.680860in}{0.078740in}}{\pgfqpoint{7.842520in}{7.842520in}}%
\pgfusepath{clip}%
\pgfsetbuttcap%
\pgfsetroundjoin%
\definecolor{currentfill}{rgb}{0.121148,0.592739,0.544641}%
\pgfsetfillcolor{currentfill}%
\pgfsetlinewidth{0.000000pt}%
\definecolor{currentstroke}{rgb}{0.123444,0.636809,0.528763}%
\pgfsetstrokecolor{currentstroke}%
\pgfsetdash{}{0pt}%
\pgfpathmoveto{\pgfqpoint{5.539015in}{4.494947in}}%
\pgfpathlineto{\pgfqpoint{5.463019in}{4.553572in}}%
\pgfpathlineto{\pgfqpoint{5.603330in}{4.571642in}}%
\pgfpathclose%
\pgfusepath{fill}%
\end{pgfscope}%
\begin{pgfscope}%
\pgfpathrectangle{\pgfqpoint{0.680860in}{0.078740in}}{\pgfqpoint{7.842520in}{7.842520in}}%
\pgfusepath{clip}%
\pgfsetbuttcap%
\pgfsetroundjoin%
\definecolor{currentfill}{rgb}{0.122606,0.585371,0.546557}%
\pgfsetfillcolor{currentfill}%
\pgfsetlinewidth{0.000000pt}%
\definecolor{currentstroke}{rgb}{0.124780,0.640461,0.527068}%
\pgfsetstrokecolor{currentstroke}%
\pgfsetdash{}{0pt}%
\pgfpathmoveto{\pgfqpoint{5.893304in}{4.440669in}}%
\pgfpathlineto{\pgfqpoint{5.819308in}{4.524369in}}%
\pgfpathlineto{\pgfqpoint{5.960494in}{4.541766in}}%
\pgfpathclose%
\pgfusepath{fill}%
\end{pgfscope}%
\begin{pgfscope}%
\pgfpathrectangle{\pgfqpoint{0.680860in}{0.078740in}}{\pgfqpoint{7.842520in}{7.842520in}}%
\pgfusepath{clip}%
\pgfsetbuttcap%
\pgfsetroundjoin%
\definecolor{currentfill}{rgb}{0.137770,0.537492,0.554906}%
\pgfsetfillcolor{currentfill}%
\pgfsetlinewidth{0.000000pt}%
\definecolor{currentstroke}{rgb}{0.126326,0.644107,0.525311}%
\pgfsetstrokecolor{currentstroke}%
\pgfsetdash{}{0pt}%
\pgfpathmoveto{\pgfqpoint{3.605349in}{4.322255in}}%
\pgfpathlineto{\pgfqpoint{3.522549in}{4.246142in}}%
\pgfpathlineto{\pgfqpoint{3.740963in}{4.332939in}}%
\pgfpathclose%
\pgfusepath{fill}%
\end{pgfscope}%
\begin{pgfscope}%
\pgfpathrectangle{\pgfqpoint{0.680860in}{0.078740in}}{\pgfqpoint{7.842520in}{7.842520in}}%
\pgfusepath{clip}%
\pgfsetbuttcap%
\pgfsetroundjoin%
\definecolor{currentfill}{rgb}{0.122606,0.585371,0.546557}%
\pgfsetfillcolor{currentfill}%
\pgfsetlinewidth{0.000000pt}%
\definecolor{currentstroke}{rgb}{0.128087,0.647749,0.523491}%
\pgfsetstrokecolor{currentstroke}%
\pgfsetdash{}{0pt}%
\pgfpathmoveto{\pgfqpoint{4.532931in}{4.517865in}}%
\pgfpathlineto{\pgfqpoint{4.395520in}{4.502548in}}%
\pgfpathlineto{\pgfqpoint{4.314611in}{4.477381in}}%
\pgfpathclose%
\pgfusepath{fill}%
\end{pgfscope}%
\begin{pgfscope}%
\pgfpathrectangle{\pgfqpoint{0.680860in}{0.078740in}}{\pgfqpoint{7.842520in}{7.842520in}}%
\pgfusepath{clip}%
\pgfsetbuttcap%
\pgfsetroundjoin%
\definecolor{currentfill}{rgb}{0.128729,0.563265,0.551229}%
\pgfsetfillcolor{currentfill}%
\pgfsetlinewidth{0.000000pt}%
\definecolor{currentstroke}{rgb}{0.130067,0.651384,0.521608}%
\pgfsetstrokecolor{currentstroke}%
\pgfsetdash{}{0pt}%
\pgfpathmoveto{\pgfqpoint{6.388150in}{4.367027in}}%
\pgfpathlineto{\pgfqpoint{6.317148in}{4.482731in}}%
\pgfpathlineto{\pgfqpoint{6.530116in}{4.378410in}}%
\pgfpathclose%
\pgfusepath{fill}%
\end{pgfscope}%
\begin{pgfscope}%
\pgfpathrectangle{\pgfqpoint{0.680860in}{0.078740in}}{\pgfqpoint{7.842520in}{7.842520in}}%
\pgfusepath{clip}%
\pgfsetbuttcap%
\pgfsetroundjoin%
\definecolor{currentfill}{rgb}{0.120565,0.596422,0.543611}%
\pgfsetfillcolor{currentfill}%
\pgfsetlinewidth{0.000000pt}%
\definecolor{currentstroke}{rgb}{0.132268,0.655014,0.519661}%
\pgfsetstrokecolor{currentstroke}%
\pgfsetdash{}{0pt}%
\pgfpathmoveto{\pgfqpoint{5.106724in}{4.557637in}}%
\pgfpathlineto{\pgfqpoint{5.246102in}{4.575513in}}%
\pgfpathlineto{\pgfqpoint{5.323395in}{4.537338in}}%
\pgfpathclose%
\pgfusepath{fill}%
\end{pgfscope}%
\begin{pgfscope}%
\pgfpathrectangle{\pgfqpoint{0.680860in}{0.078740in}}{\pgfqpoint{7.842520in}{7.842520in}}%
\pgfusepath{clip}%
\pgfsetbuttcap%
\pgfsetroundjoin%
\definecolor{currentfill}{rgb}{0.159194,0.482237,0.558073}%
\pgfsetfillcolor{currentfill}%
\pgfsetlinewidth{0.000000pt}%
\definecolor{currentstroke}{rgb}{0.134692,0.658636,0.517649}%
\pgfsetstrokecolor{currentstroke}%
\pgfsetdash{}{0pt}%
\pgfpathmoveto{\pgfqpoint{7.583875in}{3.898413in}}%
\pgfpathlineto{\pgfqpoint{7.523042in}{4.106176in}}%
\pgfpathlineto{\pgfqpoint{7.668081in}{4.113125in}}%
\pgfpathclose%
\pgfusepath{fill}%
\end{pgfscope}%
\begin{pgfscope}%
\pgfpathrectangle{\pgfqpoint{0.680860in}{0.078740in}}{\pgfqpoint{7.842520in}{7.842520in}}%
\pgfusepath{clip}%
\pgfsetbuttcap%
\pgfsetroundjoin%
\definecolor{currentfill}{rgb}{0.144759,0.519093,0.556572}%
\pgfsetfillcolor{currentfill}%
\pgfsetlinewidth{0.000000pt}%
\definecolor{currentstroke}{rgb}{0.137339,0.662252,0.515571}%
\pgfsetstrokecolor{currentstroke}%
\pgfsetdash{}{0pt}%
\pgfpathmoveto{\pgfqpoint{3.522549in}{4.246142in}}%
\pgfpathlineto{\pgfqpoint{3.387352in}{4.237421in}}%
\pgfpathlineto{\pgfqpoint{3.304198in}{4.146617in}}%
\pgfpathclose%
\pgfusepath{fill}%
\end{pgfscope}%
\begin{pgfscope}%
\pgfpathrectangle{\pgfqpoint{0.680860in}{0.078740in}}{\pgfqpoint{7.842520in}{7.842520in}}%
\pgfusepath{clip}%
\pgfsetbuttcap%
\pgfsetroundjoin%
\definecolor{currentfill}{rgb}{0.174274,0.445044,0.557792}%
\pgfsetfillcolor{currentfill}%
\pgfsetlinewidth{0.000000pt}%
\definecolor{currentstroke}{rgb}{0.140210,0.665859,0.513427}%
\pgfsetstrokecolor{currentstroke}%
\pgfsetdash{}{0pt}%
\pgfpathmoveto{\pgfqpoint{2.867990in}{3.915533in}}%
\pgfpathlineto{\pgfqpoint{2.734052in}{3.914985in}}%
\pgfpathlineto{\pgfqpoint{2.650237in}{3.787098in}}%
\pgfpathclose%
\pgfusepath{fill}%
\end{pgfscope}%
\begin{pgfscope}%
\pgfpathrectangle{\pgfqpoint{0.680860in}{0.078740in}}{\pgfqpoint{7.842520in}{7.842520in}}%
\pgfusepath{clip}%
\pgfsetbuttcap%
\pgfsetroundjoin%
\definecolor{currentfill}{rgb}{0.153364,0.497000,0.557724}%
\pgfsetfillcolor{currentfill}%
\pgfsetlinewidth{0.000000pt}%
\definecolor{currentstroke}{rgb}{0.143303,0.669459,0.511215}%
\pgfsetstrokecolor{currentstroke}%
\pgfsetdash{}{0pt}%
\pgfpathmoveto{\pgfqpoint{3.304198in}{4.146617in}}%
\pgfpathlineto{\pgfqpoint{3.169424in}{4.140274in}}%
\pgfpathlineto{\pgfqpoint{3.085992in}{4.035894in}}%
\pgfpathclose%
\pgfusepath{fill}%
\end{pgfscope}%
\begin{pgfscope}%
\pgfpathrectangle{\pgfqpoint{0.680860in}{0.078740in}}{\pgfqpoint{7.842520in}{7.842520in}}%
\pgfusepath{clip}%
\pgfsetbuttcap%
\pgfsetroundjoin%
\definecolor{currentfill}{rgb}{0.163625,0.471133,0.558148}%
\pgfsetfillcolor{currentfill}%
\pgfsetlinewidth{0.000000pt}%
\definecolor{currentstroke}{rgb}{0.146616,0.673050,0.508936}%
\pgfsetstrokecolor{currentstroke}%
\pgfsetdash{}{0pt}%
\pgfpathmoveto{\pgfqpoint{2.867990in}{3.915533in}}%
\pgfpathlineto{\pgfqpoint{3.085992in}{4.035894in}}%
\pgfpathlineto{\pgfqpoint{2.951638in}{4.032294in}}%
\pgfpathclose%
\pgfusepath{fill}%
\end{pgfscope}%
\begin{pgfscope}%
\pgfpathrectangle{\pgfqpoint{0.680860in}{0.078740in}}{\pgfqpoint{7.842520in}{7.842520in}}%
\pgfusepath{clip}%
\pgfsetbuttcap%
\pgfsetroundjoin%
\definecolor{currentfill}{rgb}{0.120092,0.600104,0.542530}%
\pgfsetfillcolor{currentfill}%
\pgfsetlinewidth{0.000000pt}%
\definecolor{currentstroke}{rgb}{0.150148,0.676631,0.506589}%
\pgfsetstrokecolor{currentstroke}%
\pgfsetdash{}{0pt}%
\pgfpathmoveto{\pgfqpoint{5.106724in}{4.557637in}}%
\pgfpathlineto{\pgfqpoint{4.968019in}{4.541588in}}%
\pgfpathlineto{\pgfqpoint{5.028273in}{4.575333in}}%
\pgfpathclose%
\pgfusepath{fill}%
\end{pgfscope}%
\begin{pgfscope}%
\pgfpathrectangle{\pgfqpoint{0.680860in}{0.078740in}}{\pgfqpoint{7.842520in}{7.842520in}}%
\pgfusepath{clip}%
\pgfsetbuttcap%
\pgfsetroundjoin%
\definecolor{currentfill}{rgb}{0.143343,0.522773,0.556295}%
\pgfsetfillcolor{currentfill}%
\pgfsetlinewidth{0.000000pt}%
\definecolor{currentstroke}{rgb}{0.153894,0.680203,0.504172}%
\pgfsetstrokecolor{currentstroke}%
\pgfsetdash{}{0pt}%
\pgfpathmoveto{\pgfqpoint{7.170785in}{4.275781in}}%
\pgfpathlineto{\pgfqpoint{7.235076in}{4.096611in}}%
\pgfpathlineto{\pgfqpoint{7.026844in}{4.265011in}}%
\pgfpathclose%
\pgfusepath{fill}%
\end{pgfscope}%
\begin{pgfscope}%
\pgfpathrectangle{\pgfqpoint{0.680860in}{0.078740in}}{\pgfqpoint{7.842520in}{7.842520in}}%
\pgfusepath{clip}%
\pgfsetbuttcap%
\pgfsetroundjoin%
\definecolor{currentfill}{rgb}{0.208623,0.367752,0.552675}%
\pgfsetfillcolor{currentfill}%
\pgfsetlinewidth{0.000000pt}%
\definecolor{currentstroke}{rgb}{0.157851,0.683765,0.501686}%
\pgfsetstrokecolor{currentstroke}%
\pgfsetdash{}{0pt}%
\pgfpathmoveto{\pgfqpoint{2.348839in}{3.503168in}}%
\pgfpathlineto{\pgfqpoint{2.299617in}{3.658412in}}%
\pgfpathlineto{\pgfqpoint{2.215559in}{3.512175in}}%
\pgfpathclose%
\pgfusepath{fill}%
\end{pgfscope}%
\begin{pgfscope}%
\pgfpathrectangle{\pgfqpoint{0.680860in}{0.078740in}}{\pgfqpoint{7.842520in}{7.842520in}}%
\pgfusepath{clip}%
\pgfsetbuttcap%
\pgfsetroundjoin%
\definecolor{currentfill}{rgb}{0.120565,0.596422,0.543611}%
\pgfsetfillcolor{currentfill}%
\pgfsetlinewidth{0.000000pt}%
\definecolor{currentstroke}{rgb}{0.162016,0.687316,0.499129}%
\pgfsetstrokecolor{currentstroke}%
\pgfsetdash{}{0pt}%
\pgfpathmoveto{\pgfqpoint{5.603330in}{4.571642in}}%
\pgfpathlineto{\pgfqpoint{5.819308in}{4.524369in}}%
\pgfpathlineto{\pgfqpoint{5.678820in}{4.508785in}}%
\pgfpathclose%
\pgfusepath{fill}%
\end{pgfscope}%
\begin{pgfscope}%
\pgfpathrectangle{\pgfqpoint{0.680860in}{0.078740in}}{\pgfqpoint{7.842520in}{7.842520in}}%
\pgfusepath{clip}%
\pgfsetbuttcap%
\pgfsetroundjoin%
\definecolor{currentfill}{rgb}{0.203063,0.379716,0.553925}%
\pgfsetfillcolor{currentfill}%
\pgfsetlinewidth{0.000000pt}%
\definecolor{currentstroke}{rgb}{0.166383,0.690856,0.496502}%
\pgfsetstrokecolor{currentstroke}%
\pgfsetdash{}{0pt}%
\pgfpathmoveto{\pgfqpoint{2.432758in}{3.652139in}}%
\pgfpathlineto{\pgfqpoint{2.299617in}{3.658412in}}%
\pgfpathlineto{\pgfqpoint{2.348839in}{3.503168in}}%
\pgfpathclose%
\pgfusepath{fill}%
\end{pgfscope}%
\begin{pgfscope}%
\pgfpathrectangle{\pgfqpoint{0.680860in}{0.078740in}}{\pgfqpoint{7.842520in}{7.842520in}}%
\pgfusepath{clip}%
\pgfsetbuttcap%
\pgfsetroundjoin%
\definecolor{currentfill}{rgb}{0.147607,0.511733,0.557049}%
\pgfsetfillcolor{currentfill}%
\pgfsetlinewidth{0.000000pt}%
\definecolor{currentstroke}{rgb}{0.170948,0.694384,0.493803}%
\pgfsetstrokecolor{currentstroke}%
\pgfsetdash{}{0pt}%
\pgfpathmoveto{\pgfqpoint{7.235076in}{4.096611in}}%
\pgfpathlineto{\pgfqpoint{7.315451in}{4.288200in}}%
\pgfpathlineto{\pgfqpoint{7.378712in}{4.100689in}}%
\pgfpathclose%
\pgfusepath{fill}%
\end{pgfscope}%
\begin{pgfscope}%
\pgfpathrectangle{\pgfqpoint{0.680860in}{0.078740in}}{\pgfqpoint{7.842520in}{7.842520in}}%
\pgfusepath{clip}%
\pgfsetbuttcap%
\pgfsetroundjoin%
\definecolor{currentfill}{rgb}{0.120565,0.596422,0.543611}%
\pgfsetfillcolor{currentfill}%
\pgfsetlinewidth{0.000000pt}%
\definecolor{currentstroke}{rgb}{0.175707,0.697900,0.491033}%
\pgfsetstrokecolor{currentstroke}%
\pgfsetdash{}{0pt}%
\pgfpathmoveto{\pgfqpoint{4.670994in}{4.534985in}}%
\pgfpathlineto{\pgfqpoint{4.750789in}{4.539623in}}%
\pgfpathlineto{\pgfqpoint{4.532931in}{4.517865in}}%
\pgfpathclose%
\pgfusepath{fill}%
\end{pgfscope}%
\begin{pgfscope}%
\pgfpathrectangle{\pgfqpoint{0.680860in}{0.078740in}}{\pgfqpoint{7.842520in}{7.842520in}}%
\pgfusepath{clip}%
\pgfsetbuttcap%
\pgfsetroundjoin%
\definecolor{currentfill}{rgb}{0.192357,0.403199,0.555836}%
\pgfsetfillcolor{currentfill}%
\pgfsetlinewidth{0.000000pt}%
\definecolor{currentstroke}{rgb}{0.180653,0.701402,0.488189}%
\pgfsetstrokecolor{currentstroke}%
\pgfsetdash{}{0pt}%
\pgfpathmoveto{\pgfqpoint{2.432758in}{3.652139in}}%
\pgfpathlineto{\pgfqpoint{2.566427in}{3.646923in}}%
\pgfpathlineto{\pgfqpoint{2.516704in}{3.789858in}}%
\pgfpathclose%
\pgfusepath{fill}%
\end{pgfscope}%
\begin{pgfscope}%
\pgfpathrectangle{\pgfqpoint{0.680860in}{0.078740in}}{\pgfqpoint{7.842520in}{7.842520in}}%
\pgfusepath{clip}%
\pgfsetbuttcap%
\pgfsetroundjoin%
\definecolor{currentfill}{rgb}{0.223925,0.334994,0.548053}%
\pgfsetfillcolor{currentfill}%
\pgfsetlinewidth{0.000000pt}%
\definecolor{currentstroke}{rgb}{0.185783,0.704891,0.485273}%
\pgfsetstrokecolor{currentstroke}%
\pgfsetdash{}{0pt}%
\pgfpathmoveto{\pgfqpoint{2.215559in}{3.512175in}}%
\pgfpathlineto{\pgfqpoint{1.998634in}{3.368677in}}%
\pgfpathlineto{\pgfqpoint{2.131544in}{3.355773in}}%
\pgfpathclose%
\pgfusepath{fill}%
\end{pgfscope}%
\begin{pgfscope}%
\pgfpathrectangle{\pgfqpoint{0.680860in}{0.078740in}}{\pgfqpoint{7.842520in}{7.842520in}}%
\pgfusepath{clip}%
\pgfsetbuttcap%
\pgfsetroundjoin%
\definecolor{currentfill}{rgb}{0.187231,0.414746,0.556547}%
\pgfsetfillcolor{currentfill}%
\pgfsetlinewidth{0.000000pt}%
\definecolor{currentstroke}{rgb}{0.191090,0.708366,0.482284}%
\pgfsetstrokecolor{currentstroke}%
\pgfsetdash{}{0pt}%
\pgfpathmoveto{\pgfqpoint{2.516704in}{3.789858in}}%
\pgfpathlineto{\pgfqpoint{2.566427in}{3.646923in}}%
\pgfpathlineto{\pgfqpoint{2.650237in}{3.787098in}}%
\pgfpathclose%
\pgfusepath{fill}%
\end{pgfscope}%
\begin{pgfscope}%
\pgfpathrectangle{\pgfqpoint{0.680860in}{0.078740in}}{\pgfqpoint{7.842520in}{7.842520in}}%
\pgfusepath{clip}%
\pgfsetbuttcap%
\pgfsetroundjoin%
\definecolor{currentfill}{rgb}{0.124395,0.578002,0.548287}%
\pgfsetfillcolor{currentfill}%
\pgfsetlinewidth{0.000000pt}%
\definecolor{currentstroke}{rgb}{0.196571,0.711827,0.479221}%
\pgfsetstrokecolor{currentstroke}%
\pgfsetdash{}{0pt}%
\pgfpathmoveto{\pgfqpoint{4.177572in}{4.462498in}}%
\pgfpathlineto{\pgfqpoint{4.095983in}{4.419396in}}%
\pgfpathlineto{\pgfqpoint{4.314611in}{4.477381in}}%
\pgfpathclose%
\pgfusepath{fill}%
\end{pgfscope}%
\begin{pgfscope}%
\pgfpathrectangle{\pgfqpoint{0.680860in}{0.078740in}}{\pgfqpoint{7.842520in}{7.842520in}}%
\pgfusepath{clip}%
\pgfsetbuttcap%
\pgfsetroundjoin%
\definecolor{currentfill}{rgb}{0.119738,0.603785,0.541400}%
\pgfsetfillcolor{currentfill}%
\pgfsetlinewidth{0.000000pt}%
\definecolor{currentstroke}{rgb}{0.202219,0.715272,0.476084}%
\pgfsetstrokecolor{currentstroke}%
\pgfsetdash{}{0pt}%
\pgfpathmoveto{\pgfqpoint{5.028273in}{4.575333in}}%
\pgfpathlineto{\pgfqpoint{4.968019in}{4.541588in}}%
\pgfpathlineto{\pgfqpoint{4.889195in}{4.556547in}}%
\pgfpathclose%
\pgfusepath{fill}%
\end{pgfscope}%
\begin{pgfscope}%
\pgfpathrectangle{\pgfqpoint{0.680860in}{0.078740in}}{\pgfqpoint{7.842520in}{7.842520in}}%
\pgfusepath{clip}%
\pgfsetbuttcap%
\pgfsetroundjoin%
\definecolor{currentfill}{rgb}{0.129933,0.559582,0.551864}%
\pgfsetfillcolor{currentfill}%
\pgfsetlinewidth{0.000000pt}%
\definecolor{currentstroke}{rgb}{0.208030,0.718701,0.472873}%
\pgfsetstrokecolor{currentstroke}%
\pgfsetdash{}{0pt}%
\pgfpathmoveto{\pgfqpoint{3.740963in}{4.332939in}}%
\pgfpathlineto{\pgfqpoint{3.877187in}{4.345262in}}%
\pgfpathlineto{\pgfqpoint{3.959342in}{4.405526in}}%
\pgfpathclose%
\pgfusepath{fill}%
\end{pgfscope}%
\begin{pgfscope}%
\pgfpathrectangle{\pgfqpoint{0.680860in}{0.078740in}}{\pgfqpoint{7.842520in}{7.842520in}}%
\pgfusepath{clip}%
\pgfsetbuttcap%
\pgfsetroundjoin%
\definecolor{currentfill}{rgb}{0.119738,0.603785,0.541400}%
\pgfsetfillcolor{currentfill}%
\pgfsetlinewidth{0.000000pt}%
\definecolor{currentstroke}{rgb}{0.214000,0.722114,0.469588}%
\pgfsetstrokecolor{currentstroke}%
\pgfsetdash{}{0pt}%
\pgfpathmoveto{\pgfqpoint{5.386170in}{4.595288in}}%
\pgfpathlineto{\pgfqpoint{5.463019in}{4.553572in}}%
\pgfpathlineto{\pgfqpoint{5.323395in}{4.537338in}}%
\pgfpathclose%
\pgfusepath{fill}%
\end{pgfscope}%
\begin{pgfscope}%
\pgfpathrectangle{\pgfqpoint{0.680860in}{0.078740in}}{\pgfqpoint{7.842520in}{7.842520in}}%
\pgfusepath{clip}%
\pgfsetbuttcap%
\pgfsetroundjoin%
\definecolor{currentfill}{rgb}{0.235526,0.309527,0.542944}%
\pgfsetfillcolor{currentfill}%
\pgfsetlinewidth{0.000000pt}%
\definecolor{currentstroke}{rgb}{0.220124,0.725509,0.466226}%
\pgfsetstrokecolor{currentstroke}%
\pgfsetdash{}{0pt}%
\pgfpathmoveto{\pgfqpoint{2.131544in}{3.355773in}}%
\pgfpathlineto{\pgfqpoint{1.998634in}{3.368677in}}%
\pgfpathlineto{\pgfqpoint{1.914518in}{3.206196in}}%
\pgfpathclose%
\pgfusepath{fill}%
\end{pgfscope}%
\begin{pgfscope}%
\pgfpathrectangle{\pgfqpoint{0.680860in}{0.078740in}}{\pgfqpoint{7.842520in}{7.842520in}}%
\pgfusepath{clip}%
\pgfsetbuttcap%
\pgfsetroundjoin%
\definecolor{currentfill}{rgb}{0.121831,0.589055,0.545623}%
\pgfsetfillcolor{currentfill}%
\pgfsetlinewidth{0.000000pt}%
\definecolor{currentstroke}{rgb}{0.226397,0.728888,0.462789}%
\pgfsetstrokecolor{currentstroke}%
\pgfsetdash{}{0pt}%
\pgfpathmoveto{\pgfqpoint{6.033892in}{4.452976in}}%
\pgfpathlineto{\pgfqpoint{6.102393in}{4.561046in}}%
\pgfpathlineto{\pgfqpoint{6.175169in}{4.466975in}}%
\pgfpathclose%
\pgfusepath{fill}%
\end{pgfscope}%
\begin{pgfscope}%
\pgfpathrectangle{\pgfqpoint{0.680860in}{0.078740in}}{\pgfqpoint{7.842520in}{7.842520in}}%
\pgfusepath{clip}%
\pgfsetbuttcap%
\pgfsetroundjoin%
\definecolor{currentfill}{rgb}{0.133743,0.548535,0.553541}%
\pgfsetfillcolor{currentfill}%
\pgfsetlinewidth{0.000000pt}%
\definecolor{currentstroke}{rgb}{0.232815,0.732247,0.459277}%
\pgfsetstrokecolor{currentstroke}%
\pgfsetdash{}{0pt}%
\pgfpathmoveto{\pgfqpoint{7.026844in}{4.265011in}}%
\pgfpathlineto{\pgfqpoint{6.883614in}{4.255831in}}%
\pgfpathlineto{\pgfqpoint{6.816173in}{4.406230in}}%
\pgfpathclose%
\pgfusepath{fill}%
\end{pgfscope}%
\begin{pgfscope}%
\pgfpathrectangle{\pgfqpoint{0.680860in}{0.078740in}}{\pgfqpoint{7.842520in}{7.842520in}}%
\pgfusepath{clip}%
\pgfsetbuttcap%
\pgfsetroundjoin%
\definecolor{currentfill}{rgb}{0.136408,0.541173,0.554483}%
\pgfsetfillcolor{currentfill}%
\pgfsetlinewidth{0.000000pt}%
\definecolor{currentstroke}{rgb}{0.239374,0.735588,0.455688}%
\pgfsetstrokecolor{currentstroke}%
\pgfsetdash{}{0pt}%
\pgfpathmoveto{\pgfqpoint{3.522549in}{4.246142in}}%
\pgfpathlineto{\pgfqpoint{3.658344in}{4.256435in}}%
\pgfpathlineto{\pgfqpoint{3.740963in}{4.332939in}}%
\pgfpathclose%
\pgfusepath{fill}%
\end{pgfscope}%
\begin{pgfscope}%
\pgfpathrectangle{\pgfqpoint{0.680860in}{0.078740in}}{\pgfqpoint{7.842520in}{7.842520in}}%
\pgfusepath{clip}%
\pgfsetbuttcap%
\pgfsetroundjoin%
\definecolor{currentfill}{rgb}{0.127568,0.566949,0.550556}%
\pgfsetfillcolor{currentfill}%
\pgfsetlinewidth{0.000000pt}%
\definecolor{currentstroke}{rgb}{0.246070,0.738910,0.452024}%
\pgfsetstrokecolor{currentstroke}%
\pgfsetdash{}{0pt}%
\pgfpathmoveto{\pgfqpoint{3.877187in}{4.345262in}}%
\pgfpathlineto{\pgfqpoint{4.095983in}{4.419396in}}%
\pgfpathlineto{\pgfqpoint{3.959342in}{4.405526in}}%
\pgfpathclose%
\pgfusepath{fill}%
\end{pgfscope}%
\begin{pgfscope}%
\pgfpathrectangle{\pgfqpoint{0.680860in}{0.078740in}}{\pgfqpoint{7.842520in}{7.842520in}}%
\pgfusepath{clip}%
\pgfsetbuttcap%
\pgfsetroundjoin%
\definecolor{currentfill}{rgb}{0.216210,0.351535,0.550627}%
\pgfsetfillcolor{currentfill}%
\pgfsetlinewidth{0.000000pt}%
\definecolor{currentstroke}{rgb}{0.252899,0.742211,0.448284}%
\pgfsetstrokecolor{currentstroke}%
\pgfsetdash{}{0pt}%
\pgfpathmoveto{\pgfqpoint{2.348839in}{3.503168in}}%
\pgfpathlineto{\pgfqpoint{2.215559in}{3.512175in}}%
\pgfpathlineto{\pgfqpoint{2.131544in}{3.355773in}}%
\pgfpathclose%
\pgfusepath{fill}%
\end{pgfscope}%
\begin{pgfscope}%
\pgfpathrectangle{\pgfqpoint{0.680860in}{0.078740in}}{\pgfqpoint{7.842520in}{7.842520in}}%
\pgfusepath{clip}%
\pgfsetbuttcap%
\pgfsetroundjoin%
\definecolor{currentfill}{rgb}{0.119423,0.611141,0.538982}%
\pgfsetfillcolor{currentfill}%
\pgfsetlinewidth{0.000000pt}%
\definecolor{currentstroke}{rgb}{0.259857,0.745492,0.444467}%
\pgfsetstrokecolor{currentstroke}%
\pgfsetdash{}{0pt}%
\pgfpathmoveto{\pgfqpoint{5.246102in}{4.575513in}}%
\pgfpathlineto{\pgfqpoint{5.386170in}{4.595288in}}%
\pgfpathlineto{\pgfqpoint{5.323395in}{4.537338in}}%
\pgfpathclose%
\pgfusepath{fill}%
\end{pgfscope}%
\begin{pgfscope}%
\pgfpathrectangle{\pgfqpoint{0.680860in}{0.078740in}}{\pgfqpoint{7.842520in}{7.842520in}}%
\pgfusepath{clip}%
\pgfsetbuttcap%
\pgfsetroundjoin%
\definecolor{currentfill}{rgb}{0.175841,0.441290,0.557685}%
\pgfsetfillcolor{currentfill}%
\pgfsetlinewidth{0.000000pt}%
\definecolor{currentstroke}{rgb}{0.266941,0.748751,0.440573}%
\pgfsetstrokecolor{currentstroke}%
\pgfsetdash{}{0pt}%
\pgfpathmoveto{\pgfqpoint{2.650237in}{3.787098in}}%
\pgfpathlineto{\pgfqpoint{2.784313in}{3.785515in}}%
\pgfpathlineto{\pgfqpoint{2.867990in}{3.915533in}}%
\pgfpathclose%
\pgfusepath{fill}%
\end{pgfscope}%
\begin{pgfscope}%
\pgfpathrectangle{\pgfqpoint{0.680860in}{0.078740in}}{\pgfqpoint{7.842520in}{7.842520in}}%
\pgfusepath{clip}%
\pgfsetbuttcap%
\pgfsetroundjoin%
\definecolor{currentfill}{rgb}{0.143343,0.522773,0.556295}%
\pgfsetfillcolor{currentfill}%
\pgfsetlinewidth{0.000000pt}%
\definecolor{currentstroke}{rgb}{0.274149,0.751988,0.436601}%
\pgfsetstrokecolor{currentstroke}%
\pgfsetdash{}{0pt}%
\pgfpathmoveto{\pgfqpoint{3.304198in}{4.146617in}}%
\pgfpathlineto{\pgfqpoint{3.439558in}{4.154450in}}%
\pgfpathlineto{\pgfqpoint{3.522549in}{4.246142in}}%
\pgfpathclose%
\pgfusepath{fill}%
\end{pgfscope}%
\begin{pgfscope}%
\pgfpathrectangle{\pgfqpoint{0.680860in}{0.078740in}}{\pgfqpoint{7.842520in}{7.842520in}}%
\pgfusepath{clip}%
\pgfsetbuttcap%
\pgfsetroundjoin%
\definecolor{currentfill}{rgb}{0.129933,0.559582,0.551864}%
\pgfsetfillcolor{currentfill}%
\pgfsetlinewidth{0.000000pt}%
\definecolor{currentstroke}{rgb}{0.281477,0.755203,0.432552}%
\pgfsetstrokecolor{currentstroke}%
\pgfsetdash{}{0pt}%
\pgfpathmoveto{\pgfqpoint{6.883614in}{4.255831in}}%
\pgfpathlineto{\pgfqpoint{6.672785in}{4.391456in}}%
\pgfpathlineto{\pgfqpoint{6.816173in}{4.406230in}}%
\pgfpathclose%
\pgfusepath{fill}%
\end{pgfscope}%
\begin{pgfscope}%
\pgfpathrectangle{\pgfqpoint{0.680860in}{0.078740in}}{\pgfqpoint{7.842520in}{7.842520in}}%
\pgfusepath{clip}%
\pgfsetbuttcap%
\pgfsetroundjoin%
\definecolor{currentfill}{rgb}{0.121148,0.592739,0.544641}%
\pgfsetfillcolor{currentfill}%
\pgfsetlinewidth{0.000000pt}%
\definecolor{currentstroke}{rgb}{0.288921,0.758394,0.428426}%
\pgfsetstrokecolor{currentstroke}%
\pgfsetdash{}{0pt}%
\pgfpathmoveto{\pgfqpoint{4.314611in}{4.477381in}}%
\pgfpathlineto{\pgfqpoint{4.452294in}{4.494054in}}%
\pgfpathlineto{\pgfqpoint{4.532931in}{4.517865in}}%
\pgfpathclose%
\pgfusepath{fill}%
\end{pgfscope}%
\begin{pgfscope}%
\pgfpathrectangle{\pgfqpoint{0.680860in}{0.078740in}}{\pgfqpoint{7.842520in}{7.842520in}}%
\pgfusepath{clip}%
\pgfsetbuttcap%
\pgfsetroundjoin%
\definecolor{currentfill}{rgb}{0.163625,0.471133,0.558148}%
\pgfsetfillcolor{currentfill}%
\pgfsetlinewidth{0.000000pt}%
\definecolor{currentstroke}{rgb}{0.296479,0.761561,0.424223}%
\pgfsetstrokecolor{currentstroke}%
\pgfsetdash{}{0pt}%
\pgfpathmoveto{\pgfqpoint{2.867990in}{3.915533in}}%
\pgfpathlineto{\pgfqpoint{3.002486in}{3.917372in}}%
\pgfpathlineto{\pgfqpoint{3.085992in}{4.035894in}}%
\pgfpathclose%
\pgfusepath{fill}%
\end{pgfscope}%
\begin{pgfscope}%
\pgfpathrectangle{\pgfqpoint{0.680860in}{0.078740in}}{\pgfqpoint{7.842520in}{7.842520in}}%
\pgfusepath{clip}%
\pgfsetbuttcap%
\pgfsetroundjoin%
\definecolor{currentfill}{rgb}{0.153364,0.497000,0.557724}%
\pgfsetfillcolor{currentfill}%
\pgfsetlinewidth{0.000000pt}%
\definecolor{currentstroke}{rgb}{0.304148,0.764704,0.419943}%
\pgfsetstrokecolor{currentstroke}%
\pgfsetdash{}{0pt}%
\pgfpathmoveto{\pgfqpoint{3.085992in}{4.035894in}}%
\pgfpathlineto{\pgfqpoint{3.220917in}{4.040891in}}%
\pgfpathlineto{\pgfqpoint{3.304198in}{4.146617in}}%
\pgfpathclose%
\pgfusepath{fill}%
\end{pgfscope}%
\begin{pgfscope}%
\pgfpathrectangle{\pgfqpoint{0.680860in}{0.078740in}}{\pgfqpoint{7.842520in}{7.842520in}}%
\pgfusepath{clip}%
\pgfsetbuttcap%
\pgfsetroundjoin%
\definecolor{currentfill}{rgb}{0.119512,0.607464,0.540218}%
\pgfsetfillcolor{currentfill}%
\pgfsetlinewidth{0.000000pt}%
\definecolor{currentstroke}{rgb}{0.311925,0.767822,0.415586}%
\pgfsetstrokecolor{currentstroke}%
\pgfsetdash{}{0pt}%
\pgfpathmoveto{\pgfqpoint{4.889195in}{4.556547in}}%
\pgfpathlineto{\pgfqpoint{4.750789in}{4.539623in}}%
\pgfpathlineto{\pgfqpoint{4.809723in}{4.553976in}}%
\pgfpathclose%
\pgfusepath{fill}%
\end{pgfscope}%
\begin{pgfscope}%
\pgfpathrectangle{\pgfqpoint{0.680860in}{0.078740in}}{\pgfqpoint{7.842520in}{7.842520in}}%
\pgfusepath{clip}%
\pgfsetbuttcap%
\pgfsetroundjoin%
\definecolor{currentfill}{rgb}{0.199430,0.387607,0.554642}%
\pgfsetfillcolor{currentfill}%
\pgfsetlinewidth{0.000000pt}%
\definecolor{currentstroke}{rgb}{0.319809,0.770914,0.411152}%
\pgfsetstrokecolor{currentstroke}%
\pgfsetdash{}{0pt}%
\pgfpathmoveto{\pgfqpoint{2.348839in}{3.503168in}}%
\pgfpathlineto{\pgfqpoint{2.566427in}{3.646923in}}%
\pgfpathlineto{\pgfqpoint{2.432758in}{3.652139in}}%
\pgfpathclose%
\pgfusepath{fill}%
\end{pgfscope}%
\begin{pgfscope}%
\pgfpathrectangle{\pgfqpoint{0.680860in}{0.078740in}}{\pgfqpoint{7.842520in}{7.842520in}}%
\pgfusepath{clip}%
\pgfsetbuttcap%
\pgfsetroundjoin%
\definecolor{currentfill}{rgb}{0.119423,0.611141,0.538982}%
\pgfsetfillcolor{currentfill}%
\pgfsetlinewidth{0.000000pt}%
\definecolor{currentstroke}{rgb}{0.327796,0.773980,0.406640}%
\pgfsetstrokecolor{currentstroke}%
\pgfsetdash{}{0pt}%
\pgfpathmoveto{\pgfqpoint{5.028273in}{4.575333in}}%
\pgfpathlineto{\pgfqpoint{5.246102in}{4.575513in}}%
\pgfpathlineto{\pgfqpoint{5.106724in}{4.557637in}}%
\pgfpathclose%
\pgfusepath{fill}%
\end{pgfscope}%
\begin{pgfscope}%
\pgfpathrectangle{\pgfqpoint{0.680860in}{0.078740in}}{\pgfqpoint{7.842520in}{7.842520in}}%
\pgfusepath{clip}%
\pgfsetbuttcap%
\pgfsetroundjoin%
\definecolor{currentfill}{rgb}{0.139147,0.533812,0.555298}%
\pgfsetfillcolor{currentfill}%
\pgfsetlinewidth{0.000000pt}%
\definecolor{currentstroke}{rgb}{0.335885,0.777018,0.402049}%
\pgfsetstrokecolor{currentstroke}%
\pgfsetdash{}{0pt}%
\pgfpathmoveto{\pgfqpoint{7.170785in}{4.275781in}}%
\pgfpathlineto{\pgfqpoint{7.315451in}{4.288200in}}%
\pgfpathlineto{\pgfqpoint{7.235076in}{4.096611in}}%
\pgfpathclose%
\pgfusepath{fill}%
\end{pgfscope}%
\begin{pgfscope}%
\pgfpathrectangle{\pgfqpoint{0.680860in}{0.078740in}}{\pgfqpoint{7.842520in}{7.842520in}}%
\pgfusepath{clip}%
\pgfsetbuttcap%
\pgfsetroundjoin%
\definecolor{currentfill}{rgb}{0.120092,0.600104,0.542530}%
\pgfsetfillcolor{currentfill}%
\pgfsetlinewidth{0.000000pt}%
\definecolor{currentstroke}{rgb}{0.344074,0.780029,0.397381}%
\pgfsetstrokecolor{currentstroke}%
\pgfsetdash{}{0pt}%
\pgfpathmoveto{\pgfqpoint{5.960494in}{4.541766in}}%
\pgfpathlineto{\pgfqpoint{6.102393in}{4.561046in}}%
\pgfpathlineto{\pgfqpoint{6.033892in}{4.452976in}}%
\pgfpathclose%
\pgfusepath{fill}%
\end{pgfscope}%
\begin{pgfscope}%
\pgfpathrectangle{\pgfqpoint{0.680860in}{0.078740in}}{\pgfqpoint{7.842520in}{7.842520in}}%
\pgfusepath{clip}%
\pgfsetbuttcap%
\pgfsetroundjoin%
\definecolor{currentfill}{rgb}{0.144759,0.519093,0.556572}%
\pgfsetfillcolor{currentfill}%
\pgfsetlinewidth{0.000000pt}%
\definecolor{currentstroke}{rgb}{0.352360,0.783011,0.392636}%
\pgfsetstrokecolor{currentstroke}%
\pgfsetdash{}{0pt}%
\pgfpathmoveto{\pgfqpoint{7.523042in}{4.106176in}}%
\pgfpathlineto{\pgfqpoint{7.378712in}{4.100689in}}%
\pgfpathlineto{\pgfqpoint{7.315451in}{4.288200in}}%
\pgfpathclose%
\pgfusepath{fill}%
\end{pgfscope}%
\begin{pgfscope}%
\pgfpathrectangle{\pgfqpoint{0.680860in}{0.078740in}}{\pgfqpoint{7.842520in}{7.842520in}}%
\pgfusepath{clip}%
\pgfsetbuttcap%
\pgfsetroundjoin%
\definecolor{currentfill}{rgb}{0.119512,0.607464,0.540218}%
\pgfsetfillcolor{currentfill}%
\pgfsetlinewidth{0.000000pt}%
\definecolor{currentstroke}{rgb}{0.360741,0.785964,0.387814}%
\pgfsetstrokecolor{currentstroke}%
\pgfsetdash{}{0pt}%
\pgfpathmoveto{\pgfqpoint{4.809723in}{4.553976in}}%
\pgfpathlineto{\pgfqpoint{4.750789in}{4.539623in}}%
\pgfpathlineto{\pgfqpoint{4.670994in}{4.534985in}}%
\pgfpathclose%
\pgfusepath{fill}%
\end{pgfscope}%
\begin{pgfscope}%
\pgfpathrectangle{\pgfqpoint{0.680860in}{0.078740in}}{\pgfqpoint{7.842520in}{7.842520in}}%
\pgfusepath{clip}%
\pgfsetbuttcap%
\pgfsetroundjoin%
\definecolor{currentfill}{rgb}{0.239346,0.300855,0.540844}%
\pgfsetfillcolor{currentfill}%
\pgfsetlinewidth{0.000000pt}%
\definecolor{currentstroke}{rgb}{0.369214,0.788888,0.382914}%
\pgfsetstrokecolor{currentstroke}%
\pgfsetdash{}{0pt}%
\pgfpathmoveto{\pgfqpoint{1.914518in}{3.206196in}}%
\pgfpathlineto{\pgfqpoint{2.047573in}{3.190040in}}%
\pgfpathlineto{\pgfqpoint{2.131544in}{3.355773in}}%
\pgfpathclose%
\pgfusepath{fill}%
\end{pgfscope}%
\begin{pgfscope}%
\pgfpathrectangle{\pgfqpoint{0.680860in}{0.078740in}}{\pgfqpoint{7.842520in}{7.842520in}}%
\pgfusepath{clip}%
\pgfsetbuttcap%
\pgfsetroundjoin%
\definecolor{currentfill}{rgb}{0.139147,0.533812,0.555298}%
\pgfsetfillcolor{currentfill}%
\pgfsetlinewidth{0.000000pt}%
\definecolor{currentstroke}{rgb}{0.377779,0.791781,0.377939}%
\pgfsetstrokecolor{currentstroke}%
\pgfsetdash{}{0pt}%
\pgfpathmoveto{\pgfqpoint{3.658344in}{4.256435in}}%
\pgfpathlineto{\pgfqpoint{3.522549in}{4.246142in}}%
\pgfpathlineto{\pgfqpoint{3.439558in}{4.154450in}}%
\pgfpathclose%
\pgfusepath{fill}%
\end{pgfscope}%
\begin{pgfscope}%
\pgfpathrectangle{\pgfqpoint{0.680860in}{0.078740in}}{\pgfqpoint{7.842520in}{7.842520in}}%
\pgfusepath{clip}%
\pgfsetbuttcap%
\pgfsetroundjoin%
\definecolor{currentfill}{rgb}{0.121831,0.589055,0.545623}%
\pgfsetfillcolor{currentfill}%
\pgfsetlinewidth{0.000000pt}%
\definecolor{currentstroke}{rgb}{0.386433,0.794644,0.372886}%
\pgfsetstrokecolor{currentstroke}%
\pgfsetdash{}{0pt}%
\pgfpathmoveto{\pgfqpoint{6.530116in}{4.378410in}}%
\pgfpathlineto{\pgfqpoint{6.317148in}{4.482731in}}%
\pgfpathlineto{\pgfqpoint{6.459846in}{4.500310in}}%
\pgfpathclose%
\pgfusepath{fill}%
\end{pgfscope}%
\begin{pgfscope}%
\pgfpathrectangle{\pgfqpoint{0.680860in}{0.078740in}}{\pgfqpoint{7.842520in}{7.842520in}}%
\pgfusepath{clip}%
\pgfsetbuttcap%
\pgfsetroundjoin%
\definecolor{currentfill}{rgb}{0.147607,0.511733,0.557049}%
\pgfsetfillcolor{currentfill}%
\pgfsetlinewidth{0.000000pt}%
\definecolor{currentstroke}{rgb}{0.395174,0.797475,0.367757}%
\pgfsetstrokecolor{currentstroke}%
\pgfsetdash{}{0pt}%
\pgfpathmoveto{\pgfqpoint{3.220917in}{4.040891in}}%
\pgfpathlineto{\pgfqpoint{3.439558in}{4.154450in}}%
\pgfpathlineto{\pgfqpoint{3.304198in}{4.146617in}}%
\pgfpathclose%
\pgfusepath{fill}%
\end{pgfscope}%
\begin{pgfscope}%
\pgfpathrectangle{\pgfqpoint{0.680860in}{0.078740in}}{\pgfqpoint{7.842520in}{7.842520in}}%
\pgfusepath{clip}%
\pgfsetbuttcap%
\pgfsetroundjoin%
\definecolor{currentfill}{rgb}{0.157729,0.485932,0.558013}%
\pgfsetfillcolor{currentfill}%
\pgfsetlinewidth{0.000000pt}%
\definecolor{currentstroke}{rgb}{0.404001,0.800275,0.362552}%
\pgfsetstrokecolor{currentstroke}%
\pgfsetdash{}{0pt}%
\pgfpathmoveto{\pgfqpoint{3.220917in}{4.040891in}}%
\pgfpathlineto{\pgfqpoint{3.085992in}{4.035894in}}%
\pgfpathlineto{\pgfqpoint{3.002486in}{3.917372in}}%
\pgfpathclose%
\pgfusepath{fill}%
\end{pgfscope}%
\begin{pgfscope}%
\pgfpathrectangle{\pgfqpoint{0.680860in}{0.078740in}}{\pgfqpoint{7.842520in}{7.842520in}}%
\pgfusepath{clip}%
\pgfsetbuttcap%
\pgfsetroundjoin%
\definecolor{currentfill}{rgb}{0.122606,0.585371,0.546557}%
\pgfsetfillcolor{currentfill}%
\pgfsetlinewidth{0.000000pt}%
\definecolor{currentstroke}{rgb}{0.412913,0.803041,0.357269}%
\pgfsetstrokecolor{currentstroke}%
\pgfsetdash{}{0pt}%
\pgfpathmoveto{\pgfqpoint{4.314611in}{4.477381in}}%
\pgfpathlineto{\pgfqpoint{4.095983in}{4.419396in}}%
\pgfpathlineto{\pgfqpoint{4.233259in}{4.435022in}}%
\pgfpathclose%
\pgfusepath{fill}%
\end{pgfscope}%
\begin{pgfscope}%
\pgfpathrectangle{\pgfqpoint{0.680860in}{0.078740in}}{\pgfqpoint{7.842520in}{7.842520in}}%
\pgfusepath{clip}%
\pgfsetbuttcap%
\pgfsetroundjoin%
\definecolor{currentfill}{rgb}{0.220057,0.343307,0.549413}%
\pgfsetfillcolor{currentfill}%
\pgfsetlinewidth{0.000000pt}%
\definecolor{currentstroke}{rgb}{0.421908,0.805774,0.351910}%
\pgfsetstrokecolor{currentstroke}%
\pgfsetdash{}{0pt}%
\pgfpathmoveto{\pgfqpoint{2.131544in}{3.355773in}}%
\pgfpathlineto{\pgfqpoint{2.264961in}{3.343707in}}%
\pgfpathlineto{\pgfqpoint{2.348839in}{3.503168in}}%
\pgfpathclose%
\pgfusepath{fill}%
\end{pgfscope}%
\begin{pgfscope}%
\pgfpathrectangle{\pgfqpoint{0.680860in}{0.078740in}}{\pgfqpoint{7.842520in}{7.842520in}}%
\pgfusepath{clip}%
\pgfsetbuttcap%
\pgfsetroundjoin%
\definecolor{currentfill}{rgb}{0.120092,0.600104,0.542530}%
\pgfsetfillcolor{currentfill}%
\pgfsetlinewidth{0.000000pt}%
\definecolor{currentstroke}{rgb}{0.430983,0.808473,0.346476}%
\pgfsetstrokecolor{currentstroke}%
\pgfsetdash{}{0pt}%
\pgfpathmoveto{\pgfqpoint{6.102393in}{4.561046in}}%
\pgfpathlineto{\pgfqpoint{6.317148in}{4.482731in}}%
\pgfpathlineto{\pgfqpoint{6.175169in}{4.466975in}}%
\pgfpathclose%
\pgfusepath{fill}%
\end{pgfscope}%
\begin{pgfscope}%
\pgfpathrectangle{\pgfqpoint{0.680860in}{0.078740in}}{\pgfqpoint{7.842520in}{7.842520in}}%
\pgfusepath{clip}%
\pgfsetbuttcap%
\pgfsetroundjoin%
\definecolor{currentfill}{rgb}{0.119483,0.614817,0.537692}%
\pgfsetfillcolor{currentfill}%
\pgfsetlinewidth{0.000000pt}%
\definecolor{currentstroke}{rgb}{0.440137,0.811138,0.340967}%
\pgfsetstrokecolor{currentstroke}%
\pgfsetdash{}{0pt}%
\pgfpathmoveto{\pgfqpoint{5.744345in}{4.591617in}}%
\pgfpathlineto{\pgfqpoint{5.819308in}{4.524369in}}%
\pgfpathlineto{\pgfqpoint{5.603330in}{4.571642in}}%
\pgfpathclose%
\pgfusepath{fill}%
\end{pgfscope}%
\begin{pgfscope}%
\pgfpathrectangle{\pgfqpoint{0.680860in}{0.078740in}}{\pgfqpoint{7.842520in}{7.842520in}}%
\pgfusepath{clip}%
\pgfsetbuttcap%
\pgfsetroundjoin%
\definecolor{currentfill}{rgb}{0.129933,0.559582,0.551864}%
\pgfsetfillcolor{currentfill}%
\pgfsetlinewidth{0.000000pt}%
\definecolor{currentstroke}{rgb}{0.449368,0.813768,0.335384}%
\pgfsetstrokecolor{currentstroke}%
\pgfsetdash{}{0pt}%
\pgfpathmoveto{\pgfqpoint{3.740963in}{4.332939in}}%
\pgfpathlineto{\pgfqpoint{3.794749in}{4.268363in}}%
\pgfpathlineto{\pgfqpoint{3.877187in}{4.345262in}}%
\pgfpathclose%
\pgfusepath{fill}%
\end{pgfscope}%
\begin{pgfscope}%
\pgfpathrectangle{\pgfqpoint{0.680860in}{0.078740in}}{\pgfqpoint{7.842520in}{7.842520in}}%
\pgfusepath{clip}%
\pgfsetbuttcap%
\pgfsetroundjoin%
\definecolor{currentfill}{rgb}{0.187231,0.414746,0.556547}%
\pgfsetfillcolor{currentfill}%
\pgfsetlinewidth{0.000000pt}%
\definecolor{currentstroke}{rgb}{0.458674,0.816363,0.329727}%
\pgfsetstrokecolor{currentstroke}%
\pgfsetdash{}{0pt}%
\pgfpathmoveto{\pgfqpoint{2.650237in}{3.787098in}}%
\pgfpathlineto{\pgfqpoint{2.566427in}{3.646923in}}%
\pgfpathlineto{\pgfqpoint{2.700633in}{3.642806in}}%
\pgfpathclose%
\pgfusepath{fill}%
\end{pgfscope}%
\begin{pgfscope}%
\pgfpathrectangle{\pgfqpoint{0.680860in}{0.078740in}}{\pgfqpoint{7.842520in}{7.842520in}}%
\pgfusepath{clip}%
\pgfsetbuttcap%
\pgfsetroundjoin%
\definecolor{currentfill}{rgb}{0.182256,0.426184,0.557120}%
\pgfsetfillcolor{currentfill}%
\pgfsetlinewidth{0.000000pt}%
\definecolor{currentstroke}{rgb}{0.468053,0.818921,0.323998}%
\pgfsetstrokecolor{currentstroke}%
\pgfsetdash{}{0pt}%
\pgfpathmoveto{\pgfqpoint{2.700633in}{3.642806in}}%
\pgfpathlineto{\pgfqpoint{2.784313in}{3.785515in}}%
\pgfpathlineto{\pgfqpoint{2.650237in}{3.787098in}}%
\pgfpathclose%
\pgfusepath{fill}%
\end{pgfscope}%
\begin{pgfscope}%
\pgfpathrectangle{\pgfqpoint{0.680860in}{0.078740in}}{\pgfqpoint{7.842520in}{7.842520in}}%
\pgfusepath{clip}%
\pgfsetbuttcap%
\pgfsetroundjoin%
\definecolor{currentfill}{rgb}{0.132444,0.552216,0.553018}%
\pgfsetfillcolor{currentfill}%
\pgfsetlinewidth{0.000000pt}%
\definecolor{currentstroke}{rgb}{0.477504,0.821444,0.318195}%
\pgfsetstrokecolor{currentstroke}%
\pgfsetdash{}{0pt}%
\pgfpathmoveto{\pgfqpoint{3.658344in}{4.256435in}}%
\pgfpathlineto{\pgfqpoint{3.794749in}{4.268363in}}%
\pgfpathlineto{\pgfqpoint{3.740963in}{4.332939in}}%
\pgfpathclose%
\pgfusepath{fill}%
\end{pgfscope}%
\begin{pgfscope}%
\pgfpathrectangle{\pgfqpoint{0.680860in}{0.078740in}}{\pgfqpoint{7.842520in}{7.842520in}}%
\pgfusepath{clip}%
\pgfsetbuttcap%
\pgfsetroundjoin%
\definecolor{currentfill}{rgb}{0.119512,0.607464,0.540218}%
\pgfsetfillcolor{currentfill}%
\pgfsetlinewidth{0.000000pt}%
\definecolor{currentstroke}{rgb}{0.487026,0.823929,0.312321}%
\pgfsetstrokecolor{currentstroke}%
\pgfsetdash{}{0pt}%
\pgfpathmoveto{\pgfqpoint{4.590636in}{4.512586in}}%
\pgfpathlineto{\pgfqpoint{4.670994in}{4.534985in}}%
\pgfpathlineto{\pgfqpoint{4.532931in}{4.517865in}}%
\pgfpathclose%
\pgfusepath{fill}%
\end{pgfscope}%
\begin{pgfscope}%
\pgfpathrectangle{\pgfqpoint{0.680860in}{0.078740in}}{\pgfqpoint{7.842520in}{7.842520in}}%
\pgfusepath{clip}%
\pgfsetbuttcap%
\pgfsetroundjoin%
\definecolor{currentfill}{rgb}{0.122606,0.585371,0.546557}%
\pgfsetfillcolor{currentfill}%
\pgfsetlinewidth{0.000000pt}%
\definecolor{currentstroke}{rgb}{0.496615,0.826376,0.306377}%
\pgfsetstrokecolor{currentstroke}%
\pgfsetdash{}{0pt}%
\pgfpathmoveto{\pgfqpoint{6.603281in}{4.519783in}}%
\pgfpathlineto{\pgfqpoint{6.672785in}{4.391456in}}%
\pgfpathlineto{\pgfqpoint{6.530116in}{4.378410in}}%
\pgfpathclose%
\pgfusepath{fill}%
\end{pgfscope}%
\begin{pgfscope}%
\pgfpathrectangle{\pgfqpoint{0.680860in}{0.078740in}}{\pgfqpoint{7.842520in}{7.842520in}}%
\pgfusepath{clip}%
\pgfsetbuttcap%
\pgfsetroundjoin%
\definecolor{currentfill}{rgb}{0.172719,0.448791,0.557885}%
\pgfsetfillcolor{currentfill}%
\pgfsetlinewidth{0.000000pt}%
\definecolor{currentstroke}{rgb}{0.506271,0.828786,0.300362}%
\pgfsetstrokecolor{currentstroke}%
\pgfsetdash{}{0pt}%
\pgfpathmoveto{\pgfqpoint{2.867990in}{3.915533in}}%
\pgfpathlineto{\pgfqpoint{2.784313in}{3.785515in}}%
\pgfpathlineto{\pgfqpoint{2.918941in}{3.785155in}}%
\pgfpathclose%
\pgfusepath{fill}%
\end{pgfscope}%
\begin{pgfscope}%
\pgfpathrectangle{\pgfqpoint{0.680860in}{0.078740in}}{\pgfqpoint{7.842520in}{7.842520in}}%
\pgfusepath{clip}%
\pgfsetbuttcap%
\pgfsetroundjoin%
\definecolor{currentfill}{rgb}{0.253935,0.265254,0.529983}%
\pgfsetfillcolor{currentfill}%
\pgfsetlinewidth{0.000000pt}%
\definecolor{currentstroke}{rgb}{0.515992,0.831158,0.294279}%
\pgfsetstrokecolor{currentstroke}%
\pgfsetdash{}{0pt}%
\pgfpathmoveto{\pgfqpoint{1.914518in}{3.206196in}}%
\pgfpathlineto{\pgfqpoint{1.830442in}{3.035560in}}%
\pgfpathlineto{\pgfqpoint{1.963644in}{3.015899in}}%
\pgfpathclose%
\pgfusepath{fill}%
\end{pgfscope}%
\begin{pgfscope}%
\pgfpathrectangle{\pgfqpoint{0.680860in}{0.078740in}}{\pgfqpoint{7.842520in}{7.842520in}}%
\pgfusepath{clip}%
\pgfsetbuttcap%
\pgfsetroundjoin%
\definecolor{currentfill}{rgb}{0.229739,0.322361,0.545706}%
\pgfsetfillcolor{currentfill}%
\pgfsetlinewidth{0.000000pt}%
\definecolor{currentstroke}{rgb}{0.525776,0.833491,0.288127}%
\pgfsetstrokecolor{currentstroke}%
\pgfsetdash{}{0pt}%
\pgfpathmoveto{\pgfqpoint{2.131544in}{3.355773in}}%
\pgfpathlineto{\pgfqpoint{2.047573in}{3.190040in}}%
\pgfpathlineto{\pgfqpoint{2.264961in}{3.343707in}}%
\pgfpathclose%
\pgfusepath{fill}%
\end{pgfscope}%
\begin{pgfscope}%
\pgfpathrectangle{\pgfqpoint{0.680860in}{0.078740in}}{\pgfqpoint{7.842520in}{7.842520in}}%
\pgfusepath{clip}%
\pgfsetbuttcap%
\pgfsetroundjoin%
\definecolor{currentfill}{rgb}{0.120565,0.596422,0.543611}%
\pgfsetfillcolor{currentfill}%
\pgfsetlinewidth{0.000000pt}%
\definecolor{currentstroke}{rgb}{0.535621,0.835785,0.281908}%
\pgfsetstrokecolor{currentstroke}%
\pgfsetdash{}{0pt}%
\pgfpathmoveto{\pgfqpoint{4.452294in}{4.494054in}}%
\pgfpathlineto{\pgfqpoint{4.314611in}{4.477381in}}%
\pgfpathlineto{\pgfqpoint{4.233259in}{4.435022in}}%
\pgfpathclose%
\pgfusepath{fill}%
\end{pgfscope}%
\begin{pgfscope}%
\pgfpathrectangle{\pgfqpoint{0.680860in}{0.078740in}}{\pgfqpoint{7.842520in}{7.842520in}}%
\pgfusepath{clip}%
\pgfsetbuttcap%
\pgfsetroundjoin%
\definecolor{currentfill}{rgb}{0.168126,0.459988,0.558082}%
\pgfsetfillcolor{currentfill}%
\pgfsetlinewidth{0.000000pt}%
\definecolor{currentstroke}{rgb}{0.545524,0.838039,0.275626}%
\pgfsetstrokecolor{currentstroke}%
\pgfsetdash{}{0pt}%
\pgfpathmoveto{\pgfqpoint{2.918941in}{3.785155in}}%
\pgfpathlineto{\pgfqpoint{3.002486in}{3.917372in}}%
\pgfpathlineto{\pgfqpoint{2.867990in}{3.915533in}}%
\pgfpathclose%
\pgfusepath{fill}%
\end{pgfscope}%
\begin{pgfscope}%
\pgfpathrectangle{\pgfqpoint{0.680860in}{0.078740in}}{\pgfqpoint{7.842520in}{7.842520in}}%
\pgfusepath{clip}%
\pgfsetbuttcap%
\pgfsetroundjoin%
\definecolor{currentfill}{rgb}{0.201239,0.383670,0.554294}%
\pgfsetfillcolor{currentfill}%
\pgfsetlinewidth{0.000000pt}%
\definecolor{currentstroke}{rgb}{0.555484,0.840254,0.269281}%
\pgfsetstrokecolor{currentstroke}%
\pgfsetdash{}{0pt}%
\pgfpathmoveto{\pgfqpoint{2.482641in}{3.495131in}}%
\pgfpathlineto{\pgfqpoint{2.566427in}{3.646923in}}%
\pgfpathlineto{\pgfqpoint{2.348839in}{3.503168in}}%
\pgfpathclose%
\pgfusepath{fill}%
\end{pgfscope}%
\begin{pgfscope}%
\pgfpathrectangle{\pgfqpoint{0.680860in}{0.078740in}}{\pgfqpoint{7.842520in}{7.842520in}}%
\pgfusepath{clip}%
\pgfsetbuttcap%
\pgfsetroundjoin%
\definecolor{currentfill}{rgb}{0.125394,0.574318,0.549086}%
\pgfsetfillcolor{currentfill}%
\pgfsetlinewidth{0.000000pt}%
\definecolor{currentstroke}{rgb}{0.565498,0.842430,0.262877}%
\pgfsetstrokecolor{currentstroke}%
\pgfsetdash{}{0pt}%
\pgfpathmoveto{\pgfqpoint{4.014034in}{4.359289in}}%
\pgfpathlineto{\pgfqpoint{4.095983in}{4.419396in}}%
\pgfpathlineto{\pgfqpoint{3.877187in}{4.345262in}}%
\pgfpathclose%
\pgfusepath{fill}%
\end{pgfscope}%
\begin{pgfscope}%
\pgfpathrectangle{\pgfqpoint{0.680860in}{0.078740in}}{\pgfqpoint{7.842520in}{7.842520in}}%
\pgfusepath{clip}%
\pgfsetbuttcap%
\pgfsetroundjoin%
\definecolor{currentfill}{rgb}{0.248629,0.278775,0.534556}%
\pgfsetfillcolor{currentfill}%
\pgfsetlinewidth{0.000000pt}%
\definecolor{currentstroke}{rgb}{0.575563,0.844566,0.256415}%
\pgfsetstrokecolor{currentstroke}%
\pgfsetdash{}{0pt}%
\pgfpathmoveto{\pgfqpoint{1.963644in}{3.015899in}}%
\pgfpathlineto{\pgfqpoint{2.047573in}{3.190040in}}%
\pgfpathlineto{\pgfqpoint{1.914518in}{3.206196in}}%
\pgfpathclose%
\pgfusepath{fill}%
\end{pgfscope}%
\begin{pgfscope}%
\pgfpathrectangle{\pgfqpoint{0.680860in}{0.078740in}}{\pgfqpoint{7.842520in}{7.842520in}}%
\pgfusepath{clip}%
\pgfsetbuttcap%
\pgfsetroundjoin%
\definecolor{currentfill}{rgb}{0.119738,0.603785,0.541400}%
\pgfsetfillcolor{currentfill}%
\pgfsetlinewidth{0.000000pt}%
\definecolor{currentstroke}{rgb}{0.585678,0.846661,0.249897}%
\pgfsetstrokecolor{currentstroke}%
\pgfsetdash{}{0pt}%
\pgfpathmoveto{\pgfqpoint{4.532931in}{4.517865in}}%
\pgfpathlineto{\pgfqpoint{4.452294in}{4.494054in}}%
\pgfpathlineto{\pgfqpoint{4.590636in}{4.512586in}}%
\pgfpathclose%
\pgfusepath{fill}%
\end{pgfscope}%
\begin{pgfscope}%
\pgfpathrectangle{\pgfqpoint{0.680860in}{0.078740in}}{\pgfqpoint{7.842520in}{7.842520in}}%
\pgfusepath{clip}%
\pgfsetbuttcap%
\pgfsetroundjoin%
\definecolor{currentfill}{rgb}{0.120081,0.622161,0.534946}%
\pgfsetfillcolor{currentfill}%
\pgfsetlinewidth{0.000000pt}%
\definecolor{currentstroke}{rgb}{0.595839,0.848717,0.243329}%
\pgfsetstrokecolor{currentstroke}%
\pgfsetdash{}{0pt}%
\pgfpathmoveto{\pgfqpoint{5.463019in}{4.553572in}}%
\pgfpathlineto{\pgfqpoint{5.526942in}{4.617032in}}%
\pgfpathlineto{\pgfqpoint{5.603330in}{4.571642in}}%
\pgfpathclose%
\pgfusepath{fill}%
\end{pgfscope}%
\begin{pgfscope}%
\pgfpathrectangle{\pgfqpoint{0.680860in}{0.078740in}}{\pgfqpoint{7.842520in}{7.842520in}}%
\pgfusepath{clip}%
\pgfsetbuttcap%
\pgfsetroundjoin%
\definecolor{currentfill}{rgb}{0.119699,0.618490,0.536347}%
\pgfsetfillcolor{currentfill}%
\pgfsetlinewidth{0.000000pt}%
\definecolor{currentstroke}{rgb}{0.606045,0.850733,0.236712}%
\pgfsetstrokecolor{currentstroke}%
\pgfsetdash{}{0pt}%
\pgfpathmoveto{\pgfqpoint{5.819308in}{4.524369in}}%
\pgfpathlineto{\pgfqpoint{5.886079in}{4.613572in}}%
\pgfpathlineto{\pgfqpoint{5.960494in}{4.541766in}}%
\pgfpathclose%
\pgfusepath{fill}%
\end{pgfscope}%
\begin{pgfscope}%
\pgfpathrectangle{\pgfqpoint{0.680860in}{0.078740in}}{\pgfqpoint{7.842520in}{7.842520in}}%
\pgfusepath{clip}%
\pgfsetbuttcap%
\pgfsetroundjoin%
\definecolor{currentfill}{rgb}{0.137770,0.537492,0.554906}%
\pgfsetfillcolor{currentfill}%
\pgfsetlinewidth{0.000000pt}%
\definecolor{currentstroke}{rgb}{0.616293,0.852709,0.230052}%
\pgfsetstrokecolor{currentstroke}%
\pgfsetdash{}{0pt}%
\pgfpathmoveto{\pgfqpoint{3.658344in}{4.256435in}}%
\pgfpathlineto{\pgfqpoint{3.439558in}{4.154450in}}%
\pgfpathlineto{\pgfqpoint{3.575515in}{4.163831in}}%
\pgfpathclose%
\pgfusepath{fill}%
\end{pgfscope}%
\begin{pgfscope}%
\pgfpathrectangle{\pgfqpoint{0.680860in}{0.078740in}}{\pgfqpoint{7.842520in}{7.842520in}}%
\pgfusepath{clip}%
\pgfsetbuttcap%
\pgfsetroundjoin%
\definecolor{currentfill}{rgb}{0.157729,0.485932,0.558013}%
\pgfsetfillcolor{currentfill}%
\pgfsetlinewidth{0.000000pt}%
\definecolor{currentstroke}{rgb}{0.626579,0.854645,0.223353}%
\pgfsetstrokecolor{currentstroke}%
\pgfsetdash{}{0pt}%
\pgfpathmoveto{\pgfqpoint{3.002486in}{3.917372in}}%
\pgfpathlineto{\pgfqpoint{3.137549in}{3.920553in}}%
\pgfpathlineto{\pgfqpoint{3.220917in}{4.040891in}}%
\pgfpathclose%
\pgfusepath{fill}%
\end{pgfscope}%
\begin{pgfscope}%
\pgfpathrectangle{\pgfqpoint{0.680860in}{0.078740in}}{\pgfqpoint{7.842520in}{7.842520in}}%
\pgfusepath{clip}%
\pgfsetbuttcap%
\pgfsetroundjoin%
\definecolor{currentfill}{rgb}{0.194100,0.399323,0.555565}%
\pgfsetfillcolor{currentfill}%
\pgfsetlinewidth{0.000000pt}%
\definecolor{currentstroke}{rgb}{0.636902,0.856542,0.216620}%
\pgfsetstrokecolor{currentstroke}%
\pgfsetdash{}{0pt}%
\pgfpathmoveto{\pgfqpoint{2.700633in}{3.642806in}}%
\pgfpathlineto{\pgfqpoint{2.566427in}{3.646923in}}%
\pgfpathlineto{\pgfqpoint{2.482641in}{3.495131in}}%
\pgfpathclose%
\pgfusepath{fill}%
\end{pgfscope}%
\begin{pgfscope}%
\pgfpathrectangle{\pgfqpoint{0.680860in}{0.078740in}}{\pgfqpoint{7.842520in}{7.842520in}}%
\pgfusepath{clip}%
\pgfsetbuttcap%
\pgfsetroundjoin%
\definecolor{currentfill}{rgb}{0.120638,0.625828,0.533488}%
\pgfsetfillcolor{currentfill}%
\pgfsetlinewidth{0.000000pt}%
\definecolor{currentstroke}{rgb}{0.647257,0.858400,0.209861}%
\pgfsetstrokecolor{currentstroke}%
\pgfsetdash{}{0pt}%
\pgfpathmoveto{\pgfqpoint{5.386170in}{4.595288in}}%
\pgfpathlineto{\pgfqpoint{5.526942in}{4.617032in}}%
\pgfpathlineto{\pgfqpoint{5.463019in}{4.553572in}}%
\pgfpathclose%
\pgfusepath{fill}%
\end{pgfscope}%
\begin{pgfscope}%
\pgfpathrectangle{\pgfqpoint{0.680860in}{0.078740in}}{\pgfqpoint{7.842520in}{7.842520in}}%
\pgfusepath{clip}%
\pgfsetbuttcap%
\pgfsetroundjoin%
\definecolor{currentfill}{rgb}{0.124395,0.578002,0.548287}%
\pgfsetfillcolor{currentfill}%
\pgfsetlinewidth{0.000000pt}%
\definecolor{currentstroke}{rgb}{0.657642,0.860219,0.203082}%
\pgfsetstrokecolor{currentstroke}%
\pgfsetdash{}{0pt}%
\pgfpathmoveto{\pgfqpoint{6.816173in}{4.406230in}}%
\pgfpathlineto{\pgfqpoint{6.960295in}{4.422796in}}%
\pgfpathlineto{\pgfqpoint{7.026844in}{4.265011in}}%
\pgfpathclose%
\pgfusepath{fill}%
\end{pgfscope}%
\begin{pgfscope}%
\pgfpathrectangle{\pgfqpoint{0.680860in}{0.078740in}}{\pgfqpoint{7.842520in}{7.842520in}}%
\pgfusepath{clip}%
\pgfsetbuttcap%
\pgfsetroundjoin%
\definecolor{currentfill}{rgb}{0.147607,0.511733,0.557049}%
\pgfsetfillcolor{currentfill}%
\pgfsetlinewidth{0.000000pt}%
\definecolor{currentstroke}{rgb}{0.668054,0.861999,0.196293}%
\pgfsetstrokecolor{currentstroke}%
\pgfsetdash{}{0pt}%
\pgfpathmoveto{\pgfqpoint{3.356424in}{4.047338in}}%
\pgfpathlineto{\pgfqpoint{3.439558in}{4.154450in}}%
\pgfpathlineto{\pgfqpoint{3.220917in}{4.040891in}}%
\pgfpathclose%
\pgfusepath{fill}%
\end{pgfscope}%
\begin{pgfscope}%
\pgfpathrectangle{\pgfqpoint{0.680860in}{0.078740in}}{\pgfqpoint{7.842520in}{7.842520in}}%
\pgfusepath{clip}%
\pgfsetbuttcap%
\pgfsetroundjoin%
\definecolor{currentfill}{rgb}{0.263663,0.237631,0.518762}%
\pgfsetfillcolor{currentfill}%
\pgfsetlinewidth{0.000000pt}%
\definecolor{currentstroke}{rgb}{0.678489,0.863742,0.189503}%
\pgfsetstrokecolor{currentstroke}%
\pgfsetdash{}{0pt}%
\pgfpathmoveto{\pgfqpoint{1.963644in}{3.015899in}}%
\pgfpathlineto{\pgfqpoint{1.830442in}{3.035560in}}%
\pgfpathlineto{\pgfqpoint{1.746392in}{2.857738in}}%
\pgfpathclose%
\pgfusepath{fill}%
\end{pgfscope}%
\begin{pgfscope}%
\pgfpathrectangle{\pgfqpoint{0.680860in}{0.078740in}}{\pgfqpoint{7.842520in}{7.842520in}}%
\pgfusepath{clip}%
\pgfsetbuttcap%
\pgfsetroundjoin%
\definecolor{currentfill}{rgb}{0.120638,0.625828,0.533488}%
\pgfsetfillcolor{currentfill}%
\pgfsetlinewidth{0.000000pt}%
\definecolor{currentstroke}{rgb}{0.688944,0.865448,0.182725}%
\pgfsetstrokecolor{currentstroke}%
\pgfsetdash{}{0pt}%
\pgfpathmoveto{\pgfqpoint{5.168036in}{4.596052in}}%
\pgfpathlineto{\pgfqpoint{5.246102in}{4.575513in}}%
\pgfpathlineto{\pgfqpoint{5.028273in}{4.575333in}}%
\pgfpathclose%
\pgfusepath{fill}%
\end{pgfscope}%
\begin{pgfscope}%
\pgfpathrectangle{\pgfqpoint{0.680860in}{0.078740in}}{\pgfqpoint{7.842520in}{7.842520in}}%
\pgfusepath{clip}%
\pgfsetbuttcap%
\pgfsetroundjoin%
\definecolor{currentfill}{rgb}{0.120092,0.600104,0.542530}%
\pgfsetfillcolor{currentfill}%
\pgfsetlinewidth{0.000000pt}%
\definecolor{currentstroke}{rgb}{0.699415,0.867117,0.175971}%
\pgfsetstrokecolor{currentstroke}%
\pgfsetdash{}{0pt}%
\pgfpathmoveto{\pgfqpoint{6.530116in}{4.378410in}}%
\pgfpathlineto{\pgfqpoint{6.459846in}{4.500310in}}%
\pgfpathlineto{\pgfqpoint{6.603281in}{4.519783in}}%
\pgfpathclose%
\pgfusepath{fill}%
\end{pgfscope}%
\begin{pgfscope}%
\pgfpathrectangle{\pgfqpoint{0.680860in}{0.078740in}}{\pgfqpoint{7.842520in}{7.842520in}}%
\pgfusepath{clip}%
\pgfsetbuttcap%
\pgfsetroundjoin%
\definecolor{currentfill}{rgb}{0.216210,0.351535,0.550627}%
\pgfsetfillcolor{currentfill}%
\pgfsetlinewidth{0.000000pt}%
\definecolor{currentstroke}{rgb}{0.709898,0.868751,0.169257}%
\pgfsetstrokecolor{currentstroke}%
\pgfsetdash{}{0pt}%
\pgfpathmoveto{\pgfqpoint{2.264961in}{3.343707in}}%
\pgfpathlineto{\pgfqpoint{2.398894in}{3.332512in}}%
\pgfpathlineto{\pgfqpoint{2.348839in}{3.503168in}}%
\pgfpathclose%
\pgfusepath{fill}%
\end{pgfscope}%
\begin{pgfscope}%
\pgfpathrectangle{\pgfqpoint{0.680860in}{0.078740in}}{\pgfqpoint{7.842520in}{7.842520in}}%
\pgfusepath{clip}%
\pgfsetbuttcap%
\pgfsetroundjoin%
\definecolor{currentfill}{rgb}{0.120638,0.625828,0.533488}%
\pgfsetfillcolor{currentfill}%
\pgfsetlinewidth{0.000000pt}%
\definecolor{currentstroke}{rgb}{0.720391,0.870350,0.162603}%
\pgfsetstrokecolor{currentstroke}%
\pgfsetdash{}{0pt}%
\pgfpathmoveto{\pgfqpoint{4.949132in}{4.574912in}}%
\pgfpathlineto{\pgfqpoint{5.028273in}{4.575333in}}%
\pgfpathlineto{\pgfqpoint{4.889195in}{4.556547in}}%
\pgfpathclose%
\pgfusepath{fill}%
\end{pgfscope}%
\begin{pgfscope}%
\pgfpathrectangle{\pgfqpoint{0.680860in}{0.078740in}}{\pgfqpoint{7.842520in}{7.842520in}}%
\pgfusepath{clip}%
\pgfsetbuttcap%
\pgfsetroundjoin%
\definecolor{currentfill}{rgb}{0.210503,0.363727,0.552206}%
\pgfsetfillcolor{currentfill}%
\pgfsetlinewidth{0.000000pt}%
\definecolor{currentstroke}{rgb}{0.730889,0.871916,0.156029}%
\pgfsetstrokecolor{currentstroke}%
\pgfsetdash{}{0pt}%
\pgfpathmoveto{\pgfqpoint{2.348839in}{3.503168in}}%
\pgfpathlineto{\pgfqpoint{2.398894in}{3.332512in}}%
\pgfpathlineto{\pgfqpoint{2.482641in}{3.495131in}}%
\pgfpathclose%
\pgfusepath{fill}%
\end{pgfscope}%
\begin{pgfscope}%
\pgfpathrectangle{\pgfqpoint{0.680860in}{0.078740in}}{\pgfqpoint{7.842520in}{7.842520in}}%
\pgfusepath{clip}%
\pgfsetbuttcap%
\pgfsetroundjoin%
\definecolor{currentfill}{rgb}{0.233603,0.313828,0.543914}%
\pgfsetfillcolor{currentfill}%
\pgfsetlinewidth{0.000000pt}%
\definecolor{currentstroke}{rgb}{0.741388,0.873449,0.149561}%
\pgfsetstrokecolor{currentstroke}%
\pgfsetdash{}{0pt}%
\pgfpathmoveto{\pgfqpoint{2.047573in}{3.190040in}}%
\pgfpathlineto{\pgfqpoint{2.181129in}{3.174618in}}%
\pgfpathlineto{\pgfqpoint{2.264961in}{3.343707in}}%
\pgfpathclose%
\pgfusepath{fill}%
\end{pgfscope}%
\begin{pgfscope}%
\pgfpathrectangle{\pgfqpoint{0.680860in}{0.078740in}}{\pgfqpoint{7.842520in}{7.842520in}}%
\pgfusepath{clip}%
\pgfsetbuttcap%
\pgfsetroundjoin%
\definecolor{currentfill}{rgb}{0.177423,0.437527,0.557565}%
\pgfsetfillcolor{currentfill}%
\pgfsetlinewidth{0.000000pt}%
\definecolor{currentstroke}{rgb}{0.751884,0.874951,0.143228}%
\pgfsetstrokecolor{currentstroke}%
\pgfsetdash{}{0pt}%
\pgfpathmoveto{\pgfqpoint{2.918941in}{3.785155in}}%
\pgfpathlineto{\pgfqpoint{2.784313in}{3.785515in}}%
\pgfpathlineto{\pgfqpoint{2.700633in}{3.642806in}}%
\pgfpathclose%
\pgfusepath{fill}%
\end{pgfscope}%
\begin{pgfscope}%
\pgfpathrectangle{\pgfqpoint{0.680860in}{0.078740in}}{\pgfqpoint{7.842520in}{7.842520in}}%
\pgfusepath{clip}%
\pgfsetbuttcap%
\pgfsetroundjoin%
\definecolor{currentfill}{rgb}{0.120638,0.625828,0.533488}%
\pgfsetfillcolor{currentfill}%
\pgfsetlinewidth{0.000000pt}%
\definecolor{currentstroke}{rgb}{0.762373,0.876424,0.137064}%
\pgfsetstrokecolor{currentstroke}%
\pgfsetdash{}{0pt}%
\pgfpathmoveto{\pgfqpoint{5.744345in}{4.591617in}}%
\pgfpathlineto{\pgfqpoint{5.886079in}{4.613572in}}%
\pgfpathlineto{\pgfqpoint{5.819308in}{4.524369in}}%
\pgfpathclose%
\pgfusepath{fill}%
\end{pgfscope}%
\begin{pgfscope}%
\pgfpathrectangle{\pgfqpoint{0.680860in}{0.078740in}}{\pgfqpoint{7.842520in}{7.842520in}}%
\pgfusepath{clip}%
\pgfsetbuttcap%
\pgfsetroundjoin%
\definecolor{currentfill}{rgb}{0.126453,0.570633,0.549841}%
\pgfsetfillcolor{currentfill}%
\pgfsetlinewidth{0.000000pt}%
\definecolor{currentstroke}{rgb}{0.772852,0.877868,0.131109}%
\pgfsetstrokecolor{currentstroke}%
\pgfsetdash{}{0pt}%
\pgfpathmoveto{\pgfqpoint{7.026844in}{4.265011in}}%
\pgfpathlineto{\pgfqpoint{7.105170in}{4.441224in}}%
\pgfpathlineto{\pgfqpoint{7.170785in}{4.275781in}}%
\pgfpathclose%
\pgfusepath{fill}%
\end{pgfscope}%
\begin{pgfscope}%
\pgfpathrectangle{\pgfqpoint{0.680860in}{0.078740in}}{\pgfqpoint{7.842520in}{7.842520in}}%
\pgfusepath{clip}%
\pgfsetbuttcap%
\pgfsetroundjoin%
\definecolor{currentfill}{rgb}{0.119699,0.618490,0.536347}%
\pgfsetfillcolor{currentfill}%
\pgfsetlinewidth{0.000000pt}%
\definecolor{currentstroke}{rgb}{0.783315,0.879285,0.125405}%
\pgfsetstrokecolor{currentstroke}%
\pgfsetdash{}{0pt}%
\pgfpathmoveto{\pgfqpoint{4.809723in}{4.553976in}}%
\pgfpathlineto{\pgfqpoint{4.670994in}{4.534985in}}%
\pgfpathlineto{\pgfqpoint{4.590636in}{4.512586in}}%
\pgfpathclose%
\pgfusepath{fill}%
\end{pgfscope}%
\begin{pgfscope}%
\pgfpathrectangle{\pgfqpoint{0.680860in}{0.078740in}}{\pgfqpoint{7.842520in}{7.842520in}}%
\pgfusepath{clip}%
\pgfsetbuttcap%
\pgfsetroundjoin%
\definecolor{currentfill}{rgb}{0.246811,0.283237,0.535941}%
\pgfsetfillcolor{currentfill}%
\pgfsetlinewidth{0.000000pt}%
\definecolor{currentstroke}{rgb}{0.793760,0.880678,0.120005}%
\pgfsetstrokecolor{currentstroke}%
\pgfsetdash{}{0pt}%
\pgfpathmoveto{\pgfqpoint{2.181129in}{3.174618in}}%
\pgfpathlineto{\pgfqpoint{2.047573in}{3.190040in}}%
\pgfpathlineto{\pgfqpoint{1.963644in}{3.015899in}}%
\pgfpathclose%
\pgfusepath{fill}%
\end{pgfscope}%
\begin{pgfscope}%
\pgfpathrectangle{\pgfqpoint{0.680860in}{0.078740in}}{\pgfqpoint{7.842520in}{7.842520in}}%
\pgfusepath{clip}%
\pgfsetbuttcap%
\pgfsetroundjoin%
\definecolor{currentfill}{rgb}{0.120638,0.625828,0.533488}%
\pgfsetfillcolor{currentfill}%
\pgfsetlinewidth{0.000000pt}%
\definecolor{currentstroke}{rgb}{0.804182,0.882046,0.114965}%
\pgfsetstrokecolor{currentstroke}%
\pgfsetdash{}{0pt}%
\pgfpathmoveto{\pgfqpoint{4.889195in}{4.556547in}}%
\pgfpathlineto{\pgfqpoint{4.809723in}{4.553976in}}%
\pgfpathlineto{\pgfqpoint{4.949132in}{4.574912in}}%
\pgfpathclose%
\pgfusepath{fill}%
\end{pgfscope}%
\begin{pgfscope}%
\pgfpathrectangle{\pgfqpoint{0.680860in}{0.078740in}}{\pgfqpoint{7.842520in}{7.842520in}}%
\pgfusepath{clip}%
\pgfsetbuttcap%
\pgfsetroundjoin%
\definecolor{currentfill}{rgb}{0.121831,0.589055,0.545623}%
\pgfsetfillcolor{currentfill}%
\pgfsetlinewidth{0.000000pt}%
\definecolor{currentstroke}{rgb}{0.814576,0.883393,0.110347}%
\pgfsetstrokecolor{currentstroke}%
\pgfsetdash{}{0pt}%
\pgfpathmoveto{\pgfqpoint{4.233259in}{4.435022in}}%
\pgfpathlineto{\pgfqpoint{4.095983in}{4.419396in}}%
\pgfpathlineto{\pgfqpoint{4.151517in}{4.375085in}}%
\pgfpathclose%
\pgfusepath{fill}%
\end{pgfscope}%
\begin{pgfscope}%
\pgfpathrectangle{\pgfqpoint{0.680860in}{0.078740in}}{\pgfqpoint{7.842520in}{7.842520in}}%
\pgfusepath{clip}%
\pgfsetbuttcap%
\pgfsetroundjoin%
\definecolor{currentfill}{rgb}{0.133743,0.548535,0.553541}%
\pgfsetfillcolor{currentfill}%
\pgfsetlinewidth{0.000000pt}%
\definecolor{currentstroke}{rgb}{0.824940,0.884720,0.106217}%
\pgfsetstrokecolor{currentstroke}%
\pgfsetdash{}{0pt}%
\pgfpathmoveto{\pgfqpoint{3.794749in}{4.268363in}}%
\pgfpathlineto{\pgfqpoint{3.658344in}{4.256435in}}%
\pgfpathlineto{\pgfqpoint{3.575515in}{4.163831in}}%
\pgfpathclose%
\pgfusepath{fill}%
\end{pgfscope}%
\begin{pgfscope}%
\pgfpathrectangle{\pgfqpoint{0.680860in}{0.078740in}}{\pgfqpoint{7.842520in}{7.842520in}}%
\pgfusepath{clip}%
\pgfsetbuttcap%
\pgfsetroundjoin%
\definecolor{currentfill}{rgb}{0.151918,0.500685,0.557587}%
\pgfsetfillcolor{currentfill}%
\pgfsetlinewidth{0.000000pt}%
\definecolor{currentstroke}{rgb}{0.835270,0.886029,0.102646}%
\pgfsetstrokecolor{currentstroke}%
\pgfsetdash{}{0pt}%
\pgfpathmoveto{\pgfqpoint{3.356424in}{4.047338in}}%
\pgfpathlineto{\pgfqpoint{3.220917in}{4.040891in}}%
\pgfpathlineto{\pgfqpoint{3.137549in}{3.920553in}}%
\pgfpathclose%
\pgfusepath{fill}%
\end{pgfscope}%
\begin{pgfscope}%
\pgfpathrectangle{\pgfqpoint{0.680860in}{0.078740in}}{\pgfqpoint{7.842520in}{7.842520in}}%
\pgfusepath{clip}%
\pgfsetbuttcap%
\pgfsetroundjoin%
\definecolor{currentfill}{rgb}{0.122606,0.585371,0.546557}%
\pgfsetfillcolor{currentfill}%
\pgfsetlinewidth{0.000000pt}%
\definecolor{currentstroke}{rgb}{0.845561,0.887322,0.099702}%
\pgfsetstrokecolor{currentstroke}%
\pgfsetdash{}{0pt}%
\pgfpathmoveto{\pgfqpoint{4.151517in}{4.375085in}}%
\pgfpathlineto{\pgfqpoint{4.095983in}{4.419396in}}%
\pgfpathlineto{\pgfqpoint{4.014034in}{4.359289in}}%
\pgfpathclose%
\pgfusepath{fill}%
\end{pgfscope}%
\begin{pgfscope}%
\pgfpathrectangle{\pgfqpoint{0.680860in}{0.078740in}}{\pgfqpoint{7.842520in}{7.842520in}}%
\pgfusepath{clip}%
\pgfsetbuttcap%
\pgfsetroundjoin%
\definecolor{currentfill}{rgb}{0.132444,0.552216,0.553018}%
\pgfsetfillcolor{currentfill}%
\pgfsetlinewidth{0.000000pt}%
\definecolor{currentstroke}{rgb}{0.855810,0.888601,0.097452}%
\pgfsetstrokecolor{currentstroke}%
\pgfsetdash{}{0pt}%
\pgfpathmoveto{\pgfqpoint{7.460861in}{4.302331in}}%
\pgfpathlineto{\pgfqpoint{7.523042in}{4.106176in}}%
\pgfpathlineto{\pgfqpoint{7.315451in}{4.288200in}}%
\pgfpathclose%
\pgfusepath{fill}%
\end{pgfscope}%
\begin{pgfscope}%
\pgfpathrectangle{\pgfqpoint{0.680860in}{0.078740in}}{\pgfqpoint{7.842520in}{7.842520in}}%
\pgfusepath{clip}%
\pgfsetbuttcap%
\pgfsetroundjoin%
\definecolor{currentfill}{rgb}{0.122312,0.633153,0.530398}%
\pgfsetfillcolor{currentfill}%
\pgfsetlinewidth{0.000000pt}%
\definecolor{currentstroke}{rgb}{0.866013,0.889868,0.095953}%
\pgfsetstrokecolor{currentstroke}%
\pgfsetdash{}{0pt}%
\pgfpathmoveto{\pgfqpoint{5.744345in}{4.591617in}}%
\pgfpathlineto{\pgfqpoint{5.603330in}{4.571642in}}%
\pgfpathlineto{\pgfqpoint{5.526942in}{4.617032in}}%
\pgfpathclose%
\pgfusepath{fill}%
\end{pgfscope}%
\begin{pgfscope}%
\pgfpathrectangle{\pgfqpoint{0.680860in}{0.078740in}}{\pgfqpoint{7.842520in}{7.842520in}}%
\pgfusepath{clip}%
\pgfsetbuttcap%
\pgfsetroundjoin%
\definecolor{currentfill}{rgb}{0.119738,0.603785,0.541400}%
\pgfsetfillcolor{currentfill}%
\pgfsetlinewidth{0.000000pt}%
\definecolor{currentstroke}{rgb}{0.876168,0.891125,0.095250}%
\pgfsetstrokecolor{currentstroke}%
\pgfsetdash{}{0pt}%
\pgfpathmoveto{\pgfqpoint{4.452294in}{4.494054in}}%
\pgfpathlineto{\pgfqpoint{4.233259in}{4.435022in}}%
\pgfpathlineto{\pgfqpoint{4.371182in}{4.452473in}}%
\pgfpathclose%
\pgfusepath{fill}%
\end{pgfscope}%
\begin{pgfscope}%
\pgfpathrectangle{\pgfqpoint{0.680860in}{0.078740in}}{\pgfqpoint{7.842520in}{7.842520in}}%
\pgfusepath{clip}%
\pgfsetbuttcap%
\pgfsetroundjoin%
\definecolor{currentfill}{rgb}{0.267968,0.223549,0.512008}%
\pgfsetfillcolor{currentfill}%
\pgfsetlinewidth{0.000000pt}%
\definecolor{currentstroke}{rgb}{0.886271,0.892374,0.095374}%
\pgfsetstrokecolor{currentstroke}%
\pgfsetdash{}{0pt}%
\pgfpathmoveto{\pgfqpoint{1.746392in}{2.857738in}}%
\pgfpathlineto{\pgfqpoint{1.879746in}{2.834351in}}%
\pgfpathlineto{\pgfqpoint{1.963644in}{3.015899in}}%
\pgfpathclose%
\pgfusepath{fill}%
\end{pgfscope}%
\begin{pgfscope}%
\pgfpathrectangle{\pgfqpoint{0.680860in}{0.078740in}}{\pgfqpoint{7.842520in}{7.842520in}}%
\pgfusepath{clip}%
\pgfsetbuttcap%
\pgfsetroundjoin%
\definecolor{currentfill}{rgb}{0.125394,0.574318,0.549086}%
\pgfsetfillcolor{currentfill}%
\pgfsetlinewidth{0.000000pt}%
\definecolor{currentstroke}{rgb}{0.896320,0.893616,0.096335}%
\pgfsetstrokecolor{currentstroke}%
\pgfsetdash{}{0pt}%
\pgfpathmoveto{\pgfqpoint{4.014034in}{4.359289in}}%
\pgfpathlineto{\pgfqpoint{3.877187in}{4.345262in}}%
\pgfpathlineto{\pgfqpoint{3.931777in}{4.281986in}}%
\pgfpathclose%
\pgfusepath{fill}%
\end{pgfscope}%
\begin{pgfscope}%
\pgfpathrectangle{\pgfqpoint{0.680860in}{0.078740in}}{\pgfqpoint{7.842520in}{7.842520in}}%
\pgfusepath{clip}%
\pgfsetbuttcap%
\pgfsetroundjoin%
\definecolor{currentfill}{rgb}{0.225863,0.330805,0.547314}%
\pgfsetfillcolor{currentfill}%
\pgfsetlinewidth{0.000000pt}%
\definecolor{currentstroke}{rgb}{0.906311,0.894855,0.098125}%
\pgfsetstrokecolor{currentstroke}%
\pgfsetdash{}{0pt}%
\pgfpathmoveto{\pgfqpoint{2.398894in}{3.332512in}}%
\pgfpathlineto{\pgfqpoint{2.264961in}{3.343707in}}%
\pgfpathlineto{\pgfqpoint{2.181129in}{3.174618in}}%
\pgfpathclose%
\pgfusepath{fill}%
\end{pgfscope}%
\begin{pgfscope}%
\pgfpathrectangle{\pgfqpoint{0.680860in}{0.078740in}}{\pgfqpoint{7.842520in}{7.842520in}}%
\pgfusepath{clip}%
\pgfsetbuttcap%
\pgfsetroundjoin%
\definecolor{currentfill}{rgb}{0.123444,0.636809,0.528763}%
\pgfsetfillcolor{currentfill}%
\pgfsetlinewidth{0.000000pt}%
\definecolor{currentstroke}{rgb}{0.916242,0.896091,0.100717}%
\pgfsetstrokecolor{currentstroke}%
\pgfsetdash{}{0pt}%
\pgfpathmoveto{\pgfqpoint{5.308503in}{4.618780in}}%
\pgfpathlineto{\pgfqpoint{5.386170in}{4.595288in}}%
\pgfpathlineto{\pgfqpoint{5.246102in}{4.575513in}}%
\pgfpathclose%
\pgfusepath{fill}%
\end{pgfscope}%
\begin{pgfscope}%
\pgfpathrectangle{\pgfqpoint{0.680860in}{0.078740in}}{\pgfqpoint{7.842520in}{7.842520in}}%
\pgfusepath{clip}%
\pgfsetbuttcap%
\pgfsetroundjoin%
\definecolor{currentfill}{rgb}{0.136408,0.541173,0.554483}%
\pgfsetfillcolor{currentfill}%
\pgfsetlinewidth{0.000000pt}%
\definecolor{currentstroke}{rgb}{0.926106,0.897330,0.104071}%
\pgfsetstrokecolor{currentstroke}%
\pgfsetdash{}{0pt}%
\pgfpathmoveto{\pgfqpoint{7.668081in}{4.113125in}}%
\pgfpathlineto{\pgfqpoint{7.523042in}{4.106176in}}%
\pgfpathlineto{\pgfqpoint{7.607030in}{4.318240in}}%
\pgfpathclose%
\pgfusepath{fill}%
\end{pgfscope}%
\begin{pgfscope}%
\pgfpathrectangle{\pgfqpoint{0.680860in}{0.078740in}}{\pgfqpoint{7.842520in}{7.842520in}}%
\pgfusepath{clip}%
\pgfsetbuttcap%
\pgfsetroundjoin%
\definecolor{currentfill}{rgb}{0.120081,0.622161,0.534946}%
\pgfsetfillcolor{currentfill}%
\pgfsetlinewidth{0.000000pt}%
\definecolor{currentstroke}{rgb}{0.935904,0.898570,0.108131}%
\pgfsetstrokecolor{currentstroke}%
\pgfsetdash{}{0pt}%
\pgfpathmoveto{\pgfqpoint{6.245023in}{4.582280in}}%
\pgfpathlineto{\pgfqpoint{6.317148in}{4.482731in}}%
\pgfpathlineto{\pgfqpoint{6.102393in}{4.561046in}}%
\pgfpathclose%
\pgfusepath{fill}%
\end{pgfscope}%
\begin{pgfscope}%
\pgfpathrectangle{\pgfqpoint{0.680860in}{0.078740in}}{\pgfqpoint{7.842520in}{7.842520in}}%
\pgfusepath{clip}%
\pgfsetbuttcap%
\pgfsetroundjoin%
\definecolor{currentfill}{rgb}{0.127568,0.566949,0.550556}%
\pgfsetfillcolor{currentfill}%
\pgfsetlinewidth{0.000000pt}%
\definecolor{currentstroke}{rgb}{0.945636,0.899815,0.112838}%
\pgfsetstrokecolor{currentstroke}%
\pgfsetdash{}{0pt}%
\pgfpathmoveto{\pgfqpoint{3.877187in}{4.345262in}}%
\pgfpathlineto{\pgfqpoint{3.794749in}{4.268363in}}%
\pgfpathlineto{\pgfqpoint{3.931777in}{4.281986in}}%
\pgfpathclose%
\pgfusepath{fill}%
\end{pgfscope}%
\begin{pgfscope}%
\pgfpathrectangle{\pgfqpoint{0.680860in}{0.078740in}}{\pgfqpoint{7.842520in}{7.842520in}}%
\pgfusepath{clip}%
\pgfsetbuttcap%
\pgfsetroundjoin%
\definecolor{currentfill}{rgb}{0.166617,0.463708,0.558119}%
\pgfsetfillcolor{currentfill}%
\pgfsetlinewidth{0.000000pt}%
\definecolor{currentstroke}{rgb}{0.955300,0.901065,0.118128}%
\pgfsetstrokecolor{currentstroke}%
\pgfsetdash{}{0pt}%
\pgfpathmoveto{\pgfqpoint{3.002486in}{3.917372in}}%
\pgfpathlineto{\pgfqpoint{2.918941in}{3.785155in}}%
\pgfpathlineto{\pgfqpoint{3.054130in}{3.786067in}}%
\pgfpathclose%
\pgfusepath{fill}%
\end{pgfscope}%
\begin{pgfscope}%
\pgfpathrectangle{\pgfqpoint{0.680860in}{0.078740in}}{\pgfqpoint{7.842520in}{7.842520in}}%
\pgfusepath{clip}%
\pgfsetbuttcap%
\pgfsetroundjoin%
\definecolor{currentfill}{rgb}{0.121380,0.629492,0.531973}%
\pgfsetfillcolor{currentfill}%
\pgfsetlinewidth{0.000000pt}%
\definecolor{currentstroke}{rgb}{0.964894,0.902323,0.123941}%
\pgfsetstrokecolor{currentstroke}%
\pgfsetdash{}{0pt}%
\pgfpathmoveto{\pgfqpoint{5.960494in}{4.541766in}}%
\pgfpathlineto{\pgfqpoint{5.886079in}{4.613572in}}%
\pgfpathlineto{\pgfqpoint{6.102393in}{4.561046in}}%
\pgfpathclose%
\pgfusepath{fill}%
\end{pgfscope}%
\begin{pgfscope}%
\pgfpathrectangle{\pgfqpoint{0.680860in}{0.078740in}}{\pgfqpoint{7.842520in}{7.842520in}}%
\pgfusepath{clip}%
\pgfsetbuttcap%
\pgfsetroundjoin%
\definecolor{currentfill}{rgb}{0.195860,0.395433,0.555276}%
\pgfsetfillcolor{currentfill}%
\pgfsetlinewidth{0.000000pt}%
\definecolor{currentstroke}{rgb}{0.974417,0.903590,0.130215}%
\pgfsetstrokecolor{currentstroke}%
\pgfsetdash{}{0pt}%
\pgfpathmoveto{\pgfqpoint{2.482641in}{3.495131in}}%
\pgfpathlineto{\pgfqpoint{2.616974in}{3.488103in}}%
\pgfpathlineto{\pgfqpoint{2.700633in}{3.642806in}}%
\pgfpathclose%
\pgfusepath{fill}%
\end{pgfscope}%
\begin{pgfscope}%
\pgfpathrectangle{\pgfqpoint{0.680860in}{0.078740in}}{\pgfqpoint{7.842520in}{7.842520in}}%
\pgfusepath{clip}%
\pgfsetbuttcap%
\pgfsetroundjoin%
\definecolor{currentfill}{rgb}{0.162142,0.474838,0.558140}%
\pgfsetfillcolor{currentfill}%
\pgfsetlinewidth{0.000000pt}%
\definecolor{currentstroke}{rgb}{0.983868,0.904867,0.136897}%
\pgfsetstrokecolor{currentstroke}%
\pgfsetdash{}{0pt}%
\pgfpathmoveto{\pgfqpoint{3.054130in}{3.786067in}}%
\pgfpathlineto{\pgfqpoint{3.137549in}{3.920553in}}%
\pgfpathlineto{\pgfqpoint{3.002486in}{3.917372in}}%
\pgfpathclose%
\pgfusepath{fill}%
\end{pgfscope}%
\begin{pgfscope}%
\pgfpathrectangle{\pgfqpoint{0.680860in}{0.078740in}}{\pgfqpoint{7.842520in}{7.842520in}}%
\pgfusepath{clip}%
\pgfsetbuttcap%
\pgfsetroundjoin%
\definecolor{currentfill}{rgb}{0.119738,0.603785,0.541400}%
\pgfsetfillcolor{currentfill}%
\pgfsetlinewidth{0.000000pt}%
\definecolor{currentstroke}{rgb}{0.993248,0.906157,0.143936}%
\pgfsetstrokecolor{currentstroke}%
\pgfsetdash{}{0pt}%
\pgfpathmoveto{\pgfqpoint{6.816173in}{4.406230in}}%
\pgfpathlineto{\pgfqpoint{6.672785in}{4.391456in}}%
\pgfpathlineto{\pgfqpoint{6.747468in}{4.541222in}}%
\pgfpathclose%
\pgfusepath{fill}%
\end{pgfscope}%
\begin{pgfscope}%
\pgfpathrectangle{\pgfqpoint{0.680860in}{0.078740in}}{\pgfqpoint{7.842520in}{7.842520in}}%
\pgfusepath{clip}%
\pgfsetbuttcap%
\pgfsetroundjoin%
\definecolor{currentfill}{rgb}{0.140536,0.530132,0.555659}%
\pgfsetfillcolor{currentfill}%
\pgfsetlinewidth{0.000000pt}%
\definecolor{currentstroke}{rgb}{0.267004,0.004874,0.329415}%
\pgfsetstrokecolor{currentstroke}%
\pgfsetdash{}{0pt}%
\pgfpathmoveto{\pgfqpoint{3.575515in}{4.163831in}}%
\pgfpathlineto{\pgfqpoint{3.439558in}{4.154450in}}%
\pgfpathlineto{\pgfqpoint{3.492525in}{4.055291in}}%
\pgfpathclose%
\pgfusepath{fill}%
\end{pgfscope}%
\begin{pgfscope}%
\pgfpathrectangle{\pgfqpoint{0.680860in}{0.078740in}}{\pgfqpoint{7.842520in}{7.842520in}}%
\pgfusepath{clip}%
\pgfsetbuttcap%
\pgfsetroundjoin%
\definecolor{currentfill}{rgb}{0.143343,0.522773,0.556295}%
\pgfsetfillcolor{currentfill}%
\pgfsetlinewidth{0.000000pt}%
\definecolor{currentstroke}{rgb}{0.268510,0.009605,0.335427}%
\pgfsetstrokecolor{currentstroke}%
\pgfsetdash{}{0pt}%
\pgfpathmoveto{\pgfqpoint{3.492525in}{4.055291in}}%
\pgfpathlineto{\pgfqpoint{3.439558in}{4.154450in}}%
\pgfpathlineto{\pgfqpoint{3.356424in}{4.047338in}}%
\pgfpathclose%
\pgfusepath{fill}%
\end{pgfscope}%
\begin{pgfscope}%
\pgfpathrectangle{\pgfqpoint{0.680860in}{0.078740in}}{\pgfqpoint{7.842520in}{7.842520in}}%
\pgfusepath{clip}%
\pgfsetbuttcap%
\pgfsetroundjoin%
\definecolor{currentfill}{rgb}{0.121831,0.589055,0.545623}%
\pgfsetfillcolor{currentfill}%
\pgfsetlinewidth{0.000000pt}%
\definecolor{currentstroke}{rgb}{0.269944,0.014625,0.341379}%
\pgfsetstrokecolor{currentstroke}%
\pgfsetdash{}{0pt}%
\pgfpathmoveto{\pgfqpoint{6.960295in}{4.422796in}}%
\pgfpathlineto{\pgfqpoint{7.105170in}{4.441224in}}%
\pgfpathlineto{\pgfqpoint{7.026844in}{4.265011in}}%
\pgfpathclose%
\pgfusepath{fill}%
\end{pgfscope}%
\begin{pgfscope}%
\pgfpathrectangle{\pgfqpoint{0.680860in}{0.078740in}}{\pgfqpoint{7.842520in}{7.842520in}}%
\pgfusepath{clip}%
\pgfsetbuttcap%
\pgfsetroundjoin%
\definecolor{currentfill}{rgb}{0.179019,0.433756,0.557430}%
\pgfsetfillcolor{currentfill}%
\pgfsetlinewidth{0.000000pt}%
\definecolor{currentstroke}{rgb}{0.271305,0.019942,0.347269}%
\pgfsetstrokecolor{currentstroke}%
\pgfsetdash{}{0pt}%
\pgfpathmoveto{\pgfqpoint{2.700633in}{3.642806in}}%
\pgfpathlineto{\pgfqpoint{2.835386in}{3.639831in}}%
\pgfpathlineto{\pgfqpoint{2.918941in}{3.785155in}}%
\pgfpathclose%
\pgfusepath{fill}%
\end{pgfscope}%
\begin{pgfscope}%
\pgfpathrectangle{\pgfqpoint{0.680860in}{0.078740in}}{\pgfqpoint{7.842520in}{7.842520in}}%
\pgfusepath{clip}%
\pgfsetbuttcap%
\pgfsetroundjoin%
\definecolor{currentfill}{rgb}{0.123444,0.636809,0.528763}%
\pgfsetfillcolor{currentfill}%
\pgfsetlinewidth{0.000000pt}%
\definecolor{currentstroke}{rgb}{0.272594,0.025563,0.353093}%
\pgfsetstrokecolor{currentstroke}%
\pgfsetdash{}{0pt}%
\pgfpathmoveto{\pgfqpoint{5.308503in}{4.618780in}}%
\pgfpathlineto{\pgfqpoint{5.246102in}{4.575513in}}%
\pgfpathlineto{\pgfqpoint{5.168036in}{4.596052in}}%
\pgfpathclose%
\pgfusepath{fill}%
\end{pgfscope}%
\begin{pgfscope}%
\pgfpathrectangle{\pgfqpoint{0.680860in}{0.078740in}}{\pgfqpoint{7.842520in}{7.842520in}}%
\pgfusepath{clip}%
\pgfsetbuttcap%
\pgfsetroundjoin%
\definecolor{currentfill}{rgb}{0.250425,0.274290,0.533103}%
\pgfsetfillcolor{currentfill}%
\pgfsetlinewidth{0.000000pt}%
\definecolor{currentstroke}{rgb}{0.273809,0.031497,0.358853}%
\pgfsetstrokecolor{currentstroke}%
\pgfsetdash{}{0pt}%
\pgfpathmoveto{\pgfqpoint{1.963644in}{3.015899in}}%
\pgfpathlineto{\pgfqpoint{2.097340in}{2.996859in}}%
\pgfpathlineto{\pgfqpoint{2.181129in}{3.174618in}}%
\pgfpathclose%
\pgfusepath{fill}%
\end{pgfscope}%
\begin{pgfscope}%
\pgfpathrectangle{\pgfqpoint{0.680860in}{0.078740in}}{\pgfqpoint{7.842520in}{7.842520in}}%
\pgfusepath{clip}%
\pgfsetbuttcap%
\pgfsetroundjoin%
\definecolor{currentfill}{rgb}{0.278012,0.180367,0.486697}%
\pgfsetfillcolor{currentfill}%
\pgfsetlinewidth{0.000000pt}%
\definecolor{currentstroke}{rgb}{0.274952,0.037752,0.364543}%
\pgfsetstrokecolor{currentstroke}%
\pgfsetdash{}{0pt}%
\pgfpathmoveto{\pgfqpoint{1.795861in}{2.646463in}}%
\pgfpathlineto{\pgfqpoint{1.746392in}{2.857738in}}%
\pgfpathlineto{\pgfqpoint{1.662351in}{2.673759in}}%
\pgfpathclose%
\pgfusepath{fill}%
\end{pgfscope}%
\begin{pgfscope}%
\pgfpathrectangle{\pgfqpoint{0.680860in}{0.078740in}}{\pgfqpoint{7.842520in}{7.842520in}}%
\pgfusepath{clip}%
\pgfsetbuttcap%
\pgfsetroundjoin%
\definecolor{currentfill}{rgb}{0.123444,0.636809,0.528763}%
\pgfsetfillcolor{currentfill}%
\pgfsetlinewidth{0.000000pt}%
\definecolor{currentstroke}{rgb}{0.276022,0.044167,0.370164}%
\pgfsetstrokecolor{currentstroke}%
\pgfsetdash{}{0pt}%
\pgfpathmoveto{\pgfqpoint{5.168036in}{4.596052in}}%
\pgfpathlineto{\pgfqpoint{5.028273in}{4.575333in}}%
\pgfpathlineto{\pgfqpoint{4.949132in}{4.574912in}}%
\pgfpathclose%
\pgfusepath{fill}%
\end{pgfscope}%
\begin{pgfscope}%
\pgfpathrectangle{\pgfqpoint{0.680860in}{0.078740in}}{\pgfqpoint{7.842520in}{7.842520in}}%
\pgfusepath{clip}%
\pgfsetbuttcap%
\pgfsetroundjoin%
\definecolor{currentfill}{rgb}{0.123463,0.581687,0.547445}%
\pgfsetfillcolor{currentfill}%
\pgfsetlinewidth{0.000000pt}%
\definecolor{currentstroke}{rgb}{0.277018,0.050344,0.375715}%
\pgfsetstrokecolor{currentstroke}%
\pgfsetdash{}{0pt}%
\pgfpathmoveto{\pgfqpoint{7.105170in}{4.441224in}}%
\pgfpathlineto{\pgfqpoint{7.315451in}{4.288200in}}%
\pgfpathlineto{\pgfqpoint{7.170785in}{4.275781in}}%
\pgfpathclose%
\pgfusepath{fill}%
\end{pgfscope}%
\begin{pgfscope}%
\pgfpathrectangle{\pgfqpoint{0.680860in}{0.078740in}}{\pgfqpoint{7.842520in}{7.842520in}}%
\pgfusepath{clip}%
\pgfsetbuttcap%
\pgfsetroundjoin%
\definecolor{currentfill}{rgb}{0.275191,0.194905,0.496005}%
\pgfsetfillcolor{currentfill}%
\pgfsetlinewidth{0.000000pt}%
\definecolor{currentstroke}{rgb}{0.277941,0.056324,0.381191}%
\pgfsetstrokecolor{currentstroke}%
\pgfsetdash{}{0pt}%
\pgfpathmoveto{\pgfqpoint{1.879746in}{2.834351in}}%
\pgfpathlineto{\pgfqpoint{1.746392in}{2.857738in}}%
\pgfpathlineto{\pgfqpoint{1.795861in}{2.646463in}}%
\pgfpathclose%
\pgfusepath{fill}%
\end{pgfscope}%
\begin{pgfscope}%
\pgfpathrectangle{\pgfqpoint{0.680860in}{0.078740in}}{\pgfqpoint{7.842520in}{7.842520in}}%
\pgfusepath{clip}%
\pgfsetbuttcap%
\pgfsetroundjoin%
\definecolor{currentfill}{rgb}{0.120638,0.625828,0.533488}%
\pgfsetfillcolor{currentfill}%
\pgfsetlinewidth{0.000000pt}%
\definecolor{currentstroke}{rgb}{0.278791,0.062145,0.386592}%
\pgfsetstrokecolor{currentstroke}%
\pgfsetdash{}{0pt}%
\pgfpathmoveto{\pgfqpoint{6.459846in}{4.500310in}}%
\pgfpathlineto{\pgfqpoint{6.317148in}{4.482731in}}%
\pgfpathlineto{\pgfqpoint{6.388401in}{4.605543in}}%
\pgfpathclose%
\pgfusepath{fill}%
\end{pgfscope}%
\begin{pgfscope}%
\pgfpathrectangle{\pgfqpoint{0.680860in}{0.078740in}}{\pgfqpoint{7.842520in}{7.842520in}}%
\pgfusepath{clip}%
\pgfsetbuttcap%
\pgfsetroundjoin%
\definecolor{currentfill}{rgb}{0.210503,0.363727,0.552206}%
\pgfsetfillcolor{currentfill}%
\pgfsetlinewidth{0.000000pt}%
\definecolor{currentstroke}{rgb}{0.279566,0.067836,0.391917}%
\pgfsetstrokecolor{currentstroke}%
\pgfsetdash{}{0pt}%
\pgfpathmoveto{\pgfqpoint{2.482641in}{3.495131in}}%
\pgfpathlineto{\pgfqpoint{2.398894in}{3.332512in}}%
\pgfpathlineto{\pgfqpoint{2.533350in}{3.322223in}}%
\pgfpathclose%
\pgfusepath{fill}%
\end{pgfscope}%
\begin{pgfscope}%
\pgfpathrectangle{\pgfqpoint{0.680860in}{0.078740in}}{\pgfqpoint{7.842520in}{7.842520in}}%
\pgfusepath{clip}%
\pgfsetbuttcap%
\pgfsetroundjoin%
\definecolor{currentfill}{rgb}{0.153364,0.497000,0.557724}%
\pgfsetfillcolor{currentfill}%
\pgfsetlinewidth{0.000000pt}%
\definecolor{currentstroke}{rgb}{0.280267,0.073417,0.397163}%
\pgfsetstrokecolor{currentstroke}%
\pgfsetdash{}{0pt}%
\pgfpathmoveto{\pgfqpoint{3.137549in}{3.920553in}}%
\pgfpathlineto{\pgfqpoint{3.273189in}{3.925127in}}%
\pgfpathlineto{\pgfqpoint{3.356424in}{4.047338in}}%
\pgfpathclose%
\pgfusepath{fill}%
\end{pgfscope}%
\begin{pgfscope}%
\pgfpathrectangle{\pgfqpoint{0.680860in}{0.078740in}}{\pgfqpoint{7.842520in}{7.842520in}}%
\pgfusepath{clip}%
\pgfsetbuttcap%
\pgfsetroundjoin%
\definecolor{currentfill}{rgb}{0.120092,0.600104,0.542530}%
\pgfsetfillcolor{currentfill}%
\pgfsetlinewidth{0.000000pt}%
\definecolor{currentstroke}{rgb}{0.280894,0.078907,0.402329}%
\pgfsetstrokecolor{currentstroke}%
\pgfsetdash{}{0pt}%
\pgfpathmoveto{\pgfqpoint{4.371182in}{4.452473in}}%
\pgfpathlineto{\pgfqpoint{4.233259in}{4.435022in}}%
\pgfpathlineto{\pgfqpoint{4.151517in}{4.375085in}}%
\pgfpathclose%
\pgfusepath{fill}%
\end{pgfscope}%
\begin{pgfscope}%
\pgfpathrectangle{\pgfqpoint{0.680860in}{0.078740in}}{\pgfqpoint{7.842520in}{7.842520in}}%
\pgfusepath{clip}%
\pgfsetbuttcap%
\pgfsetroundjoin%
\definecolor{currentfill}{rgb}{0.132444,0.552216,0.553018}%
\pgfsetfillcolor{currentfill}%
\pgfsetlinewidth{0.000000pt}%
\definecolor{currentstroke}{rgb}{0.281446,0.084320,0.407414}%
\pgfsetstrokecolor{currentstroke}%
\pgfsetdash{}{0pt}%
\pgfpathmoveto{\pgfqpoint{3.575515in}{4.163831in}}%
\pgfpathlineto{\pgfqpoint{3.712080in}{4.174821in}}%
\pgfpathlineto{\pgfqpoint{3.794749in}{4.268363in}}%
\pgfpathclose%
\pgfusepath{fill}%
\end{pgfscope}%
\begin{pgfscope}%
\pgfpathrectangle{\pgfqpoint{0.680860in}{0.078740in}}{\pgfqpoint{7.842520in}{7.842520in}}%
\pgfusepath{clip}%
\pgfsetbuttcap%
\pgfsetroundjoin%
\definecolor{currentfill}{rgb}{0.229739,0.322361,0.545706}%
\pgfsetfillcolor{currentfill}%
\pgfsetlinewidth{0.000000pt}%
\definecolor{currentstroke}{rgb}{0.281924,0.089666,0.412415}%
\pgfsetstrokecolor{currentstroke}%
\pgfsetdash{}{0pt}%
\pgfpathmoveto{\pgfqpoint{2.398894in}{3.332512in}}%
\pgfpathlineto{\pgfqpoint{2.181129in}{3.174618in}}%
\pgfpathlineto{\pgfqpoint{2.315193in}{3.159960in}}%
\pgfpathclose%
\pgfusepath{fill}%
\end{pgfscope}%
\begin{pgfscope}%
\pgfpathrectangle{\pgfqpoint{0.680860in}{0.078740in}}{\pgfqpoint{7.842520in}{7.842520in}}%
\pgfusepath{clip}%
\pgfsetbuttcap%
\pgfsetroundjoin%
\definecolor{currentfill}{rgb}{0.204903,0.375746,0.553533}%
\pgfsetfillcolor{currentfill}%
\pgfsetlinewidth{0.000000pt}%
\definecolor{currentstroke}{rgb}{0.282327,0.094955,0.417331}%
\pgfsetstrokecolor{currentstroke}%
\pgfsetdash{}{0pt}%
\pgfpathmoveto{\pgfqpoint{2.533350in}{3.322223in}}%
\pgfpathlineto{\pgfqpoint{2.616974in}{3.488103in}}%
\pgfpathlineto{\pgfqpoint{2.482641in}{3.495131in}}%
\pgfpathclose%
\pgfusepath{fill}%
\end{pgfscope}%
\begin{pgfscope}%
\pgfpathrectangle{\pgfqpoint{0.680860in}{0.078740in}}{\pgfqpoint{7.842520in}{7.842520in}}%
\pgfusepath{clip}%
\pgfsetbuttcap%
\pgfsetroundjoin%
\definecolor{currentfill}{rgb}{0.119699,0.618490,0.536347}%
\pgfsetfillcolor{currentfill}%
\pgfsetlinewidth{0.000000pt}%
\definecolor{currentstroke}{rgb}{0.282656,0.100196,0.422160}%
\pgfsetstrokecolor{currentstroke}%
\pgfsetdash{}{0pt}%
\pgfpathmoveto{\pgfqpoint{4.452294in}{4.494054in}}%
\pgfpathlineto{\pgfqpoint{4.509769in}{4.471817in}}%
\pgfpathlineto{\pgfqpoint{4.590636in}{4.512586in}}%
\pgfpathclose%
\pgfusepath{fill}%
\end{pgfscope}%
\begin{pgfscope}%
\pgfpathrectangle{\pgfqpoint{0.680860in}{0.078740in}}{\pgfqpoint{7.842520in}{7.842520in}}%
\pgfusepath{clip}%
\pgfsetbuttcap%
\pgfsetroundjoin%
\definecolor{currentfill}{rgb}{0.265145,0.232956,0.516599}%
\pgfsetfillcolor{currentfill}%
\pgfsetlinewidth{0.000000pt}%
\definecolor{currentstroke}{rgb}{0.282910,0.105393,0.426902}%
\pgfsetstrokecolor{currentstroke}%
\pgfsetdash{}{0pt}%
\pgfpathmoveto{\pgfqpoint{1.963644in}{3.015899in}}%
\pgfpathlineto{\pgfqpoint{1.879746in}{2.834351in}}%
\pgfpathlineto{\pgfqpoint{2.013586in}{2.811467in}}%
\pgfpathclose%
\pgfusepath{fill}%
\end{pgfscope}%
\begin{pgfscope}%
\pgfpathrectangle{\pgfqpoint{0.680860in}{0.078740in}}{\pgfqpoint{7.842520in}{7.842520in}}%
\pgfusepath{clip}%
\pgfsetbuttcap%
\pgfsetroundjoin%
\definecolor{currentfill}{rgb}{0.119699,0.618490,0.536347}%
\pgfsetfillcolor{currentfill}%
\pgfsetlinewidth{0.000000pt}%
\definecolor{currentstroke}{rgb}{0.283091,0.110553,0.431554}%
\pgfsetstrokecolor{currentstroke}%
\pgfsetdash{}{0pt}%
\pgfpathmoveto{\pgfqpoint{6.747468in}{4.541222in}}%
\pgfpathlineto{\pgfqpoint{6.672785in}{4.391456in}}%
\pgfpathlineto{\pgfqpoint{6.603281in}{4.519783in}}%
\pgfpathclose%
\pgfusepath{fill}%
\end{pgfscope}%
\begin{pgfscope}%
\pgfpathrectangle{\pgfqpoint{0.680860in}{0.078740in}}{\pgfqpoint{7.842520in}{7.842520in}}%
\pgfusepath{clip}%
\pgfsetbuttcap%
\pgfsetroundjoin%
\definecolor{currentfill}{rgb}{0.172719,0.448791,0.557885}%
\pgfsetfillcolor{currentfill}%
\pgfsetlinewidth{0.000000pt}%
\definecolor{currentstroke}{rgb}{0.283197,0.115680,0.436115}%
\pgfsetstrokecolor{currentstroke}%
\pgfsetdash{}{0pt}%
\pgfpathmoveto{\pgfqpoint{3.054130in}{3.786067in}}%
\pgfpathlineto{\pgfqpoint{2.918941in}{3.785155in}}%
\pgfpathlineto{\pgfqpoint{2.835386in}{3.639831in}}%
\pgfpathclose%
\pgfusepath{fill}%
\end{pgfscope}%
\begin{pgfscope}%
\pgfpathrectangle{\pgfqpoint{0.680860in}{0.078740in}}{\pgfqpoint{7.842520in}{7.842520in}}%
\pgfusepath{clip}%
\pgfsetbuttcap%
\pgfsetroundjoin%
\definecolor{currentfill}{rgb}{0.121380,0.629492,0.531973}%
\pgfsetfillcolor{currentfill}%
\pgfsetlinewidth{0.000000pt}%
\definecolor{currentstroke}{rgb}{0.283229,0.120777,0.440584}%
\pgfsetstrokecolor{currentstroke}%
\pgfsetdash{}{0pt}%
\pgfpathmoveto{\pgfqpoint{4.590636in}{4.512586in}}%
\pgfpathlineto{\pgfqpoint{4.729650in}{4.533049in}}%
\pgfpathlineto{\pgfqpoint{4.809723in}{4.553976in}}%
\pgfpathclose%
\pgfusepath{fill}%
\end{pgfscope}%
\begin{pgfscope}%
\pgfpathrectangle{\pgfqpoint{0.680860in}{0.078740in}}{\pgfqpoint{7.842520in}{7.842520in}}%
\pgfusepath{clip}%
\pgfsetbuttcap%
\pgfsetroundjoin%
\definecolor{currentfill}{rgb}{0.128729,0.563265,0.551229}%
\pgfsetfillcolor{currentfill}%
\pgfsetlinewidth{0.000000pt}%
\definecolor{currentstroke}{rgb}{0.283187,0.125848,0.444960}%
\pgfsetstrokecolor{currentstroke}%
\pgfsetdash{}{0pt}%
\pgfpathmoveto{\pgfqpoint{7.607030in}{4.318240in}}%
\pgfpathlineto{\pgfqpoint{7.523042in}{4.106176in}}%
\pgfpathlineto{\pgfqpoint{7.460861in}{4.302331in}}%
\pgfpathclose%
\pgfusepath{fill}%
\end{pgfscope}%
\begin{pgfscope}%
\pgfpathrectangle{\pgfqpoint{0.680860in}{0.078740in}}{\pgfqpoint{7.842520in}{7.842520in}}%
\pgfusepath{clip}%
\pgfsetbuttcap%
\pgfsetroundjoin%
\definecolor{currentfill}{rgb}{0.119483,0.614817,0.537692}%
\pgfsetfillcolor{currentfill}%
\pgfsetlinewidth{0.000000pt}%
\definecolor{currentstroke}{rgb}{0.283072,0.130895,0.449241}%
\pgfsetstrokecolor{currentstroke}%
\pgfsetdash{}{0pt}%
\pgfpathmoveto{\pgfqpoint{4.371182in}{4.452473in}}%
\pgfpathlineto{\pgfqpoint{4.509769in}{4.471817in}}%
\pgfpathlineto{\pgfqpoint{4.452294in}{4.494054in}}%
\pgfpathclose%
\pgfusepath{fill}%
\end{pgfscope}%
\begin{pgfscope}%
\pgfpathrectangle{\pgfqpoint{0.680860in}{0.078740in}}{\pgfqpoint{7.842520in}{7.842520in}}%
\pgfusepath{clip}%
\pgfsetbuttcap%
\pgfsetroundjoin%
\definecolor{currentfill}{rgb}{0.260571,0.246922,0.522828}%
\pgfsetfillcolor{currentfill}%
\pgfsetlinewidth{0.000000pt}%
\definecolor{currentstroke}{rgb}{0.282884,0.135920,0.453427}%
\pgfsetstrokecolor{currentstroke}%
\pgfsetdash{}{0pt}%
\pgfpathmoveto{\pgfqpoint{2.013586in}{2.811467in}}%
\pgfpathlineto{\pgfqpoint{2.097340in}{2.996859in}}%
\pgfpathlineto{\pgfqpoint{1.963644in}{3.015899in}}%
\pgfpathclose%
\pgfusepath{fill}%
\end{pgfscope}%
\begin{pgfscope}%
\pgfpathrectangle{\pgfqpoint{0.680860in}{0.078740in}}{\pgfqpoint{7.842520in}{7.842520in}}%
\pgfusepath{clip}%
\pgfsetbuttcap%
\pgfsetroundjoin%
\definecolor{currentfill}{rgb}{0.282290,0.145912,0.461510}%
\pgfsetfillcolor{currentfill}%
\pgfsetlinewidth{0.000000pt}%
\definecolor{currentstroke}{rgb}{0.282623,0.140926,0.457517}%
\pgfsetstrokecolor{currentstroke}%
\pgfsetdash{}{0pt}%
\pgfpathmoveto{\pgfqpoint{1.662351in}{2.673759in}}%
\pgfpathlineto{\pgfqpoint{1.578291in}{2.484708in}}%
\pgfpathlineto{\pgfqpoint{1.795861in}{2.646463in}}%
\pgfpathclose%
\pgfusepath{fill}%
\end{pgfscope}%
\begin{pgfscope}%
\pgfpathrectangle{\pgfqpoint{0.680860in}{0.078740in}}{\pgfqpoint{7.842520in}{7.842520in}}%
\pgfusepath{clip}%
\pgfsetbuttcap%
\pgfsetroundjoin%
\definecolor{currentfill}{rgb}{0.192357,0.403199,0.555836}%
\pgfsetfillcolor{currentfill}%
\pgfsetlinewidth{0.000000pt}%
\definecolor{currentstroke}{rgb}{0.282290,0.145912,0.461510}%
\pgfsetstrokecolor{currentstroke}%
\pgfsetdash{}{0pt}%
\pgfpathmoveto{\pgfqpoint{2.700633in}{3.642806in}}%
\pgfpathlineto{\pgfqpoint{2.616974in}{3.488103in}}%
\pgfpathlineto{\pgfqpoint{2.751845in}{3.482122in}}%
\pgfpathclose%
\pgfusepath{fill}%
\end{pgfscope}%
\begin{pgfscope}%
\pgfpathrectangle{\pgfqpoint{0.680860in}{0.078740in}}{\pgfqpoint{7.842520in}{7.842520in}}%
\pgfusepath{clip}%
\pgfsetbuttcap%
\pgfsetroundjoin%
\definecolor{currentfill}{rgb}{0.122606,0.585371,0.546557}%
\pgfsetfillcolor{currentfill}%
\pgfsetlinewidth{0.000000pt}%
\definecolor{currentstroke}{rgb}{0.281887,0.150881,0.465405}%
\pgfsetstrokecolor{currentstroke}%
\pgfsetdash{}{0pt}%
\pgfpathmoveto{\pgfqpoint{3.931777in}{4.281986in}}%
\pgfpathlineto{\pgfqpoint{4.151517in}{4.375085in}}%
\pgfpathlineto{\pgfqpoint{4.014034in}{4.359289in}}%
\pgfpathclose%
\pgfusepath{fill}%
\end{pgfscope}%
\begin{pgfscope}%
\pgfpathrectangle{\pgfqpoint{0.680860in}{0.078740in}}{\pgfqpoint{7.842520in}{7.842520in}}%
\pgfusepath{clip}%
\pgfsetbuttcap%
\pgfsetroundjoin%
\definecolor{currentfill}{rgb}{0.128087,0.647749,0.523491}%
\pgfsetfillcolor{currentfill}%
\pgfsetlinewidth{0.000000pt}%
\definecolor{currentstroke}{rgb}{0.281412,0.155834,0.469201}%
\pgfsetstrokecolor{currentstroke}%
\pgfsetdash{}{0pt}%
\pgfpathmoveto{\pgfqpoint{5.386170in}{4.595288in}}%
\pgfpathlineto{\pgfqpoint{5.308503in}{4.618780in}}%
\pgfpathlineto{\pgfqpoint{5.526942in}{4.617032in}}%
\pgfpathclose%
\pgfusepath{fill}%
\end{pgfscope}%
\begin{pgfscope}%
\pgfpathrectangle{\pgfqpoint{0.680860in}{0.078740in}}{\pgfqpoint{7.842520in}{7.842520in}}%
\pgfusepath{clip}%
\pgfsetbuttcap%
\pgfsetroundjoin%
\definecolor{currentfill}{rgb}{0.187231,0.414746,0.556547}%
\pgfsetfillcolor{currentfill}%
\pgfsetlinewidth{0.000000pt}%
\definecolor{currentstroke}{rgb}{0.280868,0.160771,0.472899}%
\pgfsetstrokecolor{currentstroke}%
\pgfsetdash{}{0pt}%
\pgfpathmoveto{\pgfqpoint{2.751845in}{3.482122in}}%
\pgfpathlineto{\pgfqpoint{2.835386in}{3.639831in}}%
\pgfpathlineto{\pgfqpoint{2.700633in}{3.642806in}}%
\pgfpathclose%
\pgfusepath{fill}%
\end{pgfscope}%
\begin{pgfscope}%
\pgfpathrectangle{\pgfqpoint{0.680860in}{0.078740in}}{\pgfqpoint{7.842520in}{7.842520in}}%
\pgfusepath{clip}%
\pgfsetbuttcap%
\pgfsetroundjoin%
\definecolor{currentfill}{rgb}{0.123444,0.636809,0.528763}%
\pgfsetfillcolor{currentfill}%
\pgfsetlinewidth{0.000000pt}%
\definecolor{currentstroke}{rgb}{0.280255,0.165693,0.476498}%
\pgfsetstrokecolor{currentstroke}%
\pgfsetdash{}{0pt}%
\pgfpathmoveto{\pgfqpoint{6.388401in}{4.605543in}}%
\pgfpathlineto{\pgfqpoint{6.317148in}{4.482731in}}%
\pgfpathlineto{\pgfqpoint{6.245023in}{4.582280in}}%
\pgfpathclose%
\pgfusepath{fill}%
\end{pgfscope}%
\begin{pgfscope}%
\pgfpathrectangle{\pgfqpoint{0.680860in}{0.078740in}}{\pgfqpoint{7.842520in}{7.842520in}}%
\pgfusepath{clip}%
\pgfsetbuttcap%
\pgfsetroundjoin%
\definecolor{currentfill}{rgb}{0.127568,0.566949,0.550556}%
\pgfsetfillcolor{currentfill}%
\pgfsetlinewidth{0.000000pt}%
\definecolor{currentstroke}{rgb}{0.279574,0.170599,0.479997}%
\pgfsetstrokecolor{currentstroke}%
\pgfsetdash{}{0pt}%
\pgfpathmoveto{\pgfqpoint{3.931777in}{4.281986in}}%
\pgfpathlineto{\pgfqpoint{3.794749in}{4.268363in}}%
\pgfpathlineto{\pgfqpoint{3.712080in}{4.174821in}}%
\pgfpathclose%
\pgfusepath{fill}%
\end{pgfscope}%
\begin{pgfscope}%
\pgfpathrectangle{\pgfqpoint{0.680860in}{0.078740in}}{\pgfqpoint{7.842520in}{7.842520in}}%
\pgfusepath{clip}%
\pgfsetbuttcap%
\pgfsetroundjoin%
\definecolor{currentfill}{rgb}{0.246811,0.283237,0.535941}%
\pgfsetfillcolor{currentfill}%
\pgfsetlinewidth{0.000000pt}%
\definecolor{currentstroke}{rgb}{0.278826,0.175490,0.483397}%
\pgfsetstrokecolor{currentstroke}%
\pgfsetdash{}{0pt}%
\pgfpathmoveto{\pgfqpoint{2.231536in}{2.978467in}}%
\pgfpathlineto{\pgfqpoint{2.181129in}{3.174618in}}%
\pgfpathlineto{\pgfqpoint{2.097340in}{2.996859in}}%
\pgfpathclose%
\pgfusepath{fill}%
\end{pgfscope}%
\begin{pgfscope}%
\pgfpathrectangle{\pgfqpoint{0.680860in}{0.078740in}}{\pgfqpoint{7.842520in}{7.842520in}}%
\pgfusepath{clip}%
\pgfsetbuttcap%
\pgfsetroundjoin%
\definecolor{currentfill}{rgb}{0.123444,0.636809,0.528763}%
\pgfsetfillcolor{currentfill}%
\pgfsetlinewidth{0.000000pt}%
\definecolor{currentstroke}{rgb}{0.278012,0.180367,0.486697}%
\pgfsetstrokecolor{currentstroke}%
\pgfsetdash{}{0pt}%
\pgfpathmoveto{\pgfqpoint{4.949132in}{4.574912in}}%
\pgfpathlineto{\pgfqpoint{4.809723in}{4.553976in}}%
\pgfpathlineto{\pgfqpoint{4.729650in}{4.533049in}}%
\pgfpathclose%
\pgfusepath{fill}%
\end{pgfscope}%
\begin{pgfscope}%
\pgfpathrectangle{\pgfqpoint{0.680860in}{0.078740in}}{\pgfqpoint{7.842520in}{7.842520in}}%
\pgfusepath{clip}%
\pgfsetbuttcap%
\pgfsetroundjoin%
\definecolor{currentfill}{rgb}{0.147607,0.511733,0.557049}%
\pgfsetfillcolor{currentfill}%
\pgfsetlinewidth{0.000000pt}%
\definecolor{currentstroke}{rgb}{0.277134,0.185228,0.489898}%
\pgfsetstrokecolor{currentstroke}%
\pgfsetdash{}{0pt}%
\pgfpathmoveto{\pgfqpoint{3.356424in}{4.047338in}}%
\pgfpathlineto{\pgfqpoint{3.273189in}{3.925127in}}%
\pgfpathlineto{\pgfqpoint{3.492525in}{4.055291in}}%
\pgfpathclose%
\pgfusepath{fill}%
\end{pgfscope}%
\begin{pgfscope}%
\pgfpathrectangle{\pgfqpoint{0.680860in}{0.078740in}}{\pgfqpoint{7.842520in}{7.842520in}}%
\pgfusepath{clip}%
\pgfsetbuttcap%
\pgfsetroundjoin%
\definecolor{currentfill}{rgb}{0.136408,0.541173,0.554483}%
\pgfsetfillcolor{currentfill}%
\pgfsetlinewidth{0.000000pt}%
\definecolor{currentstroke}{rgb}{0.276194,0.190074,0.493001}%
\pgfsetstrokecolor{currentstroke}%
\pgfsetdash{}{0pt}%
\pgfpathmoveto{\pgfqpoint{3.492525in}{4.055291in}}%
\pgfpathlineto{\pgfqpoint{3.712080in}{4.174821in}}%
\pgfpathlineto{\pgfqpoint{3.575515in}{4.163831in}}%
\pgfpathclose%
\pgfusepath{fill}%
\end{pgfscope}%
\begin{pgfscope}%
\pgfpathrectangle{\pgfqpoint{0.680860in}{0.078740in}}{\pgfqpoint{7.842520in}{7.842520in}}%
\pgfusepath{clip}%
\pgfsetbuttcap%
\pgfsetroundjoin%
\definecolor{currentfill}{rgb}{0.221989,0.339161,0.548752}%
\pgfsetfillcolor{currentfill}%
\pgfsetlinewidth{0.000000pt}%
\definecolor{currentstroke}{rgb}{0.275191,0.194905,0.496005}%
\pgfsetstrokecolor{currentstroke}%
\pgfsetdash{}{0pt}%
\pgfpathmoveto{\pgfqpoint{2.533350in}{3.322223in}}%
\pgfpathlineto{\pgfqpoint{2.398894in}{3.332512in}}%
\pgfpathlineto{\pgfqpoint{2.315193in}{3.159960in}}%
\pgfpathclose%
\pgfusepath{fill}%
\end{pgfscope}%
\begin{pgfscope}%
\pgfpathrectangle{\pgfqpoint{0.680860in}{0.078740in}}{\pgfqpoint{7.842520in}{7.842520in}}%
\pgfusepath{clip}%
\pgfsetbuttcap%
\pgfsetroundjoin%
\definecolor{currentfill}{rgb}{0.241237,0.296485,0.539709}%
\pgfsetfillcolor{currentfill}%
\pgfsetlinewidth{0.000000pt}%
\definecolor{currentstroke}{rgb}{0.274128,0.199721,0.498911}%
\pgfsetstrokecolor{currentstroke}%
\pgfsetdash{}{0pt}%
\pgfpathmoveto{\pgfqpoint{2.315193in}{3.159960in}}%
\pgfpathlineto{\pgfqpoint{2.181129in}{3.174618in}}%
\pgfpathlineto{\pgfqpoint{2.231536in}{2.978467in}}%
\pgfpathclose%
\pgfusepath{fill}%
\end{pgfscope}%
\begin{pgfscope}%
\pgfpathrectangle{\pgfqpoint{0.680860in}{0.078740in}}{\pgfqpoint{7.842520in}{7.842520in}}%
\pgfusepath{clip}%
\pgfsetbuttcap%
\pgfsetroundjoin%
\definecolor{currentfill}{rgb}{0.130067,0.651384,0.521608}%
\pgfsetfillcolor{currentfill}%
\pgfsetlinewidth{0.000000pt}%
\definecolor{currentstroke}{rgb}{0.273006,0.204520,0.501721}%
\pgfsetstrokecolor{currentstroke}%
\pgfsetdash{}{0pt}%
\pgfpathmoveto{\pgfqpoint{5.526942in}{4.617032in}}%
\pgfpathlineto{\pgfqpoint{5.668435in}{4.640824in}}%
\pgfpathlineto{\pgfqpoint{5.744345in}{4.591617in}}%
\pgfpathclose%
\pgfusepath{fill}%
\end{pgfscope}%
\begin{pgfscope}%
\pgfpathrectangle{\pgfqpoint{0.680860in}{0.078740in}}{\pgfqpoint{7.842520in}{7.842520in}}%
\pgfusepath{clip}%
\pgfsetbuttcap%
\pgfsetroundjoin%
\definecolor{currentfill}{rgb}{0.130067,0.651384,0.521608}%
\pgfsetfillcolor{currentfill}%
\pgfsetlinewidth{0.000000pt}%
\definecolor{currentstroke}{rgb}{0.271828,0.209303,0.504434}%
\pgfsetstrokecolor{currentstroke}%
\pgfsetdash{}{0pt}%
\pgfpathmoveto{\pgfqpoint{5.744345in}{4.591617in}}%
\pgfpathlineto{\pgfqpoint{5.668435in}{4.640824in}}%
\pgfpathlineto{\pgfqpoint{5.886079in}{4.613572in}}%
\pgfpathclose%
\pgfusepath{fill}%
\end{pgfscope}%
\begin{pgfscope}%
\pgfpathrectangle{\pgfqpoint{0.680860in}{0.078740in}}{\pgfqpoint{7.842520in}{7.842520in}}%
\pgfusepath{clip}%
\pgfsetbuttcap%
\pgfsetroundjoin%
\definecolor{currentfill}{rgb}{0.162142,0.474838,0.558140}%
\pgfsetfillcolor{currentfill}%
\pgfsetlinewidth{0.000000pt}%
\definecolor{currentstroke}{rgb}{0.270595,0.214069,0.507052}%
\pgfsetstrokecolor{currentstroke}%
\pgfsetdash{}{0pt}%
\pgfpathmoveto{\pgfqpoint{3.137549in}{3.920553in}}%
\pgfpathlineto{\pgfqpoint{3.054130in}{3.786067in}}%
\pgfpathlineto{\pgfqpoint{3.189891in}{3.788299in}}%
\pgfpathclose%
\pgfusepath{fill}%
\end{pgfscope}%
\begin{pgfscope}%
\pgfpathrectangle{\pgfqpoint{0.680860in}{0.078740in}}{\pgfqpoint{7.842520in}{7.842520in}}%
\pgfusepath{clip}%
\pgfsetbuttcap%
\pgfsetroundjoin%
\definecolor{currentfill}{rgb}{0.157729,0.485932,0.558013}%
\pgfsetfillcolor{currentfill}%
\pgfsetlinewidth{0.000000pt}%
\definecolor{currentstroke}{rgb}{0.269308,0.218818,0.509577}%
\pgfsetstrokecolor{currentstroke}%
\pgfsetdash{}{0pt}%
\pgfpathmoveto{\pgfqpoint{3.189891in}{3.788299in}}%
\pgfpathlineto{\pgfqpoint{3.273189in}{3.925127in}}%
\pgfpathlineto{\pgfqpoint{3.137549in}{3.920553in}}%
\pgfpathclose%
\pgfusepath{fill}%
\end{pgfscope}%
\begin{pgfscope}%
\pgfpathrectangle{\pgfqpoint{0.680860in}{0.078740in}}{\pgfqpoint{7.842520in}{7.842520in}}%
\pgfusepath{clip}%
\pgfsetbuttcap%
\pgfsetroundjoin%
\definecolor{currentfill}{rgb}{0.282884,0.135920,0.453427}%
\pgfsetfillcolor{currentfill}%
\pgfsetlinewidth{0.000000pt}%
\definecolor{currentstroke}{rgb}{0.267968,0.223549,0.512008}%
\pgfsetstrokecolor{currentstroke}%
\pgfsetdash{}{0pt}%
\pgfpathmoveto{\pgfqpoint{1.578291in}{2.484708in}}%
\pgfpathlineto{\pgfqpoint{1.711964in}{2.453359in}}%
\pgfpathlineto{\pgfqpoint{1.795861in}{2.646463in}}%
\pgfpathclose%
\pgfusepath{fill}%
\end{pgfscope}%
\begin{pgfscope}%
\pgfpathrectangle{\pgfqpoint{0.680860in}{0.078740in}}{\pgfqpoint{7.842520in}{7.842520in}}%
\pgfusepath{clip}%
\pgfsetbuttcap%
\pgfsetroundjoin%
\definecolor{currentfill}{rgb}{0.277134,0.185228,0.489898}%
\pgfsetfillcolor{currentfill}%
\pgfsetlinewidth{0.000000pt}%
\definecolor{currentstroke}{rgb}{0.266580,0.228262,0.514349}%
\pgfsetstrokecolor{currentstroke}%
\pgfsetdash{}{0pt}%
\pgfpathmoveto{\pgfqpoint{1.879746in}{2.834351in}}%
\pgfpathlineto{\pgfqpoint{1.795861in}{2.646463in}}%
\pgfpathlineto{\pgfqpoint{1.929849in}{2.619546in}}%
\pgfpathclose%
\pgfusepath{fill}%
\end{pgfscope}%
\begin{pgfscope}%
\pgfpathrectangle{\pgfqpoint{0.680860in}{0.078740in}}{\pgfqpoint{7.842520in}{7.842520in}}%
\pgfusepath{clip}%
\pgfsetbuttcap%
\pgfsetroundjoin%
\definecolor{currentfill}{rgb}{0.130067,0.651384,0.521608}%
\pgfsetfillcolor{currentfill}%
\pgfsetlinewidth{0.000000pt}%
\definecolor{currentstroke}{rgb}{0.265145,0.232956,0.516599}%
\pgfsetstrokecolor{currentstroke}%
\pgfsetdash{}{0pt}%
\pgfpathmoveto{\pgfqpoint{5.886079in}{4.613572in}}%
\pgfpathlineto{\pgfqpoint{6.028551in}{4.637580in}}%
\pgfpathlineto{\pgfqpoint{6.102393in}{4.561046in}}%
\pgfpathclose%
\pgfusepath{fill}%
\end{pgfscope}%
\begin{pgfscope}%
\pgfpathrectangle{\pgfqpoint{0.680860in}{0.078740in}}{\pgfqpoint{7.842520in}{7.842520in}}%
\pgfusepath{clip}%
\pgfsetbuttcap%
\pgfsetroundjoin%
\definecolor{currentfill}{rgb}{0.201239,0.383670,0.554294}%
\pgfsetfillcolor{currentfill}%
\pgfsetlinewidth{0.000000pt}%
\definecolor{currentstroke}{rgb}{0.263663,0.237631,0.518762}%
\pgfsetstrokecolor{currentstroke}%
\pgfsetdash{}{0pt}%
\pgfpathmoveto{\pgfqpoint{2.751845in}{3.482122in}}%
\pgfpathlineto{\pgfqpoint{2.616974in}{3.488103in}}%
\pgfpathlineto{\pgfqpoint{2.533350in}{3.322223in}}%
\pgfpathclose%
\pgfusepath{fill}%
\end{pgfscope}%
\begin{pgfscope}%
\pgfpathrectangle{\pgfqpoint{0.680860in}{0.078740in}}{\pgfqpoint{7.842520in}{7.842520in}}%
\pgfusepath{clip}%
\pgfsetbuttcap%
\pgfsetroundjoin%
\definecolor{currentfill}{rgb}{0.282656,0.100196,0.422160}%
\pgfsetfillcolor{currentfill}%
\pgfsetlinewidth{0.000000pt}%
\definecolor{currentstroke}{rgb}{0.262138,0.242286,0.520837}%
\pgfsetstrokecolor{currentstroke}%
\pgfsetdash{}{0pt}%
\pgfpathmoveto{\pgfqpoint{1.578291in}{2.484708in}}%
\pgfpathlineto{\pgfqpoint{1.494178in}{2.291711in}}%
\pgfpathlineto{\pgfqpoint{1.711964in}{2.453359in}}%
\pgfpathclose%
\pgfusepath{fill}%
\end{pgfscope}%
\begin{pgfscope}%
\pgfpathrectangle{\pgfqpoint{0.680860in}{0.078740in}}{\pgfqpoint{7.842520in}{7.842520in}}%
\pgfusepath{clip}%
\pgfsetbuttcap%
\pgfsetroundjoin%
\definecolor{currentfill}{rgb}{0.174274,0.445044,0.557792}%
\pgfsetfillcolor{currentfill}%
\pgfsetlinewidth{0.000000pt}%
\definecolor{currentstroke}{rgb}{0.260571,0.246922,0.522828}%
\pgfsetstrokecolor{currentstroke}%
\pgfsetdash{}{0pt}%
\pgfpathmoveto{\pgfqpoint{2.835386in}{3.639831in}}%
\pgfpathlineto{\pgfqpoint{2.970693in}{3.638043in}}%
\pgfpathlineto{\pgfqpoint{3.054130in}{3.786067in}}%
\pgfpathclose%
\pgfusepath{fill}%
\end{pgfscope}%
\begin{pgfscope}%
\pgfpathrectangle{\pgfqpoint{0.680860in}{0.078740in}}{\pgfqpoint{7.842520in}{7.842520in}}%
\pgfusepath{clip}%
\pgfsetbuttcap%
\pgfsetroundjoin%
\definecolor{currentfill}{rgb}{0.120081,0.622161,0.534946}%
\pgfsetfillcolor{currentfill}%
\pgfsetlinewidth{0.000000pt}%
\definecolor{currentstroke}{rgb}{0.258965,0.251537,0.524736}%
\pgfsetstrokecolor{currentstroke}%
\pgfsetdash{}{0pt}%
\pgfpathmoveto{\pgfqpoint{6.892428in}{4.564701in}}%
\pgfpathlineto{\pgfqpoint{6.960295in}{4.422796in}}%
\pgfpathlineto{\pgfqpoint{6.816173in}{4.406230in}}%
\pgfpathclose%
\pgfusepath{fill}%
\end{pgfscope}%
\begin{pgfscope}%
\pgfpathrectangle{\pgfqpoint{0.680860in}{0.078740in}}{\pgfqpoint{7.842520in}{7.842520in}}%
\pgfusepath{clip}%
\pgfsetbuttcap%
\pgfsetroundjoin%
\definecolor{currentfill}{rgb}{0.273006,0.204520,0.501721}%
\pgfsetfillcolor{currentfill}%
\pgfsetlinewidth{0.000000pt}%
\definecolor{currentstroke}{rgb}{0.257322,0.256130,0.526563}%
\pgfsetstrokecolor{currentstroke}%
\pgfsetdash{}{0pt}%
\pgfpathmoveto{\pgfqpoint{1.929849in}{2.619546in}}%
\pgfpathlineto{\pgfqpoint{2.013586in}{2.811467in}}%
\pgfpathlineto{\pgfqpoint{1.879746in}{2.834351in}}%
\pgfpathclose%
\pgfusepath{fill}%
\end{pgfscope}%
\begin{pgfscope}%
\pgfpathrectangle{\pgfqpoint{0.680860in}{0.078740in}}{\pgfqpoint{7.842520in}{7.842520in}}%
\pgfusepath{clip}%
\pgfsetbuttcap%
\pgfsetroundjoin%
\definecolor{currentfill}{rgb}{0.128087,0.647749,0.523491}%
\pgfsetfillcolor{currentfill}%
\pgfsetlinewidth{0.000000pt}%
\definecolor{currentstroke}{rgb}{0.255645,0.260703,0.528312}%
\pgfsetstrokecolor{currentstroke}%
\pgfsetdash{}{0pt}%
\pgfpathmoveto{\pgfqpoint{4.949132in}{4.574912in}}%
\pgfpathlineto{\pgfqpoint{5.089240in}{4.597867in}}%
\pgfpathlineto{\pgfqpoint{5.168036in}{4.596052in}}%
\pgfpathclose%
\pgfusepath{fill}%
\end{pgfscope}%
\begin{pgfscope}%
\pgfpathrectangle{\pgfqpoint{0.680860in}{0.078740in}}{\pgfqpoint{7.842520in}{7.842520in}}%
\pgfusepath{clip}%
\pgfsetbuttcap%
\pgfsetroundjoin%
\definecolor{currentfill}{rgb}{0.258965,0.251537,0.524736}%
\pgfsetfillcolor{currentfill}%
\pgfsetlinewidth{0.000000pt}%
\definecolor{currentstroke}{rgb}{0.253935,0.265254,0.529983}%
\pgfsetstrokecolor{currentstroke}%
\pgfsetdash{}{0pt}%
\pgfpathmoveto{\pgfqpoint{2.013586in}{2.811467in}}%
\pgfpathlineto{\pgfqpoint{2.231536in}{2.978467in}}%
\pgfpathlineto{\pgfqpoint{2.097340in}{2.996859in}}%
\pgfpathclose%
\pgfusepath{fill}%
\end{pgfscope}%
\begin{pgfscope}%
\pgfpathrectangle{\pgfqpoint{0.680860in}{0.078740in}}{\pgfqpoint{7.842520in}{7.842520in}}%
\pgfusepath{clip}%
\pgfsetbuttcap%
\pgfsetroundjoin%
\definecolor{currentfill}{rgb}{0.121380,0.629492,0.531973}%
\pgfsetfillcolor{currentfill}%
\pgfsetlinewidth{0.000000pt}%
\definecolor{currentstroke}{rgb}{0.252194,0.269783,0.531579}%
\pgfsetstrokecolor{currentstroke}%
\pgfsetdash{}{0pt}%
\pgfpathmoveto{\pgfqpoint{4.590636in}{4.512586in}}%
\pgfpathlineto{\pgfqpoint{4.509769in}{4.471817in}}%
\pgfpathlineto{\pgfqpoint{4.729650in}{4.533049in}}%
\pgfpathclose%
\pgfusepath{fill}%
\end{pgfscope}%
\begin{pgfscope}%
\pgfpathrectangle{\pgfqpoint{0.680860in}{0.078740in}}{\pgfqpoint{7.842520in}{7.842520in}}%
\pgfusepath{clip}%
\pgfsetbuttcap%
\pgfsetroundjoin%
\definecolor{currentfill}{rgb}{0.119512,0.607464,0.540218}%
\pgfsetfillcolor{currentfill}%
\pgfsetlinewidth{0.000000pt}%
\definecolor{currentstroke}{rgb}{0.250425,0.274290,0.533103}%
\pgfsetstrokecolor{currentstroke}%
\pgfsetdash{}{0pt}%
\pgfpathmoveto{\pgfqpoint{4.151517in}{4.375085in}}%
\pgfpathlineto{\pgfqpoint{4.289651in}{4.392719in}}%
\pgfpathlineto{\pgfqpoint{4.371182in}{4.452473in}}%
\pgfpathclose%
\pgfusepath{fill}%
\end{pgfscope}%
\begin{pgfscope}%
\pgfpathrectangle{\pgfqpoint{0.680860in}{0.078740in}}{\pgfqpoint{7.842520in}{7.842520in}}%
\pgfusepath{clip}%
\pgfsetbuttcap%
\pgfsetroundjoin%
\definecolor{currentfill}{rgb}{0.132268,0.655014,0.519661}%
\pgfsetfillcolor{currentfill}%
\pgfsetlinewidth{0.000000pt}%
\definecolor{currentstroke}{rgb}{0.248629,0.278775,0.534556}%
\pgfsetstrokecolor{currentstroke}%
\pgfsetdash{}{0pt}%
\pgfpathmoveto{\pgfqpoint{5.168036in}{4.596052in}}%
\pgfpathlineto{\pgfqpoint{5.089240in}{4.597867in}}%
\pgfpathlineto{\pgfqpoint{5.308503in}{4.618780in}}%
\pgfpathclose%
\pgfusepath{fill}%
\end{pgfscope}%
\begin{pgfscope}%
\pgfpathrectangle{\pgfqpoint{0.680860in}{0.078740in}}{\pgfqpoint{7.842520in}{7.842520in}}%
\pgfusepath{clip}%
\pgfsetbuttcap%
\pgfsetroundjoin%
\definecolor{currentfill}{rgb}{0.225863,0.330805,0.547314}%
\pgfsetfillcolor{currentfill}%
\pgfsetlinewidth{0.000000pt}%
\definecolor{currentstroke}{rgb}{0.246811,0.283237,0.535941}%
\pgfsetstrokecolor{currentstroke}%
\pgfsetdash{}{0pt}%
\pgfpathmoveto{\pgfqpoint{2.315193in}{3.159960in}}%
\pgfpathlineto{\pgfqpoint{2.449771in}{3.146096in}}%
\pgfpathlineto{\pgfqpoint{2.533350in}{3.322223in}}%
\pgfpathclose%
\pgfusepath{fill}%
\end{pgfscope}%
\begin{pgfscope}%
\pgfpathrectangle{\pgfqpoint{0.680860in}{0.078740in}}{\pgfqpoint{7.842520in}{7.842520in}}%
\pgfusepath{clip}%
\pgfsetbuttcap%
\pgfsetroundjoin%
\definecolor{currentfill}{rgb}{0.121831,0.589055,0.545623}%
\pgfsetfillcolor{currentfill}%
\pgfsetlinewidth{0.000000pt}%
\definecolor{currentstroke}{rgb}{0.244972,0.287675,0.537260}%
\pgfsetstrokecolor{currentstroke}%
\pgfsetdash{}{0pt}%
\pgfpathmoveto{\pgfqpoint{4.069441in}{4.297372in}}%
\pgfpathlineto{\pgfqpoint{4.151517in}{4.375085in}}%
\pgfpathlineto{\pgfqpoint{3.931777in}{4.281986in}}%
\pgfpathclose%
\pgfusepath{fill}%
\end{pgfscope}%
\begin{pgfscope}%
\pgfpathrectangle{\pgfqpoint{0.680860in}{0.078740in}}{\pgfqpoint{7.842520in}{7.842520in}}%
\pgfusepath{clip}%
\pgfsetbuttcap%
\pgfsetroundjoin%
\definecolor{currentfill}{rgb}{0.126326,0.644107,0.525311}%
\pgfsetfillcolor{currentfill}%
\pgfsetlinewidth{0.000000pt}%
\definecolor{currentstroke}{rgb}{0.243113,0.292092,0.538516}%
\pgfsetstrokecolor{currentstroke}%
\pgfsetdash{}{0pt}%
\pgfpathmoveto{\pgfqpoint{6.603281in}{4.519783in}}%
\pgfpathlineto{\pgfqpoint{6.459846in}{4.500310in}}%
\pgfpathlineto{\pgfqpoint{6.532545in}{4.630913in}}%
\pgfpathclose%
\pgfusepath{fill}%
\end{pgfscope}%
\begin{pgfscope}%
\pgfpathrectangle{\pgfqpoint{0.680860in}{0.078740in}}{\pgfqpoint{7.842520in}{7.842520in}}%
\pgfusepath{clip}%
\pgfsetbuttcap%
\pgfsetroundjoin%
\definecolor{currentfill}{rgb}{0.281412,0.155834,0.469201}%
\pgfsetfillcolor{currentfill}%
\pgfsetlinewidth{0.000000pt}%
\definecolor{currentstroke}{rgb}{0.241237,0.296485,0.539709}%
\pgfsetstrokecolor{currentstroke}%
\pgfsetdash{}{0pt}%
\pgfpathmoveto{\pgfqpoint{1.929849in}{2.619546in}}%
\pgfpathlineto{\pgfqpoint{1.795861in}{2.646463in}}%
\pgfpathlineto{\pgfqpoint{1.711964in}{2.453359in}}%
\pgfpathclose%
\pgfusepath{fill}%
\end{pgfscope}%
\begin{pgfscope}%
\pgfpathrectangle{\pgfqpoint{0.680860in}{0.078740in}}{\pgfqpoint{7.842520in}{7.842520in}}%
\pgfusepath{clip}%
\pgfsetbuttcap%
\pgfsetroundjoin%
\definecolor{currentfill}{rgb}{0.147607,0.511733,0.557049}%
\pgfsetfillcolor{currentfill}%
\pgfsetlinewidth{0.000000pt}%
\definecolor{currentstroke}{rgb}{0.239346,0.300855,0.540844}%
\pgfsetstrokecolor{currentstroke}%
\pgfsetdash{}{0pt}%
\pgfpathmoveto{\pgfqpoint{3.492525in}{4.055291in}}%
\pgfpathlineto{\pgfqpoint{3.273189in}{3.925127in}}%
\pgfpathlineto{\pgfqpoint{3.409417in}{3.931148in}}%
\pgfpathclose%
\pgfusepath{fill}%
\end{pgfscope}%
\begin{pgfscope}%
\pgfpathrectangle{\pgfqpoint{0.680860in}{0.078740in}}{\pgfqpoint{7.842520in}{7.842520in}}%
\pgfusepath{clip}%
\pgfsetbuttcap%
\pgfsetroundjoin%
\definecolor{currentfill}{rgb}{0.132268,0.655014,0.519661}%
\pgfsetfillcolor{currentfill}%
\pgfsetlinewidth{0.000000pt}%
\definecolor{currentstroke}{rgb}{0.237441,0.305202,0.541921}%
\pgfsetstrokecolor{currentstroke}%
\pgfsetdash{}{0pt}%
\pgfpathmoveto{\pgfqpoint{6.245023in}{4.582280in}}%
\pgfpathlineto{\pgfqpoint{6.102393in}{4.561046in}}%
\pgfpathlineto{\pgfqpoint{6.171777in}{4.663723in}}%
\pgfpathclose%
\pgfusepath{fill}%
\end{pgfscope}%
\begin{pgfscope}%
\pgfpathrectangle{\pgfqpoint{0.680860in}{0.078740in}}{\pgfqpoint{7.842520in}{7.842520in}}%
\pgfusepath{clip}%
\pgfsetbuttcap%
\pgfsetroundjoin%
\definecolor{currentfill}{rgb}{0.237441,0.305202,0.541921}%
\pgfsetfillcolor{currentfill}%
\pgfsetlinewidth{0.000000pt}%
\definecolor{currentstroke}{rgb}{0.235526,0.309527,0.542944}%
\pgfsetstrokecolor{currentstroke}%
\pgfsetdash{}{0pt}%
\pgfpathmoveto{\pgfqpoint{2.315193in}{3.159960in}}%
\pgfpathlineto{\pgfqpoint{2.231536in}{2.978467in}}%
\pgfpathlineto{\pgfqpoint{2.449771in}{3.146096in}}%
\pgfpathclose%
\pgfusepath{fill}%
\end{pgfscope}%
\begin{pgfscope}%
\pgfpathrectangle{\pgfqpoint{0.680860in}{0.078740in}}{\pgfqpoint{7.842520in}{7.842520in}}%
\pgfusepath{clip}%
\pgfsetbuttcap%
\pgfsetroundjoin%
\definecolor{currentfill}{rgb}{0.127568,0.566949,0.550556}%
\pgfsetfillcolor{currentfill}%
\pgfsetlinewidth{0.000000pt}%
\definecolor{currentstroke}{rgb}{0.233603,0.313828,0.543914}%
\pgfsetstrokecolor{currentstroke}%
\pgfsetdash{}{0pt}%
\pgfpathmoveto{\pgfqpoint{3.712080in}{4.174821in}}%
\pgfpathlineto{\pgfqpoint{3.849266in}{4.187481in}}%
\pgfpathlineto{\pgfqpoint{3.931777in}{4.281986in}}%
\pgfpathclose%
\pgfusepath{fill}%
\end{pgfscope}%
\begin{pgfscope}%
\pgfpathrectangle{\pgfqpoint{0.680860in}{0.078740in}}{\pgfqpoint{7.842520in}{7.842520in}}%
\pgfusepath{clip}%
\pgfsetbuttcap%
\pgfsetroundjoin%
\definecolor{currentfill}{rgb}{0.136408,0.541173,0.554483}%
\pgfsetfillcolor{currentfill}%
\pgfsetlinewidth{0.000000pt}%
\definecolor{currentstroke}{rgb}{0.231674,0.318106,0.544834}%
\pgfsetstrokecolor{currentstroke}%
\pgfsetdash{}{0pt}%
\pgfpathmoveto{\pgfqpoint{3.629230in}{4.064809in}}%
\pgfpathlineto{\pgfqpoint{3.712080in}{4.174821in}}%
\pgfpathlineto{\pgfqpoint{3.492525in}{4.055291in}}%
\pgfpathclose%
\pgfusepath{fill}%
\end{pgfscope}%
\begin{pgfscope}%
\pgfpathrectangle{\pgfqpoint{0.680860in}{0.078740in}}{\pgfqpoint{7.842520in}{7.842520in}}%
\pgfusepath{clip}%
\pgfsetbuttcap%
\pgfsetroundjoin%
\definecolor{currentfill}{rgb}{0.119699,0.618490,0.536347}%
\pgfsetfillcolor{currentfill}%
\pgfsetlinewidth{0.000000pt}%
\definecolor{currentstroke}{rgb}{0.229739,0.322361,0.545706}%
\pgfsetstrokecolor{currentstroke}%
\pgfsetdash{}{0pt}%
\pgfpathmoveto{\pgfqpoint{4.289651in}{4.392719in}}%
\pgfpathlineto{\pgfqpoint{4.509769in}{4.471817in}}%
\pgfpathlineto{\pgfqpoint{4.371182in}{4.452473in}}%
\pgfpathclose%
\pgfusepath{fill}%
\end{pgfscope}%
\begin{pgfscope}%
\pgfpathrectangle{\pgfqpoint{0.680860in}{0.078740in}}{\pgfqpoint{7.842520in}{7.842520in}}%
\pgfusepath{clip}%
\pgfsetbuttcap%
\pgfsetroundjoin%
\definecolor{currentfill}{rgb}{0.168126,0.459988,0.558082}%
\pgfsetfillcolor{currentfill}%
\pgfsetlinewidth{0.000000pt}%
\definecolor{currentstroke}{rgb}{0.227802,0.326594,0.546532}%
\pgfsetstrokecolor{currentstroke}%
\pgfsetdash{}{0pt}%
\pgfpathmoveto{\pgfqpoint{3.189891in}{3.788299in}}%
\pgfpathlineto{\pgfqpoint{3.054130in}{3.786067in}}%
\pgfpathlineto{\pgfqpoint{2.970693in}{3.638043in}}%
\pgfpathclose%
\pgfusepath{fill}%
\end{pgfscope}%
\begin{pgfscope}%
\pgfpathrectangle{\pgfqpoint{0.680860in}{0.078740in}}{\pgfqpoint{7.842520in}{7.842520in}}%
\pgfusepath{clip}%
\pgfsetbuttcap%
\pgfsetroundjoin%
\definecolor{currentfill}{rgb}{0.119423,0.611141,0.538982}%
\pgfsetfillcolor{currentfill}%
\pgfsetlinewidth{0.000000pt}%
\definecolor{currentstroke}{rgb}{0.225863,0.330805,0.547314}%
\pgfsetstrokecolor{currentstroke}%
\pgfsetdash{}{0pt}%
\pgfpathmoveto{\pgfqpoint{7.250814in}{4.461584in}}%
\pgfpathlineto{\pgfqpoint{7.315451in}{4.288200in}}%
\pgfpathlineto{\pgfqpoint{7.105170in}{4.441224in}}%
\pgfpathclose%
\pgfusepath{fill}%
\end{pgfscope}%
\begin{pgfscope}%
\pgfpathrectangle{\pgfqpoint{0.680860in}{0.078740in}}{\pgfqpoint{7.842520in}{7.842520in}}%
\pgfusepath{clip}%
\pgfsetbuttcap%
\pgfsetroundjoin%
\definecolor{currentfill}{rgb}{0.187231,0.414746,0.556547}%
\pgfsetfillcolor{currentfill}%
\pgfsetlinewidth{0.000000pt}%
\definecolor{currentstroke}{rgb}{0.223925,0.334994,0.548053}%
\pgfsetstrokecolor{currentstroke}%
\pgfsetdash{}{0pt}%
\pgfpathmoveto{\pgfqpoint{2.887264in}{3.477230in}}%
\pgfpathlineto{\pgfqpoint{2.835386in}{3.639831in}}%
\pgfpathlineto{\pgfqpoint{2.751845in}{3.482122in}}%
\pgfpathclose%
\pgfusepath{fill}%
\end{pgfscope}%
\begin{pgfscope}%
\pgfpathrectangle{\pgfqpoint{0.680860in}{0.078740in}}{\pgfqpoint{7.842520in}{7.842520in}}%
\pgfusepath{clip}%
\pgfsetbuttcap%
\pgfsetroundjoin%
\definecolor{currentfill}{rgb}{0.280894,0.078907,0.402329}%
\pgfsetfillcolor{currentfill}%
\pgfsetlinewidth{0.000000pt}%
\definecolor{currentstroke}{rgb}{0.221989,0.339161,0.548752}%
\pgfsetstrokecolor{currentstroke}%
\pgfsetdash{}{0pt}%
\pgfpathmoveto{\pgfqpoint{1.711964in}{2.453359in}}%
\pgfpathlineto{\pgfqpoint{1.494178in}{2.291711in}}%
\pgfpathlineto{\pgfqpoint{1.628022in}{2.256201in}}%
\pgfpathclose%
\pgfusepath{fill}%
\end{pgfscope}%
\begin{pgfscope}%
\pgfpathrectangle{\pgfqpoint{0.680860in}{0.078740in}}{\pgfqpoint{7.842520in}{7.842520in}}%
\pgfusepath{clip}%
\pgfsetbuttcap%
\pgfsetroundjoin%
\definecolor{currentfill}{rgb}{0.123444,0.636809,0.528763}%
\pgfsetfillcolor{currentfill}%
\pgfsetlinewidth{0.000000pt}%
\definecolor{currentstroke}{rgb}{0.220057,0.343307,0.549413}%
\pgfsetstrokecolor{currentstroke}%
\pgfsetdash{}{0pt}%
\pgfpathmoveto{\pgfqpoint{6.816173in}{4.406230in}}%
\pgfpathlineto{\pgfqpoint{6.747468in}{4.541222in}}%
\pgfpathlineto{\pgfqpoint{6.892428in}{4.564701in}}%
\pgfpathclose%
\pgfusepath{fill}%
\end{pgfscope}%
\begin{pgfscope}%
\pgfpathrectangle{\pgfqpoint{0.680860in}{0.078740in}}{\pgfqpoint{7.842520in}{7.842520in}}%
\pgfusepath{clip}%
\pgfsetbuttcap%
\pgfsetroundjoin%
\definecolor{currentfill}{rgb}{0.204903,0.375746,0.553533}%
\pgfsetfillcolor{currentfill}%
\pgfsetlinewidth{0.000000pt}%
\definecolor{currentstroke}{rgb}{0.218130,0.347432,0.550038}%
\pgfsetstrokecolor{currentstroke}%
\pgfsetdash{}{0pt}%
\pgfpathmoveto{\pgfqpoint{2.533350in}{3.322223in}}%
\pgfpathlineto{\pgfqpoint{2.668337in}{3.312877in}}%
\pgfpathlineto{\pgfqpoint{2.751845in}{3.482122in}}%
\pgfpathclose%
\pgfusepath{fill}%
\end{pgfscope}%
\begin{pgfscope}%
\pgfpathrectangle{\pgfqpoint{0.680860in}{0.078740in}}{\pgfqpoint{7.842520in}{7.842520in}}%
\pgfusepath{clip}%
\pgfsetbuttcap%
\pgfsetroundjoin%
\definecolor{currentfill}{rgb}{0.137339,0.662252,0.515571}%
\pgfsetfillcolor{currentfill}%
\pgfsetlinewidth{0.000000pt}%
\definecolor{currentstroke}{rgb}{0.216210,0.351535,0.550627}%
\pgfsetstrokecolor{currentstroke}%
\pgfsetdash{}{0pt}%
\pgfpathmoveto{\pgfqpoint{5.526942in}{4.617032in}}%
\pgfpathlineto{\pgfqpoint{5.308503in}{4.618780in}}%
\pgfpathlineto{\pgfqpoint{5.449688in}{4.643592in}}%
\pgfpathclose%
\pgfusepath{fill}%
\end{pgfscope}%
\begin{pgfscope}%
\pgfpathrectangle{\pgfqpoint{0.680860in}{0.078740in}}{\pgfqpoint{7.842520in}{7.842520in}}%
\pgfusepath{clip}%
\pgfsetbuttcap%
\pgfsetroundjoin%
\definecolor{currentfill}{rgb}{0.262138,0.242286,0.520837}%
\pgfsetfillcolor{currentfill}%
\pgfsetlinewidth{0.000000pt}%
\definecolor{currentstroke}{rgb}{0.214298,0.355619,0.551184}%
\pgfsetstrokecolor{currentstroke}%
\pgfsetdash{}{0pt}%
\pgfpathmoveto{\pgfqpoint{2.147916in}{2.789106in}}%
\pgfpathlineto{\pgfqpoint{2.231536in}{2.978467in}}%
\pgfpathlineto{\pgfqpoint{2.013586in}{2.811467in}}%
\pgfpathclose%
\pgfusepath{fill}%
\end{pgfscope}%
\begin{pgfscope}%
\pgfpathrectangle{\pgfqpoint{0.680860in}{0.078740in}}{\pgfqpoint{7.842520in}{7.842520in}}%
\pgfusepath{clip}%
\pgfsetbuttcap%
\pgfsetroundjoin%
\definecolor{currentfill}{rgb}{0.182256,0.426184,0.557120}%
\pgfsetfillcolor{currentfill}%
\pgfsetlinewidth{0.000000pt}%
\definecolor{currentstroke}{rgb}{0.212395,0.359683,0.551710}%
\pgfsetstrokecolor{currentstroke}%
\pgfsetdash{}{0pt}%
\pgfpathmoveto{\pgfqpoint{2.970693in}{3.638043in}}%
\pgfpathlineto{\pgfqpoint{2.835386in}{3.639831in}}%
\pgfpathlineto{\pgfqpoint{2.887264in}{3.477230in}}%
\pgfpathclose%
\pgfusepath{fill}%
\end{pgfscope}%
\begin{pgfscope}%
\pgfpathrectangle{\pgfqpoint{0.680860in}{0.078740in}}{\pgfqpoint{7.842520in}{7.842520in}}%
\pgfusepath{clip}%
\pgfsetbuttcap%
\pgfsetroundjoin%
\definecolor{currentfill}{rgb}{0.128087,0.647749,0.523491}%
\pgfsetfillcolor{currentfill}%
\pgfsetlinewidth{0.000000pt}%
\definecolor{currentstroke}{rgb}{0.210503,0.363727,0.552206}%
\pgfsetstrokecolor{currentstroke}%
\pgfsetdash{}{0pt}%
\pgfpathmoveto{\pgfqpoint{4.729650in}{4.533049in}}%
\pgfpathlineto{\pgfqpoint{4.869354in}{4.555516in}}%
\pgfpathlineto{\pgfqpoint{4.949132in}{4.574912in}}%
\pgfpathclose%
\pgfusepath{fill}%
\end{pgfscope}%
\begin{pgfscope}%
\pgfpathrectangle{\pgfqpoint{0.680860in}{0.078740in}}{\pgfqpoint{7.842520in}{7.842520in}}%
\pgfusepath{clip}%
\pgfsetbuttcap%
\pgfsetroundjoin%
\definecolor{currentfill}{rgb}{0.271828,0.209303,0.504434}%
\pgfsetfillcolor{currentfill}%
\pgfsetlinewidth{0.000000pt}%
\definecolor{currentstroke}{rgb}{0.208623,0.367752,0.552675}%
\pgfsetstrokecolor{currentstroke}%
\pgfsetdash{}{0pt}%
\pgfpathmoveto{\pgfqpoint{2.147916in}{2.789106in}}%
\pgfpathlineto{\pgfqpoint{2.013586in}{2.811467in}}%
\pgfpathlineto{\pgfqpoint{1.929849in}{2.619546in}}%
\pgfpathclose%
\pgfusepath{fill}%
\end{pgfscope}%
\begin{pgfscope}%
\pgfpathrectangle{\pgfqpoint{0.680860in}{0.078740in}}{\pgfqpoint{7.842520in}{7.842520in}}%
\pgfusepath{clip}%
\pgfsetbuttcap%
\pgfsetroundjoin%
\definecolor{currentfill}{rgb}{0.119738,0.603785,0.541400}%
\pgfsetfillcolor{currentfill}%
\pgfsetlinewidth{0.000000pt}%
\definecolor{currentstroke}{rgb}{0.206756,0.371758,0.553117}%
\pgfsetstrokecolor{currentstroke}%
\pgfsetdash{}{0pt}%
\pgfpathmoveto{\pgfqpoint{7.315451in}{4.288200in}}%
\pgfpathlineto{\pgfqpoint{7.397248in}{4.483949in}}%
\pgfpathlineto{\pgfqpoint{7.460861in}{4.302331in}}%
\pgfpathclose%
\pgfusepath{fill}%
\end{pgfscope}%
\begin{pgfscope}%
\pgfpathrectangle{\pgfqpoint{0.680860in}{0.078740in}}{\pgfqpoint{7.842520in}{7.842520in}}%
\pgfusepath{clip}%
\pgfsetbuttcap%
\pgfsetroundjoin%
\definecolor{currentfill}{rgb}{0.137339,0.662252,0.515571}%
\pgfsetfillcolor{currentfill}%
\pgfsetlinewidth{0.000000pt}%
\definecolor{currentstroke}{rgb}{0.204903,0.375746,0.553533}%
\pgfsetstrokecolor{currentstroke}%
\pgfsetdash{}{0pt}%
\pgfpathmoveto{\pgfqpoint{6.171777in}{4.663723in}}%
\pgfpathlineto{\pgfqpoint{6.102393in}{4.561046in}}%
\pgfpathlineto{\pgfqpoint{6.028551in}{4.637580in}}%
\pgfpathclose%
\pgfusepath{fill}%
\end{pgfscope}%
\begin{pgfscope}%
\pgfpathrectangle{\pgfqpoint{0.680860in}{0.078740in}}{\pgfqpoint{7.842520in}{7.842520in}}%
\pgfusepath{clip}%
\pgfsetbuttcap%
\pgfsetroundjoin%
\definecolor{currentfill}{rgb}{0.132268,0.655014,0.519661}%
\pgfsetfillcolor{currentfill}%
\pgfsetlinewidth{0.000000pt}%
\definecolor{currentstroke}{rgb}{0.203063,0.379716,0.553925}%
\pgfsetstrokecolor{currentstroke}%
\pgfsetdash{}{0pt}%
\pgfpathmoveto{\pgfqpoint{6.459846in}{4.500310in}}%
\pgfpathlineto{\pgfqpoint{6.388401in}{4.605543in}}%
\pgfpathlineto{\pgfqpoint{6.532545in}{4.630913in}}%
\pgfpathclose%
\pgfusepath{fill}%
\end{pgfscope}%
\begin{pgfscope}%
\pgfpathrectangle{\pgfqpoint{0.680860in}{0.078740in}}{\pgfqpoint{7.842520in}{7.842520in}}%
\pgfusepath{clip}%
\pgfsetbuttcap%
\pgfsetroundjoin%
\definecolor{currentfill}{rgb}{0.271305,0.019942,0.347269}%
\pgfsetfillcolor{currentfill}%
\pgfsetlinewidth{0.000000pt}%
\definecolor{currentstroke}{rgb}{0.201239,0.383670,0.554294}%
\pgfsetstrokecolor{currentstroke}%
\pgfsetdash{}{0pt}%
\pgfpathmoveto{\pgfqpoint{1.543995in}{2.056181in}}%
\pgfpathlineto{\pgfqpoint{1.494178in}{2.291711in}}%
\pgfpathlineto{\pgfqpoint{1.409974in}{2.095919in}}%
\pgfpathclose%
\pgfusepath{fill}%
\end{pgfscope}%
\begin{pgfscope}%
\pgfpathrectangle{\pgfqpoint{0.680860in}{0.078740in}}{\pgfqpoint{7.842520in}{7.842520in}}%
\pgfusepath{clip}%
\pgfsetbuttcap%
\pgfsetroundjoin%
\definecolor{currentfill}{rgb}{0.131172,0.555899,0.552459}%
\pgfsetfillcolor{currentfill}%
\pgfsetlinewidth{0.000000pt}%
\definecolor{currentstroke}{rgb}{0.199430,0.387607,0.554642}%
\pgfsetstrokecolor{currentstroke}%
\pgfsetdash{}{0pt}%
\pgfpathmoveto{\pgfqpoint{3.849266in}{4.187481in}}%
\pgfpathlineto{\pgfqpoint{3.712080in}{4.174821in}}%
\pgfpathlineto{\pgfqpoint{3.629230in}{4.064809in}}%
\pgfpathclose%
\pgfusepath{fill}%
\end{pgfscope}%
\begin{pgfscope}%
\pgfpathrectangle{\pgfqpoint{0.680860in}{0.078740in}}{\pgfqpoint{7.842520in}{7.842520in}}%
\pgfusepath{clip}%
\pgfsetbuttcap%
\pgfsetroundjoin%
\definecolor{currentfill}{rgb}{0.282623,0.140926,0.457517}%
\pgfsetfillcolor{currentfill}%
\pgfsetlinewidth{0.000000pt}%
\definecolor{currentstroke}{rgb}{0.197636,0.391528,0.554969}%
\pgfsetstrokecolor{currentstroke}%
\pgfsetdash{}{0pt}%
\pgfpathmoveto{\pgfqpoint{1.711964in}{2.453359in}}%
\pgfpathlineto{\pgfqpoint{1.846106in}{2.422259in}}%
\pgfpathlineto{\pgfqpoint{1.929849in}{2.619546in}}%
\pgfpathclose%
\pgfusepath{fill}%
\end{pgfscope}%
\begin{pgfscope}%
\pgfpathrectangle{\pgfqpoint{0.680860in}{0.078740in}}{\pgfqpoint{7.842520in}{7.842520in}}%
\pgfusepath{clip}%
\pgfsetbuttcap%
\pgfsetroundjoin%
\definecolor{currentfill}{rgb}{0.122312,0.633153,0.530398}%
\pgfsetfillcolor{currentfill}%
\pgfsetlinewidth{0.000000pt}%
\definecolor{currentstroke}{rgb}{0.195860,0.395433,0.555276}%
\pgfsetstrokecolor{currentstroke}%
\pgfsetdash{}{0pt}%
\pgfpathmoveto{\pgfqpoint{6.892428in}{4.564701in}}%
\pgfpathlineto{\pgfqpoint{7.105170in}{4.441224in}}%
\pgfpathlineto{\pgfqpoint{6.960295in}{4.422796in}}%
\pgfpathclose%
\pgfusepath{fill}%
\end{pgfscope}%
\begin{pgfscope}%
\pgfpathrectangle{\pgfqpoint{0.680860in}{0.078740in}}{\pgfqpoint{7.842520in}{7.842520in}}%
\pgfusepath{clip}%
\pgfsetbuttcap%
\pgfsetroundjoin%
\definecolor{currentfill}{rgb}{0.276022,0.044167,0.370164}%
\pgfsetfillcolor{currentfill}%
\pgfsetlinewidth{0.000000pt}%
\definecolor{currentstroke}{rgb}{0.194100,0.399323,0.555565}%
\pgfsetstrokecolor{currentstroke}%
\pgfsetdash{}{0pt}%
\pgfpathmoveto{\pgfqpoint{1.628022in}{2.256201in}}%
\pgfpathlineto{\pgfqpoint{1.494178in}{2.291711in}}%
\pgfpathlineto{\pgfqpoint{1.543995in}{2.056181in}}%
\pgfpathclose%
\pgfusepath{fill}%
\end{pgfscope}%
\begin{pgfscope}%
\pgfpathrectangle{\pgfqpoint{0.680860in}{0.078740in}}{\pgfqpoint{7.842520in}{7.842520in}}%
\pgfusepath{clip}%
\pgfsetbuttcap%
\pgfsetroundjoin%
\definecolor{currentfill}{rgb}{0.282656,0.100196,0.422160}%
\pgfsetfillcolor{currentfill}%
\pgfsetlinewidth{0.000000pt}%
\definecolor{currentstroke}{rgb}{0.192357,0.403199,0.555836}%
\pgfsetstrokecolor{currentstroke}%
\pgfsetdash{}{0pt}%
\pgfpathmoveto{\pgfqpoint{1.846106in}{2.422259in}}%
\pgfpathlineto{\pgfqpoint{1.711964in}{2.453359in}}%
\pgfpathlineto{\pgfqpoint{1.628022in}{2.256201in}}%
\pgfpathclose%
\pgfusepath{fill}%
\end{pgfscope}%
\begin{pgfscope}%
\pgfpathrectangle{\pgfqpoint{0.680860in}{0.078740in}}{\pgfqpoint{7.842520in}{7.842520in}}%
\pgfusepath{clip}%
\pgfsetbuttcap%
\pgfsetroundjoin%
\definecolor{currentfill}{rgb}{0.243113,0.292092,0.538516}%
\pgfsetfillcolor{currentfill}%
\pgfsetlinewidth{0.000000pt}%
\definecolor{currentstroke}{rgb}{0.190631,0.407061,0.556089}%
\pgfsetstrokecolor{currentstroke}%
\pgfsetdash{}{0pt}%
\pgfpathmoveto{\pgfqpoint{2.449771in}{3.146096in}}%
\pgfpathlineto{\pgfqpoint{2.231536in}{2.978467in}}%
\pgfpathlineto{\pgfqpoint{2.366237in}{2.960748in}}%
\pgfpathclose%
\pgfusepath{fill}%
\end{pgfscope}%
\begin{pgfscope}%
\pgfpathrectangle{\pgfqpoint{0.680860in}{0.078740in}}{\pgfqpoint{7.842520in}{7.842520in}}%
\pgfusepath{clip}%
\pgfsetbuttcap%
\pgfsetroundjoin%
\definecolor{currentfill}{rgb}{0.221989,0.339161,0.548752}%
\pgfsetfillcolor{currentfill}%
\pgfsetlinewidth{0.000000pt}%
\definecolor{currentstroke}{rgb}{0.188923,0.410910,0.556326}%
\pgfsetstrokecolor{currentstroke}%
\pgfsetdash{}{0pt}%
\pgfpathmoveto{\pgfqpoint{2.533350in}{3.322223in}}%
\pgfpathlineto{\pgfqpoint{2.449771in}{3.146096in}}%
\pgfpathlineto{\pgfqpoint{2.584871in}{3.133059in}}%
\pgfpathclose%
\pgfusepath{fill}%
\end{pgfscope}%
\begin{pgfscope}%
\pgfpathrectangle{\pgfqpoint{0.680860in}{0.078740in}}{\pgfqpoint{7.842520in}{7.842520in}}%
\pgfusepath{clip}%
\pgfsetbuttcap%
\pgfsetroundjoin%
\definecolor{currentfill}{rgb}{0.156270,0.489624,0.557936}%
\pgfsetfillcolor{currentfill}%
\pgfsetlinewidth{0.000000pt}%
\definecolor{currentstroke}{rgb}{0.187231,0.414746,0.556547}%
\pgfsetstrokecolor{currentstroke}%
\pgfsetdash{}{0pt}%
\pgfpathmoveto{\pgfqpoint{3.326234in}{3.791903in}}%
\pgfpathlineto{\pgfqpoint{3.273189in}{3.925127in}}%
\pgfpathlineto{\pgfqpoint{3.189891in}{3.788299in}}%
\pgfpathclose%
\pgfusepath{fill}%
\end{pgfscope}%
\begin{pgfscope}%
\pgfpathrectangle{\pgfqpoint{0.680860in}{0.078740in}}{\pgfqpoint{7.842520in}{7.842520in}}%
\pgfusepath{clip}%
\pgfsetbuttcap%
\pgfsetroundjoin%
\definecolor{currentfill}{rgb}{0.143303,0.669459,0.511215}%
\pgfsetfillcolor{currentfill}%
\pgfsetlinewidth{0.000000pt}%
\definecolor{currentstroke}{rgb}{0.185556,0.418570,0.556753}%
\pgfsetstrokecolor{currentstroke}%
\pgfsetdash{}{0pt}%
\pgfpathmoveto{\pgfqpoint{5.886079in}{4.613572in}}%
\pgfpathlineto{\pgfqpoint{5.668435in}{4.640824in}}%
\pgfpathlineto{\pgfqpoint{5.810668in}{4.666739in}}%
\pgfpathclose%
\pgfusepath{fill}%
\end{pgfscope}%
\begin{pgfscope}%
\pgfpathrectangle{\pgfqpoint{0.680860in}{0.078740in}}{\pgfqpoint{7.842520in}{7.842520in}}%
\pgfusepath{clip}%
\pgfsetbuttcap%
\pgfsetroundjoin%
\definecolor{currentfill}{rgb}{0.151918,0.500685,0.557587}%
\pgfsetfillcolor{currentfill}%
\pgfsetlinewidth{0.000000pt}%
\definecolor{currentstroke}{rgb}{0.183898,0.422383,0.556944}%
\pgfsetstrokecolor{currentstroke}%
\pgfsetdash{}{0pt}%
\pgfpathmoveto{\pgfqpoint{3.409417in}{3.931148in}}%
\pgfpathlineto{\pgfqpoint{3.273189in}{3.925127in}}%
\pgfpathlineto{\pgfqpoint{3.326234in}{3.791903in}}%
\pgfpathclose%
\pgfusepath{fill}%
\end{pgfscope}%
\begin{pgfscope}%
\pgfpathrectangle{\pgfqpoint{0.680860in}{0.078740in}}{\pgfqpoint{7.842520in}{7.842520in}}%
\pgfusepath{clip}%
\pgfsetbuttcap%
\pgfsetroundjoin%
\definecolor{currentfill}{rgb}{0.124780,0.640461,0.527068}%
\pgfsetfillcolor{currentfill}%
\pgfsetlinewidth{0.000000pt}%
\definecolor{currentstroke}{rgb}{0.182256,0.426184,0.557120}%
\pgfsetstrokecolor{currentstroke}%
\pgfsetdash{}{0pt}%
\pgfpathmoveto{\pgfqpoint{4.729650in}{4.533049in}}%
\pgfpathlineto{\pgfqpoint{4.509769in}{4.471817in}}%
\pgfpathlineto{\pgfqpoint{4.649033in}{4.493129in}}%
\pgfpathclose%
\pgfusepath{fill}%
\end{pgfscope}%
\begin{pgfscope}%
\pgfpathrectangle{\pgfqpoint{0.680860in}{0.078740in}}{\pgfqpoint{7.842520in}{7.842520in}}%
\pgfusepath{clip}%
\pgfsetbuttcap%
\pgfsetroundjoin%
\definecolor{currentfill}{rgb}{0.195860,0.395433,0.555276}%
\pgfsetfillcolor{currentfill}%
\pgfsetlinewidth{0.000000pt}%
\definecolor{currentstroke}{rgb}{0.180629,0.429975,0.557282}%
\pgfsetstrokecolor{currentstroke}%
\pgfsetdash{}{0pt}%
\pgfpathmoveto{\pgfqpoint{2.751845in}{3.482122in}}%
\pgfpathlineto{\pgfqpoint{2.668337in}{3.312877in}}%
\pgfpathlineto{\pgfqpoint{2.887264in}{3.477230in}}%
\pgfpathclose%
\pgfusepath{fill}%
\end{pgfscope}%
\begin{pgfscope}%
\pgfpathrectangle{\pgfqpoint{0.680860in}{0.078740in}}{\pgfqpoint{7.842520in}{7.842520in}}%
\pgfusepath{clip}%
\pgfsetbuttcap%
\pgfsetroundjoin%
\definecolor{currentfill}{rgb}{0.214298,0.355619,0.551184}%
\pgfsetfillcolor{currentfill}%
\pgfsetlinewidth{0.000000pt}%
\definecolor{currentstroke}{rgb}{0.179019,0.433756,0.557430}%
\pgfsetstrokecolor{currentstroke}%
\pgfsetdash{}{0pt}%
\pgfpathmoveto{\pgfqpoint{2.584871in}{3.133059in}}%
\pgfpathlineto{\pgfqpoint{2.668337in}{3.312877in}}%
\pgfpathlineto{\pgfqpoint{2.533350in}{3.322223in}}%
\pgfpathclose%
\pgfusepath{fill}%
\end{pgfscope}%
\begin{pgfscope}%
\pgfpathrectangle{\pgfqpoint{0.680860in}{0.078740in}}{\pgfqpoint{7.842520in}{7.842520in}}%
\pgfusepath{clip}%
\pgfsetbuttcap%
\pgfsetroundjoin%
\definecolor{currentfill}{rgb}{0.119512,0.607464,0.540218}%
\pgfsetfillcolor{currentfill}%
\pgfsetlinewidth{0.000000pt}%
\definecolor{currentstroke}{rgb}{0.177423,0.437527,0.557565}%
\pgfsetstrokecolor{currentstroke}%
\pgfsetdash{}{0pt}%
\pgfpathmoveto{\pgfqpoint{4.207755in}{4.314588in}}%
\pgfpathlineto{\pgfqpoint{4.289651in}{4.392719in}}%
\pgfpathlineto{\pgfqpoint{4.151517in}{4.375085in}}%
\pgfpathclose%
\pgfusepath{fill}%
\end{pgfscope}%
\begin{pgfscope}%
\pgfpathrectangle{\pgfqpoint{0.680860in}{0.078740in}}{\pgfqpoint{7.842520in}{7.842520in}}%
\pgfusepath{clip}%
\pgfsetbuttcap%
\pgfsetroundjoin%
\definecolor{currentfill}{rgb}{0.143303,0.669459,0.511215}%
\pgfsetfillcolor{currentfill}%
\pgfsetlinewidth{0.000000pt}%
\definecolor{currentstroke}{rgb}{0.175841,0.441290,0.557685}%
\pgfsetstrokecolor{currentstroke}%
\pgfsetdash{}{0pt}%
\pgfpathmoveto{\pgfqpoint{6.028551in}{4.637580in}}%
\pgfpathlineto{\pgfqpoint{5.886079in}{4.613572in}}%
\pgfpathlineto{\pgfqpoint{5.810668in}{4.666739in}}%
\pgfpathclose%
\pgfusepath{fill}%
\end{pgfscope}%
\begin{pgfscope}%
\pgfpathrectangle{\pgfqpoint{0.680860in}{0.078740in}}{\pgfqpoint{7.842520in}{7.842520in}}%
\pgfusepath{clip}%
\pgfsetbuttcap%
\pgfsetroundjoin%
\definecolor{currentfill}{rgb}{0.143343,0.522773,0.556295}%
\pgfsetfillcolor{currentfill}%
\pgfsetlinewidth{0.000000pt}%
\definecolor{currentstroke}{rgb}{0.174274,0.445044,0.557792}%
\pgfsetstrokecolor{currentstroke}%
\pgfsetdash{}{0pt}%
\pgfpathmoveto{\pgfqpoint{3.546246in}{3.938673in}}%
\pgfpathlineto{\pgfqpoint{3.492525in}{4.055291in}}%
\pgfpathlineto{\pgfqpoint{3.409417in}{3.931148in}}%
\pgfpathclose%
\pgfusepath{fill}%
\end{pgfscope}%
\begin{pgfscope}%
\pgfpathrectangle{\pgfqpoint{0.680860in}{0.078740in}}{\pgfqpoint{7.842520in}{7.842520in}}%
\pgfusepath{clip}%
\pgfsetbuttcap%
\pgfsetroundjoin%
\definecolor{currentfill}{rgb}{0.120092,0.600104,0.542530}%
\pgfsetfillcolor{currentfill}%
\pgfsetlinewidth{0.000000pt}%
\definecolor{currentstroke}{rgb}{0.172719,0.448791,0.557885}%
\pgfsetstrokecolor{currentstroke}%
\pgfsetdash{}{0pt}%
\pgfpathmoveto{\pgfqpoint{4.069441in}{4.297372in}}%
\pgfpathlineto{\pgfqpoint{4.207755in}{4.314588in}}%
\pgfpathlineto{\pgfqpoint{4.151517in}{4.375085in}}%
\pgfpathclose%
\pgfusepath{fill}%
\end{pgfscope}%
\begin{pgfscope}%
\pgfpathrectangle{\pgfqpoint{0.680860in}{0.078740in}}{\pgfqpoint{7.842520in}{7.842520in}}%
\pgfusepath{clip}%
\pgfsetbuttcap%
\pgfsetroundjoin%
\definecolor{currentfill}{rgb}{0.169646,0.456262,0.558030}%
\pgfsetfillcolor{currentfill}%
\pgfsetlinewidth{0.000000pt}%
\definecolor{currentstroke}{rgb}{0.171176,0.452530,0.557965}%
\pgfsetstrokecolor{currentstroke}%
\pgfsetdash{}{0pt}%
\pgfpathmoveto{\pgfqpoint{2.970693in}{3.638043in}}%
\pgfpathlineto{\pgfqpoint{3.106565in}{3.637487in}}%
\pgfpathlineto{\pgfqpoint{3.189891in}{3.788299in}}%
\pgfpathclose%
\pgfusepath{fill}%
\end{pgfscope}%
\begin{pgfscope}%
\pgfpathrectangle{\pgfqpoint{0.680860in}{0.078740in}}{\pgfqpoint{7.842520in}{7.842520in}}%
\pgfusepath{clip}%
\pgfsetbuttcap%
\pgfsetroundjoin%
\definecolor{currentfill}{rgb}{0.132268,0.655014,0.519661}%
\pgfsetfillcolor{currentfill}%
\pgfsetlinewidth{0.000000pt}%
\definecolor{currentstroke}{rgb}{0.169646,0.456262,0.558030}%
\pgfsetstrokecolor{currentstroke}%
\pgfsetdash{}{0pt}%
\pgfpathmoveto{\pgfqpoint{6.532545in}{4.630913in}}%
\pgfpathlineto{\pgfqpoint{6.747468in}{4.541222in}}%
\pgfpathlineto{\pgfqpoint{6.603281in}{4.519783in}}%
\pgfpathclose%
\pgfusepath{fill}%
\end{pgfscope}%
\begin{pgfscope}%
\pgfpathrectangle{\pgfqpoint{0.680860in}{0.078740in}}{\pgfqpoint{7.842520in}{7.842520in}}%
\pgfusepath{clip}%
\pgfsetbuttcap%
\pgfsetroundjoin%
\definecolor{currentfill}{rgb}{0.139147,0.533812,0.555298}%
\pgfsetfillcolor{currentfill}%
\pgfsetlinewidth{0.000000pt}%
\definecolor{currentstroke}{rgb}{0.168126,0.459988,0.558082}%
\pgfsetstrokecolor{currentstroke}%
\pgfsetdash{}{0pt}%
\pgfpathmoveto{\pgfqpoint{3.629230in}{4.064809in}}%
\pgfpathlineto{\pgfqpoint{3.492525in}{4.055291in}}%
\pgfpathlineto{\pgfqpoint{3.546246in}{3.938673in}}%
\pgfpathclose%
\pgfusepath{fill}%
\end{pgfscope}%
\begin{pgfscope}%
\pgfpathrectangle{\pgfqpoint{0.680860in}{0.078740in}}{\pgfqpoint{7.842520in}{7.842520in}}%
\pgfusepath{clip}%
\pgfsetbuttcap%
\pgfsetroundjoin%
\definecolor{currentfill}{rgb}{0.146616,0.673050,0.508936}%
\pgfsetfillcolor{currentfill}%
\pgfsetlinewidth{0.000000pt}%
\definecolor{currentstroke}{rgb}{0.166617,0.463708,0.558119}%
\pgfsetstrokecolor{currentstroke}%
\pgfsetdash{}{0pt}%
\pgfpathmoveto{\pgfqpoint{5.526942in}{4.617032in}}%
\pgfpathlineto{\pgfqpoint{5.591611in}{4.670570in}}%
\pgfpathlineto{\pgfqpoint{5.668435in}{4.640824in}}%
\pgfpathclose%
\pgfusepath{fill}%
\end{pgfscope}%
\begin{pgfscope}%
\pgfpathrectangle{\pgfqpoint{0.680860in}{0.078740in}}{\pgfqpoint{7.842520in}{7.842520in}}%
\pgfusepath{clip}%
\pgfsetbuttcap%
\pgfsetroundjoin%
\definecolor{currentfill}{rgb}{0.122606,0.585371,0.546557}%
\pgfsetfillcolor{currentfill}%
\pgfsetlinewidth{0.000000pt}%
\definecolor{currentstroke}{rgb}{0.165117,0.467423,0.558141}%
\pgfsetstrokecolor{currentstroke}%
\pgfsetdash{}{0pt}%
\pgfpathmoveto{\pgfqpoint{3.987086in}{4.201876in}}%
\pgfpathlineto{\pgfqpoint{4.069441in}{4.297372in}}%
\pgfpathlineto{\pgfqpoint{3.931777in}{4.281986in}}%
\pgfpathclose%
\pgfusepath{fill}%
\end{pgfscope}%
\begin{pgfscope}%
\pgfpathrectangle{\pgfqpoint{0.680860in}{0.078740in}}{\pgfqpoint{7.842520in}{7.842520in}}%
\pgfusepath{clip}%
\pgfsetbuttcap%
\pgfsetroundjoin%
\definecolor{currentfill}{rgb}{0.275191,0.194905,0.496005}%
\pgfsetfillcolor{currentfill}%
\pgfsetlinewidth{0.000000pt}%
\definecolor{currentstroke}{rgb}{0.163625,0.471133,0.558148}%
\pgfsetstrokecolor{currentstroke}%
\pgfsetdash{}{0pt}%
\pgfpathmoveto{\pgfqpoint{1.929849in}{2.619546in}}%
\pgfpathlineto{\pgfqpoint{2.064319in}{2.593024in}}%
\pgfpathlineto{\pgfqpoint{2.147916in}{2.789106in}}%
\pgfpathclose%
\pgfusepath{fill}%
\end{pgfscope}%
\begin{pgfscope}%
\pgfpathrectangle{\pgfqpoint{0.680860in}{0.078740in}}{\pgfqpoint{7.842520in}{7.842520in}}%
\pgfusepath{clip}%
\pgfsetbuttcap%
\pgfsetroundjoin%
\definecolor{currentfill}{rgb}{0.124395,0.578002,0.548287}%
\pgfsetfillcolor{currentfill}%
\pgfsetlinewidth{0.000000pt}%
\definecolor{currentstroke}{rgb}{0.162142,0.474838,0.558140}%
\pgfsetstrokecolor{currentstroke}%
\pgfsetdash{}{0pt}%
\pgfpathmoveto{\pgfqpoint{3.931777in}{4.281986in}}%
\pgfpathlineto{\pgfqpoint{3.849266in}{4.187481in}}%
\pgfpathlineto{\pgfqpoint{3.987086in}{4.201876in}}%
\pgfpathclose%
\pgfusepath{fill}%
\end{pgfscope}%
\begin{pgfscope}%
\pgfpathrectangle{\pgfqpoint{0.680860in}{0.078740in}}{\pgfqpoint{7.842520in}{7.842520in}}%
\pgfusepath{clip}%
\pgfsetbuttcap%
\pgfsetroundjoin%
\definecolor{currentfill}{rgb}{0.140210,0.665859,0.513427}%
\pgfsetfillcolor{currentfill}%
\pgfsetlinewidth{0.000000pt}%
\definecolor{currentstroke}{rgb}{0.160665,0.478540,0.558115}%
\pgfsetstrokecolor{currentstroke}%
\pgfsetdash{}{0pt}%
\pgfpathmoveto{\pgfqpoint{6.171777in}{4.663723in}}%
\pgfpathlineto{\pgfqpoint{6.388401in}{4.605543in}}%
\pgfpathlineto{\pgfqpoint{6.245023in}{4.582280in}}%
\pgfpathclose%
\pgfusepath{fill}%
\end{pgfscope}%
\begin{pgfscope}%
\pgfpathrectangle{\pgfqpoint{0.680860in}{0.078740in}}{\pgfqpoint{7.842520in}{7.842520in}}%
\pgfusepath{clip}%
\pgfsetbuttcap%
\pgfsetroundjoin%
\definecolor{currentfill}{rgb}{0.260571,0.246922,0.522828}%
\pgfsetfillcolor{currentfill}%
\pgfsetlinewidth{0.000000pt}%
\definecolor{currentstroke}{rgb}{0.159194,0.482237,0.558073}%
\pgfsetstrokecolor{currentstroke}%
\pgfsetdash{}{0pt}%
\pgfpathmoveto{\pgfqpoint{2.282743in}{2.767293in}}%
\pgfpathlineto{\pgfqpoint{2.231536in}{2.978467in}}%
\pgfpathlineto{\pgfqpoint{2.147916in}{2.789106in}}%
\pgfpathclose%
\pgfusepath{fill}%
\end{pgfscope}%
\begin{pgfscope}%
\pgfpathrectangle{\pgfqpoint{0.680860in}{0.078740in}}{\pgfqpoint{7.842520in}{7.842520in}}%
\pgfusepath{clip}%
\pgfsetbuttcap%
\pgfsetroundjoin%
\definecolor{currentfill}{rgb}{0.137339,0.662252,0.515571}%
\pgfsetfillcolor{currentfill}%
\pgfsetlinewidth{0.000000pt}%
\definecolor{currentstroke}{rgb}{0.157729,0.485932,0.558013}%
\pgfsetstrokecolor{currentstroke}%
\pgfsetdash{}{0pt}%
\pgfpathmoveto{\pgfqpoint{5.009762in}{4.580065in}}%
\pgfpathlineto{\pgfqpoint{5.089240in}{4.597867in}}%
\pgfpathlineto{\pgfqpoint{4.949132in}{4.574912in}}%
\pgfpathclose%
\pgfusepath{fill}%
\end{pgfscope}%
\begin{pgfscope}%
\pgfpathrectangle{\pgfqpoint{0.680860in}{0.078740in}}{\pgfqpoint{7.842520in}{7.842520in}}%
\pgfusepath{clip}%
\pgfsetbuttcap%
\pgfsetroundjoin%
\definecolor{currentfill}{rgb}{0.140210,0.665859,0.513427}%
\pgfsetfillcolor{currentfill}%
\pgfsetlinewidth{0.000000pt}%
\definecolor{currentstroke}{rgb}{0.156270,0.489624,0.557936}%
\pgfsetstrokecolor{currentstroke}%
\pgfsetdash{}{0pt}%
\pgfpathmoveto{\pgfqpoint{5.089240in}{4.597867in}}%
\pgfpathlineto{\pgfqpoint{5.230060in}{4.622919in}}%
\pgfpathlineto{\pgfqpoint{5.308503in}{4.618780in}}%
\pgfpathclose%
\pgfusepath{fill}%
\end{pgfscope}%
\begin{pgfscope}%
\pgfpathrectangle{\pgfqpoint{0.680860in}{0.078740in}}{\pgfqpoint{7.842520in}{7.842520in}}%
\pgfusepath{clip}%
\pgfsetbuttcap%
\pgfsetroundjoin%
\definecolor{currentfill}{rgb}{0.128087,0.647749,0.523491}%
\pgfsetfillcolor{currentfill}%
\pgfsetlinewidth{0.000000pt}%
\definecolor{currentstroke}{rgb}{0.154815,0.493313,0.557840}%
\pgfsetstrokecolor{currentstroke}%
\pgfsetdash{}{0pt}%
\pgfpathmoveto{\pgfqpoint{4.869354in}{4.555516in}}%
\pgfpathlineto{\pgfqpoint{4.729650in}{4.533049in}}%
\pgfpathlineto{\pgfqpoint{4.649033in}{4.493129in}}%
\pgfpathclose%
\pgfusepath{fill}%
\end{pgfscope}%
\begin{pgfscope}%
\pgfpathrectangle{\pgfqpoint{0.680860in}{0.078740in}}{\pgfqpoint{7.842520in}{7.842520in}}%
\pgfusepath{clip}%
\pgfsetbuttcap%
\pgfsetroundjoin%
\definecolor{currentfill}{rgb}{0.120081,0.622161,0.534946}%
\pgfsetfillcolor{currentfill}%
\pgfsetlinewidth{0.000000pt}%
\definecolor{currentstroke}{rgb}{0.153364,0.497000,0.557724}%
\pgfsetstrokecolor{currentstroke}%
\pgfsetdash{}{0pt}%
\pgfpathmoveto{\pgfqpoint{7.315451in}{4.288200in}}%
\pgfpathlineto{\pgfqpoint{7.250814in}{4.461584in}}%
\pgfpathlineto{\pgfqpoint{7.397248in}{4.483949in}}%
\pgfpathclose%
\pgfusepath{fill}%
\end{pgfscope}%
\begin{pgfscope}%
\pgfpathrectangle{\pgfqpoint{0.680860in}{0.078740in}}{\pgfqpoint{7.842520in}{7.842520in}}%
\pgfusepath{clip}%
\pgfsetbuttcap%
\pgfsetroundjoin%
\definecolor{currentfill}{rgb}{0.177423,0.437527,0.557565}%
\pgfsetfillcolor{currentfill}%
\pgfsetlinewidth{0.000000pt}%
\definecolor{currentstroke}{rgb}{0.151918,0.500685,0.557587}%
\pgfsetstrokecolor{currentstroke}%
\pgfsetdash{}{0pt}%
\pgfpathmoveto{\pgfqpoint{3.106565in}{3.637487in}}%
\pgfpathlineto{\pgfqpoint{2.970693in}{3.638043in}}%
\pgfpathlineto{\pgfqpoint{2.887264in}{3.477230in}}%
\pgfpathclose%
\pgfusepath{fill}%
\end{pgfscope}%
\begin{pgfscope}%
\pgfpathrectangle{\pgfqpoint{0.680860in}{0.078740in}}{\pgfqpoint{7.842520in}{7.842520in}}%
\pgfusepath{clip}%
\pgfsetbuttcap%
\pgfsetroundjoin%
\definecolor{currentfill}{rgb}{0.120638,0.625828,0.533488}%
\pgfsetfillcolor{currentfill}%
\pgfsetlinewidth{0.000000pt}%
\definecolor{currentstroke}{rgb}{0.150476,0.504369,0.557430}%
\pgfsetstrokecolor{currentstroke}%
\pgfsetdash{}{0pt}%
\pgfpathmoveto{\pgfqpoint{4.428449in}{4.412261in}}%
\pgfpathlineto{\pgfqpoint{4.509769in}{4.471817in}}%
\pgfpathlineto{\pgfqpoint{4.289651in}{4.392719in}}%
\pgfpathclose%
\pgfusepath{fill}%
\end{pgfscope}%
\begin{pgfscope}%
\pgfpathrectangle{\pgfqpoint{0.680860in}{0.078740in}}{\pgfqpoint{7.842520in}{7.842520in}}%
\pgfusepath{clip}%
\pgfsetbuttcap%
\pgfsetroundjoin%
\definecolor{currentfill}{rgb}{0.253935,0.265254,0.529983}%
\pgfsetfillcolor{currentfill}%
\pgfsetlinewidth{0.000000pt}%
\definecolor{currentstroke}{rgb}{0.149039,0.508051,0.557250}%
\pgfsetstrokecolor{currentstroke}%
\pgfsetdash{}{0pt}%
\pgfpathmoveto{\pgfqpoint{2.366237in}{2.960748in}}%
\pgfpathlineto{\pgfqpoint{2.231536in}{2.978467in}}%
\pgfpathlineto{\pgfqpoint{2.282743in}{2.767293in}}%
\pgfpathclose%
\pgfusepath{fill}%
\end{pgfscope}%
\begin{pgfscope}%
\pgfpathrectangle{\pgfqpoint{0.680860in}{0.078740in}}{\pgfqpoint{7.842520in}{7.842520in}}%
\pgfusepath{clip}%
\pgfsetbuttcap%
\pgfsetroundjoin%
\definecolor{currentfill}{rgb}{0.281446,0.084320,0.407414}%
\pgfsetfillcolor{currentfill}%
\pgfsetlinewidth{0.000000pt}%
\definecolor{currentstroke}{rgb}{0.147607,0.511733,0.557049}%
\pgfsetstrokecolor{currentstroke}%
\pgfsetdash{}{0pt}%
\pgfpathmoveto{\pgfqpoint{1.628022in}{2.256201in}}%
\pgfpathlineto{\pgfqpoint{1.762325in}{2.220810in}}%
\pgfpathlineto{\pgfqpoint{1.846106in}{2.422259in}}%
\pgfpathclose%
\pgfusepath{fill}%
\end{pgfscope}%
\begin{pgfscope}%
\pgfpathrectangle{\pgfqpoint{0.680860in}{0.078740in}}{\pgfqpoint{7.842520in}{7.842520in}}%
\pgfusepath{clip}%
\pgfsetbuttcap%
\pgfsetroundjoin%
\definecolor{currentfill}{rgb}{0.150148,0.676631,0.506589}%
\pgfsetfillcolor{currentfill}%
\pgfsetlinewidth{0.000000pt}%
\definecolor{currentstroke}{rgb}{0.146180,0.515413,0.556823}%
\pgfsetstrokecolor{currentstroke}%
\pgfsetdash{}{0pt}%
\pgfpathmoveto{\pgfqpoint{5.449688in}{4.643592in}}%
\pgfpathlineto{\pgfqpoint{5.591611in}{4.670570in}}%
\pgfpathlineto{\pgfqpoint{5.526942in}{4.617032in}}%
\pgfpathclose%
\pgfusepath{fill}%
\end{pgfscope}%
\begin{pgfscope}%
\pgfpathrectangle{\pgfqpoint{0.680860in}{0.078740in}}{\pgfqpoint{7.842520in}{7.842520in}}%
\pgfusepath{clip}%
\pgfsetbuttcap%
\pgfsetroundjoin%
\definecolor{currentfill}{rgb}{0.282623,0.140926,0.457517}%
\pgfsetfillcolor{currentfill}%
\pgfsetlinewidth{0.000000pt}%
\definecolor{currentstroke}{rgb}{0.144759,0.519093,0.556572}%
\pgfsetstrokecolor{currentstroke}%
\pgfsetdash{}{0pt}%
\pgfpathmoveto{\pgfqpoint{1.929849in}{2.619546in}}%
\pgfpathlineto{\pgfqpoint{1.846106in}{2.422259in}}%
\pgfpathlineto{\pgfqpoint{1.980721in}{2.391422in}}%
\pgfpathclose%
\pgfusepath{fill}%
\end{pgfscope}%
\begin{pgfscope}%
\pgfpathrectangle{\pgfqpoint{0.680860in}{0.078740in}}{\pgfqpoint{7.842520in}{7.842520in}}%
\pgfusepath{clip}%
\pgfsetbuttcap%
\pgfsetroundjoin%
\definecolor{currentfill}{rgb}{0.134692,0.658636,0.517649}%
\pgfsetfillcolor{currentfill}%
\pgfsetlinewidth{0.000000pt}%
\definecolor{currentstroke}{rgb}{0.143343,0.522773,0.556295}%
\pgfsetstrokecolor{currentstroke}%
\pgfsetdash{}{0pt}%
\pgfpathmoveto{\pgfqpoint{4.949132in}{4.574912in}}%
\pgfpathlineto{\pgfqpoint{4.869354in}{4.555516in}}%
\pgfpathlineto{\pgfqpoint{5.009762in}{4.580065in}}%
\pgfpathclose%
\pgfusepath{fill}%
\end{pgfscope}%
\begin{pgfscope}%
\pgfpathrectangle{\pgfqpoint{0.680860in}{0.078740in}}{\pgfqpoint{7.842520in}{7.842520in}}%
\pgfusepath{clip}%
\pgfsetbuttcap%
\pgfsetroundjoin%
\definecolor{currentfill}{rgb}{0.276022,0.044167,0.370164}%
\pgfsetfillcolor{currentfill}%
\pgfsetlinewidth{0.000000pt}%
\definecolor{currentstroke}{rgb}{0.141935,0.526453,0.555991}%
\pgfsetstrokecolor{currentstroke}%
\pgfsetdash{}{0pt}%
\pgfpathmoveto{\pgfqpoint{1.628022in}{2.256201in}}%
\pgfpathlineto{\pgfqpoint{1.543995in}{2.056181in}}%
\pgfpathlineto{\pgfqpoint{1.762325in}{2.220810in}}%
\pgfpathclose%
\pgfusepath{fill}%
\end{pgfscope}%
\begin{pgfscope}%
\pgfpathrectangle{\pgfqpoint{0.680860in}{0.078740in}}{\pgfqpoint{7.842520in}{7.842520in}}%
\pgfusepath{clip}%
\pgfsetbuttcap%
\pgfsetroundjoin%
\definecolor{currentfill}{rgb}{0.280868,0.160771,0.472899}%
\pgfsetfillcolor{currentfill}%
\pgfsetlinewidth{0.000000pt}%
\definecolor{currentstroke}{rgb}{0.140536,0.530132,0.555659}%
\pgfsetstrokecolor{currentstroke}%
\pgfsetdash{}{0pt}%
\pgfpathmoveto{\pgfqpoint{1.980721in}{2.391422in}}%
\pgfpathlineto{\pgfqpoint{2.064319in}{2.593024in}}%
\pgfpathlineto{\pgfqpoint{1.929849in}{2.619546in}}%
\pgfpathclose%
\pgfusepath{fill}%
\end{pgfscope}%
\begin{pgfscope}%
\pgfpathrectangle{\pgfqpoint{0.680860in}{0.078740in}}{\pgfqpoint{7.842520in}{7.842520in}}%
\pgfusepath{clip}%
\pgfsetbuttcap%
\pgfsetroundjoin%
\definecolor{currentfill}{rgb}{0.162142,0.474838,0.558140}%
\pgfsetfillcolor{currentfill}%
\pgfsetlinewidth{0.000000pt}%
\definecolor{currentstroke}{rgb}{0.139147,0.533812,0.555298}%
\pgfsetstrokecolor{currentstroke}%
\pgfsetdash{}{0pt}%
\pgfpathmoveto{\pgfqpoint{3.189891in}{3.788299in}}%
\pgfpathlineto{\pgfqpoint{3.106565in}{3.637487in}}%
\pgfpathlineto{\pgfqpoint{3.326234in}{3.791903in}}%
\pgfpathclose%
\pgfusepath{fill}%
\end{pgfscope}%
\begin{pgfscope}%
\pgfpathrectangle{\pgfqpoint{0.680860in}{0.078740in}}{\pgfqpoint{7.842520in}{7.842520in}}%
\pgfusepath{clip}%
\pgfsetbuttcap%
\pgfsetroundjoin%
\definecolor{currentfill}{rgb}{0.131172,0.555899,0.552459}%
\pgfsetfillcolor{currentfill}%
\pgfsetlinewidth{0.000000pt}%
\definecolor{currentstroke}{rgb}{0.137770,0.537492,0.554906}%
\pgfsetstrokecolor{currentstroke}%
\pgfsetdash{}{0pt}%
\pgfpathmoveto{\pgfqpoint{3.629230in}{4.064809in}}%
\pgfpathlineto{\pgfqpoint{3.766552in}{4.075952in}}%
\pgfpathlineto{\pgfqpoint{3.849266in}{4.187481in}}%
\pgfpathclose%
\pgfusepath{fill}%
\end{pgfscope}%
\begin{pgfscope}%
\pgfpathrectangle{\pgfqpoint{0.680860in}{0.078740in}}{\pgfqpoint{7.842520in}{7.842520in}}%
\pgfusepath{clip}%
\pgfsetbuttcap%
\pgfsetroundjoin%
\definecolor{currentfill}{rgb}{0.239346,0.300855,0.540844}%
\pgfsetfillcolor{currentfill}%
\pgfsetlinewidth{0.000000pt}%
\definecolor{currentstroke}{rgb}{0.136408,0.541173,0.554483}%
\pgfsetstrokecolor{currentstroke}%
\pgfsetdash{}{0pt}%
\pgfpathmoveto{\pgfqpoint{2.366237in}{2.960748in}}%
\pgfpathlineto{\pgfqpoint{2.501451in}{2.943731in}}%
\pgfpathlineto{\pgfqpoint{2.449771in}{3.146096in}}%
\pgfpathclose%
\pgfusepath{fill}%
\end{pgfscope}%
\begin{pgfscope}%
\pgfpathrectangle{\pgfqpoint{0.680860in}{0.078740in}}{\pgfqpoint{7.842520in}{7.842520in}}%
\pgfusepath{clip}%
\pgfsetbuttcap%
\pgfsetroundjoin%
\definecolor{currentfill}{rgb}{0.270595,0.214069,0.507052}%
\pgfsetfillcolor{currentfill}%
\pgfsetlinewidth{0.000000pt}%
\definecolor{currentstroke}{rgb}{0.135066,0.544853,0.554029}%
\pgfsetstrokecolor{currentstroke}%
\pgfsetdash{}{0pt}%
\pgfpathmoveto{\pgfqpoint{2.064319in}{2.593024in}}%
\pgfpathlineto{\pgfqpoint{2.282743in}{2.767293in}}%
\pgfpathlineto{\pgfqpoint{2.147916in}{2.789106in}}%
\pgfpathclose%
\pgfusepath{fill}%
\end{pgfscope}%
\begin{pgfscope}%
\pgfpathrectangle{\pgfqpoint{0.680860in}{0.078740in}}{\pgfqpoint{7.842520in}{7.842520in}}%
\pgfusepath{clip}%
\pgfsetbuttcap%
\pgfsetroundjoin%
\definecolor{currentfill}{rgb}{0.199430,0.387607,0.554642}%
\pgfsetfillcolor{currentfill}%
\pgfsetlinewidth{0.000000pt}%
\definecolor{currentstroke}{rgb}{0.133743,0.548535,0.553541}%
\pgfsetstrokecolor{currentstroke}%
\pgfsetdash{}{0pt}%
\pgfpathmoveto{\pgfqpoint{2.887264in}{3.477230in}}%
\pgfpathlineto{\pgfqpoint{2.668337in}{3.312877in}}%
\pgfpathlineto{\pgfqpoint{2.803862in}{3.304510in}}%
\pgfpathclose%
\pgfusepath{fill}%
\end{pgfscope}%
\begin{pgfscope}%
\pgfpathrectangle{\pgfqpoint{0.680860in}{0.078740in}}{\pgfqpoint{7.842520in}{7.842520in}}%
\pgfusepath{clip}%
\pgfsetbuttcap%
\pgfsetroundjoin%
\definecolor{currentfill}{rgb}{0.231674,0.318106,0.544834}%
\pgfsetfillcolor{currentfill}%
\pgfsetlinewidth{0.000000pt}%
\definecolor{currentstroke}{rgb}{0.132444,0.552216,0.553018}%
\pgfsetstrokecolor{currentstroke}%
\pgfsetdash{}{0pt}%
\pgfpathmoveto{\pgfqpoint{2.449771in}{3.146096in}}%
\pgfpathlineto{\pgfqpoint{2.501451in}{2.943731in}}%
\pgfpathlineto{\pgfqpoint{2.584871in}{3.133059in}}%
\pgfpathclose%
\pgfusepath{fill}%
\end{pgfscope}%
\begin{pgfscope}%
\pgfpathrectangle{\pgfqpoint{0.680860in}{0.078740in}}{\pgfqpoint{7.842520in}{7.842520in}}%
\pgfusepath{clip}%
\pgfsetbuttcap%
\pgfsetroundjoin%
\definecolor{currentfill}{rgb}{0.149039,0.508051,0.557250}%
\pgfsetfillcolor{currentfill}%
\pgfsetlinewidth{0.000000pt}%
\definecolor{currentstroke}{rgb}{0.131172,0.555899,0.552459}%
\pgfsetstrokecolor{currentstroke}%
\pgfsetdash{}{0pt}%
\pgfpathmoveto{\pgfqpoint{3.409417in}{3.931148in}}%
\pgfpathlineto{\pgfqpoint{3.326234in}{3.791903in}}%
\pgfpathlineto{\pgfqpoint{3.546246in}{3.938673in}}%
\pgfpathclose%
\pgfusepath{fill}%
\end{pgfscope}%
\begin{pgfscope}%
\pgfpathrectangle{\pgfqpoint{0.680860in}{0.078740in}}{\pgfqpoint{7.842520in}{7.842520in}}%
\pgfusepath{clip}%
\pgfsetbuttcap%
\pgfsetroundjoin%
\definecolor{currentfill}{rgb}{0.120081,0.622161,0.534946}%
\pgfsetfillcolor{currentfill}%
\pgfsetlinewidth{0.000000pt}%
\definecolor{currentstroke}{rgb}{0.129933,0.559582,0.551864}%
\pgfsetstrokecolor{currentstroke}%
\pgfsetdash{}{0pt}%
\pgfpathmoveto{\pgfqpoint{7.544489in}{4.508396in}}%
\pgfpathlineto{\pgfqpoint{7.607030in}{4.318240in}}%
\pgfpathlineto{\pgfqpoint{7.460861in}{4.302331in}}%
\pgfpathclose%
\pgfusepath{fill}%
\end{pgfscope}%
\begin{pgfscope}%
\pgfpathrectangle{\pgfqpoint{0.680860in}{0.078740in}}{\pgfqpoint{7.842520in}{7.842520in}}%
\pgfusepath{clip}%
\pgfsetbuttcap%
\pgfsetroundjoin%
\definecolor{currentfill}{rgb}{0.153894,0.680203,0.504172}%
\pgfsetfillcolor{currentfill}%
\pgfsetlinewidth{0.000000pt}%
\definecolor{currentstroke}{rgb}{0.128729,0.563265,0.551229}%
\pgfsetstrokecolor{currentstroke}%
\pgfsetdash{}{0pt}%
\pgfpathmoveto{\pgfqpoint{5.308503in}{4.618780in}}%
\pgfpathlineto{\pgfqpoint{5.371612in}{4.650148in}}%
\pgfpathlineto{\pgfqpoint{5.449688in}{4.643592in}}%
\pgfpathclose%
\pgfusepath{fill}%
\end{pgfscope}%
\begin{pgfscope}%
\pgfpathrectangle{\pgfqpoint{0.680860in}{0.078740in}}{\pgfqpoint{7.842520in}{7.842520in}}%
\pgfusepath{clip}%
\pgfsetbuttcap%
\pgfsetroundjoin%
\definecolor{currentfill}{rgb}{0.136408,0.541173,0.554483}%
\pgfsetfillcolor{currentfill}%
\pgfsetlinewidth{0.000000pt}%
\definecolor{currentstroke}{rgb}{0.127568,0.566949,0.550556}%
\pgfsetstrokecolor{currentstroke}%
\pgfsetdash{}{0pt}%
\pgfpathmoveto{\pgfqpoint{3.546246in}{3.938673in}}%
\pgfpathlineto{\pgfqpoint{3.766552in}{4.075952in}}%
\pgfpathlineto{\pgfqpoint{3.629230in}{4.064809in}}%
\pgfpathclose%
\pgfusepath{fill}%
\end{pgfscope}%
\begin{pgfscope}%
\pgfpathrectangle{\pgfqpoint{0.680860in}{0.078740in}}{\pgfqpoint{7.842520in}{7.842520in}}%
\pgfusepath{clip}%
\pgfsetbuttcap%
\pgfsetroundjoin%
\definecolor{currentfill}{rgb}{0.272594,0.025563,0.353093}%
\pgfsetfillcolor{currentfill}%
\pgfsetlinewidth{0.000000pt}%
\definecolor{currentstroke}{rgb}{0.126453,0.570633,0.549841}%
\pgfsetstrokecolor{currentstroke}%
\pgfsetdash{}{0pt}%
\pgfpathmoveto{\pgfqpoint{1.762325in}{2.220810in}}%
\pgfpathlineto{\pgfqpoint{1.543995in}{2.056181in}}%
\pgfpathlineto{\pgfqpoint{1.678468in}{2.016431in}}%
\pgfpathclose%
\pgfusepath{fill}%
\end{pgfscope}%
\begin{pgfscope}%
\pgfpathrectangle{\pgfqpoint{0.680860in}{0.078740in}}{\pgfqpoint{7.842520in}{7.842520in}}%
\pgfusepath{clip}%
\pgfsetbuttcap%
\pgfsetroundjoin%
\definecolor{currentfill}{rgb}{0.180629,0.429975,0.557282}%
\pgfsetfillcolor{currentfill}%
\pgfsetlinewidth{0.000000pt}%
\definecolor{currentstroke}{rgb}{0.125394,0.574318,0.549086}%
\pgfsetstrokecolor{currentstroke}%
\pgfsetdash{}{0pt}%
\pgfpathmoveto{\pgfqpoint{3.023238in}{3.473471in}}%
\pgfpathlineto{\pgfqpoint{3.106565in}{3.637487in}}%
\pgfpathlineto{\pgfqpoint{2.887264in}{3.477230in}}%
\pgfpathclose%
\pgfusepath{fill}%
\end{pgfscope}%
\begin{pgfscope}%
\pgfpathrectangle{\pgfqpoint{0.680860in}{0.078740in}}{\pgfqpoint{7.842520in}{7.842520in}}%
\pgfusepath{clip}%
\pgfsetbuttcap%
\pgfsetroundjoin%
\definecolor{currentfill}{rgb}{0.218130,0.347432,0.550038}%
\pgfsetfillcolor{currentfill}%
\pgfsetlinewidth{0.000000pt}%
\definecolor{currentstroke}{rgb}{0.124395,0.578002,0.548287}%
\pgfsetstrokecolor{currentstroke}%
\pgfsetdash{}{0pt}%
\pgfpathmoveto{\pgfqpoint{2.584871in}{3.133059in}}%
\pgfpathlineto{\pgfqpoint{2.720500in}{3.120880in}}%
\pgfpathlineto{\pgfqpoint{2.668337in}{3.312877in}}%
\pgfpathclose%
\pgfusepath{fill}%
\end{pgfscope}%
\begin{pgfscope}%
\pgfpathrectangle{\pgfqpoint{0.680860in}{0.078740in}}{\pgfqpoint{7.842520in}{7.842520in}}%
\pgfusepath{clip}%
\pgfsetbuttcap%
\pgfsetroundjoin%
\definecolor{currentfill}{rgb}{0.281924,0.089666,0.412415}%
\pgfsetfillcolor{currentfill}%
\pgfsetlinewidth{0.000000pt}%
\definecolor{currentstroke}{rgb}{0.123463,0.581687,0.547445}%
\pgfsetstrokecolor{currentstroke}%
\pgfsetdash{}{0pt}%
\pgfpathmoveto{\pgfqpoint{1.897093in}{2.185546in}}%
\pgfpathlineto{\pgfqpoint{1.846106in}{2.422259in}}%
\pgfpathlineto{\pgfqpoint{1.762325in}{2.220810in}}%
\pgfpathclose%
\pgfusepath{fill}%
\end{pgfscope}%
\begin{pgfscope}%
\pgfpathrectangle{\pgfqpoint{0.680860in}{0.078740in}}{\pgfqpoint{7.842520in}{7.842520in}}%
\pgfusepath{clip}%
\pgfsetbuttcap%
\pgfsetroundjoin%
\definecolor{currentfill}{rgb}{0.120092,0.600104,0.542530}%
\pgfsetfillcolor{currentfill}%
\pgfsetlinewidth{0.000000pt}%
\definecolor{currentstroke}{rgb}{0.122606,0.585371,0.546557}%
\pgfsetstrokecolor{currentstroke}%
\pgfsetdash{}{0pt}%
\pgfpathmoveto{\pgfqpoint{4.207755in}{4.314588in}}%
\pgfpathlineto{\pgfqpoint{4.069441in}{4.297372in}}%
\pgfpathlineto{\pgfqpoint{3.987086in}{4.201876in}}%
\pgfpathclose%
\pgfusepath{fill}%
\end{pgfscope}%
\begin{pgfscope}%
\pgfpathrectangle{\pgfqpoint{0.680860in}{0.078740in}}{\pgfqpoint{7.842520in}{7.842520in}}%
\pgfusepath{clip}%
\pgfsetbuttcap%
\pgfsetroundjoin%
\definecolor{currentfill}{rgb}{0.157851,0.683765,0.501686}%
\pgfsetfillcolor{currentfill}%
\pgfsetlinewidth{0.000000pt}%
\definecolor{currentstroke}{rgb}{0.121831,0.589055,0.545623}%
\pgfsetstrokecolor{currentstroke}%
\pgfsetdash{}{0pt}%
\pgfpathmoveto{\pgfqpoint{5.810668in}{4.666739in}}%
\pgfpathlineto{\pgfqpoint{5.668435in}{4.640824in}}%
\pgfpathlineto{\pgfqpoint{5.591611in}{4.670570in}}%
\pgfpathclose%
\pgfusepath{fill}%
\end{pgfscope}%
\begin{pgfscope}%
\pgfpathrectangle{\pgfqpoint{0.680860in}{0.078740in}}{\pgfqpoint{7.842520in}{7.842520in}}%
\pgfusepath{clip}%
\pgfsetbuttcap%
\pgfsetroundjoin%
\definecolor{currentfill}{rgb}{0.126326,0.644107,0.525311}%
\pgfsetfillcolor{currentfill}%
\pgfsetlinewidth{0.000000pt}%
\definecolor{currentstroke}{rgb}{0.121148,0.592739,0.544641}%
\pgfsetstrokecolor{currentstroke}%
\pgfsetdash{}{0pt}%
\pgfpathmoveto{\pgfqpoint{4.649033in}{4.493129in}}%
\pgfpathlineto{\pgfqpoint{4.509769in}{4.471817in}}%
\pgfpathlineto{\pgfqpoint{4.567927in}{4.433785in}}%
\pgfpathclose%
\pgfusepath{fill}%
\end{pgfscope}%
\begin{pgfscope}%
\pgfpathrectangle{\pgfqpoint{0.680860in}{0.078740in}}{\pgfqpoint{7.842520in}{7.842520in}}%
\pgfusepath{clip}%
\pgfsetbuttcap%
\pgfsetroundjoin%
\definecolor{currentfill}{rgb}{0.130067,0.651384,0.521608}%
\pgfsetfillcolor{currentfill}%
\pgfsetlinewidth{0.000000pt}%
\definecolor{currentstroke}{rgb}{0.120565,0.596422,0.543611}%
\pgfsetstrokecolor{currentstroke}%
\pgfsetdash{}{0pt}%
\pgfpathmoveto{\pgfqpoint{7.250814in}{4.461584in}}%
\pgfpathlineto{\pgfqpoint{7.105170in}{4.441224in}}%
\pgfpathlineto{\pgfqpoint{7.038179in}{4.590300in}}%
\pgfpathclose%
\pgfusepath{fill}%
\end{pgfscope}%
\begin{pgfscope}%
\pgfpathrectangle{\pgfqpoint{0.680860in}{0.078740in}}{\pgfqpoint{7.842520in}{7.842520in}}%
\pgfusepath{clip}%
\pgfsetbuttcap%
\pgfsetroundjoin%
\definecolor{currentfill}{rgb}{0.210503,0.363727,0.552206}%
\pgfsetfillcolor{currentfill}%
\pgfsetlinewidth{0.000000pt}%
\definecolor{currentstroke}{rgb}{0.120092,0.600104,0.542530}%
\pgfsetstrokecolor{currentstroke}%
\pgfsetdash{}{0pt}%
\pgfpathmoveto{\pgfqpoint{2.668337in}{3.312877in}}%
\pgfpathlineto{\pgfqpoint{2.720500in}{3.120880in}}%
\pgfpathlineto{\pgfqpoint{2.803862in}{3.304510in}}%
\pgfpathclose%
\pgfusepath{fill}%
\end{pgfscope}%
\begin{pgfscope}%
\pgfpathrectangle{\pgfqpoint{0.680860in}{0.078740in}}{\pgfqpoint{7.842520in}{7.842520in}}%
\pgfusepath{clip}%
\pgfsetbuttcap%
\pgfsetroundjoin%
\definecolor{currentfill}{rgb}{0.146616,0.673050,0.508936}%
\pgfsetfillcolor{currentfill}%
\pgfsetlinewidth{0.000000pt}%
\definecolor{currentstroke}{rgb}{0.119738,0.603785,0.541400}%
\pgfsetstrokecolor{currentstroke}%
\pgfsetdash{}{0pt}%
\pgfpathmoveto{\pgfqpoint{5.230060in}{4.622919in}}%
\pgfpathlineto{\pgfqpoint{5.089240in}{4.597867in}}%
\pgfpathlineto{\pgfqpoint{5.009762in}{4.580065in}}%
\pgfpathclose%
\pgfusepath{fill}%
\end{pgfscope}%
\begin{pgfscope}%
\pgfpathrectangle{\pgfqpoint{0.680860in}{0.078740in}}{\pgfqpoint{7.842520in}{7.842520in}}%
\pgfusepath{clip}%
\pgfsetbuttcap%
\pgfsetroundjoin%
\definecolor{currentfill}{rgb}{0.283091,0.110553,0.431554}%
\pgfsetfillcolor{currentfill}%
\pgfsetlinewidth{0.000000pt}%
\definecolor{currentstroke}{rgb}{0.119512,0.607464,0.540218}%
\pgfsetstrokecolor{currentstroke}%
\pgfsetdash{}{0pt}%
\pgfpathmoveto{\pgfqpoint{1.980721in}{2.391422in}}%
\pgfpathlineto{\pgfqpoint{1.846106in}{2.422259in}}%
\pgfpathlineto{\pgfqpoint{1.897093in}{2.185546in}}%
\pgfpathclose%
\pgfusepath{fill}%
\end{pgfscope}%
\begin{pgfscope}%
\pgfpathrectangle{\pgfqpoint{0.680860in}{0.078740in}}{\pgfqpoint{7.842520in}{7.842520in}}%
\pgfusepath{clip}%
\pgfsetbuttcap%
\pgfsetroundjoin%
\definecolor{currentfill}{rgb}{0.273006,0.204520,0.501721}%
\pgfsetfillcolor{currentfill}%
\pgfsetlinewidth{0.000000pt}%
\definecolor{currentstroke}{rgb}{0.119423,0.611141,0.538982}%
\pgfsetstrokecolor{currentstroke}%
\pgfsetdash{}{0pt}%
\pgfpathmoveto{\pgfqpoint{2.199276in}{2.566915in}}%
\pgfpathlineto{\pgfqpoint{2.282743in}{2.767293in}}%
\pgfpathlineto{\pgfqpoint{2.064319in}{2.593024in}}%
\pgfpathclose%
\pgfusepath{fill}%
\end{pgfscope}%
\begin{pgfscope}%
\pgfpathrectangle{\pgfqpoint{0.680860in}{0.078740in}}{\pgfqpoint{7.842520in}{7.842520in}}%
\pgfusepath{clip}%
\pgfsetbuttcap%
\pgfsetroundjoin%
\definecolor{currentfill}{rgb}{0.153894,0.680203,0.504172}%
\pgfsetfillcolor{currentfill}%
\pgfsetlinewidth{0.000000pt}%
\definecolor{currentstroke}{rgb}{0.119483,0.614817,0.537692}%
\pgfsetstrokecolor{currentstroke}%
\pgfsetdash{}{0pt}%
\pgfpathmoveto{\pgfqpoint{5.308503in}{4.618780in}}%
\pgfpathlineto{\pgfqpoint{5.230060in}{4.622919in}}%
\pgfpathlineto{\pgfqpoint{5.371612in}{4.650148in}}%
\pgfpathclose%
\pgfusepath{fill}%
\end{pgfscope}%
\begin{pgfscope}%
\pgfpathrectangle{\pgfqpoint{0.680860in}{0.078740in}}{\pgfqpoint{7.842520in}{7.842520in}}%
\pgfusepath{clip}%
\pgfsetbuttcap%
\pgfsetroundjoin%
\definecolor{currentfill}{rgb}{0.123444,0.636809,0.528763}%
\pgfsetfillcolor{currentfill}%
\pgfsetlinewidth{0.000000pt}%
\definecolor{currentstroke}{rgb}{0.119699,0.618490,0.536347}%
\pgfsetstrokecolor{currentstroke}%
\pgfsetdash{}{0pt}%
\pgfpathmoveto{\pgfqpoint{4.567927in}{4.433785in}}%
\pgfpathlineto{\pgfqpoint{4.509769in}{4.471817in}}%
\pgfpathlineto{\pgfqpoint{4.428449in}{4.412261in}}%
\pgfpathclose%
\pgfusepath{fill}%
\end{pgfscope}%
\begin{pgfscope}%
\pgfpathrectangle{\pgfqpoint{0.680860in}{0.078740in}}{\pgfqpoint{7.842520in}{7.842520in}}%
\pgfusepath{clip}%
\pgfsetbuttcap%
\pgfsetroundjoin%
\definecolor{currentfill}{rgb}{0.257322,0.256130,0.526563}%
\pgfsetfillcolor{currentfill}%
\pgfsetlinewidth{0.000000pt}%
\definecolor{currentstroke}{rgb}{0.120081,0.622161,0.534946}%
\pgfsetstrokecolor{currentstroke}%
\pgfsetdash{}{0pt}%
\pgfpathmoveto{\pgfqpoint{2.366237in}{2.960748in}}%
\pgfpathlineto{\pgfqpoint{2.282743in}{2.767293in}}%
\pgfpathlineto{\pgfqpoint{2.418072in}{2.746048in}}%
\pgfpathclose%
\pgfusepath{fill}%
\end{pgfscope}%
\begin{pgfscope}%
\pgfpathrectangle{\pgfqpoint{0.680860in}{0.078740in}}{\pgfqpoint{7.842520in}{7.842520in}}%
\pgfusepath{clip}%
\pgfsetbuttcap%
\pgfsetroundjoin%
\definecolor{currentfill}{rgb}{0.134692,0.658636,0.517649}%
\pgfsetfillcolor{currentfill}%
\pgfsetlinewidth{0.000000pt}%
\definecolor{currentstroke}{rgb}{0.120638,0.625828,0.533488}%
\pgfsetstrokecolor{currentstroke}%
\pgfsetdash{}{0pt}%
\pgfpathmoveto{\pgfqpoint{6.892428in}{4.564701in}}%
\pgfpathlineto{\pgfqpoint{7.038179in}{4.590300in}}%
\pgfpathlineto{\pgfqpoint{7.105170in}{4.441224in}}%
\pgfpathclose%
\pgfusepath{fill}%
\end{pgfscope}%
\begin{pgfscope}%
\pgfpathrectangle{\pgfqpoint{0.680860in}{0.078740in}}{\pgfqpoint{7.842520in}{7.842520in}}%
\pgfusepath{clip}%
\pgfsetbuttcap%
\pgfsetroundjoin%
\definecolor{currentfill}{rgb}{0.163625,0.471133,0.558148}%
\pgfsetfillcolor{currentfill}%
\pgfsetlinewidth{0.000000pt}%
\definecolor{currentstroke}{rgb}{0.121380,0.629492,0.531973}%
\pgfsetstrokecolor{currentstroke}%
\pgfsetdash{}{0pt}%
\pgfpathmoveto{\pgfqpoint{3.106565in}{3.637487in}}%
\pgfpathlineto{\pgfqpoint{3.243011in}{3.638213in}}%
\pgfpathlineto{\pgfqpoint{3.326234in}{3.791903in}}%
\pgfpathclose%
\pgfusepath{fill}%
\end{pgfscope}%
\begin{pgfscope}%
\pgfpathrectangle{\pgfqpoint{0.680860in}{0.078740in}}{\pgfqpoint{7.842520in}{7.842520in}}%
\pgfusepath{clip}%
\pgfsetbuttcap%
\pgfsetroundjoin%
\definecolor{currentfill}{rgb}{0.162016,0.687316,0.499129}%
\pgfsetfillcolor{currentfill}%
\pgfsetlinewidth{0.000000pt}%
\definecolor{currentstroke}{rgb}{0.122312,0.633153,0.530398}%
\pgfsetstrokecolor{currentstroke}%
\pgfsetdash{}{0pt}%
\pgfpathmoveto{\pgfqpoint{5.810668in}{4.666739in}}%
\pgfpathlineto{\pgfqpoint{5.953658in}{4.694862in}}%
\pgfpathlineto{\pgfqpoint{6.028551in}{4.637580in}}%
\pgfpathclose%
\pgfusepath{fill}%
\end{pgfscope}%
\begin{pgfscope}%
\pgfpathrectangle{\pgfqpoint{0.680860in}{0.078740in}}{\pgfqpoint{7.842520in}{7.842520in}}%
\pgfusepath{clip}%
\pgfsetbuttcap%
\pgfsetroundjoin%
\definecolor{currentfill}{rgb}{0.134692,0.658636,0.517649}%
\pgfsetfillcolor{currentfill}%
\pgfsetlinewidth{0.000000pt}%
\definecolor{currentstroke}{rgb}{0.123444,0.636809,0.528763}%
\pgfsetstrokecolor{currentstroke}%
\pgfsetdash{}{0pt}%
\pgfpathmoveto{\pgfqpoint{4.869354in}{4.555516in}}%
\pgfpathlineto{\pgfqpoint{4.649033in}{4.493129in}}%
\pgfpathlineto{\pgfqpoint{4.788990in}{4.516484in}}%
\pgfpathclose%
\pgfusepath{fill}%
\end{pgfscope}%
\begin{pgfscope}%
\pgfpathrectangle{\pgfqpoint{0.680860in}{0.078740in}}{\pgfqpoint{7.842520in}{7.842520in}}%
\pgfusepath{clip}%
\pgfsetbuttcap%
\pgfsetroundjoin%
\definecolor{currentfill}{rgb}{0.120638,0.625828,0.533488}%
\pgfsetfillcolor{currentfill}%
\pgfsetlinewidth{0.000000pt}%
\definecolor{currentstroke}{rgb}{0.124780,0.640461,0.527068}%
\pgfsetstrokecolor{currentstroke}%
\pgfsetdash{}{0pt}%
\pgfpathmoveto{\pgfqpoint{4.346734in}{4.333704in}}%
\pgfpathlineto{\pgfqpoint{4.428449in}{4.412261in}}%
\pgfpathlineto{\pgfqpoint{4.289651in}{4.392719in}}%
\pgfpathclose%
\pgfusepath{fill}%
\end{pgfscope}%
\begin{pgfscope}%
\pgfpathrectangle{\pgfqpoint{0.680860in}{0.078740in}}{\pgfqpoint{7.842520in}{7.842520in}}%
\pgfusepath{clip}%
\pgfsetbuttcap%
\pgfsetroundjoin%
\definecolor{currentfill}{rgb}{0.119699,0.618490,0.536347}%
\pgfsetfillcolor{currentfill}%
\pgfsetlinewidth{0.000000pt}%
\definecolor{currentstroke}{rgb}{0.126326,0.644107,0.525311}%
\pgfsetstrokecolor{currentstroke}%
\pgfsetdash{}{0pt}%
\pgfpathmoveto{\pgfqpoint{4.289651in}{4.392719in}}%
\pgfpathlineto{\pgfqpoint{4.207755in}{4.314588in}}%
\pgfpathlineto{\pgfqpoint{4.346734in}{4.333704in}}%
\pgfpathclose%
\pgfusepath{fill}%
\end{pgfscope}%
\begin{pgfscope}%
\pgfpathrectangle{\pgfqpoint{0.680860in}{0.078740in}}{\pgfqpoint{7.842520in}{7.842520in}}%
\pgfusepath{clip}%
\pgfsetbuttcap%
\pgfsetroundjoin%
\definecolor{currentfill}{rgb}{0.166383,0.690856,0.496502}%
\pgfsetfillcolor{currentfill}%
\pgfsetlinewidth{0.000000pt}%
\definecolor{currentstroke}{rgb}{0.128087,0.647749,0.523491}%
\pgfsetstrokecolor{currentstroke}%
\pgfsetdash{}{0pt}%
\pgfpathmoveto{\pgfqpoint{6.171777in}{4.663723in}}%
\pgfpathlineto{\pgfqpoint{6.028551in}{4.637580in}}%
\pgfpathlineto{\pgfqpoint{5.953658in}{4.694862in}}%
\pgfpathclose%
\pgfusepath{fill}%
\end{pgfscope}%
\begin{pgfscope}%
\pgfpathrectangle{\pgfqpoint{0.680860in}{0.078740in}}{\pgfqpoint{7.842520in}{7.842520in}}%
\pgfusepath{clip}%
\pgfsetbuttcap%
\pgfsetroundjoin%
\definecolor{currentfill}{rgb}{0.277018,0.050344,0.375715}%
\pgfsetfillcolor{currentfill}%
\pgfsetlinewidth{0.000000pt}%
\definecolor{currentstroke}{rgb}{0.130067,0.651384,0.521608}%
\pgfsetstrokecolor{currentstroke}%
\pgfsetdash{}{0pt}%
\pgfpathmoveto{\pgfqpoint{1.897093in}{2.185546in}}%
\pgfpathlineto{\pgfqpoint{1.762325in}{2.220810in}}%
\pgfpathlineto{\pgfqpoint{1.678468in}{2.016431in}}%
\pgfpathclose%
\pgfusepath{fill}%
\end{pgfscope}%
\begin{pgfscope}%
\pgfpathrectangle{\pgfqpoint{0.680860in}{0.078740in}}{\pgfqpoint{7.842520in}{7.842520in}}%
\pgfusepath{clip}%
\pgfsetbuttcap%
\pgfsetroundjoin%
\definecolor{currentfill}{rgb}{0.195860,0.395433,0.555276}%
\pgfsetfillcolor{currentfill}%
\pgfsetlinewidth{0.000000pt}%
\definecolor{currentstroke}{rgb}{0.132268,0.655014,0.519661}%
\pgfsetstrokecolor{currentstroke}%
\pgfsetdash{}{0pt}%
\pgfpathmoveto{\pgfqpoint{2.887264in}{3.477230in}}%
\pgfpathlineto{\pgfqpoint{2.803862in}{3.304510in}}%
\pgfpathlineto{\pgfqpoint{2.939933in}{3.297162in}}%
\pgfpathclose%
\pgfusepath{fill}%
\end{pgfscope}%
\begin{pgfscope}%
\pgfpathrectangle{\pgfqpoint{0.680860in}{0.078740in}}{\pgfqpoint{7.842520in}{7.842520in}}%
\pgfusepath{clip}%
\pgfsetbuttcap%
\pgfsetroundjoin%
\definecolor{currentfill}{rgb}{0.250425,0.274290,0.533103}%
\pgfsetfillcolor{currentfill}%
\pgfsetlinewidth{0.000000pt}%
\definecolor{currentstroke}{rgb}{0.134692,0.658636,0.517649}%
\pgfsetstrokecolor{currentstroke}%
\pgfsetdash{}{0pt}%
\pgfpathmoveto{\pgfqpoint{2.418072in}{2.746048in}}%
\pgfpathlineto{\pgfqpoint{2.501451in}{2.943731in}}%
\pgfpathlineto{\pgfqpoint{2.366237in}{2.960748in}}%
\pgfpathclose%
\pgfusepath{fill}%
\end{pgfscope}%
\begin{pgfscope}%
\pgfpathrectangle{\pgfqpoint{0.680860in}{0.078740in}}{\pgfqpoint{7.842520in}{7.842520in}}%
\pgfusepath{clip}%
\pgfsetbuttcap%
\pgfsetroundjoin%
\definecolor{currentfill}{rgb}{0.282290,0.145912,0.461510}%
\pgfsetfillcolor{currentfill}%
\pgfsetlinewidth{0.000000pt}%
\definecolor{currentstroke}{rgb}{0.137339,0.662252,0.515571}%
\pgfsetstrokecolor{currentstroke}%
\pgfsetdash{}{0pt}%
\pgfpathmoveto{\pgfqpoint{1.980721in}{2.391422in}}%
\pgfpathlineto{\pgfqpoint{2.115813in}{2.360861in}}%
\pgfpathlineto{\pgfqpoint{2.064319in}{2.593024in}}%
\pgfpathclose%
\pgfusepath{fill}%
\end{pgfscope}%
\begin{pgfscope}%
\pgfpathrectangle{\pgfqpoint{0.680860in}{0.078740in}}{\pgfqpoint{7.842520in}{7.842520in}}%
\pgfusepath{clip}%
\pgfsetbuttcap%
\pgfsetroundjoin%
\definecolor{currentfill}{rgb}{0.127568,0.566949,0.550556}%
\pgfsetfillcolor{currentfill}%
\pgfsetlinewidth{0.000000pt}%
\definecolor{currentstroke}{rgb}{0.140210,0.665859,0.513427}%
\pgfsetstrokecolor{currentstroke}%
\pgfsetdash{}{0pt}%
\pgfpathmoveto{\pgfqpoint{3.849266in}{4.187481in}}%
\pgfpathlineto{\pgfqpoint{3.766552in}{4.075952in}}%
\pgfpathlineto{\pgfqpoint{3.904504in}{4.088783in}}%
\pgfpathclose%
\pgfusepath{fill}%
\end{pgfscope}%
\begin{pgfscope}%
\pgfpathrectangle{\pgfqpoint{0.680860in}{0.078740in}}{\pgfqpoint{7.842520in}{7.842520in}}%
\pgfusepath{clip}%
\pgfsetbuttcap%
\pgfsetroundjoin%
\definecolor{currentfill}{rgb}{0.124395,0.578002,0.548287}%
\pgfsetfillcolor{currentfill}%
\pgfsetlinewidth{0.000000pt}%
\definecolor{currentstroke}{rgb}{0.143303,0.669459,0.511215}%
\pgfsetstrokecolor{currentstroke}%
\pgfsetdash{}{0pt}%
\pgfpathmoveto{\pgfqpoint{3.904504in}{4.088783in}}%
\pgfpathlineto{\pgfqpoint{3.987086in}{4.201876in}}%
\pgfpathlineto{\pgfqpoint{3.849266in}{4.187481in}}%
\pgfpathclose%
\pgfusepath{fill}%
\end{pgfscope}%
\begin{pgfscope}%
\pgfpathrectangle{\pgfqpoint{0.680860in}{0.078740in}}{\pgfqpoint{7.842520in}{7.842520in}}%
\pgfusepath{clip}%
\pgfsetbuttcap%
\pgfsetroundjoin%
\definecolor{currentfill}{rgb}{0.146616,0.673050,0.508936}%
\pgfsetfillcolor{currentfill}%
\pgfsetlinewidth{0.000000pt}%
\definecolor{currentstroke}{rgb}{0.146616,0.673050,0.508936}%
\pgfsetstrokecolor{currentstroke}%
\pgfsetdash{}{0pt}%
\pgfpathmoveto{\pgfqpoint{6.892428in}{4.564701in}}%
\pgfpathlineto{\pgfqpoint{6.747468in}{4.541222in}}%
\pgfpathlineto{\pgfqpoint{6.677475in}{4.658470in}}%
\pgfpathclose%
\pgfusepath{fill}%
\end{pgfscope}%
\begin{pgfscope}%
\pgfpathrectangle{\pgfqpoint{0.680860in}{0.078740in}}{\pgfqpoint{7.842520in}{7.842520in}}%
\pgfusepath{clip}%
\pgfsetbuttcap%
\pgfsetroundjoin%
\definecolor{currentfill}{rgb}{0.172719,0.448791,0.557885}%
\pgfsetfillcolor{currentfill}%
\pgfsetlinewidth{0.000000pt}%
\definecolor{currentstroke}{rgb}{0.150148,0.676631,0.506589}%
\pgfsetstrokecolor{currentstroke}%
\pgfsetdash{}{0pt}%
\pgfpathmoveto{\pgfqpoint{3.243011in}{3.638213in}}%
\pgfpathlineto{\pgfqpoint{3.106565in}{3.637487in}}%
\pgfpathlineto{\pgfqpoint{3.023238in}{3.473471in}}%
\pgfpathclose%
\pgfusepath{fill}%
\end{pgfscope}%
\begin{pgfscope}%
\pgfpathrectangle{\pgfqpoint{0.680860in}{0.078740in}}{\pgfqpoint{7.842520in}{7.842520in}}%
\pgfusepath{clip}%
\pgfsetbuttcap%
\pgfsetroundjoin%
\definecolor{currentfill}{rgb}{0.153894,0.680203,0.504172}%
\pgfsetfillcolor{currentfill}%
\pgfsetlinewidth{0.000000pt}%
\definecolor{currentstroke}{rgb}{0.153894,0.680203,0.504172}%
\pgfsetstrokecolor{currentstroke}%
\pgfsetdash{}{0pt}%
\pgfpathmoveto{\pgfqpoint{6.677475in}{4.658470in}}%
\pgfpathlineto{\pgfqpoint{6.747468in}{4.541222in}}%
\pgfpathlineto{\pgfqpoint{6.532545in}{4.630913in}}%
\pgfpathclose%
\pgfusepath{fill}%
\end{pgfscope}%
\begin{pgfscope}%
\pgfpathrectangle{\pgfqpoint{0.680860in}{0.078740in}}{\pgfqpoint{7.842520in}{7.842520in}}%
\pgfusepath{clip}%
\pgfsetbuttcap%
\pgfsetroundjoin%
\definecolor{currentfill}{rgb}{0.162016,0.687316,0.499129}%
\pgfsetfillcolor{currentfill}%
\pgfsetlinewidth{0.000000pt}%
\definecolor{currentstroke}{rgb}{0.157851,0.683765,0.501686}%
\pgfsetstrokecolor{currentstroke}%
\pgfsetdash{}{0pt}%
\pgfpathmoveto{\pgfqpoint{6.315778in}{4.692081in}}%
\pgfpathlineto{\pgfqpoint{6.388401in}{4.605543in}}%
\pgfpathlineto{\pgfqpoint{6.171777in}{4.663723in}}%
\pgfpathclose%
\pgfusepath{fill}%
\end{pgfscope}%
\begin{pgfscope}%
\pgfpathrectangle{\pgfqpoint{0.680860in}{0.078740in}}{\pgfqpoint{7.842520in}{7.842520in}}%
\pgfusepath{clip}%
\pgfsetbuttcap%
\pgfsetroundjoin%
\definecolor{currentfill}{rgb}{0.188923,0.410910,0.556326}%
\pgfsetfillcolor{currentfill}%
\pgfsetlinewidth{0.000000pt}%
\definecolor{currentstroke}{rgb}{0.162016,0.687316,0.499129}%
\pgfsetstrokecolor{currentstroke}%
\pgfsetdash{}{0pt}%
\pgfpathmoveto{\pgfqpoint{2.887264in}{3.477230in}}%
\pgfpathlineto{\pgfqpoint{2.939933in}{3.297162in}}%
\pgfpathlineto{\pgfqpoint{3.023238in}{3.473471in}}%
\pgfpathclose%
\pgfusepath{fill}%
\end{pgfscope}%
\begin{pgfscope}%
\pgfpathrectangle{\pgfqpoint{0.680860in}{0.078740in}}{\pgfqpoint{7.842520in}{7.842520in}}%
\pgfusepath{clip}%
\pgfsetbuttcap%
\pgfsetroundjoin%
\definecolor{currentfill}{rgb}{0.162016,0.687316,0.499129}%
\pgfsetfillcolor{currentfill}%
\pgfsetlinewidth{0.000000pt}%
\definecolor{currentstroke}{rgb}{0.166383,0.690856,0.496502}%
\pgfsetstrokecolor{currentstroke}%
\pgfsetdash{}{0pt}%
\pgfpathmoveto{\pgfqpoint{6.532545in}{4.630913in}}%
\pgfpathlineto{\pgfqpoint{6.388401in}{4.605543in}}%
\pgfpathlineto{\pgfqpoint{6.315778in}{4.692081in}}%
\pgfpathclose%
\pgfusepath{fill}%
\end{pgfscope}%
\begin{pgfscope}%
\pgfpathrectangle{\pgfqpoint{0.680860in}{0.078740in}}{\pgfqpoint{7.842520in}{7.842520in}}%
\pgfusepath{clip}%
\pgfsetbuttcap%
\pgfsetroundjoin%
\definecolor{currentfill}{rgb}{0.149039,0.508051,0.557250}%
\pgfsetfillcolor{currentfill}%
\pgfsetlinewidth{0.000000pt}%
\definecolor{currentstroke}{rgb}{0.170948,0.694384,0.493803}%
\pgfsetstrokecolor{currentstroke}%
\pgfsetdash{}{0pt}%
\pgfpathmoveto{\pgfqpoint{3.546246in}{3.938673in}}%
\pgfpathlineto{\pgfqpoint{3.326234in}{3.791903in}}%
\pgfpathlineto{\pgfqpoint{3.463170in}{3.796932in}}%
\pgfpathclose%
\pgfusepath{fill}%
\end{pgfscope}%
\begin{pgfscope}%
\pgfpathrectangle{\pgfqpoint{0.680860in}{0.078740in}}{\pgfqpoint{7.842520in}{7.842520in}}%
\pgfusepath{clip}%
\pgfsetbuttcap%
\pgfsetroundjoin%
\definecolor{currentfill}{rgb}{0.280255,0.165693,0.476498}%
\pgfsetfillcolor{currentfill}%
\pgfsetlinewidth{0.000000pt}%
\definecolor{currentstroke}{rgb}{0.175707,0.697900,0.491033}%
\pgfsetstrokecolor{currentstroke}%
\pgfsetdash{}{0pt}%
\pgfpathmoveto{\pgfqpoint{2.064319in}{2.593024in}}%
\pgfpathlineto{\pgfqpoint{2.115813in}{2.360861in}}%
\pgfpathlineto{\pgfqpoint{2.199276in}{2.566915in}}%
\pgfpathclose%
\pgfusepath{fill}%
\end{pgfscope}%
\begin{pgfscope}%
\pgfpathrectangle{\pgfqpoint{0.680860in}{0.078740in}}{\pgfqpoint{7.842520in}{7.842520in}}%
\pgfusepath{clip}%
\pgfsetbuttcap%
\pgfsetroundjoin%
\definecolor{currentfill}{rgb}{0.124780,0.640461,0.527068}%
\pgfsetfillcolor{currentfill}%
\pgfsetlinewidth{0.000000pt}%
\definecolor{currentstroke}{rgb}{0.180653,0.701402,0.488189}%
\pgfsetstrokecolor{currentstroke}%
\pgfsetdash{}{0pt}%
\pgfpathmoveto{\pgfqpoint{7.460861in}{4.302331in}}%
\pgfpathlineto{\pgfqpoint{7.397248in}{4.483949in}}%
\pgfpathlineto{\pgfqpoint{7.544489in}{4.508396in}}%
\pgfpathclose%
\pgfusepath{fill}%
\end{pgfscope}%
\begin{pgfscope}%
\pgfpathrectangle{\pgfqpoint{0.680860in}{0.078740in}}{\pgfqpoint{7.842520in}{7.842520in}}%
\pgfusepath{clip}%
\pgfsetbuttcap%
\pgfsetroundjoin%
\definecolor{currentfill}{rgb}{0.267968,0.223549,0.512008}%
\pgfsetfillcolor{currentfill}%
\pgfsetlinewidth{0.000000pt}%
\definecolor{currentstroke}{rgb}{0.185783,0.704891,0.485273}%
\pgfsetstrokecolor{currentstroke}%
\pgfsetdash{}{0pt}%
\pgfpathmoveto{\pgfqpoint{2.418072in}{2.746048in}}%
\pgfpathlineto{\pgfqpoint{2.282743in}{2.767293in}}%
\pgfpathlineto{\pgfqpoint{2.199276in}{2.566915in}}%
\pgfpathclose%
\pgfusepath{fill}%
\end{pgfscope}%
\begin{pgfscope}%
\pgfpathrectangle{\pgfqpoint{0.680860in}{0.078740in}}{\pgfqpoint{7.842520in}{7.842520in}}%
\pgfusepath{clip}%
\pgfsetbuttcap%
\pgfsetroundjoin%
\definecolor{currentfill}{rgb}{0.235526,0.309527,0.542944}%
\pgfsetfillcolor{currentfill}%
\pgfsetlinewidth{0.000000pt}%
\definecolor{currentstroke}{rgb}{0.191090,0.708366,0.482284}%
\pgfsetstrokecolor{currentstroke}%
\pgfsetdash{}{0pt}%
\pgfpathmoveto{\pgfqpoint{2.584871in}{3.133059in}}%
\pgfpathlineto{\pgfqpoint{2.501451in}{2.943731in}}%
\pgfpathlineto{\pgfqpoint{2.637183in}{2.927444in}}%
\pgfpathclose%
\pgfusepath{fill}%
\end{pgfscope}%
\begin{pgfscope}%
\pgfpathrectangle{\pgfqpoint{0.680860in}{0.078740in}}{\pgfqpoint{7.842520in}{7.842520in}}%
\pgfusepath{clip}%
\pgfsetbuttcap%
\pgfsetroundjoin%
\definecolor{currentfill}{rgb}{0.136408,0.541173,0.554483}%
\pgfsetfillcolor{currentfill}%
\pgfsetlinewidth{0.000000pt}%
\definecolor{currentstroke}{rgb}{0.196571,0.711827,0.479221}%
\pgfsetstrokecolor{currentstroke}%
\pgfsetdash{}{0pt}%
\pgfpathmoveto{\pgfqpoint{3.683685in}{3.947759in}}%
\pgfpathlineto{\pgfqpoint{3.766552in}{4.075952in}}%
\pgfpathlineto{\pgfqpoint{3.546246in}{3.938673in}}%
\pgfpathclose%
\pgfusepath{fill}%
\end{pgfscope}%
\begin{pgfscope}%
\pgfpathrectangle{\pgfqpoint{0.680860in}{0.078740in}}{\pgfqpoint{7.842520in}{7.842520in}}%
\pgfusepath{clip}%
\pgfsetbuttcap%
\pgfsetroundjoin%
\definecolor{currentfill}{rgb}{0.283091,0.110553,0.431554}%
\pgfsetfillcolor{currentfill}%
\pgfsetlinewidth{0.000000pt}%
\definecolor{currentstroke}{rgb}{0.202219,0.715272,0.476084}%
\pgfsetstrokecolor{currentstroke}%
\pgfsetdash{}{0pt}%
\pgfpathmoveto{\pgfqpoint{1.980721in}{2.391422in}}%
\pgfpathlineto{\pgfqpoint{1.897093in}{2.185546in}}%
\pgfpathlineto{\pgfqpoint{2.115813in}{2.360861in}}%
\pgfpathclose%
\pgfusepath{fill}%
\end{pgfscope}%
\begin{pgfscope}%
\pgfpathrectangle{\pgfqpoint{0.680860in}{0.078740in}}{\pgfqpoint{7.842520in}{7.842520in}}%
\pgfusepath{clip}%
\pgfsetbuttcap%
\pgfsetroundjoin%
\definecolor{currentfill}{rgb}{0.227802,0.326594,0.546532}%
\pgfsetfillcolor{currentfill}%
\pgfsetlinewidth{0.000000pt}%
\definecolor{currentstroke}{rgb}{0.208030,0.718701,0.472873}%
\pgfsetstrokecolor{currentstroke}%
\pgfsetdash{}{0pt}%
\pgfpathmoveto{\pgfqpoint{2.637183in}{2.927444in}}%
\pgfpathlineto{\pgfqpoint{2.720500in}{3.120880in}}%
\pgfpathlineto{\pgfqpoint{2.584871in}{3.133059in}}%
\pgfpathclose%
\pgfusepath{fill}%
\end{pgfscope}%
\begin{pgfscope}%
\pgfpathrectangle{\pgfqpoint{0.680860in}{0.078740in}}{\pgfqpoint{7.842520in}{7.842520in}}%
\pgfusepath{clip}%
\pgfsetbuttcap%
\pgfsetroundjoin%
\definecolor{currentfill}{rgb}{0.272594,0.025563,0.353093}%
\pgfsetfillcolor{currentfill}%
\pgfsetlinewidth{0.000000pt}%
\definecolor{currentstroke}{rgb}{0.214000,0.722114,0.469588}%
\pgfsetstrokecolor{currentstroke}%
\pgfsetdash{}{0pt}%
\pgfpathmoveto{\pgfqpoint{1.678468in}{2.016431in}}%
\pgfpathlineto{\pgfqpoint{1.813395in}{1.976671in}}%
\pgfpathlineto{\pgfqpoint{1.897093in}{2.185546in}}%
\pgfpathclose%
\pgfusepath{fill}%
\end{pgfscope}%
\begin{pgfscope}%
\pgfpathrectangle{\pgfqpoint{0.680860in}{0.078740in}}{\pgfqpoint{7.842520in}{7.842520in}}%
\pgfusepath{clip}%
\pgfsetbuttcap%
\pgfsetroundjoin%
\definecolor{currentfill}{rgb}{0.120092,0.600104,0.542530}%
\pgfsetfillcolor{currentfill}%
\pgfsetlinewidth{0.000000pt}%
\definecolor{currentstroke}{rgb}{0.220124,0.725509,0.466226}%
\pgfsetstrokecolor{currentstroke}%
\pgfsetdash{}{0pt}%
\pgfpathmoveto{\pgfqpoint{3.987086in}{4.201876in}}%
\pgfpathlineto{\pgfqpoint{4.125553in}{4.218073in}}%
\pgfpathlineto{\pgfqpoint{4.207755in}{4.314588in}}%
\pgfpathclose%
\pgfusepath{fill}%
\end{pgfscope}%
\begin{pgfscope}%
\pgfpathrectangle{\pgfqpoint{0.680860in}{0.078740in}}{\pgfqpoint{7.842520in}{7.842520in}}%
\pgfusepath{clip}%
\pgfsetbuttcap%
\pgfsetroundjoin%
\definecolor{currentfill}{rgb}{0.206756,0.371758,0.553117}%
\pgfsetfillcolor{currentfill}%
\pgfsetlinewidth{0.000000pt}%
\definecolor{currentstroke}{rgb}{0.226397,0.728888,0.462789}%
\pgfsetstrokecolor{currentstroke}%
\pgfsetdash{}{0pt}%
\pgfpathmoveto{\pgfqpoint{2.939933in}{3.297162in}}%
\pgfpathlineto{\pgfqpoint{2.803862in}{3.304510in}}%
\pgfpathlineto{\pgfqpoint{2.720500in}{3.120880in}}%
\pgfpathclose%
\pgfusepath{fill}%
\end{pgfscope}%
\begin{pgfscope}%
\pgfpathrectangle{\pgfqpoint{0.680860in}{0.078740in}}{\pgfqpoint{7.842520in}{7.842520in}}%
\pgfusepath{clip}%
\pgfsetbuttcap%
\pgfsetroundjoin%
\definecolor{currentfill}{rgb}{0.132268,0.655014,0.519661}%
\pgfsetfillcolor{currentfill}%
\pgfsetlinewidth{0.000000pt}%
\definecolor{currentstroke}{rgb}{0.232815,0.732247,0.459277}%
\pgfsetstrokecolor{currentstroke}%
\pgfsetdash{}{0pt}%
\pgfpathmoveto{\pgfqpoint{4.788990in}{4.516484in}}%
\pgfpathlineto{\pgfqpoint{4.649033in}{4.493129in}}%
\pgfpathlineto{\pgfqpoint{4.567927in}{4.433785in}}%
\pgfpathclose%
\pgfusepath{fill}%
\end{pgfscope}%
\begin{pgfscope}%
\pgfpathrectangle{\pgfqpoint{0.680860in}{0.078740in}}{\pgfqpoint{7.842520in}{7.842520in}}%
\pgfusepath{clip}%
\pgfsetbuttcap%
\pgfsetroundjoin%
\definecolor{currentfill}{rgb}{0.160665,0.478540,0.558115}%
\pgfsetfillcolor{currentfill}%
\pgfsetlinewidth{0.000000pt}%
\definecolor{currentstroke}{rgb}{0.239374,0.735588,0.455688}%
\pgfsetstrokecolor{currentstroke}%
\pgfsetdash{}{0pt}%
\pgfpathmoveto{\pgfqpoint{3.243011in}{3.638213in}}%
\pgfpathlineto{\pgfqpoint{3.380042in}{3.640270in}}%
\pgfpathlineto{\pgfqpoint{3.326234in}{3.791903in}}%
\pgfpathclose%
\pgfusepath{fill}%
\end{pgfscope}%
\begin{pgfscope}%
\pgfpathrectangle{\pgfqpoint{0.680860in}{0.078740in}}{\pgfqpoint{7.842520in}{7.842520in}}%
\pgfusepath{clip}%
\pgfsetbuttcap%
\pgfsetroundjoin%
\definecolor{currentfill}{rgb}{0.146616,0.673050,0.508936}%
\pgfsetfillcolor{currentfill}%
\pgfsetlinewidth{0.000000pt}%
\definecolor{currentstroke}{rgb}{0.246070,0.738910,0.452024}%
\pgfsetstrokecolor{currentstroke}%
\pgfsetdash{}{0pt}%
\pgfpathmoveto{\pgfqpoint{4.869354in}{4.555516in}}%
\pgfpathlineto{\pgfqpoint{4.929657in}{4.541960in}}%
\pgfpathlineto{\pgfqpoint{5.009762in}{4.580065in}}%
\pgfpathclose%
\pgfusepath{fill}%
\end{pgfscope}%
\begin{pgfscope}%
\pgfpathrectangle{\pgfqpoint{0.680860in}{0.078740in}}{\pgfqpoint{7.842520in}{7.842520in}}%
\pgfusepath{clip}%
\pgfsetbuttcap%
\pgfsetroundjoin%
\definecolor{currentfill}{rgb}{0.248629,0.278775,0.534556}%
\pgfsetfillcolor{currentfill}%
\pgfsetlinewidth{0.000000pt}%
\definecolor{currentstroke}{rgb}{0.252899,0.742211,0.448284}%
\pgfsetstrokecolor{currentstroke}%
\pgfsetdash{}{0pt}%
\pgfpathmoveto{\pgfqpoint{2.418072in}{2.746048in}}%
\pgfpathlineto{\pgfqpoint{2.637183in}{2.927444in}}%
\pgfpathlineto{\pgfqpoint{2.501451in}{2.943731in}}%
\pgfpathclose%
\pgfusepath{fill}%
\end{pgfscope}%
\begin{pgfscope}%
\pgfpathrectangle{\pgfqpoint{0.680860in}{0.078740in}}{\pgfqpoint{7.842520in}{7.842520in}}%
\pgfusepath{clip}%
\pgfsetbuttcap%
\pgfsetroundjoin%
\definecolor{currentfill}{rgb}{0.175841,0.441290,0.557685}%
\pgfsetfillcolor{currentfill}%
\pgfsetlinewidth{0.000000pt}%
\definecolor{currentstroke}{rgb}{0.259857,0.745492,0.444467}%
\pgfsetstrokecolor{currentstroke}%
\pgfsetdash{}{0pt}%
\pgfpathmoveto{\pgfqpoint{3.243011in}{3.638213in}}%
\pgfpathlineto{\pgfqpoint{3.023238in}{3.473471in}}%
\pgfpathlineto{\pgfqpoint{3.159778in}{3.470887in}}%
\pgfpathclose%
\pgfusepath{fill}%
\end{pgfscope}%
\begin{pgfscope}%
\pgfpathrectangle{\pgfqpoint{0.680860in}{0.078740in}}{\pgfqpoint{7.842520in}{7.842520in}}%
\pgfusepath{clip}%
\pgfsetbuttcap%
\pgfsetroundjoin%
\definecolor{currentfill}{rgb}{0.154815,0.493313,0.557840}%
\pgfsetfillcolor{currentfill}%
\pgfsetlinewidth{0.000000pt}%
\definecolor{currentstroke}{rgb}{0.266941,0.748751,0.440573}%
\pgfsetstrokecolor{currentstroke}%
\pgfsetdash{}{0pt}%
\pgfpathmoveto{\pgfqpoint{3.326234in}{3.791903in}}%
\pgfpathlineto{\pgfqpoint{3.380042in}{3.640270in}}%
\pgfpathlineto{\pgfqpoint{3.463170in}{3.796932in}}%
\pgfpathclose%
\pgfusepath{fill}%
\end{pgfscope}%
\begin{pgfscope}%
\pgfpathrectangle{\pgfqpoint{0.680860in}{0.078740in}}{\pgfqpoint{7.842520in}{7.842520in}}%
\pgfusepath{clip}%
\pgfsetbuttcap%
\pgfsetroundjoin%
\definecolor{currentfill}{rgb}{0.271828,0.209303,0.504434}%
\pgfsetfillcolor{currentfill}%
\pgfsetlinewidth{0.000000pt}%
\definecolor{currentstroke}{rgb}{0.274149,0.751988,0.436601}%
\pgfsetstrokecolor{currentstroke}%
\pgfsetdash{}{0pt}%
\pgfpathmoveto{\pgfqpoint{2.199276in}{2.566915in}}%
\pgfpathlineto{\pgfqpoint{2.334724in}{2.541238in}}%
\pgfpathlineto{\pgfqpoint{2.418072in}{2.746048in}}%
\pgfpathclose%
\pgfusepath{fill}%
\end{pgfscope}%
\begin{pgfscope}%
\pgfpathrectangle{\pgfqpoint{0.680860in}{0.078740in}}{\pgfqpoint{7.842520in}{7.842520in}}%
\pgfusepath{clip}%
\pgfsetbuttcap%
\pgfsetroundjoin%
\definecolor{currentfill}{rgb}{0.123444,0.636809,0.528763}%
\pgfsetfillcolor{currentfill}%
\pgfsetlinewidth{0.000000pt}%
\definecolor{currentstroke}{rgb}{0.281477,0.755203,0.432552}%
\pgfsetstrokecolor{currentstroke}%
\pgfsetdash{}{0pt}%
\pgfpathmoveto{\pgfqpoint{4.567927in}{4.433785in}}%
\pgfpathlineto{\pgfqpoint{4.428449in}{4.412261in}}%
\pgfpathlineto{\pgfqpoint{4.346734in}{4.333704in}}%
\pgfpathclose%
\pgfusepath{fill}%
\end{pgfscope}%
\begin{pgfscope}%
\pgfpathrectangle{\pgfqpoint{0.680860in}{0.078740in}}{\pgfqpoint{7.842520in}{7.842520in}}%
\pgfusepath{clip}%
\pgfsetbuttcap%
\pgfsetroundjoin%
\definecolor{currentfill}{rgb}{0.143303,0.669459,0.511215}%
\pgfsetfillcolor{currentfill}%
\pgfsetlinewidth{0.000000pt}%
\definecolor{currentstroke}{rgb}{0.288921,0.758394,0.428426}%
\pgfsetstrokecolor{currentstroke}%
\pgfsetdash{}{0pt}%
\pgfpathmoveto{\pgfqpoint{4.788990in}{4.516484in}}%
\pgfpathlineto{\pgfqpoint{4.929657in}{4.541960in}}%
\pgfpathlineto{\pgfqpoint{4.869354in}{4.555516in}}%
\pgfpathclose%
\pgfusepath{fill}%
\end{pgfscope}%
\begin{pgfscope}%
\pgfpathrectangle{\pgfqpoint{0.680860in}{0.078740in}}{\pgfqpoint{7.842520in}{7.842520in}}%
\pgfusepath{clip}%
\pgfsetbuttcap%
\pgfsetroundjoin%
\definecolor{currentfill}{rgb}{0.157851,0.683765,0.501686}%
\pgfsetfillcolor{currentfill}%
\pgfsetlinewidth{0.000000pt}%
\definecolor{currentstroke}{rgb}{0.296479,0.761561,0.424223}%
\pgfsetstrokecolor{currentstroke}%
\pgfsetdash{}{0pt}%
\pgfpathmoveto{\pgfqpoint{5.009762in}{4.580065in}}%
\pgfpathlineto{\pgfqpoint{5.150891in}{4.606777in}}%
\pgfpathlineto{\pgfqpoint{5.230060in}{4.622919in}}%
\pgfpathclose%
\pgfusepath{fill}%
\end{pgfscope}%
\begin{pgfscope}%
\pgfpathrectangle{\pgfqpoint{0.680860in}{0.078740in}}{\pgfqpoint{7.842520in}{7.842520in}}%
\pgfusepath{clip}%
\pgfsetbuttcap%
\pgfsetroundjoin%
\definecolor{currentfill}{rgb}{0.279574,0.170599,0.479997}%
\pgfsetfillcolor{currentfill}%
\pgfsetlinewidth{0.000000pt}%
\definecolor{currentstroke}{rgb}{0.304148,0.764704,0.419943}%
\pgfsetstrokecolor{currentstroke}%
\pgfsetdash{}{0pt}%
\pgfpathmoveto{\pgfqpoint{2.199276in}{2.566915in}}%
\pgfpathlineto{\pgfqpoint{2.115813in}{2.360861in}}%
\pgfpathlineto{\pgfqpoint{2.334724in}{2.541238in}}%
\pgfpathclose%
\pgfusepath{fill}%
\end{pgfscope}%
\begin{pgfscope}%
\pgfpathrectangle{\pgfqpoint{0.680860in}{0.078740in}}{\pgfqpoint{7.842520in}{7.842520in}}%
\pgfusepath{clip}%
\pgfsetbuttcap%
\pgfsetroundjoin%
\definecolor{currentfill}{rgb}{0.180653,0.701402,0.488189}%
\pgfsetfillcolor{currentfill}%
\pgfsetlinewidth{0.000000pt}%
\definecolor{currentstroke}{rgb}{0.311925,0.767822,0.415586}%
\pgfsetstrokecolor{currentstroke}%
\pgfsetdash{}{0pt}%
\pgfpathmoveto{\pgfqpoint{5.591611in}{4.670570in}}%
\pgfpathlineto{\pgfqpoint{5.734289in}{4.699798in}}%
\pgfpathlineto{\pgfqpoint{5.810668in}{4.666739in}}%
\pgfpathclose%
\pgfusepath{fill}%
\end{pgfscope}%
\begin{pgfscope}%
\pgfpathrectangle{\pgfqpoint{0.680860in}{0.078740in}}{\pgfqpoint{7.842520in}{7.842520in}}%
\pgfusepath{clip}%
\pgfsetbuttcap%
\pgfsetroundjoin%
\definecolor{currentfill}{rgb}{0.282327,0.094955,0.417331}%
\pgfsetfillcolor{currentfill}%
\pgfsetlinewidth{0.000000pt}%
\definecolor{currentstroke}{rgb}{0.319809,0.770914,0.411152}%
\pgfsetstrokecolor{currentstroke}%
\pgfsetdash{}{0pt}%
\pgfpathmoveto{\pgfqpoint{2.115813in}{2.360861in}}%
\pgfpathlineto{\pgfqpoint{1.897093in}{2.185546in}}%
\pgfpathlineto{\pgfqpoint{2.032327in}{2.150418in}}%
\pgfpathclose%
\pgfusepath{fill}%
\end{pgfscope}%
\begin{pgfscope}%
\pgfpathrectangle{\pgfqpoint{0.680860in}{0.078740in}}{\pgfqpoint{7.842520in}{7.842520in}}%
\pgfusepath{clip}%
\pgfsetbuttcap%
\pgfsetroundjoin%
\definecolor{currentfill}{rgb}{0.175707,0.697900,0.491033}%
\pgfsetfillcolor{currentfill}%
\pgfsetlinewidth{0.000000pt}%
\definecolor{currentstroke}{rgb}{0.327796,0.773980,0.406640}%
\pgfsetstrokecolor{currentstroke}%
\pgfsetdash{}{0pt}%
\pgfpathmoveto{\pgfqpoint{5.513913in}{4.679638in}}%
\pgfpathlineto{\pgfqpoint{5.591611in}{4.670570in}}%
\pgfpathlineto{\pgfqpoint{5.449688in}{4.643592in}}%
\pgfpathclose%
\pgfusepath{fill}%
\end{pgfscope}%
\begin{pgfscope}%
\pgfpathrectangle{\pgfqpoint{0.680860in}{0.078740in}}{\pgfqpoint{7.842520in}{7.842520in}}%
\pgfusepath{clip}%
\pgfsetbuttcap%
\pgfsetroundjoin%
\definecolor{currentfill}{rgb}{0.146180,0.515413,0.556823}%
\pgfsetfillcolor{currentfill}%
\pgfsetlinewidth{0.000000pt}%
\definecolor{currentstroke}{rgb}{0.335885,0.777018,0.402049}%
\pgfsetstrokecolor{currentstroke}%
\pgfsetdash{}{0pt}%
\pgfpathmoveto{\pgfqpoint{3.463170in}{3.796932in}}%
\pgfpathlineto{\pgfqpoint{3.600710in}{3.803441in}}%
\pgfpathlineto{\pgfqpoint{3.546246in}{3.938673in}}%
\pgfpathclose%
\pgfusepath{fill}%
\end{pgfscope}%
\begin{pgfscope}%
\pgfpathrectangle{\pgfqpoint{0.680860in}{0.078740in}}{\pgfqpoint{7.842520in}{7.842520in}}%
\pgfusepath{clip}%
\pgfsetbuttcap%
\pgfsetroundjoin%
\definecolor{currentfill}{rgb}{0.140536,0.530132,0.555659}%
\pgfsetfillcolor{currentfill}%
\pgfsetlinewidth{0.000000pt}%
\definecolor{currentstroke}{rgb}{0.344074,0.780029,0.397381}%
\pgfsetstrokecolor{currentstroke}%
\pgfsetdash{}{0pt}%
\pgfpathmoveto{\pgfqpoint{3.600710in}{3.803441in}}%
\pgfpathlineto{\pgfqpoint{3.683685in}{3.947759in}}%
\pgfpathlineto{\pgfqpoint{3.546246in}{3.938673in}}%
\pgfpathclose%
\pgfusepath{fill}%
\end{pgfscope}%
\begin{pgfscope}%
\pgfpathrectangle{\pgfqpoint{0.680860in}{0.078740in}}{\pgfqpoint{7.842520in}{7.842520in}}%
\pgfusepath{clip}%
\pgfsetbuttcap%
\pgfsetroundjoin%
\definecolor{currentfill}{rgb}{0.143303,0.669459,0.511215}%
\pgfsetfillcolor{currentfill}%
\pgfsetlinewidth{0.000000pt}%
\definecolor{currentstroke}{rgb}{0.352360,0.783011,0.392636}%
\pgfsetstrokecolor{currentstroke}%
\pgfsetdash{}{0pt}%
\pgfpathmoveto{\pgfqpoint{7.397248in}{4.483949in}}%
\pgfpathlineto{\pgfqpoint{7.250814in}{4.461584in}}%
\pgfpathlineto{\pgfqpoint{7.184740in}{4.618099in}}%
\pgfpathclose%
\pgfusepath{fill}%
\end{pgfscope}%
\begin{pgfscope}%
\pgfpathrectangle{\pgfqpoint{0.680860in}{0.078740in}}{\pgfqpoint{7.842520in}{7.842520in}}%
\pgfusepath{clip}%
\pgfsetbuttcap%
\pgfsetroundjoin%
\definecolor{currentfill}{rgb}{0.272594,0.025563,0.353093}%
\pgfsetfillcolor{currentfill}%
\pgfsetlinewidth{0.000000pt}%
\definecolor{currentstroke}{rgb}{0.360741,0.785964,0.387814}%
\pgfsetstrokecolor{currentstroke}%
\pgfsetdash{}{0pt}%
\pgfpathmoveto{\pgfqpoint{1.897093in}{2.185546in}}%
\pgfpathlineto{\pgfqpoint{1.813395in}{1.976671in}}%
\pgfpathlineto{\pgfqpoint{1.948780in}{1.936906in}}%
\pgfpathclose%
\pgfusepath{fill}%
\end{pgfscope}%
\begin{pgfscope}%
\pgfpathrectangle{\pgfqpoint{0.680860in}{0.078740in}}{\pgfqpoint{7.842520in}{7.842520in}}%
\pgfusepath{clip}%
\pgfsetbuttcap%
\pgfsetroundjoin%
\definecolor{currentfill}{rgb}{0.210503,0.363727,0.552206}%
\pgfsetfillcolor{currentfill}%
\pgfsetlinewidth{0.000000pt}%
\definecolor{currentstroke}{rgb}{0.369214,0.788888,0.382914}%
\pgfsetstrokecolor{currentstroke}%
\pgfsetdash{}{0pt}%
\pgfpathmoveto{\pgfqpoint{2.720500in}{3.120880in}}%
\pgfpathlineto{\pgfqpoint{2.856665in}{3.109596in}}%
\pgfpathlineto{\pgfqpoint{2.939933in}{3.297162in}}%
\pgfpathclose%
\pgfusepath{fill}%
\end{pgfscope}%
\begin{pgfscope}%
\pgfpathrectangle{\pgfqpoint{0.680860in}{0.078740in}}{\pgfqpoint{7.842520in}{7.842520in}}%
\pgfusepath{clip}%
\pgfsetbuttcap%
\pgfsetroundjoin%
\definecolor{currentfill}{rgb}{0.190631,0.407061,0.556089}%
\pgfsetfillcolor{currentfill}%
\pgfsetlinewidth{0.000000pt}%
\definecolor{currentstroke}{rgb}{0.377779,0.791781,0.377939}%
\pgfsetstrokecolor{currentstroke}%
\pgfsetdash{}{0pt}%
\pgfpathmoveto{\pgfqpoint{3.023238in}{3.473471in}}%
\pgfpathlineto{\pgfqpoint{2.939933in}{3.297162in}}%
\pgfpathlineto{\pgfqpoint{3.076561in}{3.290871in}}%
\pgfpathclose%
\pgfusepath{fill}%
\end{pgfscope}%
\begin{pgfscope}%
\pgfpathrectangle{\pgfqpoint{0.680860in}{0.078740in}}{\pgfqpoint{7.842520in}{7.842520in}}%
\pgfusepath{clip}%
\pgfsetbuttcap%
\pgfsetroundjoin%
\definecolor{currentfill}{rgb}{0.175707,0.697900,0.491033}%
\pgfsetfillcolor{currentfill}%
\pgfsetlinewidth{0.000000pt}%
\definecolor{currentstroke}{rgb}{0.386433,0.794644,0.372886}%
\pgfsetstrokecolor{currentstroke}%
\pgfsetdash{}{0pt}%
\pgfpathmoveto{\pgfqpoint{5.449688in}{4.643592in}}%
\pgfpathlineto{\pgfqpoint{5.371612in}{4.650148in}}%
\pgfpathlineto{\pgfqpoint{5.513913in}{4.679638in}}%
\pgfpathclose%
\pgfusepath{fill}%
\end{pgfscope}%
\begin{pgfscope}%
\pgfpathrectangle{\pgfqpoint{0.680860in}{0.078740in}}{\pgfqpoint{7.842520in}{7.842520in}}%
\pgfusepath{clip}%
\pgfsetbuttcap%
\pgfsetroundjoin%
\definecolor{currentfill}{rgb}{0.223925,0.334994,0.548053}%
\pgfsetfillcolor{currentfill}%
\pgfsetlinewidth{0.000000pt}%
\definecolor{currentstroke}{rgb}{0.395174,0.797475,0.367757}%
\pgfsetstrokecolor{currentstroke}%
\pgfsetdash{}{0pt}%
\pgfpathmoveto{\pgfqpoint{2.856665in}{3.109596in}}%
\pgfpathlineto{\pgfqpoint{2.720500in}{3.120880in}}%
\pgfpathlineto{\pgfqpoint{2.637183in}{2.927444in}}%
\pgfpathclose%
\pgfusepath{fill}%
\end{pgfscope}%
\begin{pgfscope}%
\pgfpathrectangle{\pgfqpoint{0.680860in}{0.078740in}}{\pgfqpoint{7.842520in}{7.842520in}}%
\pgfusepath{clip}%
\pgfsetbuttcap%
\pgfsetroundjoin%
\definecolor{currentfill}{rgb}{0.132444,0.552216,0.553018}%
\pgfsetfillcolor{currentfill}%
\pgfsetlinewidth{0.000000pt}%
\definecolor{currentstroke}{rgb}{0.404001,0.800275,0.362552}%
\pgfsetstrokecolor{currentstroke}%
\pgfsetdash{}{0pt}%
\pgfpathmoveto{\pgfqpoint{3.821749in}{3.958466in}}%
\pgfpathlineto{\pgfqpoint{3.766552in}{4.075952in}}%
\pgfpathlineto{\pgfqpoint{3.683685in}{3.947759in}}%
\pgfpathclose%
\pgfusepath{fill}%
\end{pgfscope}%
\begin{pgfscope}%
\pgfpathrectangle{\pgfqpoint{0.680860in}{0.078740in}}{\pgfqpoint{7.842520in}{7.842520in}}%
\pgfusepath{clip}%
\pgfsetbuttcap%
\pgfsetroundjoin%
\definecolor{currentfill}{rgb}{0.150148,0.676631,0.506589}%
\pgfsetfillcolor{currentfill}%
\pgfsetlinewidth{0.000000pt}%
\definecolor{currentstroke}{rgb}{0.412913,0.803041,0.357269}%
\pgfsetstrokecolor{currentstroke}%
\pgfsetdash{}{0pt}%
\pgfpathmoveto{\pgfqpoint{7.038179in}{4.590300in}}%
\pgfpathlineto{\pgfqpoint{7.184740in}{4.618099in}}%
\pgfpathlineto{\pgfqpoint{7.250814in}{4.461584in}}%
\pgfpathclose%
\pgfusepath{fill}%
\end{pgfscope}%
\begin{pgfscope}%
\pgfpathrectangle{\pgfqpoint{0.680860in}{0.078740in}}{\pgfqpoint{7.842520in}{7.842520in}}%
\pgfusepath{clip}%
\pgfsetbuttcap%
\pgfsetroundjoin%
\definecolor{currentfill}{rgb}{0.128729,0.563265,0.551229}%
\pgfsetfillcolor{currentfill}%
\pgfsetlinewidth{0.000000pt}%
\definecolor{currentstroke}{rgb}{0.421908,0.805774,0.351910}%
\pgfsetstrokecolor{currentstroke}%
\pgfsetdash{}{0pt}%
\pgfpathmoveto{\pgfqpoint{3.904504in}{4.088783in}}%
\pgfpathlineto{\pgfqpoint{3.766552in}{4.075952in}}%
\pgfpathlineto{\pgfqpoint{3.821749in}{3.958466in}}%
\pgfpathclose%
\pgfusepath{fill}%
\end{pgfscope}%
\begin{pgfscope}%
\pgfpathrectangle{\pgfqpoint{0.680860in}{0.078740in}}{\pgfqpoint{7.842520in}{7.842520in}}%
\pgfusepath{clip}%
\pgfsetbuttcap%
\pgfsetroundjoin%
\definecolor{currentfill}{rgb}{0.277018,0.050344,0.375715}%
\pgfsetfillcolor{currentfill}%
\pgfsetlinewidth{0.000000pt}%
\definecolor{currentstroke}{rgb}{0.430983,0.808473,0.346476}%
\pgfsetstrokecolor{currentstroke}%
\pgfsetdash{}{0pt}%
\pgfpathmoveto{\pgfqpoint{1.948780in}{1.936906in}}%
\pgfpathlineto{\pgfqpoint{2.032327in}{2.150418in}}%
\pgfpathlineto{\pgfqpoint{1.897093in}{2.185546in}}%
\pgfpathclose%
\pgfusepath{fill}%
\end{pgfscope}%
\begin{pgfscope}%
\pgfpathrectangle{\pgfqpoint{0.680860in}{0.078740in}}{\pgfqpoint{7.842520in}{7.842520in}}%
\pgfusepath{clip}%
\pgfsetbuttcap%
\pgfsetroundjoin%
\definecolor{currentfill}{rgb}{0.191090,0.708366,0.482284}%
\pgfsetfillcolor{currentfill}%
\pgfsetlinewidth{0.000000pt}%
\definecolor{currentstroke}{rgb}{0.440137,0.811138,0.340967}%
\pgfsetstrokecolor{currentstroke}%
\pgfsetdash{}{0pt}%
\pgfpathmoveto{\pgfqpoint{5.953658in}{4.694862in}}%
\pgfpathlineto{\pgfqpoint{6.097424in}{4.725276in}}%
\pgfpathlineto{\pgfqpoint{6.171777in}{4.663723in}}%
\pgfpathclose%
\pgfusepath{fill}%
\end{pgfscope}%
\begin{pgfscope}%
\pgfpathrectangle{\pgfqpoint{0.680860in}{0.078740in}}{\pgfqpoint{7.842520in}{7.842520in}}%
\pgfusepath{clip}%
\pgfsetbuttcap%
\pgfsetroundjoin%
\definecolor{currentfill}{rgb}{0.252194,0.269783,0.531579}%
\pgfsetfillcolor{currentfill}%
\pgfsetlinewidth{0.000000pt}%
\definecolor{currentstroke}{rgb}{0.449368,0.813768,0.335384}%
\pgfsetstrokecolor{currentstroke}%
\pgfsetdash{}{0pt}%
\pgfpathmoveto{\pgfqpoint{2.418072in}{2.746048in}}%
\pgfpathlineto{\pgfqpoint{2.553910in}{2.725397in}}%
\pgfpathlineto{\pgfqpoint{2.637183in}{2.927444in}}%
\pgfpathclose%
\pgfusepath{fill}%
\end{pgfscope}%
\begin{pgfscope}%
\pgfpathrectangle{\pgfqpoint{0.680860in}{0.078740in}}{\pgfqpoint{7.842520in}{7.842520in}}%
\pgfusepath{clip}%
\pgfsetbuttcap%
\pgfsetroundjoin%
\definecolor{currentfill}{rgb}{0.183898,0.422383,0.556944}%
\pgfsetfillcolor{currentfill}%
\pgfsetlinewidth{0.000000pt}%
\definecolor{currentstroke}{rgb}{0.458674,0.816363,0.329727}%
\pgfsetstrokecolor{currentstroke}%
\pgfsetdash{}{0pt}%
\pgfpathmoveto{\pgfqpoint{3.159778in}{3.470887in}}%
\pgfpathlineto{\pgfqpoint{3.023238in}{3.473471in}}%
\pgfpathlineto{\pgfqpoint{3.076561in}{3.290871in}}%
\pgfpathclose%
\pgfusepath{fill}%
\end{pgfscope}%
\begin{pgfscope}%
\pgfpathrectangle{\pgfqpoint{0.680860in}{0.078740in}}{\pgfqpoint{7.842520in}{7.842520in}}%
\pgfusepath{clip}%
\pgfsetbuttcap%
\pgfsetroundjoin%
\definecolor{currentfill}{rgb}{0.122606,0.585371,0.546557}%
\pgfsetfillcolor{currentfill}%
\pgfsetlinewidth{0.000000pt}%
\definecolor{currentstroke}{rgb}{0.468053,0.818921,0.323998}%
\pgfsetstrokecolor{currentstroke}%
\pgfsetdash{}{0pt}%
\pgfpathmoveto{\pgfqpoint{4.043100in}{4.103366in}}%
\pgfpathlineto{\pgfqpoint{3.987086in}{4.201876in}}%
\pgfpathlineto{\pgfqpoint{3.904504in}{4.088783in}}%
\pgfpathclose%
\pgfusepath{fill}%
\end{pgfscope}%
\begin{pgfscope}%
\pgfpathrectangle{\pgfqpoint{0.680860in}{0.078740in}}{\pgfqpoint{7.842520in}{7.842520in}}%
\pgfusepath{clip}%
\pgfsetbuttcap%
\pgfsetroundjoin%
\definecolor{currentfill}{rgb}{0.121148,0.592739,0.544641}%
\pgfsetfillcolor{currentfill}%
\pgfsetlinewidth{0.000000pt}%
\definecolor{currentstroke}{rgb}{0.477504,0.821444,0.318195}%
\pgfsetstrokecolor{currentstroke}%
\pgfsetdash{}{0pt}%
\pgfpathmoveto{\pgfqpoint{4.043100in}{4.103366in}}%
\pgfpathlineto{\pgfqpoint{4.125553in}{4.218073in}}%
\pgfpathlineto{\pgfqpoint{3.987086in}{4.201876in}}%
\pgfpathclose%
\pgfusepath{fill}%
\end{pgfscope}%
\begin{pgfscope}%
\pgfpathrectangle{\pgfqpoint{0.680860in}{0.078740in}}{\pgfqpoint{7.842520in}{7.842520in}}%
\pgfusepath{clip}%
\pgfsetbuttcap%
\pgfsetroundjoin%
\definecolor{currentfill}{rgb}{0.281412,0.155834,0.469201}%
\pgfsetfillcolor{currentfill}%
\pgfsetlinewidth{0.000000pt}%
\definecolor{currentstroke}{rgb}{0.487026,0.823929,0.312321}%
\pgfsetstrokecolor{currentstroke}%
\pgfsetdash{}{0pt}%
\pgfpathmoveto{\pgfqpoint{2.334724in}{2.541238in}}%
\pgfpathlineto{\pgfqpoint{2.115813in}{2.360861in}}%
\pgfpathlineto{\pgfqpoint{2.251387in}{2.330588in}}%
\pgfpathclose%
\pgfusepath{fill}%
\end{pgfscope}%
\begin{pgfscope}%
\pgfpathrectangle{\pgfqpoint{0.680860in}{0.078740in}}{\pgfqpoint{7.842520in}{7.842520in}}%
\pgfusepath{clip}%
\pgfsetbuttcap%
\pgfsetroundjoin%
\definecolor{currentfill}{rgb}{0.120081,0.622161,0.534946}%
\pgfsetfillcolor{currentfill}%
\pgfsetlinewidth{0.000000pt}%
\definecolor{currentstroke}{rgb}{0.496615,0.826376,0.306377}%
\pgfsetstrokecolor{currentstroke}%
\pgfsetdash{}{0pt}%
\pgfpathmoveto{\pgfqpoint{4.207755in}{4.314588in}}%
\pgfpathlineto{\pgfqpoint{4.264682in}{4.236141in}}%
\pgfpathlineto{\pgfqpoint{4.346734in}{4.333704in}}%
\pgfpathclose%
\pgfusepath{fill}%
\end{pgfscope}%
\begin{pgfscope}%
\pgfpathrectangle{\pgfqpoint{0.680860in}{0.078740in}}{\pgfqpoint{7.842520in}{7.842520in}}%
\pgfusepath{clip}%
\pgfsetbuttcap%
\pgfsetroundjoin%
\definecolor{currentfill}{rgb}{0.119423,0.611141,0.538982}%
\pgfsetfillcolor{currentfill}%
\pgfsetlinewidth{0.000000pt}%
\definecolor{currentstroke}{rgb}{0.506271,0.828786,0.300362}%
\pgfsetstrokecolor{currentstroke}%
\pgfsetdash{}{0pt}%
\pgfpathmoveto{\pgfqpoint{4.207755in}{4.314588in}}%
\pgfpathlineto{\pgfqpoint{4.125553in}{4.218073in}}%
\pgfpathlineto{\pgfqpoint{4.264682in}{4.236141in}}%
\pgfpathclose%
\pgfusepath{fill}%
\end{pgfscope}%
\begin{pgfscope}%
\pgfpathrectangle{\pgfqpoint{0.680860in}{0.078740in}}{\pgfqpoint{7.842520in}{7.842520in}}%
\pgfusepath{clip}%
\pgfsetbuttcap%
\pgfsetroundjoin%
\definecolor{currentfill}{rgb}{0.196571,0.711827,0.479221}%
\pgfsetfillcolor{currentfill}%
\pgfsetlinewidth{0.000000pt}%
\definecolor{currentstroke}{rgb}{0.515992,0.831158,0.294279}%
\pgfsetstrokecolor{currentstroke}%
\pgfsetdash{}{0pt}%
\pgfpathmoveto{\pgfqpoint{5.877741in}{4.731360in}}%
\pgfpathlineto{\pgfqpoint{5.953658in}{4.694862in}}%
\pgfpathlineto{\pgfqpoint{5.810668in}{4.666739in}}%
\pgfpathclose%
\pgfusepath{fill}%
\end{pgfscope}%
\begin{pgfscope}%
\pgfpathrectangle{\pgfqpoint{0.680860in}{0.078740in}}{\pgfqpoint{7.842520in}{7.842520in}}%
\pgfusepath{clip}%
\pgfsetbuttcap%
\pgfsetroundjoin%
\definecolor{currentfill}{rgb}{0.175707,0.697900,0.491033}%
\pgfsetfillcolor{currentfill}%
\pgfsetlinewidth{0.000000pt}%
\definecolor{currentstroke}{rgb}{0.525776,0.833491,0.288127}%
\pgfsetstrokecolor{currentstroke}%
\pgfsetdash{}{0pt}%
\pgfpathmoveto{\pgfqpoint{6.677475in}{4.658470in}}%
\pgfpathlineto{\pgfqpoint{6.823210in}{4.688299in}}%
\pgfpathlineto{\pgfqpoint{6.892428in}{4.564701in}}%
\pgfpathclose%
\pgfusepath{fill}%
\end{pgfscope}%
\begin{pgfscope}%
\pgfpathrectangle{\pgfqpoint{0.680860in}{0.078740in}}{\pgfqpoint{7.842520in}{7.842520in}}%
\pgfusepath{clip}%
\pgfsetbuttcap%
\pgfsetroundjoin%
\definecolor{currentfill}{rgb}{0.270595,0.214069,0.507052}%
\pgfsetfillcolor{currentfill}%
\pgfsetlinewidth{0.000000pt}%
\definecolor{currentstroke}{rgb}{0.535621,0.835785,0.281908}%
\pgfsetstrokecolor{currentstroke}%
\pgfsetdash{}{0pt}%
\pgfpathmoveto{\pgfqpoint{2.418072in}{2.746048in}}%
\pgfpathlineto{\pgfqpoint{2.334724in}{2.541238in}}%
\pgfpathlineto{\pgfqpoint{2.470670in}{2.516010in}}%
\pgfpathclose%
\pgfusepath{fill}%
\end{pgfscope}%
\begin{pgfscope}%
\pgfpathrectangle{\pgfqpoint{0.680860in}{0.078740in}}{\pgfqpoint{7.842520in}{7.842520in}}%
\pgfusepath{clip}%
\pgfsetbuttcap%
\pgfsetroundjoin%
\definecolor{currentfill}{rgb}{0.191090,0.708366,0.482284}%
\pgfsetfillcolor{currentfill}%
\pgfsetlinewidth{0.000000pt}%
\definecolor{currentstroke}{rgb}{0.545524,0.838039,0.275626}%
\pgfsetstrokecolor{currentstroke}%
\pgfsetdash{}{0pt}%
\pgfpathmoveto{\pgfqpoint{6.532545in}{4.630913in}}%
\pgfpathlineto{\pgfqpoint{6.315778in}{4.692081in}}%
\pgfpathlineto{\pgfqpoint{6.460572in}{4.722740in}}%
\pgfpathclose%
\pgfusepath{fill}%
\end{pgfscope}%
\begin{pgfscope}%
\pgfpathrectangle{\pgfqpoint{0.680860in}{0.078740in}}{\pgfqpoint{7.842520in}{7.842520in}}%
\pgfusepath{clip}%
\pgfsetbuttcap%
\pgfsetroundjoin%
\definecolor{currentfill}{rgb}{0.282327,0.094955,0.417331}%
\pgfsetfillcolor{currentfill}%
\pgfsetlinewidth{0.000000pt}%
\definecolor{currentstroke}{rgb}{0.555484,0.840254,0.269281}%
\pgfsetstrokecolor{currentstroke}%
\pgfsetdash{}{0pt}%
\pgfpathmoveto{\pgfqpoint{2.115813in}{2.360861in}}%
\pgfpathlineto{\pgfqpoint{2.032327in}{2.150418in}}%
\pgfpathlineto{\pgfqpoint{2.168032in}{2.115434in}}%
\pgfpathclose%
\pgfusepath{fill}%
\end{pgfscope}%
\begin{pgfscope}%
\pgfpathrectangle{\pgfqpoint{0.680860in}{0.078740in}}{\pgfqpoint{7.842520in}{7.842520in}}%
\pgfusepath{clip}%
\pgfsetbuttcap%
\pgfsetroundjoin%
\definecolor{currentfill}{rgb}{0.171176,0.452530,0.557965}%
\pgfsetfillcolor{currentfill}%
\pgfsetlinewidth{0.000000pt}%
\definecolor{currentstroke}{rgb}{0.565498,0.842430,0.262877}%
\pgfsetstrokecolor{currentstroke}%
\pgfsetdash{}{0pt}%
\pgfpathmoveto{\pgfqpoint{3.296894in}{3.469525in}}%
\pgfpathlineto{\pgfqpoint{3.243011in}{3.638213in}}%
\pgfpathlineto{\pgfqpoint{3.159778in}{3.470887in}}%
\pgfpathclose%
\pgfusepath{fill}%
\end{pgfscope}%
\begin{pgfscope}%
\pgfpathrectangle{\pgfqpoint{0.680860in}{0.078740in}}{\pgfqpoint{7.842520in}{7.842520in}}%
\pgfusepath{clip}%
\pgfsetbuttcap%
\pgfsetroundjoin%
\definecolor{currentfill}{rgb}{0.175707,0.697900,0.491033}%
\pgfsetfillcolor{currentfill}%
\pgfsetlinewidth{0.000000pt}%
\definecolor{currentstroke}{rgb}{0.575563,0.844566,0.256415}%
\pgfsetstrokecolor{currentstroke}%
\pgfsetdash{}{0pt}%
\pgfpathmoveto{\pgfqpoint{5.371612in}{4.650148in}}%
\pgfpathlineto{\pgfqpoint{5.230060in}{4.622919in}}%
\pgfpathlineto{\pgfqpoint{5.292761in}{4.635734in}}%
\pgfpathclose%
\pgfusepath{fill}%
\end{pgfscope}%
\begin{pgfscope}%
\pgfpathrectangle{\pgfqpoint{0.680860in}{0.078740in}}{\pgfqpoint{7.842520in}{7.842520in}}%
\pgfusepath{clip}%
\pgfsetbuttcap%
\pgfsetroundjoin%
\definecolor{currentfill}{rgb}{0.137339,0.662252,0.515571}%
\pgfsetfillcolor{currentfill}%
\pgfsetlinewidth{0.000000pt}%
\definecolor{currentstroke}{rgb}{0.585678,0.846661,0.249897}%
\pgfsetstrokecolor{currentstroke}%
\pgfsetdash{}{0pt}%
\pgfpathmoveto{\pgfqpoint{4.708100in}{4.457367in}}%
\pgfpathlineto{\pgfqpoint{4.788990in}{4.516484in}}%
\pgfpathlineto{\pgfqpoint{4.567927in}{4.433785in}}%
\pgfpathclose%
\pgfusepath{fill}%
\end{pgfscope}%
\begin{pgfscope}%
\pgfpathrectangle{\pgfqpoint{0.680860in}{0.078740in}}{\pgfqpoint{7.842520in}{7.842520in}}%
\pgfusepath{clip}%
\pgfsetbuttcap%
\pgfsetroundjoin%
\definecolor{currentfill}{rgb}{0.151918,0.500685,0.557587}%
\pgfsetfillcolor{currentfill}%
\pgfsetlinewidth{0.000000pt}%
\definecolor{currentstroke}{rgb}{0.595839,0.848717,0.243329}%
\pgfsetstrokecolor{currentstroke}%
\pgfsetdash{}{0pt}%
\pgfpathmoveto{\pgfqpoint{3.380042in}{3.640270in}}%
\pgfpathlineto{\pgfqpoint{3.600710in}{3.803441in}}%
\pgfpathlineto{\pgfqpoint{3.463170in}{3.796932in}}%
\pgfpathclose%
\pgfusepath{fill}%
\end{pgfscope}%
\begin{pgfscope}%
\pgfpathrectangle{\pgfqpoint{0.680860in}{0.078740in}}{\pgfqpoint{7.842520in}{7.842520in}}%
\pgfusepath{clip}%
\pgfsetbuttcap%
\pgfsetroundjoin%
\definecolor{currentfill}{rgb}{0.201239,0.383670,0.554294}%
\pgfsetfillcolor{currentfill}%
\pgfsetlinewidth{0.000000pt}%
\definecolor{currentstroke}{rgb}{0.606045,0.850733,0.236712}%
\pgfsetstrokecolor{currentstroke}%
\pgfsetdash{}{0pt}%
\pgfpathmoveto{\pgfqpoint{2.939933in}{3.297162in}}%
\pgfpathlineto{\pgfqpoint{2.856665in}{3.109596in}}%
\pgfpathlineto{\pgfqpoint{3.076561in}{3.290871in}}%
\pgfpathclose%
\pgfusepath{fill}%
\end{pgfscope}%
\begin{pgfscope}%
\pgfpathrectangle{\pgfqpoint{0.680860in}{0.078740in}}{\pgfqpoint{7.842520in}{7.842520in}}%
\pgfusepath{clip}%
\pgfsetbuttcap%
\pgfsetroundjoin%
\definecolor{currentfill}{rgb}{0.265145,0.232956,0.516599}%
\pgfsetfillcolor{currentfill}%
\pgfsetlinewidth{0.000000pt}%
\definecolor{currentstroke}{rgb}{0.616293,0.852709,0.230052}%
\pgfsetstrokecolor{currentstroke}%
\pgfsetdash{}{0pt}%
\pgfpathmoveto{\pgfqpoint{2.470670in}{2.516010in}}%
\pgfpathlineto{\pgfqpoint{2.553910in}{2.725397in}}%
\pgfpathlineto{\pgfqpoint{2.418072in}{2.746048in}}%
\pgfpathclose%
\pgfusepath{fill}%
\end{pgfscope}%
\begin{pgfscope}%
\pgfpathrectangle{\pgfqpoint{0.680860in}{0.078740in}}{\pgfqpoint{7.842520in}{7.842520in}}%
\pgfusepath{clip}%
\pgfsetbuttcap%
\pgfsetroundjoin%
\definecolor{currentfill}{rgb}{0.166617,0.463708,0.558119}%
\pgfsetfillcolor{currentfill}%
\pgfsetlinewidth{0.000000pt}%
\definecolor{currentstroke}{rgb}{0.626579,0.854645,0.223353}%
\pgfsetstrokecolor{currentstroke}%
\pgfsetdash{}{0pt}%
\pgfpathmoveto{\pgfqpoint{3.296894in}{3.469525in}}%
\pgfpathlineto{\pgfqpoint{3.380042in}{3.640270in}}%
\pgfpathlineto{\pgfqpoint{3.243011in}{3.638213in}}%
\pgfpathclose%
\pgfusepath{fill}%
\end{pgfscope}%
\begin{pgfscope}%
\pgfpathrectangle{\pgfqpoint{0.680860in}{0.078740in}}{\pgfqpoint{7.842520in}{7.842520in}}%
\pgfusepath{clip}%
\pgfsetbuttcap%
\pgfsetroundjoin%
\definecolor{currentfill}{rgb}{0.283197,0.115680,0.436115}%
\pgfsetfillcolor{currentfill}%
\pgfsetlinewidth{0.000000pt}%
\definecolor{currentstroke}{rgb}{0.636902,0.856542,0.216620}%
\pgfsetstrokecolor{currentstroke}%
\pgfsetdash{}{0pt}%
\pgfpathmoveto{\pgfqpoint{2.168032in}{2.115434in}}%
\pgfpathlineto{\pgfqpoint{2.251387in}{2.330588in}}%
\pgfpathlineto{\pgfqpoint{2.115813in}{2.360861in}}%
\pgfpathclose%
\pgfusepath{fill}%
\end{pgfscope}%
\begin{pgfscope}%
\pgfpathrectangle{\pgfqpoint{0.680860in}{0.078740in}}{\pgfqpoint{7.842520in}{7.842520in}}%
\pgfusepath{clip}%
\pgfsetbuttcap%
\pgfsetroundjoin%
\definecolor{currentfill}{rgb}{0.126326,0.644107,0.525311}%
\pgfsetfillcolor{currentfill}%
\pgfsetlinewidth{0.000000pt}%
\definecolor{currentstroke}{rgb}{0.647257,0.858400,0.209861}%
\pgfsetstrokecolor{currentstroke}%
\pgfsetdash{}{0pt}%
\pgfpathmoveto{\pgfqpoint{4.346734in}{4.333704in}}%
\pgfpathlineto{\pgfqpoint{4.486392in}{4.354795in}}%
\pgfpathlineto{\pgfqpoint{4.567927in}{4.433785in}}%
\pgfpathclose%
\pgfusepath{fill}%
\end{pgfscope}%
\begin{pgfscope}%
\pgfpathrectangle{\pgfqpoint{0.680860in}{0.078740in}}{\pgfqpoint{7.842520in}{7.842520in}}%
\pgfusepath{clip}%
\pgfsetbuttcap%
\pgfsetroundjoin%
\definecolor{currentfill}{rgb}{0.175707,0.697900,0.491033}%
\pgfsetfillcolor{currentfill}%
\pgfsetlinewidth{0.000000pt}%
\definecolor{currentstroke}{rgb}{0.657642,0.860219,0.203082}%
\pgfsetstrokecolor{currentstroke}%
\pgfsetdash{}{0pt}%
\pgfpathmoveto{\pgfqpoint{5.292761in}{4.635734in}}%
\pgfpathlineto{\pgfqpoint{5.230060in}{4.622919in}}%
\pgfpathlineto{\pgfqpoint{5.150891in}{4.606777in}}%
\pgfpathclose%
\pgfusepath{fill}%
\end{pgfscope}%
\begin{pgfscope}%
\pgfpathrectangle{\pgfqpoint{0.680860in}{0.078740in}}{\pgfqpoint{7.842520in}{7.842520in}}%
\pgfusepath{clip}%
\pgfsetbuttcap%
\pgfsetroundjoin%
\definecolor{currentfill}{rgb}{0.229739,0.322361,0.545706}%
\pgfsetfillcolor{currentfill}%
\pgfsetlinewidth{0.000000pt}%
\definecolor{currentstroke}{rgb}{0.668054,0.861999,0.196293}%
\pgfsetstrokecolor{currentstroke}%
\pgfsetdash{}{0pt}%
\pgfpathmoveto{\pgfqpoint{2.773441in}{2.911917in}}%
\pgfpathlineto{\pgfqpoint{2.856665in}{3.109596in}}%
\pgfpathlineto{\pgfqpoint{2.637183in}{2.927444in}}%
\pgfpathclose%
\pgfusepath{fill}%
\end{pgfscope}%
\begin{pgfscope}%
\pgfpathrectangle{\pgfqpoint{0.680860in}{0.078740in}}{\pgfqpoint{7.842520in}{7.842520in}}%
\pgfusepath{clip}%
\pgfsetbuttcap%
\pgfsetroundjoin%
\definecolor{currentfill}{rgb}{0.196571,0.711827,0.479221}%
\pgfsetfillcolor{currentfill}%
\pgfsetlinewidth{0.000000pt}%
\definecolor{currentstroke}{rgb}{0.678489,0.863742,0.189503}%
\pgfsetstrokecolor{currentstroke}%
\pgfsetdash{}{0pt}%
\pgfpathmoveto{\pgfqpoint{5.810668in}{4.666739in}}%
\pgfpathlineto{\pgfqpoint{5.734289in}{4.699798in}}%
\pgfpathlineto{\pgfqpoint{5.877741in}{4.731360in}}%
\pgfpathclose%
\pgfusepath{fill}%
\end{pgfscope}%
\begin{pgfscope}%
\pgfpathrectangle{\pgfqpoint{0.680860in}{0.078740in}}{\pgfqpoint{7.842520in}{7.842520in}}%
\pgfusepath{clip}%
\pgfsetbuttcap%
\pgfsetroundjoin%
\definecolor{currentfill}{rgb}{0.244972,0.287675,0.537260}%
\pgfsetfillcolor{currentfill}%
\pgfsetlinewidth{0.000000pt}%
\definecolor{currentstroke}{rgb}{0.688944,0.865448,0.182725}%
\pgfsetstrokecolor{currentstroke}%
\pgfsetdash{}{0pt}%
\pgfpathmoveto{\pgfqpoint{2.773441in}{2.911917in}}%
\pgfpathlineto{\pgfqpoint{2.637183in}{2.927444in}}%
\pgfpathlineto{\pgfqpoint{2.553910in}{2.725397in}}%
\pgfpathclose%
\pgfusepath{fill}%
\end{pgfscope}%
\begin{pgfscope}%
\pgfpathrectangle{\pgfqpoint{0.680860in}{0.078740in}}{\pgfqpoint{7.842520in}{7.842520in}}%
\pgfusepath{clip}%
\pgfsetbuttcap%
\pgfsetroundjoin%
\definecolor{currentfill}{rgb}{0.202219,0.715272,0.476084}%
\pgfsetfillcolor{currentfill}%
\pgfsetlinewidth{0.000000pt}%
\definecolor{currentstroke}{rgb}{0.699415,0.867117,0.175971}%
\pgfsetstrokecolor{currentstroke}%
\pgfsetdash{}{0pt}%
\pgfpathmoveto{\pgfqpoint{6.241987in}{4.758070in}}%
\pgfpathlineto{\pgfqpoint{6.315778in}{4.692081in}}%
\pgfpathlineto{\pgfqpoint{6.171777in}{4.663723in}}%
\pgfpathclose%
\pgfusepath{fill}%
\end{pgfscope}%
\begin{pgfscope}%
\pgfpathrectangle{\pgfqpoint{0.680860in}{0.078740in}}{\pgfqpoint{7.842520in}{7.842520in}}%
\pgfusepath{clip}%
\pgfsetbuttcap%
\pgfsetroundjoin%
\definecolor{currentfill}{rgb}{0.137770,0.537492,0.554906}%
\pgfsetfillcolor{currentfill}%
\pgfsetlinewidth{0.000000pt}%
\definecolor{currentstroke}{rgb}{0.709898,0.868751,0.169257}%
\pgfsetstrokecolor{currentstroke}%
\pgfsetdash{}{0pt}%
\pgfpathmoveto{\pgfqpoint{3.821749in}{3.958466in}}%
\pgfpathlineto{\pgfqpoint{3.683685in}{3.947759in}}%
\pgfpathlineto{\pgfqpoint{3.600710in}{3.803441in}}%
\pgfpathclose%
\pgfusepath{fill}%
\end{pgfscope}%
\begin{pgfscope}%
\pgfpathrectangle{\pgfqpoint{0.680860in}{0.078740in}}{\pgfqpoint{7.842520in}{7.842520in}}%
\pgfusepath{clip}%
\pgfsetbuttcap%
\pgfsetroundjoin%
\definecolor{currentfill}{rgb}{0.175707,0.697900,0.491033}%
\pgfsetfillcolor{currentfill}%
\pgfsetlinewidth{0.000000pt}%
\definecolor{currentstroke}{rgb}{0.720391,0.870350,0.162603}%
\pgfsetstrokecolor{currentstroke}%
\pgfsetdash{}{0pt}%
\pgfpathmoveto{\pgfqpoint{6.969772in}{4.720488in}}%
\pgfpathlineto{\pgfqpoint{7.038179in}{4.590300in}}%
\pgfpathlineto{\pgfqpoint{6.892428in}{4.564701in}}%
\pgfpathclose%
\pgfusepath{fill}%
\end{pgfscope}%
\begin{pgfscope}%
\pgfpathrectangle{\pgfqpoint{0.680860in}{0.078740in}}{\pgfqpoint{7.842520in}{7.842520in}}%
\pgfusepath{clip}%
\pgfsetbuttcap%
\pgfsetroundjoin%
\definecolor{currentfill}{rgb}{0.146616,0.673050,0.508936}%
\pgfsetfillcolor{currentfill}%
\pgfsetlinewidth{0.000000pt}%
\definecolor{currentstroke}{rgb}{0.730889,0.871916,0.156029}%
\pgfsetstrokecolor{currentstroke}%
\pgfsetdash{}{0pt}%
\pgfpathmoveto{\pgfqpoint{4.929657in}{4.541960in}}%
\pgfpathlineto{\pgfqpoint{4.788990in}{4.516484in}}%
\pgfpathlineto{\pgfqpoint{4.708100in}{4.457367in}}%
\pgfpathclose%
\pgfusepath{fill}%
\end{pgfscope}%
\begin{pgfscope}%
\pgfpathrectangle{\pgfqpoint{0.680860in}{0.078740in}}{\pgfqpoint{7.842520in}{7.842520in}}%
\pgfusepath{clip}%
\pgfsetbuttcap%
\pgfsetroundjoin%
\definecolor{currentfill}{rgb}{0.196571,0.711827,0.479221}%
\pgfsetfillcolor{currentfill}%
\pgfsetlinewidth{0.000000pt}%
\definecolor{currentstroke}{rgb}{0.741388,0.873449,0.149561}%
\pgfsetstrokecolor{currentstroke}%
\pgfsetdash{}{0pt}%
\pgfpathmoveto{\pgfqpoint{6.677475in}{4.658470in}}%
\pgfpathlineto{\pgfqpoint{6.532545in}{4.630913in}}%
\pgfpathlineto{\pgfqpoint{6.606180in}{4.755788in}}%
\pgfpathclose%
\pgfusepath{fill}%
\end{pgfscope}%
\begin{pgfscope}%
\pgfpathrectangle{\pgfqpoint{0.680860in}{0.078740in}}{\pgfqpoint{7.842520in}{7.842520in}}%
\pgfusepath{clip}%
\pgfsetbuttcap%
\pgfsetroundjoin%
\definecolor{currentfill}{rgb}{0.280868,0.160771,0.472899}%
\pgfsetfillcolor{currentfill}%
\pgfsetlinewidth{0.000000pt}%
\definecolor{currentstroke}{rgb}{0.751884,0.874951,0.143228}%
\pgfsetstrokecolor{currentstroke}%
\pgfsetdash{}{0pt}%
\pgfpathmoveto{\pgfqpoint{2.334724in}{2.541238in}}%
\pgfpathlineto{\pgfqpoint{2.251387in}{2.330588in}}%
\pgfpathlineto{\pgfqpoint{2.387446in}{2.300619in}}%
\pgfpathclose%
\pgfusepath{fill}%
\end{pgfscope}%
\begin{pgfscope}%
\pgfpathrectangle{\pgfqpoint{0.680860in}{0.078740in}}{\pgfqpoint{7.842520in}{7.842520in}}%
\pgfusepath{clip}%
\pgfsetbuttcap%
\pgfsetroundjoin%
\definecolor{currentfill}{rgb}{0.273809,0.031497,0.358853}%
\pgfsetfillcolor{currentfill}%
\pgfsetlinewidth{0.000000pt}%
\definecolor{currentstroke}{rgb}{0.762373,0.876424,0.137064}%
\pgfsetstrokecolor{currentstroke}%
\pgfsetdash{}{0pt}%
\pgfpathmoveto{\pgfqpoint{1.948780in}{1.936906in}}%
\pgfpathlineto{\pgfqpoint{2.084625in}{1.897138in}}%
\pgfpathlineto{\pgfqpoint{2.032327in}{2.150418in}}%
\pgfpathclose%
\pgfusepath{fill}%
\end{pgfscope}%
\begin{pgfscope}%
\pgfpathrectangle{\pgfqpoint{0.680860in}{0.078740in}}{\pgfqpoint{7.842520in}{7.842520in}}%
\pgfusepath{clip}%
\pgfsetbuttcap%
\pgfsetroundjoin%
\definecolor{currentfill}{rgb}{0.166383,0.690856,0.496502}%
\pgfsetfillcolor{currentfill}%
\pgfsetlinewidth{0.000000pt}%
\definecolor{currentstroke}{rgb}{0.772852,0.877868,0.131109}%
\pgfsetstrokecolor{currentstroke}%
\pgfsetdash{}{0pt}%
\pgfpathmoveto{\pgfqpoint{5.071052in}{4.569641in}}%
\pgfpathlineto{\pgfqpoint{5.150891in}{4.606777in}}%
\pgfpathlineto{\pgfqpoint{5.009762in}{4.580065in}}%
\pgfpathclose%
\pgfusepath{fill}%
\end{pgfscope}%
\begin{pgfscope}%
\pgfpathrectangle{\pgfqpoint{0.680860in}{0.078740in}}{\pgfqpoint{7.842520in}{7.842520in}}%
\pgfusepath{clip}%
\pgfsetbuttcap%
\pgfsetroundjoin%
\definecolor{currentfill}{rgb}{0.180629,0.429975,0.557282}%
\pgfsetfillcolor{currentfill}%
\pgfsetlinewidth{0.000000pt}%
\definecolor{currentstroke}{rgb}{0.783315,0.879285,0.125405}%
\pgfsetstrokecolor{currentstroke}%
\pgfsetdash{}{0pt}%
\pgfpathmoveto{\pgfqpoint{3.159778in}{3.470887in}}%
\pgfpathlineto{\pgfqpoint{3.076561in}{3.290871in}}%
\pgfpathlineto{\pgfqpoint{3.296894in}{3.469525in}}%
\pgfpathclose%
\pgfusepath{fill}%
\end{pgfscope}%
\begin{pgfscope}%
\pgfpathrectangle{\pgfqpoint{0.680860in}{0.078740in}}{\pgfqpoint{7.842520in}{7.842520in}}%
\pgfusepath{clip}%
\pgfsetbuttcap%
\pgfsetroundjoin%
\definecolor{currentfill}{rgb}{0.125394,0.574318,0.549086}%
\pgfsetfillcolor{currentfill}%
\pgfsetlinewidth{0.000000pt}%
\definecolor{currentstroke}{rgb}{0.793760,0.880678,0.120005}%
\pgfsetstrokecolor{currentstroke}%
\pgfsetdash{}{0pt}%
\pgfpathmoveto{\pgfqpoint{3.821749in}{3.958466in}}%
\pgfpathlineto{\pgfqpoint{4.043100in}{4.103366in}}%
\pgfpathlineto{\pgfqpoint{3.904504in}{4.088783in}}%
\pgfpathclose%
\pgfusepath{fill}%
\end{pgfscope}%
\begin{pgfscope}%
\pgfpathrectangle{\pgfqpoint{0.680860in}{0.078740in}}{\pgfqpoint{7.842520in}{7.842520in}}%
\pgfusepath{clip}%
\pgfsetbuttcap%
\pgfsetroundjoin%
\definecolor{currentfill}{rgb}{0.278012,0.180367,0.486697}%
\pgfsetfillcolor{currentfill}%
\pgfsetlinewidth{0.000000pt}%
\definecolor{currentstroke}{rgb}{0.804182,0.882046,0.114965}%
\pgfsetstrokecolor{currentstroke}%
\pgfsetdash{}{0pt}%
\pgfpathmoveto{\pgfqpoint{2.387446in}{2.300619in}}%
\pgfpathlineto{\pgfqpoint{2.470670in}{2.516010in}}%
\pgfpathlineto{\pgfqpoint{2.334724in}{2.541238in}}%
\pgfpathclose%
\pgfusepath{fill}%
\end{pgfscope}%
\begin{pgfscope}%
\pgfpathrectangle{\pgfqpoint{0.680860in}{0.078740in}}{\pgfqpoint{7.842520in}{7.842520in}}%
\pgfusepath{clip}%
\pgfsetbuttcap%
\pgfsetroundjoin%
\definecolor{currentfill}{rgb}{0.277941,0.056324,0.381191}%
\pgfsetfillcolor{currentfill}%
\pgfsetlinewidth{0.000000pt}%
\definecolor{currentstroke}{rgb}{0.814576,0.883393,0.110347}%
\pgfsetstrokecolor{currentstroke}%
\pgfsetdash{}{0pt}%
\pgfpathmoveto{\pgfqpoint{2.032327in}{2.150418in}}%
\pgfpathlineto{\pgfqpoint{2.084625in}{1.897138in}}%
\pgfpathlineto{\pgfqpoint{2.168032in}{2.115434in}}%
\pgfpathclose%
\pgfusepath{fill}%
\end{pgfscope}%
\begin{pgfscope}%
\pgfpathrectangle{\pgfqpoint{0.680860in}{0.078740in}}{\pgfqpoint{7.842520in}{7.842520in}}%
\pgfusepath{clip}%
\pgfsetbuttcap%
\pgfsetroundjoin%
\definecolor{currentfill}{rgb}{0.162016,0.687316,0.499129}%
\pgfsetfillcolor{currentfill}%
\pgfsetlinewidth{0.000000pt}%
\definecolor{currentstroke}{rgb}{0.824940,0.884720,0.106217}%
\pgfsetstrokecolor{currentstroke}%
\pgfsetdash{}{0pt}%
\pgfpathmoveto{\pgfqpoint{5.009762in}{4.580065in}}%
\pgfpathlineto{\pgfqpoint{4.929657in}{4.541960in}}%
\pgfpathlineto{\pgfqpoint{5.071052in}{4.569641in}}%
\pgfpathclose%
\pgfusepath{fill}%
\end{pgfscope}%
\begin{pgfscope}%
\pgfpathrectangle{\pgfqpoint{0.680860in}{0.078740in}}{\pgfqpoint{7.842520in}{7.842520in}}%
\pgfusepath{clip}%
\pgfsetbuttcap%
\pgfsetroundjoin%
\definecolor{currentfill}{rgb}{0.214000,0.722114,0.469588}%
\pgfsetfillcolor{currentfill}%
\pgfsetlinewidth{0.000000pt}%
\definecolor{currentstroke}{rgb}{0.835270,0.886029,0.102646}%
\pgfsetstrokecolor{currentstroke}%
\pgfsetdash{}{0pt}%
\pgfpathmoveto{\pgfqpoint{6.171777in}{4.663723in}}%
\pgfpathlineto{\pgfqpoint{6.097424in}{4.725276in}}%
\pgfpathlineto{\pgfqpoint{6.241987in}{4.758070in}}%
\pgfpathclose%
\pgfusepath{fill}%
\end{pgfscope}%
\begin{pgfscope}%
\pgfpathrectangle{\pgfqpoint{0.680860in}{0.078740in}}{\pgfqpoint{7.842520in}{7.842520in}}%
\pgfusepath{clip}%
\pgfsetbuttcap%
\pgfsetroundjoin%
\definecolor{currentfill}{rgb}{0.204903,0.375746,0.553533}%
\pgfsetfillcolor{currentfill}%
\pgfsetlinewidth{0.000000pt}%
\definecolor{currentstroke}{rgb}{0.845561,0.887322,0.099702}%
\pgfsetstrokecolor{currentstroke}%
\pgfsetdash{}{0pt}%
\pgfpathmoveto{\pgfqpoint{3.076561in}{3.290871in}}%
\pgfpathlineto{\pgfqpoint{2.856665in}{3.109596in}}%
\pgfpathlineto{\pgfqpoint{2.993376in}{3.099240in}}%
\pgfpathclose%
\pgfusepath{fill}%
\end{pgfscope}%
\begin{pgfscope}%
\pgfpathrectangle{\pgfqpoint{0.680860in}{0.078740in}}{\pgfqpoint{7.842520in}{7.842520in}}%
\pgfusepath{clip}%
\pgfsetbuttcap%
\pgfsetroundjoin%
\definecolor{currentfill}{rgb}{0.208030,0.718701,0.472873}%
\pgfsetfillcolor{currentfill}%
\pgfsetlinewidth{0.000000pt}%
\definecolor{currentstroke}{rgb}{0.855810,0.888601,0.097452}%
\pgfsetstrokecolor{currentstroke}%
\pgfsetdash{}{0pt}%
\pgfpathmoveto{\pgfqpoint{5.734289in}{4.699798in}}%
\pgfpathlineto{\pgfqpoint{5.591611in}{4.670570in}}%
\pgfpathlineto{\pgfqpoint{5.656982in}{4.711478in}}%
\pgfpathclose%
\pgfusepath{fill}%
\end{pgfscope}%
\begin{pgfscope}%
\pgfpathrectangle{\pgfqpoint{0.680860in}{0.078740in}}{\pgfqpoint{7.842520in}{7.842520in}}%
\pgfusepath{clip}%
\pgfsetbuttcap%
\pgfsetroundjoin%
\definecolor{currentfill}{rgb}{0.153364,0.497000,0.557724}%
\pgfsetfillcolor{currentfill}%
\pgfsetlinewidth{0.000000pt}%
\definecolor{currentstroke}{rgb}{0.866013,0.889868,0.095953}%
\pgfsetstrokecolor{currentstroke}%
\pgfsetdash{}{0pt}%
\pgfpathmoveto{\pgfqpoint{3.517668in}{3.643709in}}%
\pgfpathlineto{\pgfqpoint{3.600710in}{3.803441in}}%
\pgfpathlineto{\pgfqpoint{3.380042in}{3.640270in}}%
\pgfpathclose%
\pgfusepath{fill}%
\end{pgfscope}%
\begin{pgfscope}%
\pgfpathrectangle{\pgfqpoint{0.680860in}{0.078740in}}{\pgfqpoint{7.842520in}{7.842520in}}%
\pgfusepath{clip}%
\pgfsetbuttcap%
\pgfsetroundjoin%
\definecolor{currentfill}{rgb}{0.250425,0.274290,0.533103}%
\pgfsetfillcolor{currentfill}%
\pgfsetlinewidth{0.000000pt}%
\definecolor{currentstroke}{rgb}{0.876168,0.891125,0.095250}%
\pgfsetstrokecolor{currentstroke}%
\pgfsetdash{}{0pt}%
\pgfpathmoveto{\pgfqpoint{2.553910in}{2.725397in}}%
\pgfpathlineto{\pgfqpoint{2.690262in}{2.705363in}}%
\pgfpathlineto{\pgfqpoint{2.773441in}{2.911917in}}%
\pgfpathclose%
\pgfusepath{fill}%
\end{pgfscope}%
\begin{pgfscope}%
\pgfpathrectangle{\pgfqpoint{0.680860in}{0.078740in}}{\pgfqpoint{7.842520in}{7.842520in}}%
\pgfusepath{clip}%
\pgfsetbuttcap%
\pgfsetroundjoin%
\definecolor{currentfill}{rgb}{0.214000,0.722114,0.469588}%
\pgfsetfillcolor{currentfill}%
\pgfsetlinewidth{0.000000pt}%
\definecolor{currentstroke}{rgb}{0.886271,0.892374,0.095374}%
\pgfsetstrokecolor{currentstroke}%
\pgfsetdash{}{0pt}%
\pgfpathmoveto{\pgfqpoint{5.877741in}{4.731360in}}%
\pgfpathlineto{\pgfqpoint{6.097424in}{4.725276in}}%
\pgfpathlineto{\pgfqpoint{5.953658in}{4.694862in}}%
\pgfpathclose%
\pgfusepath{fill}%
\end{pgfscope}%
\begin{pgfscope}%
\pgfpathrectangle{\pgfqpoint{0.680860in}{0.078740in}}{\pgfqpoint{7.842520in}{7.842520in}}%
\pgfusepath{clip}%
\pgfsetbuttcap%
\pgfsetroundjoin%
\definecolor{currentfill}{rgb}{0.196571,0.711827,0.479221}%
\pgfsetfillcolor{currentfill}%
\pgfsetlinewidth{0.000000pt}%
\definecolor{currentstroke}{rgb}{0.896320,0.893616,0.096335}%
\pgfsetstrokecolor{currentstroke}%
\pgfsetdash{}{0pt}%
\pgfpathmoveto{\pgfqpoint{6.892428in}{4.564701in}}%
\pgfpathlineto{\pgfqpoint{6.823210in}{4.688299in}}%
\pgfpathlineto{\pgfqpoint{6.969772in}{4.720488in}}%
\pgfpathclose%
\pgfusepath{fill}%
\end{pgfscope}%
\begin{pgfscope}%
\pgfpathrectangle{\pgfqpoint{0.680860in}{0.078740in}}{\pgfqpoint{7.842520in}{7.842520in}}%
\pgfusepath{clip}%
\pgfsetbuttcap%
\pgfsetroundjoin%
\definecolor{currentfill}{rgb}{0.162016,0.687316,0.499129}%
\pgfsetfillcolor{currentfill}%
\pgfsetlinewidth{0.000000pt}%
\definecolor{currentstroke}{rgb}{0.906311,0.894855,0.098125}%
\pgfsetstrokecolor{currentstroke}%
\pgfsetdash{}{0pt}%
\pgfpathmoveto{\pgfqpoint{7.544489in}{4.508396in}}%
\pgfpathlineto{\pgfqpoint{7.397248in}{4.483949in}}%
\pgfpathlineto{\pgfqpoint{7.332134in}{4.648182in}}%
\pgfpathclose%
\pgfusepath{fill}%
\end{pgfscope}%
\begin{pgfscope}%
\pgfpathrectangle{\pgfqpoint{0.680860in}{0.078740in}}{\pgfqpoint{7.842520in}{7.842520in}}%
\pgfusepath{clip}%
\pgfsetbuttcap%
\pgfsetroundjoin%
\definecolor{currentfill}{rgb}{0.208030,0.718701,0.472873}%
\pgfsetfillcolor{currentfill}%
\pgfsetlinewidth{0.000000pt}%
\definecolor{currentstroke}{rgb}{0.916242,0.896091,0.100717}%
\pgfsetstrokecolor{currentstroke}%
\pgfsetdash{}{0pt}%
\pgfpathmoveto{\pgfqpoint{6.606180in}{4.755788in}}%
\pgfpathlineto{\pgfqpoint{6.532545in}{4.630913in}}%
\pgfpathlineto{\pgfqpoint{6.460572in}{4.722740in}}%
\pgfpathclose%
\pgfusepath{fill}%
\end{pgfscope}%
\begin{pgfscope}%
\pgfpathrectangle{\pgfqpoint{0.680860in}{0.078740in}}{\pgfqpoint{7.842520in}{7.842520in}}%
\pgfusepath{clip}%
\pgfsetbuttcap%
\pgfsetroundjoin%
\definecolor{currentfill}{rgb}{0.225863,0.330805,0.547314}%
\pgfsetfillcolor{currentfill}%
\pgfsetlinewidth{0.000000pt}%
\definecolor{currentstroke}{rgb}{0.926106,0.897330,0.104071}%
\pgfsetstrokecolor{currentstroke}%
\pgfsetdash{}{0pt}%
\pgfpathmoveto{\pgfqpoint{2.910233in}{2.897179in}}%
\pgfpathlineto{\pgfqpoint{2.856665in}{3.109596in}}%
\pgfpathlineto{\pgfqpoint{2.773441in}{2.911917in}}%
\pgfpathclose%
\pgfusepath{fill}%
\end{pgfscope}%
\begin{pgfscope}%
\pgfpathrectangle{\pgfqpoint{0.680860in}{0.078740in}}{\pgfqpoint{7.842520in}{7.842520in}}%
\pgfusepath{clip}%
\pgfsetbuttcap%
\pgfsetroundjoin%
\definecolor{currentfill}{rgb}{0.196571,0.711827,0.479221}%
\pgfsetfillcolor{currentfill}%
\pgfsetlinewidth{0.000000pt}%
\definecolor{currentstroke}{rgb}{0.935904,0.898570,0.108131}%
\pgfsetstrokecolor{currentstroke}%
\pgfsetdash{}{0pt}%
\pgfpathmoveto{\pgfqpoint{5.292761in}{4.635734in}}%
\pgfpathlineto{\pgfqpoint{5.513913in}{4.679638in}}%
\pgfpathlineto{\pgfqpoint{5.371612in}{4.650148in}}%
\pgfpathclose%
\pgfusepath{fill}%
\end{pgfscope}%
\begin{pgfscope}%
\pgfpathrectangle{\pgfqpoint{0.680860in}{0.078740in}}{\pgfqpoint{7.842520in}{7.842520in}}%
\pgfusepath{clip}%
\pgfsetbuttcap%
\pgfsetroundjoin%
\definecolor{currentfill}{rgb}{0.282656,0.100196,0.422160}%
\pgfsetfillcolor{currentfill}%
\pgfsetlinewidth{0.000000pt}%
\definecolor{currentstroke}{rgb}{0.945636,0.899815,0.112838}%
\pgfsetstrokecolor{currentstroke}%
\pgfsetdash{}{0pt}%
\pgfpathmoveto{\pgfqpoint{2.168032in}{2.115434in}}%
\pgfpathlineto{\pgfqpoint{2.304212in}{2.080603in}}%
\pgfpathlineto{\pgfqpoint{2.251387in}{2.330588in}}%
\pgfpathclose%
\pgfusepath{fill}%
\end{pgfscope}%
\begin{pgfscope}%
\pgfpathrectangle{\pgfqpoint{0.680860in}{0.078740in}}{\pgfqpoint{7.842520in}{7.842520in}}%
\pgfusepath{clip}%
\pgfsetbuttcap%
\pgfsetroundjoin%
\definecolor{currentfill}{rgb}{0.267968,0.223549,0.512008}%
\pgfsetfillcolor{currentfill}%
\pgfsetlinewidth{0.000000pt}%
\definecolor{currentstroke}{rgb}{0.955300,0.901065,0.118128}%
\pgfsetstrokecolor{currentstroke}%
\pgfsetdash{}{0pt}%
\pgfpathmoveto{\pgfqpoint{2.553910in}{2.725397in}}%
\pgfpathlineto{\pgfqpoint{2.470670in}{2.516010in}}%
\pgfpathlineto{\pgfqpoint{2.607119in}{2.491253in}}%
\pgfpathclose%
\pgfusepath{fill}%
\end{pgfscope}%
\begin{pgfscope}%
\pgfpathrectangle{\pgfqpoint{0.680860in}{0.078740in}}{\pgfqpoint{7.842520in}{7.842520in}}%
\pgfusepath{clip}%
\pgfsetbuttcap%
\pgfsetroundjoin%
\definecolor{currentfill}{rgb}{0.208030,0.718701,0.472873}%
\pgfsetfillcolor{currentfill}%
\pgfsetlinewidth{0.000000pt}%
\definecolor{currentstroke}{rgb}{0.964894,0.902323,0.123941}%
\pgfsetstrokecolor{currentstroke}%
\pgfsetdash{}{0pt}%
\pgfpathmoveto{\pgfqpoint{5.591611in}{4.670570in}}%
\pgfpathlineto{\pgfqpoint{5.513913in}{4.679638in}}%
\pgfpathlineto{\pgfqpoint{5.656982in}{4.711478in}}%
\pgfpathclose%
\pgfusepath{fill}%
\end{pgfscope}%
\begin{pgfscope}%
\pgfpathrectangle{\pgfqpoint{0.680860in}{0.078740in}}{\pgfqpoint{7.842520in}{7.842520in}}%
\pgfusepath{clip}%
\pgfsetbuttcap%
\pgfsetroundjoin%
\definecolor{currentfill}{rgb}{0.175707,0.697900,0.491033}%
\pgfsetfillcolor{currentfill}%
\pgfsetlinewidth{0.000000pt}%
\definecolor{currentstroke}{rgb}{0.974417,0.903590,0.130215}%
\pgfsetstrokecolor{currentstroke}%
\pgfsetdash{}{0pt}%
\pgfpathmoveto{\pgfqpoint{7.332134in}{4.648182in}}%
\pgfpathlineto{\pgfqpoint{7.397248in}{4.483949in}}%
\pgfpathlineto{\pgfqpoint{7.184740in}{4.618099in}}%
\pgfpathclose%
\pgfusepath{fill}%
\end{pgfscope}%
\begin{pgfscope}%
\pgfpathrectangle{\pgfqpoint{0.680860in}{0.078740in}}{\pgfqpoint{7.842520in}{7.842520in}}%
\pgfusepath{clip}%
\pgfsetbuttcap%
\pgfsetroundjoin%
\definecolor{currentfill}{rgb}{0.137339,0.662252,0.515571}%
\pgfsetfillcolor{currentfill}%
\pgfsetlinewidth{0.000000pt}%
\definecolor{currentstroke}{rgb}{0.983868,0.904867,0.136897}%
\pgfsetstrokecolor{currentstroke}%
\pgfsetdash{}{0pt}%
\pgfpathmoveto{\pgfqpoint{4.567927in}{4.433785in}}%
\pgfpathlineto{\pgfqpoint{4.626744in}{4.377935in}}%
\pgfpathlineto{\pgfqpoint{4.708100in}{4.457367in}}%
\pgfpathclose%
\pgfusepath{fill}%
\end{pgfscope}%
\begin{pgfscope}%
\pgfpathrectangle{\pgfqpoint{0.680860in}{0.078740in}}{\pgfqpoint{7.842520in}{7.842520in}}%
\pgfusepath{clip}%
\pgfsetbuttcap%
\pgfsetroundjoin%
\definecolor{currentfill}{rgb}{0.120092,0.600104,0.542530}%
\pgfsetfillcolor{currentfill}%
\pgfsetlinewidth{0.000000pt}%
\definecolor{currentstroke}{rgb}{0.993248,0.906157,0.143936}%
\pgfsetstrokecolor{currentstroke}%
\pgfsetdash{}{0pt}%
\pgfpathmoveto{\pgfqpoint{4.125553in}{4.218073in}}%
\pgfpathlineto{\pgfqpoint{4.043100in}{4.103366in}}%
\pgfpathlineto{\pgfqpoint{4.182352in}{4.119771in}}%
\pgfpathclose%
\pgfusepath{fill}%
\end{pgfscope}%
\begin{pgfscope}%
\pgfpathrectangle{\pgfqpoint{0.680860in}{0.078740in}}{\pgfqpoint{7.842520in}{7.842520in}}%
\pgfusepath{clip}%
\pgfsetbuttcap%
\pgfsetroundjoin%
\definecolor{currentfill}{rgb}{0.137770,0.537492,0.554906}%
\pgfsetfillcolor{currentfill}%
\pgfsetlinewidth{0.000000pt}%
\definecolor{currentstroke}{rgb}{0.267004,0.004874,0.329415}%
\pgfsetstrokecolor{currentstroke}%
\pgfsetdash{}{0pt}%
\pgfpathmoveto{\pgfqpoint{3.600710in}{3.803441in}}%
\pgfpathlineto{\pgfqpoint{3.738866in}{3.811488in}}%
\pgfpathlineto{\pgfqpoint{3.821749in}{3.958466in}}%
\pgfpathclose%
\pgfusepath{fill}%
\end{pgfscope}%
\begin{pgfscope}%
\pgfpathrectangle{\pgfqpoint{0.680860in}{0.078740in}}{\pgfqpoint{7.842520in}{7.842520in}}%
\pgfusepath{clip}%
\pgfsetbuttcap%
\pgfsetroundjoin%
\definecolor{currentfill}{rgb}{0.132268,0.655014,0.519661}%
\pgfsetfillcolor{currentfill}%
\pgfsetlinewidth{0.000000pt}%
\definecolor{currentstroke}{rgb}{0.268510,0.009605,0.335427}%
\pgfsetstrokecolor{currentstroke}%
\pgfsetdash{}{0pt}%
\pgfpathmoveto{\pgfqpoint{4.486392in}{4.354795in}}%
\pgfpathlineto{\pgfqpoint{4.626744in}{4.377935in}}%
\pgfpathlineto{\pgfqpoint{4.567927in}{4.433785in}}%
\pgfpathclose%
\pgfusepath{fill}%
\end{pgfscope}%
\begin{pgfscope}%
\pgfpathrectangle{\pgfqpoint{0.680860in}{0.078740in}}{\pgfqpoint{7.842520in}{7.842520in}}%
\pgfusepath{clip}%
\pgfsetbuttcap%
\pgfsetroundjoin%
\definecolor{currentfill}{rgb}{0.119423,0.611141,0.538982}%
\pgfsetfillcolor{currentfill}%
\pgfsetlinewidth{0.000000pt}%
\definecolor{currentstroke}{rgb}{0.269944,0.014625,0.341379}%
\pgfsetstrokecolor{currentstroke}%
\pgfsetdash{}{0pt}%
\pgfpathmoveto{\pgfqpoint{4.264682in}{4.236141in}}%
\pgfpathlineto{\pgfqpoint{4.125553in}{4.218073in}}%
\pgfpathlineto{\pgfqpoint{4.182352in}{4.119771in}}%
\pgfpathclose%
\pgfusepath{fill}%
\end{pgfscope}%
\begin{pgfscope}%
\pgfpathrectangle{\pgfqpoint{0.680860in}{0.078740in}}{\pgfqpoint{7.842520in}{7.842520in}}%
\pgfusepath{clip}%
\pgfsetbuttcap%
\pgfsetroundjoin%
\definecolor{currentfill}{rgb}{0.218130,0.347432,0.550038}%
\pgfsetfillcolor{currentfill}%
\pgfsetlinewidth{0.000000pt}%
\definecolor{currentstroke}{rgb}{0.271305,0.019942,0.347269}%
\pgfsetstrokecolor{currentstroke}%
\pgfsetdash{}{0pt}%
\pgfpathmoveto{\pgfqpoint{2.993376in}{3.099240in}}%
\pgfpathlineto{\pgfqpoint{2.856665in}{3.109596in}}%
\pgfpathlineto{\pgfqpoint{2.910233in}{2.897179in}}%
\pgfpathclose%
\pgfusepath{fill}%
\end{pgfscope}%
\begin{pgfscope}%
\pgfpathrectangle{\pgfqpoint{0.680860in}{0.078740in}}{\pgfqpoint{7.842520in}{7.842520in}}%
\pgfusepath{clip}%
\pgfsetbuttcap%
\pgfsetroundjoin%
\definecolor{currentfill}{rgb}{0.283229,0.120777,0.440584}%
\pgfsetfillcolor{currentfill}%
\pgfsetlinewidth{0.000000pt}%
\definecolor{currentstroke}{rgb}{0.272594,0.025563,0.353093}%
\pgfsetstrokecolor{currentstroke}%
\pgfsetdash{}{0pt}%
\pgfpathmoveto{\pgfqpoint{2.251387in}{2.330588in}}%
\pgfpathlineto{\pgfqpoint{2.304212in}{2.080603in}}%
\pgfpathlineto{\pgfqpoint{2.387446in}{2.300619in}}%
\pgfpathclose%
\pgfusepath{fill}%
\end{pgfscope}%
\begin{pgfscope}%
\pgfpathrectangle{\pgfqpoint{0.680860in}{0.078740in}}{\pgfqpoint{7.842520in}{7.842520in}}%
\pgfusepath{clip}%
\pgfsetbuttcap%
\pgfsetroundjoin%
\definecolor{currentfill}{rgb}{0.124780,0.640461,0.527068}%
\pgfsetfillcolor{currentfill}%
\pgfsetlinewidth{0.000000pt}%
\definecolor{currentstroke}{rgb}{0.273809,0.031497,0.358853}%
\pgfsetstrokecolor{currentstroke}%
\pgfsetdash{}{0pt}%
\pgfpathmoveto{\pgfqpoint{4.404488in}{4.256152in}}%
\pgfpathlineto{\pgfqpoint{4.486392in}{4.354795in}}%
\pgfpathlineto{\pgfqpoint{4.346734in}{4.333704in}}%
\pgfpathclose%
\pgfusepath{fill}%
\end{pgfscope}%
\begin{pgfscope}%
\pgfpathrectangle{\pgfqpoint{0.680860in}{0.078740in}}{\pgfqpoint{7.842520in}{7.842520in}}%
\pgfusepath{clip}%
\pgfsetbuttcap%
\pgfsetroundjoin%
\definecolor{currentfill}{rgb}{0.121380,0.629492,0.531973}%
\pgfsetfillcolor{currentfill}%
\pgfsetlinewidth{0.000000pt}%
\definecolor{currentstroke}{rgb}{0.274952,0.037752,0.364543}%
\pgfsetstrokecolor{currentstroke}%
\pgfsetdash{}{0pt}%
\pgfpathmoveto{\pgfqpoint{4.346734in}{4.333704in}}%
\pgfpathlineto{\pgfqpoint{4.264682in}{4.236141in}}%
\pgfpathlineto{\pgfqpoint{4.404488in}{4.256152in}}%
\pgfpathclose%
\pgfusepath{fill}%
\end{pgfscope}%
\begin{pgfscope}%
\pgfpathrectangle{\pgfqpoint{0.680860in}{0.078740in}}{\pgfqpoint{7.842520in}{7.842520in}}%
\pgfusepath{clip}%
\pgfsetbuttcap%
\pgfsetroundjoin%
\definecolor{currentfill}{rgb}{0.262138,0.242286,0.520837}%
\pgfsetfillcolor{currentfill}%
\pgfsetlinewidth{0.000000pt}%
\definecolor{currentstroke}{rgb}{0.276022,0.044167,0.370164}%
\pgfsetstrokecolor{currentstroke}%
\pgfsetdash{}{0pt}%
\pgfpathmoveto{\pgfqpoint{2.607119in}{2.491253in}}%
\pgfpathlineto{\pgfqpoint{2.690262in}{2.705363in}}%
\pgfpathlineto{\pgfqpoint{2.553910in}{2.725397in}}%
\pgfpathclose%
\pgfusepath{fill}%
\end{pgfscope}%
\begin{pgfscope}%
\pgfpathrectangle{\pgfqpoint{0.680860in}{0.078740in}}{\pgfqpoint{7.842520in}{7.842520in}}%
\pgfusepath{clip}%
\pgfsetbuttcap%
\pgfsetroundjoin%
\definecolor{currentfill}{rgb}{0.183898,0.422383,0.556944}%
\pgfsetfillcolor{currentfill}%
\pgfsetlinewidth{0.000000pt}%
\definecolor{currentstroke}{rgb}{0.277018,0.050344,0.375715}%
\pgfsetstrokecolor{currentstroke}%
\pgfsetdash{}{0pt}%
\pgfpathmoveto{\pgfqpoint{3.296894in}{3.469525in}}%
\pgfpathlineto{\pgfqpoint{3.076561in}{3.290871in}}%
\pgfpathlineto{\pgfqpoint{3.213753in}{3.285681in}}%
\pgfpathclose%
\pgfusepath{fill}%
\end{pgfscope}%
\begin{pgfscope}%
\pgfpathrectangle{\pgfqpoint{0.680860in}{0.078740in}}{\pgfqpoint{7.842520in}{7.842520in}}%
\pgfusepath{clip}%
\pgfsetbuttcap%
\pgfsetroundjoin%
\definecolor{currentfill}{rgb}{0.166617,0.463708,0.558119}%
\pgfsetfillcolor{currentfill}%
\pgfsetlinewidth{0.000000pt}%
\definecolor{currentstroke}{rgb}{0.277941,0.056324,0.381191}%
\pgfsetstrokecolor{currentstroke}%
\pgfsetdash{}{0pt}%
\pgfpathmoveto{\pgfqpoint{3.380042in}{3.640270in}}%
\pgfpathlineto{\pgfqpoint{3.296894in}{3.469525in}}%
\pgfpathlineto{\pgfqpoint{3.434594in}{3.469432in}}%
\pgfpathclose%
\pgfusepath{fill}%
\end{pgfscope}%
\begin{pgfscope}%
\pgfpathrectangle{\pgfqpoint{0.680860in}{0.078740in}}{\pgfqpoint{7.842520in}{7.842520in}}%
\pgfusepath{clip}%
\pgfsetbuttcap%
\pgfsetroundjoin%
\definecolor{currentfill}{rgb}{0.160665,0.478540,0.558115}%
\pgfsetfillcolor{currentfill}%
\pgfsetlinewidth{0.000000pt}%
\definecolor{currentstroke}{rgb}{0.278791,0.062145,0.386592}%
\pgfsetstrokecolor{currentstroke}%
\pgfsetdash{}{0pt}%
\pgfpathmoveto{\pgfqpoint{3.517668in}{3.643709in}}%
\pgfpathlineto{\pgfqpoint{3.380042in}{3.640270in}}%
\pgfpathlineto{\pgfqpoint{3.434594in}{3.469432in}}%
\pgfpathclose%
\pgfusepath{fill}%
\end{pgfscope}%
\begin{pgfscope}%
\pgfpathrectangle{\pgfqpoint{0.680860in}{0.078740in}}{\pgfqpoint{7.842520in}{7.842520in}}%
\pgfusepath{clip}%
\pgfsetbuttcap%
\pgfsetroundjoin%
\definecolor{currentfill}{rgb}{0.125394,0.574318,0.549086}%
\pgfsetfillcolor{currentfill}%
\pgfsetlinewidth{0.000000pt}%
\definecolor{currentstroke}{rgb}{0.279566,0.067836,0.391917}%
\pgfsetstrokecolor{currentstroke}%
\pgfsetdash{}{0pt}%
\pgfpathmoveto{\pgfqpoint{3.960448in}{3.970858in}}%
\pgfpathlineto{\pgfqpoint{4.043100in}{4.103366in}}%
\pgfpathlineto{\pgfqpoint{3.821749in}{3.958466in}}%
\pgfpathclose%
\pgfusepath{fill}%
\end{pgfscope}%
\begin{pgfscope}%
\pgfpathrectangle{\pgfqpoint{0.680860in}{0.078740in}}{\pgfqpoint{7.842520in}{7.842520in}}%
\pgfusepath{clip}%
\pgfsetbuttcap%
\pgfsetroundjoin%
\definecolor{currentfill}{rgb}{0.220124,0.725509,0.466226}%
\pgfsetfillcolor{currentfill}%
\pgfsetlinewidth{0.000000pt}%
\definecolor{currentstroke}{rgb}{0.280267,0.073417,0.397163}%
\pgfsetstrokecolor{currentstroke}%
\pgfsetdash{}{0pt}%
\pgfpathmoveto{\pgfqpoint{6.677475in}{4.658470in}}%
\pgfpathlineto{\pgfqpoint{6.606180in}{4.755788in}}%
\pgfpathlineto{\pgfqpoint{6.823210in}{4.688299in}}%
\pgfpathclose%
\pgfusepath{fill}%
\end{pgfscope}%
\begin{pgfscope}%
\pgfpathrectangle{\pgfqpoint{0.680860in}{0.078740in}}{\pgfqpoint{7.842520in}{7.842520in}}%
\pgfusepath{clip}%
\pgfsetbuttcap%
\pgfsetroundjoin%
\definecolor{currentfill}{rgb}{0.153894,0.680203,0.504172}%
\pgfsetfillcolor{currentfill}%
\pgfsetlinewidth{0.000000pt}%
\definecolor{currentstroke}{rgb}{0.280894,0.078907,0.402329}%
\pgfsetstrokecolor{currentstroke}%
\pgfsetdash{}{0pt}%
\pgfpathmoveto{\pgfqpoint{4.708100in}{4.457367in}}%
\pgfpathlineto{\pgfqpoint{4.848986in}{4.483086in}}%
\pgfpathlineto{\pgfqpoint{4.929657in}{4.541960in}}%
\pgfpathclose%
\pgfusepath{fill}%
\end{pgfscope}%
\begin{pgfscope}%
\pgfpathrectangle{\pgfqpoint{0.680860in}{0.078740in}}{\pgfqpoint{7.842520in}{7.842520in}}%
\pgfusepath{clip}%
\pgfsetbuttcap%
\pgfsetroundjoin%
\definecolor{currentfill}{rgb}{0.273809,0.031497,0.358853}%
\pgfsetfillcolor{currentfill}%
\pgfsetlinewidth{0.000000pt}%
\definecolor{currentstroke}{rgb}{0.281446,0.084320,0.407414}%
\pgfsetstrokecolor{currentstroke}%
\pgfsetdash{}{0pt}%
\pgfpathmoveto{\pgfqpoint{2.168032in}{2.115434in}}%
\pgfpathlineto{\pgfqpoint{2.084625in}{1.897138in}}%
\pgfpathlineto{\pgfqpoint{2.220933in}{1.857373in}}%
\pgfpathclose%
\pgfusepath{fill}%
\end{pgfscope}%
\begin{pgfscope}%
\pgfpathrectangle{\pgfqpoint{0.680860in}{0.078740in}}{\pgfqpoint{7.842520in}{7.842520in}}%
\pgfusepath{clip}%
\pgfsetbuttcap%
\pgfsetroundjoin%
\definecolor{currentfill}{rgb}{0.185783,0.704891,0.485273}%
\pgfsetfillcolor{currentfill}%
\pgfsetlinewidth{0.000000pt}%
\definecolor{currentstroke}{rgb}{0.281924,0.089666,0.412415}%
\pgfsetstrokecolor{currentstroke}%
\pgfsetdash{}{0pt}%
\pgfpathmoveto{\pgfqpoint{5.071052in}{4.569641in}}%
\pgfpathlineto{\pgfqpoint{5.292761in}{4.635734in}}%
\pgfpathlineto{\pgfqpoint{5.150891in}{4.606777in}}%
\pgfpathclose%
\pgfusepath{fill}%
\end{pgfscope}%
\begin{pgfscope}%
\pgfpathrectangle{\pgfqpoint{0.680860in}{0.078740in}}{\pgfqpoint{7.842520in}{7.842520in}}%
\pgfusepath{clip}%
\pgfsetbuttcap%
\pgfsetroundjoin%
\definecolor{currentfill}{rgb}{0.203063,0.379716,0.553925}%
\pgfsetfillcolor{currentfill}%
\pgfsetlinewidth{0.000000pt}%
\definecolor{currentstroke}{rgb}{0.282327,0.094955,0.417331}%
\pgfsetstrokecolor{currentstroke}%
\pgfsetdash{}{0pt}%
\pgfpathmoveto{\pgfqpoint{3.130639in}{3.089850in}}%
\pgfpathlineto{\pgfqpoint{3.076561in}{3.290871in}}%
\pgfpathlineto{\pgfqpoint{2.993376in}{3.099240in}}%
\pgfpathclose%
\pgfusepath{fill}%
\end{pgfscope}%
\begin{pgfscope}%
\pgfpathrectangle{\pgfqpoint{0.680860in}{0.078740in}}{\pgfqpoint{7.842520in}{7.842520in}}%
\pgfusepath{clip}%
\pgfsetbuttcap%
\pgfsetroundjoin%
\definecolor{currentfill}{rgb}{0.280255,0.165693,0.476498}%
\pgfsetfillcolor{currentfill}%
\pgfsetlinewidth{0.000000pt}%
\definecolor{currentstroke}{rgb}{0.282656,0.100196,0.422160}%
\pgfsetstrokecolor{currentstroke}%
\pgfsetdash{}{0pt}%
\pgfpathmoveto{\pgfqpoint{2.387446in}{2.300619in}}%
\pgfpathlineto{\pgfqpoint{2.523996in}{2.270968in}}%
\pgfpathlineto{\pgfqpoint{2.470670in}{2.516010in}}%
\pgfpathclose%
\pgfusepath{fill}%
\end{pgfscope}%
\begin{pgfscope}%
\pgfpathrectangle{\pgfqpoint{0.680860in}{0.078740in}}{\pgfqpoint{7.842520in}{7.842520in}}%
\pgfusepath{clip}%
\pgfsetbuttcap%
\pgfsetroundjoin%
\definecolor{currentfill}{rgb}{0.246811,0.283237,0.535941}%
\pgfsetfillcolor{currentfill}%
\pgfsetlinewidth{0.000000pt}%
\definecolor{currentstroke}{rgb}{0.282910,0.105393,0.426902}%
\pgfsetstrokecolor{currentstroke}%
\pgfsetdash{}{0pt}%
\pgfpathmoveto{\pgfqpoint{2.773441in}{2.911917in}}%
\pgfpathlineto{\pgfqpoint{2.690262in}{2.705363in}}%
\pgfpathlineto{\pgfqpoint{2.827135in}{2.685973in}}%
\pgfpathclose%
\pgfusepath{fill}%
\end{pgfscope}%
\begin{pgfscope}%
\pgfpathrectangle{\pgfqpoint{0.680860in}{0.078740in}}{\pgfqpoint{7.842520in}{7.842520in}}%
\pgfusepath{clip}%
\pgfsetbuttcap%
\pgfsetroundjoin%
\definecolor{currentfill}{rgb}{0.278791,0.062145,0.386592}%
\pgfsetfillcolor{currentfill}%
\pgfsetlinewidth{0.000000pt}%
\definecolor{currentstroke}{rgb}{0.283091,0.110553,0.431554}%
\pgfsetstrokecolor{currentstroke}%
\pgfsetdash{}{0pt}%
\pgfpathmoveto{\pgfqpoint{2.220933in}{1.857373in}}%
\pgfpathlineto{\pgfqpoint{2.304212in}{2.080603in}}%
\pgfpathlineto{\pgfqpoint{2.168032in}{2.115434in}}%
\pgfpathclose%
\pgfusepath{fill}%
\end{pgfscope}%
\begin{pgfscope}%
\pgfpathrectangle{\pgfqpoint{0.680860in}{0.078740in}}{\pgfqpoint{7.842520in}{7.842520in}}%
\pgfusepath{clip}%
\pgfsetbuttcap%
\pgfsetroundjoin%
\definecolor{currentfill}{rgb}{0.150476,0.504369,0.557430}%
\pgfsetfillcolor{currentfill}%
\pgfsetlinewidth{0.000000pt}%
\definecolor{currentstroke}{rgb}{0.283197,0.115680,0.436115}%
\pgfsetstrokecolor{currentstroke}%
\pgfsetdash{}{0pt}%
\pgfpathmoveto{\pgfqpoint{3.655900in}{3.648583in}}%
\pgfpathlineto{\pgfqpoint{3.600710in}{3.803441in}}%
\pgfpathlineto{\pgfqpoint{3.517668in}{3.643709in}}%
\pgfpathclose%
\pgfusepath{fill}%
\end{pgfscope}%
\begin{pgfscope}%
\pgfpathrectangle{\pgfqpoint{0.680860in}{0.078740in}}{\pgfqpoint{7.842520in}{7.842520in}}%
\pgfusepath{clip}%
\pgfsetbuttcap%
\pgfsetroundjoin%
\definecolor{currentfill}{rgb}{0.194100,0.399323,0.555565}%
\pgfsetfillcolor{currentfill}%
\pgfsetlinewidth{0.000000pt}%
\definecolor{currentstroke}{rgb}{0.283229,0.120777,0.440584}%
\pgfsetstrokecolor{currentstroke}%
\pgfsetdash{}{0pt}%
\pgfpathmoveto{\pgfqpoint{3.213753in}{3.285681in}}%
\pgfpathlineto{\pgfqpoint{3.076561in}{3.290871in}}%
\pgfpathlineto{\pgfqpoint{3.130639in}{3.089850in}}%
\pgfpathclose%
\pgfusepath{fill}%
\end{pgfscope}%
\begin{pgfscope}%
\pgfpathrectangle{\pgfqpoint{0.680860in}{0.078740in}}{\pgfqpoint{7.842520in}{7.842520in}}%
\pgfusepath{clip}%
\pgfsetbuttcap%
\pgfsetroundjoin%
\definecolor{currentfill}{rgb}{0.277134,0.185228,0.489898}%
\pgfsetfillcolor{currentfill}%
\pgfsetlinewidth{0.000000pt}%
\definecolor{currentstroke}{rgb}{0.283187,0.125848,0.444960}%
\pgfsetstrokecolor{currentstroke}%
\pgfsetdash{}{0pt}%
\pgfpathmoveto{\pgfqpoint{2.470670in}{2.516010in}}%
\pgfpathlineto{\pgfqpoint{2.523996in}{2.270968in}}%
\pgfpathlineto{\pgfqpoint{2.607119in}{2.491253in}}%
\pgfpathclose%
\pgfusepath{fill}%
\end{pgfscope}%
\begin{pgfscope}%
\pgfpathrectangle{\pgfqpoint{0.680860in}{0.078740in}}{\pgfqpoint{7.842520in}{7.842520in}}%
\pgfusepath{clip}%
\pgfsetbuttcap%
\pgfsetroundjoin%
\definecolor{currentfill}{rgb}{0.239346,0.300855,0.540844}%
\pgfsetfillcolor{currentfill}%
\pgfsetlinewidth{0.000000pt}%
\definecolor{currentstroke}{rgb}{0.283072,0.130895,0.449241}%
\pgfsetstrokecolor{currentstroke}%
\pgfsetdash{}{0pt}%
\pgfpathmoveto{\pgfqpoint{2.827135in}{2.685973in}}%
\pgfpathlineto{\pgfqpoint{2.910233in}{2.897179in}}%
\pgfpathlineto{\pgfqpoint{2.773441in}{2.911917in}}%
\pgfpathclose%
\pgfusepath{fill}%
\end{pgfscope}%
\begin{pgfscope}%
\pgfpathrectangle{\pgfqpoint{0.680860in}{0.078740in}}{\pgfqpoint{7.842520in}{7.842520in}}%
\pgfusepath{clip}%
\pgfsetbuttcap%
\pgfsetroundjoin%
\definecolor{currentfill}{rgb}{0.226397,0.728888,0.462789}%
\pgfsetfillcolor{currentfill}%
\pgfsetlinewidth{0.000000pt}%
\definecolor{currentstroke}{rgb}{0.282884,0.135920,0.453427}%
\pgfsetstrokecolor{currentstroke}%
\pgfsetdash{}{0pt}%
\pgfpathmoveto{\pgfqpoint{5.734289in}{4.699798in}}%
\pgfpathlineto{\pgfqpoint{5.656982in}{4.711478in}}%
\pgfpathlineto{\pgfqpoint{5.877741in}{4.731360in}}%
\pgfpathclose%
\pgfusepath{fill}%
\end{pgfscope}%
\begin{pgfscope}%
\pgfpathrectangle{\pgfqpoint{0.680860in}{0.078740in}}{\pgfqpoint{7.842520in}{7.842520in}}%
\pgfusepath{clip}%
\pgfsetbuttcap%
\pgfsetroundjoin%
\definecolor{currentfill}{rgb}{0.239374,0.735588,0.455688}%
\pgfsetfillcolor{currentfill}%
\pgfsetlinewidth{0.000000pt}%
\definecolor{currentstroke}{rgb}{0.282623,0.140926,0.457517}%
\pgfsetstrokecolor{currentstroke}%
\pgfsetdash{}{0pt}%
\pgfpathmoveto{\pgfqpoint{6.460572in}{4.722740in}}%
\pgfpathlineto{\pgfqpoint{6.315778in}{4.692081in}}%
\pgfpathlineto{\pgfqpoint{6.387367in}{4.793337in}}%
\pgfpathclose%
\pgfusepath{fill}%
\end{pgfscope}%
\begin{pgfscope}%
\pgfpathrectangle{\pgfqpoint{0.680860in}{0.078740in}}{\pgfqpoint{7.842520in}{7.842520in}}%
\pgfusepath{clip}%
\pgfsetbuttcap%
\pgfsetroundjoin%
\definecolor{currentfill}{rgb}{0.175841,0.441290,0.557685}%
\pgfsetfillcolor{currentfill}%
\pgfsetlinewidth{0.000000pt}%
\definecolor{currentstroke}{rgb}{0.282290,0.145912,0.461510}%
\pgfsetstrokecolor{currentstroke}%
\pgfsetdash{}{0pt}%
\pgfpathmoveto{\pgfqpoint{3.434594in}{3.469432in}}%
\pgfpathlineto{\pgfqpoint{3.296894in}{3.469525in}}%
\pgfpathlineto{\pgfqpoint{3.213753in}{3.285681in}}%
\pgfpathclose%
\pgfusepath{fill}%
\end{pgfscope}%
\begin{pgfscope}%
\pgfpathrectangle{\pgfqpoint{0.680860in}{0.078740in}}{\pgfqpoint{7.842520in}{7.842520in}}%
\pgfusepath{clip}%
\pgfsetbuttcap%
\pgfsetroundjoin%
\definecolor{currentfill}{rgb}{0.144759,0.519093,0.556572}%
\pgfsetfillcolor{currentfill}%
\pgfsetlinewidth{0.000000pt}%
\definecolor{currentstroke}{rgb}{0.281887,0.150881,0.465405}%
\pgfsetstrokecolor{currentstroke}%
\pgfsetdash{}{0pt}%
\pgfpathmoveto{\pgfqpoint{3.655900in}{3.648583in}}%
\pgfpathlineto{\pgfqpoint{3.738866in}{3.811488in}}%
\pgfpathlineto{\pgfqpoint{3.600710in}{3.803441in}}%
\pgfpathclose%
\pgfusepath{fill}%
\end{pgfscope}%
\begin{pgfscope}%
\pgfpathrectangle{\pgfqpoint{0.680860in}{0.078740in}}{\pgfqpoint{7.842520in}{7.842520in}}%
\pgfusepath{clip}%
\pgfsetbuttcap%
\pgfsetroundjoin%
\definecolor{currentfill}{rgb}{0.208030,0.718701,0.472873}%
\pgfsetfillcolor{currentfill}%
\pgfsetlinewidth{0.000000pt}%
\definecolor{currentstroke}{rgb}{0.281412,0.155834,0.469201}%
\pgfsetstrokecolor{currentstroke}%
\pgfsetdash{}{0pt}%
\pgfpathmoveto{\pgfqpoint{7.038179in}{4.590300in}}%
\pgfpathlineto{\pgfqpoint{7.117183in}{4.755126in}}%
\pgfpathlineto{\pgfqpoint{7.184740in}{4.618099in}}%
\pgfpathclose%
\pgfusepath{fill}%
\end{pgfscope}%
\begin{pgfscope}%
\pgfpathrectangle{\pgfqpoint{0.680860in}{0.078740in}}{\pgfqpoint{7.842520in}{7.842520in}}%
\pgfusepath{clip}%
\pgfsetbuttcap%
\pgfsetroundjoin%
\definecolor{currentfill}{rgb}{0.166383,0.690856,0.496502}%
\pgfsetfillcolor{currentfill}%
\pgfsetlinewidth{0.000000pt}%
\definecolor{currentstroke}{rgb}{0.280868,0.160771,0.472899}%
\pgfsetstrokecolor{currentstroke}%
\pgfsetdash{}{0pt}%
\pgfpathmoveto{\pgfqpoint{5.071052in}{4.569641in}}%
\pgfpathlineto{\pgfqpoint{4.929657in}{4.541960in}}%
\pgfpathlineto{\pgfqpoint{4.848986in}{4.483086in}}%
\pgfpathclose%
\pgfusepath{fill}%
\end{pgfscope}%
\begin{pgfscope}%
\pgfpathrectangle{\pgfqpoint{0.680860in}{0.078740in}}{\pgfqpoint{7.842520in}{7.842520in}}%
\pgfusepath{clip}%
\pgfsetbuttcap%
\pgfsetroundjoin%
\definecolor{currentfill}{rgb}{0.283187,0.125848,0.444960}%
\pgfsetfillcolor{currentfill}%
\pgfsetlinewidth{0.000000pt}%
\definecolor{currentstroke}{rgb}{0.280255,0.165693,0.476498}%
\pgfsetstrokecolor{currentstroke}%
\pgfsetdash{}{0pt}%
\pgfpathmoveto{\pgfqpoint{2.387446in}{2.300619in}}%
\pgfpathlineto{\pgfqpoint{2.304212in}{2.080603in}}%
\pgfpathlineto{\pgfqpoint{2.523996in}{2.270968in}}%
\pgfpathclose%
\pgfusepath{fill}%
\end{pgfscope}%
\begin{pgfscope}%
\pgfpathrectangle{\pgfqpoint{0.680860in}{0.078740in}}{\pgfqpoint{7.842520in}{7.842520in}}%
\pgfusepath{clip}%
\pgfsetbuttcap%
\pgfsetroundjoin%
\definecolor{currentfill}{rgb}{0.121831,0.589055,0.545623}%
\pgfsetfillcolor{currentfill}%
\pgfsetlinewidth{0.000000pt}%
\definecolor{currentstroke}{rgb}{0.279574,0.170599,0.479997}%
\pgfsetstrokecolor{currentstroke}%
\pgfsetdash{}{0pt}%
\pgfpathmoveto{\pgfqpoint{4.182352in}{4.119771in}}%
\pgfpathlineto{\pgfqpoint{4.043100in}{4.103366in}}%
\pgfpathlineto{\pgfqpoint{3.960448in}{3.970858in}}%
\pgfpathclose%
\pgfusepath{fill}%
\end{pgfscope}%
\begin{pgfscope}%
\pgfpathrectangle{\pgfqpoint{0.680860in}{0.078740in}}{\pgfqpoint{7.842520in}{7.842520in}}%
\pgfusepath{clip}%
\pgfsetbuttcap%
\pgfsetroundjoin%
\definecolor{currentfill}{rgb}{0.260571,0.246922,0.522828}%
\pgfsetfillcolor{currentfill}%
\pgfsetlinewidth{0.000000pt}%
\definecolor{currentstroke}{rgb}{0.278826,0.175490,0.483397}%
\pgfsetstrokecolor{currentstroke}%
\pgfsetdash{}{0pt}%
\pgfpathmoveto{\pgfqpoint{2.827135in}{2.685973in}}%
\pgfpathlineto{\pgfqpoint{2.690262in}{2.705363in}}%
\pgfpathlineto{\pgfqpoint{2.607119in}{2.491253in}}%
\pgfpathclose%
\pgfusepath{fill}%
\end{pgfscope}%
\begin{pgfscope}%
\pgfpathrectangle{\pgfqpoint{0.680860in}{0.078740in}}{\pgfqpoint{7.842520in}{7.842520in}}%
\pgfusepath{clip}%
\pgfsetbuttcap%
\pgfsetroundjoin%
\definecolor{currentfill}{rgb}{0.135066,0.544853,0.554029}%
\pgfsetfillcolor{currentfill}%
\pgfsetlinewidth{0.000000pt}%
\definecolor{currentstroke}{rgb}{0.278012,0.180367,0.486697}%
\pgfsetstrokecolor{currentstroke}%
\pgfsetdash{}{0pt}%
\pgfpathmoveto{\pgfqpoint{3.821749in}{3.958466in}}%
\pgfpathlineto{\pgfqpoint{3.738866in}{3.811488in}}%
\pgfpathlineto{\pgfqpoint{3.877650in}{3.821131in}}%
\pgfpathclose%
\pgfusepath{fill}%
\end{pgfscope}%
\begin{pgfscope}%
\pgfpathrectangle{\pgfqpoint{0.680860in}{0.078740in}}{\pgfqpoint{7.842520in}{7.842520in}}%
\pgfusepath{clip}%
\pgfsetbuttcap%
\pgfsetroundjoin%
\definecolor{currentfill}{rgb}{0.146616,0.673050,0.508936}%
\pgfsetfillcolor{currentfill}%
\pgfsetlinewidth{0.000000pt}%
\definecolor{currentstroke}{rgb}{0.277134,0.185228,0.489898}%
\pgfsetstrokecolor{currentstroke}%
\pgfsetdash{}{0pt}%
\pgfpathmoveto{\pgfqpoint{4.626744in}{4.377935in}}%
\pgfpathlineto{\pgfqpoint{4.848986in}{4.483086in}}%
\pgfpathlineto{\pgfqpoint{4.708100in}{4.457367in}}%
\pgfpathclose%
\pgfusepath{fill}%
\end{pgfscope}%
\begin{pgfscope}%
\pgfpathrectangle{\pgfqpoint{0.680860in}{0.078740in}}{\pgfqpoint{7.842520in}{7.842520in}}%
\pgfusepath{clip}%
\pgfsetbuttcap%
\pgfsetroundjoin%
\definecolor{currentfill}{rgb}{0.120081,0.622161,0.534946}%
\pgfsetfillcolor{currentfill}%
\pgfsetlinewidth{0.000000pt}%
\definecolor{currentstroke}{rgb}{0.276194,0.190074,0.493001}%
\pgfsetstrokecolor{currentstroke}%
\pgfsetdash{}{0pt}%
\pgfpathmoveto{\pgfqpoint{4.182352in}{4.119771in}}%
\pgfpathlineto{\pgfqpoint{4.404488in}{4.256152in}}%
\pgfpathlineto{\pgfqpoint{4.264682in}{4.236141in}}%
\pgfpathclose%
\pgfusepath{fill}%
\end{pgfscope}%
\begin{pgfscope}%
\pgfpathrectangle{\pgfqpoint{0.680860in}{0.078740in}}{\pgfqpoint{7.842520in}{7.842520in}}%
\pgfusepath{clip}%
\pgfsetbuttcap%
\pgfsetroundjoin%
\definecolor{currentfill}{rgb}{0.129933,0.559582,0.551864}%
\pgfsetfillcolor{currentfill}%
\pgfsetlinewidth{0.000000pt}%
\definecolor{currentstroke}{rgb}{0.275191,0.194905,0.496005}%
\pgfsetstrokecolor{currentstroke}%
\pgfsetdash{}{0pt}%
\pgfpathmoveto{\pgfqpoint{3.877650in}{3.821131in}}%
\pgfpathlineto{\pgfqpoint{3.960448in}{3.970858in}}%
\pgfpathlineto{\pgfqpoint{3.821749in}{3.958466in}}%
\pgfpathclose%
\pgfusepath{fill}%
\end{pgfscope}%
\begin{pgfscope}%
\pgfpathrectangle{\pgfqpoint{0.680860in}{0.078740in}}{\pgfqpoint{7.842520in}{7.842520in}}%
\pgfusepath{clip}%
\pgfsetbuttcap%
\pgfsetroundjoin%
\definecolor{currentfill}{rgb}{0.130067,0.651384,0.521608}%
\pgfsetfillcolor{currentfill}%
\pgfsetlinewidth{0.000000pt}%
\definecolor{currentstroke}{rgb}{0.274128,0.199721,0.498911}%
\pgfsetstrokecolor{currentstroke}%
\pgfsetdash{}{0pt}%
\pgfpathmoveto{\pgfqpoint{4.404488in}{4.256152in}}%
\pgfpathlineto{\pgfqpoint{4.626744in}{4.377935in}}%
\pgfpathlineto{\pgfqpoint{4.486392in}{4.354795in}}%
\pgfpathclose%
\pgfusepath{fill}%
\end{pgfscope}%
\begin{pgfscope}%
\pgfpathrectangle{\pgfqpoint{0.680860in}{0.078740in}}{\pgfqpoint{7.842520in}{7.842520in}}%
\pgfusepath{clip}%
\pgfsetbuttcap%
\pgfsetroundjoin%
\definecolor{currentfill}{rgb}{0.221989,0.339161,0.548752}%
\pgfsetfillcolor{currentfill}%
\pgfsetlinewidth{0.000000pt}%
\definecolor{currentstroke}{rgb}{0.273006,0.204520,0.501721}%
\pgfsetstrokecolor{currentstroke}%
\pgfsetdash{}{0pt}%
\pgfpathmoveto{\pgfqpoint{2.993376in}{3.099240in}}%
\pgfpathlineto{\pgfqpoint{2.910233in}{2.897179in}}%
\pgfpathlineto{\pgfqpoint{3.047565in}{2.883264in}}%
\pgfpathclose%
\pgfusepath{fill}%
\end{pgfscope}%
\begin{pgfscope}%
\pgfpathrectangle{\pgfqpoint{0.680860in}{0.078740in}}{\pgfqpoint{7.842520in}{7.842520in}}%
\pgfusepath{clip}%
\pgfsetbuttcap%
\pgfsetroundjoin%
\definecolor{currentfill}{rgb}{0.214000,0.722114,0.469588}%
\pgfsetfillcolor{currentfill}%
\pgfsetlinewidth{0.000000pt}%
\definecolor{currentstroke}{rgb}{0.271828,0.209303,0.504434}%
\pgfsetstrokecolor{currentstroke}%
\pgfsetdash{}{0pt}%
\pgfpathmoveto{\pgfqpoint{5.435389in}{4.667024in}}%
\pgfpathlineto{\pgfqpoint{5.513913in}{4.679638in}}%
\pgfpathlineto{\pgfqpoint{5.292761in}{4.635734in}}%
\pgfpathclose%
\pgfusepath{fill}%
\end{pgfscope}%
\begin{pgfscope}%
\pgfpathrectangle{\pgfqpoint{0.680860in}{0.078740in}}{\pgfqpoint{7.842520in}{7.842520in}}%
\pgfusepath{clip}%
\pgfsetbuttcap%
\pgfsetroundjoin%
\definecolor{currentfill}{rgb}{0.246070,0.738910,0.452024}%
\pgfsetfillcolor{currentfill}%
\pgfsetlinewidth{0.000000pt}%
\definecolor{currentstroke}{rgb}{0.270595,0.214069,0.507052}%
\pgfsetstrokecolor{currentstroke}%
\pgfsetdash{}{0pt}%
\pgfpathmoveto{\pgfqpoint{6.387367in}{4.793337in}}%
\pgfpathlineto{\pgfqpoint{6.315778in}{4.692081in}}%
\pgfpathlineto{\pgfqpoint{6.241987in}{4.758070in}}%
\pgfpathclose%
\pgfusepath{fill}%
\end{pgfscope}%
\begin{pgfscope}%
\pgfpathrectangle{\pgfqpoint{0.680860in}{0.078740in}}{\pgfqpoint{7.842520in}{7.842520in}}%
\pgfusepath{clip}%
\pgfsetbuttcap%
\pgfsetroundjoin%
\definecolor{currentfill}{rgb}{0.246070,0.738910,0.452024}%
\pgfsetfillcolor{currentfill}%
\pgfsetlinewidth{0.000000pt}%
\definecolor{currentstroke}{rgb}{0.269308,0.218818,0.509577}%
\pgfsetstrokecolor{currentstroke}%
\pgfsetdash{}{0pt}%
\pgfpathmoveto{\pgfqpoint{5.877741in}{4.731360in}}%
\pgfpathlineto{\pgfqpoint{6.021988in}{4.765349in}}%
\pgfpathlineto{\pgfqpoint{6.097424in}{4.725276in}}%
\pgfpathclose%
\pgfusepath{fill}%
\end{pgfscope}%
\begin{pgfscope}%
\pgfpathrectangle{\pgfqpoint{0.680860in}{0.078740in}}{\pgfqpoint{7.842520in}{7.842520in}}%
\pgfusepath{clip}%
\pgfsetbuttcap%
\pgfsetroundjoin%
\definecolor{currentfill}{rgb}{0.157729,0.485932,0.558013}%
\pgfsetfillcolor{currentfill}%
\pgfsetlinewidth{0.000000pt}%
\definecolor{currentstroke}{rgb}{0.267968,0.223549,0.512008}%
\pgfsetstrokecolor{currentstroke}%
\pgfsetdash{}{0pt}%
\pgfpathmoveto{\pgfqpoint{3.434594in}{3.469432in}}%
\pgfpathlineto{\pgfqpoint{3.655900in}{3.648583in}}%
\pgfpathlineto{\pgfqpoint{3.517668in}{3.643709in}}%
\pgfpathclose%
\pgfusepath{fill}%
\end{pgfscope}%
\begin{pgfscope}%
\pgfpathrectangle{\pgfqpoint{0.680860in}{0.078740in}}{\pgfqpoint{7.842520in}{7.842520in}}%
\pgfusepath{clip}%
\pgfsetbuttcap%
\pgfsetroundjoin%
\definecolor{currentfill}{rgb}{0.278791,0.062145,0.386592}%
\pgfsetfillcolor{currentfill}%
\pgfsetlinewidth{0.000000pt}%
\definecolor{currentstroke}{rgb}{0.266580,0.228262,0.514349}%
\pgfsetstrokecolor{currentstroke}%
\pgfsetdash{}{0pt}%
\pgfpathmoveto{\pgfqpoint{2.440870in}{2.045935in}}%
\pgfpathlineto{\pgfqpoint{2.304212in}{2.080603in}}%
\pgfpathlineto{\pgfqpoint{2.220933in}{1.857373in}}%
\pgfpathclose%
\pgfusepath{fill}%
\end{pgfscope}%
\begin{pgfscope}%
\pgfpathrectangle{\pgfqpoint{0.680860in}{0.078740in}}{\pgfqpoint{7.842520in}{7.842520in}}%
\pgfusepath{clip}%
\pgfsetbuttcap%
\pgfsetroundjoin%
\definecolor{currentfill}{rgb}{0.214298,0.355619,0.551184}%
\pgfsetfillcolor{currentfill}%
\pgfsetlinewidth{0.000000pt}%
\definecolor{currentstroke}{rgb}{0.265145,0.232956,0.516599}%
\pgfsetstrokecolor{currentstroke}%
\pgfsetdash{}{0pt}%
\pgfpathmoveto{\pgfqpoint{3.047565in}{2.883264in}}%
\pgfpathlineto{\pgfqpoint{3.130639in}{3.089850in}}%
\pgfpathlineto{\pgfqpoint{2.993376in}{3.099240in}}%
\pgfpathclose%
\pgfusepath{fill}%
\end{pgfscope}%
\begin{pgfscope}%
\pgfpathrectangle{\pgfqpoint{0.680860in}{0.078740in}}{\pgfqpoint{7.842520in}{7.842520in}}%
\pgfusepath{clip}%
\pgfsetbuttcap%
\pgfsetroundjoin%
\definecolor{currentfill}{rgb}{0.282910,0.105393,0.426902}%
\pgfsetfillcolor{currentfill}%
\pgfsetlinewidth{0.000000pt}%
\definecolor{currentstroke}{rgb}{0.263663,0.237631,0.518762}%
\pgfsetstrokecolor{currentstroke}%
\pgfsetdash{}{0pt}%
\pgfpathmoveto{\pgfqpoint{2.523996in}{2.270968in}}%
\pgfpathlineto{\pgfqpoint{2.304212in}{2.080603in}}%
\pgfpathlineto{\pgfqpoint{2.440870in}{2.045935in}}%
\pgfpathclose%
\pgfusepath{fill}%
\end{pgfscope}%
\begin{pgfscope}%
\pgfpathrectangle{\pgfqpoint{0.680860in}{0.078740in}}{\pgfqpoint{7.842520in}{7.842520in}}%
\pgfusepath{clip}%
\pgfsetbuttcap%
\pgfsetroundjoin%
\definecolor{currentfill}{rgb}{0.226397,0.728888,0.462789}%
\pgfsetfillcolor{currentfill}%
\pgfsetlinewidth{0.000000pt}%
\definecolor{currentstroke}{rgb}{0.262138,0.242286,0.520837}%
\pgfsetstrokecolor{currentstroke}%
\pgfsetdash{}{0pt}%
\pgfpathmoveto{\pgfqpoint{6.969772in}{4.720488in}}%
\pgfpathlineto{\pgfqpoint{7.117183in}{4.755126in}}%
\pgfpathlineto{\pgfqpoint{7.038179in}{4.590300in}}%
\pgfpathclose%
\pgfusepath{fill}%
\end{pgfscope}%
\begin{pgfscope}%
\pgfpathrectangle{\pgfqpoint{0.680860in}{0.078740in}}{\pgfqpoint{7.842520in}{7.842520in}}%
\pgfusepath{clip}%
\pgfsetbuttcap%
\pgfsetroundjoin%
\definecolor{currentfill}{rgb}{0.276194,0.190074,0.493001}%
\pgfsetfillcolor{currentfill}%
\pgfsetlinewidth{0.000000pt}%
\definecolor{currentstroke}{rgb}{0.260571,0.246922,0.522828}%
\pgfsetstrokecolor{currentstroke}%
\pgfsetdash{}{0pt}%
\pgfpathmoveto{\pgfqpoint{2.523996in}{2.270968in}}%
\pgfpathlineto{\pgfqpoint{2.744075in}{2.466986in}}%
\pgfpathlineto{\pgfqpoint{2.607119in}{2.491253in}}%
\pgfpathclose%
\pgfusepath{fill}%
\end{pgfscope}%
\begin{pgfscope}%
\pgfpathrectangle{\pgfqpoint{0.680860in}{0.078740in}}{\pgfqpoint{7.842520in}{7.842520in}}%
\pgfusepath{clip}%
\pgfsetbuttcap%
\pgfsetroundjoin%
\definecolor{currentfill}{rgb}{0.179019,0.433756,0.557430}%
\pgfsetfillcolor{currentfill}%
\pgfsetlinewidth{0.000000pt}%
\definecolor{currentstroke}{rgb}{0.258965,0.251537,0.524736}%
\pgfsetstrokecolor{currentstroke}%
\pgfsetdash{}{0pt}%
\pgfpathmoveto{\pgfqpoint{3.213753in}{3.285681in}}%
\pgfpathlineto{\pgfqpoint{3.351519in}{3.281632in}}%
\pgfpathlineto{\pgfqpoint{3.434594in}{3.469432in}}%
\pgfpathclose%
\pgfusepath{fill}%
\end{pgfscope}%
\begin{pgfscope}%
\pgfpathrectangle{\pgfqpoint{0.680860in}{0.078740in}}{\pgfqpoint{7.842520in}{7.842520in}}%
\pgfusepath{clip}%
\pgfsetbuttcap%
\pgfsetroundjoin%
\definecolor{currentfill}{rgb}{0.226397,0.728888,0.462789}%
\pgfsetfillcolor{currentfill}%
\pgfsetlinewidth{0.000000pt}%
\definecolor{currentstroke}{rgb}{0.257322,0.256130,0.526563}%
\pgfsetstrokecolor{currentstroke}%
\pgfsetdash{}{0pt}%
\pgfpathmoveto{\pgfqpoint{5.656982in}{4.711478in}}%
\pgfpathlineto{\pgfqpoint{5.513913in}{4.679638in}}%
\pgfpathlineto{\pgfqpoint{5.435389in}{4.667024in}}%
\pgfpathclose%
\pgfusepath{fill}%
\end{pgfscope}%
\begin{pgfscope}%
\pgfpathrectangle{\pgfqpoint{0.680860in}{0.078740in}}{\pgfqpoint{7.842520in}{7.842520in}}%
\pgfusepath{clip}%
\pgfsetbuttcap%
\pgfsetroundjoin%
\definecolor{currentfill}{rgb}{0.265145,0.232956,0.516599}%
\pgfsetfillcolor{currentfill}%
\pgfsetlinewidth{0.000000pt}%
\definecolor{currentstroke}{rgb}{0.255645,0.260703,0.528312}%
\pgfsetstrokecolor{currentstroke}%
\pgfsetdash{}{0pt}%
\pgfpathmoveto{\pgfqpoint{2.607119in}{2.491253in}}%
\pgfpathlineto{\pgfqpoint{2.744075in}{2.466986in}}%
\pgfpathlineto{\pgfqpoint{2.827135in}{2.685973in}}%
\pgfpathclose%
\pgfusepath{fill}%
\end{pgfscope}%
\begin{pgfscope}%
\pgfpathrectangle{\pgfqpoint{0.680860in}{0.078740in}}{\pgfqpoint{7.842520in}{7.842520in}}%
\pgfusepath{clip}%
\pgfsetbuttcap%
\pgfsetroundjoin%
\definecolor{currentfill}{rgb}{0.235526,0.309527,0.542944}%
\pgfsetfillcolor{currentfill}%
\pgfsetlinewidth{0.000000pt}%
\definecolor{currentstroke}{rgb}{0.253935,0.265254,0.529983}%
\pgfsetstrokecolor{currentstroke}%
\pgfsetdash{}{0pt}%
\pgfpathmoveto{\pgfqpoint{2.910233in}{2.897179in}}%
\pgfpathlineto{\pgfqpoint{2.827135in}{2.685973in}}%
\pgfpathlineto{\pgfqpoint{3.047565in}{2.883264in}}%
\pgfpathclose%
\pgfusepath{fill}%
\end{pgfscope}%
\begin{pgfscope}%
\pgfpathrectangle{\pgfqpoint{0.680860in}{0.078740in}}{\pgfqpoint{7.842520in}{7.842520in}}%
\pgfusepath{clip}%
\pgfsetbuttcap%
\pgfsetroundjoin%
\definecolor{currentfill}{rgb}{0.202219,0.715272,0.476084}%
\pgfsetfillcolor{currentfill}%
\pgfsetlinewidth{0.000000pt}%
\definecolor{currentstroke}{rgb}{0.252194,0.269783,0.531579}%
\pgfsetstrokecolor{currentstroke}%
\pgfsetdash{}{0pt}%
\pgfpathmoveto{\pgfqpoint{7.332134in}{4.648182in}}%
\pgfpathlineto{\pgfqpoint{7.480381in}{4.680637in}}%
\pgfpathlineto{\pgfqpoint{7.544489in}{4.508396in}}%
\pgfpathclose%
\pgfusepath{fill}%
\end{pgfscope}%
\begin{pgfscope}%
\pgfpathrectangle{\pgfqpoint{0.680860in}{0.078740in}}{\pgfqpoint{7.842520in}{7.842520in}}%
\pgfusepath{clip}%
\pgfsetbuttcap%
\pgfsetroundjoin%
\definecolor{currentfill}{rgb}{0.202219,0.715272,0.476084}%
\pgfsetfillcolor{currentfill}%
\pgfsetlinewidth{0.000000pt}%
\definecolor{currentstroke}{rgb}{0.250425,0.274290,0.533103}%
\pgfsetstrokecolor{currentstroke}%
\pgfsetdash{}{0pt}%
\pgfpathmoveto{\pgfqpoint{5.213191in}{4.599610in}}%
\pgfpathlineto{\pgfqpoint{5.292761in}{4.635734in}}%
\pgfpathlineto{\pgfqpoint{5.071052in}{4.569641in}}%
\pgfpathclose%
\pgfusepath{fill}%
\end{pgfscope}%
\begin{pgfscope}%
\pgfpathrectangle{\pgfqpoint{0.680860in}{0.078740in}}{\pgfqpoint{7.842520in}{7.842520in}}%
\pgfusepath{clip}%
\pgfsetbuttcap%
\pgfsetroundjoin%
\definecolor{currentfill}{rgb}{0.140536,0.530132,0.555659}%
\pgfsetfillcolor{currentfill}%
\pgfsetlinewidth{0.000000pt}%
\definecolor{currentstroke}{rgb}{0.248629,0.278775,0.534556}%
\pgfsetstrokecolor{currentstroke}%
\pgfsetdash{}{0pt}%
\pgfpathmoveto{\pgfqpoint{3.877650in}{3.821131in}}%
\pgfpathlineto{\pgfqpoint{3.738866in}{3.811488in}}%
\pgfpathlineto{\pgfqpoint{3.655900in}{3.648583in}}%
\pgfpathclose%
\pgfusepath{fill}%
\end{pgfscope}%
\begin{pgfscope}%
\pgfpathrectangle{\pgfqpoint{0.680860in}{0.078740in}}{\pgfqpoint{7.842520in}{7.842520in}}%
\pgfusepath{clip}%
\pgfsetbuttcap%
\pgfsetroundjoin%
\definecolor{currentfill}{rgb}{0.197636,0.391528,0.554969}%
\pgfsetfillcolor{currentfill}%
\pgfsetlinewidth{0.000000pt}%
\definecolor{currentstroke}{rgb}{0.246811,0.283237,0.535941}%
\pgfsetstrokecolor{currentstroke}%
\pgfsetdash{}{0pt}%
\pgfpathmoveto{\pgfqpoint{3.213753in}{3.285681in}}%
\pgfpathlineto{\pgfqpoint{3.130639in}{3.089850in}}%
\pgfpathlineto{\pgfqpoint{3.268464in}{3.081464in}}%
\pgfpathclose%
\pgfusepath{fill}%
\end{pgfscope}%
\begin{pgfscope}%
\pgfpathrectangle{\pgfqpoint{0.680860in}{0.078740in}}{\pgfqpoint{7.842520in}{7.842520in}}%
\pgfusepath{clip}%
\pgfsetbuttcap%
\pgfsetroundjoin%
\definecolor{currentfill}{rgb}{0.259857,0.745492,0.444467}%
\pgfsetfillcolor{currentfill}%
\pgfsetlinewidth{0.000000pt}%
\definecolor{currentstroke}{rgb}{0.244972,0.287675,0.537260}%
\pgfsetstrokecolor{currentstroke}%
\pgfsetdash{}{0pt}%
\pgfpathmoveto{\pgfqpoint{6.387367in}{4.793337in}}%
\pgfpathlineto{\pgfqpoint{6.606180in}{4.755788in}}%
\pgfpathlineto{\pgfqpoint{6.460572in}{4.722740in}}%
\pgfpathclose%
\pgfusepath{fill}%
\end{pgfscope}%
\begin{pgfscope}%
\pgfpathrectangle{\pgfqpoint{0.680860in}{0.078740in}}{\pgfqpoint{7.842520in}{7.842520in}}%
\pgfusepath{clip}%
\pgfsetbuttcap%
\pgfsetroundjoin%
\definecolor{currentfill}{rgb}{0.121831,0.589055,0.545623}%
\pgfsetfillcolor{currentfill}%
\pgfsetlinewidth{0.000000pt}%
\definecolor{currentstroke}{rgb}{0.243113,0.292092,0.538516}%
\pgfsetstrokecolor{currentstroke}%
\pgfsetdash{}{0pt}%
\pgfpathmoveto{\pgfqpoint{4.182352in}{4.119771in}}%
\pgfpathlineto{\pgfqpoint{3.960448in}{3.970858in}}%
\pgfpathlineto{\pgfqpoint{4.099798in}{3.985000in}}%
\pgfpathclose%
\pgfusepath{fill}%
\end{pgfscope}%
\begin{pgfscope}%
\pgfpathrectangle{\pgfqpoint{0.680860in}{0.078740in}}{\pgfqpoint{7.842520in}{7.842520in}}%
\pgfusepath{clip}%
\pgfsetbuttcap%
\pgfsetroundjoin%
\definecolor{currentfill}{rgb}{0.232815,0.732247,0.459277}%
\pgfsetfillcolor{currentfill}%
\pgfsetlinewidth{0.000000pt}%
\definecolor{currentstroke}{rgb}{0.241237,0.296485,0.539709}%
\pgfsetstrokecolor{currentstroke}%
\pgfsetdash{}{0pt}%
\pgfpathmoveto{\pgfqpoint{7.332134in}{4.648182in}}%
\pgfpathlineto{\pgfqpoint{7.184740in}{4.618099in}}%
\pgfpathlineto{\pgfqpoint{7.117183in}{4.755126in}}%
\pgfpathclose%
\pgfusepath{fill}%
\end{pgfscope}%
\begin{pgfscope}%
\pgfpathrectangle{\pgfqpoint{0.680860in}{0.078740in}}{\pgfqpoint{7.842520in}{7.842520in}}%
\pgfusepath{clip}%
\pgfsetbuttcap%
\pgfsetroundjoin%
\definecolor{currentfill}{rgb}{0.274952,0.037752,0.364543}%
\pgfsetfillcolor{currentfill}%
\pgfsetlinewidth{0.000000pt}%
\definecolor{currentstroke}{rgb}{0.239346,0.300855,0.540844}%
\pgfsetstrokecolor{currentstroke}%
\pgfsetdash{}{0pt}%
\pgfpathmoveto{\pgfqpoint{2.220933in}{1.857373in}}%
\pgfpathlineto{\pgfqpoint{2.357708in}{1.817615in}}%
\pgfpathlineto{\pgfqpoint{2.440870in}{2.045935in}}%
\pgfpathclose%
\pgfusepath{fill}%
\end{pgfscope}%
\begin{pgfscope}%
\pgfpathrectangle{\pgfqpoint{0.680860in}{0.078740in}}{\pgfqpoint{7.842520in}{7.842520in}}%
\pgfusepath{clip}%
\pgfsetbuttcap%
\pgfsetroundjoin%
\definecolor{currentfill}{rgb}{0.266941,0.748751,0.440573}%
\pgfsetfillcolor{currentfill}%
\pgfsetlinewidth{0.000000pt}%
\definecolor{currentstroke}{rgb}{0.237441,0.305202,0.541921}%
\pgfsetstrokecolor{currentstroke}%
\pgfsetdash{}{0pt}%
\pgfpathmoveto{\pgfqpoint{6.241987in}{4.758070in}}%
\pgfpathlineto{\pgfqpoint{6.097424in}{4.725276in}}%
\pgfpathlineto{\pgfqpoint{6.167049in}{4.801858in}}%
\pgfpathclose%
\pgfusepath{fill}%
\end{pgfscope}%
\begin{pgfscope}%
\pgfpathrectangle{\pgfqpoint{0.680860in}{0.078740in}}{\pgfqpoint{7.842520in}{7.842520in}}%
\pgfusepath{clip}%
\pgfsetbuttcap%
\pgfsetroundjoin%
\definecolor{currentfill}{rgb}{0.259857,0.745492,0.444467}%
\pgfsetfillcolor{currentfill}%
\pgfsetlinewidth{0.000000pt}%
\definecolor{currentstroke}{rgb}{0.235526,0.309527,0.542944}%
\pgfsetstrokecolor{currentstroke}%
\pgfsetdash{}{0pt}%
\pgfpathmoveto{\pgfqpoint{6.823210in}{4.688299in}}%
\pgfpathlineto{\pgfqpoint{6.606180in}{4.755788in}}%
\pgfpathlineto{\pgfqpoint{6.752625in}{4.791317in}}%
\pgfpathclose%
\pgfusepath{fill}%
\end{pgfscope}%
\begin{pgfscope}%
\pgfpathrectangle{\pgfqpoint{0.680860in}{0.078740in}}{\pgfqpoint{7.842520in}{7.842520in}}%
\pgfusepath{clip}%
\pgfsetbuttcap%
\pgfsetroundjoin%
\definecolor{currentfill}{rgb}{0.188923,0.410910,0.556326}%
\pgfsetfillcolor{currentfill}%
\pgfsetlinewidth{0.000000pt}%
\definecolor{currentstroke}{rgb}{0.233603,0.313828,0.543914}%
\pgfsetstrokecolor{currentstroke}%
\pgfsetdash{}{0pt}%
\pgfpathmoveto{\pgfqpoint{3.268464in}{3.081464in}}%
\pgfpathlineto{\pgfqpoint{3.351519in}{3.281632in}}%
\pgfpathlineto{\pgfqpoint{3.213753in}{3.285681in}}%
\pgfpathclose%
\pgfusepath{fill}%
\end{pgfscope}%
\begin{pgfscope}%
\pgfpathrectangle{\pgfqpoint{0.680860in}{0.078740in}}{\pgfqpoint{7.842520in}{7.842520in}}%
\pgfusepath{clip}%
\pgfsetbuttcap%
\pgfsetroundjoin%
\definecolor{currentfill}{rgb}{0.180653,0.701402,0.488189}%
\pgfsetfillcolor{currentfill}%
\pgfsetlinewidth{0.000000pt}%
\definecolor{currentstroke}{rgb}{0.231674,0.318106,0.544834}%
\pgfsetstrokecolor{currentstroke}%
\pgfsetdash{}{0pt}%
\pgfpathmoveto{\pgfqpoint{5.071052in}{4.569641in}}%
\pgfpathlineto{\pgfqpoint{4.848986in}{4.483086in}}%
\pgfpathlineto{\pgfqpoint{4.990600in}{4.511026in}}%
\pgfpathclose%
\pgfusepath{fill}%
\end{pgfscope}%
\begin{pgfscope}%
\pgfpathrectangle{\pgfqpoint{0.680860in}{0.078740in}}{\pgfqpoint{7.842520in}{7.842520in}}%
\pgfusepath{clip}%
\pgfsetbuttcap%
\pgfsetroundjoin%
\definecolor{currentfill}{rgb}{0.159194,0.482237,0.558073}%
\pgfsetfillcolor{currentfill}%
\pgfsetlinewidth{0.000000pt}%
\definecolor{currentstroke}{rgb}{0.229739,0.322361,0.545706}%
\pgfsetstrokecolor{currentstroke}%
\pgfsetdash{}{0pt}%
\pgfpathmoveto{\pgfqpoint{3.572890in}{3.470658in}}%
\pgfpathlineto{\pgfqpoint{3.655900in}{3.648583in}}%
\pgfpathlineto{\pgfqpoint{3.434594in}{3.469432in}}%
\pgfpathclose%
\pgfusepath{fill}%
\end{pgfscope}%
\begin{pgfscope}%
\pgfpathrectangle{\pgfqpoint{0.680860in}{0.078740in}}{\pgfqpoint{7.842520in}{7.842520in}}%
\pgfusepath{clip}%
\pgfsetbuttcap%
\pgfsetroundjoin%
\definecolor{currentfill}{rgb}{0.278826,0.175490,0.483397}%
\pgfsetfillcolor{currentfill}%
\pgfsetlinewidth{0.000000pt}%
\definecolor{currentstroke}{rgb}{0.227802,0.326594,0.546532}%
\pgfsetstrokecolor{currentstroke}%
\pgfsetdash{}{0pt}%
\pgfpathmoveto{\pgfqpoint{2.661041in}{2.241649in}}%
\pgfpathlineto{\pgfqpoint{2.744075in}{2.466986in}}%
\pgfpathlineto{\pgfqpoint{2.523996in}{2.270968in}}%
\pgfpathclose%
\pgfusepath{fill}%
\end{pgfscope}%
\begin{pgfscope}%
\pgfpathrectangle{\pgfqpoint{0.680860in}{0.078740in}}{\pgfqpoint{7.842520in}{7.842520in}}%
\pgfusepath{clip}%
\pgfsetbuttcap%
\pgfsetroundjoin%
\definecolor{currentfill}{rgb}{0.120638,0.625828,0.533488}%
\pgfsetfillcolor{currentfill}%
\pgfsetlinewidth{0.000000pt}%
\definecolor{currentstroke}{rgb}{0.225863,0.330805,0.547314}%
\pgfsetstrokecolor{currentstroke}%
\pgfsetdash{}{0pt}%
\pgfpathmoveto{\pgfqpoint{4.322275in}{4.138066in}}%
\pgfpathlineto{\pgfqpoint{4.404488in}{4.256152in}}%
\pgfpathlineto{\pgfqpoint{4.182352in}{4.119771in}}%
\pgfpathclose%
\pgfusepath{fill}%
\end{pgfscope}%
\begin{pgfscope}%
\pgfpathrectangle{\pgfqpoint{0.680860in}{0.078740in}}{\pgfqpoint{7.842520in}{7.842520in}}%
\pgfusepath{clip}%
\pgfsetbuttcap%
\pgfsetroundjoin%
\definecolor{currentfill}{rgb}{0.252899,0.742211,0.448284}%
\pgfsetfillcolor{currentfill}%
\pgfsetlinewidth{0.000000pt}%
\definecolor{currentstroke}{rgb}{0.223925,0.334994,0.548053}%
\pgfsetstrokecolor{currentstroke}%
\pgfsetdash{}{0pt}%
\pgfpathmoveto{\pgfqpoint{5.877741in}{4.731360in}}%
\pgfpathlineto{\pgfqpoint{5.656982in}{4.711478in}}%
\pgfpathlineto{\pgfqpoint{5.800838in}{4.745757in}}%
\pgfpathclose%
\pgfusepath{fill}%
\end{pgfscope}%
\begin{pgfscope}%
\pgfpathrectangle{\pgfqpoint{0.680860in}{0.078740in}}{\pgfqpoint{7.842520in}{7.842520in}}%
\pgfusepath{clip}%
\pgfsetbuttcap%
\pgfsetroundjoin%
\definecolor{currentfill}{rgb}{0.283091,0.110553,0.431554}%
\pgfsetfillcolor{currentfill}%
\pgfsetlinewidth{0.000000pt}%
\definecolor{currentstroke}{rgb}{0.221989,0.339161,0.548752}%
\pgfsetstrokecolor{currentstroke}%
\pgfsetdash{}{0pt}%
\pgfpathmoveto{\pgfqpoint{2.578010in}{2.011439in}}%
\pgfpathlineto{\pgfqpoint{2.523996in}{2.270968in}}%
\pgfpathlineto{\pgfqpoint{2.440870in}{2.045935in}}%
\pgfpathclose%
\pgfusepath{fill}%
\end{pgfscope}%
\begin{pgfscope}%
\pgfpathrectangle{\pgfqpoint{0.680860in}{0.078740in}}{\pgfqpoint{7.842520in}{7.842520in}}%
\pgfusepath{clip}%
\pgfsetbuttcap%
\pgfsetroundjoin%
\definecolor{currentfill}{rgb}{0.153894,0.680203,0.504172}%
\pgfsetfillcolor{currentfill}%
\pgfsetlinewidth{0.000000pt}%
\definecolor{currentstroke}{rgb}{0.220057,0.343307,0.549413}%
\pgfsetstrokecolor{currentstroke}%
\pgfsetdash{}{0pt}%
\pgfpathmoveto{\pgfqpoint{4.626744in}{4.377935in}}%
\pgfpathlineto{\pgfqpoint{4.767809in}{4.403204in}}%
\pgfpathlineto{\pgfqpoint{4.848986in}{4.483086in}}%
\pgfpathclose%
\pgfusepath{fill}%
\end{pgfscope}%
\begin{pgfscope}%
\pgfpathrectangle{\pgfqpoint{0.680860in}{0.078740in}}{\pgfqpoint{7.842520in}{7.842520in}}%
\pgfusepath{clip}%
\pgfsetbuttcap%
\pgfsetroundjoin%
\definecolor{currentfill}{rgb}{0.132268,0.655014,0.519661}%
\pgfsetfillcolor{currentfill}%
\pgfsetlinewidth{0.000000pt}%
\definecolor{currentstroke}{rgb}{0.218130,0.347432,0.550038}%
\pgfsetstrokecolor{currentstroke}%
\pgfsetdash{}{0pt}%
\pgfpathmoveto{\pgfqpoint{4.544985in}{4.278182in}}%
\pgfpathlineto{\pgfqpoint{4.626744in}{4.377935in}}%
\pgfpathlineto{\pgfqpoint{4.404488in}{4.256152in}}%
\pgfpathclose%
\pgfusepath{fill}%
\end{pgfscope}%
\begin{pgfscope}%
\pgfpathrectangle{\pgfqpoint{0.680860in}{0.078740in}}{\pgfqpoint{7.842520in}{7.842520in}}%
\pgfusepath{clip}%
\pgfsetbuttcap%
\pgfsetroundjoin%
\definecolor{currentfill}{rgb}{0.241237,0.296485,0.539709}%
\pgfsetfillcolor{currentfill}%
\pgfsetlinewidth{0.000000pt}%
\definecolor{currentstroke}{rgb}{0.216210,0.351535,0.550627}%
\pgfsetstrokecolor{currentstroke}%
\pgfsetdash{}{0pt}%
\pgfpathmoveto{\pgfqpoint{3.047565in}{2.883264in}}%
\pgfpathlineto{\pgfqpoint{2.827135in}{2.685973in}}%
\pgfpathlineto{\pgfqpoint{2.964535in}{2.667253in}}%
\pgfpathclose%
\pgfusepath{fill}%
\end{pgfscope}%
\begin{pgfscope}%
\pgfpathrectangle{\pgfqpoint{0.680860in}{0.078740in}}{\pgfqpoint{7.842520in}{7.842520in}}%
\pgfusepath{clip}%
\pgfsetbuttcap%
\pgfsetroundjoin%
\definecolor{currentfill}{rgb}{0.210503,0.363727,0.552206}%
\pgfsetfillcolor{currentfill}%
\pgfsetlinewidth{0.000000pt}%
\definecolor{currentstroke}{rgb}{0.214298,0.355619,0.551184}%
\pgfsetstrokecolor{currentstroke}%
\pgfsetdash{}{0pt}%
\pgfpathmoveto{\pgfqpoint{3.130639in}{3.089850in}}%
\pgfpathlineto{\pgfqpoint{3.047565in}{2.883264in}}%
\pgfpathlineto{\pgfqpoint{3.268464in}{3.081464in}}%
\pgfpathclose%
\pgfusepath{fill}%
\end{pgfscope}%
\begin{pgfscope}%
\pgfpathrectangle{\pgfqpoint{0.680860in}{0.078740in}}{\pgfqpoint{7.842520in}{7.842520in}}%
\pgfusepath{clip}%
\pgfsetbuttcap%
\pgfsetroundjoin%
\definecolor{currentfill}{rgb}{0.274149,0.751988,0.436601}%
\pgfsetfillcolor{currentfill}%
\pgfsetlinewidth{0.000000pt}%
\definecolor{currentstroke}{rgb}{0.212395,0.359683,0.551710}%
\pgfsetstrokecolor{currentstroke}%
\pgfsetdash{}{0pt}%
\pgfpathmoveto{\pgfqpoint{6.097424in}{4.725276in}}%
\pgfpathlineto{\pgfqpoint{6.021988in}{4.765349in}}%
\pgfpathlineto{\pgfqpoint{6.167049in}{4.801858in}}%
\pgfpathclose%
\pgfusepath{fill}%
\end{pgfscope}%
\begin{pgfscope}%
\pgfpathrectangle{\pgfqpoint{0.680860in}{0.078740in}}{\pgfqpoint{7.842520in}{7.842520in}}%
\pgfusepath{clip}%
\pgfsetbuttcap%
\pgfsetroundjoin%
\definecolor{currentfill}{rgb}{0.263663,0.237631,0.518762}%
\pgfsetfillcolor{currentfill}%
\pgfsetlinewidth{0.000000pt}%
\definecolor{currentstroke}{rgb}{0.210503,0.363727,0.552206}%
\pgfsetstrokecolor{currentstroke}%
\pgfsetdash{}{0pt}%
\pgfpathmoveto{\pgfqpoint{2.744075in}{2.466986in}}%
\pgfpathlineto{\pgfqpoint{2.881546in}{2.443230in}}%
\pgfpathlineto{\pgfqpoint{2.827135in}{2.685973in}}%
\pgfpathclose%
\pgfusepath{fill}%
\end{pgfscope}%
\begin{pgfscope}%
\pgfpathrectangle{\pgfqpoint{0.680860in}{0.078740in}}{\pgfqpoint{7.842520in}{7.842520in}}%
\pgfusepath{clip}%
\pgfsetbuttcap%
\pgfsetroundjoin%
\definecolor{currentfill}{rgb}{0.282884,0.135920,0.453427}%
\pgfsetfillcolor{currentfill}%
\pgfsetlinewidth{0.000000pt}%
\definecolor{currentstroke}{rgb}{0.208623,0.367752,0.552675}%
\pgfsetstrokecolor{currentstroke}%
\pgfsetdash{}{0pt}%
\pgfpathmoveto{\pgfqpoint{2.661041in}{2.241649in}}%
\pgfpathlineto{\pgfqpoint{2.523996in}{2.270968in}}%
\pgfpathlineto{\pgfqpoint{2.578010in}{2.011439in}}%
\pgfpathclose%
\pgfusepath{fill}%
\end{pgfscope}%
\begin{pgfscope}%
\pgfpathrectangle{\pgfqpoint{0.680860in}{0.078740in}}{\pgfqpoint{7.842520in}{7.842520in}}%
\pgfusepath{clip}%
\pgfsetbuttcap%
\pgfsetroundjoin%
\definecolor{currentfill}{rgb}{0.175841,0.441290,0.557685}%
\pgfsetfillcolor{currentfill}%
\pgfsetlinewidth{0.000000pt}%
\definecolor{currentstroke}{rgb}{0.206756,0.371758,0.553117}%
\pgfsetstrokecolor{currentstroke}%
\pgfsetdash{}{0pt}%
\pgfpathmoveto{\pgfqpoint{3.351519in}{3.281632in}}%
\pgfpathlineto{\pgfqpoint{3.489868in}{3.278771in}}%
\pgfpathlineto{\pgfqpoint{3.434594in}{3.469432in}}%
\pgfpathclose%
\pgfusepath{fill}%
\end{pgfscope}%
\begin{pgfscope}%
\pgfpathrectangle{\pgfqpoint{0.680860in}{0.078740in}}{\pgfqpoint{7.842520in}{7.842520in}}%
\pgfusepath{clip}%
\pgfsetbuttcap%
\pgfsetroundjoin%
\definecolor{currentfill}{rgb}{0.279566,0.067836,0.391917}%
\pgfsetfillcolor{currentfill}%
\pgfsetlinewidth{0.000000pt}%
\definecolor{currentstroke}{rgb}{0.204903,0.375746,0.553533}%
\pgfsetstrokecolor{currentstroke}%
\pgfsetdash{}{0pt}%
\pgfpathmoveto{\pgfqpoint{2.440870in}{2.045935in}}%
\pgfpathlineto{\pgfqpoint{2.357708in}{1.817615in}}%
\pgfpathlineto{\pgfqpoint{2.578010in}{2.011439in}}%
\pgfpathclose%
\pgfusepath{fill}%
\end{pgfscope}%
\begin{pgfscope}%
\pgfpathrectangle{\pgfqpoint{0.680860in}{0.078740in}}{\pgfqpoint{7.842520in}{7.842520in}}%
\pgfusepath{clip}%
\pgfsetbuttcap%
\pgfsetroundjoin%
\definecolor{currentfill}{rgb}{0.128729,0.563265,0.551229}%
\pgfsetfillcolor{currentfill}%
\pgfsetlinewidth{0.000000pt}%
\definecolor{currentstroke}{rgb}{0.203063,0.379716,0.553925}%
\pgfsetstrokecolor{currentstroke}%
\pgfsetdash{}{0pt}%
\pgfpathmoveto{\pgfqpoint{4.017076in}{3.832432in}}%
\pgfpathlineto{\pgfqpoint{3.960448in}{3.970858in}}%
\pgfpathlineto{\pgfqpoint{3.877650in}{3.821131in}}%
\pgfpathclose%
\pgfusepath{fill}%
\end{pgfscope}%
\begin{pgfscope}%
\pgfpathrectangle{\pgfqpoint{0.680860in}{0.078740in}}{\pgfqpoint{7.842520in}{7.842520in}}%
\pgfusepath{clip}%
\pgfsetbuttcap%
\pgfsetroundjoin%
\definecolor{currentfill}{rgb}{0.141935,0.526453,0.555991}%
\pgfsetfillcolor{currentfill}%
\pgfsetlinewidth{0.000000pt}%
\definecolor{currentstroke}{rgb}{0.201239,0.383670,0.554294}%
\pgfsetstrokecolor{currentstroke}%
\pgfsetdash{}{0pt}%
\pgfpathmoveto{\pgfqpoint{3.655900in}{3.648583in}}%
\pgfpathlineto{\pgfqpoint{3.794751in}{3.654950in}}%
\pgfpathlineto{\pgfqpoint{3.877650in}{3.821131in}}%
\pgfpathclose%
\pgfusepath{fill}%
\end{pgfscope}%
\begin{pgfscope}%
\pgfpathrectangle{\pgfqpoint{0.680860in}{0.078740in}}{\pgfqpoint{7.842520in}{7.842520in}}%
\pgfusepath{clip}%
\pgfsetbuttcap%
\pgfsetroundjoin%
\definecolor{currentfill}{rgb}{0.255645,0.260703,0.528312}%
\pgfsetfillcolor{currentfill}%
\pgfsetlinewidth{0.000000pt}%
\definecolor{currentstroke}{rgb}{0.199430,0.387607,0.554642}%
\pgfsetstrokecolor{currentstroke}%
\pgfsetdash{}{0pt}%
\pgfpathmoveto{\pgfqpoint{2.827135in}{2.685973in}}%
\pgfpathlineto{\pgfqpoint{2.881546in}{2.443230in}}%
\pgfpathlineto{\pgfqpoint{2.964535in}{2.667253in}}%
\pgfpathclose%
\pgfusepath{fill}%
\end{pgfscope}%
\begin{pgfscope}%
\pgfpathrectangle{\pgfqpoint{0.680860in}{0.078740in}}{\pgfqpoint{7.842520in}{7.842520in}}%
\pgfusepath{clip}%
\pgfsetbuttcap%
\pgfsetroundjoin%
\definecolor{currentfill}{rgb}{0.266941,0.748751,0.440573}%
\pgfsetfillcolor{currentfill}%
\pgfsetlinewidth{0.000000pt}%
\definecolor{currentstroke}{rgb}{0.197636,0.391528,0.554969}%
\pgfsetstrokecolor{currentstroke}%
\pgfsetdash{}{0pt}%
\pgfpathmoveto{\pgfqpoint{6.969772in}{4.720488in}}%
\pgfpathlineto{\pgfqpoint{6.823210in}{4.688299in}}%
\pgfpathlineto{\pgfqpoint{6.899928in}{4.829425in}}%
\pgfpathclose%
\pgfusepath{fill}%
\end{pgfscope}%
\begin{pgfscope}%
\pgfpathrectangle{\pgfqpoint{0.680860in}{0.078740in}}{\pgfqpoint{7.842520in}{7.842520in}}%
\pgfusepath{clip}%
\pgfsetbuttcap%
\pgfsetroundjoin%
\definecolor{currentfill}{rgb}{0.168126,0.459988,0.558082}%
\pgfsetfillcolor{currentfill}%
\pgfsetlinewidth{0.000000pt}%
\definecolor{currentstroke}{rgb}{0.195860,0.395433,0.555276}%
\pgfsetstrokecolor{currentstroke}%
\pgfsetdash{}{0pt}%
\pgfpathmoveto{\pgfqpoint{3.434594in}{3.469432in}}%
\pgfpathlineto{\pgfqpoint{3.489868in}{3.278771in}}%
\pgfpathlineto{\pgfqpoint{3.572890in}{3.470658in}}%
\pgfpathclose%
\pgfusepath{fill}%
\end{pgfscope}%
\begin{pgfscope}%
\pgfpathrectangle{\pgfqpoint{0.680860in}{0.078740in}}{\pgfqpoint{7.842520in}{7.842520in}}%
\pgfusepath{clip}%
\pgfsetbuttcap%
\pgfsetroundjoin%
\definecolor{currentfill}{rgb}{0.232815,0.732247,0.459277}%
\pgfsetfillcolor{currentfill}%
\pgfsetlinewidth{0.000000pt}%
\definecolor{currentstroke}{rgb}{0.194100,0.399323,0.555565}%
\pgfsetstrokecolor{currentstroke}%
\pgfsetdash{}{0pt}%
\pgfpathmoveto{\pgfqpoint{5.292761in}{4.635734in}}%
\pgfpathlineto{\pgfqpoint{5.356094in}{4.631957in}}%
\pgfpathlineto{\pgfqpoint{5.435389in}{4.667024in}}%
\pgfpathclose%
\pgfusepath{fill}%
\end{pgfscope}%
\begin{pgfscope}%
\pgfpathrectangle{\pgfqpoint{0.680860in}{0.078740in}}{\pgfqpoint{7.842520in}{7.842520in}}%
\pgfusepath{clip}%
\pgfsetbuttcap%
\pgfsetroundjoin%
\definecolor{currentfill}{rgb}{0.125394,0.574318,0.549086}%
\pgfsetfillcolor{currentfill}%
\pgfsetlinewidth{0.000000pt}%
\definecolor{currentstroke}{rgb}{0.192357,0.403199,0.555836}%
\pgfsetstrokecolor{currentstroke}%
\pgfsetdash{}{0pt}%
\pgfpathmoveto{\pgfqpoint{4.099798in}{3.985000in}}%
\pgfpathlineto{\pgfqpoint{3.960448in}{3.970858in}}%
\pgfpathlineto{\pgfqpoint{4.017076in}{3.832432in}}%
\pgfpathclose%
\pgfusepath{fill}%
\end{pgfscope}%
\begin{pgfscope}%
\pgfpathrectangle{\pgfqpoint{0.680860in}{0.078740in}}{\pgfqpoint{7.842520in}{7.842520in}}%
\pgfusepath{clip}%
\pgfsetbuttcap%
\pgfsetroundjoin%
\definecolor{currentfill}{rgb}{0.170948,0.694384,0.493803}%
\pgfsetfillcolor{currentfill}%
\pgfsetlinewidth{0.000000pt}%
\definecolor{currentstroke}{rgb}{0.190631,0.407061,0.556089}%
\pgfsetstrokecolor{currentstroke}%
\pgfsetdash{}{0pt}%
\pgfpathmoveto{\pgfqpoint{4.990600in}{4.511026in}}%
\pgfpathlineto{\pgfqpoint{4.848986in}{4.483086in}}%
\pgfpathlineto{\pgfqpoint{4.767809in}{4.403204in}}%
\pgfpathclose%
\pgfusepath{fill}%
\end{pgfscope}%
\begin{pgfscope}%
\pgfpathrectangle{\pgfqpoint{0.680860in}{0.078740in}}{\pgfqpoint{7.842520in}{7.842520in}}%
\pgfusepath{clip}%
\pgfsetbuttcap%
\pgfsetroundjoin%
\definecolor{currentfill}{rgb}{0.220124,0.725509,0.466226}%
\pgfsetfillcolor{currentfill}%
\pgfsetlinewidth{0.000000pt}%
\definecolor{currentstroke}{rgb}{0.188923,0.410910,0.556326}%
\pgfsetstrokecolor{currentstroke}%
\pgfsetdash{}{0pt}%
\pgfpathmoveto{\pgfqpoint{5.292761in}{4.635734in}}%
\pgfpathlineto{\pgfqpoint{5.213191in}{4.599610in}}%
\pgfpathlineto{\pgfqpoint{5.356094in}{4.631957in}}%
\pgfpathclose%
\pgfusepath{fill}%
\end{pgfscope}%
\begin{pgfscope}%
\pgfpathrectangle{\pgfqpoint{0.680860in}{0.078740in}}{\pgfqpoint{7.842520in}{7.842520in}}%
\pgfusepath{clip}%
\pgfsetbuttcap%
\pgfsetroundjoin%
\definecolor{currentfill}{rgb}{0.252899,0.742211,0.448284}%
\pgfsetfillcolor{currentfill}%
\pgfsetlinewidth{0.000000pt}%
\definecolor{currentstroke}{rgb}{0.187231,0.414746,0.556547}%
\pgfsetstrokecolor{currentstroke}%
\pgfsetdash{}{0pt}%
\pgfpathmoveto{\pgfqpoint{5.435389in}{4.667024in}}%
\pgfpathlineto{\pgfqpoint{5.578794in}{4.700737in}}%
\pgfpathlineto{\pgfqpoint{5.656982in}{4.711478in}}%
\pgfpathclose%
\pgfusepath{fill}%
\end{pgfscope}%
\begin{pgfscope}%
\pgfpathrectangle{\pgfqpoint{0.680860in}{0.078740in}}{\pgfqpoint{7.842520in}{7.842520in}}%
\pgfusepath{clip}%
\pgfsetbuttcap%
\pgfsetroundjoin%
\definecolor{currentfill}{rgb}{0.278012,0.180367,0.486697}%
\pgfsetfillcolor{currentfill}%
\pgfsetlinewidth{0.000000pt}%
\definecolor{currentstroke}{rgb}{0.185556,0.418570,0.556753}%
\pgfsetstrokecolor{currentstroke}%
\pgfsetdash{}{0pt}%
\pgfpathmoveto{\pgfqpoint{2.798587in}{2.212678in}}%
\pgfpathlineto{\pgfqpoint{2.744075in}{2.466986in}}%
\pgfpathlineto{\pgfqpoint{2.661041in}{2.241649in}}%
\pgfpathclose%
\pgfusepath{fill}%
\end{pgfscope}%
\begin{pgfscope}%
\pgfpathrectangle{\pgfqpoint{0.680860in}{0.078740in}}{\pgfqpoint{7.842520in}{7.842520in}}%
\pgfusepath{clip}%
\pgfsetbuttcap%
\pgfsetroundjoin%
\definecolor{currentfill}{rgb}{0.120092,0.600104,0.542530}%
\pgfsetfillcolor{currentfill}%
\pgfsetlinewidth{0.000000pt}%
\definecolor{currentstroke}{rgb}{0.183898,0.422383,0.556944}%
\pgfsetstrokecolor{currentstroke}%
\pgfsetdash{}{0pt}%
\pgfpathmoveto{\pgfqpoint{4.099798in}{3.985000in}}%
\pgfpathlineto{\pgfqpoint{4.239812in}{4.000958in}}%
\pgfpathlineto{\pgfqpoint{4.182352in}{4.119771in}}%
\pgfpathclose%
\pgfusepath{fill}%
\end{pgfscope}%
\begin{pgfscope}%
\pgfpathrectangle{\pgfqpoint{0.680860in}{0.078740in}}{\pgfqpoint{7.842520in}{7.842520in}}%
\pgfusepath{clip}%
\pgfsetbuttcap%
\pgfsetroundjoin%
\definecolor{currentfill}{rgb}{0.216210,0.351535,0.550627}%
\pgfsetfillcolor{currentfill}%
\pgfsetlinewidth{0.000000pt}%
\definecolor{currentstroke}{rgb}{0.182256,0.426184,0.557120}%
\pgfsetstrokecolor{currentstroke}%
\pgfsetdash{}{0pt}%
\pgfpathmoveto{\pgfqpoint{3.268464in}{3.081464in}}%
\pgfpathlineto{\pgfqpoint{3.047565in}{2.883264in}}%
\pgfpathlineto{\pgfqpoint{3.185445in}{2.870203in}}%
\pgfpathclose%
\pgfusepath{fill}%
\end{pgfscope}%
\begin{pgfscope}%
\pgfpathrectangle{\pgfqpoint{0.680860in}{0.078740in}}{\pgfqpoint{7.842520in}{7.842520in}}%
\pgfusepath{clip}%
\pgfsetbuttcap%
\pgfsetroundjoin%
\definecolor{currentfill}{rgb}{0.185556,0.418570,0.556753}%
\pgfsetfillcolor{currentfill}%
\pgfsetlinewidth{0.000000pt}%
\definecolor{currentstroke}{rgb}{0.180629,0.429975,0.557282}%
\pgfsetstrokecolor{currentstroke}%
\pgfsetdash{}{0pt}%
\pgfpathmoveto{\pgfqpoint{3.268464in}{3.081464in}}%
\pgfpathlineto{\pgfqpoint{3.489868in}{3.278771in}}%
\pgfpathlineto{\pgfqpoint{3.351519in}{3.281632in}}%
\pgfpathclose%
\pgfusepath{fill}%
\end{pgfscope}%
\begin{pgfscope}%
\pgfpathrectangle{\pgfqpoint{0.680860in}{0.078740in}}{\pgfqpoint{7.842520in}{7.842520in}}%
\pgfusepath{clip}%
\pgfsetbuttcap%
\pgfsetroundjoin%
\definecolor{currentfill}{rgb}{0.119483,0.614817,0.537692}%
\pgfsetfillcolor{currentfill}%
\pgfsetlinewidth{0.000000pt}%
\definecolor{currentstroke}{rgb}{0.179019,0.433756,0.557430}%
\pgfsetstrokecolor{currentstroke}%
\pgfsetdash{}{0pt}%
\pgfpathmoveto{\pgfqpoint{4.182352in}{4.119771in}}%
\pgfpathlineto{\pgfqpoint{4.239812in}{4.000958in}}%
\pgfpathlineto{\pgfqpoint{4.322275in}{4.138066in}}%
\pgfpathclose%
\pgfusepath{fill}%
\end{pgfscope}%
\begin{pgfscope}%
\pgfpathrectangle{\pgfqpoint{0.680860in}{0.078740in}}{\pgfqpoint{7.842520in}{7.842520in}}%
\pgfusepath{clip}%
\pgfsetbuttcap%
\pgfsetroundjoin%
\definecolor{currentfill}{rgb}{0.281477,0.755203,0.432552}%
\pgfsetfillcolor{currentfill}%
\pgfsetlinewidth{0.000000pt}%
\definecolor{currentstroke}{rgb}{0.177423,0.437527,0.557565}%
\pgfsetstrokecolor{currentstroke}%
\pgfsetdash{}{0pt}%
\pgfpathmoveto{\pgfqpoint{5.945504in}{4.782570in}}%
\pgfpathlineto{\pgfqpoint{6.021988in}{4.765349in}}%
\pgfpathlineto{\pgfqpoint{5.877741in}{4.731360in}}%
\pgfpathclose%
\pgfusepath{fill}%
\end{pgfscope}%
\begin{pgfscope}%
\pgfpathrectangle{\pgfqpoint{0.680860in}{0.078740in}}{\pgfqpoint{7.842520in}{7.842520in}}%
\pgfusepath{clip}%
\pgfsetbuttcap%
\pgfsetroundjoin%
\definecolor{currentfill}{rgb}{0.276022,0.044167,0.370164}%
\pgfsetfillcolor{currentfill}%
\pgfsetlinewidth{0.000000pt}%
\definecolor{currentstroke}{rgb}{0.175841,0.441290,0.557685}%
\pgfsetstrokecolor{currentstroke}%
\pgfsetdash{}{0pt}%
\pgfpathmoveto{\pgfqpoint{2.357708in}{1.817615in}}%
\pgfpathlineto{\pgfqpoint{2.494952in}{1.777867in}}%
\pgfpathlineto{\pgfqpoint{2.578010in}{2.011439in}}%
\pgfpathclose%
\pgfusepath{fill}%
\end{pgfscope}%
\begin{pgfscope}%
\pgfpathrectangle{\pgfqpoint{0.680860in}{0.078740in}}{\pgfqpoint{7.842520in}{7.842520in}}%
\pgfusepath{clip}%
\pgfsetbuttcap%
\pgfsetroundjoin%
\definecolor{currentfill}{rgb}{0.156270,0.489624,0.557936}%
\pgfsetfillcolor{currentfill}%
\pgfsetlinewidth{0.000000pt}%
\definecolor{currentstroke}{rgb}{0.174274,0.445044,0.557792}%
\pgfsetstrokecolor{currentstroke}%
\pgfsetdash{}{0pt}%
\pgfpathmoveto{\pgfqpoint{3.572890in}{3.470658in}}%
\pgfpathlineto{\pgfqpoint{3.711794in}{3.473254in}}%
\pgfpathlineto{\pgfqpoint{3.655900in}{3.648583in}}%
\pgfpathclose%
\pgfusepath{fill}%
\end{pgfscope}%
\begin{pgfscope}%
\pgfpathrectangle{\pgfqpoint{0.680860in}{0.078740in}}{\pgfqpoint{7.842520in}{7.842520in}}%
\pgfusepath{clip}%
\pgfsetbuttcap%
\pgfsetroundjoin%
\definecolor{currentfill}{rgb}{0.208030,0.718701,0.472873}%
\pgfsetfillcolor{currentfill}%
\pgfsetlinewidth{0.000000pt}%
\definecolor{currentstroke}{rgb}{0.172719,0.448791,0.557885}%
\pgfsetstrokecolor{currentstroke}%
\pgfsetdash{}{0pt}%
\pgfpathmoveto{\pgfqpoint{5.132963in}{4.541273in}}%
\pgfpathlineto{\pgfqpoint{5.213191in}{4.599610in}}%
\pgfpathlineto{\pgfqpoint{5.071052in}{4.569641in}}%
\pgfpathclose%
\pgfusepath{fill}%
\end{pgfscope}%
\begin{pgfscope}%
\pgfpathrectangle{\pgfqpoint{0.680860in}{0.078740in}}{\pgfqpoint{7.842520in}{7.842520in}}%
\pgfusepath{clip}%
\pgfsetbuttcap%
\pgfsetroundjoin%
\definecolor{currentfill}{rgb}{0.239346,0.300855,0.540844}%
\pgfsetfillcolor{currentfill}%
\pgfsetlinewidth{0.000000pt}%
\definecolor{currentstroke}{rgb}{0.171176,0.452530,0.557965}%
\pgfsetstrokecolor{currentstroke}%
\pgfsetdash{}{0pt}%
\pgfpathmoveto{\pgfqpoint{2.964535in}{2.667253in}}%
\pgfpathlineto{\pgfqpoint{3.102470in}{2.649229in}}%
\pgfpathlineto{\pgfqpoint{3.047565in}{2.883264in}}%
\pgfpathclose%
\pgfusepath{fill}%
\end{pgfscope}%
\begin{pgfscope}%
\pgfpathrectangle{\pgfqpoint{0.680860in}{0.078740in}}{\pgfqpoint{7.842520in}{7.842520in}}%
\pgfusepath{clip}%
\pgfsetbuttcap%
\pgfsetroundjoin%
\definecolor{currentfill}{rgb}{0.274128,0.199721,0.498911}%
\pgfsetfillcolor{currentfill}%
\pgfsetlinewidth{0.000000pt}%
\definecolor{currentstroke}{rgb}{0.169646,0.456262,0.558030}%
\pgfsetstrokecolor{currentstroke}%
\pgfsetdash{}{0pt}%
\pgfpathmoveto{\pgfqpoint{2.798587in}{2.212678in}}%
\pgfpathlineto{\pgfqpoint{2.881546in}{2.443230in}}%
\pgfpathlineto{\pgfqpoint{2.744075in}{2.466986in}}%
\pgfpathclose%
\pgfusepath{fill}%
\end{pgfscope}%
\begin{pgfscope}%
\pgfpathrectangle{\pgfqpoint{0.680860in}{0.078740in}}{\pgfqpoint{7.842520in}{7.842520in}}%
\pgfusepath{clip}%
\pgfsetbuttcap%
\pgfsetroundjoin%
\definecolor{currentfill}{rgb}{0.296479,0.761561,0.424223}%
\pgfsetfillcolor{currentfill}%
\pgfsetlinewidth{0.000000pt}%
\definecolor{currentstroke}{rgb}{0.168126,0.459988,0.558082}%
\pgfsetstrokecolor{currentstroke}%
\pgfsetdash{}{0pt}%
\pgfpathmoveto{\pgfqpoint{6.241987in}{4.758070in}}%
\pgfpathlineto{\pgfqpoint{6.167049in}{4.801858in}}%
\pgfpathlineto{\pgfqpoint{6.387367in}{4.793337in}}%
\pgfpathclose%
\pgfusepath{fill}%
\end{pgfscope}%
\begin{pgfscope}%
\pgfpathrectangle{\pgfqpoint{0.680860in}{0.078740in}}{\pgfqpoint{7.842520in}{7.842520in}}%
\pgfusepath{clip}%
\pgfsetbuttcap%
\pgfsetroundjoin%
\definecolor{currentfill}{rgb}{0.196571,0.711827,0.479221}%
\pgfsetfillcolor{currentfill}%
\pgfsetlinewidth{0.000000pt}%
\definecolor{currentstroke}{rgb}{0.166617,0.463708,0.558119}%
\pgfsetstrokecolor{currentstroke}%
\pgfsetdash{}{0pt}%
\pgfpathmoveto{\pgfqpoint{4.990600in}{4.511026in}}%
\pgfpathlineto{\pgfqpoint{5.132963in}{4.541273in}}%
\pgfpathlineto{\pgfqpoint{5.071052in}{4.569641in}}%
\pgfpathclose%
\pgfusepath{fill}%
\end{pgfscope}%
\begin{pgfscope}%
\pgfpathrectangle{\pgfqpoint{0.680860in}{0.078740in}}{\pgfqpoint{7.842520in}{7.842520in}}%
\pgfusepath{clip}%
\pgfsetbuttcap%
\pgfsetroundjoin%
\definecolor{currentfill}{rgb}{0.123444,0.636809,0.528763}%
\pgfsetfillcolor{currentfill}%
\pgfsetlinewidth{0.000000pt}%
\definecolor{currentstroke}{rgb}{0.165117,0.467423,0.558141}%
\pgfsetstrokecolor{currentstroke}%
\pgfsetdash{}{0pt}%
\pgfpathmoveto{\pgfqpoint{4.462885in}{4.158325in}}%
\pgfpathlineto{\pgfqpoint{4.404488in}{4.256152in}}%
\pgfpathlineto{\pgfqpoint{4.322275in}{4.138066in}}%
\pgfpathclose%
\pgfusepath{fill}%
\end{pgfscope}%
\begin{pgfscope}%
\pgfpathrectangle{\pgfqpoint{0.680860in}{0.078740in}}{\pgfqpoint{7.842520in}{7.842520in}}%
\pgfusepath{clip}%
\pgfsetbuttcap%
\pgfsetroundjoin%
\definecolor{currentfill}{rgb}{0.288921,0.758394,0.428426}%
\pgfsetfillcolor{currentfill}%
\pgfsetlinewidth{0.000000pt}%
\definecolor{currentstroke}{rgb}{0.163625,0.471133,0.558148}%
\pgfsetstrokecolor{currentstroke}%
\pgfsetdash{}{0pt}%
\pgfpathmoveto{\pgfqpoint{6.899928in}{4.829425in}}%
\pgfpathlineto{\pgfqpoint{6.823210in}{4.688299in}}%
\pgfpathlineto{\pgfqpoint{6.752625in}{4.791317in}}%
\pgfpathclose%
\pgfusepath{fill}%
\end{pgfscope}%
\begin{pgfscope}%
\pgfpathrectangle{\pgfqpoint{0.680860in}{0.078740in}}{\pgfqpoint{7.842520in}{7.842520in}}%
\pgfusepath{clip}%
\pgfsetbuttcap%
\pgfsetroundjoin%
\definecolor{currentfill}{rgb}{0.282884,0.135920,0.453427}%
\pgfsetfillcolor{currentfill}%
\pgfsetlinewidth{0.000000pt}%
\definecolor{currentstroke}{rgb}{0.162142,0.474838,0.558140}%
\pgfsetstrokecolor{currentstroke}%
\pgfsetdash{}{0pt}%
\pgfpathmoveto{\pgfqpoint{2.661041in}{2.241649in}}%
\pgfpathlineto{\pgfqpoint{2.578010in}{2.011439in}}%
\pgfpathlineto{\pgfqpoint{2.798587in}{2.212678in}}%
\pgfpathclose%
\pgfusepath{fill}%
\end{pgfscope}%
\begin{pgfscope}%
\pgfpathrectangle{\pgfqpoint{0.680860in}{0.078740in}}{\pgfqpoint{7.842520in}{7.842520in}}%
\pgfusepath{clip}%
\pgfsetbuttcap%
\pgfsetroundjoin%
\definecolor{currentfill}{rgb}{0.128087,0.647749,0.523491}%
\pgfsetfillcolor{currentfill}%
\pgfsetlinewidth{0.000000pt}%
\definecolor{currentstroke}{rgb}{0.160665,0.478540,0.558115}%
\pgfsetstrokecolor{currentstroke}%
\pgfsetdash{}{0pt}%
\pgfpathmoveto{\pgfqpoint{4.544985in}{4.278182in}}%
\pgfpathlineto{\pgfqpoint{4.404488in}{4.256152in}}%
\pgfpathlineto{\pgfqpoint{4.462885in}{4.158325in}}%
\pgfpathclose%
\pgfusepath{fill}%
\end{pgfscope}%
\begin{pgfscope}%
\pgfpathrectangle{\pgfqpoint{0.680860in}{0.078740in}}{\pgfqpoint{7.842520in}{7.842520in}}%
\pgfusepath{clip}%
\pgfsetbuttcap%
\pgfsetroundjoin%
\definecolor{currentfill}{rgb}{0.150476,0.504369,0.557430}%
\pgfsetfillcolor{currentfill}%
\pgfsetlinewidth{0.000000pt}%
\definecolor{currentstroke}{rgb}{0.159194,0.482237,0.558073}%
\pgfsetstrokecolor{currentstroke}%
\pgfsetdash{}{0pt}%
\pgfpathmoveto{\pgfqpoint{3.655900in}{3.648583in}}%
\pgfpathlineto{\pgfqpoint{3.711794in}{3.473254in}}%
\pgfpathlineto{\pgfqpoint{3.794751in}{3.654950in}}%
\pgfpathclose%
\pgfusepath{fill}%
\end{pgfscope}%
\begin{pgfscope}%
\pgfpathrectangle{\pgfqpoint{0.680860in}{0.078740in}}{\pgfqpoint{7.842520in}{7.842520in}}%
\pgfusepath{clip}%
\pgfsetbuttcap%
\pgfsetroundjoin%
\definecolor{currentfill}{rgb}{0.281477,0.755203,0.432552}%
\pgfsetfillcolor{currentfill}%
\pgfsetlinewidth{0.000000pt}%
\definecolor{currentstroke}{rgb}{0.157729,0.485932,0.558013}%
\pgfsetstrokecolor{currentstroke}%
\pgfsetdash{}{0pt}%
\pgfpathmoveto{\pgfqpoint{5.800838in}{4.745757in}}%
\pgfpathlineto{\pgfqpoint{5.945504in}{4.782570in}}%
\pgfpathlineto{\pgfqpoint{5.877741in}{4.731360in}}%
\pgfpathclose%
\pgfusepath{fill}%
\end{pgfscope}%
\begin{pgfscope}%
\pgfpathrectangle{\pgfqpoint{0.680860in}{0.078740in}}{\pgfqpoint{7.842520in}{7.842520in}}%
\pgfusepath{clip}%
\pgfsetbuttcap%
\pgfsetroundjoin%
\definecolor{currentfill}{rgb}{0.140210,0.665859,0.513427}%
\pgfsetfillcolor{currentfill}%
\pgfsetlinewidth{0.000000pt}%
\definecolor{currentstroke}{rgb}{0.156270,0.489624,0.557936}%
\pgfsetstrokecolor{currentstroke}%
\pgfsetdash{}{0pt}%
\pgfpathmoveto{\pgfqpoint{4.686191in}{4.302308in}}%
\pgfpathlineto{\pgfqpoint{4.626744in}{4.377935in}}%
\pgfpathlineto{\pgfqpoint{4.544985in}{4.278182in}}%
\pgfpathclose%
\pgfusepath{fill}%
\end{pgfscope}%
\begin{pgfscope}%
\pgfpathrectangle{\pgfqpoint{0.680860in}{0.078740in}}{\pgfqpoint{7.842520in}{7.842520in}}%
\pgfusepath{clip}%
\pgfsetbuttcap%
\pgfsetroundjoin%
\definecolor{currentfill}{rgb}{0.150148,0.676631,0.506589}%
\pgfsetfillcolor{currentfill}%
\pgfsetlinewidth{0.000000pt}%
\definecolor{currentstroke}{rgb}{0.154815,0.493313,0.557840}%
\pgfsetstrokecolor{currentstroke}%
\pgfsetdash{}{0pt}%
\pgfpathmoveto{\pgfqpoint{4.767809in}{4.403204in}}%
\pgfpathlineto{\pgfqpoint{4.626744in}{4.377935in}}%
\pgfpathlineto{\pgfqpoint{4.686191in}{4.302308in}}%
\pgfpathclose%
\pgfusepath{fill}%
\end{pgfscope}%
\begin{pgfscope}%
\pgfpathrectangle{\pgfqpoint{0.680860in}{0.078740in}}{\pgfqpoint{7.842520in}{7.842520in}}%
\pgfusepath{clip}%
\pgfsetbuttcap%
\pgfsetroundjoin%
\definecolor{currentfill}{rgb}{0.229739,0.322361,0.545706}%
\pgfsetfillcolor{currentfill}%
\pgfsetlinewidth{0.000000pt}%
\definecolor{currentstroke}{rgb}{0.153364,0.497000,0.557724}%
\pgfsetstrokecolor{currentstroke}%
\pgfsetdash{}{0pt}%
\pgfpathmoveto{\pgfqpoint{3.102470in}{2.649229in}}%
\pgfpathlineto{\pgfqpoint{3.185445in}{2.870203in}}%
\pgfpathlineto{\pgfqpoint{3.047565in}{2.883264in}}%
\pgfpathclose%
\pgfusepath{fill}%
\end{pgfscope}%
\begin{pgfscope}%
\pgfpathrectangle{\pgfqpoint{0.680860in}{0.078740in}}{\pgfqpoint{7.842520in}{7.842520in}}%
\pgfusepath{clip}%
\pgfsetbuttcap%
\pgfsetroundjoin%
\definecolor{currentfill}{rgb}{0.304148,0.764704,0.419943}%
\pgfsetfillcolor{currentfill}%
\pgfsetlinewidth{0.000000pt}%
\definecolor{currentstroke}{rgb}{0.151918,0.500685,0.557587}%
\pgfsetstrokecolor{currentstroke}%
\pgfsetdash{}{0pt}%
\pgfpathmoveto{\pgfqpoint{6.387367in}{4.793337in}}%
\pgfpathlineto{\pgfqpoint{6.533586in}{4.831170in}}%
\pgfpathlineto{\pgfqpoint{6.606180in}{4.755788in}}%
\pgfpathclose%
\pgfusepath{fill}%
\end{pgfscope}%
\begin{pgfscope}%
\pgfpathrectangle{\pgfqpoint{0.680860in}{0.078740in}}{\pgfqpoint{7.842520in}{7.842520in}}%
\pgfusepath{clip}%
\pgfsetbuttcap%
\pgfsetroundjoin%
\definecolor{currentfill}{rgb}{0.304148,0.764704,0.419943}%
\pgfsetfillcolor{currentfill}%
\pgfsetlinewidth{0.000000pt}%
\definecolor{currentstroke}{rgb}{0.150476,0.504369,0.557430}%
\pgfsetstrokecolor{currentstroke}%
\pgfsetdash{}{0pt}%
\pgfpathmoveto{\pgfqpoint{6.752625in}{4.791317in}}%
\pgfpathlineto{\pgfqpoint{6.606180in}{4.755788in}}%
\pgfpathlineto{\pgfqpoint{6.533586in}{4.831170in}}%
\pgfpathclose%
\pgfusepath{fill}%
\end{pgfscope}%
\begin{pgfscope}%
\pgfpathrectangle{\pgfqpoint{0.680860in}{0.078740in}}{\pgfqpoint{7.842520in}{7.842520in}}%
\pgfusepath{clip}%
\pgfsetbuttcap%
\pgfsetroundjoin%
\definecolor{currentfill}{rgb}{0.266941,0.748751,0.440573}%
\pgfsetfillcolor{currentfill}%
\pgfsetlinewidth{0.000000pt}%
\definecolor{currentstroke}{rgb}{0.149039,0.508051,0.557250}%
\pgfsetstrokecolor{currentstroke}%
\pgfsetdash{}{0pt}%
\pgfpathmoveto{\pgfqpoint{7.265465in}{4.792308in}}%
\pgfpathlineto{\pgfqpoint{7.480381in}{4.680637in}}%
\pgfpathlineto{\pgfqpoint{7.332134in}{4.648182in}}%
\pgfpathclose%
\pgfusepath{fill}%
\end{pgfscope}%
\begin{pgfscope}%
\pgfpathrectangle{\pgfqpoint{0.680860in}{0.078740in}}{\pgfqpoint{7.842520in}{7.842520in}}%
\pgfusepath{clip}%
\pgfsetbuttcap%
\pgfsetroundjoin%
\definecolor{currentfill}{rgb}{0.139147,0.533812,0.555298}%
\pgfsetfillcolor{currentfill}%
\pgfsetlinewidth{0.000000pt}%
\definecolor{currentstroke}{rgb}{0.147607,0.511733,0.557049}%
\pgfsetstrokecolor{currentstroke}%
\pgfsetdash{}{0pt}%
\pgfpathmoveto{\pgfqpoint{3.877650in}{3.821131in}}%
\pgfpathlineto{\pgfqpoint{3.794751in}{3.654950in}}%
\pgfpathlineto{\pgfqpoint{3.934233in}{3.662865in}}%
\pgfpathclose%
\pgfusepath{fill}%
\end{pgfscope}%
\begin{pgfscope}%
\pgfpathrectangle{\pgfqpoint{0.680860in}{0.078740in}}{\pgfqpoint{7.842520in}{7.842520in}}%
\pgfusepath{clip}%
\pgfsetbuttcap%
\pgfsetroundjoin%
\definecolor{currentfill}{rgb}{0.165117,0.467423,0.558141}%
\pgfsetfillcolor{currentfill}%
\pgfsetlinewidth{0.000000pt}%
\definecolor{currentstroke}{rgb}{0.146180,0.515413,0.556823}%
\pgfsetstrokecolor{currentstroke}%
\pgfsetdash{}{0pt}%
\pgfpathmoveto{\pgfqpoint{3.489868in}{3.278771in}}%
\pgfpathlineto{\pgfqpoint{3.711794in}{3.473254in}}%
\pgfpathlineto{\pgfqpoint{3.572890in}{3.470658in}}%
\pgfpathclose%
\pgfusepath{fill}%
\end{pgfscope}%
\begin{pgfscope}%
\pgfpathrectangle{\pgfqpoint{0.680860in}{0.078740in}}{\pgfqpoint{7.842520in}{7.842520in}}%
\pgfusepath{clip}%
\pgfsetbuttcap%
\pgfsetroundjoin%
\definecolor{currentfill}{rgb}{0.260571,0.246922,0.522828}%
\pgfsetfillcolor{currentfill}%
\pgfsetlinewidth{0.000000pt}%
\definecolor{currentstroke}{rgb}{0.144759,0.519093,0.556572}%
\pgfsetstrokecolor{currentstroke}%
\pgfsetdash{}{0pt}%
\pgfpathmoveto{\pgfqpoint{2.964535in}{2.667253in}}%
\pgfpathlineto{\pgfqpoint{2.881546in}{2.443230in}}%
\pgfpathlineto{\pgfqpoint{3.019537in}{2.420008in}}%
\pgfpathclose%
\pgfusepath{fill}%
\end{pgfscope}%
\begin{pgfscope}%
\pgfpathrectangle{\pgfqpoint{0.680860in}{0.078740in}}{\pgfqpoint{7.842520in}{7.842520in}}%
\pgfusepath{clip}%
\pgfsetbuttcap%
\pgfsetroundjoin%
\definecolor{currentfill}{rgb}{0.190631,0.407061,0.556089}%
\pgfsetfillcolor{currentfill}%
\pgfsetlinewidth{0.000000pt}%
\definecolor{currentstroke}{rgb}{0.143343,0.522773,0.556295}%
\pgfsetstrokecolor{currentstroke}%
\pgfsetdash{}{0pt}%
\pgfpathmoveto{\pgfqpoint{3.406859in}{3.074121in}}%
\pgfpathlineto{\pgfqpoint{3.489868in}{3.278771in}}%
\pgfpathlineto{\pgfqpoint{3.268464in}{3.081464in}}%
\pgfpathclose%
\pgfusepath{fill}%
\end{pgfscope}%
\begin{pgfscope}%
\pgfpathrectangle{\pgfqpoint{0.680860in}{0.078740in}}{\pgfqpoint{7.842520in}{7.842520in}}%
\pgfusepath{clip}%
\pgfsetbuttcap%
\pgfsetroundjoin%
\definecolor{currentfill}{rgb}{0.281477,0.755203,0.432552}%
\pgfsetfillcolor{currentfill}%
\pgfsetlinewidth{0.000000pt}%
\definecolor{currentstroke}{rgb}{0.141935,0.526453,0.555991}%
\pgfsetstrokecolor{currentstroke}%
\pgfsetdash{}{0pt}%
\pgfpathmoveto{\pgfqpoint{7.117183in}{4.755126in}}%
\pgfpathlineto{\pgfqpoint{7.265465in}{4.792308in}}%
\pgfpathlineto{\pgfqpoint{7.332134in}{4.648182in}}%
\pgfpathclose%
\pgfusepath{fill}%
\end{pgfscope}%
\begin{pgfscope}%
\pgfpathrectangle{\pgfqpoint{0.680860in}{0.078740in}}{\pgfqpoint{7.842520in}{7.842520in}}%
\pgfusepath{clip}%
\pgfsetbuttcap%
\pgfsetroundjoin%
\definecolor{currentfill}{rgb}{0.296479,0.761561,0.424223}%
\pgfsetfillcolor{currentfill}%
\pgfsetlinewidth{0.000000pt}%
\definecolor{currentstroke}{rgb}{0.140536,0.530132,0.555659}%
\pgfsetstrokecolor{currentstroke}%
\pgfsetdash{}{0pt}%
\pgfpathmoveto{\pgfqpoint{6.969772in}{4.720488in}}%
\pgfpathlineto{\pgfqpoint{6.899928in}{4.829425in}}%
\pgfpathlineto{\pgfqpoint{7.117183in}{4.755126in}}%
\pgfpathclose%
\pgfusepath{fill}%
\end{pgfscope}%
\begin{pgfscope}%
\pgfpathrectangle{\pgfqpoint{0.680860in}{0.078740in}}{\pgfqpoint{7.842520in}{7.842520in}}%
\pgfusepath{clip}%
\pgfsetbuttcap%
\pgfsetroundjoin%
\definecolor{currentfill}{rgb}{0.283197,0.115680,0.436115}%
\pgfsetfillcolor{currentfill}%
\pgfsetlinewidth{0.000000pt}%
\definecolor{currentstroke}{rgb}{0.139147,0.533812,0.555298}%
\pgfsetstrokecolor{currentstroke}%
\pgfsetdash{}{0pt}%
\pgfpathmoveto{\pgfqpoint{2.798587in}{2.212678in}}%
\pgfpathlineto{\pgfqpoint{2.578010in}{2.011439in}}%
\pgfpathlineto{\pgfqpoint{2.715637in}{1.977126in}}%
\pgfpathclose%
\pgfusepath{fill}%
\end{pgfscope}%
\begin{pgfscope}%
\pgfpathrectangle{\pgfqpoint{0.680860in}{0.078740in}}{\pgfqpoint{7.842520in}{7.842520in}}%
\pgfusepath{clip}%
\pgfsetbuttcap%
\pgfsetroundjoin%
\definecolor{currentfill}{rgb}{0.133743,0.548535,0.553541}%
\pgfsetfillcolor{currentfill}%
\pgfsetlinewidth{0.000000pt}%
\definecolor{currentstroke}{rgb}{0.137770,0.537492,0.554906}%
\pgfsetstrokecolor{currentstroke}%
\pgfsetdash{}{0pt}%
\pgfpathmoveto{\pgfqpoint{3.934233in}{3.662865in}}%
\pgfpathlineto{\pgfqpoint{4.017076in}{3.832432in}}%
\pgfpathlineto{\pgfqpoint{3.877650in}{3.821131in}}%
\pgfpathclose%
\pgfusepath{fill}%
\end{pgfscope}%
\begin{pgfscope}%
\pgfpathrectangle{\pgfqpoint{0.680860in}{0.078740in}}{\pgfqpoint{7.842520in}{7.842520in}}%
\pgfusepath{clip}%
\pgfsetbuttcap%
\pgfsetroundjoin%
\definecolor{currentfill}{rgb}{0.276022,0.044167,0.370164}%
\pgfsetfillcolor{currentfill}%
\pgfsetlinewidth{0.000000pt}%
\definecolor{currentstroke}{rgb}{0.136408,0.541173,0.554483}%
\pgfsetstrokecolor{currentstroke}%
\pgfsetdash{}{0pt}%
\pgfpathmoveto{\pgfqpoint{2.578010in}{2.011439in}}%
\pgfpathlineto{\pgfqpoint{2.494952in}{1.777867in}}%
\pgfpathlineto{\pgfqpoint{2.632669in}{1.738135in}}%
\pgfpathclose%
\pgfusepath{fill}%
\end{pgfscope}%
\begin{pgfscope}%
\pgfpathrectangle{\pgfqpoint{0.680860in}{0.078740in}}{\pgfqpoint{7.842520in}{7.842520in}}%
\pgfusepath{clip}%
\pgfsetbuttcap%
\pgfsetroundjoin%
\definecolor{currentfill}{rgb}{0.252899,0.742211,0.448284}%
\pgfsetfillcolor{currentfill}%
\pgfsetlinewidth{0.000000pt}%
\definecolor{currentstroke}{rgb}{0.135066,0.544853,0.554029}%
\pgfsetstrokecolor{currentstroke}%
\pgfsetdash{}{0pt}%
\pgfpathmoveto{\pgfqpoint{5.356094in}{4.631957in}}%
\pgfpathlineto{\pgfqpoint{5.578794in}{4.700737in}}%
\pgfpathlineto{\pgfqpoint{5.435389in}{4.667024in}}%
\pgfpathclose%
\pgfusepath{fill}%
\end{pgfscope}%
\begin{pgfscope}%
\pgfpathrectangle{\pgfqpoint{0.680860in}{0.078740in}}{\pgfqpoint{7.842520in}{7.842520in}}%
\pgfusepath{clip}%
\pgfsetbuttcap%
\pgfsetroundjoin%
\definecolor{currentfill}{rgb}{0.281477,0.755203,0.432552}%
\pgfsetfillcolor{currentfill}%
\pgfsetlinewidth{0.000000pt}%
\definecolor{currentstroke}{rgb}{0.133743,0.548535,0.553541}%
\pgfsetstrokecolor{currentstroke}%
\pgfsetdash{}{0pt}%
\pgfpathmoveto{\pgfqpoint{5.656982in}{4.711478in}}%
\pgfpathlineto{\pgfqpoint{5.722997in}{4.736967in}}%
\pgfpathlineto{\pgfqpoint{5.800838in}{4.745757in}}%
\pgfpathclose%
\pgfusepath{fill}%
\end{pgfscope}%
\begin{pgfscope}%
\pgfpathrectangle{\pgfqpoint{0.680860in}{0.078740in}}{\pgfqpoint{7.842520in}{7.842520in}}%
\pgfusepath{clip}%
\pgfsetbuttcap%
\pgfsetroundjoin%
\definecolor{currentfill}{rgb}{0.212395,0.359683,0.551710}%
\pgfsetfillcolor{currentfill}%
\pgfsetlinewidth{0.000000pt}%
\definecolor{currentstroke}{rgb}{0.132444,0.552216,0.553018}%
\pgfsetstrokecolor{currentstroke}%
\pgfsetdash{}{0pt}%
\pgfpathmoveto{\pgfqpoint{3.268464in}{3.081464in}}%
\pgfpathlineto{\pgfqpoint{3.185445in}{2.870203in}}%
\pgfpathlineto{\pgfqpoint{3.323882in}{2.858031in}}%
\pgfpathclose%
\pgfusepath{fill}%
\end{pgfscope}%
\begin{pgfscope}%
\pgfpathrectangle{\pgfqpoint{0.680860in}{0.078740in}}{\pgfqpoint{7.842520in}{7.842520in}}%
\pgfusepath{clip}%
\pgfsetbuttcap%
\pgfsetroundjoin%
\definecolor{currentfill}{rgb}{0.175707,0.697900,0.491033}%
\pgfsetfillcolor{currentfill}%
\pgfsetlinewidth{0.000000pt}%
\definecolor{currentstroke}{rgb}{0.131172,0.555899,0.552459}%
\pgfsetstrokecolor{currentstroke}%
\pgfsetdash{}{0pt}%
\pgfpathmoveto{\pgfqpoint{4.767809in}{4.403204in}}%
\pgfpathlineto{\pgfqpoint{4.909602in}{4.430686in}}%
\pgfpathlineto{\pgfqpoint{4.990600in}{4.511026in}}%
\pgfpathclose%
\pgfusepath{fill}%
\end{pgfscope}%
\begin{pgfscope}%
\pgfpathrectangle{\pgfqpoint{0.680860in}{0.078740in}}{\pgfqpoint{7.842520in}{7.842520in}}%
\pgfusepath{clip}%
\pgfsetbuttcap%
\pgfsetroundjoin%
\definecolor{currentfill}{rgb}{0.252194,0.269783,0.531579}%
\pgfsetfillcolor{currentfill}%
\pgfsetlinewidth{0.000000pt}%
\definecolor{currentstroke}{rgb}{0.129933,0.559582,0.551864}%
\pgfsetstrokecolor{currentstroke}%
\pgfsetdash{}{0pt}%
\pgfpathmoveto{\pgfqpoint{3.019537in}{2.420008in}}%
\pgfpathlineto{\pgfqpoint{3.102470in}{2.649229in}}%
\pgfpathlineto{\pgfqpoint{2.964535in}{2.667253in}}%
\pgfpathclose%
\pgfusepath{fill}%
\end{pgfscope}%
\begin{pgfscope}%
\pgfpathrectangle{\pgfqpoint{0.680860in}{0.078740in}}{\pgfqpoint{7.842520in}{7.842520in}}%
\pgfusepath{clip}%
\pgfsetbuttcap%
\pgfsetroundjoin%
\definecolor{currentfill}{rgb}{0.273006,0.204520,0.501721}%
\pgfsetfillcolor{currentfill}%
\pgfsetlinewidth{0.000000pt}%
\definecolor{currentstroke}{rgb}{0.128729,0.563265,0.551229}%
\pgfsetstrokecolor{currentstroke}%
\pgfsetdash{}{0pt}%
\pgfpathmoveto{\pgfqpoint{2.881546in}{2.443230in}}%
\pgfpathlineto{\pgfqpoint{2.798587in}{2.212678in}}%
\pgfpathlineto{\pgfqpoint{3.019537in}{2.420008in}}%
\pgfpathclose%
\pgfusepath{fill}%
\end{pgfscope}%
\begin{pgfscope}%
\pgfpathrectangle{\pgfqpoint{0.680860in}{0.078740in}}{\pgfqpoint{7.842520in}{7.842520in}}%
\pgfusepath{clip}%
\pgfsetbuttcap%
\pgfsetroundjoin%
\definecolor{currentfill}{rgb}{0.281477,0.755203,0.432552}%
\pgfsetfillcolor{currentfill}%
\pgfsetlinewidth{0.000000pt}%
\definecolor{currentstroke}{rgb}{0.127568,0.566949,0.550556}%
\pgfsetstrokecolor{currentstroke}%
\pgfsetdash{}{0pt}%
\pgfpathmoveto{\pgfqpoint{5.656982in}{4.711478in}}%
\pgfpathlineto{\pgfqpoint{5.578794in}{4.700737in}}%
\pgfpathlineto{\pgfqpoint{5.722997in}{4.736967in}}%
\pgfpathclose%
\pgfusepath{fill}%
\end{pgfscope}%
\begin{pgfscope}%
\pgfpathrectangle{\pgfqpoint{0.680860in}{0.078740in}}{\pgfqpoint{7.842520in}{7.842520in}}%
\pgfusepath{clip}%
\pgfsetbuttcap%
\pgfsetroundjoin%
\definecolor{currentfill}{rgb}{0.279566,0.067836,0.391917}%
\pgfsetfillcolor{currentfill}%
\pgfsetlinewidth{0.000000pt}%
\definecolor{currentstroke}{rgb}{0.126453,0.570633,0.549841}%
\pgfsetstrokecolor{currentstroke}%
\pgfsetdash{}{0pt}%
\pgfpathmoveto{\pgfqpoint{2.578010in}{2.011439in}}%
\pgfpathlineto{\pgfqpoint{2.632669in}{1.738135in}}%
\pgfpathlineto{\pgfqpoint{2.715637in}{1.977126in}}%
\pgfpathclose%
\pgfusepath{fill}%
\end{pgfscope}%
\begin{pgfscope}%
\pgfpathrectangle{\pgfqpoint{0.680860in}{0.078740in}}{\pgfqpoint{7.842520in}{7.842520in}}%
\pgfusepath{clip}%
\pgfsetbuttcap%
\pgfsetroundjoin%
\definecolor{currentfill}{rgb}{0.120638,0.625828,0.533488}%
\pgfsetfillcolor{currentfill}%
\pgfsetlinewidth{0.000000pt}%
\definecolor{currentstroke}{rgb}{0.125394,0.574318,0.549086}%
\pgfsetstrokecolor{currentstroke}%
\pgfsetdash{}{0pt}%
\pgfpathmoveto{\pgfqpoint{4.322275in}{4.138066in}}%
\pgfpathlineto{\pgfqpoint{4.239812in}{4.000958in}}%
\pgfpathlineto{\pgfqpoint{4.462885in}{4.158325in}}%
\pgfpathclose%
\pgfusepath{fill}%
\end{pgfscope}%
\begin{pgfscope}%
\pgfpathrectangle{\pgfqpoint{0.680860in}{0.078740in}}{\pgfqpoint{7.842520in}{7.842520in}}%
\pgfusepath{clip}%
\pgfsetbuttcap%
\pgfsetroundjoin%
\definecolor{currentfill}{rgb}{0.124395,0.578002,0.548287}%
\pgfsetfillcolor{currentfill}%
\pgfsetlinewidth{0.000000pt}%
\definecolor{currentstroke}{rgb}{0.124395,0.578002,0.548287}%
\pgfsetstrokecolor{currentstroke}%
\pgfsetdash{}{0pt}%
\pgfpathmoveto{\pgfqpoint{4.099798in}{3.985000in}}%
\pgfpathlineto{\pgfqpoint{4.017076in}{3.832432in}}%
\pgfpathlineto{\pgfqpoint{4.157156in}{3.845456in}}%
\pgfpathclose%
\pgfusepath{fill}%
\end{pgfscope}%
\begin{pgfscope}%
\pgfpathrectangle{\pgfqpoint{0.680860in}{0.078740in}}{\pgfqpoint{7.842520in}{7.842520in}}%
\pgfusepath{clip}%
\pgfsetbuttcap%
\pgfsetroundjoin%
\definecolor{currentfill}{rgb}{0.203063,0.379716,0.553925}%
\pgfsetfillcolor{currentfill}%
\pgfsetlinewidth{0.000000pt}%
\definecolor{currentstroke}{rgb}{0.123463,0.581687,0.547445}%
\pgfsetstrokecolor{currentstroke}%
\pgfsetdash{}{0pt}%
\pgfpathmoveto{\pgfqpoint{3.268464in}{3.081464in}}%
\pgfpathlineto{\pgfqpoint{3.323882in}{2.858031in}}%
\pgfpathlineto{\pgfqpoint{3.406859in}{3.074121in}}%
\pgfpathclose%
\pgfusepath{fill}%
\end{pgfscope}%
\begin{pgfscope}%
\pgfpathrectangle{\pgfqpoint{0.680860in}{0.078740in}}{\pgfqpoint{7.842520in}{7.842520in}}%
\pgfusepath{clip}%
\pgfsetbuttcap%
\pgfsetroundjoin%
\definecolor{currentfill}{rgb}{0.121148,0.592739,0.544641}%
\pgfsetfillcolor{currentfill}%
\pgfsetlinewidth{0.000000pt}%
\definecolor{currentstroke}{rgb}{0.122606,0.585371,0.546557}%
\pgfsetstrokecolor{currentstroke}%
\pgfsetdash{}{0pt}%
\pgfpathmoveto{\pgfqpoint{4.157156in}{3.845456in}}%
\pgfpathlineto{\pgfqpoint{4.239812in}{4.000958in}}%
\pgfpathlineto{\pgfqpoint{4.099798in}{3.985000in}}%
\pgfpathclose%
\pgfusepath{fill}%
\end{pgfscope}%
\begin{pgfscope}%
\pgfpathrectangle{\pgfqpoint{0.680860in}{0.078740in}}{\pgfqpoint{7.842520in}{7.842520in}}%
\pgfusepath{clip}%
\pgfsetbuttcap%
\pgfsetroundjoin%
\definecolor{currentfill}{rgb}{0.146180,0.515413,0.556823}%
\pgfsetfillcolor{currentfill}%
\pgfsetlinewidth{0.000000pt}%
\definecolor{currentstroke}{rgb}{0.121831,0.589055,0.545623}%
\pgfsetstrokecolor{currentstroke}%
\pgfsetdash{}{0pt}%
\pgfpathmoveto{\pgfqpoint{3.794751in}{3.654950in}}%
\pgfpathlineto{\pgfqpoint{3.711794in}{3.473254in}}%
\pgfpathlineto{\pgfqpoint{3.934233in}{3.662865in}}%
\pgfpathclose%
\pgfusepath{fill}%
\end{pgfscope}%
\begin{pgfscope}%
\pgfpathrectangle{\pgfqpoint{0.680860in}{0.078740in}}{\pgfqpoint{7.842520in}{7.842520in}}%
\pgfusepath{clip}%
\pgfsetbuttcap%
\pgfsetroundjoin%
\definecolor{currentfill}{rgb}{0.134692,0.658636,0.517649}%
\pgfsetfillcolor{currentfill}%
\pgfsetlinewidth{0.000000pt}%
\definecolor{currentstroke}{rgb}{0.121148,0.592739,0.544641}%
\pgfsetstrokecolor{currentstroke}%
\pgfsetdash{}{0pt}%
\pgfpathmoveto{\pgfqpoint{4.462885in}{4.158325in}}%
\pgfpathlineto{\pgfqpoint{4.686191in}{4.302308in}}%
\pgfpathlineto{\pgfqpoint{4.544985in}{4.278182in}}%
\pgfpathclose%
\pgfusepath{fill}%
\end{pgfscope}%
\begin{pgfscope}%
\pgfpathrectangle{\pgfqpoint{0.680860in}{0.078740in}}{\pgfqpoint{7.842520in}{7.842520in}}%
\pgfusepath{clip}%
\pgfsetbuttcap%
\pgfsetroundjoin%
\definecolor{currentfill}{rgb}{0.168126,0.459988,0.558082}%
\pgfsetfillcolor{currentfill}%
\pgfsetlinewidth{0.000000pt}%
\definecolor{currentstroke}{rgb}{0.120565,0.596422,0.543611}%
\pgfsetstrokecolor{currentstroke}%
\pgfsetdash{}{0pt}%
\pgfpathmoveto{\pgfqpoint{3.628812in}{3.277143in}}%
\pgfpathlineto{\pgfqpoint{3.711794in}{3.473254in}}%
\pgfpathlineto{\pgfqpoint{3.489868in}{3.278771in}}%
\pgfpathclose%
\pgfusepath{fill}%
\end{pgfscope}%
\begin{pgfscope}%
\pgfpathrectangle{\pgfqpoint{0.680860in}{0.078740in}}{\pgfqpoint{7.842520in}{7.842520in}}%
\pgfusepath{clip}%
\pgfsetbuttcap%
\pgfsetroundjoin%
\definecolor{currentfill}{rgb}{0.235526,0.309527,0.542944}%
\pgfsetfillcolor{currentfill}%
\pgfsetlinewidth{0.000000pt}%
\definecolor{currentstroke}{rgb}{0.120092,0.600104,0.542530}%
\pgfsetstrokecolor{currentstroke}%
\pgfsetdash{}{0pt}%
\pgfpathmoveto{\pgfqpoint{3.102470in}{2.649229in}}%
\pgfpathlineto{\pgfqpoint{3.240947in}{2.631931in}}%
\pgfpathlineto{\pgfqpoint{3.185445in}{2.870203in}}%
\pgfpathclose%
\pgfusepath{fill}%
\end{pgfscope}%
\begin{pgfscope}%
\pgfpathrectangle{\pgfqpoint{0.680860in}{0.078740in}}{\pgfqpoint{7.842520in}{7.842520in}}%
\pgfusepath{clip}%
\pgfsetbuttcap%
\pgfsetroundjoin%
\definecolor{currentfill}{rgb}{0.162016,0.687316,0.499129}%
\pgfsetfillcolor{currentfill}%
\pgfsetlinewidth{0.000000pt}%
\definecolor{currentstroke}{rgb}{0.119738,0.603785,0.541400}%
\pgfsetstrokecolor{currentstroke}%
\pgfsetdash{}{0pt}%
\pgfpathmoveto{\pgfqpoint{4.909602in}{4.430686in}}%
\pgfpathlineto{\pgfqpoint{4.767809in}{4.403204in}}%
\pgfpathlineto{\pgfqpoint{4.686191in}{4.302308in}}%
\pgfpathclose%
\pgfusepath{fill}%
\end{pgfscope}%
\begin{pgfscope}%
\pgfpathrectangle{\pgfqpoint{0.680860in}{0.078740in}}{\pgfqpoint{7.842520in}{7.842520in}}%
\pgfusepath{clip}%
\pgfsetbuttcap%
\pgfsetroundjoin%
\definecolor{currentfill}{rgb}{0.276194,0.190074,0.493001}%
\pgfsetfillcolor{currentfill}%
\pgfsetlinewidth{0.000000pt}%
\definecolor{currentstroke}{rgb}{0.119512,0.607464,0.540218}%
\pgfsetstrokecolor{currentstroke}%
\pgfsetdash{}{0pt}%
\pgfpathmoveto{\pgfqpoint{3.019537in}{2.420008in}}%
\pgfpathlineto{\pgfqpoint{2.798587in}{2.212678in}}%
\pgfpathlineto{\pgfqpoint{2.936638in}{2.184072in}}%
\pgfpathclose%
\pgfusepath{fill}%
\end{pgfscope}%
\begin{pgfscope}%
\pgfpathrectangle{\pgfqpoint{0.680860in}{0.078740in}}{\pgfqpoint{7.842520in}{7.842520in}}%
\pgfusepath{clip}%
\pgfsetbuttcap%
\pgfsetroundjoin%
\definecolor{currentfill}{rgb}{0.283229,0.120777,0.440584}%
\pgfsetfillcolor{currentfill}%
\pgfsetlinewidth{0.000000pt}%
\definecolor{currentstroke}{rgb}{0.119423,0.611141,0.538982}%
\pgfsetstrokecolor{currentstroke}%
\pgfsetdash{}{0pt}%
\pgfpathmoveto{\pgfqpoint{2.853755in}{1.943006in}}%
\pgfpathlineto{\pgfqpoint{2.798587in}{2.212678in}}%
\pgfpathlineto{\pgfqpoint{2.715637in}{1.977126in}}%
\pgfpathclose%
\pgfusepath{fill}%
\end{pgfscope}%
\begin{pgfscope}%
\pgfpathrectangle{\pgfqpoint{0.680860in}{0.078740in}}{\pgfqpoint{7.842520in}{7.842520in}}%
\pgfusepath{clip}%
\pgfsetbuttcap%
\pgfsetroundjoin%
\definecolor{currentfill}{rgb}{0.187231,0.414746,0.556547}%
\pgfsetfillcolor{currentfill}%
\pgfsetlinewidth{0.000000pt}%
\definecolor{currentstroke}{rgb}{0.119483,0.614817,0.537692}%
\pgfsetstrokecolor{currentstroke}%
\pgfsetdash{}{0pt}%
\pgfpathmoveto{\pgfqpoint{3.406859in}{3.074121in}}%
\pgfpathlineto{\pgfqpoint{3.545835in}{3.067862in}}%
\pgfpathlineto{\pgfqpoint{3.489868in}{3.278771in}}%
\pgfpathclose%
\pgfusepath{fill}%
\end{pgfscope}%
\begin{pgfscope}%
\pgfpathrectangle{\pgfqpoint{0.680860in}{0.078740in}}{\pgfqpoint{7.842520in}{7.842520in}}%
\pgfusepath{clip}%
\pgfsetbuttcap%
\pgfsetroundjoin%
\definecolor{currentfill}{rgb}{0.335885,0.777018,0.402049}%
\pgfsetfillcolor{currentfill}%
\pgfsetlinewidth{0.000000pt}%
\definecolor{currentstroke}{rgb}{0.119699,0.618490,0.536347}%
\pgfsetstrokecolor{currentstroke}%
\pgfsetdash{}{0pt}%
\pgfpathmoveto{\pgfqpoint{6.167049in}{4.801858in}}%
\pgfpathlineto{\pgfqpoint{6.312949in}{4.840984in}}%
\pgfpathlineto{\pgfqpoint{6.387367in}{4.793337in}}%
\pgfpathclose%
\pgfusepath{fill}%
\end{pgfscope}%
\begin{pgfscope}%
\pgfpathrectangle{\pgfqpoint{0.680860in}{0.078740in}}{\pgfqpoint{7.842520in}{7.842520in}}%
\pgfusepath{clip}%
\pgfsetbuttcap%
\pgfsetroundjoin%
\definecolor{currentfill}{rgb}{0.225863,0.330805,0.547314}%
\pgfsetfillcolor{currentfill}%
\pgfsetlinewidth{0.000000pt}%
\definecolor{currentstroke}{rgb}{0.120081,0.622161,0.534946}%
\pgfsetstrokecolor{currentstroke}%
\pgfsetdash{}{0pt}%
\pgfpathmoveto{\pgfqpoint{3.185445in}{2.870203in}}%
\pgfpathlineto{\pgfqpoint{3.240947in}{2.631931in}}%
\pgfpathlineto{\pgfqpoint{3.323882in}{2.858031in}}%
\pgfpathclose%
\pgfusepath{fill}%
\end{pgfscope}%
\begin{pgfscope}%
\pgfpathrectangle{\pgfqpoint{0.680860in}{0.078740in}}{\pgfqpoint{7.842520in}{7.842520in}}%
\pgfusepath{clip}%
\pgfsetbuttcap%
\pgfsetroundjoin%
\definecolor{currentfill}{rgb}{0.327796,0.773980,0.406640}%
\pgfsetfillcolor{currentfill}%
\pgfsetlinewidth{0.000000pt}%
\definecolor{currentstroke}{rgb}{0.120638,0.625828,0.533488}%
\pgfsetstrokecolor{currentstroke}%
\pgfsetdash{}{0pt}%
\pgfpathmoveto{\pgfqpoint{6.091000in}{4.822016in}}%
\pgfpathlineto{\pgfqpoint{6.167049in}{4.801858in}}%
\pgfpathlineto{\pgfqpoint{6.021988in}{4.765349in}}%
\pgfpathclose%
\pgfusepath{fill}%
\end{pgfscope}%
\begin{pgfscope}%
\pgfpathrectangle{\pgfqpoint{0.680860in}{0.078740in}}{\pgfqpoint{7.842520in}{7.842520in}}%
\pgfusepath{clip}%
\pgfsetbuttcap%
\pgfsetroundjoin%
\definecolor{currentfill}{rgb}{0.246070,0.738910,0.452024}%
\pgfsetfillcolor{currentfill}%
\pgfsetlinewidth{0.000000pt}%
\definecolor{currentstroke}{rgb}{0.121380,0.629492,0.531973}%
\pgfsetstrokecolor{currentstroke}%
\pgfsetdash{}{0pt}%
\pgfpathmoveto{\pgfqpoint{5.276092in}{4.573915in}}%
\pgfpathlineto{\pgfqpoint{5.356094in}{4.631957in}}%
\pgfpathlineto{\pgfqpoint{5.213191in}{4.599610in}}%
\pgfpathclose%
\pgfusepath{fill}%
\end{pgfscope}%
\begin{pgfscope}%
\pgfpathrectangle{\pgfqpoint{0.680860in}{0.078740in}}{\pgfqpoint{7.842520in}{7.842520in}}%
\pgfusepath{clip}%
\pgfsetbuttcap%
\pgfsetroundjoin%
\definecolor{currentfill}{rgb}{0.250425,0.274290,0.533103}%
\pgfsetfillcolor{currentfill}%
\pgfsetlinewidth{0.000000pt}%
\definecolor{currentstroke}{rgb}{0.122312,0.633153,0.530398}%
\pgfsetstrokecolor{currentstroke}%
\pgfsetdash{}{0pt}%
\pgfpathmoveto{\pgfqpoint{3.019537in}{2.420008in}}%
\pgfpathlineto{\pgfqpoint{3.240947in}{2.631931in}}%
\pgfpathlineto{\pgfqpoint{3.102470in}{2.649229in}}%
\pgfpathclose%
\pgfusepath{fill}%
\end{pgfscope}%
\begin{pgfscope}%
\pgfpathrectangle{\pgfqpoint{0.680860in}{0.078740in}}{\pgfqpoint{7.842520in}{7.842520in}}%
\pgfusepath{clip}%
\pgfsetbuttcap%
\pgfsetroundjoin%
\definecolor{currentfill}{rgb}{0.282290,0.145912,0.461510}%
\pgfsetfillcolor{currentfill}%
\pgfsetlinewidth{0.000000pt}%
\definecolor{currentstroke}{rgb}{0.123444,0.636809,0.528763}%
\pgfsetstrokecolor{currentstroke}%
\pgfsetdash{}{0pt}%
\pgfpathmoveto{\pgfqpoint{2.936638in}{2.184072in}}%
\pgfpathlineto{\pgfqpoint{2.798587in}{2.212678in}}%
\pgfpathlineto{\pgfqpoint{2.853755in}{1.943006in}}%
\pgfpathclose%
\pgfusepath{fill}%
\end{pgfscope}%
\begin{pgfscope}%
\pgfpathrectangle{\pgfqpoint{0.680860in}{0.078740in}}{\pgfqpoint{7.842520in}{7.842520in}}%
\pgfusepath{clip}%
\pgfsetbuttcap%
\pgfsetroundjoin%
\definecolor{currentfill}{rgb}{0.129933,0.559582,0.551864}%
\pgfsetfillcolor{currentfill}%
\pgfsetlinewidth{0.000000pt}%
\definecolor{currentstroke}{rgb}{0.124780,0.640461,0.527068}%
\pgfsetstrokecolor{currentstroke}%
\pgfsetdash{}{0pt}%
\pgfpathmoveto{\pgfqpoint{4.017076in}{3.832432in}}%
\pgfpathlineto{\pgfqpoint{3.934233in}{3.662865in}}%
\pgfpathlineto{\pgfqpoint{4.157156in}{3.845456in}}%
\pgfpathclose%
\pgfusepath{fill}%
\end{pgfscope}%
\begin{pgfscope}%
\pgfpathrectangle{\pgfqpoint{0.680860in}{0.078740in}}{\pgfqpoint{7.842520in}{7.842520in}}%
\pgfusepath{clip}%
\pgfsetbuttcap%
\pgfsetroundjoin%
\definecolor{currentfill}{rgb}{0.280267,0.073417,0.397163}%
\pgfsetfillcolor{currentfill}%
\pgfsetlinewidth{0.000000pt}%
\definecolor{currentstroke}{rgb}{0.126326,0.644107,0.525311}%
\pgfsetstrokecolor{currentstroke}%
\pgfsetdash{}{0pt}%
\pgfpathmoveto{\pgfqpoint{2.715637in}{1.977126in}}%
\pgfpathlineto{\pgfqpoint{2.632669in}{1.738135in}}%
\pgfpathlineto{\pgfqpoint{2.853755in}{1.943006in}}%
\pgfpathclose%
\pgfusepath{fill}%
\end{pgfscope}%
\begin{pgfscope}%
\pgfpathrectangle{\pgfqpoint{0.680860in}{0.078740in}}{\pgfqpoint{7.842520in}{7.842520in}}%
\pgfusepath{clip}%
\pgfsetbuttcap%
\pgfsetroundjoin%
\definecolor{currentfill}{rgb}{0.232815,0.732247,0.459277}%
\pgfsetfillcolor{currentfill}%
\pgfsetlinewidth{0.000000pt}%
\definecolor{currentstroke}{rgb}{0.128087,0.647749,0.523491}%
\pgfsetstrokecolor{currentstroke}%
\pgfsetdash{}{0pt}%
\pgfpathmoveto{\pgfqpoint{5.213191in}{4.599610in}}%
\pgfpathlineto{\pgfqpoint{5.132963in}{4.541273in}}%
\pgfpathlineto{\pgfqpoint{5.276092in}{4.573915in}}%
\pgfpathclose%
\pgfusepath{fill}%
\end{pgfscope}%
\begin{pgfscope}%
\pgfpathrectangle{\pgfqpoint{0.680860in}{0.078740in}}{\pgfqpoint{7.842520in}{7.842520in}}%
\pgfusepath{clip}%
\pgfsetbuttcap%
\pgfsetroundjoin%
\definecolor{currentfill}{rgb}{0.179019,0.433756,0.557430}%
\pgfsetfillcolor{currentfill}%
\pgfsetlinewidth{0.000000pt}%
\definecolor{currentstroke}{rgb}{0.130067,0.651384,0.521608}%
\pgfsetstrokecolor{currentstroke}%
\pgfsetdash{}{0pt}%
\pgfpathmoveto{\pgfqpoint{3.489868in}{3.278771in}}%
\pgfpathlineto{\pgfqpoint{3.545835in}{3.067862in}}%
\pgfpathlineto{\pgfqpoint{3.628812in}{3.277143in}}%
\pgfpathclose%
\pgfusepath{fill}%
\end{pgfscope}%
\begin{pgfscope}%
\pgfpathrectangle{\pgfqpoint{0.680860in}{0.078740in}}{\pgfqpoint{7.842520in}{7.842520in}}%
\pgfusepath{clip}%
\pgfsetbuttcap%
\pgfsetroundjoin%
\definecolor{currentfill}{rgb}{0.208030,0.718701,0.472873}%
\pgfsetfillcolor{currentfill}%
\pgfsetlinewidth{0.000000pt}%
\definecolor{currentstroke}{rgb}{0.132268,0.655014,0.519661}%
\pgfsetstrokecolor{currentstroke}%
\pgfsetdash{}{0pt}%
\pgfpathmoveto{\pgfqpoint{5.052142in}{4.460464in}}%
\pgfpathlineto{\pgfqpoint{5.132963in}{4.541273in}}%
\pgfpathlineto{\pgfqpoint{4.990600in}{4.511026in}}%
\pgfpathclose%
\pgfusepath{fill}%
\end{pgfscope}%
\begin{pgfscope}%
\pgfpathrectangle{\pgfqpoint{0.680860in}{0.078740in}}{\pgfqpoint{7.842520in}{7.842520in}}%
\pgfusepath{clip}%
\pgfsetbuttcap%
\pgfsetroundjoin%
\definecolor{currentfill}{rgb}{0.196571,0.711827,0.479221}%
\pgfsetfillcolor{currentfill}%
\pgfsetlinewidth{0.000000pt}%
\definecolor{currentstroke}{rgb}{0.134692,0.658636,0.517649}%
\pgfsetstrokecolor{currentstroke}%
\pgfsetdash{}{0pt}%
\pgfpathmoveto{\pgfqpoint{4.990600in}{4.511026in}}%
\pgfpathlineto{\pgfqpoint{4.909602in}{4.430686in}}%
\pgfpathlineto{\pgfqpoint{5.052142in}{4.460464in}}%
\pgfpathclose%
\pgfusepath{fill}%
\end{pgfscope}%
\begin{pgfscope}%
\pgfpathrectangle{\pgfqpoint{0.680860in}{0.078740in}}{\pgfqpoint{7.842520in}{7.842520in}}%
\pgfusepath{clip}%
\pgfsetbuttcap%
\pgfsetroundjoin%
\definecolor{currentfill}{rgb}{0.120638,0.625828,0.533488}%
\pgfsetfillcolor{currentfill}%
\pgfsetlinewidth{0.000000pt}%
\definecolor{currentstroke}{rgb}{0.137339,0.662252,0.515571}%
\pgfsetstrokecolor{currentstroke}%
\pgfsetdash{}{0pt}%
\pgfpathmoveto{\pgfqpoint{4.462885in}{4.158325in}}%
\pgfpathlineto{\pgfqpoint{4.239812in}{4.000958in}}%
\pgfpathlineto{\pgfqpoint{4.380505in}{4.018802in}}%
\pgfpathclose%
\pgfusepath{fill}%
\end{pgfscope}%
\begin{pgfscope}%
\pgfpathrectangle{\pgfqpoint{0.680860in}{0.078740in}}{\pgfqpoint{7.842520in}{7.842520in}}%
\pgfusepath{clip}%
\pgfsetbuttcap%
\pgfsetroundjoin%
\definecolor{currentfill}{rgb}{0.149039,0.508051,0.557250}%
\pgfsetfillcolor{currentfill}%
\pgfsetlinewidth{0.000000pt}%
\definecolor{currentstroke}{rgb}{0.140210,0.665859,0.513427}%
\pgfsetstrokecolor{currentstroke}%
\pgfsetdash{}{0pt}%
\pgfpathmoveto{\pgfqpoint{3.934233in}{3.662865in}}%
\pgfpathlineto{\pgfqpoint{3.711794in}{3.473254in}}%
\pgfpathlineto{\pgfqpoint{3.851315in}{3.477272in}}%
\pgfpathclose%
\pgfusepath{fill}%
\end{pgfscope}%
\begin{pgfscope}%
\pgfpathrectangle{\pgfqpoint{0.680860in}{0.078740in}}{\pgfqpoint{7.842520in}{7.842520in}}%
\pgfusepath{clip}%
\pgfsetbuttcap%
\pgfsetroundjoin%
\definecolor{currentfill}{rgb}{0.335885,0.777018,0.402049}%
\pgfsetfillcolor{currentfill}%
\pgfsetlinewidth{0.000000pt}%
\definecolor{currentstroke}{rgb}{0.143303,0.669459,0.511215}%
\pgfsetstrokecolor{currentstroke}%
\pgfsetdash{}{0pt}%
\pgfpathmoveto{\pgfqpoint{6.091000in}{4.822016in}}%
\pgfpathlineto{\pgfqpoint{6.021988in}{4.765349in}}%
\pgfpathlineto{\pgfqpoint{5.945504in}{4.782570in}}%
\pgfpathclose%
\pgfusepath{fill}%
\end{pgfscope}%
\begin{pgfscope}%
\pgfpathrectangle{\pgfqpoint{0.680860in}{0.078740in}}{\pgfqpoint{7.842520in}{7.842520in}}%
\pgfusepath{clip}%
\pgfsetbuttcap%
\pgfsetroundjoin%
\definecolor{currentfill}{rgb}{0.274149,0.751988,0.436601}%
\pgfsetfillcolor{currentfill}%
\pgfsetlinewidth{0.000000pt}%
\definecolor{currentstroke}{rgb}{0.146616,0.673050,0.508936}%
\pgfsetstrokecolor{currentstroke}%
\pgfsetdash{}{0pt}%
\pgfpathmoveto{\pgfqpoint{5.499781in}{4.666774in}}%
\pgfpathlineto{\pgfqpoint{5.578794in}{4.700737in}}%
\pgfpathlineto{\pgfqpoint{5.356094in}{4.631957in}}%
\pgfpathclose%
\pgfusepath{fill}%
\end{pgfscope}%
\begin{pgfscope}%
\pgfpathrectangle{\pgfqpoint{0.680860in}{0.078740in}}{\pgfqpoint{7.842520in}{7.842520in}}%
\pgfusepath{clip}%
\pgfsetbuttcap%
\pgfsetroundjoin%
\definecolor{currentfill}{rgb}{0.352360,0.783011,0.392636}%
\pgfsetfillcolor{currentfill}%
\pgfsetlinewidth{0.000000pt}%
\definecolor{currentstroke}{rgb}{0.150148,0.676631,0.506589}%
\pgfsetstrokecolor{currentstroke}%
\pgfsetdash{}{0pt}%
\pgfpathmoveto{\pgfqpoint{6.533586in}{4.831170in}}%
\pgfpathlineto{\pgfqpoint{6.680668in}{4.871671in}}%
\pgfpathlineto{\pgfqpoint{6.752625in}{4.791317in}}%
\pgfpathclose%
\pgfusepath{fill}%
\end{pgfscope}%
\begin{pgfscope}%
\pgfpathrectangle{\pgfqpoint{0.680860in}{0.078740in}}{\pgfqpoint{7.842520in}{7.842520in}}%
\pgfusepath{clip}%
\pgfsetbuttcap%
\pgfsetroundjoin%
\definecolor{currentfill}{rgb}{0.206756,0.371758,0.553117}%
\pgfsetfillcolor{currentfill}%
\pgfsetlinewidth{0.000000pt}%
\definecolor{currentstroke}{rgb}{0.153894,0.680203,0.504172}%
\pgfsetstrokecolor{currentstroke}%
\pgfsetdash{}{0pt}%
\pgfpathmoveto{\pgfqpoint{3.406859in}{3.074121in}}%
\pgfpathlineto{\pgfqpoint{3.323882in}{2.858031in}}%
\pgfpathlineto{\pgfqpoint{3.462884in}{2.846783in}}%
\pgfpathclose%
\pgfusepath{fill}%
\end{pgfscope}%
\begin{pgfscope}%
\pgfpathrectangle{\pgfqpoint{0.680860in}{0.078740in}}{\pgfqpoint{7.842520in}{7.842520in}}%
\pgfusepath{clip}%
\pgfsetbuttcap%
\pgfsetroundjoin%
\definecolor{currentfill}{rgb}{0.119738,0.603785,0.541400}%
\pgfsetfillcolor{currentfill}%
\pgfsetlinewidth{0.000000pt}%
\definecolor{currentstroke}{rgb}{0.157851,0.683765,0.501686}%
\pgfsetstrokecolor{currentstroke}%
\pgfsetdash{}{0pt}%
\pgfpathmoveto{\pgfqpoint{4.380505in}{4.018802in}}%
\pgfpathlineto{\pgfqpoint{4.239812in}{4.000958in}}%
\pgfpathlineto{\pgfqpoint{4.157156in}{3.845456in}}%
\pgfpathclose%
\pgfusepath{fill}%
\end{pgfscope}%
\begin{pgfscope}%
\pgfpathrectangle{\pgfqpoint{0.680860in}{0.078740in}}{\pgfqpoint{7.842520in}{7.842520in}}%
\pgfusepath{clip}%
\pgfsetbuttcap%
\pgfsetroundjoin%
\definecolor{currentfill}{rgb}{0.277018,0.050344,0.375715}%
\pgfsetfillcolor{currentfill}%
\pgfsetlinewidth{0.000000pt}%
\definecolor{currentstroke}{rgb}{0.162016,0.687316,0.499129}%
\pgfsetstrokecolor{currentstroke}%
\pgfsetdash{}{0pt}%
\pgfpathmoveto{\pgfqpoint{2.632669in}{1.738135in}}%
\pgfpathlineto{\pgfqpoint{2.770862in}{1.698424in}}%
\pgfpathlineto{\pgfqpoint{2.853755in}{1.943006in}}%
\pgfpathclose%
\pgfusepath{fill}%
\end{pgfscope}%
\begin{pgfscope}%
\pgfpathrectangle{\pgfqpoint{0.680860in}{0.078740in}}{\pgfqpoint{7.842520in}{7.842520in}}%
\pgfusepath{clip}%
\pgfsetbuttcap%
\pgfsetroundjoin%
\definecolor{currentfill}{rgb}{0.137339,0.662252,0.515571}%
\pgfsetfillcolor{currentfill}%
\pgfsetlinewidth{0.000000pt}%
\definecolor{currentstroke}{rgb}{0.166383,0.690856,0.496502}%
\pgfsetstrokecolor{currentstroke}%
\pgfsetdash{}{0pt}%
\pgfpathmoveto{\pgfqpoint{4.604197in}{4.180624in}}%
\pgfpathlineto{\pgfqpoint{4.686191in}{4.302308in}}%
\pgfpathlineto{\pgfqpoint{4.462885in}{4.158325in}}%
\pgfpathclose%
\pgfusepath{fill}%
\end{pgfscope}%
\begin{pgfscope}%
\pgfpathrectangle{\pgfqpoint{0.680860in}{0.078740in}}{\pgfqpoint{7.842520in}{7.842520in}}%
\pgfusepath{clip}%
\pgfsetbuttcap%
\pgfsetroundjoin%
\definecolor{currentfill}{rgb}{0.165117,0.467423,0.558141}%
\pgfsetfillcolor{currentfill}%
\pgfsetlinewidth{0.000000pt}%
\definecolor{currentstroke}{rgb}{0.170948,0.694384,0.493803}%
\pgfsetstrokecolor{currentstroke}%
\pgfsetdash{}{0pt}%
\pgfpathmoveto{\pgfqpoint{3.628812in}{3.277143in}}%
\pgfpathlineto{\pgfqpoint{3.768360in}{3.276797in}}%
\pgfpathlineto{\pgfqpoint{3.711794in}{3.473254in}}%
\pgfpathclose%
\pgfusepath{fill}%
\end{pgfscope}%
\begin{pgfscope}%
\pgfpathrectangle{\pgfqpoint{0.680860in}{0.078740in}}{\pgfqpoint{7.842520in}{7.842520in}}%
\pgfusepath{clip}%
\pgfsetbuttcap%
\pgfsetroundjoin%
\definecolor{currentfill}{rgb}{0.360741,0.785964,0.387814}%
\pgfsetfillcolor{currentfill}%
\pgfsetlinewidth{0.000000pt}%
\definecolor{currentstroke}{rgb}{0.175707,0.697900,0.491033}%
\pgfsetstrokecolor{currentstroke}%
\pgfsetdash{}{0pt}%
\pgfpathmoveto{\pgfqpoint{6.752625in}{4.791317in}}%
\pgfpathlineto{\pgfqpoint{6.680668in}{4.871671in}}%
\pgfpathlineto{\pgfqpoint{6.899928in}{4.829425in}}%
\pgfpathclose%
\pgfusepath{fill}%
\end{pgfscope}%
\begin{pgfscope}%
\pgfpathrectangle{\pgfqpoint{0.680860in}{0.078740in}}{\pgfqpoint{7.842520in}{7.842520in}}%
\pgfusepath{clip}%
\pgfsetbuttcap%
\pgfsetroundjoin%
\definecolor{currentfill}{rgb}{0.276194,0.190074,0.493001}%
\pgfsetfillcolor{currentfill}%
\pgfsetlinewidth{0.000000pt}%
\definecolor{currentstroke}{rgb}{0.180653,0.701402,0.488189}%
\pgfsetstrokecolor{currentstroke}%
\pgfsetdash{}{0pt}%
\pgfpathmoveto{\pgfqpoint{2.936638in}{2.184072in}}%
\pgfpathlineto{\pgfqpoint{3.075200in}{2.155846in}}%
\pgfpathlineto{\pgfqpoint{3.019537in}{2.420008in}}%
\pgfpathclose%
\pgfusepath{fill}%
\end{pgfscope}%
\begin{pgfscope}%
\pgfpathrectangle{\pgfqpoint{0.680860in}{0.078740in}}{\pgfqpoint{7.842520in}{7.842520in}}%
\pgfusepath{clip}%
\pgfsetbuttcap%
\pgfsetroundjoin%
\definecolor{currentfill}{rgb}{0.257322,0.256130,0.526563}%
\pgfsetfillcolor{currentfill}%
\pgfsetlinewidth{0.000000pt}%
\definecolor{currentstroke}{rgb}{0.185783,0.704891,0.485273}%
\pgfsetstrokecolor{currentstroke}%
\pgfsetdash{}{0pt}%
\pgfpathmoveto{\pgfqpoint{3.158055in}{2.397340in}}%
\pgfpathlineto{\pgfqpoint{3.240947in}{2.631931in}}%
\pgfpathlineto{\pgfqpoint{3.019537in}{2.420008in}}%
\pgfpathclose%
\pgfusepath{fill}%
\end{pgfscope}%
\begin{pgfscope}%
\pgfpathrectangle{\pgfqpoint{0.680860in}{0.078740in}}{\pgfqpoint{7.842520in}{7.842520in}}%
\pgfusepath{clip}%
\pgfsetbuttcap%
\pgfsetroundjoin%
\definecolor{currentfill}{rgb}{0.327796,0.773980,0.406640}%
\pgfsetfillcolor{currentfill}%
\pgfsetlinewidth{0.000000pt}%
\definecolor{currentstroke}{rgb}{0.191090,0.708366,0.482284}%
\pgfsetstrokecolor{currentstroke}%
\pgfsetdash{}{0pt}%
\pgfpathmoveto{\pgfqpoint{7.265465in}{4.792308in}}%
\pgfpathlineto{\pgfqpoint{7.414642in}{4.832133in}}%
\pgfpathlineto{\pgfqpoint{7.480381in}{4.680637in}}%
\pgfpathclose%
\pgfusepath{fill}%
\end{pgfscope}%
\begin{pgfscope}%
\pgfpathrectangle{\pgfqpoint{0.680860in}{0.078740in}}{\pgfqpoint{7.842520in}{7.842520in}}%
\pgfusepath{clip}%
\pgfsetbuttcap%
\pgfsetroundjoin%
\definecolor{currentfill}{rgb}{0.360741,0.785964,0.387814}%
\pgfsetfillcolor{currentfill}%
\pgfsetlinewidth{0.000000pt}%
\definecolor{currentstroke}{rgb}{0.196571,0.711827,0.479221}%
\pgfsetstrokecolor{currentstroke}%
\pgfsetdash{}{0pt}%
\pgfpathmoveto{\pgfqpoint{6.459709in}{4.882829in}}%
\pgfpathlineto{\pgfqpoint{6.533586in}{4.831170in}}%
\pgfpathlineto{\pgfqpoint{6.387367in}{4.793337in}}%
\pgfpathclose%
\pgfusepath{fill}%
\end{pgfscope}%
\begin{pgfscope}%
\pgfpathrectangle{\pgfqpoint{0.680860in}{0.078740in}}{\pgfqpoint{7.842520in}{7.842520in}}%
\pgfusepath{clip}%
\pgfsetbuttcap%
\pgfsetroundjoin%
\definecolor{currentfill}{rgb}{0.352360,0.783011,0.392636}%
\pgfsetfillcolor{currentfill}%
\pgfsetlinewidth{0.000000pt}%
\definecolor{currentstroke}{rgb}{0.202219,0.715272,0.476084}%
\pgfsetstrokecolor{currentstroke}%
\pgfsetdash{}{0pt}%
\pgfpathmoveto{\pgfqpoint{7.117183in}{4.755126in}}%
\pgfpathlineto{\pgfqpoint{6.899928in}{4.829425in}}%
\pgfpathlineto{\pgfqpoint{7.048113in}{4.870209in}}%
\pgfpathclose%
\pgfusepath{fill}%
\end{pgfscope}%
\begin{pgfscope}%
\pgfpathrectangle{\pgfqpoint{0.680860in}{0.078740in}}{\pgfqpoint{7.842520in}{7.842520in}}%
\pgfusepath{clip}%
\pgfsetbuttcap%
\pgfsetroundjoin%
\definecolor{currentfill}{rgb}{0.197636,0.391528,0.554969}%
\pgfsetfillcolor{currentfill}%
\pgfsetlinewidth{0.000000pt}%
\definecolor{currentstroke}{rgb}{0.208030,0.718701,0.472873}%
\pgfsetstrokecolor{currentstroke}%
\pgfsetdash{}{0pt}%
\pgfpathmoveto{\pgfqpoint{3.462884in}{2.846783in}}%
\pgfpathlineto{\pgfqpoint{3.545835in}{3.067862in}}%
\pgfpathlineto{\pgfqpoint{3.406859in}{3.074121in}}%
\pgfpathclose%
\pgfusepath{fill}%
\end{pgfscope}%
\begin{pgfscope}%
\pgfpathrectangle{\pgfqpoint{0.680860in}{0.078740in}}{\pgfqpoint{7.842520in}{7.842520in}}%
\pgfusepath{clip}%
\pgfsetbuttcap%
\pgfsetroundjoin%
\definecolor{currentfill}{rgb}{0.166383,0.690856,0.496502}%
\pgfsetfillcolor{currentfill}%
\pgfsetlinewidth{0.000000pt}%
\definecolor{currentstroke}{rgb}{0.214000,0.722114,0.469588}%
\pgfsetstrokecolor{currentstroke}%
\pgfsetdash{}{0pt}%
\pgfpathmoveto{\pgfqpoint{4.828122in}{4.328612in}}%
\pgfpathlineto{\pgfqpoint{4.909602in}{4.430686in}}%
\pgfpathlineto{\pgfqpoint{4.686191in}{4.302308in}}%
\pgfpathclose%
\pgfusepath{fill}%
\end{pgfscope}%
\begin{pgfscope}%
\pgfpathrectangle{\pgfqpoint{0.680860in}{0.078740in}}{\pgfqpoint{7.842520in}{7.842520in}}%
\pgfusepath{clip}%
\pgfsetbuttcap%
\pgfsetroundjoin%
\definecolor{currentfill}{rgb}{0.327796,0.773980,0.406640}%
\pgfsetfillcolor{currentfill}%
\pgfsetlinewidth{0.000000pt}%
\definecolor{currentstroke}{rgb}{0.220124,0.725509,0.466226}%
\pgfsetstrokecolor{currentstroke}%
\pgfsetdash{}{0pt}%
\pgfpathmoveto{\pgfqpoint{5.868020in}{4.775811in}}%
\pgfpathlineto{\pgfqpoint{5.945504in}{4.782570in}}%
\pgfpathlineto{\pgfqpoint{5.800838in}{4.745757in}}%
\pgfpathclose%
\pgfusepath{fill}%
\end{pgfscope}%
\begin{pgfscope}%
\pgfpathrectangle{\pgfqpoint{0.680860in}{0.078740in}}{\pgfqpoint{7.842520in}{7.842520in}}%
\pgfusepath{clip}%
\pgfsetbuttcap%
\pgfsetroundjoin%
\definecolor{currentfill}{rgb}{0.157729,0.485932,0.558013}%
\pgfsetfillcolor{currentfill}%
\pgfsetlinewidth{0.000000pt}%
\definecolor{currentstroke}{rgb}{0.226397,0.728888,0.462789}%
\pgfsetstrokecolor{currentstroke}%
\pgfsetdash{}{0pt}%
\pgfpathmoveto{\pgfqpoint{3.711794in}{3.473254in}}%
\pgfpathlineto{\pgfqpoint{3.768360in}{3.276797in}}%
\pgfpathlineto{\pgfqpoint{3.851315in}{3.477272in}}%
\pgfpathclose%
\pgfusepath{fill}%
\end{pgfscope}%
\begin{pgfscope}%
\pgfpathrectangle{\pgfqpoint{0.680860in}{0.078740in}}{\pgfqpoint{7.842520in}{7.842520in}}%
\pgfusepath{clip}%
\pgfsetbuttcap%
\pgfsetroundjoin%
\definecolor{currentfill}{rgb}{0.221989,0.339161,0.548752}%
\pgfsetfillcolor{currentfill}%
\pgfsetlinewidth{0.000000pt}%
\definecolor{currentstroke}{rgb}{0.232815,0.732247,0.459277}%
\pgfsetstrokecolor{currentstroke}%
\pgfsetdash{}{0pt}%
\pgfpathmoveto{\pgfqpoint{3.240947in}{2.631931in}}%
\pgfpathlineto{\pgfqpoint{3.462884in}{2.846783in}}%
\pgfpathlineto{\pgfqpoint{3.323882in}{2.858031in}}%
\pgfpathclose%
\pgfusepath{fill}%
\end{pgfscope}%
\begin{pgfscope}%
\pgfpathrectangle{\pgfqpoint{0.680860in}{0.078740in}}{\pgfqpoint{7.842520in}{7.842520in}}%
\pgfusepath{clip}%
\pgfsetbuttcap%
\pgfsetroundjoin%
\definecolor{currentfill}{rgb}{0.281887,0.150881,0.465405}%
\pgfsetfillcolor{currentfill}%
\pgfsetlinewidth{0.000000pt}%
\definecolor{currentstroke}{rgb}{0.239374,0.735588,0.455688}%
\pgfsetstrokecolor{currentstroke}%
\pgfsetdash{}{0pt}%
\pgfpathmoveto{\pgfqpoint{2.853755in}{1.943006in}}%
\pgfpathlineto{\pgfqpoint{3.075200in}{2.155846in}}%
\pgfpathlineto{\pgfqpoint{2.936638in}{2.184072in}}%
\pgfpathclose%
\pgfusepath{fill}%
\end{pgfscope}%
\begin{pgfscope}%
\pgfpathrectangle{\pgfqpoint{0.680860in}{0.078740in}}{\pgfqpoint{7.842520in}{7.842520in}}%
\pgfusepath{clip}%
\pgfsetbuttcap%
\pgfsetroundjoin%
\definecolor{currentfill}{rgb}{0.270595,0.214069,0.507052}%
\pgfsetfillcolor{currentfill}%
\pgfsetlinewidth{0.000000pt}%
\definecolor{currentstroke}{rgb}{0.246070,0.738910,0.452024}%
\pgfsetstrokecolor{currentstroke}%
\pgfsetdash{}{0pt}%
\pgfpathmoveto{\pgfqpoint{3.075200in}{2.155846in}}%
\pgfpathlineto{\pgfqpoint{3.158055in}{2.397340in}}%
\pgfpathlineto{\pgfqpoint{3.019537in}{2.420008in}}%
\pgfpathclose%
\pgfusepath{fill}%
\end{pgfscope}%
\begin{pgfscope}%
\pgfpathrectangle{\pgfqpoint{0.680860in}{0.078740in}}{\pgfqpoint{7.842520in}{7.842520in}}%
\pgfusepath{clip}%
\pgfsetbuttcap%
\pgfsetroundjoin%
\definecolor{currentfill}{rgb}{0.131172,0.555899,0.552459}%
\pgfsetfillcolor{currentfill}%
\pgfsetlinewidth{0.000000pt}%
\definecolor{currentstroke}{rgb}{0.252899,0.742211,0.448284}%
\pgfsetstrokecolor{currentstroke}%
\pgfsetdash{}{0pt}%
\pgfpathmoveto{\pgfqpoint{3.934233in}{3.662865in}}%
\pgfpathlineto{\pgfqpoint{4.074358in}{3.672388in}}%
\pgfpathlineto{\pgfqpoint{4.157156in}{3.845456in}}%
\pgfpathclose%
\pgfusepath{fill}%
\end{pgfscope}%
\begin{pgfscope}%
\pgfpathrectangle{\pgfqpoint{0.680860in}{0.078740in}}{\pgfqpoint{7.842520in}{7.842520in}}%
\pgfusepath{clip}%
\pgfsetbuttcap%
\pgfsetroundjoin%
\definecolor{currentfill}{rgb}{0.140536,0.530132,0.555659}%
\pgfsetfillcolor{currentfill}%
\pgfsetlinewidth{0.000000pt}%
\definecolor{currentstroke}{rgb}{0.259857,0.745492,0.444467}%
\pgfsetstrokecolor{currentstroke}%
\pgfsetdash{}{0pt}%
\pgfpathmoveto{\pgfqpoint{4.074358in}{3.672388in}}%
\pgfpathlineto{\pgfqpoint{3.934233in}{3.662865in}}%
\pgfpathlineto{\pgfqpoint{3.851315in}{3.477272in}}%
\pgfpathclose%
\pgfusepath{fill}%
\end{pgfscope}%
\begin{pgfscope}%
\pgfpathrectangle{\pgfqpoint{0.680860in}{0.078740in}}{\pgfqpoint{7.842520in}{7.842520in}}%
\pgfusepath{clip}%
\pgfsetbuttcap%
\pgfsetroundjoin%
\definecolor{currentfill}{rgb}{0.327796,0.773980,0.406640}%
\pgfsetfillcolor{currentfill}%
\pgfsetlinewidth{0.000000pt}%
\definecolor{currentstroke}{rgb}{0.266941,0.748751,0.440573}%
\pgfsetstrokecolor{currentstroke}%
\pgfsetdash{}{0pt}%
\pgfpathmoveto{\pgfqpoint{5.800838in}{4.745757in}}%
\pgfpathlineto{\pgfqpoint{5.722997in}{4.736967in}}%
\pgfpathlineto{\pgfqpoint{5.868020in}{4.775811in}}%
\pgfpathclose%
\pgfusepath{fill}%
\end{pgfscope}%
\begin{pgfscope}%
\pgfpathrectangle{\pgfqpoint{0.680860in}{0.078740in}}{\pgfqpoint{7.842520in}{7.842520in}}%
\pgfusepath{clip}%
\pgfsetbuttcap%
\pgfsetroundjoin%
\definecolor{currentfill}{rgb}{0.175841,0.441290,0.557685}%
\pgfsetfillcolor{currentfill}%
\pgfsetlinewidth{0.000000pt}%
\definecolor{currentstroke}{rgb}{0.274149,0.751988,0.436601}%
\pgfsetstrokecolor{currentstroke}%
\pgfsetdash{}{0pt}%
\pgfpathmoveto{\pgfqpoint{3.628812in}{3.277143in}}%
\pgfpathlineto{\pgfqpoint{3.545835in}{3.067862in}}%
\pgfpathlineto{\pgfqpoint{3.768360in}{3.276797in}}%
\pgfpathclose%
\pgfusepath{fill}%
\end{pgfscope}%
\begin{pgfscope}%
\pgfpathrectangle{\pgfqpoint{0.680860in}{0.078740in}}{\pgfqpoint{7.842520in}{7.842520in}}%
\pgfusepath{clip}%
\pgfsetbuttcap%
\pgfsetroundjoin%
\definecolor{currentfill}{rgb}{0.369214,0.788888,0.382914}%
\pgfsetfillcolor{currentfill}%
\pgfsetlinewidth{0.000000pt}%
\definecolor{currentstroke}{rgb}{0.281477,0.755203,0.432552}%
\pgfsetstrokecolor{currentstroke}%
\pgfsetdash{}{0pt}%
\pgfpathmoveto{\pgfqpoint{6.387367in}{4.793337in}}%
\pgfpathlineto{\pgfqpoint{6.312949in}{4.840984in}}%
\pgfpathlineto{\pgfqpoint{6.459709in}{4.882829in}}%
\pgfpathclose%
\pgfusepath{fill}%
\end{pgfscope}%
\begin{pgfscope}%
\pgfpathrectangle{\pgfqpoint{0.680860in}{0.078740in}}{\pgfqpoint{7.842520in}{7.842520in}}%
\pgfusepath{clip}%
\pgfsetbuttcap%
\pgfsetroundjoin%
\definecolor{currentfill}{rgb}{0.360741,0.785964,0.387814}%
\pgfsetfillcolor{currentfill}%
\pgfsetlinewidth{0.000000pt}%
\definecolor{currentstroke}{rgb}{0.288921,0.758394,0.428426}%
\pgfsetstrokecolor{currentstroke}%
\pgfsetdash{}{0pt}%
\pgfpathmoveto{\pgfqpoint{6.167049in}{4.801858in}}%
\pgfpathlineto{\pgfqpoint{6.091000in}{4.822016in}}%
\pgfpathlineto{\pgfqpoint{6.312949in}{4.840984in}}%
\pgfpathclose%
\pgfusepath{fill}%
\end{pgfscope}%
\begin{pgfscope}%
\pgfpathrectangle{\pgfqpoint{0.680860in}{0.078740in}}{\pgfqpoint{7.842520in}{7.842520in}}%
\pgfusepath{clip}%
\pgfsetbuttcap%
\pgfsetroundjoin%
\definecolor{currentfill}{rgb}{0.277018,0.050344,0.375715}%
\pgfsetfillcolor{currentfill}%
\pgfsetlinewidth{0.000000pt}%
\definecolor{currentstroke}{rgb}{0.296479,0.761561,0.424223}%
\pgfsetstrokecolor{currentstroke}%
\pgfsetdash{}{0pt}%
\pgfpathmoveto{\pgfqpoint{2.770862in}{1.698424in}}%
\pgfpathlineto{\pgfqpoint{2.909534in}{1.658738in}}%
\pgfpathlineto{\pgfqpoint{2.853755in}{1.943006in}}%
\pgfpathclose%
\pgfusepath{fill}%
\end{pgfscope}%
\begin{pgfscope}%
\pgfpathrectangle{\pgfqpoint{0.680860in}{0.078740in}}{\pgfqpoint{7.842520in}{7.842520in}}%
\pgfusepath{clip}%
\pgfsetbuttcap%
\pgfsetroundjoin%
\definecolor{currentfill}{rgb}{0.232815,0.732247,0.459277}%
\pgfsetfillcolor{currentfill}%
\pgfsetlinewidth{0.000000pt}%
\definecolor{currentstroke}{rgb}{0.304148,0.764704,0.419943}%
\pgfsetstrokecolor{currentstroke}%
\pgfsetdash{}{0pt}%
\pgfpathmoveto{\pgfqpoint{5.276092in}{4.573915in}}%
\pgfpathlineto{\pgfqpoint{5.132963in}{4.541273in}}%
\pgfpathlineto{\pgfqpoint{5.052142in}{4.460464in}}%
\pgfpathclose%
\pgfusepath{fill}%
\end{pgfscope}%
\begin{pgfscope}%
\pgfpathrectangle{\pgfqpoint{0.680860in}{0.078740in}}{\pgfqpoint{7.842520in}{7.842520in}}%
\pgfusepath{clip}%
\pgfsetbuttcap%
\pgfsetroundjoin%
\definecolor{currentfill}{rgb}{0.123444,0.636809,0.528763}%
\pgfsetfillcolor{currentfill}%
\pgfsetlinewidth{0.000000pt}%
\definecolor{currentstroke}{rgb}{0.311925,0.767822,0.415586}%
\pgfsetstrokecolor{currentstroke}%
\pgfsetdash{}{0pt}%
\pgfpathmoveto{\pgfqpoint{4.380505in}{4.018802in}}%
\pgfpathlineto{\pgfqpoint{4.521892in}{4.038607in}}%
\pgfpathlineto{\pgfqpoint{4.462885in}{4.158325in}}%
\pgfpathclose%
\pgfusepath{fill}%
\end{pgfscope}%
\begin{pgfscope}%
\pgfpathrectangle{\pgfqpoint{0.680860in}{0.078740in}}{\pgfqpoint{7.842520in}{7.842520in}}%
\pgfusepath{clip}%
\pgfsetbuttcap%
\pgfsetroundjoin%
\definecolor{currentfill}{rgb}{0.369214,0.788888,0.382914}%
\pgfsetfillcolor{currentfill}%
\pgfsetlinewidth{0.000000pt}%
\definecolor{currentstroke}{rgb}{0.319809,0.770914,0.411152}%
\pgfsetstrokecolor{currentstroke}%
\pgfsetdash{}{0pt}%
\pgfpathmoveto{\pgfqpoint{7.265465in}{4.792308in}}%
\pgfpathlineto{\pgfqpoint{7.117183in}{4.755126in}}%
\pgfpathlineto{\pgfqpoint{7.197205in}{4.913775in}}%
\pgfpathclose%
\pgfusepath{fill}%
\end{pgfscope}%
\begin{pgfscope}%
\pgfpathrectangle{\pgfqpoint{0.680860in}{0.078740in}}{\pgfqpoint{7.842520in}{7.842520in}}%
\pgfusepath{clip}%
\pgfsetbuttcap%
\pgfsetroundjoin%
\definecolor{currentfill}{rgb}{0.311925,0.767822,0.415586}%
\pgfsetfillcolor{currentfill}%
\pgfsetlinewidth{0.000000pt}%
\definecolor{currentstroke}{rgb}{0.327796,0.773980,0.406640}%
\pgfsetstrokecolor{currentstroke}%
\pgfsetdash{}{0pt}%
\pgfpathmoveto{\pgfqpoint{5.644273in}{4.704157in}}%
\pgfpathlineto{\pgfqpoint{5.722997in}{4.736967in}}%
\pgfpathlineto{\pgfqpoint{5.578794in}{4.700737in}}%
\pgfpathclose%
\pgfusepath{fill}%
\end{pgfscope}%
\begin{pgfscope}%
\pgfpathrectangle{\pgfqpoint{0.680860in}{0.078740in}}{\pgfqpoint{7.842520in}{7.842520in}}%
\pgfusepath{clip}%
\pgfsetbuttcap%
\pgfsetroundjoin%
\definecolor{currentfill}{rgb}{0.283187,0.125848,0.444960}%
\pgfsetfillcolor{currentfill}%
\pgfsetlinewidth{0.000000pt}%
\definecolor{currentstroke}{rgb}{0.335885,0.777018,0.402049}%
\pgfsetstrokecolor{currentstroke}%
\pgfsetdash{}{0pt}%
\pgfpathmoveto{\pgfqpoint{2.853755in}{1.943006in}}%
\pgfpathlineto{\pgfqpoint{2.992368in}{1.909088in}}%
\pgfpathlineto{\pgfqpoint{3.075200in}{2.155846in}}%
\pgfpathclose%
\pgfusepath{fill}%
\end{pgfscope}%
\begin{pgfscope}%
\pgfpathrectangle{\pgfqpoint{0.680860in}{0.078740in}}{\pgfqpoint{7.842520in}{7.842520in}}%
\pgfusepath{clip}%
\pgfsetbuttcap%
\pgfsetroundjoin%
\definecolor{currentfill}{rgb}{0.120092,0.600104,0.542530}%
\pgfsetfillcolor{currentfill}%
\pgfsetlinewidth{0.000000pt}%
\definecolor{currentstroke}{rgb}{0.344074,0.780029,0.397381}%
\pgfsetstrokecolor{currentstroke}%
\pgfsetdash{}{0pt}%
\pgfpathmoveto{\pgfqpoint{4.157156in}{3.845456in}}%
\pgfpathlineto{\pgfqpoint{4.297905in}{3.860268in}}%
\pgfpathlineto{\pgfqpoint{4.380505in}{4.018802in}}%
\pgfpathclose%
\pgfusepath{fill}%
\end{pgfscope}%
\begin{pgfscope}%
\pgfpathrectangle{\pgfqpoint{0.680860in}{0.078740in}}{\pgfqpoint{7.842520in}{7.842520in}}%
\pgfusepath{clip}%
\pgfsetbuttcap%
\pgfsetroundjoin%
\definecolor{currentfill}{rgb}{0.253935,0.265254,0.529983}%
\pgfsetfillcolor{currentfill}%
\pgfsetlinewidth{0.000000pt}%
\definecolor{currentstroke}{rgb}{0.352360,0.783011,0.392636}%
\pgfsetstrokecolor{currentstroke}%
\pgfsetdash{}{0pt}%
\pgfpathmoveto{\pgfqpoint{3.297105in}{2.375252in}}%
\pgfpathlineto{\pgfqpoint{3.240947in}{2.631931in}}%
\pgfpathlineto{\pgfqpoint{3.158055in}{2.397340in}}%
\pgfpathclose%
\pgfusepath{fill}%
\end{pgfscope}%
\begin{pgfscope}%
\pgfpathrectangle{\pgfqpoint{0.680860in}{0.078740in}}{\pgfqpoint{7.842520in}{7.842520in}}%
\pgfusepath{clip}%
\pgfsetbuttcap%
\pgfsetroundjoin%
\definecolor{currentfill}{rgb}{0.229739,0.322361,0.545706}%
\pgfsetfillcolor{currentfill}%
\pgfsetlinewidth{0.000000pt}%
\definecolor{currentstroke}{rgb}{0.360741,0.785964,0.387814}%
\pgfsetstrokecolor{currentstroke}%
\pgfsetdash{}{0pt}%
\pgfpathmoveto{\pgfqpoint{3.379972in}{2.615388in}}%
\pgfpathlineto{\pgfqpoint{3.462884in}{2.846783in}}%
\pgfpathlineto{\pgfqpoint{3.240947in}{2.631931in}}%
\pgfpathclose%
\pgfusepath{fill}%
\end{pgfscope}%
\begin{pgfscope}%
\pgfpathrectangle{\pgfqpoint{0.680860in}{0.078740in}}{\pgfqpoint{7.842520in}{7.842520in}}%
\pgfusepath{clip}%
\pgfsetbuttcap%
\pgfsetroundjoin%
\definecolor{currentfill}{rgb}{0.128087,0.647749,0.523491}%
\pgfsetfillcolor{currentfill}%
\pgfsetlinewidth{0.000000pt}%
\definecolor{currentstroke}{rgb}{0.369214,0.788888,0.382914}%
\pgfsetstrokecolor{currentstroke}%
\pgfsetdash{}{0pt}%
\pgfpathmoveto{\pgfqpoint{4.462885in}{4.158325in}}%
\pgfpathlineto{\pgfqpoint{4.521892in}{4.038607in}}%
\pgfpathlineto{\pgfqpoint{4.604197in}{4.180624in}}%
\pgfpathclose%
\pgfusepath{fill}%
\end{pgfscope}%
\begin{pgfscope}%
\pgfpathrectangle{\pgfqpoint{0.680860in}{0.078740in}}{\pgfqpoint{7.842520in}{7.842520in}}%
\pgfusepath{clip}%
\pgfsetbuttcap%
\pgfsetroundjoin%
\definecolor{currentfill}{rgb}{0.304148,0.764704,0.419943}%
\pgfsetfillcolor{currentfill}%
\pgfsetlinewidth{0.000000pt}%
\definecolor{currentstroke}{rgb}{0.377779,0.791781,0.377939}%
\pgfsetstrokecolor{currentstroke}%
\pgfsetdash{}{0pt}%
\pgfpathmoveto{\pgfqpoint{5.578794in}{4.700737in}}%
\pgfpathlineto{\pgfqpoint{5.499781in}{4.666774in}}%
\pgfpathlineto{\pgfqpoint{5.644273in}{4.704157in}}%
\pgfpathclose%
\pgfusepath{fill}%
\end{pgfscope}%
\begin{pgfscope}%
\pgfpathrectangle{\pgfqpoint{0.680860in}{0.078740in}}{\pgfqpoint{7.842520in}{7.842520in}}%
\pgfusepath{clip}%
\pgfsetbuttcap%
\pgfsetroundjoin%
\definecolor{currentfill}{rgb}{0.280894,0.078907,0.402329}%
\pgfsetfillcolor{currentfill}%
\pgfsetlinewidth{0.000000pt}%
\definecolor{currentstroke}{rgb}{0.386433,0.794644,0.372886}%
\pgfsetstrokecolor{currentstroke}%
\pgfsetdash{}{0pt}%
\pgfpathmoveto{\pgfqpoint{2.853755in}{1.943006in}}%
\pgfpathlineto{\pgfqpoint{2.909534in}{1.658738in}}%
\pgfpathlineto{\pgfqpoint{2.992368in}{1.909088in}}%
\pgfpathclose%
\pgfusepath{fill}%
\end{pgfscope}%
\begin{pgfscope}%
\pgfpathrectangle{\pgfqpoint{0.680860in}{0.078740in}}{\pgfqpoint{7.842520in}{7.842520in}}%
\pgfusepath{clip}%
\pgfsetbuttcap%
\pgfsetroundjoin%
\definecolor{currentfill}{rgb}{0.194100,0.399323,0.555565}%
\pgfsetfillcolor{currentfill}%
\pgfsetlinewidth{0.000000pt}%
\definecolor{currentstroke}{rgb}{0.395174,0.797475,0.367757}%
\pgfsetstrokecolor{currentstroke}%
\pgfsetdash{}{0pt}%
\pgfpathmoveto{\pgfqpoint{3.685400in}{3.062729in}}%
\pgfpathlineto{\pgfqpoint{3.545835in}{3.067862in}}%
\pgfpathlineto{\pgfqpoint{3.462884in}{2.846783in}}%
\pgfpathclose%
\pgfusepath{fill}%
\end{pgfscope}%
\begin{pgfscope}%
\pgfpathrectangle{\pgfqpoint{0.680860in}{0.078740in}}{\pgfqpoint{7.842520in}{7.842520in}}%
\pgfusepath{clip}%
\pgfsetbuttcap%
\pgfsetroundjoin%
\definecolor{currentfill}{rgb}{0.146616,0.673050,0.508936}%
\pgfsetfillcolor{currentfill}%
\pgfsetlinewidth{0.000000pt}%
\definecolor{currentstroke}{rgb}{0.404001,0.800275,0.362552}%
\pgfsetstrokecolor{currentstroke}%
\pgfsetdash{}{0pt}%
\pgfpathmoveto{\pgfqpoint{4.686191in}{4.302308in}}%
\pgfpathlineto{\pgfqpoint{4.604197in}{4.180624in}}%
\pgfpathlineto{\pgfqpoint{4.746229in}{4.205041in}}%
\pgfpathclose%
\pgfusepath{fill}%
\end{pgfscope}%
\begin{pgfscope}%
\pgfpathrectangle{\pgfqpoint{0.680860in}{0.078740in}}{\pgfqpoint{7.842520in}{7.842520in}}%
\pgfusepath{clip}%
\pgfsetbuttcap%
\pgfsetroundjoin%
\definecolor{currentfill}{rgb}{0.288921,0.758394,0.428426}%
\pgfsetfillcolor{currentfill}%
\pgfsetlinewidth{0.000000pt}%
\definecolor{currentstroke}{rgb}{0.412913,0.803041,0.357269}%
\pgfsetstrokecolor{currentstroke}%
\pgfsetdash{}{0pt}%
\pgfpathmoveto{\pgfqpoint{5.499781in}{4.666774in}}%
\pgfpathlineto{\pgfqpoint{5.356094in}{4.631957in}}%
\pgfpathlineto{\pgfqpoint{5.420007in}{4.609045in}}%
\pgfpathclose%
\pgfusepath{fill}%
\end{pgfscope}%
\begin{pgfscope}%
\pgfpathrectangle{\pgfqpoint{0.680860in}{0.078740in}}{\pgfqpoint{7.842520in}{7.842520in}}%
\pgfusepath{clip}%
\pgfsetbuttcap%
\pgfsetroundjoin%
\definecolor{currentfill}{rgb}{0.244972,0.287675,0.537260}%
\pgfsetfillcolor{currentfill}%
\pgfsetlinewidth{0.000000pt}%
\definecolor{currentstroke}{rgb}{0.421908,0.805774,0.351910}%
\pgfsetstrokecolor{currentstroke}%
\pgfsetdash{}{0pt}%
\pgfpathmoveto{\pgfqpoint{3.379972in}{2.615388in}}%
\pgfpathlineto{\pgfqpoint{3.240947in}{2.631931in}}%
\pgfpathlineto{\pgfqpoint{3.297105in}{2.375252in}}%
\pgfpathclose%
\pgfusepath{fill}%
\end{pgfscope}%
\begin{pgfscope}%
\pgfpathrectangle{\pgfqpoint{0.680860in}{0.078740in}}{\pgfqpoint{7.842520in}{7.842520in}}%
\pgfusepath{clip}%
\pgfsetbuttcap%
\pgfsetroundjoin%
\definecolor{currentfill}{rgb}{0.180629,0.429975,0.557282}%
\pgfsetfillcolor{currentfill}%
\pgfsetlinewidth{0.000000pt}%
\definecolor{currentstroke}{rgb}{0.430983,0.808473,0.346476}%
\pgfsetstrokecolor{currentstroke}%
\pgfsetdash{}{0pt}%
\pgfpathmoveto{\pgfqpoint{3.768360in}{3.276797in}}%
\pgfpathlineto{\pgfqpoint{3.545835in}{3.067862in}}%
\pgfpathlineto{\pgfqpoint{3.685400in}{3.062729in}}%
\pgfpathclose%
\pgfusepath{fill}%
\end{pgfscope}%
\begin{pgfscope}%
\pgfpathrectangle{\pgfqpoint{0.680860in}{0.078740in}}{\pgfqpoint{7.842520in}{7.842520in}}%
\pgfusepath{clip}%
\pgfsetbuttcap%
\pgfsetroundjoin%
\definecolor{currentfill}{rgb}{0.157851,0.683765,0.501686}%
\pgfsetfillcolor{currentfill}%
\pgfsetlinewidth{0.000000pt}%
\definecolor{currentstroke}{rgb}{0.440137,0.811138,0.340967}%
\pgfsetstrokecolor{currentstroke}%
\pgfsetdash{}{0pt}%
\pgfpathmoveto{\pgfqpoint{4.686191in}{4.302308in}}%
\pgfpathlineto{\pgfqpoint{4.746229in}{4.205041in}}%
\pgfpathlineto{\pgfqpoint{4.828122in}{4.328612in}}%
\pgfpathclose%
\pgfusepath{fill}%
\end{pgfscope}%
\begin{pgfscope}%
\pgfpathrectangle{\pgfqpoint{0.680860in}{0.078740in}}{\pgfqpoint{7.842520in}{7.842520in}}%
\pgfusepath{clip}%
\pgfsetbuttcap%
\pgfsetroundjoin%
\definecolor{currentfill}{rgb}{0.143343,0.522773,0.556295}%
\pgfsetfillcolor{currentfill}%
\pgfsetlinewidth{0.000000pt}%
\definecolor{currentstroke}{rgb}{0.449368,0.813768,0.335384}%
\pgfsetstrokecolor{currentstroke}%
\pgfsetdash{}{0pt}%
\pgfpathmoveto{\pgfqpoint{3.851315in}{3.477272in}}%
\pgfpathlineto{\pgfqpoint{3.991467in}{3.482768in}}%
\pgfpathlineto{\pgfqpoint{4.074358in}{3.672388in}}%
\pgfpathclose%
\pgfusepath{fill}%
\end{pgfscope}%
\begin{pgfscope}%
\pgfpathrectangle{\pgfqpoint{0.680860in}{0.078740in}}{\pgfqpoint{7.842520in}{7.842520in}}%
\pgfusepath{clip}%
\pgfsetbuttcap%
\pgfsetroundjoin%
\definecolor{currentfill}{rgb}{0.274149,0.751988,0.436601}%
\pgfsetfillcolor{currentfill}%
\pgfsetlinewidth{0.000000pt}%
\definecolor{currentstroke}{rgb}{0.458674,0.816363,0.329727}%
\pgfsetstrokecolor{currentstroke}%
\pgfsetdash{}{0pt}%
\pgfpathmoveto{\pgfqpoint{5.420007in}{4.609045in}}%
\pgfpathlineto{\pgfqpoint{5.356094in}{4.631957in}}%
\pgfpathlineto{\pgfqpoint{5.276092in}{4.573915in}}%
\pgfpathclose%
\pgfusepath{fill}%
\end{pgfscope}%
\begin{pgfscope}%
\pgfpathrectangle{\pgfqpoint{0.680860in}{0.078740in}}{\pgfqpoint{7.842520in}{7.842520in}}%
\pgfusepath{clip}%
\pgfsetbuttcap%
\pgfsetroundjoin%
\definecolor{currentfill}{rgb}{0.154815,0.493313,0.557840}%
\pgfsetfillcolor{currentfill}%
\pgfsetlinewidth{0.000000pt}%
\definecolor{currentstroke}{rgb}{0.468053,0.818921,0.323998}%
\pgfsetstrokecolor{currentstroke}%
\pgfsetdash{}{0pt}%
\pgfpathmoveto{\pgfqpoint{3.768360in}{3.276797in}}%
\pgfpathlineto{\pgfqpoint{3.991467in}{3.482768in}}%
\pgfpathlineto{\pgfqpoint{3.851315in}{3.477272in}}%
\pgfpathclose%
\pgfusepath{fill}%
\end{pgfscope}%
\begin{pgfscope}%
\pgfpathrectangle{\pgfqpoint{0.680860in}{0.078740in}}{\pgfqpoint{7.842520in}{7.842520in}}%
\pgfusepath{clip}%
\pgfsetbuttcap%
\pgfsetroundjoin%
\definecolor{currentfill}{rgb}{0.274128,0.199721,0.498911}%
\pgfsetfillcolor{currentfill}%
\pgfsetlinewidth{0.000000pt}%
\definecolor{currentstroke}{rgb}{0.477504,0.821444,0.318195}%
\pgfsetstrokecolor{currentstroke}%
\pgfsetdash{}{0pt}%
\pgfpathmoveto{\pgfqpoint{3.214279in}{2.128019in}}%
\pgfpathlineto{\pgfqpoint{3.158055in}{2.397340in}}%
\pgfpathlineto{\pgfqpoint{3.075200in}{2.155846in}}%
\pgfpathclose%
\pgfusepath{fill}%
\end{pgfscope}%
\begin{pgfscope}%
\pgfpathrectangle{\pgfqpoint{0.680860in}{0.078740in}}{\pgfqpoint{7.842520in}{7.842520in}}%
\pgfusepath{clip}%
\pgfsetbuttcap%
\pgfsetroundjoin%
\definecolor{currentfill}{rgb}{0.185783,0.704891,0.485273}%
\pgfsetfillcolor{currentfill}%
\pgfsetlinewidth{0.000000pt}%
\definecolor{currentstroke}{rgb}{0.487026,0.823929,0.312321}%
\pgfsetstrokecolor{currentstroke}%
\pgfsetdash{}{0pt}%
\pgfpathmoveto{\pgfqpoint{4.970796in}{4.357177in}}%
\pgfpathlineto{\pgfqpoint{4.909602in}{4.430686in}}%
\pgfpathlineto{\pgfqpoint{4.828122in}{4.328612in}}%
\pgfpathclose%
\pgfusepath{fill}%
\end{pgfscope}%
\begin{pgfscope}%
\pgfpathrectangle{\pgfqpoint{0.680860in}{0.078740in}}{\pgfqpoint{7.842520in}{7.842520in}}%
\pgfusepath{clip}%
\pgfsetbuttcap%
\pgfsetroundjoin%
\definecolor{currentfill}{rgb}{0.202219,0.715272,0.476084}%
\pgfsetfillcolor{currentfill}%
\pgfsetlinewidth{0.000000pt}%
\definecolor{currentstroke}{rgb}{0.496615,0.826376,0.306377}%
\pgfsetstrokecolor{currentstroke}%
\pgfsetdash{}{0pt}%
\pgfpathmoveto{\pgfqpoint{4.970796in}{4.357177in}}%
\pgfpathlineto{\pgfqpoint{5.052142in}{4.460464in}}%
\pgfpathlineto{\pgfqpoint{4.909602in}{4.430686in}}%
\pgfpathclose%
\pgfusepath{fill}%
\end{pgfscope}%
\begin{pgfscope}%
\pgfpathrectangle{\pgfqpoint{0.680860in}{0.078740in}}{\pgfqpoint{7.842520in}{7.842520in}}%
\pgfusepath{clip}%
\pgfsetbuttcap%
\pgfsetroundjoin%
\definecolor{currentfill}{rgb}{0.127568,0.566949,0.550556}%
\pgfsetfillcolor{currentfill}%
\pgfsetlinewidth{0.000000pt}%
\definecolor{currentstroke}{rgb}{0.506271,0.828786,0.300362}%
\pgfsetstrokecolor{currentstroke}%
\pgfsetdash{}{0pt}%
\pgfpathmoveto{\pgfqpoint{4.157156in}{3.845456in}}%
\pgfpathlineto{\pgfqpoint{4.074358in}{3.672388in}}%
\pgfpathlineto{\pgfqpoint{4.215140in}{3.683583in}}%
\pgfpathclose%
\pgfusepath{fill}%
\end{pgfscope}%
\begin{pgfscope}%
\pgfpathrectangle{\pgfqpoint{0.680860in}{0.078740in}}{\pgfqpoint{7.842520in}{7.842520in}}%
\pgfusepath{clip}%
\pgfsetbuttcap%
\pgfsetroundjoin%
\definecolor{currentfill}{rgb}{0.395174,0.797475,0.367757}%
\pgfsetfillcolor{currentfill}%
\pgfsetlinewidth{0.000000pt}%
\definecolor{currentstroke}{rgb}{0.515992,0.831158,0.294279}%
\pgfsetstrokecolor{currentstroke}%
\pgfsetdash{}{0pt}%
\pgfpathmoveto{\pgfqpoint{7.197205in}{4.913775in}}%
\pgfpathlineto{\pgfqpoint{7.117183in}{4.755126in}}%
\pgfpathlineto{\pgfqpoint{7.048113in}{4.870209in}}%
\pgfpathclose%
\pgfusepath{fill}%
\end{pgfscope}%
\begin{pgfscope}%
\pgfpathrectangle{\pgfqpoint{0.680860in}{0.078740in}}{\pgfqpoint{7.842520in}{7.842520in}}%
\pgfusepath{clip}%
\pgfsetbuttcap%
\pgfsetroundjoin%
\definecolor{currentfill}{rgb}{0.360741,0.785964,0.387814}%
\pgfsetfillcolor{currentfill}%
\pgfsetlinewidth{0.000000pt}%
\definecolor{currentstroke}{rgb}{0.525776,0.833491,0.288127}%
\pgfsetstrokecolor{currentstroke}%
\pgfsetdash{}{0pt}%
\pgfpathmoveto{\pgfqpoint{5.945504in}{4.782570in}}%
\pgfpathlineto{\pgfqpoint{5.868020in}{4.775811in}}%
\pgfpathlineto{\pgfqpoint{6.091000in}{4.822016in}}%
\pgfpathclose%
\pgfusepath{fill}%
\end{pgfscope}%
\begin{pgfscope}%
\pgfpathrectangle{\pgfqpoint{0.680860in}{0.078740in}}{\pgfqpoint{7.842520in}{7.842520in}}%
\pgfusepath{clip}%
\pgfsetbuttcap%
\pgfsetroundjoin%
\definecolor{currentfill}{rgb}{0.281412,0.155834,0.469201}%
\pgfsetfillcolor{currentfill}%
\pgfsetlinewidth{0.000000pt}%
\definecolor{currentstroke}{rgb}{0.535621,0.835785,0.281908}%
\pgfsetstrokecolor{currentstroke}%
\pgfsetdash{}{0pt}%
\pgfpathmoveto{\pgfqpoint{3.075200in}{2.155846in}}%
\pgfpathlineto{\pgfqpoint{2.992368in}{1.909088in}}%
\pgfpathlineto{\pgfqpoint{3.214279in}{2.128019in}}%
\pgfpathclose%
\pgfusepath{fill}%
\end{pgfscope}%
\begin{pgfscope}%
\pgfpathrectangle{\pgfqpoint{0.680860in}{0.078740in}}{\pgfqpoint{7.842520in}{7.842520in}}%
\pgfusepath{clip}%
\pgfsetbuttcap%
\pgfsetroundjoin%
\definecolor{currentfill}{rgb}{0.123463,0.581687,0.547445}%
\pgfsetfillcolor{currentfill}%
\pgfsetlinewidth{0.000000pt}%
\definecolor{currentstroke}{rgb}{0.545524,0.838039,0.275626}%
\pgfsetstrokecolor{currentstroke}%
\pgfsetdash{}{0pt}%
\pgfpathmoveto{\pgfqpoint{4.215140in}{3.683583in}}%
\pgfpathlineto{\pgfqpoint{4.297905in}{3.860268in}}%
\pgfpathlineto{\pgfqpoint{4.157156in}{3.845456in}}%
\pgfpathclose%
\pgfusepath{fill}%
\end{pgfscope}%
\begin{pgfscope}%
\pgfpathrectangle{\pgfqpoint{0.680860in}{0.078740in}}{\pgfqpoint{7.842520in}{7.842520in}}%
\pgfusepath{clip}%
\pgfsetbuttcap%
\pgfsetroundjoin%
\definecolor{currentfill}{rgb}{0.267968,0.223549,0.512008}%
\pgfsetfillcolor{currentfill}%
\pgfsetlinewidth{0.000000pt}%
\definecolor{currentstroke}{rgb}{0.555484,0.840254,0.269281}%
\pgfsetstrokecolor{currentstroke}%
\pgfsetdash{}{0pt}%
\pgfpathmoveto{\pgfqpoint{3.214279in}{2.128019in}}%
\pgfpathlineto{\pgfqpoint{3.297105in}{2.375252in}}%
\pgfpathlineto{\pgfqpoint{3.158055in}{2.397340in}}%
\pgfpathclose%
\pgfusepath{fill}%
\end{pgfscope}%
\begin{pgfscope}%
\pgfpathrectangle{\pgfqpoint{0.680860in}{0.078740in}}{\pgfqpoint{7.842520in}{7.842520in}}%
\pgfusepath{clip}%
\pgfsetbuttcap%
\pgfsetroundjoin%
\definecolor{currentfill}{rgb}{0.120081,0.622161,0.534946}%
\pgfsetfillcolor{currentfill}%
\pgfsetlinewidth{0.000000pt}%
\definecolor{currentstroke}{rgb}{0.565498,0.842430,0.262877}%
\pgfsetstrokecolor{currentstroke}%
\pgfsetdash{}{0pt}%
\pgfpathmoveto{\pgfqpoint{4.380505in}{4.018802in}}%
\pgfpathlineto{\pgfqpoint{4.297905in}{3.860268in}}%
\pgfpathlineto{\pgfqpoint{4.521892in}{4.038607in}}%
\pgfpathclose%
\pgfusepath{fill}%
\end{pgfscope}%
\begin{pgfscope}%
\pgfpathrectangle{\pgfqpoint{0.680860in}{0.078740in}}{\pgfqpoint{7.842520in}{7.842520in}}%
\pgfusepath{clip}%
\pgfsetbuttcap%
\pgfsetroundjoin%
\definecolor{currentfill}{rgb}{0.412913,0.803041,0.357269}%
\pgfsetfillcolor{currentfill}%
\pgfsetlinewidth{0.000000pt}%
\definecolor{currentstroke}{rgb}{0.575563,0.844566,0.256415}%
\pgfsetstrokecolor{currentstroke}%
\pgfsetdash{}{0pt}%
\pgfpathmoveto{\pgfqpoint{6.899928in}{4.829425in}}%
\pgfpathlineto{\pgfqpoint{6.680668in}{4.871671in}}%
\pgfpathlineto{\pgfqpoint{6.828636in}{4.914943in}}%
\pgfpathclose%
\pgfusepath{fill}%
\end{pgfscope}%
\begin{pgfscope}%
\pgfpathrectangle{\pgfqpoint{0.680860in}{0.078740in}}{\pgfqpoint{7.842520in}{7.842520in}}%
\pgfusepath{clip}%
\pgfsetbuttcap%
\pgfsetroundjoin%
\definecolor{currentfill}{rgb}{0.201239,0.383670,0.554294}%
\pgfsetfillcolor{currentfill}%
\pgfsetlinewidth{0.000000pt}%
\definecolor{currentstroke}{rgb}{0.585678,0.846661,0.249897}%
\pgfsetstrokecolor{currentstroke}%
\pgfsetdash{}{0pt}%
\pgfpathmoveto{\pgfqpoint{3.462884in}{2.846783in}}%
\pgfpathlineto{\pgfqpoint{3.602460in}{2.836495in}}%
\pgfpathlineto{\pgfqpoint{3.685400in}{3.062729in}}%
\pgfpathclose%
\pgfusepath{fill}%
\end{pgfscope}%
\begin{pgfscope}%
\pgfpathrectangle{\pgfqpoint{0.680860in}{0.078740in}}{\pgfqpoint{7.842520in}{7.842520in}}%
\pgfusepath{clip}%
\pgfsetbuttcap%
\pgfsetroundjoin%
\definecolor{currentfill}{rgb}{0.225863,0.330805,0.547314}%
\pgfsetfillcolor{currentfill}%
\pgfsetlinewidth{0.000000pt}%
\definecolor{currentstroke}{rgb}{0.595839,0.848717,0.243329}%
\pgfsetstrokecolor{currentstroke}%
\pgfsetdash{}{0pt}%
\pgfpathmoveto{\pgfqpoint{3.519555in}{2.599630in}}%
\pgfpathlineto{\pgfqpoint{3.462884in}{2.846783in}}%
\pgfpathlineto{\pgfqpoint{3.379972in}{2.615388in}}%
\pgfpathclose%
\pgfusepath{fill}%
\end{pgfscope}%
\begin{pgfscope}%
\pgfpathrectangle{\pgfqpoint{0.680860in}{0.078740in}}{\pgfqpoint{7.842520in}{7.842520in}}%
\pgfusepath{clip}%
\pgfsetbuttcap%
\pgfsetroundjoin%
\definecolor{currentfill}{rgb}{0.281446,0.084320,0.407414}%
\pgfsetfillcolor{currentfill}%
\pgfsetlinewidth{0.000000pt}%
\definecolor{currentstroke}{rgb}{0.606045,0.850733,0.236712}%
\pgfsetstrokecolor{currentstroke}%
\pgfsetdash{}{0pt}%
\pgfpathmoveto{\pgfqpoint{3.131480in}{1.875386in}}%
\pgfpathlineto{\pgfqpoint{2.992368in}{1.909088in}}%
\pgfpathlineto{\pgfqpoint{2.909534in}{1.658738in}}%
\pgfpathclose%
\pgfusepath{fill}%
\end{pgfscope}%
\begin{pgfscope}%
\pgfpathrectangle{\pgfqpoint{0.680860in}{0.078740in}}{\pgfqpoint{7.842520in}{7.842520in}}%
\pgfusepath{clip}%
\pgfsetbuttcap%
\pgfsetroundjoin%
\definecolor{currentfill}{rgb}{0.246070,0.738910,0.452024}%
\pgfsetfillcolor{currentfill}%
\pgfsetlinewidth{0.000000pt}%
\definecolor{currentstroke}{rgb}{0.616293,0.852709,0.230052}%
\pgfsetstrokecolor{currentstroke}%
\pgfsetdash{}{0pt}%
\pgfpathmoveto{\pgfqpoint{5.052142in}{4.460464in}}%
\pgfpathlineto{\pgfqpoint{5.195448in}{4.492628in}}%
\pgfpathlineto{\pgfqpoint{5.276092in}{4.573915in}}%
\pgfpathclose%
\pgfusepath{fill}%
\end{pgfscope}%
\begin{pgfscope}%
\pgfpathrectangle{\pgfqpoint{0.680860in}{0.078740in}}{\pgfqpoint{7.842520in}{7.842520in}}%
\pgfusepath{clip}%
\pgfsetbuttcap%
\pgfsetroundjoin%
\definecolor{currentfill}{rgb}{0.421908,0.805774,0.351910}%
\pgfsetfillcolor{currentfill}%
\pgfsetlinewidth{0.000000pt}%
\definecolor{currentstroke}{rgb}{0.626579,0.854645,0.223353}%
\pgfsetstrokecolor{currentstroke}%
\pgfsetdash{}{0pt}%
\pgfpathmoveto{\pgfqpoint{6.607353in}{4.927499in}}%
\pgfpathlineto{\pgfqpoint{6.680668in}{4.871671in}}%
\pgfpathlineto{\pgfqpoint{6.533586in}{4.831170in}}%
\pgfpathclose%
\pgfusepath{fill}%
\end{pgfscope}%
\begin{pgfscope}%
\pgfpathrectangle{\pgfqpoint{0.680860in}{0.078740in}}{\pgfqpoint{7.842520in}{7.842520in}}%
\pgfusepath{clip}%
\pgfsetbuttcap%
\pgfsetroundjoin%
\definecolor{currentfill}{rgb}{0.404001,0.800275,0.362552}%
\pgfsetfillcolor{currentfill}%
\pgfsetlinewidth{0.000000pt}%
\definecolor{currentstroke}{rgb}{0.636902,0.856542,0.216620}%
\pgfsetstrokecolor{currentstroke}%
\pgfsetdash{}{0pt}%
\pgfpathmoveto{\pgfqpoint{7.414642in}{4.832133in}}%
\pgfpathlineto{\pgfqpoint{7.265465in}{4.792308in}}%
\pgfpathlineto{\pgfqpoint{7.197205in}{4.913775in}}%
\pgfpathclose%
\pgfusepath{fill}%
\end{pgfscope}%
\begin{pgfscope}%
\pgfpathrectangle{\pgfqpoint{0.680860in}{0.078740in}}{\pgfqpoint{7.842520in}{7.842520in}}%
\pgfusepath{clip}%
\pgfsetbuttcap%
\pgfsetroundjoin%
\definecolor{currentfill}{rgb}{0.157729,0.485932,0.558013}%
\pgfsetfillcolor{currentfill}%
\pgfsetlinewidth{0.000000pt}%
\definecolor{currentstroke}{rgb}{0.647257,0.858400,0.209861}%
\pgfsetstrokecolor{currentstroke}%
\pgfsetdash{}{0pt}%
\pgfpathmoveto{\pgfqpoint{3.908524in}{3.277781in}}%
\pgfpathlineto{\pgfqpoint{3.991467in}{3.482768in}}%
\pgfpathlineto{\pgfqpoint{3.768360in}{3.276797in}}%
\pgfpathclose%
\pgfusepath{fill}%
\end{pgfscope}%
\begin{pgfscope}%
\pgfpathrectangle{\pgfqpoint{0.680860in}{0.078740in}}{\pgfqpoint{7.842520in}{7.842520in}}%
\pgfusepath{clip}%
\pgfsetbuttcap%
\pgfsetroundjoin%
\definecolor{currentfill}{rgb}{0.137339,0.662252,0.515571}%
\pgfsetfillcolor{currentfill}%
\pgfsetlinewidth{0.000000pt}%
\definecolor{currentstroke}{rgb}{0.657642,0.860219,0.203082}%
\pgfsetstrokecolor{currentstroke}%
\pgfsetdash{}{0pt}%
\pgfpathmoveto{\pgfqpoint{4.746229in}{4.205041in}}%
\pgfpathlineto{\pgfqpoint{4.604197in}{4.180624in}}%
\pgfpathlineto{\pgfqpoint{4.521892in}{4.038607in}}%
\pgfpathclose%
\pgfusepath{fill}%
\end{pgfscope}%
\begin{pgfscope}%
\pgfpathrectangle{\pgfqpoint{0.680860in}{0.078740in}}{\pgfqpoint{7.842520in}{7.842520in}}%
\pgfusepath{clip}%
\pgfsetbuttcap%
\pgfsetroundjoin%
\definecolor{currentfill}{rgb}{0.283072,0.130895,0.449241}%
\pgfsetfillcolor{currentfill}%
\pgfsetlinewidth{0.000000pt}%
\definecolor{currentstroke}{rgb}{0.668054,0.861999,0.196293}%
\pgfsetstrokecolor{currentstroke}%
\pgfsetdash{}{0pt}%
\pgfpathmoveto{\pgfqpoint{3.214279in}{2.128019in}}%
\pgfpathlineto{\pgfqpoint{2.992368in}{1.909088in}}%
\pgfpathlineto{\pgfqpoint{3.131480in}{1.875386in}}%
\pgfpathclose%
\pgfusepath{fill}%
\end{pgfscope}%
\begin{pgfscope}%
\pgfpathrectangle{\pgfqpoint{0.680860in}{0.078740in}}{\pgfqpoint{7.842520in}{7.842520in}}%
\pgfusepath{clip}%
\pgfsetbuttcap%
\pgfsetroundjoin%
\definecolor{currentfill}{rgb}{0.243113,0.292092,0.538516}%
\pgfsetfillcolor{currentfill}%
\pgfsetlinewidth{0.000000pt}%
\definecolor{currentstroke}{rgb}{0.678489,0.863742,0.189503}%
\pgfsetstrokecolor{currentstroke}%
\pgfsetdash{}{0pt}%
\pgfpathmoveto{\pgfqpoint{3.379972in}{2.615388in}}%
\pgfpathlineto{\pgfqpoint{3.297105in}{2.375252in}}%
\pgfpathlineto{\pgfqpoint{3.519555in}{2.599630in}}%
\pgfpathclose%
\pgfusepath{fill}%
\end{pgfscope}%
\begin{pgfscope}%
\pgfpathrectangle{\pgfqpoint{0.680860in}{0.078740in}}{\pgfqpoint{7.842520in}{7.842520in}}%
\pgfusepath{clip}%
\pgfsetbuttcap%
\pgfsetroundjoin%
\definecolor{currentfill}{rgb}{0.216210,0.351535,0.550627}%
\pgfsetfillcolor{currentfill}%
\pgfsetlinewidth{0.000000pt}%
\definecolor{currentstroke}{rgb}{0.688944,0.865448,0.182725}%
\pgfsetstrokecolor{currentstroke}%
\pgfsetdash{}{0pt}%
\pgfpathmoveto{\pgfqpoint{3.602460in}{2.836495in}}%
\pgfpathlineto{\pgfqpoint{3.462884in}{2.846783in}}%
\pgfpathlineto{\pgfqpoint{3.519555in}{2.599630in}}%
\pgfpathclose%
\pgfusepath{fill}%
\end{pgfscope}%
\begin{pgfscope}%
\pgfpathrectangle{\pgfqpoint{0.680860in}{0.078740in}}{\pgfqpoint{7.842520in}{7.842520in}}%
\pgfusepath{clip}%
\pgfsetbuttcap%
\pgfsetroundjoin%
\definecolor{currentfill}{rgb}{0.404001,0.800275,0.362552}%
\pgfsetfillcolor{currentfill}%
\pgfsetlinewidth{0.000000pt}%
\definecolor{currentstroke}{rgb}{0.699415,0.867117,0.175971}%
\pgfsetstrokecolor{currentstroke}%
\pgfsetdash{}{0pt}%
\pgfpathmoveto{\pgfqpoint{6.312949in}{4.840984in}}%
\pgfpathlineto{\pgfqpoint{6.091000in}{4.822016in}}%
\pgfpathlineto{\pgfqpoint{6.237350in}{4.864197in}}%
\pgfpathclose%
\pgfusepath{fill}%
\end{pgfscope}%
\begin{pgfscope}%
\pgfpathrectangle{\pgfqpoint{0.680860in}{0.078740in}}{\pgfqpoint{7.842520in}{7.842520in}}%
\pgfusepath{clip}%
\pgfsetbuttcap%
\pgfsetroundjoin%
\definecolor{currentfill}{rgb}{0.177423,0.437527,0.557565}%
\pgfsetfillcolor{currentfill}%
\pgfsetlinewidth{0.000000pt}%
\definecolor{currentstroke}{rgb}{0.709898,0.868751,0.169257}%
\pgfsetstrokecolor{currentstroke}%
\pgfsetdash{}{0pt}%
\pgfpathmoveto{\pgfqpoint{3.685400in}{3.062729in}}%
\pgfpathlineto{\pgfqpoint{3.825565in}{3.058767in}}%
\pgfpathlineto{\pgfqpoint{3.768360in}{3.276797in}}%
\pgfpathclose%
\pgfusepath{fill}%
\end{pgfscope}%
\begin{pgfscope}%
\pgfpathrectangle{\pgfqpoint{0.680860in}{0.078740in}}{\pgfqpoint{7.842520in}{7.842520in}}%
\pgfusepath{clip}%
\pgfsetbuttcap%
\pgfsetroundjoin%
\definecolor{currentfill}{rgb}{0.139147,0.533812,0.555298}%
\pgfsetfillcolor{currentfill}%
\pgfsetlinewidth{0.000000pt}%
\definecolor{currentstroke}{rgb}{0.720391,0.870350,0.162603}%
\pgfsetstrokecolor{currentstroke}%
\pgfsetdash{}{0pt}%
\pgfpathmoveto{\pgfqpoint{3.991467in}{3.482768in}}%
\pgfpathlineto{\pgfqpoint{4.132261in}{3.489798in}}%
\pgfpathlineto{\pgfqpoint{4.074358in}{3.672388in}}%
\pgfpathclose%
\pgfusepath{fill}%
\end{pgfscope}%
\begin{pgfscope}%
\pgfpathrectangle{\pgfqpoint{0.680860in}{0.078740in}}{\pgfqpoint{7.842520in}{7.842520in}}%
\pgfusepath{clip}%
\pgfsetbuttcap%
\pgfsetroundjoin%
\definecolor{currentfill}{rgb}{0.277941,0.056324,0.381191}%
\pgfsetfillcolor{currentfill}%
\pgfsetlinewidth{0.000000pt}%
\definecolor{currentstroke}{rgb}{0.730889,0.871916,0.156029}%
\pgfsetstrokecolor{currentstroke}%
\pgfsetdash{}{0pt}%
\pgfpathmoveto{\pgfqpoint{2.909534in}{1.658738in}}%
\pgfpathlineto{\pgfqpoint{3.048688in}{1.619082in}}%
\pgfpathlineto{\pgfqpoint{3.131480in}{1.875386in}}%
\pgfpathclose%
\pgfusepath{fill}%
\end{pgfscope}%
\begin{pgfscope}%
\pgfpathrectangle{\pgfqpoint{0.680860in}{0.078740in}}{\pgfqpoint{7.842520in}{7.842520in}}%
\pgfusepath{clip}%
\pgfsetbuttcap%
\pgfsetroundjoin%
\definecolor{currentfill}{rgb}{0.175707,0.697900,0.491033}%
\pgfsetfillcolor{currentfill}%
\pgfsetlinewidth{0.000000pt}%
\definecolor{currentstroke}{rgb}{0.741388,0.873449,0.149561}%
\pgfsetstrokecolor{currentstroke}%
\pgfsetdash{}{0pt}%
\pgfpathmoveto{\pgfqpoint{4.828122in}{4.328612in}}%
\pgfpathlineto{\pgfqpoint{4.746229in}{4.205041in}}%
\pgfpathlineto{\pgfqpoint{4.970796in}{4.357177in}}%
\pgfpathclose%
\pgfusepath{fill}%
\end{pgfscope}%
\begin{pgfscope}%
\pgfpathrectangle{\pgfqpoint{0.680860in}{0.078740in}}{\pgfqpoint{7.842520in}{7.842520in}}%
\pgfusepath{clip}%
\pgfsetbuttcap%
\pgfsetroundjoin%
\definecolor{currentfill}{rgb}{0.430983,0.808473,0.346476}%
\pgfsetfillcolor{currentfill}%
\pgfsetlinewidth{0.000000pt}%
\definecolor{currentstroke}{rgb}{0.751884,0.874951,0.143228}%
\pgfsetstrokecolor{currentstroke}%
\pgfsetdash{}{0pt}%
\pgfpathmoveto{\pgfqpoint{6.533586in}{4.831170in}}%
\pgfpathlineto{\pgfqpoint{6.459709in}{4.882829in}}%
\pgfpathlineto{\pgfqpoint{6.607353in}{4.927499in}}%
\pgfpathclose%
\pgfusepath{fill}%
\end{pgfscope}%
\begin{pgfscope}%
\pgfpathrectangle{\pgfqpoint{0.680860in}{0.078740in}}{\pgfqpoint{7.842520in}{7.842520in}}%
\pgfusepath{clip}%
\pgfsetbuttcap%
\pgfsetroundjoin%
\definecolor{currentfill}{rgb}{0.319809,0.770914,0.411152}%
\pgfsetfillcolor{currentfill}%
\pgfsetlinewidth{0.000000pt}%
\definecolor{currentstroke}{rgb}{0.762373,0.876424,0.137064}%
\pgfsetstrokecolor{currentstroke}%
\pgfsetdash{}{0pt}%
\pgfpathmoveto{\pgfqpoint{5.644273in}{4.704157in}}%
\pgfpathlineto{\pgfqpoint{5.499781in}{4.666774in}}%
\pgfpathlineto{\pgfqpoint{5.420007in}{4.609045in}}%
\pgfpathclose%
\pgfusepath{fill}%
\end{pgfscope}%
\begin{pgfscope}%
\pgfpathrectangle{\pgfqpoint{0.680860in}{0.078740in}}{\pgfqpoint{7.842520in}{7.842520in}}%
\pgfusepath{clip}%
\pgfsetbuttcap%
\pgfsetroundjoin%
\definecolor{currentfill}{rgb}{0.430983,0.808473,0.346476}%
\pgfsetfillcolor{currentfill}%
\pgfsetlinewidth{0.000000pt}%
\definecolor{currentstroke}{rgb}{0.772852,0.877868,0.131109}%
\pgfsetstrokecolor{currentstroke}%
\pgfsetdash{}{0pt}%
\pgfpathmoveto{\pgfqpoint{6.899928in}{4.829425in}}%
\pgfpathlineto{\pgfqpoint{6.977516in}{4.961093in}}%
\pgfpathlineto{\pgfqpoint{7.048113in}{4.870209in}}%
\pgfpathclose%
\pgfusepath{fill}%
\end{pgfscope}%
\begin{pgfscope}%
\pgfpathrectangle{\pgfqpoint{0.680860in}{0.078740in}}{\pgfqpoint{7.842520in}{7.842520in}}%
\pgfusepath{clip}%
\pgfsetbuttcap%
\pgfsetroundjoin%
\definecolor{currentfill}{rgb}{0.266580,0.228262,0.514349}%
\pgfsetfillcolor{currentfill}%
\pgfsetlinewidth{0.000000pt}%
\definecolor{currentstroke}{rgb}{0.783315,0.879285,0.125405}%
\pgfsetstrokecolor{currentstroke}%
\pgfsetdash{}{0pt}%
\pgfpathmoveto{\pgfqpoint{3.297105in}{2.375252in}}%
\pgfpathlineto{\pgfqpoint{3.214279in}{2.128019in}}%
\pgfpathlineto{\pgfqpoint{3.436695in}{2.353766in}}%
\pgfpathclose%
\pgfusepath{fill}%
\end{pgfscope}%
\begin{pgfscope}%
\pgfpathrectangle{\pgfqpoint{0.680860in}{0.078740in}}{\pgfqpoint{7.842520in}{7.842520in}}%
\pgfusepath{clip}%
\pgfsetbuttcap%
\pgfsetroundjoin%
\definecolor{currentfill}{rgb}{0.169646,0.456262,0.558030}%
\pgfsetfillcolor{currentfill}%
\pgfsetlinewidth{0.000000pt}%
\definecolor{currentstroke}{rgb}{0.793760,0.880678,0.120005}%
\pgfsetstrokecolor{currentstroke}%
\pgfsetdash{}{0pt}%
\pgfpathmoveto{\pgfqpoint{3.825565in}{3.058767in}}%
\pgfpathlineto{\pgfqpoint{3.908524in}{3.277781in}}%
\pgfpathlineto{\pgfqpoint{3.768360in}{3.276797in}}%
\pgfpathclose%
\pgfusepath{fill}%
\end{pgfscope}%
\begin{pgfscope}%
\pgfpathrectangle{\pgfqpoint{0.680860in}{0.078740in}}{\pgfqpoint{7.842520in}{7.842520in}}%
\pgfusepath{clip}%
\pgfsetbuttcap%
\pgfsetroundjoin%
\definecolor{currentfill}{rgb}{0.119699,0.618490,0.536347}%
\pgfsetfillcolor{currentfill}%
\pgfsetlinewidth{0.000000pt}%
\definecolor{currentstroke}{rgb}{0.804182,0.882046,0.114965}%
\pgfsetstrokecolor{currentstroke}%
\pgfsetdash{}{0pt}%
\pgfpathmoveto{\pgfqpoint{4.521892in}{4.038607in}}%
\pgfpathlineto{\pgfqpoint{4.297905in}{3.860268in}}%
\pgfpathlineto{\pgfqpoint{4.439337in}{3.876939in}}%
\pgfpathclose%
\pgfusepath{fill}%
\end{pgfscope}%
\begin{pgfscope}%
\pgfpathrectangle{\pgfqpoint{0.680860in}{0.078740in}}{\pgfqpoint{7.842520in}{7.842520in}}%
\pgfusepath{clip}%
\pgfsetbuttcap%
\pgfsetroundjoin%
\definecolor{currentfill}{rgb}{0.133743,0.548535,0.553541}%
\pgfsetfillcolor{currentfill}%
\pgfsetlinewidth{0.000000pt}%
\definecolor{currentstroke}{rgb}{0.814576,0.883393,0.110347}%
\pgfsetstrokecolor{currentstroke}%
\pgfsetdash{}{0pt}%
\pgfpathmoveto{\pgfqpoint{4.074358in}{3.672388in}}%
\pgfpathlineto{\pgfqpoint{4.132261in}{3.489798in}}%
\pgfpathlineto{\pgfqpoint{4.215140in}{3.683583in}}%
\pgfpathclose%
\pgfusepath{fill}%
\end{pgfscope}%
\begin{pgfscope}%
\pgfpathrectangle{\pgfqpoint{0.680860in}{0.078740in}}{\pgfqpoint{7.842520in}{7.842520in}}%
\pgfusepath{clip}%
\pgfsetbuttcap%
\pgfsetroundjoin%
\definecolor{currentfill}{rgb}{0.369214,0.788888,0.382914}%
\pgfsetfillcolor{currentfill}%
\pgfsetlinewidth{0.000000pt}%
\definecolor{currentstroke}{rgb}{0.824940,0.884720,0.106217}%
\pgfsetstrokecolor{currentstroke}%
\pgfsetdash{}{0pt}%
\pgfpathmoveto{\pgfqpoint{5.789590in}{4.744205in}}%
\pgfpathlineto{\pgfqpoint{5.868020in}{4.775811in}}%
\pgfpathlineto{\pgfqpoint{5.722997in}{4.736967in}}%
\pgfpathclose%
\pgfusepath{fill}%
\end{pgfscope}%
\begin{pgfscope}%
\pgfpathrectangle{\pgfqpoint{0.680860in}{0.078740in}}{\pgfqpoint{7.842520in}{7.842520in}}%
\pgfusepath{clip}%
\pgfsetbuttcap%
\pgfsetroundjoin%
\definecolor{currentfill}{rgb}{0.250425,0.274290,0.533103}%
\pgfsetfillcolor{currentfill}%
\pgfsetlinewidth{0.000000pt}%
\definecolor{currentstroke}{rgb}{0.835270,0.886029,0.102646}%
\pgfsetstrokecolor{currentstroke}%
\pgfsetdash{}{0pt}%
\pgfpathmoveto{\pgfqpoint{3.297105in}{2.375252in}}%
\pgfpathlineto{\pgfqpoint{3.436695in}{2.353766in}}%
\pgfpathlineto{\pgfqpoint{3.519555in}{2.599630in}}%
\pgfpathclose%
\pgfusepath{fill}%
\end{pgfscope}%
\begin{pgfscope}%
\pgfpathrectangle{\pgfqpoint{0.680860in}{0.078740in}}{\pgfqpoint{7.842520in}{7.842520in}}%
\pgfusepath{clip}%
\pgfsetbuttcap%
\pgfsetroundjoin%
\definecolor{currentfill}{rgb}{0.197636,0.391528,0.554969}%
\pgfsetfillcolor{currentfill}%
\pgfsetlinewidth{0.000000pt}%
\definecolor{currentstroke}{rgb}{0.845561,0.887322,0.099702}%
\pgfsetstrokecolor{currentstroke}%
\pgfsetdash{}{0pt}%
\pgfpathmoveto{\pgfqpoint{3.685400in}{3.062729in}}%
\pgfpathlineto{\pgfqpoint{3.602460in}{2.836495in}}%
\pgfpathlineto{\pgfqpoint{3.742618in}{2.827205in}}%
\pgfpathclose%
\pgfusepath{fill}%
\end{pgfscope}%
\begin{pgfscope}%
\pgfpathrectangle{\pgfqpoint{0.680860in}{0.078740in}}{\pgfqpoint{7.842520in}{7.842520in}}%
\pgfusepath{clip}%
\pgfsetbuttcap%
\pgfsetroundjoin%
\definecolor{currentfill}{rgb}{0.352360,0.783011,0.392636}%
\pgfsetfillcolor{currentfill}%
\pgfsetlinewidth{0.000000pt}%
\definecolor{currentstroke}{rgb}{0.855810,0.888601,0.097452}%
\pgfsetstrokecolor{currentstroke}%
\pgfsetdash{}{0pt}%
\pgfpathmoveto{\pgfqpoint{5.789590in}{4.744205in}}%
\pgfpathlineto{\pgfqpoint{5.722997in}{4.736967in}}%
\pgfpathlineto{\pgfqpoint{5.644273in}{4.704157in}}%
\pgfpathclose%
\pgfusepath{fill}%
\end{pgfscope}%
\begin{pgfscope}%
\pgfpathrectangle{\pgfqpoint{0.680860in}{0.078740in}}{\pgfqpoint{7.842520in}{7.842520in}}%
\pgfusepath{clip}%
\pgfsetbuttcap%
\pgfsetroundjoin%
\definecolor{currentfill}{rgb}{0.149039,0.508051,0.557250}%
\pgfsetfillcolor{currentfill}%
\pgfsetlinewidth{0.000000pt}%
\definecolor{currentstroke}{rgb}{0.866013,0.889868,0.095953}%
\pgfsetstrokecolor{currentstroke}%
\pgfsetdash{}{0pt}%
\pgfpathmoveto{\pgfqpoint{4.132261in}{3.489798in}}%
\pgfpathlineto{\pgfqpoint{3.991467in}{3.482768in}}%
\pgfpathlineto{\pgfqpoint{3.908524in}{3.277781in}}%
\pgfpathclose%
\pgfusepath{fill}%
\end{pgfscope}%
\begin{pgfscope}%
\pgfpathrectangle{\pgfqpoint{0.680860in}{0.078740in}}{\pgfqpoint{7.842520in}{7.842520in}}%
\pgfusepath{clip}%
\pgfsetbuttcap%
\pgfsetroundjoin%
\definecolor{currentfill}{rgb}{0.282884,0.135920,0.453427}%
\pgfsetfillcolor{currentfill}%
\pgfsetlinewidth{0.000000pt}%
\definecolor{currentstroke}{rgb}{0.876168,0.891125,0.095250}%
\pgfsetstrokecolor{currentstroke}%
\pgfsetdash{}{0pt}%
\pgfpathmoveto{\pgfqpoint{3.131480in}{1.875386in}}%
\pgfpathlineto{\pgfqpoint{3.271096in}{1.841909in}}%
\pgfpathlineto{\pgfqpoint{3.214279in}{2.128019in}}%
\pgfpathclose%
\pgfusepath{fill}%
\end{pgfscope}%
\begin{pgfscope}%
\pgfpathrectangle{\pgfqpoint{0.680860in}{0.078740in}}{\pgfqpoint{7.842520in}{7.842520in}}%
\pgfusepath{clip}%
\pgfsetbuttcap%
\pgfsetroundjoin%
\definecolor{currentfill}{rgb}{0.395174,0.797475,0.367757}%
\pgfsetfillcolor{currentfill}%
\pgfsetlinewidth{0.000000pt}%
\definecolor{currentstroke}{rgb}{0.886271,0.892374,0.095374}%
\pgfsetstrokecolor{currentstroke}%
\pgfsetdash{}{0pt}%
\pgfpathmoveto{\pgfqpoint{6.091000in}{4.822016in}}%
\pgfpathlineto{\pgfqpoint{5.868020in}{4.775811in}}%
\pgfpathlineto{\pgfqpoint{6.013884in}{4.817371in}}%
\pgfpathclose%
\pgfusepath{fill}%
\end{pgfscope}%
\begin{pgfscope}%
\pgfpathrectangle{\pgfqpoint{0.680860in}{0.078740in}}{\pgfqpoint{7.842520in}{7.842520in}}%
\pgfusepath{clip}%
\pgfsetbuttcap%
\pgfsetroundjoin%
\definecolor{currentfill}{rgb}{0.449368,0.813768,0.335384}%
\pgfsetfillcolor{currentfill}%
\pgfsetlinewidth{0.000000pt}%
\definecolor{currentstroke}{rgb}{0.896320,0.893616,0.096335}%
\pgfsetstrokecolor{currentstroke}%
\pgfsetdash{}{0pt}%
\pgfpathmoveto{\pgfqpoint{6.828636in}{4.914943in}}%
\pgfpathlineto{\pgfqpoint{6.977516in}{4.961093in}}%
\pgfpathlineto{\pgfqpoint{6.899928in}{4.829425in}}%
\pgfpathclose%
\pgfusepath{fill}%
\end{pgfscope}%
\begin{pgfscope}%
\pgfpathrectangle{\pgfqpoint{0.680860in}{0.078740in}}{\pgfqpoint{7.842520in}{7.842520in}}%
\pgfusepath{clip}%
\pgfsetbuttcap%
\pgfsetroundjoin%
\definecolor{currentfill}{rgb}{0.123463,0.581687,0.547445}%
\pgfsetfillcolor{currentfill}%
\pgfsetlinewidth{0.000000pt}%
\definecolor{currentstroke}{rgb}{0.906311,0.894855,0.098125}%
\pgfsetstrokecolor{currentstroke}%
\pgfsetdash{}{0pt}%
\pgfpathmoveto{\pgfqpoint{4.356592in}{3.696513in}}%
\pgfpathlineto{\pgfqpoint{4.297905in}{3.860268in}}%
\pgfpathlineto{\pgfqpoint{4.215140in}{3.683583in}}%
\pgfpathclose%
\pgfusepath{fill}%
\end{pgfscope}%
\begin{pgfscope}%
\pgfpathrectangle{\pgfqpoint{0.680860in}{0.078740in}}{\pgfqpoint{7.842520in}{7.842520in}}%
\pgfusepath{clip}%
\pgfsetbuttcap%
\pgfsetroundjoin%
\definecolor{currentfill}{rgb}{0.271828,0.209303,0.504434}%
\pgfsetfillcolor{currentfill}%
\pgfsetlinewidth{0.000000pt}%
\definecolor{currentstroke}{rgb}{0.916242,0.896091,0.100717}%
\pgfsetstrokecolor{currentstroke}%
\pgfsetdash{}{0pt}%
\pgfpathmoveto{\pgfqpoint{3.436695in}{2.353766in}}%
\pgfpathlineto{\pgfqpoint{3.214279in}{2.128019in}}%
\pgfpathlineto{\pgfqpoint{3.353879in}{2.100607in}}%
\pgfpathclose%
\pgfusepath{fill}%
\end{pgfscope}%
\begin{pgfscope}%
\pgfpathrectangle{\pgfqpoint{0.680860in}{0.078740in}}{\pgfqpoint{7.842520in}{7.842520in}}%
\pgfusepath{clip}%
\pgfsetbuttcap%
\pgfsetroundjoin%
\definecolor{currentfill}{rgb}{0.134692,0.658636,0.517649}%
\pgfsetfillcolor{currentfill}%
\pgfsetlinewidth{0.000000pt}%
\definecolor{currentstroke}{rgb}{0.926106,0.897330,0.104071}%
\pgfsetstrokecolor{currentstroke}%
\pgfsetdash{}{0pt}%
\pgfpathmoveto{\pgfqpoint{4.521892in}{4.038607in}}%
\pgfpathlineto{\pgfqpoint{4.663989in}{4.060447in}}%
\pgfpathlineto{\pgfqpoint{4.746229in}{4.205041in}}%
\pgfpathclose%
\pgfusepath{fill}%
\end{pgfscope}%
\begin{pgfscope}%
\pgfpathrectangle{\pgfqpoint{0.680860in}{0.078740in}}{\pgfqpoint{7.842520in}{7.842520in}}%
\pgfusepath{clip}%
\pgfsetbuttcap%
\pgfsetroundjoin%
\definecolor{currentfill}{rgb}{0.188923,0.410910,0.556326}%
\pgfsetfillcolor{currentfill}%
\pgfsetlinewidth{0.000000pt}%
\definecolor{currentstroke}{rgb}{0.935904,0.898570,0.108131}%
\pgfsetstrokecolor{currentstroke}%
\pgfsetdash{}{0pt}%
\pgfpathmoveto{\pgfqpoint{3.742618in}{2.827205in}}%
\pgfpathlineto{\pgfqpoint{3.825565in}{3.058767in}}%
\pgfpathlineto{\pgfqpoint{3.685400in}{3.062729in}}%
\pgfpathclose%
\pgfusepath{fill}%
\end{pgfscope}%
\begin{pgfscope}%
\pgfpathrectangle{\pgfqpoint{0.680860in}{0.078740in}}{\pgfqpoint{7.842520in}{7.842520in}}%
\pgfusepath{clip}%
\pgfsetbuttcap%
\pgfsetroundjoin%
\definecolor{currentfill}{rgb}{0.440137,0.811138,0.340967}%
\pgfsetfillcolor{currentfill}%
\pgfsetlinewidth{0.000000pt}%
\definecolor{currentstroke}{rgb}{0.945636,0.899815,0.112838}%
\pgfsetstrokecolor{currentstroke}%
\pgfsetdash{}{0pt}%
\pgfpathmoveto{\pgfqpoint{6.384576in}{4.909220in}}%
\pgfpathlineto{\pgfqpoint{6.459709in}{4.882829in}}%
\pgfpathlineto{\pgfqpoint{6.312949in}{4.840984in}}%
\pgfpathclose%
\pgfusepath{fill}%
\end{pgfscope}%
\begin{pgfscope}%
\pgfpathrectangle{\pgfqpoint{0.680860in}{0.078740in}}{\pgfqpoint{7.842520in}{7.842520in}}%
\pgfusepath{clip}%
\pgfsetbuttcap%
\pgfsetroundjoin%
\definecolor{currentfill}{rgb}{0.220124,0.725509,0.466226}%
\pgfsetfillcolor{currentfill}%
\pgfsetlinewidth{0.000000pt}%
\definecolor{currentstroke}{rgb}{0.955300,0.901065,0.118128}%
\pgfsetstrokecolor{currentstroke}%
\pgfsetdash{}{0pt}%
\pgfpathmoveto{\pgfqpoint{5.052142in}{4.460464in}}%
\pgfpathlineto{\pgfqpoint{4.970796in}{4.357177in}}%
\pgfpathlineto{\pgfqpoint{5.114232in}{4.388092in}}%
\pgfpathclose%
\pgfusepath{fill}%
\end{pgfscope}%
\begin{pgfscope}%
\pgfpathrectangle{\pgfqpoint{0.680860in}{0.078740in}}{\pgfqpoint{7.842520in}{7.842520in}}%
\pgfusepath{clip}%
\pgfsetbuttcap%
\pgfsetroundjoin%
\definecolor{currentfill}{rgb}{0.277941,0.056324,0.381191}%
\pgfsetfillcolor{currentfill}%
\pgfsetlinewidth{0.000000pt}%
\definecolor{currentstroke}{rgb}{0.964894,0.902323,0.123941}%
\pgfsetstrokecolor{currentstroke}%
\pgfsetdash{}{0pt}%
\pgfpathmoveto{\pgfqpoint{3.131480in}{1.875386in}}%
\pgfpathlineto{\pgfqpoint{3.048688in}{1.619082in}}%
\pgfpathlineto{\pgfqpoint{3.188328in}{1.579461in}}%
\pgfpathclose%
\pgfusepath{fill}%
\end{pgfscope}%
\begin{pgfscope}%
\pgfpathrectangle{\pgfqpoint{0.680860in}{0.078740in}}{\pgfqpoint{7.842520in}{7.842520in}}%
\pgfusepath{clip}%
\pgfsetbuttcap%
\pgfsetroundjoin%
\definecolor{currentfill}{rgb}{0.232815,0.732247,0.459277}%
\pgfsetfillcolor{currentfill}%
\pgfsetlinewidth{0.000000pt}%
\definecolor{currentstroke}{rgb}{0.974417,0.903590,0.130215}%
\pgfsetstrokecolor{currentstroke}%
\pgfsetdash{}{0pt}%
\pgfpathmoveto{\pgfqpoint{5.195448in}{4.492628in}}%
\pgfpathlineto{\pgfqpoint{5.052142in}{4.460464in}}%
\pgfpathlineto{\pgfqpoint{5.114232in}{4.388092in}}%
\pgfpathclose%
\pgfusepath{fill}%
\end{pgfscope}%
\begin{pgfscope}%
\pgfpathrectangle{\pgfqpoint{0.680860in}{0.078740in}}{\pgfqpoint{7.842520in}{7.842520in}}%
\pgfusepath{clip}%
\pgfsetbuttcap%
\pgfsetroundjoin%
\definecolor{currentfill}{rgb}{0.288921,0.758394,0.428426}%
\pgfsetfillcolor{currentfill}%
\pgfsetlinewidth{0.000000pt}%
\definecolor{currentstroke}{rgb}{0.983868,0.904867,0.136897}%
\pgfsetstrokecolor{currentstroke}%
\pgfsetdash{}{0pt}%
\pgfpathmoveto{\pgfqpoint{5.339539in}{4.527272in}}%
\pgfpathlineto{\pgfqpoint{5.420007in}{4.609045in}}%
\pgfpathlineto{\pgfqpoint{5.276092in}{4.573915in}}%
\pgfpathclose%
\pgfusepath{fill}%
\end{pgfscope}%
\begin{pgfscope}%
\pgfpathrectangle{\pgfqpoint{0.680860in}{0.078740in}}{\pgfqpoint{7.842520in}{7.842520in}}%
\pgfusepath{clip}%
\pgfsetbuttcap%
\pgfsetroundjoin%
\definecolor{currentfill}{rgb}{0.274149,0.751988,0.436601}%
\pgfsetfillcolor{currentfill}%
\pgfsetlinewidth{0.000000pt}%
\definecolor{currentstroke}{rgb}{0.993248,0.906157,0.143936}%
\pgfsetstrokecolor{currentstroke}%
\pgfsetdash{}{0pt}%
\pgfpathmoveto{\pgfqpoint{5.276092in}{4.573915in}}%
\pgfpathlineto{\pgfqpoint{5.195448in}{4.492628in}}%
\pgfpathlineto{\pgfqpoint{5.339539in}{4.527272in}}%
\pgfpathclose%
\pgfusepath{fill}%
\end{pgfscope}%
\begin{pgfscope}%
\pgfpathrectangle{\pgfqpoint{0.680860in}{0.078740in}}{\pgfqpoint{7.842520in}{7.842520in}}%
\pgfusepath{clip}%
\pgfsetbuttcap%
\pgfsetroundjoin%
\definecolor{currentfill}{rgb}{0.280868,0.160771,0.472899}%
\pgfsetfillcolor{currentfill}%
\pgfsetlinewidth{0.000000pt}%
\definecolor{currentstroke}{rgb}{0.267004,0.004874,0.329415}%
\pgfsetstrokecolor{currentstroke}%
\pgfsetdash{}{0pt}%
\pgfpathmoveto{\pgfqpoint{3.271096in}{1.841909in}}%
\pgfpathlineto{\pgfqpoint{3.353879in}{2.100607in}}%
\pgfpathlineto{\pgfqpoint{3.214279in}{2.128019in}}%
\pgfpathclose%
\pgfusepath{fill}%
\end{pgfscope}%
\begin{pgfscope}%
\pgfpathrectangle{\pgfqpoint{0.680860in}{0.078740in}}{\pgfqpoint{7.842520in}{7.842520in}}%
\pgfusepath{clip}%
\pgfsetbuttcap%
\pgfsetroundjoin%
\definecolor{currentfill}{rgb}{0.221989,0.339161,0.548752}%
\pgfsetfillcolor{currentfill}%
\pgfsetlinewidth{0.000000pt}%
\definecolor{currentstroke}{rgb}{0.268510,0.009605,0.335427}%
\pgfsetstrokecolor{currentstroke}%
\pgfsetdash{}{0pt}%
\pgfpathmoveto{\pgfqpoint{3.602460in}{2.836495in}}%
\pgfpathlineto{\pgfqpoint{3.519555in}{2.599630in}}%
\pgfpathlineto{\pgfqpoint{3.659703in}{2.584688in}}%
\pgfpathclose%
\pgfusepath{fill}%
\end{pgfscope}%
\begin{pgfscope}%
\pgfpathrectangle{\pgfqpoint{0.680860in}{0.078740in}}{\pgfqpoint{7.842520in}{7.842520in}}%
\pgfusepath{clip}%
\pgfsetbuttcap%
\pgfsetroundjoin%
\definecolor{currentfill}{rgb}{0.120565,0.596422,0.543611}%
\pgfsetfillcolor{currentfill}%
\pgfsetlinewidth{0.000000pt}%
\definecolor{currentstroke}{rgb}{0.269944,0.014625,0.341379}%
\pgfsetstrokecolor{currentstroke}%
\pgfsetdash{}{0pt}%
\pgfpathmoveto{\pgfqpoint{4.439337in}{3.876939in}}%
\pgfpathlineto{\pgfqpoint{4.297905in}{3.860268in}}%
\pgfpathlineto{\pgfqpoint{4.356592in}{3.696513in}}%
\pgfpathclose%
\pgfusepath{fill}%
\end{pgfscope}%
\begin{pgfscope}%
\pgfpathrectangle{\pgfqpoint{0.680860in}{0.078740in}}{\pgfqpoint{7.842520in}{7.842520in}}%
\pgfusepath{clip}%
\pgfsetbuttcap%
\pgfsetroundjoin%
\definecolor{currentfill}{rgb}{0.440137,0.811138,0.340967}%
\pgfsetfillcolor{currentfill}%
\pgfsetlinewidth{0.000000pt}%
\definecolor{currentstroke}{rgb}{0.271305,0.019942,0.347269}%
\pgfsetstrokecolor{currentstroke}%
\pgfsetdash{}{0pt}%
\pgfpathmoveto{\pgfqpoint{6.237350in}{4.864197in}}%
\pgfpathlineto{\pgfqpoint{6.384576in}{4.909220in}}%
\pgfpathlineto{\pgfqpoint{6.312949in}{4.840984in}}%
\pgfpathclose%
\pgfusepath{fill}%
\end{pgfscope}%
\begin{pgfscope}%
\pgfpathrectangle{\pgfqpoint{0.680860in}{0.078740in}}{\pgfqpoint{7.842520in}{7.842520in}}%
\pgfusepath{clip}%
\pgfsetbuttcap%
\pgfsetroundjoin%
\definecolor{currentfill}{rgb}{0.175707,0.697900,0.491033}%
\pgfsetfillcolor{currentfill}%
\pgfsetlinewidth{0.000000pt}%
\definecolor{currentstroke}{rgb}{0.272594,0.025563,0.353093}%
\pgfsetstrokecolor{currentstroke}%
\pgfsetdash{}{0pt}%
\pgfpathmoveto{\pgfqpoint{4.746229in}{4.205041in}}%
\pgfpathlineto{\pgfqpoint{4.888997in}{4.231658in}}%
\pgfpathlineto{\pgfqpoint{4.970796in}{4.357177in}}%
\pgfpathclose%
\pgfusepath{fill}%
\end{pgfscope}%
\begin{pgfscope}%
\pgfpathrectangle{\pgfqpoint{0.680860in}{0.078740in}}{\pgfqpoint{7.842520in}{7.842520in}}%
\pgfusepath{clip}%
\pgfsetbuttcap%
\pgfsetroundjoin%
\definecolor{currentfill}{rgb}{0.281446,0.084320,0.407414}%
\pgfsetfillcolor{currentfill}%
\pgfsetlinewidth{0.000000pt}%
\definecolor{currentstroke}{rgb}{0.273809,0.031497,0.358853}%
\pgfsetstrokecolor{currentstroke}%
\pgfsetdash{}{0pt}%
\pgfpathmoveto{\pgfqpoint{3.188328in}{1.579461in}}%
\pgfpathlineto{\pgfqpoint{3.271096in}{1.841909in}}%
\pgfpathlineto{\pgfqpoint{3.131480in}{1.875386in}}%
\pgfpathclose%
\pgfusepath{fill}%
\end{pgfscope}%
\begin{pgfscope}%
\pgfpathrectangle{\pgfqpoint{0.680860in}{0.078740in}}{\pgfqpoint{7.842520in}{7.842520in}}%
\pgfusepath{clip}%
\pgfsetbuttcap%
\pgfsetroundjoin%
\definecolor{currentfill}{rgb}{0.239346,0.300855,0.540844}%
\pgfsetfillcolor{currentfill}%
\pgfsetlinewidth{0.000000pt}%
\definecolor{currentstroke}{rgb}{0.274952,0.037752,0.364543}%
\pgfsetstrokecolor{currentstroke}%
\pgfsetdash{}{0pt}%
\pgfpathmoveto{\pgfqpoint{3.659703in}{2.584688in}}%
\pgfpathlineto{\pgfqpoint{3.519555in}{2.599630in}}%
\pgfpathlineto{\pgfqpoint{3.436695in}{2.353766in}}%
\pgfpathclose%
\pgfusepath{fill}%
\end{pgfscope}%
\begin{pgfscope}%
\pgfpathrectangle{\pgfqpoint{0.680860in}{0.078740in}}{\pgfqpoint{7.842520in}{7.842520in}}%
\pgfusepath{clip}%
\pgfsetbuttcap%
\pgfsetroundjoin%
\definecolor{currentfill}{rgb}{0.210503,0.363727,0.552206}%
\pgfsetfillcolor{currentfill}%
\pgfsetlinewidth{0.000000pt}%
\definecolor{currentstroke}{rgb}{0.276022,0.044167,0.370164}%
\pgfsetstrokecolor{currentstroke}%
\pgfsetdash{}{0pt}%
\pgfpathmoveto{\pgfqpoint{3.659703in}{2.584688in}}%
\pgfpathlineto{\pgfqpoint{3.742618in}{2.827205in}}%
\pgfpathlineto{\pgfqpoint{3.602460in}{2.836495in}}%
\pgfpathclose%
\pgfusepath{fill}%
\end{pgfscope}%
\begin{pgfscope}%
\pgfpathrectangle{\pgfqpoint{0.680860in}{0.078740in}}{\pgfqpoint{7.842520in}{7.842520in}}%
\pgfusepath{clip}%
\pgfsetbuttcap%
\pgfsetroundjoin%
\definecolor{currentfill}{rgb}{0.151918,0.500685,0.557587}%
\pgfsetfillcolor{currentfill}%
\pgfsetlinewidth{0.000000pt}%
\definecolor{currentstroke}{rgb}{0.277018,0.050344,0.375715}%
\pgfsetstrokecolor{currentstroke}%
\pgfsetdash{}{0pt}%
\pgfpathmoveto{\pgfqpoint{4.132261in}{3.489798in}}%
\pgfpathlineto{\pgfqpoint{3.908524in}{3.277781in}}%
\pgfpathlineto{\pgfqpoint{4.049315in}{3.280147in}}%
\pgfpathclose%
\pgfusepath{fill}%
\end{pgfscope}%
\begin{pgfscope}%
\pgfpathrectangle{\pgfqpoint{0.680860in}{0.078740in}}{\pgfqpoint{7.842520in}{7.842520in}}%
\pgfusepath{clip}%
\pgfsetbuttcap%
\pgfsetroundjoin%
\definecolor{currentfill}{rgb}{0.129933,0.559582,0.551864}%
\pgfsetfillcolor{currentfill}%
\pgfsetlinewidth{0.000000pt}%
\definecolor{currentstroke}{rgb}{0.277941,0.056324,0.381191}%
\pgfsetstrokecolor{currentstroke}%
\pgfsetdash{}{0pt}%
\pgfpathmoveto{\pgfqpoint{4.215140in}{3.683583in}}%
\pgfpathlineto{\pgfqpoint{4.132261in}{3.489798in}}%
\pgfpathlineto{\pgfqpoint{4.356592in}{3.696513in}}%
\pgfpathclose%
\pgfusepath{fill}%
\end{pgfscope}%
\begin{pgfscope}%
\pgfpathrectangle{\pgfqpoint{0.680860in}{0.078740in}}{\pgfqpoint{7.842520in}{7.842520in}}%
\pgfusepath{clip}%
\pgfsetbuttcap%
\pgfsetroundjoin%
\definecolor{currentfill}{rgb}{0.121380,0.629492,0.531973}%
\pgfsetfillcolor{currentfill}%
\pgfsetlinewidth{0.000000pt}%
\definecolor{currentstroke}{rgb}{0.278791,0.062145,0.386592}%
\pgfsetstrokecolor{currentstroke}%
\pgfsetdash{}{0pt}%
\pgfpathmoveto{\pgfqpoint{4.581468in}{3.895539in}}%
\pgfpathlineto{\pgfqpoint{4.521892in}{4.038607in}}%
\pgfpathlineto{\pgfqpoint{4.439337in}{3.876939in}}%
\pgfpathclose%
\pgfusepath{fill}%
\end{pgfscope}%
\begin{pgfscope}%
\pgfpathrectangle{\pgfqpoint{0.680860in}{0.078740in}}{\pgfqpoint{7.842520in}{7.842520in}}%
\pgfusepath{clip}%
\pgfsetbuttcap%
\pgfsetroundjoin%
\definecolor{currentfill}{rgb}{0.335885,0.777018,0.402049}%
\pgfsetfillcolor{currentfill}%
\pgfsetlinewidth{0.000000pt}%
\definecolor{currentstroke}{rgb}{0.279566,0.067836,0.391917}%
\pgfsetstrokecolor{currentstroke}%
\pgfsetdash{}{0pt}%
\pgfpathmoveto{\pgfqpoint{5.420007in}{4.609045in}}%
\pgfpathlineto{\pgfqpoint{5.564729in}{4.646761in}}%
\pgfpathlineto{\pgfqpoint{5.644273in}{4.704157in}}%
\pgfpathclose%
\pgfusepath{fill}%
\end{pgfscope}%
\begin{pgfscope}%
\pgfpathrectangle{\pgfqpoint{0.680860in}{0.078740in}}{\pgfqpoint{7.842520in}{7.842520in}}%
\pgfusepath{clip}%
\pgfsetbuttcap%
\pgfsetroundjoin%
\definecolor{currentfill}{rgb}{0.171176,0.452530,0.557965}%
\pgfsetfillcolor{currentfill}%
\pgfsetlinewidth{0.000000pt}%
\definecolor{currentstroke}{rgb}{0.280267,0.073417,0.397163}%
\pgfsetstrokecolor{currentstroke}%
\pgfsetdash{}{0pt}%
\pgfpathmoveto{\pgfqpoint{3.966340in}{3.056020in}}%
\pgfpathlineto{\pgfqpoint{3.908524in}{3.277781in}}%
\pgfpathlineto{\pgfqpoint{3.825565in}{3.058767in}}%
\pgfpathclose%
\pgfusepath{fill}%
\end{pgfscope}%
\begin{pgfscope}%
\pgfpathrectangle{\pgfqpoint{0.680860in}{0.078740in}}{\pgfqpoint{7.842520in}{7.842520in}}%
\pgfusepath{clip}%
\pgfsetbuttcap%
\pgfsetroundjoin%
\definecolor{currentfill}{rgb}{0.477504,0.821444,0.318195}%
\pgfsetfillcolor{currentfill}%
\pgfsetlinewidth{0.000000pt}%
\definecolor{currentstroke}{rgb}{0.280894,0.078907,0.402329}%
\pgfsetstrokecolor{currentstroke}%
\pgfsetdash{}{0pt}%
\pgfpathmoveto{\pgfqpoint{7.048113in}{4.870209in}}%
\pgfpathlineto{\pgfqpoint{6.977516in}{4.961093in}}%
\pgfpathlineto{\pgfqpoint{7.197205in}{4.913775in}}%
\pgfpathclose%
\pgfusepath{fill}%
\end{pgfscope}%
\begin{pgfscope}%
\pgfpathrectangle{\pgfqpoint{0.680860in}{0.078740in}}{\pgfqpoint{7.842520in}{7.842520in}}%
\pgfusepath{clip}%
\pgfsetbuttcap%
\pgfsetroundjoin%
\definecolor{currentfill}{rgb}{0.153894,0.680203,0.504172}%
\pgfsetfillcolor{currentfill}%
\pgfsetlinewidth{0.000000pt}%
\definecolor{currentstroke}{rgb}{0.281446,0.084320,0.407414}%
\pgfsetstrokecolor{currentstroke}%
\pgfsetdash{}{0pt}%
\pgfpathmoveto{\pgfqpoint{4.746229in}{4.205041in}}%
\pgfpathlineto{\pgfqpoint{4.663989in}{4.060447in}}%
\pgfpathlineto{\pgfqpoint{4.888997in}{4.231658in}}%
\pgfpathclose%
\pgfusepath{fill}%
\end{pgfscope}%
\begin{pgfscope}%
\pgfpathrectangle{\pgfqpoint{0.680860in}{0.078740in}}{\pgfqpoint{7.842520in}{7.842520in}}%
\pgfusepath{clip}%
\pgfsetbuttcap%
\pgfsetroundjoin%
\definecolor{currentfill}{rgb}{0.263663,0.237631,0.518762}%
\pgfsetfillcolor{currentfill}%
\pgfsetlinewidth{0.000000pt}%
\definecolor{currentstroke}{rgb}{0.281924,0.089666,0.412415}%
\pgfsetstrokecolor{currentstroke}%
\pgfsetdash{}{0pt}%
\pgfpathmoveto{\pgfqpoint{3.576832in}{2.332908in}}%
\pgfpathlineto{\pgfqpoint{3.436695in}{2.353766in}}%
\pgfpathlineto{\pgfqpoint{3.353879in}{2.100607in}}%
\pgfpathclose%
\pgfusepath{fill}%
\end{pgfscope}%
\begin{pgfscope}%
\pgfpathrectangle{\pgfqpoint{0.680860in}{0.078740in}}{\pgfqpoint{7.842520in}{7.842520in}}%
\pgfusepath{clip}%
\pgfsetbuttcap%
\pgfsetroundjoin%
\definecolor{currentfill}{rgb}{0.126326,0.644107,0.525311}%
\pgfsetfillcolor{currentfill}%
\pgfsetlinewidth{0.000000pt}%
\definecolor{currentstroke}{rgb}{0.282327,0.094955,0.417331}%
\pgfsetstrokecolor{currentstroke}%
\pgfsetdash{}{0pt}%
\pgfpathmoveto{\pgfqpoint{4.663989in}{4.060447in}}%
\pgfpathlineto{\pgfqpoint{4.521892in}{4.038607in}}%
\pgfpathlineto{\pgfqpoint{4.581468in}{3.895539in}}%
\pgfpathclose%
\pgfusepath{fill}%
\end{pgfscope}%
\begin{pgfscope}%
\pgfpathrectangle{\pgfqpoint{0.680860in}{0.078740in}}{\pgfqpoint{7.842520in}{7.842520in}}%
\pgfusepath{clip}%
\pgfsetbuttcap%
\pgfsetroundjoin%
\definecolor{currentfill}{rgb}{0.468053,0.818921,0.323998}%
\pgfsetfillcolor{currentfill}%
\pgfsetlinewidth{0.000000pt}%
\definecolor{currentstroke}{rgb}{0.282656,0.100196,0.422160}%
\pgfsetstrokecolor{currentstroke}%
\pgfsetdash{}{0pt}%
\pgfpathmoveto{\pgfqpoint{7.347229in}{4.960230in}}%
\pgfpathlineto{\pgfqpoint{7.414642in}{4.832133in}}%
\pgfpathlineto{\pgfqpoint{7.197205in}{4.913775in}}%
\pgfpathclose%
\pgfusepath{fill}%
\end{pgfscope}%
\begin{pgfscope}%
\pgfpathrectangle{\pgfqpoint{0.680860in}{0.078740in}}{\pgfqpoint{7.842520in}{7.842520in}}%
\pgfusepath{clip}%
\pgfsetbuttcap%
\pgfsetroundjoin%
\definecolor{currentfill}{rgb}{0.477504,0.821444,0.318195}%
\pgfsetfillcolor{currentfill}%
\pgfsetlinewidth{0.000000pt}%
\definecolor{currentstroke}{rgb}{0.282910,0.105393,0.426902}%
\pgfsetstrokecolor{currentstroke}%
\pgfsetdash{}{0pt}%
\pgfpathmoveto{\pgfqpoint{6.755907in}{4.975105in}}%
\pgfpathlineto{\pgfqpoint{6.828636in}{4.914943in}}%
\pgfpathlineto{\pgfqpoint{6.680668in}{4.871671in}}%
\pgfpathclose%
\pgfusepath{fill}%
\end{pgfscope}%
\begin{pgfscope}%
\pgfpathrectangle{\pgfqpoint{0.680860in}{0.078740in}}{\pgfqpoint{7.842520in}{7.842520in}}%
\pgfusepath{clip}%
\pgfsetbuttcap%
\pgfsetroundjoin%
\definecolor{currentfill}{rgb}{0.404001,0.800275,0.362552}%
\pgfsetfillcolor{currentfill}%
\pgfsetlinewidth{0.000000pt}%
\definecolor{currentstroke}{rgb}{0.283091,0.110553,0.431554}%
\pgfsetstrokecolor{currentstroke}%
\pgfsetdash{}{0pt}%
\pgfpathmoveto{\pgfqpoint{6.013884in}{4.817371in}}%
\pgfpathlineto{\pgfqpoint{5.868020in}{4.775811in}}%
\pgfpathlineto{\pgfqpoint{5.789590in}{4.744205in}}%
\pgfpathclose%
\pgfusepath{fill}%
\end{pgfscope}%
\begin{pgfscope}%
\pgfpathrectangle{\pgfqpoint{0.680860in}{0.078740in}}{\pgfqpoint{7.842520in}{7.842520in}}%
\pgfusepath{clip}%
\pgfsetbuttcap%
\pgfsetroundjoin%
\definecolor{currentfill}{rgb}{0.440137,0.811138,0.340967}%
\pgfsetfillcolor{currentfill}%
\pgfsetlinewidth{0.000000pt}%
\definecolor{currentstroke}{rgb}{0.283197,0.115680,0.436115}%
\pgfsetstrokecolor{currentstroke}%
\pgfsetdash{}{0pt}%
\pgfpathmoveto{\pgfqpoint{6.091000in}{4.822016in}}%
\pgfpathlineto{\pgfqpoint{6.160613in}{4.861754in}}%
\pgfpathlineto{\pgfqpoint{6.237350in}{4.864197in}}%
\pgfpathclose%
\pgfusepath{fill}%
\end{pgfscope}%
\begin{pgfscope}%
\pgfpathrectangle{\pgfqpoint{0.680860in}{0.078740in}}{\pgfqpoint{7.842520in}{7.842520in}}%
\pgfusepath{clip}%
\pgfsetbuttcap%
\pgfsetroundjoin%
\definecolor{currentfill}{rgb}{0.163625,0.471133,0.558148}%
\pgfsetfillcolor{currentfill}%
\pgfsetlinewidth{0.000000pt}%
\definecolor{currentstroke}{rgb}{0.283229,0.120777,0.440584}%
\pgfsetstrokecolor{currentstroke}%
\pgfsetdash{}{0pt}%
\pgfpathmoveto{\pgfqpoint{4.049315in}{3.280147in}}%
\pgfpathlineto{\pgfqpoint{3.908524in}{3.277781in}}%
\pgfpathlineto{\pgfqpoint{3.966340in}{3.056020in}}%
\pgfpathclose%
\pgfusepath{fill}%
\end{pgfscope}%
\begin{pgfscope}%
\pgfpathrectangle{\pgfqpoint{0.680860in}{0.078740in}}{\pgfqpoint{7.842520in}{7.842520in}}%
\pgfusepath{clip}%
\pgfsetbuttcap%
\pgfsetroundjoin%
\definecolor{currentfill}{rgb}{0.246811,0.283237,0.535941}%
\pgfsetfillcolor{currentfill}%
\pgfsetlinewidth{0.000000pt}%
\definecolor{currentstroke}{rgb}{0.283187,0.125848,0.444960}%
\pgfsetstrokecolor{currentstroke}%
\pgfsetdash{}{0pt}%
\pgfpathmoveto{\pgfqpoint{3.436695in}{2.353766in}}%
\pgfpathlineto{\pgfqpoint{3.576832in}{2.332908in}}%
\pgfpathlineto{\pgfqpoint{3.659703in}{2.584688in}}%
\pgfpathclose%
\pgfusepath{fill}%
\end{pgfscope}%
\begin{pgfscope}%
\pgfpathrectangle{\pgfqpoint{0.680860in}{0.078740in}}{\pgfqpoint{7.842520in}{7.842520in}}%
\pgfusepath{clip}%
\pgfsetbuttcap%
\pgfsetroundjoin%
\definecolor{currentfill}{rgb}{0.183898,0.422383,0.556944}%
\pgfsetfillcolor{currentfill}%
\pgfsetlinewidth{0.000000pt}%
\definecolor{currentstroke}{rgb}{0.283072,0.130895,0.449241}%
\pgfsetstrokecolor{currentstroke}%
\pgfsetdash{}{0pt}%
\pgfpathmoveto{\pgfqpoint{3.742618in}{2.827205in}}%
\pgfpathlineto{\pgfqpoint{3.966340in}{3.056020in}}%
\pgfpathlineto{\pgfqpoint{3.825565in}{3.058767in}}%
\pgfpathclose%
\pgfusepath{fill}%
\end{pgfscope}%
\begin{pgfscope}%
\pgfpathrectangle{\pgfqpoint{0.680860in}{0.078740in}}{\pgfqpoint{7.842520in}{7.842520in}}%
\pgfusepath{clip}%
\pgfsetbuttcap%
\pgfsetroundjoin%
\definecolor{currentfill}{rgb}{0.280255,0.165693,0.476498}%
\pgfsetfillcolor{currentfill}%
\pgfsetlinewidth{0.000000pt}%
\definecolor{currentstroke}{rgb}{0.282884,0.135920,0.453427}%
\pgfsetstrokecolor{currentstroke}%
\pgfsetdash{}{0pt}%
\pgfpathmoveto{\pgfqpoint{3.494006in}{2.073629in}}%
\pgfpathlineto{\pgfqpoint{3.353879in}{2.100607in}}%
\pgfpathlineto{\pgfqpoint{3.271096in}{1.841909in}}%
\pgfpathclose%
\pgfusepath{fill}%
\end{pgfscope}%
\begin{pgfscope}%
\pgfpathrectangle{\pgfqpoint{0.680860in}{0.078740in}}{\pgfqpoint{7.842520in}{7.842520in}}%
\pgfusepath{clip}%
\pgfsetbuttcap%
\pgfsetroundjoin%
\definecolor{currentfill}{rgb}{0.202219,0.715272,0.476084}%
\pgfsetfillcolor{currentfill}%
\pgfsetlinewidth{0.000000pt}%
\definecolor{currentstroke}{rgb}{0.282623,0.140926,0.457517}%
\pgfsetstrokecolor{currentstroke}%
\pgfsetdash{}{0pt}%
\pgfpathmoveto{\pgfqpoint{5.114232in}{4.388092in}}%
\pgfpathlineto{\pgfqpoint{4.970796in}{4.357177in}}%
\pgfpathlineto{\pgfqpoint{4.888997in}{4.231658in}}%
\pgfpathclose%
\pgfusepath{fill}%
\end{pgfscope}%
\begin{pgfscope}%
\pgfpathrectangle{\pgfqpoint{0.680860in}{0.078740in}}{\pgfqpoint{7.842520in}{7.842520in}}%
\pgfusepath{clip}%
\pgfsetbuttcap%
\pgfsetroundjoin%
\definecolor{currentfill}{rgb}{0.440137,0.811138,0.340967}%
\pgfsetfillcolor{currentfill}%
\pgfsetlinewidth{0.000000pt}%
\definecolor{currentstroke}{rgb}{0.282290,0.145912,0.461510}%
\pgfsetstrokecolor{currentstroke}%
\pgfsetdash{}{0pt}%
\pgfpathmoveto{\pgfqpoint{6.013884in}{4.817371in}}%
\pgfpathlineto{\pgfqpoint{6.160613in}{4.861754in}}%
\pgfpathlineto{\pgfqpoint{6.091000in}{4.822016in}}%
\pgfpathclose%
\pgfusepath{fill}%
\end{pgfscope}%
\begin{pgfscope}%
\pgfpathrectangle{\pgfqpoint{0.680860in}{0.078740in}}{\pgfqpoint{7.842520in}{7.842520in}}%
\pgfusepath{clip}%
\pgfsetbuttcap%
\pgfsetroundjoin%
\definecolor{currentfill}{rgb}{0.369214,0.788888,0.382914}%
\pgfsetfillcolor{currentfill}%
\pgfsetlinewidth{0.000000pt}%
\definecolor{currentstroke}{rgb}{0.281887,0.150881,0.465405}%
\pgfsetstrokecolor{currentstroke}%
\pgfsetdash{}{0pt}%
\pgfpathmoveto{\pgfqpoint{5.644273in}{4.704157in}}%
\pgfpathlineto{\pgfqpoint{5.564729in}{4.646761in}}%
\pgfpathlineto{\pgfqpoint{5.789590in}{4.744205in}}%
\pgfpathclose%
\pgfusepath{fill}%
\end{pgfscope}%
\begin{pgfscope}%
\pgfpathrectangle{\pgfqpoint{0.680860in}{0.078740in}}{\pgfqpoint{7.842520in}{7.842520in}}%
\pgfusepath{clip}%
\pgfsetbuttcap%
\pgfsetroundjoin%
\definecolor{currentfill}{rgb}{0.487026,0.823929,0.312321}%
\pgfsetfillcolor{currentfill}%
\pgfsetlinewidth{0.000000pt}%
\definecolor{currentstroke}{rgb}{0.281412,0.155834,0.469201}%
\pgfsetstrokecolor{currentstroke}%
\pgfsetdash{}{0pt}%
\pgfpathmoveto{\pgfqpoint{6.680668in}{4.871671in}}%
\pgfpathlineto{\pgfqpoint{6.607353in}{4.927499in}}%
\pgfpathlineto{\pgfqpoint{6.755907in}{4.975105in}}%
\pgfpathclose%
\pgfusepath{fill}%
\end{pgfscope}%
\begin{pgfscope}%
\pgfpathrectangle{\pgfqpoint{0.680860in}{0.078740in}}{\pgfqpoint{7.842520in}{7.842520in}}%
\pgfusepath{clip}%
\pgfsetbuttcap%
\pgfsetroundjoin%
\definecolor{currentfill}{rgb}{0.281924,0.089666,0.412415}%
\pgfsetfillcolor{currentfill}%
\pgfsetlinewidth{0.000000pt}%
\definecolor{currentstroke}{rgb}{0.280868,0.160771,0.472899}%
\pgfsetstrokecolor{currentstroke}%
\pgfsetdash{}{0pt}%
\pgfpathmoveto{\pgfqpoint{3.188328in}{1.579461in}}%
\pgfpathlineto{\pgfqpoint{3.411220in}{1.808670in}}%
\pgfpathlineto{\pgfqpoint{3.271096in}{1.841909in}}%
\pgfpathclose%
\pgfusepath{fill}%
\end{pgfscope}%
\begin{pgfscope}%
\pgfpathrectangle{\pgfqpoint{0.680860in}{0.078740in}}{\pgfqpoint{7.842520in}{7.842520in}}%
\pgfusepath{clip}%
\pgfsetbuttcap%
\pgfsetroundjoin%
\definecolor{currentfill}{rgb}{0.269308,0.218818,0.509577}%
\pgfsetfillcolor{currentfill}%
\pgfsetlinewidth{0.000000pt}%
\definecolor{currentstroke}{rgb}{0.280255,0.165693,0.476498}%
\pgfsetstrokecolor{currentstroke}%
\pgfsetdash{}{0pt}%
\pgfpathmoveto{\pgfqpoint{3.353879in}{2.100607in}}%
\pgfpathlineto{\pgfqpoint{3.494006in}{2.073629in}}%
\pgfpathlineto{\pgfqpoint{3.576832in}{2.332908in}}%
\pgfpathclose%
\pgfusepath{fill}%
\end{pgfscope}%
\begin{pgfscope}%
\pgfpathrectangle{\pgfqpoint{0.680860in}{0.078740in}}{\pgfqpoint{7.842520in}{7.842520in}}%
\pgfusepath{clip}%
\pgfsetbuttcap%
\pgfsetroundjoin%
\definecolor{currentfill}{rgb}{0.259857,0.745492,0.444467}%
\pgfsetfillcolor{currentfill}%
\pgfsetlinewidth{0.000000pt}%
\definecolor{currentstroke}{rgb}{0.279574,0.170599,0.479997}%
\pgfsetstrokecolor{currentstroke}%
\pgfsetdash{}{0pt}%
\pgfpathmoveto{\pgfqpoint{5.114232in}{4.388092in}}%
\pgfpathlineto{\pgfqpoint{5.339539in}{4.527272in}}%
\pgfpathlineto{\pgfqpoint{5.195448in}{4.492628in}}%
\pgfpathclose%
\pgfusepath{fill}%
\end{pgfscope}%
\begin{pgfscope}%
\pgfpathrectangle{\pgfqpoint{0.680860in}{0.078740in}}{\pgfqpoint{7.842520in}{7.842520in}}%
\pgfusepath{clip}%
\pgfsetbuttcap%
\pgfsetroundjoin%
\definecolor{currentfill}{rgb}{0.119423,0.611141,0.538982}%
\pgfsetfillcolor{currentfill}%
\pgfsetlinewidth{0.000000pt}%
\definecolor{currentstroke}{rgb}{0.278826,0.175490,0.483397}%
\pgfsetstrokecolor{currentstroke}%
\pgfsetdash{}{0pt}%
\pgfpathmoveto{\pgfqpoint{4.439337in}{3.876939in}}%
\pgfpathlineto{\pgfqpoint{4.356592in}{3.696513in}}%
\pgfpathlineto{\pgfqpoint{4.581468in}{3.895539in}}%
\pgfpathclose%
\pgfusepath{fill}%
\end{pgfscope}%
\begin{pgfscope}%
\pgfpathrectangle{\pgfqpoint{0.680860in}{0.078740in}}{\pgfqpoint{7.842520in}{7.842520in}}%
\pgfusepath{clip}%
\pgfsetbuttcap%
\pgfsetroundjoin%
\definecolor{currentfill}{rgb}{0.132444,0.552216,0.553018}%
\pgfsetfillcolor{currentfill}%
\pgfsetlinewidth{0.000000pt}%
\definecolor{currentstroke}{rgb}{0.278012,0.180367,0.486697}%
\pgfsetstrokecolor{currentstroke}%
\pgfsetdash{}{0pt}%
\pgfpathmoveto{\pgfqpoint{4.356592in}{3.696513in}}%
\pgfpathlineto{\pgfqpoint{4.132261in}{3.489798in}}%
\pgfpathlineto{\pgfqpoint{4.273711in}{3.498423in}}%
\pgfpathclose%
\pgfusepath{fill}%
\end{pgfscope}%
\begin{pgfscope}%
\pgfpathrectangle{\pgfqpoint{0.680860in}{0.078740in}}{\pgfqpoint{7.842520in}{7.842520in}}%
\pgfusepath{clip}%
\pgfsetbuttcap%
\pgfsetroundjoin%
\definecolor{currentfill}{rgb}{0.208623,0.367752,0.552675}%
\pgfsetfillcolor{currentfill}%
\pgfsetlinewidth{0.000000pt}%
\definecolor{currentstroke}{rgb}{0.277134,0.185228,0.489898}%
\pgfsetstrokecolor{currentstroke}%
\pgfsetdash{}{0pt}%
\pgfpathmoveto{\pgfqpoint{3.883369in}{2.818952in}}%
\pgfpathlineto{\pgfqpoint{3.742618in}{2.827205in}}%
\pgfpathlineto{\pgfqpoint{3.659703in}{2.584688in}}%
\pgfpathclose%
\pgfusepath{fill}%
\end{pgfscope}%
\begin{pgfscope}%
\pgfpathrectangle{\pgfqpoint{0.680860in}{0.078740in}}{\pgfqpoint{7.842520in}{7.842520in}}%
\pgfusepath{clip}%
\pgfsetbuttcap%
\pgfsetroundjoin%
\definecolor{currentfill}{rgb}{0.282290,0.145912,0.461510}%
\pgfsetfillcolor{currentfill}%
\pgfsetlinewidth{0.000000pt}%
\definecolor{currentstroke}{rgb}{0.276194,0.190074,0.493001}%
\pgfsetstrokecolor{currentstroke}%
\pgfsetdash{}{0pt}%
\pgfpathmoveto{\pgfqpoint{3.271096in}{1.841909in}}%
\pgfpathlineto{\pgfqpoint{3.411220in}{1.808670in}}%
\pgfpathlineto{\pgfqpoint{3.494006in}{2.073629in}}%
\pgfpathclose%
\pgfusepath{fill}%
\end{pgfscope}%
\begin{pgfscope}%
\pgfpathrectangle{\pgfqpoint{0.680860in}{0.078740in}}{\pgfqpoint{7.842520in}{7.842520in}}%
\pgfusepath{clip}%
\pgfsetbuttcap%
\pgfsetroundjoin%
\definecolor{currentfill}{rgb}{0.149039,0.508051,0.557250}%
\pgfsetfillcolor{currentfill}%
\pgfsetlinewidth{0.000000pt}%
\definecolor{currentstroke}{rgb}{0.275191,0.194905,0.496005}%
\pgfsetstrokecolor{currentstroke}%
\pgfsetdash{}{0pt}%
\pgfpathmoveto{\pgfqpoint{4.049315in}{3.280147in}}%
\pgfpathlineto{\pgfqpoint{4.190745in}{3.283949in}}%
\pgfpathlineto{\pgfqpoint{4.132261in}{3.489798in}}%
\pgfpathclose%
\pgfusepath{fill}%
\end{pgfscope}%
\begin{pgfscope}%
\pgfpathrectangle{\pgfqpoint{0.680860in}{0.078740in}}{\pgfqpoint{7.842520in}{7.842520in}}%
\pgfusepath{clip}%
\pgfsetbuttcap%
\pgfsetroundjoin%
\definecolor{currentfill}{rgb}{0.190631,0.407061,0.556089}%
\pgfsetfillcolor{currentfill}%
\pgfsetlinewidth{0.000000pt}%
\definecolor{currentstroke}{rgb}{0.274128,0.199721,0.498911}%
\pgfsetstrokecolor{currentstroke}%
\pgfsetdash{}{0pt}%
\pgfpathmoveto{\pgfqpoint{3.883369in}{2.818952in}}%
\pgfpathlineto{\pgfqpoint{3.966340in}{3.056020in}}%
\pgfpathlineto{\pgfqpoint{3.742618in}{2.827205in}}%
\pgfpathclose%
\pgfusepath{fill}%
\end{pgfscope}%
\begin{pgfscope}%
\pgfpathrectangle{\pgfqpoint{0.680860in}{0.078740in}}{\pgfqpoint{7.842520in}{7.842520in}}%
\pgfusepath{clip}%
\pgfsetbuttcap%
\pgfsetroundjoin%
\definecolor{currentfill}{rgb}{0.278791,0.062145,0.386592}%
\pgfsetfillcolor{currentfill}%
\pgfsetlinewidth{0.000000pt}%
\definecolor{currentstroke}{rgb}{0.273006,0.204520,0.501721}%
\pgfsetstrokecolor{currentstroke}%
\pgfsetdash{}{0pt}%
\pgfpathmoveto{\pgfqpoint{3.328458in}{1.539882in}}%
\pgfpathlineto{\pgfqpoint{3.411220in}{1.808670in}}%
\pgfpathlineto{\pgfqpoint{3.188328in}{1.579461in}}%
\pgfpathclose%
\pgfusepath{fill}%
\end{pgfscope}%
\begin{pgfscope}%
\pgfpathrectangle{\pgfqpoint{0.680860in}{0.078740in}}{\pgfqpoint{7.842520in}{7.842520in}}%
\pgfusepath{clip}%
\pgfsetbuttcap%
\pgfsetroundjoin%
\definecolor{currentfill}{rgb}{0.153894,0.680203,0.504172}%
\pgfsetfillcolor{currentfill}%
\pgfsetlinewidth{0.000000pt}%
\definecolor{currentstroke}{rgb}{0.271828,0.209303,0.504434}%
\pgfsetstrokecolor{currentstroke}%
\pgfsetdash{}{0pt}%
\pgfpathmoveto{\pgfqpoint{4.888997in}{4.231658in}}%
\pgfpathlineto{\pgfqpoint{4.663989in}{4.060447in}}%
\pgfpathlineto{\pgfqpoint{4.806813in}{4.084401in}}%
\pgfpathclose%
\pgfusepath{fill}%
\end{pgfscope}%
\begin{pgfscope}%
\pgfpathrectangle{\pgfqpoint{0.680860in}{0.078740in}}{\pgfqpoint{7.842520in}{7.842520in}}%
\pgfusepath{clip}%
\pgfsetbuttcap%
\pgfsetroundjoin%
\definecolor{currentfill}{rgb}{0.141935,0.526453,0.555991}%
\pgfsetfillcolor{currentfill}%
\pgfsetlinewidth{0.000000pt}%
\definecolor{currentstroke}{rgb}{0.270595,0.214069,0.507052}%
\pgfsetstrokecolor{currentstroke}%
\pgfsetdash{}{0pt}%
\pgfpathmoveto{\pgfqpoint{4.190745in}{3.283949in}}%
\pgfpathlineto{\pgfqpoint{4.273711in}{3.498423in}}%
\pgfpathlineto{\pgfqpoint{4.132261in}{3.489798in}}%
\pgfpathclose%
\pgfusepath{fill}%
\end{pgfscope}%
\begin{pgfscope}%
\pgfpathrectangle{\pgfqpoint{0.680860in}{0.078740in}}{\pgfqpoint{7.842520in}{7.842520in}}%
\pgfusepath{clip}%
\pgfsetbuttcap%
\pgfsetroundjoin%
\definecolor{currentfill}{rgb}{0.243113,0.292092,0.538516}%
\pgfsetfillcolor{currentfill}%
\pgfsetlinewidth{0.000000pt}%
\definecolor{currentstroke}{rgb}{0.269308,0.218818,0.509577}%
\pgfsetstrokecolor{currentstroke}%
\pgfsetdash{}{0pt}%
\pgfpathmoveto{\pgfqpoint{3.659703in}{2.584688in}}%
\pgfpathlineto{\pgfqpoint{3.576832in}{2.332908in}}%
\pgfpathlineto{\pgfqpoint{3.717522in}{2.312703in}}%
\pgfpathclose%
\pgfusepath{fill}%
\end{pgfscope}%
\begin{pgfscope}%
\pgfpathrectangle{\pgfqpoint{0.680860in}{0.078740in}}{\pgfqpoint{7.842520in}{7.842520in}}%
\pgfusepath{clip}%
\pgfsetbuttcap%
\pgfsetroundjoin%
\definecolor{currentfill}{rgb}{0.506271,0.828786,0.300362}%
\pgfsetfillcolor{currentfill}%
\pgfsetlinewidth{0.000000pt}%
\definecolor{currentstroke}{rgb}{0.267968,0.223549,0.512008}%
\pgfsetstrokecolor{currentstroke}%
\pgfsetdash{}{0pt}%
\pgfpathmoveto{\pgfqpoint{6.459709in}{4.882829in}}%
\pgfpathlineto{\pgfqpoint{6.532705in}{4.957195in}}%
\pgfpathlineto{\pgfqpoint{6.607353in}{4.927499in}}%
\pgfpathclose%
\pgfusepath{fill}%
\end{pgfscope}%
\begin{pgfscope}%
\pgfpathrectangle{\pgfqpoint{0.680860in}{0.078740in}}{\pgfqpoint{7.842520in}{7.842520in}}%
\pgfusepath{clip}%
\pgfsetbuttcap%
\pgfsetroundjoin%
\definecolor{currentfill}{rgb}{0.132268,0.655014,0.519661}%
\pgfsetfillcolor{currentfill}%
\pgfsetlinewidth{0.000000pt}%
\definecolor{currentstroke}{rgb}{0.266580,0.228262,0.514349}%
\pgfsetstrokecolor{currentstroke}%
\pgfsetdash{}{0pt}%
\pgfpathmoveto{\pgfqpoint{4.806813in}{4.084401in}}%
\pgfpathlineto{\pgfqpoint{4.663989in}{4.060447in}}%
\pgfpathlineto{\pgfqpoint{4.581468in}{3.895539in}}%
\pgfpathclose%
\pgfusepath{fill}%
\end{pgfscope}%
\begin{pgfscope}%
\pgfpathrectangle{\pgfqpoint{0.680860in}{0.078740in}}{\pgfqpoint{7.842520in}{7.842520in}}%
\pgfusepath{clip}%
\pgfsetbuttcap%
\pgfsetroundjoin%
\definecolor{currentfill}{rgb}{0.335885,0.777018,0.402049}%
\pgfsetfillcolor{currentfill}%
\pgfsetlinewidth{0.000000pt}%
\definecolor{currentstroke}{rgb}{0.265145,0.232956,0.516599}%
\pgfsetstrokecolor{currentstroke}%
\pgfsetdash{}{0pt}%
\pgfpathmoveto{\pgfqpoint{5.564729in}{4.646761in}}%
\pgfpathlineto{\pgfqpoint{5.420007in}{4.609045in}}%
\pgfpathlineto{\pgfqpoint{5.484436in}{4.564490in}}%
\pgfpathclose%
\pgfusepath{fill}%
\end{pgfscope}%
\begin{pgfscope}%
\pgfpathrectangle{\pgfqpoint{0.680860in}{0.078740in}}{\pgfqpoint{7.842520in}{7.842520in}}%
\pgfusepath{clip}%
\pgfsetbuttcap%
\pgfsetroundjoin%
\definecolor{currentfill}{rgb}{0.319809,0.770914,0.411152}%
\pgfsetfillcolor{currentfill}%
\pgfsetlinewidth{0.000000pt}%
\definecolor{currentstroke}{rgb}{0.263663,0.237631,0.518762}%
\pgfsetstrokecolor{currentstroke}%
\pgfsetdash{}{0pt}%
\pgfpathmoveto{\pgfqpoint{5.420007in}{4.609045in}}%
\pgfpathlineto{\pgfqpoint{5.339539in}{4.527272in}}%
\pgfpathlineto{\pgfqpoint{5.484436in}{4.564490in}}%
\pgfpathclose%
\pgfusepath{fill}%
\end{pgfscope}%
\begin{pgfscope}%
\pgfpathrectangle{\pgfqpoint{0.680860in}{0.078740in}}{\pgfqpoint{7.842520in}{7.842520in}}%
\pgfusepath{clip}%
\pgfsetbuttcap%
\pgfsetroundjoin%
\definecolor{currentfill}{rgb}{0.160665,0.478540,0.558115}%
\pgfsetfillcolor{currentfill}%
\pgfsetlinewidth{0.000000pt}%
\definecolor{currentstroke}{rgb}{0.262138,0.242286,0.520837}%
\pgfsetstrokecolor{currentstroke}%
\pgfsetdash{}{0pt}%
\pgfpathmoveto{\pgfqpoint{4.049315in}{3.280147in}}%
\pgfpathlineto{\pgfqpoint{3.966340in}{3.056020in}}%
\pgfpathlineto{\pgfqpoint{4.190745in}{3.283949in}}%
\pgfpathclose%
\pgfusepath{fill}%
\end{pgfscope}%
\begin{pgfscope}%
\pgfpathrectangle{\pgfqpoint{0.680860in}{0.078740in}}{\pgfqpoint{7.842520in}{7.842520in}}%
\pgfusepath{clip}%
\pgfsetbuttcap%
\pgfsetroundjoin%
\definecolor{currentfill}{rgb}{0.214298,0.355619,0.551184}%
\pgfsetfillcolor{currentfill}%
\pgfsetlinewidth{0.000000pt}%
\definecolor{currentstroke}{rgb}{0.260571,0.246922,0.522828}%
\pgfsetstrokecolor{currentstroke}%
\pgfsetdash{}{0pt}%
\pgfpathmoveto{\pgfqpoint{3.659703in}{2.584688in}}%
\pgfpathlineto{\pgfqpoint{3.800424in}{2.570595in}}%
\pgfpathlineto{\pgfqpoint{3.883369in}{2.818952in}}%
\pgfpathclose%
\pgfusepath{fill}%
\end{pgfscope}%
\begin{pgfscope}%
\pgfpathrectangle{\pgfqpoint{0.680860in}{0.078740in}}{\pgfqpoint{7.842520in}{7.842520in}}%
\pgfusepath{clip}%
\pgfsetbuttcap%
\pgfsetroundjoin%
\definecolor{currentfill}{rgb}{0.487026,0.823929,0.312321}%
\pgfsetfillcolor{currentfill}%
\pgfsetlinewidth{0.000000pt}%
\definecolor{currentstroke}{rgb}{0.258965,0.251537,0.524736}%
\pgfsetstrokecolor{currentstroke}%
\pgfsetdash{}{0pt}%
\pgfpathmoveto{\pgfqpoint{6.160613in}{4.861754in}}%
\pgfpathlineto{\pgfqpoint{6.384576in}{4.909220in}}%
\pgfpathlineto{\pgfqpoint{6.237350in}{4.864197in}}%
\pgfpathclose%
\pgfusepath{fill}%
\end{pgfscope}%
\begin{pgfscope}%
\pgfpathrectangle{\pgfqpoint{0.680860in}{0.078740in}}{\pgfqpoint{7.842520in}{7.842520in}}%
\pgfusepath{clip}%
\pgfsetbuttcap%
\pgfsetroundjoin%
\definecolor{currentfill}{rgb}{0.267968,0.223549,0.512008}%
\pgfsetfillcolor{currentfill}%
\pgfsetlinewidth{0.000000pt}%
\definecolor{currentstroke}{rgb}{0.257322,0.256130,0.526563}%
\pgfsetstrokecolor{currentstroke}%
\pgfsetdash{}{0pt}%
\pgfpathmoveto{\pgfqpoint{3.576832in}{2.332908in}}%
\pgfpathlineto{\pgfqpoint{3.494006in}{2.073629in}}%
\pgfpathlineto{\pgfqpoint{3.634667in}{2.047105in}}%
\pgfpathclose%
\pgfusepath{fill}%
\end{pgfscope}%
\begin{pgfscope}%
\pgfpathrectangle{\pgfqpoint{0.680860in}{0.078740in}}{\pgfqpoint{7.842520in}{7.842520in}}%
\pgfusepath{clip}%
\pgfsetbuttcap%
\pgfsetroundjoin%
\definecolor{currentfill}{rgb}{0.208030,0.718701,0.472873}%
\pgfsetfillcolor{currentfill}%
\pgfsetlinewidth{0.000000pt}%
\definecolor{currentstroke}{rgb}{0.255645,0.260703,0.528312}%
\pgfsetstrokecolor{currentstroke}%
\pgfsetdash{}{0pt}%
\pgfpathmoveto{\pgfqpoint{4.888997in}{4.231658in}}%
\pgfpathlineto{\pgfqpoint{5.032519in}{4.260562in}}%
\pgfpathlineto{\pgfqpoint{5.114232in}{4.388092in}}%
\pgfpathclose%
\pgfusepath{fill}%
\end{pgfscope}%
\begin{pgfscope}%
\pgfpathrectangle{\pgfqpoint{0.680860in}{0.078740in}}{\pgfqpoint{7.842520in}{7.842520in}}%
\pgfusepath{clip}%
\pgfsetbuttcap%
\pgfsetroundjoin%
\definecolor{currentfill}{rgb}{0.119738,0.603785,0.541400}%
\pgfsetfillcolor{currentfill}%
\pgfsetlinewidth{0.000000pt}%
\definecolor{currentstroke}{rgb}{0.253935,0.265254,0.529983}%
\pgfsetstrokecolor{currentstroke}%
\pgfsetdash{}{0pt}%
\pgfpathmoveto{\pgfqpoint{4.581468in}{3.895539in}}%
\pgfpathlineto{\pgfqpoint{4.356592in}{3.696513in}}%
\pgfpathlineto{\pgfqpoint{4.498729in}{3.711245in}}%
\pgfpathclose%
\pgfusepath{fill}%
\end{pgfscope}%
\begin{pgfscope}%
\pgfpathrectangle{\pgfqpoint{0.680860in}{0.078740in}}{\pgfqpoint{7.842520in}{7.842520in}}%
\pgfusepath{clip}%
\pgfsetbuttcap%
\pgfsetroundjoin%
\definecolor{currentfill}{rgb}{0.430983,0.808473,0.346476}%
\pgfsetfillcolor{currentfill}%
\pgfsetlinewidth{0.000000pt}%
\definecolor{currentstroke}{rgb}{0.252194,0.269783,0.531579}%
\pgfsetstrokecolor{currentstroke}%
\pgfsetdash{}{0pt}%
\pgfpathmoveto{\pgfqpoint{5.789590in}{4.744205in}}%
\pgfpathlineto{\pgfqpoint{5.935756in}{4.787022in}}%
\pgfpathlineto{\pgfqpoint{6.013884in}{4.817371in}}%
\pgfpathclose%
\pgfusepath{fill}%
\end{pgfscope}%
\begin{pgfscope}%
\pgfpathrectangle{\pgfqpoint{0.680860in}{0.078740in}}{\pgfqpoint{7.842520in}{7.842520in}}%
\pgfusepath{clip}%
\pgfsetbuttcap%
\pgfsetroundjoin%
\definecolor{currentfill}{rgb}{0.506271,0.828786,0.300362}%
\pgfsetfillcolor{currentfill}%
\pgfsetlinewidth{0.000000pt}%
\definecolor{currentstroke}{rgb}{0.250425,0.274290,0.533103}%
\pgfsetstrokecolor{currentstroke}%
\pgfsetdash{}{0pt}%
\pgfpathmoveto{\pgfqpoint{6.459709in}{4.882829in}}%
\pgfpathlineto{\pgfqpoint{6.384576in}{4.909220in}}%
\pgfpathlineto{\pgfqpoint{6.532705in}{4.957195in}}%
\pgfpathclose%
\pgfusepath{fill}%
\end{pgfscope}%
\begin{pgfscope}%
\pgfpathrectangle{\pgfqpoint{0.680860in}{0.078740in}}{\pgfqpoint{7.842520in}{7.842520in}}%
\pgfusepath{clip}%
\pgfsetbuttcap%
\pgfsetroundjoin%
\definecolor{currentfill}{rgb}{0.233603,0.313828,0.543914}%
\pgfsetfillcolor{currentfill}%
\pgfsetlinewidth{0.000000pt}%
\definecolor{currentstroke}{rgb}{0.248629,0.278775,0.534556}%
\pgfsetstrokecolor{currentstroke}%
\pgfsetdash{}{0pt}%
\pgfpathmoveto{\pgfqpoint{3.717522in}{2.312703in}}%
\pgfpathlineto{\pgfqpoint{3.800424in}{2.570595in}}%
\pgfpathlineto{\pgfqpoint{3.659703in}{2.584688in}}%
\pgfpathclose%
\pgfusepath{fill}%
\end{pgfscope}%
\begin{pgfscope}%
\pgfpathrectangle{\pgfqpoint{0.680860in}{0.078740in}}{\pgfqpoint{7.842520in}{7.842520in}}%
\pgfusepath{clip}%
\pgfsetbuttcap%
\pgfsetroundjoin%
\definecolor{currentfill}{rgb}{0.282290,0.145912,0.461510}%
\pgfsetfillcolor{currentfill}%
\pgfsetlinewidth{0.000000pt}%
\definecolor{currentstroke}{rgb}{0.246811,0.283237,0.535941}%
\pgfsetstrokecolor{currentstroke}%
\pgfsetdash{}{0pt}%
\pgfpathmoveto{\pgfqpoint{3.411220in}{1.808670in}}%
\pgfpathlineto{\pgfqpoint{3.551858in}{1.775682in}}%
\pgfpathlineto{\pgfqpoint{3.494006in}{2.073629in}}%
\pgfpathclose%
\pgfusepath{fill}%
\end{pgfscope}%
\begin{pgfscope}%
\pgfpathrectangle{\pgfqpoint{0.680860in}{0.078740in}}{\pgfqpoint{7.842520in}{7.842520in}}%
\pgfusepath{clip}%
\pgfsetbuttcap%
\pgfsetroundjoin%
\definecolor{currentfill}{rgb}{0.266941,0.748751,0.440573}%
\pgfsetfillcolor{currentfill}%
\pgfsetlinewidth{0.000000pt}%
\definecolor{currentstroke}{rgb}{0.244972,0.287675,0.537260}%
\pgfsetstrokecolor{currentstroke}%
\pgfsetdash{}{0pt}%
\pgfpathmoveto{\pgfqpoint{5.114232in}{4.388092in}}%
\pgfpathlineto{\pgfqpoint{5.258449in}{4.421448in}}%
\pgfpathlineto{\pgfqpoint{5.339539in}{4.527272in}}%
\pgfpathclose%
\pgfusepath{fill}%
\end{pgfscope}%
\begin{pgfscope}%
\pgfpathrectangle{\pgfqpoint{0.680860in}{0.078740in}}{\pgfqpoint{7.842520in}{7.842520in}}%
\pgfusepath{clip}%
\pgfsetbuttcap%
\pgfsetroundjoin%
\definecolor{currentfill}{rgb}{0.535621,0.835785,0.281908}%
\pgfsetfillcolor{currentfill}%
\pgfsetlinewidth{0.000000pt}%
\definecolor{currentstroke}{rgb}{0.243113,0.292092,0.538516}%
\pgfsetstrokecolor{currentstroke}%
\pgfsetdash{}{0pt}%
\pgfpathmoveto{\pgfqpoint{7.197205in}{4.913775in}}%
\pgfpathlineto{\pgfqpoint{6.977516in}{4.961093in}}%
\pgfpathlineto{\pgfqpoint{7.127334in}{5.010234in}}%
\pgfpathclose%
\pgfusepath{fill}%
\end{pgfscope}%
\begin{pgfscope}%
\pgfpathrectangle{\pgfqpoint{0.680860in}{0.078740in}}{\pgfqpoint{7.842520in}{7.842520in}}%
\pgfusepath{clip}%
\pgfsetbuttcap%
\pgfsetroundjoin%
\definecolor{currentfill}{rgb}{0.128729,0.563265,0.551229}%
\pgfsetfillcolor{currentfill}%
\pgfsetlinewidth{0.000000pt}%
\definecolor{currentstroke}{rgb}{0.241237,0.296485,0.539709}%
\pgfsetstrokecolor{currentstroke}%
\pgfsetdash{}{0pt}%
\pgfpathmoveto{\pgfqpoint{4.273711in}{3.498423in}}%
\pgfpathlineto{\pgfqpoint{4.415831in}{3.508703in}}%
\pgfpathlineto{\pgfqpoint{4.356592in}{3.696513in}}%
\pgfpathclose%
\pgfusepath{fill}%
\end{pgfscope}%
\begin{pgfscope}%
\pgfpathrectangle{\pgfqpoint{0.680860in}{0.078740in}}{\pgfqpoint{7.842520in}{7.842520in}}%
\pgfusepath{clip}%
\pgfsetbuttcap%
\pgfsetroundjoin%
\definecolor{currentfill}{rgb}{0.282327,0.094955,0.417331}%
\pgfsetfillcolor{currentfill}%
\pgfsetlinewidth{0.000000pt}%
\definecolor{currentstroke}{rgb}{0.239346,0.300855,0.540844}%
\pgfsetstrokecolor{currentstroke}%
\pgfsetdash{}{0pt}%
\pgfpathmoveto{\pgfqpoint{3.551858in}{1.775682in}}%
\pgfpathlineto{\pgfqpoint{3.411220in}{1.808670in}}%
\pgfpathlineto{\pgfqpoint{3.328458in}{1.539882in}}%
\pgfpathclose%
\pgfusepath{fill}%
\end{pgfscope}%
\begin{pgfscope}%
\pgfpathrectangle{\pgfqpoint{0.680860in}{0.078740in}}{\pgfqpoint{7.842520in}{7.842520in}}%
\pgfusepath{clip}%
\pgfsetbuttcap%
\pgfsetroundjoin%
\definecolor{currentfill}{rgb}{0.175707,0.697900,0.491033}%
\pgfsetfillcolor{currentfill}%
\pgfsetlinewidth{0.000000pt}%
\definecolor{currentstroke}{rgb}{0.237441,0.305202,0.541921}%
\pgfsetstrokecolor{currentstroke}%
\pgfsetdash{}{0pt}%
\pgfpathmoveto{\pgfqpoint{5.032519in}{4.260562in}}%
\pgfpathlineto{\pgfqpoint{4.888997in}{4.231658in}}%
\pgfpathlineto{\pgfqpoint{4.806813in}{4.084401in}}%
\pgfpathclose%
\pgfusepath{fill}%
\end{pgfscope}%
\begin{pgfscope}%
\pgfpathrectangle{\pgfqpoint{0.680860in}{0.078740in}}{\pgfqpoint{7.842520in}{7.842520in}}%
\pgfusepath{clip}%
\pgfsetbuttcap%
\pgfsetroundjoin%
\definecolor{currentfill}{rgb}{0.386433,0.794644,0.372886}%
\pgfsetfillcolor{currentfill}%
\pgfsetlinewidth{0.000000pt}%
\definecolor{currentstroke}{rgb}{0.235526,0.309527,0.542944}%
\pgfsetstrokecolor{currentstroke}%
\pgfsetdash{}{0pt}%
\pgfpathmoveto{\pgfqpoint{5.789590in}{4.744205in}}%
\pgfpathlineto{\pgfqpoint{5.564729in}{4.646761in}}%
\pgfpathlineto{\pgfqpoint{5.710280in}{4.687163in}}%
\pgfpathclose%
\pgfusepath{fill}%
\end{pgfscope}%
\begin{pgfscope}%
\pgfpathrectangle{\pgfqpoint{0.680860in}{0.078740in}}{\pgfqpoint{7.842520in}{7.842520in}}%
\pgfusepath{clip}%
\pgfsetbuttcap%
\pgfsetroundjoin%
\definecolor{currentfill}{rgb}{0.260571,0.246922,0.522828}%
\pgfsetfillcolor{currentfill}%
\pgfsetlinewidth{0.000000pt}%
\definecolor{currentstroke}{rgb}{0.233603,0.313828,0.543914}%
\pgfsetstrokecolor{currentstroke}%
\pgfsetdash{}{0pt}%
\pgfpathmoveto{\pgfqpoint{3.634667in}{2.047105in}}%
\pgfpathlineto{\pgfqpoint{3.717522in}{2.312703in}}%
\pgfpathlineto{\pgfqpoint{3.576832in}{2.332908in}}%
\pgfpathclose%
\pgfusepath{fill}%
\end{pgfscope}%
\begin{pgfscope}%
\pgfpathrectangle{\pgfqpoint{0.680860in}{0.078740in}}{\pgfqpoint{7.842520in}{7.842520in}}%
\pgfusepath{clip}%
\pgfsetbuttcap%
\pgfsetroundjoin%
\definecolor{currentfill}{rgb}{0.187231,0.414746,0.556547}%
\pgfsetfillcolor{currentfill}%
\pgfsetlinewidth{0.000000pt}%
\definecolor{currentstroke}{rgb}{0.231674,0.318106,0.544834}%
\pgfsetstrokecolor{currentstroke}%
\pgfsetdash{}{0pt}%
\pgfpathmoveto{\pgfqpoint{3.883369in}{2.818952in}}%
\pgfpathlineto{\pgfqpoint{4.024722in}{2.811777in}}%
\pgfpathlineto{\pgfqpoint{3.966340in}{3.056020in}}%
\pgfpathclose%
\pgfusepath{fill}%
\end{pgfscope}%
\begin{pgfscope}%
\pgfpathrectangle{\pgfqpoint{0.680860in}{0.078740in}}{\pgfqpoint{7.842520in}{7.842520in}}%
\pgfusepath{clip}%
\pgfsetbuttcap%
\pgfsetroundjoin%
\definecolor{currentfill}{rgb}{0.165117,0.467423,0.558141}%
\pgfsetfillcolor{currentfill}%
\pgfsetlinewidth{0.000000pt}%
\definecolor{currentstroke}{rgb}{0.229739,0.322361,0.545706}%
\pgfsetstrokecolor{currentstroke}%
\pgfsetdash{}{0pt}%
\pgfpathmoveto{\pgfqpoint{4.190745in}{3.283949in}}%
\pgfpathlineto{\pgfqpoint{3.966340in}{3.056020in}}%
\pgfpathlineto{\pgfqpoint{4.107737in}{3.054537in}}%
\pgfpathclose%
\pgfusepath{fill}%
\end{pgfscope}%
\begin{pgfscope}%
\pgfpathrectangle{\pgfqpoint{0.680860in}{0.078740in}}{\pgfqpoint{7.842520in}{7.842520in}}%
\pgfusepath{clip}%
\pgfsetbuttcap%
\pgfsetroundjoin%
\definecolor{currentfill}{rgb}{0.458674,0.816363,0.329727}%
\pgfsetfillcolor{currentfill}%
\pgfsetlinewidth{0.000000pt}%
\definecolor{currentstroke}{rgb}{0.227802,0.326594,0.546532}%
\pgfsetstrokecolor{currentstroke}%
\pgfsetdash{}{0pt}%
\pgfpathmoveto{\pgfqpoint{6.013884in}{4.817371in}}%
\pgfpathlineto{\pgfqpoint{5.935756in}{4.787022in}}%
\pgfpathlineto{\pgfqpoint{6.160613in}{4.861754in}}%
\pgfpathclose%
\pgfusepath{fill}%
\end{pgfscope}%
\begin{pgfscope}%
\pgfpathrectangle{\pgfqpoint{0.680860in}{0.078740in}}{\pgfqpoint{7.842520in}{7.842520in}}%
\pgfusepath{clip}%
\pgfsetbuttcap%
\pgfsetroundjoin%
\definecolor{currentfill}{rgb}{0.278826,0.175490,0.483397}%
\pgfsetfillcolor{currentfill}%
\pgfsetlinewidth{0.000000pt}%
\definecolor{currentstroke}{rgb}{0.225863,0.330805,0.547314}%
\pgfsetstrokecolor{currentstroke}%
\pgfsetdash{}{0pt}%
\pgfpathmoveto{\pgfqpoint{3.494006in}{2.073629in}}%
\pgfpathlineto{\pgfqpoint{3.551858in}{1.775682in}}%
\pgfpathlineto{\pgfqpoint{3.634667in}{2.047105in}}%
\pgfpathclose%
\pgfusepath{fill}%
\end{pgfscope}%
\begin{pgfscope}%
\pgfpathrectangle{\pgfqpoint{0.680860in}{0.078740in}}{\pgfqpoint{7.842520in}{7.842520in}}%
\pgfusepath{clip}%
\pgfsetbuttcap%
\pgfsetroundjoin%
\definecolor{currentfill}{rgb}{0.545524,0.838039,0.275626}%
\pgfsetfillcolor{currentfill}%
\pgfsetlinewidth{0.000000pt}%
\definecolor{currentstroke}{rgb}{0.223925,0.334994,0.548053}%
\pgfsetstrokecolor{currentstroke}%
\pgfsetdash{}{0pt}%
\pgfpathmoveto{\pgfqpoint{6.905397in}{5.025761in}}%
\pgfpathlineto{\pgfqpoint{6.977516in}{4.961093in}}%
\pgfpathlineto{\pgfqpoint{6.828636in}{4.914943in}}%
\pgfpathclose%
\pgfusepath{fill}%
\end{pgfscope}%
\begin{pgfscope}%
\pgfpathrectangle{\pgfqpoint{0.680860in}{0.078740in}}{\pgfqpoint{7.842520in}{7.842520in}}%
\pgfusepath{clip}%
\pgfsetbuttcap%
\pgfsetroundjoin%
\definecolor{currentfill}{rgb}{0.123463,0.581687,0.547445}%
\pgfsetfillcolor{currentfill}%
\pgfsetlinewidth{0.000000pt}%
\definecolor{currentstroke}{rgb}{0.221989,0.339161,0.548752}%
\pgfsetstrokecolor{currentstroke}%
\pgfsetdash{}{0pt}%
\pgfpathmoveto{\pgfqpoint{4.356592in}{3.696513in}}%
\pgfpathlineto{\pgfqpoint{4.415831in}{3.508703in}}%
\pgfpathlineto{\pgfqpoint{4.498729in}{3.711245in}}%
\pgfpathclose%
\pgfusepath{fill}%
\end{pgfscope}%
\begin{pgfscope}%
\pgfpathrectangle{\pgfqpoint{0.680860in}{0.078740in}}{\pgfqpoint{7.842520in}{7.842520in}}%
\pgfusepath{clip}%
\pgfsetbuttcap%
\pgfsetroundjoin%
\definecolor{currentfill}{rgb}{0.132268,0.655014,0.519661}%
\pgfsetfillcolor{currentfill}%
\pgfsetlinewidth{0.000000pt}%
\definecolor{currentstroke}{rgb}{0.220057,0.343307,0.549413}%
\pgfsetstrokecolor{currentstroke}%
\pgfsetdash{}{0pt}%
\pgfpathmoveto{\pgfqpoint{4.581468in}{3.895539in}}%
\pgfpathlineto{\pgfqpoint{4.724314in}{3.916143in}}%
\pgfpathlineto{\pgfqpoint{4.806813in}{4.084401in}}%
\pgfpathclose%
\pgfusepath{fill}%
\end{pgfscope}%
\begin{pgfscope}%
\pgfpathrectangle{\pgfqpoint{0.680860in}{0.078740in}}{\pgfqpoint{7.842520in}{7.842520in}}%
\pgfusepath{clip}%
\pgfsetbuttcap%
\pgfsetroundjoin%
\definecolor{currentfill}{rgb}{0.203063,0.379716,0.553925}%
\pgfsetfillcolor{currentfill}%
\pgfsetlinewidth{0.000000pt}%
\definecolor{currentstroke}{rgb}{0.218130,0.347432,0.550038}%
\pgfsetstrokecolor{currentstroke}%
\pgfsetdash{}{0pt}%
\pgfpathmoveto{\pgfqpoint{3.800424in}{2.570595in}}%
\pgfpathlineto{\pgfqpoint{4.024722in}{2.811777in}}%
\pgfpathlineto{\pgfqpoint{3.883369in}{2.818952in}}%
\pgfpathclose%
\pgfusepath{fill}%
\end{pgfscope}%
\begin{pgfscope}%
\pgfpathrectangle{\pgfqpoint{0.680860in}{0.078740in}}{\pgfqpoint{7.842520in}{7.842520in}}%
\pgfusepath{clip}%
\pgfsetbuttcap%
\pgfsetroundjoin%
\definecolor{currentfill}{rgb}{0.279566,0.067836,0.391917}%
\pgfsetfillcolor{currentfill}%
\pgfsetlinewidth{0.000000pt}%
\definecolor{currentstroke}{rgb}{0.216210,0.351535,0.550627}%
\pgfsetstrokecolor{currentstroke}%
\pgfsetdash{}{0pt}%
\pgfpathmoveto{\pgfqpoint{3.551858in}{1.775682in}}%
\pgfpathlineto{\pgfqpoint{3.328458in}{1.539882in}}%
\pgfpathlineto{\pgfqpoint{3.469081in}{1.500349in}}%
\pgfpathclose%
\pgfusepath{fill}%
\end{pgfscope}%
\begin{pgfscope}%
\pgfpathrectangle{\pgfqpoint{0.680860in}{0.078740in}}{\pgfqpoint{7.842520in}{7.842520in}}%
\pgfusepath{clip}%
\pgfsetbuttcap%
\pgfsetroundjoin%
\definecolor{currentfill}{rgb}{0.137770,0.537492,0.554906}%
\pgfsetfillcolor{currentfill}%
\pgfsetlinewidth{0.000000pt}%
\definecolor{currentstroke}{rgb}{0.214298,0.355619,0.551184}%
\pgfsetstrokecolor{currentstroke}%
\pgfsetdash{}{0pt}%
\pgfpathmoveto{\pgfqpoint{4.415831in}{3.508703in}}%
\pgfpathlineto{\pgfqpoint{4.273711in}{3.498423in}}%
\pgfpathlineto{\pgfqpoint{4.190745in}{3.283949in}}%
\pgfpathclose%
\pgfusepath{fill}%
\end{pgfscope}%
\begin{pgfscope}%
\pgfpathrectangle{\pgfqpoint{0.680860in}{0.078740in}}{\pgfqpoint{7.842520in}{7.842520in}}%
\pgfusepath{clip}%
\pgfsetbuttcap%
\pgfsetroundjoin%
\definecolor{currentfill}{rgb}{0.177423,0.437527,0.557565}%
\pgfsetfillcolor{currentfill}%
\pgfsetlinewidth{0.000000pt}%
\definecolor{currentstroke}{rgb}{0.212395,0.359683,0.551710}%
\pgfsetstrokecolor{currentstroke}%
\pgfsetdash{}{0pt}%
\pgfpathmoveto{\pgfqpoint{3.966340in}{3.056020in}}%
\pgfpathlineto{\pgfqpoint{4.024722in}{2.811777in}}%
\pgfpathlineto{\pgfqpoint{4.107737in}{3.054537in}}%
\pgfpathclose%
\pgfusepath{fill}%
\end{pgfscope}%
\begin{pgfscope}%
\pgfpathrectangle{\pgfqpoint{0.680860in}{0.078740in}}{\pgfqpoint{7.842520in}{7.842520in}}%
\pgfusepath{clip}%
\pgfsetbuttcap%
\pgfsetroundjoin%
\definecolor{currentfill}{rgb}{0.555484,0.840254,0.269281}%
\pgfsetfillcolor{currentfill}%
\pgfsetlinewidth{0.000000pt}%
\definecolor{currentstroke}{rgb}{0.210503,0.363727,0.552206}%
\pgfsetstrokecolor{currentstroke}%
\pgfsetdash{}{0pt}%
\pgfpathmoveto{\pgfqpoint{6.905397in}{5.025761in}}%
\pgfpathlineto{\pgfqpoint{6.828636in}{4.914943in}}%
\pgfpathlineto{\pgfqpoint{6.755907in}{4.975105in}}%
\pgfpathclose%
\pgfusepath{fill}%
\end{pgfscope}%
\begin{pgfscope}%
\pgfpathrectangle{\pgfqpoint{0.680860in}{0.078740in}}{\pgfqpoint{7.842520in}{7.842520in}}%
\pgfusepath{clip}%
\pgfsetbuttcap%
\pgfsetroundjoin%
\definecolor{currentfill}{rgb}{0.369214,0.788888,0.382914}%
\pgfsetfillcolor{currentfill}%
\pgfsetlinewidth{0.000000pt}%
\definecolor{currentstroke}{rgb}{0.208623,0.367752,0.552675}%
\pgfsetstrokecolor{currentstroke}%
\pgfsetdash{}{0pt}%
\pgfpathmoveto{\pgfqpoint{5.710280in}{4.687163in}}%
\pgfpathlineto{\pgfqpoint{5.564729in}{4.646761in}}%
\pgfpathlineto{\pgfqpoint{5.484436in}{4.564490in}}%
\pgfpathclose%
\pgfusepath{fill}%
\end{pgfscope}%
\begin{pgfscope}%
\pgfpathrectangle{\pgfqpoint{0.680860in}{0.078740in}}{\pgfqpoint{7.842520in}{7.842520in}}%
\pgfusepath{clip}%
\pgfsetbuttcap%
\pgfsetroundjoin%
\definecolor{currentfill}{rgb}{0.545524,0.838039,0.275626}%
\pgfsetfillcolor{currentfill}%
\pgfsetlinewidth{0.000000pt}%
\definecolor{currentstroke}{rgb}{0.206756,0.371758,0.553117}%
\pgfsetstrokecolor{currentstroke}%
\pgfsetdash{}{0pt}%
\pgfpathmoveto{\pgfqpoint{6.755907in}{4.975105in}}%
\pgfpathlineto{\pgfqpoint{6.607353in}{4.927499in}}%
\pgfpathlineto{\pgfqpoint{6.532705in}{4.957195in}}%
\pgfpathclose%
\pgfusepath{fill}%
\end{pgfscope}%
\begin{pgfscope}%
\pgfpathrectangle{\pgfqpoint{0.680860in}{0.078740in}}{\pgfqpoint{7.842520in}{7.842520in}}%
\pgfusepath{clip}%
\pgfsetbuttcap%
\pgfsetroundjoin%
\definecolor{currentfill}{rgb}{0.565498,0.842430,0.262877}%
\pgfsetfillcolor{currentfill}%
\pgfsetlinewidth{0.000000pt}%
\definecolor{currentstroke}{rgb}{0.204903,0.375746,0.553533}%
\pgfsetstrokecolor{currentstroke}%
\pgfsetdash{}{0pt}%
\pgfpathmoveto{\pgfqpoint{7.197205in}{4.913775in}}%
\pgfpathlineto{\pgfqpoint{7.278117in}{5.062483in}}%
\pgfpathlineto{\pgfqpoint{7.347229in}{4.960230in}}%
\pgfpathclose%
\pgfusepath{fill}%
\end{pgfscope}%
\begin{pgfscope}%
\pgfpathrectangle{\pgfqpoint{0.680860in}{0.078740in}}{\pgfqpoint{7.842520in}{7.842520in}}%
\pgfusepath{clip}%
\pgfsetbuttcap%
\pgfsetroundjoin%
\definecolor{currentfill}{rgb}{0.119483,0.614817,0.537692}%
\pgfsetfillcolor{currentfill}%
\pgfsetlinewidth{0.000000pt}%
\definecolor{currentstroke}{rgb}{0.203063,0.379716,0.553925}%
\pgfsetstrokecolor{currentstroke}%
\pgfsetdash{}{0pt}%
\pgfpathmoveto{\pgfqpoint{4.641567in}{3.727849in}}%
\pgfpathlineto{\pgfqpoint{4.581468in}{3.895539in}}%
\pgfpathlineto{\pgfqpoint{4.498729in}{3.711245in}}%
\pgfpathclose%
\pgfusepath{fill}%
\end{pgfscope}%
\begin{pgfscope}%
\pgfpathrectangle{\pgfqpoint{0.680860in}{0.078740in}}{\pgfqpoint{7.842520in}{7.842520in}}%
\pgfusepath{clip}%
\pgfsetbuttcap%
\pgfsetroundjoin%
\definecolor{currentfill}{rgb}{0.239346,0.300855,0.540844}%
\pgfsetfillcolor{currentfill}%
\pgfsetlinewidth{0.000000pt}%
\definecolor{currentstroke}{rgb}{0.201239,0.383670,0.554294}%
\pgfsetstrokecolor{currentstroke}%
\pgfsetdash{}{0pt}%
\pgfpathmoveto{\pgfqpoint{3.858773in}{2.293179in}}%
\pgfpathlineto{\pgfqpoint{3.800424in}{2.570595in}}%
\pgfpathlineto{\pgfqpoint{3.717522in}{2.312703in}}%
\pgfpathclose%
\pgfusepath{fill}%
\end{pgfscope}%
\begin{pgfscope}%
\pgfpathrectangle{\pgfqpoint{0.680860in}{0.078740in}}{\pgfqpoint{7.842520in}{7.842520in}}%
\pgfusepath{clip}%
\pgfsetbuttcap%
\pgfsetroundjoin%
\definecolor{currentfill}{rgb}{0.226397,0.728888,0.462789}%
\pgfsetfillcolor{currentfill}%
\pgfsetlinewidth{0.000000pt}%
\definecolor{currentstroke}{rgb}{0.199430,0.387607,0.554642}%
\pgfsetstrokecolor{currentstroke}%
\pgfsetdash{}{0pt}%
\pgfpathmoveto{\pgfqpoint{5.032519in}{4.260562in}}%
\pgfpathlineto{\pgfqpoint{5.176815in}{4.291839in}}%
\pgfpathlineto{\pgfqpoint{5.114232in}{4.388092in}}%
\pgfpathclose%
\pgfusepath{fill}%
\end{pgfscope}%
\begin{pgfscope}%
\pgfpathrectangle{\pgfqpoint{0.680860in}{0.078740in}}{\pgfqpoint{7.842520in}{7.842520in}}%
\pgfusepath{clip}%
\pgfsetbuttcap%
\pgfsetroundjoin%
\definecolor{currentfill}{rgb}{0.257322,0.256130,0.526563}%
\pgfsetfillcolor{currentfill}%
\pgfsetlinewidth{0.000000pt}%
\definecolor{currentstroke}{rgb}{0.197636,0.391528,0.554969}%
\pgfsetstrokecolor{currentstroke}%
\pgfsetdash{}{0pt}%
\pgfpathmoveto{\pgfqpoint{3.858773in}{2.293179in}}%
\pgfpathlineto{\pgfqpoint{3.717522in}{2.312703in}}%
\pgfpathlineto{\pgfqpoint{3.634667in}{2.047105in}}%
\pgfpathclose%
\pgfusepath{fill}%
\end{pgfscope}%
\begin{pgfscope}%
\pgfpathrectangle{\pgfqpoint{0.680860in}{0.078740in}}{\pgfqpoint{7.842520in}{7.842520in}}%
\pgfusepath{clip}%
\pgfsetbuttcap%
\pgfsetroundjoin%
\definecolor{currentfill}{rgb}{0.252899,0.742211,0.448284}%
\pgfsetfillcolor{currentfill}%
\pgfsetlinewidth{0.000000pt}%
\definecolor{currentstroke}{rgb}{0.195860,0.395433,0.555276}%
\pgfsetstrokecolor{currentstroke}%
\pgfsetdash{}{0pt}%
\pgfpathmoveto{\pgfqpoint{5.176815in}{4.291839in}}%
\pgfpathlineto{\pgfqpoint{5.258449in}{4.421448in}}%
\pgfpathlineto{\pgfqpoint{5.114232in}{4.388092in}}%
\pgfpathclose%
\pgfusepath{fill}%
\end{pgfscope}%
\begin{pgfscope}%
\pgfpathrectangle{\pgfqpoint{0.680860in}{0.078740in}}{\pgfqpoint{7.842520in}{7.842520in}}%
\pgfusepath{clip}%
\pgfsetbuttcap%
\pgfsetroundjoin%
\definecolor{currentfill}{rgb}{0.175707,0.697900,0.491033}%
\pgfsetfillcolor{currentfill}%
\pgfsetlinewidth{0.000000pt}%
\definecolor{currentstroke}{rgb}{0.194100,0.399323,0.555565}%
\pgfsetstrokecolor{currentstroke}%
\pgfsetdash{}{0pt}%
\pgfpathmoveto{\pgfqpoint{4.806813in}{4.084401in}}%
\pgfpathlineto{\pgfqpoint{4.950381in}{4.110552in}}%
\pgfpathlineto{\pgfqpoint{5.032519in}{4.260562in}}%
\pgfpathclose%
\pgfusepath{fill}%
\end{pgfscope}%
\begin{pgfscope}%
\pgfpathrectangle{\pgfqpoint{0.680860in}{0.078740in}}{\pgfqpoint{7.842520in}{7.842520in}}%
\pgfusepath{clip}%
\pgfsetbuttcap%
\pgfsetroundjoin%
\definecolor{currentfill}{rgb}{0.210503,0.363727,0.552206}%
\pgfsetfillcolor{currentfill}%
\pgfsetlinewidth{0.000000pt}%
\definecolor{currentstroke}{rgb}{0.192357,0.403199,0.555836}%
\pgfsetstrokecolor{currentstroke}%
\pgfsetdash{}{0pt}%
\pgfpathmoveto{\pgfqpoint{3.800424in}{2.570595in}}%
\pgfpathlineto{\pgfqpoint{3.941727in}{2.557386in}}%
\pgfpathlineto{\pgfqpoint{4.024722in}{2.811777in}}%
\pgfpathclose%
\pgfusepath{fill}%
\end{pgfscope}%
\begin{pgfscope}%
\pgfpathrectangle{\pgfqpoint{0.680860in}{0.078740in}}{\pgfqpoint{7.842520in}{7.842520in}}%
\pgfusepath{clip}%
\pgfsetbuttcap%
\pgfsetroundjoin%
\definecolor{currentfill}{rgb}{0.122312,0.633153,0.530398}%
\pgfsetfillcolor{currentfill}%
\pgfsetlinewidth{0.000000pt}%
\definecolor{currentstroke}{rgb}{0.190631,0.407061,0.556089}%
\pgfsetstrokecolor{currentstroke}%
\pgfsetdash{}{0pt}%
\pgfpathmoveto{\pgfqpoint{4.724314in}{3.916143in}}%
\pgfpathlineto{\pgfqpoint{4.581468in}{3.895539in}}%
\pgfpathlineto{\pgfqpoint{4.641567in}{3.727849in}}%
\pgfpathclose%
\pgfusepath{fill}%
\end{pgfscope}%
\begin{pgfscope}%
\pgfpathrectangle{\pgfqpoint{0.680860in}{0.078740in}}{\pgfqpoint{7.842520in}{7.842520in}}%
\pgfusepath{clip}%
\pgfsetbuttcap%
\pgfsetroundjoin%
\definecolor{currentfill}{rgb}{0.278012,0.180367,0.486697}%
\pgfsetfillcolor{currentfill}%
\pgfsetlinewidth{0.000000pt}%
\definecolor{currentstroke}{rgb}{0.188923,0.410910,0.556326}%
\pgfsetstrokecolor{currentstroke}%
\pgfsetdash{}{0pt}%
\pgfpathmoveto{\pgfqpoint{3.634667in}{2.047105in}}%
\pgfpathlineto{\pgfqpoint{3.551858in}{1.775682in}}%
\pgfpathlineto{\pgfqpoint{3.775868in}{2.021053in}}%
\pgfpathclose%
\pgfusepath{fill}%
\end{pgfscope}%
\begin{pgfscope}%
\pgfpathrectangle{\pgfqpoint{0.680860in}{0.078740in}}{\pgfqpoint{7.842520in}{7.842520in}}%
\pgfusepath{clip}%
\pgfsetbuttcap%
\pgfsetroundjoin%
\definecolor{currentfill}{rgb}{0.525776,0.833491,0.288127}%
\pgfsetfillcolor{currentfill}%
\pgfsetlinewidth{0.000000pt}%
\definecolor{currentstroke}{rgb}{0.187231,0.414746,0.556547}%
\pgfsetstrokecolor{currentstroke}%
\pgfsetdash{}{0pt}%
\pgfpathmoveto{\pgfqpoint{6.308232in}{4.909070in}}%
\pgfpathlineto{\pgfqpoint{6.384576in}{4.909220in}}%
\pgfpathlineto{\pgfqpoint{6.160613in}{4.861754in}}%
\pgfpathclose%
\pgfusepath{fill}%
\end{pgfscope}%
\begin{pgfscope}%
\pgfpathrectangle{\pgfqpoint{0.680860in}{0.078740in}}{\pgfqpoint{7.842520in}{7.842520in}}%
\pgfusepath{clip}%
\pgfsetbuttcap%
\pgfsetroundjoin%
\definecolor{currentfill}{rgb}{0.296479,0.761561,0.424223}%
\pgfsetfillcolor{currentfill}%
\pgfsetlinewidth{0.000000pt}%
\definecolor{currentstroke}{rgb}{0.185556,0.418570,0.556753}%
\pgfsetstrokecolor{currentstroke}%
\pgfsetdash{}{0pt}%
\pgfpathmoveto{\pgfqpoint{5.339539in}{4.527272in}}%
\pgfpathlineto{\pgfqpoint{5.258449in}{4.421448in}}%
\pgfpathlineto{\pgfqpoint{5.403468in}{4.457339in}}%
\pgfpathclose%
\pgfusepath{fill}%
\end{pgfscope}%
\begin{pgfscope}%
\pgfpathrectangle{\pgfqpoint{0.680860in}{0.078740in}}{\pgfqpoint{7.842520in}{7.842520in}}%
\pgfusepath{clip}%
\pgfsetbuttcap%
\pgfsetroundjoin%
\definecolor{currentfill}{rgb}{0.229739,0.322361,0.545706}%
\pgfsetfillcolor{currentfill}%
\pgfsetlinewidth{0.000000pt}%
\definecolor{currentstroke}{rgb}{0.183898,0.422383,0.556944}%
\pgfsetstrokecolor{currentstroke}%
\pgfsetdash{}{0pt}%
\pgfpathmoveto{\pgfqpoint{3.941727in}{2.557386in}}%
\pgfpathlineto{\pgfqpoint{3.800424in}{2.570595in}}%
\pgfpathlineto{\pgfqpoint{3.858773in}{2.293179in}}%
\pgfpathclose%
\pgfusepath{fill}%
\end{pgfscope}%
\begin{pgfscope}%
\pgfpathrectangle{\pgfqpoint{0.680860in}{0.078740in}}{\pgfqpoint{7.842520in}{7.842520in}}%
\pgfusepath{clip}%
\pgfsetbuttcap%
\pgfsetroundjoin%
\definecolor{currentfill}{rgb}{0.440137,0.811138,0.340967}%
\pgfsetfillcolor{currentfill}%
\pgfsetlinewidth{0.000000pt}%
\definecolor{currentstroke}{rgb}{0.182256,0.426184,0.557120}%
\pgfsetstrokecolor{currentstroke}%
\pgfsetdash{}{0pt}%
\pgfpathmoveto{\pgfqpoint{5.856683in}{4.730356in}}%
\pgfpathlineto{\pgfqpoint{5.935756in}{4.787022in}}%
\pgfpathlineto{\pgfqpoint{5.789590in}{4.744205in}}%
\pgfpathclose%
\pgfusepath{fill}%
\end{pgfscope}%
\begin{pgfscope}%
\pgfpathrectangle{\pgfqpoint{0.680860in}{0.078740in}}{\pgfqpoint{7.842520in}{7.842520in}}%
\pgfusepath{clip}%
\pgfsetbuttcap%
\pgfsetroundjoin%
\definecolor{currentfill}{rgb}{0.141935,0.526453,0.555991}%
\pgfsetfillcolor{currentfill}%
\pgfsetlinewidth{0.000000pt}%
\definecolor{currentstroke}{rgb}{0.180629,0.429975,0.557282}%
\pgfsetstrokecolor{currentstroke}%
\pgfsetdash{}{0pt}%
\pgfpathmoveto{\pgfqpoint{4.190745in}{3.283949in}}%
\pgfpathlineto{\pgfqpoint{4.332828in}{3.289243in}}%
\pgfpathlineto{\pgfqpoint{4.415831in}{3.508703in}}%
\pgfpathclose%
\pgfusepath{fill}%
\end{pgfscope}%
\begin{pgfscope}%
\pgfpathrectangle{\pgfqpoint{0.680860in}{0.078740in}}{\pgfqpoint{7.842520in}{7.842520in}}%
\pgfusepath{clip}%
\pgfsetbuttcap%
\pgfsetroundjoin%
\definecolor{currentfill}{rgb}{0.160665,0.478540,0.558115}%
\pgfsetfillcolor{currentfill}%
\pgfsetlinewidth{0.000000pt}%
\definecolor{currentstroke}{rgb}{0.179019,0.433756,0.557430}%
\pgfsetstrokecolor{currentstroke}%
\pgfsetdash{}{0pt}%
\pgfpathmoveto{\pgfqpoint{4.249766in}{3.054365in}}%
\pgfpathlineto{\pgfqpoint{4.190745in}{3.283949in}}%
\pgfpathlineto{\pgfqpoint{4.107737in}{3.054537in}}%
\pgfpathclose%
\pgfusepath{fill}%
\end{pgfscope}%
\begin{pgfscope}%
\pgfpathrectangle{\pgfqpoint{0.680860in}{0.078740in}}{\pgfqpoint{7.842520in}{7.842520in}}%
\pgfusepath{clip}%
\pgfsetbuttcap%
\pgfsetroundjoin%
\definecolor{currentfill}{rgb}{0.279566,0.067836,0.391917}%
\pgfsetfillcolor{currentfill}%
\pgfsetlinewidth{0.000000pt}%
\definecolor{currentstroke}{rgb}{0.177423,0.437527,0.557565}%
\pgfsetstrokecolor{currentstroke}%
\pgfsetdash{}{0pt}%
\pgfpathmoveto{\pgfqpoint{3.610200in}{1.460869in}}%
\pgfpathlineto{\pgfqpoint{3.551858in}{1.775682in}}%
\pgfpathlineto{\pgfqpoint{3.469081in}{1.500349in}}%
\pgfpathclose%
\pgfusepath{fill}%
\end{pgfscope}%
\begin{pgfscope}%
\pgfpathrectangle{\pgfqpoint{0.680860in}{0.078740in}}{\pgfqpoint{7.842520in}{7.842520in}}%
\pgfusepath{clip}%
\pgfsetbuttcap%
\pgfsetroundjoin%
\definecolor{currentfill}{rgb}{0.319809,0.770914,0.411152}%
\pgfsetfillcolor{currentfill}%
\pgfsetlinewidth{0.000000pt}%
\definecolor{currentstroke}{rgb}{0.175841,0.441290,0.557685}%
\pgfsetstrokecolor{currentstroke}%
\pgfsetdash{}{0pt}%
\pgfpathmoveto{\pgfqpoint{5.403468in}{4.457339in}}%
\pgfpathlineto{\pgfqpoint{5.484436in}{4.564490in}}%
\pgfpathlineto{\pgfqpoint{5.339539in}{4.527272in}}%
\pgfpathclose%
\pgfusepath{fill}%
\end{pgfscope}%
\begin{pgfscope}%
\pgfpathrectangle{\pgfqpoint{0.680860in}{0.078740in}}{\pgfqpoint{7.842520in}{7.842520in}}%
\pgfusepath{clip}%
\pgfsetbuttcap%
\pgfsetroundjoin%
\definecolor{currentfill}{rgb}{0.430983,0.808473,0.346476}%
\pgfsetfillcolor{currentfill}%
\pgfsetlinewidth{0.000000pt}%
\definecolor{currentstroke}{rgb}{0.174274,0.445044,0.557792}%
\pgfsetstrokecolor{currentstroke}%
\pgfsetdash{}{0pt}%
\pgfpathmoveto{\pgfqpoint{5.789590in}{4.744205in}}%
\pgfpathlineto{\pgfqpoint{5.710280in}{4.687163in}}%
\pgfpathlineto{\pgfqpoint{5.856683in}{4.730356in}}%
\pgfpathclose%
\pgfusepath{fill}%
\end{pgfscope}%
\begin{pgfscope}%
\pgfpathrectangle{\pgfqpoint{0.680860in}{0.078740in}}{\pgfqpoint{7.842520in}{7.842520in}}%
\pgfusepath{clip}%
\pgfsetbuttcap%
\pgfsetroundjoin%
\definecolor{currentfill}{rgb}{0.585678,0.846661,0.249897}%
\pgfsetfillcolor{currentfill}%
\pgfsetlinewidth{0.000000pt}%
\definecolor{currentstroke}{rgb}{0.172719,0.448791,0.557885}%
\pgfsetstrokecolor{currentstroke}%
\pgfsetdash{}{0pt}%
\pgfpathmoveto{\pgfqpoint{7.127334in}{5.010234in}}%
\pgfpathlineto{\pgfqpoint{7.278117in}{5.062483in}}%
\pgfpathlineto{\pgfqpoint{7.197205in}{4.913775in}}%
\pgfpathclose%
\pgfusepath{fill}%
\end{pgfscope}%
\begin{pgfscope}%
\pgfpathrectangle{\pgfqpoint{0.680860in}{0.078740in}}{\pgfqpoint{7.842520in}{7.842520in}}%
\pgfusepath{clip}%
\pgfsetbuttcap%
\pgfsetroundjoin%
\definecolor{currentfill}{rgb}{0.121148,0.592739,0.544641}%
\pgfsetfillcolor{currentfill}%
\pgfsetlinewidth{0.000000pt}%
\definecolor{currentstroke}{rgb}{0.171176,0.452530,0.557965}%
\pgfsetstrokecolor{currentstroke}%
\pgfsetdash{}{0pt}%
\pgfpathmoveto{\pgfqpoint{4.498729in}{3.711245in}}%
\pgfpathlineto{\pgfqpoint{4.415831in}{3.508703in}}%
\pgfpathlineto{\pgfqpoint{4.641567in}{3.727849in}}%
\pgfpathclose%
\pgfusepath{fill}%
\end{pgfscope}%
\begin{pgfscope}%
\pgfpathrectangle{\pgfqpoint{0.680860in}{0.078740in}}{\pgfqpoint{7.842520in}{7.842520in}}%
\pgfusepath{clip}%
\pgfsetbuttcap%
\pgfsetroundjoin%
\definecolor{currentfill}{rgb}{0.265145,0.232956,0.516599}%
\pgfsetfillcolor{currentfill}%
\pgfsetlinewidth{0.000000pt}%
\definecolor{currentstroke}{rgb}{0.169646,0.456262,0.558030}%
\pgfsetstrokecolor{currentstroke}%
\pgfsetdash{}{0pt}%
\pgfpathmoveto{\pgfqpoint{3.634667in}{2.047105in}}%
\pgfpathlineto{\pgfqpoint{3.775868in}{2.021053in}}%
\pgfpathlineto{\pgfqpoint{3.858773in}{2.293179in}}%
\pgfpathclose%
\pgfusepath{fill}%
\end{pgfscope}%
\begin{pgfscope}%
\pgfpathrectangle{\pgfqpoint{0.680860in}{0.078740in}}{\pgfqpoint{7.842520in}{7.842520in}}%
\pgfusepath{clip}%
\pgfsetbuttcap%
\pgfsetroundjoin%
\definecolor{currentfill}{rgb}{0.595839,0.848717,0.243329}%
\pgfsetfillcolor{currentfill}%
\pgfsetlinewidth{0.000000pt}%
\definecolor{currentstroke}{rgb}{0.168126,0.459988,0.558082}%
\pgfsetstrokecolor{currentstroke}%
\pgfsetdash{}{0pt}%
\pgfpathmoveto{\pgfqpoint{7.127334in}{5.010234in}}%
\pgfpathlineto{\pgfqpoint{6.977516in}{4.961093in}}%
\pgfpathlineto{\pgfqpoint{6.905397in}{5.025761in}}%
\pgfpathclose%
\pgfusepath{fill}%
\end{pgfscope}%
\begin{pgfscope}%
\pgfpathrectangle{\pgfqpoint{0.680860in}{0.078740in}}{\pgfqpoint{7.842520in}{7.842520in}}%
\pgfusepath{clip}%
\pgfsetbuttcap%
\pgfsetroundjoin%
\definecolor{currentfill}{rgb}{0.140210,0.665859,0.513427}%
\pgfsetfillcolor{currentfill}%
\pgfsetlinewidth{0.000000pt}%
\definecolor{currentstroke}{rgb}{0.166617,0.463708,0.558119}%
\pgfsetstrokecolor{currentstroke}%
\pgfsetdash{}{0pt}%
\pgfpathmoveto{\pgfqpoint{4.806813in}{4.084401in}}%
\pgfpathlineto{\pgfqpoint{4.724314in}{3.916143in}}%
\pgfpathlineto{\pgfqpoint{4.867891in}{3.938828in}}%
\pgfpathclose%
\pgfusepath{fill}%
\end{pgfscope}%
\begin{pgfscope}%
\pgfpathrectangle{\pgfqpoint{0.680860in}{0.078740in}}{\pgfqpoint{7.842520in}{7.842520in}}%
\pgfusepath{clip}%
\pgfsetbuttcap%
\pgfsetroundjoin%
\definecolor{currentfill}{rgb}{0.281412,0.155834,0.469201}%
\pgfsetfillcolor{currentfill}%
\pgfsetlinewidth{0.000000pt}%
\definecolor{currentstroke}{rgb}{0.165117,0.467423,0.558141}%
\pgfsetstrokecolor{currentstroke}%
\pgfsetdash{}{0pt}%
\pgfpathmoveto{\pgfqpoint{3.775868in}{2.021053in}}%
\pgfpathlineto{\pgfqpoint{3.551858in}{1.775682in}}%
\pgfpathlineto{\pgfqpoint{3.693014in}{1.742956in}}%
\pgfpathclose%
\pgfusepath{fill}%
\end{pgfscope}%
\begin{pgfscope}%
\pgfpathrectangle{\pgfqpoint{0.680860in}{0.078740in}}{\pgfqpoint{7.842520in}{7.842520in}}%
\pgfusepath{clip}%
\pgfsetbuttcap%
\pgfsetroundjoin%
\definecolor{currentfill}{rgb}{0.153364,0.497000,0.557724}%
\pgfsetfillcolor{currentfill}%
\pgfsetlinewidth{0.000000pt}%
\definecolor{currentstroke}{rgb}{0.163625,0.471133,0.558148}%
\pgfsetstrokecolor{currentstroke}%
\pgfsetdash{}{0pt}%
\pgfpathmoveto{\pgfqpoint{4.249766in}{3.054365in}}%
\pgfpathlineto{\pgfqpoint{4.332828in}{3.289243in}}%
\pgfpathlineto{\pgfqpoint{4.190745in}{3.283949in}}%
\pgfpathclose%
\pgfusepath{fill}%
\end{pgfscope}%
\begin{pgfscope}%
\pgfpathrectangle{\pgfqpoint{0.680860in}{0.078740in}}{\pgfqpoint{7.842520in}{7.842520in}}%
\pgfusepath{clip}%
\pgfsetbuttcap%
\pgfsetroundjoin%
\definecolor{currentfill}{rgb}{0.282656,0.100196,0.422160}%
\pgfsetfillcolor{currentfill}%
\pgfsetlinewidth{0.000000pt}%
\definecolor{currentstroke}{rgb}{0.162142,0.474838,0.558140}%
\pgfsetstrokecolor{currentstroke}%
\pgfsetdash{}{0pt}%
\pgfpathmoveto{\pgfqpoint{3.693014in}{1.742956in}}%
\pgfpathlineto{\pgfqpoint{3.551858in}{1.775682in}}%
\pgfpathlineto{\pgfqpoint{3.610200in}{1.460869in}}%
\pgfpathclose%
\pgfusepath{fill}%
\end{pgfscope}%
\begin{pgfscope}%
\pgfpathrectangle{\pgfqpoint{0.680860in}{0.078740in}}{\pgfqpoint{7.842520in}{7.842520in}}%
\pgfusepath{clip}%
\pgfsetbuttcap%
\pgfsetroundjoin%
\definecolor{currentfill}{rgb}{0.496615,0.826376,0.306377}%
\pgfsetfillcolor{currentfill}%
\pgfsetlinewidth{0.000000pt}%
\definecolor{currentstroke}{rgb}{0.160665,0.478540,0.558115}%
\pgfsetstrokecolor{currentstroke}%
\pgfsetdash{}{0pt}%
\pgfpathmoveto{\pgfqpoint{5.935756in}{4.787022in}}%
\pgfpathlineto{\pgfqpoint{6.082795in}{4.832718in}}%
\pgfpathlineto{\pgfqpoint{6.160613in}{4.861754in}}%
\pgfpathclose%
\pgfusepath{fill}%
\end{pgfscope}%
\begin{pgfscope}%
\pgfpathrectangle{\pgfqpoint{0.680860in}{0.078740in}}{\pgfqpoint{7.842520in}{7.842520in}}%
\pgfusepath{clip}%
\pgfsetbuttcap%
\pgfsetroundjoin%
\definecolor{currentfill}{rgb}{0.182256,0.426184,0.557120}%
\pgfsetfillcolor{currentfill}%
\pgfsetlinewidth{0.000000pt}%
\definecolor{currentstroke}{rgb}{0.159194,0.482237,0.558073}%
\pgfsetstrokecolor{currentstroke}%
\pgfsetdash{}{0pt}%
\pgfpathmoveto{\pgfqpoint{4.107737in}{3.054537in}}%
\pgfpathlineto{\pgfqpoint{4.024722in}{2.811777in}}%
\pgfpathlineto{\pgfqpoint{4.166687in}{2.805722in}}%
\pgfpathclose%
\pgfusepath{fill}%
\end{pgfscope}%
\begin{pgfscope}%
\pgfpathrectangle{\pgfqpoint{0.680860in}{0.078740in}}{\pgfqpoint{7.842520in}{7.842520in}}%
\pgfusepath{clip}%
\pgfsetbuttcap%
\pgfsetroundjoin%
\definecolor{currentfill}{rgb}{0.208030,0.718701,0.472873}%
\pgfsetfillcolor{currentfill}%
\pgfsetlinewidth{0.000000pt}%
\definecolor{currentstroke}{rgb}{0.157729,0.485932,0.558013}%
\pgfsetstrokecolor{currentstroke}%
\pgfsetdash{}{0pt}%
\pgfpathmoveto{\pgfqpoint{5.176815in}{4.291839in}}%
\pgfpathlineto{\pgfqpoint{5.032519in}{4.260562in}}%
\pgfpathlineto{\pgfqpoint{4.950381in}{4.110552in}}%
\pgfpathclose%
\pgfusepath{fill}%
\end{pgfscope}%
\begin{pgfscope}%
\pgfpathrectangle{\pgfqpoint{0.680860in}{0.078740in}}{\pgfqpoint{7.842520in}{7.842520in}}%
\pgfusepath{clip}%
\pgfsetbuttcap%
\pgfsetroundjoin%
\definecolor{currentfill}{rgb}{0.153894,0.680203,0.504172}%
\pgfsetfillcolor{currentfill}%
\pgfsetlinewidth{0.000000pt}%
\definecolor{currentstroke}{rgb}{0.156270,0.489624,0.557936}%
\pgfsetstrokecolor{currentstroke}%
\pgfsetdash{}{0pt}%
\pgfpathmoveto{\pgfqpoint{4.867891in}{3.938828in}}%
\pgfpathlineto{\pgfqpoint{4.950381in}{4.110552in}}%
\pgfpathlineto{\pgfqpoint{4.806813in}{4.084401in}}%
\pgfpathclose%
\pgfusepath{fill}%
\end{pgfscope}%
\begin{pgfscope}%
\pgfpathrectangle{\pgfqpoint{0.680860in}{0.078740in}}{\pgfqpoint{7.842520in}{7.842520in}}%
\pgfusepath{clip}%
\pgfsetbuttcap%
\pgfsetroundjoin%
\definecolor{currentfill}{rgb}{0.197636,0.391528,0.554969}%
\pgfsetfillcolor{currentfill}%
\pgfsetlinewidth{0.000000pt}%
\definecolor{currentstroke}{rgb}{0.154815,0.493313,0.557840}%
\pgfsetstrokecolor{currentstroke}%
\pgfsetdash{}{0pt}%
\pgfpathmoveto{\pgfqpoint{4.024722in}{2.811777in}}%
\pgfpathlineto{\pgfqpoint{3.941727in}{2.557386in}}%
\pgfpathlineto{\pgfqpoint{4.166687in}{2.805722in}}%
\pgfpathclose%
\pgfusepath{fill}%
\end{pgfscope}%
\begin{pgfscope}%
\pgfpathrectangle{\pgfqpoint{0.680860in}{0.078740in}}{\pgfqpoint{7.842520in}{7.842520in}}%
\pgfusepath{clip}%
\pgfsetbuttcap%
\pgfsetroundjoin%
\definecolor{currentfill}{rgb}{0.386433,0.794644,0.372886}%
\pgfsetfillcolor{currentfill}%
\pgfsetlinewidth{0.000000pt}%
\definecolor{currentstroke}{rgb}{0.153364,0.497000,0.557724}%
\pgfsetstrokecolor{currentstroke}%
\pgfsetdash{}{0pt}%
\pgfpathmoveto{\pgfqpoint{5.630161in}{4.604385in}}%
\pgfpathlineto{\pgfqpoint{5.710280in}{4.687163in}}%
\pgfpathlineto{\pgfqpoint{5.484436in}{4.564490in}}%
\pgfpathclose%
\pgfusepath{fill}%
\end{pgfscope}%
\begin{pgfscope}%
\pgfpathrectangle{\pgfqpoint{0.680860in}{0.078740in}}{\pgfqpoint{7.842520in}{7.842520in}}%
\pgfusepath{clip}%
\pgfsetbuttcap%
\pgfsetroundjoin%
\definecolor{currentfill}{rgb}{0.172719,0.448791,0.557885}%
\pgfsetfillcolor{currentfill}%
\pgfsetlinewidth{0.000000pt}%
\definecolor{currentstroke}{rgb}{0.151918,0.500685,0.557587}%
\pgfsetstrokecolor{currentstroke}%
\pgfsetdash{}{0pt}%
\pgfpathmoveto{\pgfqpoint{4.166687in}{2.805722in}}%
\pgfpathlineto{\pgfqpoint{4.249766in}{3.054365in}}%
\pgfpathlineto{\pgfqpoint{4.107737in}{3.054537in}}%
\pgfpathclose%
\pgfusepath{fill}%
\end{pgfscope}%
\begin{pgfscope}%
\pgfpathrectangle{\pgfqpoint{0.680860in}{0.078740in}}{\pgfqpoint{7.842520in}{7.842520in}}%
\pgfusepath{clip}%
\pgfsetbuttcap%
\pgfsetroundjoin%
\definecolor{currentfill}{rgb}{0.132444,0.552216,0.553018}%
\pgfsetfillcolor{currentfill}%
\pgfsetlinewidth{0.000000pt}%
\definecolor{currentstroke}{rgb}{0.150476,0.504369,0.557430}%
\pgfsetstrokecolor{currentstroke}%
\pgfsetdash{}{0pt}%
\pgfpathmoveto{\pgfqpoint{4.558634in}{3.520702in}}%
\pgfpathlineto{\pgfqpoint{4.415831in}{3.508703in}}%
\pgfpathlineto{\pgfqpoint{4.332828in}{3.289243in}}%
\pgfpathclose%
\pgfusepath{fill}%
\end{pgfscope}%
\begin{pgfscope}%
\pgfpathrectangle{\pgfqpoint{0.680860in}{0.078740in}}{\pgfqpoint{7.842520in}{7.842520in}}%
\pgfusepath{clip}%
\pgfsetbuttcap%
\pgfsetroundjoin%
\definecolor{currentfill}{rgb}{0.225863,0.330805,0.547314}%
\pgfsetfillcolor{currentfill}%
\pgfsetlinewidth{0.000000pt}%
\definecolor{currentstroke}{rgb}{0.149039,0.508051,0.557250}%
\pgfsetstrokecolor{currentstroke}%
\pgfsetdash{}{0pt}%
\pgfpathmoveto{\pgfqpoint{4.083621in}{2.545094in}}%
\pgfpathlineto{\pgfqpoint{3.941727in}{2.557386in}}%
\pgfpathlineto{\pgfqpoint{3.858773in}{2.293179in}}%
\pgfpathclose%
\pgfusepath{fill}%
\end{pgfscope}%
\begin{pgfscope}%
\pgfpathrectangle{\pgfqpoint{0.680860in}{0.078740in}}{\pgfqpoint{7.842520in}{7.842520in}}%
\pgfusepath{clip}%
\pgfsetbuttcap%
\pgfsetroundjoin%
\definecolor{currentfill}{rgb}{0.122606,0.585371,0.546557}%
\pgfsetfillcolor{currentfill}%
\pgfsetlinewidth{0.000000pt}%
\definecolor{currentstroke}{rgb}{0.147607,0.511733,0.557049}%
\pgfsetstrokecolor{currentstroke}%
\pgfsetdash{}{0pt}%
\pgfpathmoveto{\pgfqpoint{4.415831in}{3.508703in}}%
\pgfpathlineto{\pgfqpoint{4.558634in}{3.520702in}}%
\pgfpathlineto{\pgfqpoint{4.641567in}{3.727849in}}%
\pgfpathclose%
\pgfusepath{fill}%
\end{pgfscope}%
\begin{pgfscope}%
\pgfpathrectangle{\pgfqpoint{0.680860in}{0.078740in}}{\pgfqpoint{7.842520in}{7.842520in}}%
\pgfusepath{clip}%
\pgfsetbuttcap%
\pgfsetroundjoin%
\definecolor{currentfill}{rgb}{0.126326,0.644107,0.525311}%
\pgfsetfillcolor{currentfill}%
\pgfsetlinewidth{0.000000pt}%
\definecolor{currentstroke}{rgb}{0.146180,0.515413,0.556823}%
\pgfsetstrokecolor{currentstroke}%
\pgfsetdash{}{0pt}%
\pgfpathmoveto{\pgfqpoint{4.641567in}{3.727849in}}%
\pgfpathlineto{\pgfqpoint{4.867891in}{3.938828in}}%
\pgfpathlineto{\pgfqpoint{4.724314in}{3.916143in}}%
\pgfpathclose%
\pgfusepath{fill}%
\end{pgfscope}%
\begin{pgfscope}%
\pgfpathrectangle{\pgfqpoint{0.680860in}{0.078740in}}{\pgfqpoint{7.842520in}{7.842520in}}%
\pgfusepath{clip}%
\pgfsetbuttcap%
\pgfsetroundjoin%
\definecolor{currentfill}{rgb}{0.281477,0.755203,0.432552}%
\pgfsetfillcolor{currentfill}%
\pgfsetlinewidth{0.000000pt}%
\definecolor{currentstroke}{rgb}{0.144759,0.519093,0.556572}%
\pgfsetstrokecolor{currentstroke}%
\pgfsetdash{}{0pt}%
\pgfpathmoveto{\pgfqpoint{5.403468in}{4.457339in}}%
\pgfpathlineto{\pgfqpoint{5.258449in}{4.421448in}}%
\pgfpathlineto{\pgfqpoint{5.176815in}{4.291839in}}%
\pgfpathclose%
\pgfusepath{fill}%
\end{pgfscope}%
\begin{pgfscope}%
\pgfpathrectangle{\pgfqpoint{0.680860in}{0.078740in}}{\pgfqpoint{7.842520in}{7.842520in}}%
\pgfusepath{clip}%
\pgfsetbuttcap%
\pgfsetroundjoin%
\definecolor{currentfill}{rgb}{0.575563,0.844566,0.256415}%
\pgfsetfillcolor{currentfill}%
\pgfsetlinewidth{0.000000pt}%
\definecolor{currentstroke}{rgb}{0.143343,0.522773,0.556295}%
\pgfsetstrokecolor{currentstroke}%
\pgfsetdash{}{0pt}%
\pgfpathmoveto{\pgfqpoint{6.456767in}{4.959433in}}%
\pgfpathlineto{\pgfqpoint{6.532705in}{4.957195in}}%
\pgfpathlineto{\pgfqpoint{6.384576in}{4.909220in}}%
\pgfpathclose%
\pgfusepath{fill}%
\end{pgfscope}%
\begin{pgfscope}%
\pgfpathrectangle{\pgfqpoint{0.680860in}{0.078740in}}{\pgfqpoint{7.842520in}{7.842520in}}%
\pgfusepath{clip}%
\pgfsetbuttcap%
\pgfsetroundjoin%
\definecolor{currentfill}{rgb}{0.595839,0.848717,0.243329}%
\pgfsetfillcolor{currentfill}%
\pgfsetlinewidth{0.000000pt}%
\definecolor{currentstroke}{rgb}{0.141935,0.526453,0.555991}%
\pgfsetstrokecolor{currentstroke}%
\pgfsetdash{}{0pt}%
\pgfpathmoveto{\pgfqpoint{6.532705in}{4.957195in}}%
\pgfpathlineto{\pgfqpoint{6.681763in}{5.008239in}}%
\pgfpathlineto{\pgfqpoint{6.755907in}{4.975105in}}%
\pgfpathclose%
\pgfusepath{fill}%
\end{pgfscope}%
\begin{pgfscope}%
\pgfpathrectangle{\pgfqpoint{0.680860in}{0.078740in}}{\pgfqpoint{7.842520in}{7.842520in}}%
\pgfusepath{clip}%
\pgfsetbuttcap%
\pgfsetroundjoin%
\definecolor{currentfill}{rgb}{0.263663,0.237631,0.518762}%
\pgfsetfillcolor{currentfill}%
\pgfsetlinewidth{0.000000pt}%
\definecolor{currentstroke}{rgb}{0.140536,0.530132,0.555659}%
\pgfsetstrokecolor{currentstroke}%
\pgfsetdash{}{0pt}%
\pgfpathmoveto{\pgfqpoint{3.775868in}{2.021053in}}%
\pgfpathlineto{\pgfqpoint{3.917615in}{1.995495in}}%
\pgfpathlineto{\pgfqpoint{3.858773in}{2.293179in}}%
\pgfpathclose%
\pgfusepath{fill}%
\end{pgfscope}%
\begin{pgfscope}%
\pgfpathrectangle{\pgfqpoint{0.680860in}{0.078740in}}{\pgfqpoint{7.842520in}{7.842520in}}%
\pgfusepath{clip}%
\pgfsetbuttcap%
\pgfsetroundjoin%
\definecolor{currentfill}{rgb}{0.277134,0.185228,0.489898}%
\pgfsetfillcolor{currentfill}%
\pgfsetlinewidth{0.000000pt}%
\definecolor{currentstroke}{rgb}{0.139147,0.533812,0.555298}%
\pgfsetstrokecolor{currentstroke}%
\pgfsetdash{}{0pt}%
\pgfpathmoveto{\pgfqpoint{3.917615in}{1.995495in}}%
\pgfpathlineto{\pgfqpoint{3.775868in}{2.021053in}}%
\pgfpathlineto{\pgfqpoint{3.693014in}{1.742956in}}%
\pgfpathclose%
\pgfusepath{fill}%
\end{pgfscope}%
\begin{pgfscope}%
\pgfpathrectangle{\pgfqpoint{0.680860in}{0.078740in}}{\pgfqpoint{7.842520in}{7.842520in}}%
\pgfusepath{clip}%
\pgfsetbuttcap%
\pgfsetroundjoin%
\definecolor{currentfill}{rgb}{0.565498,0.842430,0.262877}%
\pgfsetfillcolor{currentfill}%
\pgfsetlinewidth{0.000000pt}%
\definecolor{currentstroke}{rgb}{0.137770,0.537492,0.554906}%
\pgfsetstrokecolor{currentstroke}%
\pgfsetdash{}{0pt}%
\pgfpathmoveto{\pgfqpoint{6.384576in}{4.909220in}}%
\pgfpathlineto{\pgfqpoint{6.308232in}{4.909070in}}%
\pgfpathlineto{\pgfqpoint{6.456767in}{4.959433in}}%
\pgfpathclose%
\pgfusepath{fill}%
\end{pgfscope}%
\begin{pgfscope}%
\pgfpathrectangle{\pgfqpoint{0.680860in}{0.078740in}}{\pgfqpoint{7.842520in}{7.842520in}}%
\pgfusepath{clip}%
\pgfsetbuttcap%
\pgfsetroundjoin%
\definecolor{currentfill}{rgb}{0.282910,0.105393,0.426902}%
\pgfsetfillcolor{currentfill}%
\pgfsetlinewidth{0.000000pt}%
\definecolor{currentstroke}{rgb}{0.136408,0.541173,0.554483}%
\pgfsetstrokecolor{currentstroke}%
\pgfsetdash{}{0pt}%
\pgfpathmoveto{\pgfqpoint{3.834693in}{1.710506in}}%
\pgfpathlineto{\pgfqpoint{3.693014in}{1.742956in}}%
\pgfpathlineto{\pgfqpoint{3.610200in}{1.460869in}}%
\pgfpathclose%
\pgfusepath{fill}%
\end{pgfscope}%
\begin{pgfscope}%
\pgfpathrectangle{\pgfqpoint{0.680860in}{0.078740in}}{\pgfqpoint{7.842520in}{7.842520in}}%
\pgfusepath{clip}%
\pgfsetbuttcap%
\pgfsetroundjoin%
\definecolor{currentfill}{rgb}{0.204903,0.375746,0.553533}%
\pgfsetfillcolor{currentfill}%
\pgfsetlinewidth{0.000000pt}%
\definecolor{currentstroke}{rgb}{0.135066,0.544853,0.554029}%
\pgfsetstrokecolor{currentstroke}%
\pgfsetdash{}{0pt}%
\pgfpathmoveto{\pgfqpoint{4.166687in}{2.805722in}}%
\pgfpathlineto{\pgfqpoint{3.941727in}{2.557386in}}%
\pgfpathlineto{\pgfqpoint{4.083621in}{2.545094in}}%
\pgfpathclose%
\pgfusepath{fill}%
\end{pgfscope}%
\begin{pgfscope}%
\pgfpathrectangle{\pgfqpoint{0.680860in}{0.078740in}}{\pgfqpoint{7.842520in}{7.842520in}}%
\pgfusepath{clip}%
\pgfsetbuttcap%
\pgfsetroundjoin%
\definecolor{currentfill}{rgb}{0.487026,0.823929,0.312321}%
\pgfsetfillcolor{currentfill}%
\pgfsetlinewidth{0.000000pt}%
\definecolor{currentstroke}{rgb}{0.133743,0.548535,0.553541}%
\pgfsetstrokecolor{currentstroke}%
\pgfsetdash{}{0pt}%
\pgfpathmoveto{\pgfqpoint{6.082795in}{4.832718in}}%
\pgfpathlineto{\pgfqpoint{5.935756in}{4.787022in}}%
\pgfpathlineto{\pgfqpoint{5.856683in}{4.730356in}}%
\pgfpathclose%
\pgfusepath{fill}%
\end{pgfscope}%
\begin{pgfscope}%
\pgfpathrectangle{\pgfqpoint{0.680860in}{0.078740in}}{\pgfqpoint{7.842520in}{7.842520in}}%
\pgfusepath{clip}%
\pgfsetbuttcap%
\pgfsetroundjoin%
\definecolor{currentfill}{rgb}{0.352360,0.783011,0.392636}%
\pgfsetfillcolor{currentfill}%
\pgfsetlinewidth{0.000000pt}%
\definecolor{currentstroke}{rgb}{0.132444,0.552216,0.553018}%
\pgfsetstrokecolor{currentstroke}%
\pgfsetdash{}{0pt}%
\pgfpathmoveto{\pgfqpoint{5.630161in}{4.604385in}}%
\pgfpathlineto{\pgfqpoint{5.484436in}{4.564490in}}%
\pgfpathlineto{\pgfqpoint{5.403468in}{4.457339in}}%
\pgfpathclose%
\pgfusepath{fill}%
\end{pgfscope}%
\begin{pgfscope}%
\pgfpathrectangle{\pgfqpoint{0.680860in}{0.078740in}}{\pgfqpoint{7.842520in}{7.842520in}}%
\pgfusepath{clip}%
\pgfsetbuttcap%
\pgfsetroundjoin%
\definecolor{currentfill}{rgb}{0.149039,0.508051,0.557250}%
\pgfsetfillcolor{currentfill}%
\pgfsetlinewidth{0.000000pt}%
\definecolor{currentstroke}{rgb}{0.131172,0.555899,0.552459}%
\pgfsetstrokecolor{currentstroke}%
\pgfsetdash{}{0pt}%
\pgfpathmoveto{\pgfqpoint{4.332828in}{3.289243in}}%
\pgfpathlineto{\pgfqpoint{4.249766in}{3.054365in}}%
\pgfpathlineto{\pgfqpoint{4.475575in}{3.296085in}}%
\pgfpathclose%
\pgfusepath{fill}%
\end{pgfscope}%
\begin{pgfscope}%
\pgfpathrectangle{\pgfqpoint{0.680860in}{0.078740in}}{\pgfqpoint{7.842520in}{7.842520in}}%
\pgfusepath{clip}%
\pgfsetbuttcap%
\pgfsetroundjoin%
\definecolor{currentfill}{rgb}{0.616293,0.852709,0.230052}%
\pgfsetfillcolor{currentfill}%
\pgfsetlinewidth{0.000000pt}%
\definecolor{currentstroke}{rgb}{0.129933,0.559582,0.551864}%
\pgfsetstrokecolor{currentstroke}%
\pgfsetdash{}{0pt}%
\pgfpathmoveto{\pgfqpoint{6.755907in}{4.975105in}}%
\pgfpathlineto{\pgfqpoint{6.681763in}{5.008239in}}%
\pgfpathlineto{\pgfqpoint{6.905397in}{5.025761in}}%
\pgfpathclose%
\pgfusepath{fill}%
\end{pgfscope}%
\begin{pgfscope}%
\pgfpathrectangle{\pgfqpoint{0.680860in}{0.078740in}}{\pgfqpoint{7.842520in}{7.842520in}}%
\pgfusepath{clip}%
\pgfsetbuttcap%
\pgfsetroundjoin%
\definecolor{currentfill}{rgb}{0.233603,0.313828,0.543914}%
\pgfsetfillcolor{currentfill}%
\pgfsetlinewidth{0.000000pt}%
\definecolor{currentstroke}{rgb}{0.128729,0.563265,0.551229}%
\pgfsetstrokecolor{currentstroke}%
\pgfsetdash{}{0pt}%
\pgfpathmoveto{\pgfqpoint{3.858773in}{2.293179in}}%
\pgfpathlineto{\pgfqpoint{4.000593in}{2.274363in}}%
\pgfpathlineto{\pgfqpoint{4.083621in}{2.545094in}}%
\pgfpathclose%
\pgfusepath{fill}%
\end{pgfscope}%
\begin{pgfscope}%
\pgfpathrectangle{\pgfqpoint{0.680860in}{0.078740in}}{\pgfqpoint{7.842520in}{7.842520in}}%
\pgfusepath{clip}%
\pgfsetbuttcap%
\pgfsetroundjoin%
\definecolor{currentfill}{rgb}{0.253935,0.265254,0.529983}%
\pgfsetfillcolor{currentfill}%
\pgfsetlinewidth{0.000000pt}%
\definecolor{currentstroke}{rgb}{0.127568,0.566949,0.550556}%
\pgfsetstrokecolor{currentstroke}%
\pgfsetdash{}{0pt}%
\pgfpathmoveto{\pgfqpoint{3.858773in}{2.293179in}}%
\pgfpathlineto{\pgfqpoint{3.917615in}{1.995495in}}%
\pgfpathlineto{\pgfqpoint{4.000593in}{2.274363in}}%
\pgfpathclose%
\pgfusepath{fill}%
\end{pgfscope}%
\begin{pgfscope}%
\pgfpathrectangle{\pgfqpoint{0.680860in}{0.078740in}}{\pgfqpoint{7.842520in}{7.842520in}}%
\pgfusepath{clip}%
\pgfsetbuttcap%
\pgfsetroundjoin%
\definecolor{currentfill}{rgb}{0.208030,0.718701,0.472873}%
\pgfsetfillcolor{currentfill}%
\pgfsetlinewidth{0.000000pt}%
\definecolor{currentstroke}{rgb}{0.126453,0.570633,0.549841}%
\pgfsetstrokecolor{currentstroke}%
\pgfsetdash{}{0pt}%
\pgfpathmoveto{\pgfqpoint{5.176815in}{4.291839in}}%
\pgfpathlineto{\pgfqpoint{4.950381in}{4.110552in}}%
\pgfpathlineto{\pgfqpoint{5.094712in}{4.138983in}}%
\pgfpathclose%
\pgfusepath{fill}%
\end{pgfscope}%
\begin{pgfscope}%
\pgfpathrectangle{\pgfqpoint{0.680860in}{0.078740in}}{\pgfqpoint{7.842520in}{7.842520in}}%
\pgfusepath{clip}%
\pgfsetbuttcap%
\pgfsetroundjoin%
\definecolor{currentfill}{rgb}{0.545524,0.838039,0.275626}%
\pgfsetfillcolor{currentfill}%
\pgfsetlinewidth{0.000000pt}%
\definecolor{currentstroke}{rgb}{0.125394,0.574318,0.549086}%
\pgfsetstrokecolor{currentstroke}%
\pgfsetdash{}{0pt}%
\pgfpathmoveto{\pgfqpoint{6.230731in}{4.881403in}}%
\pgfpathlineto{\pgfqpoint{6.308232in}{4.909070in}}%
\pgfpathlineto{\pgfqpoint{6.160613in}{4.861754in}}%
\pgfpathclose%
\pgfusepath{fill}%
\end{pgfscope}%
\begin{pgfscope}%
\pgfpathrectangle{\pgfqpoint{0.680860in}{0.078740in}}{\pgfqpoint{7.842520in}{7.842520in}}%
\pgfusepath{clip}%
\pgfsetbuttcap%
\pgfsetroundjoin%
\definecolor{currentfill}{rgb}{0.136408,0.541173,0.554483}%
\pgfsetfillcolor{currentfill}%
\pgfsetlinewidth{0.000000pt}%
\definecolor{currentstroke}{rgb}{0.124395,0.578002,0.548287}%
\pgfsetstrokecolor{currentstroke}%
\pgfsetdash{}{0pt}%
\pgfpathmoveto{\pgfqpoint{4.332828in}{3.289243in}}%
\pgfpathlineto{\pgfqpoint{4.475575in}{3.296085in}}%
\pgfpathlineto{\pgfqpoint{4.558634in}{3.520702in}}%
\pgfpathclose%
\pgfusepath{fill}%
\end{pgfscope}%
\begin{pgfscope}%
\pgfpathrectangle{\pgfqpoint{0.680860in}{0.078740in}}{\pgfqpoint{7.842520in}{7.842520in}}%
\pgfusepath{clip}%
\pgfsetbuttcap%
\pgfsetroundjoin%
\definecolor{currentfill}{rgb}{0.280868,0.160771,0.472899}%
\pgfsetfillcolor{currentfill}%
\pgfsetlinewidth{0.000000pt}%
\definecolor{currentstroke}{rgb}{0.123463,0.581687,0.547445}%
\pgfsetstrokecolor{currentstroke}%
\pgfsetdash{}{0pt}%
\pgfpathmoveto{\pgfqpoint{3.917615in}{1.995495in}}%
\pgfpathlineto{\pgfqpoint{3.693014in}{1.742956in}}%
\pgfpathlineto{\pgfqpoint{3.834693in}{1.710506in}}%
\pgfpathclose%
\pgfusepath{fill}%
\end{pgfscope}%
\begin{pgfscope}%
\pgfpathrectangle{\pgfqpoint{0.680860in}{0.078740in}}{\pgfqpoint{7.842520in}{7.842520in}}%
\pgfusepath{clip}%
\pgfsetbuttcap%
\pgfsetroundjoin%
\definecolor{currentfill}{rgb}{0.280267,0.073417,0.397163}%
\pgfsetfillcolor{currentfill}%
\pgfsetlinewidth{0.000000pt}%
\definecolor{currentstroke}{rgb}{0.122606,0.585371,0.546557}%
\pgfsetstrokecolor{currentstroke}%
\pgfsetdash{}{0pt}%
\pgfpathmoveto{\pgfqpoint{3.610200in}{1.460869in}}%
\pgfpathlineto{\pgfqpoint{3.751820in}{1.421448in}}%
\pgfpathlineto{\pgfqpoint{3.834693in}{1.710506in}}%
\pgfpathclose%
\pgfusepath{fill}%
\end{pgfscope}%
\begin{pgfscope}%
\pgfpathrectangle{\pgfqpoint{0.680860in}{0.078740in}}{\pgfqpoint{7.842520in}{7.842520in}}%
\pgfusepath{clip}%
\pgfsetbuttcap%
\pgfsetroundjoin%
\definecolor{currentfill}{rgb}{0.535621,0.835785,0.281908}%
\pgfsetfillcolor{currentfill}%
\pgfsetlinewidth{0.000000pt}%
\definecolor{currentstroke}{rgb}{0.121831,0.589055,0.545623}%
\pgfsetstrokecolor{currentstroke}%
\pgfsetdash{}{0pt}%
\pgfpathmoveto{\pgfqpoint{6.160613in}{4.861754in}}%
\pgfpathlineto{\pgfqpoint{6.082795in}{4.832718in}}%
\pgfpathlineto{\pgfqpoint{6.230731in}{4.881403in}}%
\pgfpathclose%
\pgfusepath{fill}%
\end{pgfscope}%
\begin{pgfscope}%
\pgfpathrectangle{\pgfqpoint{0.680860in}{0.078740in}}{\pgfqpoint{7.842520in}{7.842520in}}%
\pgfusepath{clip}%
\pgfsetbuttcap%
\pgfsetroundjoin%
\definecolor{currentfill}{rgb}{0.170948,0.694384,0.493803}%
\pgfsetfillcolor{currentfill}%
\pgfsetlinewidth{0.000000pt}%
\definecolor{currentstroke}{rgb}{0.121148,0.592739,0.544641}%
\pgfsetstrokecolor{currentstroke}%
\pgfsetdash{}{0pt}%
\pgfpathmoveto{\pgfqpoint{5.094712in}{4.138983in}}%
\pgfpathlineto{\pgfqpoint{4.950381in}{4.110552in}}%
\pgfpathlineto{\pgfqpoint{4.867891in}{3.938828in}}%
\pgfpathclose%
\pgfusepath{fill}%
\end{pgfscope}%
\begin{pgfscope}%
\pgfpathrectangle{\pgfqpoint{0.680860in}{0.078740in}}{\pgfqpoint{7.842520in}{7.842520in}}%
\pgfusepath{clip}%
\pgfsetbuttcap%
\pgfsetroundjoin%
\definecolor{currentfill}{rgb}{0.169646,0.456262,0.558030}%
\pgfsetfillcolor{currentfill}%
\pgfsetlinewidth{0.000000pt}%
\definecolor{currentstroke}{rgb}{0.120565,0.596422,0.543611}%
\pgfsetstrokecolor{currentstroke}%
\pgfsetdash{}{0pt}%
\pgfpathmoveto{\pgfqpoint{4.392441in}{3.055556in}}%
\pgfpathlineto{\pgfqpoint{4.249766in}{3.054365in}}%
\pgfpathlineto{\pgfqpoint{4.166687in}{2.805722in}}%
\pgfpathclose%
\pgfusepath{fill}%
\end{pgfscope}%
\begin{pgfscope}%
\pgfpathrectangle{\pgfqpoint{0.680860in}{0.078740in}}{\pgfqpoint{7.842520in}{7.842520in}}%
\pgfusepath{clip}%
\pgfsetbuttcap%
\pgfsetroundjoin%
\definecolor{currentfill}{rgb}{0.124780,0.640461,0.527068}%
\pgfsetfillcolor{currentfill}%
\pgfsetlinewidth{0.000000pt}%
\definecolor{currentstroke}{rgb}{0.120092,0.600104,0.542530}%
\pgfsetstrokecolor{currentstroke}%
\pgfsetdash{}{0pt}%
\pgfpathmoveto{\pgfqpoint{4.785120in}{3.746396in}}%
\pgfpathlineto{\pgfqpoint{4.867891in}{3.938828in}}%
\pgfpathlineto{\pgfqpoint{4.641567in}{3.727849in}}%
\pgfpathclose%
\pgfusepath{fill}%
\end{pgfscope}%
\begin{pgfscope}%
\pgfpathrectangle{\pgfqpoint{0.680860in}{0.078740in}}{\pgfqpoint{7.842520in}{7.842520in}}%
\pgfusepath{clip}%
\pgfsetbuttcap%
\pgfsetroundjoin%
\definecolor{currentfill}{rgb}{0.421908,0.805774,0.351910}%
\pgfsetfillcolor{currentfill}%
\pgfsetlinewidth{0.000000pt}%
\definecolor{currentstroke}{rgb}{0.119738,0.603785,0.541400}%
\pgfsetstrokecolor{currentstroke}%
\pgfsetdash{}{0pt}%
\pgfpathmoveto{\pgfqpoint{5.776736in}{4.647059in}}%
\pgfpathlineto{\pgfqpoint{5.710280in}{4.687163in}}%
\pgfpathlineto{\pgfqpoint{5.630161in}{4.604385in}}%
\pgfpathclose%
\pgfusepath{fill}%
\end{pgfscope}%
\begin{pgfscope}%
\pgfpathrectangle{\pgfqpoint{0.680860in}{0.078740in}}{\pgfqpoint{7.842520in}{7.842520in}}%
\pgfusepath{clip}%
\pgfsetbuttcap%
\pgfsetroundjoin%
\definecolor{currentfill}{rgb}{0.449368,0.813768,0.335384}%
\pgfsetfillcolor{currentfill}%
\pgfsetlinewidth{0.000000pt}%
\definecolor{currentstroke}{rgb}{0.119512,0.607464,0.540218}%
\pgfsetstrokecolor{currentstroke}%
\pgfsetdash{}{0pt}%
\pgfpathmoveto{\pgfqpoint{5.856683in}{4.730356in}}%
\pgfpathlineto{\pgfqpoint{5.710280in}{4.687163in}}%
\pgfpathlineto{\pgfqpoint{5.776736in}{4.647059in}}%
\pgfpathclose%
\pgfusepath{fill}%
\end{pgfscope}%
\begin{pgfscope}%
\pgfpathrectangle{\pgfqpoint{0.680860in}{0.078740in}}{\pgfqpoint{7.842520in}{7.842520in}}%
\pgfusepath{clip}%
\pgfsetbuttcap%
\pgfsetroundjoin%
\definecolor{currentfill}{rgb}{0.657642,0.860219,0.203082}%
\pgfsetfillcolor{currentfill}%
\pgfsetlinewidth{0.000000pt}%
\definecolor{currentstroke}{rgb}{0.119423,0.611141,0.538982}%
\pgfsetstrokecolor{currentstroke}%
\pgfsetdash{}{0pt}%
\pgfpathmoveto{\pgfqpoint{7.055852in}{5.079587in}}%
\pgfpathlineto{\pgfqpoint{7.127334in}{5.010234in}}%
\pgfpathlineto{\pgfqpoint{6.905397in}{5.025761in}}%
\pgfpathclose%
\pgfusepath{fill}%
\end{pgfscope}%
\begin{pgfscope}%
\pgfpathrectangle{\pgfqpoint{0.680860in}{0.078740in}}{\pgfqpoint{7.842520in}{7.842520in}}%
\pgfusepath{clip}%
\pgfsetbuttcap%
\pgfsetroundjoin%
\definecolor{currentfill}{rgb}{0.120565,0.596422,0.543611}%
\pgfsetfillcolor{currentfill}%
\pgfsetlinewidth{0.000000pt}%
\definecolor{currentstroke}{rgb}{0.119483,0.614817,0.537692}%
\pgfsetstrokecolor{currentstroke}%
\pgfsetdash{}{0pt}%
\pgfpathmoveto{\pgfqpoint{4.641567in}{3.727849in}}%
\pgfpathlineto{\pgfqpoint{4.558634in}{3.520702in}}%
\pgfpathlineto{\pgfqpoint{4.702136in}{3.534487in}}%
\pgfpathclose%
\pgfusepath{fill}%
\end{pgfscope}%
\begin{pgfscope}%
\pgfpathrectangle{\pgfqpoint{0.680860in}{0.078740in}}{\pgfqpoint{7.842520in}{7.842520in}}%
\pgfusepath{clip}%
\pgfsetbuttcap%
\pgfsetroundjoin%
\definecolor{currentfill}{rgb}{0.281477,0.755203,0.432552}%
\pgfsetfillcolor{currentfill}%
\pgfsetlinewidth{0.000000pt}%
\definecolor{currentstroke}{rgb}{0.119699,0.618490,0.536347}%
\pgfsetstrokecolor{currentstroke}%
\pgfsetdash{}{0pt}%
\pgfpathmoveto{\pgfqpoint{5.403468in}{4.457339in}}%
\pgfpathlineto{\pgfqpoint{5.176815in}{4.291839in}}%
\pgfpathlineto{\pgfqpoint{5.321904in}{4.325583in}}%
\pgfpathclose%
\pgfusepath{fill}%
\end{pgfscope}%
\begin{pgfscope}%
\pgfpathrectangle{\pgfqpoint{0.680860in}{0.078740in}}{\pgfqpoint{7.842520in}{7.842520in}}%
\pgfusepath{clip}%
\pgfsetbuttcap%
\pgfsetroundjoin%
\definecolor{currentfill}{rgb}{0.153364,0.497000,0.557724}%
\pgfsetfillcolor{currentfill}%
\pgfsetlinewidth{0.000000pt}%
\definecolor{currentstroke}{rgb}{0.120081,0.622161,0.534946}%
\pgfsetstrokecolor{currentstroke}%
\pgfsetdash{}{0pt}%
\pgfpathmoveto{\pgfqpoint{4.475575in}{3.296085in}}%
\pgfpathlineto{\pgfqpoint{4.249766in}{3.054365in}}%
\pgfpathlineto{\pgfqpoint{4.392441in}{3.055556in}}%
\pgfpathclose%
\pgfusepath{fill}%
\end{pgfscope}%
\begin{pgfscope}%
\pgfpathrectangle{\pgfqpoint{0.680860in}{0.078740in}}{\pgfqpoint{7.842520in}{7.842520in}}%
\pgfusepath{clip}%
\pgfsetbuttcap%
\pgfsetroundjoin%
\definecolor{currentfill}{rgb}{0.201239,0.383670,0.554294}%
\pgfsetfillcolor{currentfill}%
\pgfsetlinewidth{0.000000pt}%
\definecolor{currentstroke}{rgb}{0.120638,0.625828,0.533488}%
\pgfsetstrokecolor{currentstroke}%
\pgfsetdash{}{0pt}%
\pgfpathmoveto{\pgfqpoint{4.083621in}{2.545094in}}%
\pgfpathlineto{\pgfqpoint{4.226116in}{2.533756in}}%
\pgfpathlineto{\pgfqpoint{4.166687in}{2.805722in}}%
\pgfpathclose%
\pgfusepath{fill}%
\end{pgfscope}%
\begin{pgfscope}%
\pgfpathrectangle{\pgfqpoint{0.680860in}{0.078740in}}{\pgfqpoint{7.842520in}{7.842520in}}%
\pgfusepath{clip}%
\pgfsetbuttcap%
\pgfsetroundjoin%
\definecolor{currentfill}{rgb}{0.668054,0.861999,0.196293}%
\pgfsetfillcolor{currentfill}%
\pgfsetlinewidth{0.000000pt}%
\definecolor{currentstroke}{rgb}{0.121380,0.629492,0.531973}%
\pgfsetstrokecolor{currentstroke}%
\pgfsetdash{}{0pt}%
\pgfpathmoveto{\pgfqpoint{7.278117in}{5.062483in}}%
\pgfpathlineto{\pgfqpoint{7.127334in}{5.010234in}}%
\pgfpathlineto{\pgfqpoint{7.055852in}{5.079587in}}%
\pgfpathclose%
\pgfusepath{fill}%
\end{pgfscope}%
\begin{pgfscope}%
\pgfpathrectangle{\pgfqpoint{0.680860in}{0.078740in}}{\pgfqpoint{7.842520in}{7.842520in}}%
\pgfusepath{clip}%
\pgfsetbuttcap%
\pgfsetroundjoin%
\definecolor{currentfill}{rgb}{0.119483,0.614817,0.537692}%
\pgfsetfillcolor{currentfill}%
\pgfsetlinewidth{0.000000pt}%
\definecolor{currentstroke}{rgb}{0.122312,0.633153,0.530398}%
\pgfsetstrokecolor{currentstroke}%
\pgfsetdash{}{0pt}%
\pgfpathmoveto{\pgfqpoint{4.641567in}{3.727849in}}%
\pgfpathlineto{\pgfqpoint{4.702136in}{3.534487in}}%
\pgfpathlineto{\pgfqpoint{4.785120in}{3.746396in}}%
\pgfpathclose%
\pgfusepath{fill}%
\end{pgfscope}%
\begin{pgfscope}%
\pgfpathrectangle{\pgfqpoint{0.680860in}{0.078740in}}{\pgfqpoint{7.842520in}{7.842520in}}%
\pgfusepath{clip}%
\pgfsetbuttcap%
\pgfsetroundjoin%
\definecolor{currentfill}{rgb}{0.231674,0.318106,0.544834}%
\pgfsetfillcolor{currentfill}%
\pgfsetlinewidth{0.000000pt}%
\definecolor{currentstroke}{rgb}{0.123444,0.636809,0.528763}%
\pgfsetstrokecolor{currentstroke}%
\pgfsetdash{}{0pt}%
\pgfpathmoveto{\pgfqpoint{4.083621in}{2.545094in}}%
\pgfpathlineto{\pgfqpoint{4.000593in}{2.274363in}}%
\pgfpathlineto{\pgfqpoint{4.142990in}{2.256284in}}%
\pgfpathclose%
\pgfusepath{fill}%
\end{pgfscope}%
\begin{pgfscope}%
\pgfpathrectangle{\pgfqpoint{0.680860in}{0.078740in}}{\pgfqpoint{7.842520in}{7.842520in}}%
\pgfusepath{clip}%
\pgfsetbuttcap%
\pgfsetroundjoin%
\definecolor{currentfill}{rgb}{0.369214,0.788888,0.382914}%
\pgfsetfillcolor{currentfill}%
\pgfsetlinewidth{0.000000pt}%
\definecolor{currentstroke}{rgb}{0.124780,0.640461,0.527068}%
\pgfsetstrokecolor{currentstroke}%
\pgfsetdash{}{0pt}%
\pgfpathmoveto{\pgfqpoint{5.403468in}{4.457339in}}%
\pgfpathlineto{\pgfqpoint{5.549309in}{4.495864in}}%
\pgfpathlineto{\pgfqpoint{5.630161in}{4.604385in}}%
\pgfpathclose%
\pgfusepath{fill}%
\end{pgfscope}%
\begin{pgfscope}%
\pgfpathrectangle{\pgfqpoint{0.680860in}{0.078740in}}{\pgfqpoint{7.842520in}{7.842520in}}%
\pgfusepath{clip}%
\pgfsetbuttcap%
\pgfsetroundjoin%
\definecolor{currentfill}{rgb}{0.280894,0.078907,0.402329}%
\pgfsetfillcolor{currentfill}%
\pgfsetlinewidth{0.000000pt}%
\definecolor{currentstroke}{rgb}{0.126326,0.644107,0.525311}%
\pgfsetstrokecolor{currentstroke}%
\pgfsetdash{}{0pt}%
\pgfpathmoveto{\pgfqpoint{3.834693in}{1.710506in}}%
\pgfpathlineto{\pgfqpoint{3.751820in}{1.421448in}}%
\pgfpathlineto{\pgfqpoint{3.893944in}{1.382091in}}%
\pgfpathclose%
\pgfusepath{fill}%
\end{pgfscope}%
\begin{pgfscope}%
\pgfpathrectangle{\pgfqpoint{0.680860in}{0.078740in}}{\pgfqpoint{7.842520in}{7.842520in}}%
\pgfusepath{clip}%
\pgfsetbuttcap%
\pgfsetroundjoin%
\definecolor{currentfill}{rgb}{0.260571,0.246922,0.522828}%
\pgfsetfillcolor{currentfill}%
\pgfsetlinewidth{0.000000pt}%
\definecolor{currentstroke}{rgb}{0.128087,0.647749,0.523491}%
\pgfsetstrokecolor{currentstroke}%
\pgfsetdash{}{0pt}%
\pgfpathmoveto{\pgfqpoint{4.000593in}{2.274363in}}%
\pgfpathlineto{\pgfqpoint{3.917615in}{1.995495in}}%
\pgfpathlineto{\pgfqpoint{4.059915in}{1.970451in}}%
\pgfpathclose%
\pgfusepath{fill}%
\end{pgfscope}%
\begin{pgfscope}%
\pgfpathrectangle{\pgfqpoint{0.680860in}{0.078740in}}{\pgfqpoint{7.842520in}{7.842520in}}%
\pgfusepath{clip}%
\pgfsetbuttcap%
\pgfsetroundjoin%
\definecolor{currentfill}{rgb}{0.174274,0.445044,0.557792}%
\pgfsetfillcolor{currentfill}%
\pgfsetlinewidth{0.000000pt}%
\definecolor{currentstroke}{rgb}{0.130067,0.651384,0.521608}%
\pgfsetstrokecolor{currentstroke}%
\pgfsetdash{}{0pt}%
\pgfpathmoveto{\pgfqpoint{4.166687in}{2.805722in}}%
\pgfpathlineto{\pgfqpoint{4.309275in}{2.800831in}}%
\pgfpathlineto{\pgfqpoint{4.392441in}{3.055556in}}%
\pgfpathclose%
\pgfusepath{fill}%
\end{pgfscope}%
\begin{pgfscope}%
\pgfpathrectangle{\pgfqpoint{0.680860in}{0.078740in}}{\pgfqpoint{7.842520in}{7.842520in}}%
\pgfusepath{clip}%
\pgfsetbuttcap%
\pgfsetroundjoin%
\definecolor{currentfill}{rgb}{0.275191,0.194905,0.496005}%
\pgfsetfillcolor{currentfill}%
\pgfsetlinewidth{0.000000pt}%
\definecolor{currentstroke}{rgb}{0.132268,0.655014,0.519661}%
\pgfsetstrokecolor{currentstroke}%
\pgfsetdash{}{0pt}%
\pgfpathmoveto{\pgfqpoint{4.059915in}{1.970451in}}%
\pgfpathlineto{\pgfqpoint{3.917615in}{1.995495in}}%
\pgfpathlineto{\pgfqpoint{3.834693in}{1.710506in}}%
\pgfpathclose%
\pgfusepath{fill}%
\end{pgfscope}%
\begin{pgfscope}%
\pgfpathrectangle{\pgfqpoint{0.680860in}{0.078740in}}{\pgfqpoint{7.842520in}{7.842520in}}%
\pgfusepath{clip}%
\pgfsetbuttcap%
\pgfsetroundjoin%
\definecolor{currentfill}{rgb}{0.506271,0.828786,0.300362}%
\pgfsetfillcolor{currentfill}%
\pgfsetlinewidth{0.000000pt}%
\definecolor{currentstroke}{rgb}{0.134692,0.658636,0.517649}%
\pgfsetstrokecolor{currentstroke}%
\pgfsetdash{}{0pt}%
\pgfpathmoveto{\pgfqpoint{5.856683in}{4.730356in}}%
\pgfpathlineto{\pgfqpoint{6.003961in}{4.776449in}}%
\pgfpathlineto{\pgfqpoint{6.082795in}{4.832718in}}%
\pgfpathclose%
\pgfusepath{fill}%
\end{pgfscope}%
\begin{pgfscope}%
\pgfpathrectangle{\pgfqpoint{0.680860in}{0.078740in}}{\pgfqpoint{7.842520in}{7.842520in}}%
\pgfusepath{clip}%
\pgfsetbuttcap%
\pgfsetroundjoin%
\definecolor{currentfill}{rgb}{0.190631,0.407061,0.556089}%
\pgfsetfillcolor{currentfill}%
\pgfsetlinewidth{0.000000pt}%
\definecolor{currentstroke}{rgb}{0.137339,0.662252,0.515571}%
\pgfsetstrokecolor{currentstroke}%
\pgfsetdash{}{0pt}%
\pgfpathmoveto{\pgfqpoint{4.226116in}{2.533756in}}%
\pgfpathlineto{\pgfqpoint{4.309275in}{2.800831in}}%
\pgfpathlineto{\pgfqpoint{4.166687in}{2.805722in}}%
\pgfpathclose%
\pgfusepath{fill}%
\end{pgfscope}%
\begin{pgfscope}%
\pgfpathrectangle{\pgfqpoint{0.680860in}{0.078740in}}{\pgfqpoint{7.842520in}{7.842520in}}%
\pgfusepath{clip}%
\pgfsetbuttcap%
\pgfsetroundjoin%
\definecolor{currentfill}{rgb}{0.166383,0.690856,0.496502}%
\pgfsetfillcolor{currentfill}%
\pgfsetlinewidth{0.000000pt}%
\definecolor{currentstroke}{rgb}{0.140210,0.665859,0.513427}%
\pgfsetstrokecolor{currentstroke}%
\pgfsetdash{}{0pt}%
\pgfpathmoveto{\pgfqpoint{4.867891in}{3.938828in}}%
\pgfpathlineto{\pgfqpoint{5.012218in}{3.963675in}}%
\pgfpathlineto{\pgfqpoint{5.094712in}{4.138983in}}%
\pgfpathclose%
\pgfusepath{fill}%
\end{pgfscope}%
\begin{pgfscope}%
\pgfpathrectangle{\pgfqpoint{0.680860in}{0.078740in}}{\pgfqpoint{7.842520in}{7.842520in}}%
\pgfusepath{clip}%
\pgfsetbuttcap%
\pgfsetroundjoin%
\definecolor{currentfill}{rgb}{0.132444,0.552216,0.553018}%
\pgfsetfillcolor{currentfill}%
\pgfsetlinewidth{0.000000pt}%
\definecolor{currentstroke}{rgb}{0.143303,0.669459,0.511215}%
\pgfsetstrokecolor{currentstroke}%
\pgfsetdash{}{0pt}%
\pgfpathmoveto{\pgfqpoint{4.558634in}{3.520702in}}%
\pgfpathlineto{\pgfqpoint{4.475575in}{3.296085in}}%
\pgfpathlineto{\pgfqpoint{4.619001in}{3.304536in}}%
\pgfpathclose%
\pgfusepath{fill}%
\end{pgfscope}%
\begin{pgfscope}%
\pgfpathrectangle{\pgfqpoint{0.680860in}{0.078740in}}{\pgfqpoint{7.842520in}{7.842520in}}%
\pgfusepath{clip}%
\pgfsetbuttcap%
\pgfsetroundjoin%
\definecolor{currentfill}{rgb}{0.226397,0.728888,0.462789}%
\pgfsetfillcolor{currentfill}%
\pgfsetlinewidth{0.000000pt}%
\definecolor{currentstroke}{rgb}{0.146616,0.673050,0.508936}%
\pgfsetstrokecolor{currentstroke}%
\pgfsetdash{}{0pt}%
\pgfpathmoveto{\pgfqpoint{5.176815in}{4.291839in}}%
\pgfpathlineto{\pgfqpoint{5.094712in}{4.138983in}}%
\pgfpathlineto{\pgfqpoint{5.239825in}{4.169783in}}%
\pgfpathclose%
\pgfusepath{fill}%
\end{pgfscope}%
\begin{pgfscope}%
\pgfpathrectangle{\pgfqpoint{0.680860in}{0.078740in}}{\pgfqpoint{7.842520in}{7.842520in}}%
\pgfusepath{clip}%
\pgfsetbuttcap%
\pgfsetroundjoin%
\definecolor{currentfill}{rgb}{0.220057,0.343307,0.549413}%
\pgfsetfillcolor{currentfill}%
\pgfsetlinewidth{0.000000pt}%
\definecolor{currentstroke}{rgb}{0.150148,0.676631,0.506589}%
\pgfsetstrokecolor{currentstroke}%
\pgfsetdash{}{0pt}%
\pgfpathmoveto{\pgfqpoint{4.142990in}{2.256284in}}%
\pgfpathlineto{\pgfqpoint{4.226116in}{2.533756in}}%
\pgfpathlineto{\pgfqpoint{4.083621in}{2.545094in}}%
\pgfpathclose%
\pgfusepath{fill}%
\end{pgfscope}%
\begin{pgfscope}%
\pgfpathrectangle{\pgfqpoint{0.680860in}{0.078740in}}{\pgfqpoint{7.842520in}{7.842520in}}%
\pgfusepath{clip}%
\pgfsetbuttcap%
\pgfsetroundjoin%
\definecolor{currentfill}{rgb}{0.283091,0.110553,0.431554}%
\pgfsetfillcolor{currentfill}%
\pgfsetlinewidth{0.000000pt}%
\definecolor{currentstroke}{rgb}{0.153894,0.680203,0.504172}%
\pgfsetstrokecolor{currentstroke}%
\pgfsetdash{}{0pt}%
\pgfpathmoveto{\pgfqpoint{3.893944in}{1.382091in}}%
\pgfpathlineto{\pgfqpoint{3.976900in}{1.678346in}}%
\pgfpathlineto{\pgfqpoint{3.834693in}{1.710506in}}%
\pgfpathclose%
\pgfusepath{fill}%
\end{pgfscope}%
\begin{pgfscope}%
\pgfpathrectangle{\pgfqpoint{0.680860in}{0.078740in}}{\pgfqpoint{7.842520in}{7.842520in}}%
\pgfusepath{clip}%
\pgfsetbuttcap%
\pgfsetroundjoin%
\definecolor{currentfill}{rgb}{0.250425,0.274290,0.533103}%
\pgfsetfillcolor{currentfill}%
\pgfsetlinewidth{0.000000pt}%
\definecolor{currentstroke}{rgb}{0.157851,0.683765,0.501686}%
\pgfsetstrokecolor{currentstroke}%
\pgfsetdash{}{0pt}%
\pgfpathmoveto{\pgfqpoint{4.059915in}{1.970451in}}%
\pgfpathlineto{\pgfqpoint{4.142990in}{2.256284in}}%
\pgfpathlineto{\pgfqpoint{4.000593in}{2.274363in}}%
\pgfpathclose%
\pgfusepath{fill}%
\end{pgfscope}%
\begin{pgfscope}%
\pgfpathrectangle{\pgfqpoint{0.680860in}{0.078740in}}{\pgfqpoint{7.842520in}{7.842520in}}%
\pgfusepath{clip}%
\pgfsetbuttcap%
\pgfsetroundjoin%
\definecolor{currentfill}{rgb}{0.280255,0.165693,0.476498}%
\pgfsetfillcolor{currentfill}%
\pgfsetlinewidth{0.000000pt}%
\definecolor{currentstroke}{rgb}{0.162016,0.687316,0.499129}%
\pgfsetstrokecolor{currentstroke}%
\pgfsetdash{}{0pt}%
\pgfpathmoveto{\pgfqpoint{3.834693in}{1.710506in}}%
\pgfpathlineto{\pgfqpoint{3.976900in}{1.678346in}}%
\pgfpathlineto{\pgfqpoint{4.059915in}{1.970451in}}%
\pgfpathclose%
\pgfusepath{fill}%
\end{pgfscope}%
\begin{pgfscope}%
\pgfpathrectangle{\pgfqpoint{0.680860in}{0.078740in}}{\pgfqpoint{7.842520in}{7.842520in}}%
\pgfusepath{clip}%
\pgfsetbuttcap%
\pgfsetroundjoin%
\definecolor{currentfill}{rgb}{0.252899,0.742211,0.448284}%
\pgfsetfillcolor{currentfill}%
\pgfsetlinewidth{0.000000pt}%
\definecolor{currentstroke}{rgb}{0.166383,0.690856,0.496502}%
\pgfsetstrokecolor{currentstroke}%
\pgfsetdash{}{0pt}%
\pgfpathmoveto{\pgfqpoint{5.239825in}{4.169783in}}%
\pgfpathlineto{\pgfqpoint{5.321904in}{4.325583in}}%
\pgfpathlineto{\pgfqpoint{5.176815in}{4.291839in}}%
\pgfpathclose%
\pgfusepath{fill}%
\end{pgfscope}%
\begin{pgfscope}%
\pgfpathrectangle{\pgfqpoint{0.680860in}{0.078740in}}{\pgfqpoint{7.842520in}{7.842520in}}%
\pgfusepath{clip}%
\pgfsetbuttcap%
\pgfsetroundjoin%
\definecolor{currentfill}{rgb}{0.647257,0.858400,0.209861}%
\pgfsetfillcolor{currentfill}%
\pgfsetlinewidth{0.000000pt}%
\definecolor{currentstroke}{rgb}{0.170948,0.694384,0.493803}%
\pgfsetstrokecolor{currentstroke}%
\pgfsetdash{}{0pt}%
\pgfpathmoveto{\pgfqpoint{6.681763in}{5.008239in}}%
\pgfpathlineto{\pgfqpoint{6.532705in}{4.957195in}}%
\pgfpathlineto{\pgfqpoint{6.606243in}{5.012964in}}%
\pgfpathclose%
\pgfusepath{fill}%
\end{pgfscope}%
\begin{pgfscope}%
\pgfpathrectangle{\pgfqpoint{0.680860in}{0.078740in}}{\pgfqpoint{7.842520in}{7.842520in}}%
\pgfusepath{clip}%
\pgfsetbuttcap%
\pgfsetroundjoin%
\definecolor{currentfill}{rgb}{0.130067,0.651384,0.521608}%
\pgfsetfillcolor{currentfill}%
\pgfsetlinewidth{0.000000pt}%
\definecolor{currentstroke}{rgb}{0.175707,0.697900,0.491033}%
\pgfsetstrokecolor{currentstroke}%
\pgfsetdash{}{0pt}%
\pgfpathmoveto{\pgfqpoint{4.929407in}{3.766964in}}%
\pgfpathlineto{\pgfqpoint{4.867891in}{3.938828in}}%
\pgfpathlineto{\pgfqpoint{4.785120in}{3.746396in}}%
\pgfpathclose%
\pgfusepath{fill}%
\end{pgfscope}%
\begin{pgfscope}%
\pgfpathrectangle{\pgfqpoint{0.680860in}{0.078740in}}{\pgfqpoint{7.842520in}{7.842520in}}%
\pgfusepath{clip}%
\pgfsetbuttcap%
\pgfsetroundjoin%
\definecolor{currentfill}{rgb}{0.125394,0.574318,0.549086}%
\pgfsetfillcolor{currentfill}%
\pgfsetlinewidth{0.000000pt}%
\definecolor{currentstroke}{rgb}{0.180653,0.701402,0.488189}%
\pgfsetstrokecolor{currentstroke}%
\pgfsetdash{}{0pt}%
\pgfpathmoveto{\pgfqpoint{4.619001in}{3.304536in}}%
\pgfpathlineto{\pgfqpoint{4.702136in}{3.534487in}}%
\pgfpathlineto{\pgfqpoint{4.558634in}{3.520702in}}%
\pgfpathclose%
\pgfusepath{fill}%
\end{pgfscope}%
\begin{pgfscope}%
\pgfpathrectangle{\pgfqpoint{0.680860in}{0.078740in}}{\pgfqpoint{7.842520in}{7.842520in}}%
\pgfusepath{clip}%
\pgfsetbuttcap%
\pgfsetroundjoin%
\definecolor{currentfill}{rgb}{0.668054,0.861999,0.196293}%
\pgfsetfillcolor{currentfill}%
\pgfsetlinewidth{0.000000pt}%
\definecolor{currentstroke}{rgb}{0.185783,0.704891,0.485273}%
\pgfsetstrokecolor{currentstroke}%
\pgfsetdash{}{0pt}%
\pgfpathmoveto{\pgfqpoint{6.681763in}{5.008239in}}%
\pgfpathlineto{\pgfqpoint{6.831776in}{5.062471in}}%
\pgfpathlineto{\pgfqpoint{6.905397in}{5.025761in}}%
\pgfpathclose%
\pgfusepath{fill}%
\end{pgfscope}%
\begin{pgfscope}%
\pgfpathrectangle{\pgfqpoint{0.680860in}{0.078740in}}{\pgfqpoint{7.842520in}{7.842520in}}%
\pgfusepath{clip}%
\pgfsetbuttcap%
\pgfsetroundjoin%
\definecolor{currentfill}{rgb}{0.150476,0.504369,0.557430}%
\pgfsetfillcolor{currentfill}%
\pgfsetlinewidth{0.000000pt}%
\definecolor{currentstroke}{rgb}{0.191090,0.708366,0.482284}%
\pgfsetstrokecolor{currentstroke}%
\pgfsetdash{}{0pt}%
\pgfpathmoveto{\pgfqpoint{4.392441in}{3.055556in}}%
\pgfpathlineto{\pgfqpoint{4.535772in}{3.058162in}}%
\pgfpathlineto{\pgfqpoint{4.475575in}{3.296085in}}%
\pgfpathclose%
\pgfusepath{fill}%
\end{pgfscope}%
\begin{pgfscope}%
\pgfpathrectangle{\pgfqpoint{0.680860in}{0.078740in}}{\pgfqpoint{7.842520in}{7.842520in}}%
\pgfusepath{clip}%
\pgfsetbuttcap%
\pgfsetroundjoin%
\definecolor{currentfill}{rgb}{0.636902,0.856542,0.216620}%
\pgfsetfillcolor{currentfill}%
\pgfsetlinewidth{0.000000pt}%
\definecolor{currentstroke}{rgb}{0.196571,0.711827,0.479221}%
\pgfsetstrokecolor{currentstroke}%
\pgfsetdash{}{0pt}%
\pgfpathmoveto{\pgfqpoint{6.606243in}{5.012964in}}%
\pgfpathlineto{\pgfqpoint{6.532705in}{4.957195in}}%
\pgfpathlineto{\pgfqpoint{6.456767in}{4.959433in}}%
\pgfpathclose%
\pgfusepath{fill}%
\end{pgfscope}%
\begin{pgfscope}%
\pgfpathrectangle{\pgfqpoint{0.680860in}{0.078740in}}{\pgfqpoint{7.842520in}{7.842520in}}%
\pgfusepath{clip}%
\pgfsetbuttcap%
\pgfsetroundjoin%
\definecolor{currentfill}{rgb}{0.487026,0.823929,0.312321}%
\pgfsetfillcolor{currentfill}%
\pgfsetlinewidth{0.000000pt}%
\definecolor{currentstroke}{rgb}{0.202219,0.715272,0.476084}%
\pgfsetstrokecolor{currentstroke}%
\pgfsetdash{}{0pt}%
\pgfpathmoveto{\pgfqpoint{5.776736in}{4.647059in}}%
\pgfpathlineto{\pgfqpoint{6.003961in}{4.776449in}}%
\pgfpathlineto{\pgfqpoint{5.856683in}{4.730356in}}%
\pgfpathclose%
\pgfusepath{fill}%
\end{pgfscope}%
\begin{pgfscope}%
\pgfpathrectangle{\pgfqpoint{0.680860in}{0.078740in}}{\pgfqpoint{7.842520in}{7.842520in}}%
\pgfusepath{clip}%
\pgfsetbuttcap%
\pgfsetroundjoin%
\definecolor{currentfill}{rgb}{0.143303,0.669459,0.511215}%
\pgfsetfillcolor{currentfill}%
\pgfsetlinewidth{0.000000pt}%
\definecolor{currentstroke}{rgb}{0.208030,0.718701,0.472873}%
\pgfsetstrokecolor{currentstroke}%
\pgfsetdash{}{0pt}%
\pgfpathmoveto{\pgfqpoint{5.012218in}{3.963675in}}%
\pgfpathlineto{\pgfqpoint{4.867891in}{3.938828in}}%
\pgfpathlineto{\pgfqpoint{4.929407in}{3.766964in}}%
\pgfpathclose%
\pgfusepath{fill}%
\end{pgfscope}%
\begin{pgfscope}%
\pgfpathrectangle{\pgfqpoint{0.680860in}{0.078740in}}{\pgfqpoint{7.842520in}{7.842520in}}%
\pgfusepath{clip}%
\pgfsetbuttcap%
\pgfsetroundjoin%
\definecolor{currentfill}{rgb}{0.311925,0.767822,0.415586}%
\pgfsetfillcolor{currentfill}%
\pgfsetlinewidth{0.000000pt}%
\definecolor{currentstroke}{rgb}{0.214000,0.722114,0.469588}%
\pgfsetstrokecolor{currentstroke}%
\pgfsetdash{}{0pt}%
\pgfpathmoveto{\pgfqpoint{5.321904in}{4.325583in}}%
\pgfpathlineto{\pgfqpoint{5.467808in}{4.361890in}}%
\pgfpathlineto{\pgfqpoint{5.403468in}{4.457339in}}%
\pgfpathclose%
\pgfusepath{fill}%
\end{pgfscope}%
\begin{pgfscope}%
\pgfpathrectangle{\pgfqpoint{0.680860in}{0.078740in}}{\pgfqpoint{7.842520in}{7.842520in}}%
\pgfusepath{clip}%
\pgfsetbuttcap%
\pgfsetroundjoin%
\definecolor{currentfill}{rgb}{0.163625,0.471133,0.558148}%
\pgfsetfillcolor{currentfill}%
\pgfsetlinewidth{0.000000pt}%
\definecolor{currentstroke}{rgb}{0.220124,0.725509,0.466226}%
\pgfsetstrokecolor{currentstroke}%
\pgfsetdash{}{0pt}%
\pgfpathmoveto{\pgfqpoint{4.309275in}{2.800831in}}%
\pgfpathlineto{\pgfqpoint{4.535772in}{3.058162in}}%
\pgfpathlineto{\pgfqpoint{4.392441in}{3.055556in}}%
\pgfpathclose%
\pgfusepath{fill}%
\end{pgfscope}%
\begin{pgfscope}%
\pgfpathrectangle{\pgfqpoint{0.680860in}{0.078740in}}{\pgfqpoint{7.842520in}{7.842520in}}%
\pgfusepath{clip}%
\pgfsetbuttcap%
\pgfsetroundjoin%
\definecolor{currentfill}{rgb}{0.344074,0.780029,0.397381}%
\pgfsetfillcolor{currentfill}%
\pgfsetlinewidth{0.000000pt}%
\definecolor{currentstroke}{rgb}{0.226397,0.728888,0.462789}%
\pgfsetstrokecolor{currentstroke}%
\pgfsetdash{}{0pt}%
\pgfpathmoveto{\pgfqpoint{5.467808in}{4.361890in}}%
\pgfpathlineto{\pgfqpoint{5.549309in}{4.495864in}}%
\pgfpathlineto{\pgfqpoint{5.403468in}{4.457339in}}%
\pgfpathclose%
\pgfusepath{fill}%
\end{pgfscope}%
\begin{pgfscope}%
\pgfpathrectangle{\pgfqpoint{0.680860in}{0.078740in}}{\pgfqpoint{7.842520in}{7.842520in}}%
\pgfusepath{clip}%
\pgfsetbuttcap%
\pgfsetroundjoin%
\definecolor{currentfill}{rgb}{0.196571,0.711827,0.479221}%
\pgfsetfillcolor{currentfill}%
\pgfsetlinewidth{0.000000pt}%
\definecolor{currentstroke}{rgb}{0.232815,0.732247,0.459277}%
\pgfsetstrokecolor{currentstroke}%
\pgfsetdash{}{0pt}%
\pgfpathmoveto{\pgfqpoint{5.239825in}{4.169783in}}%
\pgfpathlineto{\pgfqpoint{5.094712in}{4.138983in}}%
\pgfpathlineto{\pgfqpoint{5.012218in}{3.963675in}}%
\pgfpathclose%
\pgfusepath{fill}%
\end{pgfscope}%
\begin{pgfscope}%
\pgfpathrectangle{\pgfqpoint{0.680860in}{0.078740in}}{\pgfqpoint{7.842520in}{7.842520in}}%
\pgfusepath{clip}%
\pgfsetbuttcap%
\pgfsetroundjoin%
\definecolor{currentfill}{rgb}{0.626579,0.854645,0.223353}%
\pgfsetfillcolor{currentfill}%
\pgfsetlinewidth{0.000000pt}%
\definecolor{currentstroke}{rgb}{0.239374,0.735588,0.455688}%
\pgfsetstrokecolor{currentstroke}%
\pgfsetdash{}{0pt}%
\pgfpathmoveto{\pgfqpoint{6.308232in}{4.909070in}}%
\pgfpathlineto{\pgfqpoint{6.379591in}{4.933197in}}%
\pgfpathlineto{\pgfqpoint{6.456767in}{4.959433in}}%
\pgfpathclose%
\pgfusepath{fill}%
\end{pgfscope}%
\begin{pgfscope}%
\pgfpathrectangle{\pgfqpoint{0.680860in}{0.078740in}}{\pgfqpoint{7.842520in}{7.842520in}}%
\pgfusepath{clip}%
\pgfsetbuttcap%
\pgfsetroundjoin%
\definecolor{currentfill}{rgb}{0.141935,0.526453,0.555991}%
\pgfsetfillcolor{currentfill}%
\pgfsetlinewidth{0.000000pt}%
\definecolor{currentstroke}{rgb}{0.246070,0.738910,0.452024}%
\pgfsetstrokecolor{currentstroke}%
\pgfsetdash{}{0pt}%
\pgfpathmoveto{\pgfqpoint{4.475575in}{3.296085in}}%
\pgfpathlineto{\pgfqpoint{4.535772in}{3.058162in}}%
\pgfpathlineto{\pgfqpoint{4.619001in}{3.304536in}}%
\pgfpathclose%
\pgfusepath{fill}%
\end{pgfscope}%
\begin{pgfscope}%
\pgfpathrectangle{\pgfqpoint{0.680860in}{0.078740in}}{\pgfqpoint{7.842520in}{7.842520in}}%
\pgfusepath{clip}%
\pgfsetbuttcap%
\pgfsetroundjoin%
\definecolor{currentfill}{rgb}{0.606045,0.850733,0.236712}%
\pgfsetfillcolor{currentfill}%
\pgfsetlinewidth{0.000000pt}%
\definecolor{currentstroke}{rgb}{0.252899,0.742211,0.448284}%
\pgfsetstrokecolor{currentstroke}%
\pgfsetdash{}{0pt}%
\pgfpathmoveto{\pgfqpoint{6.230731in}{4.881403in}}%
\pgfpathlineto{\pgfqpoint{6.379591in}{4.933197in}}%
\pgfpathlineto{\pgfqpoint{6.308232in}{4.909070in}}%
\pgfpathclose%
\pgfusepath{fill}%
\end{pgfscope}%
\begin{pgfscope}%
\pgfpathrectangle{\pgfqpoint{0.680860in}{0.078740in}}{\pgfqpoint{7.842520in}{7.842520in}}%
\pgfusepath{clip}%
\pgfsetbuttcap%
\pgfsetroundjoin%
\definecolor{currentfill}{rgb}{0.195860,0.395433,0.555276}%
\pgfsetfillcolor{currentfill}%
\pgfsetlinewidth{0.000000pt}%
\definecolor{currentstroke}{rgb}{0.259857,0.745492,0.444467}%
\pgfsetstrokecolor{currentstroke}%
\pgfsetdash{}{0pt}%
\pgfpathmoveto{\pgfqpoint{4.309275in}{2.800831in}}%
\pgfpathlineto{\pgfqpoint{4.226116in}{2.533756in}}%
\pgfpathlineto{\pgfqpoint{4.369221in}{2.523409in}}%
\pgfpathclose%
\pgfusepath{fill}%
\end{pgfscope}%
\begin{pgfscope}%
\pgfpathrectangle{\pgfqpoint{0.680860in}{0.078740in}}{\pgfqpoint{7.842520in}{7.842520in}}%
\pgfusepath{clip}%
\pgfsetbuttcap%
\pgfsetroundjoin%
\definecolor{currentfill}{rgb}{0.404001,0.800275,0.362552}%
\pgfsetfillcolor{currentfill}%
\pgfsetlinewidth{0.000000pt}%
\definecolor{currentstroke}{rgb}{0.266941,0.748751,0.440573}%
\pgfsetstrokecolor{currentstroke}%
\pgfsetdash{}{0pt}%
\pgfpathmoveto{\pgfqpoint{5.630161in}{4.604385in}}%
\pgfpathlineto{\pgfqpoint{5.549309in}{4.495864in}}%
\pgfpathlineto{\pgfqpoint{5.695996in}{4.537126in}}%
\pgfpathclose%
\pgfusepath{fill}%
\end{pgfscope}%
\begin{pgfscope}%
\pgfpathrectangle{\pgfqpoint{0.680860in}{0.078740in}}{\pgfqpoint{7.842520in}{7.842520in}}%
\pgfusepath{clip}%
\pgfsetbuttcap%
\pgfsetroundjoin%
\definecolor{currentfill}{rgb}{0.283091,0.110553,0.431554}%
\pgfsetfillcolor{currentfill}%
\pgfsetlinewidth{0.000000pt}%
\definecolor{currentstroke}{rgb}{0.274149,0.751988,0.436601}%
\pgfsetstrokecolor{currentstroke}%
\pgfsetdash{}{0pt}%
\pgfpathmoveto{\pgfqpoint{4.119642in}{1.646490in}}%
\pgfpathlineto{\pgfqpoint{3.976900in}{1.678346in}}%
\pgfpathlineto{\pgfqpoint{3.893944in}{1.382091in}}%
\pgfpathclose%
\pgfusepath{fill}%
\end{pgfscope}%
\begin{pgfscope}%
\pgfpathrectangle{\pgfqpoint{0.680860in}{0.078740in}}{\pgfqpoint{7.842520in}{7.842520in}}%
\pgfusepath{clip}%
\pgfsetbuttcap%
\pgfsetroundjoin%
\definecolor{currentfill}{rgb}{0.119423,0.611141,0.538982}%
\pgfsetfillcolor{currentfill}%
\pgfsetlinewidth{0.000000pt}%
\definecolor{currentstroke}{rgb}{0.281477,0.755203,0.432552}%
\pgfsetstrokecolor{currentstroke}%
\pgfsetdash{}{0pt}%
\pgfpathmoveto{\pgfqpoint{4.785120in}{3.746396in}}%
\pgfpathlineto{\pgfqpoint{4.702136in}{3.534487in}}%
\pgfpathlineto{\pgfqpoint{4.846352in}{3.550127in}}%
\pgfpathclose%
\pgfusepath{fill}%
\end{pgfscope}%
\begin{pgfscope}%
\pgfpathrectangle{\pgfqpoint{0.680860in}{0.078740in}}{\pgfqpoint{7.842520in}{7.842520in}}%
\pgfusepath{clip}%
\pgfsetbuttcap%
\pgfsetroundjoin%
\definecolor{currentfill}{rgb}{0.279574,0.170599,0.479997}%
\pgfsetfillcolor{currentfill}%
\pgfsetlinewidth{0.000000pt}%
\definecolor{currentstroke}{rgb}{0.288921,0.758394,0.428426}%
\pgfsetstrokecolor{currentstroke}%
\pgfsetdash{}{0pt}%
\pgfpathmoveto{\pgfqpoint{4.059915in}{1.970451in}}%
\pgfpathlineto{\pgfqpoint{3.976900in}{1.678346in}}%
\pgfpathlineto{\pgfqpoint{4.119642in}{1.646490in}}%
\pgfpathclose%
\pgfusepath{fill}%
\end{pgfscope}%
\begin{pgfscope}%
\pgfpathrectangle{\pgfqpoint{0.680860in}{0.078740in}}{\pgfqpoint{7.842520in}{7.842520in}}%
\pgfusepath{clip}%
\pgfsetbuttcap%
\pgfsetroundjoin%
\definecolor{currentfill}{rgb}{0.216210,0.351535,0.550627}%
\pgfsetfillcolor{currentfill}%
\pgfsetlinewidth{0.000000pt}%
\definecolor{currentstroke}{rgb}{0.296479,0.761561,0.424223}%
\pgfsetstrokecolor{currentstroke}%
\pgfsetdash{}{0pt}%
\pgfpathmoveto{\pgfqpoint{4.369221in}{2.523409in}}%
\pgfpathlineto{\pgfqpoint{4.226116in}{2.533756in}}%
\pgfpathlineto{\pgfqpoint{4.142990in}{2.256284in}}%
\pgfpathclose%
\pgfusepath{fill}%
\end{pgfscope}%
\begin{pgfscope}%
\pgfpathrectangle{\pgfqpoint{0.680860in}{0.078740in}}{\pgfqpoint{7.842520in}{7.842520in}}%
\pgfusepath{clip}%
\pgfsetbuttcap%
\pgfsetroundjoin%
\definecolor{currentfill}{rgb}{0.430983,0.808473,0.346476}%
\pgfsetfillcolor{currentfill}%
\pgfsetlinewidth{0.000000pt}%
\definecolor{currentstroke}{rgb}{0.304148,0.764704,0.419943}%
\pgfsetstrokecolor{currentstroke}%
\pgfsetdash{}{0pt}%
\pgfpathmoveto{\pgfqpoint{5.695996in}{4.537126in}}%
\pgfpathlineto{\pgfqpoint{5.776736in}{4.647059in}}%
\pgfpathlineto{\pgfqpoint{5.630161in}{4.604385in}}%
\pgfpathclose%
\pgfusepath{fill}%
\end{pgfscope}%
\begin{pgfscope}%
\pgfpathrectangle{\pgfqpoint{0.680860in}{0.078740in}}{\pgfqpoint{7.842520in}{7.842520in}}%
\pgfusepath{clip}%
\pgfsetbuttcap%
\pgfsetroundjoin%
\definecolor{currentfill}{rgb}{0.248629,0.278775,0.534556}%
\pgfsetfillcolor{currentfill}%
\pgfsetlinewidth{0.000000pt}%
\definecolor{currentstroke}{rgb}{0.311925,0.767822,0.415586}%
\pgfsetstrokecolor{currentstroke}%
\pgfsetdash{}{0pt}%
\pgfpathmoveto{\pgfqpoint{4.285972in}{2.238972in}}%
\pgfpathlineto{\pgfqpoint{4.142990in}{2.256284in}}%
\pgfpathlineto{\pgfqpoint{4.059915in}{1.970451in}}%
\pgfpathclose%
\pgfusepath{fill}%
\end{pgfscope}%
\begin{pgfscope}%
\pgfpathrectangle{\pgfqpoint{0.680860in}{0.078740in}}{\pgfqpoint{7.842520in}{7.842520in}}%
\pgfusepath{clip}%
\pgfsetbuttcap%
\pgfsetroundjoin%
\definecolor{currentfill}{rgb}{0.575563,0.844566,0.256415}%
\pgfsetfillcolor{currentfill}%
\pgfsetlinewidth{0.000000pt}%
\definecolor{currentstroke}{rgb}{0.319809,0.770914,0.411152}%
\pgfsetstrokecolor{currentstroke}%
\pgfsetdash{}{0pt}%
\pgfpathmoveto{\pgfqpoint{6.082795in}{4.832718in}}%
\pgfpathlineto{\pgfqpoint{6.152140in}{4.825555in}}%
\pgfpathlineto{\pgfqpoint{6.230731in}{4.881403in}}%
\pgfpathclose%
\pgfusepath{fill}%
\end{pgfscope}%
\begin{pgfscope}%
\pgfpathrectangle{\pgfqpoint{0.680860in}{0.078740in}}{\pgfqpoint{7.842520in}{7.842520in}}%
\pgfusepath{clip}%
\pgfsetbuttcap%
\pgfsetroundjoin%
\definecolor{currentfill}{rgb}{0.555484,0.840254,0.269281}%
\pgfsetfillcolor{currentfill}%
\pgfsetlinewidth{0.000000pt}%
\definecolor{currentstroke}{rgb}{0.327796,0.773980,0.406640}%
\pgfsetstrokecolor{currentstroke}%
\pgfsetdash{}{0pt}%
\pgfpathmoveto{\pgfqpoint{6.003961in}{4.776449in}}%
\pgfpathlineto{\pgfqpoint{6.152140in}{4.825555in}}%
\pgfpathlineto{\pgfqpoint{6.082795in}{4.832718in}}%
\pgfpathclose%
\pgfusepath{fill}%
\end{pgfscope}%
\begin{pgfscope}%
\pgfpathrectangle{\pgfqpoint{0.680860in}{0.078740in}}{\pgfqpoint{7.842520in}{7.842520in}}%
\pgfusepath{clip}%
\pgfsetbuttcap%
\pgfsetroundjoin%
\definecolor{currentfill}{rgb}{0.169646,0.456262,0.558030}%
\pgfsetfillcolor{currentfill}%
\pgfsetlinewidth{0.000000pt}%
\definecolor{currentstroke}{rgb}{0.335885,0.777018,0.402049}%
\pgfsetstrokecolor{currentstroke}%
\pgfsetdash{}{0pt}%
\pgfpathmoveto{\pgfqpoint{4.452498in}{2.797148in}}%
\pgfpathlineto{\pgfqpoint{4.535772in}{3.058162in}}%
\pgfpathlineto{\pgfqpoint{4.309275in}{2.800831in}}%
\pgfpathclose%
\pgfusepath{fill}%
\end{pgfscope}%
\begin{pgfscope}%
\pgfpathrectangle{\pgfqpoint{0.680860in}{0.078740in}}{\pgfqpoint{7.842520in}{7.842520in}}%
\pgfusepath{clip}%
\pgfsetbuttcap%
\pgfsetroundjoin%
\definecolor{currentfill}{rgb}{0.730889,0.871916,0.156029}%
\pgfsetfillcolor{currentfill}%
\pgfsetlinewidth{0.000000pt}%
\definecolor{currentstroke}{rgb}{0.344074,0.780029,0.397381}%
\pgfsetstrokecolor{currentstroke}%
\pgfsetdash{}{0pt}%
\pgfpathmoveto{\pgfqpoint{7.055852in}{5.079587in}}%
\pgfpathlineto{\pgfqpoint{7.207299in}{5.136707in}}%
\pgfpathlineto{\pgfqpoint{7.278117in}{5.062483in}}%
\pgfpathclose%
\pgfusepath{fill}%
\end{pgfscope}%
\begin{pgfscope}%
\pgfpathrectangle{\pgfqpoint{0.680860in}{0.078740in}}{\pgfqpoint{7.842520in}{7.842520in}}%
\pgfusepath{clip}%
\pgfsetbuttcap%
\pgfsetroundjoin%
\definecolor{currentfill}{rgb}{0.122312,0.633153,0.530398}%
\pgfsetfillcolor{currentfill}%
\pgfsetlinewidth{0.000000pt}%
\definecolor{currentstroke}{rgb}{0.352360,0.783011,0.392636}%
\pgfsetstrokecolor{currentstroke}%
\pgfsetdash{}{0pt}%
\pgfpathmoveto{\pgfqpoint{4.846352in}{3.550127in}}%
\pgfpathlineto{\pgfqpoint{4.929407in}{3.766964in}}%
\pgfpathlineto{\pgfqpoint{4.785120in}{3.746396in}}%
\pgfpathclose%
\pgfusepath{fill}%
\end{pgfscope}%
\begin{pgfscope}%
\pgfpathrectangle{\pgfqpoint{0.680860in}{0.078740in}}{\pgfqpoint{7.842520in}{7.842520in}}%
\pgfusepath{clip}%
\pgfsetbuttcap%
\pgfsetroundjoin%
\definecolor{currentfill}{rgb}{0.281446,0.084320,0.407414}%
\pgfsetfillcolor{currentfill}%
\pgfsetlinewidth{0.000000pt}%
\definecolor{currentstroke}{rgb}{0.360741,0.785964,0.387814}%
\pgfsetstrokecolor{currentstroke}%
\pgfsetdash{}{0pt}%
\pgfpathmoveto{\pgfqpoint{3.893944in}{1.382091in}}%
\pgfpathlineto{\pgfqpoint{4.036577in}{1.342806in}}%
\pgfpathlineto{\pgfqpoint{4.119642in}{1.646490in}}%
\pgfpathclose%
\pgfusepath{fill}%
\end{pgfscope}%
\begin{pgfscope}%
\pgfpathrectangle{\pgfqpoint{0.680860in}{0.078740in}}{\pgfqpoint{7.842520in}{7.842520in}}%
\pgfusepath{clip}%
\pgfsetbuttcap%
\pgfsetroundjoin%
\definecolor{currentfill}{rgb}{0.720391,0.870350,0.162603}%
\pgfsetfillcolor{currentfill}%
\pgfsetlinewidth{0.000000pt}%
\definecolor{currentstroke}{rgb}{0.369214,0.788888,0.382914}%
\pgfsetstrokecolor{currentstroke}%
\pgfsetdash{}{0pt}%
\pgfpathmoveto{\pgfqpoint{6.905397in}{5.025761in}}%
\pgfpathlineto{\pgfqpoint{6.982775in}{5.120019in}}%
\pgfpathlineto{\pgfqpoint{7.055852in}{5.079587in}}%
\pgfpathclose%
\pgfusepath{fill}%
\end{pgfscope}%
\begin{pgfscope}%
\pgfpathrectangle{\pgfqpoint{0.680860in}{0.078740in}}{\pgfqpoint{7.842520in}{7.842520in}}%
\pgfusepath{clip}%
\pgfsetbuttcap%
\pgfsetroundjoin%
\definecolor{currentfill}{rgb}{0.185556,0.418570,0.556753}%
\pgfsetfillcolor{currentfill}%
\pgfsetlinewidth{0.000000pt}%
\definecolor{currentstroke}{rgb}{0.377779,0.791781,0.377939}%
\pgfsetstrokecolor{currentstroke}%
\pgfsetdash{}{0pt}%
\pgfpathmoveto{\pgfqpoint{4.309275in}{2.800831in}}%
\pgfpathlineto{\pgfqpoint{4.369221in}{2.523409in}}%
\pgfpathlineto{\pgfqpoint{4.452498in}{2.797148in}}%
\pgfpathclose%
\pgfusepath{fill}%
\end{pgfscope}%
\begin{pgfscope}%
\pgfpathrectangle{\pgfqpoint{0.680860in}{0.078740in}}{\pgfqpoint{7.842520in}{7.842520in}}%
\pgfusepath{clip}%
\pgfsetbuttcap%
\pgfsetroundjoin%
\definecolor{currentfill}{rgb}{0.122606,0.585371,0.546557}%
\pgfsetfillcolor{currentfill}%
\pgfsetlinewidth{0.000000pt}%
\definecolor{currentstroke}{rgb}{0.386433,0.794644,0.372886}%
\pgfsetstrokecolor{currentstroke}%
\pgfsetdash{}{0pt}%
\pgfpathmoveto{\pgfqpoint{4.846352in}{3.550127in}}%
\pgfpathlineto{\pgfqpoint{4.702136in}{3.534487in}}%
\pgfpathlineto{\pgfqpoint{4.619001in}{3.304536in}}%
\pgfpathclose%
\pgfusepath{fill}%
\end{pgfscope}%
\begin{pgfscope}%
\pgfpathrectangle{\pgfqpoint{0.680860in}{0.078740in}}{\pgfqpoint{7.842520in}{7.842520in}}%
\pgfusepath{clip}%
\pgfsetbuttcap%
\pgfsetroundjoin%
\definecolor{currentfill}{rgb}{0.281477,0.755203,0.432552}%
\pgfsetfillcolor{currentfill}%
\pgfsetlinewidth{0.000000pt}%
\definecolor{currentstroke}{rgb}{0.395174,0.797475,0.367757}%
\pgfsetstrokecolor{currentstroke}%
\pgfsetdash{}{0pt}%
\pgfpathmoveto{\pgfqpoint{5.467808in}{4.361890in}}%
\pgfpathlineto{\pgfqpoint{5.321904in}{4.325583in}}%
\pgfpathlineto{\pgfqpoint{5.239825in}{4.169783in}}%
\pgfpathclose%
\pgfusepath{fill}%
\end{pgfscope}%
\begin{pgfscope}%
\pgfpathrectangle{\pgfqpoint{0.680860in}{0.078740in}}{\pgfqpoint{7.842520in}{7.842520in}}%
\pgfusepath{clip}%
\pgfsetbuttcap%
\pgfsetroundjoin%
\definecolor{currentfill}{rgb}{0.273006,0.204520,0.501721}%
\pgfsetfillcolor{currentfill}%
\pgfsetlinewidth{0.000000pt}%
\definecolor{currentstroke}{rgb}{0.404001,0.800275,0.362552}%
\pgfsetstrokecolor{currentstroke}%
\pgfsetdash{}{0pt}%
\pgfpathmoveto{\pgfqpoint{4.119642in}{1.646490in}}%
\pgfpathlineto{\pgfqpoint{4.202774in}{1.945944in}}%
\pgfpathlineto{\pgfqpoint{4.059915in}{1.970451in}}%
\pgfpathclose%
\pgfusepath{fill}%
\end{pgfscope}%
\begin{pgfscope}%
\pgfpathrectangle{\pgfqpoint{0.680860in}{0.078740in}}{\pgfqpoint{7.842520in}{7.842520in}}%
\pgfusepath{clip}%
\pgfsetbuttcap%
\pgfsetroundjoin%
\definecolor{currentfill}{rgb}{0.225863,0.330805,0.547314}%
\pgfsetfillcolor{currentfill}%
\pgfsetlinewidth{0.000000pt}%
\definecolor{currentstroke}{rgb}{0.412913,0.803041,0.357269}%
\pgfsetstrokecolor{currentstroke}%
\pgfsetdash{}{0pt}%
\pgfpathmoveto{\pgfqpoint{4.142990in}{2.256284in}}%
\pgfpathlineto{\pgfqpoint{4.285972in}{2.238972in}}%
\pgfpathlineto{\pgfqpoint{4.369221in}{2.523409in}}%
\pgfpathclose%
\pgfusepath{fill}%
\end{pgfscope}%
\begin{pgfscope}%
\pgfpathrectangle{\pgfqpoint{0.680860in}{0.078740in}}{\pgfqpoint{7.842520in}{7.842520in}}%
\pgfusepath{clip}%
\pgfsetbuttcap%
\pgfsetroundjoin%
\definecolor{currentfill}{rgb}{0.257322,0.256130,0.526563}%
\pgfsetfillcolor{currentfill}%
\pgfsetlinewidth{0.000000pt}%
\definecolor{currentstroke}{rgb}{0.421908,0.805774,0.351910}%
\pgfsetstrokecolor{currentstroke}%
\pgfsetdash{}{0pt}%
\pgfpathmoveto{\pgfqpoint{4.059915in}{1.970451in}}%
\pgfpathlineto{\pgfqpoint{4.202774in}{1.945944in}}%
\pgfpathlineto{\pgfqpoint{4.285972in}{2.238972in}}%
\pgfpathclose%
\pgfusepath{fill}%
\end{pgfscope}%
\begin{pgfscope}%
\pgfpathrectangle{\pgfqpoint{0.680860in}{0.078740in}}{\pgfqpoint{7.842520in}{7.842520in}}%
\pgfusepath{clip}%
\pgfsetbuttcap%
\pgfsetroundjoin%
\definecolor{currentfill}{rgb}{0.699415,0.867117,0.175971}%
\pgfsetfillcolor{currentfill}%
\pgfsetlinewidth{0.000000pt}%
\definecolor{currentstroke}{rgb}{0.430983,0.808473,0.346476}%
\pgfsetstrokecolor{currentstroke}%
\pgfsetdash{}{0pt}%
\pgfpathmoveto{\pgfqpoint{6.606243in}{5.012964in}}%
\pgfpathlineto{\pgfqpoint{6.831776in}{5.062471in}}%
\pgfpathlineto{\pgfqpoint{6.681763in}{5.008239in}}%
\pgfpathclose%
\pgfusepath{fill}%
\end{pgfscope}%
\begin{pgfscope}%
\pgfpathrectangle{\pgfqpoint{0.680860in}{0.078740in}}{\pgfqpoint{7.842520in}{7.842520in}}%
\pgfusepath{clip}%
\pgfsetbuttcap%
\pgfsetroundjoin%
\definecolor{currentfill}{rgb}{0.720391,0.870350,0.162603}%
\pgfsetfillcolor{currentfill}%
\pgfsetlinewidth{0.000000pt}%
\definecolor{currentstroke}{rgb}{0.440137,0.811138,0.340967}%
\pgfsetstrokecolor{currentstroke}%
\pgfsetdash{}{0pt}%
\pgfpathmoveto{\pgfqpoint{6.831776in}{5.062471in}}%
\pgfpathlineto{\pgfqpoint{6.982775in}{5.120019in}}%
\pgfpathlineto{\pgfqpoint{6.905397in}{5.025761in}}%
\pgfpathclose%
\pgfusepath{fill}%
\end{pgfscope}%
\begin{pgfscope}%
\pgfpathrectangle{\pgfqpoint{0.680860in}{0.078740in}}{\pgfqpoint{7.842520in}{7.842520in}}%
\pgfusepath{clip}%
\pgfsetbuttcap%
\pgfsetroundjoin%
\definecolor{currentfill}{rgb}{0.196571,0.711827,0.479221}%
\pgfsetfillcolor{currentfill}%
\pgfsetlinewidth{0.000000pt}%
\definecolor{currentstroke}{rgb}{0.449368,0.813768,0.335384}%
\pgfsetstrokecolor{currentstroke}%
\pgfsetdash{}{0pt}%
\pgfpathmoveto{\pgfqpoint{5.012218in}{3.963675in}}%
\pgfpathlineto{\pgfqpoint{5.157312in}{3.990769in}}%
\pgfpathlineto{\pgfqpoint{5.239825in}{4.169783in}}%
\pgfpathclose%
\pgfusepath{fill}%
\end{pgfscope}%
\begin{pgfscope}%
\pgfpathrectangle{\pgfqpoint{0.680860in}{0.078740in}}{\pgfqpoint{7.842520in}{7.842520in}}%
\pgfusepath{clip}%
\pgfsetbuttcap%
\pgfsetroundjoin%
\definecolor{currentfill}{rgb}{0.506271,0.828786,0.300362}%
\pgfsetfillcolor{currentfill}%
\pgfsetlinewidth{0.000000pt}%
\definecolor{currentstroke}{rgb}{0.458674,0.816363,0.329727}%
\pgfsetstrokecolor{currentstroke}%
\pgfsetdash{}{0pt}%
\pgfpathmoveto{\pgfqpoint{5.776736in}{4.647059in}}%
\pgfpathlineto{\pgfqpoint{5.924186in}{4.692622in}}%
\pgfpathlineto{\pgfqpoint{6.003961in}{4.776449in}}%
\pgfpathclose%
\pgfusepath{fill}%
\end{pgfscope}%
\begin{pgfscope}%
\pgfpathrectangle{\pgfqpoint{0.680860in}{0.078740in}}{\pgfqpoint{7.842520in}{7.842520in}}%
\pgfusepath{clip}%
\pgfsetbuttcap%
\pgfsetroundjoin%
\definecolor{currentfill}{rgb}{0.377779,0.791781,0.377939}%
\pgfsetfillcolor{currentfill}%
\pgfsetlinewidth{0.000000pt}%
\definecolor{currentstroke}{rgb}{0.468053,0.818921,0.323998}%
\pgfsetstrokecolor{currentstroke}%
\pgfsetdash{}{0pt}%
\pgfpathmoveto{\pgfqpoint{5.695996in}{4.537126in}}%
\pgfpathlineto{\pgfqpoint{5.549309in}{4.495864in}}%
\pgfpathlineto{\pgfqpoint{5.467808in}{4.361890in}}%
\pgfpathclose%
\pgfusepath{fill}%
\end{pgfscope}%
\begin{pgfscope}%
\pgfpathrectangle{\pgfqpoint{0.680860in}{0.078740in}}{\pgfqpoint{7.842520in}{7.842520in}}%
\pgfusepath{clip}%
\pgfsetbuttcap%
\pgfsetroundjoin%
\definecolor{currentfill}{rgb}{0.144759,0.519093,0.556572}%
\pgfsetfillcolor{currentfill}%
\pgfsetlinewidth{0.000000pt}%
\definecolor{currentstroke}{rgb}{0.477504,0.821444,0.318195}%
\pgfsetstrokecolor{currentstroke}%
\pgfsetdash{}{0pt}%
\pgfpathmoveto{\pgfqpoint{4.619001in}{3.304536in}}%
\pgfpathlineto{\pgfqpoint{4.535772in}{3.058162in}}%
\pgfpathlineto{\pgfqpoint{4.679774in}{3.062239in}}%
\pgfpathclose%
\pgfusepath{fill}%
\end{pgfscope}%
\begin{pgfscope}%
\pgfpathrectangle{\pgfqpoint{0.680860in}{0.078740in}}{\pgfqpoint{7.842520in}{7.842520in}}%
\pgfusepath{clip}%
\pgfsetbuttcap%
\pgfsetroundjoin%
\definecolor{currentfill}{rgb}{0.143303,0.669459,0.511215}%
\pgfsetfillcolor{currentfill}%
\pgfsetlinewidth{0.000000pt}%
\definecolor{currentstroke}{rgb}{0.487026,0.823929,0.312321}%
\pgfsetstrokecolor{currentstroke}%
\pgfsetdash{}{0pt}%
\pgfpathmoveto{\pgfqpoint{5.012218in}{3.963675in}}%
\pgfpathlineto{\pgfqpoint{4.929407in}{3.766964in}}%
\pgfpathlineto{\pgfqpoint{5.074444in}{3.789628in}}%
\pgfpathclose%
\pgfusepath{fill}%
\end{pgfscope}%
\begin{pgfscope}%
\pgfpathrectangle{\pgfqpoint{0.680860in}{0.078740in}}{\pgfqpoint{7.842520in}{7.842520in}}%
\pgfusepath{clip}%
\pgfsetbuttcap%
\pgfsetroundjoin%
\definecolor{currentfill}{rgb}{0.668054,0.861999,0.196293}%
\pgfsetfillcolor{currentfill}%
\pgfsetlinewidth{0.000000pt}%
\definecolor{currentstroke}{rgb}{0.496615,0.826376,0.306377}%
\pgfsetstrokecolor{currentstroke}%
\pgfsetdash{}{0pt}%
\pgfpathmoveto{\pgfqpoint{6.456767in}{4.959433in}}%
\pgfpathlineto{\pgfqpoint{6.379591in}{4.933197in}}%
\pgfpathlineto{\pgfqpoint{6.606243in}{5.012964in}}%
\pgfpathclose%
\pgfusepath{fill}%
\end{pgfscope}%
\begin{pgfscope}%
\pgfpathrectangle{\pgfqpoint{0.680860in}{0.078740in}}{\pgfqpoint{7.842520in}{7.842520in}}%
\pgfusepath{clip}%
\pgfsetbuttcap%
\pgfsetroundjoin%
\definecolor{currentfill}{rgb}{0.125394,0.574318,0.549086}%
\pgfsetfillcolor{currentfill}%
\pgfsetlinewidth{0.000000pt}%
\definecolor{currentstroke}{rgb}{0.506271,0.828786,0.300362}%
\pgfsetstrokecolor{currentstroke}%
\pgfsetdash{}{0pt}%
\pgfpathmoveto{\pgfqpoint{4.619001in}{3.304536in}}%
\pgfpathlineto{\pgfqpoint{4.763121in}{3.314657in}}%
\pgfpathlineto{\pgfqpoint{4.846352in}{3.550127in}}%
\pgfpathclose%
\pgfusepath{fill}%
\end{pgfscope}%
\begin{pgfscope}%
\pgfpathrectangle{\pgfqpoint{0.680860in}{0.078740in}}{\pgfqpoint{7.842520in}{7.842520in}}%
\pgfusepath{clip}%
\pgfsetbuttcap%
\pgfsetroundjoin%
\definecolor{currentfill}{rgb}{0.281446,0.084320,0.407414}%
\pgfsetfillcolor{currentfill}%
\pgfsetlinewidth{0.000000pt}%
\definecolor{currentstroke}{rgb}{0.515992,0.831158,0.294279}%
\pgfsetstrokecolor{currentstroke}%
\pgfsetdash{}{0pt}%
\pgfpathmoveto{\pgfqpoint{4.119642in}{1.646490in}}%
\pgfpathlineto{\pgfqpoint{4.036577in}{1.342806in}}%
\pgfpathlineto{\pgfqpoint{4.179721in}{1.303599in}}%
\pgfpathclose%
\pgfusepath{fill}%
\end{pgfscope}%
\begin{pgfscope}%
\pgfpathrectangle{\pgfqpoint{0.680860in}{0.078740in}}{\pgfqpoint{7.842520in}{7.842520in}}%
\pgfusepath{clip}%
\pgfsetbuttcap%
\pgfsetroundjoin%
\definecolor{currentfill}{rgb}{0.159194,0.482237,0.558073}%
\pgfsetfillcolor{currentfill}%
\pgfsetlinewidth{0.000000pt}%
\definecolor{currentstroke}{rgb}{0.525776,0.833491,0.288127}%
\pgfsetstrokecolor{currentstroke}%
\pgfsetdash{}{0pt}%
\pgfpathmoveto{\pgfqpoint{4.679774in}{3.062239in}}%
\pgfpathlineto{\pgfqpoint{4.535772in}{3.058162in}}%
\pgfpathlineto{\pgfqpoint{4.452498in}{2.797148in}}%
\pgfpathclose%
\pgfusepath{fill}%
\end{pgfscope}%
\begin{pgfscope}%
\pgfpathrectangle{\pgfqpoint{0.680860in}{0.078740in}}{\pgfqpoint{7.842520in}{7.842520in}}%
\pgfusepath{clip}%
\pgfsetbuttcap%
\pgfsetroundjoin%
\definecolor{currentfill}{rgb}{0.162016,0.687316,0.499129}%
\pgfsetfillcolor{currentfill}%
\pgfsetlinewidth{0.000000pt}%
\definecolor{currentstroke}{rgb}{0.535621,0.835785,0.281908}%
\pgfsetstrokecolor{currentstroke}%
\pgfsetdash{}{0pt}%
\pgfpathmoveto{\pgfqpoint{5.157312in}{3.990769in}}%
\pgfpathlineto{\pgfqpoint{5.012218in}{3.963675in}}%
\pgfpathlineto{\pgfqpoint{5.074444in}{3.789628in}}%
\pgfpathclose%
\pgfusepath{fill}%
\end{pgfscope}%
\begin{pgfscope}%
\pgfpathrectangle{\pgfqpoint{0.680860in}{0.078740in}}{\pgfqpoint{7.842520in}{7.842520in}}%
\pgfusepath{clip}%
\pgfsetbuttcap%
\pgfsetroundjoin%
\definecolor{currentfill}{rgb}{0.468053,0.818921,0.323998}%
\pgfsetfillcolor{currentfill}%
\pgfsetlinewidth{0.000000pt}%
\definecolor{currentstroke}{rgb}{0.545524,0.838039,0.275626}%
\pgfsetstrokecolor{currentstroke}%
\pgfsetdash{}{0pt}%
\pgfpathmoveto{\pgfqpoint{5.924186in}{4.692622in}}%
\pgfpathlineto{\pgfqpoint{5.776736in}{4.647059in}}%
\pgfpathlineto{\pgfqpoint{5.695996in}{4.537126in}}%
\pgfpathclose%
\pgfusepath{fill}%
\end{pgfscope}%
\begin{pgfscope}%
\pgfpathrectangle{\pgfqpoint{0.680860in}{0.078740in}}{\pgfqpoint{7.842520in}{7.842520in}}%
\pgfusepath{clip}%
\pgfsetbuttcap%
\pgfsetroundjoin%
\definecolor{currentfill}{rgb}{0.271828,0.209303,0.504434}%
\pgfsetfillcolor{currentfill}%
\pgfsetlinewidth{0.000000pt}%
\definecolor{currentstroke}{rgb}{0.555484,0.840254,0.269281}%
\pgfsetstrokecolor{currentstroke}%
\pgfsetdash{}{0pt}%
\pgfpathmoveto{\pgfqpoint{4.346201in}{1.921995in}}%
\pgfpathlineto{\pgfqpoint{4.202774in}{1.945944in}}%
\pgfpathlineto{\pgfqpoint{4.119642in}{1.646490in}}%
\pgfpathclose%
\pgfusepath{fill}%
\end{pgfscope}%
\begin{pgfscope}%
\pgfpathrectangle{\pgfqpoint{0.680860in}{0.078740in}}{\pgfqpoint{7.842520in}{7.842520in}}%
\pgfusepath{clip}%
\pgfsetbuttcap%
\pgfsetroundjoin%
\definecolor{currentfill}{rgb}{0.182256,0.426184,0.557120}%
\pgfsetfillcolor{currentfill}%
\pgfsetlinewidth{0.000000pt}%
\definecolor{currentstroke}{rgb}{0.565498,0.842430,0.262877}%
\pgfsetstrokecolor{currentstroke}%
\pgfsetdash{}{0pt}%
\pgfpathmoveto{\pgfqpoint{4.452498in}{2.797148in}}%
\pgfpathlineto{\pgfqpoint{4.369221in}{2.523409in}}%
\pgfpathlineto{\pgfqpoint{4.596367in}{2.794721in}}%
\pgfpathclose%
\pgfusepath{fill}%
\end{pgfscope}%
\begin{pgfscope}%
\pgfpathrectangle{\pgfqpoint{0.680860in}{0.078740in}}{\pgfqpoint{7.842520in}{7.842520in}}%
\pgfusepath{clip}%
\pgfsetbuttcap%
\pgfsetroundjoin%
\definecolor{currentfill}{rgb}{0.136408,0.541173,0.554483}%
\pgfsetfillcolor{currentfill}%
\pgfsetlinewidth{0.000000pt}%
\definecolor{currentstroke}{rgb}{0.575563,0.844566,0.256415}%
\pgfsetstrokecolor{currentstroke}%
\pgfsetdash{}{0pt}%
\pgfpathmoveto{\pgfqpoint{4.679774in}{3.062239in}}%
\pgfpathlineto{\pgfqpoint{4.763121in}{3.314657in}}%
\pgfpathlineto{\pgfqpoint{4.619001in}{3.304536in}}%
\pgfpathclose%
\pgfusepath{fill}%
\end{pgfscope}%
\begin{pgfscope}%
\pgfpathrectangle{\pgfqpoint{0.680860in}{0.078740in}}{\pgfqpoint{7.842520in}{7.842520in}}%
\pgfusepath{clip}%
\pgfsetbuttcap%
\pgfsetroundjoin%
\definecolor{currentfill}{rgb}{0.626579,0.854645,0.223353}%
\pgfsetfillcolor{currentfill}%
\pgfsetlinewidth{0.000000pt}%
\definecolor{currentstroke}{rgb}{0.585678,0.846661,0.249897}%
\pgfsetstrokecolor{currentstroke}%
\pgfsetdash{}{0pt}%
\pgfpathmoveto{\pgfqpoint{6.230731in}{4.881403in}}%
\pgfpathlineto{\pgfqpoint{6.152140in}{4.825555in}}%
\pgfpathlineto{\pgfqpoint{6.379591in}{4.933197in}}%
\pgfpathclose%
\pgfusepath{fill}%
\end{pgfscope}%
\begin{pgfscope}%
\pgfpathrectangle{\pgfqpoint{0.680860in}{0.078740in}}{\pgfqpoint{7.842520in}{7.842520in}}%
\pgfusepath{clip}%
\pgfsetbuttcap%
\pgfsetroundjoin%
\definecolor{currentfill}{rgb}{0.281477,0.755203,0.432552}%
\pgfsetfillcolor{currentfill}%
\pgfsetlinewidth{0.000000pt}%
\definecolor{currentstroke}{rgb}{0.595839,0.848717,0.243329}%
\pgfsetstrokecolor{currentstroke}%
\pgfsetdash{}{0pt}%
\pgfpathmoveto{\pgfqpoint{5.239825in}{4.169783in}}%
\pgfpathlineto{\pgfqpoint{5.385741in}{4.203046in}}%
\pgfpathlineto{\pgfqpoint{5.467808in}{4.361890in}}%
\pgfpathclose%
\pgfusepath{fill}%
\end{pgfscope}%
\begin{pgfscope}%
\pgfpathrectangle{\pgfqpoint{0.680860in}{0.078740in}}{\pgfqpoint{7.842520in}{7.842520in}}%
\pgfusepath{clip}%
\pgfsetbuttcap%
\pgfsetroundjoin%
\definecolor{currentfill}{rgb}{0.555484,0.840254,0.269281}%
\pgfsetfillcolor{currentfill}%
\pgfsetlinewidth{0.000000pt}%
\definecolor{currentstroke}{rgb}{0.606045,0.850733,0.236712}%
\pgfsetstrokecolor{currentstroke}%
\pgfsetdash{}{0pt}%
\pgfpathmoveto{\pgfqpoint{6.003961in}{4.776449in}}%
\pgfpathlineto{\pgfqpoint{5.924186in}{4.692622in}}%
\pgfpathlineto{\pgfqpoint{6.152140in}{4.825555in}}%
\pgfpathclose%
\pgfusepath{fill}%
\end{pgfscope}%
\begin{pgfscope}%
\pgfpathrectangle{\pgfqpoint{0.680860in}{0.078740in}}{\pgfqpoint{7.842520in}{7.842520in}}%
\pgfusepath{clip}%
\pgfsetbuttcap%
\pgfsetroundjoin%
\definecolor{currentfill}{rgb}{0.244972,0.287675,0.537260}%
\pgfsetfillcolor{currentfill}%
\pgfsetlinewidth{0.000000pt}%
\definecolor{currentstroke}{rgb}{0.616293,0.852709,0.230052}%
\pgfsetstrokecolor{currentstroke}%
\pgfsetdash{}{0pt}%
\pgfpathmoveto{\pgfqpoint{4.285972in}{2.238972in}}%
\pgfpathlineto{\pgfqpoint{4.202774in}{1.945944in}}%
\pgfpathlineto{\pgfqpoint{4.429547in}{2.222456in}}%
\pgfpathclose%
\pgfusepath{fill}%
\end{pgfscope}%
\begin{pgfscope}%
\pgfpathrectangle{\pgfqpoint{0.680860in}{0.078740in}}{\pgfqpoint{7.842520in}{7.842520in}}%
\pgfusepath{clip}%
\pgfsetbuttcap%
\pgfsetroundjoin%
\definecolor{currentfill}{rgb}{0.221989,0.339161,0.548752}%
\pgfsetfillcolor{currentfill}%
\pgfsetlinewidth{0.000000pt}%
\definecolor{currentstroke}{rgb}{0.626579,0.854645,0.223353}%
\pgfsetstrokecolor{currentstroke}%
\pgfsetdash{}{0pt}%
\pgfpathmoveto{\pgfqpoint{4.369221in}{2.523409in}}%
\pgfpathlineto{\pgfqpoint{4.285972in}{2.238972in}}%
\pgfpathlineto{\pgfqpoint{4.429547in}{2.222456in}}%
\pgfpathclose%
\pgfusepath{fill}%
\end{pgfscope}%
\begin{pgfscope}%
\pgfpathrectangle{\pgfqpoint{0.680860in}{0.078740in}}{\pgfqpoint{7.842520in}{7.842520in}}%
\pgfusepath{clip}%
\pgfsetbuttcap%
\pgfsetroundjoin%
\definecolor{currentfill}{rgb}{0.283229,0.120777,0.440584}%
\pgfsetfillcolor{currentfill}%
\pgfsetlinewidth{0.000000pt}%
\definecolor{currentstroke}{rgb}{0.636902,0.856542,0.216620}%
\pgfsetstrokecolor{currentstroke}%
\pgfsetdash{}{0pt}%
\pgfpathmoveto{\pgfqpoint{4.179721in}{1.303599in}}%
\pgfpathlineto{\pgfqpoint{4.262923in}{1.614952in}}%
\pgfpathlineto{\pgfqpoint{4.119642in}{1.646490in}}%
\pgfpathclose%
\pgfusepath{fill}%
\end{pgfscope}%
\begin{pgfscope}%
\pgfpathrectangle{\pgfqpoint{0.680860in}{0.078740in}}{\pgfqpoint{7.842520in}{7.842520in}}%
\pgfusepath{clip}%
\pgfsetbuttcap%
\pgfsetroundjoin%
\definecolor{currentfill}{rgb}{0.121380,0.629492,0.531973}%
\pgfsetfillcolor{currentfill}%
\pgfsetlinewidth{0.000000pt}%
\definecolor{currentstroke}{rgb}{0.647257,0.858400,0.209861}%
\pgfsetstrokecolor{currentstroke}%
\pgfsetdash{}{0pt}%
\pgfpathmoveto{\pgfqpoint{4.929407in}{3.766964in}}%
\pgfpathlineto{\pgfqpoint{4.846352in}{3.550127in}}%
\pgfpathlineto{\pgfqpoint{4.991298in}{3.567692in}}%
\pgfpathclose%
\pgfusepath{fill}%
\end{pgfscope}%
\begin{pgfscope}%
\pgfpathrectangle{\pgfqpoint{0.680860in}{0.078740in}}{\pgfqpoint{7.842520in}{7.842520in}}%
\pgfusepath{clip}%
\pgfsetbuttcap%
\pgfsetroundjoin%
\definecolor{currentfill}{rgb}{0.278012,0.180367,0.486697}%
\pgfsetfillcolor{currentfill}%
\pgfsetlinewidth{0.000000pt}%
\definecolor{currentstroke}{rgb}{0.657642,0.860219,0.203082}%
\pgfsetstrokecolor{currentstroke}%
\pgfsetdash{}{0pt}%
\pgfpathmoveto{\pgfqpoint{4.119642in}{1.646490in}}%
\pgfpathlineto{\pgfqpoint{4.262923in}{1.614952in}}%
\pgfpathlineto{\pgfqpoint{4.346201in}{1.921995in}}%
\pgfpathclose%
\pgfusepath{fill}%
\end{pgfscope}%
\begin{pgfscope}%
\pgfpathrectangle{\pgfqpoint{0.680860in}{0.078740in}}{\pgfqpoint{7.842520in}{7.842520in}}%
\pgfusepath{clip}%
\pgfsetbuttcap%
\pgfsetroundjoin%
\definecolor{currentfill}{rgb}{0.163625,0.471133,0.558148}%
\pgfsetfillcolor{currentfill}%
\pgfsetlinewidth{0.000000pt}%
\definecolor{currentstroke}{rgb}{0.668054,0.861999,0.196293}%
\pgfsetstrokecolor{currentstroke}%
\pgfsetdash{}{0pt}%
\pgfpathmoveto{\pgfqpoint{4.452498in}{2.797148in}}%
\pgfpathlineto{\pgfqpoint{4.596367in}{2.794721in}}%
\pgfpathlineto{\pgfqpoint{4.679774in}{3.062239in}}%
\pgfpathclose%
\pgfusepath{fill}%
\end{pgfscope}%
\begin{pgfscope}%
\pgfpathrectangle{\pgfqpoint{0.680860in}{0.078740in}}{\pgfqpoint{7.842520in}{7.842520in}}%
\pgfusepath{clip}%
\pgfsetbuttcap%
\pgfsetroundjoin%
\definecolor{currentfill}{rgb}{0.214000,0.722114,0.469588}%
\pgfsetfillcolor{currentfill}%
\pgfsetlinewidth{0.000000pt}%
\definecolor{currentstroke}{rgb}{0.678489,0.863742,0.189503}%
\pgfsetstrokecolor{currentstroke}%
\pgfsetdash{}{0pt}%
\pgfpathmoveto{\pgfqpoint{5.239825in}{4.169783in}}%
\pgfpathlineto{\pgfqpoint{5.157312in}{3.990769in}}%
\pgfpathlineto{\pgfqpoint{5.303193in}{4.020195in}}%
\pgfpathclose%
\pgfusepath{fill}%
\end{pgfscope}%
\begin{pgfscope}%
\pgfpathrectangle{\pgfqpoint{0.680860in}{0.078740in}}{\pgfqpoint{7.842520in}{7.842520in}}%
\pgfusepath{clip}%
\pgfsetbuttcap%
\pgfsetroundjoin%
\definecolor{currentfill}{rgb}{0.253935,0.265254,0.529983}%
\pgfsetfillcolor{currentfill}%
\pgfsetlinewidth{0.000000pt}%
\definecolor{currentstroke}{rgb}{0.688944,0.865448,0.182725}%
\pgfsetstrokecolor{currentstroke}%
\pgfsetdash{}{0pt}%
\pgfpathmoveto{\pgfqpoint{4.202774in}{1.945944in}}%
\pgfpathlineto{\pgfqpoint{4.346201in}{1.921995in}}%
\pgfpathlineto{\pgfqpoint{4.429547in}{2.222456in}}%
\pgfpathclose%
\pgfusepath{fill}%
\end{pgfscope}%
\begin{pgfscope}%
\pgfpathrectangle{\pgfqpoint{0.680860in}{0.078740in}}{\pgfqpoint{7.842520in}{7.842520in}}%
\pgfusepath{clip}%
\pgfsetbuttcap%
\pgfsetroundjoin%
\definecolor{currentfill}{rgb}{0.188923,0.410910,0.556326}%
\pgfsetfillcolor{currentfill}%
\pgfsetlinewidth{0.000000pt}%
\definecolor{currentstroke}{rgb}{0.699415,0.867117,0.175971}%
\pgfsetstrokecolor{currentstroke}%
\pgfsetdash{}{0pt}%
\pgfpathmoveto{\pgfqpoint{4.369221in}{2.523409in}}%
\pgfpathlineto{\pgfqpoint{4.512945in}{2.514093in}}%
\pgfpathlineto{\pgfqpoint{4.596367in}{2.794721in}}%
\pgfpathclose%
\pgfusepath{fill}%
\end{pgfscope}%
\begin{pgfscope}%
\pgfpathrectangle{\pgfqpoint{0.680860in}{0.078740in}}{\pgfqpoint{7.842520in}{7.842520in}}%
\pgfusepath{clip}%
\pgfsetbuttcap%
\pgfsetroundjoin%
\definecolor{currentfill}{rgb}{0.210503,0.363727,0.552206}%
\pgfsetfillcolor{currentfill}%
\pgfsetlinewidth{0.000000pt}%
\definecolor{currentstroke}{rgb}{0.709898,0.868751,0.169257}%
\pgfsetstrokecolor{currentstroke}%
\pgfsetdash{}{0pt}%
\pgfpathmoveto{\pgfqpoint{4.429547in}{2.222456in}}%
\pgfpathlineto{\pgfqpoint{4.512945in}{2.514093in}}%
\pgfpathlineto{\pgfqpoint{4.369221in}{2.523409in}}%
\pgfpathclose%
\pgfusepath{fill}%
\end{pgfscope}%
\begin{pgfscope}%
\pgfpathrectangle{\pgfqpoint{0.680860in}{0.078740in}}{\pgfqpoint{7.842520in}{7.842520in}}%
\pgfusepath{clip}%
\pgfsetbuttcap%
\pgfsetroundjoin%
\definecolor{currentfill}{rgb}{0.386433,0.794644,0.372886}%
\pgfsetfillcolor{currentfill}%
\pgfsetlinewidth{0.000000pt}%
\definecolor{currentstroke}{rgb}{0.720391,0.870350,0.162603}%
\pgfsetstrokecolor{currentstroke}%
\pgfsetdash{}{0pt}%
\pgfpathmoveto{\pgfqpoint{5.467808in}{4.361890in}}%
\pgfpathlineto{\pgfqpoint{5.614548in}{4.400859in}}%
\pgfpathlineto{\pgfqpoint{5.695996in}{4.537126in}}%
\pgfpathclose%
\pgfusepath{fill}%
\end{pgfscope}%
\begin{pgfscope}%
\pgfpathrectangle{\pgfqpoint{0.680860in}{0.078740in}}{\pgfqpoint{7.842520in}{7.842520in}}%
\pgfusepath{clip}%
\pgfsetbuttcap%
\pgfsetroundjoin%
\definecolor{currentfill}{rgb}{0.130067,0.651384,0.521608}%
\pgfsetfillcolor{currentfill}%
\pgfsetlinewidth{0.000000pt}%
\definecolor{currentstroke}{rgb}{0.730889,0.871916,0.156029}%
\pgfsetstrokecolor{currentstroke}%
\pgfsetdash{}{0pt}%
\pgfpathmoveto{\pgfqpoint{4.991298in}{3.567692in}}%
\pgfpathlineto{\pgfqpoint{5.074444in}{3.789628in}}%
\pgfpathlineto{\pgfqpoint{4.929407in}{3.766964in}}%
\pgfpathclose%
\pgfusepath{fill}%
\end{pgfscope}%
\begin{pgfscope}%
\pgfpathrectangle{\pgfqpoint{0.680860in}{0.078740in}}{\pgfqpoint{7.842520in}{7.842520in}}%
\pgfusepath{clip}%
\pgfsetbuttcap%
\pgfsetroundjoin%
\definecolor{currentfill}{rgb}{0.120092,0.600104,0.542530}%
\pgfsetfillcolor{currentfill}%
\pgfsetlinewidth{0.000000pt}%
\definecolor{currentstroke}{rgb}{0.741388,0.873449,0.149561}%
\pgfsetstrokecolor{currentstroke}%
\pgfsetdash{}{0pt}%
\pgfpathmoveto{\pgfqpoint{4.991298in}{3.567692in}}%
\pgfpathlineto{\pgfqpoint{4.846352in}{3.550127in}}%
\pgfpathlineto{\pgfqpoint{4.763121in}{3.314657in}}%
\pgfpathclose%
\pgfusepath{fill}%
\end{pgfscope}%
\begin{pgfscope}%
\pgfpathrectangle{\pgfqpoint{0.680860in}{0.078740in}}{\pgfqpoint{7.842520in}{7.842520in}}%
\pgfusepath{clip}%
\pgfsetbuttcap%
\pgfsetroundjoin%
\definecolor{currentfill}{rgb}{0.741388,0.873449,0.149561}%
\pgfsetfillcolor{currentfill}%
\pgfsetlinewidth{0.000000pt}%
\definecolor{currentstroke}{rgb}{0.751884,0.874951,0.143228}%
\pgfsetstrokecolor{currentstroke}%
\pgfsetdash{}{0pt}%
\pgfpathmoveto{\pgfqpoint{6.756690in}{5.069787in}}%
\pgfpathlineto{\pgfqpoint{6.831776in}{5.062471in}}%
\pgfpathlineto{\pgfqpoint{6.606243in}{5.012964in}}%
\pgfpathclose%
\pgfusepath{fill}%
\end{pgfscope}%
\begin{pgfscope}%
\pgfpathrectangle{\pgfqpoint{0.680860in}{0.078740in}}{\pgfqpoint{7.842520in}{7.842520in}}%
\pgfusepath{clip}%
\pgfsetbuttcap%
\pgfsetroundjoin%
\definecolor{currentfill}{rgb}{0.246070,0.738910,0.452024}%
\pgfsetfillcolor{currentfill}%
\pgfsetlinewidth{0.000000pt}%
\definecolor{currentstroke}{rgb}{0.762373,0.876424,0.137064}%
\pgfsetstrokecolor{currentstroke}%
\pgfsetdash{}{0pt}%
\pgfpathmoveto{\pgfqpoint{5.303193in}{4.020195in}}%
\pgfpathlineto{\pgfqpoint{5.385741in}{4.203046in}}%
\pgfpathlineto{\pgfqpoint{5.239825in}{4.169783in}}%
\pgfpathclose%
\pgfusepath{fill}%
\end{pgfscope}%
\begin{pgfscope}%
\pgfpathrectangle{\pgfqpoint{0.680860in}{0.078740in}}{\pgfqpoint{7.842520in}{7.842520in}}%
\pgfusepath{clip}%
\pgfsetbuttcap%
\pgfsetroundjoin%
\definecolor{currentfill}{rgb}{0.804182,0.882046,0.114965}%
\pgfsetfillcolor{currentfill}%
\pgfsetlinewidth{0.000000pt}%
\definecolor{currentstroke}{rgb}{0.772852,0.877868,0.131109}%
\pgfsetstrokecolor{currentstroke}%
\pgfsetdash{}{0pt}%
\pgfpathmoveto{\pgfqpoint{7.134788in}{5.181011in}}%
\pgfpathlineto{\pgfqpoint{7.207299in}{5.136707in}}%
\pgfpathlineto{\pgfqpoint{7.055852in}{5.079587in}}%
\pgfpathclose%
\pgfusepath{fill}%
\end{pgfscope}%
\begin{pgfscope}%
\pgfpathrectangle{\pgfqpoint{0.680860in}{0.078740in}}{\pgfqpoint{7.842520in}{7.842520in}}%
\pgfusepath{clip}%
\pgfsetbuttcap%
\pgfsetroundjoin%
\definecolor{currentfill}{rgb}{0.487026,0.823929,0.312321}%
\pgfsetfillcolor{currentfill}%
\pgfsetlinewidth{0.000000pt}%
\definecolor{currentstroke}{rgb}{0.783315,0.879285,0.125405}%
\pgfsetstrokecolor{currentstroke}%
\pgfsetdash{}{0pt}%
\pgfpathmoveto{\pgfqpoint{5.843552in}{4.581232in}}%
\pgfpathlineto{\pgfqpoint{5.924186in}{4.692622in}}%
\pgfpathlineto{\pgfqpoint{5.695996in}{4.537126in}}%
\pgfpathclose%
\pgfusepath{fill}%
\end{pgfscope}%
\begin{pgfscope}%
\pgfpathrectangle{\pgfqpoint{0.680860in}{0.078740in}}{\pgfqpoint{7.842520in}{7.842520in}}%
\pgfusepath{clip}%
\pgfsetbuttcap%
\pgfsetroundjoin%
\definecolor{currentfill}{rgb}{0.180653,0.701402,0.488189}%
\pgfsetfillcolor{currentfill}%
\pgfsetlinewidth{0.000000pt}%
\definecolor{currentstroke}{rgb}{0.793760,0.880678,0.120005}%
\pgfsetstrokecolor{currentstroke}%
\pgfsetdash{}{0pt}%
\pgfpathmoveto{\pgfqpoint{5.074444in}{3.789628in}}%
\pgfpathlineto{\pgfqpoint{5.303193in}{4.020195in}}%
\pgfpathlineto{\pgfqpoint{5.157312in}{3.990769in}}%
\pgfpathclose%
\pgfusepath{fill}%
\end{pgfscope}%
\begin{pgfscope}%
\pgfpathrectangle{\pgfqpoint{0.680860in}{0.078740in}}{\pgfqpoint{7.842520in}{7.842520in}}%
\pgfusepath{clip}%
\pgfsetbuttcap%
\pgfsetroundjoin%
\definecolor{currentfill}{rgb}{0.132444,0.552216,0.553018}%
\pgfsetfillcolor{currentfill}%
\pgfsetlinewidth{0.000000pt}%
\definecolor{currentstroke}{rgb}{0.804182,0.882046,0.114965}%
\pgfsetstrokecolor{currentstroke}%
\pgfsetdash{}{0pt}%
\pgfpathmoveto{\pgfqpoint{4.907948in}{3.326514in}}%
\pgfpathlineto{\pgfqpoint{4.763121in}{3.314657in}}%
\pgfpathlineto{\pgfqpoint{4.679774in}{3.062239in}}%
\pgfpathclose%
\pgfusepath{fill}%
\end{pgfscope}%
\begin{pgfscope}%
\pgfpathrectangle{\pgfqpoint{0.680860in}{0.078740in}}{\pgfqpoint{7.842520in}{7.842520in}}%
\pgfusepath{clip}%
\pgfsetbuttcap%
\pgfsetroundjoin%
\definecolor{currentfill}{rgb}{0.283229,0.120777,0.440584}%
\pgfsetfillcolor{currentfill}%
\pgfsetlinewidth{0.000000pt}%
\definecolor{currentstroke}{rgb}{0.814576,0.883393,0.110347}%
\pgfsetstrokecolor{currentstroke}%
\pgfsetdash{}{0pt}%
\pgfpathmoveto{\pgfqpoint{4.262923in}{1.614952in}}%
\pgfpathlineto{\pgfqpoint{4.179721in}{1.303599in}}%
\pgfpathlineto{\pgfqpoint{4.406749in}{1.583748in}}%
\pgfpathclose%
\pgfusepath{fill}%
\end{pgfscope}%
\begin{pgfscope}%
\pgfpathrectangle{\pgfqpoint{0.680860in}{0.078740in}}{\pgfqpoint{7.842520in}{7.842520in}}%
\pgfusepath{clip}%
\pgfsetbuttcap%
\pgfsetroundjoin%
\definecolor{currentfill}{rgb}{0.772852,0.877868,0.131109}%
\pgfsetfillcolor{currentfill}%
\pgfsetlinewidth{0.000000pt}%
\definecolor{currentstroke}{rgb}{0.824940,0.884720,0.106217}%
\pgfsetstrokecolor{currentstroke}%
\pgfsetdash{}{0pt}%
\pgfpathmoveto{\pgfqpoint{6.982775in}{5.120019in}}%
\pgfpathlineto{\pgfqpoint{6.831776in}{5.062471in}}%
\pgfpathlineto{\pgfqpoint{6.756690in}{5.069787in}}%
\pgfpathclose%
\pgfusepath{fill}%
\end{pgfscope}%
\begin{pgfscope}%
\pgfpathrectangle{\pgfqpoint{0.680860in}{0.078740in}}{\pgfqpoint{7.842520in}{7.842520in}}%
\pgfusepath{clip}%
\pgfsetbuttcap%
\pgfsetroundjoin%
\definecolor{currentfill}{rgb}{0.709898,0.868751,0.169257}%
\pgfsetfillcolor{currentfill}%
\pgfsetlinewidth{0.000000pt}%
\definecolor{currentstroke}{rgb}{0.835270,0.886029,0.102646}%
\pgfsetstrokecolor{currentstroke}%
\pgfsetdash{}{0pt}%
\pgfpathmoveto{\pgfqpoint{6.606243in}{5.012964in}}%
\pgfpathlineto{\pgfqpoint{6.379591in}{4.933197in}}%
\pgfpathlineto{\pgfqpoint{6.529402in}{4.988221in}}%
\pgfpathclose%
\pgfusepath{fill}%
\end{pgfscope}%
\begin{pgfscope}%
\pgfpathrectangle{\pgfqpoint{0.680860in}{0.078740in}}{\pgfqpoint{7.842520in}{7.842520in}}%
\pgfusepath{clip}%
\pgfsetbuttcap%
\pgfsetroundjoin%
\definecolor{currentfill}{rgb}{0.277134,0.185228,0.489898}%
\pgfsetfillcolor{currentfill}%
\pgfsetlinewidth{0.000000pt}%
\definecolor{currentstroke}{rgb}{0.845561,0.887322,0.099702}%
\pgfsetstrokecolor{currentstroke}%
\pgfsetdash{}{0pt}%
\pgfpathmoveto{\pgfqpoint{4.262923in}{1.614952in}}%
\pgfpathlineto{\pgfqpoint{4.406749in}{1.583748in}}%
\pgfpathlineto{\pgfqpoint{4.346201in}{1.921995in}}%
\pgfpathclose%
\pgfusepath{fill}%
\end{pgfscope}%
\begin{pgfscope}%
\pgfpathrectangle{\pgfqpoint{0.680860in}{0.078740in}}{\pgfqpoint{7.842520in}{7.842520in}}%
\pgfusepath{clip}%
\pgfsetbuttcap%
\pgfsetroundjoin%
\definecolor{currentfill}{rgb}{0.311925,0.767822,0.415586}%
\pgfsetfillcolor{currentfill}%
\pgfsetlinewidth{0.000000pt}%
\definecolor{currentstroke}{rgb}{0.855810,0.888601,0.097452}%
\pgfsetstrokecolor{currentstroke}%
\pgfsetdash{}{0pt}%
\pgfpathmoveto{\pgfqpoint{5.532480in}{4.238865in}}%
\pgfpathlineto{\pgfqpoint{5.467808in}{4.361890in}}%
\pgfpathlineto{\pgfqpoint{5.385741in}{4.203046in}}%
\pgfpathclose%
\pgfusepath{fill}%
\end{pgfscope}%
\begin{pgfscope}%
\pgfpathrectangle{\pgfqpoint{0.680860in}{0.078740in}}{\pgfqpoint{7.842520in}{7.842520in}}%
\pgfusepath{clip}%
\pgfsetbuttcap%
\pgfsetroundjoin%
\definecolor{currentfill}{rgb}{0.804182,0.882046,0.114965}%
\pgfsetfillcolor{currentfill}%
\pgfsetlinewidth{0.000000pt}%
\definecolor{currentstroke}{rgb}{0.866013,0.889868,0.095953}%
\pgfsetstrokecolor{currentstroke}%
\pgfsetdash{}{0pt}%
\pgfpathmoveto{\pgfqpoint{7.055852in}{5.079587in}}%
\pgfpathlineto{\pgfqpoint{6.982775in}{5.120019in}}%
\pgfpathlineto{\pgfqpoint{7.134788in}{5.181011in}}%
\pgfpathclose%
\pgfusepath{fill}%
\end{pgfscope}%
\begin{pgfscope}%
\pgfpathrectangle{\pgfqpoint{0.680860in}{0.078740in}}{\pgfqpoint{7.842520in}{7.842520in}}%
\pgfusepath{clip}%
\pgfsetbuttcap%
\pgfsetroundjoin%
\definecolor{currentfill}{rgb}{0.575563,0.844566,0.256415}%
\pgfsetfillcolor{currentfill}%
\pgfsetlinewidth{0.000000pt}%
\definecolor{currentstroke}{rgb}{0.876168,0.891125,0.095250}%
\pgfsetstrokecolor{currentstroke}%
\pgfsetdash{}{0pt}%
\pgfpathmoveto{\pgfqpoint{5.924186in}{4.692622in}}%
\pgfpathlineto{\pgfqpoint{6.072535in}{4.741187in}}%
\pgfpathlineto{\pgfqpoint{6.152140in}{4.825555in}}%
\pgfpathclose%
\pgfusepath{fill}%
\end{pgfscope}%
\begin{pgfscope}%
\pgfpathrectangle{\pgfqpoint{0.680860in}{0.078740in}}{\pgfqpoint{7.842520in}{7.842520in}}%
\pgfusepath{clip}%
\pgfsetbuttcap%
\pgfsetroundjoin%
\definecolor{currentfill}{rgb}{0.647257,0.858400,0.209861}%
\pgfsetfillcolor{currentfill}%
\pgfsetlinewidth{0.000000pt}%
\definecolor{currentstroke}{rgb}{0.886271,0.892374,0.095374}%
\pgfsetstrokecolor{currentstroke}%
\pgfsetdash{}{0pt}%
\pgfpathmoveto{\pgfqpoint{6.379591in}{4.933197in}}%
\pgfpathlineto{\pgfqpoint{6.152140in}{4.825555in}}%
\pgfpathlineto{\pgfqpoint{6.301246in}{4.877794in}}%
\pgfpathclose%
\pgfusepath{fill}%
\end{pgfscope}%
\begin{pgfscope}%
\pgfpathrectangle{\pgfqpoint{0.680860in}{0.078740in}}{\pgfqpoint{7.842520in}{7.842520in}}%
\pgfusepath{clip}%
\pgfsetbuttcap%
\pgfsetroundjoin%
\definecolor{currentfill}{rgb}{0.121831,0.589055,0.545623}%
\pgfsetfillcolor{currentfill}%
\pgfsetlinewidth{0.000000pt}%
\definecolor{currentstroke}{rgb}{0.896320,0.893616,0.096335}%
\pgfsetstrokecolor{currentstroke}%
\pgfsetdash{}{0pt}%
\pgfpathmoveto{\pgfqpoint{4.991298in}{3.567692in}}%
\pgfpathlineto{\pgfqpoint{4.763121in}{3.314657in}}%
\pgfpathlineto{\pgfqpoint{4.907948in}{3.326514in}}%
\pgfpathclose%
\pgfusepath{fill}%
\end{pgfscope}%
\begin{pgfscope}%
\pgfpathrectangle{\pgfqpoint{0.680860in}{0.078740in}}{\pgfqpoint{7.842520in}{7.842520in}}%
\pgfusepath{clip}%
\pgfsetbuttcap%
\pgfsetroundjoin%
\definecolor{currentfill}{rgb}{0.252194,0.269783,0.531579}%
\pgfsetfillcolor{currentfill}%
\pgfsetlinewidth{0.000000pt}%
\definecolor{currentstroke}{rgb}{0.906311,0.894855,0.098125}%
\pgfsetstrokecolor{currentstroke}%
\pgfsetdash{}{0pt}%
\pgfpathmoveto{\pgfqpoint{4.429547in}{2.222456in}}%
\pgfpathlineto{\pgfqpoint{4.346201in}{1.921995in}}%
\pgfpathlineto{\pgfqpoint{4.490201in}{1.898629in}}%
\pgfpathclose%
\pgfusepath{fill}%
\end{pgfscope}%
\begin{pgfscope}%
\pgfpathrectangle{\pgfqpoint{0.680860in}{0.078740in}}{\pgfqpoint{7.842520in}{7.842520in}}%
\pgfusepath{clip}%
\pgfsetbuttcap%
\pgfsetroundjoin%
\definecolor{currentfill}{rgb}{0.160665,0.478540,0.558115}%
\pgfsetfillcolor{currentfill}%
\pgfsetlinewidth{0.000000pt}%
\definecolor{currentstroke}{rgb}{0.916242,0.896091,0.100717}%
\pgfsetstrokecolor{currentstroke}%
\pgfsetdash{}{0pt}%
\pgfpathmoveto{\pgfqpoint{4.679774in}{3.062239in}}%
\pgfpathlineto{\pgfqpoint{4.596367in}{2.794721in}}%
\pgfpathlineto{\pgfqpoint{4.740893in}{2.793599in}}%
\pgfpathclose%
\pgfusepath{fill}%
\end{pgfscope}%
\begin{pgfscope}%
\pgfpathrectangle{\pgfqpoint{0.680860in}{0.078740in}}{\pgfqpoint{7.842520in}{7.842520in}}%
\pgfusepath{clip}%
\pgfsetbuttcap%
\pgfsetroundjoin%
\definecolor{currentfill}{rgb}{0.281924,0.089666,0.412415}%
\pgfsetfillcolor{currentfill}%
\pgfsetlinewidth{0.000000pt}%
\definecolor{currentstroke}{rgb}{0.926106,0.897330,0.104071}%
\pgfsetstrokecolor{currentstroke}%
\pgfsetdash{}{0pt}%
\pgfpathmoveto{\pgfqpoint{4.406749in}{1.583748in}}%
\pgfpathlineto{\pgfqpoint{4.179721in}{1.303599in}}%
\pgfpathlineto{\pgfqpoint{4.323382in}{1.264477in}}%
\pgfpathclose%
\pgfusepath{fill}%
\end{pgfscope}%
\begin{pgfscope}%
\pgfpathrectangle{\pgfqpoint{0.680860in}{0.078740in}}{\pgfqpoint{7.842520in}{7.842520in}}%
\pgfusepath{clip}%
\pgfsetbuttcap%
\pgfsetroundjoin%
\definecolor{currentfill}{rgb}{0.352360,0.783011,0.392636}%
\pgfsetfillcolor{currentfill}%
\pgfsetlinewidth{0.000000pt}%
\definecolor{currentstroke}{rgb}{0.935904,0.898570,0.108131}%
\pgfsetstrokecolor{currentstroke}%
\pgfsetdash{}{0pt}%
\pgfpathmoveto{\pgfqpoint{5.532480in}{4.238865in}}%
\pgfpathlineto{\pgfqpoint{5.614548in}{4.400859in}}%
\pgfpathlineto{\pgfqpoint{5.467808in}{4.361890in}}%
\pgfpathclose%
\pgfusepath{fill}%
\end{pgfscope}%
\begin{pgfscope}%
\pgfpathrectangle{\pgfqpoint{0.680860in}{0.078740in}}{\pgfqpoint{7.842520in}{7.842520in}}%
\pgfusepath{clip}%
\pgfsetbuttcap%
\pgfsetroundjoin%
\definecolor{currentfill}{rgb}{0.185556,0.418570,0.556753}%
\pgfsetfillcolor{currentfill}%
\pgfsetlinewidth{0.000000pt}%
\definecolor{currentstroke}{rgb}{0.945636,0.899815,0.112838}%
\pgfsetstrokecolor{currentstroke}%
\pgfsetdash{}{0pt}%
\pgfpathmoveto{\pgfqpoint{4.596367in}{2.794721in}}%
\pgfpathlineto{\pgfqpoint{4.512945in}{2.514093in}}%
\pgfpathlineto{\pgfqpoint{4.657301in}{2.505849in}}%
\pgfpathclose%
\pgfusepath{fill}%
\end{pgfscope}%
\begin{pgfscope}%
\pgfpathrectangle{\pgfqpoint{0.680860in}{0.078740in}}{\pgfqpoint{7.842520in}{7.842520in}}%
\pgfusepath{clip}%
\pgfsetbuttcap%
\pgfsetroundjoin%
\definecolor{currentfill}{rgb}{0.206756,0.371758,0.553117}%
\pgfsetfillcolor{currentfill}%
\pgfsetlinewidth{0.000000pt}%
\definecolor{currentstroke}{rgb}{0.955300,0.901065,0.118128}%
\pgfsetstrokecolor{currentstroke}%
\pgfsetdash{}{0pt}%
\pgfpathmoveto{\pgfqpoint{4.657301in}{2.505849in}}%
\pgfpathlineto{\pgfqpoint{4.512945in}{2.514093in}}%
\pgfpathlineto{\pgfqpoint{4.429547in}{2.222456in}}%
\pgfpathclose%
\pgfusepath{fill}%
\end{pgfscope}%
\begin{pgfscope}%
\pgfpathrectangle{\pgfqpoint{0.680860in}{0.078740in}}{\pgfqpoint{7.842520in}{7.842520in}}%
\pgfusepath{clip}%
\pgfsetbuttcap%
\pgfsetroundjoin%
\definecolor{currentfill}{rgb}{0.269308,0.218818,0.509577}%
\pgfsetfillcolor{currentfill}%
\pgfsetlinewidth{0.000000pt}%
\definecolor{currentstroke}{rgb}{0.964894,0.902323,0.123941}%
\pgfsetstrokecolor{currentstroke}%
\pgfsetdash{}{0pt}%
\pgfpathmoveto{\pgfqpoint{4.346201in}{1.921995in}}%
\pgfpathlineto{\pgfqpoint{4.406749in}{1.583748in}}%
\pgfpathlineto{\pgfqpoint{4.490201in}{1.898629in}}%
\pgfpathclose%
\pgfusepath{fill}%
\end{pgfscope}%
\begin{pgfscope}%
\pgfpathrectangle{\pgfqpoint{0.680860in}{0.078740in}}{\pgfqpoint{7.842520in}{7.842520in}}%
\pgfusepath{clip}%
\pgfsetbuttcap%
\pgfsetroundjoin%
\definecolor{currentfill}{rgb}{0.137770,0.537492,0.554906}%
\pgfsetfillcolor{currentfill}%
\pgfsetlinewidth{0.000000pt}%
\definecolor{currentstroke}{rgb}{0.974417,0.903590,0.130215}%
\pgfsetstrokecolor{currentstroke}%
\pgfsetdash{}{0pt}%
\pgfpathmoveto{\pgfqpoint{4.679774in}{3.062239in}}%
\pgfpathlineto{\pgfqpoint{4.824460in}{3.067843in}}%
\pgfpathlineto{\pgfqpoint{4.907948in}{3.326514in}}%
\pgfpathclose%
\pgfusepath{fill}%
\end{pgfscope}%
\begin{pgfscope}%
\pgfpathrectangle{\pgfqpoint{0.680860in}{0.078740in}}{\pgfqpoint{7.842520in}{7.842520in}}%
\pgfusepath{clip}%
\pgfsetbuttcap%
\pgfsetroundjoin%
\definecolor{currentfill}{rgb}{0.137339,0.662252,0.515571}%
\pgfsetfillcolor{currentfill}%
\pgfsetlinewidth{0.000000pt}%
\definecolor{currentstroke}{rgb}{0.983868,0.904867,0.136897}%
\pgfsetstrokecolor{currentstroke}%
\pgfsetdash{}{0pt}%
\pgfpathmoveto{\pgfqpoint{5.220249in}{3.814471in}}%
\pgfpathlineto{\pgfqpoint{5.074444in}{3.789628in}}%
\pgfpathlineto{\pgfqpoint{4.991298in}{3.567692in}}%
\pgfpathclose%
\pgfusepath{fill}%
\end{pgfscope}%
\begin{pgfscope}%
\pgfpathrectangle{\pgfqpoint{0.680860in}{0.078740in}}{\pgfqpoint{7.842520in}{7.842520in}}%
\pgfusepath{clip}%
\pgfsetbuttcap%
\pgfsetroundjoin%
\definecolor{currentfill}{rgb}{0.151918,0.500685,0.557587}%
\pgfsetfillcolor{currentfill}%
\pgfsetlinewidth{0.000000pt}%
\definecolor{currentstroke}{rgb}{0.993248,0.906157,0.143936}%
\pgfsetstrokecolor{currentstroke}%
\pgfsetdash{}{0pt}%
\pgfpathmoveto{\pgfqpoint{4.740893in}{2.793599in}}%
\pgfpathlineto{\pgfqpoint{4.824460in}{3.067843in}}%
\pgfpathlineto{\pgfqpoint{4.679774in}{3.062239in}}%
\pgfpathclose%
\pgfusepath{fill}%
\end{pgfscope}%
\begin{pgfscope}%
\pgfpathrectangle{\pgfqpoint{0.680860in}{0.078740in}}{\pgfqpoint{7.842520in}{7.842520in}}%
\pgfusepath{clip}%
\pgfsetbuttcap%
\pgfsetroundjoin%
\definecolor{currentfill}{rgb}{0.421908,0.805774,0.351910}%
\pgfsetfillcolor{currentfill}%
\pgfsetlinewidth{0.000000pt}%
\definecolor{currentstroke}{rgb}{0.267004,0.004874,0.329415}%
\pgfsetstrokecolor{currentstroke}%
\pgfsetdash{}{0pt}%
\pgfpathmoveto{\pgfqpoint{5.695996in}{4.537126in}}%
\pgfpathlineto{\pgfqpoint{5.614548in}{4.400859in}}%
\pgfpathlineto{\pgfqpoint{5.762147in}{4.442595in}}%
\pgfpathclose%
\pgfusepath{fill}%
\end{pgfscope}%
\begin{pgfscope}%
\pgfpathrectangle{\pgfqpoint{0.680860in}{0.078740in}}{\pgfqpoint{7.842520in}{7.842520in}}%
\pgfusepath{clip}%
\pgfsetbuttcap%
\pgfsetroundjoin%
\definecolor{currentfill}{rgb}{0.239346,0.300855,0.540844}%
\pgfsetfillcolor{currentfill}%
\pgfsetlinewidth{0.000000pt}%
\definecolor{currentstroke}{rgb}{0.268510,0.009605,0.335427}%
\pgfsetstrokecolor{currentstroke}%
\pgfsetdash{}{0pt}%
\pgfpathmoveto{\pgfqpoint{4.490201in}{1.898629in}}%
\pgfpathlineto{\pgfqpoint{4.573726in}{2.206770in}}%
\pgfpathlineto{\pgfqpoint{4.429547in}{2.222456in}}%
\pgfpathclose%
\pgfusepath{fill}%
\end{pgfscope}%
\begin{pgfscope}%
\pgfpathrectangle{\pgfqpoint{0.680860in}{0.078740in}}{\pgfqpoint{7.842520in}{7.842520in}}%
\pgfusepath{clip}%
\pgfsetbuttcap%
\pgfsetroundjoin%
\definecolor{currentfill}{rgb}{0.170948,0.694384,0.493803}%
\pgfsetfillcolor{currentfill}%
\pgfsetlinewidth{0.000000pt}%
\definecolor{currentstroke}{rgb}{0.269944,0.014625,0.341379}%
\pgfsetstrokecolor{currentstroke}%
\pgfsetdash{}{0pt}%
\pgfpathmoveto{\pgfqpoint{5.074444in}{3.789628in}}%
\pgfpathlineto{\pgfqpoint{5.220249in}{3.814471in}}%
\pgfpathlineto{\pgfqpoint{5.303193in}{4.020195in}}%
\pgfpathclose%
\pgfusepath{fill}%
\end{pgfscope}%
\begin{pgfscope}%
\pgfpathrectangle{\pgfqpoint{0.680860in}{0.078740in}}{\pgfqpoint{7.842520in}{7.842520in}}%
\pgfusepath{clip}%
\pgfsetbuttcap%
\pgfsetroundjoin%
\definecolor{currentfill}{rgb}{0.216210,0.351535,0.550627}%
\pgfsetfillcolor{currentfill}%
\pgfsetlinewidth{0.000000pt}%
\definecolor{currentstroke}{rgb}{0.271305,0.019942,0.347269}%
\pgfsetstrokecolor{currentstroke}%
\pgfsetdash{}{0pt}%
\pgfpathmoveto{\pgfqpoint{4.429547in}{2.222456in}}%
\pgfpathlineto{\pgfqpoint{4.573726in}{2.206770in}}%
\pgfpathlineto{\pgfqpoint{4.657301in}{2.505849in}}%
\pgfpathclose%
\pgfusepath{fill}%
\end{pgfscope}%
\begin{pgfscope}%
\pgfpathrectangle{\pgfqpoint{0.680860in}{0.078740in}}{\pgfqpoint{7.842520in}{7.842520in}}%
\pgfusepath{clip}%
\pgfsetbuttcap%
\pgfsetroundjoin%
\definecolor{currentfill}{rgb}{0.175841,0.441290,0.557685}%
\pgfsetfillcolor{currentfill}%
\pgfsetlinewidth{0.000000pt}%
\definecolor{currentstroke}{rgb}{0.272594,0.025563,0.353093}%
\pgfsetstrokecolor{currentstroke}%
\pgfsetdash{}{0pt}%
\pgfpathmoveto{\pgfqpoint{4.657301in}{2.505849in}}%
\pgfpathlineto{\pgfqpoint{4.740893in}{2.793599in}}%
\pgfpathlineto{\pgfqpoint{4.596367in}{2.794721in}}%
\pgfpathclose%
\pgfusepath{fill}%
\end{pgfscope}%
\begin{pgfscope}%
\pgfpathrectangle{\pgfqpoint{0.680860in}{0.078740in}}{\pgfqpoint{7.842520in}{7.842520in}}%
\pgfusepath{clip}%
\pgfsetbuttcap%
\pgfsetroundjoin%
\definecolor{currentfill}{rgb}{0.458674,0.816363,0.329727}%
\pgfsetfillcolor{currentfill}%
\pgfsetlinewidth{0.000000pt}%
\definecolor{currentstroke}{rgb}{0.273809,0.031497,0.358853}%
\pgfsetstrokecolor{currentstroke}%
\pgfsetdash{}{0pt}%
\pgfpathmoveto{\pgfqpoint{5.695996in}{4.537126in}}%
\pgfpathlineto{\pgfqpoint{5.762147in}{4.442595in}}%
\pgfpathlineto{\pgfqpoint{5.843552in}{4.581232in}}%
\pgfpathclose%
\pgfusepath{fill}%
\end{pgfscope}%
\begin{pgfscope}%
\pgfpathrectangle{\pgfqpoint{0.680860in}{0.078740in}}{\pgfqpoint{7.842520in}{7.842520in}}%
\pgfusepath{clip}%
\pgfsetbuttcap%
\pgfsetroundjoin%
\definecolor{currentfill}{rgb}{0.282327,0.094955,0.417331}%
\pgfsetfillcolor{currentfill}%
\pgfsetlinewidth{0.000000pt}%
\definecolor{currentstroke}{rgb}{0.274952,0.037752,0.364543}%
\pgfsetstrokecolor{currentstroke}%
\pgfsetdash{}{0pt}%
\pgfpathmoveto{\pgfqpoint{4.323382in}{1.264477in}}%
\pgfpathlineto{\pgfqpoint{4.467562in}{1.225448in}}%
\pgfpathlineto{\pgfqpoint{4.406749in}{1.583748in}}%
\pgfpathclose%
\pgfusepath{fill}%
\end{pgfscope}%
\begin{pgfscope}%
\pgfpathrectangle{\pgfqpoint{0.680860in}{0.078740in}}{\pgfqpoint{7.842520in}{7.842520in}}%
\pgfusepath{clip}%
\pgfsetbuttcap%
\pgfsetroundjoin%
\definecolor{currentfill}{rgb}{0.252899,0.742211,0.448284}%
\pgfsetfillcolor{currentfill}%
\pgfsetlinewidth{0.000000pt}%
\definecolor{currentstroke}{rgb}{0.276022,0.044167,0.370164}%
\pgfsetstrokecolor{currentstroke}%
\pgfsetdash{}{0pt}%
\pgfpathmoveto{\pgfqpoint{5.385741in}{4.203046in}}%
\pgfpathlineto{\pgfqpoint{5.303193in}{4.020195in}}%
\pgfpathlineto{\pgfqpoint{5.449881in}{4.052044in}}%
\pgfpathclose%
\pgfusepath{fill}%
\end{pgfscope}%
\begin{pgfscope}%
\pgfpathrectangle{\pgfqpoint{0.680860in}{0.078740in}}{\pgfqpoint{7.842520in}{7.842520in}}%
\pgfusepath{clip}%
\pgfsetbuttcap%
\pgfsetroundjoin%
\definecolor{currentfill}{rgb}{0.525776,0.833491,0.288127}%
\pgfsetfillcolor{currentfill}%
\pgfsetlinewidth{0.000000pt}%
\definecolor{currentstroke}{rgb}{0.277018,0.050344,0.375715}%
\pgfsetstrokecolor{currentstroke}%
\pgfsetdash{}{0pt}%
\pgfpathmoveto{\pgfqpoint{5.992001in}{4.628294in}}%
\pgfpathlineto{\pgfqpoint{5.924186in}{4.692622in}}%
\pgfpathlineto{\pgfqpoint{5.843552in}{4.581232in}}%
\pgfpathclose%
\pgfusepath{fill}%
\end{pgfscope}%
\begin{pgfscope}%
\pgfpathrectangle{\pgfqpoint{0.680860in}{0.078740in}}{\pgfqpoint{7.842520in}{7.842520in}}%
\pgfusepath{clip}%
\pgfsetbuttcap%
\pgfsetroundjoin%
\definecolor{currentfill}{rgb}{0.772852,0.877868,0.131109}%
\pgfsetfillcolor{currentfill}%
\pgfsetlinewidth{0.000000pt}%
\definecolor{currentstroke}{rgb}{0.277941,0.056324,0.381191}%
\pgfsetstrokecolor{currentstroke}%
\pgfsetdash{}{0pt}%
\pgfpathmoveto{\pgfqpoint{6.756690in}{5.069787in}}%
\pgfpathlineto{\pgfqpoint{6.606243in}{5.012964in}}%
\pgfpathlineto{\pgfqpoint{6.680192in}{5.046603in}}%
\pgfpathclose%
\pgfusepath{fill}%
\end{pgfscope}%
\begin{pgfscope}%
\pgfpathrectangle{\pgfqpoint{0.680860in}{0.078740in}}{\pgfqpoint{7.842520in}{7.842520in}}%
\pgfusepath{clip}%
\pgfsetbuttcap%
\pgfsetroundjoin%
\definecolor{currentfill}{rgb}{0.565498,0.842430,0.262877}%
\pgfsetfillcolor{currentfill}%
\pgfsetlinewidth{0.000000pt}%
\definecolor{currentstroke}{rgb}{0.278791,0.062145,0.386592}%
\pgfsetstrokecolor{currentstroke}%
\pgfsetdash{}{0pt}%
\pgfpathmoveto{\pgfqpoint{5.992001in}{4.628294in}}%
\pgfpathlineto{\pgfqpoint{6.072535in}{4.741187in}}%
\pgfpathlineto{\pgfqpoint{5.924186in}{4.692622in}}%
\pgfpathclose%
\pgfusepath{fill}%
\end{pgfscope}%
\begin{pgfscope}%
\pgfpathrectangle{\pgfqpoint{0.680860in}{0.078740in}}{\pgfqpoint{7.842520in}{7.842520in}}%
\pgfusepath{clip}%
\pgfsetbuttcap%
\pgfsetroundjoin%
\definecolor{currentfill}{rgb}{0.120092,0.600104,0.542530}%
\pgfsetfillcolor{currentfill}%
\pgfsetlinewidth{0.000000pt}%
\definecolor{currentstroke}{rgb}{0.279566,0.067836,0.391917}%
\pgfsetstrokecolor{currentstroke}%
\pgfsetdash{}{0pt}%
\pgfpathmoveto{\pgfqpoint{4.907948in}{3.326514in}}%
\pgfpathlineto{\pgfqpoint{5.053499in}{3.340173in}}%
\pgfpathlineto{\pgfqpoint{4.991298in}{3.567692in}}%
\pgfpathclose%
\pgfusepath{fill}%
\end{pgfscope}%
\begin{pgfscope}%
\pgfpathrectangle{\pgfqpoint{0.680860in}{0.078740in}}{\pgfqpoint{7.842520in}{7.842520in}}%
\pgfusepath{clip}%
\pgfsetbuttcap%
\pgfsetroundjoin%
\definecolor{currentfill}{rgb}{0.762373,0.876424,0.137064}%
\pgfsetfillcolor{currentfill}%
\pgfsetlinewidth{0.000000pt}%
\definecolor{currentstroke}{rgb}{0.280267,0.073417,0.397163}%
\pgfsetstrokecolor{currentstroke}%
\pgfsetdash{}{0pt}%
\pgfpathmoveto{\pgfqpoint{6.680192in}{5.046603in}}%
\pgfpathlineto{\pgfqpoint{6.606243in}{5.012964in}}%
\pgfpathlineto{\pgfqpoint{6.529402in}{4.988221in}}%
\pgfpathclose%
\pgfusepath{fill}%
\end{pgfscope}%
\begin{pgfscope}%
\pgfpathrectangle{\pgfqpoint{0.680860in}{0.078740in}}{\pgfqpoint{7.842520in}{7.842520in}}%
\pgfusepath{clip}%
\pgfsetbuttcap%
\pgfsetroundjoin%
\definecolor{currentfill}{rgb}{0.288921,0.758394,0.428426}%
\pgfsetfillcolor{currentfill}%
\pgfsetlinewidth{0.000000pt}%
\definecolor{currentstroke}{rgb}{0.280894,0.078907,0.402329}%
\pgfsetstrokecolor{currentstroke}%
\pgfsetdash{}{0pt}%
\pgfpathmoveto{\pgfqpoint{5.449881in}{4.052044in}}%
\pgfpathlineto{\pgfqpoint{5.532480in}{4.238865in}}%
\pgfpathlineto{\pgfqpoint{5.385741in}{4.203046in}}%
\pgfpathclose%
\pgfusepath{fill}%
\end{pgfscope}%
\begin{pgfscope}%
\pgfpathrectangle{\pgfqpoint{0.680860in}{0.078740in}}{\pgfqpoint{7.842520in}{7.842520in}}%
\pgfusepath{clip}%
\pgfsetbuttcap%
\pgfsetroundjoin%
\definecolor{currentfill}{rgb}{0.132268,0.655014,0.519661}%
\pgfsetfillcolor{currentfill}%
\pgfsetlinewidth{0.000000pt}%
\definecolor{currentstroke}{rgb}{0.281446,0.084320,0.407414}%
\pgfsetstrokecolor{currentstroke}%
\pgfsetdash{}{0pt}%
\pgfpathmoveto{\pgfqpoint{5.136992in}{3.587259in}}%
\pgfpathlineto{\pgfqpoint{5.220249in}{3.814471in}}%
\pgfpathlineto{\pgfqpoint{4.991298in}{3.567692in}}%
\pgfpathclose%
\pgfusepath{fill}%
\end{pgfscope}%
\begin{pgfscope}%
\pgfpathrectangle{\pgfqpoint{0.680860in}{0.078740in}}{\pgfqpoint{7.842520in}{7.842520in}}%
\pgfusepath{clip}%
\pgfsetbuttcap%
\pgfsetroundjoin%
\definecolor{currentfill}{rgb}{0.267968,0.223549,0.512008}%
\pgfsetfillcolor{currentfill}%
\pgfsetlinewidth{0.000000pt}%
\definecolor{currentstroke}{rgb}{0.281924,0.089666,0.412415}%
\pgfsetstrokecolor{currentstroke}%
\pgfsetdash{}{0pt}%
\pgfpathmoveto{\pgfqpoint{4.490201in}{1.898629in}}%
\pgfpathlineto{\pgfqpoint{4.406749in}{1.583748in}}%
\pgfpathlineto{\pgfqpoint{4.634782in}{1.875870in}}%
\pgfpathclose%
\pgfusepath{fill}%
\end{pgfscope}%
\begin{pgfscope}%
\pgfpathrectangle{\pgfqpoint{0.680860in}{0.078740in}}{\pgfqpoint{7.842520in}{7.842520in}}%
\pgfusepath{clip}%
\pgfsetbuttcap%
\pgfsetroundjoin%
\definecolor{currentfill}{rgb}{0.626579,0.854645,0.223353}%
\pgfsetfillcolor{currentfill}%
\pgfsetlinewidth{0.000000pt}%
\definecolor{currentstroke}{rgb}{0.282327,0.094955,0.417331}%
\pgfsetstrokecolor{currentstroke}%
\pgfsetdash{}{0pt}%
\pgfpathmoveto{\pgfqpoint{6.072535in}{4.741187in}}%
\pgfpathlineto{\pgfqpoint{6.221809in}{4.792873in}}%
\pgfpathlineto{\pgfqpoint{6.152140in}{4.825555in}}%
\pgfpathclose%
\pgfusepath{fill}%
\end{pgfscope}%
\begin{pgfscope}%
\pgfpathrectangle{\pgfqpoint{0.680860in}{0.078740in}}{\pgfqpoint{7.842520in}{7.842520in}}%
\pgfusepath{clip}%
\pgfsetbuttcap%
\pgfsetroundjoin%
\definecolor{currentfill}{rgb}{0.720391,0.870350,0.162603}%
\pgfsetfillcolor{currentfill}%
\pgfsetlinewidth{0.000000pt}%
\definecolor{currentstroke}{rgb}{0.282656,0.100196,0.422160}%
\pgfsetstrokecolor{currentstroke}%
\pgfsetdash{}{0pt}%
\pgfpathmoveto{\pgfqpoint{6.529402in}{4.988221in}}%
\pgfpathlineto{\pgfqpoint{6.379591in}{4.933197in}}%
\pgfpathlineto{\pgfqpoint{6.451307in}{4.933289in}}%
\pgfpathclose%
\pgfusepath{fill}%
\end{pgfscope}%
\begin{pgfscope}%
\pgfpathrectangle{\pgfqpoint{0.680860in}{0.078740in}}{\pgfqpoint{7.842520in}{7.842520in}}%
\pgfusepath{clip}%
\pgfsetbuttcap%
\pgfsetroundjoin%
\definecolor{currentfill}{rgb}{0.647257,0.858400,0.209861}%
\pgfsetfillcolor{currentfill}%
\pgfsetlinewidth{0.000000pt}%
\definecolor{currentstroke}{rgb}{0.282910,0.105393,0.426902}%
\pgfsetstrokecolor{currentstroke}%
\pgfsetdash{}{0pt}%
\pgfpathmoveto{\pgfqpoint{6.152140in}{4.825555in}}%
\pgfpathlineto{\pgfqpoint{6.221809in}{4.792873in}}%
\pgfpathlineto{\pgfqpoint{6.301246in}{4.877794in}}%
\pgfpathclose%
\pgfusepath{fill}%
\end{pgfscope}%
\begin{pgfscope}%
\pgfpathrectangle{\pgfqpoint{0.680860in}{0.078740in}}{\pgfqpoint{7.842520in}{7.842520in}}%
\pgfusepath{clip}%
\pgfsetbuttcap%
\pgfsetroundjoin%
\definecolor{currentfill}{rgb}{0.699415,0.867117,0.175971}%
\pgfsetfillcolor{currentfill}%
\pgfsetlinewidth{0.000000pt}%
\definecolor{currentstroke}{rgb}{0.283091,0.110553,0.431554}%
\pgfsetstrokecolor{currentstroke}%
\pgfsetdash{}{0pt}%
\pgfpathmoveto{\pgfqpoint{6.451307in}{4.933289in}}%
\pgfpathlineto{\pgfqpoint{6.379591in}{4.933197in}}%
\pgfpathlineto{\pgfqpoint{6.301246in}{4.877794in}}%
\pgfpathclose%
\pgfusepath{fill}%
\end{pgfscope}%
\begin{pgfscope}%
\pgfpathrectangle{\pgfqpoint{0.680860in}{0.078740in}}{\pgfqpoint{7.842520in}{7.842520in}}%
\pgfusepath{clip}%
\pgfsetbuttcap%
\pgfsetroundjoin%
\definecolor{currentfill}{rgb}{0.824940,0.884720,0.106217}%
\pgfsetfillcolor{currentfill}%
\pgfsetlinewidth{0.000000pt}%
\definecolor{currentstroke}{rgb}{0.283197,0.115680,0.436115}%
\pgfsetstrokecolor{currentstroke}%
\pgfsetdash{}{0pt}%
\pgfpathmoveto{\pgfqpoint{6.756690in}{5.069787in}}%
\pgfpathlineto{\pgfqpoint{6.908138in}{5.130032in}}%
\pgfpathlineto{\pgfqpoint{6.982775in}{5.120019in}}%
\pgfpathclose%
\pgfusepath{fill}%
\end{pgfscope}%
\begin{pgfscope}%
\pgfpathrectangle{\pgfqpoint{0.680860in}{0.078740in}}{\pgfqpoint{7.842520in}{7.842520in}}%
\pgfusepath{clip}%
\pgfsetbuttcap%
\pgfsetroundjoin%
\definecolor{currentfill}{rgb}{0.127568,0.566949,0.550556}%
\pgfsetfillcolor{currentfill}%
\pgfsetlinewidth{0.000000pt}%
\definecolor{currentstroke}{rgb}{0.283229,0.120777,0.440584}%
\pgfsetstrokecolor{currentstroke}%
\pgfsetdash{}{0pt}%
\pgfpathmoveto{\pgfqpoint{4.824460in}{3.067843in}}%
\pgfpathlineto{\pgfqpoint{5.053499in}{3.340173in}}%
\pgfpathlineto{\pgfqpoint{4.907948in}{3.326514in}}%
\pgfpathclose%
\pgfusepath{fill}%
\end{pgfscope}%
\begin{pgfscope}%
\pgfpathrectangle{\pgfqpoint{0.680860in}{0.078740in}}{\pgfqpoint{7.842520in}{7.842520in}}%
\pgfusepath{clip}%
\pgfsetbuttcap%
\pgfsetroundjoin%
\definecolor{currentfill}{rgb}{0.386433,0.794644,0.372886}%
\pgfsetfillcolor{currentfill}%
\pgfsetlinewidth{0.000000pt}%
\definecolor{currentstroke}{rgb}{0.283187,0.125848,0.444960}%
\pgfsetstrokecolor{currentstroke}%
\pgfsetdash{}{0pt}%
\pgfpathmoveto{\pgfqpoint{5.762147in}{4.442595in}}%
\pgfpathlineto{\pgfqpoint{5.614548in}{4.400859in}}%
\pgfpathlineto{\pgfqpoint{5.532480in}{4.238865in}}%
\pgfpathclose%
\pgfusepath{fill}%
\end{pgfscope}%
\begin{pgfscope}%
\pgfpathrectangle{\pgfqpoint{0.680860in}{0.078740in}}{\pgfqpoint{7.842520in}{7.842520in}}%
\pgfusepath{clip}%
\pgfsetbuttcap%
\pgfsetroundjoin%
\definecolor{currentfill}{rgb}{0.283072,0.130895,0.449241}%
\pgfsetfillcolor{currentfill}%
\pgfsetlinewidth{0.000000pt}%
\definecolor{currentstroke}{rgb}{0.283072,0.130895,0.449241}%
\pgfsetstrokecolor{currentstroke}%
\pgfsetdash{}{0pt}%
\pgfpathmoveto{\pgfqpoint{4.467562in}{1.225448in}}%
\pgfpathlineto{\pgfqpoint{4.551125in}{1.552893in}}%
\pgfpathlineto{\pgfqpoint{4.406749in}{1.583748in}}%
\pgfpathclose%
\pgfusepath{fill}%
\end{pgfscope}%
\begin{pgfscope}%
\pgfpathrectangle{\pgfqpoint{0.680860in}{0.078740in}}{\pgfqpoint{7.842520in}{7.842520in}}%
\pgfusepath{clip}%
\pgfsetbuttcap%
\pgfsetroundjoin%
\definecolor{currentfill}{rgb}{0.248629,0.278775,0.534556}%
\pgfsetfillcolor{currentfill}%
\pgfsetlinewidth{0.000000pt}%
\definecolor{currentstroke}{rgb}{0.282884,0.135920,0.453427}%
\pgfsetstrokecolor{currentstroke}%
\pgfsetdash{}{0pt}%
\pgfpathmoveto{\pgfqpoint{4.634782in}{1.875870in}}%
\pgfpathlineto{\pgfqpoint{4.573726in}{2.206770in}}%
\pgfpathlineto{\pgfqpoint{4.490201in}{1.898629in}}%
\pgfpathclose%
\pgfusepath{fill}%
\end{pgfscope}%
\begin{pgfscope}%
\pgfpathrectangle{\pgfqpoint{0.680860in}{0.078740in}}{\pgfqpoint{7.842520in}{7.842520in}}%
\pgfusepath{clip}%
\pgfsetbuttcap%
\pgfsetroundjoin%
\definecolor{currentfill}{rgb}{0.214000,0.722114,0.469588}%
\pgfsetfillcolor{currentfill}%
\pgfsetlinewidth{0.000000pt}%
\definecolor{currentstroke}{rgb}{0.282623,0.140926,0.457517}%
\pgfsetstrokecolor{currentstroke}%
\pgfsetdash{}{0pt}%
\pgfpathmoveto{\pgfqpoint{5.303193in}{4.020195in}}%
\pgfpathlineto{\pgfqpoint{5.220249in}{3.814471in}}%
\pgfpathlineto{\pgfqpoint{5.449881in}{4.052044in}}%
\pgfpathclose%
\pgfusepath{fill}%
\end{pgfscope}%
\begin{pgfscope}%
\pgfpathrectangle{\pgfqpoint{0.680860in}{0.078740in}}{\pgfqpoint{7.842520in}{7.842520in}}%
\pgfusepath{clip}%
\pgfsetbuttcap%
\pgfsetroundjoin%
\definecolor{currentfill}{rgb}{0.120081,0.622161,0.534946}%
\pgfsetfillcolor{currentfill}%
\pgfsetlinewidth{0.000000pt}%
\definecolor{currentstroke}{rgb}{0.282290,0.145912,0.461510}%
\pgfsetstrokecolor{currentstroke}%
\pgfsetdash{}{0pt}%
\pgfpathmoveto{\pgfqpoint{4.991298in}{3.567692in}}%
\pgfpathlineto{\pgfqpoint{5.053499in}{3.340173in}}%
\pgfpathlineto{\pgfqpoint{5.136992in}{3.587259in}}%
\pgfpathclose%
\pgfusepath{fill}%
\end{pgfscope}%
\begin{pgfscope}%
\pgfpathrectangle{\pgfqpoint{0.680860in}{0.078740in}}{\pgfqpoint{7.842520in}{7.842520in}}%
\pgfusepath{clip}%
\pgfsetbuttcap%
\pgfsetroundjoin%
\definecolor{currentfill}{rgb}{0.212395,0.359683,0.551710}%
\pgfsetfillcolor{currentfill}%
\pgfsetlinewidth{0.000000pt}%
\definecolor{currentstroke}{rgb}{0.281887,0.150881,0.465405}%
\pgfsetstrokecolor{currentstroke}%
\pgfsetdash{}{0pt}%
\pgfpathmoveto{\pgfqpoint{4.657301in}{2.505849in}}%
\pgfpathlineto{\pgfqpoint{4.573726in}{2.206770in}}%
\pgfpathlineto{\pgfqpoint{4.718516in}{2.191946in}}%
\pgfpathclose%
\pgfusepath{fill}%
\end{pgfscope}%
\begin{pgfscope}%
\pgfpathrectangle{\pgfqpoint{0.680860in}{0.078740in}}{\pgfqpoint{7.842520in}{7.842520in}}%
\pgfusepath{clip}%
\pgfsetbuttcap%
\pgfsetroundjoin%
\definecolor{currentfill}{rgb}{0.156270,0.489624,0.557936}%
\pgfsetfillcolor{currentfill}%
\pgfsetlinewidth{0.000000pt}%
\definecolor{currentstroke}{rgb}{0.281412,0.155834,0.469201}%
\pgfsetstrokecolor{currentstroke}%
\pgfsetdash{}{0pt}%
\pgfpathmoveto{\pgfqpoint{4.824460in}{3.067843in}}%
\pgfpathlineto{\pgfqpoint{4.740893in}{2.793599in}}%
\pgfpathlineto{\pgfqpoint{4.886090in}{2.793832in}}%
\pgfpathclose%
\pgfusepath{fill}%
\end{pgfscope}%
\begin{pgfscope}%
\pgfpathrectangle{\pgfqpoint{0.680860in}{0.078740in}}{\pgfqpoint{7.842520in}{7.842520in}}%
\pgfusepath{clip}%
\pgfsetbuttcap%
\pgfsetroundjoin%
\definecolor{currentfill}{rgb}{0.275191,0.194905,0.496005}%
\pgfsetfillcolor{currentfill}%
\pgfsetlinewidth{0.000000pt}%
\definecolor{currentstroke}{rgb}{0.280868,0.160771,0.472899}%
\pgfsetstrokecolor{currentstroke}%
\pgfsetdash{}{0pt}%
\pgfpathmoveto{\pgfqpoint{4.634782in}{1.875870in}}%
\pgfpathlineto{\pgfqpoint{4.406749in}{1.583748in}}%
\pgfpathlineto{\pgfqpoint{4.551125in}{1.552893in}}%
\pgfpathclose%
\pgfusepath{fill}%
\end{pgfscope}%
\begin{pgfscope}%
\pgfpathrectangle{\pgfqpoint{0.680860in}{0.078740in}}{\pgfqpoint{7.842520in}{7.842520in}}%
\pgfusepath{clip}%
\pgfsetbuttcap%
\pgfsetroundjoin%
\definecolor{currentfill}{rgb}{0.180629,0.429975,0.557282}%
\pgfsetfillcolor{currentfill}%
\pgfsetlinewidth{0.000000pt}%
\definecolor{currentstroke}{rgb}{0.280255,0.165693,0.476498}%
\pgfsetstrokecolor{currentstroke}%
\pgfsetdash{}{0pt}%
\pgfpathmoveto{\pgfqpoint{4.657301in}{2.505849in}}%
\pgfpathlineto{\pgfqpoint{4.802298in}{2.498717in}}%
\pgfpathlineto{\pgfqpoint{4.740893in}{2.793599in}}%
\pgfpathclose%
\pgfusepath{fill}%
\end{pgfscope}%
\begin{pgfscope}%
\pgfpathrectangle{\pgfqpoint{0.680860in}{0.078740in}}{\pgfqpoint{7.842520in}{7.842520in}}%
\pgfusepath{clip}%
\pgfsetbuttcap%
\pgfsetroundjoin%
\definecolor{currentfill}{rgb}{0.855810,0.888601,0.097452}%
\pgfsetfillcolor{currentfill}%
\pgfsetlinewidth{0.000000pt}%
\definecolor{currentstroke}{rgb}{0.279574,0.170599,0.479997}%
\pgfsetstrokecolor{currentstroke}%
\pgfsetdash{}{0pt}%
\pgfpathmoveto{\pgfqpoint{7.134788in}{5.181011in}}%
\pgfpathlineto{\pgfqpoint{6.982775in}{5.120019in}}%
\pgfpathlineto{\pgfqpoint{6.908138in}{5.130032in}}%
\pgfpathclose%
\pgfusepath{fill}%
\end{pgfscope}%
\begin{pgfscope}%
\pgfpathrectangle{\pgfqpoint{0.680860in}{0.078740in}}{\pgfqpoint{7.842520in}{7.842520in}}%
\pgfusepath{clip}%
\pgfsetbuttcap%
\pgfsetroundjoin%
\definecolor{currentfill}{rgb}{0.235526,0.309527,0.542944}%
\pgfsetfillcolor{currentfill}%
\pgfsetlinewidth{0.000000pt}%
\definecolor{currentstroke}{rgb}{0.278826,0.175490,0.483397}%
\pgfsetstrokecolor{currentstroke}%
\pgfsetdash{}{0pt}%
\pgfpathmoveto{\pgfqpoint{4.634782in}{1.875870in}}%
\pgfpathlineto{\pgfqpoint{4.718516in}{2.191946in}}%
\pgfpathlineto{\pgfqpoint{4.573726in}{2.206770in}}%
\pgfpathclose%
\pgfusepath{fill}%
\end{pgfscope}%
\begin{pgfscope}%
\pgfpathrectangle{\pgfqpoint{0.680860in}{0.078740in}}{\pgfqpoint{7.842520in}{7.842520in}}%
\pgfusepath{clip}%
\pgfsetbuttcap%
\pgfsetroundjoin%
\definecolor{currentfill}{rgb}{0.131172,0.555899,0.552459}%
\pgfsetfillcolor{currentfill}%
\pgfsetlinewidth{0.000000pt}%
\definecolor{currentstroke}{rgb}{0.278012,0.180367,0.486697}%
\pgfsetstrokecolor{currentstroke}%
\pgfsetdash{}{0pt}%
\pgfpathmoveto{\pgfqpoint{4.824460in}{3.067843in}}%
\pgfpathlineto{\pgfqpoint{4.969843in}{3.075032in}}%
\pgfpathlineto{\pgfqpoint{5.053499in}{3.340173in}}%
\pgfpathclose%
\pgfusepath{fill}%
\end{pgfscope}%
\begin{pgfscope}%
\pgfpathrectangle{\pgfqpoint{0.680860in}{0.078740in}}{\pgfqpoint{7.842520in}{7.842520in}}%
\pgfusepath{clip}%
\pgfsetbuttcap%
\pgfsetroundjoin%
\definecolor{currentfill}{rgb}{0.199430,0.387607,0.554642}%
\pgfsetfillcolor{currentfill}%
\pgfsetlinewidth{0.000000pt}%
\definecolor{currentstroke}{rgb}{0.277134,0.185228,0.489898}%
\pgfsetstrokecolor{currentstroke}%
\pgfsetdash{}{0pt}%
\pgfpathmoveto{\pgfqpoint{4.718516in}{2.191946in}}%
\pgfpathlineto{\pgfqpoint{4.802298in}{2.498717in}}%
\pgfpathlineto{\pgfqpoint{4.657301in}{2.505849in}}%
\pgfpathclose%
\pgfusepath{fill}%
\end{pgfscope}%
\begin{pgfscope}%
\pgfpathrectangle{\pgfqpoint{0.680860in}{0.078740in}}{\pgfqpoint{7.842520in}{7.842520in}}%
\pgfusepath{clip}%
\pgfsetbuttcap%
\pgfsetroundjoin%
\definecolor{currentfill}{rgb}{0.146180,0.515413,0.556823}%
\pgfsetfillcolor{currentfill}%
\pgfsetlinewidth{0.000000pt}%
\definecolor{currentstroke}{rgb}{0.276194,0.190074,0.493001}%
\pgfsetstrokecolor{currentstroke}%
\pgfsetdash{}{0pt}%
\pgfpathmoveto{\pgfqpoint{4.886090in}{2.793832in}}%
\pgfpathlineto{\pgfqpoint{4.969843in}{3.075032in}}%
\pgfpathlineto{\pgfqpoint{4.824460in}{3.067843in}}%
\pgfpathclose%
\pgfusepath{fill}%
\end{pgfscope}%
\begin{pgfscope}%
\pgfpathrectangle{\pgfqpoint{0.680860in}{0.078740in}}{\pgfqpoint{7.842520in}{7.842520in}}%
\pgfusepath{clip}%
\pgfsetbuttcap%
\pgfsetroundjoin%
\definecolor{currentfill}{rgb}{0.282656,0.100196,0.422160}%
\pgfsetfillcolor{currentfill}%
\pgfsetlinewidth{0.000000pt}%
\definecolor{currentstroke}{rgb}{0.275191,0.194905,0.496005}%
\pgfsetstrokecolor{currentstroke}%
\pgfsetdash{}{0pt}%
\pgfpathmoveto{\pgfqpoint{4.551125in}{1.552893in}}%
\pgfpathlineto{\pgfqpoint{4.467562in}{1.225448in}}%
\pgfpathlineto{\pgfqpoint{4.612268in}{1.186519in}}%
\pgfpathclose%
\pgfusepath{fill}%
\end{pgfscope}%
\begin{pgfscope}%
\pgfpathrectangle{\pgfqpoint{0.680860in}{0.078740in}}{\pgfqpoint{7.842520in}{7.842520in}}%
\pgfusepath{clip}%
\pgfsetbuttcap%
\pgfsetroundjoin%
\definecolor{currentfill}{rgb}{0.169646,0.456262,0.558030}%
\pgfsetfillcolor{currentfill}%
\pgfsetlinewidth{0.000000pt}%
\definecolor{currentstroke}{rgb}{0.274128,0.199721,0.498911}%
\pgfsetstrokecolor{currentstroke}%
\pgfsetdash{}{0pt}%
\pgfpathmoveto{\pgfqpoint{4.740893in}{2.793599in}}%
\pgfpathlineto{\pgfqpoint{4.802298in}{2.498717in}}%
\pgfpathlineto{\pgfqpoint{4.886090in}{2.793832in}}%
\pgfpathclose%
\pgfusepath{fill}%
\end{pgfscope}%
\begin{pgfscope}%
\pgfpathrectangle{\pgfqpoint{0.680860in}{0.078740in}}{\pgfqpoint{7.842520in}{7.842520in}}%
\pgfusepath{clip}%
\pgfsetbuttcap%
\pgfsetroundjoin%
\definecolor{currentfill}{rgb}{0.202219,0.715272,0.476084}%
\pgfsetfillcolor{currentfill}%
\pgfsetlinewidth{0.000000pt}%
\definecolor{currentstroke}{rgb}{0.273006,0.204520,0.501721}%
\pgfsetstrokecolor{currentstroke}%
\pgfsetdash{}{0pt}%
\pgfpathmoveto{\pgfqpoint{5.449881in}{4.052044in}}%
\pgfpathlineto{\pgfqpoint{5.220249in}{3.814471in}}%
\pgfpathlineto{\pgfqpoint{5.366842in}{3.841576in}}%
\pgfpathclose%
\pgfusepath{fill}%
\end{pgfscope}%
\begin{pgfscope}%
\pgfpathrectangle{\pgfqpoint{0.680860in}{0.078740in}}{\pgfqpoint{7.842520in}{7.842520in}}%
\pgfusepath{clip}%
\pgfsetbuttcap%
\pgfsetroundjoin%
\definecolor{currentfill}{rgb}{0.824940,0.884720,0.106217}%
\pgfsetfillcolor{currentfill}%
\pgfsetlinewidth{0.000000pt}%
\definecolor{currentstroke}{rgb}{0.271828,0.209303,0.504434}%
\pgfsetstrokecolor{currentstroke}%
\pgfsetdash{}{0pt}%
\pgfpathmoveto{\pgfqpoint{6.680192in}{5.046603in}}%
\pgfpathlineto{\pgfqpoint{6.908138in}{5.130032in}}%
\pgfpathlineto{\pgfqpoint{6.756690in}{5.069787in}}%
\pgfpathclose%
\pgfusepath{fill}%
\end{pgfscope}%
\begin{pgfscope}%
\pgfpathrectangle{\pgfqpoint{0.680860in}{0.078740in}}{\pgfqpoint{7.842520in}{7.842520in}}%
\pgfusepath{clip}%
\pgfsetbuttcap%
\pgfsetroundjoin%
\definecolor{currentfill}{rgb}{0.140210,0.665859,0.513427}%
\pgfsetfillcolor{currentfill}%
\pgfsetlinewidth{0.000000pt}%
\definecolor{currentstroke}{rgb}{0.270595,0.214069,0.507052}%
\pgfsetstrokecolor{currentstroke}%
\pgfsetdash{}{0pt}%
\pgfpathmoveto{\pgfqpoint{5.283450in}{3.608903in}}%
\pgfpathlineto{\pgfqpoint{5.220249in}{3.814471in}}%
\pgfpathlineto{\pgfqpoint{5.136992in}{3.587259in}}%
\pgfpathclose%
\pgfusepath{fill}%
\end{pgfscope}%
\begin{pgfscope}%
\pgfpathrectangle{\pgfqpoint{0.680860in}{0.078740in}}{\pgfqpoint{7.842520in}{7.842520in}}%
\pgfusepath{clip}%
\pgfsetbuttcap%
\pgfsetroundjoin%
\definecolor{currentfill}{rgb}{0.386433,0.794644,0.372886}%
\pgfsetfillcolor{currentfill}%
\pgfsetlinewidth{0.000000pt}%
\definecolor{currentstroke}{rgb}{0.269308,0.218818,0.509577}%
\pgfsetstrokecolor{currentstroke}%
\pgfsetdash{}{0pt}%
\pgfpathmoveto{\pgfqpoint{5.532480in}{4.238865in}}%
\pgfpathlineto{\pgfqpoint{5.680064in}{4.277342in}}%
\pgfpathlineto{\pgfqpoint{5.762147in}{4.442595in}}%
\pgfpathclose%
\pgfusepath{fill}%
\end{pgfscope}%
\begin{pgfscope}%
\pgfpathrectangle{\pgfqpoint{0.680860in}{0.078740in}}{\pgfqpoint{7.842520in}{7.842520in}}%
\pgfusepath{clip}%
\pgfsetbuttcap%
\pgfsetroundjoin%
\definecolor{currentfill}{rgb}{0.487026,0.823929,0.312321}%
\pgfsetfillcolor{currentfill}%
\pgfsetlinewidth{0.000000pt}%
\definecolor{currentstroke}{rgb}{0.267968,0.223549,0.512008}%
\pgfsetstrokecolor{currentstroke}%
\pgfsetdash{}{0pt}%
\pgfpathmoveto{\pgfqpoint{5.843552in}{4.581232in}}%
\pgfpathlineto{\pgfqpoint{5.762147in}{4.442595in}}%
\pgfpathlineto{\pgfqpoint{5.910629in}{4.487206in}}%
\pgfpathclose%
\pgfusepath{fill}%
\end{pgfscope}%
\begin{pgfscope}%
\pgfpathrectangle{\pgfqpoint{0.680860in}{0.078740in}}{\pgfqpoint{7.842520in}{7.842520in}}%
\pgfusepath{clip}%
\pgfsetbuttcap%
\pgfsetroundjoin%
\definecolor{currentfill}{rgb}{0.319809,0.770914,0.411152}%
\pgfsetfillcolor{currentfill}%
\pgfsetlinewidth{0.000000pt}%
\definecolor{currentstroke}{rgb}{0.266580,0.228262,0.514349}%
\pgfsetstrokecolor{currentstroke}%
\pgfsetdash{}{0pt}%
\pgfpathmoveto{\pgfqpoint{5.680064in}{4.277342in}}%
\pgfpathlineto{\pgfqpoint{5.532480in}{4.238865in}}%
\pgfpathlineto{\pgfqpoint{5.449881in}{4.052044in}}%
\pgfpathclose%
\pgfusepath{fill}%
\end{pgfscope}%
\begin{pgfscope}%
\pgfpathrectangle{\pgfqpoint{0.680860in}{0.078740in}}{\pgfqpoint{7.842520in}{7.842520in}}%
\pgfusepath{clip}%
\pgfsetbuttcap%
\pgfsetroundjoin%
\definecolor{currentfill}{rgb}{0.274128,0.199721,0.498911}%
\pgfsetfillcolor{currentfill}%
\pgfsetlinewidth{0.000000pt}%
\definecolor{currentstroke}{rgb}{0.265145,0.232956,0.516599}%
\pgfsetstrokecolor{currentstroke}%
\pgfsetdash{}{0pt}%
\pgfpathmoveto{\pgfqpoint{4.634782in}{1.875870in}}%
\pgfpathlineto{\pgfqpoint{4.551125in}{1.552893in}}%
\pgfpathlineto{\pgfqpoint{4.696059in}{1.522403in}}%
\pgfpathclose%
\pgfusepath{fill}%
\end{pgfscope}%
\begin{pgfscope}%
\pgfpathrectangle{\pgfqpoint{0.680860in}{0.078740in}}{\pgfqpoint{7.842520in}{7.842520in}}%
\pgfusepath{clip}%
\pgfsetbuttcap%
\pgfsetroundjoin%
\definecolor{currentfill}{rgb}{0.772852,0.877868,0.131109}%
\pgfsetfillcolor{currentfill}%
\pgfsetlinewidth{0.000000pt}%
\definecolor{currentstroke}{rgb}{0.263663,0.237631,0.518762}%
\pgfsetstrokecolor{currentstroke}%
\pgfsetdash{}{0pt}%
\pgfpathmoveto{\pgfqpoint{6.529402in}{4.988221in}}%
\pgfpathlineto{\pgfqpoint{6.451307in}{4.933289in}}%
\pgfpathlineto{\pgfqpoint{6.680192in}{5.046603in}}%
\pgfpathclose%
\pgfusepath{fill}%
\end{pgfscope}%
\begin{pgfscope}%
\pgfpathrectangle{\pgfqpoint{0.680860in}{0.078740in}}{\pgfqpoint{7.842520in}{7.842520in}}%
\pgfusepath{clip}%
\pgfsetbuttcap%
\pgfsetroundjoin%
\definecolor{currentfill}{rgb}{0.525776,0.833491,0.288127}%
\pgfsetfillcolor{currentfill}%
\pgfsetlinewidth{0.000000pt}%
\definecolor{currentstroke}{rgb}{0.262138,0.242286,0.520837}%
\pgfsetstrokecolor{currentstroke}%
\pgfsetdash{}{0pt}%
\pgfpathmoveto{\pgfqpoint{5.910629in}{4.487206in}}%
\pgfpathlineto{\pgfqpoint{5.992001in}{4.628294in}}%
\pgfpathlineto{\pgfqpoint{5.843552in}{4.581232in}}%
\pgfpathclose%
\pgfusepath{fill}%
\end{pgfscope}%
\begin{pgfscope}%
\pgfpathrectangle{\pgfqpoint{0.680860in}{0.078740in}}{\pgfqpoint{7.842520in}{7.842520in}}%
\pgfusepath{clip}%
\pgfsetbuttcap%
\pgfsetroundjoin%
\definecolor{currentfill}{rgb}{0.699415,0.867117,0.175971}%
\pgfsetfillcolor{currentfill}%
\pgfsetlinewidth{0.000000pt}%
\definecolor{currentstroke}{rgb}{0.260571,0.246922,0.522828}%
\pgfsetstrokecolor{currentstroke}%
\pgfsetdash{}{0pt}%
\pgfpathmoveto{\pgfqpoint{6.301246in}{4.877794in}}%
\pgfpathlineto{\pgfqpoint{6.221809in}{4.792873in}}%
\pgfpathlineto{\pgfqpoint{6.451307in}{4.933289in}}%
\pgfpathclose%
\pgfusepath{fill}%
\end{pgfscope}%
\begin{pgfscope}%
\pgfpathrectangle{\pgfqpoint{0.680860in}{0.078740in}}{\pgfqpoint{7.842520in}{7.842520in}}%
\pgfusepath{clip}%
\pgfsetbuttcap%
\pgfsetroundjoin%
\definecolor{currentfill}{rgb}{0.282884,0.135920,0.453427}%
\pgfsetfillcolor{currentfill}%
\pgfsetlinewidth{0.000000pt}%
\definecolor{currentstroke}{rgb}{0.258965,0.251537,0.524736}%
\pgfsetstrokecolor{currentstroke}%
\pgfsetdash{}{0pt}%
\pgfpathmoveto{\pgfqpoint{4.551125in}{1.552893in}}%
\pgfpathlineto{\pgfqpoint{4.612268in}{1.186519in}}%
\pgfpathlineto{\pgfqpoint{4.696059in}{1.522403in}}%
\pgfpathclose%
\pgfusepath{fill}%
\end{pgfscope}%
\begin{pgfscope}%
\pgfpathrectangle{\pgfqpoint{0.680860in}{0.078740in}}{\pgfqpoint{7.842520in}{7.842520in}}%
\pgfusepath{clip}%
\pgfsetbuttcap%
\pgfsetroundjoin%
\definecolor{currentfill}{rgb}{0.123444,0.636809,0.528763}%
\pgfsetfillcolor{currentfill}%
\pgfsetlinewidth{0.000000pt}%
\definecolor{currentstroke}{rgb}{0.257322,0.256130,0.526563}%
\pgfsetstrokecolor{currentstroke}%
\pgfsetdash{}{0pt}%
\pgfpathmoveto{\pgfqpoint{5.136992in}{3.587259in}}%
\pgfpathlineto{\pgfqpoint{5.053499in}{3.340173in}}%
\pgfpathlineto{\pgfqpoint{5.283450in}{3.608903in}}%
\pgfpathclose%
\pgfusepath{fill}%
\end{pgfscope}%
\begin{pgfscope}%
\pgfpathrectangle{\pgfqpoint{0.680860in}{0.078740in}}{\pgfqpoint{7.842520in}{7.842520in}}%
\pgfusepath{clip}%
\pgfsetbuttcap%
\pgfsetroundjoin%
\definecolor{currentfill}{rgb}{0.162016,0.687316,0.499129}%
\pgfsetfillcolor{currentfill}%
\pgfsetlinewidth{0.000000pt}%
\definecolor{currentstroke}{rgb}{0.255645,0.260703,0.528312}%
\pgfsetstrokecolor{currentstroke}%
\pgfsetdash{}{0pt}%
\pgfpathmoveto{\pgfqpoint{5.366842in}{3.841576in}}%
\pgfpathlineto{\pgfqpoint{5.220249in}{3.814471in}}%
\pgfpathlineto{\pgfqpoint{5.283450in}{3.608903in}}%
\pgfpathclose%
\pgfusepath{fill}%
\end{pgfscope}%
\begin{pgfscope}%
\pgfpathrectangle{\pgfqpoint{0.680860in}{0.078740in}}{\pgfqpoint{7.842520in}{7.842520in}}%
\pgfusepath{clip}%
\pgfsetbuttcap%
\pgfsetroundjoin%
\definecolor{currentfill}{rgb}{0.231674,0.318106,0.544834}%
\pgfsetfillcolor{currentfill}%
\pgfsetlinewidth{0.000000pt}%
\definecolor{currentstroke}{rgb}{0.253935,0.265254,0.529983}%
\pgfsetstrokecolor{currentstroke}%
\pgfsetdash{}{0pt}%
\pgfpathmoveto{\pgfqpoint{4.863928in}{2.178019in}}%
\pgfpathlineto{\pgfqpoint{4.718516in}{2.191946in}}%
\pgfpathlineto{\pgfqpoint{4.634782in}{1.875870in}}%
\pgfpathclose%
\pgfusepath{fill}%
\end{pgfscope}%
\begin{pgfscope}%
\pgfpathrectangle{\pgfqpoint{0.680860in}{0.078740in}}{\pgfqpoint{7.842520in}{7.842520in}}%
\pgfusepath{clip}%
\pgfsetbuttcap%
\pgfsetroundjoin%
\definecolor{currentfill}{rgb}{0.595839,0.848717,0.243329}%
\pgfsetfillcolor{currentfill}%
\pgfsetlinewidth{0.000000pt}%
\definecolor{currentstroke}{rgb}{0.252194,0.269783,0.531579}%
\pgfsetstrokecolor{currentstroke}%
\pgfsetdash{}{0pt}%
\pgfpathmoveto{\pgfqpoint{6.072535in}{4.741187in}}%
\pgfpathlineto{\pgfqpoint{5.992001in}{4.628294in}}%
\pgfpathlineto{\pgfqpoint{6.141369in}{4.678428in}}%
\pgfpathclose%
\pgfusepath{fill}%
\end{pgfscope}%
\begin{pgfscope}%
\pgfpathrectangle{\pgfqpoint{0.680860in}{0.078740in}}{\pgfqpoint{7.842520in}{7.842520in}}%
\pgfusepath{clip}%
\pgfsetbuttcap%
\pgfsetroundjoin%
\definecolor{currentfill}{rgb}{0.123463,0.581687,0.547445}%
\pgfsetfillcolor{currentfill}%
\pgfsetlinewidth{0.000000pt}%
\definecolor{currentstroke}{rgb}{0.250425,0.274290,0.533103}%
\pgfsetstrokecolor{currentstroke}%
\pgfsetdash{}{0pt}%
\pgfpathmoveto{\pgfqpoint{4.969843in}{3.075032in}}%
\pgfpathlineto{\pgfqpoint{5.199790in}{3.355703in}}%
\pgfpathlineto{\pgfqpoint{5.053499in}{3.340173in}}%
\pgfpathclose%
\pgfusepath{fill}%
\end{pgfscope}%
\begin{pgfscope}%
\pgfpathrectangle{\pgfqpoint{0.680860in}{0.078740in}}{\pgfqpoint{7.842520in}{7.842520in}}%
\pgfusepath{clip}%
\pgfsetbuttcap%
\pgfsetroundjoin%
\definecolor{currentfill}{rgb}{0.265145,0.232956,0.516599}%
\pgfsetfillcolor{currentfill}%
\pgfsetlinewidth{0.000000pt}%
\definecolor{currentstroke}{rgb}{0.248629,0.278775,0.534556}%
\pgfsetstrokecolor{currentstroke}%
\pgfsetdash{}{0pt}%
\pgfpathmoveto{\pgfqpoint{4.696059in}{1.522403in}}%
\pgfpathlineto{\pgfqpoint{4.779954in}{1.853743in}}%
\pgfpathlineto{\pgfqpoint{4.634782in}{1.875870in}}%
\pgfpathclose%
\pgfusepath{fill}%
\end{pgfscope}%
\begin{pgfscope}%
\pgfpathrectangle{\pgfqpoint{0.680860in}{0.078740in}}{\pgfqpoint{7.842520in}{7.842520in}}%
\pgfusepath{clip}%
\pgfsetbuttcap%
\pgfsetroundjoin%
\definecolor{currentfill}{rgb}{0.636902,0.856542,0.216620}%
\pgfsetfillcolor{currentfill}%
\pgfsetlinewidth{0.000000pt}%
\definecolor{currentstroke}{rgb}{0.246811,0.283237,0.535941}%
\pgfsetstrokecolor{currentstroke}%
\pgfsetdash{}{0pt}%
\pgfpathmoveto{\pgfqpoint{6.141369in}{4.678428in}}%
\pgfpathlineto{\pgfqpoint{6.221809in}{4.792873in}}%
\pgfpathlineto{\pgfqpoint{6.072535in}{4.741187in}}%
\pgfpathclose%
\pgfusepath{fill}%
\end{pgfscope}%
\begin{pgfscope}%
\pgfpathrectangle{\pgfqpoint{0.680860in}{0.078740in}}{\pgfqpoint{7.842520in}{7.842520in}}%
\pgfusepath{clip}%
\pgfsetbuttcap%
\pgfsetroundjoin%
\definecolor{currentfill}{rgb}{0.195860,0.395433,0.555276}%
\pgfsetfillcolor{currentfill}%
\pgfsetlinewidth{0.000000pt}%
\definecolor{currentstroke}{rgb}{0.244972,0.287675,0.537260}%
\pgfsetstrokecolor{currentstroke}%
\pgfsetdash{}{0pt}%
\pgfpathmoveto{\pgfqpoint{4.802298in}{2.498717in}}%
\pgfpathlineto{\pgfqpoint{4.718516in}{2.191946in}}%
\pgfpathlineto{\pgfqpoint{4.947949in}{2.492741in}}%
\pgfpathclose%
\pgfusepath{fill}%
\end{pgfscope}%
\begin{pgfscope}%
\pgfpathrectangle{\pgfqpoint{0.680860in}{0.078740in}}{\pgfqpoint{7.842520in}{7.842520in}}%
\pgfusepath{clip}%
\pgfsetbuttcap%
\pgfsetroundjoin%
\definecolor{currentfill}{rgb}{0.175841,0.441290,0.557685}%
\pgfsetfillcolor{currentfill}%
\pgfsetlinewidth{0.000000pt}%
\definecolor{currentstroke}{rgb}{0.243113,0.292092,0.538516}%
\pgfsetstrokecolor{currentstroke}%
\pgfsetdash{}{0pt}%
\pgfpathmoveto{\pgfqpoint{4.886090in}{2.793832in}}%
\pgfpathlineto{\pgfqpoint{4.802298in}{2.498717in}}%
\pgfpathlineto{\pgfqpoint{4.947949in}{2.492741in}}%
\pgfpathclose%
\pgfusepath{fill}%
\end{pgfscope}%
\begin{pgfscope}%
\pgfpathrectangle{\pgfqpoint{0.680860in}{0.078740in}}{\pgfqpoint{7.842520in}{7.842520in}}%
\pgfusepath{clip}%
\pgfsetbuttcap%
\pgfsetroundjoin%
\definecolor{currentfill}{rgb}{0.150476,0.504369,0.557430}%
\pgfsetfillcolor{currentfill}%
\pgfsetlinewidth{0.000000pt}%
\definecolor{currentstroke}{rgb}{0.241237,0.296485,0.539709}%
\pgfsetstrokecolor{currentstroke}%
\pgfsetdash{}{0pt}%
\pgfpathmoveto{\pgfqpoint{4.886090in}{2.793832in}}%
\pgfpathlineto{\pgfqpoint{5.031971in}{2.795473in}}%
\pgfpathlineto{\pgfqpoint{4.969843in}{3.075032in}}%
\pgfpathclose%
\pgfusepath{fill}%
\end{pgfscope}%
\begin{pgfscope}%
\pgfpathrectangle{\pgfqpoint{0.680860in}{0.078740in}}{\pgfqpoint{7.842520in}{7.842520in}}%
\pgfusepath{clip}%
\pgfsetbuttcap%
\pgfsetroundjoin%
\definecolor{currentfill}{rgb}{0.243113,0.292092,0.538516}%
\pgfsetfillcolor{currentfill}%
\pgfsetlinewidth{0.000000pt}%
\definecolor{currentstroke}{rgb}{0.239346,0.300855,0.540844}%
\pgfsetstrokecolor{currentstroke}%
\pgfsetdash{}{0pt}%
\pgfpathmoveto{\pgfqpoint{4.634782in}{1.875870in}}%
\pgfpathlineto{\pgfqpoint{4.779954in}{1.853743in}}%
\pgfpathlineto{\pgfqpoint{4.863928in}{2.178019in}}%
\pgfpathclose%
\pgfusepath{fill}%
\end{pgfscope}%
\begin{pgfscope}%
\pgfpathrectangle{\pgfqpoint{0.680860in}{0.078740in}}{\pgfqpoint{7.842520in}{7.842520in}}%
\pgfusepath{clip}%
\pgfsetbuttcap%
\pgfsetroundjoin%
\definecolor{currentfill}{rgb}{0.440137,0.811138,0.340967}%
\pgfsetfillcolor{currentfill}%
\pgfsetlinewidth{0.000000pt}%
\definecolor{currentstroke}{rgb}{0.237441,0.305202,0.541921}%
\pgfsetstrokecolor{currentstroke}%
\pgfsetdash{}{0pt}%
\pgfpathmoveto{\pgfqpoint{5.762147in}{4.442595in}}%
\pgfpathlineto{\pgfqpoint{5.680064in}{4.277342in}}%
\pgfpathlineto{\pgfqpoint{5.910629in}{4.487206in}}%
\pgfpathclose%
\pgfusepath{fill}%
\end{pgfscope}%
\begin{pgfscope}%
\pgfpathrectangle{\pgfqpoint{0.680860in}{0.078740in}}{\pgfqpoint{7.842520in}{7.842520in}}%
\pgfusepath{clip}%
\pgfsetbuttcap%
\pgfsetroundjoin%
\definecolor{currentfill}{rgb}{0.282656,0.100196,0.422160}%
\pgfsetfillcolor{currentfill}%
\pgfsetlinewidth{0.000000pt}%
\definecolor{currentstroke}{rgb}{0.235526,0.309527,0.542944}%
\pgfsetstrokecolor{currentstroke}%
\pgfsetdash{}{0pt}%
\pgfpathmoveto{\pgfqpoint{4.696059in}{1.522403in}}%
\pgfpathlineto{\pgfqpoint{4.612268in}{1.186519in}}%
\pgfpathlineto{\pgfqpoint{4.757503in}{1.147698in}}%
\pgfpathclose%
\pgfusepath{fill}%
\end{pgfscope}%
\begin{pgfscope}%
\pgfpathrectangle{\pgfqpoint{0.680860in}{0.078740in}}{\pgfqpoint{7.842520in}{7.842520in}}%
\pgfusepath{clip}%
\pgfsetbuttcap%
\pgfsetroundjoin%
\definecolor{currentfill}{rgb}{0.311925,0.767822,0.415586}%
\pgfsetfillcolor{currentfill}%
\pgfsetlinewidth{0.000000pt}%
\definecolor{currentstroke}{rgb}{0.233603,0.313828,0.543914}%
\pgfsetstrokecolor{currentstroke}%
\pgfsetdash{}{0pt}%
\pgfpathmoveto{\pgfqpoint{5.449881in}{4.052044in}}%
\pgfpathlineto{\pgfqpoint{5.597397in}{4.086411in}}%
\pgfpathlineto{\pgfqpoint{5.680064in}{4.277342in}}%
\pgfpathclose%
\pgfusepath{fill}%
\end{pgfscope}%
\begin{pgfscope}%
\pgfpathrectangle{\pgfqpoint{0.680860in}{0.078740in}}{\pgfqpoint{7.842520in}{7.842520in}}%
\pgfusepath{clip}%
\pgfsetbuttcap%
\pgfsetroundjoin%
\definecolor{currentfill}{rgb}{0.120638,0.625828,0.533488}%
\pgfsetfillcolor{currentfill}%
\pgfsetlinewidth{0.000000pt}%
\definecolor{currentstroke}{rgb}{0.231674,0.318106,0.544834}%
\pgfsetstrokecolor{currentstroke}%
\pgfsetdash{}{0pt}%
\pgfpathmoveto{\pgfqpoint{5.053499in}{3.340173in}}%
\pgfpathlineto{\pgfqpoint{5.199790in}{3.355703in}}%
\pgfpathlineto{\pgfqpoint{5.283450in}{3.608903in}}%
\pgfpathclose%
\pgfusepath{fill}%
\end{pgfscope}%
\begin{pgfscope}%
\pgfpathrectangle{\pgfqpoint{0.680860in}{0.078740in}}{\pgfqpoint{7.842520in}{7.842520in}}%
\pgfusepath{clip}%
\pgfsetbuttcap%
\pgfsetroundjoin%
\definecolor{currentfill}{rgb}{0.906311,0.894855,0.098125}%
\pgfsetfillcolor{currentfill}%
\pgfsetlinewidth{0.000000pt}%
\definecolor{currentstroke}{rgb}{0.229739,0.322361,0.545706}%
\pgfsetstrokecolor{currentstroke}%
\pgfsetdash{}{0pt}%
\pgfpathmoveto{\pgfqpoint{6.908138in}{5.130032in}}%
\pgfpathlineto{\pgfqpoint{7.060616in}{5.193835in}}%
\pgfpathlineto{\pgfqpoint{7.134788in}{5.181011in}}%
\pgfpathclose%
\pgfusepath{fill}%
\end{pgfscope}%
\begin{pgfscope}%
\pgfpathrectangle{\pgfqpoint{0.680860in}{0.078740in}}{\pgfqpoint{7.842520in}{7.842520in}}%
\pgfusepath{clip}%
\pgfsetbuttcap%
\pgfsetroundjoin%
\definecolor{currentfill}{rgb}{0.204903,0.375746,0.553533}%
\pgfsetfillcolor{currentfill}%
\pgfsetlinewidth{0.000000pt}%
\definecolor{currentstroke}{rgb}{0.227802,0.326594,0.546532}%
\pgfsetstrokecolor{currentstroke}%
\pgfsetdash{}{0pt}%
\pgfpathmoveto{\pgfqpoint{4.947949in}{2.492741in}}%
\pgfpathlineto{\pgfqpoint{4.718516in}{2.191946in}}%
\pgfpathlineto{\pgfqpoint{4.863928in}{2.178019in}}%
\pgfpathclose%
\pgfusepath{fill}%
\end{pgfscope}%
\begin{pgfscope}%
\pgfpathrectangle{\pgfqpoint{0.680860in}{0.078740in}}{\pgfqpoint{7.842520in}{7.842520in}}%
\pgfusepath{clip}%
\pgfsetbuttcap%
\pgfsetroundjoin%
\definecolor{currentfill}{rgb}{0.226397,0.728888,0.462789}%
\pgfsetfillcolor{currentfill}%
\pgfsetlinewidth{0.000000pt}%
\definecolor{currentstroke}{rgb}{0.225863,0.330805,0.547314}%
\pgfsetstrokecolor{currentstroke}%
\pgfsetdash{}{0pt}%
\pgfpathmoveto{\pgfqpoint{5.366842in}{3.841576in}}%
\pgfpathlineto{\pgfqpoint{5.514242in}{3.871032in}}%
\pgfpathlineto{\pgfqpoint{5.449881in}{4.052044in}}%
\pgfpathclose%
\pgfusepath{fill}%
\end{pgfscope}%
\begin{pgfscope}%
\pgfpathrectangle{\pgfqpoint{0.680860in}{0.078740in}}{\pgfqpoint{7.842520in}{7.842520in}}%
\pgfusepath{clip}%
\pgfsetbuttcap%
\pgfsetroundjoin%
\definecolor{currentfill}{rgb}{0.126453,0.570633,0.549841}%
\pgfsetfillcolor{currentfill}%
\pgfsetlinewidth{0.000000pt}%
\definecolor{currentstroke}{rgb}{0.223925,0.334994,0.548053}%
\pgfsetstrokecolor{currentstroke}%
\pgfsetdash{}{0pt}%
\pgfpathmoveto{\pgfqpoint{5.115939in}{3.083869in}}%
\pgfpathlineto{\pgfqpoint{5.199790in}{3.355703in}}%
\pgfpathlineto{\pgfqpoint{4.969843in}{3.075032in}}%
\pgfpathclose%
\pgfusepath{fill}%
\end{pgfscope}%
\begin{pgfscope}%
\pgfpathrectangle{\pgfqpoint{0.680860in}{0.078740in}}{\pgfqpoint{7.842520in}{7.842520in}}%
\pgfusepath{clip}%
\pgfsetbuttcap%
\pgfsetroundjoin%
\definecolor{currentfill}{rgb}{0.165117,0.467423,0.558141}%
\pgfsetfillcolor{currentfill}%
\pgfsetlinewidth{0.000000pt}%
\definecolor{currentstroke}{rgb}{0.221989,0.339161,0.548752}%
\pgfsetstrokecolor{currentstroke}%
\pgfsetdash{}{0pt}%
\pgfpathmoveto{\pgfqpoint{4.947949in}{2.492741in}}%
\pgfpathlineto{\pgfqpoint{5.031971in}{2.795473in}}%
\pgfpathlineto{\pgfqpoint{4.886090in}{2.793832in}}%
\pgfpathclose%
\pgfusepath{fill}%
\end{pgfscope}%
\begin{pgfscope}%
\pgfpathrectangle{\pgfqpoint{0.680860in}{0.078740in}}{\pgfqpoint{7.842520in}{7.842520in}}%
\pgfusepath{clip}%
\pgfsetbuttcap%
\pgfsetroundjoin%
\definecolor{currentfill}{rgb}{0.140536,0.530132,0.555659}%
\pgfsetfillcolor{currentfill}%
\pgfsetlinewidth{0.000000pt}%
\definecolor{currentstroke}{rgb}{0.220057,0.343307,0.549413}%
\pgfsetstrokecolor{currentstroke}%
\pgfsetdash{}{0pt}%
\pgfpathmoveto{\pgfqpoint{4.969843in}{3.075032in}}%
\pgfpathlineto{\pgfqpoint{5.031971in}{2.795473in}}%
\pgfpathlineto{\pgfqpoint{5.115939in}{3.083869in}}%
\pgfpathclose%
\pgfusepath{fill}%
\end{pgfscope}%
\begin{pgfscope}%
\pgfpathrectangle{\pgfqpoint{0.680860in}{0.078740in}}{\pgfqpoint{7.842520in}{7.842520in}}%
\pgfusepath{clip}%
\pgfsetbuttcap%
\pgfsetroundjoin%
\definecolor{currentfill}{rgb}{0.866013,0.889868,0.095953}%
\pgfsetfillcolor{currentfill}%
\pgfsetlinewidth{0.000000pt}%
\definecolor{currentstroke}{rgb}{0.218130,0.347432,0.550038}%
\pgfsetstrokecolor{currentstroke}%
\pgfsetdash{}{0pt}%
\pgfpathmoveto{\pgfqpoint{6.680192in}{5.046603in}}%
\pgfpathlineto{\pgfqpoint{6.831993in}{5.108475in}}%
\pgfpathlineto{\pgfqpoint{6.908138in}{5.130032in}}%
\pgfpathclose%
\pgfusepath{fill}%
\end{pgfscope}%
\begin{pgfscope}%
\pgfpathrectangle{\pgfqpoint{0.680860in}{0.078740in}}{\pgfqpoint{7.842520in}{7.842520in}}%
\pgfusepath{clip}%
\pgfsetbuttcap%
\pgfsetroundjoin%
\definecolor{currentfill}{rgb}{0.259857,0.745492,0.444467}%
\pgfsetfillcolor{currentfill}%
\pgfsetlinewidth{0.000000pt}%
\definecolor{currentstroke}{rgb}{0.216210,0.351535,0.550627}%
\pgfsetstrokecolor{currentstroke}%
\pgfsetdash{}{0pt}%
\pgfpathmoveto{\pgfqpoint{5.449881in}{4.052044in}}%
\pgfpathlineto{\pgfqpoint{5.514242in}{3.871032in}}%
\pgfpathlineto{\pgfqpoint{5.597397in}{4.086411in}}%
\pgfpathclose%
\pgfusepath{fill}%
\end{pgfscope}%
\begin{pgfscope}%
\pgfpathrectangle{\pgfqpoint{0.680860in}{0.078740in}}{\pgfqpoint{7.842520in}{7.842520in}}%
\pgfusepath{clip}%
\pgfsetbuttcap%
\pgfsetroundjoin%
\definecolor{currentfill}{rgb}{0.282623,0.140926,0.457517}%
\pgfsetfillcolor{currentfill}%
\pgfsetlinewidth{0.000000pt}%
\definecolor{currentstroke}{rgb}{0.214298,0.355619,0.551184}%
\pgfsetstrokecolor{currentstroke}%
\pgfsetdash{}{0pt}%
\pgfpathmoveto{\pgfqpoint{4.757503in}{1.147698in}}%
\pgfpathlineto{\pgfqpoint{4.841556in}{1.492295in}}%
\pgfpathlineto{\pgfqpoint{4.696059in}{1.522403in}}%
\pgfpathclose%
\pgfusepath{fill}%
\end{pgfscope}%
\begin{pgfscope}%
\pgfpathrectangle{\pgfqpoint{0.680860in}{0.078740in}}{\pgfqpoint{7.842520in}{7.842520in}}%
\pgfusepath{clip}%
\pgfsetbuttcap%
\pgfsetroundjoin%
\definecolor{currentfill}{rgb}{0.720391,0.870350,0.162603}%
\pgfsetfillcolor{currentfill}%
\pgfsetlinewidth{0.000000pt}%
\definecolor{currentstroke}{rgb}{0.212395,0.359683,0.551710}%
\pgfsetstrokecolor{currentstroke}%
\pgfsetdash{}{0pt}%
\pgfpathmoveto{\pgfqpoint{6.451307in}{4.933289in}}%
\pgfpathlineto{\pgfqpoint{6.221809in}{4.792873in}}%
\pgfpathlineto{\pgfqpoint{6.372037in}{4.847802in}}%
\pgfpathclose%
\pgfusepath{fill}%
\end{pgfscope}%
\begin{pgfscope}%
\pgfpathrectangle{\pgfqpoint{0.680860in}{0.078740in}}{\pgfqpoint{7.842520in}{7.842520in}}%
\pgfusepath{clip}%
\pgfsetbuttcap%
\pgfsetroundjoin%
\definecolor{currentfill}{rgb}{0.565498,0.842430,0.262877}%
\pgfsetfillcolor{currentfill}%
\pgfsetlinewidth{0.000000pt}%
\definecolor{currentstroke}{rgb}{0.210503,0.363727,0.552206}%
\pgfsetstrokecolor{currentstroke}%
\pgfsetdash{}{0pt}%
\pgfpathmoveto{\pgfqpoint{5.910629in}{4.487206in}}%
\pgfpathlineto{\pgfqpoint{6.141369in}{4.678428in}}%
\pgfpathlineto{\pgfqpoint{5.992001in}{4.628294in}}%
\pgfpathclose%
\pgfusepath{fill}%
\end{pgfscope}%
\begin{pgfscope}%
\pgfpathrectangle{\pgfqpoint{0.680860in}{0.078740in}}{\pgfqpoint{7.842520in}{7.842520in}}%
\pgfusepath{clip}%
\pgfsetbuttcap%
\pgfsetroundjoin%
\definecolor{currentfill}{rgb}{0.263663,0.237631,0.518762}%
\pgfsetfillcolor{currentfill}%
\pgfsetlinewidth{0.000000pt}%
\definecolor{currentstroke}{rgb}{0.208623,0.367752,0.552675}%
\pgfsetstrokecolor{currentstroke}%
\pgfsetdash{}{0pt}%
\pgfpathmoveto{\pgfqpoint{4.925722in}{1.832273in}}%
\pgfpathlineto{\pgfqpoint{4.779954in}{1.853743in}}%
\pgfpathlineto{\pgfqpoint{4.696059in}{1.522403in}}%
\pgfpathclose%
\pgfusepath{fill}%
\end{pgfscope}%
\begin{pgfscope}%
\pgfpathrectangle{\pgfqpoint{0.680860in}{0.078740in}}{\pgfqpoint{7.842520in}{7.842520in}}%
\pgfusepath{clip}%
\pgfsetbuttcap%
\pgfsetroundjoin%
\definecolor{currentfill}{rgb}{0.180653,0.701402,0.488189}%
\pgfsetfillcolor{currentfill}%
\pgfsetlinewidth{0.000000pt}%
\definecolor{currentstroke}{rgb}{0.206756,0.371758,0.553117}%
\pgfsetstrokecolor{currentstroke}%
\pgfsetdash{}{0pt}%
\pgfpathmoveto{\pgfqpoint{5.366842in}{3.841576in}}%
\pgfpathlineto{\pgfqpoint{5.283450in}{3.608903in}}%
\pgfpathlineto{\pgfqpoint{5.514242in}{3.871032in}}%
\pgfpathclose%
\pgfusepath{fill}%
\end{pgfscope}%
\begin{pgfscope}%
\pgfpathrectangle{\pgfqpoint{0.680860in}{0.078740in}}{\pgfqpoint{7.842520in}{7.842520in}}%
\pgfusepath{clip}%
\pgfsetbuttcap%
\pgfsetroundjoin%
\definecolor{currentfill}{rgb}{0.804182,0.882046,0.114965}%
\pgfsetfillcolor{currentfill}%
\pgfsetlinewidth{0.000000pt}%
\definecolor{currentstroke}{rgb}{0.204903,0.375746,0.553533}%
\pgfsetstrokecolor{currentstroke}%
\pgfsetdash{}{0pt}%
\pgfpathmoveto{\pgfqpoint{6.680192in}{5.046603in}}%
\pgfpathlineto{\pgfqpoint{6.451307in}{4.933289in}}%
\pgfpathlineto{\pgfqpoint{6.602351in}{4.992168in}}%
\pgfpathclose%
\pgfusepath{fill}%
\end{pgfscope}%
\begin{pgfscope}%
\pgfpathrectangle{\pgfqpoint{0.680860in}{0.078740in}}{\pgfqpoint{7.842520in}{7.842520in}}%
\pgfusepath{clip}%
\pgfsetbuttcap%
\pgfsetroundjoin%
\definecolor{currentfill}{rgb}{0.271828,0.209303,0.504434}%
\pgfsetfillcolor{currentfill}%
\pgfsetlinewidth{0.000000pt}%
\definecolor{currentstroke}{rgb}{0.203063,0.379716,0.553925}%
\pgfsetstrokecolor{currentstroke}%
\pgfsetdash{}{0pt}%
\pgfpathmoveto{\pgfqpoint{4.696059in}{1.522403in}}%
\pgfpathlineto{\pgfqpoint{4.841556in}{1.492295in}}%
\pgfpathlineto{\pgfqpoint{4.925722in}{1.832273in}}%
\pgfpathclose%
\pgfusepath{fill}%
\end{pgfscope}%
\begin{pgfscope}%
\pgfpathrectangle{\pgfqpoint{0.680860in}{0.078740in}}{\pgfqpoint{7.842520in}{7.842520in}}%
\pgfusepath{clip}%
\pgfsetbuttcap%
\pgfsetroundjoin%
\definecolor{currentfill}{rgb}{0.440137,0.811138,0.340967}%
\pgfsetfillcolor{currentfill}%
\pgfsetlinewidth{0.000000pt}%
\definecolor{currentstroke}{rgb}{0.201239,0.383670,0.554294}%
\pgfsetstrokecolor{currentstroke}%
\pgfsetdash{}{0pt}%
\pgfpathmoveto{\pgfqpoint{5.910629in}{4.487206in}}%
\pgfpathlineto{\pgfqpoint{5.680064in}{4.277342in}}%
\pgfpathlineto{\pgfqpoint{5.828517in}{4.318579in}}%
\pgfpathclose%
\pgfusepath{fill}%
\end{pgfscope}%
\begin{pgfscope}%
\pgfpathrectangle{\pgfqpoint{0.680860in}{0.078740in}}{\pgfqpoint{7.842520in}{7.842520in}}%
\pgfusepath{clip}%
\pgfsetbuttcap%
\pgfsetroundjoin%
\definecolor{currentfill}{rgb}{0.227802,0.326594,0.546532}%
\pgfsetfillcolor{currentfill}%
\pgfsetlinewidth{0.000000pt}%
\definecolor{currentstroke}{rgb}{0.199430,0.387607,0.554642}%
\pgfsetstrokecolor{currentstroke}%
\pgfsetdash{}{0pt}%
\pgfpathmoveto{\pgfqpoint{4.863928in}{2.178019in}}%
\pgfpathlineto{\pgfqpoint{4.779954in}{1.853743in}}%
\pgfpathlineto{\pgfqpoint{5.009972in}{2.165024in}}%
\pgfpathclose%
\pgfusepath{fill}%
\end{pgfscope}%
\begin{pgfscope}%
\pgfpathrectangle{\pgfqpoint{0.680860in}{0.078740in}}{\pgfqpoint{7.842520in}{7.842520in}}%
\pgfusepath{clip}%
\pgfsetbuttcap%
\pgfsetroundjoin%
\definecolor{currentfill}{rgb}{0.123444,0.636809,0.528763}%
\pgfsetfillcolor{currentfill}%
\pgfsetlinewidth{0.000000pt}%
\definecolor{currentstroke}{rgb}{0.197636,0.391528,0.554969}%
\pgfsetstrokecolor{currentstroke}%
\pgfsetdash{}{0pt}%
\pgfpathmoveto{\pgfqpoint{5.199790in}{3.355703in}}%
\pgfpathlineto{\pgfqpoint{5.346837in}{3.373178in}}%
\pgfpathlineto{\pgfqpoint{5.283450in}{3.608903in}}%
\pgfpathclose%
\pgfusepath{fill}%
\end{pgfscope}%
\begin{pgfscope}%
\pgfpathrectangle{\pgfqpoint{0.680860in}{0.078740in}}{\pgfqpoint{7.842520in}{7.842520in}}%
\pgfusepath{clip}%
\pgfsetbuttcap%
\pgfsetroundjoin%
\definecolor{currentfill}{rgb}{0.190631,0.407061,0.556089}%
\pgfsetfillcolor{currentfill}%
\pgfsetlinewidth{0.000000pt}%
\definecolor{currentstroke}{rgb}{0.195860,0.395433,0.555276}%
\pgfsetstrokecolor{currentstroke}%
\pgfsetdash{}{0pt}%
\pgfpathmoveto{\pgfqpoint{5.094264in}{2.487966in}}%
\pgfpathlineto{\pgfqpoint{4.947949in}{2.492741in}}%
\pgfpathlineto{\pgfqpoint{4.863928in}{2.178019in}}%
\pgfpathclose%
\pgfusepath{fill}%
\end{pgfscope}%
\begin{pgfscope}%
\pgfpathrectangle{\pgfqpoint{0.680860in}{0.078740in}}{\pgfqpoint{7.842520in}{7.842520in}}%
\pgfusepath{clip}%
\pgfsetbuttcap%
\pgfsetroundjoin%
\definecolor{currentfill}{rgb}{0.344074,0.780029,0.397381}%
\pgfsetfillcolor{currentfill}%
\pgfsetlinewidth{0.000000pt}%
\definecolor{currentstroke}{rgb}{0.194100,0.399323,0.555565}%
\pgfsetstrokecolor{currentstroke}%
\pgfsetdash{}{0pt}%
\pgfpathmoveto{\pgfqpoint{5.680064in}{4.277342in}}%
\pgfpathlineto{\pgfqpoint{5.597397in}{4.086411in}}%
\pgfpathlineto{\pgfqpoint{5.745764in}{4.123394in}}%
\pgfpathclose%
\pgfusepath{fill}%
\end{pgfscope}%
\begin{pgfscope}%
\pgfpathrectangle{\pgfqpoint{0.680860in}{0.078740in}}{\pgfqpoint{7.842520in}{7.842520in}}%
\pgfusepath{clip}%
\pgfsetbuttcap%
\pgfsetroundjoin%
\definecolor{currentfill}{rgb}{0.237441,0.305202,0.541921}%
\pgfsetfillcolor{currentfill}%
\pgfsetlinewidth{0.000000pt}%
\definecolor{currentstroke}{rgb}{0.192357,0.403199,0.555836}%
\pgfsetstrokecolor{currentstroke}%
\pgfsetdash{}{0pt}%
\pgfpathmoveto{\pgfqpoint{4.779954in}{1.853743in}}%
\pgfpathlineto{\pgfqpoint{4.925722in}{1.832273in}}%
\pgfpathlineto{\pgfqpoint{5.009972in}{2.165024in}}%
\pgfpathclose%
\pgfusepath{fill}%
\end{pgfscope}%
\begin{pgfscope}%
\pgfpathrectangle{\pgfqpoint{0.680860in}{0.078740in}}{\pgfqpoint{7.842520in}{7.842520in}}%
\pgfusepath{clip}%
\pgfsetbuttcap%
\pgfsetroundjoin%
\definecolor{currentfill}{rgb}{0.120092,0.600104,0.542530}%
\pgfsetfillcolor{currentfill}%
\pgfsetlinewidth{0.000000pt}%
\definecolor{currentstroke}{rgb}{0.190631,0.407061,0.556089}%
\pgfsetstrokecolor{currentstroke}%
\pgfsetdash{}{0pt}%
\pgfpathmoveto{\pgfqpoint{5.346837in}{3.373178in}}%
\pgfpathlineto{\pgfqpoint{5.199790in}{3.355703in}}%
\pgfpathlineto{\pgfqpoint{5.115939in}{3.083869in}}%
\pgfpathclose%
\pgfusepath{fill}%
\end{pgfscope}%
\begin{pgfscope}%
\pgfpathrectangle{\pgfqpoint{0.680860in}{0.078740in}}{\pgfqpoint{7.842520in}{7.842520in}}%
\pgfusepath{clip}%
\pgfsetbuttcap%
\pgfsetroundjoin%
\definecolor{currentfill}{rgb}{0.171176,0.452530,0.557965}%
\pgfsetfillcolor{currentfill}%
\pgfsetlinewidth{0.000000pt}%
\definecolor{currentstroke}{rgb}{0.188923,0.410910,0.556326}%
\pgfsetstrokecolor{currentstroke}%
\pgfsetdash{}{0pt}%
\pgfpathmoveto{\pgfqpoint{5.031971in}{2.795473in}}%
\pgfpathlineto{\pgfqpoint{4.947949in}{2.492741in}}%
\pgfpathlineto{\pgfqpoint{5.094264in}{2.487966in}}%
\pgfpathclose%
\pgfusepath{fill}%
\end{pgfscope}%
\begin{pgfscope}%
\pgfpathrectangle{\pgfqpoint{0.680860in}{0.078740in}}{\pgfqpoint{7.842520in}{7.842520in}}%
\pgfusepath{clip}%
\pgfsetbuttcap%
\pgfsetroundjoin%
\definecolor{currentfill}{rgb}{0.166383,0.690856,0.496502}%
\pgfsetfillcolor{currentfill}%
\pgfsetlinewidth{0.000000pt}%
\definecolor{currentstroke}{rgb}{0.187231,0.414746,0.556547}%
\pgfsetstrokecolor{currentstroke}%
\pgfsetdash{}{0pt}%
\pgfpathmoveto{\pgfqpoint{5.514242in}{3.871032in}}%
\pgfpathlineto{\pgfqpoint{5.283450in}{3.608903in}}%
\pgfpathlineto{\pgfqpoint{5.430692in}{3.632705in}}%
\pgfpathclose%
\pgfusepath{fill}%
\end{pgfscope}%
\begin{pgfscope}%
\pgfpathrectangle{\pgfqpoint{0.680860in}{0.078740in}}{\pgfqpoint{7.842520in}{7.842520in}}%
\pgfusepath{clip}%
\pgfsetbuttcap%
\pgfsetroundjoin%
\definecolor{currentfill}{rgb}{0.144759,0.519093,0.556572}%
\pgfsetfillcolor{currentfill}%
\pgfsetlinewidth{0.000000pt}%
\definecolor{currentstroke}{rgb}{0.185556,0.418570,0.556753}%
\pgfsetstrokecolor{currentstroke}%
\pgfsetdash{}{0pt}%
\pgfpathmoveto{\pgfqpoint{5.115939in}{3.083869in}}%
\pgfpathlineto{\pgfqpoint{5.031971in}{2.795473in}}%
\pgfpathlineto{\pgfqpoint{5.178549in}{2.798575in}}%
\pgfpathclose%
\pgfusepath{fill}%
\end{pgfscope}%
\begin{pgfscope}%
\pgfpathrectangle{\pgfqpoint{0.680860in}{0.078740in}}{\pgfqpoint{7.842520in}{7.842520in}}%
\pgfusepath{clip}%
\pgfsetbuttcap%
\pgfsetroundjoin%
\definecolor{currentfill}{rgb}{0.282290,0.145912,0.461510}%
\pgfsetfillcolor{currentfill}%
\pgfsetlinewidth{0.000000pt}%
\definecolor{currentstroke}{rgb}{0.183898,0.422383,0.556944}%
\pgfsetstrokecolor{currentstroke}%
\pgfsetdash{}{0pt}%
\pgfpathmoveto{\pgfqpoint{4.987623in}{1.462587in}}%
\pgfpathlineto{\pgfqpoint{4.841556in}{1.492295in}}%
\pgfpathlineto{\pgfqpoint{4.757503in}{1.147698in}}%
\pgfpathclose%
\pgfusepath{fill}%
\end{pgfscope}%
\begin{pgfscope}%
\pgfpathrectangle{\pgfqpoint{0.680860in}{0.078740in}}{\pgfqpoint{7.842520in}{7.842520in}}%
\pgfusepath{clip}%
\pgfsetbuttcap%
\pgfsetroundjoin%
\definecolor{currentfill}{rgb}{0.668054,0.861999,0.196293}%
\pgfsetfillcolor{currentfill}%
\pgfsetlinewidth{0.000000pt}%
\definecolor{currentstroke}{rgb}{0.182256,0.426184,0.557120}%
\pgfsetstrokecolor{currentstroke}%
\pgfsetdash{}{0pt}%
\pgfpathmoveto{\pgfqpoint{6.221809in}{4.792873in}}%
\pgfpathlineto{\pgfqpoint{6.141369in}{4.678428in}}%
\pgfpathlineto{\pgfqpoint{6.291683in}{4.731755in}}%
\pgfpathclose%
\pgfusepath{fill}%
\end{pgfscope}%
\begin{pgfscope}%
\pgfpathrectangle{\pgfqpoint{0.680860in}{0.078740in}}{\pgfqpoint{7.842520in}{7.842520in}}%
\pgfusepath{clip}%
\pgfsetbuttcap%
\pgfsetroundjoin%
\definecolor{currentfill}{rgb}{0.386433,0.794644,0.372886}%
\pgfsetfillcolor{currentfill}%
\pgfsetlinewidth{0.000000pt}%
\definecolor{currentstroke}{rgb}{0.180629,0.429975,0.557282}%
\pgfsetstrokecolor{currentstroke}%
\pgfsetdash{}{0pt}%
\pgfpathmoveto{\pgfqpoint{5.828517in}{4.318579in}}%
\pgfpathlineto{\pgfqpoint{5.680064in}{4.277342in}}%
\pgfpathlineto{\pgfqpoint{5.745764in}{4.123394in}}%
\pgfpathclose%
\pgfusepath{fill}%
\end{pgfscope}%
\begin{pgfscope}%
\pgfpathrectangle{\pgfqpoint{0.680860in}{0.078740in}}{\pgfqpoint{7.842520in}{7.842520in}}%
\pgfusepath{clip}%
\pgfsetbuttcap%
\pgfsetroundjoin%
\definecolor{currentfill}{rgb}{0.134692,0.658636,0.517649}%
\pgfsetfillcolor{currentfill}%
\pgfsetlinewidth{0.000000pt}%
\definecolor{currentstroke}{rgb}{0.179019,0.433756,0.557430}%
\pgfsetstrokecolor{currentstroke}%
\pgfsetdash{}{0pt}%
\pgfpathmoveto{\pgfqpoint{5.283450in}{3.608903in}}%
\pgfpathlineto{\pgfqpoint{5.346837in}{3.373178in}}%
\pgfpathlineto{\pgfqpoint{5.430692in}{3.632705in}}%
\pgfpathclose%
\pgfusepath{fill}%
\end{pgfscope}%
\begin{pgfscope}%
\pgfpathrectangle{\pgfqpoint{0.680860in}{0.078740in}}{\pgfqpoint{7.842520in}{7.842520in}}%
\pgfusepath{clip}%
\pgfsetbuttcap%
\pgfsetroundjoin%
\definecolor{currentfill}{rgb}{0.201239,0.383670,0.554294}%
\pgfsetfillcolor{currentfill}%
\pgfsetlinewidth{0.000000pt}%
\definecolor{currentstroke}{rgb}{0.177423,0.437527,0.557565}%
\pgfsetstrokecolor{currentstroke}%
\pgfsetdash{}{0pt}%
\pgfpathmoveto{\pgfqpoint{4.863928in}{2.178019in}}%
\pgfpathlineto{\pgfqpoint{5.009972in}{2.165024in}}%
\pgfpathlineto{\pgfqpoint{5.094264in}{2.487966in}}%
\pgfpathclose%
\pgfusepath{fill}%
\end{pgfscope}%
\begin{pgfscope}%
\pgfpathrectangle{\pgfqpoint{0.680860in}{0.078740in}}{\pgfqpoint{7.842520in}{7.842520in}}%
\pgfusepath{clip}%
\pgfsetbuttcap%
\pgfsetroundjoin%
\definecolor{currentfill}{rgb}{0.283091,0.110553,0.431554}%
\pgfsetfillcolor{currentfill}%
\pgfsetlinewidth{0.000000pt}%
\definecolor{currentstroke}{rgb}{0.175841,0.441290,0.557685}%
\pgfsetstrokecolor{currentstroke}%
\pgfsetdash{}{0pt}%
\pgfpathmoveto{\pgfqpoint{4.757503in}{1.147698in}}%
\pgfpathlineto{\pgfqpoint{4.903271in}{1.108992in}}%
\pgfpathlineto{\pgfqpoint{4.987623in}{1.462587in}}%
\pgfpathclose%
\pgfusepath{fill}%
\end{pgfscope}%
\begin{pgfscope}%
\pgfpathrectangle{\pgfqpoint{0.680860in}{0.078740in}}{\pgfqpoint{7.842520in}{7.842520in}}%
\pgfusepath{clip}%
\pgfsetbuttcap%
\pgfsetroundjoin%
\definecolor{currentfill}{rgb}{0.575563,0.844566,0.256415}%
\pgfsetfillcolor{currentfill}%
\pgfsetlinewidth{0.000000pt}%
\definecolor{currentstroke}{rgb}{0.174274,0.445044,0.557792}%
\pgfsetstrokecolor{currentstroke}%
\pgfsetdash{}{0pt}%
\pgfpathmoveto{\pgfqpoint{6.060019in}{4.534803in}}%
\pgfpathlineto{\pgfqpoint{6.141369in}{4.678428in}}%
\pgfpathlineto{\pgfqpoint{5.910629in}{4.487206in}}%
\pgfpathclose%
\pgfusepath{fill}%
\end{pgfscope}%
\begin{pgfscope}%
\pgfpathrectangle{\pgfqpoint{0.680860in}{0.078740in}}{\pgfqpoint{7.842520in}{7.842520in}}%
\pgfusepath{clip}%
\pgfsetbuttcap%
\pgfsetroundjoin%
\definecolor{currentfill}{rgb}{0.709898,0.868751,0.169257}%
\pgfsetfillcolor{currentfill}%
\pgfsetlinewidth{0.000000pt}%
\definecolor{currentstroke}{rgb}{0.172719,0.448791,0.557885}%
\pgfsetstrokecolor{currentstroke}%
\pgfsetdash{}{0pt}%
\pgfpathmoveto{\pgfqpoint{6.372037in}{4.847802in}}%
\pgfpathlineto{\pgfqpoint{6.221809in}{4.792873in}}%
\pgfpathlineto{\pgfqpoint{6.291683in}{4.731755in}}%
\pgfpathclose%
\pgfusepath{fill}%
\end{pgfscope}%
\begin{pgfscope}%
\pgfpathrectangle{\pgfqpoint{0.680860in}{0.078740in}}{\pgfqpoint{7.842520in}{7.842520in}}%
\pgfusepath{clip}%
\pgfsetbuttcap%
\pgfsetroundjoin%
\definecolor{currentfill}{rgb}{0.935904,0.898570,0.108131}%
\pgfsetfillcolor{currentfill}%
\pgfsetlinewidth{0.000000pt}%
\definecolor{currentstroke}{rgb}{0.171176,0.452530,0.557965}%
\pgfsetstrokecolor{currentstroke}%
\pgfsetdash{}{0pt}%
\pgfpathmoveto{\pgfqpoint{6.984835in}{5.173977in}}%
\pgfpathlineto{\pgfqpoint{7.060616in}{5.193835in}}%
\pgfpathlineto{\pgfqpoint{6.908138in}{5.130032in}}%
\pgfpathclose%
\pgfusepath{fill}%
\end{pgfscope}%
\begin{pgfscope}%
\pgfpathrectangle{\pgfqpoint{0.680860in}{0.078740in}}{\pgfqpoint{7.842520in}{7.842520in}}%
\pgfusepath{clip}%
\pgfsetbuttcap%
\pgfsetroundjoin%
\definecolor{currentfill}{rgb}{0.288921,0.758394,0.428426}%
\pgfsetfillcolor{currentfill}%
\pgfsetlinewidth{0.000000pt}%
\definecolor{currentstroke}{rgb}{0.169646,0.456262,0.558030}%
\pgfsetstrokecolor{currentstroke}%
\pgfsetdash{}{0pt}%
\pgfpathmoveto{\pgfqpoint{5.597397in}{4.086411in}}%
\pgfpathlineto{\pgfqpoint{5.514242in}{3.871032in}}%
\pgfpathlineto{\pgfqpoint{5.745764in}{4.123394in}}%
\pgfpathclose%
\pgfusepath{fill}%
\end{pgfscope}%
\begin{pgfscope}%
\pgfpathrectangle{\pgfqpoint{0.680860in}{0.078740in}}{\pgfqpoint{7.842520in}{7.842520in}}%
\pgfusepath{clip}%
\pgfsetbuttcap%
\pgfsetroundjoin%
\definecolor{currentfill}{rgb}{0.926106,0.897330,0.104071}%
\pgfsetfillcolor{currentfill}%
\pgfsetlinewidth{0.000000pt}%
\definecolor{currentstroke}{rgb}{0.168126,0.459988,0.558082}%
\pgfsetstrokecolor{currentstroke}%
\pgfsetdash{}{0pt}%
\pgfpathmoveto{\pgfqpoint{6.908138in}{5.130032in}}%
\pgfpathlineto{\pgfqpoint{6.831993in}{5.108475in}}%
\pgfpathlineto{\pgfqpoint{6.984835in}{5.173977in}}%
\pgfpathclose%
\pgfusepath{fill}%
\end{pgfscope}%
\begin{pgfscope}%
\pgfpathrectangle{\pgfqpoint{0.680860in}{0.078740in}}{\pgfqpoint{7.842520in}{7.842520in}}%
\pgfusepath{clip}%
\pgfsetbuttcap%
\pgfsetroundjoin%
\definecolor{currentfill}{rgb}{0.160665,0.478540,0.558115}%
\pgfsetfillcolor{currentfill}%
\pgfsetlinewidth{0.000000pt}%
\definecolor{currentstroke}{rgb}{0.166617,0.463708,0.558119}%
\pgfsetstrokecolor{currentstroke}%
\pgfsetdash{}{0pt}%
\pgfpathmoveto{\pgfqpoint{5.094264in}{2.487966in}}%
\pgfpathlineto{\pgfqpoint{5.178549in}{2.798575in}}%
\pgfpathlineto{\pgfqpoint{5.031971in}{2.795473in}}%
\pgfpathclose%
\pgfusepath{fill}%
\end{pgfscope}%
\begin{pgfscope}%
\pgfpathrectangle{\pgfqpoint{0.680860in}{0.078740in}}{\pgfqpoint{7.842520in}{7.842520in}}%
\pgfusepath{clip}%
\pgfsetbuttcap%
\pgfsetroundjoin%
\definecolor{currentfill}{rgb}{0.772852,0.877868,0.131109}%
\pgfsetfillcolor{currentfill}%
\pgfsetlinewidth{0.000000pt}%
\definecolor{currentstroke}{rgb}{0.165117,0.467423,0.558141}%
\pgfsetstrokecolor{currentstroke}%
\pgfsetdash{}{0pt}%
\pgfpathmoveto{\pgfqpoint{6.372037in}{4.847802in}}%
\pgfpathlineto{\pgfqpoint{6.523246in}{4.906102in}}%
\pgfpathlineto{\pgfqpoint{6.451307in}{4.933289in}}%
\pgfpathclose%
\pgfusepath{fill}%
\end{pgfscope}%
\begin{pgfscope}%
\pgfpathrectangle{\pgfqpoint{0.680860in}{0.078740in}}{\pgfqpoint{7.842520in}{7.842520in}}%
\pgfusepath{clip}%
\pgfsetbuttcap%
\pgfsetroundjoin%
\definecolor{currentfill}{rgb}{0.122606,0.585371,0.546557}%
\pgfsetfillcolor{currentfill}%
\pgfsetlinewidth{0.000000pt}%
\definecolor{currentstroke}{rgb}{0.163625,0.471133,0.558148}%
\pgfsetstrokecolor{currentstroke}%
\pgfsetdash{}{0pt}%
\pgfpathmoveto{\pgfqpoint{5.115939in}{3.083869in}}%
\pgfpathlineto{\pgfqpoint{5.262763in}{3.094416in}}%
\pgfpathlineto{\pgfqpoint{5.346837in}{3.373178in}}%
\pgfpathclose%
\pgfusepath{fill}%
\end{pgfscope}%
\begin{pgfscope}%
\pgfpathrectangle{\pgfqpoint{0.680860in}{0.078740in}}{\pgfqpoint{7.842520in}{7.842520in}}%
\pgfusepath{clip}%
\pgfsetbuttcap%
\pgfsetroundjoin%
\definecolor{currentfill}{rgb}{0.260571,0.246922,0.522828}%
\pgfsetfillcolor{currentfill}%
\pgfsetlinewidth{0.000000pt}%
\definecolor{currentstroke}{rgb}{0.162142,0.474838,0.558140}%
\pgfsetstrokecolor{currentstroke}%
\pgfsetdash{}{0pt}%
\pgfpathmoveto{\pgfqpoint{5.072097in}{1.811488in}}%
\pgfpathlineto{\pgfqpoint{4.925722in}{1.832273in}}%
\pgfpathlineto{\pgfqpoint{4.841556in}{1.492295in}}%
\pgfpathclose%
\pgfusepath{fill}%
\end{pgfscope}%
\begin{pgfscope}%
\pgfpathrectangle{\pgfqpoint{0.680860in}{0.078740in}}{\pgfqpoint{7.842520in}{7.842520in}}%
\pgfusepath{clip}%
\pgfsetbuttcap%
\pgfsetroundjoin%
\definecolor{currentfill}{rgb}{0.135066,0.544853,0.554029}%
\pgfsetfillcolor{currentfill}%
\pgfsetlinewidth{0.000000pt}%
\definecolor{currentstroke}{rgb}{0.160665,0.478540,0.558115}%
\pgfsetstrokecolor{currentstroke}%
\pgfsetdash{}{0pt}%
\pgfpathmoveto{\pgfqpoint{5.178549in}{2.798575in}}%
\pgfpathlineto{\pgfqpoint{5.262763in}{3.094416in}}%
\pgfpathlineto{\pgfqpoint{5.115939in}{3.083869in}}%
\pgfpathclose%
\pgfusepath{fill}%
\end{pgfscope}%
\begin{pgfscope}%
\pgfpathrectangle{\pgfqpoint{0.680860in}{0.078740in}}{\pgfqpoint{7.842520in}{7.842520in}}%
\pgfusepath{clip}%
\pgfsetbuttcap%
\pgfsetroundjoin%
\definecolor{currentfill}{rgb}{0.804182,0.882046,0.114965}%
\pgfsetfillcolor{currentfill}%
\pgfsetlinewidth{0.000000pt}%
\definecolor{currentstroke}{rgb}{0.159194,0.482237,0.558073}%
\pgfsetstrokecolor{currentstroke}%
\pgfsetdash{}{0pt}%
\pgfpathmoveto{\pgfqpoint{6.451307in}{4.933289in}}%
\pgfpathlineto{\pgfqpoint{6.523246in}{4.906102in}}%
\pgfpathlineto{\pgfqpoint{6.602351in}{4.992168in}}%
\pgfpathclose%
\pgfusepath{fill}%
\end{pgfscope}%
\begin{pgfscope}%
\pgfpathrectangle{\pgfqpoint{0.680860in}{0.078740in}}{\pgfqpoint{7.842520in}{7.842520in}}%
\pgfusepath{clip}%
\pgfsetbuttcap%
\pgfsetroundjoin%
\definecolor{currentfill}{rgb}{0.886271,0.892374,0.095374}%
\pgfsetfillcolor{currentfill}%
\pgfsetlinewidth{0.000000pt}%
\definecolor{currentstroke}{rgb}{0.157729,0.485932,0.558013}%
\pgfsetstrokecolor{currentstroke}%
\pgfsetdash{}{0pt}%
\pgfpathmoveto{\pgfqpoint{6.680192in}{5.046603in}}%
\pgfpathlineto{\pgfqpoint{6.754408in}{5.054566in}}%
\pgfpathlineto{\pgfqpoint{6.831993in}{5.108475in}}%
\pgfpathclose%
\pgfusepath{fill}%
\end{pgfscope}%
\begin{pgfscope}%
\pgfpathrectangle{\pgfqpoint{0.680860in}{0.078740in}}{\pgfqpoint{7.842520in}{7.842520in}}%
\pgfusepath{clip}%
\pgfsetbuttcap%
\pgfsetroundjoin%
\definecolor{currentfill}{rgb}{0.866013,0.889868,0.095953}%
\pgfsetfillcolor{currentfill}%
\pgfsetlinewidth{0.000000pt}%
\definecolor{currentstroke}{rgb}{0.156270,0.489624,0.557936}%
\pgfsetstrokecolor{currentstroke}%
\pgfsetdash{}{0pt}%
\pgfpathmoveto{\pgfqpoint{6.602351in}{4.992168in}}%
\pgfpathlineto{\pgfqpoint{6.754408in}{5.054566in}}%
\pgfpathlineto{\pgfqpoint{6.680192in}{5.046603in}}%
\pgfpathclose%
\pgfusepath{fill}%
\end{pgfscope}%
\begin{pgfscope}%
\pgfpathrectangle{\pgfqpoint{0.680860in}{0.078740in}}{\pgfqpoint{7.842520in}{7.842520in}}%
\pgfusepath{clip}%
\pgfsetbuttcap%
\pgfsetroundjoin%
\definecolor{currentfill}{rgb}{0.477504,0.821444,0.318195}%
\pgfsetfillcolor{currentfill}%
\pgfsetlinewidth{0.000000pt}%
\definecolor{currentstroke}{rgb}{0.154815,0.493313,0.557840}%
\pgfsetstrokecolor{currentstroke}%
\pgfsetdash{}{0pt}%
\pgfpathmoveto{\pgfqpoint{5.828517in}{4.318579in}}%
\pgfpathlineto{\pgfqpoint{5.977862in}{4.362686in}}%
\pgfpathlineto{\pgfqpoint{5.910629in}{4.487206in}}%
\pgfpathclose%
\pgfusepath{fill}%
\end{pgfscope}%
\begin{pgfscope}%
\pgfpathrectangle{\pgfqpoint{0.680860in}{0.078740in}}{\pgfqpoint{7.842520in}{7.842520in}}%
\pgfusepath{clip}%
\pgfsetbuttcap%
\pgfsetroundjoin%
\definecolor{currentfill}{rgb}{0.270595,0.214069,0.507052}%
\pgfsetfillcolor{currentfill}%
\pgfsetlinewidth{0.000000pt}%
\definecolor{currentstroke}{rgb}{0.153364,0.497000,0.557724}%
\pgfsetstrokecolor{currentstroke}%
\pgfsetdash{}{0pt}%
\pgfpathmoveto{\pgfqpoint{5.072097in}{1.811488in}}%
\pgfpathlineto{\pgfqpoint{4.841556in}{1.492295in}}%
\pgfpathlineto{\pgfqpoint{4.987623in}{1.462587in}}%
\pgfpathclose%
\pgfusepath{fill}%
\end{pgfscope}%
\begin{pgfscope}%
\pgfpathrectangle{\pgfqpoint{0.680860in}{0.078740in}}{\pgfqpoint{7.842520in}{7.842520in}}%
\pgfusepath{clip}%
\pgfsetbuttcap%
\pgfsetroundjoin%
\definecolor{currentfill}{rgb}{0.221989,0.339161,0.548752}%
\pgfsetfillcolor{currentfill}%
\pgfsetlinewidth{0.000000pt}%
\definecolor{currentstroke}{rgb}{0.151918,0.500685,0.557587}%
\pgfsetstrokecolor{currentstroke}%
\pgfsetdash{}{0pt}%
\pgfpathmoveto{\pgfqpoint{5.156658in}{2.152997in}}%
\pgfpathlineto{\pgfqpoint{5.009972in}{2.165024in}}%
\pgfpathlineto{\pgfqpoint{4.925722in}{1.832273in}}%
\pgfpathclose%
\pgfusepath{fill}%
\end{pgfscope}%
\begin{pgfscope}%
\pgfpathrectangle{\pgfqpoint{0.680860in}{0.078740in}}{\pgfqpoint{7.842520in}{7.842520in}}%
\pgfusepath{clip}%
\pgfsetbuttcap%
\pgfsetroundjoin%
\definecolor{currentfill}{rgb}{0.525776,0.833491,0.288127}%
\pgfsetfillcolor{currentfill}%
\pgfsetlinewidth{0.000000pt}%
\definecolor{currentstroke}{rgb}{0.150476,0.504369,0.557430}%
\pgfsetstrokecolor{currentstroke}%
\pgfsetdash{}{0pt}%
\pgfpathmoveto{\pgfqpoint{5.910629in}{4.487206in}}%
\pgfpathlineto{\pgfqpoint{5.977862in}{4.362686in}}%
\pgfpathlineto{\pgfqpoint{6.060019in}{4.534803in}}%
\pgfpathclose%
\pgfusepath{fill}%
\end{pgfscope}%
\begin{pgfscope}%
\pgfpathrectangle{\pgfqpoint{0.680860in}{0.078740in}}{\pgfqpoint{7.842520in}{7.842520in}}%
\pgfusepath{clip}%
\pgfsetbuttcap%
\pgfsetroundjoin%
\definecolor{currentfill}{rgb}{0.283091,0.110553,0.431554}%
\pgfsetfillcolor{currentfill}%
\pgfsetlinewidth{0.000000pt}%
\definecolor{currentstroke}{rgb}{0.149039,0.508051,0.557250}%
\pgfsetstrokecolor{currentstroke}%
\pgfsetdash{}{0pt}%
\pgfpathmoveto{\pgfqpoint{4.987623in}{1.462587in}}%
\pgfpathlineto{\pgfqpoint{4.903271in}{1.108992in}}%
\pgfpathlineto{\pgfqpoint{5.049577in}{1.070410in}}%
\pgfpathclose%
\pgfusepath{fill}%
\end{pgfscope}%
\begin{pgfscope}%
\pgfpathrectangle{\pgfqpoint{0.680860in}{0.078740in}}{\pgfqpoint{7.842520in}{7.842520in}}%
\pgfusepath{clip}%
\pgfsetbuttcap%
\pgfsetroundjoin%
\definecolor{currentfill}{rgb}{0.185783,0.704891,0.485273}%
\pgfsetfillcolor{currentfill}%
\pgfsetlinewidth{0.000000pt}%
\definecolor{currentstroke}{rgb}{0.147607,0.511733,0.557049}%
\pgfsetstrokecolor{currentstroke}%
\pgfsetdash{}{0pt}%
\pgfpathmoveto{\pgfqpoint{5.578737in}{3.658750in}}%
\pgfpathlineto{\pgfqpoint{5.514242in}{3.871032in}}%
\pgfpathlineto{\pgfqpoint{5.430692in}{3.632705in}}%
\pgfpathclose%
\pgfusepath{fill}%
\end{pgfscope}%
\begin{pgfscope}%
\pgfpathrectangle{\pgfqpoint{0.680860in}{0.078740in}}{\pgfqpoint{7.842520in}{7.842520in}}%
\pgfusepath{clip}%
\pgfsetbuttcap%
\pgfsetroundjoin%
\definecolor{currentfill}{rgb}{0.636902,0.856542,0.216620}%
\pgfsetfillcolor{currentfill}%
\pgfsetlinewidth{0.000000pt}%
\definecolor{currentstroke}{rgb}{0.146180,0.515413,0.556823}%
\pgfsetstrokecolor{currentstroke}%
\pgfsetdash{}{0pt}%
\pgfpathmoveto{\pgfqpoint{6.291683in}{4.731755in}}%
\pgfpathlineto{\pgfqpoint{6.141369in}{4.678428in}}%
\pgfpathlineto{\pgfqpoint{6.060019in}{4.534803in}}%
\pgfpathclose%
\pgfusepath{fill}%
\end{pgfscope}%
\begin{pgfscope}%
\pgfpathrectangle{\pgfqpoint{0.680860in}{0.078740in}}{\pgfqpoint{7.842520in}{7.842520in}}%
\pgfusepath{clip}%
\pgfsetbuttcap%
\pgfsetroundjoin%
\definecolor{currentfill}{rgb}{0.197636,0.391528,0.554969}%
\pgfsetfillcolor{currentfill}%
\pgfsetlinewidth{0.000000pt}%
\definecolor{currentstroke}{rgb}{0.144759,0.519093,0.556572}%
\pgfsetstrokecolor{currentstroke}%
\pgfsetdash{}{0pt}%
\pgfpathmoveto{\pgfqpoint{5.156658in}{2.152997in}}%
\pgfpathlineto{\pgfqpoint{5.094264in}{2.487966in}}%
\pgfpathlineto{\pgfqpoint{5.009972in}{2.165024in}}%
\pgfpathclose%
\pgfusepath{fill}%
\end{pgfscope}%
\begin{pgfscope}%
\pgfpathrectangle{\pgfqpoint{0.680860in}{0.078740in}}{\pgfqpoint{7.842520in}{7.842520in}}%
\pgfusepath{clip}%
\pgfsetbuttcap%
\pgfsetroundjoin%
\definecolor{currentfill}{rgb}{0.281477,0.755203,0.432552}%
\pgfsetfillcolor{currentfill}%
\pgfsetlinewidth{0.000000pt}%
\definecolor{currentstroke}{rgb}{0.143343,0.522773,0.556295}%
\pgfsetstrokecolor{currentstroke}%
\pgfsetdash{}{0pt}%
\pgfpathmoveto{\pgfqpoint{5.745764in}{4.123394in}}%
\pgfpathlineto{\pgfqpoint{5.514242in}{3.871032in}}%
\pgfpathlineto{\pgfqpoint{5.662470in}{3.902930in}}%
\pgfpathclose%
\pgfusepath{fill}%
\end{pgfscope}%
\begin{pgfscope}%
\pgfpathrectangle{\pgfqpoint{0.680860in}{0.078740in}}{\pgfqpoint{7.842520in}{7.842520in}}%
\pgfusepath{clip}%
\pgfsetbuttcap%
\pgfsetroundjoin%
\definecolor{currentfill}{rgb}{0.233603,0.313828,0.543914}%
\pgfsetfillcolor{currentfill}%
\pgfsetlinewidth{0.000000pt}%
\definecolor{currentstroke}{rgb}{0.141935,0.526453,0.555991}%
\pgfsetstrokecolor{currentstroke}%
\pgfsetdash{}{0pt}%
\pgfpathmoveto{\pgfqpoint{4.925722in}{1.832273in}}%
\pgfpathlineto{\pgfqpoint{5.072097in}{1.811488in}}%
\pgfpathlineto{\pgfqpoint{5.156658in}{2.152997in}}%
\pgfpathclose%
\pgfusepath{fill}%
\end{pgfscope}%
\begin{pgfscope}%
\pgfpathrectangle{\pgfqpoint{0.680860in}{0.078740in}}{\pgfqpoint{7.842520in}{7.842520in}}%
\pgfusepath{clip}%
\pgfsetbuttcap%
\pgfsetroundjoin%
\definecolor{currentfill}{rgb}{0.146616,0.673050,0.508936}%
\pgfsetfillcolor{currentfill}%
\pgfsetlinewidth{0.000000pt}%
\definecolor{currentstroke}{rgb}{0.140536,0.530132,0.555659}%
\pgfsetstrokecolor{currentstroke}%
\pgfsetdash{}{0pt}%
\pgfpathmoveto{\pgfqpoint{5.430692in}{3.632705in}}%
\pgfpathlineto{\pgfqpoint{5.346837in}{3.373178in}}%
\pgfpathlineto{\pgfqpoint{5.578737in}{3.658750in}}%
\pgfpathclose%
\pgfusepath{fill}%
\end{pgfscope}%
\begin{pgfscope}%
\pgfpathrectangle{\pgfqpoint{0.680860in}{0.078740in}}{\pgfqpoint{7.842520in}{7.842520in}}%
\pgfusepath{clip}%
\pgfsetbuttcap%
\pgfsetroundjoin%
\definecolor{currentfill}{rgb}{0.430983,0.808473,0.346476}%
\pgfsetfillcolor{currentfill}%
\pgfsetlinewidth{0.000000pt}%
\definecolor{currentstroke}{rgb}{0.139147,0.533812,0.555298}%
\pgfsetstrokecolor{currentstroke}%
\pgfsetdash{}{0pt}%
\pgfpathmoveto{\pgfqpoint{5.745764in}{4.123394in}}%
\pgfpathlineto{\pgfqpoint{5.977862in}{4.362686in}}%
\pgfpathlineto{\pgfqpoint{5.828517in}{4.318579in}}%
\pgfpathclose%
\pgfusepath{fill}%
\end{pgfscope}%
\begin{pgfscope}%
\pgfpathrectangle{\pgfqpoint{0.680860in}{0.078740in}}{\pgfqpoint{7.842520in}{7.842520in}}%
\pgfusepath{clip}%
\pgfsetbuttcap%
\pgfsetroundjoin%
\definecolor{currentfill}{rgb}{0.156270,0.489624,0.557936}%
\pgfsetfillcolor{currentfill}%
\pgfsetlinewidth{0.000000pt}%
\definecolor{currentstroke}{rgb}{0.137770,0.537492,0.554906}%
\pgfsetstrokecolor{currentstroke}%
\pgfsetdash{}{0pt}%
\pgfpathmoveto{\pgfqpoint{5.325838in}{2.803195in}}%
\pgfpathlineto{\pgfqpoint{5.178549in}{2.798575in}}%
\pgfpathlineto{\pgfqpoint{5.094264in}{2.487966in}}%
\pgfpathclose%
\pgfusepath{fill}%
\end{pgfscope}%
\begin{pgfscope}%
\pgfpathrectangle{\pgfqpoint{0.680860in}{0.078740in}}{\pgfqpoint{7.842520in}{7.842520in}}%
\pgfusepath{clip}%
\pgfsetbuttcap%
\pgfsetroundjoin%
\definecolor{currentfill}{rgb}{0.220124,0.725509,0.466226}%
\pgfsetfillcolor{currentfill}%
\pgfsetlinewidth{0.000000pt}%
\definecolor{currentstroke}{rgb}{0.136408,0.541173,0.554483}%
\pgfsetstrokecolor{currentstroke}%
\pgfsetdash{}{0pt}%
\pgfpathmoveto{\pgfqpoint{5.662470in}{3.902930in}}%
\pgfpathlineto{\pgfqpoint{5.514242in}{3.871032in}}%
\pgfpathlineto{\pgfqpoint{5.578737in}{3.658750in}}%
\pgfpathclose%
\pgfusepath{fill}%
\end{pgfscope}%
\begin{pgfscope}%
\pgfpathrectangle{\pgfqpoint{0.680860in}{0.078740in}}{\pgfqpoint{7.842520in}{7.842520in}}%
\pgfusepath{clip}%
\pgfsetbuttcap%
\pgfsetroundjoin%
\definecolor{currentfill}{rgb}{0.119699,0.618490,0.536347}%
\pgfsetfillcolor{currentfill}%
\pgfsetlinewidth{0.000000pt}%
\definecolor{currentstroke}{rgb}{0.135066,0.544853,0.554029}%
\pgfsetstrokecolor{currentstroke}%
\pgfsetdash{}{0pt}%
\pgfpathmoveto{\pgfqpoint{5.494660in}{3.392671in}}%
\pgfpathlineto{\pgfqpoint{5.346837in}{3.373178in}}%
\pgfpathlineto{\pgfqpoint{5.262763in}{3.094416in}}%
\pgfpathclose%
\pgfusepath{fill}%
\end{pgfscope}%
\begin{pgfscope}%
\pgfpathrectangle{\pgfqpoint{0.680860in}{0.078740in}}{\pgfqpoint{7.842520in}{7.842520in}}%
\pgfusepath{clip}%
\pgfsetbuttcap%
\pgfsetroundjoin%
\definecolor{currentfill}{rgb}{0.762373,0.876424,0.137064}%
\pgfsetfillcolor{currentfill}%
\pgfsetlinewidth{0.000000pt}%
\definecolor{currentstroke}{rgb}{0.133743,0.548535,0.553541}%
\pgfsetstrokecolor{currentstroke}%
\pgfsetdash{}{0pt}%
\pgfpathmoveto{\pgfqpoint{6.291683in}{4.731755in}}%
\pgfpathlineto{\pgfqpoint{6.523246in}{4.906102in}}%
\pgfpathlineto{\pgfqpoint{6.372037in}{4.847802in}}%
\pgfpathclose%
\pgfusepath{fill}%
\end{pgfscope}%
\begin{pgfscope}%
\pgfpathrectangle{\pgfqpoint{0.680860in}{0.078740in}}{\pgfqpoint{7.842520in}{7.842520in}}%
\pgfusepath{clip}%
\pgfsetbuttcap%
\pgfsetroundjoin%
\definecolor{currentfill}{rgb}{0.131172,0.555899,0.552459}%
\pgfsetfillcolor{currentfill}%
\pgfsetlinewidth{0.000000pt}%
\definecolor{currentstroke}{rgb}{0.132444,0.552216,0.553018}%
\pgfsetstrokecolor{currentstroke}%
\pgfsetdash{}{0pt}%
\pgfpathmoveto{\pgfqpoint{5.410332in}{3.106740in}}%
\pgfpathlineto{\pgfqpoint{5.262763in}{3.094416in}}%
\pgfpathlineto{\pgfqpoint{5.178549in}{2.798575in}}%
\pgfpathclose%
\pgfusepath{fill}%
\end{pgfscope}%
\begin{pgfscope}%
\pgfpathrectangle{\pgfqpoint{0.680860in}{0.078740in}}{\pgfqpoint{7.842520in}{7.842520in}}%
\pgfusepath{clip}%
\pgfsetbuttcap%
\pgfsetroundjoin%
\definecolor{currentfill}{rgb}{0.269308,0.218818,0.509577}%
\pgfsetfillcolor{currentfill}%
\pgfsetlinewidth{0.000000pt}%
\definecolor{currentstroke}{rgb}{0.131172,0.555899,0.552459}%
\pgfsetstrokecolor{currentstroke}%
\pgfsetdash{}{0pt}%
\pgfpathmoveto{\pgfqpoint{4.987623in}{1.462587in}}%
\pgfpathlineto{\pgfqpoint{5.134266in}{1.433296in}}%
\pgfpathlineto{\pgfqpoint{5.072097in}{1.811488in}}%
\pgfpathclose%
\pgfusepath{fill}%
\end{pgfscope}%
\begin{pgfscope}%
\pgfpathrectangle{\pgfqpoint{0.680860in}{0.078740in}}{\pgfqpoint{7.842520in}{7.842520in}}%
\pgfusepath{clip}%
\pgfsetbuttcap%
\pgfsetroundjoin%
\definecolor{currentfill}{rgb}{0.281887,0.150881,0.465405}%
\pgfsetfillcolor{currentfill}%
\pgfsetlinewidth{0.000000pt}%
\definecolor{currentstroke}{rgb}{0.129933,0.559582,0.551864}%
\pgfsetstrokecolor{currentstroke}%
\pgfsetdash{}{0pt}%
\pgfpathmoveto{\pgfqpoint{5.049577in}{1.070410in}}%
\pgfpathlineto{\pgfqpoint{5.134266in}{1.433296in}}%
\pgfpathlineto{\pgfqpoint{4.987623in}{1.462587in}}%
\pgfpathclose%
\pgfusepath{fill}%
\end{pgfscope}%
\begin{pgfscope}%
\pgfpathrectangle{\pgfqpoint{0.680860in}{0.078740in}}{\pgfqpoint{7.842520in}{7.842520in}}%
\pgfusepath{clip}%
\pgfsetbuttcap%
\pgfsetroundjoin%
\definecolor{currentfill}{rgb}{0.183898,0.422383,0.556944}%
\pgfsetfillcolor{currentfill}%
\pgfsetlinewidth{0.000000pt}%
\definecolor{currentstroke}{rgb}{0.128729,0.563265,0.551229}%
\pgfsetstrokecolor{currentstroke}%
\pgfsetdash{}{0pt}%
\pgfpathmoveto{\pgfqpoint{5.241257in}{2.484439in}}%
\pgfpathlineto{\pgfqpoint{5.094264in}{2.487966in}}%
\pgfpathlineto{\pgfqpoint{5.156658in}{2.152997in}}%
\pgfpathclose%
\pgfusepath{fill}%
\end{pgfscope}%
\begin{pgfscope}%
\pgfpathrectangle{\pgfqpoint{0.680860in}{0.078740in}}{\pgfqpoint{7.842520in}{7.842520in}}%
\pgfusepath{clip}%
\pgfsetbuttcap%
\pgfsetroundjoin%
\definecolor{currentfill}{rgb}{0.935904,0.898570,0.108131}%
\pgfsetfillcolor{currentfill}%
\pgfsetlinewidth{0.000000pt}%
\definecolor{currentstroke}{rgb}{0.127568,0.566949,0.550556}%
\pgfsetstrokecolor{currentstroke}%
\pgfsetdash{}{0pt}%
\pgfpathmoveto{\pgfqpoint{6.831993in}{5.108475in}}%
\pgfpathlineto{\pgfqpoint{6.754408in}{5.054566in}}%
\pgfpathlineto{\pgfqpoint{6.984835in}{5.173977in}}%
\pgfpathclose%
\pgfusepath{fill}%
\end{pgfscope}%
\begin{pgfscope}%
\pgfpathrectangle{\pgfqpoint{0.680860in}{0.078740in}}{\pgfqpoint{7.842520in}{7.842520in}}%
\pgfusepath{clip}%
\pgfsetbuttcap%
\pgfsetroundjoin%
\definecolor{currentfill}{rgb}{0.163625,0.471133,0.558148}%
\pgfsetfillcolor{currentfill}%
\pgfsetlinewidth{0.000000pt}%
\definecolor{currentstroke}{rgb}{0.126453,0.570633,0.549841}%
\pgfsetstrokecolor{currentstroke}%
\pgfsetdash{}{0pt}%
\pgfpathmoveto{\pgfqpoint{5.241257in}{2.484439in}}%
\pgfpathlineto{\pgfqpoint{5.325838in}{2.803195in}}%
\pgfpathlineto{\pgfqpoint{5.094264in}{2.487966in}}%
\pgfpathclose%
\pgfusepath{fill}%
\end{pgfscope}%
\begin{pgfscope}%
\pgfpathrectangle{\pgfqpoint{0.680860in}{0.078740in}}{\pgfqpoint{7.842520in}{7.842520in}}%
\pgfusepath{clip}%
\pgfsetbuttcap%
\pgfsetroundjoin%
\definecolor{currentfill}{rgb}{0.137339,0.662252,0.515571}%
\pgfsetfillcolor{currentfill}%
\pgfsetlinewidth{0.000000pt}%
\definecolor{currentstroke}{rgb}{0.125394,0.574318,0.549086}%
\pgfsetstrokecolor{currentstroke}%
\pgfsetdash{}{0pt}%
\pgfpathmoveto{\pgfqpoint{5.346837in}{3.373178in}}%
\pgfpathlineto{\pgfqpoint{5.494660in}{3.392671in}}%
\pgfpathlineto{\pgfqpoint{5.578737in}{3.658750in}}%
\pgfpathclose%
\pgfusepath{fill}%
\end{pgfscope}%
\begin{pgfscope}%
\pgfpathrectangle{\pgfqpoint{0.680860in}{0.078740in}}{\pgfqpoint{7.842520in}{7.842520in}}%
\pgfusepath{clip}%
\pgfsetbuttcap%
\pgfsetroundjoin%
\definecolor{currentfill}{rgb}{0.136408,0.541173,0.554483}%
\pgfsetfillcolor{currentfill}%
\pgfsetlinewidth{0.000000pt}%
\definecolor{currentstroke}{rgb}{0.124395,0.578002,0.548287}%
\pgfsetstrokecolor{currentstroke}%
\pgfsetdash{}{0pt}%
\pgfpathmoveto{\pgfqpoint{5.178549in}{2.798575in}}%
\pgfpathlineto{\pgfqpoint{5.325838in}{2.803195in}}%
\pgfpathlineto{\pgfqpoint{5.410332in}{3.106740in}}%
\pgfpathclose%
\pgfusepath{fill}%
\end{pgfscope}%
\begin{pgfscope}%
\pgfpathrectangle{\pgfqpoint{0.680860in}{0.078740in}}{\pgfqpoint{7.842520in}{7.842520in}}%
\pgfusepath{clip}%
\pgfsetbuttcap%
\pgfsetroundjoin%
\definecolor{currentfill}{rgb}{0.119738,0.603785,0.541400}%
\pgfsetfillcolor{currentfill}%
\pgfsetlinewidth{0.000000pt}%
\definecolor{currentstroke}{rgb}{0.123463,0.581687,0.547445}%
\pgfsetstrokecolor{currentstroke}%
\pgfsetdash{}{0pt}%
\pgfpathmoveto{\pgfqpoint{5.494660in}{3.392671in}}%
\pgfpathlineto{\pgfqpoint{5.262763in}{3.094416in}}%
\pgfpathlineto{\pgfqpoint{5.410332in}{3.106740in}}%
\pgfpathclose%
\pgfusepath{fill}%
\end{pgfscope}%
\begin{pgfscope}%
\pgfpathrectangle{\pgfqpoint{0.680860in}{0.078740in}}{\pgfqpoint{7.842520in}{7.842520in}}%
\pgfusepath{clip}%
\pgfsetbuttcap%
\pgfsetroundjoin%
\definecolor{currentfill}{rgb}{0.231674,0.318106,0.544834}%
\pgfsetfillcolor{currentfill}%
\pgfsetlinewidth{0.000000pt}%
\definecolor{currentstroke}{rgb}{0.122606,0.585371,0.546557}%
\pgfsetstrokecolor{currentstroke}%
\pgfsetdash{}{0pt}%
\pgfpathmoveto{\pgfqpoint{5.156658in}{2.152997in}}%
\pgfpathlineto{\pgfqpoint{5.072097in}{1.811488in}}%
\pgfpathlineto{\pgfqpoint{5.219087in}{1.791417in}}%
\pgfpathclose%
\pgfusepath{fill}%
\end{pgfscope}%
\begin{pgfscope}%
\pgfpathrectangle{\pgfqpoint{0.680860in}{0.078740in}}{\pgfqpoint{7.842520in}{7.842520in}}%
\pgfusepath{clip}%
\pgfsetbuttcap%
\pgfsetroundjoin%
\definecolor{currentfill}{rgb}{0.647257,0.858400,0.209861}%
\pgfsetfillcolor{currentfill}%
\pgfsetlinewidth{0.000000pt}%
\definecolor{currentstroke}{rgb}{0.121831,0.589055,0.545623}%
\pgfsetstrokecolor{currentstroke}%
\pgfsetdash{}{0pt}%
\pgfpathmoveto{\pgfqpoint{6.291683in}{4.731755in}}%
\pgfpathlineto{\pgfqpoint{6.060019in}{4.534803in}}%
\pgfpathlineto{\pgfqpoint{6.210344in}{4.585506in}}%
\pgfpathclose%
\pgfusepath{fill}%
\end{pgfscope}%
\begin{pgfscope}%
\pgfpathrectangle{\pgfqpoint{0.680860in}{0.078740in}}{\pgfqpoint{7.842520in}{7.842520in}}%
\pgfusepath{clip}%
\pgfsetbuttcap%
\pgfsetroundjoin%
\definecolor{currentfill}{rgb}{0.257322,0.256130,0.526563}%
\pgfsetfillcolor{currentfill}%
\pgfsetlinewidth{0.000000pt}%
\definecolor{currentstroke}{rgb}{0.121148,0.592739,0.544641}%
\pgfsetstrokecolor{currentstroke}%
\pgfsetdash{}{0pt}%
\pgfpathmoveto{\pgfqpoint{5.134266in}{1.433296in}}%
\pgfpathlineto{\pgfqpoint{5.219087in}{1.791417in}}%
\pgfpathlineto{\pgfqpoint{5.072097in}{1.811488in}}%
\pgfpathclose%
\pgfusepath{fill}%
\end{pgfscope}%
\begin{pgfscope}%
\pgfpathrectangle{\pgfqpoint{0.680860in}{0.078740in}}{\pgfqpoint{7.842520in}{7.842520in}}%
\pgfusepath{clip}%
\pgfsetbuttcap%
\pgfsetroundjoin%
\definecolor{currentfill}{rgb}{0.421908,0.805774,0.351910}%
\pgfsetfillcolor{currentfill}%
\pgfsetlinewidth{0.000000pt}%
\definecolor{currentstroke}{rgb}{0.120565,0.596422,0.543611}%
\pgfsetstrokecolor{currentstroke}%
\pgfsetdash{}{0pt}%
\pgfpathmoveto{\pgfqpoint{5.745764in}{4.123394in}}%
\pgfpathlineto{\pgfqpoint{5.895003in}{4.163095in}}%
\pgfpathlineto{\pgfqpoint{5.977862in}{4.362686in}}%
\pgfpathclose%
\pgfusepath{fill}%
\end{pgfscope}%
\begin{pgfscope}%
\pgfpathrectangle{\pgfqpoint{0.680860in}{0.078740in}}{\pgfqpoint{7.842520in}{7.842520in}}%
\pgfusepath{clip}%
\pgfsetbuttcap%
\pgfsetroundjoin%
\definecolor{currentfill}{rgb}{0.575563,0.844566,0.256415}%
\pgfsetfillcolor{currentfill}%
\pgfsetlinewidth{0.000000pt}%
\definecolor{currentstroke}{rgb}{0.120092,0.600104,0.542530}%
\pgfsetstrokecolor{currentstroke}%
\pgfsetdash{}{0pt}%
\pgfpathmoveto{\pgfqpoint{6.210344in}{4.585506in}}%
\pgfpathlineto{\pgfqpoint{6.060019in}{4.534803in}}%
\pgfpathlineto{\pgfqpoint{5.977862in}{4.362686in}}%
\pgfpathclose%
\pgfusepath{fill}%
\end{pgfscope}%
\begin{pgfscope}%
\pgfpathrectangle{\pgfqpoint{0.680860in}{0.078740in}}{\pgfqpoint{7.842520in}{7.842520in}}%
\pgfusepath{clip}%
\pgfsetbuttcap%
\pgfsetroundjoin%
\definecolor{currentfill}{rgb}{0.304148,0.764704,0.419943}%
\pgfsetfillcolor{currentfill}%
\pgfsetlinewidth{0.000000pt}%
\definecolor{currentstroke}{rgb}{0.119738,0.603785,0.541400}%
\pgfsetstrokecolor{currentstroke}%
\pgfsetdash{}{0pt}%
\pgfpathmoveto{\pgfqpoint{5.662470in}{3.902930in}}%
\pgfpathlineto{\pgfqpoint{5.811549in}{3.937365in}}%
\pgfpathlineto{\pgfqpoint{5.745764in}{4.123394in}}%
\pgfpathclose%
\pgfusepath{fill}%
\end{pgfscope}%
\begin{pgfscope}%
\pgfpathrectangle{\pgfqpoint{0.680860in}{0.078740in}}{\pgfqpoint{7.842520in}{7.842520in}}%
\pgfusepath{clip}%
\pgfsetbuttcap%
\pgfsetroundjoin%
\definecolor{currentfill}{rgb}{0.855810,0.888601,0.097452}%
\pgfsetfillcolor{currentfill}%
\pgfsetlinewidth{0.000000pt}%
\definecolor{currentstroke}{rgb}{0.119512,0.607464,0.540218}%
\pgfsetstrokecolor{currentstroke}%
\pgfsetdash{}{0pt}%
\pgfpathmoveto{\pgfqpoint{6.602351in}{4.992168in}}%
\pgfpathlineto{\pgfqpoint{6.523246in}{4.906102in}}%
\pgfpathlineto{\pgfqpoint{6.675467in}{4.967908in}}%
\pgfpathclose%
\pgfusepath{fill}%
\end{pgfscope}%
\begin{pgfscope}%
\pgfpathrectangle{\pgfqpoint{0.680860in}{0.078740in}}{\pgfqpoint{7.842520in}{7.842520in}}%
\pgfusepath{clip}%
\pgfsetbuttcap%
\pgfsetroundjoin%
\definecolor{currentfill}{rgb}{0.281412,0.155834,0.469201}%
\pgfsetfillcolor{currentfill}%
\pgfsetlinewidth{0.000000pt}%
\definecolor{currentstroke}{rgb}{0.119423,0.611141,0.538982}%
\pgfsetstrokecolor{currentstroke}%
\pgfsetdash{}{0pt}%
\pgfpathmoveto{\pgfqpoint{5.049577in}{1.070410in}}%
\pgfpathlineto{\pgfqpoint{5.281493in}{1.404441in}}%
\pgfpathlineto{\pgfqpoint{5.134266in}{1.433296in}}%
\pgfpathclose%
\pgfusepath{fill}%
\end{pgfscope}%
\begin{pgfscope}%
\pgfpathrectangle{\pgfqpoint{0.680860in}{0.078740in}}{\pgfqpoint{7.842520in}{7.842520in}}%
\pgfusepath{clip}%
\pgfsetbuttcap%
\pgfsetroundjoin%
\definecolor{currentfill}{rgb}{0.886271,0.892374,0.095374}%
\pgfsetfillcolor{currentfill}%
\pgfsetlinewidth{0.000000pt}%
\definecolor{currentstroke}{rgb}{0.119483,0.614817,0.537692}%
\pgfsetstrokecolor{currentstroke}%
\pgfsetdash{}{0pt}%
\pgfpathmoveto{\pgfqpoint{6.675467in}{4.967908in}}%
\pgfpathlineto{\pgfqpoint{6.754408in}{5.054566in}}%
\pgfpathlineto{\pgfqpoint{6.602351in}{4.992168in}}%
\pgfpathclose%
\pgfusepath{fill}%
\end{pgfscope}%
\begin{pgfscope}%
\pgfpathrectangle{\pgfqpoint{0.680860in}{0.078740in}}{\pgfqpoint{7.842520in}{7.842520in}}%
\pgfusepath{clip}%
\pgfsetbuttcap%
\pgfsetroundjoin%
\definecolor{currentfill}{rgb}{0.283229,0.120777,0.440584}%
\pgfsetfillcolor{currentfill}%
\pgfsetlinewidth{0.000000pt}%
\definecolor{currentstroke}{rgb}{0.119699,0.618490,0.536347}%
\pgfsetstrokecolor{currentstroke}%
\pgfsetdash{}{0pt}%
\pgfpathmoveto{\pgfqpoint{5.196426in}{1.031961in}}%
\pgfpathlineto{\pgfqpoint{5.281493in}{1.404441in}}%
\pgfpathlineto{\pgfqpoint{5.049577in}{1.070410in}}%
\pgfpathclose%
\pgfusepath{fill}%
\end{pgfscope}%
\begin{pgfscope}%
\pgfpathrectangle{\pgfqpoint{0.680860in}{0.078740in}}{\pgfqpoint{7.842520in}{7.842520in}}%
\pgfusepath{clip}%
\pgfsetbuttcap%
\pgfsetroundjoin%
\definecolor{currentfill}{rgb}{0.246070,0.738910,0.452024}%
\pgfsetfillcolor{currentfill}%
\pgfsetlinewidth{0.000000pt}%
\definecolor{currentstroke}{rgb}{0.120081,0.622161,0.534946}%
\pgfsetstrokecolor{currentstroke}%
\pgfsetdash{}{0pt}%
\pgfpathmoveto{\pgfqpoint{5.578737in}{3.658750in}}%
\pgfpathlineto{\pgfqpoint{5.811549in}{3.937365in}}%
\pgfpathlineto{\pgfqpoint{5.662470in}{3.902930in}}%
\pgfpathclose%
\pgfusepath{fill}%
\end{pgfscope}%
\begin{pgfscope}%
\pgfpathrectangle{\pgfqpoint{0.680860in}{0.078740in}}{\pgfqpoint{7.842520in}{7.842520in}}%
\pgfusepath{clip}%
\pgfsetbuttcap%
\pgfsetroundjoin%
\definecolor{currentfill}{rgb}{0.180629,0.429975,0.557282}%
\pgfsetfillcolor{currentfill}%
\pgfsetlinewidth{0.000000pt}%
\definecolor{currentstroke}{rgb}{0.120638,0.625828,0.533488}%
\pgfsetstrokecolor{currentstroke}%
\pgfsetdash{}{0pt}%
\pgfpathmoveto{\pgfqpoint{5.388941in}{2.482208in}}%
\pgfpathlineto{\pgfqpoint{5.241257in}{2.484439in}}%
\pgfpathlineto{\pgfqpoint{5.156658in}{2.152997in}}%
\pgfpathclose%
\pgfusepath{fill}%
\end{pgfscope}%
\begin{pgfscope}%
\pgfpathrectangle{\pgfqpoint{0.680860in}{0.078740in}}{\pgfqpoint{7.842520in}{7.842520in}}%
\pgfusepath{clip}%
\pgfsetbuttcap%
\pgfsetroundjoin%
\definecolor{currentfill}{rgb}{0.772852,0.877868,0.131109}%
\pgfsetfillcolor{currentfill}%
\pgfsetlinewidth{0.000000pt}%
\definecolor{currentstroke}{rgb}{0.121380,0.629492,0.531973}%
\pgfsetstrokecolor{currentstroke}%
\pgfsetdash{}{0pt}%
\pgfpathmoveto{\pgfqpoint{6.291683in}{4.731755in}}%
\pgfpathlineto{\pgfqpoint{6.442972in}{4.788400in}}%
\pgfpathlineto{\pgfqpoint{6.523246in}{4.906102in}}%
\pgfpathclose%
\pgfusepath{fill}%
\end{pgfscope}%
\begin{pgfscope}%
\pgfpathrectangle{\pgfqpoint{0.680860in}{0.078740in}}{\pgfqpoint{7.842520in}{7.842520in}}%
\pgfusepath{clip}%
\pgfsetbuttcap%
\pgfsetroundjoin%
\definecolor{currentfill}{rgb}{0.352360,0.783011,0.392636}%
\pgfsetfillcolor{currentfill}%
\pgfsetlinewidth{0.000000pt}%
\definecolor{currentstroke}{rgb}{0.122312,0.633153,0.530398}%
\pgfsetstrokecolor{currentstroke}%
\pgfsetdash{}{0pt}%
\pgfpathmoveto{\pgfqpoint{5.811549in}{3.937365in}}%
\pgfpathlineto{\pgfqpoint{5.895003in}{4.163095in}}%
\pgfpathlineto{\pgfqpoint{5.745764in}{4.123394in}}%
\pgfpathclose%
\pgfusepath{fill}%
\end{pgfscope}%
\begin{pgfscope}%
\pgfpathrectangle{\pgfqpoint{0.680860in}{0.078740in}}{\pgfqpoint{7.842520in}{7.842520in}}%
\pgfusepath{clip}%
\pgfsetbuttcap%
\pgfsetroundjoin%
\definecolor{currentfill}{rgb}{0.216210,0.351535,0.550627}%
\pgfsetfillcolor{currentfill}%
\pgfsetlinewidth{0.000000pt}%
\definecolor{currentstroke}{rgb}{0.123444,0.636809,0.528763}%
\pgfsetstrokecolor{currentstroke}%
\pgfsetdash{}{0pt}%
\pgfpathmoveto{\pgfqpoint{5.219087in}{1.791417in}}%
\pgfpathlineto{\pgfqpoint{5.303996in}{2.141978in}}%
\pgfpathlineto{\pgfqpoint{5.156658in}{2.152997in}}%
\pgfpathclose%
\pgfusepath{fill}%
\end{pgfscope}%
\begin{pgfscope}%
\pgfpathrectangle{\pgfqpoint{0.680860in}{0.078740in}}{\pgfqpoint{7.842520in}{7.842520in}}%
\pgfusepath{clip}%
\pgfsetbuttcap%
\pgfsetroundjoin%
\definecolor{currentfill}{rgb}{0.266580,0.228262,0.514349}%
\pgfsetfillcolor{currentfill}%
\pgfsetlinewidth{0.000000pt}%
\definecolor{currentstroke}{rgb}{0.124780,0.640461,0.527068}%
\pgfsetstrokecolor{currentstroke}%
\pgfsetdash{}{0pt}%
\pgfpathmoveto{\pgfqpoint{5.134266in}{1.433296in}}%
\pgfpathlineto{\pgfqpoint{5.281493in}{1.404441in}}%
\pgfpathlineto{\pgfqpoint{5.219087in}{1.791417in}}%
\pgfpathclose%
\pgfusepath{fill}%
\end{pgfscope}%
\begin{pgfscope}%
\pgfpathrectangle{\pgfqpoint{0.680860in}{0.078740in}}{\pgfqpoint{7.842520in}{7.842520in}}%
\pgfusepath{clip}%
\pgfsetbuttcap%
\pgfsetroundjoin%
\definecolor{currentfill}{rgb}{0.150476,0.504369,0.557430}%
\pgfsetfillcolor{currentfill}%
\pgfsetlinewidth{0.000000pt}%
\definecolor{currentstroke}{rgb}{0.126326,0.644107,0.525311}%
\pgfsetstrokecolor{currentstroke}%
\pgfsetdash{}{0pt}%
\pgfpathmoveto{\pgfqpoint{5.473854in}{2.809393in}}%
\pgfpathlineto{\pgfqpoint{5.325838in}{2.803195in}}%
\pgfpathlineto{\pgfqpoint{5.241257in}{2.484439in}}%
\pgfpathclose%
\pgfusepath{fill}%
\end{pgfscope}%
\begin{pgfscope}%
\pgfpathrectangle{\pgfqpoint{0.680860in}{0.078740in}}{\pgfqpoint{7.842520in}{7.842520in}}%
\pgfusepath{clip}%
\pgfsetbuttcap%
\pgfsetroundjoin%
\definecolor{currentfill}{rgb}{0.190631,0.407061,0.556089}%
\pgfsetfillcolor{currentfill}%
\pgfsetlinewidth{0.000000pt}%
\definecolor{currentstroke}{rgb}{0.128087,0.647749,0.523491}%
\pgfsetstrokecolor{currentstroke}%
\pgfsetdash{}{0pt}%
\pgfpathmoveto{\pgfqpoint{5.156658in}{2.152997in}}%
\pgfpathlineto{\pgfqpoint{5.303996in}{2.141978in}}%
\pgfpathlineto{\pgfqpoint{5.388941in}{2.482208in}}%
\pgfpathclose%
\pgfusepath{fill}%
\end{pgfscope}%
\begin{pgfscope}%
\pgfpathrectangle{\pgfqpoint{0.680860in}{0.078740in}}{\pgfqpoint{7.842520in}{7.842520in}}%
\pgfusepath{clip}%
\pgfsetbuttcap%
\pgfsetroundjoin%
\definecolor{currentfill}{rgb}{0.146616,0.673050,0.508936}%
\pgfsetfillcolor{currentfill}%
\pgfsetlinewidth{0.000000pt}%
\definecolor{currentstroke}{rgb}{0.130067,0.651384,0.521608}%
\pgfsetstrokecolor{currentstroke}%
\pgfsetdash{}{0pt}%
\pgfpathmoveto{\pgfqpoint{5.578737in}{3.658750in}}%
\pgfpathlineto{\pgfqpoint{5.494660in}{3.392671in}}%
\pgfpathlineto{\pgfqpoint{5.643276in}{3.414261in}}%
\pgfpathclose%
\pgfusepath{fill}%
\end{pgfscope}%
\begin{pgfscope}%
\pgfpathrectangle{\pgfqpoint{0.680860in}{0.078740in}}{\pgfqpoint{7.842520in}{7.842520in}}%
\pgfusepath{clip}%
\pgfsetbuttcap%
\pgfsetroundjoin%
\definecolor{currentfill}{rgb}{0.133743,0.548535,0.553541}%
\pgfsetfillcolor{currentfill}%
\pgfsetlinewidth{0.000000pt}%
\definecolor{currentstroke}{rgb}{0.132268,0.655014,0.519661}%
\pgfsetstrokecolor{currentstroke}%
\pgfsetdash{}{0pt}%
\pgfpathmoveto{\pgfqpoint{5.410332in}{3.106740in}}%
\pgfpathlineto{\pgfqpoint{5.325838in}{2.803195in}}%
\pgfpathlineto{\pgfqpoint{5.473854in}{2.809393in}}%
\pgfpathclose%
\pgfusepath{fill}%
\end{pgfscope}%
\begin{pgfscope}%
\pgfpathrectangle{\pgfqpoint{0.680860in}{0.078740in}}{\pgfqpoint{7.842520in}{7.842520in}}%
\pgfusepath{clip}%
\pgfsetbuttcap%
\pgfsetroundjoin%
\definecolor{currentfill}{rgb}{0.119483,0.614817,0.537692}%
\pgfsetfillcolor{currentfill}%
\pgfsetlinewidth{0.000000pt}%
\definecolor{currentstroke}{rgb}{0.134692,0.658636,0.517649}%
\pgfsetstrokecolor{currentstroke}%
\pgfsetdash{}{0pt}%
\pgfpathmoveto{\pgfqpoint{5.410332in}{3.106740in}}%
\pgfpathlineto{\pgfqpoint{5.558661in}{3.120909in}}%
\pgfpathlineto{\pgfqpoint{5.494660in}{3.392671in}}%
\pgfpathclose%
\pgfusepath{fill}%
\end{pgfscope}%
\begin{pgfscope}%
\pgfpathrectangle{\pgfqpoint{0.680860in}{0.078740in}}{\pgfqpoint{7.842520in}{7.842520in}}%
\pgfusepath{clip}%
\pgfsetbuttcap%
\pgfsetroundjoin%
\definecolor{currentfill}{rgb}{0.487026,0.823929,0.312321}%
\pgfsetfillcolor{currentfill}%
\pgfsetlinewidth{0.000000pt}%
\definecolor{currentstroke}{rgb}{0.137339,0.662252,0.515571}%
\pgfsetstrokecolor{currentstroke}%
\pgfsetdash{}{0pt}%
\pgfpathmoveto{\pgfqpoint{6.128126in}{4.409774in}}%
\pgfpathlineto{\pgfqpoint{5.977862in}{4.362686in}}%
\pgfpathlineto{\pgfqpoint{5.895003in}{4.163095in}}%
\pgfpathclose%
\pgfusepath{fill}%
\end{pgfscope}%
\begin{pgfscope}%
\pgfpathrectangle{\pgfqpoint{0.680860in}{0.078740in}}{\pgfqpoint{7.842520in}{7.842520in}}%
\pgfusepath{clip}%
\pgfsetbuttcap%
\pgfsetroundjoin%
\definecolor{currentfill}{rgb}{0.575563,0.844566,0.256415}%
\pgfsetfillcolor{currentfill}%
\pgfsetlinewidth{0.000000pt}%
\definecolor{currentstroke}{rgb}{0.140210,0.665859,0.513427}%
\pgfsetstrokecolor{currentstroke}%
\pgfsetdash{}{0pt}%
\pgfpathmoveto{\pgfqpoint{5.977862in}{4.362686in}}%
\pgfpathlineto{\pgfqpoint{6.128126in}{4.409774in}}%
\pgfpathlineto{\pgfqpoint{6.210344in}{4.585506in}}%
\pgfpathclose%
\pgfusepath{fill}%
\end{pgfscope}%
\begin{pgfscope}%
\pgfpathrectangle{\pgfqpoint{0.680860in}{0.078740in}}{\pgfqpoint{7.842520in}{7.842520in}}%
\pgfusepath{clip}%
\pgfsetbuttcap%
\pgfsetroundjoin%
\definecolor{currentfill}{rgb}{0.964894,0.902323,0.123941}%
\pgfsetfillcolor{currentfill}%
\pgfsetlinewidth{0.000000pt}%
\definecolor{currentstroke}{rgb}{0.143303,0.669459,0.511215}%
\pgfsetstrokecolor{currentstroke}%
\pgfsetdash{}{0pt}%
\pgfpathmoveto{\pgfqpoint{6.984835in}{5.173977in}}%
\pgfpathlineto{\pgfqpoint{6.754408in}{5.054566in}}%
\pgfpathlineto{\pgfqpoint{6.907512in}{5.120623in}}%
\pgfpathclose%
\pgfusepath{fill}%
\end{pgfscope}%
\begin{pgfscope}%
\pgfpathrectangle{\pgfqpoint{0.680860in}{0.078740in}}{\pgfqpoint{7.842520in}{7.842520in}}%
\pgfusepath{clip}%
\pgfsetbuttcap%
\pgfsetroundjoin%
\definecolor{currentfill}{rgb}{0.232815,0.732247,0.459277}%
\pgfsetfillcolor{currentfill}%
\pgfsetlinewidth{0.000000pt}%
\definecolor{currentstroke}{rgb}{0.146616,0.673050,0.508936}%
\pgfsetstrokecolor{currentstroke}%
\pgfsetdash{}{0pt}%
\pgfpathmoveto{\pgfqpoint{5.578737in}{3.658750in}}%
\pgfpathlineto{\pgfqpoint{5.727605in}{3.687123in}}%
\pgfpathlineto{\pgfqpoint{5.811549in}{3.937365in}}%
\pgfpathclose%
\pgfusepath{fill}%
\end{pgfscope}%
\begin{pgfscope}%
\pgfpathrectangle{\pgfqpoint{0.680860in}{0.078740in}}{\pgfqpoint{7.842520in}{7.842520in}}%
\pgfusepath{clip}%
\pgfsetbuttcap%
\pgfsetroundjoin%
\definecolor{currentfill}{rgb}{0.157729,0.485932,0.558013}%
\pgfsetfillcolor{currentfill}%
\pgfsetlinewidth{0.000000pt}%
\definecolor{currentstroke}{rgb}{0.150148,0.676631,0.506589}%
\pgfsetstrokecolor{currentstroke}%
\pgfsetdash{}{0pt}%
\pgfpathmoveto{\pgfqpoint{5.473854in}{2.809393in}}%
\pgfpathlineto{\pgfqpoint{5.241257in}{2.484439in}}%
\pgfpathlineto{\pgfqpoint{5.388941in}{2.482208in}}%
\pgfpathclose%
\pgfusepath{fill}%
\end{pgfscope}%
\begin{pgfscope}%
\pgfpathrectangle{\pgfqpoint{0.680860in}{0.078740in}}{\pgfqpoint{7.842520in}{7.842520in}}%
\pgfusepath{clip}%
\pgfsetbuttcap%
\pgfsetroundjoin%
\definecolor{currentfill}{rgb}{0.699415,0.867117,0.175971}%
\pgfsetfillcolor{currentfill}%
\pgfsetlinewidth{0.000000pt}%
\definecolor{currentstroke}{rgb}{0.153894,0.680203,0.504172}%
\pgfsetstrokecolor{currentstroke}%
\pgfsetdash{}{0pt}%
\pgfpathmoveto{\pgfqpoint{6.210344in}{4.585506in}}%
\pgfpathlineto{\pgfqpoint{6.361631in}{4.639437in}}%
\pgfpathlineto{\pgfqpoint{6.291683in}{4.731755in}}%
\pgfpathclose%
\pgfusepath{fill}%
\end{pgfscope}%
\begin{pgfscope}%
\pgfpathrectangle{\pgfqpoint{0.680860in}{0.078740in}}{\pgfqpoint{7.842520in}{7.842520in}}%
\pgfusepath{clip}%
\pgfsetbuttcap%
\pgfsetroundjoin%
\definecolor{currentfill}{rgb}{0.253935,0.265254,0.529983}%
\pgfsetfillcolor{currentfill}%
\pgfsetlinewidth{0.000000pt}%
\definecolor{currentstroke}{rgb}{0.157851,0.683765,0.501686}%
\pgfsetstrokecolor{currentstroke}%
\pgfsetdash{}{0pt}%
\pgfpathmoveto{\pgfqpoint{5.219087in}{1.791417in}}%
\pgfpathlineto{\pgfqpoint{5.281493in}{1.404441in}}%
\pgfpathlineto{\pgfqpoint{5.366700in}{1.772086in}}%
\pgfpathclose%
\pgfusepath{fill}%
\end{pgfscope}%
\begin{pgfscope}%
\pgfpathrectangle{\pgfqpoint{0.680860in}{0.078740in}}{\pgfqpoint{7.842520in}{7.842520in}}%
\pgfusepath{clip}%
\pgfsetbuttcap%
\pgfsetroundjoin%
\definecolor{currentfill}{rgb}{0.845561,0.887322,0.099702}%
\pgfsetfillcolor{currentfill}%
\pgfsetlinewidth{0.000000pt}%
\definecolor{currentstroke}{rgb}{0.162016,0.687316,0.499129}%
\pgfsetstrokecolor{currentstroke}%
\pgfsetdash{}{0pt}%
\pgfpathmoveto{\pgfqpoint{6.523246in}{4.906102in}}%
\pgfpathlineto{\pgfqpoint{6.442972in}{4.788400in}}%
\pgfpathlineto{\pgfqpoint{6.675467in}{4.967908in}}%
\pgfpathclose%
\pgfusepath{fill}%
\end{pgfscope}%
\begin{pgfscope}%
\pgfpathrectangle{\pgfqpoint{0.680860in}{0.078740in}}{\pgfqpoint{7.842520in}{7.842520in}}%
\pgfusepath{clip}%
\pgfsetbuttcap%
\pgfsetroundjoin%
\definecolor{currentfill}{rgb}{0.175707,0.697900,0.491033}%
\pgfsetfillcolor{currentfill}%
\pgfsetlinewidth{0.000000pt}%
\definecolor{currentstroke}{rgb}{0.166383,0.690856,0.496502}%
\pgfsetstrokecolor{currentstroke}%
\pgfsetdash{}{0pt}%
\pgfpathmoveto{\pgfqpoint{5.643276in}{3.414261in}}%
\pgfpathlineto{\pgfqpoint{5.727605in}{3.687123in}}%
\pgfpathlineto{\pgfqpoint{5.578737in}{3.658750in}}%
\pgfpathclose%
\pgfusepath{fill}%
\end{pgfscope}%
\begin{pgfscope}%
\pgfpathrectangle{\pgfqpoint{0.680860in}{0.078740in}}{\pgfqpoint{7.842520in}{7.842520in}}%
\pgfusepath{clip}%
\pgfsetbuttcap%
\pgfsetroundjoin%
\definecolor{currentfill}{rgb}{0.741388,0.873449,0.149561}%
\pgfsetfillcolor{currentfill}%
\pgfsetlinewidth{0.000000pt}%
\definecolor{currentstroke}{rgb}{0.170948,0.694384,0.493803}%
\pgfsetstrokecolor{currentstroke}%
\pgfsetdash{}{0pt}%
\pgfpathmoveto{\pgfqpoint{6.361631in}{4.639437in}}%
\pgfpathlineto{\pgfqpoint{6.442972in}{4.788400in}}%
\pgfpathlineto{\pgfqpoint{6.291683in}{4.731755in}}%
\pgfpathclose%
\pgfusepath{fill}%
\end{pgfscope}%
\begin{pgfscope}%
\pgfpathrectangle{\pgfqpoint{0.680860in}{0.078740in}}{\pgfqpoint{7.842520in}{7.842520in}}%
\pgfusepath{clip}%
\pgfsetbuttcap%
\pgfsetroundjoin%
\definecolor{currentfill}{rgb}{0.124395,0.578002,0.548287}%
\pgfsetfillcolor{currentfill}%
\pgfsetlinewidth{0.000000pt}%
\definecolor{currentstroke}{rgb}{0.175707,0.697900,0.491033}%
\pgfsetstrokecolor{currentstroke}%
\pgfsetdash{}{0pt}%
\pgfpathmoveto{\pgfqpoint{5.473854in}{2.809393in}}%
\pgfpathlineto{\pgfqpoint{5.558661in}{3.120909in}}%
\pgfpathlineto{\pgfqpoint{5.410332in}{3.106740in}}%
\pgfpathclose%
\pgfusepath{fill}%
\end{pgfscope}%
\begin{pgfscope}%
\pgfpathrectangle{\pgfqpoint{0.680860in}{0.078740in}}{\pgfqpoint{7.842520in}{7.842520in}}%
\pgfusepath{clip}%
\pgfsetbuttcap%
\pgfsetroundjoin%
\definecolor{currentfill}{rgb}{0.212395,0.359683,0.551710}%
\pgfsetfillcolor{currentfill}%
\pgfsetlinewidth{0.000000pt}%
\definecolor{currentstroke}{rgb}{0.180653,0.701402,0.488189}%
\pgfsetstrokecolor{currentstroke}%
\pgfsetdash{}{0pt}%
\pgfpathmoveto{\pgfqpoint{5.451999in}{2.132005in}}%
\pgfpathlineto{\pgfqpoint{5.303996in}{2.141978in}}%
\pgfpathlineto{\pgfqpoint{5.219087in}{1.791417in}}%
\pgfpathclose%
\pgfusepath{fill}%
\end{pgfscope}%
\begin{pgfscope}%
\pgfpathrectangle{\pgfqpoint{0.680860in}{0.078740in}}{\pgfqpoint{7.842520in}{7.842520in}}%
\pgfusepath{clip}%
\pgfsetbuttcap%
\pgfsetroundjoin%
\definecolor{currentfill}{rgb}{0.124780,0.640461,0.527068}%
\pgfsetfillcolor{currentfill}%
\pgfsetlinewidth{0.000000pt}%
\definecolor{currentstroke}{rgb}{0.185783,0.704891,0.485273}%
\pgfsetstrokecolor{currentstroke}%
\pgfsetdash{}{0pt}%
\pgfpathmoveto{\pgfqpoint{5.558661in}{3.120909in}}%
\pgfpathlineto{\pgfqpoint{5.643276in}{3.414261in}}%
\pgfpathlineto{\pgfqpoint{5.494660in}{3.392671in}}%
\pgfpathclose%
\pgfusepath{fill}%
\end{pgfscope}%
\begin{pgfscope}%
\pgfpathrectangle{\pgfqpoint{0.680860in}{0.078740in}}{\pgfqpoint{7.842520in}{7.842520in}}%
\pgfusepath{clip}%
\pgfsetbuttcap%
\pgfsetroundjoin%
\definecolor{currentfill}{rgb}{0.280868,0.160771,0.472899}%
\pgfsetfillcolor{currentfill}%
\pgfsetlinewidth{0.000000pt}%
\definecolor{currentstroke}{rgb}{0.191090,0.708366,0.482284}%
\pgfsetstrokecolor{currentstroke}%
\pgfsetdash{}{0pt}%
\pgfpathmoveto{\pgfqpoint{5.429309in}{1.376042in}}%
\pgfpathlineto{\pgfqpoint{5.281493in}{1.404441in}}%
\pgfpathlineto{\pgfqpoint{5.196426in}{1.031961in}}%
\pgfpathclose%
\pgfusepath{fill}%
\end{pgfscope}%
\begin{pgfscope}%
\pgfpathrectangle{\pgfqpoint{0.680860in}{0.078740in}}{\pgfqpoint{7.842520in}{7.842520in}}%
\pgfusepath{clip}%
\pgfsetbuttcap%
\pgfsetroundjoin%
\definecolor{currentfill}{rgb}{0.945636,0.899815,0.112838}%
\pgfsetfillcolor{currentfill}%
\pgfsetlinewidth{0.000000pt}%
\definecolor{currentstroke}{rgb}{0.196571,0.711827,0.479221}%
\pgfsetstrokecolor{currentstroke}%
\pgfsetdash{}{0pt}%
\pgfpathmoveto{\pgfqpoint{6.907512in}{5.120623in}}%
\pgfpathlineto{\pgfqpoint{6.754408in}{5.054566in}}%
\pgfpathlineto{\pgfqpoint{6.675467in}{4.967908in}}%
\pgfpathclose%
\pgfusepath{fill}%
\end{pgfscope}%
\begin{pgfscope}%
\pgfpathrectangle{\pgfqpoint{0.680860in}{0.078740in}}{\pgfqpoint{7.842520in}{7.842520in}}%
\pgfusepath{clip}%
\pgfsetbuttcap%
\pgfsetroundjoin%
\definecolor{currentfill}{rgb}{0.283187,0.125848,0.444960}%
\pgfsetfillcolor{currentfill}%
\pgfsetlinewidth{0.000000pt}%
\definecolor{currentstroke}{rgb}{0.202219,0.715272,0.476084}%
\pgfsetstrokecolor{currentstroke}%
\pgfsetdash{}{0pt}%
\pgfpathmoveto{\pgfqpoint{5.196426in}{1.031961in}}%
\pgfpathlineto{\pgfqpoint{5.343822in}{0.993654in}}%
\pgfpathlineto{\pgfqpoint{5.429309in}{1.376042in}}%
\pgfpathclose%
\pgfusepath{fill}%
\end{pgfscope}%
\begin{pgfscope}%
\pgfpathrectangle{\pgfqpoint{0.680860in}{0.078740in}}{\pgfqpoint{7.842520in}{7.842520in}}%
\pgfusepath{clip}%
\pgfsetbuttcap%
\pgfsetroundjoin%
\definecolor{currentfill}{rgb}{0.225863,0.330805,0.547314}%
\pgfsetfillcolor{currentfill}%
\pgfsetlinewidth{0.000000pt}%
\definecolor{currentstroke}{rgb}{0.208030,0.718701,0.472873}%
\pgfsetstrokecolor{currentstroke}%
\pgfsetdash{}{0pt}%
\pgfpathmoveto{\pgfqpoint{5.451999in}{2.132005in}}%
\pgfpathlineto{\pgfqpoint{5.219087in}{1.791417in}}%
\pgfpathlineto{\pgfqpoint{5.366700in}{1.772086in}}%
\pgfpathclose%
\pgfusepath{fill}%
\end{pgfscope}%
\begin{pgfscope}%
\pgfpathrectangle{\pgfqpoint{0.680860in}{0.078740in}}{\pgfqpoint{7.842520in}{7.842520in}}%
\pgfusepath{clip}%
\pgfsetbuttcap%
\pgfsetroundjoin%
\definecolor{currentfill}{rgb}{0.187231,0.414746,0.556547}%
\pgfsetfillcolor{currentfill}%
\pgfsetlinewidth{0.000000pt}%
\definecolor{currentstroke}{rgb}{0.214000,0.722114,0.469588}%
\pgfsetstrokecolor{currentstroke}%
\pgfsetdash{}{0pt}%
\pgfpathmoveto{\pgfqpoint{5.388941in}{2.482208in}}%
\pgfpathlineto{\pgfqpoint{5.303996in}{2.141978in}}%
\pgfpathlineto{\pgfqpoint{5.451999in}{2.132005in}}%
\pgfpathclose%
\pgfusepath{fill}%
\end{pgfscope}%
\begin{pgfscope}%
\pgfpathrectangle{\pgfqpoint{0.680860in}{0.078740in}}{\pgfqpoint{7.842520in}{7.842520in}}%
\pgfusepath{clip}%
\pgfsetbuttcap%
\pgfsetroundjoin%
\definecolor{currentfill}{rgb}{0.477504,0.821444,0.318195}%
\pgfsetfillcolor{currentfill}%
\pgfsetlinewidth{0.000000pt}%
\definecolor{currentstroke}{rgb}{0.220124,0.725509,0.466226}%
\pgfsetstrokecolor{currentstroke}%
\pgfsetdash{}{0pt}%
\pgfpathmoveto{\pgfqpoint{6.045140in}{4.205620in}}%
\pgfpathlineto{\pgfqpoint{6.128126in}{4.409774in}}%
\pgfpathlineto{\pgfqpoint{5.895003in}{4.163095in}}%
\pgfpathclose%
\pgfusepath{fill}%
\end{pgfscope}%
\begin{pgfscope}%
\pgfpathrectangle{\pgfqpoint{0.680860in}{0.078740in}}{\pgfqpoint{7.842520in}{7.842520in}}%
\pgfusepath{clip}%
\pgfsetbuttcap%
\pgfsetroundjoin%
\definecolor{currentfill}{rgb}{0.360741,0.785964,0.387814}%
\pgfsetfillcolor{currentfill}%
\pgfsetlinewidth{0.000000pt}%
\definecolor{currentstroke}{rgb}{0.226397,0.728888,0.462789}%
\pgfsetstrokecolor{currentstroke}%
\pgfsetdash{}{0pt}%
\pgfpathmoveto{\pgfqpoint{5.895003in}{4.163095in}}%
\pgfpathlineto{\pgfqpoint{5.811549in}{3.937365in}}%
\pgfpathlineto{\pgfqpoint{5.961500in}{3.974435in}}%
\pgfpathclose%
\pgfusepath{fill}%
\end{pgfscope}%
\begin{pgfscope}%
\pgfpathrectangle{\pgfqpoint{0.680860in}{0.078740in}}{\pgfqpoint{7.842520in}{7.842520in}}%
\pgfusepath{clip}%
\pgfsetbuttcap%
\pgfsetroundjoin%
\definecolor{currentfill}{rgb}{0.647257,0.858400,0.209861}%
\pgfsetfillcolor{currentfill}%
\pgfsetlinewidth{0.000000pt}%
\definecolor{currentstroke}{rgb}{0.232815,0.732247,0.459277}%
\pgfsetstrokecolor{currentstroke}%
\pgfsetdash{}{0pt}%
\pgfpathmoveto{\pgfqpoint{6.210344in}{4.585506in}}%
\pgfpathlineto{\pgfqpoint{6.128126in}{4.409774in}}%
\pgfpathlineto{\pgfqpoint{6.361631in}{4.639437in}}%
\pgfpathclose%
\pgfusepath{fill}%
\end{pgfscope}%
\begin{pgfscope}%
\pgfpathrectangle{\pgfqpoint{0.680860in}{0.078740in}}{\pgfqpoint{7.842520in}{7.842520in}}%
\pgfusepath{clip}%
\pgfsetbuttcap%
\pgfsetroundjoin%
\definecolor{currentfill}{rgb}{0.263663,0.237631,0.518762}%
\pgfsetfillcolor{currentfill}%
\pgfsetlinewidth{0.000000pt}%
\definecolor{currentstroke}{rgb}{0.239374,0.735588,0.455688}%
\pgfsetstrokecolor{currentstroke}%
\pgfsetdash{}{0pt}%
\pgfpathmoveto{\pgfqpoint{5.366700in}{1.772086in}}%
\pgfpathlineto{\pgfqpoint{5.281493in}{1.404441in}}%
\pgfpathlineto{\pgfqpoint{5.429309in}{1.376042in}}%
\pgfpathclose%
\pgfusepath{fill}%
\end{pgfscope}%
\begin{pgfscope}%
\pgfpathrectangle{\pgfqpoint{0.680860in}{0.078740in}}{\pgfqpoint{7.842520in}{7.842520in}}%
\pgfusepath{clip}%
\pgfsetbuttcap%
\pgfsetroundjoin%
\definecolor{currentfill}{rgb}{0.288921,0.758394,0.428426}%
\pgfsetfillcolor{currentfill}%
\pgfsetlinewidth{0.000000pt}%
\definecolor{currentstroke}{rgb}{0.246070,0.738910,0.452024}%
\pgfsetstrokecolor{currentstroke}%
\pgfsetdash{}{0pt}%
\pgfpathmoveto{\pgfqpoint{5.961500in}{3.974435in}}%
\pgfpathlineto{\pgfqpoint{5.811549in}{3.937365in}}%
\pgfpathlineto{\pgfqpoint{5.727605in}{3.687123in}}%
\pgfpathclose%
\pgfusepath{fill}%
\end{pgfscope}%
\begin{pgfscope}%
\pgfpathrectangle{\pgfqpoint{0.680860in}{0.078740in}}{\pgfqpoint{7.842520in}{7.842520in}}%
\pgfusepath{clip}%
\pgfsetbuttcap%
\pgfsetroundjoin%
\definecolor{currentfill}{rgb}{0.144759,0.519093,0.556572}%
\pgfsetfillcolor{currentfill}%
\pgfsetlinewidth{0.000000pt}%
\definecolor{currentstroke}{rgb}{0.252899,0.742211,0.448284}%
\pgfsetstrokecolor{currentstroke}%
\pgfsetdash{}{0pt}%
\pgfpathmoveto{\pgfqpoint{5.622612in}{2.817228in}}%
\pgfpathlineto{\pgfqpoint{5.473854in}{2.809393in}}%
\pgfpathlineto{\pgfqpoint{5.388941in}{2.482208in}}%
\pgfpathclose%
\pgfusepath{fill}%
\end{pgfscope}%
\begin{pgfscope}%
\pgfpathrectangle{\pgfqpoint{0.680860in}{0.078740in}}{\pgfqpoint{7.842520in}{7.842520in}}%
\pgfusepath{clip}%
\pgfsetbuttcap%
\pgfsetroundjoin%
\definecolor{currentfill}{rgb}{0.412913,0.803041,0.357269}%
\pgfsetfillcolor{currentfill}%
\pgfsetlinewidth{0.000000pt}%
\definecolor{currentstroke}{rgb}{0.259857,0.745492,0.444467}%
\pgfsetstrokecolor{currentstroke}%
\pgfsetdash{}{0pt}%
\pgfpathmoveto{\pgfqpoint{5.961500in}{3.974435in}}%
\pgfpathlineto{\pgfqpoint{6.045140in}{4.205620in}}%
\pgfpathlineto{\pgfqpoint{5.895003in}{4.163095in}}%
\pgfpathclose%
\pgfusepath{fill}%
\end{pgfscope}%
\begin{pgfscope}%
\pgfpathrectangle{\pgfqpoint{0.680860in}{0.078740in}}{\pgfqpoint{7.842520in}{7.842520in}}%
\pgfusepath{clip}%
\pgfsetbuttcap%
\pgfsetroundjoin%
\definecolor{currentfill}{rgb}{0.174274,0.445044,0.557792}%
\pgfsetfillcolor{currentfill}%
\pgfsetlinewidth{0.000000pt}%
\definecolor{currentstroke}{rgb}{0.266941,0.748751,0.440573}%
\pgfsetstrokecolor{currentstroke}%
\pgfsetdash{}{0pt}%
\pgfpathmoveto{\pgfqpoint{5.451999in}{2.132005in}}%
\pgfpathlineto{\pgfqpoint{5.537327in}{2.481324in}}%
\pgfpathlineto{\pgfqpoint{5.388941in}{2.482208in}}%
\pgfpathclose%
\pgfusepath{fill}%
\end{pgfscope}%
\begin{pgfscope}%
\pgfpathrectangle{\pgfqpoint{0.680860in}{0.078740in}}{\pgfqpoint{7.842520in}{7.842520in}}%
\pgfusepath{clip}%
\pgfsetbuttcap%
\pgfsetroundjoin%
\definecolor{currentfill}{rgb}{0.283187,0.125848,0.444960}%
\pgfsetfillcolor{currentfill}%
\pgfsetlinewidth{0.000000pt}%
\definecolor{currentstroke}{rgb}{0.274149,0.751988,0.436601}%
\pgfsetstrokecolor{currentstroke}%
\pgfsetdash{}{0pt}%
\pgfpathmoveto{\pgfqpoint{5.343822in}{0.993654in}}%
\pgfpathlineto{\pgfqpoint{5.491771in}{0.955498in}}%
\pgfpathlineto{\pgfqpoint{5.429309in}{1.376042in}}%
\pgfpathclose%
\pgfusepath{fill}%
\end{pgfscope}%
\begin{pgfscope}%
\pgfpathrectangle{\pgfqpoint{0.680860in}{0.078740in}}{\pgfqpoint{7.842520in}{7.842520in}}%
\pgfusepath{clip}%
\pgfsetbuttcap%
\pgfsetroundjoin%
\definecolor{currentfill}{rgb}{0.128729,0.563265,0.551229}%
\pgfsetfillcolor{currentfill}%
\pgfsetlinewidth{0.000000pt}%
\definecolor{currentstroke}{rgb}{0.281477,0.755203,0.432552}%
\pgfsetstrokecolor{currentstroke}%
\pgfsetdash{}{0pt}%
\pgfpathmoveto{\pgfqpoint{5.558661in}{3.120909in}}%
\pgfpathlineto{\pgfqpoint{5.473854in}{2.809393in}}%
\pgfpathlineto{\pgfqpoint{5.622612in}{2.817228in}}%
\pgfpathclose%
\pgfusepath{fill}%
\end{pgfscope}%
\begin{pgfscope}%
\pgfpathrectangle{\pgfqpoint{0.680860in}{0.078740in}}{\pgfqpoint{7.842520in}{7.842520in}}%
\pgfusepath{clip}%
\pgfsetbuttcap%
\pgfsetroundjoin%
\definecolor{currentfill}{rgb}{0.855810,0.888601,0.097452}%
\pgfsetfillcolor{currentfill}%
\pgfsetlinewidth{0.000000pt}%
\definecolor{currentstroke}{rgb}{0.288921,0.758394,0.428426}%
\pgfsetstrokecolor{currentstroke}%
\pgfsetdash{}{0pt}%
\pgfpathmoveto{\pgfqpoint{6.675467in}{4.967908in}}%
\pgfpathlineto{\pgfqpoint{6.442972in}{4.788400in}}%
\pgfpathlineto{\pgfqpoint{6.595265in}{4.848496in}}%
\pgfpathclose%
\pgfusepath{fill}%
\end{pgfscope}%
\begin{pgfscope}%
\pgfpathrectangle{\pgfqpoint{0.680860in}{0.078740in}}{\pgfqpoint{7.842520in}{7.842520in}}%
\pgfusepath{clip}%
\pgfsetbuttcap%
\pgfsetroundjoin%
\definecolor{currentfill}{rgb}{0.166383,0.690856,0.496502}%
\pgfsetfillcolor{currentfill}%
\pgfsetlinewidth{0.000000pt}%
\definecolor{currentstroke}{rgb}{0.296479,0.761561,0.424223}%
\pgfsetstrokecolor{currentstroke}%
\pgfsetdash{}{0pt}%
\pgfpathmoveto{\pgfqpoint{5.727605in}{3.687123in}}%
\pgfpathlineto{\pgfqpoint{5.643276in}{3.414261in}}%
\pgfpathlineto{\pgfqpoint{5.792704in}{3.438030in}}%
\pgfpathclose%
\pgfusepath{fill}%
\end{pgfscope}%
\begin{pgfscope}%
\pgfpathrectangle{\pgfqpoint{0.680860in}{0.078740in}}{\pgfqpoint{7.842520in}{7.842520in}}%
\pgfusepath{clip}%
\pgfsetbuttcap%
\pgfsetroundjoin%
\definecolor{currentfill}{rgb}{0.122312,0.633153,0.530398}%
\pgfsetfillcolor{currentfill}%
\pgfsetlinewidth{0.000000pt}%
\definecolor{currentstroke}{rgb}{0.304148,0.764704,0.419943}%
\pgfsetstrokecolor{currentstroke}%
\pgfsetdash{}{0pt}%
\pgfpathmoveto{\pgfqpoint{5.643276in}{3.414261in}}%
\pgfpathlineto{\pgfqpoint{5.558661in}{3.120909in}}%
\pgfpathlineto{\pgfqpoint{5.707769in}{3.136994in}}%
\pgfpathclose%
\pgfusepath{fill}%
\end{pgfscope}%
\begin{pgfscope}%
\pgfpathrectangle{\pgfqpoint{0.680860in}{0.078740in}}{\pgfqpoint{7.842520in}{7.842520in}}%
\pgfusepath{clip}%
\pgfsetbuttcap%
\pgfsetroundjoin%
\definecolor{currentfill}{rgb}{0.221989,0.339161,0.548752}%
\pgfsetfillcolor{currentfill}%
\pgfsetlinewidth{0.000000pt}%
\definecolor{currentstroke}{rgb}{0.311925,0.767822,0.415586}%
\pgfsetstrokecolor{currentstroke}%
\pgfsetdash{}{0pt}%
\pgfpathmoveto{\pgfqpoint{5.366700in}{1.772086in}}%
\pgfpathlineto{\pgfqpoint{5.514947in}{1.753528in}}%
\pgfpathlineto{\pgfqpoint{5.451999in}{2.132005in}}%
\pgfpathclose%
\pgfusepath{fill}%
\end{pgfscope}%
\begin{pgfscope}%
\pgfpathrectangle{\pgfqpoint{0.680860in}{0.078740in}}{\pgfqpoint{7.842520in}{7.842520in}}%
\pgfusepath{clip}%
\pgfsetbuttcap%
\pgfsetroundjoin%
\definecolor{currentfill}{rgb}{0.250425,0.274290,0.533103}%
\pgfsetfillcolor{currentfill}%
\pgfsetlinewidth{0.000000pt}%
\definecolor{currentstroke}{rgb}{0.319809,0.770914,0.411152}%
\pgfsetstrokecolor{currentstroke}%
\pgfsetdash{}{0pt}%
\pgfpathmoveto{\pgfqpoint{5.429309in}{1.376042in}}%
\pgfpathlineto{\pgfqpoint{5.514947in}{1.753528in}}%
\pgfpathlineto{\pgfqpoint{5.366700in}{1.772086in}}%
\pgfpathclose%
\pgfusepath{fill}%
\end{pgfscope}%
\begin{pgfscope}%
\pgfpathrectangle{\pgfqpoint{0.680860in}{0.078740in}}{\pgfqpoint{7.842520in}{7.842520in}}%
\pgfusepath{clip}%
\pgfsetbuttcap%
\pgfsetroundjoin%
\definecolor{currentfill}{rgb}{0.153364,0.497000,0.557724}%
\pgfsetfillcolor{currentfill}%
\pgfsetlinewidth{0.000000pt}%
\definecolor{currentstroke}{rgb}{0.327796,0.773980,0.406640}%
\pgfsetstrokecolor{currentstroke}%
\pgfsetdash{}{0pt}%
\pgfpathmoveto{\pgfqpoint{5.388941in}{2.482208in}}%
\pgfpathlineto{\pgfqpoint{5.537327in}{2.481324in}}%
\pgfpathlineto{\pgfqpoint{5.622612in}{2.817228in}}%
\pgfpathclose%
\pgfusepath{fill}%
\end{pgfscope}%
\begin{pgfscope}%
\pgfpathrectangle{\pgfqpoint{0.680860in}{0.078740in}}{\pgfqpoint{7.842520in}{7.842520in}}%
\pgfusepath{clip}%
\pgfsetbuttcap%
\pgfsetroundjoin%
\definecolor{currentfill}{rgb}{0.964894,0.902323,0.123941}%
\pgfsetfillcolor{currentfill}%
\pgfsetlinewidth{0.000000pt}%
\definecolor{currentstroke}{rgb}{0.335885,0.777018,0.402049}%
\pgfsetstrokecolor{currentstroke}%
\pgfsetdash{}{0pt}%
\pgfpathmoveto{\pgfqpoint{6.675467in}{4.967908in}}%
\pgfpathlineto{\pgfqpoint{6.828731in}{5.033358in}}%
\pgfpathlineto{\pgfqpoint{6.907512in}{5.120623in}}%
\pgfpathclose%
\pgfusepath{fill}%
\end{pgfscope}%
\begin{pgfscope}%
\pgfpathrectangle{\pgfqpoint{0.680860in}{0.078740in}}{\pgfqpoint{7.842520in}{7.842520in}}%
\pgfusepath{clip}%
\pgfsetbuttcap%
\pgfsetroundjoin%
\definecolor{currentfill}{rgb}{0.647257,0.858400,0.209861}%
\pgfsetfillcolor{currentfill}%
\pgfsetlinewidth{0.000000pt}%
\definecolor{currentstroke}{rgb}{0.344074,0.780029,0.397381}%
\pgfsetstrokecolor{currentstroke}%
\pgfsetdash{}{0pt}%
\pgfpathmoveto{\pgfqpoint{6.361631in}{4.639437in}}%
\pgfpathlineto{\pgfqpoint{6.128126in}{4.409774in}}%
\pgfpathlineto{\pgfqpoint{6.279335in}{4.459960in}}%
\pgfpathclose%
\pgfusepath{fill}%
\end{pgfscope}%
\begin{pgfscope}%
\pgfpathrectangle{\pgfqpoint{0.680860in}{0.078740in}}{\pgfqpoint{7.842520in}{7.842520in}}%
\pgfusepath{clip}%
\pgfsetbuttcap%
\pgfsetroundjoin%
\definecolor{currentfill}{rgb}{0.266941,0.748751,0.440573}%
\pgfsetfillcolor{currentfill}%
\pgfsetlinewidth{0.000000pt}%
\definecolor{currentstroke}{rgb}{0.352360,0.783011,0.392636}%
\pgfsetstrokecolor{currentstroke}%
\pgfsetdash{}{0pt}%
\pgfpathmoveto{\pgfqpoint{5.727605in}{3.687123in}}%
\pgfpathlineto{\pgfqpoint{5.877317in}{3.717915in}}%
\pgfpathlineto{\pgfqpoint{5.961500in}{3.974435in}}%
\pgfpathclose%
\pgfusepath{fill}%
\end{pgfscope}%
\begin{pgfscope}%
\pgfpathrectangle{\pgfqpoint{0.680860in}{0.078740in}}{\pgfqpoint{7.842520in}{7.842520in}}%
\pgfusepath{clip}%
\pgfsetbuttcap%
\pgfsetroundjoin%
\definecolor{currentfill}{rgb}{0.121148,0.592739,0.544641}%
\pgfsetfillcolor{currentfill}%
\pgfsetlinewidth{0.000000pt}%
\definecolor{currentstroke}{rgb}{0.360741,0.785964,0.387814}%
\pgfsetstrokecolor{currentstroke}%
\pgfsetdash{}{0pt}%
\pgfpathmoveto{\pgfqpoint{5.707769in}{3.136994in}}%
\pgfpathlineto{\pgfqpoint{5.558661in}{3.120909in}}%
\pgfpathlineto{\pgfqpoint{5.622612in}{2.817228in}}%
\pgfpathclose%
\pgfusepath{fill}%
\end{pgfscope}%
\begin{pgfscope}%
\pgfpathrectangle{\pgfqpoint{0.680860in}{0.078740in}}{\pgfqpoint{7.842520in}{7.842520in}}%
\pgfusepath{clip}%
\pgfsetbuttcap%
\pgfsetroundjoin%
\definecolor{currentfill}{rgb}{0.202219,0.715272,0.476084}%
\pgfsetfillcolor{currentfill}%
\pgfsetlinewidth{0.000000pt}%
\definecolor{currentstroke}{rgb}{0.369214,0.788888,0.382914}%
\pgfsetstrokecolor{currentstroke}%
\pgfsetdash{}{0pt}%
\pgfpathmoveto{\pgfqpoint{5.727605in}{3.687123in}}%
\pgfpathlineto{\pgfqpoint{5.792704in}{3.438030in}}%
\pgfpathlineto{\pgfqpoint{5.877317in}{3.717915in}}%
\pgfpathclose%
\pgfusepath{fill}%
\end{pgfscope}%
\begin{pgfscope}%
\pgfpathrectangle{\pgfqpoint{0.680860in}{0.078740in}}{\pgfqpoint{7.842520in}{7.842520in}}%
\pgfusepath{clip}%
\pgfsetbuttcap%
\pgfsetroundjoin%
\definecolor{currentfill}{rgb}{0.525776,0.833491,0.288127}%
\pgfsetfillcolor{currentfill}%
\pgfsetlinewidth{0.000000pt}%
\definecolor{currentstroke}{rgb}{0.377779,0.791781,0.377939}%
\pgfsetstrokecolor{currentstroke}%
\pgfsetdash{}{0pt}%
\pgfpathmoveto{\pgfqpoint{6.045140in}{4.205620in}}%
\pgfpathlineto{\pgfqpoint{6.196200in}{4.251079in}}%
\pgfpathlineto{\pgfqpoint{6.128126in}{4.409774in}}%
\pgfpathclose%
\pgfusepath{fill}%
\end{pgfscope}%
\begin{pgfscope}%
\pgfpathrectangle{\pgfqpoint{0.680860in}{0.078740in}}{\pgfqpoint{7.842520in}{7.842520in}}%
\pgfusepath{clip}%
\pgfsetbuttcap%
\pgfsetroundjoin%
\definecolor{currentfill}{rgb}{0.279574,0.170599,0.479997}%
\pgfsetfillcolor{currentfill}%
\pgfsetlinewidth{0.000000pt}%
\definecolor{currentstroke}{rgb}{0.386433,0.794644,0.372886}%
\pgfsetstrokecolor{currentstroke}%
\pgfsetdash{}{0pt}%
\pgfpathmoveto{\pgfqpoint{5.491771in}{0.955498in}}%
\pgfpathlineto{\pgfqpoint{5.577724in}{1.348117in}}%
\pgfpathlineto{\pgfqpoint{5.429309in}{1.376042in}}%
\pgfpathclose%
\pgfusepath{fill}%
\end{pgfscope}%
\begin{pgfscope}%
\pgfpathrectangle{\pgfqpoint{0.680860in}{0.078740in}}{\pgfqpoint{7.842520in}{7.842520in}}%
\pgfusepath{clip}%
\pgfsetbuttcap%
\pgfsetroundjoin%
\definecolor{currentfill}{rgb}{0.772852,0.877868,0.131109}%
\pgfsetfillcolor{currentfill}%
\pgfsetlinewidth{0.000000pt}%
\definecolor{currentstroke}{rgb}{0.395174,0.797475,0.367757}%
\pgfsetstrokecolor{currentstroke}%
\pgfsetdash{}{0pt}%
\pgfpathmoveto{\pgfqpoint{6.442972in}{4.788400in}}%
\pgfpathlineto{\pgfqpoint{6.361631in}{4.639437in}}%
\pgfpathlineto{\pgfqpoint{6.513909in}{4.696723in}}%
\pgfpathclose%
\pgfusepath{fill}%
\end{pgfscope}%
\begin{pgfscope}%
\pgfpathrectangle{\pgfqpoint{0.680860in}{0.078740in}}{\pgfqpoint{7.842520in}{7.842520in}}%
\pgfusepath{clip}%
\pgfsetbuttcap%
\pgfsetroundjoin%
\definecolor{currentfill}{rgb}{0.134692,0.658636,0.517649}%
\pgfsetfillcolor{currentfill}%
\pgfsetlinewidth{0.000000pt}%
\definecolor{currentstroke}{rgb}{0.404001,0.800275,0.362552}%
\pgfsetstrokecolor{currentstroke}%
\pgfsetdash{}{0pt}%
\pgfpathmoveto{\pgfqpoint{5.707769in}{3.136994in}}%
\pgfpathlineto{\pgfqpoint{5.792704in}{3.438030in}}%
\pgfpathlineto{\pgfqpoint{5.643276in}{3.414261in}}%
\pgfpathclose%
\pgfusepath{fill}%
\end{pgfscope}%
\begin{pgfscope}%
\pgfpathrectangle{\pgfqpoint{0.680860in}{0.078740in}}{\pgfqpoint{7.842520in}{7.842520in}}%
\pgfusepath{clip}%
\pgfsetbuttcap%
\pgfsetroundjoin%
\definecolor{currentfill}{rgb}{0.206756,0.371758,0.553117}%
\pgfsetfillcolor{currentfill}%
\pgfsetlinewidth{0.000000pt}%
\definecolor{currentstroke}{rgb}{0.412913,0.803041,0.357269}%
\pgfsetstrokecolor{currentstroke}%
\pgfsetdash{}{0pt}%
\pgfpathmoveto{\pgfqpoint{5.514947in}{1.753528in}}%
\pgfpathlineto{\pgfqpoint{5.600678in}{2.123119in}}%
\pgfpathlineto{\pgfqpoint{5.451999in}{2.132005in}}%
\pgfpathclose%
\pgfusepath{fill}%
\end{pgfscope}%
\begin{pgfscope}%
\pgfpathrectangle{\pgfqpoint{0.680860in}{0.078740in}}{\pgfqpoint{7.842520in}{7.842520in}}%
\pgfusepath{clip}%
\pgfsetbuttcap%
\pgfsetroundjoin%
\definecolor{currentfill}{rgb}{0.824940,0.884720,0.106217}%
\pgfsetfillcolor{currentfill}%
\pgfsetlinewidth{0.000000pt}%
\definecolor{currentstroke}{rgb}{0.421908,0.805774,0.351910}%
\pgfsetstrokecolor{currentstroke}%
\pgfsetdash{}{0pt}%
\pgfpathmoveto{\pgfqpoint{6.442972in}{4.788400in}}%
\pgfpathlineto{\pgfqpoint{6.513909in}{4.696723in}}%
\pgfpathlineto{\pgfqpoint{6.595265in}{4.848496in}}%
\pgfpathclose%
\pgfusepath{fill}%
\end{pgfscope}%
\begin{pgfscope}%
\pgfpathrectangle{\pgfqpoint{0.680860in}{0.078740in}}{\pgfqpoint{7.842520in}{7.842520in}}%
\pgfusepath{clip}%
\pgfsetbuttcap%
\pgfsetroundjoin%
\definecolor{currentfill}{rgb}{0.169646,0.456262,0.558030}%
\pgfsetfillcolor{currentfill}%
\pgfsetlinewidth{0.000000pt}%
\definecolor{currentstroke}{rgb}{0.430983,0.808473,0.346476}%
\pgfsetstrokecolor{currentstroke}%
\pgfsetdash{}{0pt}%
\pgfpathmoveto{\pgfqpoint{5.537327in}{2.481324in}}%
\pgfpathlineto{\pgfqpoint{5.451999in}{2.132005in}}%
\pgfpathlineto{\pgfqpoint{5.686432in}{2.481838in}}%
\pgfpathclose%
\pgfusepath{fill}%
\end{pgfscope}%
\begin{pgfscope}%
\pgfpathrectangle{\pgfqpoint{0.680860in}{0.078740in}}{\pgfqpoint{7.842520in}{7.842520in}}%
\pgfusepath{clip}%
\pgfsetbuttcap%
\pgfsetroundjoin%
\definecolor{currentfill}{rgb}{0.575563,0.844566,0.256415}%
\pgfsetfillcolor{currentfill}%
\pgfsetlinewidth{0.000000pt}%
\definecolor{currentstroke}{rgb}{0.440137,0.811138,0.340967}%
\pgfsetstrokecolor{currentstroke}%
\pgfsetdash{}{0pt}%
\pgfpathmoveto{\pgfqpoint{6.128126in}{4.409774in}}%
\pgfpathlineto{\pgfqpoint{6.196200in}{4.251079in}}%
\pgfpathlineto{\pgfqpoint{6.279335in}{4.459960in}}%
\pgfpathclose%
\pgfusepath{fill}%
\end{pgfscope}%
\begin{pgfscope}%
\pgfpathrectangle{\pgfqpoint{0.680860in}{0.078740in}}{\pgfqpoint{7.842520in}{7.842520in}}%
\pgfusepath{clip}%
\pgfsetbuttcap%
\pgfsetroundjoin%
\definecolor{currentfill}{rgb}{0.246811,0.283237,0.535941}%
\pgfsetfillcolor{currentfill}%
\pgfsetlinewidth{0.000000pt}%
\definecolor{currentstroke}{rgb}{0.449368,0.813768,0.335384}%
\pgfsetstrokecolor{currentstroke}%
\pgfsetdash{}{0pt}%
\pgfpathmoveto{\pgfqpoint{5.663836in}{1.735772in}}%
\pgfpathlineto{\pgfqpoint{5.514947in}{1.753528in}}%
\pgfpathlineto{\pgfqpoint{5.429309in}{1.376042in}}%
\pgfpathclose%
\pgfusepath{fill}%
\end{pgfscope}%
\begin{pgfscope}%
\pgfpathrectangle{\pgfqpoint{0.680860in}{0.078740in}}{\pgfqpoint{7.842520in}{7.842520in}}%
\pgfusepath{clip}%
\pgfsetbuttcap%
\pgfsetroundjoin%
\definecolor{currentfill}{rgb}{0.283072,0.130895,0.449241}%
\pgfsetfillcolor{currentfill}%
\pgfsetlinewidth{0.000000pt}%
\definecolor{currentstroke}{rgb}{0.458674,0.816363,0.329727}%
\pgfsetstrokecolor{currentstroke}%
\pgfsetdash{}{0pt}%
\pgfpathmoveto{\pgfqpoint{5.577724in}{1.348117in}}%
\pgfpathlineto{\pgfqpoint{5.491771in}{0.955498in}}%
\pgfpathlineto{\pgfqpoint{5.640278in}{0.917502in}}%
\pgfpathclose%
\pgfusepath{fill}%
\end{pgfscope}%
\begin{pgfscope}%
\pgfpathrectangle{\pgfqpoint{0.680860in}{0.078740in}}{\pgfqpoint{7.842520in}{7.842520in}}%
\pgfusepath{clip}%
\pgfsetbuttcap%
\pgfsetroundjoin%
\definecolor{currentfill}{rgb}{0.149039,0.508051,0.557250}%
\pgfsetfillcolor{currentfill}%
\pgfsetlinewidth{0.000000pt}%
\definecolor{currentstroke}{rgb}{0.468053,0.818921,0.323998}%
\pgfsetstrokecolor{currentstroke}%
\pgfsetdash{}{0pt}%
\pgfpathmoveto{\pgfqpoint{5.537327in}{2.481324in}}%
\pgfpathlineto{\pgfqpoint{5.686432in}{2.481838in}}%
\pgfpathlineto{\pgfqpoint{5.622612in}{2.817228in}}%
\pgfpathclose%
\pgfusepath{fill}%
\end{pgfscope}%
\begin{pgfscope}%
\pgfpathrectangle{\pgfqpoint{0.680860in}{0.078740in}}{\pgfqpoint{7.842520in}{7.842520in}}%
\pgfusepath{clip}%
\pgfsetbuttcap%
\pgfsetroundjoin%
\definecolor{currentfill}{rgb}{0.260571,0.246922,0.522828}%
\pgfsetfillcolor{currentfill}%
\pgfsetlinewidth{0.000000pt}%
\definecolor{currentstroke}{rgb}{0.477504,0.821444,0.318195}%
\pgfsetstrokecolor{currentstroke}%
\pgfsetdash{}{0pt}%
\pgfpathmoveto{\pgfqpoint{5.429309in}{1.376042in}}%
\pgfpathlineto{\pgfqpoint{5.577724in}{1.348117in}}%
\pgfpathlineto{\pgfqpoint{5.663836in}{1.735772in}}%
\pgfpathclose%
\pgfusepath{fill}%
\end{pgfscope}%
\begin{pgfscope}%
\pgfpathrectangle{\pgfqpoint{0.680860in}{0.078740in}}{\pgfqpoint{7.842520in}{7.842520in}}%
\pgfusepath{clip}%
\pgfsetbuttcap%
\pgfsetroundjoin%
\definecolor{currentfill}{rgb}{0.412913,0.803041,0.357269}%
\pgfsetfillcolor{currentfill}%
\pgfsetlinewidth{0.000000pt}%
\definecolor{currentstroke}{rgb}{0.487026,0.823929,0.312321}%
\pgfsetstrokecolor{currentstroke}%
\pgfsetdash{}{0pt}%
\pgfpathmoveto{\pgfqpoint{6.112347in}{4.014244in}}%
\pgfpathlineto{\pgfqpoint{6.045140in}{4.205620in}}%
\pgfpathlineto{\pgfqpoint{5.961500in}{3.974435in}}%
\pgfpathclose%
\pgfusepath{fill}%
\end{pgfscope}%
\begin{pgfscope}%
\pgfpathrectangle{\pgfqpoint{0.680860in}{0.078740in}}{\pgfqpoint{7.842520in}{7.842520in}}%
\pgfusepath{clip}%
\pgfsetbuttcap%
\pgfsetroundjoin%
\definecolor{currentfill}{rgb}{0.180629,0.429975,0.557282}%
\pgfsetfillcolor{currentfill}%
\pgfsetlinewidth{0.000000pt}%
\definecolor{currentstroke}{rgb}{0.496615,0.826376,0.306377}%
\pgfsetstrokecolor{currentstroke}%
\pgfsetdash{}{0pt}%
\pgfpathmoveto{\pgfqpoint{5.686432in}{2.481838in}}%
\pgfpathlineto{\pgfqpoint{5.451999in}{2.132005in}}%
\pgfpathlineto{\pgfqpoint{5.600678in}{2.123119in}}%
\pgfpathclose%
\pgfusepath{fill}%
\end{pgfscope}%
\begin{pgfscope}%
\pgfpathrectangle{\pgfqpoint{0.680860in}{0.078740in}}{\pgfqpoint{7.842520in}{7.842520in}}%
\pgfusepath{clip}%
\pgfsetbuttcap%
\pgfsetroundjoin%
\definecolor{currentfill}{rgb}{0.906311,0.894855,0.098125}%
\pgfsetfillcolor{currentfill}%
\pgfsetlinewidth{0.000000pt}%
\definecolor{currentstroke}{rgb}{0.506271,0.828786,0.300362}%
\pgfsetstrokecolor{currentstroke}%
\pgfsetdash{}{0pt}%
\pgfpathmoveto{\pgfqpoint{6.595265in}{4.848496in}}%
\pgfpathlineto{\pgfqpoint{6.748593in}{4.912180in}}%
\pgfpathlineto{\pgfqpoint{6.675467in}{4.967908in}}%
\pgfpathclose%
\pgfusepath{fill}%
\end{pgfscope}%
\begin{pgfscope}%
\pgfpathrectangle{\pgfqpoint{0.680860in}{0.078740in}}{\pgfqpoint{7.842520in}{7.842520in}}%
\pgfusepath{clip}%
\pgfsetbuttcap%
\pgfsetroundjoin%
\definecolor{currentfill}{rgb}{0.945636,0.899815,0.112838}%
\pgfsetfillcolor{currentfill}%
\pgfsetlinewidth{0.000000pt}%
\definecolor{currentstroke}{rgb}{0.515992,0.831158,0.294279}%
\pgfsetstrokecolor{currentstroke}%
\pgfsetdash{}{0pt}%
\pgfpathmoveto{\pgfqpoint{6.748593in}{4.912180in}}%
\pgfpathlineto{\pgfqpoint{6.828731in}{5.033358in}}%
\pgfpathlineto{\pgfqpoint{6.675467in}{4.967908in}}%
\pgfpathclose%
\pgfusepath{fill}%
\end{pgfscope}%
\begin{pgfscope}%
\pgfpathrectangle{\pgfqpoint{0.680860in}{0.078740in}}{\pgfqpoint{7.842520in}{7.842520in}}%
\pgfusepath{clip}%
\pgfsetbuttcap%
\pgfsetroundjoin%
\definecolor{currentfill}{rgb}{0.335885,0.777018,0.402049}%
\pgfsetfillcolor{currentfill}%
\pgfsetlinewidth{0.000000pt}%
\definecolor{currentstroke}{rgb}{0.525776,0.833491,0.288127}%
\pgfsetstrokecolor{currentstroke}%
\pgfsetdash{}{0pt}%
\pgfpathmoveto{\pgfqpoint{5.961500in}{3.974435in}}%
\pgfpathlineto{\pgfqpoint{5.877317in}{3.717915in}}%
\pgfpathlineto{\pgfqpoint{6.112347in}{4.014244in}}%
\pgfpathclose%
\pgfusepath{fill}%
\end{pgfscope}%
\begin{pgfscope}%
\pgfpathrectangle{\pgfqpoint{0.680860in}{0.078740in}}{\pgfqpoint{7.842520in}{7.842520in}}%
\pgfusepath{clip}%
\pgfsetbuttcap%
\pgfsetroundjoin%
\definecolor{currentfill}{rgb}{0.218130,0.347432,0.550038}%
\pgfsetfillcolor{currentfill}%
\pgfsetlinewidth{0.000000pt}%
\definecolor{currentstroke}{rgb}{0.535621,0.835785,0.281908}%
\pgfsetstrokecolor{currentstroke}%
\pgfsetdash{}{0pt}%
\pgfpathmoveto{\pgfqpoint{5.600678in}{2.123119in}}%
\pgfpathlineto{\pgfqpoint{5.514947in}{1.753528in}}%
\pgfpathlineto{\pgfqpoint{5.663836in}{1.735772in}}%
\pgfpathclose%
\pgfusepath{fill}%
\end{pgfscope}%
\begin{pgfscope}%
\pgfpathrectangle{\pgfqpoint{0.680860in}{0.078740in}}{\pgfqpoint{7.842520in}{7.842520in}}%
\pgfusepath{clip}%
\pgfsetbuttcap%
\pgfsetroundjoin%
\definecolor{currentfill}{rgb}{0.119738,0.603785,0.541400}%
\pgfsetfillcolor{currentfill}%
\pgfsetlinewidth{0.000000pt}%
\definecolor{currentstroke}{rgb}{0.545524,0.838039,0.275626}%
\pgfsetstrokecolor{currentstroke}%
\pgfsetdash{}{0pt}%
\pgfpathmoveto{\pgfqpoint{5.622612in}{2.817228in}}%
\pgfpathlineto{\pgfqpoint{5.857674in}{3.155069in}}%
\pgfpathlineto{\pgfqpoint{5.707769in}{3.136994in}}%
\pgfpathclose%
\pgfusepath{fill}%
\end{pgfscope}%
\begin{pgfscope}%
\pgfpathrectangle{\pgfqpoint{0.680860in}{0.078740in}}{\pgfqpoint{7.842520in}{7.842520in}}%
\pgfusepath{clip}%
\pgfsetbuttcap%
\pgfsetroundjoin%
\definecolor{currentfill}{rgb}{0.468053,0.818921,0.323998}%
\pgfsetfillcolor{currentfill}%
\pgfsetlinewidth{0.000000pt}%
\definecolor{currentstroke}{rgb}{0.555484,0.840254,0.269281}%
\pgfsetstrokecolor{currentstroke}%
\pgfsetdash{}{0pt}%
\pgfpathmoveto{\pgfqpoint{6.112347in}{4.014244in}}%
\pgfpathlineto{\pgfqpoint{6.196200in}{4.251079in}}%
\pgfpathlineto{\pgfqpoint{6.045140in}{4.205620in}}%
\pgfpathclose%
\pgfusepath{fill}%
\end{pgfscope}%
\begin{pgfscope}%
\pgfpathrectangle{\pgfqpoint{0.680860in}{0.078740in}}{\pgfqpoint{7.842520in}{7.842520in}}%
\pgfusepath{clip}%
\pgfsetbuttcap%
\pgfsetroundjoin%
\definecolor{currentfill}{rgb}{0.699415,0.867117,0.175971}%
\pgfsetfillcolor{currentfill}%
\pgfsetlinewidth{0.000000pt}%
\definecolor{currentstroke}{rgb}{0.565498,0.842430,0.262877}%
\pgfsetstrokecolor{currentstroke}%
\pgfsetdash{}{0pt}%
\pgfpathmoveto{\pgfqpoint{6.431517in}{4.513368in}}%
\pgfpathlineto{\pgfqpoint{6.361631in}{4.639437in}}%
\pgfpathlineto{\pgfqpoint{6.279335in}{4.459960in}}%
\pgfpathclose%
\pgfusepath{fill}%
\end{pgfscope}%
\begin{pgfscope}%
\pgfpathrectangle{\pgfqpoint{0.680860in}{0.078740in}}{\pgfqpoint{7.842520in}{7.842520in}}%
\pgfusepath{clip}%
\pgfsetbuttcap%
\pgfsetroundjoin%
\definecolor{currentfill}{rgb}{0.130067,0.651384,0.521608}%
\pgfsetfillcolor{currentfill}%
\pgfsetlinewidth{0.000000pt}%
\definecolor{currentstroke}{rgb}{0.575563,0.844566,0.256415}%
\pgfsetstrokecolor{currentstroke}%
\pgfsetdash{}{0pt}%
\pgfpathmoveto{\pgfqpoint{5.857674in}{3.155069in}}%
\pgfpathlineto{\pgfqpoint{5.792704in}{3.438030in}}%
\pgfpathlineto{\pgfqpoint{5.707769in}{3.136994in}}%
\pgfpathclose%
\pgfusepath{fill}%
\end{pgfscope}%
\begin{pgfscope}%
\pgfpathrectangle{\pgfqpoint{0.680860in}{0.078740in}}{\pgfqpoint{7.842520in}{7.842520in}}%
\pgfusepath{clip}%
\pgfsetbuttcap%
\pgfsetroundjoin%
\definecolor{currentfill}{rgb}{0.226397,0.728888,0.462789}%
\pgfsetfillcolor{currentfill}%
\pgfsetlinewidth{0.000000pt}%
\definecolor{currentstroke}{rgb}{0.585678,0.846661,0.249897}%
\pgfsetstrokecolor{currentstroke}%
\pgfsetdash{}{0pt}%
\pgfpathmoveto{\pgfqpoint{5.877317in}{3.717915in}}%
\pgfpathlineto{\pgfqpoint{5.792704in}{3.438030in}}%
\pgfpathlineto{\pgfqpoint{6.027896in}{3.751220in}}%
\pgfpathclose%
\pgfusepath{fill}%
\end{pgfscope}%
\begin{pgfscope}%
\pgfpathrectangle{\pgfqpoint{0.680860in}{0.078740in}}{\pgfqpoint{7.842520in}{7.842520in}}%
\pgfusepath{clip}%
\pgfsetbuttcap%
\pgfsetroundjoin%
\definecolor{currentfill}{rgb}{0.137770,0.537492,0.554906}%
\pgfsetfillcolor{currentfill}%
\pgfsetlinewidth{0.000000pt}%
\definecolor{currentstroke}{rgb}{0.595839,0.848717,0.243329}%
\pgfsetstrokecolor{currentstroke}%
\pgfsetdash{}{0pt}%
\pgfpathmoveto{\pgfqpoint{5.622612in}{2.817228in}}%
\pgfpathlineto{\pgfqpoint{5.686432in}{2.481838in}}%
\pgfpathlineto{\pgfqpoint{5.772128in}{2.826764in}}%
\pgfpathclose%
\pgfusepath{fill}%
\end{pgfscope}%
\begin{pgfscope}%
\pgfpathrectangle{\pgfqpoint{0.680860in}{0.078740in}}{\pgfqpoint{7.842520in}{7.842520in}}%
\pgfusepath{clip}%
\pgfsetbuttcap%
\pgfsetroundjoin%
\definecolor{currentfill}{rgb}{0.278826,0.175490,0.483397}%
\pgfsetfillcolor{currentfill}%
\pgfsetlinewidth{0.000000pt}%
\definecolor{currentstroke}{rgb}{0.606045,0.850733,0.236712}%
\pgfsetstrokecolor{currentstroke}%
\pgfsetdash{}{0pt}%
\pgfpathmoveto{\pgfqpoint{5.640278in}{0.917502in}}%
\pgfpathlineto{\pgfqpoint{5.726743in}{1.320688in}}%
\pgfpathlineto{\pgfqpoint{5.577724in}{1.348117in}}%
\pgfpathclose%
\pgfusepath{fill}%
\end{pgfscope}%
\begin{pgfscope}%
\pgfpathrectangle{\pgfqpoint{0.680860in}{0.078740in}}{\pgfqpoint{7.842520in}{7.842520in}}%
\pgfusepath{clip}%
\pgfsetbuttcap%
\pgfsetroundjoin%
\definecolor{currentfill}{rgb}{0.751884,0.874951,0.143228}%
\pgfsetfillcolor{currentfill}%
\pgfsetlinewidth{0.000000pt}%
\definecolor{currentstroke}{rgb}{0.616293,0.852709,0.230052}%
\pgfsetstrokecolor{currentstroke}%
\pgfsetdash{}{0pt}%
\pgfpathmoveto{\pgfqpoint{6.513909in}{4.696723in}}%
\pgfpathlineto{\pgfqpoint{6.361631in}{4.639437in}}%
\pgfpathlineto{\pgfqpoint{6.431517in}{4.513368in}}%
\pgfpathclose%
\pgfusepath{fill}%
\end{pgfscope}%
\begin{pgfscope}%
\pgfpathrectangle{\pgfqpoint{0.680860in}{0.078740in}}{\pgfqpoint{7.842520in}{7.842520in}}%
\pgfusepath{clip}%
\pgfsetbuttcap%
\pgfsetroundjoin%
\definecolor{currentfill}{rgb}{0.122606,0.585371,0.546557}%
\pgfsetfillcolor{currentfill}%
\pgfsetlinewidth{0.000000pt}%
\definecolor{currentstroke}{rgb}{0.626579,0.854645,0.223353}%
\pgfsetstrokecolor{currentstroke}%
\pgfsetdash{}{0pt}%
\pgfpathmoveto{\pgfqpoint{5.772128in}{2.826764in}}%
\pgfpathlineto{\pgfqpoint{5.857674in}{3.155069in}}%
\pgfpathlineto{\pgfqpoint{5.622612in}{2.817228in}}%
\pgfpathclose%
\pgfusepath{fill}%
\end{pgfscope}%
\begin{pgfscope}%
\pgfpathrectangle{\pgfqpoint{0.680860in}{0.078740in}}{\pgfqpoint{7.842520in}{7.842520in}}%
\pgfusepath{clip}%
\pgfsetbuttcap%
\pgfsetroundjoin%
\definecolor{currentfill}{rgb}{0.177423,0.437527,0.557565}%
\pgfsetfillcolor{currentfill}%
\pgfsetlinewidth{0.000000pt}%
\definecolor{currentstroke}{rgb}{0.636902,0.856542,0.216620}%
\pgfsetstrokecolor{currentstroke}%
\pgfsetdash{}{0pt}%
\pgfpathmoveto{\pgfqpoint{5.686432in}{2.481838in}}%
\pgfpathlineto{\pgfqpoint{5.600678in}{2.123119in}}%
\pgfpathlineto{\pgfqpoint{5.750044in}{2.115363in}}%
\pgfpathclose%
\pgfusepath{fill}%
\end{pgfscope}%
\begin{pgfscope}%
\pgfpathrectangle{\pgfqpoint{0.680860in}{0.078740in}}{\pgfqpoint{7.842520in}{7.842520in}}%
\pgfusepath{clip}%
\pgfsetbuttcap%
\pgfsetroundjoin%
\definecolor{currentfill}{rgb}{0.319809,0.770914,0.411152}%
\pgfsetfillcolor{currentfill}%
\pgfsetlinewidth{0.000000pt}%
\definecolor{currentstroke}{rgb}{0.647257,0.858400,0.209861}%
\pgfsetstrokecolor{currentstroke}%
\pgfsetdash{}{0pt}%
\pgfpathmoveto{\pgfqpoint{5.877317in}{3.717915in}}%
\pgfpathlineto{\pgfqpoint{6.027896in}{3.751220in}}%
\pgfpathlineto{\pgfqpoint{6.112347in}{4.014244in}}%
\pgfpathclose%
\pgfusepath{fill}%
\end{pgfscope}%
\begin{pgfscope}%
\pgfpathrectangle{\pgfqpoint{0.680860in}{0.078740in}}{\pgfqpoint{7.842520in}{7.842520in}}%
\pgfusepath{clip}%
\pgfsetbuttcap%
\pgfsetroundjoin%
\definecolor{currentfill}{rgb}{0.201239,0.383670,0.554294}%
\pgfsetfillcolor{currentfill}%
\pgfsetlinewidth{0.000000pt}%
\definecolor{currentstroke}{rgb}{0.657642,0.860219,0.203082}%
\pgfsetstrokecolor{currentstroke}%
\pgfsetdash{}{0pt}%
\pgfpathmoveto{\pgfqpoint{5.663836in}{1.735772in}}%
\pgfpathlineto{\pgfqpoint{5.750044in}{2.115363in}}%
\pgfpathlineto{\pgfqpoint{5.600678in}{2.123119in}}%
\pgfpathclose%
\pgfusepath{fill}%
\end{pgfscope}%
\begin{pgfscope}%
\pgfpathrectangle{\pgfqpoint{0.680860in}{0.078740in}}{\pgfqpoint{7.842520in}{7.842520in}}%
\pgfusepath{clip}%
\pgfsetbuttcap%
\pgfsetroundjoin%
\definecolor{currentfill}{rgb}{0.626579,0.854645,0.223353}%
\pgfsetfillcolor{currentfill}%
\pgfsetlinewidth{0.000000pt}%
\definecolor{currentstroke}{rgb}{0.668054,0.861999,0.196293}%
\pgfsetstrokecolor{currentstroke}%
\pgfsetdash{}{0pt}%
\pgfpathmoveto{\pgfqpoint{6.279335in}{4.459960in}}%
\pgfpathlineto{\pgfqpoint{6.196200in}{4.251079in}}%
\pgfpathlineto{\pgfqpoint{6.431517in}{4.513368in}}%
\pgfpathclose%
\pgfusepath{fill}%
\end{pgfscope}%
\begin{pgfscope}%
\pgfpathrectangle{\pgfqpoint{0.680860in}{0.078740in}}{\pgfqpoint{7.842520in}{7.842520in}}%
\pgfusepath{clip}%
\pgfsetbuttcap%
\pgfsetroundjoin%
\definecolor{currentfill}{rgb}{0.150148,0.676631,0.506589}%
\pgfsetfillcolor{currentfill}%
\pgfsetlinewidth{0.000000pt}%
\definecolor{currentstroke}{rgb}{0.678489,0.863742,0.189503}%
\pgfsetstrokecolor{currentstroke}%
\pgfsetdash{}{0pt}%
\pgfpathmoveto{\pgfqpoint{5.942966in}{3.464060in}}%
\pgfpathlineto{\pgfqpoint{5.792704in}{3.438030in}}%
\pgfpathlineto{\pgfqpoint{5.857674in}{3.155069in}}%
\pgfpathclose%
\pgfusepath{fill}%
\end{pgfscope}%
\begin{pgfscope}%
\pgfpathrectangle{\pgfqpoint{0.680860in}{0.078740in}}{\pgfqpoint{7.842520in}{7.842520in}}%
\pgfusepath{clip}%
\pgfsetbuttcap%
\pgfsetroundjoin%
\definecolor{currentfill}{rgb}{0.208030,0.718701,0.472873}%
\pgfsetfillcolor{currentfill}%
\pgfsetlinewidth{0.000000pt}%
\definecolor{currentstroke}{rgb}{0.688944,0.865448,0.182725}%
\pgfsetstrokecolor{currentstroke}%
\pgfsetdash{}{0pt}%
\pgfpathmoveto{\pgfqpoint{6.027896in}{3.751220in}}%
\pgfpathlineto{\pgfqpoint{5.792704in}{3.438030in}}%
\pgfpathlineto{\pgfqpoint{5.942966in}{3.464060in}}%
\pgfpathclose%
\pgfusepath{fill}%
\end{pgfscope}%
\begin{pgfscope}%
\pgfpathrectangle{\pgfqpoint{0.680860in}{0.078740in}}{\pgfqpoint{7.842520in}{7.842520in}}%
\pgfusepath{clip}%
\pgfsetbuttcap%
\pgfsetroundjoin%
\definecolor{currentfill}{rgb}{0.243113,0.292092,0.538516}%
\pgfsetfillcolor{currentfill}%
\pgfsetlinewidth{0.000000pt}%
\definecolor{currentstroke}{rgb}{0.699415,0.867117,0.175971}%
\pgfsetstrokecolor{currentstroke}%
\pgfsetdash{}{0pt}%
\pgfpathmoveto{\pgfqpoint{5.577724in}{1.348117in}}%
\pgfpathlineto{\pgfqpoint{5.813379in}{1.718851in}}%
\pgfpathlineto{\pgfqpoint{5.663836in}{1.735772in}}%
\pgfpathclose%
\pgfusepath{fill}%
\end{pgfscope}%
\begin{pgfscope}%
\pgfpathrectangle{\pgfqpoint{0.680860in}{0.078740in}}{\pgfqpoint{7.842520in}{7.842520in}}%
\pgfusepath{clip}%
\pgfsetbuttcap%
\pgfsetroundjoin%
\definecolor{currentfill}{rgb}{0.257322,0.256130,0.526563}%
\pgfsetfillcolor{currentfill}%
\pgfsetlinewidth{0.000000pt}%
\definecolor{currentstroke}{rgb}{0.709898,0.868751,0.169257}%
\pgfsetstrokecolor{currentstroke}%
\pgfsetdash{}{0pt}%
\pgfpathmoveto{\pgfqpoint{5.577724in}{1.348117in}}%
\pgfpathlineto{\pgfqpoint{5.726743in}{1.320688in}}%
\pgfpathlineto{\pgfqpoint{5.813379in}{1.718851in}}%
\pgfpathclose%
\pgfusepath{fill}%
\end{pgfscope}%
\begin{pgfscope}%
\pgfpathrectangle{\pgfqpoint{0.680860in}{0.078740in}}{\pgfqpoint{7.842520in}{7.842520in}}%
\pgfusepath{clip}%
\pgfsetbuttcap%
\pgfsetroundjoin%
\definecolor{currentfill}{rgb}{0.855810,0.888601,0.097452}%
\pgfsetfillcolor{currentfill}%
\pgfsetlinewidth{0.000000pt}%
\definecolor{currentstroke}{rgb}{0.720391,0.870350,0.162603}%
\pgfsetstrokecolor{currentstroke}%
\pgfsetdash{}{0pt}%
\pgfpathmoveto{\pgfqpoint{6.595265in}{4.848496in}}%
\pgfpathlineto{\pgfqpoint{6.513909in}{4.696723in}}%
\pgfpathlineto{\pgfqpoint{6.667209in}{4.757496in}}%
\pgfpathclose%
\pgfusepath{fill}%
\end{pgfscope}%
\begin{pgfscope}%
\pgfpathrectangle{\pgfqpoint{0.680860in}{0.078740in}}{\pgfqpoint{7.842520in}{7.842520in}}%
\pgfusepath{clip}%
\pgfsetbuttcap%
\pgfsetroundjoin%
\definecolor{currentfill}{rgb}{0.906311,0.894855,0.098125}%
\pgfsetfillcolor{currentfill}%
\pgfsetlinewidth{0.000000pt}%
\definecolor{currentstroke}{rgb}{0.730889,0.871916,0.156029}%
\pgfsetstrokecolor{currentstroke}%
\pgfsetdash{}{0pt}%
\pgfpathmoveto{\pgfqpoint{6.667209in}{4.757496in}}%
\pgfpathlineto{\pgfqpoint{6.748593in}{4.912180in}}%
\pgfpathlineto{\pgfqpoint{6.595265in}{4.848496in}}%
\pgfpathclose%
\pgfusepath{fill}%
\end{pgfscope}%
\begin{pgfscope}%
\pgfpathrectangle{\pgfqpoint{0.680860in}{0.078740in}}{\pgfqpoint{7.842520in}{7.842520in}}%
\pgfusepath{clip}%
\pgfsetbuttcap%
\pgfsetroundjoin%
\definecolor{currentfill}{rgb}{0.163625,0.471133,0.558148}%
\pgfsetfillcolor{currentfill}%
\pgfsetlinewidth{0.000000pt}%
\definecolor{currentstroke}{rgb}{0.741388,0.873449,0.149561}%
\pgfsetstrokecolor{currentstroke}%
\pgfsetdash{}{0pt}%
\pgfpathmoveto{\pgfqpoint{5.686432in}{2.481838in}}%
\pgfpathlineto{\pgfqpoint{5.750044in}{2.115363in}}%
\pgfpathlineto{\pgfqpoint{5.836269in}{2.483804in}}%
\pgfpathclose%
\pgfusepath{fill}%
\end{pgfscope}%
\begin{pgfscope}%
\pgfpathrectangle{\pgfqpoint{0.680860in}{0.078740in}}{\pgfqpoint{7.842520in}{7.842520in}}%
\pgfusepath{clip}%
\pgfsetbuttcap%
\pgfsetroundjoin%
\definecolor{currentfill}{rgb}{0.212395,0.359683,0.551710}%
\pgfsetfillcolor{currentfill}%
\pgfsetlinewidth{0.000000pt}%
\definecolor{currentstroke}{rgb}{0.751884,0.874951,0.143228}%
\pgfsetstrokecolor{currentstroke}%
\pgfsetdash{}{0pt}%
\pgfpathmoveto{\pgfqpoint{5.663836in}{1.735772in}}%
\pgfpathlineto{\pgfqpoint{5.813379in}{1.718851in}}%
\pgfpathlineto{\pgfqpoint{5.750044in}{2.115363in}}%
\pgfpathclose%
\pgfusepath{fill}%
\end{pgfscope}%
\begin{pgfscope}%
\pgfpathrectangle{\pgfqpoint{0.680860in}{0.078740in}}{\pgfqpoint{7.842520in}{7.842520in}}%
\pgfusepath{clip}%
\pgfsetbuttcap%
\pgfsetroundjoin%
\definecolor{currentfill}{rgb}{0.133743,0.548535,0.553541}%
\pgfsetfillcolor{currentfill}%
\pgfsetlinewidth{0.000000pt}%
\definecolor{currentstroke}{rgb}{0.762373,0.876424,0.137064}%
\pgfsetstrokecolor{currentstroke}%
\pgfsetdash{}{0pt}%
\pgfpathmoveto{\pgfqpoint{5.922420in}{2.838066in}}%
\pgfpathlineto{\pgfqpoint{5.772128in}{2.826764in}}%
\pgfpathlineto{\pgfqpoint{5.686432in}{2.481838in}}%
\pgfpathclose%
\pgfusepath{fill}%
\end{pgfscope}%
\begin{pgfscope}%
\pgfpathrectangle{\pgfqpoint{0.680860in}{0.078740in}}{\pgfqpoint{7.842520in}{7.842520in}}%
\pgfusepath{clip}%
\pgfsetbuttcap%
\pgfsetroundjoin%
\definecolor{currentfill}{rgb}{0.477504,0.821444,0.318195}%
\pgfsetfillcolor{currentfill}%
\pgfsetlinewidth{0.000000pt}%
\definecolor{currentstroke}{rgb}{0.772852,0.877868,0.131109}%
\pgfsetstrokecolor{currentstroke}%
\pgfsetdash{}{0pt}%
\pgfpathmoveto{\pgfqpoint{6.196200in}{4.251079in}}%
\pgfpathlineto{\pgfqpoint{6.112347in}{4.014244in}}%
\pgfpathlineto{\pgfqpoint{6.264117in}{4.056898in}}%
\pgfpathclose%
\pgfusepath{fill}%
\end{pgfscope}%
\begin{pgfscope}%
\pgfpathrectangle{\pgfqpoint{0.680860in}{0.078740in}}{\pgfqpoint{7.842520in}{7.842520in}}%
\pgfusepath{clip}%
\pgfsetbuttcap%
\pgfsetroundjoin%
\definecolor{currentfill}{rgb}{0.616293,0.852709,0.230052}%
\pgfsetfillcolor{currentfill}%
\pgfsetlinewidth{0.000000pt}%
\definecolor{currentstroke}{rgb}{0.783315,0.879285,0.125405}%
\pgfsetstrokecolor{currentstroke}%
\pgfsetdash{}{0pt}%
\pgfpathmoveto{\pgfqpoint{6.431517in}{4.513368in}}%
\pgfpathlineto{\pgfqpoint{6.196200in}{4.251079in}}%
\pgfpathlineto{\pgfqpoint{6.348210in}{4.299589in}}%
\pgfpathclose%
\pgfusepath{fill}%
\end{pgfscope}%
\begin{pgfscope}%
\pgfpathrectangle{\pgfqpoint{0.680860in}{0.078740in}}{\pgfqpoint{7.842520in}{7.842520in}}%
\pgfusepath{clip}%
\pgfsetbuttcap%
\pgfsetroundjoin%
\definecolor{currentfill}{rgb}{0.120565,0.596422,0.543611}%
\pgfsetfillcolor{currentfill}%
\pgfsetlinewidth{0.000000pt}%
\definecolor{currentstroke}{rgb}{0.793760,0.880678,0.120005}%
\pgfsetstrokecolor{currentstroke}%
\pgfsetdash{}{0pt}%
\pgfpathmoveto{\pgfqpoint{5.857674in}{3.155069in}}%
\pgfpathlineto{\pgfqpoint{5.772128in}{2.826764in}}%
\pgfpathlineto{\pgfqpoint{5.922420in}{2.838066in}}%
\pgfpathclose%
\pgfusepath{fill}%
\end{pgfscope}%
\begin{pgfscope}%
\pgfpathrectangle{\pgfqpoint{0.680860in}{0.078740in}}{\pgfqpoint{7.842520in}{7.842520in}}%
\pgfusepath{clip}%
\pgfsetbuttcap%
\pgfsetroundjoin%
\definecolor{currentfill}{rgb}{0.141935,0.526453,0.555991}%
\pgfsetfillcolor{currentfill}%
\pgfsetlinewidth{0.000000pt}%
\definecolor{currentstroke}{rgb}{0.804182,0.882046,0.114965}%
\pgfsetstrokecolor{currentstroke}%
\pgfsetdash{}{0pt}%
\pgfpathmoveto{\pgfqpoint{5.686432in}{2.481838in}}%
\pgfpathlineto{\pgfqpoint{5.836269in}{2.483804in}}%
\pgfpathlineto{\pgfqpoint{5.922420in}{2.838066in}}%
\pgfpathclose%
\pgfusepath{fill}%
\end{pgfscope}%
\begin{pgfscope}%
\pgfpathrectangle{\pgfqpoint{0.680860in}{0.078740in}}{\pgfqpoint{7.842520in}{7.842520in}}%
\pgfusepath{clip}%
\pgfsetbuttcap%
\pgfsetroundjoin%
\definecolor{currentfill}{rgb}{0.395174,0.797475,0.367757}%
\pgfsetfillcolor{currentfill}%
\pgfsetlinewidth{0.000000pt}%
\definecolor{currentstroke}{rgb}{0.814576,0.883393,0.110347}%
\pgfsetstrokecolor{currentstroke}%
\pgfsetdash{}{0pt}%
\pgfpathmoveto{\pgfqpoint{6.264117in}{4.056898in}}%
\pgfpathlineto{\pgfqpoint{6.112347in}{4.014244in}}%
\pgfpathlineto{\pgfqpoint{6.027896in}{3.751220in}}%
\pgfpathclose%
\pgfusepath{fill}%
\end{pgfscope}%
\begin{pgfscope}%
\pgfpathrectangle{\pgfqpoint{0.680860in}{0.078740in}}{\pgfqpoint{7.842520in}{7.842520in}}%
\pgfusepath{clip}%
\pgfsetbuttcap%
\pgfsetroundjoin%
\definecolor{currentfill}{rgb}{0.804182,0.882046,0.114965}%
\pgfsetfillcolor{currentfill}%
\pgfsetlinewidth{0.000000pt}%
\definecolor{currentstroke}{rgb}{0.824940,0.884720,0.106217}%
\pgfsetstrokecolor{currentstroke}%
\pgfsetdash{}{0pt}%
\pgfpathmoveto{\pgfqpoint{6.431517in}{4.513368in}}%
\pgfpathlineto{\pgfqpoint{6.667209in}{4.757496in}}%
\pgfpathlineto{\pgfqpoint{6.513909in}{4.696723in}}%
\pgfpathclose%
\pgfusepath{fill}%
\end{pgfscope}%
\begin{pgfscope}%
\pgfpathrectangle{\pgfqpoint{0.680860in}{0.078740in}}{\pgfqpoint{7.842520in}{7.842520in}}%
\pgfusepath{clip}%
\pgfsetbuttcap%
\pgfsetroundjoin%
\definecolor{currentfill}{rgb}{0.166383,0.690856,0.496502}%
\pgfsetfillcolor{currentfill}%
\pgfsetlinewidth{0.000000pt}%
\definecolor{currentstroke}{rgb}{0.835270,0.886029,0.102646}%
\pgfsetstrokecolor{currentstroke}%
\pgfsetdash{}{0pt}%
\pgfpathmoveto{\pgfqpoint{6.094083in}{3.492441in}}%
\pgfpathlineto{\pgfqpoint{5.942966in}{3.464060in}}%
\pgfpathlineto{\pgfqpoint{5.857674in}{3.155069in}}%
\pgfpathclose%
\pgfusepath{fill}%
\end{pgfscope}%
\begin{pgfscope}%
\pgfpathrectangle{\pgfqpoint{0.680860in}{0.078740in}}{\pgfqpoint{7.842520in}{7.842520in}}%
\pgfusepath{clip}%
\pgfsetbuttcap%
\pgfsetroundjoin%
\definecolor{currentfill}{rgb}{0.266941,0.748751,0.440573}%
\pgfsetfillcolor{currentfill}%
\pgfsetlinewidth{0.000000pt}%
\definecolor{currentstroke}{rgb}{0.845561,0.887322,0.099702}%
\pgfsetstrokecolor{currentstroke}%
\pgfsetdash{}{0pt}%
\pgfpathmoveto{\pgfqpoint{6.179365in}{3.787136in}}%
\pgfpathlineto{\pgfqpoint{6.027896in}{3.751220in}}%
\pgfpathlineto{\pgfqpoint{5.942966in}{3.464060in}}%
\pgfpathclose%
\pgfusepath{fill}%
\end{pgfscope}%
\begin{pgfscope}%
\pgfpathrectangle{\pgfqpoint{0.680860in}{0.078740in}}{\pgfqpoint{7.842520in}{7.842520in}}%
\pgfusepath{clip}%
\pgfsetbuttcap%
\pgfsetroundjoin%
\definecolor{currentfill}{rgb}{0.535621,0.835785,0.281908}%
\pgfsetfillcolor{currentfill}%
\pgfsetlinewidth{0.000000pt}%
\definecolor{currentstroke}{rgb}{0.855810,0.888601,0.097452}%
\pgfsetstrokecolor{currentstroke}%
\pgfsetdash{}{0pt}%
\pgfpathmoveto{\pgfqpoint{6.348210in}{4.299589in}}%
\pgfpathlineto{\pgfqpoint{6.196200in}{4.251079in}}%
\pgfpathlineto{\pgfqpoint{6.264117in}{4.056898in}}%
\pgfpathclose%
\pgfusepath{fill}%
\end{pgfscope}%
\begin{pgfscope}%
\pgfpathrectangle{\pgfqpoint{0.680860in}{0.078740in}}{\pgfqpoint{7.842520in}{7.842520in}}%
\pgfusepath{clip}%
\pgfsetbuttcap%
\pgfsetroundjoin%
\definecolor{currentfill}{rgb}{0.195860,0.395433,0.555276}%
\pgfsetfillcolor{currentfill}%
\pgfsetlinewidth{0.000000pt}%
\definecolor{currentstroke}{rgb}{0.866013,0.889868,0.095953}%
\pgfsetstrokecolor{currentstroke}%
\pgfsetdash{}{0pt}%
\pgfpathmoveto{\pgfqpoint{5.750044in}{2.115363in}}%
\pgfpathlineto{\pgfqpoint{5.813379in}{1.718851in}}%
\pgfpathlineto{\pgfqpoint{5.900111in}{2.108780in}}%
\pgfpathclose%
\pgfusepath{fill}%
\end{pgfscope}%
\begin{pgfscope}%
\pgfpathrectangle{\pgfqpoint{0.680860in}{0.078740in}}{\pgfqpoint{7.842520in}{7.842520in}}%
\pgfusepath{clip}%
\pgfsetbuttcap%
\pgfsetroundjoin%
\definecolor{currentfill}{rgb}{0.121380,0.629492,0.531973}%
\pgfsetfillcolor{currentfill}%
\pgfsetlinewidth{0.000000pt}%
\definecolor{currentstroke}{rgb}{0.876168,0.891125,0.095250}%
\pgfsetstrokecolor{currentstroke}%
\pgfsetdash{}{0pt}%
\pgfpathmoveto{\pgfqpoint{5.922420in}{2.838066in}}%
\pgfpathlineto{\pgfqpoint{6.008395in}{3.175211in}}%
\pgfpathlineto{\pgfqpoint{5.857674in}{3.155069in}}%
\pgfpathclose%
\pgfusepath{fill}%
\end{pgfscope}%
\begin{pgfscope}%
\pgfpathrectangle{\pgfqpoint{0.680860in}{0.078740in}}{\pgfqpoint{7.842520in}{7.842520in}}%
\pgfusepath{clip}%
\pgfsetbuttcap%
\pgfsetroundjoin%
\definecolor{currentfill}{rgb}{0.159194,0.482237,0.558073}%
\pgfsetfillcolor{currentfill}%
\pgfsetlinewidth{0.000000pt}%
\definecolor{currentstroke}{rgb}{0.886271,0.892374,0.095374}%
\pgfsetstrokecolor{currentstroke}%
\pgfsetdash{}{0pt}%
\pgfpathmoveto{\pgfqpoint{5.836269in}{2.483804in}}%
\pgfpathlineto{\pgfqpoint{5.750044in}{2.115363in}}%
\pgfpathlineto{\pgfqpoint{5.986853in}{2.487279in}}%
\pgfpathclose%
\pgfusepath{fill}%
\end{pgfscope}%
\begin{pgfscope}%
\pgfpathrectangle{\pgfqpoint{0.680860in}{0.078740in}}{\pgfqpoint{7.842520in}{7.842520in}}%
\pgfusepath{clip}%
\pgfsetbuttcap%
\pgfsetroundjoin%
\definecolor{currentfill}{rgb}{0.150148,0.676631,0.506589}%
\pgfsetfillcolor{currentfill}%
\pgfsetlinewidth{0.000000pt}%
\definecolor{currentstroke}{rgb}{0.896320,0.893616,0.096335}%
\pgfsetstrokecolor{currentstroke}%
\pgfsetdash{}{0pt}%
\pgfpathmoveto{\pgfqpoint{5.857674in}{3.155069in}}%
\pgfpathlineto{\pgfqpoint{6.008395in}{3.175211in}}%
\pgfpathlineto{\pgfqpoint{6.094083in}{3.492441in}}%
\pgfpathclose%
\pgfusepath{fill}%
\end{pgfscope}%
\begin{pgfscope}%
\pgfpathrectangle{\pgfqpoint{0.680860in}{0.078740in}}{\pgfqpoint{7.842520in}{7.842520in}}%
\pgfusepath{clip}%
\pgfsetbuttcap%
\pgfsetroundjoin%
\definecolor{currentfill}{rgb}{0.369214,0.788888,0.382914}%
\pgfsetfillcolor{currentfill}%
\pgfsetlinewidth{0.000000pt}%
\definecolor{currentstroke}{rgb}{0.906311,0.894855,0.098125}%
\pgfsetstrokecolor{currentstroke}%
\pgfsetdash{}{0pt}%
\pgfpathmoveto{\pgfqpoint{6.027896in}{3.751220in}}%
\pgfpathlineto{\pgfqpoint{6.179365in}{3.787136in}}%
\pgfpathlineto{\pgfqpoint{6.264117in}{4.056898in}}%
\pgfpathclose%
\pgfusepath{fill}%
\end{pgfscope}%
\begin{pgfscope}%
\pgfpathrectangle{\pgfqpoint{0.680860in}{0.078740in}}{\pgfqpoint{7.842520in}{7.842520in}}%
\pgfusepath{clip}%
\pgfsetbuttcap%
\pgfsetroundjoin%
\definecolor{currentfill}{rgb}{0.246070,0.738910,0.452024}%
\pgfsetfillcolor{currentfill}%
\pgfsetlinewidth{0.000000pt}%
\definecolor{currentstroke}{rgb}{0.916242,0.896091,0.100717}%
\pgfsetstrokecolor{currentstroke}%
\pgfsetdash{}{0pt}%
\pgfpathmoveto{\pgfqpoint{6.179365in}{3.787136in}}%
\pgfpathlineto{\pgfqpoint{5.942966in}{3.464060in}}%
\pgfpathlineto{\pgfqpoint{6.094083in}{3.492441in}}%
\pgfpathclose%
\pgfusepath{fill}%
\end{pgfscope}%
\begin{pgfscope}%
\pgfpathrectangle{\pgfqpoint{0.680860in}{0.078740in}}{\pgfqpoint{7.842520in}{7.842520in}}%
\pgfusepath{clip}%
\pgfsetbuttcap%
\pgfsetroundjoin%
\definecolor{currentfill}{rgb}{0.169646,0.456262,0.558030}%
\pgfsetfillcolor{currentfill}%
\pgfsetlinewidth{0.000000pt}%
\definecolor{currentstroke}{rgb}{0.926106,0.897330,0.104071}%
\pgfsetstrokecolor{currentstroke}%
\pgfsetdash{}{0pt}%
\pgfpathmoveto{\pgfqpoint{5.986853in}{2.487279in}}%
\pgfpathlineto{\pgfqpoint{5.750044in}{2.115363in}}%
\pgfpathlineto{\pgfqpoint{5.900111in}{2.108780in}}%
\pgfpathclose%
\pgfusepath{fill}%
\end{pgfscope}%
\begin{pgfscope}%
\pgfpathrectangle{\pgfqpoint{0.680860in}{0.078740in}}{\pgfqpoint{7.842520in}{7.842520in}}%
\pgfusepath{clip}%
\pgfsetbuttcap%
\pgfsetroundjoin%
\definecolor{currentfill}{rgb}{0.139147,0.533812,0.555298}%
\pgfsetfillcolor{currentfill}%
\pgfsetlinewidth{0.000000pt}%
\definecolor{currentstroke}{rgb}{0.935904,0.898570,0.108131}%
\pgfsetstrokecolor{currentstroke}%
\pgfsetdash{}{0pt}%
\pgfpathmoveto{\pgfqpoint{5.922420in}{2.838066in}}%
\pgfpathlineto{\pgfqpoint{5.836269in}{2.483804in}}%
\pgfpathlineto{\pgfqpoint{5.986853in}{2.487279in}}%
\pgfpathclose%
\pgfusepath{fill}%
\end{pgfscope}%
\begin{pgfscope}%
\pgfpathrectangle{\pgfqpoint{0.680860in}{0.078740in}}{\pgfqpoint{7.842520in}{7.842520in}}%
\pgfusepath{clip}%
\pgfsetbuttcap%
\pgfsetroundjoin%
\definecolor{currentfill}{rgb}{0.804182,0.882046,0.114965}%
\pgfsetfillcolor{currentfill}%
\pgfsetlinewidth{0.000000pt}%
\definecolor{currentstroke}{rgb}{0.945636,0.899815,0.112838}%
\pgfsetstrokecolor{currentstroke}%
\pgfsetdash{}{0pt}%
\pgfpathmoveto{\pgfqpoint{6.584700in}{4.570124in}}%
\pgfpathlineto{\pgfqpoint{6.667209in}{4.757496in}}%
\pgfpathlineto{\pgfqpoint{6.431517in}{4.513368in}}%
\pgfpathclose%
\pgfusepath{fill}%
\end{pgfscope}%
\begin{pgfscope}%
\pgfpathrectangle{\pgfqpoint{0.680860in}{0.078740in}}{\pgfqpoint{7.842520in}{7.842520in}}%
\pgfusepath{clip}%
\pgfsetbuttcap%
\pgfsetroundjoin%
\definecolor{currentfill}{rgb}{0.668054,0.861999,0.196293}%
\pgfsetfillcolor{currentfill}%
\pgfsetlinewidth{0.000000pt}%
\definecolor{currentstroke}{rgb}{0.955300,0.901065,0.118128}%
\pgfsetstrokecolor{currentstroke}%
\pgfsetdash{}{0pt}%
\pgfpathmoveto{\pgfqpoint{6.348210in}{4.299589in}}%
\pgfpathlineto{\pgfqpoint{6.501197in}{4.351269in}}%
\pgfpathlineto{\pgfqpoint{6.431517in}{4.513368in}}%
\pgfpathclose%
\pgfusepath{fill}%
\end{pgfscope}%
\begin{pgfscope}%
\pgfpathrectangle{\pgfqpoint{0.680860in}{0.078740in}}{\pgfqpoint{7.842520in}{7.842520in}}%
\pgfusepath{clip}%
\pgfsetbuttcap%
\pgfsetroundjoin%
\definecolor{currentfill}{rgb}{0.585678,0.846661,0.249897}%
\pgfsetfillcolor{currentfill}%
\pgfsetlinewidth{0.000000pt}%
\definecolor{currentstroke}{rgb}{0.964894,0.902323,0.123941}%
\pgfsetstrokecolor{currentstroke}%
\pgfsetdash{}{0pt}%
\pgfpathmoveto{\pgfqpoint{6.264117in}{4.056898in}}%
\pgfpathlineto{\pgfqpoint{6.501197in}{4.351269in}}%
\pgfpathlineto{\pgfqpoint{6.348210in}{4.299589in}}%
\pgfpathclose%
\pgfusepath{fill}%
\end{pgfscope}%
\begin{pgfscope}%
\pgfpathrectangle{\pgfqpoint{0.680860in}{0.078740in}}{\pgfqpoint{7.842520in}{7.842520in}}%
\pgfusepath{clip}%
\pgfsetbuttcap%
\pgfsetroundjoin%
\definecolor{currentfill}{rgb}{0.730889,0.871916,0.156029}%
\pgfsetfillcolor{currentfill}%
\pgfsetlinewidth{0.000000pt}%
\definecolor{currentstroke}{rgb}{0.974417,0.903590,0.130215}%
\pgfsetstrokecolor{currentstroke}%
\pgfsetdash{}{0pt}%
\pgfpathmoveto{\pgfqpoint{6.431517in}{4.513368in}}%
\pgfpathlineto{\pgfqpoint{6.501197in}{4.351269in}}%
\pgfpathlineto{\pgfqpoint{6.584700in}{4.570124in}}%
\pgfpathclose%
\pgfusepath{fill}%
\end{pgfscope}%
\begin{pgfscope}%
\pgfpathrectangle{\pgfqpoint{0.680860in}{0.078740in}}{\pgfqpoint{7.842520in}{7.842520in}}%
\pgfusepath{clip}%
\pgfsetbuttcap%
\pgfsetroundjoin%
\definecolor{currentfill}{rgb}{0.124780,0.640461,0.527068}%
\pgfsetfillcolor{currentfill}%
\pgfsetlinewidth{0.000000pt}%
\definecolor{currentstroke}{rgb}{0.983868,0.904867,0.136897}%
\pgfsetstrokecolor{currentstroke}%
\pgfsetdash{}{0pt}%
\pgfpathmoveto{\pgfqpoint{6.159951in}{3.197500in}}%
\pgfpathlineto{\pgfqpoint{6.008395in}{3.175211in}}%
\pgfpathlineto{\pgfqpoint{5.922420in}{2.838066in}}%
\pgfpathclose%
\pgfusepath{fill}%
\end{pgfscope}%
\begin{pgfscope}%
\pgfpathrectangle{\pgfqpoint{0.680860in}{0.078740in}}{\pgfqpoint{7.842520in}{7.842520in}}%
\pgfusepath{clip}%
\pgfsetbuttcap%
\pgfsetroundjoin%
\definecolor{currentfill}{rgb}{0.127568,0.566949,0.550556}%
\pgfsetfillcolor{currentfill}%
\pgfsetlinewidth{0.000000pt}%
\definecolor{currentstroke}{rgb}{0.993248,0.906157,0.143936}%
\pgfsetstrokecolor{currentstroke}%
\pgfsetdash{}{0pt}%
\pgfpathmoveto{\pgfqpoint{5.986853in}{2.487279in}}%
\pgfpathlineto{\pgfqpoint{6.073503in}{2.851203in}}%
\pgfpathlineto{\pgfqpoint{5.922420in}{2.838066in}}%
\pgfpathclose%
\pgfusepath{fill}%
\end{pgfscope}%
\begin{pgfscope}%
\pgfpathrectangle{\pgfqpoint{0.680860in}{0.078740in}}{\pgfqpoint{7.842520in}{7.842520in}}%
\pgfusepath{clip}%
\pgfsetbuttcap%
\pgfsetroundjoin%
\definecolor{currentfill}{rgb}{0.162016,0.687316,0.499129}%
\pgfsetfillcolor{currentfill}%
\pgfsetlinewidth{0.000000pt}%
\definecolor{currentstroke}{rgb}{0.267004,0.004874,0.329415}%
\pgfsetstrokecolor{currentstroke}%
\pgfsetdash{}{0pt}%
\pgfpathmoveto{\pgfqpoint{6.094083in}{3.492441in}}%
\pgfpathlineto{\pgfqpoint{6.008395in}{3.175211in}}%
\pgfpathlineto{\pgfqpoint{6.159951in}{3.197500in}}%
\pgfpathclose%
\pgfusepath{fill}%
\end{pgfscope}%
\begin{pgfscope}%
\pgfpathrectangle{\pgfqpoint{0.680860in}{0.078740in}}{\pgfqpoint{7.842520in}{7.842520in}}%
\pgfusepath{clip}%
\pgfsetbuttcap%
\pgfsetroundjoin%
\definecolor{currentfill}{rgb}{0.404001,0.800275,0.362552}%
\pgfsetfillcolor{currentfill}%
\pgfsetlinewidth{0.000000pt}%
\definecolor{currentstroke}{rgb}{0.268510,0.009605,0.335427}%
\pgfsetstrokecolor{currentstroke}%
\pgfsetdash{}{0pt}%
\pgfpathmoveto{\pgfqpoint{6.264117in}{4.056898in}}%
\pgfpathlineto{\pgfqpoint{6.179365in}{3.787136in}}%
\pgfpathlineto{\pgfqpoint{6.331748in}{3.825766in}}%
\pgfpathclose%
\pgfusepath{fill}%
\end{pgfscope}%
\begin{pgfscope}%
\pgfpathrectangle{\pgfqpoint{0.680860in}{0.078740in}}{\pgfqpoint{7.842520in}{7.842520in}}%
\pgfusepath{clip}%
\pgfsetbuttcap%
\pgfsetroundjoin%
\definecolor{currentfill}{rgb}{0.266941,0.748751,0.440573}%
\pgfsetfillcolor{currentfill}%
\pgfsetlinewidth{0.000000pt}%
\definecolor{currentstroke}{rgb}{0.269944,0.014625,0.341379}%
\pgfsetstrokecolor{currentstroke}%
\pgfsetdash{}{0pt}%
\pgfpathmoveto{\pgfqpoint{6.094083in}{3.492441in}}%
\pgfpathlineto{\pgfqpoint{6.246076in}{3.523263in}}%
\pgfpathlineto{\pgfqpoint{6.179365in}{3.787136in}}%
\pgfpathclose%
\pgfusepath{fill}%
\end{pgfscope}%
\begin{pgfscope}%
\pgfpathrectangle{\pgfqpoint{0.680860in}{0.078740in}}{\pgfqpoint{7.842520in}{7.842520in}}%
\pgfusepath{clip}%
\pgfsetbuttcap%
\pgfsetroundjoin%
\definecolor{currentfill}{rgb}{0.120081,0.622161,0.534946}%
\pgfsetfillcolor{currentfill}%
\pgfsetlinewidth{0.000000pt}%
\definecolor{currentstroke}{rgb}{0.271305,0.019942,0.347269}%
\pgfsetstrokecolor{currentstroke}%
\pgfsetdash{}{0pt}%
\pgfpathmoveto{\pgfqpoint{5.922420in}{2.838066in}}%
\pgfpathlineto{\pgfqpoint{6.073503in}{2.851203in}}%
\pgfpathlineto{\pgfqpoint{6.159951in}{3.197500in}}%
\pgfpathclose%
\pgfusepath{fill}%
\end{pgfscope}%
\begin{pgfscope}%
\pgfpathrectangle{\pgfqpoint{0.680860in}{0.078740in}}{\pgfqpoint{7.842520in}{7.842520in}}%
\pgfusepath{clip}%
\pgfsetbuttcap%
\pgfsetroundjoin%
\definecolor{currentfill}{rgb}{0.565498,0.842430,0.262877}%
\pgfsetfillcolor{currentfill}%
\pgfsetlinewidth{0.000000pt}%
\definecolor{currentstroke}{rgb}{0.272594,0.025563,0.353093}%
\pgfsetstrokecolor{currentstroke}%
\pgfsetdash{}{0pt}%
\pgfpathmoveto{\pgfqpoint{6.416834in}{4.102510in}}%
\pgfpathlineto{\pgfqpoint{6.501197in}{4.351269in}}%
\pgfpathlineto{\pgfqpoint{6.264117in}{4.056898in}}%
\pgfpathclose%
\pgfusepath{fill}%
\end{pgfscope}%
\begin{pgfscope}%
\pgfpathrectangle{\pgfqpoint{0.680860in}{0.078740in}}{\pgfqpoint{7.842520in}{7.842520in}}%
\pgfusepath{clip}%
\pgfsetbuttcap%
\pgfsetroundjoin%
\definecolor{currentfill}{rgb}{0.468053,0.818921,0.323998}%
\pgfsetfillcolor{currentfill}%
\pgfsetlinewidth{0.000000pt}%
\definecolor{currentstroke}{rgb}{0.273809,0.031497,0.358853}%
\pgfsetstrokecolor{currentstroke}%
\pgfsetdash{}{0pt}%
\pgfpathmoveto{\pgfqpoint{6.264117in}{4.056898in}}%
\pgfpathlineto{\pgfqpoint{6.331748in}{3.825766in}}%
\pgfpathlineto{\pgfqpoint{6.416834in}{4.102510in}}%
\pgfpathclose%
\pgfusepath{fill}%
\end{pgfscope}%
\begin{pgfscope}%
\pgfpathrectangle{\pgfqpoint{0.680860in}{0.078740in}}{\pgfqpoint{7.842520in}{7.842520in}}%
\pgfusepath{clip}%
\pgfsetbuttcap%
\pgfsetroundjoin%
\definecolor{currentfill}{rgb}{0.202219,0.715272,0.476084}%
\pgfsetfillcolor{currentfill}%
\pgfsetlinewidth{0.000000pt}%
\definecolor{currentstroke}{rgb}{0.274952,0.037752,0.364543}%
\pgfsetstrokecolor{currentstroke}%
\pgfsetdash{}{0pt}%
\pgfpathmoveto{\pgfqpoint{6.159951in}{3.197500in}}%
\pgfpathlineto{\pgfqpoint{6.246076in}{3.523263in}}%
\pgfpathlineto{\pgfqpoint{6.094083in}{3.492441in}}%
\pgfpathclose%
\pgfusepath{fill}%
\end{pgfscope}%
\begin{pgfscope}%
\pgfpathrectangle{\pgfqpoint{0.680860in}{0.078740in}}{\pgfqpoint{7.842520in}{7.842520in}}%
\pgfusepath{clip}%
\pgfsetbuttcap%
\pgfsetroundjoin%
\definecolor{currentfill}{rgb}{0.327796,0.773980,0.406640}%
\pgfsetfillcolor{currentfill}%
\pgfsetlinewidth{0.000000pt}%
\definecolor{currentstroke}{rgb}{0.276022,0.044167,0.370164}%
\pgfsetstrokecolor{currentstroke}%
\pgfsetdash{}{0pt}%
\pgfpathmoveto{\pgfqpoint{6.246076in}{3.523263in}}%
\pgfpathlineto{\pgfqpoint{6.331748in}{3.825766in}}%
\pgfpathlineto{\pgfqpoint{6.179365in}{3.787136in}}%
\pgfpathclose%
\pgfusepath{fill}%
\end{pgfscope}%
\begin{pgfscope}%
\pgfsetbuttcap%
\pgfsetmiterjoin%
\definecolor{currentfill}{rgb}{1.000000,1.000000,1.000000}%
\pgfsetfillcolor{currentfill}%
\pgfsetlinewidth{0.000000pt}%
\definecolor{currentstroke}{rgb}{0.000000,0.000000,0.000000}%
\pgfsetstrokecolor{currentstroke}%
\pgfsetstrokeopacity{0.000000}%
\pgfsetdash{}{0pt}%
\pgfpathmoveto{\pgfqpoint{11.463982in}{0.157480in}}%
\pgfpathlineto{\pgfqpoint{11.778943in}{0.157480in}}%
\pgfpathlineto{\pgfqpoint{11.778943in}{7.842520in}}%
\pgfpathlineto{\pgfqpoint{11.463982in}{7.842520in}}%
\pgfpathclose%
\pgfusepath{fill}%
\end{pgfscope}%
\begin{pgfscope}%
\pgfpathrectangle{\pgfqpoint{11.463982in}{0.157480in}}{\pgfqpoint{0.314961in}{7.685039in}}%
\pgfusepath{clip}%
\pgfsetbuttcap%
\pgfsetmiterjoin%
\definecolor{currentfill}{rgb}{1.000000,1.000000,1.000000}%
\pgfsetfillcolor{currentfill}%
\pgfsetlinewidth{0.010037pt}%
\definecolor{currentstroke}{rgb}{1.000000,1.000000,1.000000}%
\pgfsetstrokecolor{currentstroke}%
\pgfsetdash{}{0pt}%
\pgfpathmoveto{\pgfqpoint{11.463982in}{0.157480in}}%
\pgfpathlineto{\pgfqpoint{11.463982in}{0.187500in}}%
\pgfpathlineto{\pgfqpoint{11.463982in}{7.812500in}}%
\pgfpathlineto{\pgfqpoint{11.463982in}{7.842520in}}%
\pgfpathlineto{\pgfqpoint{11.778943in}{7.842520in}}%
\pgfpathlineto{\pgfqpoint{11.778943in}{7.812500in}}%
\pgfpathlineto{\pgfqpoint{11.778943in}{0.187500in}}%
\pgfpathlineto{\pgfqpoint{11.778943in}{0.157480in}}%
\pgfpathlineto{\pgfqpoint{11.778943in}{0.157480in}}%
\pgfpathclose%
\pgfusepath{stroke,fill}%
\end{pgfscope}%
\begin{pgfscope}%
\pgfsys@transformshift{11.460000in}{0.160000in}%
\pgftext[left,bottom]{\includegraphics[interpolate=true,width=0.320000in,height=7.680000in]{known1-img0.png}}%
\end{pgfscope}%
\begin{pgfscope}%
\pgfsetbuttcap%
\pgfsetroundjoin%
\definecolor{currentfill}{rgb}{0.000000,0.000000,0.000000}%
\pgfsetfillcolor{currentfill}%
\pgfsetlinewidth{0.501875pt}%
\definecolor{currentstroke}{rgb}{0.000000,0.000000,0.000000}%
\pgfsetstrokecolor{currentstroke}%
\pgfsetdash{}{0pt}%
\pgfsys@defobject{currentmarker}{\pgfqpoint{-0.034722in}{0.000000in}}{\pgfqpoint{-0.000000in}{0.000000in}}{%
\pgfpathmoveto{\pgfqpoint{-0.000000in}{0.000000in}}%
\pgfpathlineto{\pgfqpoint{-0.034722in}{0.000000in}}%
\pgfusepath{stroke,fill}%
}%
\begin{pgfscope}%
\pgfsys@transformshift{11.778943in}{1.367301in}%
\pgfsys@useobject{currentmarker}{}%
\end{pgfscope}%
\end{pgfscope}%
\begin{pgfscope}%
\definecolor{textcolor}{rgb}{0.980392,0.811765,0.352941}%
\pgfsetstrokecolor{textcolor}%
\pgfsetfillcolor{textcolor}%
\pgftext[x=11.327554in, y=1.325092in, right, base]{\color{textcolor}\sffamily\fontsize{18.000000}{9.600000}\selectfont $\displaystyle 0.5$}%
\end{pgfscope}%
\begin{pgfscope}%
\pgfsetbuttcap%
\pgfsetroundjoin%
\definecolor{currentfill}{rgb}{0.000000,0.000000,0.000000}%
\pgfsetfillcolor{currentfill}%
\pgfsetlinewidth{0.501875pt}%
\definecolor{currentstroke}{rgb}{0.000000,0.000000,0.000000}%
\pgfsetstrokecolor{currentstroke}%
\pgfsetdash{}{0pt}%
\pgfsys@defobject{currentmarker}{\pgfqpoint{-0.034722in}{0.000000in}}{\pgfqpoint{-0.000000in}{0.000000in}}{%
\pgfpathmoveto{\pgfqpoint{-0.000000in}{0.000000in}}%
\pgfpathlineto{\pgfqpoint{-0.034722in}{0.000000in}}%
\pgfusepath{stroke,fill}%
}%
\begin{pgfscope}%
\pgfsys@transformshift{11.778943in}{2.883891in}%
\pgfsys@useobject{currentmarker}{}%
\end{pgfscope}%
\end{pgfscope}%
\begin{pgfscope}%
\definecolor{textcolor}{rgb}{0.980392,0.811765,0.352941}%
\pgfsetstrokecolor{textcolor}%
\pgfsetfillcolor{textcolor}%
\pgftext[x=11.327554in, y=2.841681in, right, base]{\color{textcolor}\sffamily\fontsize{18.000000}{9.600000}\selectfont $\displaystyle 1.0$}%
\end{pgfscope}%
\begin{pgfscope}%
\pgfsetbuttcap%
\pgfsetroundjoin%
\definecolor{currentfill}{rgb}{0.000000,0.000000,0.000000}%
\pgfsetfillcolor{currentfill}%
\pgfsetlinewidth{0.501875pt}%
\definecolor{currentstroke}{rgb}{0.000000,0.000000,0.000000}%
\pgfsetstrokecolor{currentstroke}%
\pgfsetdash{}{0pt}%
\pgfsys@defobject{currentmarker}{\pgfqpoint{-0.034722in}{0.000000in}}{\pgfqpoint{-0.000000in}{0.000000in}}{%
\pgfpathmoveto{\pgfqpoint{-0.000000in}{0.000000in}}%
\pgfpathlineto{\pgfqpoint{-0.034722in}{0.000000in}}%
\pgfusepath{stroke,fill}%
}%
\begin{pgfscope}%
\pgfsys@transformshift{11.778943in}{4.400480in}%
\pgfsys@useobject{currentmarker}{}%
\end{pgfscope}%
\end{pgfscope}%
\begin{pgfscope}%
\definecolor{textcolor}{rgb}{0.980392,0.811765,0.352941}%
\pgfsetstrokecolor{textcolor}%
\pgfsetfillcolor{textcolor}%
\pgftext[x=11.327554in, y=4.358271in, right, base]{\color{textcolor}\sffamily\fontsize{18.000000}{9.600000}\selectfont $\displaystyle 1.5$}%
\end{pgfscope}%
\begin{pgfscope}%
\pgfsetbuttcap%
\pgfsetroundjoin%
\definecolor{currentfill}{rgb}{0.000000,0.000000,0.000000}%
\pgfsetfillcolor{currentfill}%
\pgfsetlinewidth{0.501875pt}%
\definecolor{currentstroke}{rgb}{0.000000,0.000000,0.000000}%
\pgfsetstrokecolor{currentstroke}%
\pgfsetdash{}{0pt}%
\pgfsys@defobject{currentmarker}{\pgfqpoint{-0.034722in}{0.000000in}}{\pgfqpoint{-0.000000in}{0.000000in}}{%
\pgfpathmoveto{\pgfqpoint{-0.000000in}{0.000000in}}%
\pgfpathlineto{\pgfqpoint{-0.034722in}{0.000000in}}%
\pgfusepath{stroke,fill}%
}%
\begin{pgfscope}%
\pgfsys@transformshift{11.778943in}{5.917069in}%
\pgfsys@useobject{currentmarker}{}%
\end{pgfscope}%
\end{pgfscope}%
\begin{pgfscope}%
\definecolor{textcolor}{rgb}{0.980392,0.811765,0.352941}%
\pgfsetstrokecolor{textcolor}%
\pgfsetfillcolor{textcolor}%
\pgftext[x=11.327554in, y=5.874860in, right, base]{\color{textcolor}\sffamily\fontsize{18.000000}{9.600000}\selectfont $\displaystyle 2.0$}%
\end{pgfscope}%
\begin{pgfscope}%
\pgfsetbuttcap%
\pgfsetroundjoin%
\definecolor{currentfill}{rgb}{0.000000,0.000000,0.000000}%
\pgfsetfillcolor{currentfill}%
\pgfsetlinewidth{0.501875pt}%
\definecolor{currentstroke}{rgb}{0.000000,0.000000,0.000000}%
\pgfsetstrokecolor{currentstroke}%
\pgfsetdash{}{0pt}%
\pgfsys@defobject{currentmarker}{\pgfqpoint{-0.034722in}{0.000000in}}{\pgfqpoint{-0.000000in}{0.000000in}}{%
\pgfpathmoveto{\pgfqpoint{-0.000000in}{0.000000in}}%
\pgfpathlineto{\pgfqpoint{-0.034722in}{0.000000in}}%
\pgfusepath{stroke,fill}%
}%
\begin{pgfscope}%
\pgfsys@transformshift{11.778943in}{7.433659in}%
\pgfsys@useobject{currentmarker}{}%
\end{pgfscope}%
\end{pgfscope}%
\begin{pgfscope}%
\definecolor{textcolor}{rgb}{0.980392,0.811765,0.352941}%
\pgfsetstrokecolor{textcolor}%
\pgfsetfillcolor{textcolor}%
\pgftext[x=11.327554in, y=7.391449in, right, base]{\color{textcolor}\sffamily\fontsize{18.000000}{9.600000}\selectfont $\displaystyle 2.5$}%
\end{pgfscope}%
\begin{pgfscope}%
\pgfsetrectcap%
\pgfsetmiterjoin%
\pgfsetlinewidth{1.003750pt}%
\definecolor{currentstroke}{rgb}{0.000000,0.000000,0.000000}%
\pgfsetstrokecolor{currentstroke}%
\pgfsetdash{}{0pt}%
\pgfpathmoveto{\pgfqpoint{11.463982in}{0.157480in}}%
\pgfpathlineto{\pgfqpoint{11.463982in}{0.187500in}}%
\pgfpathlineto{\pgfqpoint{11.463982in}{7.812500in}}%
\pgfpathlineto{\pgfqpoint{11.463982in}{7.842520in}}%
\pgfpathlineto{\pgfqpoint{11.778943in}{7.842520in}}%
\pgfpathlineto{\pgfqpoint{11.778943in}{7.812500in}}%
\pgfpathlineto{\pgfqpoint{11.778943in}{0.187500in}}%
\pgfpathlineto{\pgfqpoint{11.778943in}{0.157480in}}%
\pgfpathclose%
\pgfusepath{stroke}%
\end{pgfscope}%
\end{pgfpicture}%
\makeatother%
\endgroup%
}
	\caption{$h = 2^{-5}$ 时差分逼近解}\label{fig:known1}
\end{figure}

不同步长时误差的 $\mathbb{L}^\infty$ 范数及收敛速度如表 \ref{tab:errorNorm1} 所示. 可以看出, 随着 $h$ 的减小, 误差的 $\mathbb{L}^\infty$ 范数逐渐减小. $h$ 从 $2^{-1}$ 到 $2^{-9}$ 时, 收敛阶接近于 $2$, 但是再缩小网格时, 因 $\lrvv{e_h}_\infty$ 接近于机器精度, 收敛阶又变小了.

\begin{table}[H]\centering\heiti\zihao{-5}
	\caption{不同步长时误差的 $\mathbb{L}^\infty$ 范数及收敛速度}\label{tab:errorNorm1}
	\begin{tabular}{|c|c|c|}\hline
		$-\log_2 h$	&	$\lrvv{e_h}_\infty$	&	$\log_2\(\lrvv{e_h}_\infty / \lrvv{e_{h / 2}}_\infty\)$\\\hline
		$1$	&	$1.94818276337851 \times 10^{-1}$	&	$1.92405$	\\\hline
		$2$	&	$5.13373125805703 \times 10^{-2}$	&	$1.96026$	\\\hline
		$3$	&	$1.31927593688654 \times 10^{-2}$	&	$1.98207$	\\\hline
		$4$	&	$3.33943713696105 \times 10^{-3}$	&	$1.99837$	\\\hline
		$5$	&	$8.35802510806616 \times 10^{-4}$	&	$1.99841$	\\\hline
		$6$	&	$2.09181357501365 \times 10^{-4}$	&	$1.99990$	\\\hline
		$7$	&	$5.22991071318923 \times 10^{-5}$	&	$1.99997$	\\\hline
		$8$	&	$1.30750132527613 \times 10^{-5}$	&	$1.99999$	\\\hline
		$9$	&	$3.26877185896635 \times 10^{-6}$	&	$1.99997$	\\\hline
		$10$	&	$8.17209409920139 \times 10^{-7}$	&	$1.99956$	\\\hline
		$11$	&	$2.04365120159977 \times 10^{-7}$	&	$1.99291$	\\\hline
		$12$	&	$5.13429063708770 \times 10^{-8}$	&	$1.89080$	\\\hline
		$13$	&	$1.38449633979576 \times 10^{-8}$	&	$0.88309$	\\\hline
		$14$	&	$7.50680606564913 \times 10^{-9}$	&	\\\hline
	\end{tabular}
\end{table}

% [0.19481827633785143, 0.05133731258057028, 0.013192759368865437, 0.0033394371369610543, 0.0008358025108066158, 0.00020918135750136457, 5.229910713189234e-5, 1.307501325276128e-5, 3.268771858966346e-6, 8.172094099201388e-7, 2.043651201599772e-7, 5.134290637087702e-8, 1.38449633979576e-8, 7.50680606564913e-9]

\subsection{未知真解算例}

取
\begin{equation*}
	\lb\begin{aligned}
		f(x, y) &= \dfrac{\sin(4\pi xy)}{4\pi xy},\quad (x, y) 
		\in \Omega,\\
		u_1(y) &= u_3(y) = u_2(x) = u_4(x) = 0.
	\end{aligned}\rd
\end{equation*}

$h = 2^{-5}$ 时求得的差分逼近解如图 \ref{fig:unknown1} 所示.

\begin{figure}[H]\centering
	\resizebox{0.9\linewidth}{!}{%% Creator: Matplotlib, PGF backend
%%
%% To include the figure in your LaTeX document, write
%%   \input{<filename>.pgf}
%%
%% Make sure the required packages are loaded in your preamble
%%   \usepackage{pgf}
%%
%% Figures using additional raster images can only be included by \input if
%% they are in the same directory as the main LaTeX file. For loading figures
%% from other directories you can use the `import` package
%%   \usepackage{import}
%%
%% and then include the figures with
%%   \import{<path to file>}{<filename>.pgf}
%%
%% Matplotlib used the following preamble
%%   \usepackage{fontspec}
%%   \setmainfont{DejaVuSerif.ttf}[Path=\detokenize{/Users/quejiahao/.julia/conda/3/lib/python3.9/site-packages/matplotlib/mpl-data/fonts/ttf/}]
%%   \setsansfont{DejaVuSans.ttf}[Path=\detokenize{/Users/quejiahao/.julia/conda/3/lib/python3.9/site-packages/matplotlib/mpl-data/fonts/ttf/}]
%%   \setmonofont{DejaVuSansMono.ttf}[Path=\detokenize{/Users/quejiahao/.julia/conda/3/lib/python3.9/site-packages/matplotlib/mpl-data/fonts/ttf/}]
%%
\begingroup%
\makeatletter%
\begin{pgfpicture}%
\pgfpathrectangle{\pgfpointorigin}{\pgfqpoint{12.000000in}{8.000000in}}%
\pgfusepath{use as bounding box, clip}%
\begin{pgfscope}%
\pgfsetbuttcap%
\pgfsetmiterjoin%
\definecolor{currentfill}{rgb}{0.152941,0.098039,0.141176}%
\pgfsetfillcolor{currentfill}%
\pgfsetlinewidth{0.000000pt}%
\definecolor{currentstroke}{rgb}{1.000000,1.000000,1.000000}%
\pgfsetstrokecolor{currentstroke}%
\pgfsetdash{}{0pt}%
\pgfpathmoveto{\pgfqpoint{0.000000in}{0.000000in}}%
\pgfpathlineto{\pgfqpoint{12.000000in}{0.000000in}}%
\pgfpathlineto{\pgfqpoint{12.000000in}{8.000000in}}%
\pgfpathlineto{\pgfqpoint{0.000000in}{8.000000in}}%
\pgfpathclose%
\pgfusepath{fill}%
\end{pgfscope}%
\begin{pgfscope}%
\pgfsetbuttcap%
\pgfsetmiterjoin%
\definecolor{currentfill}{rgb}{0.152941,0.098039,0.141176}%
\pgfsetfillcolor{currentfill}%
\pgfsetlinewidth{0.000000pt}%
\definecolor{currentstroke}{rgb}{0.000000,0.000000,0.000000}%
\pgfsetstrokecolor{currentstroke}%
\pgfsetstrokeopacity{0.000000}%
\pgfsetdash{}{0pt}%
\pgfpathmoveto{\pgfqpoint{0.539299in}{0.078740in}}%
\pgfpathlineto{\pgfqpoint{8.381819in}{0.078740in}}%
\pgfpathlineto{\pgfqpoint{8.381819in}{7.921260in}}%
\pgfpathlineto{\pgfqpoint{0.539299in}{7.921260in}}%
\pgfpathclose%
\pgfusepath{fill}%
\end{pgfscope}%
\begin{pgfscope}%
\pgfsetbuttcap%
\pgfsetmiterjoin%
\definecolor{currentfill}{rgb}{0.950000,0.950000,0.950000}%
\pgfsetfillcolor{currentfill}%
\pgfsetfillopacity{0.500000}%
\pgfsetlinewidth{1.003750pt}%
\definecolor{currentstroke}{rgb}{0.950000,0.950000,0.950000}%
\pgfsetstrokecolor{currentstroke}%
\pgfsetstrokeopacity{0.500000}%
\pgfsetdash{}{0pt}%
\pgfpathmoveto{\pgfqpoint{1.131463in}{2.012454in}}%
\pgfpathlineto{\pgfqpoint{3.721319in}{4.183323in}}%
\pgfpathlineto{\pgfqpoint{3.685318in}{7.314104in}}%
\pgfpathlineto{\pgfqpoint{0.971524in}{5.333700in}}%
\pgfusepath{stroke,fill}%
\end{pgfscope}%
\begin{pgfscope}%
\pgfsetbuttcap%
\pgfsetmiterjoin%
\definecolor{currentfill}{rgb}{0.900000,0.900000,0.900000}%
\pgfsetfillcolor{currentfill}%
\pgfsetfillopacity{0.500000}%
\pgfsetlinewidth{1.003750pt}%
\definecolor{currentstroke}{rgb}{0.900000,0.900000,0.900000}%
\pgfsetstrokecolor{currentstroke}%
\pgfsetstrokeopacity{0.500000}%
\pgfsetdash{}{0pt}%
\pgfpathmoveto{\pgfqpoint{3.721319in}{4.183323in}}%
\pgfpathlineto{\pgfqpoint{7.877114in}{2.975397in}}%
\pgfpathlineto{\pgfqpoint{8.025420in}{6.214014in}}%
\pgfpathlineto{\pgfqpoint{3.685318in}{7.314104in}}%
\pgfusepath{stroke,fill}%
\end{pgfscope}%
\begin{pgfscope}%
\pgfsetbuttcap%
\pgfsetmiterjoin%
\definecolor{currentfill}{rgb}{0.925000,0.925000,0.925000}%
\pgfsetfillcolor{currentfill}%
\pgfsetfillopacity{0.500000}%
\pgfsetlinewidth{1.003750pt}%
\definecolor{currentstroke}{rgb}{0.925000,0.925000,0.925000}%
\pgfsetstrokecolor{currentstroke}%
\pgfsetstrokeopacity{0.500000}%
\pgfsetdash{}{0pt}%
\pgfpathmoveto{\pgfqpoint{1.131463in}{2.012454in}}%
\pgfpathlineto{\pgfqpoint{5.536809in}{0.573668in}}%
\pgfpathlineto{\pgfqpoint{7.877114in}{2.975397in}}%
\pgfpathlineto{\pgfqpoint{3.721319in}{4.183323in}}%
\pgfusepath{stroke,fill}%
\end{pgfscope}%
\begin{pgfscope}%
\pgfsetrectcap%
\pgfsetroundjoin%
\pgfsetlinewidth{0.803000pt}%
\definecolor{currentstroke}{rgb}{0.000000,0.000000,0.000000}%
\pgfsetstrokecolor{currentstroke}%
\pgfsetdash{}{0pt}%
\pgfpathmoveto{\pgfqpoint{1.131463in}{2.012454in}}%
\pgfpathlineto{\pgfqpoint{5.536809in}{0.573668in}}%
\pgfusepath{stroke}%
\end{pgfscope}%
\begin{pgfscope}%
\definecolor{textcolor}{rgb}{0.525490,0.694118,0.356863}%
\pgfsetstrokecolor{textcolor}%
\pgfsetfillcolor{textcolor}%
\pgftext[x=3.050786in,y=0.824279in,,]{\color{textcolor}\sffamily\fontsize{24.000000}{13.200000}\bfseries\selectfont $x$}%
\end{pgfscope}%
\begin{pgfscope}%
\pgfsetbuttcap%
\pgfsetroundjoin%
\pgfsetlinewidth{0.803000pt}%
\definecolor{currentstroke}{rgb}{0.690196,0.690196,0.690196}%
\pgfsetstrokecolor{currentstroke}%
\pgfsetdash{}{0pt}%
\pgfpathmoveto{\pgfqpoint{1.945241in}{1.746674in}}%
\pgfpathlineto{\pgfqpoint{4.491766in}{3.959384in}}%
\pgfpathlineto{\pgfqpoint{4.488551in}{7.110508in}}%
\pgfusepath{stroke}%
\end{pgfscope}%
\begin{pgfscope}%
\pgfsetbuttcap%
\pgfsetroundjoin%
\pgfsetlinewidth{0.803000pt}%
\definecolor{currentstroke}{rgb}{0.690196,0.690196,0.690196}%
\pgfsetstrokecolor{currentstroke}%
\pgfsetdash{}{0pt}%
\pgfpathmoveto{\pgfqpoint{2.827694in}{1.458465in}}%
\pgfpathlineto{\pgfqpoint{5.325810in}{3.716961in}}%
\pgfpathlineto{\pgfqpoint{5.358796in}{6.889926in}}%
\pgfusepath{stroke}%
\end{pgfscope}%
\begin{pgfscope}%
\pgfsetbuttcap%
\pgfsetroundjoin%
\pgfsetlinewidth{0.803000pt}%
\definecolor{currentstroke}{rgb}{0.690196,0.690196,0.690196}%
\pgfsetstrokecolor{currentstroke}%
\pgfsetdash{}{0pt}%
\pgfpathmoveto{\pgfqpoint{3.729308in}{1.163998in}}%
\pgfpathlineto{\pgfqpoint{6.176441in}{3.469716in}}%
\pgfpathlineto{\pgfqpoint{6.247107in}{6.664765in}}%
\pgfusepath{stroke}%
\end{pgfscope}%
\begin{pgfscope}%
\pgfsetbuttcap%
\pgfsetroundjoin%
\pgfsetlinewidth{0.803000pt}%
\definecolor{currentstroke}{rgb}{0.690196,0.690196,0.690196}%
\pgfsetstrokecolor{currentstroke}%
\pgfsetdash{}{0pt}%
\pgfpathmoveto{\pgfqpoint{4.650714in}{0.863067in}}%
\pgfpathlineto{\pgfqpoint{7.044159in}{3.217505in}}%
\pgfpathlineto{\pgfqpoint{7.154054in}{6.434880in}}%
\pgfusepath{stroke}%
\end{pgfscope}%
\begin{pgfscope}%
\pgfpathrectangle{\pgfqpoint{0.539299in}{0.078740in}}{\pgfqpoint{7.842520in}{7.842520in}}%
\pgfusepath{clip}%
\pgfsetrectcap%
\pgfsetroundjoin%
\pgfsetlinewidth{0.501875pt}%
\definecolor{currentstroke}{rgb}{0.980392,0.811765,0.352941}%
\pgfsetstrokecolor{currentstroke}%
\pgfsetstrokeopacity{0.100000}%
\pgfsetdash{}{0pt}%
\pgfpathmoveto{\pgfqpoint{4.566539in}{4.105980in}}%
\pgfusepath{stroke}%
\end{pgfscope}%
\begin{pgfscope}%
\pgfsetrectcap%
\pgfsetroundjoin%
\pgfsetlinewidth{0.803000pt}%
\definecolor{currentstroke}{rgb}{0.980392,0.811765,0.352941}%
\pgfsetstrokecolor{currentstroke}%
\pgfsetdash{}{0pt}%
\pgfpathmoveto{\pgfqpoint{1.967428in}{1.765953in}}%
\pgfpathlineto{\pgfqpoint{1.900771in}{1.708034in}}%
\pgfusepath{stroke}%
\end{pgfscope}%
\begin{pgfscope}%
\definecolor{textcolor}{rgb}{0.525490,0.694118,0.356863}%
\pgfsetstrokecolor{textcolor}%
\pgfsetfillcolor{textcolor}%
\pgftext[x=1.841138in,y=1.527121in,,top]{\color{textcolor}\sffamily\fontsize{18.000000}{9.600000}\selectfont $\displaystyle 0.2$}%
\end{pgfscope}%
\begin{pgfscope}%
\pgfpathrectangle{\pgfqpoint{0.539299in}{0.078740in}}{\pgfqpoint{7.842520in}{7.842520in}}%
\pgfusepath{clip}%
\pgfsetrectcap%
\pgfsetroundjoin%
\pgfsetlinewidth{0.501875pt}%
\definecolor{currentstroke}{rgb}{0.980392,0.811765,0.352941}%
\pgfsetstrokecolor{currentstroke}%
\pgfsetstrokeopacity{0.100000}%
\pgfsetdash{}{0pt}%
\pgfpathmoveto{\pgfqpoint{4.566539in}{4.105980in}}%
\pgfusepath{stroke}%
\end{pgfscope}%
\begin{pgfscope}%
\pgfsetrectcap%
\pgfsetroundjoin%
\pgfsetlinewidth{0.803000pt}%
\definecolor{currentstroke}{rgb}{0.980392,0.811765,0.352941}%
\pgfsetstrokecolor{currentstroke}%
\pgfsetdash{}{0pt}%
\pgfpathmoveto{\pgfqpoint{2.849478in}{1.478160in}}%
\pgfpathlineto{\pgfqpoint{2.784030in}{1.418990in}}%
\pgfusepath{stroke}%
\end{pgfscope}%
\begin{pgfscope}%
\definecolor{textcolor}{rgb}{0.525490,0.694118,0.356863}%
\pgfsetstrokecolor{textcolor}%
\pgfsetfillcolor{textcolor}%
\pgftext[x=2.724041in,y=1.236037in,,top]{\color{textcolor}\sffamily\fontsize{18.000000}{9.600000}\selectfont $\displaystyle 0.4$}%
\end{pgfscope}%
\begin{pgfscope}%
\pgfpathrectangle{\pgfqpoint{0.539299in}{0.078740in}}{\pgfqpoint{7.842520in}{7.842520in}}%
\pgfusepath{clip}%
\pgfsetrectcap%
\pgfsetroundjoin%
\pgfsetlinewidth{0.501875pt}%
\definecolor{currentstroke}{rgb}{0.980392,0.811765,0.352941}%
\pgfsetstrokecolor{currentstroke}%
\pgfsetstrokeopacity{0.100000}%
\pgfsetdash{}{0pt}%
\pgfpathmoveto{\pgfqpoint{4.566539in}{4.105980in}}%
\pgfusepath{stroke}%
\end{pgfscope}%
\begin{pgfscope}%
\pgfsetrectcap%
\pgfsetroundjoin%
\pgfsetlinewidth{0.803000pt}%
\definecolor{currentstroke}{rgb}{0.980392,0.811765,0.352941}%
\pgfsetstrokecolor{currentstroke}%
\pgfsetdash{}{0pt}%
\pgfpathmoveto{\pgfqpoint{3.750667in}{1.184123in}}%
\pgfpathlineto{\pgfqpoint{3.686496in}{1.123660in}}%
\pgfusepath{stroke}%
\end{pgfscope}%
\begin{pgfscope}%
\definecolor{textcolor}{rgb}{0.525490,0.694118,0.356863}%
\pgfsetstrokecolor{textcolor}%
\pgfsetfillcolor{textcolor}%
\pgftext[x=3.626149in,y=0.938621in,,top]{\color{textcolor}\sffamily\fontsize{18.000000}{9.600000}\selectfont $\displaystyle 0.6$}%
\end{pgfscope}%
\begin{pgfscope}%
\pgfpathrectangle{\pgfqpoint{0.539299in}{0.078740in}}{\pgfqpoint{7.842520in}{7.842520in}}%
\pgfusepath{clip}%
\pgfsetrectcap%
\pgfsetroundjoin%
\pgfsetlinewidth{0.501875pt}%
\definecolor{currentstroke}{rgb}{0.980392,0.811765,0.352941}%
\pgfsetstrokecolor{currentstroke}%
\pgfsetstrokeopacity{0.100000}%
\pgfsetdash{}{0pt}%
\pgfpathmoveto{\pgfqpoint{4.566539in}{4.105980in}}%
\pgfusepath{stroke}%
\end{pgfscope}%
\begin{pgfscope}%
\pgfsetrectcap%
\pgfsetroundjoin%
\pgfsetlinewidth{0.803000pt}%
\definecolor{currentstroke}{rgb}{0.980392,0.811765,0.352941}%
\pgfsetstrokecolor{currentstroke}%
\pgfsetdash{}{0pt}%
\pgfpathmoveto{\pgfqpoint{4.671624in}{0.883635in}}%
\pgfpathlineto{\pgfqpoint{4.608802in}{0.821838in}}%
\pgfusepath{stroke}%
\end{pgfscope}%
\begin{pgfscope}%
\definecolor{textcolor}{rgb}{0.525490,0.694118,0.356863}%
\pgfsetstrokecolor{textcolor}%
\pgfsetfillcolor{textcolor}%
\pgftext[x=4.548096in,y=0.634665in,,top]{\color{textcolor}\sffamily\fontsize{18.000000}{9.600000}\selectfont $\displaystyle 0.8$}%
\end{pgfscope}%
\begin{pgfscope}%
\pgfsetrectcap%
\pgfsetroundjoin%
\pgfsetlinewidth{0.803000pt}%
\definecolor{currentstroke}{rgb}{0.000000,0.000000,0.000000}%
\pgfsetstrokecolor{currentstroke}%
\pgfsetdash{}{0pt}%
\pgfpathmoveto{\pgfqpoint{7.877114in}{2.975397in}}%
\pgfpathlineto{\pgfqpoint{5.536809in}{0.573668in}}%
\pgfusepath{stroke}%
\end{pgfscope}%
\begin{pgfscope}%
\definecolor{textcolor}{rgb}{0.525490,0.694118,0.356863}%
\pgfsetstrokecolor{textcolor}%
\pgfsetfillcolor{textcolor}%
\pgftext[x=7.133707in,y=1.439527in,,]{\color{textcolor}\sffamily\fontsize{24.000000}{13.200000}\bfseries\selectfont $y$}%
\end{pgfscope}%
\begin{pgfscope}%
\pgfsetbuttcap%
\pgfsetroundjoin%
\pgfsetlinewidth{0.803000pt}%
\definecolor{currentstroke}{rgb}{0.690196,0.690196,0.690196}%
\pgfsetstrokecolor{currentstroke}%
\pgfsetdash{}{0pt}%
\pgfpathmoveto{\pgfqpoint{1.533379in}{5.743717in}}%
\pgfpathlineto{\pgfqpoint{1.666046in}{2.460552in}}%
\pgfpathlineto{\pgfqpoint{6.021576in}{1.071158in}}%
\pgfusepath{stroke}%
\end{pgfscope}%
\begin{pgfscope}%
\pgfsetbuttcap%
\pgfsetroundjoin%
\pgfsetlinewidth{0.803000pt}%
\definecolor{currentstroke}{rgb}{0.690196,0.690196,0.690196}%
\pgfsetstrokecolor{currentstroke}%
\pgfsetdash{}{0pt}%
\pgfpathmoveto{\pgfqpoint{2.108895in}{6.163702in}}%
\pgfpathlineto{\pgfqpoint{2.214498in}{2.920275in}}%
\pgfpathlineto{\pgfqpoint{6.518001in}{1.580612in}}%
\pgfusepath{stroke}%
\end{pgfscope}%
\begin{pgfscope}%
\pgfsetbuttcap%
\pgfsetroundjoin%
\pgfsetlinewidth{0.803000pt}%
\definecolor{currentstroke}{rgb}{0.690196,0.690196,0.690196}%
\pgfsetstrokecolor{currentstroke}%
\pgfsetdash{}{0pt}%
\pgfpathmoveto{\pgfqpoint{2.663627in}{6.568520in}}%
\pgfpathlineto{\pgfqpoint{2.743979in}{3.364097in}}%
\pgfpathlineto{\pgfqpoint{6.996374in}{2.071541in}}%
\pgfusepath{stroke}%
\end{pgfscope}%
\begin{pgfscope}%
\pgfsetbuttcap%
\pgfsetroundjoin%
\pgfsetlinewidth{0.803000pt}%
\definecolor{currentstroke}{rgb}{0.690196,0.690196,0.690196}%
\pgfsetstrokecolor{currentstroke}%
\pgfsetdash{}{0pt}%
\pgfpathmoveto{\pgfqpoint{3.198678in}{6.958976in}}%
\pgfpathlineto{\pgfqpoint{3.255455in}{3.792826in}}%
\pgfpathlineto{\pgfqpoint{7.457662in}{2.544936in}}%
\pgfusepath{stroke}%
\end{pgfscope}%
\begin{pgfscope}%
\pgfpathrectangle{\pgfqpoint{0.539299in}{0.078740in}}{\pgfqpoint{7.842520in}{7.842520in}}%
\pgfusepath{clip}%
\pgfsetrectcap%
\pgfsetroundjoin%
\pgfsetlinewidth{0.501875pt}%
\definecolor{currentstroke}{rgb}{0.980392,0.811765,0.352941}%
\pgfsetstrokecolor{currentstroke}%
\pgfsetstrokeopacity{0.100000}%
\pgfsetdash{}{0pt}%
\pgfpathmoveto{\pgfqpoint{4.566539in}{4.105980in}}%
\pgfusepath{stroke}%
\end{pgfscope}%
\begin{pgfscope}%
\pgfsetrectcap%
\pgfsetroundjoin%
\pgfsetlinewidth{0.803000pt}%
\definecolor{currentstroke}{rgb}{0.980392,0.811765,0.352941}%
\pgfsetstrokecolor{currentstroke}%
\pgfsetdash{}{0pt}%
\pgfpathmoveto{\pgfqpoint{5.984893in}{1.082859in}}%
\pgfpathlineto{\pgfqpoint{6.095033in}{1.047725in}}%
\pgfusepath{stroke}%
\end{pgfscope}%
\begin{pgfscope}%
\definecolor{textcolor}{rgb}{0.525490,0.694118,0.356863}%
\pgfsetstrokecolor{textcolor}%
\pgfsetfillcolor{textcolor}%
\pgftext[x=6.198378in,y=0.887093in,,top]{\color{textcolor}\sffamily\fontsize{18.000000}{9.600000}\selectfont $\displaystyle 0.2$}%
\end{pgfscope}%
\begin{pgfscope}%
\pgfpathrectangle{\pgfqpoint{0.539299in}{0.078740in}}{\pgfqpoint{7.842520in}{7.842520in}}%
\pgfusepath{clip}%
\pgfsetrectcap%
\pgfsetroundjoin%
\pgfsetlinewidth{0.501875pt}%
\definecolor{currentstroke}{rgb}{0.980392,0.811765,0.352941}%
\pgfsetstrokecolor{currentstroke}%
\pgfsetstrokeopacity{0.100000}%
\pgfsetdash{}{0pt}%
\pgfpathmoveto{\pgfqpoint{4.566539in}{4.105980in}}%
\pgfusepath{stroke}%
\end{pgfscope}%
\begin{pgfscope}%
\pgfsetrectcap%
\pgfsetroundjoin%
\pgfsetlinewidth{0.803000pt}%
\definecolor{currentstroke}{rgb}{0.980392,0.811765,0.352941}%
\pgfsetstrokecolor{currentstroke}%
\pgfsetdash{}{0pt}%
\pgfpathmoveto{\pgfqpoint{6.481791in}{1.591885in}}%
\pgfpathlineto{\pgfqpoint{6.590512in}{1.558040in}}%
\pgfusepath{stroke}%
\end{pgfscope}%
\begin{pgfscope}%
\definecolor{textcolor}{rgb}{0.525490,0.694118,0.356863}%
\pgfsetstrokecolor{textcolor}%
\pgfsetfillcolor{textcolor}%
\pgftext[x=6.691664in,y=1.400097in,,top]{\color{textcolor}\sffamily\fontsize{18.000000}{9.600000}\selectfont $\displaystyle 0.4$}%
\end{pgfscope}%
\begin{pgfscope}%
\pgfpathrectangle{\pgfqpoint{0.539299in}{0.078740in}}{\pgfqpoint{7.842520in}{7.842520in}}%
\pgfusepath{clip}%
\pgfsetrectcap%
\pgfsetroundjoin%
\pgfsetlinewidth{0.501875pt}%
\definecolor{currentstroke}{rgb}{0.980392,0.811765,0.352941}%
\pgfsetstrokecolor{currentstroke}%
\pgfsetstrokeopacity{0.100000}%
\pgfsetdash{}{0pt}%
\pgfpathmoveto{\pgfqpoint{4.566539in}{4.105980in}}%
\pgfusepath{stroke}%
\end{pgfscope}%
\begin{pgfscope}%
\pgfsetrectcap%
\pgfsetroundjoin%
\pgfsetlinewidth{0.803000pt}%
\definecolor{currentstroke}{rgb}{0.980392,0.811765,0.352941}%
\pgfsetstrokecolor{currentstroke}%
\pgfsetdash{}{0pt}%
\pgfpathmoveto{\pgfqpoint{6.960625in}{2.082407in}}%
\pgfpathlineto{\pgfqpoint{7.067958in}{2.049782in}}%
\pgfusepath{stroke}%
\end{pgfscope}%
\begin{pgfscope}%
\definecolor{textcolor}{rgb}{0.525490,0.694118,0.356863}%
\pgfsetstrokecolor{textcolor}%
\pgfsetfillcolor{textcolor}%
\pgftext[x=7.167007in,y=1.894440in,,top]{\color{textcolor}\sffamily\fontsize{18.000000}{9.600000}\selectfont $\displaystyle 0.6$}%
\end{pgfscope}%
\begin{pgfscope}%
\pgfpathrectangle{\pgfqpoint{0.539299in}{0.078740in}}{\pgfqpoint{7.842520in}{7.842520in}}%
\pgfusepath{clip}%
\pgfsetrectcap%
\pgfsetroundjoin%
\pgfsetlinewidth{0.501875pt}%
\definecolor{currentstroke}{rgb}{0.980392,0.811765,0.352941}%
\pgfsetstrokecolor{currentstroke}%
\pgfsetstrokeopacity{0.100000}%
\pgfsetdash{}{0pt}%
\pgfpathmoveto{\pgfqpoint{4.566539in}{4.105980in}}%
\pgfusepath{stroke}%
\end{pgfscope}%
\begin{pgfscope}%
\pgfsetrectcap%
\pgfsetroundjoin%
\pgfsetlinewidth{0.803000pt}%
\definecolor{currentstroke}{rgb}{0.980392,0.811765,0.352941}%
\pgfsetstrokecolor{currentstroke}%
\pgfsetdash{}{0pt}%
\pgfpathmoveto{\pgfqpoint{7.422366in}{2.555417in}}%
\pgfpathlineto{\pgfqpoint{7.528339in}{2.523947in}}%
\pgfusepath{stroke}%
\end{pgfscope}%
\begin{pgfscope}%
\definecolor{textcolor}{rgb}{0.525490,0.694118,0.356863}%
\pgfsetstrokecolor{textcolor}%
\pgfsetfillcolor{textcolor}%
\pgftext[x=7.625368in,y=2.371123in,,top]{\color{textcolor}\sffamily\fontsize{18.000000}{9.600000}\selectfont $\displaystyle 0.8$}%
\end{pgfscope}%
\begin{pgfscope}%
\pgfsetrectcap%
\pgfsetroundjoin%
\pgfsetlinewidth{0.803000pt}%
\definecolor{currentstroke}{rgb}{0.000000,0.000000,0.000000}%
\pgfsetstrokecolor{currentstroke}%
\pgfsetdash{}{0pt}%
\pgfpathmoveto{\pgfqpoint{7.877114in}{2.975397in}}%
\pgfpathlineto{\pgfqpoint{8.025420in}{6.214014in}}%
\pgfusepath{stroke}%
\end{pgfscope}%
\begin{pgfscope}%
\definecolor{textcolor}{rgb}{0.525490,0.694118,0.356863}%
\pgfsetstrokecolor{textcolor}%
\pgfsetfillcolor{textcolor}%
\pgftext[x=8.812392in,y=4.634624in,,]{\color{textcolor}\sffamily\fontsize{24.000000}{13.200000}\bfseries\selectfont $U$}%
\end{pgfscope}%
\begin{pgfscope}%
\pgfsetbuttcap%
\pgfsetroundjoin%
\pgfsetlinewidth{0.803000pt}%
\definecolor{currentstroke}{rgb}{0.690196,0.690196,0.690196}%
\pgfsetstrokecolor{currentstroke}%
\pgfsetdash{}{0pt}%
\pgfpathmoveto{\pgfqpoint{7.903878in}{3.559855in}}%
\pgfpathlineto{\pgfqpoint{3.714811in}{4.749298in}}%
\pgfpathlineto{\pgfqpoint{1.102640in}{2.610996in}}%
\pgfusepath{stroke}%
\end{pgfscope}%
\begin{pgfscope}%
\pgfsetbuttcap%
\pgfsetroundjoin%
\pgfsetlinewidth{0.803000pt}%
\definecolor{currentstroke}{rgb}{0.690196,0.690196,0.690196}%
\pgfsetstrokecolor{currentstroke}%
\pgfsetdash{}{0pt}%
\pgfpathmoveto{\pgfqpoint{7.928165in}{4.090216in}}%
\pgfpathlineto{\pgfqpoint{3.708910in}{5.262514in}}%
\pgfpathlineto{\pgfqpoint{1.076469in}{3.154452in}}%
\pgfusepath{stroke}%
\end{pgfscope}%
\begin{pgfscope}%
\pgfsetbuttcap%
\pgfsetroundjoin%
\pgfsetlinewidth{0.803000pt}%
\definecolor{currentstroke}{rgb}{0.690196,0.690196,0.690196}%
\pgfsetstrokecolor{currentstroke}%
\pgfsetdash{}{0pt}%
\pgfpathmoveto{\pgfqpoint{7.952808in}{4.628352in}}%
\pgfpathlineto{\pgfqpoint{3.702926in}{5.782891in}}%
\pgfpathlineto{\pgfqpoint{1.049899in}{3.706182in}}%
\pgfusepath{stroke}%
\end{pgfscope}%
\begin{pgfscope}%
\pgfsetbuttcap%
\pgfsetroundjoin%
\pgfsetlinewidth{0.803000pt}%
\definecolor{currentstroke}{rgb}{0.690196,0.690196,0.690196}%
\pgfsetstrokecolor{currentstroke}%
\pgfsetdash{}{0pt}%
\pgfpathmoveto{\pgfqpoint{7.977815in}{5.174436in}}%
\pgfpathlineto{\pgfqpoint{3.696858in}{6.310580in}}%
\pgfpathlineto{\pgfqpoint{1.022922in}{4.266378in}}%
\pgfusepath{stroke}%
\end{pgfscope}%
\begin{pgfscope}%
\pgfsetbuttcap%
\pgfsetroundjoin%
\pgfsetlinewidth{0.803000pt}%
\definecolor{currentstroke}{rgb}{0.690196,0.690196,0.690196}%
\pgfsetstrokecolor{currentstroke}%
\pgfsetdash{}{0pt}%
\pgfpathmoveto{\pgfqpoint{8.003194in}{5.728645in}}%
\pgfpathlineto{\pgfqpoint{3.690704in}{6.845738in}}%
\pgfpathlineto{\pgfqpoint{0.995528in}{4.835234in}}%
\pgfusepath{stroke}%
\end{pgfscope}%
\begin{pgfscope}%
\pgfpathrectangle{\pgfqpoint{0.539299in}{0.078740in}}{\pgfqpoint{7.842520in}{7.842520in}}%
\pgfusepath{clip}%
\pgfsetrectcap%
\pgfsetroundjoin%
\pgfsetlinewidth{0.501875pt}%
\definecolor{currentstroke}{rgb}{0.980392,0.811765,0.352941}%
\pgfsetstrokecolor{currentstroke}%
\pgfsetstrokeopacity{0.100000}%
\pgfsetdash{}{0pt}%
\pgfpathmoveto{\pgfqpoint{4.566539in}{4.105980in}}%
\pgfusepath{stroke}%
\end{pgfscope}%
\begin{pgfscope}%
\pgfsetrectcap%
\pgfsetroundjoin%
\pgfsetlinewidth{0.803000pt}%
\definecolor{currentstroke}{rgb}{0.980392,0.811765,0.352941}%
\pgfsetstrokecolor{currentstroke}%
\pgfsetdash{}{0pt}%
\pgfpathmoveto{\pgfqpoint{7.868707in}{3.569842in}}%
\pgfpathlineto{\pgfqpoint{7.974305in}{3.539859in}}%
\pgfusepath{stroke}%
\end{pgfscope}%
\begin{pgfscope}%
\definecolor{textcolor}{rgb}{0.525490,0.694118,0.356863}%
\pgfsetstrokecolor{textcolor}%
\pgfsetfillcolor{textcolor}%
\pgftext[x=8.359745in,y=3.595552in,,top]{\color{textcolor}\sffamily\fontsize{18.000000}{9.600000}\selectfont $\displaystyle 0.005$}%
\end{pgfscope}%
\begin{pgfscope}%
\pgfpathrectangle{\pgfqpoint{0.539299in}{0.078740in}}{\pgfqpoint{7.842520in}{7.842520in}}%
\pgfusepath{clip}%
\pgfsetrectcap%
\pgfsetroundjoin%
\pgfsetlinewidth{0.501875pt}%
\definecolor{currentstroke}{rgb}{0.980392,0.811765,0.352941}%
\pgfsetstrokecolor{currentstroke}%
\pgfsetstrokeopacity{0.100000}%
\pgfsetdash{}{0pt}%
\pgfpathmoveto{\pgfqpoint{4.566539in}{4.105980in}}%
\pgfusepath{stroke}%
\end{pgfscope}%
\begin{pgfscope}%
\pgfsetrectcap%
\pgfsetroundjoin%
\pgfsetlinewidth{0.803000pt}%
\definecolor{currentstroke}{rgb}{0.980392,0.811765,0.352941}%
\pgfsetstrokecolor{currentstroke}%
\pgfsetdash{}{0pt}%
\pgfpathmoveto{\pgfqpoint{7.892728in}{4.100062in}}%
\pgfpathlineto{\pgfqpoint{7.999124in}{4.070500in}}%
\pgfusepath{stroke}%
\end{pgfscope}%
\begin{pgfscope}%
\definecolor{textcolor}{rgb}{0.525490,0.694118,0.356863}%
\pgfsetstrokecolor{textcolor}%
\pgfsetfillcolor{textcolor}%
\pgftext[x=8.385857in,y=4.125410in,,top]{\color{textcolor}\sffamily\fontsize{18.000000}{9.600000}\selectfont $\displaystyle 0.010$}%
\end{pgfscope}%
\begin{pgfscope}%
\pgfpathrectangle{\pgfqpoint{0.539299in}{0.078740in}}{\pgfqpoint{7.842520in}{7.842520in}}%
\pgfusepath{clip}%
\pgfsetrectcap%
\pgfsetroundjoin%
\pgfsetlinewidth{0.501875pt}%
\definecolor{currentstroke}{rgb}{0.980392,0.811765,0.352941}%
\pgfsetstrokecolor{currentstroke}%
\pgfsetstrokeopacity{0.100000}%
\pgfsetdash{}{0pt}%
\pgfpathmoveto{\pgfqpoint{4.566539in}{4.105980in}}%
\pgfusepath{stroke}%
\end{pgfscope}%
\begin{pgfscope}%
\pgfsetrectcap%
\pgfsetroundjoin%
\pgfsetlinewidth{0.803000pt}%
\definecolor{currentstroke}{rgb}{0.980392,0.811765,0.352941}%
\pgfsetstrokecolor{currentstroke}%
\pgfsetdash{}{0pt}%
\pgfpathmoveto{\pgfqpoint{7.917102in}{4.638052in}}%
\pgfpathlineto{\pgfqpoint{8.024308in}{4.608928in}}%
\pgfusepath{stroke}%
\end{pgfscope}%
\begin{pgfscope}%
\definecolor{textcolor}{rgb}{0.525490,0.694118,0.356863}%
\pgfsetstrokecolor{textcolor}%
\pgfsetfillcolor{textcolor}%
\pgftext[x=8.412350in,y=4.663024in,,top]{\color{textcolor}\sffamily\fontsize{18.000000}{9.600000}\selectfont $\displaystyle 0.015$}%
\end{pgfscope}%
\begin{pgfscope}%
\pgfpathrectangle{\pgfqpoint{0.539299in}{0.078740in}}{\pgfqpoint{7.842520in}{7.842520in}}%
\pgfusepath{clip}%
\pgfsetrectcap%
\pgfsetroundjoin%
\pgfsetlinewidth{0.501875pt}%
\definecolor{currentstroke}{rgb}{0.980392,0.811765,0.352941}%
\pgfsetstrokecolor{currentstroke}%
\pgfsetstrokeopacity{0.100000}%
\pgfsetdash{}{0pt}%
\pgfpathmoveto{\pgfqpoint{4.566539in}{4.105980in}}%
\pgfusepath{stroke}%
\end{pgfscope}%
\begin{pgfscope}%
\pgfsetrectcap%
\pgfsetroundjoin%
\pgfsetlinewidth{0.803000pt}%
\definecolor{currentstroke}{rgb}{0.980392,0.811765,0.352941}%
\pgfsetstrokecolor{currentstroke}%
\pgfsetdash{}{0pt}%
\pgfpathmoveto{\pgfqpoint{7.941835in}{5.183985in}}%
\pgfpathlineto{\pgfqpoint{8.049863in}{5.155315in}}%
\pgfusepath{stroke}%
\end{pgfscope}%
\begin{pgfscope}%
\definecolor{textcolor}{rgb}{0.525490,0.694118,0.356863}%
\pgfsetstrokecolor{textcolor}%
\pgfsetfillcolor{textcolor}%
\pgftext[x=8.439234in,y=5.208568in,,top]{\color{textcolor}\sffamily\fontsize{18.000000}{9.600000}\selectfont $\displaystyle 0.020$}%
\end{pgfscope}%
\begin{pgfscope}%
\pgfpathrectangle{\pgfqpoint{0.539299in}{0.078740in}}{\pgfqpoint{7.842520in}{7.842520in}}%
\pgfusepath{clip}%
\pgfsetrectcap%
\pgfsetroundjoin%
\pgfsetlinewidth{0.501875pt}%
\definecolor{currentstroke}{rgb}{0.980392,0.811765,0.352941}%
\pgfsetstrokecolor{currentstroke}%
\pgfsetstrokeopacity{0.100000}%
\pgfsetdash{}{0pt}%
\pgfpathmoveto{\pgfqpoint{4.566539in}{4.105980in}}%
\pgfusepath{stroke}%
\end{pgfscope}%
\begin{pgfscope}%
\pgfsetrectcap%
\pgfsetroundjoin%
\pgfsetlinewidth{0.803000pt}%
\definecolor{currentstroke}{rgb}{0.980392,0.811765,0.352941}%
\pgfsetstrokecolor{currentstroke}%
\pgfsetdash{}{0pt}%
\pgfpathmoveto{\pgfqpoint{7.966936in}{5.738038in}}%
\pgfpathlineto{\pgfqpoint{8.075799in}{5.709838in}}%
\pgfusepath{stroke}%
\end{pgfscope}%
\begin{pgfscope}%
\definecolor{textcolor}{rgb}{0.525490,0.694118,0.356863}%
\pgfsetstrokecolor{textcolor}%
\pgfsetfillcolor{textcolor}%
\pgftext[x=8.466518in,y=5.762217in,,top]{\color{textcolor}\sffamily\fontsize{18.000000}{9.600000}\selectfont $\displaystyle 0.025$}%
\end{pgfscope}%
\begin{pgfscope}%
\pgfpathrectangle{\pgfqpoint{0.539299in}{0.078740in}}{\pgfqpoint{7.842520in}{7.842520in}}%
\pgfusepath{clip}%
\pgfsetbuttcap%
\pgfsetroundjoin%
\definecolor{currentfill}{rgb}{0.283091,0.110553,0.431554}%
\pgfsetfillcolor{currentfill}%
\pgfsetlinewidth{0.000000pt}%
\definecolor{currentstroke}{rgb}{0.267004,0.004874,0.329415}%
\pgfsetstrokecolor{currentstroke}%
\pgfsetdash{}{0pt}%
\pgfpathmoveto{\pgfqpoint{3.799922in}{4.421780in}}%
\pgfpathlineto{\pgfqpoint{3.877896in}{4.341545in}}%
\pgfpathlineto{\pgfqpoint{3.750979in}{4.305549in}}%
\pgfpathclose%
\pgfusepath{fill}%
\end{pgfscope}%
\begin{pgfscope}%
\pgfpathrectangle{\pgfqpoint{0.539299in}{0.078740in}}{\pgfqpoint{7.842520in}{7.842520in}}%
\pgfusepath{clip}%
\pgfsetbuttcap%
\pgfsetroundjoin%
\definecolor{currentfill}{rgb}{0.283091,0.110553,0.431554}%
\pgfsetfillcolor{currentfill}%
\pgfsetlinewidth{0.000000pt}%
\definecolor{currentstroke}{rgb}{0.268510,0.009605,0.335427}%
\pgfsetstrokecolor{currentstroke}%
\pgfsetdash{}{0pt}%
\pgfpathmoveto{\pgfqpoint{3.750979in}{4.305549in}}%
\pgfpathlineto{\pgfqpoint{3.673363in}{4.323977in}}%
\pgfpathlineto{\pgfqpoint{3.799922in}{4.421780in}}%
\pgfpathclose%
\pgfusepath{fill}%
\end{pgfscope}%
\begin{pgfscope}%
\pgfpathrectangle{\pgfqpoint{0.539299in}{0.078740in}}{\pgfqpoint{7.842520in}{7.842520in}}%
\pgfusepath{clip}%
\pgfsetbuttcap%
\pgfsetroundjoin%
\definecolor{currentfill}{rgb}{0.283187,0.125848,0.444960}%
\pgfsetfillcolor{currentfill}%
\pgfsetlinewidth{0.000000pt}%
\definecolor{currentstroke}{rgb}{0.269944,0.014625,0.341379}%
\pgfsetstrokecolor{currentstroke}%
\pgfsetdash{}{0pt}%
\pgfpathmoveto{\pgfqpoint{4.005716in}{4.343136in}}%
\pgfpathlineto{\pgfqpoint{3.877896in}{4.341545in}}%
\pgfpathlineto{\pgfqpoint{3.799922in}{4.421780in}}%
\pgfpathclose%
\pgfusepath{fill}%
\end{pgfscope}%
\begin{pgfscope}%
\pgfpathrectangle{\pgfqpoint{0.539299in}{0.078740in}}{\pgfqpoint{7.842520in}{7.842520in}}%
\pgfusepath{clip}%
\pgfsetbuttcap%
\pgfsetroundjoin%
\definecolor{currentfill}{rgb}{0.280868,0.160771,0.472899}%
\pgfsetfillcolor{currentfill}%
\pgfsetlinewidth{0.000000pt}%
\definecolor{currentstroke}{rgb}{0.271305,0.019942,0.347269}%
\pgfsetstrokecolor{currentstroke}%
\pgfsetdash{}{0pt}%
\pgfpathmoveto{\pgfqpoint{3.927826in}{4.463428in}}%
\pgfpathlineto{\pgfqpoint{4.005716in}{4.343136in}}%
\pgfpathlineto{\pgfqpoint{3.799922in}{4.421780in}}%
\pgfpathclose%
\pgfusepath{fill}%
\end{pgfscope}%
\begin{pgfscope}%
\pgfpathrectangle{\pgfqpoint{0.539299in}{0.078740in}}{\pgfqpoint{7.842520in}{7.842520in}}%
\pgfusepath{clip}%
\pgfsetbuttcap%
\pgfsetroundjoin%
\definecolor{currentfill}{rgb}{0.280868,0.160771,0.472899}%
\pgfsetfillcolor{currentfill}%
\pgfsetlinewidth{0.000000pt}%
\definecolor{currentstroke}{rgb}{0.272594,0.025563,0.353093}%
\pgfsetstrokecolor{currentstroke}%
\pgfsetdash{}{0pt}%
\pgfpathmoveto{\pgfqpoint{3.721620in}{4.468366in}}%
\pgfpathlineto{\pgfqpoint{3.799922in}{4.421780in}}%
\pgfpathlineto{\pgfqpoint{3.673363in}{4.323977in}}%
\pgfpathclose%
\pgfusepath{fill}%
\end{pgfscope}%
\begin{pgfscope}%
\pgfpathrectangle{\pgfqpoint{0.539299in}{0.078740in}}{\pgfqpoint{7.842520in}{7.842520in}}%
\pgfusepath{clip}%
\pgfsetbuttcap%
\pgfsetroundjoin%
\definecolor{currentfill}{rgb}{0.281887,0.150881,0.465405}%
\pgfsetfillcolor{currentfill}%
\pgfsetlinewidth{0.000000pt}%
\definecolor{currentstroke}{rgb}{0.273809,0.031497,0.358853}%
\pgfsetstrokecolor{currentstroke}%
\pgfsetdash{}{0pt}%
\pgfpathmoveto{\pgfqpoint{3.721620in}{4.468366in}}%
\pgfpathlineto{\pgfqpoint{3.673363in}{4.323977in}}%
\pgfpathlineto{\pgfqpoint{3.595462in}{4.319772in}}%
\pgfpathclose%
\pgfusepath{fill}%
\end{pgfscope}%
\begin{pgfscope}%
\pgfpathrectangle{\pgfqpoint{0.539299in}{0.078740in}}{\pgfqpoint{7.842520in}{7.842520in}}%
\pgfusepath{clip}%
\pgfsetbuttcap%
\pgfsetroundjoin%
\definecolor{currentfill}{rgb}{0.281887,0.150881,0.465405}%
\pgfsetfillcolor{currentfill}%
\pgfsetlinewidth{0.000000pt}%
\definecolor{currentstroke}{rgb}{0.274952,0.037752,0.364543}%
\pgfsetstrokecolor{currentstroke}%
\pgfsetdash{}{0pt}%
\pgfpathmoveto{\pgfqpoint{4.056700in}{4.461449in}}%
\pgfpathlineto{\pgfqpoint{4.134205in}{4.319900in}}%
\pgfpathlineto{\pgfqpoint{4.005716in}{4.343136in}}%
\pgfpathclose%
\pgfusepath{fill}%
\end{pgfscope}%
\begin{pgfscope}%
\pgfpathrectangle{\pgfqpoint{0.539299in}{0.078740in}}{\pgfqpoint{7.842520in}{7.842520in}}%
\pgfusepath{clip}%
\pgfsetbuttcap%
\pgfsetroundjoin%
\definecolor{currentfill}{rgb}{0.281887,0.150881,0.465405}%
\pgfsetfillcolor{currentfill}%
\pgfsetlinewidth{0.000000pt}%
\definecolor{currentstroke}{rgb}{0.276022,0.044167,0.370164}%
\pgfsetstrokecolor{currentstroke}%
\pgfsetdash{}{0pt}%
\pgfpathmoveto{\pgfqpoint{4.263211in}{4.278418in}}%
\pgfpathlineto{\pgfqpoint{4.134205in}{4.319900in}}%
\pgfpathlineto{\pgfqpoint{4.056700in}{4.461449in}}%
\pgfpathclose%
\pgfusepath{fill}%
\end{pgfscope}%
\begin{pgfscope}%
\pgfpathrectangle{\pgfqpoint{0.539299in}{0.078740in}}{\pgfqpoint{7.842520in}{7.842520in}}%
\pgfusepath{clip}%
\pgfsetbuttcap%
\pgfsetroundjoin%
\definecolor{currentfill}{rgb}{0.278826,0.175490,0.483397}%
\pgfsetfillcolor{currentfill}%
\pgfsetlinewidth{0.000000pt}%
\definecolor{currentstroke}{rgb}{0.277018,0.050344,0.375715}%
\pgfsetstrokecolor{currentstroke}%
\pgfsetdash{}{0pt}%
\pgfpathmoveto{\pgfqpoint{4.056700in}{4.461449in}}%
\pgfpathlineto{\pgfqpoint{4.005716in}{4.343136in}}%
\pgfpathlineto{\pgfqpoint{3.927826in}{4.463428in}}%
\pgfpathclose%
\pgfusepath{fill}%
\end{pgfscope}%
\begin{pgfscope}%
\pgfpathrectangle{\pgfqpoint{0.539299in}{0.078740in}}{\pgfqpoint{7.842520in}{7.842520in}}%
\pgfusepath{clip}%
\pgfsetbuttcap%
\pgfsetroundjoin%
\definecolor{currentfill}{rgb}{0.276194,0.190074,0.493001}%
\pgfsetfillcolor{currentfill}%
\pgfsetlinewidth{0.000000pt}%
\definecolor{currentstroke}{rgb}{0.277941,0.056324,0.381191}%
\pgfsetstrokecolor{currentstroke}%
\pgfsetdash{}{0pt}%
\pgfpathmoveto{\pgfqpoint{3.927826in}{4.463428in}}%
\pgfpathlineto{\pgfqpoint{3.799922in}{4.421780in}}%
\pgfpathlineto{\pgfqpoint{3.721620in}{4.468366in}}%
\pgfpathclose%
\pgfusepath{fill}%
\end{pgfscope}%
\begin{pgfscope}%
\pgfpathrectangle{\pgfqpoint{0.539299in}{0.078740in}}{\pgfqpoint{7.842520in}{7.842520in}}%
\pgfusepath{clip}%
\pgfsetbuttcap%
\pgfsetroundjoin%
\definecolor{currentfill}{rgb}{0.280255,0.165693,0.476498}%
\pgfsetfillcolor{currentfill}%
\pgfsetlinewidth{0.000000pt}%
\definecolor{currentstroke}{rgb}{0.278791,0.062145,0.386592}%
\pgfsetstrokecolor{currentstroke}%
\pgfsetdash{}{0pt}%
\pgfpathmoveto{\pgfqpoint{3.595462in}{4.319772in}}%
\pgfpathlineto{\pgfqpoint{3.517208in}{4.302391in}}%
\pgfpathlineto{\pgfqpoint{3.721620in}{4.468366in}}%
\pgfpathclose%
\pgfusepath{fill}%
\end{pgfscope}%
\begin{pgfscope}%
\pgfpathrectangle{\pgfqpoint{0.539299in}{0.078740in}}{\pgfqpoint{7.842520in}{7.842520in}}%
\pgfusepath{clip}%
\pgfsetbuttcap%
\pgfsetroundjoin%
\definecolor{currentfill}{rgb}{0.281887,0.150881,0.465405}%
\pgfsetfillcolor{currentfill}%
\pgfsetlinewidth{0.000000pt}%
\definecolor{currentstroke}{rgb}{0.279566,0.067836,0.391917}%
\pgfsetstrokecolor{currentstroke}%
\pgfsetdash{}{0pt}%
\pgfpathmoveto{\pgfqpoint{4.186265in}{4.426031in}}%
\pgfpathlineto{\pgfqpoint{4.392627in}{4.224057in}}%
\pgfpathlineto{\pgfqpoint{4.263211in}{4.278418in}}%
\pgfpathclose%
\pgfusepath{fill}%
\end{pgfscope}%
\begin{pgfscope}%
\pgfpathrectangle{\pgfqpoint{0.539299in}{0.078740in}}{\pgfqpoint{7.842520in}{7.842520in}}%
\pgfusepath{clip}%
\pgfsetbuttcap%
\pgfsetroundjoin%
\definecolor{currentfill}{rgb}{0.278012,0.180367,0.486697}%
\pgfsetfillcolor{currentfill}%
\pgfsetlinewidth{0.000000pt}%
\definecolor{currentstroke}{rgb}{0.280267,0.073417,0.397163}%
\pgfsetstrokecolor{currentstroke}%
\pgfsetdash{}{0pt}%
\pgfpathmoveto{\pgfqpoint{4.056700in}{4.461449in}}%
\pgfpathlineto{\pgfqpoint{4.186265in}{4.426031in}}%
\pgfpathlineto{\pgfqpoint{4.263211in}{4.278418in}}%
\pgfpathclose%
\pgfusepath{fill}%
\end{pgfscope}%
\begin{pgfscope}%
\pgfpathrectangle{\pgfqpoint{0.539299in}{0.078740in}}{\pgfqpoint{7.842520in}{7.842520in}}%
\pgfusepath{clip}%
\pgfsetbuttcap%
\pgfsetroundjoin%
\definecolor{currentfill}{rgb}{0.282884,0.135920,0.453427}%
\pgfsetfillcolor{currentfill}%
\pgfsetlinewidth{0.000000pt}%
\definecolor{currentstroke}{rgb}{0.280894,0.078907,0.402329}%
\pgfsetstrokecolor{currentstroke}%
\pgfsetdash{}{0pt}%
\pgfpathmoveto{\pgfqpoint{4.522384in}{4.161514in}}%
\pgfpathlineto{\pgfqpoint{4.392627in}{4.224057in}}%
\pgfpathlineto{\pgfqpoint{4.446717in}{4.289781in}}%
\pgfpathclose%
\pgfusepath{fill}%
\end{pgfscope}%
\begin{pgfscope}%
\pgfpathrectangle{\pgfqpoint{0.539299in}{0.078740in}}{\pgfqpoint{7.842520in}{7.842520in}}%
\pgfusepath{clip}%
\pgfsetbuttcap%
\pgfsetroundjoin%
\definecolor{currentfill}{rgb}{0.267968,0.223549,0.512008}%
\pgfsetfillcolor{currentfill}%
\pgfsetlinewidth{0.000000pt}%
\definecolor{currentstroke}{rgb}{0.281446,0.084320,0.407414}%
\pgfsetstrokecolor{currentstroke}%
\pgfsetdash{}{0pt}%
\pgfpathmoveto{\pgfqpoint{3.721620in}{4.468366in}}%
\pgfpathlineto{\pgfqpoint{3.849491in}{4.547694in}}%
\pgfpathlineto{\pgfqpoint{3.927826in}{4.463428in}}%
\pgfpathclose%
\pgfusepath{fill}%
\end{pgfscope}%
\begin{pgfscope}%
\pgfpathrectangle{\pgfqpoint{0.539299in}{0.078740in}}{\pgfqpoint{7.842520in}{7.842520in}}%
\pgfusepath{clip}%
\pgfsetbuttcap%
\pgfsetroundjoin%
\definecolor{currentfill}{rgb}{0.267968,0.223549,0.512008}%
\pgfsetfillcolor{currentfill}%
\pgfsetlinewidth{0.000000pt}%
\definecolor{currentstroke}{rgb}{0.281924,0.089666,0.412415}%
\pgfsetstrokecolor{currentstroke}%
\pgfsetdash{}{0pt}%
\pgfpathmoveto{\pgfqpoint{3.927826in}{4.463428in}}%
\pgfpathlineto{\pgfqpoint{3.849491in}{4.547694in}}%
\pgfpathlineto{\pgfqpoint{4.056700in}{4.461449in}}%
\pgfpathclose%
\pgfusepath{fill}%
\end{pgfscope}%
\begin{pgfscope}%
\pgfpathrectangle{\pgfqpoint{0.539299in}{0.078740in}}{\pgfqpoint{7.842520in}{7.842520in}}%
\pgfusepath{clip}%
\pgfsetbuttcap%
\pgfsetroundjoin%
\definecolor{currentfill}{rgb}{0.278826,0.175490,0.483397}%
\pgfsetfillcolor{currentfill}%
\pgfsetlinewidth{0.000000pt}%
\definecolor{currentstroke}{rgb}{0.282327,0.094955,0.417331}%
\pgfsetstrokecolor{currentstroke}%
\pgfsetdash{}{0pt}%
\pgfpathmoveto{\pgfqpoint{4.316318in}{4.366115in}}%
\pgfpathlineto{\pgfqpoint{4.392627in}{4.224057in}}%
\pgfpathlineto{\pgfqpoint{4.186265in}{4.426031in}}%
\pgfpathclose%
\pgfusepath{fill}%
\end{pgfscope}%
\begin{pgfscope}%
\pgfpathrectangle{\pgfqpoint{0.539299in}{0.078740in}}{\pgfqpoint{7.842520in}{7.842520in}}%
\pgfusepath{clip}%
\pgfsetbuttcap%
\pgfsetroundjoin%
\definecolor{currentfill}{rgb}{0.273006,0.204520,0.501721}%
\pgfsetfillcolor{currentfill}%
\pgfsetlinewidth{0.000000pt}%
\definecolor{currentstroke}{rgb}{0.282656,0.100196,0.422160}%
\pgfsetstrokecolor{currentstroke}%
\pgfsetdash{}{0pt}%
\pgfpathmoveto{\pgfqpoint{3.517208in}{4.302391in}}%
\pgfpathlineto{\pgfqpoint{3.642937in}{4.493560in}}%
\pgfpathlineto{\pgfqpoint{3.721620in}{4.468366in}}%
\pgfpathclose%
\pgfusepath{fill}%
\end{pgfscope}%
\begin{pgfscope}%
\pgfpathrectangle{\pgfqpoint{0.539299in}{0.078740in}}{\pgfqpoint{7.842520in}{7.842520in}}%
\pgfusepath{clip}%
\pgfsetbuttcap%
\pgfsetroundjoin%
\definecolor{currentfill}{rgb}{0.283187,0.125848,0.444960}%
\pgfsetfillcolor{currentfill}%
\pgfsetlinewidth{0.000000pt}%
\definecolor{currentstroke}{rgb}{0.282910,0.105393,0.426902}%
\pgfsetstrokecolor{currentstroke}%
\pgfsetdash{}{0pt}%
\pgfpathmoveto{\pgfqpoint{4.652446in}{4.094919in}}%
\pgfpathlineto{\pgfqpoint{4.522384in}{4.161514in}}%
\pgfpathlineto{\pgfqpoint{4.446717in}{4.289781in}}%
\pgfpathclose%
\pgfusepath{fill}%
\end{pgfscope}%
\begin{pgfscope}%
\pgfpathrectangle{\pgfqpoint{0.539299in}{0.078740in}}{\pgfqpoint{7.842520in}{7.842520in}}%
\pgfusepath{clip}%
\pgfsetbuttcap%
\pgfsetroundjoin%
\definecolor{currentfill}{rgb}{0.279574,0.170599,0.479997}%
\pgfsetfillcolor{currentfill}%
\pgfsetlinewidth{0.000000pt}%
\definecolor{currentstroke}{rgb}{0.283091,0.110553,0.431554}%
\pgfsetstrokecolor{currentstroke}%
\pgfsetdash{}{0pt}%
\pgfpathmoveto{\pgfqpoint{4.446717in}{4.289781in}}%
\pgfpathlineto{\pgfqpoint{4.392627in}{4.224057in}}%
\pgfpathlineto{\pgfqpoint{4.316318in}{4.366115in}}%
\pgfpathclose%
\pgfusepath{fill}%
\end{pgfscope}%
\begin{pgfscope}%
\pgfpathrectangle{\pgfqpoint{0.539299in}{0.078740in}}{\pgfqpoint{7.842520in}{7.842520in}}%
\pgfusepath{clip}%
\pgfsetbuttcap%
\pgfsetroundjoin%
\definecolor{currentfill}{rgb}{0.276194,0.190074,0.493001}%
\pgfsetfillcolor{currentfill}%
\pgfsetlinewidth{0.000000pt}%
\definecolor{currentstroke}{rgb}{0.283197,0.115680,0.436115}%
\pgfsetstrokecolor{currentstroke}%
\pgfsetdash{}{0pt}%
\pgfpathmoveto{\pgfqpoint{3.642937in}{4.493560in}}%
\pgfpathlineto{\pgfqpoint{3.517208in}{4.302391in}}%
\pgfpathlineto{\pgfqpoint{3.438560in}{4.276751in}}%
\pgfpathclose%
\pgfusepath{fill}%
\end{pgfscope}%
\begin{pgfscope}%
\pgfpathrectangle{\pgfqpoint{0.539299in}{0.078740in}}{\pgfqpoint{7.842520in}{7.842520in}}%
\pgfusepath{clip}%
\pgfsetbuttcap%
\pgfsetroundjoin%
\definecolor{currentfill}{rgb}{0.282910,0.105393,0.426902}%
\pgfsetfillcolor{currentfill}%
\pgfsetlinewidth{0.000000pt}%
\definecolor{currentstroke}{rgb}{0.283229,0.120777,0.440584}%
\pgfsetstrokecolor{currentstroke}%
\pgfsetdash{}{0pt}%
\pgfpathmoveto{\pgfqpoint{4.708270in}{4.115796in}}%
\pgfpathlineto{\pgfqpoint{4.782803in}{4.027779in}}%
\pgfpathlineto{\pgfqpoint{4.652446in}{4.094919in}}%
\pgfpathclose%
\pgfusepath{fill}%
\end{pgfscope}%
\begin{pgfscope}%
\pgfpathrectangle{\pgfqpoint{0.539299in}{0.078740in}}{\pgfqpoint{7.842520in}{7.842520in}}%
\pgfusepath{clip}%
\pgfsetbuttcap%
\pgfsetroundjoin%
\definecolor{currentfill}{rgb}{0.258965,0.251537,0.524736}%
\pgfsetfillcolor{currentfill}%
\pgfsetlinewidth{0.000000pt}%
\definecolor{currentstroke}{rgb}{0.283187,0.125848,0.444960}%
\pgfsetstrokecolor{currentstroke}%
\pgfsetdash{}{0pt}%
\pgfpathmoveto{\pgfqpoint{4.056700in}{4.461449in}}%
\pgfpathlineto{\pgfqpoint{3.849491in}{4.547694in}}%
\pgfpathlineto{\pgfqpoint{3.978618in}{4.570065in}}%
\pgfpathclose%
\pgfusepath{fill}%
\end{pgfscope}%
\begin{pgfscope}%
\pgfpathrectangle{\pgfqpoint{0.539299in}{0.078740in}}{\pgfqpoint{7.842520in}{7.842520in}}%
\pgfusepath{clip}%
\pgfsetbuttcap%
\pgfsetroundjoin%
\definecolor{currentfill}{rgb}{0.282290,0.145912,0.461510}%
\pgfsetfillcolor{currentfill}%
\pgfsetlinewidth{0.000000pt}%
\definecolor{currentstroke}{rgb}{0.283072,0.130895,0.449241}%
\pgfsetstrokecolor{currentstroke}%
\pgfsetdash{}{0pt}%
\pgfpathmoveto{\pgfqpoint{4.577380in}{4.204259in}}%
\pgfpathlineto{\pgfqpoint{4.652446in}{4.094919in}}%
\pgfpathlineto{\pgfqpoint{4.446717in}{4.289781in}}%
\pgfpathclose%
\pgfusepath{fill}%
\end{pgfscope}%
\begin{pgfscope}%
\pgfpathrectangle{\pgfqpoint{0.539299in}{0.078740in}}{\pgfqpoint{7.842520in}{7.842520in}}%
\pgfusepath{clip}%
\pgfsetbuttcap%
\pgfsetroundjoin%
\definecolor{currentfill}{rgb}{0.282327,0.094955,0.417331}%
\pgfsetfillcolor{currentfill}%
\pgfsetlinewidth{0.000000pt}%
\definecolor{currentstroke}{rgb}{0.282884,0.135920,0.453427}%
\pgfsetstrokecolor{currentstroke}%
\pgfsetdash{}{0pt}%
\pgfpathmoveto{\pgfqpoint{4.913467in}{3.962882in}}%
\pgfpathlineto{\pgfqpoint{4.782803in}{4.027779in}}%
\pgfpathlineto{\pgfqpoint{4.708270in}{4.115796in}}%
\pgfpathclose%
\pgfusepath{fill}%
\end{pgfscope}%
\begin{pgfscope}%
\pgfpathrectangle{\pgfqpoint{0.539299in}{0.078740in}}{\pgfqpoint{7.842520in}{7.842520in}}%
\pgfusepath{clip}%
\pgfsetbuttcap%
\pgfsetroundjoin%
\definecolor{currentfill}{rgb}{0.263663,0.237631,0.518762}%
\pgfsetfillcolor{currentfill}%
\pgfsetlinewidth{0.000000pt}%
\definecolor{currentstroke}{rgb}{0.282623,0.140926,0.457517}%
\pgfsetstrokecolor{currentstroke}%
\pgfsetdash{}{0pt}%
\pgfpathmoveto{\pgfqpoint{4.108635in}{4.547005in}}%
\pgfpathlineto{\pgfqpoint{4.186265in}{4.426031in}}%
\pgfpathlineto{\pgfqpoint{4.056700in}{4.461449in}}%
\pgfpathclose%
\pgfusepath{fill}%
\end{pgfscope}%
\begin{pgfscope}%
\pgfpathrectangle{\pgfqpoint{0.539299in}{0.078740in}}{\pgfqpoint{7.842520in}{7.842520in}}%
\pgfusepath{clip}%
\pgfsetbuttcap%
\pgfsetroundjoin%
\definecolor{currentfill}{rgb}{0.255645,0.260703,0.528312}%
\pgfsetfillcolor{currentfill}%
\pgfsetlinewidth{0.000000pt}%
\definecolor{currentstroke}{rgb}{0.282290,0.145912,0.461510}%
\pgfsetstrokecolor{currentstroke}%
\pgfsetdash{}{0pt}%
\pgfpathmoveto{\pgfqpoint{3.770691in}{4.607440in}}%
\pgfpathlineto{\pgfqpoint{3.849491in}{4.547694in}}%
\pgfpathlineto{\pgfqpoint{3.721620in}{4.468366in}}%
\pgfpathclose%
\pgfusepath{fill}%
\end{pgfscope}%
\begin{pgfscope}%
\pgfpathrectangle{\pgfqpoint{0.539299in}{0.078740in}}{\pgfqpoint{7.842520in}{7.842520in}}%
\pgfusepath{clip}%
\pgfsetbuttcap%
\pgfsetroundjoin%
\definecolor{currentfill}{rgb}{0.282884,0.135920,0.453427}%
\pgfsetfillcolor{currentfill}%
\pgfsetlinewidth{0.000000pt}%
\definecolor{currentstroke}{rgb}{0.281887,0.150881,0.465405}%
\pgfsetstrokecolor{currentstroke}%
\pgfsetdash{}{0pt}%
\pgfpathmoveto{\pgfqpoint{4.708270in}{4.115796in}}%
\pgfpathlineto{\pgfqpoint{4.652446in}{4.094919in}}%
\pgfpathlineto{\pgfqpoint{4.577380in}{4.204259in}}%
\pgfpathclose%
\pgfusepath{fill}%
\end{pgfscope}%
\begin{pgfscope}%
\pgfpathrectangle{\pgfqpoint{0.539299in}{0.078740in}}{\pgfqpoint{7.842520in}{7.842520in}}%
\pgfusepath{clip}%
\pgfsetbuttcap%
\pgfsetroundjoin%
\definecolor{currentfill}{rgb}{0.257322,0.256130,0.526563}%
\pgfsetfillcolor{currentfill}%
\pgfsetlinewidth{0.000000pt}%
\definecolor{currentstroke}{rgb}{0.281412,0.155834,0.469201}%
\pgfsetstrokecolor{currentstroke}%
\pgfsetdash{}{0pt}%
\pgfpathmoveto{\pgfqpoint{3.721620in}{4.468366in}}%
\pgfpathlineto{\pgfqpoint{3.642937in}{4.493560in}}%
\pgfpathlineto{\pgfqpoint{3.770691in}{4.607440in}}%
\pgfpathclose%
\pgfusepath{fill}%
\end{pgfscope}%
\begin{pgfscope}%
\pgfpathrectangle{\pgfqpoint{0.539299in}{0.078740in}}{\pgfqpoint{7.842520in}{7.842520in}}%
\pgfusepath{clip}%
\pgfsetbuttcap%
\pgfsetroundjoin%
\definecolor{currentfill}{rgb}{0.280894,0.078907,0.402329}%
\pgfsetfillcolor{currentfill}%
\pgfsetlinewidth{0.000000pt}%
\definecolor{currentstroke}{rgb}{0.280868,0.160771,0.472899}%
\pgfsetstrokecolor{currentstroke}%
\pgfsetdash{}{0pt}%
\pgfpathmoveto{\pgfqpoint{4.839392in}{4.029475in}}%
\pgfpathlineto{\pgfqpoint{5.044466in}{3.902225in}}%
\pgfpathlineto{\pgfqpoint{4.913467in}{3.962882in}}%
\pgfpathclose%
\pgfusepath{fill}%
\end{pgfscope}%
\begin{pgfscope}%
\pgfpathrectangle{\pgfqpoint{0.539299in}{0.078740in}}{\pgfqpoint{7.842520in}{7.842520in}}%
\pgfusepath{clip}%
\pgfsetbuttcap%
\pgfsetroundjoin%
\definecolor{currentfill}{rgb}{0.255645,0.260703,0.528312}%
\pgfsetfillcolor{currentfill}%
\pgfsetlinewidth{0.000000pt}%
\definecolor{currentstroke}{rgb}{0.280255,0.165693,0.476498}%
\pgfsetstrokecolor{currentstroke}%
\pgfsetdash{}{0pt}%
\pgfpathmoveto{\pgfqpoint{3.978618in}{4.570065in}}%
\pgfpathlineto{\pgfqpoint{4.108635in}{4.547005in}}%
\pgfpathlineto{\pgfqpoint{4.056700in}{4.461449in}}%
\pgfpathclose%
\pgfusepath{fill}%
\end{pgfscope}%
\begin{pgfscope}%
\pgfpathrectangle{\pgfqpoint{0.539299in}{0.078740in}}{\pgfqpoint{7.842520in}{7.842520in}}%
\pgfusepath{clip}%
\pgfsetbuttcap%
\pgfsetroundjoin%
\definecolor{currentfill}{rgb}{0.265145,0.232956,0.516599}%
\pgfsetfillcolor{currentfill}%
\pgfsetlinewidth{0.000000pt}%
\definecolor{currentstroke}{rgb}{0.279574,0.170599,0.479997}%
\pgfsetstrokecolor{currentstroke}%
\pgfsetdash{}{0pt}%
\pgfpathmoveto{\pgfqpoint{4.186265in}{4.426031in}}%
\pgfpathlineto{\pgfqpoint{4.239257in}{4.489420in}}%
\pgfpathlineto{\pgfqpoint{4.316318in}{4.366115in}}%
\pgfpathclose%
\pgfusepath{fill}%
\end{pgfscope}%
\begin{pgfscope}%
\pgfpathrectangle{\pgfqpoint{0.539299in}{0.078740in}}{\pgfqpoint{7.842520in}{7.842520in}}%
\pgfusepath{clip}%
\pgfsetbuttcap%
\pgfsetroundjoin%
\definecolor{currentfill}{rgb}{0.265145,0.232956,0.516599}%
\pgfsetfillcolor{currentfill}%
\pgfsetlinewidth{0.000000pt}%
\definecolor{currentstroke}{rgb}{0.278826,0.175490,0.483397}%
\pgfsetstrokecolor{currentstroke}%
\pgfsetdash{}{0pt}%
\pgfpathmoveto{\pgfqpoint{3.438560in}{4.276751in}}%
\pgfpathlineto{\pgfqpoint{3.563833in}{4.504742in}}%
\pgfpathlineto{\pgfqpoint{3.642937in}{4.493560in}}%
\pgfpathclose%
\pgfusepath{fill}%
\end{pgfscope}%
\begin{pgfscope}%
\pgfpathrectangle{\pgfqpoint{0.539299in}{0.078740in}}{\pgfqpoint{7.842520in}{7.842520in}}%
\pgfusepath{clip}%
\pgfsetbuttcap%
\pgfsetroundjoin%
\definecolor{currentfill}{rgb}{0.283091,0.110553,0.431554}%
\pgfsetfillcolor{currentfill}%
\pgfsetlinewidth{0.000000pt}%
\definecolor{currentstroke}{rgb}{0.278012,0.180367,0.486697}%
\pgfsetstrokecolor{currentstroke}%
\pgfsetdash{}{0pt}%
\pgfpathmoveto{\pgfqpoint{4.708270in}{4.115796in}}%
\pgfpathlineto{\pgfqpoint{4.839392in}{4.029475in}}%
\pgfpathlineto{\pgfqpoint{4.913467in}{3.962882in}}%
\pgfpathclose%
\pgfusepath{fill}%
\end{pgfscope}%
\begin{pgfscope}%
\pgfpathrectangle{\pgfqpoint{0.539299in}{0.078740in}}{\pgfqpoint{7.842520in}{7.842520in}}%
\pgfusepath{clip}%
\pgfsetbuttcap%
\pgfsetroundjoin%
\definecolor{currentfill}{rgb}{0.271828,0.209303,0.504434}%
\pgfsetfillcolor{currentfill}%
\pgfsetlinewidth{0.000000pt}%
\definecolor{currentstroke}{rgb}{0.277134,0.185228,0.489898}%
\pgfsetstrokecolor{currentstroke}%
\pgfsetdash{}{0pt}%
\pgfpathmoveto{\pgfqpoint{3.438560in}{4.276751in}}%
\pgfpathlineto{\pgfqpoint{3.359496in}{4.245638in}}%
\pgfpathlineto{\pgfqpoint{3.563833in}{4.504742in}}%
\pgfpathclose%
\pgfusepath{fill}%
\end{pgfscope}%
\begin{pgfscope}%
\pgfpathrectangle{\pgfqpoint{0.539299in}{0.078740in}}{\pgfqpoint{7.842520in}{7.842520in}}%
\pgfusepath{clip}%
\pgfsetbuttcap%
\pgfsetroundjoin%
\definecolor{currentfill}{rgb}{0.267968,0.223549,0.512008}%
\pgfsetfillcolor{currentfill}%
\pgfsetlinewidth{0.000000pt}%
\definecolor{currentstroke}{rgb}{0.276194,0.190074,0.493001}%
\pgfsetstrokecolor{currentstroke}%
\pgfsetdash{}{0pt}%
\pgfpathmoveto{\pgfqpoint{4.239257in}{4.489420in}}%
\pgfpathlineto{\pgfqpoint{4.446717in}{4.289781in}}%
\pgfpathlineto{\pgfqpoint{4.316318in}{4.366115in}}%
\pgfpathclose%
\pgfusepath{fill}%
\end{pgfscope}%
\begin{pgfscope}%
\pgfpathrectangle{\pgfqpoint{0.539299in}{0.078740in}}{\pgfqpoint{7.842520in}{7.842520in}}%
\pgfusepath{clip}%
\pgfsetbuttcap%
\pgfsetroundjoin%
\definecolor{currentfill}{rgb}{0.278791,0.062145,0.386592}%
\pgfsetfillcolor{currentfill}%
\pgfsetlinewidth{0.000000pt}%
\definecolor{currentstroke}{rgb}{0.275191,0.194905,0.496005}%
\pgfsetstrokecolor{currentstroke}%
\pgfsetdash{}{0pt}%
\pgfpathmoveto{\pgfqpoint{5.175833in}{3.846987in}}%
\pgfpathlineto{\pgfqpoint{5.044466in}{3.902225in}}%
\pgfpathlineto{\pgfqpoint{5.102484in}{3.877072in}}%
\pgfpathclose%
\pgfusepath{fill}%
\end{pgfscope}%
\begin{pgfscope}%
\pgfpathrectangle{\pgfqpoint{0.539299in}{0.078740in}}{\pgfqpoint{7.842520in}{7.842520in}}%
\pgfusepath{clip}%
\pgfsetbuttcap%
\pgfsetroundjoin%
\definecolor{currentfill}{rgb}{0.255645,0.260703,0.528312}%
\pgfsetfillcolor{currentfill}%
\pgfsetlinewidth{0.000000pt}%
\definecolor{currentstroke}{rgb}{0.274128,0.199721,0.498911}%
\pgfsetstrokecolor{currentstroke}%
\pgfsetdash{}{0pt}%
\pgfpathmoveto{\pgfqpoint{4.108635in}{4.547005in}}%
\pgfpathlineto{\pgfqpoint{4.239257in}{4.489420in}}%
\pgfpathlineto{\pgfqpoint{4.186265in}{4.426031in}}%
\pgfpathclose%
\pgfusepath{fill}%
\end{pgfscope}%
\begin{pgfscope}%
\pgfpathrectangle{\pgfqpoint{0.539299in}{0.078740in}}{\pgfqpoint{7.842520in}{7.842520in}}%
\pgfusepath{clip}%
\pgfsetbuttcap%
\pgfsetroundjoin%
\definecolor{currentfill}{rgb}{0.282327,0.094955,0.417331}%
\pgfsetfillcolor{currentfill}%
\pgfsetlinewidth{0.000000pt}%
\definecolor{currentstroke}{rgb}{0.273006,0.204520,0.501721}%
\pgfsetstrokecolor{currentstroke}%
\pgfsetdash{}{0pt}%
\pgfpathmoveto{\pgfqpoint{4.970779in}{3.949088in}}%
\pgfpathlineto{\pgfqpoint{5.044466in}{3.902225in}}%
\pgfpathlineto{\pgfqpoint{4.839392in}{4.029475in}}%
\pgfpathclose%
\pgfusepath{fill}%
\end{pgfscope}%
\begin{pgfscope}%
\pgfpathrectangle{\pgfqpoint{0.539299in}{0.078740in}}{\pgfqpoint{7.842520in}{7.842520in}}%
\pgfusepath{clip}%
\pgfsetbuttcap%
\pgfsetroundjoin%
\definecolor{currentfill}{rgb}{0.243113,0.292092,0.538516}%
\pgfsetfillcolor{currentfill}%
\pgfsetlinewidth{0.000000pt}%
\definecolor{currentstroke}{rgb}{0.271828,0.209303,0.504434}%
\pgfsetstrokecolor{currentstroke}%
\pgfsetdash{}{0pt}%
\pgfpathmoveto{\pgfqpoint{3.978618in}{4.570065in}}%
\pgfpathlineto{\pgfqpoint{3.849491in}{4.547694in}}%
\pgfpathlineto{\pgfqpoint{3.899971in}{4.655078in}}%
\pgfpathclose%
\pgfusepath{fill}%
\end{pgfscope}%
\begin{pgfscope}%
\pgfpathrectangle{\pgfqpoint{0.539299in}{0.078740in}}{\pgfqpoint{7.842520in}{7.842520in}}%
\pgfusepath{clip}%
\pgfsetbuttcap%
\pgfsetroundjoin%
\definecolor{currentfill}{rgb}{0.275191,0.194905,0.496005}%
\pgfsetfillcolor{currentfill}%
\pgfsetlinewidth{0.000000pt}%
\definecolor{currentstroke}{rgb}{0.270595,0.214069,0.507052}%
\pgfsetstrokecolor{currentstroke}%
\pgfsetdash{}{0pt}%
\pgfpathmoveto{\pgfqpoint{4.577380in}{4.204259in}}%
\pgfpathlineto{\pgfqpoint{4.446717in}{4.289781in}}%
\pgfpathlineto{\pgfqpoint{4.501556in}{4.310839in}}%
\pgfpathclose%
\pgfusepath{fill}%
\end{pgfscope}%
\begin{pgfscope}%
\pgfpathrectangle{\pgfqpoint{0.539299in}{0.078740in}}{\pgfqpoint{7.842520in}{7.842520in}}%
\pgfusepath{clip}%
\pgfsetbuttcap%
\pgfsetroundjoin%
\definecolor{currentfill}{rgb}{0.277941,0.056324,0.381191}%
\pgfsetfillcolor{currentfill}%
\pgfsetlinewidth{0.000000pt}%
\definecolor{currentstroke}{rgb}{0.269308,0.218818,0.509577}%
\pgfsetstrokecolor{currentstroke}%
\pgfsetdash{}{0pt}%
\pgfpathmoveto{\pgfqpoint{5.102484in}{3.877072in}}%
\pgfpathlineto{\pgfqpoint{5.307605in}{3.797555in}}%
\pgfpathlineto{\pgfqpoint{5.175833in}{3.846987in}}%
\pgfpathclose%
\pgfusepath{fill}%
\end{pgfscope}%
\begin{pgfscope}%
\pgfpathrectangle{\pgfqpoint{0.539299in}{0.078740in}}{\pgfqpoint{7.842520in}{7.842520in}}%
\pgfusepath{clip}%
\pgfsetbuttcap%
\pgfsetroundjoin%
\definecolor{currentfill}{rgb}{0.281446,0.084320,0.407414}%
\pgfsetfillcolor{currentfill}%
\pgfsetlinewidth{0.000000pt}%
\definecolor{currentstroke}{rgb}{0.267968,0.223549,0.512008}%
\pgfsetstrokecolor{currentstroke}%
\pgfsetdash{}{0pt}%
\pgfpathmoveto{\pgfqpoint{5.102484in}{3.877072in}}%
\pgfpathlineto{\pgfqpoint{5.044466in}{3.902225in}}%
\pgfpathlineto{\pgfqpoint{4.970779in}{3.949088in}}%
\pgfpathclose%
\pgfusepath{fill}%
\end{pgfscope}%
\begin{pgfscope}%
\pgfpathrectangle{\pgfqpoint{0.539299in}{0.078740in}}{\pgfqpoint{7.842520in}{7.842520in}}%
\pgfusepath{clip}%
\pgfsetbuttcap%
\pgfsetroundjoin%
\definecolor{currentfill}{rgb}{0.239346,0.300855,0.540844}%
\pgfsetfillcolor{currentfill}%
\pgfsetlinewidth{0.000000pt}%
\definecolor{currentstroke}{rgb}{0.266580,0.228262,0.514349}%
\pgfsetstrokecolor{currentstroke}%
\pgfsetdash{}{0pt}%
\pgfpathmoveto{\pgfqpoint{3.899971in}{4.655078in}}%
\pgfpathlineto{\pgfqpoint{3.849491in}{4.547694in}}%
\pgfpathlineto{\pgfqpoint{3.770691in}{4.607440in}}%
\pgfpathclose%
\pgfusepath{fill}%
\end{pgfscope}%
\begin{pgfscope}%
\pgfpathrectangle{\pgfqpoint{0.539299in}{0.078740in}}{\pgfqpoint{7.842520in}{7.842520in}}%
\pgfusepath{clip}%
\pgfsetbuttcap%
\pgfsetroundjoin%
\definecolor{currentfill}{rgb}{0.277134,0.185228,0.489898}%
\pgfsetfillcolor{currentfill}%
\pgfsetlinewidth{0.000000pt}%
\definecolor{currentstroke}{rgb}{0.265145,0.232956,0.516599}%
\pgfsetstrokecolor{currentstroke}%
\pgfsetdash{}{0pt}%
\pgfpathmoveto{\pgfqpoint{4.577380in}{4.204259in}}%
\pgfpathlineto{\pgfqpoint{4.501556in}{4.310839in}}%
\pgfpathlineto{\pgfqpoint{4.708270in}{4.115796in}}%
\pgfpathclose%
\pgfusepath{fill}%
\end{pgfscope}%
\begin{pgfscope}%
\pgfpathrectangle{\pgfqpoint{0.539299in}{0.078740in}}{\pgfqpoint{7.842520in}{7.842520in}}%
\pgfusepath{clip}%
\pgfsetbuttcap%
\pgfsetroundjoin%
\definecolor{currentfill}{rgb}{0.262138,0.242286,0.520837}%
\pgfsetfillcolor{currentfill}%
\pgfsetlinewidth{0.000000pt}%
\definecolor{currentstroke}{rgb}{0.263663,0.237631,0.518762}%
\pgfsetstrokecolor{currentstroke}%
\pgfsetdash{}{0pt}%
\pgfpathmoveto{\pgfqpoint{4.370276in}{4.407562in}}%
\pgfpathlineto{\pgfqpoint{4.446717in}{4.289781in}}%
\pgfpathlineto{\pgfqpoint{4.239257in}{4.489420in}}%
\pgfpathclose%
\pgfusepath{fill}%
\end{pgfscope}%
\begin{pgfscope}%
\pgfpathrectangle{\pgfqpoint{0.539299in}{0.078740in}}{\pgfqpoint{7.842520in}{7.842520in}}%
\pgfusepath{clip}%
\pgfsetbuttcap%
\pgfsetroundjoin%
\definecolor{currentfill}{rgb}{0.239346,0.300855,0.540844}%
\pgfsetfillcolor{currentfill}%
\pgfsetlinewidth{0.000000pt}%
\definecolor{currentstroke}{rgb}{0.262138,0.242286,0.520837}%
\pgfsetstrokecolor{currentstroke}%
\pgfsetdash{}{0pt}%
\pgfpathmoveto{\pgfqpoint{3.899971in}{4.655078in}}%
\pgfpathlineto{\pgfqpoint{4.108635in}{4.547005in}}%
\pgfpathlineto{\pgfqpoint{3.978618in}{4.570065in}}%
\pgfpathclose%
\pgfusepath{fill}%
\end{pgfscope}%
\begin{pgfscope}%
\pgfpathrectangle{\pgfqpoint{0.539299in}{0.078740in}}{\pgfqpoint{7.842520in}{7.842520in}}%
\pgfusepath{clip}%
\pgfsetbuttcap%
\pgfsetroundjoin%
\definecolor{currentfill}{rgb}{0.239346,0.300855,0.540844}%
\pgfsetfillcolor{currentfill}%
\pgfsetlinewidth{0.000000pt}%
\definecolor{currentstroke}{rgb}{0.260571,0.246922,0.522828}%
\pgfsetstrokecolor{currentstroke}%
\pgfsetdash{}{0pt}%
\pgfpathmoveto{\pgfqpoint{3.642937in}{4.493560in}}%
\pgfpathlineto{\pgfqpoint{3.691406in}{4.650385in}}%
\pgfpathlineto{\pgfqpoint{3.770691in}{4.607440in}}%
\pgfpathclose%
\pgfusepath{fill}%
\end{pgfscope}%
\begin{pgfscope}%
\pgfpathrectangle{\pgfqpoint{0.539299in}{0.078740in}}{\pgfqpoint{7.842520in}{7.842520in}}%
\pgfusepath{clip}%
\pgfsetbuttcap%
\pgfsetroundjoin%
\definecolor{currentfill}{rgb}{0.243113,0.292092,0.538516}%
\pgfsetfillcolor{currentfill}%
\pgfsetlinewidth{0.000000pt}%
\definecolor{currentstroke}{rgb}{0.258965,0.251537,0.524736}%
\pgfsetstrokecolor{currentstroke}%
\pgfsetdash{}{0pt}%
\pgfpathmoveto{\pgfqpoint{3.563833in}{4.504742in}}%
\pgfpathlineto{\pgfqpoint{3.691406in}{4.650385in}}%
\pgfpathlineto{\pgfqpoint{3.642937in}{4.493560in}}%
\pgfpathclose%
\pgfusepath{fill}%
\end{pgfscope}%
\begin{pgfscope}%
\pgfpathrectangle{\pgfqpoint{0.539299in}{0.078740in}}{\pgfqpoint{7.842520in}{7.842520in}}%
\pgfusepath{clip}%
\pgfsetbuttcap%
\pgfsetroundjoin%
\definecolor{currentfill}{rgb}{0.266580,0.228262,0.514349}%
\pgfsetfillcolor{currentfill}%
\pgfsetlinewidth{0.000000pt}%
\definecolor{currentstroke}{rgb}{0.257322,0.256130,0.526563}%
\pgfsetstrokecolor{currentstroke}%
\pgfsetdash{}{0pt}%
\pgfpathmoveto{\pgfqpoint{4.501556in}{4.310839in}}%
\pgfpathlineto{\pgfqpoint{4.446717in}{4.289781in}}%
\pgfpathlineto{\pgfqpoint{4.370276in}{4.407562in}}%
\pgfpathclose%
\pgfusepath{fill}%
\end{pgfscope}%
\begin{pgfscope}%
\pgfpathrectangle{\pgfqpoint{0.539299in}{0.078740in}}{\pgfqpoint{7.842520in}{7.842520in}}%
\pgfusepath{clip}%
\pgfsetbuttcap%
\pgfsetroundjoin%
\definecolor{currentfill}{rgb}{0.257322,0.256130,0.526563}%
\pgfsetfillcolor{currentfill}%
\pgfsetlinewidth{0.000000pt}%
\definecolor{currentstroke}{rgb}{0.255645,0.260703,0.528312}%
\pgfsetstrokecolor{currentstroke}%
\pgfsetdash{}{0pt}%
\pgfpathmoveto{\pgfqpoint{3.563833in}{4.504742in}}%
\pgfpathlineto{\pgfqpoint{3.359496in}{4.245638in}}%
\pgfpathlineto{\pgfqpoint{3.484281in}{4.506457in}}%
\pgfpathclose%
\pgfusepath{fill}%
\end{pgfscope}%
\begin{pgfscope}%
\pgfpathrectangle{\pgfqpoint{0.539299in}{0.078740in}}{\pgfqpoint{7.842520in}{7.842520in}}%
\pgfusepath{clip}%
\pgfsetbuttcap%
\pgfsetroundjoin%
\definecolor{currentfill}{rgb}{0.281887,0.150881,0.465405}%
\pgfsetfillcolor{currentfill}%
\pgfsetlinewidth{0.000000pt}%
\definecolor{currentstroke}{rgb}{0.253935,0.265254,0.529983}%
\pgfsetstrokecolor{currentstroke}%
\pgfsetdash{}{0pt}%
\pgfpathmoveto{\pgfqpoint{4.839392in}{4.029475in}}%
\pgfpathlineto{\pgfqpoint{4.708270in}{4.115796in}}%
\pgfpathlineto{\pgfqpoint{4.764671in}{4.104606in}}%
\pgfpathclose%
\pgfusepath{fill}%
\end{pgfscope}%
\begin{pgfscope}%
\pgfpathrectangle{\pgfqpoint{0.539299in}{0.078740in}}{\pgfqpoint{7.842520in}{7.842520in}}%
\pgfusepath{clip}%
\pgfsetbuttcap%
\pgfsetroundjoin%
\definecolor{currentfill}{rgb}{0.266580,0.228262,0.514349}%
\pgfsetfillcolor{currentfill}%
\pgfsetlinewidth{0.000000pt}%
\definecolor{currentstroke}{rgb}{0.252194,0.269783,0.531579}%
\pgfsetstrokecolor{currentstroke}%
\pgfsetdash{}{0pt}%
\pgfpathmoveto{\pgfqpoint{3.484281in}{4.506457in}}%
\pgfpathlineto{\pgfqpoint{3.359496in}{4.245638in}}%
\pgfpathlineto{\pgfqpoint{3.279998in}{4.210712in}}%
\pgfpathclose%
\pgfusepath{fill}%
\end{pgfscope}%
\begin{pgfscope}%
\pgfpathrectangle{\pgfqpoint{0.539299in}{0.078740in}}{\pgfqpoint{7.842520in}{7.842520in}}%
\pgfusepath{clip}%
\pgfsetbuttcap%
\pgfsetroundjoin%
\definecolor{currentfill}{rgb}{0.279566,0.067836,0.391917}%
\pgfsetfillcolor{currentfill}%
\pgfsetlinewidth{0.000000pt}%
\definecolor{currentstroke}{rgb}{0.250425,0.274290,0.533103}%
\pgfsetstrokecolor{currentstroke}%
\pgfsetdash{}{0pt}%
\pgfpathmoveto{\pgfqpoint{5.234563in}{3.814537in}}%
\pgfpathlineto{\pgfqpoint{5.307605in}{3.797555in}}%
\pgfpathlineto{\pgfqpoint{5.102484in}{3.877072in}}%
\pgfpathclose%
\pgfusepath{fill}%
\end{pgfscope}%
\begin{pgfscope}%
\pgfpathrectangle{\pgfqpoint{0.539299in}{0.078740in}}{\pgfqpoint{7.842520in}{7.842520in}}%
\pgfusepath{clip}%
\pgfsetbuttcap%
\pgfsetroundjoin%
\definecolor{currentfill}{rgb}{0.276022,0.044167,0.370164}%
\pgfsetfillcolor{currentfill}%
\pgfsetlinewidth{0.000000pt}%
\definecolor{currentstroke}{rgb}{0.248629,0.278775,0.534556}%
\pgfsetstrokecolor{currentstroke}%
\pgfsetdash{}{0pt}%
\pgfpathmoveto{\pgfqpoint{5.439811in}{3.753615in}}%
\pgfpathlineto{\pgfqpoint{5.307605in}{3.797555in}}%
\pgfpathlineto{\pgfqpoint{5.367069in}{3.761390in}}%
\pgfpathclose%
\pgfusepath{fill}%
\end{pgfscope}%
\begin{pgfscope}%
\pgfpathrectangle{\pgfqpoint{0.539299in}{0.078740in}}{\pgfqpoint{7.842520in}{7.842520in}}%
\pgfusepath{clip}%
\pgfsetbuttcap%
\pgfsetroundjoin%
\definecolor{currentfill}{rgb}{0.282884,0.135920,0.453427}%
\pgfsetfillcolor{currentfill}%
\pgfsetlinewidth{0.000000pt}%
\definecolor{currentstroke}{rgb}{0.246811,0.283237,0.535941}%
\pgfsetstrokecolor{currentstroke}%
\pgfsetdash{}{0pt}%
\pgfpathmoveto{\pgfqpoint{4.970779in}{3.949088in}}%
\pgfpathlineto{\pgfqpoint{4.839392in}{4.029475in}}%
\pgfpathlineto{\pgfqpoint{4.764671in}{4.104606in}}%
\pgfpathclose%
\pgfusepath{fill}%
\end{pgfscope}%
\begin{pgfscope}%
\pgfpathrectangle{\pgfqpoint{0.539299in}{0.078740in}}{\pgfqpoint{7.842520in}{7.842520in}}%
\pgfusepath{clip}%
\pgfsetbuttcap%
\pgfsetroundjoin%
\definecolor{currentfill}{rgb}{0.274128,0.199721,0.498911}%
\pgfsetfillcolor{currentfill}%
\pgfsetlinewidth{0.000000pt}%
\definecolor{currentstroke}{rgb}{0.244972,0.287675,0.537260}%
\pgfsetstrokecolor{currentstroke}%
\pgfsetdash{}{0pt}%
\pgfpathmoveto{\pgfqpoint{4.501556in}{4.310839in}}%
\pgfpathlineto{\pgfqpoint{4.633026in}{4.207547in}}%
\pgfpathlineto{\pgfqpoint{4.708270in}{4.115796in}}%
\pgfpathclose%
\pgfusepath{fill}%
\end{pgfscope}%
\begin{pgfscope}%
\pgfpathrectangle{\pgfqpoint{0.539299in}{0.078740in}}{\pgfqpoint{7.842520in}{7.842520in}}%
\pgfusepath{clip}%
\pgfsetbuttcap%
\pgfsetroundjoin%
\definecolor{currentfill}{rgb}{0.278791,0.062145,0.386592}%
\pgfsetfillcolor{currentfill}%
\pgfsetlinewidth{0.000000pt}%
\definecolor{currentstroke}{rgb}{0.243113,0.292092,0.538516}%
\pgfsetstrokecolor{currentstroke}%
\pgfsetdash{}{0pt}%
\pgfpathmoveto{\pgfqpoint{5.367069in}{3.761390in}}%
\pgfpathlineto{\pgfqpoint{5.307605in}{3.797555in}}%
\pgfpathlineto{\pgfqpoint{5.234563in}{3.814537in}}%
\pgfpathclose%
\pgfusepath{fill}%
\end{pgfscope}%
\begin{pgfscope}%
\pgfpathrectangle{\pgfqpoint{0.539299in}{0.078740in}}{\pgfqpoint{7.842520in}{7.842520in}}%
\pgfusepath{clip}%
\pgfsetbuttcap%
\pgfsetroundjoin%
\definecolor{currentfill}{rgb}{0.278012,0.180367,0.486697}%
\pgfsetfillcolor{currentfill}%
\pgfsetlinewidth{0.000000pt}%
\definecolor{currentstroke}{rgb}{0.241237,0.296485,0.539709}%
\pgfsetstrokecolor{currentstroke}%
\pgfsetdash{}{0pt}%
\pgfpathmoveto{\pgfqpoint{4.764671in}{4.104606in}}%
\pgfpathlineto{\pgfqpoint{4.708270in}{4.115796in}}%
\pgfpathlineto{\pgfqpoint{4.633026in}{4.207547in}}%
\pgfpathclose%
\pgfusepath{fill}%
\end{pgfscope}%
\begin{pgfscope}%
\pgfpathrectangle{\pgfqpoint{0.539299in}{0.078740in}}{\pgfqpoint{7.842520in}{7.842520in}}%
\pgfusepath{clip}%
\pgfsetbuttcap%
\pgfsetroundjoin%
\definecolor{currentfill}{rgb}{0.276022,0.044167,0.370164}%
\pgfsetfillcolor{currentfill}%
\pgfsetlinewidth{0.000000pt}%
\definecolor{currentstroke}{rgb}{0.239346,0.300855,0.540844}%
\pgfsetstrokecolor{currentstroke}%
\pgfsetdash{}{0pt}%
\pgfpathmoveto{\pgfqpoint{5.367069in}{3.761390in}}%
\pgfpathlineto{\pgfqpoint{5.572472in}{3.714282in}}%
\pgfpathlineto{\pgfqpoint{5.439811in}{3.753615in}}%
\pgfpathclose%
\pgfusepath{fill}%
\end{pgfscope}%
\begin{pgfscope}%
\pgfpathrectangle{\pgfqpoint{0.539299in}{0.078740in}}{\pgfqpoint{7.842520in}{7.842520in}}%
\pgfusepath{clip}%
\pgfsetbuttcap%
\pgfsetroundjoin%
\definecolor{currentfill}{rgb}{0.237441,0.305202,0.541921}%
\pgfsetfillcolor{currentfill}%
\pgfsetlinewidth{0.000000pt}%
\definecolor{currentstroke}{rgb}{0.237441,0.305202,0.541921}%
\pgfsetstrokecolor{currentstroke}%
\pgfsetdash{}{0pt}%
\pgfpathmoveto{\pgfqpoint{4.161473in}{4.598110in}}%
\pgfpathlineto{\pgfqpoint{4.239257in}{4.489420in}}%
\pgfpathlineto{\pgfqpoint{4.108635in}{4.547005in}}%
\pgfpathclose%
\pgfusepath{fill}%
\end{pgfscope}%
\begin{pgfscope}%
\pgfpathrectangle{\pgfqpoint{0.539299in}{0.078740in}}{\pgfqpoint{7.842520in}{7.842520in}}%
\pgfusepath{clip}%
\pgfsetbuttcap%
\pgfsetroundjoin%
\definecolor{currentfill}{rgb}{0.225863,0.330805,0.547314}%
\pgfsetfillcolor{currentfill}%
\pgfsetlinewidth{0.000000pt}%
\definecolor{currentstroke}{rgb}{0.235526,0.309527,0.542944}%
\pgfsetstrokecolor{currentstroke}%
\pgfsetdash{}{0pt}%
\pgfpathmoveto{\pgfqpoint{3.770691in}{4.607440in}}%
\pgfpathlineto{\pgfqpoint{3.691406in}{4.650385in}}%
\pgfpathlineto{\pgfqpoint{3.899971in}{4.655078in}}%
\pgfpathclose%
\pgfusepath{fill}%
\end{pgfscope}%
\begin{pgfscope}%
\pgfpathrectangle{\pgfqpoint{0.539299in}{0.078740in}}{\pgfqpoint{7.842520in}{7.842520in}}%
\pgfusepath{clip}%
\pgfsetbuttcap%
\pgfsetroundjoin%
\definecolor{currentfill}{rgb}{0.229739,0.322361,0.545706}%
\pgfsetfillcolor{currentfill}%
\pgfsetlinewidth{0.000000pt}%
\definecolor{currentstroke}{rgb}{0.233603,0.313828,0.543914}%
\pgfsetstrokecolor{currentstroke}%
\pgfsetdash{}{0pt}%
\pgfpathmoveto{\pgfqpoint{3.899971in}{4.655078in}}%
\pgfpathlineto{\pgfqpoint{4.030348in}{4.648062in}}%
\pgfpathlineto{\pgfqpoint{4.108635in}{4.547005in}}%
\pgfpathclose%
\pgfusepath{fill}%
\end{pgfscope}%
\begin{pgfscope}%
\pgfpathrectangle{\pgfqpoint{0.539299in}{0.078740in}}{\pgfqpoint{7.842520in}{7.842520in}}%
\pgfusepath{clip}%
\pgfsetbuttcap%
\pgfsetroundjoin%
\definecolor{currentfill}{rgb}{0.283091,0.110553,0.431554}%
\pgfsetfillcolor{currentfill}%
\pgfsetlinewidth{0.000000pt}%
\definecolor{currentstroke}{rgb}{0.231674,0.318106,0.544834}%
\pgfsetstrokecolor{currentstroke}%
\pgfsetdash{}{0pt}%
\pgfpathmoveto{\pgfqpoint{5.028622in}{3.919505in}}%
\pgfpathlineto{\pgfqpoint{5.102484in}{3.877072in}}%
\pgfpathlineto{\pgfqpoint{4.970779in}{3.949088in}}%
\pgfpathclose%
\pgfusepath{fill}%
\end{pgfscope}%
\begin{pgfscope}%
\pgfpathrectangle{\pgfqpoint{0.539299in}{0.078740in}}{\pgfqpoint{7.842520in}{7.842520in}}%
\pgfusepath{clip}%
\pgfsetbuttcap%
\pgfsetroundjoin%
\definecolor{currentfill}{rgb}{0.281887,0.150881,0.465405}%
\pgfsetfillcolor{currentfill}%
\pgfsetlinewidth{0.000000pt}%
\definecolor{currentstroke}{rgb}{0.229739,0.322361,0.545706}%
\pgfsetstrokecolor{currentstroke}%
\pgfsetdash{}{0pt}%
\pgfpathmoveto{\pgfqpoint{4.896518in}{4.007367in}}%
\pgfpathlineto{\pgfqpoint{4.970779in}{3.949088in}}%
\pgfpathlineto{\pgfqpoint{4.764671in}{4.104606in}}%
\pgfpathclose%
\pgfusepath{fill}%
\end{pgfscope}%
\begin{pgfscope}%
\pgfpathrectangle{\pgfqpoint{0.539299in}{0.078740in}}{\pgfqpoint{7.842520in}{7.842520in}}%
\pgfusepath{clip}%
\pgfsetbuttcap%
\pgfsetroundjoin%
\definecolor{currentfill}{rgb}{0.282656,0.100196,0.422160}%
\pgfsetfillcolor{currentfill}%
\pgfsetlinewidth{0.000000pt}%
\definecolor{currentstroke}{rgb}{0.227802,0.326594,0.546532}%
\pgfsetstrokecolor{currentstroke}%
\pgfsetdash{}{0pt}%
\pgfpathmoveto{\pgfqpoint{5.234563in}{3.814537in}}%
\pgfpathlineto{\pgfqpoint{5.102484in}{3.877072in}}%
\pgfpathlineto{\pgfqpoint{5.028622in}{3.919505in}}%
\pgfpathclose%
\pgfusepath{fill}%
\end{pgfscope}%
\begin{pgfscope}%
\pgfpathrectangle{\pgfqpoint{0.539299in}{0.078740in}}{\pgfqpoint{7.842520in}{7.842520in}}%
\pgfusepath{clip}%
\pgfsetbuttcap%
\pgfsetroundjoin%
\definecolor{currentfill}{rgb}{0.244972,0.287675,0.537260}%
\pgfsetfillcolor{currentfill}%
\pgfsetlinewidth{0.000000pt}%
\definecolor{currentstroke}{rgb}{0.225863,0.330805,0.547314}%
\pgfsetstrokecolor{currentstroke}%
\pgfsetdash{}{0pt}%
\pgfpathmoveto{\pgfqpoint{4.370276in}{4.407562in}}%
\pgfpathlineto{\pgfqpoint{4.239257in}{4.489420in}}%
\pgfpathlineto{\pgfqpoint{4.293076in}{4.516692in}}%
\pgfpathclose%
\pgfusepath{fill}%
\end{pgfscope}%
\begin{pgfscope}%
\pgfpathrectangle{\pgfqpoint{0.539299in}{0.078740in}}{\pgfqpoint{7.842520in}{7.842520in}}%
\pgfusepath{clip}%
\pgfsetbuttcap%
\pgfsetroundjoin%
\definecolor{currentfill}{rgb}{0.227802,0.326594,0.546532}%
\pgfsetfillcolor{currentfill}%
\pgfsetlinewidth{0.000000pt}%
\definecolor{currentstroke}{rgb}{0.223925,0.334994,0.548053}%
\pgfsetstrokecolor{currentstroke}%
\pgfsetdash{}{0pt}%
\pgfpathmoveto{\pgfqpoint{4.108635in}{4.547005in}}%
\pgfpathlineto{\pgfqpoint{4.030348in}{4.648062in}}%
\pgfpathlineto{\pgfqpoint{4.161473in}{4.598110in}}%
\pgfpathclose%
\pgfusepath{fill}%
\end{pgfscope}%
\begin{pgfscope}%
\pgfpathrectangle{\pgfqpoint{0.539299in}{0.078740in}}{\pgfqpoint{7.842520in}{7.842520in}}%
\pgfusepath{clip}%
\pgfsetbuttcap%
\pgfsetroundjoin%
\definecolor{currentfill}{rgb}{0.282884,0.135920,0.453427}%
\pgfsetfillcolor{currentfill}%
\pgfsetlinewidth{0.000000pt}%
\definecolor{currentstroke}{rgb}{0.221989,0.339161,0.548752}%
\pgfsetstrokecolor{currentstroke}%
\pgfsetdash{}{0pt}%
\pgfpathmoveto{\pgfqpoint{5.028622in}{3.919505in}}%
\pgfpathlineto{\pgfqpoint{4.970779in}{3.949088in}}%
\pgfpathlineto{\pgfqpoint{4.896518in}{4.007367in}}%
\pgfpathclose%
\pgfusepath{fill}%
\end{pgfscope}%
\begin{pgfscope}%
\pgfpathrectangle{\pgfqpoint{0.539299in}{0.078740in}}{\pgfqpoint{7.842520in}{7.842520in}}%
\pgfusepath{clip}%
\pgfsetbuttcap%
\pgfsetroundjoin%
\definecolor{currentfill}{rgb}{0.277941,0.056324,0.381191}%
\pgfsetfillcolor{currentfill}%
\pgfsetlinewidth{0.000000pt}%
\definecolor{currentstroke}{rgb}{0.220057,0.343307,0.549413}%
\pgfsetstrokecolor{currentstroke}%
\pgfsetdash{}{0pt}%
\pgfpathmoveto{\pgfqpoint{5.500041in}{3.716527in}}%
\pgfpathlineto{\pgfqpoint{5.572472in}{3.714282in}}%
\pgfpathlineto{\pgfqpoint{5.367069in}{3.761390in}}%
\pgfpathclose%
\pgfusepath{fill}%
\end{pgfscope}%
\begin{pgfscope}%
\pgfpathrectangle{\pgfqpoint{0.539299in}{0.078740in}}{\pgfqpoint{7.842520in}{7.842520in}}%
\pgfusepath{clip}%
\pgfsetbuttcap%
\pgfsetroundjoin%
\definecolor{currentfill}{rgb}{0.250425,0.274290,0.533103}%
\pgfsetfillcolor{currentfill}%
\pgfsetlinewidth{0.000000pt}%
\definecolor{currentstroke}{rgb}{0.218130,0.347432,0.550038}%
\pgfsetstrokecolor{currentstroke}%
\pgfsetdash{}{0pt}%
\pgfpathmoveto{\pgfqpoint{4.293076in}{4.516692in}}%
\pgfpathlineto{\pgfqpoint{4.501556in}{4.310839in}}%
\pgfpathlineto{\pgfqpoint{4.370276in}{4.407562in}}%
\pgfpathclose%
\pgfusepath{fill}%
\end{pgfscope}%
\begin{pgfscope}%
\pgfpathrectangle{\pgfqpoint{0.539299in}{0.078740in}}{\pgfqpoint{7.842520in}{7.842520in}}%
\pgfusepath{clip}%
\pgfsetbuttcap%
\pgfsetroundjoin%
\definecolor{currentfill}{rgb}{0.231674,0.318106,0.544834}%
\pgfsetfillcolor{currentfill}%
\pgfsetlinewidth{0.000000pt}%
\definecolor{currentstroke}{rgb}{0.216210,0.351535,0.550627}%
\pgfsetstrokecolor{currentstroke}%
\pgfsetdash{}{0pt}%
\pgfpathmoveto{\pgfqpoint{3.611618in}{4.681644in}}%
\pgfpathlineto{\pgfqpoint{3.563833in}{4.504742in}}%
\pgfpathlineto{\pgfqpoint{3.484281in}{4.506457in}}%
\pgfpathclose%
\pgfusepath{fill}%
\end{pgfscope}%
\begin{pgfscope}%
\pgfpathrectangle{\pgfqpoint{0.539299in}{0.078740in}}{\pgfqpoint{7.842520in}{7.842520in}}%
\pgfusepath{clip}%
\pgfsetbuttcap%
\pgfsetroundjoin%
\definecolor{currentfill}{rgb}{0.250425,0.274290,0.533103}%
\pgfsetfillcolor{currentfill}%
\pgfsetlinewidth{0.000000pt}%
\definecolor{currentstroke}{rgb}{0.214298,0.355619,0.551184}%
\pgfsetstrokecolor{currentstroke}%
\pgfsetdash{}{0pt}%
\pgfpathmoveto{\pgfqpoint{3.279998in}{4.210712in}}%
\pgfpathlineto{\pgfqpoint{3.404261in}{4.501552in}}%
\pgfpathlineto{\pgfqpoint{3.484281in}{4.506457in}}%
\pgfpathclose%
\pgfusepath{fill}%
\end{pgfscope}%
\begin{pgfscope}%
\pgfpathrectangle{\pgfqpoint{0.539299in}{0.078740in}}{\pgfqpoint{7.842520in}{7.842520in}}%
\pgfusepath{clip}%
\pgfsetbuttcap%
\pgfsetroundjoin%
\definecolor{currentfill}{rgb}{0.223925,0.334994,0.548053}%
\pgfsetfillcolor{currentfill}%
\pgfsetlinewidth{0.000000pt}%
\definecolor{currentstroke}{rgb}{0.212395,0.359683,0.551710}%
\pgfsetstrokecolor{currentstroke}%
\pgfsetdash{}{0pt}%
\pgfpathmoveto{\pgfqpoint{3.611618in}{4.681644in}}%
\pgfpathlineto{\pgfqpoint{3.691406in}{4.650385in}}%
\pgfpathlineto{\pgfqpoint{3.563833in}{4.504742in}}%
\pgfpathclose%
\pgfusepath{fill}%
\end{pgfscope}%
\begin{pgfscope}%
\pgfpathrectangle{\pgfqpoint{0.539299in}{0.078740in}}{\pgfqpoint{7.842520in}{7.842520in}}%
\pgfusepath{clip}%
\pgfsetbuttcap%
\pgfsetroundjoin%
\definecolor{currentfill}{rgb}{0.276022,0.044167,0.370164}%
\pgfsetfillcolor{currentfill}%
\pgfsetlinewidth{0.000000pt}%
\definecolor{currentstroke}{rgb}{0.210503,0.363727,0.552206}%
\pgfsetstrokecolor{currentstroke}%
\pgfsetdash{}{0pt}%
\pgfpathmoveto{\pgfqpoint{5.705595in}{3.678274in}}%
\pgfpathlineto{\pgfqpoint{5.572472in}{3.714282in}}%
\pgfpathlineto{\pgfqpoint{5.633500in}{3.678097in}}%
\pgfpathclose%
\pgfusepath{fill}%
\end{pgfscope}%
\begin{pgfscope}%
\pgfpathrectangle{\pgfqpoint{0.539299in}{0.078740in}}{\pgfqpoint{7.842520in}{7.842520in}}%
\pgfusepath{clip}%
\pgfsetbuttcap%
\pgfsetroundjoin%
\definecolor{currentfill}{rgb}{0.258965,0.251537,0.524736}%
\pgfsetfillcolor{currentfill}%
\pgfsetlinewidth{0.000000pt}%
\definecolor{currentstroke}{rgb}{0.208623,0.367752,0.552675}%
\pgfsetstrokecolor{currentstroke}%
\pgfsetdash{}{0pt}%
\pgfpathmoveto{\pgfqpoint{4.501556in}{4.310839in}}%
\pgfpathlineto{\pgfqpoint{4.424973in}{4.414618in}}%
\pgfpathlineto{\pgfqpoint{4.633026in}{4.207547in}}%
\pgfpathclose%
\pgfusepath{fill}%
\end{pgfscope}%
\begin{pgfscope}%
\pgfpathrectangle{\pgfqpoint{0.539299in}{0.078740in}}{\pgfqpoint{7.842520in}{7.842520in}}%
\pgfusepath{clip}%
\pgfsetbuttcap%
\pgfsetroundjoin%
\definecolor{currentfill}{rgb}{0.214298,0.355619,0.551184}%
\pgfsetfillcolor{currentfill}%
\pgfsetlinewidth{0.000000pt}%
\definecolor{currentstroke}{rgb}{0.206756,0.371758,0.553117}%
\pgfsetstrokecolor{currentstroke}%
\pgfsetdash{}{0pt}%
\pgfpathmoveto{\pgfqpoint{3.691406in}{4.650385in}}%
\pgfpathlineto{\pgfqpoint{3.820757in}{4.723273in}}%
\pgfpathlineto{\pgfqpoint{3.899971in}{4.655078in}}%
\pgfpathclose%
\pgfusepath{fill}%
\end{pgfscope}%
\begin{pgfscope}%
\pgfpathrectangle{\pgfqpoint{0.539299in}{0.078740in}}{\pgfqpoint{7.842520in}{7.842520in}}%
\pgfusepath{clip}%
\pgfsetbuttcap%
\pgfsetroundjoin%
\definecolor{currentfill}{rgb}{0.231674,0.318106,0.544834}%
\pgfsetfillcolor{currentfill}%
\pgfsetlinewidth{0.000000pt}%
\definecolor{currentstroke}{rgb}{0.204903,0.375746,0.553533}%
\pgfsetstrokecolor{currentstroke}%
\pgfsetdash{}{0pt}%
\pgfpathmoveto{\pgfqpoint{4.239257in}{4.489420in}}%
\pgfpathlineto{\pgfqpoint{4.161473in}{4.598110in}}%
\pgfpathlineto{\pgfqpoint{4.293076in}{4.516692in}}%
\pgfpathclose%
\pgfusepath{fill}%
\end{pgfscope}%
\begin{pgfscope}%
\pgfpathrectangle{\pgfqpoint{0.539299in}{0.078740in}}{\pgfqpoint{7.842520in}{7.842520in}}%
\pgfusepath{clip}%
\pgfsetbuttcap%
\pgfsetroundjoin%
\definecolor{currentfill}{rgb}{0.277941,0.056324,0.381191}%
\pgfsetfillcolor{currentfill}%
\pgfsetlinewidth{0.000000pt}%
\definecolor{currentstroke}{rgb}{0.203063,0.379716,0.553925}%
\pgfsetstrokecolor{currentstroke}%
\pgfsetdash{}{0pt}%
\pgfpathmoveto{\pgfqpoint{5.633500in}{3.678097in}}%
\pgfpathlineto{\pgfqpoint{5.572472in}{3.714282in}}%
\pgfpathlineto{\pgfqpoint{5.500041in}{3.716527in}}%
\pgfpathclose%
\pgfusepath{fill}%
\end{pgfscope}%
\begin{pgfscope}%
\pgfpathrectangle{\pgfqpoint{0.539299in}{0.078740in}}{\pgfqpoint{7.842520in}{7.842520in}}%
\pgfusepath{clip}%
\pgfsetbuttcap%
\pgfsetroundjoin%
\definecolor{currentfill}{rgb}{0.281446,0.084320,0.407414}%
\pgfsetfillcolor{currentfill}%
\pgfsetlinewidth{0.000000pt}%
\definecolor{currentstroke}{rgb}{0.201239,0.383670,0.554294}%
\pgfsetstrokecolor{currentstroke}%
\pgfsetdash{}{0pt}%
\pgfpathmoveto{\pgfqpoint{5.293876in}{3.778414in}}%
\pgfpathlineto{\pgfqpoint{5.367069in}{3.761390in}}%
\pgfpathlineto{\pgfqpoint{5.234563in}{3.814537in}}%
\pgfpathclose%
\pgfusepath{fill}%
\end{pgfscope}%
\begin{pgfscope}%
\pgfpathrectangle{\pgfqpoint{0.539299in}{0.078740in}}{\pgfqpoint{7.842520in}{7.842520in}}%
\pgfusepath{clip}%
\pgfsetbuttcap%
\pgfsetroundjoin%
\definecolor{currentfill}{rgb}{0.283091,0.110553,0.431554}%
\pgfsetfillcolor{currentfill}%
\pgfsetlinewidth{0.000000pt}%
\definecolor{currentstroke}{rgb}{0.199430,0.387607,0.554642}%
\pgfsetstrokecolor{currentstroke}%
\pgfsetdash{}{0pt}%
\pgfpathmoveto{\pgfqpoint{5.028622in}{3.919505in}}%
\pgfpathlineto{\pgfqpoint{5.161052in}{3.843033in}}%
\pgfpathlineto{\pgfqpoint{5.234563in}{3.814537in}}%
\pgfpathclose%
\pgfusepath{fill}%
\end{pgfscope}%
\begin{pgfscope}%
\pgfpathrectangle{\pgfqpoint{0.539299in}{0.078740in}}{\pgfqpoint{7.842520in}{7.842520in}}%
\pgfusepath{clip}%
\pgfsetbuttcap%
\pgfsetroundjoin%
\definecolor{currentfill}{rgb}{0.267968,0.223549,0.512008}%
\pgfsetfillcolor{currentfill}%
\pgfsetlinewidth{0.000000pt}%
\definecolor{currentstroke}{rgb}{0.197636,0.391528,0.554969}%
\pgfsetstrokecolor{currentstroke}%
\pgfsetdash{}{0pt}%
\pgfpathmoveto{\pgfqpoint{4.764671in}{4.104606in}}%
\pgfpathlineto{\pgfqpoint{4.633026in}{4.207547in}}%
\pgfpathlineto{\pgfqpoint{4.557051in}{4.301638in}}%
\pgfpathclose%
\pgfusepath{fill}%
\end{pgfscope}%
\begin{pgfscope}%
\pgfpathrectangle{\pgfqpoint{0.539299in}{0.078740in}}{\pgfqpoint{7.842520in}{7.842520in}}%
\pgfusepath{clip}%
\pgfsetbuttcap%
\pgfsetroundjoin%
\definecolor{currentfill}{rgb}{0.258965,0.251537,0.524736}%
\pgfsetfillcolor{currentfill}%
\pgfsetlinewidth{0.000000pt}%
\definecolor{currentstroke}{rgb}{0.195860,0.395433,0.555276}%
\pgfsetstrokecolor{currentstroke}%
\pgfsetdash{}{0pt}%
\pgfpathmoveto{\pgfqpoint{3.279998in}{4.210712in}}%
\pgfpathlineto{\pgfqpoint{3.200055in}{4.172995in}}%
\pgfpathlineto{\pgfqpoint{3.323758in}{4.491822in}}%
\pgfpathclose%
\pgfusepath{fill}%
\end{pgfscope}%
\begin{pgfscope}%
\pgfpathrectangle{\pgfqpoint{0.539299in}{0.078740in}}{\pgfqpoint{7.842520in}{7.842520in}}%
\pgfusepath{clip}%
\pgfsetbuttcap%
\pgfsetroundjoin%
\definecolor{currentfill}{rgb}{0.212395,0.359683,0.551710}%
\pgfsetfillcolor{currentfill}%
\pgfsetlinewidth{0.000000pt}%
\definecolor{currentstroke}{rgb}{0.194100,0.399323,0.555565}%
\pgfsetstrokecolor{currentstroke}%
\pgfsetdash{}{0pt}%
\pgfpathmoveto{\pgfqpoint{3.951415in}{4.734438in}}%
\pgfpathlineto{\pgfqpoint{4.030348in}{4.648062in}}%
\pgfpathlineto{\pgfqpoint{3.899971in}{4.655078in}}%
\pgfpathclose%
\pgfusepath{fill}%
\end{pgfscope}%
\begin{pgfscope}%
\pgfpathrectangle{\pgfqpoint{0.539299in}{0.078740in}}{\pgfqpoint{7.842520in}{7.842520in}}%
\pgfusepath{clip}%
\pgfsetbuttcap%
\pgfsetroundjoin%
\definecolor{currentfill}{rgb}{0.276022,0.044167,0.370164}%
\pgfsetfillcolor{currentfill}%
\pgfsetlinewidth{0.000000pt}%
\definecolor{currentstroke}{rgb}{0.192357,0.403199,0.555836}%
\pgfsetstrokecolor{currentstroke}%
\pgfsetdash{}{0pt}%
\pgfpathmoveto{\pgfqpoint{5.767446in}{3.643788in}}%
\pgfpathlineto{\pgfqpoint{5.839174in}{3.644098in}}%
\pgfpathlineto{\pgfqpoint{5.705595in}{3.678274in}}%
\pgfpathclose%
\pgfusepath{fill}%
\end{pgfscope}%
\begin{pgfscope}%
\pgfpathrectangle{\pgfqpoint{0.539299in}{0.078740in}}{\pgfqpoint{7.842520in}{7.842520in}}%
\pgfusepath{clip}%
\pgfsetbuttcap%
\pgfsetroundjoin%
\definecolor{currentfill}{rgb}{0.282910,0.105393,0.426902}%
\pgfsetfillcolor{currentfill}%
\pgfsetlinewidth{0.000000pt}%
\definecolor{currentstroke}{rgb}{0.190631,0.407061,0.556089}%
\pgfsetstrokecolor{currentstroke}%
\pgfsetdash{}{0pt}%
\pgfpathmoveto{\pgfqpoint{5.234563in}{3.814537in}}%
\pgfpathlineto{\pgfqpoint{5.161052in}{3.843033in}}%
\pgfpathlineto{\pgfqpoint{5.293876in}{3.778414in}}%
\pgfpathclose%
\pgfusepath{fill}%
\end{pgfscope}%
\begin{pgfscope}%
\pgfpathrectangle{\pgfqpoint{0.539299in}{0.078740in}}{\pgfqpoint{7.842520in}{7.842520in}}%
\pgfusepath{clip}%
\pgfsetbuttcap%
\pgfsetroundjoin%
\definecolor{currentfill}{rgb}{0.243113,0.292092,0.538516}%
\pgfsetfillcolor{currentfill}%
\pgfsetlinewidth{0.000000pt}%
\definecolor{currentstroke}{rgb}{0.188923,0.410910,0.556326}%
\pgfsetstrokecolor{currentstroke}%
\pgfsetdash{}{0pt}%
\pgfpathmoveto{\pgfqpoint{4.293076in}{4.516692in}}%
\pgfpathlineto{\pgfqpoint{4.424973in}{4.414618in}}%
\pgfpathlineto{\pgfqpoint{4.501556in}{4.310839in}}%
\pgfpathclose%
\pgfusepath{fill}%
\end{pgfscope}%
\begin{pgfscope}%
\pgfpathrectangle{\pgfqpoint{0.539299in}{0.078740in}}{\pgfqpoint{7.842520in}{7.842520in}}%
\pgfusepath{clip}%
\pgfsetbuttcap%
\pgfsetroundjoin%
\definecolor{currentfill}{rgb}{0.277134,0.185228,0.489898}%
\pgfsetfillcolor{currentfill}%
\pgfsetlinewidth{0.000000pt}%
\definecolor{currentstroke}{rgb}{0.187231,0.414746,0.556547}%
\pgfsetstrokecolor{currentstroke}%
\pgfsetdash{}{0pt}%
\pgfpathmoveto{\pgfqpoint{4.764671in}{4.104606in}}%
\pgfpathlineto{\pgfqpoint{4.821630in}{4.074629in}}%
\pgfpathlineto{\pgfqpoint{4.896518in}{4.007367in}}%
\pgfpathclose%
\pgfusepath{fill}%
\end{pgfscope}%
\begin{pgfscope}%
\pgfpathrectangle{\pgfqpoint{0.539299in}{0.078740in}}{\pgfqpoint{7.842520in}{7.842520in}}%
\pgfusepath{clip}%
\pgfsetbuttcap%
\pgfsetroundjoin%
\definecolor{currentfill}{rgb}{0.280894,0.078907,0.402329}%
\pgfsetfillcolor{currentfill}%
\pgfsetlinewidth{0.000000pt}%
\definecolor{currentstroke}{rgb}{0.185556,0.418570,0.556753}%
\pgfsetstrokecolor{currentstroke}%
\pgfsetdash{}{0pt}%
\pgfpathmoveto{\pgfqpoint{5.500041in}{3.716527in}}%
\pgfpathlineto{\pgfqpoint{5.367069in}{3.761390in}}%
\pgfpathlineto{\pgfqpoint{5.427151in}{3.724780in}}%
\pgfpathclose%
\pgfusepath{fill}%
\end{pgfscope}%
\begin{pgfscope}%
\pgfpathrectangle{\pgfqpoint{0.539299in}{0.078740in}}{\pgfqpoint{7.842520in}{7.842520in}}%
\pgfusepath{clip}%
\pgfsetbuttcap%
\pgfsetroundjoin%
\definecolor{currentfill}{rgb}{0.223925,0.334994,0.548053}%
\pgfsetfillcolor{currentfill}%
\pgfsetlinewidth{0.000000pt}%
\definecolor{currentstroke}{rgb}{0.183898,0.422383,0.556944}%
\pgfsetstrokecolor{currentstroke}%
\pgfsetdash{}{0pt}%
\pgfpathmoveto{\pgfqpoint{3.484281in}{4.506457in}}%
\pgfpathlineto{\pgfqpoint{3.404261in}{4.501552in}}%
\pgfpathlineto{\pgfqpoint{3.611618in}{4.681644in}}%
\pgfpathclose%
\pgfusepath{fill}%
\end{pgfscope}%
\begin{pgfscope}%
\pgfpathrectangle{\pgfqpoint{0.539299in}{0.078740in}}{\pgfqpoint{7.842520in}{7.842520in}}%
\pgfusepath{clip}%
\pgfsetbuttcap%
\pgfsetroundjoin%
\definecolor{currentfill}{rgb}{0.243113,0.292092,0.538516}%
\pgfsetfillcolor{currentfill}%
\pgfsetlinewidth{0.000000pt}%
\definecolor{currentstroke}{rgb}{0.182256,0.426184,0.557120}%
\pgfsetstrokecolor{currentstroke}%
\pgfsetdash{}{0pt}%
\pgfpathmoveto{\pgfqpoint{3.323758in}{4.491822in}}%
\pgfpathlineto{\pgfqpoint{3.404261in}{4.501552in}}%
\pgfpathlineto{\pgfqpoint{3.279998in}{4.210712in}}%
\pgfpathclose%
\pgfusepath{fill}%
\end{pgfscope}%
\begin{pgfscope}%
\pgfpathrectangle{\pgfqpoint{0.539299in}{0.078740in}}{\pgfqpoint{7.842520in}{7.842520in}}%
\pgfusepath{clip}%
\pgfsetbuttcap%
\pgfsetroundjoin%
\definecolor{currentfill}{rgb}{0.277941,0.056324,0.381191}%
\pgfsetfillcolor{currentfill}%
\pgfsetlinewidth{0.000000pt}%
\definecolor{currentstroke}{rgb}{0.180629,0.429975,0.557282}%
\pgfsetstrokecolor{currentstroke}%
\pgfsetdash{}{0pt}%
\pgfpathmoveto{\pgfqpoint{5.705595in}{3.678274in}}%
\pgfpathlineto{\pgfqpoint{5.633500in}{3.678097in}}%
\pgfpathlineto{\pgfqpoint{5.767446in}{3.643788in}}%
\pgfpathclose%
\pgfusepath{fill}%
\end{pgfscope}%
\begin{pgfscope}%
\pgfpathrectangle{\pgfqpoint{0.539299in}{0.078740in}}{\pgfqpoint{7.842520in}{7.842520in}}%
\pgfusepath{clip}%
\pgfsetbuttcap%
\pgfsetroundjoin%
\definecolor{currentfill}{rgb}{0.253935,0.265254,0.529983}%
\pgfsetfillcolor{currentfill}%
\pgfsetlinewidth{0.000000pt}%
\definecolor{currentstroke}{rgb}{0.179019,0.433756,0.557430}%
\pgfsetstrokecolor{currentstroke}%
\pgfsetdash{}{0pt}%
\pgfpathmoveto{\pgfqpoint{4.633026in}{4.207547in}}%
\pgfpathlineto{\pgfqpoint{4.424973in}{4.414618in}}%
\pgfpathlineto{\pgfqpoint{4.557051in}{4.301638in}}%
\pgfpathclose%
\pgfusepath{fill}%
\end{pgfscope}%
\begin{pgfscope}%
\pgfpathrectangle{\pgfqpoint{0.539299in}{0.078740in}}{\pgfqpoint{7.842520in}{7.842520in}}%
\pgfusepath{clip}%
\pgfsetbuttcap%
\pgfsetroundjoin%
\definecolor{currentfill}{rgb}{0.206756,0.371758,0.553117}%
\pgfsetfillcolor{currentfill}%
\pgfsetlinewidth{0.000000pt}%
\definecolor{currentstroke}{rgb}{0.177423,0.437527,0.557565}%
\pgfsetstrokecolor{currentstroke}%
\pgfsetdash{}{0pt}%
\pgfpathmoveto{\pgfqpoint{3.899971in}{4.655078in}}%
\pgfpathlineto{\pgfqpoint{3.820757in}{4.723273in}}%
\pgfpathlineto{\pgfqpoint{3.951415in}{4.734438in}}%
\pgfpathclose%
\pgfusepath{fill}%
\end{pgfscope}%
\begin{pgfscope}%
\pgfpathrectangle{\pgfqpoint{0.539299in}{0.078740in}}{\pgfqpoint{7.842520in}{7.842520in}}%
\pgfusepath{clip}%
\pgfsetbuttcap%
\pgfsetroundjoin%
\definecolor{currentfill}{rgb}{0.208623,0.367752,0.552675}%
\pgfsetfillcolor{currentfill}%
\pgfsetlinewidth{0.000000pt}%
\definecolor{currentstroke}{rgb}{0.175841,0.441290,0.557685}%
\pgfsetstrokecolor{currentstroke}%
\pgfsetdash{}{0pt}%
\pgfpathmoveto{\pgfqpoint{3.611618in}{4.681644in}}%
\pgfpathlineto{\pgfqpoint{3.820757in}{4.723273in}}%
\pgfpathlineto{\pgfqpoint{3.691406in}{4.650385in}}%
\pgfpathclose%
\pgfusepath{fill}%
\end{pgfscope}%
\begin{pgfscope}%
\pgfpathrectangle{\pgfqpoint{0.539299in}{0.078740in}}{\pgfqpoint{7.842520in}{7.842520in}}%
\pgfusepath{clip}%
\pgfsetbuttcap%
\pgfsetroundjoin%
\definecolor{currentfill}{rgb}{0.279574,0.170599,0.479997}%
\pgfsetfillcolor{currentfill}%
\pgfsetlinewidth{0.000000pt}%
\definecolor{currentstroke}{rgb}{0.174274,0.445044,0.557792}%
\pgfsetstrokecolor{currentstroke}%
\pgfsetdash{}{0pt}%
\pgfpathmoveto{\pgfqpoint{4.821630in}{4.074629in}}%
\pgfpathlineto{\pgfqpoint{5.028622in}{3.919505in}}%
\pgfpathlineto{\pgfqpoint{4.896518in}{4.007367in}}%
\pgfpathclose%
\pgfusepath{fill}%
\end{pgfscope}%
\begin{pgfscope}%
\pgfpathrectangle{\pgfqpoint{0.539299in}{0.078740in}}{\pgfqpoint{7.842520in}{7.842520in}}%
\pgfusepath{clip}%
\pgfsetbuttcap%
\pgfsetroundjoin%
\definecolor{currentfill}{rgb}{0.277018,0.050344,0.375715}%
\pgfsetfillcolor{currentfill}%
\pgfsetlinewidth{0.000000pt}%
\definecolor{currentstroke}{rgb}{0.172719,0.448791,0.557885}%
\pgfsetstrokecolor{currentstroke}%
\pgfsetdash{}{0pt}%
\pgfpathmoveto{\pgfqpoint{5.973195in}{3.610242in}}%
\pgfpathlineto{\pgfqpoint{5.839174in}{3.644098in}}%
\pgfpathlineto{\pgfqpoint{5.767446in}{3.643788in}}%
\pgfpathclose%
\pgfusepath{fill}%
\end{pgfscope}%
\begin{pgfscope}%
\pgfpathrectangle{\pgfqpoint{0.539299in}{0.078740in}}{\pgfqpoint{7.842520in}{7.842520in}}%
\pgfusepath{clip}%
\pgfsetbuttcap%
\pgfsetroundjoin%
\definecolor{currentfill}{rgb}{0.265145,0.232956,0.516599}%
\pgfsetfillcolor{currentfill}%
\pgfsetlinewidth{0.000000pt}%
\definecolor{currentstroke}{rgb}{0.171176,0.452530,0.557965}%
\pgfsetstrokecolor{currentstroke}%
\pgfsetdash{}{0pt}%
\pgfpathmoveto{\pgfqpoint{4.557051in}{4.301638in}}%
\pgfpathlineto{\pgfqpoint{4.689266in}{4.186089in}}%
\pgfpathlineto{\pgfqpoint{4.764671in}{4.104606in}}%
\pgfpathclose%
\pgfusepath{fill}%
\end{pgfscope}%
\begin{pgfscope}%
\pgfpathrectangle{\pgfqpoint{0.539299in}{0.078740in}}{\pgfqpoint{7.842520in}{7.842520in}}%
\pgfusepath{clip}%
\pgfsetbuttcap%
\pgfsetroundjoin%
\definecolor{currentfill}{rgb}{0.282327,0.094955,0.417331}%
\pgfsetfillcolor{currentfill}%
\pgfsetlinewidth{0.000000pt}%
\definecolor{currentstroke}{rgb}{0.169646,0.456262,0.558030}%
\pgfsetstrokecolor{currentstroke}%
\pgfsetdash{}{0pt}%
\pgfpathmoveto{\pgfqpoint{5.367069in}{3.761390in}}%
\pgfpathlineto{\pgfqpoint{5.293876in}{3.778414in}}%
\pgfpathlineto{\pgfqpoint{5.427151in}{3.724780in}}%
\pgfpathclose%
\pgfusepath{fill}%
\end{pgfscope}%
\begin{pgfscope}%
\pgfpathrectangle{\pgfqpoint{0.539299in}{0.078740in}}{\pgfqpoint{7.842520in}{7.842520in}}%
\pgfusepath{clip}%
\pgfsetbuttcap%
\pgfsetroundjoin%
\definecolor{currentfill}{rgb}{0.210503,0.363727,0.552206}%
\pgfsetfillcolor{currentfill}%
\pgfsetlinewidth{0.000000pt}%
\definecolor{currentstroke}{rgb}{0.168126,0.459988,0.558082}%
\pgfsetstrokecolor{currentstroke}%
\pgfsetdash{}{0pt}%
\pgfpathmoveto{\pgfqpoint{4.082979in}{4.695694in}}%
\pgfpathlineto{\pgfqpoint{4.161473in}{4.598110in}}%
\pgfpathlineto{\pgfqpoint{4.030348in}{4.648062in}}%
\pgfpathclose%
\pgfusepath{fill}%
\end{pgfscope}%
\begin{pgfscope}%
\pgfpathrectangle{\pgfqpoint{0.539299in}{0.078740in}}{\pgfqpoint{7.842520in}{7.842520in}}%
\pgfusepath{clip}%
\pgfsetbuttcap%
\pgfsetroundjoin%
\definecolor{currentfill}{rgb}{0.280894,0.078907,0.402329}%
\pgfsetfillcolor{currentfill}%
\pgfsetlinewidth{0.000000pt}%
\definecolor{currentstroke}{rgb}{0.166617,0.463708,0.558119}%
\pgfsetstrokecolor{currentstroke}%
\pgfsetdash{}{0pt}%
\pgfpathmoveto{\pgfqpoint{5.427151in}{3.724780in}}%
\pgfpathlineto{\pgfqpoint{5.633500in}{3.678097in}}%
\pgfpathlineto{\pgfqpoint{5.500041in}{3.716527in}}%
\pgfpathclose%
\pgfusepath{fill}%
\end{pgfscope}%
\begin{pgfscope}%
\pgfpathrectangle{\pgfqpoint{0.539299in}{0.078740in}}{\pgfqpoint{7.842520in}{7.842520in}}%
\pgfusepath{clip}%
\pgfsetbuttcap%
\pgfsetroundjoin%
\definecolor{currentfill}{rgb}{0.270595,0.214069,0.507052}%
\pgfsetfillcolor{currentfill}%
\pgfsetlinewidth{0.000000pt}%
\definecolor{currentstroke}{rgb}{0.165117,0.467423,0.558141}%
\pgfsetstrokecolor{currentstroke}%
\pgfsetdash{}{0pt}%
\pgfpathmoveto{\pgfqpoint{4.689266in}{4.186089in}}%
\pgfpathlineto{\pgfqpoint{4.821630in}{4.074629in}}%
\pgfpathlineto{\pgfqpoint{4.764671in}{4.104606in}}%
\pgfpathclose%
\pgfusepath{fill}%
\end{pgfscope}%
\begin{pgfscope}%
\pgfpathrectangle{\pgfqpoint{0.539299in}{0.078740in}}{\pgfqpoint{7.842520in}{7.842520in}}%
\pgfusepath{clip}%
\pgfsetbuttcap%
\pgfsetroundjoin%
\definecolor{currentfill}{rgb}{0.282623,0.140926,0.457517}%
\pgfsetfillcolor{currentfill}%
\pgfsetlinewidth{0.000000pt}%
\definecolor{currentstroke}{rgb}{0.163625,0.471133,0.558148}%
\pgfsetstrokecolor{currentstroke}%
\pgfsetdash{}{0pt}%
\pgfpathmoveto{\pgfqpoint{5.087019in}{3.881455in}}%
\pgfpathlineto{\pgfqpoint{5.161052in}{3.843033in}}%
\pgfpathlineto{\pgfqpoint{5.028622in}{3.919505in}}%
\pgfpathclose%
\pgfusepath{fill}%
\end{pgfscope}%
\begin{pgfscope}%
\pgfpathrectangle{\pgfqpoint{0.539299in}{0.078740in}}{\pgfqpoint{7.842520in}{7.842520in}}%
\pgfusepath{clip}%
\pgfsetbuttcap%
\pgfsetroundjoin%
\definecolor{currentfill}{rgb}{0.214298,0.355619,0.551184}%
\pgfsetfillcolor{currentfill}%
\pgfsetlinewidth{0.000000pt}%
\definecolor{currentstroke}{rgb}{0.162142,0.474838,0.558140}%
\pgfsetstrokecolor{currentstroke}%
\pgfsetdash{}{0pt}%
\pgfpathmoveto{\pgfqpoint{4.082979in}{4.695694in}}%
\pgfpathlineto{\pgfqpoint{4.293076in}{4.516692in}}%
\pgfpathlineto{\pgfqpoint{4.161473in}{4.598110in}}%
\pgfpathclose%
\pgfusepath{fill}%
\end{pgfscope}%
\begin{pgfscope}%
\pgfpathrectangle{\pgfqpoint{0.539299in}{0.078740in}}{\pgfqpoint{7.842520in}{7.842520in}}%
\pgfusepath{clip}%
\pgfsetbuttcap%
\pgfsetroundjoin%
\definecolor{currentfill}{rgb}{0.278012,0.180367,0.486697}%
\pgfsetfillcolor{currentfill}%
\pgfsetlinewidth{0.000000pt}%
\definecolor{currentstroke}{rgb}{0.160665,0.478540,0.558115}%
\pgfsetstrokecolor{currentstroke}%
\pgfsetdash{}{0pt}%
\pgfpathmoveto{\pgfqpoint{4.954191in}{3.972092in}}%
\pgfpathlineto{\pgfqpoint{5.028622in}{3.919505in}}%
\pgfpathlineto{\pgfqpoint{4.821630in}{4.074629in}}%
\pgfpathclose%
\pgfusepath{fill}%
\end{pgfscope}%
\begin{pgfscope}%
\pgfpathrectangle{\pgfqpoint{0.539299in}{0.078740in}}{\pgfqpoint{7.842520in}{7.842520in}}%
\pgfusepath{clip}%
\pgfsetbuttcap%
\pgfsetroundjoin%
\definecolor{currentfill}{rgb}{0.203063,0.379716,0.553925}%
\pgfsetfillcolor{currentfill}%
\pgfsetlinewidth{0.000000pt}%
\definecolor{currentstroke}{rgb}{0.159194,0.482237,0.558073}%
\pgfsetstrokecolor{currentstroke}%
\pgfsetdash{}{0pt}%
\pgfpathmoveto{\pgfqpoint{4.082979in}{4.695694in}}%
\pgfpathlineto{\pgfqpoint{4.030348in}{4.648062in}}%
\pgfpathlineto{\pgfqpoint{3.951415in}{4.734438in}}%
\pgfpathclose%
\pgfusepath{fill}%
\end{pgfscope}%
\begin{pgfscope}%
\pgfpathrectangle{\pgfqpoint{0.539299in}{0.078740in}}{\pgfqpoint{7.842520in}{7.842520in}}%
\pgfusepath{clip}%
\pgfsetbuttcap%
\pgfsetroundjoin%
\definecolor{currentfill}{rgb}{0.277018,0.050344,0.375715}%
\pgfsetfillcolor{currentfill}%
\pgfsetlinewidth{0.000000pt}%
\definecolor{currentstroke}{rgb}{0.157729,0.485932,0.558013}%
\pgfsetstrokecolor{currentstroke}%
\pgfsetdash{}{0pt}%
\pgfpathmoveto{\pgfqpoint{6.107631in}{3.575340in}}%
\pgfpathlineto{\pgfqpoint{5.973195in}{3.610242in}}%
\pgfpathlineto{\pgfqpoint{5.901860in}{3.611134in}}%
\pgfpathclose%
\pgfusepath{fill}%
\end{pgfscope}%
\begin{pgfscope}%
\pgfpathrectangle{\pgfqpoint{0.539299in}{0.078740in}}{\pgfqpoint{7.842520in}{7.842520in}}%
\pgfusepath{clip}%
\pgfsetbuttcap%
\pgfsetroundjoin%
\definecolor{currentfill}{rgb}{0.253935,0.265254,0.529983}%
\pgfsetfillcolor{currentfill}%
\pgfsetlinewidth{0.000000pt}%
\definecolor{currentstroke}{rgb}{0.156270,0.489624,0.557936}%
\pgfsetstrokecolor{currentstroke}%
\pgfsetdash{}{0pt}%
\pgfpathmoveto{\pgfqpoint{3.200055in}{4.172995in}}%
\pgfpathlineto{\pgfqpoint{3.119658in}{4.133109in}}%
\pgfpathlineto{\pgfqpoint{3.242761in}{4.478380in}}%
\pgfpathclose%
\pgfusepath{fill}%
\end{pgfscope}%
\begin{pgfscope}%
\pgfpathrectangle{\pgfqpoint{0.539299in}{0.078740in}}{\pgfqpoint{7.842520in}{7.842520in}}%
\pgfusepath{clip}%
\pgfsetbuttcap%
\pgfsetroundjoin%
\definecolor{currentfill}{rgb}{0.278791,0.062145,0.386592}%
\pgfsetfillcolor{currentfill}%
\pgfsetlinewidth{0.000000pt}%
\definecolor{currentstroke}{rgb}{0.154815,0.493313,0.557840}%
\pgfsetstrokecolor{currentstroke}%
\pgfsetdash{}{0pt}%
\pgfpathmoveto{\pgfqpoint{5.901860in}{3.611134in}}%
\pgfpathlineto{\pgfqpoint{5.973195in}{3.610242in}}%
\pgfpathlineto{\pgfqpoint{5.767446in}{3.643788in}}%
\pgfpathclose%
\pgfusepath{fill}%
\end{pgfscope}%
\begin{pgfscope}%
\pgfpathrectangle{\pgfqpoint{0.539299in}{0.078740in}}{\pgfqpoint{7.842520in}{7.842520in}}%
\pgfusepath{clip}%
\pgfsetbuttcap%
\pgfsetroundjoin%
\definecolor{currentfill}{rgb}{0.283187,0.125848,0.444960}%
\pgfsetfillcolor{currentfill}%
\pgfsetlinewidth{0.000000pt}%
\definecolor{currentstroke}{rgb}{0.153364,0.497000,0.557724}%
\pgfsetstrokecolor{currentstroke}%
\pgfsetdash{}{0pt}%
\pgfpathmoveto{\pgfqpoint{5.220192in}{3.803938in}}%
\pgfpathlineto{\pgfqpoint{5.293876in}{3.778414in}}%
\pgfpathlineto{\pgfqpoint{5.161052in}{3.843033in}}%
\pgfpathclose%
\pgfusepath{fill}%
\end{pgfscope}%
\begin{pgfscope}%
\pgfpathrectangle{\pgfqpoint{0.539299in}{0.078740in}}{\pgfqpoint{7.842520in}{7.842520in}}%
\pgfusepath{clip}%
\pgfsetbuttcap%
\pgfsetroundjoin%
\definecolor{currentfill}{rgb}{0.280255,0.165693,0.476498}%
\pgfsetfillcolor{currentfill}%
\pgfsetlinewidth{0.000000pt}%
\definecolor{currentstroke}{rgb}{0.151918,0.500685,0.557587}%
\pgfsetstrokecolor{currentstroke}%
\pgfsetdash{}{0pt}%
\pgfpathmoveto{\pgfqpoint{5.028622in}{3.919505in}}%
\pgfpathlineto{\pgfqpoint{4.954191in}{3.972092in}}%
\pgfpathlineto{\pgfqpoint{5.087019in}{3.881455in}}%
\pgfpathclose%
\pgfusepath{fill}%
\end{pgfscope}%
\begin{pgfscope}%
\pgfpathrectangle{\pgfqpoint{0.539299in}{0.078740in}}{\pgfqpoint{7.842520in}{7.842520in}}%
\pgfusepath{clip}%
\pgfsetbuttcap%
\pgfsetroundjoin%
\definecolor{currentfill}{rgb}{0.280894,0.078907,0.402329}%
\pgfsetfillcolor{currentfill}%
\pgfsetlinewidth{0.000000pt}%
\definecolor{currentstroke}{rgb}{0.150476,0.504369,0.557430}%
\pgfsetstrokecolor{currentstroke}%
\pgfsetdash{}{0pt}%
\pgfpathmoveto{\pgfqpoint{5.560913in}{3.680214in}}%
\pgfpathlineto{\pgfqpoint{5.767446in}{3.643788in}}%
\pgfpathlineto{\pgfqpoint{5.633500in}{3.678097in}}%
\pgfpathclose%
\pgfusepath{fill}%
\end{pgfscope}%
\begin{pgfscope}%
\pgfpathrectangle{\pgfqpoint{0.539299in}{0.078740in}}{\pgfqpoint{7.842520in}{7.842520in}}%
\pgfusepath{clip}%
\pgfsetbuttcap%
\pgfsetroundjoin%
\definecolor{currentfill}{rgb}{0.195860,0.395433,0.555276}%
\pgfsetfillcolor{currentfill}%
\pgfsetlinewidth{0.000000pt}%
\definecolor{currentstroke}{rgb}{0.149039,0.508051,0.557250}%
\pgfsetstrokecolor{currentstroke}%
\pgfsetdash{}{0pt}%
\pgfpathmoveto{\pgfqpoint{3.740970in}{4.779433in}}%
\pgfpathlineto{\pgfqpoint{3.820757in}{4.723273in}}%
\pgfpathlineto{\pgfqpoint{3.611618in}{4.681644in}}%
\pgfpathclose%
\pgfusepath{fill}%
\end{pgfscope}%
\begin{pgfscope}%
\pgfpathrectangle{\pgfqpoint{0.539299in}{0.078740in}}{\pgfqpoint{7.842520in}{7.842520in}}%
\pgfusepath{clip}%
\pgfsetbuttcap%
\pgfsetroundjoin%
\definecolor{currentfill}{rgb}{0.281924,0.089666,0.412415}%
\pgfsetfillcolor{currentfill}%
\pgfsetlinewidth{0.000000pt}%
\definecolor{currentstroke}{rgb}{0.147607,0.511733,0.557049}%
\pgfsetstrokecolor{currentstroke}%
\pgfsetdash{}{0pt}%
\pgfpathmoveto{\pgfqpoint{5.427151in}{3.724780in}}%
\pgfpathlineto{\pgfqpoint{5.560913in}{3.680214in}}%
\pgfpathlineto{\pgfqpoint{5.633500in}{3.678097in}}%
\pgfpathclose%
\pgfusepath{fill}%
\end{pgfscope}%
\begin{pgfscope}%
\pgfpathrectangle{\pgfqpoint{0.539299in}{0.078740in}}{\pgfqpoint{7.842520in}{7.842520in}}%
\pgfusepath{clip}%
\pgfsetbuttcap%
\pgfsetroundjoin%
\definecolor{currentfill}{rgb}{0.225863,0.330805,0.547314}%
\pgfsetfillcolor{currentfill}%
\pgfsetlinewidth{0.000000pt}%
\definecolor{currentstroke}{rgb}{0.146180,0.515413,0.556823}%
\pgfsetstrokecolor{currentstroke}%
\pgfsetdash{}{0pt}%
\pgfpathmoveto{\pgfqpoint{4.347630in}{4.516065in}}%
\pgfpathlineto{\pgfqpoint{4.424973in}{4.414618in}}%
\pgfpathlineto{\pgfqpoint{4.293076in}{4.516692in}}%
\pgfpathclose%
\pgfusepath{fill}%
\end{pgfscope}%
\begin{pgfscope}%
\pgfpathrectangle{\pgfqpoint{0.539299in}{0.078740in}}{\pgfqpoint{7.842520in}{7.842520in}}%
\pgfusepath{clip}%
\pgfsetbuttcap%
\pgfsetroundjoin%
\definecolor{currentfill}{rgb}{0.206756,0.371758,0.553117}%
\pgfsetfillcolor{currentfill}%
\pgfsetlinewidth{0.000000pt}%
\definecolor{currentstroke}{rgb}{0.144759,0.519093,0.556572}%
\pgfsetstrokecolor{currentstroke}%
\pgfsetdash{}{0pt}%
\pgfpathmoveto{\pgfqpoint{3.404261in}{4.501552in}}%
\pgfpathlineto{\pgfqpoint{3.531309in}{4.704589in}}%
\pgfpathlineto{\pgfqpoint{3.611618in}{4.681644in}}%
\pgfpathclose%
\pgfusepath{fill}%
\end{pgfscope}%
\begin{pgfscope}%
\pgfpathrectangle{\pgfqpoint{0.539299in}{0.078740in}}{\pgfqpoint{7.842520in}{7.842520in}}%
\pgfusepath{clip}%
\pgfsetbuttcap%
\pgfsetroundjoin%
\definecolor{currentfill}{rgb}{0.283229,0.120777,0.440584}%
\pgfsetfillcolor{currentfill}%
\pgfsetlinewidth{0.000000pt}%
\definecolor{currentstroke}{rgb}{0.143343,0.522773,0.556295}%
\pgfsetstrokecolor{currentstroke}%
\pgfsetdash{}{0pt}%
\pgfpathmoveto{\pgfqpoint{5.220192in}{3.803938in}}%
\pgfpathlineto{\pgfqpoint{5.427151in}{3.724780in}}%
\pgfpathlineto{\pgfqpoint{5.293876in}{3.778414in}}%
\pgfpathclose%
\pgfusepath{fill}%
\end{pgfscope}%
\begin{pgfscope}%
\pgfpathrectangle{\pgfqpoint{0.539299in}{0.078740in}}{\pgfqpoint{7.842520in}{7.842520in}}%
\pgfusepath{clip}%
\pgfsetbuttcap%
\pgfsetroundjoin%
\definecolor{currentfill}{rgb}{0.237441,0.305202,0.541921}%
\pgfsetfillcolor{currentfill}%
\pgfsetlinewidth{0.000000pt}%
\definecolor{currentstroke}{rgb}{0.141935,0.526453,0.555991}%
\pgfsetstrokecolor{currentstroke}%
\pgfsetdash{}{0pt}%
\pgfpathmoveto{\pgfqpoint{3.323758in}{4.491822in}}%
\pgfpathlineto{\pgfqpoint{3.200055in}{4.172995in}}%
\pgfpathlineto{\pgfqpoint{3.242761in}{4.478380in}}%
\pgfpathclose%
\pgfusepath{fill}%
\end{pgfscope}%
\begin{pgfscope}%
\pgfpathrectangle{\pgfqpoint{0.539299in}{0.078740in}}{\pgfqpoint{7.842520in}{7.842520in}}%
\pgfusepath{clip}%
\pgfsetbuttcap%
\pgfsetroundjoin%
\definecolor{currentfill}{rgb}{0.237441,0.305202,0.541921}%
\pgfsetfillcolor{currentfill}%
\pgfsetlinewidth{0.000000pt}%
\definecolor{currentstroke}{rgb}{0.140536,0.530132,0.555659}%
\pgfsetstrokecolor{currentstroke}%
\pgfsetdash{}{0pt}%
\pgfpathmoveto{\pgfqpoint{4.557051in}{4.301638in}}%
\pgfpathlineto{\pgfqpoint{4.424973in}{4.414618in}}%
\pgfpathlineto{\pgfqpoint{4.480330in}{4.397499in}}%
\pgfpathclose%
\pgfusepath{fill}%
\end{pgfscope}%
\begin{pgfscope}%
\pgfpathrectangle{\pgfqpoint{0.539299in}{0.078740in}}{\pgfqpoint{7.842520in}{7.842520in}}%
\pgfusepath{clip}%
\pgfsetbuttcap%
\pgfsetroundjoin%
\definecolor{currentfill}{rgb}{0.282290,0.145912,0.461510}%
\pgfsetfillcolor{currentfill}%
\pgfsetlinewidth{0.000000pt}%
\definecolor{currentstroke}{rgb}{0.139147,0.533812,0.555298}%
\pgfsetstrokecolor{currentstroke}%
\pgfsetdash{}{0pt}%
\pgfpathmoveto{\pgfqpoint{5.220192in}{3.803938in}}%
\pgfpathlineto{\pgfqpoint{5.161052in}{3.843033in}}%
\pgfpathlineto{\pgfqpoint{5.087019in}{3.881455in}}%
\pgfpathclose%
\pgfusepath{fill}%
\end{pgfscope}%
\begin{pgfscope}%
\pgfpathrectangle{\pgfqpoint{0.539299in}{0.078740in}}{\pgfqpoint{7.842520in}{7.842520in}}%
\pgfusepath{clip}%
\pgfsetbuttcap%
\pgfsetroundjoin%
\definecolor{currentfill}{rgb}{0.212395,0.359683,0.551710}%
\pgfsetfillcolor{currentfill}%
\pgfsetlinewidth{0.000000pt}%
\definecolor{currentstroke}{rgb}{0.137770,0.537492,0.554906}%
\pgfsetstrokecolor{currentstroke}%
\pgfsetdash{}{0pt}%
\pgfpathmoveto{\pgfqpoint{3.531309in}{4.704589in}}%
\pgfpathlineto{\pgfqpoint{3.404261in}{4.501552in}}%
\pgfpathlineto{\pgfqpoint{3.323758in}{4.491822in}}%
\pgfpathclose%
\pgfusepath{fill}%
\end{pgfscope}%
\begin{pgfscope}%
\pgfpathrectangle{\pgfqpoint{0.539299in}{0.078740in}}{\pgfqpoint{7.842520in}{7.842520in}}%
\pgfusepath{clip}%
\pgfsetbuttcap%
\pgfsetroundjoin%
\definecolor{currentfill}{rgb}{0.206756,0.371758,0.553117}%
\pgfsetfillcolor{currentfill}%
\pgfsetlinewidth{0.000000pt}%
\definecolor{currentstroke}{rgb}{0.136408,0.541173,0.554483}%
\pgfsetstrokecolor{currentstroke}%
\pgfsetdash{}{0pt}%
\pgfpathmoveto{\pgfqpoint{4.215129in}{4.619034in}}%
\pgfpathlineto{\pgfqpoint{4.293076in}{4.516692in}}%
\pgfpathlineto{\pgfqpoint{4.082979in}{4.695694in}}%
\pgfpathclose%
\pgfusepath{fill}%
\end{pgfscope}%
\begin{pgfscope}%
\pgfpathrectangle{\pgfqpoint{0.539299in}{0.078740in}}{\pgfqpoint{7.842520in}{7.842520in}}%
\pgfusepath{clip}%
\pgfsetbuttcap%
\pgfsetroundjoin%
\definecolor{currentfill}{rgb}{0.246811,0.283237,0.535941}%
\pgfsetfillcolor{currentfill}%
\pgfsetlinewidth{0.000000pt}%
\definecolor{currentstroke}{rgb}{0.135066,0.544853,0.554029}%
\pgfsetstrokecolor{currentstroke}%
\pgfsetdash{}{0pt}%
\pgfpathmoveto{\pgfqpoint{4.480330in}{4.397499in}}%
\pgfpathlineto{\pgfqpoint{4.689266in}{4.186089in}}%
\pgfpathlineto{\pgfqpoint{4.557051in}{4.301638in}}%
\pgfpathclose%
\pgfusepath{fill}%
\end{pgfscope}%
\begin{pgfscope}%
\pgfpathrectangle{\pgfqpoint{0.539299in}{0.078740in}}{\pgfqpoint{7.842520in}{7.842520in}}%
\pgfusepath{clip}%
\pgfsetbuttcap%
\pgfsetroundjoin%
\definecolor{currentfill}{rgb}{0.190631,0.407061,0.556089}%
\pgfsetfillcolor{currentfill}%
\pgfsetlinewidth{0.000000pt}%
\definecolor{currentstroke}{rgb}{0.133743,0.548535,0.553541}%
\pgfsetstrokecolor{currentstroke}%
\pgfsetdash{}{0pt}%
\pgfpathmoveto{\pgfqpoint{3.951415in}{4.734438in}}%
\pgfpathlineto{\pgfqpoint{3.820757in}{4.723273in}}%
\pgfpathlineto{\pgfqpoint{3.871838in}{4.810044in}}%
\pgfpathclose%
\pgfusepath{fill}%
\end{pgfscope}%
\begin{pgfscope}%
\pgfpathrectangle{\pgfqpoint{0.539299in}{0.078740in}}{\pgfqpoint{7.842520in}{7.842520in}}%
\pgfusepath{clip}%
\pgfsetbuttcap%
\pgfsetroundjoin%
\definecolor{currentfill}{rgb}{0.278791,0.062145,0.386592}%
\pgfsetfillcolor{currentfill}%
\pgfsetlinewidth{0.000000pt}%
\definecolor{currentstroke}{rgb}{0.132444,0.552216,0.553018}%
\pgfsetstrokecolor{currentstroke}%
\pgfsetdash{}{0pt}%
\pgfpathmoveto{\pgfqpoint{5.901860in}{3.611134in}}%
\pgfpathlineto{\pgfqpoint{6.036706in}{3.577796in}}%
\pgfpathlineto{\pgfqpoint{6.107631in}{3.575340in}}%
\pgfpathclose%
\pgfusepath{fill}%
\end{pgfscope}%
\begin{pgfscope}%
\pgfpathrectangle{\pgfqpoint{0.539299in}{0.078740in}}{\pgfqpoint{7.842520in}{7.842520in}}%
\pgfusepath{clip}%
\pgfsetbuttcap%
\pgfsetroundjoin%
\definecolor{currentfill}{rgb}{0.257322,0.256130,0.526563}%
\pgfsetfillcolor{currentfill}%
\pgfsetlinewidth{0.000000pt}%
\definecolor{currentstroke}{rgb}{0.131172,0.555899,0.552459}%
\pgfsetstrokecolor{currentstroke}%
\pgfsetdash{}{0pt}%
\pgfpathmoveto{\pgfqpoint{4.821630in}{4.074629in}}%
\pgfpathlineto{\pgfqpoint{4.689266in}{4.186089in}}%
\pgfpathlineto{\pgfqpoint{4.613151in}{4.272715in}}%
\pgfpathclose%
\pgfusepath{fill}%
\end{pgfscope}%
\begin{pgfscope}%
\pgfpathrectangle{\pgfqpoint{0.539299in}{0.078740in}}{\pgfqpoint{7.842520in}{7.842520in}}%
\pgfusepath{clip}%
\pgfsetbuttcap%
\pgfsetroundjoin%
\definecolor{currentfill}{rgb}{0.212395,0.359683,0.551710}%
\pgfsetfillcolor{currentfill}%
\pgfsetlinewidth{0.000000pt}%
\definecolor{currentstroke}{rgb}{0.129933,0.559582,0.551864}%
\pgfsetstrokecolor{currentstroke}%
\pgfsetdash{}{0pt}%
\pgfpathmoveto{\pgfqpoint{4.293076in}{4.516692in}}%
\pgfpathlineto{\pgfqpoint{4.215129in}{4.619034in}}%
\pgfpathlineto{\pgfqpoint{4.347630in}{4.516065in}}%
\pgfpathclose%
\pgfusepath{fill}%
\end{pgfscope}%
\begin{pgfscope}%
\pgfpathrectangle{\pgfqpoint{0.539299in}{0.078740in}}{\pgfqpoint{7.842520in}{7.842520in}}%
\pgfusepath{clip}%
\pgfsetbuttcap%
\pgfsetroundjoin%
\definecolor{currentfill}{rgb}{0.252194,0.269783,0.531579}%
\pgfsetfillcolor{currentfill}%
\pgfsetlinewidth{0.000000pt}%
\definecolor{currentstroke}{rgb}{0.128729,0.563265,0.551229}%
\pgfsetstrokecolor{currentstroke}%
\pgfsetdash{}{0pt}%
\pgfpathmoveto{\pgfqpoint{3.242761in}{4.478380in}}%
\pgfpathlineto{\pgfqpoint{3.119658in}{4.133109in}}%
\pgfpathlineto{\pgfqpoint{3.038801in}{4.091418in}}%
\pgfpathclose%
\pgfusepath{fill}%
\end{pgfscope}%
\begin{pgfscope}%
\pgfpathrectangle{\pgfqpoint{0.539299in}{0.078740in}}{\pgfqpoint{7.842520in}{7.842520in}}%
\pgfusepath{clip}%
\pgfsetbuttcap%
\pgfsetroundjoin%
\definecolor{currentfill}{rgb}{0.281924,0.089666,0.412415}%
\pgfsetfillcolor{currentfill}%
\pgfsetlinewidth{0.000000pt}%
\definecolor{currentstroke}{rgb}{0.127568,0.566949,0.550556}%
\pgfsetstrokecolor{currentstroke}%
\pgfsetdash{}{0pt}%
\pgfpathmoveto{\pgfqpoint{5.695175in}{3.642100in}}%
\pgfpathlineto{\pgfqpoint{5.767446in}{3.643788in}}%
\pgfpathlineto{\pgfqpoint{5.560913in}{3.680214in}}%
\pgfpathclose%
\pgfusepath{fill}%
\end{pgfscope}%
\begin{pgfscope}%
\pgfpathrectangle{\pgfqpoint{0.539299in}{0.078740in}}{\pgfqpoint{7.842520in}{7.842520in}}%
\pgfusepath{clip}%
\pgfsetbuttcap%
\pgfsetroundjoin%
\definecolor{currentfill}{rgb}{0.277941,0.056324,0.381191}%
\pgfsetfillcolor{currentfill}%
\pgfsetlinewidth{0.000000pt}%
\definecolor{currentstroke}{rgb}{0.126453,0.570633,0.549841}%
\pgfsetstrokecolor{currentstroke}%
\pgfsetdash{}{0pt}%
\pgfpathmoveto{\pgfqpoint{6.107631in}{3.575340in}}%
\pgfpathlineto{\pgfqpoint{6.171939in}{3.541802in}}%
\pgfpathlineto{\pgfqpoint{6.242456in}{3.538310in}}%
\pgfpathclose%
\pgfusepath{fill}%
\end{pgfscope}%
\begin{pgfscope}%
\pgfpathrectangle{\pgfqpoint{0.539299in}{0.078740in}}{\pgfqpoint{7.842520in}{7.842520in}}%
\pgfusepath{clip}%
\pgfsetbuttcap%
\pgfsetroundjoin%
\definecolor{currentfill}{rgb}{0.187231,0.414746,0.556547}%
\pgfsetfillcolor{currentfill}%
\pgfsetlinewidth{0.000000pt}%
\definecolor{currentstroke}{rgb}{0.125394,0.574318,0.549086}%
\pgfsetstrokecolor{currentstroke}%
\pgfsetdash{}{0pt}%
\pgfpathmoveto{\pgfqpoint{3.871838in}{4.810044in}}%
\pgfpathlineto{\pgfqpoint{3.820757in}{4.723273in}}%
\pgfpathlineto{\pgfqpoint{3.740970in}{4.779433in}}%
\pgfpathclose%
\pgfusepath{fill}%
\end{pgfscope}%
\begin{pgfscope}%
\pgfpathrectangle{\pgfqpoint{0.539299in}{0.078740in}}{\pgfqpoint{7.842520in}{7.842520in}}%
\pgfusepath{clip}%
\pgfsetbuttcap%
\pgfsetroundjoin%
\definecolor{currentfill}{rgb}{0.281446,0.084320,0.407414}%
\pgfsetfillcolor{currentfill}%
\pgfsetlinewidth{0.000000pt}%
\definecolor{currentstroke}{rgb}{0.124395,0.578002,0.548287}%
\pgfsetstrokecolor{currentstroke}%
\pgfsetdash{}{0pt}%
\pgfpathmoveto{\pgfqpoint{5.767446in}{3.643788in}}%
\pgfpathlineto{\pgfqpoint{5.829924in}{3.607475in}}%
\pgfpathlineto{\pgfqpoint{5.901860in}{3.611134in}}%
\pgfpathclose%
\pgfusepath{fill}%
\end{pgfscope}%
\begin{pgfscope}%
\pgfpathrectangle{\pgfqpoint{0.539299in}{0.078740in}}{\pgfqpoint{7.842520in}{7.842520in}}%
\pgfusepath{clip}%
\pgfsetbuttcap%
\pgfsetroundjoin%
\definecolor{currentfill}{rgb}{0.223925,0.334994,0.548053}%
\pgfsetfillcolor{currentfill}%
\pgfsetlinewidth{0.000000pt}%
\definecolor{currentstroke}{rgb}{0.123463,0.581687,0.547445}%
\pgfsetstrokecolor{currentstroke}%
\pgfsetdash{}{0pt}%
\pgfpathmoveto{\pgfqpoint{4.480330in}{4.397499in}}%
\pgfpathlineto{\pgfqpoint{4.424973in}{4.414618in}}%
\pgfpathlineto{\pgfqpoint{4.347630in}{4.516065in}}%
\pgfpathclose%
\pgfusepath{fill}%
\end{pgfscope}%
\begin{pgfscope}%
\pgfpathrectangle{\pgfqpoint{0.539299in}{0.078740in}}{\pgfqpoint{7.842520in}{7.842520in}}%
\pgfusepath{clip}%
\pgfsetbuttcap%
\pgfsetroundjoin%
\definecolor{currentfill}{rgb}{0.270595,0.214069,0.507052}%
\pgfsetfillcolor{currentfill}%
\pgfsetlinewidth{0.000000pt}%
\definecolor{currentstroke}{rgb}{0.122606,0.585371,0.546557}%
\pgfsetstrokecolor{currentstroke}%
\pgfsetdash{}{0pt}%
\pgfpathmoveto{\pgfqpoint{4.879148in}{4.033520in}}%
\pgfpathlineto{\pgfqpoint{4.954191in}{3.972092in}}%
\pgfpathlineto{\pgfqpoint{4.821630in}{4.074629in}}%
\pgfpathclose%
\pgfusepath{fill}%
\end{pgfscope}%
\begin{pgfscope}%
\pgfpathrectangle{\pgfqpoint{0.539299in}{0.078740in}}{\pgfqpoint{7.842520in}{7.842520in}}%
\pgfusepath{clip}%
\pgfsetbuttcap%
\pgfsetroundjoin%
\definecolor{currentfill}{rgb}{0.278791,0.062145,0.386592}%
\pgfsetfillcolor{currentfill}%
\pgfsetlinewidth{0.000000pt}%
\definecolor{currentstroke}{rgb}{0.121831,0.589055,0.545623}%
\pgfsetstrokecolor{currentstroke}%
\pgfsetdash{}{0pt}%
\pgfpathmoveto{\pgfqpoint{6.036706in}{3.577796in}}%
\pgfpathlineto{\pgfqpoint{6.171939in}{3.541802in}}%
\pgfpathlineto{\pgfqpoint{6.107631in}{3.575340in}}%
\pgfpathclose%
\pgfusepath{fill}%
\end{pgfscope}%
\begin{pgfscope}%
\pgfpathrectangle{\pgfqpoint{0.539299in}{0.078740in}}{\pgfqpoint{7.842520in}{7.842520in}}%
\pgfusepath{clip}%
\pgfsetbuttcap%
\pgfsetroundjoin%
\definecolor{currentfill}{rgb}{0.283072,0.130895,0.449241}%
\pgfsetfillcolor{currentfill}%
\pgfsetlinewidth{0.000000pt}%
\definecolor{currentstroke}{rgb}{0.121148,0.592739,0.544641}%
\pgfsetstrokecolor{currentstroke}%
\pgfsetdash{}{0pt}%
\pgfpathmoveto{\pgfqpoint{5.353780in}{3.739197in}}%
\pgfpathlineto{\pgfqpoint{5.427151in}{3.724780in}}%
\pgfpathlineto{\pgfqpoint{5.220192in}{3.803938in}}%
\pgfpathclose%
\pgfusepath{fill}%
\end{pgfscope}%
\begin{pgfscope}%
\pgfpathrectangle{\pgfqpoint{0.539299in}{0.078740in}}{\pgfqpoint{7.842520in}{7.842520in}}%
\pgfusepath{clip}%
\pgfsetbuttcap%
\pgfsetroundjoin%
\definecolor{currentfill}{rgb}{0.187231,0.414746,0.556547}%
\pgfsetfillcolor{currentfill}%
\pgfsetlinewidth{0.000000pt}%
\definecolor{currentstroke}{rgb}{0.120565,0.596422,0.543611}%
\pgfsetstrokecolor{currentstroke}%
\pgfsetdash{}{0pt}%
\pgfpathmoveto{\pgfqpoint{3.951415in}{4.734438in}}%
\pgfpathlineto{\pgfqpoint{3.871838in}{4.810044in}}%
\pgfpathlineto{\pgfqpoint{4.082979in}{4.695694in}}%
\pgfpathclose%
\pgfusepath{fill}%
\end{pgfscope}%
\begin{pgfscope}%
\pgfpathrectangle{\pgfqpoint{0.539299in}{0.078740in}}{\pgfqpoint{7.842520in}{7.842520in}}%
\pgfusepath{clip}%
\pgfsetbuttcap%
\pgfsetroundjoin%
\definecolor{currentfill}{rgb}{0.283091,0.110553,0.431554}%
\pgfsetfillcolor{currentfill}%
\pgfsetlinewidth{0.000000pt}%
\definecolor{currentstroke}{rgb}{0.120092,0.600104,0.542530}%
\pgfsetstrokecolor{currentstroke}%
\pgfsetdash{}{0pt}%
\pgfpathmoveto{\pgfqpoint{5.487835in}{3.685615in}}%
\pgfpathlineto{\pgfqpoint{5.560913in}{3.680214in}}%
\pgfpathlineto{\pgfqpoint{5.427151in}{3.724780in}}%
\pgfpathclose%
\pgfusepath{fill}%
\end{pgfscope}%
\begin{pgfscope}%
\pgfpathrectangle{\pgfqpoint{0.539299in}{0.078740in}}{\pgfqpoint{7.842520in}{7.842520in}}%
\pgfusepath{clip}%
\pgfsetbuttcap%
\pgfsetroundjoin%
\definecolor{currentfill}{rgb}{0.281924,0.089666,0.412415}%
\pgfsetfillcolor{currentfill}%
\pgfsetlinewidth{0.000000pt}%
\definecolor{currentstroke}{rgb}{0.119738,0.603785,0.541400}%
\pgfsetstrokecolor{currentstroke}%
\pgfsetdash{}{0pt}%
\pgfpathmoveto{\pgfqpoint{5.695175in}{3.642100in}}%
\pgfpathlineto{\pgfqpoint{5.829924in}{3.607475in}}%
\pgfpathlineto{\pgfqpoint{5.767446in}{3.643788in}}%
\pgfpathclose%
\pgfusepath{fill}%
\end{pgfscope}%
\begin{pgfscope}%
\pgfpathrectangle{\pgfqpoint{0.539299in}{0.078740in}}{\pgfqpoint{7.842520in}{7.842520in}}%
\pgfusepath{clip}%
\pgfsetbuttcap%
\pgfsetroundjoin%
\definecolor{currentfill}{rgb}{0.277941,0.056324,0.381191}%
\pgfsetfillcolor{currentfill}%
\pgfsetlinewidth{0.000000pt}%
\definecolor{currentstroke}{rgb}{0.119512,0.607464,0.540218}%
\pgfsetstrokecolor{currentstroke}%
\pgfsetdash{}{0pt}%
\pgfpathmoveto{\pgfqpoint{6.242456in}{3.538310in}}%
\pgfpathlineto{\pgfqpoint{6.171939in}{3.541802in}}%
\pgfpathlineto{\pgfqpoint{6.377643in}{3.498447in}}%
\pgfpathclose%
\pgfusepath{fill}%
\end{pgfscope}%
\begin{pgfscope}%
\pgfpathrectangle{\pgfqpoint{0.539299in}{0.078740in}}{\pgfqpoint{7.842520in}{7.842520in}}%
\pgfusepath{clip}%
\pgfsetbuttcap%
\pgfsetroundjoin%
\definecolor{currentfill}{rgb}{0.241237,0.296485,0.539709}%
\pgfsetfillcolor{currentfill}%
\pgfsetlinewidth{0.000000pt}%
\definecolor{currentstroke}{rgb}{0.119423,0.611141,0.538982}%
\pgfsetstrokecolor{currentstroke}%
\pgfsetdash{}{0pt}%
\pgfpathmoveto{\pgfqpoint{4.613151in}{4.272715in}}%
\pgfpathlineto{\pgfqpoint{4.689266in}{4.186089in}}%
\pgfpathlineto{\pgfqpoint{4.480330in}{4.397499in}}%
\pgfpathclose%
\pgfusepath{fill}%
\end{pgfscope}%
\begin{pgfscope}%
\pgfpathrectangle{\pgfqpoint{0.539299in}{0.078740in}}{\pgfqpoint{7.842520in}{7.842520in}}%
\pgfusepath{clip}%
\pgfsetbuttcap%
\pgfsetroundjoin%
\definecolor{currentfill}{rgb}{0.185556,0.418570,0.556753}%
\pgfsetfillcolor{currentfill}%
\pgfsetlinewidth{0.000000pt}%
\definecolor{currentstroke}{rgb}{0.119483,0.614817,0.537692}%
\pgfsetstrokecolor{currentstroke}%
\pgfsetdash{}{0pt}%
\pgfpathmoveto{\pgfqpoint{3.611618in}{4.681644in}}%
\pgfpathlineto{\pgfqpoint{3.660600in}{4.826856in}}%
\pgfpathlineto{\pgfqpoint{3.740970in}{4.779433in}}%
\pgfpathclose%
\pgfusepath{fill}%
\end{pgfscope}%
\begin{pgfscope}%
\pgfpathrectangle{\pgfqpoint{0.539299in}{0.078740in}}{\pgfqpoint{7.842520in}{7.842520in}}%
\pgfusepath{clip}%
\pgfsetbuttcap%
\pgfsetroundjoin%
\definecolor{currentfill}{rgb}{0.187231,0.414746,0.556547}%
\pgfsetfillcolor{currentfill}%
\pgfsetlinewidth{0.000000pt}%
\definecolor{currentstroke}{rgb}{0.119699,0.618490,0.536347}%
\pgfsetstrokecolor{currentstroke}%
\pgfsetdash{}{0pt}%
\pgfpathmoveto{\pgfqpoint{3.611618in}{4.681644in}}%
\pgfpathlineto{\pgfqpoint{3.531309in}{4.704589in}}%
\pgfpathlineto{\pgfqpoint{3.660600in}{4.826856in}}%
\pgfpathclose%
\pgfusepath{fill}%
\end{pgfscope}%
\begin{pgfscope}%
\pgfpathrectangle{\pgfqpoint{0.539299in}{0.078740in}}{\pgfqpoint{7.842520in}{7.842520in}}%
\pgfusepath{clip}%
\pgfsetbuttcap%
\pgfsetroundjoin%
\definecolor{currentfill}{rgb}{0.276194,0.190074,0.493001}%
\pgfsetfillcolor{currentfill}%
\pgfsetlinewidth{0.000000pt}%
\definecolor{currentstroke}{rgb}{0.120081,0.622161,0.534946}%
\pgfsetstrokecolor{currentstroke}%
\pgfsetdash{}{0pt}%
\pgfpathmoveto{\pgfqpoint{4.954191in}{3.972092in}}%
\pgfpathlineto{\pgfqpoint{5.012423in}{3.928911in}}%
\pgfpathlineto{\pgfqpoint{5.087019in}{3.881455in}}%
\pgfpathclose%
\pgfusepath{fill}%
\end{pgfscope}%
\begin{pgfscope}%
\pgfpathrectangle{\pgfqpoint{0.539299in}{0.078740in}}{\pgfqpoint{7.842520in}{7.842520in}}%
\pgfusepath{clip}%
\pgfsetbuttcap%
\pgfsetroundjoin%
\definecolor{currentfill}{rgb}{0.253935,0.265254,0.529983}%
\pgfsetfillcolor{currentfill}%
\pgfsetlinewidth{0.000000pt}%
\definecolor{currentstroke}{rgb}{0.120638,0.625828,0.533488}%
\pgfsetstrokecolor{currentstroke}%
\pgfsetdash{}{0pt}%
\pgfpathmoveto{\pgfqpoint{4.613151in}{4.272715in}}%
\pgfpathlineto{\pgfqpoint{4.746079in}{4.149433in}}%
\pgfpathlineto{\pgfqpoint{4.821630in}{4.074629in}}%
\pgfpathclose%
\pgfusepath{fill}%
\end{pgfscope}%
\begin{pgfscope}%
\pgfpathrectangle{\pgfqpoint{0.539299in}{0.078740in}}{\pgfqpoint{7.842520in}{7.842520in}}%
\pgfusepath{clip}%
\pgfsetbuttcap%
\pgfsetroundjoin%
\definecolor{currentfill}{rgb}{0.283187,0.125848,0.444960}%
\pgfsetfillcolor{currentfill}%
\pgfsetlinewidth{0.000000pt}%
\definecolor{currentstroke}{rgb}{0.121380,0.629492,0.531973}%
\pgfsetstrokecolor{currentstroke}%
\pgfsetdash{}{0pt}%
\pgfpathmoveto{\pgfqpoint{5.353780in}{3.739197in}}%
\pgfpathlineto{\pgfqpoint{5.487835in}{3.685615in}}%
\pgfpathlineto{\pgfqpoint{5.427151in}{3.724780in}}%
\pgfpathclose%
\pgfusepath{fill}%
\end{pgfscope}%
\begin{pgfscope}%
\pgfpathrectangle{\pgfqpoint{0.539299in}{0.078740in}}{\pgfqpoint{7.842520in}{7.842520in}}%
\pgfusepath{clip}%
\pgfsetbuttcap%
\pgfsetroundjoin%
\definecolor{currentfill}{rgb}{0.262138,0.242286,0.520837}%
\pgfsetfillcolor{currentfill}%
\pgfsetlinewidth{0.000000pt}%
\definecolor{currentstroke}{rgb}{0.122312,0.633153,0.530398}%
\pgfsetstrokecolor{currentstroke}%
\pgfsetdash{}{0pt}%
\pgfpathmoveto{\pgfqpoint{4.821630in}{4.074629in}}%
\pgfpathlineto{\pgfqpoint{4.746079in}{4.149433in}}%
\pgfpathlineto{\pgfqpoint{4.879148in}{4.033520in}}%
\pgfpathclose%
\pgfusepath{fill}%
\end{pgfscope}%
\begin{pgfscope}%
\pgfpathrectangle{\pgfqpoint{0.539299in}{0.078740in}}{\pgfqpoint{7.842520in}{7.842520in}}%
\pgfusepath{clip}%
\pgfsetbuttcap%
\pgfsetroundjoin%
\definecolor{currentfill}{rgb}{0.278826,0.175490,0.483397}%
\pgfsetfillcolor{currentfill}%
\pgfsetlinewidth{0.000000pt}%
\definecolor{currentstroke}{rgb}{0.123444,0.636809,0.528763}%
\pgfsetstrokecolor{currentstroke}%
\pgfsetdash{}{0pt}%
\pgfpathmoveto{\pgfqpoint{5.087019in}{3.881455in}}%
\pgfpathlineto{\pgfqpoint{5.012423in}{3.928911in}}%
\pgfpathlineto{\pgfqpoint{5.220192in}{3.803938in}}%
\pgfpathclose%
\pgfusepath{fill}%
\end{pgfscope}%
\begin{pgfscope}%
\pgfpathrectangle{\pgfqpoint{0.539299in}{0.078740in}}{\pgfqpoint{7.842520in}{7.842520in}}%
\pgfusepath{clip}%
\pgfsetbuttcap%
\pgfsetroundjoin%
\definecolor{currentfill}{rgb}{0.281446,0.084320,0.407414}%
\pgfsetfillcolor{currentfill}%
\pgfsetlinewidth{0.000000pt}%
\definecolor{currentstroke}{rgb}{0.124780,0.640461,0.527068}%
\pgfsetstrokecolor{currentstroke}%
\pgfsetdash{}{0pt}%
\pgfpathmoveto{\pgfqpoint{5.965125in}{3.573380in}}%
\pgfpathlineto{\pgfqpoint{6.036706in}{3.577796in}}%
\pgfpathlineto{\pgfqpoint{5.901860in}{3.611134in}}%
\pgfpathclose%
\pgfusepath{fill}%
\end{pgfscope}%
\begin{pgfscope}%
\pgfpathrectangle{\pgfqpoint{0.539299in}{0.078740in}}{\pgfqpoint{7.842520in}{7.842520in}}%
\pgfusepath{clip}%
\pgfsetbuttcap%
\pgfsetroundjoin%
\definecolor{currentfill}{rgb}{0.195860,0.395433,0.555276}%
\pgfsetfillcolor{currentfill}%
\pgfsetlinewidth{0.000000pt}%
\definecolor{currentstroke}{rgb}{0.126326,0.644107,0.525311}%
\pgfsetstrokecolor{currentstroke}%
\pgfsetdash{}{0pt}%
\pgfpathmoveto{\pgfqpoint{3.323758in}{4.491822in}}%
\pgfpathlineto{\pgfqpoint{3.450467in}{4.721409in}}%
\pgfpathlineto{\pgfqpoint{3.531309in}{4.704589in}}%
\pgfpathclose%
\pgfusepath{fill}%
\end{pgfscope}%
\begin{pgfscope}%
\pgfpathrectangle{\pgfqpoint{0.539299in}{0.078740in}}{\pgfqpoint{7.842520in}{7.842520in}}%
\pgfusepath{clip}%
\pgfsetbuttcap%
\pgfsetroundjoin%
\definecolor{currentfill}{rgb}{0.283091,0.110553,0.431554}%
\pgfsetfillcolor{currentfill}%
\pgfsetlinewidth{0.000000pt}%
\definecolor{currentstroke}{rgb}{0.128087,0.647749,0.523491}%
\pgfsetstrokecolor{currentstroke}%
\pgfsetdash{}{0pt}%
\pgfpathmoveto{\pgfqpoint{5.560913in}{3.680214in}}%
\pgfpathlineto{\pgfqpoint{5.622384in}{3.640649in}}%
\pgfpathlineto{\pgfqpoint{5.695175in}{3.642100in}}%
\pgfpathclose%
\pgfusepath{fill}%
\end{pgfscope}%
\begin{pgfscope}%
\pgfpathrectangle{\pgfqpoint{0.539299in}{0.078740in}}{\pgfqpoint{7.842520in}{7.842520in}}%
\pgfusepath{clip}%
\pgfsetbuttcap%
\pgfsetroundjoin%
\definecolor{currentfill}{rgb}{0.270595,0.214069,0.507052}%
\pgfsetfillcolor{currentfill}%
\pgfsetlinewidth{0.000000pt}%
\definecolor{currentstroke}{rgb}{0.130067,0.651384,0.521608}%
\pgfsetstrokecolor{currentstroke}%
\pgfsetdash{}{0pt}%
\pgfpathmoveto{\pgfqpoint{4.879148in}{4.033520in}}%
\pgfpathlineto{\pgfqpoint{5.012423in}{3.928911in}}%
\pgfpathlineto{\pgfqpoint{4.954191in}{3.972092in}}%
\pgfpathclose%
\pgfusepath{fill}%
\end{pgfscope}%
\begin{pgfscope}%
\pgfpathrectangle{\pgfqpoint{0.539299in}{0.078740in}}{\pgfqpoint{7.842520in}{7.842520in}}%
\pgfusepath{clip}%
\pgfsetbuttcap%
\pgfsetroundjoin%
\definecolor{currentfill}{rgb}{0.190631,0.407061,0.556089}%
\pgfsetfillcolor{currentfill}%
\pgfsetlinewidth{0.000000pt}%
\definecolor{currentstroke}{rgb}{0.132268,0.655014,0.519661}%
\pgfsetstrokecolor{currentstroke}%
\pgfsetdash{}{0pt}%
\pgfpathmoveto{\pgfqpoint{4.136438in}{4.716235in}}%
\pgfpathlineto{\pgfqpoint{4.215129in}{4.619034in}}%
\pgfpathlineto{\pgfqpoint{4.082979in}{4.695694in}}%
\pgfpathclose%
\pgfusepath{fill}%
\end{pgfscope}%
\begin{pgfscope}%
\pgfpathrectangle{\pgfqpoint{0.539299in}{0.078740in}}{\pgfqpoint{7.842520in}{7.842520in}}%
\pgfusepath{clip}%
\pgfsetbuttcap%
\pgfsetroundjoin%
\definecolor{currentfill}{rgb}{0.282327,0.094955,0.417331}%
\pgfsetfillcolor{currentfill}%
\pgfsetlinewidth{0.000000pt}%
\definecolor{currentstroke}{rgb}{0.134692,0.658636,0.517649}%
\pgfsetstrokecolor{currentstroke}%
\pgfsetdash{}{0pt}%
\pgfpathmoveto{\pgfqpoint{5.901860in}{3.611134in}}%
\pgfpathlineto{\pgfqpoint{5.829924in}{3.607475in}}%
\pgfpathlineto{\pgfqpoint{5.965125in}{3.573380in}}%
\pgfpathclose%
\pgfusepath{fill}%
\end{pgfscope}%
\begin{pgfscope}%
\pgfpathrectangle{\pgfqpoint{0.539299in}{0.078740in}}{\pgfqpoint{7.842520in}{7.842520in}}%
\pgfusepath{clip}%
\pgfsetbuttcap%
\pgfsetroundjoin%
\definecolor{currentfill}{rgb}{0.279566,0.067836,0.391917}%
\pgfsetfillcolor{currentfill}%
\pgfsetlinewidth{0.000000pt}%
\definecolor{currentstroke}{rgb}{0.137339,0.662252,0.515571}%
\pgfsetstrokecolor{currentstroke}%
\pgfsetdash{}{0pt}%
\pgfpathmoveto{\pgfqpoint{6.171939in}{3.541802in}}%
\pgfpathlineto{\pgfqpoint{6.307514in}{3.501723in}}%
\pgfpathlineto{\pgfqpoint{6.377643in}{3.498447in}}%
\pgfpathclose%
\pgfusepath{fill}%
\end{pgfscope}%
\begin{pgfscope}%
\pgfpathrectangle{\pgfqpoint{0.539299in}{0.078740in}}{\pgfqpoint{7.842520in}{7.842520in}}%
\pgfusepath{clip}%
\pgfsetbuttcap%
\pgfsetroundjoin%
\definecolor{currentfill}{rgb}{0.203063,0.379716,0.553925}%
\pgfsetfillcolor{currentfill}%
\pgfsetlinewidth{0.000000pt}%
\definecolor{currentstroke}{rgb}{0.140210,0.665859,0.513427}%
\pgfsetstrokecolor{currentstroke}%
\pgfsetdash{}{0pt}%
\pgfpathmoveto{\pgfqpoint{3.242761in}{4.478380in}}%
\pgfpathlineto{\pgfqpoint{3.450467in}{4.721409in}}%
\pgfpathlineto{\pgfqpoint{3.323758in}{4.491822in}}%
\pgfpathclose%
\pgfusepath{fill}%
\end{pgfscope}%
\begin{pgfscope}%
\pgfpathrectangle{\pgfqpoint{0.539299in}{0.078740in}}{\pgfqpoint{7.842520in}{7.842520in}}%
\pgfusepath{clip}%
\pgfsetbuttcap%
\pgfsetroundjoin%
\definecolor{currentfill}{rgb}{0.180629,0.429975,0.557282}%
\pgfsetfillcolor{currentfill}%
\pgfsetlinewidth{0.000000pt}%
\definecolor{currentstroke}{rgb}{0.143303,0.669459,0.511215}%
\pgfsetstrokecolor{currentstroke}%
\pgfsetdash{}{0pt}%
\pgfpathmoveto{\pgfqpoint{4.082979in}{4.695694in}}%
\pgfpathlineto{\pgfqpoint{3.871838in}{4.810044in}}%
\pgfpathlineto{\pgfqpoint{4.003782in}{4.784962in}}%
\pgfpathclose%
\pgfusepath{fill}%
\end{pgfscope}%
\begin{pgfscope}%
\pgfpathrectangle{\pgfqpoint{0.539299in}{0.078740in}}{\pgfqpoint{7.842520in}{7.842520in}}%
\pgfusepath{clip}%
\pgfsetbuttcap%
\pgfsetroundjoin%
\definecolor{currentfill}{rgb}{0.277018,0.050344,0.375715}%
\pgfsetfillcolor{currentfill}%
\pgfsetlinewidth{0.000000pt}%
\definecolor{currentstroke}{rgb}{0.146616,0.673050,0.508936}%
\pgfsetstrokecolor{currentstroke}%
\pgfsetdash{}{0pt}%
\pgfpathmoveto{\pgfqpoint{6.377643in}{3.498447in}}%
\pgfpathlineto{\pgfqpoint{6.443391in}{3.456772in}}%
\pgfpathlineto{\pgfqpoint{6.513170in}{3.455462in}}%
\pgfpathclose%
\pgfusepath{fill}%
\end{pgfscope}%
\begin{pgfscope}%
\pgfpathrectangle{\pgfqpoint{0.539299in}{0.078740in}}{\pgfqpoint{7.842520in}{7.842520in}}%
\pgfusepath{clip}%
\pgfsetbuttcap%
\pgfsetroundjoin%
\definecolor{currentfill}{rgb}{0.283229,0.120777,0.440584}%
\pgfsetfillcolor{currentfill}%
\pgfsetlinewidth{0.000000pt}%
\definecolor{currentstroke}{rgb}{0.150148,0.676631,0.506589}%
\pgfsetstrokecolor{currentstroke}%
\pgfsetdash{}{0pt}%
\pgfpathmoveto{\pgfqpoint{5.560913in}{3.680214in}}%
\pgfpathlineto{\pgfqpoint{5.487835in}{3.685615in}}%
\pgfpathlineto{\pgfqpoint{5.622384in}{3.640649in}}%
\pgfpathclose%
\pgfusepath{fill}%
\end{pgfscope}%
\begin{pgfscope}%
\pgfpathrectangle{\pgfqpoint{0.539299in}{0.078740in}}{\pgfqpoint{7.842520in}{7.842520in}}%
\pgfusepath{clip}%
\pgfsetbuttcap%
\pgfsetroundjoin%
\definecolor{currentfill}{rgb}{0.281446,0.084320,0.407414}%
\pgfsetfillcolor{currentfill}%
\pgfsetlinewidth{0.000000pt}%
\definecolor{currentstroke}{rgb}{0.153894,0.680203,0.504172}%
\pgfsetstrokecolor{currentstroke}%
\pgfsetdash{}{0pt}%
\pgfpathmoveto{\pgfqpoint{5.965125in}{3.573380in}}%
\pgfpathlineto{\pgfqpoint{6.171939in}{3.541802in}}%
\pgfpathlineto{\pgfqpoint{6.036706in}{3.577796in}}%
\pgfpathclose%
\pgfusepath{fill}%
\end{pgfscope}%
\begin{pgfscope}%
\pgfpathrectangle{\pgfqpoint{0.539299in}{0.078740in}}{\pgfqpoint{7.842520in}{7.842520in}}%
\pgfusepath{clip}%
\pgfsetbuttcap%
\pgfsetroundjoin%
\definecolor{currentfill}{rgb}{0.246811,0.283237,0.535941}%
\pgfsetfillcolor{currentfill}%
\pgfsetlinewidth{0.000000pt}%
\definecolor{currentstroke}{rgb}{0.157851,0.683765,0.501686}%
\pgfsetstrokecolor{currentstroke}%
\pgfsetdash{}{0pt}%
\pgfpathmoveto{\pgfqpoint{2.957481in}{4.048102in}}%
\pgfpathlineto{\pgfqpoint{3.161266in}{4.461876in}}%
\pgfpathlineto{\pgfqpoint{3.038801in}{4.091418in}}%
\pgfpathclose%
\pgfusepath{fill}%
\end{pgfscope}%
\begin{pgfscope}%
\pgfpathrectangle{\pgfqpoint{0.539299in}{0.078740in}}{\pgfqpoint{7.842520in}{7.842520in}}%
\pgfusepath{clip}%
\pgfsetbuttcap%
\pgfsetroundjoin%
\definecolor{currentfill}{rgb}{0.227802,0.326594,0.546532}%
\pgfsetfillcolor{currentfill}%
\pgfsetlinewidth{0.000000pt}%
\definecolor{currentstroke}{rgb}{0.162016,0.687316,0.499129}%
\pgfsetstrokecolor{currentstroke}%
\pgfsetdash{}{0pt}%
\pgfpathmoveto{\pgfqpoint{3.038801in}{4.091418in}}%
\pgfpathlineto{\pgfqpoint{3.161266in}{4.461876in}}%
\pgfpathlineto{\pgfqpoint{3.242761in}{4.478380in}}%
\pgfpathclose%
\pgfusepath{fill}%
\end{pgfscope}%
\begin{pgfscope}%
\pgfpathrectangle{\pgfqpoint{0.539299in}{0.078740in}}{\pgfqpoint{7.842520in}{7.842520in}}%
\pgfusepath{clip}%
\pgfsetbuttcap%
\pgfsetroundjoin%
\definecolor{currentfill}{rgb}{0.281412,0.155834,0.469201}%
\pgfsetfillcolor{currentfill}%
\pgfsetlinewidth{0.000000pt}%
\definecolor{currentstroke}{rgb}{0.166383,0.690856,0.496502}%
\pgfsetstrokecolor{currentstroke}%
\pgfsetdash{}{0pt}%
\pgfpathmoveto{\pgfqpoint{5.220192in}{3.803938in}}%
\pgfpathlineto{\pgfqpoint{5.279908in}{3.760172in}}%
\pgfpathlineto{\pgfqpoint{5.353780in}{3.739197in}}%
\pgfpathclose%
\pgfusepath{fill}%
\end{pgfscope}%
\begin{pgfscope}%
\pgfpathrectangle{\pgfqpoint{0.539299in}{0.078740in}}{\pgfqpoint{7.842520in}{7.842520in}}%
\pgfusepath{clip}%
\pgfsetbuttcap%
\pgfsetroundjoin%
\definecolor{currentfill}{rgb}{0.283091,0.110553,0.431554}%
\pgfsetfillcolor{currentfill}%
\pgfsetlinewidth{0.000000pt}%
\definecolor{currentstroke}{rgb}{0.170948,0.694384,0.493803}%
\pgfsetstrokecolor{currentstroke}%
\pgfsetdash{}{0pt}%
\pgfpathmoveto{\pgfqpoint{5.622384in}{3.640649in}}%
\pgfpathlineto{\pgfqpoint{5.829924in}{3.607475in}}%
\pgfpathlineto{\pgfqpoint{5.695175in}{3.642100in}}%
\pgfpathclose%
\pgfusepath{fill}%
\end{pgfscope}%
\begin{pgfscope}%
\pgfpathrectangle{\pgfqpoint{0.539299in}{0.078740in}}{\pgfqpoint{7.842520in}{7.842520in}}%
\pgfusepath{clip}%
\pgfsetbuttcap%
\pgfsetroundjoin%
\definecolor{currentfill}{rgb}{0.199430,0.387607,0.554642}%
\pgfsetfillcolor{currentfill}%
\pgfsetlinewidth{0.000000pt}%
\definecolor{currentstroke}{rgb}{0.175707,0.697900,0.491033}%
\pgfsetstrokecolor{currentstroke}%
\pgfsetdash{}{0pt}%
\pgfpathmoveto{\pgfqpoint{4.215129in}{4.619034in}}%
\pgfpathlineto{\pgfqpoint{4.269526in}{4.615818in}}%
\pgfpathlineto{\pgfqpoint{4.347630in}{4.516065in}}%
\pgfpathclose%
\pgfusepath{fill}%
\end{pgfscope}%
\begin{pgfscope}%
\pgfpathrectangle{\pgfqpoint{0.539299in}{0.078740in}}{\pgfqpoint{7.842520in}{7.842520in}}%
\pgfusepath{clip}%
\pgfsetbuttcap%
\pgfsetroundjoin%
\definecolor{currentfill}{rgb}{0.277134,0.185228,0.489898}%
\pgfsetfillcolor{currentfill}%
\pgfsetlinewidth{0.000000pt}%
\definecolor{currentstroke}{rgb}{0.180653,0.701402,0.488189}%
\pgfsetstrokecolor{currentstroke}%
\pgfsetdash{}{0pt}%
\pgfpathmoveto{\pgfqpoint{5.012423in}{3.928911in}}%
\pgfpathlineto{\pgfqpoint{5.145984in}{3.837674in}}%
\pgfpathlineto{\pgfqpoint{5.220192in}{3.803938in}}%
\pgfpathclose%
\pgfusepath{fill}%
\end{pgfscope}%
\begin{pgfscope}%
\pgfpathrectangle{\pgfqpoint{0.539299in}{0.078740in}}{\pgfqpoint{7.842520in}{7.842520in}}%
\pgfusepath{clip}%
\pgfsetbuttcap%
\pgfsetroundjoin%
\definecolor{currentfill}{rgb}{0.278791,0.062145,0.386592}%
\pgfsetfillcolor{currentfill}%
\pgfsetlinewidth{0.000000pt}%
\definecolor{currentstroke}{rgb}{0.185783,0.704891,0.485273}%
\pgfsetstrokecolor{currentstroke}%
\pgfsetdash{}{0pt}%
\pgfpathmoveto{\pgfqpoint{6.377643in}{3.498447in}}%
\pgfpathlineto{\pgfqpoint{6.307514in}{3.501723in}}%
\pgfpathlineto{\pgfqpoint{6.443391in}{3.456772in}}%
\pgfpathclose%
\pgfusepath{fill}%
\end{pgfscope}%
\begin{pgfscope}%
\pgfpathrectangle{\pgfqpoint{0.539299in}{0.078740in}}{\pgfqpoint{7.842520in}{7.842520in}}%
\pgfusepath{clip}%
\pgfsetbuttcap%
\pgfsetroundjoin%
\definecolor{currentfill}{rgb}{0.206756,0.371758,0.553117}%
\pgfsetfillcolor{currentfill}%
\pgfsetlinewidth{0.000000pt}%
\definecolor{currentstroke}{rgb}{0.191090,0.708366,0.482284}%
\pgfsetstrokecolor{currentstroke}%
\pgfsetdash{}{0pt}%
\pgfpathmoveto{\pgfqpoint{4.347630in}{4.516065in}}%
\pgfpathlineto{\pgfqpoint{4.269526in}{4.615818in}}%
\pgfpathlineto{\pgfqpoint{4.480330in}{4.397499in}}%
\pgfpathclose%
\pgfusepath{fill}%
\end{pgfscope}%
\begin{pgfscope}%
\pgfpathrectangle{\pgfqpoint{0.539299in}{0.078740in}}{\pgfqpoint{7.842520in}{7.842520in}}%
\pgfusepath{clip}%
\pgfsetbuttcap%
\pgfsetroundjoin%
\definecolor{currentfill}{rgb}{0.180629,0.429975,0.557282}%
\pgfsetfillcolor{currentfill}%
\pgfsetlinewidth{0.000000pt}%
\definecolor{currentstroke}{rgb}{0.196571,0.711827,0.479221}%
\pgfsetstrokecolor{currentstroke}%
\pgfsetdash{}{0pt}%
\pgfpathmoveto{\pgfqpoint{4.082979in}{4.695694in}}%
\pgfpathlineto{\pgfqpoint{4.003782in}{4.784962in}}%
\pgfpathlineto{\pgfqpoint{4.136438in}{4.716235in}}%
\pgfpathclose%
\pgfusepath{fill}%
\end{pgfscope}%
\begin{pgfscope}%
\pgfpathrectangle{\pgfqpoint{0.539299in}{0.078740in}}{\pgfqpoint{7.842520in}{7.842520in}}%
\pgfusepath{clip}%
\pgfsetbuttcap%
\pgfsetroundjoin%
\definecolor{currentfill}{rgb}{0.171176,0.452530,0.557965}%
\pgfsetfillcolor{currentfill}%
\pgfsetlinewidth{0.000000pt}%
\definecolor{currentstroke}{rgb}{0.202219,0.715272,0.476084}%
\pgfsetstrokecolor{currentstroke}%
\pgfsetdash{}{0pt}%
\pgfpathmoveto{\pgfqpoint{3.791614in}{4.877687in}}%
\pgfpathlineto{\pgfqpoint{3.871838in}{4.810044in}}%
\pgfpathlineto{\pgfqpoint{3.740970in}{4.779433in}}%
\pgfpathclose%
\pgfusepath{fill}%
\end{pgfscope}%
\begin{pgfscope}%
\pgfpathrectangle{\pgfqpoint{0.539299in}{0.078740in}}{\pgfqpoint{7.842520in}{7.842520in}}%
\pgfusepath{clip}%
\pgfsetbuttcap%
\pgfsetroundjoin%
\definecolor{currentfill}{rgb}{0.281887,0.150881,0.465405}%
\pgfsetfillcolor{currentfill}%
\pgfsetlinewidth{0.000000pt}%
\definecolor{currentstroke}{rgb}{0.208030,0.718701,0.472873}%
\pgfsetstrokecolor{currentstroke}%
\pgfsetdash{}{0pt}%
\pgfpathmoveto{\pgfqpoint{5.279908in}{3.760172in}}%
\pgfpathlineto{\pgfqpoint{5.487835in}{3.685615in}}%
\pgfpathlineto{\pgfqpoint{5.353780in}{3.739197in}}%
\pgfpathclose%
\pgfusepath{fill}%
\end{pgfscope}%
\begin{pgfscope}%
\pgfpathrectangle{\pgfqpoint{0.539299in}{0.078740in}}{\pgfqpoint{7.842520in}{7.842520in}}%
\pgfusepath{clip}%
\pgfsetbuttcap%
\pgfsetroundjoin%
\definecolor{currentfill}{rgb}{0.279574,0.170599,0.479997}%
\pgfsetfillcolor{currentfill}%
\pgfsetlinewidth{0.000000pt}%
\definecolor{currentstroke}{rgb}{0.214000,0.722114,0.469588}%
\pgfsetstrokecolor{currentstroke}%
\pgfsetdash{}{0pt}%
\pgfpathmoveto{\pgfqpoint{5.220192in}{3.803938in}}%
\pgfpathlineto{\pgfqpoint{5.145984in}{3.837674in}}%
\pgfpathlineto{\pgfqpoint{5.279908in}{3.760172in}}%
\pgfpathclose%
\pgfusepath{fill}%
\end{pgfscope}%
\begin{pgfscope}%
\pgfpathrectangle{\pgfqpoint{0.539299in}{0.078740in}}{\pgfqpoint{7.842520in}{7.842520in}}%
\pgfusepath{clip}%
\pgfsetbuttcap%
\pgfsetroundjoin%
\definecolor{currentfill}{rgb}{0.225863,0.330805,0.547314}%
\pgfsetfillcolor{currentfill}%
\pgfsetlinewidth{0.000000pt}%
\definecolor{currentstroke}{rgb}{0.220124,0.725509,0.466226}%
\pgfsetstrokecolor{currentstroke}%
\pgfsetdash{}{0pt}%
\pgfpathmoveto{\pgfqpoint{4.536301in}{4.363851in}}%
\pgfpathlineto{\pgfqpoint{4.613151in}{4.272715in}}%
\pgfpathlineto{\pgfqpoint{4.480330in}{4.397499in}}%
\pgfpathclose%
\pgfusepath{fill}%
\end{pgfscope}%
\begin{pgfscope}%
\pgfpathrectangle{\pgfqpoint{0.539299in}{0.078740in}}{\pgfqpoint{7.842520in}{7.842520in}}%
\pgfusepath{clip}%
\pgfsetbuttcap%
\pgfsetroundjoin%
\definecolor{currentfill}{rgb}{0.277018,0.050344,0.375715}%
\pgfsetfillcolor{currentfill}%
\pgfsetlinewidth{0.000000pt}%
\definecolor{currentstroke}{rgb}{0.226397,0.728888,0.462789}%
\pgfsetstrokecolor{currentstroke}%
\pgfsetdash{}{0pt}%
\pgfpathmoveto{\pgfqpoint{6.579539in}{3.406827in}}%
\pgfpathlineto{\pgfqpoint{6.649027in}{3.409472in}}%
\pgfpathlineto{\pgfqpoint{6.513170in}{3.455462in}}%
\pgfpathclose%
\pgfusepath{fill}%
\end{pgfscope}%
\begin{pgfscope}%
\pgfpathrectangle{\pgfqpoint{0.539299in}{0.078740in}}{\pgfqpoint{7.842520in}{7.842520in}}%
\pgfusepath{clip}%
\pgfsetbuttcap%
\pgfsetroundjoin%
\definecolor{currentfill}{rgb}{0.169646,0.456262,0.558030}%
\pgfsetfillcolor{currentfill}%
\pgfsetlinewidth{0.000000pt}%
\definecolor{currentstroke}{rgb}{0.232815,0.732247,0.459277}%
\pgfsetstrokecolor{currentstroke}%
\pgfsetdash{}{0pt}%
\pgfpathmoveto{\pgfqpoint{3.740970in}{4.779433in}}%
\pgfpathlineto{\pgfqpoint{3.660600in}{4.826856in}}%
\pgfpathlineto{\pgfqpoint{3.791614in}{4.877687in}}%
\pgfpathclose%
\pgfusepath{fill}%
\end{pgfscope}%
\begin{pgfscope}%
\pgfpathrectangle{\pgfqpoint{0.539299in}{0.078740in}}{\pgfqpoint{7.842520in}{7.842520in}}%
\pgfusepath{clip}%
\pgfsetbuttcap%
\pgfsetroundjoin%
\definecolor{currentfill}{rgb}{0.281446,0.084320,0.407414}%
\pgfsetfillcolor{currentfill}%
\pgfsetlinewidth{0.000000pt}%
\definecolor{currentstroke}{rgb}{0.239374,0.735588,0.455688}%
\pgfsetstrokecolor{currentstroke}%
\pgfsetdash{}{0pt}%
\pgfpathmoveto{\pgfqpoint{6.307514in}{3.501723in}}%
\pgfpathlineto{\pgfqpoint{6.171939in}{3.541802in}}%
\pgfpathlineto{\pgfqpoint{6.100728in}{3.537174in}}%
\pgfpathclose%
\pgfusepath{fill}%
\end{pgfscope}%
\begin{pgfscope}%
\pgfpathrectangle{\pgfqpoint{0.539299in}{0.078740in}}{\pgfqpoint{7.842520in}{7.842520in}}%
\pgfusepath{clip}%
\pgfsetbuttcap%
\pgfsetroundjoin%
\definecolor{currentfill}{rgb}{0.239346,0.300855,0.540844}%
\pgfsetfillcolor{currentfill}%
\pgfsetlinewidth{0.000000pt}%
\definecolor{currentstroke}{rgb}{0.246070,0.738910,0.452024}%
\pgfsetstrokecolor{currentstroke}%
\pgfsetdash{}{0pt}%
\pgfpathmoveto{\pgfqpoint{4.669832in}{4.230954in}}%
\pgfpathlineto{\pgfqpoint{4.746079in}{4.149433in}}%
\pgfpathlineto{\pgfqpoint{4.613151in}{4.272715in}}%
\pgfpathclose%
\pgfusepath{fill}%
\end{pgfscope}%
\begin{pgfscope}%
\pgfpathrectangle{\pgfqpoint{0.539299in}{0.078740in}}{\pgfqpoint{7.842520in}{7.842520in}}%
\pgfusepath{clip}%
\pgfsetbuttcap%
\pgfsetroundjoin%
\definecolor{currentfill}{rgb}{0.282327,0.094955,0.417331}%
\pgfsetfillcolor{currentfill}%
\pgfsetlinewidth{0.000000pt}%
\definecolor{currentstroke}{rgb}{0.252899,0.742211,0.448284}%
\pgfsetstrokecolor{currentstroke}%
\pgfsetdash{}{0pt}%
\pgfpathmoveto{\pgfqpoint{6.100728in}{3.537174in}}%
\pgfpathlineto{\pgfqpoint{6.171939in}{3.541802in}}%
\pgfpathlineto{\pgfqpoint{5.965125in}{3.573380in}}%
\pgfpathclose%
\pgfusepath{fill}%
\end{pgfscope}%
\begin{pgfscope}%
\pgfpathrectangle{\pgfqpoint{0.539299in}{0.078740in}}{\pgfqpoint{7.842520in}{7.842520in}}%
\pgfusepath{clip}%
\pgfsetbuttcap%
\pgfsetroundjoin%
\definecolor{currentfill}{rgb}{0.187231,0.414746,0.556547}%
\pgfsetfillcolor{currentfill}%
\pgfsetlinewidth{0.000000pt}%
\definecolor{currentstroke}{rgb}{0.259857,0.745492,0.444467}%
\pgfsetstrokecolor{currentstroke}%
\pgfsetdash{}{0pt}%
\pgfpathmoveto{\pgfqpoint{4.136438in}{4.716235in}}%
\pgfpathlineto{\pgfqpoint{4.269526in}{4.615818in}}%
\pgfpathlineto{\pgfqpoint{4.215129in}{4.619034in}}%
\pgfpathclose%
\pgfusepath{fill}%
\end{pgfscope}%
\begin{pgfscope}%
\pgfpathrectangle{\pgfqpoint{0.539299in}{0.078740in}}{\pgfqpoint{7.842520in}{7.842520in}}%
\pgfusepath{clip}%
\pgfsetbuttcap%
\pgfsetroundjoin%
\definecolor{currentfill}{rgb}{0.283091,0.110553,0.431554}%
\pgfsetfillcolor{currentfill}%
\pgfsetlinewidth{0.000000pt}%
\definecolor{currentstroke}{rgb}{0.266941,0.748751,0.440573}%
\pgfsetstrokecolor{currentstroke}%
\pgfsetdash{}{0pt}%
\pgfpathmoveto{\pgfqpoint{5.965125in}{3.573380in}}%
\pgfpathlineto{\pgfqpoint{5.829924in}{3.607475in}}%
\pgfpathlineto{\pgfqpoint{5.757427in}{3.601206in}}%
\pgfpathclose%
\pgfusepath{fill}%
\end{pgfscope}%
\begin{pgfscope}%
\pgfpathrectangle{\pgfqpoint{0.539299in}{0.078740in}}{\pgfqpoint{7.842520in}{7.842520in}}%
\pgfusepath{clip}%
\pgfsetbuttcap%
\pgfsetroundjoin%
\definecolor{currentfill}{rgb}{0.248629,0.278775,0.534556}%
\pgfsetfillcolor{currentfill}%
\pgfsetlinewidth{0.000000pt}%
\definecolor{currentstroke}{rgb}{0.274149,0.751988,0.436601}%
\pgfsetstrokecolor{currentstroke}%
\pgfsetdash{}{0pt}%
\pgfpathmoveto{\pgfqpoint{4.879148in}{4.033520in}}%
\pgfpathlineto{\pgfqpoint{4.746079in}{4.149433in}}%
\pgfpathlineto{\pgfqpoint{4.669832in}{4.230954in}}%
\pgfpathclose%
\pgfusepath{fill}%
\end{pgfscope}%
\begin{pgfscope}%
\pgfpathrectangle{\pgfqpoint{0.539299in}{0.078740in}}{\pgfqpoint{7.842520in}{7.842520in}}%
\pgfusepath{clip}%
\pgfsetbuttcap%
\pgfsetroundjoin%
\definecolor{currentfill}{rgb}{0.283229,0.120777,0.440584}%
\pgfsetfillcolor{currentfill}%
\pgfsetlinewidth{0.000000pt}%
\definecolor{currentstroke}{rgb}{0.281477,0.755203,0.432552}%
\pgfsetstrokecolor{currentstroke}%
\pgfsetdash{}{0pt}%
\pgfpathmoveto{\pgfqpoint{5.757427in}{3.601206in}}%
\pgfpathlineto{\pgfqpoint{5.829924in}{3.607475in}}%
\pgfpathlineto{\pgfqpoint{5.622384in}{3.640649in}}%
\pgfpathclose%
\pgfusepath{fill}%
\end{pgfscope}%
\begin{pgfscope}%
\pgfpathrectangle{\pgfqpoint{0.539299in}{0.078740in}}{\pgfqpoint{7.842520in}{7.842520in}}%
\pgfusepath{clip}%
\pgfsetbuttcap%
\pgfsetroundjoin%
\definecolor{currentfill}{rgb}{0.171176,0.452530,0.557965}%
\pgfsetfillcolor{currentfill}%
\pgfsetlinewidth{0.000000pt}%
\definecolor{currentstroke}{rgb}{0.288921,0.758394,0.428426}%
\pgfsetstrokecolor{currentstroke}%
\pgfsetdash{}{0pt}%
\pgfpathmoveto{\pgfqpoint{3.579639in}{4.867750in}}%
\pgfpathlineto{\pgfqpoint{3.660600in}{4.826856in}}%
\pgfpathlineto{\pgfqpoint{3.531309in}{4.704589in}}%
\pgfpathclose%
\pgfusepath{fill}%
\end{pgfscope}%
\begin{pgfscope}%
\pgfpathrectangle{\pgfqpoint{0.539299in}{0.078740in}}{\pgfqpoint{7.842520in}{7.842520in}}%
\pgfusepath{clip}%
\pgfsetbuttcap%
\pgfsetroundjoin%
\definecolor{currentfill}{rgb}{0.278791,0.062145,0.386592}%
\pgfsetfillcolor{currentfill}%
\pgfsetlinewidth{0.000000pt}%
\definecolor{currentstroke}{rgb}{0.296479,0.761561,0.424223}%
\pgfsetstrokecolor{currentstroke}%
\pgfsetdash{}{0pt}%
\pgfpathmoveto{\pgfqpoint{6.513170in}{3.455462in}}%
\pgfpathlineto{\pgfqpoint{6.443391in}{3.456772in}}%
\pgfpathlineto{\pgfqpoint{6.579539in}{3.406827in}}%
\pgfpathclose%
\pgfusepath{fill}%
\end{pgfscope}%
\begin{pgfscope}%
\pgfpathrectangle{\pgfqpoint{0.539299in}{0.078740in}}{\pgfqpoint{7.842520in}{7.842520in}}%
\pgfusepath{clip}%
\pgfsetbuttcap%
\pgfsetroundjoin%
\definecolor{currentfill}{rgb}{0.174274,0.445044,0.557792}%
\pgfsetfillcolor{currentfill}%
\pgfsetlinewidth{0.000000pt}%
\definecolor{currentstroke}{rgb}{0.304148,0.764704,0.419943}%
\pgfsetstrokecolor{currentstroke}%
\pgfsetdash{}{0pt}%
\pgfpathmoveto{\pgfqpoint{3.531309in}{4.704589in}}%
\pgfpathlineto{\pgfqpoint{3.450467in}{4.721409in}}%
\pgfpathlineto{\pgfqpoint{3.579639in}{4.867750in}}%
\pgfpathclose%
\pgfusepath{fill}%
\end{pgfscope}%
\begin{pgfscope}%
\pgfpathrectangle{\pgfqpoint{0.539299in}{0.078740in}}{\pgfqpoint{7.842520in}{7.842520in}}%
\pgfusepath{clip}%
\pgfsetbuttcap%
\pgfsetroundjoin%
\definecolor{currentfill}{rgb}{0.168126,0.459988,0.558082}%
\pgfsetfillcolor{currentfill}%
\pgfsetlinewidth{0.000000pt}%
\definecolor{currentstroke}{rgb}{0.311925,0.767822,0.415586}%
\pgfsetstrokecolor{currentstroke}%
\pgfsetdash{}{0pt}%
\pgfpathmoveto{\pgfqpoint{4.003782in}{4.784962in}}%
\pgfpathlineto{\pgfqpoint{3.871838in}{4.810044in}}%
\pgfpathlineto{\pgfqpoint{3.791614in}{4.877687in}}%
\pgfpathclose%
\pgfusepath{fill}%
\end{pgfscope}%
\begin{pgfscope}%
\pgfpathrectangle{\pgfqpoint{0.539299in}{0.078740in}}{\pgfqpoint{7.842520in}{7.842520in}}%
\pgfusepath{clip}%
\pgfsetbuttcap%
\pgfsetroundjoin%
\definecolor{currentfill}{rgb}{0.276022,0.044167,0.370164}%
\pgfsetfillcolor{currentfill}%
\pgfsetlinewidth{0.000000pt}%
\definecolor{currentstroke}{rgb}{0.319809,0.770914,0.411152}%
\pgfsetstrokecolor{currentstroke}%
\pgfsetdash{}{0pt}%
\pgfpathmoveto{\pgfqpoint{6.785212in}{3.360948in}}%
\pgfpathlineto{\pgfqpoint{6.649027in}{3.409472in}}%
\pgfpathlineto{\pgfqpoint{6.579539in}{3.406827in}}%
\pgfpathclose%
\pgfusepath{fill}%
\end{pgfscope}%
\begin{pgfscope}%
\pgfpathrectangle{\pgfqpoint{0.539299in}{0.078740in}}{\pgfqpoint{7.842520in}{7.842520in}}%
\pgfusepath{clip}%
\pgfsetbuttcap%
\pgfsetroundjoin%
\definecolor{currentfill}{rgb}{0.258965,0.251537,0.524736}%
\pgfsetfillcolor{currentfill}%
\pgfsetlinewidth{0.000000pt}%
\definecolor{currentstroke}{rgb}{0.327796,0.773980,0.406640}%
\pgfsetstrokecolor{currentstroke}%
\pgfsetdash{}{0pt}%
\pgfpathmoveto{\pgfqpoint{4.803456in}{4.103048in}}%
\pgfpathlineto{\pgfqpoint{5.012423in}{3.928911in}}%
\pgfpathlineto{\pgfqpoint{4.879148in}{4.033520in}}%
\pgfpathclose%
\pgfusepath{fill}%
\end{pgfscope}%
\begin{pgfscope}%
\pgfpathrectangle{\pgfqpoint{0.539299in}{0.078740in}}{\pgfqpoint{7.842520in}{7.842520in}}%
\pgfusepath{clip}%
\pgfsetbuttcap%
\pgfsetroundjoin%
\definecolor{currentfill}{rgb}{0.201239,0.383670,0.554294}%
\pgfsetfillcolor{currentfill}%
\pgfsetlinewidth{0.000000pt}%
\definecolor{currentstroke}{rgb}{0.335885,0.777018,0.402049}%
\pgfsetstrokecolor{currentstroke}%
\pgfsetdash{}{0pt}%
\pgfpathmoveto{\pgfqpoint{4.480330in}{4.397499in}}%
\pgfpathlineto{\pgfqpoint{4.269526in}{4.615818in}}%
\pgfpathlineto{\pgfqpoint{4.402851in}{4.494989in}}%
\pgfpathclose%
\pgfusepath{fill}%
\end{pgfscope}%
\begin{pgfscope}%
\pgfpathrectangle{\pgfqpoint{0.539299in}{0.078740in}}{\pgfqpoint{7.842520in}{7.842520in}}%
\pgfusepath{clip}%
\pgfsetbuttcap%
\pgfsetroundjoin%
\definecolor{currentfill}{rgb}{0.185556,0.418570,0.556753}%
\pgfsetfillcolor{currentfill}%
\pgfsetlinewidth{0.000000pt}%
\definecolor{currentstroke}{rgb}{0.344074,0.780029,0.397381}%
\pgfsetstrokecolor{currentstroke}%
\pgfsetdash{}{0pt}%
\pgfpathmoveto{\pgfqpoint{3.369083in}{4.733483in}}%
\pgfpathlineto{\pgfqpoint{3.450467in}{4.721409in}}%
\pgfpathlineto{\pgfqpoint{3.242761in}{4.478380in}}%
\pgfpathclose%
\pgfusepath{fill}%
\end{pgfscope}%
\begin{pgfscope}%
\pgfpathrectangle{\pgfqpoint{0.539299in}{0.078740in}}{\pgfqpoint{7.842520in}{7.842520in}}%
\pgfusepath{clip}%
\pgfsetbuttcap%
\pgfsetroundjoin%
\definecolor{currentfill}{rgb}{0.210503,0.363727,0.552206}%
\pgfsetfillcolor{currentfill}%
\pgfsetlinewidth{0.000000pt}%
\definecolor{currentstroke}{rgb}{0.352360,0.783011,0.392636}%
\pgfsetstrokecolor{currentstroke}%
\pgfsetdash{}{0pt}%
\pgfpathmoveto{\pgfqpoint{4.480330in}{4.397499in}}%
\pgfpathlineto{\pgfqpoint{4.402851in}{4.494989in}}%
\pgfpathlineto{\pgfqpoint{4.536301in}{4.363851in}}%
\pgfpathclose%
\pgfusepath{fill}%
\end{pgfscope}%
\begin{pgfscope}%
\pgfpathrectangle{\pgfqpoint{0.539299in}{0.078740in}}{\pgfqpoint{7.842520in}{7.842520in}}%
\pgfusepath{clip}%
\pgfsetbuttcap%
\pgfsetroundjoin%
\definecolor{currentfill}{rgb}{0.281412,0.155834,0.469201}%
\pgfsetfillcolor{currentfill}%
\pgfsetlinewidth{0.000000pt}%
\definecolor{currentstroke}{rgb}{0.360741,0.785964,0.387814}%
\pgfsetstrokecolor{currentstroke}%
\pgfsetdash{}{0pt}%
\pgfpathmoveto{\pgfqpoint{5.414261in}{3.695329in}}%
\pgfpathlineto{\pgfqpoint{5.487835in}{3.685615in}}%
\pgfpathlineto{\pgfqpoint{5.279908in}{3.760172in}}%
\pgfpathclose%
\pgfusepath{fill}%
\end{pgfscope}%
\begin{pgfscope}%
\pgfpathrectangle{\pgfqpoint{0.539299in}{0.078740in}}{\pgfqpoint{7.842520in}{7.842520in}}%
\pgfusepath{clip}%
\pgfsetbuttcap%
\pgfsetroundjoin%
\definecolor{currentfill}{rgb}{0.282623,0.140926,0.457517}%
\pgfsetfillcolor{currentfill}%
\pgfsetlinewidth{0.000000pt}%
\definecolor{currentstroke}{rgb}{0.369214,0.788888,0.382914}%
\pgfsetstrokecolor{currentstroke}%
\pgfsetdash{}{0pt}%
\pgfpathmoveto{\pgfqpoint{5.487835in}{3.685615in}}%
\pgfpathlineto{\pgfqpoint{5.549086in}{3.640953in}}%
\pgfpathlineto{\pgfqpoint{5.622384in}{3.640649in}}%
\pgfpathclose%
\pgfusepath{fill}%
\end{pgfscope}%
\begin{pgfscope}%
\pgfpathrectangle{\pgfqpoint{0.539299in}{0.078740in}}{\pgfqpoint{7.842520in}{7.842520in}}%
\pgfusepath{clip}%
\pgfsetbuttcap%
\pgfsetroundjoin%
\definecolor{currentfill}{rgb}{0.225863,0.330805,0.547314}%
\pgfsetfillcolor{currentfill}%
\pgfsetlinewidth{0.000000pt}%
\definecolor{currentstroke}{rgb}{0.377779,0.791781,0.377939}%
\pgfsetstrokecolor{currentstroke}%
\pgfsetdash{}{0pt}%
\pgfpathmoveto{\pgfqpoint{4.669832in}{4.230954in}}%
\pgfpathlineto{\pgfqpoint{4.613151in}{4.272715in}}%
\pgfpathlineto{\pgfqpoint{4.536301in}{4.363851in}}%
\pgfpathclose%
\pgfusepath{fill}%
\end{pgfscope}%
\begin{pgfscope}%
\pgfpathrectangle{\pgfqpoint{0.539299in}{0.078740in}}{\pgfqpoint{7.842520in}{7.842520in}}%
\pgfusepath{clip}%
\pgfsetbuttcap%
\pgfsetroundjoin%
\definecolor{currentfill}{rgb}{0.267968,0.223549,0.512008}%
\pgfsetfillcolor{currentfill}%
\pgfsetlinewidth{0.000000pt}%
\definecolor{currentstroke}{rgb}{0.386433,0.794644,0.372886}%
\pgfsetstrokecolor{currentstroke}%
\pgfsetdash{}{0pt}%
\pgfpathmoveto{\pgfqpoint{5.012423in}{3.928911in}}%
\pgfpathlineto{\pgfqpoint{4.937227in}{3.984970in}}%
\pgfpathlineto{\pgfqpoint{5.145984in}{3.837674in}}%
\pgfpathclose%
\pgfusepath{fill}%
\end{pgfscope}%
\begin{pgfscope}%
\pgfpathrectangle{\pgfqpoint{0.539299in}{0.078740in}}{\pgfqpoint{7.842520in}{7.842520in}}%
\pgfusepath{clip}%
\pgfsetbuttcap%
\pgfsetroundjoin%
\definecolor{currentfill}{rgb}{0.282327,0.094955,0.417331}%
\pgfsetfillcolor{currentfill}%
\pgfsetlinewidth{0.000000pt}%
\definecolor{currentstroke}{rgb}{0.395174,0.797475,0.367757}%
\pgfsetstrokecolor{currentstroke}%
\pgfsetdash{}{0pt}%
\pgfpathmoveto{\pgfqpoint{6.100728in}{3.537174in}}%
\pgfpathlineto{\pgfqpoint{6.236676in}{3.496772in}}%
\pgfpathlineto{\pgfqpoint{6.307514in}{3.501723in}}%
\pgfpathclose%
\pgfusepath{fill}%
\end{pgfscope}%
\begin{pgfscope}%
\pgfpathrectangle{\pgfqpoint{0.539299in}{0.078740in}}{\pgfqpoint{7.842520in}{7.842520in}}%
\pgfusepath{clip}%
\pgfsetbuttcap%
\pgfsetroundjoin%
\definecolor{currentfill}{rgb}{0.221989,0.339161,0.548752}%
\pgfsetfillcolor{currentfill}%
\pgfsetlinewidth{0.000000pt}%
\definecolor{currentstroke}{rgb}{0.404001,0.800275,0.362552}%
\pgfsetstrokecolor{currentstroke}%
\pgfsetdash{}{0pt}%
\pgfpathmoveto{\pgfqpoint{2.957481in}{4.048102in}}%
\pgfpathlineto{\pgfqpoint{3.079268in}{4.442627in}}%
\pgfpathlineto{\pgfqpoint{3.161266in}{4.461876in}}%
\pgfpathclose%
\pgfusepath{fill}%
\end{pgfscope}%
\begin{pgfscope}%
\pgfpathrectangle{\pgfqpoint{0.539299in}{0.078740in}}{\pgfqpoint{7.842520in}{7.842520in}}%
\pgfusepath{clip}%
\pgfsetbuttcap%
\pgfsetroundjoin%
\definecolor{currentfill}{rgb}{0.281446,0.084320,0.407414}%
\pgfsetfillcolor{currentfill}%
\pgfsetlinewidth{0.000000pt}%
\definecolor{currentstroke}{rgb}{0.412913,0.803041,0.357269}%
\pgfsetstrokecolor{currentstroke}%
\pgfsetdash{}{0pt}%
\pgfpathmoveto{\pgfqpoint{6.372915in}{3.450815in}}%
\pgfpathlineto{\pgfqpoint{6.443391in}{3.456772in}}%
\pgfpathlineto{\pgfqpoint{6.307514in}{3.501723in}}%
\pgfpathclose%
\pgfusepath{fill}%
\end{pgfscope}%
\begin{pgfscope}%
\pgfpathrectangle{\pgfqpoint{0.539299in}{0.078740in}}{\pgfqpoint{7.842520in}{7.842520in}}%
\pgfusepath{clip}%
\pgfsetbuttcap%
\pgfsetroundjoin%
\definecolor{currentfill}{rgb}{0.244972,0.287675,0.537260}%
\pgfsetfillcolor{currentfill}%
\pgfsetlinewidth{0.000000pt}%
\definecolor{currentstroke}{rgb}{0.421908,0.805774,0.351910}%
\pgfsetstrokecolor{currentstroke}%
\pgfsetdash{}{0pt}%
\pgfpathmoveto{\pgfqpoint{4.669832in}{4.230954in}}%
\pgfpathlineto{\pgfqpoint{4.803456in}{4.103048in}}%
\pgfpathlineto{\pgfqpoint{4.879148in}{4.033520in}}%
\pgfpathclose%
\pgfusepath{fill}%
\end{pgfscope}%
\begin{pgfscope}%
\pgfpathrectangle{\pgfqpoint{0.539299in}{0.078740in}}{\pgfqpoint{7.842520in}{7.842520in}}%
\pgfusepath{clip}%
\pgfsetbuttcap%
\pgfsetroundjoin%
\definecolor{currentfill}{rgb}{0.283229,0.120777,0.440584}%
\pgfsetfillcolor{currentfill}%
\pgfsetlinewidth{0.000000pt}%
\definecolor{currentstroke}{rgb}{0.430983,0.808473,0.346476}%
\pgfsetstrokecolor{currentstroke}%
\pgfsetdash{}{0pt}%
\pgfpathmoveto{\pgfqpoint{5.757427in}{3.601206in}}%
\pgfpathlineto{\pgfqpoint{5.892938in}{3.564025in}}%
\pgfpathlineto{\pgfqpoint{5.965125in}{3.573380in}}%
\pgfpathclose%
\pgfusepath{fill}%
\end{pgfscope}%
\begin{pgfscope}%
\pgfpathrectangle{\pgfqpoint{0.539299in}{0.078740in}}{\pgfqpoint{7.842520in}{7.842520in}}%
\pgfusepath{clip}%
\pgfsetbuttcap%
\pgfsetroundjoin%
\definecolor{currentfill}{rgb}{0.281887,0.150881,0.465405}%
\pgfsetfillcolor{currentfill}%
\pgfsetlinewidth{0.000000pt}%
\definecolor{currentstroke}{rgb}{0.440137,0.811138,0.340967}%
\pgfsetstrokecolor{currentstroke}%
\pgfsetdash{}{0pt}%
\pgfpathmoveto{\pgfqpoint{5.414261in}{3.695329in}}%
\pgfpathlineto{\pgfqpoint{5.549086in}{3.640953in}}%
\pgfpathlineto{\pgfqpoint{5.487835in}{3.685615in}}%
\pgfpathclose%
\pgfusepath{fill}%
\end{pgfscope}%
\begin{pgfscope}%
\pgfpathrectangle{\pgfqpoint{0.539299in}{0.078740in}}{\pgfqpoint{7.842520in}{7.842520in}}%
\pgfusepath{clip}%
\pgfsetbuttcap%
\pgfsetroundjoin%
\definecolor{currentfill}{rgb}{0.274952,0.037752,0.364543}%
\pgfsetfillcolor{currentfill}%
\pgfsetlinewidth{0.000000pt}%
\definecolor{currentstroke}{rgb}{0.449368,0.813768,0.335384}%
\pgfsetstrokecolor{currentstroke}%
\pgfsetdash{}{0pt}%
\pgfpathmoveto{\pgfqpoint{6.921738in}{3.310623in}}%
\pgfpathlineto{\pgfqpoint{6.785212in}{3.360948in}}%
\pgfpathlineto{\pgfqpoint{6.715949in}{3.352378in}}%
\pgfpathclose%
\pgfusepath{fill}%
\end{pgfscope}%
\begin{pgfscope}%
\pgfpathrectangle{\pgfqpoint{0.539299in}{0.078740in}}{\pgfqpoint{7.842520in}{7.842520in}}%
\pgfusepath{clip}%
\pgfsetbuttcap%
\pgfsetroundjoin%
\definecolor{currentfill}{rgb}{0.277941,0.056324,0.381191}%
\pgfsetfillcolor{currentfill}%
\pgfsetlinewidth{0.000000pt}%
\definecolor{currentstroke}{rgb}{0.458674,0.816363,0.329727}%
\pgfsetstrokecolor{currentstroke}%
\pgfsetdash{}{0pt}%
\pgfpathmoveto{\pgfqpoint{6.715949in}{3.352378in}}%
\pgfpathlineto{\pgfqpoint{6.785212in}{3.360948in}}%
\pgfpathlineto{\pgfqpoint{6.579539in}{3.406827in}}%
\pgfpathclose%
\pgfusepath{fill}%
\end{pgfscope}%
\begin{pgfscope}%
\pgfpathrectangle{\pgfqpoint{0.539299in}{0.078740in}}{\pgfqpoint{7.842520in}{7.842520in}}%
\pgfusepath{clip}%
\pgfsetbuttcap%
\pgfsetroundjoin%
\definecolor{currentfill}{rgb}{0.283197,0.115680,0.436115}%
\pgfsetfillcolor{currentfill}%
\pgfsetlinewidth{0.000000pt}%
\definecolor{currentstroke}{rgb}{0.468053,0.818921,0.323998}%
\pgfsetstrokecolor{currentstroke}%
\pgfsetdash{}{0pt}%
\pgfpathmoveto{\pgfqpoint{6.100728in}{3.537174in}}%
\pgfpathlineto{\pgfqpoint{5.965125in}{3.573380in}}%
\pgfpathlineto{\pgfqpoint{6.028872in}{3.526024in}}%
\pgfpathclose%
\pgfusepath{fill}%
\end{pgfscope}%
\begin{pgfscope}%
\pgfpathrectangle{\pgfqpoint{0.539299in}{0.078740in}}{\pgfqpoint{7.842520in}{7.842520in}}%
\pgfusepath{clip}%
\pgfsetbuttcap%
\pgfsetroundjoin%
\definecolor{currentfill}{rgb}{0.257322,0.256130,0.526563}%
\pgfsetfillcolor{currentfill}%
\pgfsetlinewidth{0.000000pt}%
\definecolor{currentstroke}{rgb}{0.477504,0.821444,0.318195}%
\pgfsetstrokecolor{currentstroke}%
\pgfsetdash{}{0pt}%
\pgfpathmoveto{\pgfqpoint{4.803456in}{4.103048in}}%
\pgfpathlineto{\pgfqpoint{4.937227in}{3.984970in}}%
\pgfpathlineto{\pgfqpoint{5.012423in}{3.928911in}}%
\pgfpathclose%
\pgfusepath{fill}%
\end{pgfscope}%
\begin{pgfscope}%
\pgfpathrectangle{\pgfqpoint{0.539299in}{0.078740in}}{\pgfqpoint{7.842520in}{7.842520in}}%
\pgfusepath{clip}%
\pgfsetbuttcap%
\pgfsetroundjoin%
\definecolor{currentfill}{rgb}{0.239346,0.300855,0.540844}%
\pgfsetfillcolor{currentfill}%
\pgfsetlinewidth{0.000000pt}%
\definecolor{currentstroke}{rgb}{0.487026,0.823929,0.312321}%
\pgfsetstrokecolor{currentstroke}%
\pgfsetdash{}{0pt}%
\pgfpathmoveto{\pgfqpoint{2.875698in}{4.003213in}}%
\pgfpathlineto{\pgfqpoint{2.996771in}{4.420700in}}%
\pgfpathlineto{\pgfqpoint{2.957481in}{4.048102in}}%
\pgfpathclose%
\pgfusepath{fill}%
\end{pgfscope}%
\begin{pgfscope}%
\pgfpathrectangle{\pgfqpoint{0.539299in}{0.078740in}}{\pgfqpoint{7.842520in}{7.842520in}}%
\pgfusepath{clip}%
\pgfsetbuttcap%
\pgfsetroundjoin%
\definecolor{currentfill}{rgb}{0.275191,0.194905,0.496005}%
\pgfsetfillcolor{currentfill}%
\pgfsetlinewidth{0.000000pt}%
\definecolor{currentstroke}{rgb}{0.496615,0.826376,0.306377}%
\pgfsetstrokecolor{currentstroke}%
\pgfsetdash{}{0pt}%
\pgfpathmoveto{\pgfqpoint{5.205512in}{3.788278in}}%
\pgfpathlineto{\pgfqpoint{5.279908in}{3.760172in}}%
\pgfpathlineto{\pgfqpoint{5.145984in}{3.837674in}}%
\pgfpathclose%
\pgfusepath{fill}%
\end{pgfscope}%
\begin{pgfscope}%
\pgfpathrectangle{\pgfqpoint{0.539299in}{0.078740in}}{\pgfqpoint{7.842520in}{7.842520in}}%
\pgfusepath{clip}%
\pgfsetbuttcap%
\pgfsetroundjoin%
\definecolor{currentfill}{rgb}{0.160665,0.478540,0.558115}%
\pgfsetfillcolor{currentfill}%
\pgfsetlinewidth{0.000000pt}%
\definecolor{currentstroke}{rgb}{0.506271,0.828786,0.300362}%
\pgfsetstrokecolor{currentstroke}%
\pgfsetdash{}{0pt}%
\pgfpathmoveto{\pgfqpoint{3.791614in}{4.877687in}}%
\pgfpathlineto{\pgfqpoint{3.660600in}{4.826856in}}%
\pgfpathlineto{\pgfqpoint{3.579639in}{4.867750in}}%
\pgfpathclose%
\pgfusepath{fill}%
\end{pgfscope}%
\begin{pgfscope}%
\pgfpathrectangle{\pgfqpoint{0.539299in}{0.078740in}}{\pgfqpoint{7.842520in}{7.842520in}}%
\pgfusepath{clip}%
\pgfsetbuttcap%
\pgfsetroundjoin%
\definecolor{currentfill}{rgb}{0.282327,0.094955,0.417331}%
\pgfsetfillcolor{currentfill}%
\pgfsetlinewidth{0.000000pt}%
\definecolor{currentstroke}{rgb}{0.515992,0.831158,0.294279}%
\pgfsetstrokecolor{currentstroke}%
\pgfsetdash{}{0pt}%
\pgfpathmoveto{\pgfqpoint{6.307514in}{3.501723in}}%
\pgfpathlineto{\pgfqpoint{6.236676in}{3.496772in}}%
\pgfpathlineto{\pgfqpoint{6.372915in}{3.450815in}}%
\pgfpathclose%
\pgfusepath{fill}%
\end{pgfscope}%
\begin{pgfscope}%
\pgfpathrectangle{\pgfqpoint{0.539299in}{0.078740in}}{\pgfqpoint{7.842520in}{7.842520in}}%
\pgfusepath{clip}%
\pgfsetbuttcap%
\pgfsetroundjoin%
\definecolor{currentfill}{rgb}{0.169646,0.456262,0.558030}%
\pgfsetfillcolor{currentfill}%
\pgfsetlinewidth{0.000000pt}%
\definecolor{currentstroke}{rgb}{0.525776,0.833491,0.288127}%
\pgfsetstrokecolor{currentstroke}%
\pgfsetdash{}{0pt}%
\pgfpathmoveto{\pgfqpoint{4.003782in}{4.784962in}}%
\pgfpathlineto{\pgfqpoint{4.057001in}{4.809605in}}%
\pgfpathlineto{\pgfqpoint{4.136438in}{4.716235in}}%
\pgfpathclose%
\pgfusepath{fill}%
\end{pgfscope}%
\begin{pgfscope}%
\pgfpathrectangle{\pgfqpoint{0.539299in}{0.078740in}}{\pgfqpoint{7.842520in}{7.842520in}}%
\pgfusepath{clip}%
\pgfsetbuttcap%
\pgfsetroundjoin%
\definecolor{currentfill}{rgb}{0.282884,0.135920,0.453427}%
\pgfsetfillcolor{currentfill}%
\pgfsetlinewidth{0.000000pt}%
\definecolor{currentstroke}{rgb}{0.535621,0.835785,0.281908}%
\pgfsetstrokecolor{currentstroke}%
\pgfsetdash{}{0pt}%
\pgfpathmoveto{\pgfqpoint{5.757427in}{3.601206in}}%
\pgfpathlineto{\pgfqpoint{5.622384in}{3.640649in}}%
\pgfpathlineto{\pgfqpoint{5.684398in}{3.594113in}}%
\pgfpathclose%
\pgfusepath{fill}%
\end{pgfscope}%
\begin{pgfscope}%
\pgfpathrectangle{\pgfqpoint{0.539299in}{0.078740in}}{\pgfqpoint{7.842520in}{7.842520in}}%
\pgfusepath{clip}%
\pgfsetbuttcap%
\pgfsetroundjoin%
\definecolor{currentfill}{rgb}{0.162142,0.474838,0.558140}%
\pgfsetfillcolor{currentfill}%
\pgfsetlinewidth{0.000000pt}%
\definecolor{currentstroke}{rgb}{0.545524,0.838039,0.275626}%
\pgfsetstrokecolor{currentstroke}%
\pgfsetdash{}{0pt}%
\pgfpathmoveto{\pgfqpoint{3.923882in}{4.868004in}}%
\pgfpathlineto{\pgfqpoint{4.003782in}{4.784962in}}%
\pgfpathlineto{\pgfqpoint{3.791614in}{4.877687in}}%
\pgfpathclose%
\pgfusepath{fill}%
\end{pgfscope}%
\begin{pgfscope}%
\pgfpathrectangle{\pgfqpoint{0.539299in}{0.078740in}}{\pgfqpoint{7.842520in}{7.842520in}}%
\pgfusepath{clip}%
\pgfsetbuttcap%
\pgfsetroundjoin%
\definecolor{currentfill}{rgb}{0.190631,0.407061,0.556089}%
\pgfsetfillcolor{currentfill}%
\pgfsetlinewidth{0.000000pt}%
\definecolor{currentstroke}{rgb}{0.555484,0.840254,0.269281}%
\pgfsetstrokecolor{currentstroke}%
\pgfsetdash{}{0pt}%
\pgfpathmoveto{\pgfqpoint{3.242761in}{4.478380in}}%
\pgfpathlineto{\pgfqpoint{3.161266in}{4.461876in}}%
\pgfpathlineto{\pgfqpoint{3.287152in}{4.741607in}}%
\pgfpathclose%
\pgfusepath{fill}%
\end{pgfscope}%
\begin{pgfscope}%
\pgfpathrectangle{\pgfqpoint{0.539299in}{0.078740in}}{\pgfqpoint{7.842520in}{7.842520in}}%
\pgfusepath{clip}%
\pgfsetbuttcap%
\pgfsetroundjoin%
\definecolor{currentfill}{rgb}{0.281446,0.084320,0.407414}%
\pgfsetfillcolor{currentfill}%
\pgfsetlinewidth{0.000000pt}%
\definecolor{currentstroke}{rgb}{0.565498,0.842430,0.262877}%
\pgfsetstrokecolor{currentstroke}%
\pgfsetdash{}{0pt}%
\pgfpathmoveto{\pgfqpoint{6.579539in}{3.406827in}}%
\pgfpathlineto{\pgfqpoint{6.443391in}{3.456772in}}%
\pgfpathlineto{\pgfqpoint{6.372915in}{3.450815in}}%
\pgfpathclose%
\pgfusepath{fill}%
\end{pgfscope}%
\begin{pgfscope}%
\pgfpathrectangle{\pgfqpoint{0.539299in}{0.078740in}}{\pgfqpoint{7.842520in}{7.842520in}}%
\pgfusepath{clip}%
\pgfsetbuttcap%
\pgfsetroundjoin%
\definecolor{currentfill}{rgb}{0.283229,0.120777,0.440584}%
\pgfsetfillcolor{currentfill}%
\pgfsetlinewidth{0.000000pt}%
\definecolor{currentstroke}{rgb}{0.575563,0.844566,0.256415}%
\pgfsetstrokecolor{currentstroke}%
\pgfsetdash{}{0pt}%
\pgfpathmoveto{\pgfqpoint{6.028872in}{3.526024in}}%
\pgfpathlineto{\pgfqpoint{5.965125in}{3.573380in}}%
\pgfpathlineto{\pgfqpoint{5.892938in}{3.564025in}}%
\pgfpathclose%
\pgfusepath{fill}%
\end{pgfscope}%
\begin{pgfscope}%
\pgfpathrectangle{\pgfqpoint{0.539299in}{0.078740in}}{\pgfqpoint{7.842520in}{7.842520in}}%
\pgfusepath{clip}%
\pgfsetbuttcap%
\pgfsetroundjoin%
\definecolor{currentfill}{rgb}{0.265145,0.232956,0.516599}%
\pgfsetfillcolor{currentfill}%
\pgfsetlinewidth{0.000000pt}%
\definecolor{currentstroke}{rgb}{0.585678,0.846661,0.249897}%
\pgfsetstrokecolor{currentstroke}%
\pgfsetdash{}{0pt}%
\pgfpathmoveto{\pgfqpoint{5.145984in}{3.837674in}}%
\pgfpathlineto{\pgfqpoint{4.937227in}{3.984970in}}%
\pgfpathlineto{\pgfqpoint{5.071218in}{3.879659in}}%
\pgfpathclose%
\pgfusepath{fill}%
\end{pgfscope}%
\begin{pgfscope}%
\pgfpathrectangle{\pgfqpoint{0.539299in}{0.078740in}}{\pgfqpoint{7.842520in}{7.842520in}}%
\pgfusepath{clip}%
\pgfsetbuttcap%
\pgfsetroundjoin%
\definecolor{currentfill}{rgb}{0.172719,0.448791,0.557885}%
\pgfsetfillcolor{currentfill}%
\pgfsetlinewidth{0.000000pt}%
\definecolor{currentstroke}{rgb}{0.595839,0.848717,0.243329}%
\pgfsetstrokecolor{currentstroke}%
\pgfsetdash{}{0pt}%
\pgfpathmoveto{\pgfqpoint{4.136438in}{4.716235in}}%
\pgfpathlineto{\pgfqpoint{4.057001in}{4.809605in}}%
\pgfpathlineto{\pgfqpoint{4.269526in}{4.615818in}}%
\pgfpathclose%
\pgfusepath{fill}%
\end{pgfscope}%
\begin{pgfscope}%
\pgfpathrectangle{\pgfqpoint{0.539299in}{0.078740in}}{\pgfqpoint{7.842520in}{7.842520in}}%
\pgfusepath{clip}%
\pgfsetbuttcap%
\pgfsetroundjoin%
\definecolor{currentfill}{rgb}{0.282290,0.145912,0.461510}%
\pgfsetfillcolor{currentfill}%
\pgfsetlinewidth{0.000000pt}%
\definecolor{currentstroke}{rgb}{0.606045,0.850733,0.236712}%
\pgfsetstrokecolor{currentstroke}%
\pgfsetdash{}{0pt}%
\pgfpathmoveto{\pgfqpoint{5.622384in}{3.640649in}}%
\pgfpathlineto{\pgfqpoint{5.549086in}{3.640953in}}%
\pgfpathlineto{\pgfqpoint{5.684398in}{3.594113in}}%
\pgfpathclose%
\pgfusepath{fill}%
\end{pgfscope}%
\begin{pgfscope}%
\pgfpathrectangle{\pgfqpoint{0.539299in}{0.078740in}}{\pgfqpoint{7.842520in}{7.842520in}}%
\pgfusepath{clip}%
\pgfsetbuttcap%
\pgfsetroundjoin%
\definecolor{currentfill}{rgb}{0.179019,0.433756,0.557430}%
\pgfsetfillcolor{currentfill}%
\pgfsetlinewidth{0.000000pt}%
\definecolor{currentstroke}{rgb}{0.616293,0.852709,0.230052}%
\pgfsetstrokecolor{currentstroke}%
\pgfsetdash{}{0pt}%
\pgfpathmoveto{\pgfqpoint{3.287152in}{4.741607in}}%
\pgfpathlineto{\pgfqpoint{3.369083in}{4.733483in}}%
\pgfpathlineto{\pgfqpoint{3.242761in}{4.478380in}}%
\pgfpathclose%
\pgfusepath{fill}%
\end{pgfscope}%
\begin{pgfscope}%
\pgfpathrectangle{\pgfqpoint{0.539299in}{0.078740in}}{\pgfqpoint{7.842520in}{7.842520in}}%
\pgfusepath{clip}%
\pgfsetbuttcap%
\pgfsetroundjoin%
\definecolor{currentfill}{rgb}{0.270595,0.214069,0.507052}%
\pgfsetfillcolor{currentfill}%
\pgfsetlinewidth{0.000000pt}%
\definecolor{currentstroke}{rgb}{0.626579,0.854645,0.223353}%
\pgfsetstrokecolor{currentstroke}%
\pgfsetdash{}{0pt}%
\pgfpathmoveto{\pgfqpoint{5.071218in}{3.879659in}}%
\pgfpathlineto{\pgfqpoint{5.205512in}{3.788278in}}%
\pgfpathlineto{\pgfqpoint{5.145984in}{3.837674in}}%
\pgfpathclose%
\pgfusepath{fill}%
\end{pgfscope}%
\begin{pgfscope}%
\pgfpathrectangle{\pgfqpoint{0.539299in}{0.078740in}}{\pgfqpoint{7.842520in}{7.842520in}}%
\pgfusepath{clip}%
\pgfsetbuttcap%
\pgfsetroundjoin%
\definecolor{currentfill}{rgb}{0.278012,0.180367,0.486697}%
\pgfsetfillcolor{currentfill}%
\pgfsetlinewidth{0.000000pt}%
\definecolor{currentstroke}{rgb}{0.636902,0.856542,0.216620}%
\pgfsetstrokecolor{currentstroke}%
\pgfsetdash{}{0pt}%
\pgfpathmoveto{\pgfqpoint{5.414261in}{3.695329in}}%
\pgfpathlineto{\pgfqpoint{5.279908in}{3.760172in}}%
\pgfpathlineto{\pgfqpoint{5.340183in}{3.710444in}}%
\pgfpathclose%
\pgfusepath{fill}%
\end{pgfscope}%
\begin{pgfscope}%
\pgfpathrectangle{\pgfqpoint{0.539299in}{0.078740in}}{\pgfqpoint{7.842520in}{7.842520in}}%
\pgfusepath{clip}%
\pgfsetbuttcap%
\pgfsetroundjoin%
\definecolor{currentfill}{rgb}{0.276022,0.044167,0.370164}%
\pgfsetfillcolor{currentfill}%
\pgfsetlinewidth{0.000000pt}%
\definecolor{currentstroke}{rgb}{0.647257,0.858400,0.209861}%
\pgfsetstrokecolor{currentstroke}%
\pgfsetdash{}{0pt}%
\pgfpathmoveto{\pgfqpoint{6.715949in}{3.352378in}}%
\pgfpathlineto{\pgfqpoint{6.852629in}{3.294416in}}%
\pgfpathlineto{\pgfqpoint{6.921738in}{3.310623in}}%
\pgfpathclose%
\pgfusepath{fill}%
\end{pgfscope}%
\begin{pgfscope}%
\pgfpathrectangle{\pgfqpoint{0.539299in}{0.078740in}}{\pgfqpoint{7.842520in}{7.842520in}}%
\pgfusepath{clip}%
\pgfsetbuttcap%
\pgfsetroundjoin%
\definecolor{currentfill}{rgb}{0.160665,0.478540,0.558115}%
\pgfsetfillcolor{currentfill}%
\pgfsetlinewidth{0.000000pt}%
\definecolor{currentstroke}{rgb}{0.657642,0.860219,0.203082}%
\pgfsetstrokecolor{currentstroke}%
\pgfsetdash{}{0pt}%
\pgfpathmoveto{\pgfqpoint{3.923882in}{4.868004in}}%
\pgfpathlineto{\pgfqpoint{4.057001in}{4.809605in}}%
\pgfpathlineto{\pgfqpoint{4.003782in}{4.784962in}}%
\pgfpathclose%
\pgfusepath{fill}%
\end{pgfscope}%
\begin{pgfscope}%
\pgfpathrectangle{\pgfqpoint{0.539299in}{0.078740in}}{\pgfqpoint{7.842520in}{7.842520in}}%
\pgfusepath{clip}%
\pgfsetbuttcap%
\pgfsetroundjoin%
\definecolor{currentfill}{rgb}{0.272594,0.025563,0.353093}%
\pgfsetfillcolor{currentfill}%
\pgfsetlinewidth{0.000000pt}%
\definecolor{currentstroke}{rgb}{0.668054,0.861999,0.196293}%
\pgfsetstrokecolor{currentstroke}%
\pgfsetdash{}{0pt}%
\pgfpathmoveto{\pgfqpoint{6.989604in}{3.234274in}}%
\pgfpathlineto{\pgfqpoint{7.058628in}{3.259387in}}%
\pgfpathlineto{\pgfqpoint{6.921738in}{3.310623in}}%
\pgfpathclose%
\pgfusepath{fill}%
\end{pgfscope}%
\begin{pgfscope}%
\pgfpathrectangle{\pgfqpoint{0.539299in}{0.078740in}}{\pgfqpoint{7.842520in}{7.842520in}}%
\pgfusepath{clip}%
\pgfsetbuttcap%
\pgfsetroundjoin%
\definecolor{currentfill}{rgb}{0.283197,0.115680,0.436115}%
\pgfsetfillcolor{currentfill}%
\pgfsetlinewidth{0.000000pt}%
\definecolor{currentstroke}{rgb}{0.678489,0.863742,0.189503}%
\pgfsetstrokecolor{currentstroke}%
\pgfsetdash{}{0pt}%
\pgfpathmoveto{\pgfqpoint{6.165167in}{3.484596in}}%
\pgfpathlineto{\pgfqpoint{6.236676in}{3.496772in}}%
\pgfpathlineto{\pgfqpoint{6.100728in}{3.537174in}}%
\pgfpathclose%
\pgfusepath{fill}%
\end{pgfscope}%
\begin{pgfscope}%
\pgfpathrectangle{\pgfqpoint{0.539299in}{0.078740in}}{\pgfqpoint{7.842520in}{7.842520in}}%
\pgfusepath{clip}%
\pgfsetbuttcap%
\pgfsetroundjoin%
\definecolor{currentfill}{rgb}{0.216210,0.351535,0.550627}%
\pgfsetfillcolor{currentfill}%
\pgfsetlinewidth{0.000000pt}%
\definecolor{currentstroke}{rgb}{0.688944,0.865448,0.182725}%
\pgfsetstrokecolor{currentstroke}%
\pgfsetdash{}{0pt}%
\pgfpathmoveto{\pgfqpoint{2.996771in}{4.420700in}}%
\pgfpathlineto{\pgfqpoint{3.079268in}{4.442627in}}%
\pgfpathlineto{\pgfqpoint{2.957481in}{4.048102in}}%
\pgfpathclose%
\pgfusepath{fill}%
\end{pgfscope}%
\begin{pgfscope}%
\pgfpathrectangle{\pgfqpoint{0.539299in}{0.078740in}}{\pgfqpoint{7.842520in}{7.842520in}}%
\pgfusepath{clip}%
\pgfsetbuttcap%
\pgfsetroundjoin%
\definecolor{currentfill}{rgb}{0.185556,0.418570,0.556753}%
\pgfsetfillcolor{currentfill}%
\pgfsetlinewidth{0.000000pt}%
\definecolor{currentstroke}{rgb}{0.699415,0.867117,0.175971}%
\pgfsetstrokecolor{currentstroke}%
\pgfsetdash{}{0pt}%
\pgfpathmoveto{\pgfqpoint{4.402851in}{4.494989in}}%
\pgfpathlineto{\pgfqpoint{4.269526in}{4.615818in}}%
\pgfpathlineto{\pgfqpoint{4.324600in}{4.594132in}}%
\pgfpathclose%
\pgfusepath{fill}%
\end{pgfscope}%
\begin{pgfscope}%
\pgfpathrectangle{\pgfqpoint{0.539299in}{0.078740in}}{\pgfqpoint{7.842520in}{7.842520in}}%
\pgfusepath{clip}%
\pgfsetbuttcap%
\pgfsetroundjoin%
\definecolor{currentfill}{rgb}{0.282884,0.135920,0.453427}%
\pgfsetfillcolor{currentfill}%
\pgfsetlinewidth{0.000000pt}%
\definecolor{currentstroke}{rgb}{0.709898,0.868751,0.169257}%
\pgfsetstrokecolor{currentstroke}%
\pgfsetdash{}{0pt}%
\pgfpathmoveto{\pgfqpoint{5.820187in}{3.551515in}}%
\pgfpathlineto{\pgfqpoint{5.892938in}{3.564025in}}%
\pgfpathlineto{\pgfqpoint{5.757427in}{3.601206in}}%
\pgfpathclose%
\pgfusepath{fill}%
\end{pgfscope}%
\begin{pgfscope}%
\pgfpathrectangle{\pgfqpoint{0.539299in}{0.078740in}}{\pgfqpoint{7.842520in}{7.842520in}}%
\pgfusepath{clip}%
\pgfsetbuttcap%
\pgfsetroundjoin%
\definecolor{currentfill}{rgb}{0.163625,0.471133,0.558148}%
\pgfsetfillcolor{currentfill}%
\pgfsetlinewidth{0.000000pt}%
\definecolor{currentstroke}{rgb}{0.720391,0.870350,0.162603}%
\pgfsetstrokecolor{currentstroke}%
\pgfsetdash{}{0pt}%
\pgfpathmoveto{\pgfqpoint{3.498080in}{4.903513in}}%
\pgfpathlineto{\pgfqpoint{3.450467in}{4.721409in}}%
\pgfpathlineto{\pgfqpoint{3.369083in}{4.733483in}}%
\pgfpathclose%
\pgfusepath{fill}%
\end{pgfscope}%
\begin{pgfscope}%
\pgfpathrectangle{\pgfqpoint{0.539299in}{0.078740in}}{\pgfqpoint{7.842520in}{7.842520in}}%
\pgfusepath{clip}%
\pgfsetbuttcap%
\pgfsetroundjoin%
\definecolor{currentfill}{rgb}{0.194100,0.399323,0.555565}%
\pgfsetfillcolor{currentfill}%
\pgfsetlinewidth{0.000000pt}%
\definecolor{currentstroke}{rgb}{0.730889,0.871916,0.156029}%
\pgfsetstrokecolor{currentstroke}%
\pgfsetdash{}{0pt}%
\pgfpathmoveto{\pgfqpoint{4.536301in}{4.363851in}}%
\pgfpathlineto{\pgfqpoint{4.402851in}{4.494989in}}%
\pgfpathlineto{\pgfqpoint{4.324600in}{4.594132in}}%
\pgfpathclose%
\pgfusepath{fill}%
\end{pgfscope}%
\begin{pgfscope}%
\pgfpathrectangle{\pgfqpoint{0.539299in}{0.078740in}}{\pgfqpoint{7.842520in}{7.842520in}}%
\pgfusepath{clip}%
\pgfsetbuttcap%
\pgfsetroundjoin%
\definecolor{currentfill}{rgb}{0.159194,0.482237,0.558073}%
\pgfsetfillcolor{currentfill}%
\pgfsetlinewidth{0.000000pt}%
\definecolor{currentstroke}{rgb}{0.741388,0.873449,0.149561}%
\pgfsetstrokecolor{currentstroke}%
\pgfsetdash{}{0pt}%
\pgfpathmoveto{\pgfqpoint{3.579639in}{4.867750in}}%
\pgfpathlineto{\pgfqpoint{3.450467in}{4.721409in}}%
\pgfpathlineto{\pgfqpoint{3.498080in}{4.903513in}}%
\pgfpathclose%
\pgfusepath{fill}%
\end{pgfscope}%
\begin{pgfscope}%
\pgfpathrectangle{\pgfqpoint{0.539299in}{0.078740in}}{\pgfqpoint{7.842520in}{7.842520in}}%
\pgfusepath{clip}%
\pgfsetbuttcap%
\pgfsetroundjoin%
\definecolor{currentfill}{rgb}{0.278826,0.175490,0.483397}%
\pgfsetfillcolor{currentfill}%
\pgfsetlinewidth{0.000000pt}%
\definecolor{currentstroke}{rgb}{0.751884,0.874951,0.143228}%
\pgfsetstrokecolor{currentstroke}%
\pgfsetdash{}{0pt}%
\pgfpathmoveto{\pgfqpoint{5.340183in}{3.710444in}}%
\pgfpathlineto{\pgfqpoint{5.549086in}{3.640953in}}%
\pgfpathlineto{\pgfqpoint{5.414261in}{3.695329in}}%
\pgfpathclose%
\pgfusepath{fill}%
\end{pgfscope}%
\begin{pgfscope}%
\pgfpathrectangle{\pgfqpoint{0.539299in}{0.078740in}}{\pgfqpoint{7.842520in}{7.842520in}}%
\pgfusepath{clip}%
\pgfsetbuttcap%
\pgfsetroundjoin%
\definecolor{currentfill}{rgb}{0.237441,0.305202,0.541921}%
\pgfsetfillcolor{currentfill}%
\pgfsetlinewidth{0.000000pt}%
\definecolor{currentstroke}{rgb}{0.762373,0.876424,0.137064}%
\pgfsetstrokecolor{currentstroke}%
\pgfsetdash{}{0pt}%
\pgfpathmoveto{\pgfqpoint{2.793452in}{3.956696in}}%
\pgfpathlineto{\pgfqpoint{2.996771in}{4.420700in}}%
\pgfpathlineto{\pgfqpoint{2.875698in}{4.003213in}}%
\pgfpathclose%
\pgfusepath{fill}%
\end{pgfscope}%
\begin{pgfscope}%
\pgfpathrectangle{\pgfqpoint{0.539299in}{0.078740in}}{\pgfqpoint{7.842520in}{7.842520in}}%
\pgfusepath{clip}%
\pgfsetbuttcap%
\pgfsetroundjoin%
\definecolor{currentfill}{rgb}{0.281924,0.089666,0.412415}%
\pgfsetfillcolor{currentfill}%
\pgfsetlinewidth{0.000000pt}%
\definecolor{currentstroke}{rgb}{0.772852,0.877868,0.131109}%
\pgfsetstrokecolor{currentstroke}%
\pgfsetdash{}{0pt}%
\pgfpathmoveto{\pgfqpoint{6.372915in}{3.450815in}}%
\pgfpathlineto{\pgfqpoint{6.509400in}{3.398734in}}%
\pgfpathlineto{\pgfqpoint{6.579539in}{3.406827in}}%
\pgfpathclose%
\pgfusepath{fill}%
\end{pgfscope}%
\begin{pgfscope}%
\pgfpathrectangle{\pgfqpoint{0.539299in}{0.078740in}}{\pgfqpoint{7.842520in}{7.842520in}}%
\pgfusepath{clip}%
\pgfsetbuttcap%
\pgfsetroundjoin%
\definecolor{currentfill}{rgb}{0.275191,0.194905,0.496005}%
\pgfsetfillcolor{currentfill}%
\pgfsetlinewidth{0.000000pt}%
\definecolor{currentstroke}{rgb}{0.783315,0.879285,0.125405}%
\pgfsetstrokecolor{currentstroke}%
\pgfsetdash{}{0pt}%
\pgfpathmoveto{\pgfqpoint{5.340183in}{3.710444in}}%
\pgfpathlineto{\pgfqpoint{5.279908in}{3.760172in}}%
\pgfpathlineto{\pgfqpoint{5.205512in}{3.788278in}}%
\pgfpathclose%
\pgfusepath{fill}%
\end{pgfscope}%
\begin{pgfscope}%
\pgfpathrectangle{\pgfqpoint{0.539299in}{0.078740in}}{\pgfqpoint{7.842520in}{7.842520in}}%
\pgfusepath{clip}%
\pgfsetbuttcap%
\pgfsetroundjoin%
\definecolor{currentfill}{rgb}{0.283229,0.120777,0.440584}%
\pgfsetfillcolor{currentfill}%
\pgfsetlinewidth{0.000000pt}%
\definecolor{currentstroke}{rgb}{0.793760,0.880678,0.120005}%
\pgfsetstrokecolor{currentstroke}%
\pgfsetdash{}{0pt}%
\pgfpathmoveto{\pgfqpoint{6.100728in}{3.537174in}}%
\pgfpathlineto{\pgfqpoint{6.028872in}{3.526024in}}%
\pgfpathlineto{\pgfqpoint{6.165167in}{3.484596in}}%
\pgfpathclose%
\pgfusepath{fill}%
\end{pgfscope}%
\begin{pgfscope}%
\pgfpathrectangle{\pgfqpoint{0.539299in}{0.078740in}}{\pgfqpoint{7.842520in}{7.842520in}}%
\pgfusepath{clip}%
\pgfsetbuttcap%
\pgfsetroundjoin%
\definecolor{currentfill}{rgb}{0.274952,0.037752,0.364543}%
\pgfsetfillcolor{currentfill}%
\pgfsetlinewidth{0.000000pt}%
\definecolor{currentstroke}{rgb}{0.804182,0.882046,0.114965}%
\pgfsetstrokecolor{currentstroke}%
\pgfsetdash{}{0pt}%
\pgfpathmoveto{\pgfqpoint{6.921738in}{3.310623in}}%
\pgfpathlineto{\pgfqpoint{6.852629in}{3.294416in}}%
\pgfpathlineto{\pgfqpoint{6.989604in}{3.234274in}}%
\pgfpathclose%
\pgfusepath{fill}%
\end{pgfscope}%
\begin{pgfscope}%
\pgfpathrectangle{\pgfqpoint{0.539299in}{0.078740in}}{\pgfqpoint{7.842520in}{7.842520in}}%
\pgfusepath{clip}%
\pgfsetbuttcap%
\pgfsetroundjoin%
\definecolor{currentfill}{rgb}{0.208623,0.367752,0.552675}%
\pgfsetfillcolor{currentfill}%
\pgfsetlinewidth{0.000000pt}%
\definecolor{currentstroke}{rgb}{0.814576,0.883393,0.110347}%
\pgfsetstrokecolor{currentstroke}%
\pgfsetdash{}{0pt}%
\pgfpathmoveto{\pgfqpoint{4.536301in}{4.363851in}}%
\pgfpathlineto{\pgfqpoint{4.458695in}{4.459156in}}%
\pgfpathlineto{\pgfqpoint{4.669832in}{4.230954in}}%
\pgfpathclose%
\pgfusepath{fill}%
\end{pgfscope}%
\begin{pgfscope}%
\pgfpathrectangle{\pgfqpoint{0.539299in}{0.078740in}}{\pgfqpoint{7.842520in}{7.842520in}}%
\pgfusepath{clip}%
\pgfsetbuttcap%
\pgfsetroundjoin%
\definecolor{currentfill}{rgb}{0.280267,0.073417,0.397163}%
\pgfsetfillcolor{currentfill}%
\pgfsetlinewidth{0.000000pt}%
\definecolor{currentstroke}{rgb}{0.824940,0.884720,0.106217}%
\pgfsetstrokecolor{currentstroke}%
\pgfsetdash{}{0pt}%
\pgfpathmoveto{\pgfqpoint{6.579539in}{3.406827in}}%
\pgfpathlineto{\pgfqpoint{6.646106in}{3.340736in}}%
\pgfpathlineto{\pgfqpoint{6.715949in}{3.352378in}}%
\pgfpathclose%
\pgfusepath{fill}%
\end{pgfscope}%
\begin{pgfscope}%
\pgfpathrectangle{\pgfqpoint{0.539299in}{0.078740in}}{\pgfqpoint{7.842520in}{7.842520in}}%
\pgfusepath{clip}%
\pgfsetbuttcap%
\pgfsetroundjoin%
\definecolor{currentfill}{rgb}{0.282290,0.145912,0.461510}%
\pgfsetfillcolor{currentfill}%
\pgfsetlinewidth{0.000000pt}%
\definecolor{currentstroke}{rgb}{0.835270,0.886029,0.102646}%
\pgfsetstrokecolor{currentstroke}%
\pgfsetdash{}{0pt}%
\pgfpathmoveto{\pgfqpoint{5.757427in}{3.601206in}}%
\pgfpathlineto{\pgfqpoint{5.684398in}{3.594113in}}%
\pgfpathlineto{\pgfqpoint{5.820187in}{3.551515in}}%
\pgfpathclose%
\pgfusepath{fill}%
\end{pgfscope}%
\begin{pgfscope}%
\pgfpathrectangle{\pgfqpoint{0.539299in}{0.078740in}}{\pgfqpoint{7.842520in}{7.842520in}}%
\pgfusepath{clip}%
\pgfsetbuttcap%
\pgfsetroundjoin%
\definecolor{currentfill}{rgb}{0.283091,0.110553,0.431554}%
\pgfsetfillcolor{currentfill}%
\pgfsetlinewidth{0.000000pt}%
\definecolor{currentstroke}{rgb}{0.845561,0.887322,0.099702}%
\pgfsetstrokecolor{currentstroke}%
\pgfsetdash{}{0pt}%
\pgfpathmoveto{\pgfqpoint{6.372915in}{3.450815in}}%
\pgfpathlineto{\pgfqpoint{6.236676in}{3.496772in}}%
\pgfpathlineto{\pgfqpoint{6.165167in}{3.484596in}}%
\pgfpathclose%
\pgfusepath{fill}%
\end{pgfscope}%
\begin{pgfscope}%
\pgfpathrectangle{\pgfqpoint{0.539299in}{0.078740in}}{\pgfqpoint{7.842520in}{7.842520in}}%
\pgfusepath{clip}%
\pgfsetbuttcap%
\pgfsetroundjoin%
\definecolor{currentfill}{rgb}{0.151918,0.500685,0.557587}%
\pgfsetfillcolor{currentfill}%
\pgfsetlinewidth{0.000000pt}%
\definecolor{currentstroke}{rgb}{0.855810,0.888601,0.097452}%
\pgfsetstrokecolor{currentstroke}%
\pgfsetdash{}{0pt}%
\pgfpathmoveto{\pgfqpoint{3.579639in}{4.867750in}}%
\pgfpathlineto{\pgfqpoint{3.710739in}{4.939290in}}%
\pgfpathlineto{\pgfqpoint{3.791614in}{4.877687in}}%
\pgfpathclose%
\pgfusepath{fill}%
\end{pgfscope}%
\begin{pgfscope}%
\pgfpathrectangle{\pgfqpoint{0.539299in}{0.078740in}}{\pgfqpoint{7.842520in}{7.842520in}}%
\pgfusepath{clip}%
\pgfsetbuttcap%
\pgfsetroundjoin%
\definecolor{currentfill}{rgb}{0.229739,0.322361,0.545706}%
\pgfsetfillcolor{currentfill}%
\pgfsetlinewidth{0.000000pt}%
\definecolor{currentstroke}{rgb}{0.866013,0.889868,0.095953}%
\pgfsetstrokecolor{currentstroke}%
\pgfsetdash{}{0pt}%
\pgfpathmoveto{\pgfqpoint{4.727082in}{4.180259in}}%
\pgfpathlineto{\pgfqpoint{4.803456in}{4.103048in}}%
\pgfpathlineto{\pgfqpoint{4.669832in}{4.230954in}}%
\pgfpathclose%
\pgfusepath{fill}%
\end{pgfscope}%
\begin{pgfscope}%
\pgfpathrectangle{\pgfqpoint{0.539299in}{0.078740in}}{\pgfqpoint{7.842520in}{7.842520in}}%
\pgfusepath{clip}%
\pgfsetbuttcap%
\pgfsetroundjoin%
\definecolor{currentfill}{rgb}{0.185556,0.418570,0.556753}%
\pgfsetfillcolor{currentfill}%
\pgfsetlinewidth{0.000000pt}%
\definecolor{currentstroke}{rgb}{0.876168,0.891125,0.095250}%
\pgfsetstrokecolor{currentstroke}%
\pgfsetdash{}{0pt}%
\pgfpathmoveto{\pgfqpoint{3.287152in}{4.741607in}}%
\pgfpathlineto{\pgfqpoint{3.161266in}{4.461876in}}%
\pgfpathlineto{\pgfqpoint{3.079268in}{4.442627in}}%
\pgfpathclose%
\pgfusepath{fill}%
\end{pgfscope}%
\begin{pgfscope}%
\pgfpathrectangle{\pgfqpoint{0.539299in}{0.078740in}}{\pgfqpoint{7.842520in}{7.842520in}}%
\pgfusepath{clip}%
\pgfsetbuttcap%
\pgfsetroundjoin%
\definecolor{currentfill}{rgb}{0.168126,0.459988,0.558082}%
\pgfsetfillcolor{currentfill}%
\pgfsetlinewidth{0.000000pt}%
\definecolor{currentstroke}{rgb}{0.886271,0.892374,0.095374}%
\pgfsetstrokecolor{currentstroke}%
\pgfsetdash{}{0pt}%
\pgfpathmoveto{\pgfqpoint{4.269526in}{4.615818in}}%
\pgfpathlineto{\pgfqpoint{4.057001in}{4.809605in}}%
\pgfpathlineto{\pgfqpoint{4.190652in}{4.714468in}}%
\pgfpathclose%
\pgfusepath{fill}%
\end{pgfscope}%
\begin{pgfscope}%
\pgfpathrectangle{\pgfqpoint{0.539299in}{0.078740in}}{\pgfqpoint{7.842520in}{7.842520in}}%
\pgfusepath{clip}%
\pgfsetbuttcap%
\pgfsetroundjoin%
\definecolor{currentfill}{rgb}{0.271305,0.019942,0.347269}%
\pgfsetfillcolor{currentfill}%
\pgfsetlinewidth{0.000000pt}%
\definecolor{currentstroke}{rgb}{0.896320,0.893616,0.096335}%
\pgfsetstrokecolor{currentstroke}%
\pgfsetdash{}{0pt}%
\pgfpathmoveto{\pgfqpoint{7.126919in}{3.173464in}}%
\pgfpathlineto{\pgfqpoint{7.195913in}{3.208174in}}%
\pgfpathlineto{\pgfqpoint{7.058628in}{3.259387in}}%
\pgfpathclose%
\pgfusepath{fill}%
\end{pgfscope}%
\begin{pgfscope}%
\pgfpathrectangle{\pgfqpoint{0.539299in}{0.078740in}}{\pgfqpoint{7.842520in}{7.842520in}}%
\pgfusepath{clip}%
\pgfsetbuttcap%
\pgfsetroundjoin%
\definecolor{currentfill}{rgb}{0.282623,0.140926,0.457517}%
\pgfsetfillcolor{currentfill}%
\pgfsetlinewidth{0.000000pt}%
\definecolor{currentstroke}{rgb}{0.906311,0.894855,0.098125}%
\pgfsetstrokecolor{currentstroke}%
\pgfsetdash{}{0pt}%
\pgfpathmoveto{\pgfqpoint{5.820187in}{3.551515in}}%
\pgfpathlineto{\pgfqpoint{6.028872in}{3.526024in}}%
\pgfpathlineto{\pgfqpoint{5.892938in}{3.564025in}}%
\pgfpathclose%
\pgfusepath{fill}%
\end{pgfscope}%
\begin{pgfscope}%
\pgfpathrectangle{\pgfqpoint{0.539299in}{0.078740in}}{\pgfqpoint{7.842520in}{7.842520in}}%
\pgfusepath{clip}%
\pgfsetbuttcap%
\pgfsetroundjoin%
\definecolor{currentfill}{rgb}{0.244972,0.287675,0.537260}%
\pgfsetfillcolor{currentfill}%
\pgfsetlinewidth{0.000000pt}%
\definecolor{currentstroke}{rgb}{0.916242,0.896091,0.100717}%
\pgfsetstrokecolor{currentstroke}%
\pgfsetdash{}{0pt}%
\pgfpathmoveto{\pgfqpoint{4.803456in}{4.103048in}}%
\pgfpathlineto{\pgfqpoint{4.861395in}{4.049469in}}%
\pgfpathlineto{\pgfqpoint{4.937227in}{3.984970in}}%
\pgfpathclose%
\pgfusepath{fill}%
\end{pgfscope}%
\begin{pgfscope}%
\pgfpathrectangle{\pgfqpoint{0.539299in}{0.078740in}}{\pgfqpoint{7.842520in}{7.842520in}}%
\pgfusepath{clip}%
\pgfsetbuttcap%
\pgfsetroundjoin%
\definecolor{currentfill}{rgb}{0.281446,0.084320,0.407414}%
\pgfsetfillcolor{currentfill}%
\pgfsetlinewidth{0.000000pt}%
\definecolor{currentstroke}{rgb}{0.926106,0.897330,0.104071}%
\pgfsetstrokecolor{currentstroke}%
\pgfsetdash{}{0pt}%
\pgfpathmoveto{\pgfqpoint{6.509400in}{3.398734in}}%
\pgfpathlineto{\pgfqpoint{6.646106in}{3.340736in}}%
\pgfpathlineto{\pgfqpoint{6.579539in}{3.406827in}}%
\pgfpathclose%
\pgfusepath{fill}%
\end{pgfscope}%
\begin{pgfscope}%
\pgfpathrectangle{\pgfqpoint{0.539299in}{0.078740in}}{\pgfqpoint{7.842520in}{7.842520in}}%
\pgfusepath{clip}%
\pgfsetbuttcap%
\pgfsetroundjoin%
\definecolor{currentfill}{rgb}{0.174274,0.445044,0.557792}%
\pgfsetfillcolor{currentfill}%
\pgfsetlinewidth{0.000000pt}%
\definecolor{currentstroke}{rgb}{0.935904,0.898570,0.108131}%
\pgfsetstrokecolor{currentstroke}%
\pgfsetdash{}{0pt}%
\pgfpathmoveto{\pgfqpoint{4.324600in}{4.594132in}}%
\pgfpathlineto{\pgfqpoint{4.269526in}{4.615818in}}%
\pgfpathlineto{\pgfqpoint{4.190652in}{4.714468in}}%
\pgfpathclose%
\pgfusepath{fill}%
\end{pgfscope}%
\begin{pgfscope}%
\pgfpathrectangle{\pgfqpoint{0.539299in}{0.078740in}}{\pgfqpoint{7.842520in}{7.842520in}}%
\pgfusepath{clip}%
\pgfsetbuttcap%
\pgfsetroundjoin%
\definecolor{currentfill}{rgb}{0.252194,0.269783,0.531579}%
\pgfsetfillcolor{currentfill}%
\pgfsetlinewidth{0.000000pt}%
\definecolor{currentstroke}{rgb}{0.945636,0.899815,0.112838}%
\pgfsetstrokecolor{currentstroke}%
\pgfsetdash{}{0pt}%
\pgfpathmoveto{\pgfqpoint{5.071218in}{3.879659in}}%
\pgfpathlineto{\pgfqpoint{4.937227in}{3.984970in}}%
\pgfpathlineto{\pgfqpoint{4.861395in}{4.049469in}}%
\pgfpathclose%
\pgfusepath{fill}%
\end{pgfscope}%
\begin{pgfscope}%
\pgfpathrectangle{\pgfqpoint{0.539299in}{0.078740in}}{\pgfqpoint{7.842520in}{7.842520in}}%
\pgfusepath{clip}%
\pgfsetbuttcap%
\pgfsetroundjoin%
\definecolor{currentfill}{rgb}{0.280255,0.165693,0.476498}%
\pgfsetfillcolor{currentfill}%
\pgfsetlinewidth{0.000000pt}%
\definecolor{currentstroke}{rgb}{0.955300,0.901065,0.118128}%
\pgfsetstrokecolor{currentstroke}%
\pgfsetdash{}{0pt}%
\pgfpathmoveto{\pgfqpoint{5.684398in}{3.594113in}}%
\pgfpathlineto{\pgfqpoint{5.549086in}{3.640953in}}%
\pgfpathlineto{\pgfqpoint{5.475289in}{3.644517in}}%
\pgfpathclose%
\pgfusepath{fill}%
\end{pgfscope}%
\begin{pgfscope}%
\pgfpathrectangle{\pgfqpoint{0.539299in}{0.078740in}}{\pgfqpoint{7.842520in}{7.842520in}}%
\pgfusepath{clip}%
\pgfsetbuttcap%
\pgfsetroundjoin%
\definecolor{currentfill}{rgb}{0.150476,0.504369,0.557430}%
\pgfsetfillcolor{currentfill}%
\pgfsetlinewidth{0.000000pt}%
\definecolor{currentstroke}{rgb}{0.964894,0.902323,0.123941}%
\pgfsetstrokecolor{currentstroke}%
\pgfsetdash{}{0pt}%
\pgfpathmoveto{\pgfqpoint{3.791614in}{4.877687in}}%
\pgfpathlineto{\pgfqpoint{3.843276in}{4.946277in}}%
\pgfpathlineto{\pgfqpoint{3.923882in}{4.868004in}}%
\pgfpathclose%
\pgfusepath{fill}%
\end{pgfscope}%
\begin{pgfscope}%
\pgfpathrectangle{\pgfqpoint{0.539299in}{0.078740in}}{\pgfqpoint{7.842520in}{7.842520in}}%
\pgfusepath{clip}%
\pgfsetbuttcap%
\pgfsetroundjoin%
\definecolor{currentfill}{rgb}{0.273809,0.031497,0.358853}%
\pgfsetfillcolor{currentfill}%
\pgfsetlinewidth{0.000000pt}%
\definecolor{currentstroke}{rgb}{0.974417,0.903590,0.130215}%
\pgfsetstrokecolor{currentstroke}%
\pgfsetdash{}{0pt}%
\pgfpathmoveto{\pgfqpoint{7.126919in}{3.173464in}}%
\pgfpathlineto{\pgfqpoint{7.058628in}{3.259387in}}%
\pgfpathlineto{\pgfqpoint{6.989604in}{3.234274in}}%
\pgfpathclose%
\pgfusepath{fill}%
\end{pgfscope}%
\begin{pgfscope}%
\pgfpathrectangle{\pgfqpoint{0.539299in}{0.078740in}}{\pgfqpoint{7.842520in}{7.842520in}}%
\pgfusepath{clip}%
\pgfsetbuttcap%
\pgfsetroundjoin%
\definecolor{currentfill}{rgb}{0.278791,0.062145,0.386592}%
\pgfsetfillcolor{currentfill}%
\pgfsetlinewidth{0.000000pt}%
\definecolor{currentstroke}{rgb}{0.983868,0.904867,0.136897}%
\pgfsetstrokecolor{currentstroke}%
\pgfsetdash{}{0pt}%
\pgfpathmoveto{\pgfqpoint{6.783027in}{3.277708in}}%
\pgfpathlineto{\pgfqpoint{6.852629in}{3.294416in}}%
\pgfpathlineto{\pgfqpoint{6.715949in}{3.352378in}}%
\pgfpathclose%
\pgfusepath{fill}%
\end{pgfscope}%
\begin{pgfscope}%
\pgfpathrectangle{\pgfqpoint{0.539299in}{0.078740in}}{\pgfqpoint{7.842520in}{7.842520in}}%
\pgfusepath{clip}%
\pgfsetbuttcap%
\pgfsetroundjoin%
\definecolor{currentfill}{rgb}{0.190631,0.407061,0.556089}%
\pgfsetfillcolor{currentfill}%
\pgfsetlinewidth{0.000000pt}%
\definecolor{currentstroke}{rgb}{0.993248,0.906157,0.143936}%
\pgfsetstrokecolor{currentstroke}%
\pgfsetdash{}{0pt}%
\pgfpathmoveto{\pgfqpoint{4.324600in}{4.594132in}}%
\pgfpathlineto{\pgfqpoint{4.458695in}{4.459156in}}%
\pgfpathlineto{\pgfqpoint{4.536301in}{4.363851in}}%
\pgfpathclose%
\pgfusepath{fill}%
\end{pgfscope}%
\begin{pgfscope}%
\pgfpathrectangle{\pgfqpoint{0.539299in}{0.078740in}}{\pgfqpoint{7.842520in}{7.842520in}}%
\pgfusepath{clip}%
\pgfsetbuttcap%
\pgfsetroundjoin%
\definecolor{currentfill}{rgb}{0.278012,0.180367,0.486697}%
\pgfsetfillcolor{currentfill}%
\pgfsetlinewidth{0.000000pt}%
\definecolor{currentstroke}{rgb}{0.267004,0.004874,0.329415}%
\pgfsetstrokecolor{currentstroke}%
\pgfsetdash{}{0pt}%
\pgfpathmoveto{\pgfqpoint{5.475289in}{3.644517in}}%
\pgfpathlineto{\pgfqpoint{5.549086in}{3.640953in}}%
\pgfpathlineto{\pgfqpoint{5.340183in}{3.710444in}}%
\pgfpathclose%
\pgfusepath{fill}%
\end{pgfscope}%
\begin{pgfscope}%
\pgfpathrectangle{\pgfqpoint{0.539299in}{0.078740in}}{\pgfqpoint{7.842520in}{7.842520in}}%
\pgfusepath{clip}%
\pgfsetbuttcap%
\pgfsetroundjoin%
\definecolor{currentfill}{rgb}{0.271305,0.019942,0.347269}%
\pgfsetfillcolor{currentfill}%
\pgfsetlinewidth{0.000000pt}%
\definecolor{currentstroke}{rgb}{0.268510,0.009605,0.335427}%
\pgfsetstrokecolor{currentstroke}%
\pgfsetdash{}{0pt}%
\pgfpathmoveto{\pgfqpoint{7.333628in}{3.157861in}}%
\pgfpathlineto{\pgfqpoint{7.195913in}{3.208174in}}%
\pgfpathlineto{\pgfqpoint{7.126919in}{3.173464in}}%
\pgfpathclose%
\pgfusepath{fill}%
\end{pgfscope}%
\begin{pgfscope}%
\pgfpathrectangle{\pgfqpoint{0.539299in}{0.078740in}}{\pgfqpoint{7.842520in}{7.842520in}}%
\pgfusepath{clip}%
\pgfsetbuttcap%
\pgfsetroundjoin%
\definecolor{currentfill}{rgb}{0.206756,0.371758,0.553117}%
\pgfsetfillcolor{currentfill}%
\pgfsetlinewidth{0.000000pt}%
\definecolor{currentstroke}{rgb}{0.269944,0.014625,0.341379}%
\pgfsetstrokecolor{currentstroke}%
\pgfsetdash{}{0pt}%
\pgfpathmoveto{\pgfqpoint{4.458695in}{4.459156in}}%
\pgfpathlineto{\pgfqpoint{4.592860in}{4.318703in}}%
\pgfpathlineto{\pgfqpoint{4.669832in}{4.230954in}}%
\pgfpathclose%
\pgfusepath{fill}%
\end{pgfscope}%
\begin{pgfscope}%
\pgfpathrectangle{\pgfqpoint{0.539299in}{0.078740in}}{\pgfqpoint{7.842520in}{7.842520in}}%
\pgfusepath{clip}%
\pgfsetbuttcap%
\pgfsetroundjoin%
\definecolor{currentfill}{rgb}{0.283091,0.110553,0.431554}%
\pgfsetfillcolor{currentfill}%
\pgfsetlinewidth{0.000000pt}%
\definecolor{currentstroke}{rgb}{0.271305,0.019942,0.347269}%
\pgfsetstrokecolor{currentstroke}%
\pgfsetdash{}{0pt}%
\pgfpathmoveto{\pgfqpoint{6.301764in}{3.437824in}}%
\pgfpathlineto{\pgfqpoint{6.509400in}{3.398734in}}%
\pgfpathlineto{\pgfqpoint{6.372915in}{3.450815in}}%
\pgfpathclose%
\pgfusepath{fill}%
\end{pgfscope}%
\begin{pgfscope}%
\pgfpathrectangle{\pgfqpoint{0.539299in}{0.078740in}}{\pgfqpoint{7.842520in}{7.842520in}}%
\pgfusepath{clip}%
\pgfsetbuttcap%
\pgfsetroundjoin%
\definecolor{currentfill}{rgb}{0.216210,0.351535,0.550627}%
\pgfsetfillcolor{currentfill}%
\pgfsetlinewidth{0.000000pt}%
\definecolor{currentstroke}{rgb}{0.272594,0.025563,0.353093}%
\pgfsetstrokecolor{currentstroke}%
\pgfsetdash{}{0pt}%
\pgfpathmoveto{\pgfqpoint{4.669832in}{4.230954in}}%
\pgfpathlineto{\pgfqpoint{4.592860in}{4.318703in}}%
\pgfpathlineto{\pgfqpoint{4.727082in}{4.180259in}}%
\pgfpathclose%
\pgfusepath{fill}%
\end{pgfscope}%
\begin{pgfscope}%
\pgfpathrectangle{\pgfqpoint{0.539299in}{0.078740in}}{\pgfqpoint{7.842520in}{7.842520in}}%
\pgfusepath{clip}%
\pgfsetbuttcap%
\pgfsetroundjoin%
\definecolor{currentfill}{rgb}{0.283229,0.120777,0.440584}%
\pgfsetfillcolor{currentfill}%
\pgfsetlinewidth{0.000000pt}%
\definecolor{currentstroke}{rgb}{0.273809,0.031497,0.358853}%
\pgfsetstrokecolor{currentstroke}%
\pgfsetdash{}{0pt}%
\pgfpathmoveto{\pgfqpoint{6.165167in}{3.484596in}}%
\pgfpathlineto{\pgfqpoint{6.301764in}{3.437824in}}%
\pgfpathlineto{\pgfqpoint{6.372915in}{3.450815in}}%
\pgfpathclose%
\pgfusepath{fill}%
\end{pgfscope}%
\begin{pgfscope}%
\pgfpathrectangle{\pgfqpoint{0.539299in}{0.078740in}}{\pgfqpoint{7.842520in}{7.842520in}}%
\pgfusepath{clip}%
\pgfsetbuttcap%
\pgfsetroundjoin%
\definecolor{currentfill}{rgb}{0.263663,0.237631,0.518762}%
\pgfsetfillcolor{currentfill}%
\pgfsetlinewidth{0.000000pt}%
\definecolor{currentstroke}{rgb}{0.274952,0.037752,0.364543}%
\pgfsetstrokecolor{currentstroke}%
\pgfsetdash{}{0pt}%
\pgfpathmoveto{\pgfqpoint{5.130570in}{3.824189in}}%
\pgfpathlineto{\pgfqpoint{5.205512in}{3.788278in}}%
\pgfpathlineto{\pgfqpoint{5.071218in}{3.879659in}}%
\pgfpathclose%
\pgfusepath{fill}%
\end{pgfscope}%
\begin{pgfscope}%
\pgfpathrectangle{\pgfqpoint{0.539299in}{0.078740in}}{\pgfqpoint{7.842520in}{7.842520in}}%
\pgfusepath{clip}%
\pgfsetbuttcap%
\pgfsetroundjoin%
\definecolor{currentfill}{rgb}{0.146180,0.515413,0.556823}%
\pgfsetfillcolor{currentfill}%
\pgfsetlinewidth{0.000000pt}%
\definecolor{currentstroke}{rgb}{0.276022,0.044167,0.370164}%
\pgfsetstrokecolor{currentstroke}%
\pgfsetdash{}{0pt}%
\pgfpathmoveto{\pgfqpoint{3.710739in}{4.939290in}}%
\pgfpathlineto{\pgfqpoint{3.843276in}{4.946277in}}%
\pgfpathlineto{\pgfqpoint{3.791614in}{4.877687in}}%
\pgfpathclose%
\pgfusepath{fill}%
\end{pgfscope}%
\begin{pgfscope}%
\pgfpathrectangle{\pgfqpoint{0.539299in}{0.078740in}}{\pgfqpoint{7.842520in}{7.842520in}}%
\pgfusepath{clip}%
\pgfsetbuttcap%
\pgfsetroundjoin%
\definecolor{currentfill}{rgb}{0.280894,0.078907,0.402329}%
\pgfsetfillcolor{currentfill}%
\pgfsetlinewidth{0.000000pt}%
\definecolor{currentstroke}{rgb}{0.277018,0.050344,0.375715}%
\pgfsetstrokecolor{currentstroke}%
\pgfsetdash{}{0pt}%
\pgfpathmoveto{\pgfqpoint{6.715949in}{3.352378in}}%
\pgfpathlineto{\pgfqpoint{6.646106in}{3.340736in}}%
\pgfpathlineto{\pgfqpoint{6.783027in}{3.277708in}}%
\pgfpathclose%
\pgfusepath{fill}%
\end{pgfscope}%
\begin{pgfscope}%
\pgfpathrectangle{\pgfqpoint{0.539299in}{0.078740in}}{\pgfqpoint{7.842520in}{7.842520in}}%
\pgfusepath{clip}%
\pgfsetbuttcap%
\pgfsetroundjoin%
\definecolor{currentfill}{rgb}{0.231674,0.318106,0.544834}%
\pgfsetfillcolor{currentfill}%
\pgfsetlinewidth{0.000000pt}%
\definecolor{currentstroke}{rgb}{0.277941,0.056324,0.381191}%
\pgfsetstrokecolor{currentstroke}%
\pgfsetdash{}{0pt}%
\pgfpathmoveto{\pgfqpoint{4.727082in}{4.180259in}}%
\pgfpathlineto{\pgfqpoint{4.861395in}{4.049469in}}%
\pgfpathlineto{\pgfqpoint{4.803456in}{4.103048in}}%
\pgfpathclose%
\pgfusepath{fill}%
\end{pgfscope}%
\begin{pgfscope}%
\pgfpathrectangle{\pgfqpoint{0.539299in}{0.078740in}}{\pgfqpoint{7.842520in}{7.842520in}}%
\pgfusepath{clip}%
\pgfsetbuttcap%
\pgfsetroundjoin%
\definecolor{currentfill}{rgb}{0.147607,0.511733,0.557049}%
\pgfsetfillcolor{currentfill}%
\pgfsetlinewidth{0.000000pt}%
\definecolor{currentstroke}{rgb}{0.278791,0.062145,0.386592}%
\pgfsetstrokecolor{currentstroke}%
\pgfsetdash{}{0pt}%
\pgfpathmoveto{\pgfqpoint{3.498080in}{4.903513in}}%
\pgfpathlineto{\pgfqpoint{3.710739in}{4.939290in}}%
\pgfpathlineto{\pgfqpoint{3.579639in}{4.867750in}}%
\pgfpathclose%
\pgfusepath{fill}%
\end{pgfscope}%
\begin{pgfscope}%
\pgfpathrectangle{\pgfqpoint{0.539299in}{0.078740in}}{\pgfqpoint{7.842520in}{7.842520in}}%
\pgfusepath{clip}%
\pgfsetbuttcap%
\pgfsetroundjoin%
\definecolor{currentfill}{rgb}{0.277941,0.056324,0.381191}%
\pgfsetfillcolor{currentfill}%
\pgfsetlinewidth{0.000000pt}%
\definecolor{currentstroke}{rgb}{0.279566,0.067836,0.391917}%
\pgfsetstrokecolor{currentstroke}%
\pgfsetdash{}{0pt}%
\pgfpathmoveto{\pgfqpoint{6.989604in}{3.234274in}}%
\pgfpathlineto{\pgfqpoint{6.852629in}{3.294416in}}%
\pgfpathlineto{\pgfqpoint{6.783027in}{3.277708in}}%
\pgfpathclose%
\pgfusepath{fill}%
\end{pgfscope}%
\begin{pgfscope}%
\pgfpathrectangle{\pgfqpoint{0.539299in}{0.078740in}}{\pgfqpoint{7.842520in}{7.842520in}}%
\pgfusepath{clip}%
\pgfsetbuttcap%
\pgfsetroundjoin%
\definecolor{currentfill}{rgb}{0.282290,0.145912,0.461510}%
\pgfsetfillcolor{currentfill}%
\pgfsetlinewidth{0.000000pt}%
\definecolor{currentstroke}{rgb}{0.280267,0.073417,0.397163}%
\pgfsetstrokecolor{currentstroke}%
\pgfsetdash{}{0pt}%
\pgfpathmoveto{\pgfqpoint{5.956414in}{3.509868in}}%
\pgfpathlineto{\pgfqpoint{6.028872in}{3.526024in}}%
\pgfpathlineto{\pgfqpoint{5.820187in}{3.551515in}}%
\pgfpathclose%
\pgfusepath{fill}%
\end{pgfscope}%
\begin{pgfscope}%
\pgfpathrectangle{\pgfqpoint{0.539299in}{0.078740in}}{\pgfqpoint{7.842520in}{7.842520in}}%
\pgfusepath{clip}%
\pgfsetbuttcap%
\pgfsetroundjoin%
\definecolor{currentfill}{rgb}{0.280255,0.165693,0.476498}%
\pgfsetfillcolor{currentfill}%
\pgfsetlinewidth{0.000000pt}%
\definecolor{currentstroke}{rgb}{0.280894,0.078907,0.402329}%
\pgfsetstrokecolor{currentstroke}%
\pgfsetdash{}{0pt}%
\pgfpathmoveto{\pgfqpoint{5.820187in}{3.551515in}}%
\pgfpathlineto{\pgfqpoint{5.684398in}{3.594113in}}%
\pgfpathlineto{\pgfqpoint{5.610860in}{3.587946in}}%
\pgfpathclose%
\pgfusepath{fill}%
\end{pgfscope}%
\begin{pgfscope}%
\pgfpathrectangle{\pgfqpoint{0.539299in}{0.078740in}}{\pgfqpoint{7.842520in}{7.842520in}}%
\pgfusepath{clip}%
\pgfsetbuttcap%
\pgfsetroundjoin%
\definecolor{currentfill}{rgb}{0.267968,0.223549,0.512008}%
\pgfsetfillcolor{currentfill}%
\pgfsetlinewidth{0.000000pt}%
\definecolor{currentstroke}{rgb}{0.281446,0.084320,0.407414}%
\pgfsetstrokecolor{currentstroke}%
\pgfsetdash{}{0pt}%
\pgfpathmoveto{\pgfqpoint{5.130570in}{3.824189in}}%
\pgfpathlineto{\pgfqpoint{5.340183in}{3.710444in}}%
\pgfpathlineto{\pgfqpoint{5.205512in}{3.788278in}}%
\pgfpathclose%
\pgfusepath{fill}%
\end{pgfscope}%
\begin{pgfscope}%
\pgfpathrectangle{\pgfqpoint{0.539299in}{0.078740in}}{\pgfqpoint{7.842520in}{7.842520in}}%
\pgfusepath{clip}%
\pgfsetbuttcap%
\pgfsetroundjoin%
\definecolor{currentfill}{rgb}{0.282884,0.135920,0.453427}%
\pgfsetfillcolor{currentfill}%
\pgfsetlinewidth{0.000000pt}%
\definecolor{currentstroke}{rgb}{0.281924,0.089666,0.412415}%
\pgfsetstrokecolor{currentstroke}%
\pgfsetdash{}{0pt}%
\pgfpathmoveto{\pgfqpoint{6.093026in}{3.466206in}}%
\pgfpathlineto{\pgfqpoint{6.165167in}{3.484596in}}%
\pgfpathlineto{\pgfqpoint{6.028872in}{3.526024in}}%
\pgfpathclose%
\pgfusepath{fill}%
\end{pgfscope}%
\begin{pgfscope}%
\pgfpathrectangle{\pgfqpoint{0.539299in}{0.078740in}}{\pgfqpoint{7.842520in}{7.842520in}}%
\pgfusepath{clip}%
\pgfsetbuttcap%
\pgfsetroundjoin%
\definecolor{currentfill}{rgb}{0.248629,0.278775,0.534556}%
\pgfsetfillcolor{currentfill}%
\pgfsetlinewidth{0.000000pt}%
\definecolor{currentstroke}{rgb}{0.282327,0.094955,0.417331}%
\pgfsetstrokecolor{currentstroke}%
\pgfsetdash{}{0pt}%
\pgfpathmoveto{\pgfqpoint{4.995865in}{3.930113in}}%
\pgfpathlineto{\pgfqpoint{5.071218in}{3.879659in}}%
\pgfpathlineto{\pgfqpoint{4.861395in}{4.049469in}}%
\pgfpathclose%
\pgfusepath{fill}%
\end{pgfscope}%
\begin{pgfscope}%
\pgfpathrectangle{\pgfqpoint{0.539299in}{0.078740in}}{\pgfqpoint{7.842520in}{7.842520in}}%
\pgfusepath{clip}%
\pgfsetbuttcap%
\pgfsetroundjoin%
\definecolor{currentfill}{rgb}{0.150476,0.504369,0.557430}%
\pgfsetfillcolor{currentfill}%
\pgfsetlinewidth{0.000000pt}%
\definecolor{currentstroke}{rgb}{0.282656,0.100196,0.422160}%
\pgfsetstrokecolor{currentstroke}%
\pgfsetdash{}{0pt}%
\pgfpathmoveto{\pgfqpoint{3.976815in}{4.900063in}}%
\pgfpathlineto{\pgfqpoint{4.057001in}{4.809605in}}%
\pgfpathlineto{\pgfqpoint{3.923882in}{4.868004in}}%
\pgfpathclose%
\pgfusepath{fill}%
\end{pgfscope}%
\begin{pgfscope}%
\pgfpathrectangle{\pgfqpoint{0.539299in}{0.078740in}}{\pgfqpoint{7.842520in}{7.842520in}}%
\pgfusepath{clip}%
\pgfsetbuttcap%
\pgfsetroundjoin%
\definecolor{currentfill}{rgb}{0.278826,0.175490,0.483397}%
\pgfsetfillcolor{currentfill}%
\pgfsetlinewidth{0.000000pt}%
\definecolor{currentstroke}{rgb}{0.282910,0.105393,0.426902}%
\pgfsetstrokecolor{currentstroke}%
\pgfsetdash{}{0pt}%
\pgfpathmoveto{\pgfqpoint{5.475289in}{3.644517in}}%
\pgfpathlineto{\pgfqpoint{5.610860in}{3.587946in}}%
\pgfpathlineto{\pgfqpoint{5.684398in}{3.594113in}}%
\pgfpathclose%
\pgfusepath{fill}%
\end{pgfscope}%
\begin{pgfscope}%
\pgfpathrectangle{\pgfqpoint{0.539299in}{0.078740in}}{\pgfqpoint{7.842520in}{7.842520in}}%
\pgfusepath{clip}%
\pgfsetbuttcap%
\pgfsetroundjoin%
\definecolor{currentfill}{rgb}{0.269944,0.014625,0.341379}%
\pgfsetfillcolor{currentfill}%
\pgfsetlinewidth{0.000000pt}%
\definecolor{currentstroke}{rgb}{0.283091,0.110553,0.431554}%
\pgfsetstrokecolor{currentstroke}%
\pgfsetdash{}{0pt}%
\pgfpathmoveto{\pgfqpoint{7.264628in}{3.113510in}}%
\pgfpathlineto{\pgfqpoint{7.471813in}{3.109193in}}%
\pgfpathlineto{\pgfqpoint{7.333628in}{3.157861in}}%
\pgfpathclose%
\pgfusepath{fill}%
\end{pgfscope}%
\begin{pgfscope}%
\pgfpathrectangle{\pgfqpoint{0.539299in}{0.078740in}}{\pgfqpoint{7.842520in}{7.842520in}}%
\pgfusepath{clip}%
\pgfsetbuttcap%
\pgfsetroundjoin%
\definecolor{currentfill}{rgb}{0.271305,0.019942,0.347269}%
\pgfsetfillcolor{currentfill}%
\pgfsetlinewidth{0.000000pt}%
\definecolor{currentstroke}{rgb}{0.283197,0.115680,0.436115}%
\pgfsetstrokecolor{currentstroke}%
\pgfsetdash{}{0pt}%
\pgfpathmoveto{\pgfqpoint{7.126919in}{3.173464in}}%
\pgfpathlineto{\pgfqpoint{7.264628in}{3.113510in}}%
\pgfpathlineto{\pgfqpoint{7.333628in}{3.157861in}}%
\pgfpathclose%
\pgfusepath{fill}%
\end{pgfscope}%
\begin{pgfscope}%
\pgfpathrectangle{\pgfqpoint{0.539299in}{0.078740in}}{\pgfqpoint{7.842520in}{7.842520in}}%
\pgfusepath{clip}%
\pgfsetbuttcap%
\pgfsetroundjoin%
\definecolor{currentfill}{rgb}{0.255645,0.260703,0.528312}%
\pgfsetfillcolor{currentfill}%
\pgfsetlinewidth{0.000000pt}%
\definecolor{currentstroke}{rgb}{0.283229,0.120777,0.440584}%
\pgfsetstrokecolor{currentstroke}%
\pgfsetdash{}{0pt}%
\pgfpathmoveto{\pgfqpoint{5.130570in}{3.824189in}}%
\pgfpathlineto{\pgfqpoint{5.071218in}{3.879659in}}%
\pgfpathlineto{\pgfqpoint{4.995865in}{3.930113in}}%
\pgfpathclose%
\pgfusepath{fill}%
\end{pgfscope}%
\begin{pgfscope}%
\pgfpathrectangle{\pgfqpoint{0.539299in}{0.078740in}}{\pgfqpoint{7.842520in}{7.842520in}}%
\pgfusepath{clip}%
\pgfsetbuttcap%
\pgfsetroundjoin%
\definecolor{currentfill}{rgb}{0.283197,0.115680,0.436115}%
\pgfsetfillcolor{currentfill}%
\pgfsetlinewidth{0.000000pt}%
\definecolor{currentstroke}{rgb}{0.283187,0.125848,0.444960}%
\pgfsetstrokecolor{currentstroke}%
\pgfsetdash{}{0pt}%
\pgfpathmoveto{\pgfqpoint{6.301764in}{3.437824in}}%
\pgfpathlineto{\pgfqpoint{6.438604in}{3.384604in}}%
\pgfpathlineto{\pgfqpoint{6.509400in}{3.398734in}}%
\pgfpathclose%
\pgfusepath{fill}%
\end{pgfscope}%
\begin{pgfscope}%
\pgfpathrectangle{\pgfqpoint{0.539299in}{0.078740in}}{\pgfqpoint{7.842520in}{7.842520in}}%
\pgfusepath{clip}%
\pgfsetbuttcap%
\pgfsetroundjoin%
\definecolor{currentfill}{rgb}{0.282290,0.145912,0.461510}%
\pgfsetfillcolor{currentfill}%
\pgfsetlinewidth{0.000000pt}%
\definecolor{currentstroke}{rgb}{0.283072,0.130895,0.449241}%
\pgfsetstrokecolor{currentstroke}%
\pgfsetdash{}{0pt}%
\pgfpathmoveto{\pgfqpoint{6.093026in}{3.466206in}}%
\pgfpathlineto{\pgfqpoint{6.028872in}{3.526024in}}%
\pgfpathlineto{\pgfqpoint{5.956414in}{3.509868in}}%
\pgfpathclose%
\pgfusepath{fill}%
\end{pgfscope}%
\begin{pgfscope}%
\pgfpathrectangle{\pgfqpoint{0.539299in}{0.078740in}}{\pgfqpoint{7.842520in}{7.842520in}}%
\pgfusepath{clip}%
\pgfsetbuttcap%
\pgfsetroundjoin%
\definecolor{currentfill}{rgb}{0.153364,0.497000,0.557724}%
\pgfsetfillcolor{currentfill}%
\pgfsetlinewidth{0.000000pt}%
\definecolor{currentstroke}{rgb}{0.282884,0.135920,0.453427}%
\pgfsetstrokecolor{currentstroke}%
\pgfsetdash{}{0pt}%
\pgfpathmoveto{\pgfqpoint{4.190652in}{4.714468in}}%
\pgfpathlineto{\pgfqpoint{4.057001in}{4.809605in}}%
\pgfpathlineto{\pgfqpoint{3.976815in}{4.900063in}}%
\pgfpathclose%
\pgfusepath{fill}%
\end{pgfscope}%
\begin{pgfscope}%
\pgfpathrectangle{\pgfqpoint{0.539299in}{0.078740in}}{\pgfqpoint{7.842520in}{7.842520in}}%
\pgfusepath{clip}%
\pgfsetbuttcap%
\pgfsetroundjoin%
\definecolor{currentfill}{rgb}{0.153364,0.497000,0.557724}%
\pgfsetfillcolor{currentfill}%
\pgfsetlinewidth{0.000000pt}%
\definecolor{currentstroke}{rgb}{0.282623,0.140926,0.457517}%
\pgfsetstrokecolor{currentstroke}%
\pgfsetdash{}{0pt}%
\pgfpathmoveto{\pgfqpoint{3.415921in}{4.934934in}}%
\pgfpathlineto{\pgfqpoint{3.369083in}{4.733483in}}%
\pgfpathlineto{\pgfqpoint{3.287152in}{4.741607in}}%
\pgfpathclose%
\pgfusepath{fill}%
\end{pgfscope}%
\begin{pgfscope}%
\pgfpathrectangle{\pgfqpoint{0.539299in}{0.078740in}}{\pgfqpoint{7.842520in}{7.842520in}}%
\pgfusepath{clip}%
\pgfsetbuttcap%
\pgfsetroundjoin%
\definecolor{currentfill}{rgb}{0.210503,0.363727,0.552206}%
\pgfsetfillcolor{currentfill}%
\pgfsetlinewidth{0.000000pt}%
\definecolor{currentstroke}{rgb}{0.282290,0.145912,0.461510}%
\pgfsetstrokecolor{currentstroke}%
\pgfsetdash{}{0pt}%
\pgfpathmoveto{\pgfqpoint{2.913780in}{4.395967in}}%
\pgfpathlineto{\pgfqpoint{2.996771in}{4.420700in}}%
\pgfpathlineto{\pgfqpoint{2.793452in}{3.956696in}}%
\pgfpathclose%
\pgfusepath{fill}%
\end{pgfscope}%
\begin{pgfscope}%
\pgfpathrectangle{\pgfqpoint{0.539299in}{0.078740in}}{\pgfqpoint{7.842520in}{7.842520in}}%
\pgfusepath{clip}%
\pgfsetbuttcap%
\pgfsetroundjoin%
\definecolor{currentfill}{rgb}{0.282656,0.100196,0.422160}%
\pgfsetfillcolor{currentfill}%
\pgfsetlinewidth{0.000000pt}%
\definecolor{currentstroke}{rgb}{0.281887,0.150881,0.465405}%
\pgfsetstrokecolor{currentstroke}%
\pgfsetdash{}{0pt}%
\pgfpathmoveto{\pgfqpoint{6.575647in}{3.324679in}}%
\pgfpathlineto{\pgfqpoint{6.646106in}{3.340736in}}%
\pgfpathlineto{\pgfqpoint{6.509400in}{3.398734in}}%
\pgfpathclose%
\pgfusepath{fill}%
\end{pgfscope}%
\begin{pgfscope}%
\pgfpathrectangle{\pgfqpoint{0.539299in}{0.078740in}}{\pgfqpoint{7.842520in}{7.842520in}}%
\pgfusepath{clip}%
\pgfsetbuttcap%
\pgfsetroundjoin%
\definecolor{currentfill}{rgb}{0.168126,0.459988,0.558082}%
\pgfsetfillcolor{currentfill}%
\pgfsetlinewidth{0.000000pt}%
\definecolor{currentstroke}{rgb}{0.281412,0.155834,0.469201}%
\pgfsetstrokecolor{currentstroke}%
\pgfsetdash{}{0pt}%
\pgfpathmoveto{\pgfqpoint{3.079268in}{4.442627in}}%
\pgfpathlineto{\pgfqpoint{3.204675in}{4.746147in}}%
\pgfpathlineto{\pgfqpoint{3.287152in}{4.741607in}}%
\pgfpathclose%
\pgfusepath{fill}%
\end{pgfscope}%
\begin{pgfscope}%
\pgfpathrectangle{\pgfqpoint{0.539299in}{0.078740in}}{\pgfqpoint{7.842520in}{7.842520in}}%
\pgfusepath{clip}%
\pgfsetbuttcap%
\pgfsetroundjoin%
\definecolor{currentfill}{rgb}{0.147607,0.511733,0.557049}%
\pgfsetfillcolor{currentfill}%
\pgfsetlinewidth{0.000000pt}%
\definecolor{currentstroke}{rgb}{0.280868,0.160771,0.472899}%
\pgfsetstrokecolor{currentstroke}%
\pgfsetdash{}{0pt}%
\pgfpathmoveto{\pgfqpoint{3.415921in}{4.934934in}}%
\pgfpathlineto{\pgfqpoint{3.498080in}{4.903513in}}%
\pgfpathlineto{\pgfqpoint{3.369083in}{4.733483in}}%
\pgfpathclose%
\pgfusepath{fill}%
\end{pgfscope}%
\begin{pgfscope}%
\pgfpathrectangle{\pgfqpoint{0.539299in}{0.078740in}}{\pgfqpoint{7.842520in}{7.842520in}}%
\pgfusepath{clip}%
\pgfsetbuttcap%
\pgfsetroundjoin%
\definecolor{currentfill}{rgb}{0.282884,0.135920,0.453427}%
\pgfsetfillcolor{currentfill}%
\pgfsetlinewidth{0.000000pt}%
\definecolor{currentstroke}{rgb}{0.280255,0.165693,0.476498}%
\pgfsetstrokecolor{currentstroke}%
\pgfsetdash{}{0pt}%
\pgfpathmoveto{\pgfqpoint{6.301764in}{3.437824in}}%
\pgfpathlineto{\pgfqpoint{6.165167in}{3.484596in}}%
\pgfpathlineto{\pgfqpoint{6.093026in}{3.466206in}}%
\pgfpathclose%
\pgfusepath{fill}%
\end{pgfscope}%
\begin{pgfscope}%
\pgfpathrectangle{\pgfqpoint{0.539299in}{0.078740in}}{\pgfqpoint{7.842520in}{7.842520in}}%
\pgfusepath{clip}%
\pgfsetbuttcap%
\pgfsetroundjoin%
\definecolor{currentfill}{rgb}{0.278791,0.062145,0.386592}%
\pgfsetfillcolor{currentfill}%
\pgfsetlinewidth{0.000000pt}%
\definecolor{currentstroke}{rgb}{0.279574,0.170599,0.479997}%
\pgfsetstrokecolor{currentstroke}%
\pgfsetdash{}{0pt}%
\pgfpathmoveto{\pgfqpoint{6.783027in}{3.277708in}}%
\pgfpathlineto{\pgfqpoint{6.920184in}{3.211063in}}%
\pgfpathlineto{\pgfqpoint{6.989604in}{3.234274in}}%
\pgfpathclose%
\pgfusepath{fill}%
\end{pgfscope}%
\begin{pgfscope}%
\pgfpathrectangle{\pgfqpoint{0.539299in}{0.078740in}}{\pgfqpoint{7.842520in}{7.842520in}}%
\pgfusepath{clip}%
\pgfsetbuttcap%
\pgfsetroundjoin%
\definecolor{currentfill}{rgb}{0.276022,0.044167,0.370164}%
\pgfsetfillcolor{currentfill}%
\pgfsetlinewidth{0.000000pt}%
\definecolor{currentstroke}{rgb}{0.278826,0.175490,0.483397}%
\pgfsetstrokecolor{currentstroke}%
\pgfsetdash{}{0pt}%
\pgfpathmoveto{\pgfqpoint{7.057617in}{3.142554in}}%
\pgfpathlineto{\pgfqpoint{7.126919in}{3.173464in}}%
\pgfpathlineto{\pgfqpoint{6.989604in}{3.234274in}}%
\pgfpathclose%
\pgfusepath{fill}%
\end{pgfscope}%
\begin{pgfscope}%
\pgfpathrectangle{\pgfqpoint{0.539299in}{0.078740in}}{\pgfqpoint{7.842520in}{7.842520in}}%
\pgfusepath{clip}%
\pgfsetbuttcap%
\pgfsetroundjoin%
\definecolor{currentfill}{rgb}{0.143343,0.522773,0.556295}%
\pgfsetfillcolor{currentfill}%
\pgfsetlinewidth{0.000000pt}%
\definecolor{currentstroke}{rgb}{0.278012,0.180367,0.486697}%
\pgfsetstrokecolor{currentstroke}%
\pgfsetdash{}{0pt}%
\pgfpathmoveto{\pgfqpoint{3.923882in}{4.868004in}}%
\pgfpathlineto{\pgfqpoint{3.843276in}{4.946277in}}%
\pgfpathlineto{\pgfqpoint{3.976815in}{4.900063in}}%
\pgfpathclose%
\pgfusepath{fill}%
\end{pgfscope}%
\begin{pgfscope}%
\pgfpathrectangle{\pgfqpoint{0.539299in}{0.078740in}}{\pgfqpoint{7.842520in}{7.842520in}}%
\pgfusepath{clip}%
\pgfsetbuttcap%
\pgfsetroundjoin%
\definecolor{currentfill}{rgb}{0.273006,0.204520,0.501721}%
\pgfsetfillcolor{currentfill}%
\pgfsetlinewidth{0.000000pt}%
\definecolor{currentstroke}{rgb}{0.277134,0.185228,0.489898}%
\pgfsetstrokecolor{currentstroke}%
\pgfsetdash{}{0pt}%
\pgfpathmoveto{\pgfqpoint{5.340183in}{3.710444in}}%
\pgfpathlineto{\pgfqpoint{5.400994in}{3.652853in}}%
\pgfpathlineto{\pgfqpoint{5.475289in}{3.644517in}}%
\pgfpathclose%
\pgfusepath{fill}%
\end{pgfscope}%
\begin{pgfscope}%
\pgfpathrectangle{\pgfqpoint{0.539299in}{0.078740in}}{\pgfqpoint{7.842520in}{7.842520in}}%
\pgfusepath{clip}%
\pgfsetbuttcap%
\pgfsetroundjoin%
\definecolor{currentfill}{rgb}{0.229739,0.322361,0.545706}%
\pgfsetfillcolor{currentfill}%
\pgfsetlinewidth{0.000000pt}%
\definecolor{currentstroke}{rgb}{0.276194,0.190074,0.493001}%
\pgfsetstrokecolor{currentstroke}%
\pgfsetdash{}{0pt}%
\pgfpathmoveto{\pgfqpoint{2.793452in}{3.956696in}}%
\pgfpathlineto{\pgfqpoint{2.710747in}{3.908413in}}%
\pgfpathlineto{\pgfqpoint{2.830306in}{4.368134in}}%
\pgfpathclose%
\pgfusepath{fill}%
\end{pgfscope}%
\begin{pgfscope}%
\pgfpathrectangle{\pgfqpoint{0.539299in}{0.078740in}}{\pgfqpoint{7.842520in}{7.842520in}}%
\pgfusepath{clip}%
\pgfsetbuttcap%
\pgfsetroundjoin%
\definecolor{currentfill}{rgb}{0.160665,0.478540,0.558115}%
\pgfsetfillcolor{currentfill}%
\pgfsetlinewidth{0.000000pt}%
\definecolor{currentstroke}{rgb}{0.275191,0.194905,0.496005}%
\pgfsetstrokecolor{currentstroke}%
\pgfsetdash{}{0pt}%
\pgfpathmoveto{\pgfqpoint{4.190652in}{4.714468in}}%
\pgfpathlineto{\pgfqpoint{4.111002in}{4.812426in}}%
\pgfpathlineto{\pgfqpoint{4.324600in}{4.594132in}}%
\pgfpathclose%
\pgfusepath{fill}%
\end{pgfscope}%
\begin{pgfscope}%
\pgfpathrectangle{\pgfqpoint{0.539299in}{0.078740in}}{\pgfqpoint{7.842520in}{7.842520in}}%
\pgfusepath{clip}%
\pgfsetbuttcap%
\pgfsetroundjoin%
\definecolor{currentfill}{rgb}{0.283091,0.110553,0.431554}%
\pgfsetfillcolor{currentfill}%
\pgfsetlinewidth{0.000000pt}%
\definecolor{currentstroke}{rgb}{0.274128,0.199721,0.498911}%
\pgfsetstrokecolor{currentstroke}%
\pgfsetdash{}{0pt}%
\pgfpathmoveto{\pgfqpoint{6.509400in}{3.398734in}}%
\pgfpathlineto{\pgfqpoint{6.438604in}{3.384604in}}%
\pgfpathlineto{\pgfqpoint{6.575647in}{3.324679in}}%
\pgfpathclose%
\pgfusepath{fill}%
\end{pgfscope}%
\begin{pgfscope}%
\pgfpathrectangle{\pgfqpoint{0.539299in}{0.078740in}}{\pgfqpoint{7.842520in}{7.842520in}}%
\pgfusepath{clip}%
\pgfsetbuttcap%
\pgfsetroundjoin%
\definecolor{currentfill}{rgb}{0.265145,0.232956,0.516599}%
\pgfsetfillcolor{currentfill}%
\pgfsetlinewidth{0.000000pt}%
\definecolor{currentstroke}{rgb}{0.273006,0.204520,0.501721}%
\pgfsetstrokecolor{currentstroke}%
\pgfsetdash{}{0pt}%
\pgfpathmoveto{\pgfqpoint{5.265591in}{3.732089in}}%
\pgfpathlineto{\pgfqpoint{5.340183in}{3.710444in}}%
\pgfpathlineto{\pgfqpoint{5.130570in}{3.824189in}}%
\pgfpathclose%
\pgfusepath{fill}%
\end{pgfscope}%
\begin{pgfscope}%
\pgfpathrectangle{\pgfqpoint{0.539299in}{0.078740in}}{\pgfqpoint{7.842520in}{7.842520in}}%
\pgfusepath{clip}%
\pgfsetbuttcap%
\pgfsetroundjoin%
\definecolor{currentfill}{rgb}{0.279574,0.170599,0.479997}%
\pgfsetfillcolor{currentfill}%
\pgfsetlinewidth{0.000000pt}%
\definecolor{currentstroke}{rgb}{0.271828,0.209303,0.504434}%
\pgfsetstrokecolor{currentstroke}%
\pgfsetdash{}{0pt}%
\pgfpathmoveto{\pgfqpoint{5.610860in}{3.587946in}}%
\pgfpathlineto{\pgfqpoint{5.746902in}{3.537634in}}%
\pgfpathlineto{\pgfqpoint{5.820187in}{3.551515in}}%
\pgfpathclose%
\pgfusepath{fill}%
\end{pgfscope}%
\begin{pgfscope}%
\pgfpathrectangle{\pgfqpoint{0.539299in}{0.078740in}}{\pgfqpoint{7.842520in}{7.842520in}}%
\pgfusepath{clip}%
\pgfsetbuttcap%
\pgfsetroundjoin%
\definecolor{currentfill}{rgb}{0.282327,0.094955,0.417331}%
\pgfsetfillcolor{currentfill}%
\pgfsetlinewidth{0.000000pt}%
\definecolor{currentstroke}{rgb}{0.270595,0.214069,0.507052}%
\pgfsetstrokecolor{currentstroke}%
\pgfsetdash{}{0pt}%
\pgfpathmoveto{\pgfqpoint{6.575647in}{3.324679in}}%
\pgfpathlineto{\pgfqpoint{6.783027in}{3.277708in}}%
\pgfpathlineto{\pgfqpoint{6.646106in}{3.340736in}}%
\pgfpathclose%
\pgfusepath{fill}%
\end{pgfscope}%
\begin{pgfscope}%
\pgfpathrectangle{\pgfqpoint{0.539299in}{0.078740in}}{\pgfqpoint{7.842520in}{7.842520in}}%
\pgfusepath{clip}%
\pgfsetbuttcap%
\pgfsetroundjoin%
\definecolor{currentfill}{rgb}{0.269944,0.014625,0.341379}%
\pgfsetfillcolor{currentfill}%
\pgfsetlinewidth{0.000000pt}%
\definecolor{currentstroke}{rgb}{0.269308,0.218818,0.509577}%
\pgfsetstrokecolor{currentstroke}%
\pgfsetdash{}{0pt}%
\pgfpathmoveto{\pgfqpoint{7.402794in}{3.055818in}}%
\pgfpathlineto{\pgfqpoint{7.471813in}{3.109193in}}%
\pgfpathlineto{\pgfqpoint{7.264628in}{3.113510in}}%
\pgfpathclose%
\pgfusepath{fill}%
\end{pgfscope}%
\begin{pgfscope}%
\pgfpathrectangle{\pgfqpoint{0.539299in}{0.078740in}}{\pgfqpoint{7.842520in}{7.842520in}}%
\pgfusepath{clip}%
\pgfsetbuttcap%
\pgfsetroundjoin%
\definecolor{currentfill}{rgb}{0.277941,0.056324,0.381191}%
\pgfsetfillcolor{currentfill}%
\pgfsetlinewidth{0.000000pt}%
\definecolor{currentstroke}{rgb}{0.267968,0.223549,0.512008}%
\pgfsetstrokecolor{currentstroke}%
\pgfsetdash{}{0pt}%
\pgfpathmoveto{\pgfqpoint{6.989604in}{3.234274in}}%
\pgfpathlineto{\pgfqpoint{6.920184in}{3.211063in}}%
\pgfpathlineto{\pgfqpoint{7.057617in}{3.142554in}}%
\pgfpathclose%
\pgfusepath{fill}%
\end{pgfscope}%
\begin{pgfscope}%
\pgfpathrectangle{\pgfqpoint{0.539299in}{0.078740in}}{\pgfqpoint{7.842520in}{7.842520in}}%
\pgfusepath{clip}%
\pgfsetbuttcap%
\pgfsetroundjoin%
\definecolor{currentfill}{rgb}{0.137770,0.537492,0.554906}%
\pgfsetfillcolor{currentfill}%
\pgfsetlinewidth{0.000000pt}%
\definecolor{currentstroke}{rgb}{0.266580,0.228262,0.514349}%
\pgfsetstrokecolor{currentstroke}%
\pgfsetdash{}{0pt}%
\pgfpathmoveto{\pgfqpoint{3.629210in}{4.996070in}}%
\pgfpathlineto{\pgfqpoint{3.710739in}{4.939290in}}%
\pgfpathlineto{\pgfqpoint{3.498080in}{4.903513in}}%
\pgfpathclose%
\pgfusepath{fill}%
\end{pgfscope}%
\begin{pgfscope}%
\pgfpathrectangle{\pgfqpoint{0.539299in}{0.078740in}}{\pgfqpoint{7.842520in}{7.842520in}}%
\pgfusepath{clip}%
\pgfsetbuttcap%
\pgfsetroundjoin%
\definecolor{currentfill}{rgb}{0.280868,0.160771,0.472899}%
\pgfsetfillcolor{currentfill}%
\pgfsetlinewidth{0.000000pt}%
\definecolor{currentstroke}{rgb}{0.265145,0.232956,0.516599}%
\pgfsetstrokecolor{currentstroke}%
\pgfsetdash{}{0pt}%
\pgfpathmoveto{\pgfqpoint{5.820187in}{3.551515in}}%
\pgfpathlineto{\pgfqpoint{5.883392in}{3.490293in}}%
\pgfpathlineto{\pgfqpoint{5.956414in}{3.509868in}}%
\pgfpathclose%
\pgfusepath{fill}%
\end{pgfscope}%
\begin{pgfscope}%
\pgfpathrectangle{\pgfqpoint{0.539299in}{0.078740in}}{\pgfqpoint{7.842520in}{7.842520in}}%
\pgfusepath{clip}%
\pgfsetbuttcap%
\pgfsetroundjoin%
\definecolor{currentfill}{rgb}{0.175841,0.441290,0.557685}%
\pgfsetfillcolor{currentfill}%
\pgfsetlinewidth{0.000000pt}%
\definecolor{currentstroke}{rgb}{0.263663,0.237631,0.518762}%
\pgfsetstrokecolor{currentstroke}%
\pgfsetdash{}{0pt}%
\pgfpathmoveto{\pgfqpoint{4.324600in}{4.594132in}}%
\pgfpathlineto{\pgfqpoint{4.380311in}{4.558363in}}%
\pgfpathlineto{\pgfqpoint{4.458695in}{4.459156in}}%
\pgfpathclose%
\pgfusepath{fill}%
\end{pgfscope}%
\begin{pgfscope}%
\pgfpathrectangle{\pgfqpoint{0.539299in}{0.078740in}}{\pgfqpoint{7.842520in}{7.842520in}}%
\pgfusepath{clip}%
\pgfsetbuttcap%
\pgfsetroundjoin%
\definecolor{currentfill}{rgb}{0.273809,0.031497,0.358853}%
\pgfsetfillcolor{currentfill}%
\pgfsetlinewidth{0.000000pt}%
\definecolor{currentstroke}{rgb}{0.262138,0.242286,0.520837}%
\pgfsetstrokecolor{currentstroke}%
\pgfsetdash{}{0pt}%
\pgfpathmoveto{\pgfqpoint{7.264628in}{3.113510in}}%
\pgfpathlineto{\pgfqpoint{7.126919in}{3.173464in}}%
\pgfpathlineto{\pgfqpoint{7.057617in}{3.142554in}}%
\pgfpathclose%
\pgfusepath{fill}%
\end{pgfscope}%
\begin{pgfscope}%
\pgfpathrectangle{\pgfqpoint{0.539299in}{0.078740in}}{\pgfqpoint{7.842520in}{7.842520in}}%
\pgfusepath{clip}%
\pgfsetbuttcap%
\pgfsetroundjoin%
\definecolor{currentfill}{rgb}{0.185556,0.418570,0.556753}%
\pgfsetfillcolor{currentfill}%
\pgfsetlinewidth{0.000000pt}%
\definecolor{currentstroke}{rgb}{0.260571,0.246922,0.522828}%
\pgfsetstrokecolor{currentstroke}%
\pgfsetdash{}{0pt}%
\pgfpathmoveto{\pgfqpoint{4.380311in}{4.558363in}}%
\pgfpathlineto{\pgfqpoint{4.592860in}{4.318703in}}%
\pgfpathlineto{\pgfqpoint{4.458695in}{4.459156in}}%
\pgfpathclose%
\pgfusepath{fill}%
\end{pgfscope}%
\begin{pgfscope}%
\pgfpathrectangle{\pgfqpoint{0.539299in}{0.078740in}}{\pgfqpoint{7.842520in}{7.842520in}}%
\pgfusepath{clip}%
\pgfsetbuttcap%
\pgfsetroundjoin%
\definecolor{currentfill}{rgb}{0.267004,0.004874,0.329415}%
\pgfsetfillcolor{currentfill}%
\pgfsetlinewidth{0.000000pt}%
\definecolor{currentstroke}{rgb}{0.258965,0.251537,0.524736}%
\pgfsetstrokecolor{currentstroke}%
\pgfsetdash{}{0pt}%
\pgfpathmoveto{\pgfqpoint{7.541482in}{3.001589in}}%
\pgfpathlineto{\pgfqpoint{7.610502in}{3.062741in}}%
\pgfpathlineto{\pgfqpoint{7.471813in}{3.109193in}}%
\pgfpathclose%
\pgfusepath{fill}%
\end{pgfscope}%
\begin{pgfscope}%
\pgfpathrectangle{\pgfqpoint{0.539299in}{0.078740in}}{\pgfqpoint{7.842520in}{7.842520in}}%
\pgfusepath{clip}%
\pgfsetbuttcap%
\pgfsetroundjoin%
\definecolor{currentfill}{rgb}{0.269308,0.218818,0.509577}%
\pgfsetfillcolor{currentfill}%
\pgfsetlinewidth{0.000000pt}%
\definecolor{currentstroke}{rgb}{0.257322,0.256130,0.526563}%
\pgfsetstrokecolor{currentstroke}%
\pgfsetdash{}{0pt}%
\pgfpathmoveto{\pgfqpoint{5.265591in}{3.732089in}}%
\pgfpathlineto{\pgfqpoint{5.400994in}{3.652853in}}%
\pgfpathlineto{\pgfqpoint{5.340183in}{3.710444in}}%
\pgfpathclose%
\pgfusepath{fill}%
\end{pgfscope}%
\begin{pgfscope}%
\pgfpathrectangle{\pgfqpoint{0.539299in}{0.078740in}}{\pgfqpoint{7.842520in}{7.842520in}}%
\pgfusepath{clip}%
\pgfsetbuttcap%
\pgfsetroundjoin%
\definecolor{currentfill}{rgb}{0.136408,0.541173,0.554483}%
\pgfsetfillcolor{currentfill}%
\pgfsetlinewidth{0.000000pt}%
\definecolor{currentstroke}{rgb}{0.255645,0.260703,0.528312}%
\pgfsetstrokecolor{currentstroke}%
\pgfsetdash{}{0pt}%
\pgfpathmoveto{\pgfqpoint{3.843276in}{4.946277in}}%
\pgfpathlineto{\pgfqpoint{3.710739in}{4.939290in}}%
\pgfpathlineto{\pgfqpoint{3.629210in}{4.996070in}}%
\pgfpathclose%
\pgfusepath{fill}%
\end{pgfscope}%
\begin{pgfscope}%
\pgfpathrectangle{\pgfqpoint{0.539299in}{0.078740in}}{\pgfqpoint{7.842520in}{7.842520in}}%
\pgfusepath{clip}%
\pgfsetbuttcap%
\pgfsetroundjoin%
\definecolor{currentfill}{rgb}{0.175841,0.441290,0.557685}%
\pgfsetfillcolor{currentfill}%
\pgfsetlinewidth{0.000000pt}%
\definecolor{currentstroke}{rgb}{0.253935,0.265254,0.529983}%
\pgfsetstrokecolor{currentstroke}%
\pgfsetdash{}{0pt}%
\pgfpathmoveto{\pgfqpoint{3.079268in}{4.442627in}}%
\pgfpathlineto{\pgfqpoint{2.996771in}{4.420700in}}%
\pgfpathlineto{\pgfqpoint{3.121655in}{4.747134in}}%
\pgfpathclose%
\pgfusepath{fill}%
\end{pgfscope}%
\begin{pgfscope}%
\pgfpathrectangle{\pgfqpoint{0.539299in}{0.078740in}}{\pgfqpoint{7.842520in}{7.842520in}}%
\pgfusepath{clip}%
\pgfsetbuttcap%
\pgfsetroundjoin%
\definecolor{currentfill}{rgb}{0.204903,0.375746,0.553533}%
\pgfsetfillcolor{currentfill}%
\pgfsetlinewidth{0.000000pt}%
\definecolor{currentstroke}{rgb}{0.252194,0.269783,0.531579}%
\pgfsetstrokecolor{currentstroke}%
\pgfsetdash{}{0pt}%
\pgfpathmoveto{\pgfqpoint{4.592860in}{4.318703in}}%
\pgfpathlineto{\pgfqpoint{4.649993in}{4.264866in}}%
\pgfpathlineto{\pgfqpoint{4.727082in}{4.180259in}}%
\pgfpathclose%
\pgfusepath{fill}%
\end{pgfscope}%
\begin{pgfscope}%
\pgfpathrectangle{\pgfqpoint{0.539299in}{0.078740in}}{\pgfqpoint{7.842520in}{7.842520in}}%
\pgfusepath{clip}%
\pgfsetbuttcap%
\pgfsetroundjoin%
\definecolor{currentfill}{rgb}{0.216210,0.351535,0.550627}%
\pgfsetfillcolor{currentfill}%
\pgfsetlinewidth{0.000000pt}%
\definecolor{currentstroke}{rgb}{0.250425,0.274290,0.533103}%
\pgfsetstrokecolor{currentstroke}%
\pgfsetdash{}{0pt}%
\pgfpathmoveto{\pgfqpoint{4.727082in}{4.180259in}}%
\pgfpathlineto{\pgfqpoint{4.649993in}{4.264866in}}%
\pgfpathlineto{\pgfqpoint{4.861395in}{4.049469in}}%
\pgfpathclose%
\pgfusepath{fill}%
\end{pgfscope}%
\begin{pgfscope}%
\pgfpathrectangle{\pgfqpoint{0.539299in}{0.078740in}}{\pgfqpoint{7.842520in}{7.842520in}}%
\pgfusepath{clip}%
\pgfsetbuttcap%
\pgfsetroundjoin%
\definecolor{currentfill}{rgb}{0.275191,0.194905,0.496005}%
\pgfsetfillcolor{currentfill}%
\pgfsetlinewidth{0.000000pt}%
\definecolor{currentstroke}{rgb}{0.248629,0.278775,0.534556}%
\pgfsetstrokecolor{currentstroke}%
\pgfsetdash{}{0pt}%
\pgfpathmoveto{\pgfqpoint{5.536826in}{3.584474in}}%
\pgfpathlineto{\pgfqpoint{5.610860in}{3.587946in}}%
\pgfpathlineto{\pgfqpoint{5.475289in}{3.644517in}}%
\pgfpathclose%
\pgfusepath{fill}%
\end{pgfscope}%
\begin{pgfscope}%
\pgfpathrectangle{\pgfqpoint{0.539299in}{0.078740in}}{\pgfqpoint{7.842520in}{7.842520in}}%
\pgfusepath{clip}%
\pgfsetbuttcap%
\pgfsetroundjoin%
\definecolor{currentfill}{rgb}{0.279574,0.170599,0.479997}%
\pgfsetfillcolor{currentfill}%
\pgfsetlinewidth{0.000000pt}%
\definecolor{currentstroke}{rgb}{0.246811,0.283237,0.535941}%
\pgfsetstrokecolor{currentstroke}%
\pgfsetdash{}{0pt}%
\pgfpathmoveto{\pgfqpoint{5.820187in}{3.551515in}}%
\pgfpathlineto{\pgfqpoint{5.746902in}{3.537634in}}%
\pgfpathlineto{\pgfqpoint{5.883392in}{3.490293in}}%
\pgfpathclose%
\pgfusepath{fill}%
\end{pgfscope}%
\begin{pgfscope}%
\pgfpathrectangle{\pgfqpoint{0.539299in}{0.078740in}}{\pgfqpoint{7.842520in}{7.842520in}}%
\pgfusepath{clip}%
\pgfsetbuttcap%
\pgfsetroundjoin%
\definecolor{currentfill}{rgb}{0.268510,0.009605,0.335427}%
\pgfsetfillcolor{currentfill}%
\pgfsetlinewidth{0.000000pt}%
\definecolor{currentstroke}{rgb}{0.244972,0.287675,0.537260}%
\pgfsetstrokecolor{currentstroke}%
\pgfsetdash{}{0pt}%
\pgfpathmoveto{\pgfqpoint{7.541482in}{3.001589in}}%
\pgfpathlineto{\pgfqpoint{7.471813in}{3.109193in}}%
\pgfpathlineto{\pgfqpoint{7.402794in}{3.055818in}}%
\pgfpathclose%
\pgfusepath{fill}%
\end{pgfscope}%
\begin{pgfscope}%
\pgfpathrectangle{\pgfqpoint{0.539299in}{0.078740in}}{\pgfqpoint{7.842520in}{7.842520in}}%
\pgfusepath{clip}%
\pgfsetbuttcap%
\pgfsetroundjoin%
\definecolor{currentfill}{rgb}{0.282623,0.140926,0.457517}%
\pgfsetfillcolor{currentfill}%
\pgfsetlinewidth{0.000000pt}%
\definecolor{currentstroke}{rgb}{0.243113,0.292092,0.538516}%
\pgfsetstrokecolor{currentstroke}%
\pgfsetdash{}{0pt}%
\pgfpathmoveto{\pgfqpoint{6.229961in}{3.418141in}}%
\pgfpathlineto{\pgfqpoint{6.301764in}{3.437824in}}%
\pgfpathlineto{\pgfqpoint{6.093026in}{3.466206in}}%
\pgfpathclose%
\pgfusepath{fill}%
\end{pgfscope}%
\begin{pgfscope}%
\pgfpathrectangle{\pgfqpoint{0.539299in}{0.078740in}}{\pgfqpoint{7.842520in}{7.842520in}}%
\pgfusepath{clip}%
\pgfsetbuttcap%
\pgfsetroundjoin%
\definecolor{currentfill}{rgb}{0.149039,0.508051,0.557250}%
\pgfsetfillcolor{currentfill}%
\pgfsetlinewidth{0.000000pt}%
\definecolor{currentstroke}{rgb}{0.241237,0.296485,0.539709}%
\pgfsetstrokecolor{currentstroke}%
\pgfsetdash{}{0pt}%
\pgfpathmoveto{\pgfqpoint{3.976815in}{4.900063in}}%
\pgfpathlineto{\pgfqpoint{4.111002in}{4.812426in}}%
\pgfpathlineto{\pgfqpoint{4.190652in}{4.714468in}}%
\pgfpathclose%
\pgfusepath{fill}%
\end{pgfscope}%
\begin{pgfscope}%
\pgfpathrectangle{\pgfqpoint{0.539299in}{0.078740in}}{\pgfqpoint{7.842520in}{7.842520in}}%
\pgfusepath{clip}%
\pgfsetbuttcap%
\pgfsetroundjoin%
\definecolor{currentfill}{rgb}{0.283072,0.130895,0.449241}%
\pgfsetfillcolor{currentfill}%
\pgfsetlinewidth{0.000000pt}%
\definecolor{currentstroke}{rgb}{0.239346,0.300855,0.540844}%
\pgfsetstrokecolor{currentstroke}%
\pgfsetdash{}{0pt}%
\pgfpathmoveto{\pgfqpoint{6.438604in}{3.384604in}}%
\pgfpathlineto{\pgfqpoint{6.301764in}{3.437824in}}%
\pgfpathlineto{\pgfqpoint{6.367154in}{3.364036in}}%
\pgfpathclose%
\pgfusepath{fill}%
\end{pgfscope}%
\begin{pgfscope}%
\pgfpathrectangle{\pgfqpoint{0.539299in}{0.078740in}}{\pgfqpoint{7.842520in}{7.842520in}}%
\pgfusepath{clip}%
\pgfsetbuttcap%
\pgfsetroundjoin%
\definecolor{currentfill}{rgb}{0.235526,0.309527,0.542944}%
\pgfsetfillcolor{currentfill}%
\pgfsetlinewidth{0.000000pt}%
\definecolor{currentstroke}{rgb}{0.237441,0.305202,0.541921}%
\pgfsetstrokecolor{currentstroke}%
\pgfsetdash{}{0pt}%
\pgfpathmoveto{\pgfqpoint{4.861395in}{4.049469in}}%
\pgfpathlineto{\pgfqpoint{4.919892in}{3.989315in}}%
\pgfpathlineto{\pgfqpoint{4.995865in}{3.930113in}}%
\pgfpathclose%
\pgfusepath{fill}%
\end{pgfscope}%
\begin{pgfscope}%
\pgfpathrectangle{\pgfqpoint{0.539299in}{0.078740in}}{\pgfqpoint{7.842520in}{7.842520in}}%
\pgfusepath{clip}%
\pgfsetbuttcap%
\pgfsetroundjoin%
\definecolor{currentfill}{rgb}{0.204903,0.375746,0.553533}%
\pgfsetfillcolor{currentfill}%
\pgfsetlinewidth{0.000000pt}%
\definecolor{currentstroke}{rgb}{0.235526,0.309527,0.542944}%
\pgfsetstrokecolor{currentstroke}%
\pgfsetdash{}{0pt}%
\pgfpathmoveto{\pgfqpoint{2.830306in}{4.368134in}}%
\pgfpathlineto{\pgfqpoint{2.913780in}{4.395967in}}%
\pgfpathlineto{\pgfqpoint{2.793452in}{3.956696in}}%
\pgfpathclose%
\pgfusepath{fill}%
\end{pgfscope}%
\begin{pgfscope}%
\pgfpathrectangle{\pgfqpoint{0.539299in}{0.078740in}}{\pgfqpoint{7.842520in}{7.842520in}}%
\pgfusepath{clip}%
\pgfsetbuttcap%
\pgfsetroundjoin%
\definecolor{currentfill}{rgb}{0.267004,0.004874,0.329415}%
\pgfsetfillcolor{currentfill}%
\pgfsetlinewidth{0.000000pt}%
\definecolor{currentstroke}{rgb}{0.233603,0.313828,0.543914}%
\pgfsetstrokecolor{currentstroke}%
\pgfsetdash{}{0pt}%
\pgfpathmoveto{\pgfqpoint{7.749729in}{3.018900in}}%
\pgfpathlineto{\pgfqpoint{7.610502in}{3.062741in}}%
\pgfpathlineto{\pgfqpoint{7.541482in}{3.001589in}}%
\pgfpathclose%
\pgfusepath{fill}%
\end{pgfscope}%
\begin{pgfscope}%
\pgfpathrectangle{\pgfqpoint{0.539299in}{0.078740in}}{\pgfqpoint{7.842520in}{7.842520in}}%
\pgfusepath{clip}%
\pgfsetbuttcap%
\pgfsetroundjoin%
\definecolor{currentfill}{rgb}{0.282656,0.100196,0.422160}%
\pgfsetfillcolor{currentfill}%
\pgfsetlinewidth{0.000000pt}%
\definecolor{currentstroke}{rgb}{0.231674,0.318106,0.544834}%
\pgfsetstrokecolor{currentstroke}%
\pgfsetdash{}{0pt}%
\pgfpathmoveto{\pgfqpoint{6.712873in}{3.258583in}}%
\pgfpathlineto{\pgfqpoint{6.783027in}{3.277708in}}%
\pgfpathlineto{\pgfqpoint{6.575647in}{3.324679in}}%
\pgfpathclose%
\pgfusepath{fill}%
\end{pgfscope}%
\begin{pgfscope}%
\pgfpathrectangle{\pgfqpoint{0.539299in}{0.078740in}}{\pgfqpoint{7.842520in}{7.842520in}}%
\pgfusepath{clip}%
\pgfsetbuttcap%
\pgfsetroundjoin%
\definecolor{currentfill}{rgb}{0.162142,0.474838,0.558140}%
\pgfsetfillcolor{currentfill}%
\pgfsetlinewidth{0.000000pt}%
\definecolor{currentstroke}{rgb}{0.229739,0.322361,0.545706}%
\pgfsetstrokecolor{currentstroke}%
\pgfsetdash{}{0pt}%
\pgfpathmoveto{\pgfqpoint{3.121655in}{4.747134in}}%
\pgfpathlineto{\pgfqpoint{3.204675in}{4.746147in}}%
\pgfpathlineto{\pgfqpoint{3.079268in}{4.442627in}}%
\pgfpathclose%
\pgfusepath{fill}%
\end{pgfscope}%
\begin{pgfscope}%
\pgfpathrectangle{\pgfqpoint{0.539299in}{0.078740in}}{\pgfqpoint{7.842520in}{7.842520in}}%
\pgfusepath{clip}%
\pgfsetbuttcap%
\pgfsetroundjoin%
\definecolor{currentfill}{rgb}{0.157729,0.485932,0.558013}%
\pgfsetfillcolor{currentfill}%
\pgfsetlinewidth{0.000000pt}%
\definecolor{currentstroke}{rgb}{0.227802,0.326594,0.546532}%
\pgfsetstrokecolor{currentstroke}%
\pgfsetdash{}{0pt}%
\pgfpathmoveto{\pgfqpoint{4.324600in}{4.594132in}}%
\pgfpathlineto{\pgfqpoint{4.111002in}{4.812426in}}%
\pgfpathlineto{\pgfqpoint{4.245563in}{4.694927in}}%
\pgfpathclose%
\pgfusepath{fill}%
\end{pgfscope}%
\begin{pgfscope}%
\pgfpathrectangle{\pgfqpoint{0.539299in}{0.078740in}}{\pgfqpoint{7.842520in}{7.842520in}}%
\pgfusepath{clip}%
\pgfsetbuttcap%
\pgfsetroundjoin%
\definecolor{currentfill}{rgb}{0.280894,0.078907,0.402329}%
\pgfsetfillcolor{currentfill}%
\pgfsetlinewidth{0.000000pt}%
\definecolor{currentstroke}{rgb}{0.225863,0.330805,0.547314}%
\pgfsetstrokecolor{currentstroke}%
\pgfsetdash{}{0pt}%
\pgfpathmoveto{\pgfqpoint{6.783027in}{3.277708in}}%
\pgfpathlineto{\pgfqpoint{6.850286in}{3.187517in}}%
\pgfpathlineto{\pgfqpoint{6.920184in}{3.211063in}}%
\pgfpathclose%
\pgfusepath{fill}%
\end{pgfscope}%
\begin{pgfscope}%
\pgfpathrectangle{\pgfqpoint{0.539299in}{0.078740in}}{\pgfqpoint{7.842520in}{7.842520in}}%
\pgfusepath{clip}%
\pgfsetbuttcap%
\pgfsetroundjoin%
\definecolor{currentfill}{rgb}{0.273006,0.204520,0.501721}%
\pgfsetfillcolor{currentfill}%
\pgfsetlinewidth{0.000000pt}%
\definecolor{currentstroke}{rgb}{0.223925,0.334994,0.548053}%
\pgfsetstrokecolor{currentstroke}%
\pgfsetdash{}{0pt}%
\pgfpathmoveto{\pgfqpoint{5.475289in}{3.644517in}}%
\pgfpathlineto{\pgfqpoint{5.400994in}{3.652853in}}%
\pgfpathlineto{\pgfqpoint{5.536826in}{3.584474in}}%
\pgfpathclose%
\pgfusepath{fill}%
\end{pgfscope}%
\begin{pgfscope}%
\pgfpathrectangle{\pgfqpoint{0.539299in}{0.078740in}}{\pgfqpoint{7.842520in}{7.842520in}}%
\pgfusepath{clip}%
\pgfsetbuttcap%
\pgfsetroundjoin%
\definecolor{currentfill}{rgb}{0.243113,0.292092,0.538516}%
\pgfsetfillcolor{currentfill}%
\pgfsetlinewidth{0.000000pt}%
\definecolor{currentstroke}{rgb}{0.221989,0.339161,0.548752}%
\pgfsetstrokecolor{currentstroke}%
\pgfsetdash{}{0pt}%
\pgfpathmoveto{\pgfqpoint{4.995865in}{3.930113in}}%
\pgfpathlineto{\pgfqpoint{4.919892in}{3.989315in}}%
\pgfpathlineto{\pgfqpoint{5.130570in}{3.824189in}}%
\pgfpathclose%
\pgfusepath{fill}%
\end{pgfscope}%
\begin{pgfscope}%
\pgfpathrectangle{\pgfqpoint{0.539299in}{0.078740in}}{\pgfqpoint{7.842520in}{7.842520in}}%
\pgfusepath{clip}%
\pgfsetbuttcap%
\pgfsetroundjoin%
\definecolor{currentfill}{rgb}{0.280868,0.160771,0.472899}%
\pgfsetfillcolor{currentfill}%
\pgfsetlinewidth{0.000000pt}%
\definecolor{currentstroke}{rgb}{0.220057,0.343307,0.549413}%
\pgfsetstrokecolor{currentstroke}%
\pgfsetdash{}{0pt}%
\pgfpathmoveto{\pgfqpoint{6.020289in}{3.442779in}}%
\pgfpathlineto{\pgfqpoint{6.093026in}{3.466206in}}%
\pgfpathlineto{\pgfqpoint{5.956414in}{3.509868in}}%
\pgfpathclose%
\pgfusepath{fill}%
\end{pgfscope}%
\begin{pgfscope}%
\pgfpathrectangle{\pgfqpoint{0.539299in}{0.078740in}}{\pgfqpoint{7.842520in}{7.842520in}}%
\pgfusepath{clip}%
\pgfsetbuttcap%
\pgfsetroundjoin%
\definecolor{currentfill}{rgb}{0.276194,0.190074,0.493001}%
\pgfsetfillcolor{currentfill}%
\pgfsetlinewidth{0.000000pt}%
\definecolor{currentstroke}{rgb}{0.218130,0.347432,0.550038}%
\pgfsetstrokecolor{currentstroke}%
\pgfsetdash{}{0pt}%
\pgfpathmoveto{\pgfqpoint{5.536826in}{3.584474in}}%
\pgfpathlineto{\pgfqpoint{5.746902in}{3.537634in}}%
\pgfpathlineto{\pgfqpoint{5.610860in}{3.587946in}}%
\pgfpathclose%
\pgfusepath{fill}%
\end{pgfscope}%
\begin{pgfscope}%
\pgfpathrectangle{\pgfqpoint{0.539299in}{0.078740in}}{\pgfqpoint{7.842520in}{7.842520in}}%
\pgfusepath{clip}%
\pgfsetbuttcap%
\pgfsetroundjoin%
\definecolor{currentfill}{rgb}{0.274952,0.037752,0.364543}%
\pgfsetfillcolor{currentfill}%
\pgfsetlinewidth{0.000000pt}%
\definecolor{currentstroke}{rgb}{0.216210,0.351535,0.550627}%
\pgfsetstrokecolor{currentstroke}%
\pgfsetdash{}{0pt}%
\pgfpathmoveto{\pgfqpoint{7.195388in}{3.074084in}}%
\pgfpathlineto{\pgfqpoint{7.264628in}{3.113510in}}%
\pgfpathlineto{\pgfqpoint{7.057617in}{3.142554in}}%
\pgfpathclose%
\pgfusepath{fill}%
\end{pgfscope}%
\begin{pgfscope}%
\pgfpathrectangle{\pgfqpoint{0.539299in}{0.078740in}}{\pgfqpoint{7.842520in}{7.842520in}}%
\pgfusepath{clip}%
\pgfsetbuttcap%
\pgfsetroundjoin%
\definecolor{currentfill}{rgb}{0.165117,0.467423,0.558141}%
\pgfsetfillcolor{currentfill}%
\pgfsetlinewidth{0.000000pt}%
\definecolor{currentstroke}{rgb}{0.214298,0.355619,0.551184}%
\pgfsetstrokecolor{currentstroke}%
\pgfsetdash{}{0pt}%
\pgfpathmoveto{\pgfqpoint{4.245563in}{4.694927in}}%
\pgfpathlineto{\pgfqpoint{4.380311in}{4.558363in}}%
\pgfpathlineto{\pgfqpoint{4.324600in}{4.594132in}}%
\pgfpathclose%
\pgfusepath{fill}%
\end{pgfscope}%
\begin{pgfscope}%
\pgfpathrectangle{\pgfqpoint{0.539299in}{0.078740in}}{\pgfqpoint{7.842520in}{7.842520in}}%
\pgfusepath{clip}%
\pgfsetbuttcap%
\pgfsetroundjoin%
\definecolor{currentfill}{rgb}{0.135066,0.544853,0.554029}%
\pgfsetfillcolor{currentfill}%
\pgfsetlinewidth{0.000000pt}%
\definecolor{currentstroke}{rgb}{0.212395,0.359683,0.551710}%
\pgfsetstrokecolor{currentstroke}%
\pgfsetdash{}{0pt}%
\pgfpathmoveto{\pgfqpoint{3.629210in}{4.996070in}}%
\pgfpathlineto{\pgfqpoint{3.498080in}{4.903513in}}%
\pgfpathlineto{\pgfqpoint{3.415921in}{4.934934in}}%
\pgfpathclose%
\pgfusepath{fill}%
\end{pgfscope}%
\begin{pgfscope}%
\pgfpathrectangle{\pgfqpoint{0.539299in}{0.078740in}}{\pgfqpoint{7.842520in}{7.842520in}}%
\pgfusepath{clip}%
\pgfsetbuttcap%
\pgfsetroundjoin%
\definecolor{currentfill}{rgb}{0.282623,0.140926,0.457517}%
\pgfsetfillcolor{currentfill}%
\pgfsetlinewidth{0.000000pt}%
\definecolor{currentstroke}{rgb}{0.210503,0.363727,0.552206}%
\pgfsetstrokecolor{currentstroke}%
\pgfsetdash{}{0pt}%
\pgfpathmoveto{\pgfqpoint{6.367154in}{3.364036in}}%
\pgfpathlineto{\pgfqpoint{6.301764in}{3.437824in}}%
\pgfpathlineto{\pgfqpoint{6.229961in}{3.418141in}}%
\pgfpathclose%
\pgfusepath{fill}%
\end{pgfscope}%
\begin{pgfscope}%
\pgfpathrectangle{\pgfqpoint{0.539299in}{0.078740in}}{\pgfqpoint{7.842520in}{7.842520in}}%
\pgfusepath{clip}%
\pgfsetbuttcap%
\pgfsetroundjoin%
\definecolor{currentfill}{rgb}{0.271305,0.019942,0.347269}%
\pgfsetfillcolor{currentfill}%
\pgfsetlinewidth{0.000000pt}%
\definecolor{currentstroke}{rgb}{0.208623,0.367752,0.552675}%
\pgfsetstrokecolor{currentstroke}%
\pgfsetdash{}{0pt}%
\pgfpathmoveto{\pgfqpoint{7.264628in}{3.113510in}}%
\pgfpathlineto{\pgfqpoint{7.333570in}{3.007542in}}%
\pgfpathlineto{\pgfqpoint{7.402794in}{3.055818in}}%
\pgfpathclose%
\pgfusepath{fill}%
\end{pgfscope}%
\begin{pgfscope}%
\pgfpathrectangle{\pgfqpoint{0.539299in}{0.078740in}}{\pgfqpoint{7.842520in}{7.842520in}}%
\pgfusepath{clip}%
\pgfsetbuttcap%
\pgfsetroundjoin%
\definecolor{currentfill}{rgb}{0.182256,0.426184,0.557120}%
\pgfsetfillcolor{currentfill}%
\pgfsetlinewidth{0.000000pt}%
\definecolor{currentstroke}{rgb}{0.206756,0.371758,0.553117}%
\pgfsetstrokecolor{currentstroke}%
\pgfsetdash{}{0pt}%
\pgfpathmoveto{\pgfqpoint{4.515135in}{4.412321in}}%
\pgfpathlineto{\pgfqpoint{4.592860in}{4.318703in}}%
\pgfpathlineto{\pgfqpoint{4.380311in}{4.558363in}}%
\pgfpathclose%
\pgfusepath{fill}%
\end{pgfscope}%
\begin{pgfscope}%
\pgfpathrectangle{\pgfqpoint{0.539299in}{0.078740in}}{\pgfqpoint{7.842520in}{7.842520in}}%
\pgfusepath{clip}%
\pgfsetbuttcap%
\pgfsetroundjoin%
\definecolor{currentfill}{rgb}{0.192357,0.403199,0.555836}%
\pgfsetfillcolor{currentfill}%
\pgfsetlinewidth{0.000000pt}%
\definecolor{currentstroke}{rgb}{0.204903,0.375746,0.553533}%
\pgfsetstrokecolor{currentstroke}%
\pgfsetdash{}{0pt}%
\pgfpathmoveto{\pgfqpoint{4.515135in}{4.412321in}}%
\pgfpathlineto{\pgfqpoint{4.649993in}{4.264866in}}%
\pgfpathlineto{\pgfqpoint{4.592860in}{4.318703in}}%
\pgfpathclose%
\pgfusepath{fill}%
\end{pgfscope}%
\begin{pgfscope}%
\pgfpathrectangle{\pgfqpoint{0.539299in}{0.078740in}}{\pgfqpoint{7.842520in}{7.842520in}}%
\pgfusepath{clip}%
\pgfsetbuttcap%
\pgfsetroundjoin%
\definecolor{currentfill}{rgb}{0.281924,0.089666,0.412415}%
\pgfsetfillcolor{currentfill}%
\pgfsetlinewidth{0.000000pt}%
\definecolor{currentstroke}{rgb}{0.203063,0.379716,0.553925}%
\pgfsetstrokecolor{currentstroke}%
\pgfsetdash{}{0pt}%
\pgfpathmoveto{\pgfqpoint{6.712873in}{3.258583in}}%
\pgfpathlineto{\pgfqpoint{6.850286in}{3.187517in}}%
\pgfpathlineto{\pgfqpoint{6.783027in}{3.277708in}}%
\pgfpathclose%
\pgfusepath{fill}%
\end{pgfscope}%
\begin{pgfscope}%
\pgfpathrectangle{\pgfqpoint{0.539299in}{0.078740in}}{\pgfqpoint{7.842520in}{7.842520in}}%
\pgfusepath{clip}%
\pgfsetbuttcap%
\pgfsetroundjoin%
\definecolor{currentfill}{rgb}{0.280255,0.165693,0.476498}%
\pgfsetfillcolor{currentfill}%
\pgfsetlinewidth{0.000000pt}%
\definecolor{currentstroke}{rgb}{0.201239,0.383670,0.554294}%
\pgfsetstrokecolor{currentstroke}%
\pgfsetdash{}{0pt}%
\pgfpathmoveto{\pgfqpoint{5.956414in}{3.509868in}}%
\pgfpathlineto{\pgfqpoint{5.883392in}{3.490293in}}%
\pgfpathlineto{\pgfqpoint{6.020289in}{3.442779in}}%
\pgfpathclose%
\pgfusepath{fill}%
\end{pgfscope}%
\begin{pgfscope}%
\pgfpathrectangle{\pgfqpoint{0.539299in}{0.078740in}}{\pgfqpoint{7.842520in}{7.842520in}}%
\pgfusepath{clip}%
\pgfsetbuttcap%
\pgfsetroundjoin%
\definecolor{currentfill}{rgb}{0.279566,0.067836,0.391917}%
\pgfsetfillcolor{currentfill}%
\pgfsetlinewidth{0.000000pt}%
\definecolor{currentstroke}{rgb}{0.199430,0.387607,0.554642}%
\pgfsetstrokecolor{currentstroke}%
\pgfsetdash{}{0pt}%
\pgfpathmoveto{\pgfqpoint{6.920184in}{3.211063in}}%
\pgfpathlineto{\pgfqpoint{6.850286in}{3.187517in}}%
\pgfpathlineto{\pgfqpoint{7.057617in}{3.142554in}}%
\pgfpathclose%
\pgfusepath{fill}%
\end{pgfscope}%
\begin{pgfscope}%
\pgfpathrectangle{\pgfqpoint{0.539299in}{0.078740in}}{\pgfqpoint{7.842520in}{7.842520in}}%
\pgfusepath{clip}%
\pgfsetbuttcap%
\pgfsetroundjoin%
\definecolor{currentfill}{rgb}{0.214298,0.355619,0.551184}%
\pgfsetfillcolor{currentfill}%
\pgfsetlinewidth{0.000000pt}%
\definecolor{currentstroke}{rgb}{0.197636,0.391528,0.554969}%
\pgfsetstrokecolor{currentstroke}%
\pgfsetdash{}{0pt}%
\pgfpathmoveto{\pgfqpoint{4.861395in}{4.049469in}}%
\pgfpathlineto{\pgfqpoint{4.649993in}{4.264866in}}%
\pgfpathlineto{\pgfqpoint{4.784895in}{4.122344in}}%
\pgfpathclose%
\pgfusepath{fill}%
\end{pgfscope}%
\begin{pgfscope}%
\pgfpathrectangle{\pgfqpoint{0.539299in}{0.078740in}}{\pgfqpoint{7.842520in}{7.842520in}}%
\pgfusepath{clip}%
\pgfsetbuttcap%
\pgfsetroundjoin%
\definecolor{currentfill}{rgb}{0.257322,0.256130,0.526563}%
\pgfsetfillcolor{currentfill}%
\pgfsetlinewidth{0.000000pt}%
\definecolor{currentstroke}{rgb}{0.195860,0.395433,0.555276}%
\pgfsetstrokecolor{currentstroke}%
\pgfsetdash{}{0pt}%
\pgfpathmoveto{\pgfqpoint{5.190465in}{3.761377in}}%
\pgfpathlineto{\pgfqpoint{5.265591in}{3.732089in}}%
\pgfpathlineto{\pgfqpoint{5.130570in}{3.824189in}}%
\pgfpathclose%
\pgfusepath{fill}%
\end{pgfscope}%
\begin{pgfscope}%
\pgfpathrectangle{\pgfqpoint{0.539299in}{0.078740in}}{\pgfqpoint{7.842520in}{7.842520in}}%
\pgfusepath{clip}%
\pgfsetbuttcap%
\pgfsetroundjoin%
\definecolor{currentfill}{rgb}{0.132444,0.552216,0.553018}%
\pgfsetfillcolor{currentfill}%
\pgfsetlinewidth{0.000000pt}%
\definecolor{currentstroke}{rgb}{0.194100,0.399323,0.555565}%
\pgfsetstrokecolor{currentstroke}%
\pgfsetdash{}{0pt}%
\pgfpathmoveto{\pgfqpoint{3.976815in}{4.900063in}}%
\pgfpathlineto{\pgfqpoint{3.843276in}{4.946277in}}%
\pgfpathlineto{\pgfqpoint{3.761962in}{5.020676in}}%
\pgfpathclose%
\pgfusepath{fill}%
\end{pgfscope}%
\begin{pgfscope}%
\pgfpathrectangle{\pgfqpoint{0.539299in}{0.078740in}}{\pgfqpoint{7.842520in}{7.842520in}}%
\pgfusepath{clip}%
\pgfsetbuttcap%
\pgfsetroundjoin%
\definecolor{currentfill}{rgb}{0.283229,0.120777,0.440584}%
\pgfsetfillcolor{currentfill}%
\pgfsetlinewidth{0.000000pt}%
\definecolor{currentstroke}{rgb}{0.192357,0.403199,0.555836}%
\pgfsetstrokecolor{currentstroke}%
\pgfsetdash{}{0pt}%
\pgfpathmoveto{\pgfqpoint{6.504552in}{3.303089in}}%
\pgfpathlineto{\pgfqpoint{6.575647in}{3.324679in}}%
\pgfpathlineto{\pgfqpoint{6.438604in}{3.384604in}}%
\pgfpathclose%
\pgfusepath{fill}%
\end{pgfscope}%
\begin{pgfscope}%
\pgfpathrectangle{\pgfqpoint{0.539299in}{0.078740in}}{\pgfqpoint{7.842520in}{7.842520in}}%
\pgfusepath{clip}%
\pgfsetbuttcap%
\pgfsetroundjoin%
\definecolor{currentfill}{rgb}{0.272594,0.025563,0.353093}%
\pgfsetfillcolor{currentfill}%
\pgfsetlinewidth{0.000000pt}%
\definecolor{currentstroke}{rgb}{0.190631,0.407061,0.556089}%
\pgfsetstrokecolor{currentstroke}%
\pgfsetdash{}{0pt}%
\pgfpathmoveto{\pgfqpoint{7.195388in}{3.074084in}}%
\pgfpathlineto{\pgfqpoint{7.333570in}{3.007542in}}%
\pgfpathlineto{\pgfqpoint{7.264628in}{3.113510in}}%
\pgfpathclose%
\pgfusepath{fill}%
\end{pgfscope}%
\begin{pgfscope}%
\pgfpathrectangle{\pgfqpoint{0.539299in}{0.078740in}}{\pgfqpoint{7.842520in}{7.842520in}}%
\pgfusepath{clip}%
\pgfsetbuttcap%
\pgfsetroundjoin%
\definecolor{currentfill}{rgb}{0.223925,0.334994,0.548053}%
\pgfsetfillcolor{currentfill}%
\pgfsetlinewidth{0.000000pt}%
\definecolor{currentstroke}{rgb}{0.188923,0.410910,0.556326}%
\pgfsetstrokecolor{currentstroke}%
\pgfsetdash{}{0pt}%
\pgfpathmoveto{\pgfqpoint{4.784895in}{4.122344in}}%
\pgfpathlineto{\pgfqpoint{4.919892in}{3.989315in}}%
\pgfpathlineto{\pgfqpoint{4.861395in}{4.049469in}}%
\pgfpathclose%
\pgfusepath{fill}%
\end{pgfscope}%
\begin{pgfscope}%
\pgfpathrectangle{\pgfqpoint{0.539299in}{0.078740in}}{\pgfqpoint{7.842520in}{7.842520in}}%
\pgfusepath{clip}%
\pgfsetbuttcap%
\pgfsetroundjoin%
\definecolor{currentfill}{rgb}{0.225863,0.330805,0.547314}%
\pgfsetfillcolor{currentfill}%
\pgfsetlinewidth{0.000000pt}%
\definecolor{currentstroke}{rgb}{0.187231,0.414746,0.556547}%
\pgfsetstrokecolor{currentstroke}%
\pgfsetdash{}{0pt}%
\pgfpathmoveto{\pgfqpoint{2.710747in}{3.908413in}}%
\pgfpathlineto{\pgfqpoint{2.627589in}{3.858151in}}%
\pgfpathlineto{\pgfqpoint{2.746363in}{4.336760in}}%
\pgfpathclose%
\pgfusepath{fill}%
\end{pgfscope}%
\begin{pgfscope}%
\pgfpathrectangle{\pgfqpoint{0.539299in}{0.078740in}}{\pgfqpoint{7.842520in}{7.842520in}}%
\pgfusepath{clip}%
\pgfsetbuttcap%
\pgfsetroundjoin%
\definecolor{currentfill}{rgb}{0.129933,0.559582,0.551864}%
\pgfsetfillcolor{currentfill}%
\pgfsetlinewidth{0.000000pt}%
\definecolor{currentstroke}{rgb}{0.185556,0.418570,0.556753}%
\pgfsetstrokecolor{currentstroke}%
\pgfsetdash{}{0pt}%
\pgfpathmoveto{\pgfqpoint{3.629210in}{4.996070in}}%
\pgfpathlineto{\pgfqpoint{3.761962in}{5.020676in}}%
\pgfpathlineto{\pgfqpoint{3.843276in}{4.946277in}}%
\pgfpathclose%
\pgfusepath{fill}%
\end{pgfscope}%
\begin{pgfscope}%
\pgfpathrectangle{\pgfqpoint{0.539299in}{0.078740in}}{\pgfqpoint{7.842520in}{7.842520in}}%
\pgfusepath{clip}%
\pgfsetbuttcap%
\pgfsetroundjoin%
\definecolor{currentfill}{rgb}{0.143343,0.522773,0.556295}%
\pgfsetfillcolor{currentfill}%
\pgfsetlinewidth{0.000000pt}%
\definecolor{currentstroke}{rgb}{0.183898,0.422383,0.556944}%
\pgfsetstrokecolor{currentstroke}%
\pgfsetdash{}{0pt}%
\pgfpathmoveto{\pgfqpoint{3.287152in}{4.741607in}}%
\pgfpathlineto{\pgfqpoint{3.204675in}{4.746147in}}%
\pgfpathlineto{\pgfqpoint{3.333165in}{4.962328in}}%
\pgfpathclose%
\pgfusepath{fill}%
\end{pgfscope}%
\begin{pgfscope}%
\pgfpathrectangle{\pgfqpoint{0.539299in}{0.078740in}}{\pgfqpoint{7.842520in}{7.842520in}}%
\pgfusepath{clip}%
\pgfsetbuttcap%
\pgfsetroundjoin%
\definecolor{currentfill}{rgb}{0.267004,0.004874,0.329415}%
\pgfsetfillcolor{currentfill}%
\pgfsetlinewidth{0.000000pt}%
\definecolor{currentstroke}{rgb}{0.182256,0.426184,0.557120}%
\pgfsetstrokecolor{currentstroke}%
\pgfsetdash{}{0pt}%
\pgfpathmoveto{\pgfqpoint{7.541482in}{3.001589in}}%
\pgfpathlineto{\pgfqpoint{7.680757in}{2.951802in}}%
\pgfpathlineto{\pgfqpoint{7.749729in}{3.018900in}}%
\pgfpathclose%
\pgfusepath{fill}%
\end{pgfscope}%
\begin{pgfscope}%
\pgfpathrectangle{\pgfqpoint{0.539299in}{0.078740in}}{\pgfqpoint{7.842520in}{7.842520in}}%
\pgfusepath{clip}%
\pgfsetbuttcap%
\pgfsetroundjoin%
\definecolor{currentfill}{rgb}{0.262138,0.242286,0.520837}%
\pgfsetfillcolor{currentfill}%
\pgfsetlinewidth{0.000000pt}%
\definecolor{currentstroke}{rgb}{0.180629,0.429975,0.557282}%
\pgfsetstrokecolor{currentstroke}%
\pgfsetdash{}{0pt}%
\pgfpathmoveto{\pgfqpoint{5.400994in}{3.652853in}}%
\pgfpathlineto{\pgfqpoint{5.265591in}{3.732089in}}%
\pgfpathlineto{\pgfqpoint{5.190465in}{3.761377in}}%
\pgfpathclose%
\pgfusepath{fill}%
\end{pgfscope}%
\begin{pgfscope}%
\pgfpathrectangle{\pgfqpoint{0.539299in}{0.078740in}}{\pgfqpoint{7.842520in}{7.842520in}}%
\pgfusepath{clip}%
\pgfsetbuttcap%
\pgfsetroundjoin%
\definecolor{currentfill}{rgb}{0.281412,0.155834,0.469201}%
\pgfsetfillcolor{currentfill}%
\pgfsetlinewidth{0.000000pt}%
\definecolor{currentstroke}{rgb}{0.179019,0.433756,0.557430}%
\pgfsetstrokecolor{currentstroke}%
\pgfsetdash{}{0pt}%
\pgfpathmoveto{\pgfqpoint{6.229961in}{3.418141in}}%
\pgfpathlineto{\pgfqpoint{6.093026in}{3.466206in}}%
\pgfpathlineto{\pgfqpoint{6.157532in}{3.392366in}}%
\pgfpathclose%
\pgfusepath{fill}%
\end{pgfscope}%
\begin{pgfscope}%
\pgfpathrectangle{\pgfqpoint{0.539299in}{0.078740in}}{\pgfqpoint{7.842520in}{7.842520in}}%
\pgfusepath{clip}%
\pgfsetbuttcap%
\pgfsetroundjoin%
\definecolor{currentfill}{rgb}{0.137770,0.537492,0.554906}%
\pgfsetfillcolor{currentfill}%
\pgfsetlinewidth{0.000000pt}%
\definecolor{currentstroke}{rgb}{0.177423,0.437527,0.557565}%
\pgfsetstrokecolor{currentstroke}%
\pgfsetdash{}{0pt}%
\pgfpathmoveto{\pgfqpoint{3.333165in}{4.962328in}}%
\pgfpathlineto{\pgfqpoint{3.415921in}{4.934934in}}%
\pgfpathlineto{\pgfqpoint{3.287152in}{4.741607in}}%
\pgfpathclose%
\pgfusepath{fill}%
\end{pgfscope}%
\begin{pgfscope}%
\pgfpathrectangle{\pgfqpoint{0.539299in}{0.078740in}}{\pgfqpoint{7.842520in}{7.842520in}}%
\pgfusepath{clip}%
\pgfsetbuttcap%
\pgfsetroundjoin%
\definecolor{currentfill}{rgb}{0.283072,0.130895,0.449241}%
\pgfsetfillcolor{currentfill}%
\pgfsetlinewidth{0.000000pt}%
\definecolor{currentstroke}{rgb}{0.175841,0.441290,0.557685}%
\pgfsetstrokecolor{currentstroke}%
\pgfsetdash{}{0pt}%
\pgfpathmoveto{\pgfqpoint{6.438604in}{3.384604in}}%
\pgfpathlineto{\pgfqpoint{6.367154in}{3.364036in}}%
\pgfpathlineto{\pgfqpoint{6.504552in}{3.303089in}}%
\pgfpathclose%
\pgfusepath{fill}%
\end{pgfscope}%
\begin{pgfscope}%
\pgfpathrectangle{\pgfqpoint{0.539299in}{0.078740in}}{\pgfqpoint{7.842520in}{7.842520in}}%
\pgfusepath{clip}%
\pgfsetbuttcap%
\pgfsetroundjoin%
\definecolor{currentfill}{rgb}{0.241237,0.296485,0.539709}%
\pgfsetfillcolor{currentfill}%
\pgfsetlinewidth{0.000000pt}%
\definecolor{currentstroke}{rgb}{0.174274,0.445044,0.557792}%
\pgfsetstrokecolor{currentstroke}%
\pgfsetdash{}{0pt}%
\pgfpathmoveto{\pgfqpoint{5.130570in}{3.824189in}}%
\pgfpathlineto{\pgfqpoint{4.919892in}{3.989315in}}%
\pgfpathlineto{\pgfqpoint{5.055056in}{3.868593in}}%
\pgfpathclose%
\pgfusepath{fill}%
\end{pgfscope}%
\begin{pgfscope}%
\pgfpathrectangle{\pgfqpoint{0.539299in}{0.078740in}}{\pgfqpoint{7.842520in}{7.842520in}}%
\pgfusepath{clip}%
\pgfsetbuttcap%
\pgfsetroundjoin%
\definecolor{currentfill}{rgb}{0.268510,0.009605,0.335427}%
\pgfsetfillcolor{currentfill}%
\pgfsetlinewidth{0.000000pt}%
\definecolor{currentstroke}{rgb}{0.172719,0.448791,0.557885}%
\pgfsetstrokecolor{currentstroke}%
\pgfsetdash{}{0pt}%
\pgfpathmoveto{\pgfqpoint{7.472249in}{2.944684in}}%
\pgfpathlineto{\pgfqpoint{7.541482in}{3.001589in}}%
\pgfpathlineto{\pgfqpoint{7.402794in}{3.055818in}}%
\pgfpathclose%
\pgfusepath{fill}%
\end{pgfscope}%
\begin{pgfscope}%
\pgfpathrectangle{\pgfqpoint{0.539299in}{0.078740in}}{\pgfqpoint{7.842520in}{7.842520in}}%
\pgfusepath{clip}%
\pgfsetbuttcap%
\pgfsetroundjoin%
\definecolor{currentfill}{rgb}{0.283197,0.115680,0.436115}%
\pgfsetfillcolor{currentfill}%
\pgfsetlinewidth{0.000000pt}%
\definecolor{currentstroke}{rgb}{0.171176,0.452530,0.557965}%
\pgfsetstrokecolor{currentstroke}%
\pgfsetdash{}{0pt}%
\pgfpathmoveto{\pgfqpoint{6.712873in}{3.258583in}}%
\pgfpathlineto{\pgfqpoint{6.575647in}{3.324679in}}%
\pgfpathlineto{\pgfqpoint{6.504552in}{3.303089in}}%
\pgfpathclose%
\pgfusepath{fill}%
\end{pgfscope}%
\begin{pgfscope}%
\pgfpathrectangle{\pgfqpoint{0.539299in}{0.078740in}}{\pgfqpoint{7.842520in}{7.842520in}}%
\pgfusepath{clip}%
\pgfsetbuttcap%
\pgfsetroundjoin%
\definecolor{currentfill}{rgb}{0.275191,0.194905,0.496005}%
\pgfsetfillcolor{currentfill}%
\pgfsetlinewidth{0.000000pt}%
\definecolor{currentstroke}{rgb}{0.169646,0.456262,0.558030}%
\pgfsetstrokecolor{currentstroke}%
\pgfsetdash{}{0pt}%
\pgfpathmoveto{\pgfqpoint{5.673111in}{3.524234in}}%
\pgfpathlineto{\pgfqpoint{5.746902in}{3.537634in}}%
\pgfpathlineto{\pgfqpoint{5.536826in}{3.584474in}}%
\pgfpathclose%
\pgfusepath{fill}%
\end{pgfscope}%
\begin{pgfscope}%
\pgfpathrectangle{\pgfqpoint{0.539299in}{0.078740in}}{\pgfqpoint{7.842520in}{7.842520in}}%
\pgfusepath{clip}%
\pgfsetbuttcap%
\pgfsetroundjoin%
\definecolor{currentfill}{rgb}{0.278012,0.180367,0.486697}%
\pgfsetfillcolor{currentfill}%
\pgfsetlinewidth{0.000000pt}%
\definecolor{currentstroke}{rgb}{0.168126,0.459988,0.558082}%
\pgfsetstrokecolor{currentstroke}%
\pgfsetdash{}{0pt}%
\pgfpathmoveto{\pgfqpoint{5.809842in}{3.469040in}}%
\pgfpathlineto{\pgfqpoint{5.883392in}{3.490293in}}%
\pgfpathlineto{\pgfqpoint{5.746902in}{3.537634in}}%
\pgfpathclose%
\pgfusepath{fill}%
\end{pgfscope}%
\begin{pgfscope}%
\pgfpathrectangle{\pgfqpoint{0.539299in}{0.078740in}}{\pgfqpoint{7.842520in}{7.842520in}}%
\pgfusepath{clip}%
\pgfsetbuttcap%
\pgfsetroundjoin%
\definecolor{currentfill}{rgb}{0.248629,0.278775,0.534556}%
\pgfsetfillcolor{currentfill}%
\pgfsetlinewidth{0.000000pt}%
\definecolor{currentstroke}{rgb}{0.166617,0.463708,0.558119}%
\pgfsetstrokecolor{currentstroke}%
\pgfsetdash{}{0pt}%
\pgfpathmoveto{\pgfqpoint{5.130570in}{3.824189in}}%
\pgfpathlineto{\pgfqpoint{5.055056in}{3.868593in}}%
\pgfpathlineto{\pgfqpoint{5.190465in}{3.761377in}}%
\pgfpathclose%
\pgfusepath{fill}%
\end{pgfscope}%
\begin{pgfscope}%
\pgfpathrectangle{\pgfqpoint{0.539299in}{0.078740in}}{\pgfqpoint{7.842520in}{7.842520in}}%
\pgfusepath{clip}%
\pgfsetbuttcap%
\pgfsetroundjoin%
\definecolor{currentfill}{rgb}{0.280255,0.165693,0.476498}%
\pgfsetfillcolor{currentfill}%
\pgfsetlinewidth{0.000000pt}%
\definecolor{currentstroke}{rgb}{0.165117,0.467423,0.558141}%
\pgfsetstrokecolor{currentstroke}%
\pgfsetdash{}{0pt}%
\pgfpathmoveto{\pgfqpoint{6.093026in}{3.466206in}}%
\pgfpathlineto{\pgfqpoint{6.020289in}{3.442779in}}%
\pgfpathlineto{\pgfqpoint{6.157532in}{3.392366in}}%
\pgfpathclose%
\pgfusepath{fill}%
\end{pgfscope}%
\begin{pgfscope}%
\pgfpathrectangle{\pgfqpoint{0.539299in}{0.078740in}}{\pgfqpoint{7.842520in}{7.842520in}}%
\pgfusepath{clip}%
\pgfsetbuttcap%
\pgfsetroundjoin%
\definecolor{currentfill}{rgb}{0.280267,0.073417,0.397163}%
\pgfsetfillcolor{currentfill}%
\pgfsetlinewidth{0.000000pt}%
\definecolor{currentstroke}{rgb}{0.163625,0.471133,0.558148}%
\pgfsetstrokecolor{currentstroke}%
\pgfsetdash{}{0pt}%
\pgfpathmoveto{\pgfqpoint{6.850286in}{3.187517in}}%
\pgfpathlineto{\pgfqpoint{6.987917in}{3.113178in}}%
\pgfpathlineto{\pgfqpoint{7.057617in}{3.142554in}}%
\pgfpathclose%
\pgfusepath{fill}%
\end{pgfscope}%
\begin{pgfscope}%
\pgfpathrectangle{\pgfqpoint{0.539299in}{0.078740in}}{\pgfqpoint{7.842520in}{7.842520in}}%
\pgfusepath{clip}%
\pgfsetbuttcap%
\pgfsetroundjoin%
\definecolor{currentfill}{rgb}{0.269944,0.014625,0.341379}%
\pgfsetfillcolor{currentfill}%
\pgfsetlinewidth{0.000000pt}%
\definecolor{currentstroke}{rgb}{0.162142,0.474838,0.558140}%
\pgfsetstrokecolor{currentstroke}%
\pgfsetdash{}{0pt}%
\pgfpathmoveto{\pgfqpoint{7.402794in}{3.055818in}}%
\pgfpathlineto{\pgfqpoint{7.333570in}{3.007542in}}%
\pgfpathlineto{\pgfqpoint{7.472249in}{2.944684in}}%
\pgfpathclose%
\pgfusepath{fill}%
\end{pgfscope}%
\begin{pgfscope}%
\pgfpathrectangle{\pgfqpoint{0.539299in}{0.078740in}}{\pgfqpoint{7.842520in}{7.842520in}}%
\pgfusepath{clip}%
\pgfsetbuttcap%
\pgfsetroundjoin%
\definecolor{currentfill}{rgb}{0.277018,0.050344,0.375715}%
\pgfsetfillcolor{currentfill}%
\pgfsetlinewidth{0.000000pt}%
\definecolor{currentstroke}{rgb}{0.160665,0.478540,0.558115}%
\pgfsetstrokecolor{currentstroke}%
\pgfsetdash{}{0pt}%
\pgfpathmoveto{\pgfqpoint{7.195388in}{3.074084in}}%
\pgfpathlineto{\pgfqpoint{7.057617in}{3.142554in}}%
\pgfpathlineto{\pgfqpoint{7.125818in}{3.037569in}}%
\pgfpathclose%
\pgfusepath{fill}%
\end{pgfscope}%
\begin{pgfscope}%
\pgfpathrectangle{\pgfqpoint{0.539299in}{0.078740in}}{\pgfqpoint{7.842520in}{7.842520in}}%
\pgfusepath{clip}%
\pgfsetbuttcap%
\pgfsetroundjoin%
\definecolor{currentfill}{rgb}{0.168126,0.459988,0.558082}%
\pgfsetfillcolor{currentfill}%
\pgfsetlinewidth{0.000000pt}%
\definecolor{currentstroke}{rgb}{0.159194,0.482237,0.558073}%
\pgfsetstrokecolor{currentstroke}%
\pgfsetdash{}{0pt}%
\pgfpathmoveto{\pgfqpoint{3.038103in}{4.744324in}}%
\pgfpathlineto{\pgfqpoint{2.996771in}{4.420700in}}%
\pgfpathlineto{\pgfqpoint{2.913780in}{4.395967in}}%
\pgfpathclose%
\pgfusepath{fill}%
\end{pgfscope}%
\begin{pgfscope}%
\pgfpathrectangle{\pgfqpoint{0.539299in}{0.078740in}}{\pgfqpoint{7.842520in}{7.842520in}}%
\pgfusepath{clip}%
\pgfsetbuttcap%
\pgfsetroundjoin%
\definecolor{currentfill}{rgb}{0.281887,0.150881,0.465405}%
\pgfsetfillcolor{currentfill}%
\pgfsetlinewidth{0.000000pt}%
\definecolor{currentstroke}{rgb}{0.157729,0.485932,0.558013}%
\pgfsetstrokecolor{currentstroke}%
\pgfsetdash{}{0pt}%
\pgfpathmoveto{\pgfqpoint{6.157532in}{3.392366in}}%
\pgfpathlineto{\pgfqpoint{6.367154in}{3.364036in}}%
\pgfpathlineto{\pgfqpoint{6.229961in}{3.418141in}}%
\pgfpathclose%
\pgfusepath{fill}%
\end{pgfscope}%
\begin{pgfscope}%
\pgfpathrectangle{\pgfqpoint{0.539299in}{0.078740in}}{\pgfqpoint{7.842520in}{7.842520in}}%
\pgfusepath{clip}%
\pgfsetbuttcap%
\pgfsetroundjoin%
\definecolor{currentfill}{rgb}{0.266580,0.228262,0.514349}%
\pgfsetfillcolor{currentfill}%
\pgfsetlinewidth{0.000000pt}%
\definecolor{currentstroke}{rgb}{0.156270,0.489624,0.557936}%
\pgfsetstrokecolor{currentstroke}%
\pgfsetdash{}{0pt}%
\pgfpathmoveto{\pgfqpoint{5.536826in}{3.584474in}}%
\pgfpathlineto{\pgfqpoint{5.400994in}{3.652853in}}%
\pgfpathlineto{\pgfqpoint{5.326196in}{3.667457in}}%
\pgfpathclose%
\pgfusepath{fill}%
\end{pgfscope}%
\begin{pgfscope}%
\pgfpathrectangle{\pgfqpoint{0.539299in}{0.078740in}}{\pgfqpoint{7.842520in}{7.842520in}}%
\pgfusepath{clip}%
\pgfsetbuttcap%
\pgfsetroundjoin%
\definecolor{currentfill}{rgb}{0.136408,0.541173,0.554483}%
\pgfsetfillcolor{currentfill}%
\pgfsetlinewidth{0.000000pt}%
\definecolor{currentstroke}{rgb}{0.154815,0.493313,0.557840}%
\pgfsetstrokecolor{currentstroke}%
\pgfsetdash{}{0pt}%
\pgfpathmoveto{\pgfqpoint{3.976815in}{4.900063in}}%
\pgfpathlineto{\pgfqpoint{4.030569in}{4.909838in}}%
\pgfpathlineto{\pgfqpoint{4.111002in}{4.812426in}}%
\pgfpathclose%
\pgfusepath{fill}%
\end{pgfscope}%
\begin{pgfscope}%
\pgfpathrectangle{\pgfqpoint{0.539299in}{0.078740in}}{\pgfqpoint{7.842520in}{7.842520in}}%
\pgfusepath{clip}%
\pgfsetbuttcap%
\pgfsetroundjoin%
\definecolor{currentfill}{rgb}{0.268510,0.009605,0.335427}%
\pgfsetfillcolor{currentfill}%
\pgfsetlinewidth{0.000000pt}%
\definecolor{currentstroke}{rgb}{0.153364,0.497000,0.557724}%
\pgfsetstrokecolor{currentstroke}%
\pgfsetdash{}{0pt}%
\pgfpathmoveto{\pgfqpoint{7.680757in}{2.951802in}}%
\pgfpathlineto{\pgfqpoint{7.541482in}{3.001589in}}%
\pgfpathlineto{\pgfqpoint{7.472249in}{2.944684in}}%
\pgfpathclose%
\pgfusepath{fill}%
\end{pgfscope}%
\begin{pgfscope}%
\pgfpathrectangle{\pgfqpoint{0.539299in}{0.078740in}}{\pgfqpoint{7.842520in}{7.842520in}}%
\pgfusepath{clip}%
\pgfsetbuttcap%
\pgfsetroundjoin%
\definecolor{currentfill}{rgb}{0.276194,0.190074,0.493001}%
\pgfsetfillcolor{currentfill}%
\pgfsetlinewidth{0.000000pt}%
\definecolor{currentstroke}{rgb}{0.151918,0.500685,0.557587}%
\pgfsetstrokecolor{currentstroke}%
\pgfsetdash{}{0pt}%
\pgfpathmoveto{\pgfqpoint{5.673111in}{3.524234in}}%
\pgfpathlineto{\pgfqpoint{5.809842in}{3.469040in}}%
\pgfpathlineto{\pgfqpoint{5.746902in}{3.537634in}}%
\pgfpathclose%
\pgfusepath{fill}%
\end{pgfscope}%
\begin{pgfscope}%
\pgfpathrectangle{\pgfqpoint{0.539299in}{0.078740in}}{\pgfqpoint{7.842520in}{7.842520in}}%
\pgfusepath{clip}%
\pgfsetbuttcap%
\pgfsetroundjoin%
\definecolor{currentfill}{rgb}{0.126453,0.570633,0.549841}%
\pgfsetfillcolor{currentfill}%
\pgfsetlinewidth{0.000000pt}%
\definecolor{currentstroke}{rgb}{0.150476,0.504369,0.557430}%
\pgfsetstrokecolor{currentstroke}%
\pgfsetdash{}{0pt}%
\pgfpathmoveto{\pgfqpoint{3.547026in}{5.048670in}}%
\pgfpathlineto{\pgfqpoint{3.629210in}{4.996070in}}%
\pgfpathlineto{\pgfqpoint{3.415921in}{4.934934in}}%
\pgfpathclose%
\pgfusepath{fill}%
\end{pgfscope}%
\begin{pgfscope}%
\pgfpathrectangle{\pgfqpoint{0.539299in}{0.078740in}}{\pgfqpoint{7.842520in}{7.842520in}}%
\pgfusepath{clip}%
\pgfsetbuttcap%
\pgfsetroundjoin%
\definecolor{currentfill}{rgb}{0.201239,0.383670,0.554294}%
\pgfsetfillcolor{currentfill}%
\pgfsetlinewidth{0.000000pt}%
\definecolor{currentstroke}{rgb}{0.149039,0.508051,0.557250}%
\pgfsetstrokecolor{currentstroke}%
\pgfsetdash{}{0pt}%
\pgfpathmoveto{\pgfqpoint{2.830306in}{4.368134in}}%
\pgfpathlineto{\pgfqpoint{2.710747in}{3.908413in}}%
\pgfpathlineto{\pgfqpoint{2.746363in}{4.336760in}}%
\pgfpathclose%
\pgfusepath{fill}%
\end{pgfscope}%
\begin{pgfscope}%
\pgfpathrectangle{\pgfqpoint{0.539299in}{0.078740in}}{\pgfqpoint{7.842520in}{7.842520in}}%
\pgfusepath{clip}%
\pgfsetbuttcap%
\pgfsetroundjoin%
\definecolor{currentfill}{rgb}{0.282910,0.105393,0.426902}%
\pgfsetfillcolor{currentfill}%
\pgfsetlinewidth{0.000000pt}%
\definecolor{currentstroke}{rgb}{0.147607,0.511733,0.557049}%
\pgfsetstrokecolor{currentstroke}%
\pgfsetdash{}{0pt}%
\pgfpathmoveto{\pgfqpoint{6.712873in}{3.258583in}}%
\pgfpathlineto{\pgfqpoint{6.642120in}{3.235328in}}%
\pgfpathlineto{\pgfqpoint{6.850286in}{3.187517in}}%
\pgfpathclose%
\pgfusepath{fill}%
\end{pgfscope}%
\begin{pgfscope}%
\pgfpathrectangle{\pgfqpoint{0.539299in}{0.078740in}}{\pgfqpoint{7.842520in}{7.842520in}}%
\pgfusepath{clip}%
\pgfsetbuttcap%
\pgfsetroundjoin%
\definecolor{currentfill}{rgb}{0.278791,0.062145,0.386592}%
\pgfsetfillcolor{currentfill}%
\pgfsetlinewidth{0.000000pt}%
\definecolor{currentstroke}{rgb}{0.146180,0.515413,0.556823}%
\pgfsetstrokecolor{currentstroke}%
\pgfsetdash{}{0pt}%
\pgfpathmoveto{\pgfqpoint{7.125818in}{3.037569in}}%
\pgfpathlineto{\pgfqpoint{7.057617in}{3.142554in}}%
\pgfpathlineto{\pgfqpoint{6.987917in}{3.113178in}}%
\pgfpathclose%
\pgfusepath{fill}%
\end{pgfscope}%
\begin{pgfscope}%
\pgfpathrectangle{\pgfqpoint{0.539299in}{0.078740in}}{\pgfqpoint{7.842520in}{7.842520in}}%
\pgfusepath{clip}%
\pgfsetbuttcap%
\pgfsetroundjoin%
\definecolor{currentfill}{rgb}{0.127568,0.566949,0.550556}%
\pgfsetfillcolor{currentfill}%
\pgfsetlinewidth{0.000000pt}%
\definecolor{currentstroke}{rgb}{0.144759,0.519093,0.556572}%
\pgfsetstrokecolor{currentstroke}%
\pgfsetdash{}{0pt}%
\pgfpathmoveto{\pgfqpoint{3.761962in}{5.020676in}}%
\pgfpathlineto{\pgfqpoint{3.895878in}{4.988126in}}%
\pgfpathlineto{\pgfqpoint{3.976815in}{4.900063in}}%
\pgfpathclose%
\pgfusepath{fill}%
\end{pgfscope}%
\begin{pgfscope}%
\pgfpathrectangle{\pgfqpoint{0.539299in}{0.078740in}}{\pgfqpoint{7.842520in}{7.842520in}}%
\pgfusepath{clip}%
\pgfsetbuttcap%
\pgfsetroundjoin%
\definecolor{currentfill}{rgb}{0.140536,0.530132,0.555659}%
\pgfsetfillcolor{currentfill}%
\pgfsetlinewidth{0.000000pt}%
\definecolor{currentstroke}{rgb}{0.143343,0.522773,0.556295}%
\pgfsetstrokecolor{currentstroke}%
\pgfsetdash{}{0pt}%
\pgfpathmoveto{\pgfqpoint{4.245563in}{4.694927in}}%
\pgfpathlineto{\pgfqpoint{4.111002in}{4.812426in}}%
\pgfpathlineto{\pgfqpoint{4.030569in}{4.909838in}}%
\pgfpathclose%
\pgfusepath{fill}%
\end{pgfscope}%
\begin{pgfscope}%
\pgfpathrectangle{\pgfqpoint{0.539299in}{0.078740in}}{\pgfqpoint{7.842520in}{7.842520in}}%
\pgfusepath{clip}%
\pgfsetbuttcap%
\pgfsetroundjoin%
\definecolor{currentfill}{rgb}{0.274952,0.037752,0.364543}%
\pgfsetfillcolor{currentfill}%
\pgfsetlinewidth{0.000000pt}%
\definecolor{currentstroke}{rgb}{0.141935,0.526453,0.555991}%
\pgfsetstrokecolor{currentstroke}%
\pgfsetdash{}{0pt}%
\pgfpathmoveto{\pgfqpoint{7.125818in}{3.037569in}}%
\pgfpathlineto{\pgfqpoint{7.333570in}{3.007542in}}%
\pgfpathlineto{\pgfqpoint{7.195388in}{3.074084in}}%
\pgfpathclose%
\pgfusepath{fill}%
\end{pgfscope}%
\begin{pgfscope}%
\pgfpathrectangle{\pgfqpoint{0.539299in}{0.078740in}}{\pgfqpoint{7.842520in}{7.842520in}}%
\pgfusepath{clip}%
\pgfsetbuttcap%
\pgfsetroundjoin%
\definecolor{currentfill}{rgb}{0.258965,0.251537,0.524736}%
\pgfsetfillcolor{currentfill}%
\pgfsetlinewidth{0.000000pt}%
\definecolor{currentstroke}{rgb}{0.140536,0.530132,0.555659}%
\pgfsetstrokecolor{currentstroke}%
\pgfsetdash{}{0pt}%
\pgfpathmoveto{\pgfqpoint{5.190465in}{3.761377in}}%
\pgfpathlineto{\pgfqpoint{5.326196in}{3.667457in}}%
\pgfpathlineto{\pgfqpoint{5.400994in}{3.652853in}}%
\pgfpathclose%
\pgfusepath{fill}%
\end{pgfscope}%
\begin{pgfscope}%
\pgfpathrectangle{\pgfqpoint{0.539299in}{0.078740in}}{\pgfqpoint{7.842520in}{7.842520in}}%
\pgfusepath{clip}%
\pgfsetbuttcap%
\pgfsetroundjoin%
\definecolor{currentfill}{rgb}{0.283229,0.120777,0.440584}%
\pgfsetfillcolor{currentfill}%
\pgfsetlinewidth{0.000000pt}%
\definecolor{currentstroke}{rgb}{0.139147,0.533812,0.555298}%
\pgfsetstrokecolor{currentstroke}%
\pgfsetdash{}{0pt}%
\pgfpathmoveto{\pgfqpoint{6.504552in}{3.303089in}}%
\pgfpathlineto{\pgfqpoint{6.642120in}{3.235328in}}%
\pgfpathlineto{\pgfqpoint{6.712873in}{3.258583in}}%
\pgfpathclose%
\pgfusepath{fill}%
\end{pgfscope}%
\begin{pgfscope}%
\pgfpathrectangle{\pgfqpoint{0.539299in}{0.078740in}}{\pgfqpoint{7.842520in}{7.842520in}}%
\pgfusepath{clip}%
\pgfsetbuttcap%
\pgfsetroundjoin%
\definecolor{currentfill}{rgb}{0.278012,0.180367,0.486697}%
\pgfsetfillcolor{currentfill}%
\pgfsetlinewidth{0.000000pt}%
\definecolor{currentstroke}{rgb}{0.137770,0.537492,0.554906}%
\pgfsetstrokecolor{currentstroke}%
\pgfsetdash{}{0pt}%
\pgfpathmoveto{\pgfqpoint{5.946992in}{3.415751in}}%
\pgfpathlineto{\pgfqpoint{6.020289in}{3.442779in}}%
\pgfpathlineto{\pgfqpoint{5.883392in}{3.490293in}}%
\pgfpathclose%
\pgfusepath{fill}%
\end{pgfscope}%
\begin{pgfscope}%
\pgfpathrectangle{\pgfqpoint{0.539299in}{0.078740in}}{\pgfqpoint{7.842520in}{7.842520in}}%
\pgfusepath{clip}%
\pgfsetbuttcap%
\pgfsetroundjoin%
\definecolor{currentfill}{rgb}{0.154815,0.493313,0.557840}%
\pgfsetfillcolor{currentfill}%
\pgfsetlinewidth{0.000000pt}%
\definecolor{currentstroke}{rgb}{0.136408,0.541173,0.554483}%
\pgfsetstrokecolor{currentstroke}%
\pgfsetdash{}{0pt}%
\pgfpathmoveto{\pgfqpoint{3.121655in}{4.747134in}}%
\pgfpathlineto{\pgfqpoint{2.996771in}{4.420700in}}%
\pgfpathlineto{\pgfqpoint{3.038103in}{4.744324in}}%
\pgfpathclose%
\pgfusepath{fill}%
\end{pgfscope}%
\begin{pgfscope}%
\pgfpathrectangle{\pgfqpoint{0.539299in}{0.078740in}}{\pgfqpoint{7.842520in}{7.842520in}}%
\pgfusepath{clip}%
\pgfsetbuttcap%
\pgfsetroundjoin%
\definecolor{currentfill}{rgb}{0.150476,0.504369,0.557430}%
\pgfsetfillcolor{currentfill}%
\pgfsetlinewidth{0.000000pt}%
\definecolor{currentstroke}{rgb}{0.135066,0.544853,0.554029}%
\pgfsetstrokecolor{currentstroke}%
\pgfsetdash{}{0pt}%
\pgfpathmoveto{\pgfqpoint{4.380311in}{4.558363in}}%
\pgfpathlineto{\pgfqpoint{4.245563in}{4.694927in}}%
\pgfpathlineto{\pgfqpoint{4.165728in}{4.797221in}}%
\pgfpathclose%
\pgfusepath{fill}%
\end{pgfscope}%
\begin{pgfscope}%
\pgfpathrectangle{\pgfqpoint{0.539299in}{0.078740in}}{\pgfqpoint{7.842520in}{7.842520in}}%
\pgfusepath{clip}%
\pgfsetbuttcap%
\pgfsetroundjoin%
\definecolor{currentfill}{rgb}{0.139147,0.533812,0.555298}%
\pgfsetfillcolor{currentfill}%
\pgfsetlinewidth{0.000000pt}%
\definecolor{currentstroke}{rgb}{0.133743,0.548535,0.553541}%
\pgfsetstrokecolor{currentstroke}%
\pgfsetdash{}{0pt}%
\pgfpathmoveto{\pgfqpoint{3.333165in}{4.962328in}}%
\pgfpathlineto{\pgfqpoint{3.204675in}{4.746147in}}%
\pgfpathlineto{\pgfqpoint{3.121655in}{4.747134in}}%
\pgfpathclose%
\pgfusepath{fill}%
\end{pgfscope}%
\begin{pgfscope}%
\pgfpathrectangle{\pgfqpoint{0.539299in}{0.078740in}}{\pgfqpoint{7.842520in}{7.842520in}}%
\pgfusepath{clip}%
\pgfsetbuttcap%
\pgfsetroundjoin%
\definecolor{currentfill}{rgb}{0.168126,0.459988,0.558082}%
\pgfsetfillcolor{currentfill}%
\pgfsetlinewidth{0.000000pt}%
\definecolor{currentstroke}{rgb}{0.132444,0.552216,0.553018}%
\pgfsetstrokecolor{currentstroke}%
\pgfsetdash{}{0pt}%
\pgfpathmoveto{\pgfqpoint{4.380311in}{4.558363in}}%
\pgfpathlineto{\pgfqpoint{4.436631in}{4.511408in}}%
\pgfpathlineto{\pgfqpoint{4.515135in}{4.412321in}}%
\pgfpathclose%
\pgfusepath{fill}%
\end{pgfscope}%
\begin{pgfscope}%
\pgfpathrectangle{\pgfqpoint{0.539299in}{0.078740in}}{\pgfqpoint{7.842520in}{7.842520in}}%
\pgfusepath{clip}%
\pgfsetbuttcap%
\pgfsetroundjoin%
\definecolor{currentfill}{rgb}{0.282290,0.145912,0.461510}%
\pgfsetfillcolor{currentfill}%
\pgfsetlinewidth{0.000000pt}%
\definecolor{currentstroke}{rgb}{0.131172,0.555899,0.552459}%
\pgfsetstrokecolor{currentstroke}%
\pgfsetdash{}{0pt}%
\pgfpathmoveto{\pgfqpoint{6.504552in}{3.303089in}}%
\pgfpathlineto{\pgfqpoint{6.367154in}{3.364036in}}%
\pgfpathlineto{\pgfqpoint{6.295060in}{3.336953in}}%
\pgfpathclose%
\pgfusepath{fill}%
\end{pgfscope}%
\begin{pgfscope}%
\pgfpathrectangle{\pgfqpoint{0.539299in}{0.078740in}}{\pgfqpoint{7.842520in}{7.842520in}}%
\pgfusepath{clip}%
\pgfsetbuttcap%
\pgfsetroundjoin%
\definecolor{currentfill}{rgb}{0.177423,0.437527,0.557565}%
\pgfsetfillcolor{currentfill}%
\pgfsetlinewidth{0.000000pt}%
\definecolor{currentstroke}{rgb}{0.129933,0.559582,0.551864}%
\pgfsetstrokecolor{currentstroke}%
\pgfsetdash{}{0pt}%
\pgfpathmoveto{\pgfqpoint{4.515135in}{4.412321in}}%
\pgfpathlineto{\pgfqpoint{4.436631in}{4.511408in}}%
\pgfpathlineto{\pgfqpoint{4.649993in}{4.264866in}}%
\pgfpathclose%
\pgfusepath{fill}%
\end{pgfscope}%
\begin{pgfscope}%
\pgfpathrectangle{\pgfqpoint{0.539299in}{0.078740in}}{\pgfqpoint{7.842520in}{7.842520in}}%
\pgfusepath{clip}%
\pgfsetbuttcap%
\pgfsetroundjoin%
\definecolor{currentfill}{rgb}{0.128729,0.563265,0.551229}%
\pgfsetfillcolor{currentfill}%
\pgfsetlinewidth{0.000000pt}%
\definecolor{currentstroke}{rgb}{0.128729,0.563265,0.551229}%
\pgfsetstrokecolor{currentstroke}%
\pgfsetdash{}{0pt}%
\pgfpathmoveto{\pgfqpoint{3.976815in}{4.900063in}}%
\pgfpathlineto{\pgfqpoint{3.895878in}{4.988126in}}%
\pgfpathlineto{\pgfqpoint{4.030569in}{4.909838in}}%
\pgfpathclose%
\pgfusepath{fill}%
\end{pgfscope}%
\begin{pgfscope}%
\pgfpathrectangle{\pgfqpoint{0.539299in}{0.078740in}}{\pgfqpoint{7.842520in}{7.842520in}}%
\pgfusepath{clip}%
\pgfsetbuttcap%
\pgfsetroundjoin%
\definecolor{currentfill}{rgb}{0.197636,0.391528,0.554969}%
\pgfsetfillcolor{currentfill}%
\pgfsetlinewidth{0.000000pt}%
\definecolor{currentstroke}{rgb}{0.127568,0.566949,0.550556}%
\pgfsetstrokecolor{currentstroke}%
\pgfsetdash{}{0pt}%
\pgfpathmoveto{\pgfqpoint{4.784895in}{4.122344in}}%
\pgfpathlineto{\pgfqpoint{4.649993in}{4.264866in}}%
\pgfpathlineto{\pgfqpoint{4.707691in}{4.203491in}}%
\pgfpathclose%
\pgfusepath{fill}%
\end{pgfscope}%
\begin{pgfscope}%
\pgfpathrectangle{\pgfqpoint{0.539299in}{0.078740in}}{\pgfqpoint{7.842520in}{7.842520in}}%
\pgfusepath{clip}%
\pgfsetbuttcap%
\pgfsetroundjoin%
\definecolor{currentfill}{rgb}{0.223925,0.334994,0.548053}%
\pgfsetfillcolor{currentfill}%
\pgfsetlinewidth{0.000000pt}%
\definecolor{currentstroke}{rgb}{0.126453,0.570633,0.549841}%
\pgfsetstrokecolor{currentstroke}%
\pgfsetdash{}{0pt}%
\pgfpathmoveto{\pgfqpoint{2.661974in}{4.301274in}}%
\pgfpathlineto{\pgfqpoint{2.627589in}{3.858151in}}%
\pgfpathlineto{\pgfqpoint{2.543987in}{3.805632in}}%
\pgfpathclose%
\pgfusepath{fill}%
\end{pgfscope}%
\begin{pgfscope}%
\pgfpathrectangle{\pgfqpoint{0.539299in}{0.078740in}}{\pgfqpoint{7.842520in}{7.842520in}}%
\pgfusepath{clip}%
\pgfsetbuttcap%
\pgfsetroundjoin%
\definecolor{currentfill}{rgb}{0.281412,0.155834,0.469201}%
\pgfsetfillcolor{currentfill}%
\pgfsetlinewidth{0.000000pt}%
\definecolor{currentstroke}{rgb}{0.125394,0.574318,0.549086}%
\pgfsetstrokecolor{currentstroke}%
\pgfsetdash{}{0pt}%
\pgfpathmoveto{\pgfqpoint{6.295060in}{3.336953in}}%
\pgfpathlineto{\pgfqpoint{6.367154in}{3.364036in}}%
\pgfpathlineto{\pgfqpoint{6.157532in}{3.392366in}}%
\pgfpathclose%
\pgfusepath{fill}%
\end{pgfscope}%
\begin{pgfscope}%
\pgfpathrectangle{\pgfqpoint{0.539299in}{0.078740in}}{\pgfqpoint{7.842520in}{7.842520in}}%
\pgfusepath{clip}%
\pgfsetbuttcap%
\pgfsetroundjoin%
\definecolor{currentfill}{rgb}{0.277134,0.185228,0.489898}%
\pgfsetfillcolor{currentfill}%
\pgfsetlinewidth{0.000000pt}%
\definecolor{currentstroke}{rgb}{0.124395,0.578002,0.548287}%
\pgfsetstrokecolor{currentstroke}%
\pgfsetdash{}{0pt}%
\pgfpathmoveto{\pgfqpoint{5.946992in}{3.415751in}}%
\pgfpathlineto{\pgfqpoint{5.883392in}{3.490293in}}%
\pgfpathlineto{\pgfqpoint{5.809842in}{3.469040in}}%
\pgfpathclose%
\pgfusepath{fill}%
\end{pgfscope}%
\begin{pgfscope}%
\pgfpathrectangle{\pgfqpoint{0.539299in}{0.078740in}}{\pgfqpoint{7.842520in}{7.842520in}}%
\pgfusepath{clip}%
\pgfsetbuttcap%
\pgfsetroundjoin%
\definecolor{currentfill}{rgb}{0.268510,0.009605,0.335427}%
\pgfsetfillcolor{currentfill}%
\pgfsetlinewidth{0.000000pt}%
\definecolor{currentstroke}{rgb}{0.123463,0.581687,0.547445}%
\pgfsetstrokecolor{currentstroke}%
\pgfsetdash{}{0pt}%
\pgfpathmoveto{\pgfqpoint{7.611514in}{2.887087in}}%
\pgfpathlineto{\pgfqpoint{7.680757in}{2.951802in}}%
\pgfpathlineto{\pgfqpoint{7.472249in}{2.944684in}}%
\pgfpathclose%
\pgfusepath{fill}%
\end{pgfscope}%
\begin{pgfscope}%
\pgfpathrectangle{\pgfqpoint{0.539299in}{0.078740in}}{\pgfqpoint{7.842520in}{7.842520in}}%
\pgfusepath{clip}%
\pgfsetbuttcap%
\pgfsetroundjoin%
\definecolor{currentfill}{rgb}{0.281924,0.089666,0.412415}%
\pgfsetfillcolor{currentfill}%
\pgfsetlinewidth{0.000000pt}%
\definecolor{currentstroke}{rgb}{0.122606,0.585371,0.546557}%
\pgfsetstrokecolor{currentstroke}%
\pgfsetdash{}{0pt}%
\pgfpathmoveto{\pgfqpoint{6.987917in}{3.113178in}}%
\pgfpathlineto{\pgfqpoint{6.850286in}{3.187517in}}%
\pgfpathlineto{\pgfqpoint{6.779843in}{3.161531in}}%
\pgfpathclose%
\pgfusepath{fill}%
\end{pgfscope}%
\begin{pgfscope}%
\pgfpathrectangle{\pgfqpoint{0.539299in}{0.078740in}}{\pgfqpoint{7.842520in}{7.842520in}}%
\pgfusepath{clip}%
\pgfsetbuttcap%
\pgfsetroundjoin%
\definecolor{currentfill}{rgb}{0.271828,0.209303,0.504434}%
\pgfsetfillcolor{currentfill}%
\pgfsetlinewidth{0.000000pt}%
\definecolor{currentstroke}{rgb}{0.121831,0.589055,0.545623}%
\pgfsetstrokecolor{currentstroke}%
\pgfsetdash{}{0pt}%
\pgfpathmoveto{\pgfqpoint{5.536826in}{3.584474in}}%
\pgfpathlineto{\pgfqpoint{5.598833in}{3.513257in}}%
\pgfpathlineto{\pgfqpoint{5.673111in}{3.524234in}}%
\pgfpathclose%
\pgfusepath{fill}%
\end{pgfscope}%
\begin{pgfscope}%
\pgfpathrectangle{\pgfqpoint{0.539299in}{0.078740in}}{\pgfqpoint{7.842520in}{7.842520in}}%
\pgfusepath{clip}%
\pgfsetbuttcap%
\pgfsetroundjoin%
\definecolor{currentfill}{rgb}{0.208623,0.367752,0.552675}%
\pgfsetfillcolor{currentfill}%
\pgfsetlinewidth{0.000000pt}%
\definecolor{currentstroke}{rgb}{0.121148,0.592739,0.544641}%
\pgfsetstrokecolor{currentstroke}%
\pgfsetdash{}{0pt}%
\pgfpathmoveto{\pgfqpoint{4.707691in}{4.203491in}}%
\pgfpathlineto{\pgfqpoint{4.919892in}{3.989315in}}%
\pgfpathlineto{\pgfqpoint{4.784895in}{4.122344in}}%
\pgfpathclose%
\pgfusepath{fill}%
\end{pgfscope}%
\begin{pgfscope}%
\pgfpathrectangle{\pgfqpoint{0.539299in}{0.078740in}}{\pgfqpoint{7.842520in}{7.842520in}}%
\pgfusepath{clip}%
\pgfsetbuttcap%
\pgfsetroundjoin%
\definecolor{currentfill}{rgb}{0.265145,0.232956,0.516599}%
\pgfsetfillcolor{currentfill}%
\pgfsetlinewidth{0.000000pt}%
\definecolor{currentstroke}{rgb}{0.120565,0.596422,0.543611}%
\pgfsetstrokecolor{currentstroke}%
\pgfsetdash{}{0pt}%
\pgfpathmoveto{\pgfqpoint{5.326196in}{3.667457in}}%
\pgfpathlineto{\pgfqpoint{5.462306in}{3.585484in}}%
\pgfpathlineto{\pgfqpoint{5.536826in}{3.584474in}}%
\pgfpathclose%
\pgfusepath{fill}%
\end{pgfscope}%
\begin{pgfscope}%
\pgfpathrectangle{\pgfqpoint{0.539299in}{0.078740in}}{\pgfqpoint{7.842520in}{7.842520in}}%
\pgfusepath{clip}%
\pgfsetbuttcap%
\pgfsetroundjoin%
\definecolor{currentfill}{rgb}{0.121831,0.589055,0.545623}%
\pgfsetfillcolor{currentfill}%
\pgfsetlinewidth{0.000000pt}%
\definecolor{currentstroke}{rgb}{0.120092,0.600104,0.542530}%
\pgfsetstrokecolor{currentstroke}%
\pgfsetdash{}{0pt}%
\pgfpathmoveto{\pgfqpoint{3.679940in}{5.091602in}}%
\pgfpathlineto{\pgfqpoint{3.761962in}{5.020676in}}%
\pgfpathlineto{\pgfqpoint{3.629210in}{4.996070in}}%
\pgfpathclose%
\pgfusepath{fill}%
\end{pgfscope}%
\begin{pgfscope}%
\pgfpathrectangle{\pgfqpoint{0.539299in}{0.078740in}}{\pgfqpoint{7.842520in}{7.842520in}}%
\pgfusepath{clip}%
\pgfsetbuttcap%
\pgfsetroundjoin%
\definecolor{currentfill}{rgb}{0.272594,0.025563,0.353093}%
\pgfsetfillcolor{currentfill}%
\pgfsetlinewidth{0.000000pt}%
\definecolor{currentstroke}{rgb}{0.119738,0.603785,0.541400}%
\pgfsetstrokecolor{currentstroke}%
\pgfsetdash{}{0pt}%
\pgfpathmoveto{\pgfqpoint{7.472249in}{2.944684in}}%
\pgfpathlineto{\pgfqpoint{7.333570in}{3.007542in}}%
\pgfpathlineto{\pgfqpoint{7.264065in}{2.962827in}}%
\pgfpathclose%
\pgfusepath{fill}%
\end{pgfscope}%
\begin{pgfscope}%
\pgfpathrectangle{\pgfqpoint{0.539299in}{0.078740in}}{\pgfqpoint{7.842520in}{7.842520in}}%
\pgfusepath{clip}%
\pgfsetbuttcap%
\pgfsetroundjoin%
\definecolor{currentfill}{rgb}{0.124395,0.578002,0.548287}%
\pgfsetfillcolor{currentfill}%
\pgfsetlinewidth{0.000000pt}%
\definecolor{currentstroke}{rgb}{0.119512,0.607464,0.540218}%
\pgfsetstrokecolor{currentstroke}%
\pgfsetdash{}{0pt}%
\pgfpathmoveto{\pgfqpoint{3.547026in}{5.048670in}}%
\pgfpathlineto{\pgfqpoint{3.415921in}{4.934934in}}%
\pgfpathlineto{\pgfqpoint{3.333165in}{4.962328in}}%
\pgfpathclose%
\pgfusepath{fill}%
\end{pgfscope}%
\begin{pgfscope}%
\pgfpathrectangle{\pgfqpoint{0.539299in}{0.078740in}}{\pgfqpoint{7.842520in}{7.842520in}}%
\pgfusepath{clip}%
\pgfsetbuttcap%
\pgfsetroundjoin%
\definecolor{currentfill}{rgb}{0.278826,0.175490,0.483397}%
\pgfsetfillcolor{currentfill}%
\pgfsetlinewidth{0.000000pt}%
\definecolor{currentstroke}{rgb}{0.119423,0.611141,0.538982}%
\pgfsetstrokecolor{currentstroke}%
\pgfsetdash{}{0pt}%
\pgfpathmoveto{\pgfqpoint{5.946992in}{3.415751in}}%
\pgfpathlineto{\pgfqpoint{6.157532in}{3.392366in}}%
\pgfpathlineto{\pgfqpoint{6.020289in}{3.442779in}}%
\pgfpathclose%
\pgfusepath{fill}%
\end{pgfscope}%
\begin{pgfscope}%
\pgfpathrectangle{\pgfqpoint{0.539299in}{0.078740in}}{\pgfqpoint{7.842520in}{7.842520in}}%
\pgfusepath{clip}%
\pgfsetbuttcap%
\pgfsetroundjoin%
\definecolor{currentfill}{rgb}{0.282910,0.105393,0.426902}%
\pgfsetfillcolor{currentfill}%
\pgfsetlinewidth{0.000000pt}%
\definecolor{currentstroke}{rgb}{0.119483,0.614817,0.537692}%
\pgfsetstrokecolor{currentstroke}%
\pgfsetdash{}{0pt}%
\pgfpathmoveto{\pgfqpoint{6.850286in}{3.187517in}}%
\pgfpathlineto{\pgfqpoint{6.642120in}{3.235328in}}%
\pgfpathlineto{\pgfqpoint{6.779843in}{3.161531in}}%
\pgfpathclose%
\pgfusepath{fill}%
\end{pgfscope}%
\begin{pgfscope}%
\pgfpathrectangle{\pgfqpoint{0.539299in}{0.078740in}}{\pgfqpoint{7.842520in}{7.842520in}}%
\pgfusepath{clip}%
\pgfsetbuttcap%
\pgfsetroundjoin%
\definecolor{currentfill}{rgb}{0.274952,0.037752,0.364543}%
\pgfsetfillcolor{currentfill}%
\pgfsetlinewidth{0.000000pt}%
\definecolor{currentstroke}{rgb}{0.119699,0.618490,0.536347}%
\pgfsetstrokecolor{currentstroke}%
\pgfsetdash{}{0pt}%
\pgfpathmoveto{\pgfqpoint{7.264065in}{2.962827in}}%
\pgfpathlineto{\pgfqpoint{7.333570in}{3.007542in}}%
\pgfpathlineto{\pgfqpoint{7.125818in}{3.037569in}}%
\pgfpathclose%
\pgfusepath{fill}%
\end{pgfscope}%
\begin{pgfscope}%
\pgfpathrectangle{\pgfqpoint{0.539299in}{0.078740in}}{\pgfqpoint{7.842520in}{7.842520in}}%
\pgfusepath{clip}%
\pgfsetbuttcap%
\pgfsetroundjoin%
\definecolor{currentfill}{rgb}{0.229739,0.322361,0.545706}%
\pgfsetfillcolor{currentfill}%
\pgfsetlinewidth{0.000000pt}%
\definecolor{currentstroke}{rgb}{0.120081,0.622161,0.534946}%
\pgfsetstrokecolor{currentstroke}%
\pgfsetdash{}{0pt}%
\pgfpathmoveto{\pgfqpoint{5.055056in}{3.868593in}}%
\pgfpathlineto{\pgfqpoint{4.919892in}{3.989315in}}%
\pgfpathlineto{\pgfqpoint{4.978938in}{3.922091in}}%
\pgfpathclose%
\pgfusepath{fill}%
\end{pgfscope}%
\begin{pgfscope}%
\pgfpathrectangle{\pgfqpoint{0.539299in}{0.078740in}}{\pgfqpoint{7.842520in}{7.842520in}}%
\pgfusepath{clip}%
\pgfsetbuttcap%
\pgfsetroundjoin%
\definecolor{currentfill}{rgb}{0.121148,0.592739,0.544641}%
\pgfsetfillcolor{currentfill}%
\pgfsetlinewidth{0.000000pt}%
\definecolor{currentstroke}{rgb}{0.120638,0.625828,0.533488}%
\pgfsetstrokecolor{currentstroke}%
\pgfsetdash{}{0pt}%
\pgfpathmoveto{\pgfqpoint{3.679940in}{5.091602in}}%
\pgfpathlineto{\pgfqpoint{3.629210in}{4.996070in}}%
\pgfpathlineto{\pgfqpoint{3.547026in}{5.048670in}}%
\pgfpathclose%
\pgfusepath{fill}%
\end{pgfscope}%
\begin{pgfscope}%
\pgfpathrectangle{\pgfqpoint{0.539299in}{0.078740in}}{\pgfqpoint{7.842520in}{7.842520in}}%
\pgfusepath{clip}%
\pgfsetbuttcap%
\pgfsetroundjoin%
\definecolor{currentfill}{rgb}{0.267968,0.223549,0.512008}%
\pgfsetfillcolor{currentfill}%
\pgfsetlinewidth{0.000000pt}%
\definecolor{currentstroke}{rgb}{0.121380,0.629492,0.531973}%
\pgfsetstrokecolor{currentstroke}%
\pgfsetdash{}{0pt}%
\pgfpathmoveto{\pgfqpoint{5.462306in}{3.585484in}}%
\pgfpathlineto{\pgfqpoint{5.598833in}{3.513257in}}%
\pgfpathlineto{\pgfqpoint{5.536826in}{3.584474in}}%
\pgfpathclose%
\pgfusepath{fill}%
\end{pgfscope}%
\begin{pgfscope}%
\pgfpathrectangle{\pgfqpoint{0.539299in}{0.078740in}}{\pgfqpoint{7.842520in}{7.842520in}}%
\pgfusepath{clip}%
\pgfsetbuttcap%
\pgfsetroundjoin%
\definecolor{currentfill}{rgb}{0.137770,0.537492,0.554906}%
\pgfsetfillcolor{currentfill}%
\pgfsetlinewidth{0.000000pt}%
\definecolor{currentstroke}{rgb}{0.122312,0.633153,0.530398}%
\pgfsetstrokecolor{currentstroke}%
\pgfsetdash{}{0pt}%
\pgfpathmoveto{\pgfqpoint{4.165728in}{4.797221in}}%
\pgfpathlineto{\pgfqpoint{4.245563in}{4.694927in}}%
\pgfpathlineto{\pgfqpoint{4.030569in}{4.909838in}}%
\pgfpathclose%
\pgfusepath{fill}%
\end{pgfscope}%
\begin{pgfscope}%
\pgfpathrectangle{\pgfqpoint{0.539299in}{0.078740in}}{\pgfqpoint{7.842520in}{7.842520in}}%
\pgfusepath{clip}%
\pgfsetbuttcap%
\pgfsetroundjoin%
\definecolor{currentfill}{rgb}{0.241237,0.296485,0.539709}%
\pgfsetfillcolor{currentfill}%
\pgfsetlinewidth{0.000000pt}%
\definecolor{currentstroke}{rgb}{0.123444,0.636809,0.528763}%
\pgfsetstrokecolor{currentstroke}%
\pgfsetdash{}{0pt}%
\pgfpathmoveto{\pgfqpoint{5.190465in}{3.761377in}}%
\pgfpathlineto{\pgfqpoint{5.055056in}{3.868593in}}%
\pgfpathlineto{\pgfqpoint{5.114785in}{3.799326in}}%
\pgfpathclose%
\pgfusepath{fill}%
\end{pgfscope}%
\begin{pgfscope}%
\pgfpathrectangle{\pgfqpoint{0.539299in}{0.078740in}}{\pgfqpoint{7.842520in}{7.842520in}}%
\pgfusepath{clip}%
\pgfsetbuttcap%
\pgfsetroundjoin%
\definecolor{currentfill}{rgb}{0.280267,0.073417,0.397163}%
\pgfsetfillcolor{currentfill}%
\pgfsetlinewidth{0.000000pt}%
\definecolor{currentstroke}{rgb}{0.124780,0.640461,0.527068}%
\pgfsetstrokecolor{currentstroke}%
\pgfsetdash{}{0pt}%
\pgfpathmoveto{\pgfqpoint{7.125818in}{3.037569in}}%
\pgfpathlineto{\pgfqpoint{6.987917in}{3.113178in}}%
\pgfpathlineto{\pgfqpoint{6.917735in}{3.083093in}}%
\pgfpathclose%
\pgfusepath{fill}%
\end{pgfscope}%
\begin{pgfscope}%
\pgfpathrectangle{\pgfqpoint{0.539299in}{0.078740in}}{\pgfqpoint{7.842520in}{7.842520in}}%
\pgfusepath{clip}%
\pgfsetbuttcap%
\pgfsetroundjoin%
\definecolor{currentfill}{rgb}{0.147607,0.511733,0.557049}%
\pgfsetfillcolor{currentfill}%
\pgfsetlinewidth{0.000000pt}%
\definecolor{currentstroke}{rgb}{0.126326,0.644107,0.525311}%
\pgfsetstrokecolor{currentstroke}%
\pgfsetdash{}{0pt}%
\pgfpathmoveto{\pgfqpoint{4.165728in}{4.797221in}}%
\pgfpathlineto{\pgfqpoint{4.301131in}{4.661130in}}%
\pgfpathlineto{\pgfqpoint{4.380311in}{4.558363in}}%
\pgfpathclose%
\pgfusepath{fill}%
\end{pgfscope}%
\begin{pgfscope}%
\pgfpathrectangle{\pgfqpoint{0.539299in}{0.078740in}}{\pgfqpoint{7.842520in}{7.842520in}}%
\pgfusepath{clip}%
\pgfsetbuttcap%
\pgfsetroundjoin%
\definecolor{currentfill}{rgb}{0.281887,0.150881,0.465405}%
\pgfsetfillcolor{currentfill}%
\pgfsetlinewidth{0.000000pt}%
\definecolor{currentstroke}{rgb}{0.128087,0.647749,0.523491}%
\pgfsetstrokecolor{currentstroke}%
\pgfsetdash{}{0pt}%
\pgfpathmoveto{\pgfqpoint{6.504552in}{3.303089in}}%
\pgfpathlineto{\pgfqpoint{6.295060in}{3.336953in}}%
\pgfpathlineto{\pgfqpoint{6.432810in}{3.275195in}}%
\pgfpathclose%
\pgfusepath{fill}%
\end{pgfscope}%
\begin{pgfscope}%
\pgfpathrectangle{\pgfqpoint{0.539299in}{0.078740in}}{\pgfqpoint{7.842520in}{7.842520in}}%
\pgfusepath{clip}%
\pgfsetbuttcap%
\pgfsetroundjoin%
\definecolor{currentfill}{rgb}{0.156270,0.489624,0.557936}%
\pgfsetfillcolor{currentfill}%
\pgfsetlinewidth{0.000000pt}%
\definecolor{currentstroke}{rgb}{0.130067,0.651384,0.521608}%
\pgfsetstrokecolor{currentstroke}%
\pgfsetdash{}{0pt}%
\pgfpathmoveto{\pgfqpoint{4.380311in}{4.558363in}}%
\pgfpathlineto{\pgfqpoint{4.301131in}{4.661130in}}%
\pgfpathlineto{\pgfqpoint{4.436631in}{4.511408in}}%
\pgfpathclose%
\pgfusepath{fill}%
\end{pgfscope}%
\begin{pgfscope}%
\pgfpathrectangle{\pgfqpoint{0.539299in}{0.078740in}}{\pgfqpoint{7.842520in}{7.842520in}}%
\pgfusepath{clip}%
\pgfsetbuttcap%
\pgfsetroundjoin%
\definecolor{currentfill}{rgb}{0.273006,0.204520,0.501721}%
\pgfsetfillcolor{currentfill}%
\pgfsetlinewidth{0.000000pt}%
\definecolor{currentstroke}{rgb}{0.132268,0.655014,0.519661}%
\pgfsetstrokecolor{currentstroke}%
\pgfsetdash{}{0pt}%
\pgfpathmoveto{\pgfqpoint{5.735793in}{3.448039in}}%
\pgfpathlineto{\pgfqpoint{5.809842in}{3.469040in}}%
\pgfpathlineto{\pgfqpoint{5.673111in}{3.524234in}}%
\pgfpathclose%
\pgfusepath{fill}%
\end{pgfscope}%
\begin{pgfscope}%
\pgfpathrectangle{\pgfqpoint{0.539299in}{0.078740in}}{\pgfqpoint{7.842520in}{7.842520in}}%
\pgfusepath{clip}%
\pgfsetbuttcap%
\pgfsetroundjoin%
\definecolor{currentfill}{rgb}{0.283072,0.130895,0.449241}%
\pgfsetfillcolor{currentfill}%
\pgfsetlinewidth{0.000000pt}%
\definecolor{currentstroke}{rgb}{0.134692,0.658636,0.517649}%
\pgfsetstrokecolor{currentstroke}%
\pgfsetdash{}{0pt}%
\pgfpathmoveto{\pgfqpoint{6.570734in}{3.206547in}}%
\pgfpathlineto{\pgfqpoint{6.642120in}{3.235328in}}%
\pgfpathlineto{\pgfqpoint{6.504552in}{3.303089in}}%
\pgfpathclose%
\pgfusepath{fill}%
\end{pgfscope}%
\begin{pgfscope}%
\pgfpathrectangle{\pgfqpoint{0.539299in}{0.078740in}}{\pgfqpoint{7.842520in}{7.842520in}}%
\pgfusepath{clip}%
\pgfsetbuttcap%
\pgfsetroundjoin%
\definecolor{currentfill}{rgb}{0.175841,0.441290,0.557685}%
\pgfsetfillcolor{currentfill}%
\pgfsetlinewidth{0.000000pt}%
\definecolor{currentstroke}{rgb}{0.137339,0.662252,0.515571}%
\pgfsetstrokecolor{currentstroke}%
\pgfsetdash{}{0pt}%
\pgfpathmoveto{\pgfqpoint{4.436631in}{4.511408in}}%
\pgfpathlineto{\pgfqpoint{4.572157in}{4.356552in}}%
\pgfpathlineto{\pgfqpoint{4.649993in}{4.264866in}}%
\pgfpathclose%
\pgfusepath{fill}%
\end{pgfscope}%
\begin{pgfscope}%
\pgfpathrectangle{\pgfqpoint{0.539299in}{0.078740in}}{\pgfqpoint{7.842520in}{7.842520in}}%
\pgfusepath{clip}%
\pgfsetbuttcap%
\pgfsetroundjoin%
\definecolor{currentfill}{rgb}{0.185556,0.418570,0.556753}%
\pgfsetfillcolor{currentfill}%
\pgfsetlinewidth{0.000000pt}%
\definecolor{currentstroke}{rgb}{0.140210,0.665859,0.513427}%
\pgfsetstrokecolor{currentstroke}%
\pgfsetdash{}{0pt}%
\pgfpathmoveto{\pgfqpoint{4.707691in}{4.203491in}}%
\pgfpathlineto{\pgfqpoint{4.649993in}{4.264866in}}%
\pgfpathlineto{\pgfqpoint{4.572157in}{4.356552in}}%
\pgfpathclose%
\pgfusepath{fill}%
\end{pgfscope}%
\begin{pgfscope}%
\pgfpathrectangle{\pgfqpoint{0.539299in}{0.078740in}}{\pgfqpoint{7.842520in}{7.842520in}}%
\pgfusepath{clip}%
\pgfsetbuttcap%
\pgfsetroundjoin%
\definecolor{currentfill}{rgb}{0.281924,0.089666,0.412415}%
\pgfsetfillcolor{currentfill}%
\pgfsetlinewidth{0.000000pt}%
\definecolor{currentstroke}{rgb}{0.143303,0.669459,0.511215}%
\pgfsetstrokecolor{currentstroke}%
\pgfsetdash{}{0pt}%
\pgfpathmoveto{\pgfqpoint{6.779843in}{3.161531in}}%
\pgfpathlineto{\pgfqpoint{6.917735in}{3.083093in}}%
\pgfpathlineto{\pgfqpoint{6.987917in}{3.113178in}}%
\pgfpathclose%
\pgfusepath{fill}%
\end{pgfscope}%
\begin{pgfscope}%
\pgfpathrectangle{\pgfqpoint{0.539299in}{0.078740in}}{\pgfqpoint{7.842520in}{7.842520in}}%
\pgfusepath{clip}%
\pgfsetbuttcap%
\pgfsetroundjoin%
\definecolor{currentfill}{rgb}{0.120565,0.596422,0.543611}%
\pgfsetfillcolor{currentfill}%
\pgfsetlinewidth{0.000000pt}%
\definecolor{currentstroke}{rgb}{0.146616,0.673050,0.508936}%
\pgfsetstrokecolor{currentstroke}%
\pgfsetdash{}{0pt}%
\pgfpathmoveto{\pgfqpoint{3.761962in}{5.020676in}}%
\pgfpathlineto{\pgfqpoint{3.679940in}{5.091602in}}%
\pgfpathlineto{\pgfqpoint{3.895878in}{4.988126in}}%
\pgfpathclose%
\pgfusepath{fill}%
\end{pgfscope}%
\begin{pgfscope}%
\pgfpathrectangle{\pgfqpoint{0.539299in}{0.078740in}}{\pgfqpoint{7.842520in}{7.842520in}}%
\pgfusepath{clip}%
\pgfsetbuttcap%
\pgfsetroundjoin%
\definecolor{currentfill}{rgb}{0.272594,0.025563,0.353093}%
\pgfsetfillcolor{currentfill}%
\pgfsetlinewidth{0.000000pt}%
\definecolor{currentstroke}{rgb}{0.150148,0.676631,0.506589}%
\pgfsetstrokecolor{currentstroke}%
\pgfsetdash{}{0pt}%
\pgfpathmoveto{\pgfqpoint{7.264065in}{2.962827in}}%
\pgfpathlineto{\pgfqpoint{7.402749in}{2.891078in}}%
\pgfpathlineto{\pgfqpoint{7.472249in}{2.944684in}}%
\pgfpathclose%
\pgfusepath{fill}%
\end{pgfscope}%
\begin{pgfscope}%
\pgfpathrectangle{\pgfqpoint{0.539299in}{0.078740in}}{\pgfqpoint{7.842520in}{7.842520in}}%
\pgfusepath{clip}%
\pgfsetbuttcap%
\pgfsetroundjoin%
\definecolor{currentfill}{rgb}{0.163625,0.471133,0.558148}%
\pgfsetfillcolor{currentfill}%
\pgfsetlinewidth{0.000000pt}%
\definecolor{currentstroke}{rgb}{0.153894,0.680203,0.504172}%
\pgfsetstrokecolor{currentstroke}%
\pgfsetdash{}{0pt}%
\pgfpathmoveto{\pgfqpoint{2.954036in}{4.737238in}}%
\pgfpathlineto{\pgfqpoint{2.913780in}{4.395967in}}%
\pgfpathlineto{\pgfqpoint{2.830306in}{4.368134in}}%
\pgfpathclose%
\pgfusepath{fill}%
\end{pgfscope}%
\begin{pgfscope}%
\pgfpathrectangle{\pgfqpoint{0.539299in}{0.078740in}}{\pgfqpoint{7.842520in}{7.842520in}}%
\pgfusepath{clip}%
\pgfsetbuttcap%
\pgfsetroundjoin%
\definecolor{currentfill}{rgb}{0.206756,0.371758,0.553117}%
\pgfsetfillcolor{currentfill}%
\pgfsetlinewidth{0.000000pt}%
\definecolor{currentstroke}{rgb}{0.157851,0.683765,0.501686}%
\pgfsetstrokecolor{currentstroke}%
\pgfsetdash{}{0pt}%
\pgfpathmoveto{\pgfqpoint{4.843266in}{4.057477in}}%
\pgfpathlineto{\pgfqpoint{4.919892in}{3.989315in}}%
\pgfpathlineto{\pgfqpoint{4.707691in}{4.203491in}}%
\pgfpathclose%
\pgfusepath{fill}%
\end{pgfscope}%
\begin{pgfscope}%
\pgfpathrectangle{\pgfqpoint{0.539299in}{0.078740in}}{\pgfqpoint{7.842520in}{7.842520in}}%
\pgfusepath{clip}%
\pgfsetbuttcap%
\pgfsetroundjoin%
\definecolor{currentfill}{rgb}{0.252194,0.269783,0.531579}%
\pgfsetfillcolor{currentfill}%
\pgfsetlinewidth{0.000000pt}%
\definecolor{currentstroke}{rgb}{0.162016,0.687316,0.499129}%
\pgfsetstrokecolor{currentstroke}%
\pgfsetdash{}{0pt}%
\pgfpathmoveto{\pgfqpoint{5.250882in}{3.689752in}}%
\pgfpathlineto{\pgfqpoint{5.326196in}{3.667457in}}%
\pgfpathlineto{\pgfqpoint{5.190465in}{3.761377in}}%
\pgfpathclose%
\pgfusepath{fill}%
\end{pgfscope}%
\begin{pgfscope}%
\pgfpathrectangle{\pgfqpoint{0.539299in}{0.078740in}}{\pgfqpoint{7.842520in}{7.842520in}}%
\pgfusepath{clip}%
\pgfsetbuttcap%
\pgfsetroundjoin%
\definecolor{currentfill}{rgb}{0.216210,0.351535,0.550627}%
\pgfsetfillcolor{currentfill}%
\pgfsetlinewidth{0.000000pt}%
\definecolor{currentstroke}{rgb}{0.166383,0.690856,0.496502}%
\pgfsetstrokecolor{currentstroke}%
\pgfsetdash{}{0pt}%
\pgfpathmoveto{\pgfqpoint{4.978938in}{3.922091in}}%
\pgfpathlineto{\pgfqpoint{4.919892in}{3.989315in}}%
\pgfpathlineto{\pgfqpoint{4.843266in}{4.057477in}}%
\pgfpathclose%
\pgfusepath{fill}%
\end{pgfscope}%
\begin{pgfscope}%
\pgfpathrectangle{\pgfqpoint{0.539299in}{0.078740in}}{\pgfqpoint{7.842520in}{7.842520in}}%
\pgfusepath{clip}%
\pgfsetbuttcap%
\pgfsetroundjoin%
\definecolor{currentfill}{rgb}{0.195860,0.395433,0.555276}%
\pgfsetfillcolor{currentfill}%
\pgfsetlinewidth{0.000000pt}%
\definecolor{currentstroke}{rgb}{0.170948,0.694384,0.493803}%
\pgfsetstrokecolor{currentstroke}%
\pgfsetdash{}{0pt}%
\pgfpathmoveto{\pgfqpoint{2.661974in}{4.301274in}}%
\pgfpathlineto{\pgfqpoint{2.746363in}{4.336760in}}%
\pgfpathlineto{\pgfqpoint{2.627589in}{3.858151in}}%
\pgfpathclose%
\pgfusepath{fill}%
\end{pgfscope}%
\begin{pgfscope}%
\pgfpathrectangle{\pgfqpoint{0.539299in}{0.078740in}}{\pgfqpoint{7.842520in}{7.842520in}}%
\pgfusepath{clip}%
\pgfsetbuttcap%
\pgfsetroundjoin%
\definecolor{currentfill}{rgb}{0.278012,0.180367,0.486697}%
\pgfsetfillcolor{currentfill}%
\pgfsetlinewidth{0.000000pt}%
\definecolor{currentstroke}{rgb}{0.175707,0.697900,0.491033}%
\pgfsetstrokecolor{currentstroke}%
\pgfsetdash{}{0pt}%
\pgfpathmoveto{\pgfqpoint{6.084512in}{3.361457in}}%
\pgfpathlineto{\pgfqpoint{6.157532in}{3.392366in}}%
\pgfpathlineto{\pgfqpoint{5.946992in}{3.415751in}}%
\pgfpathclose%
\pgfusepath{fill}%
\end{pgfscope}%
\begin{pgfscope}%
\pgfpathrectangle{\pgfqpoint{0.539299in}{0.078740in}}{\pgfqpoint{7.842520in}{7.842520in}}%
\pgfusepath{clip}%
\pgfsetbuttcap%
\pgfsetroundjoin%
\definecolor{currentfill}{rgb}{0.268510,0.009605,0.335427}%
\pgfsetfillcolor{currentfill}%
\pgfsetlinewidth{0.000000pt}%
\definecolor{currentstroke}{rgb}{0.180653,0.701402,0.488189}%
\pgfsetstrokecolor{currentstroke}%
\pgfsetdash{}{0pt}%
\pgfpathmoveto{\pgfqpoint{7.472249in}{2.944684in}}%
\pgfpathlineto{\pgfqpoint{7.541973in}{2.824378in}}%
\pgfpathlineto{\pgfqpoint{7.611514in}{2.887087in}}%
\pgfpathclose%
\pgfusepath{fill}%
\end{pgfscope}%
\begin{pgfscope}%
\pgfpathrectangle{\pgfqpoint{0.539299in}{0.078740in}}{\pgfqpoint{7.842520in}{7.842520in}}%
\pgfusepath{clip}%
\pgfsetbuttcap%
\pgfsetroundjoin%
\definecolor{currentfill}{rgb}{0.221989,0.339161,0.548752}%
\pgfsetfillcolor{currentfill}%
\pgfsetlinewidth{0.000000pt}%
\definecolor{currentstroke}{rgb}{0.185783,0.704891,0.485273}%
\pgfsetstrokecolor{currentstroke}%
\pgfsetdash{}{0pt}%
\pgfpathmoveto{\pgfqpoint{2.543987in}{3.805632in}}%
\pgfpathlineto{\pgfqpoint{2.459954in}{3.750514in}}%
\pgfpathlineto{\pgfqpoint{2.661974in}{4.301274in}}%
\pgfpathclose%
\pgfusepath{fill}%
\end{pgfscope}%
\begin{pgfscope}%
\pgfpathrectangle{\pgfqpoint{0.539299in}{0.078740in}}{\pgfqpoint{7.842520in}{7.842520in}}%
\pgfusepath{clip}%
\pgfsetbuttcap%
\pgfsetroundjoin%
\definecolor{currentfill}{rgb}{0.270595,0.214069,0.507052}%
\pgfsetfillcolor{currentfill}%
\pgfsetlinewidth{0.000000pt}%
\definecolor{currentstroke}{rgb}{0.191090,0.708366,0.482284}%
\pgfsetstrokecolor{currentstroke}%
\pgfsetdash{}{0pt}%
\pgfpathmoveto{\pgfqpoint{5.673111in}{3.524234in}}%
\pgfpathlineto{\pgfqpoint{5.598833in}{3.513257in}}%
\pgfpathlineto{\pgfqpoint{5.735793in}{3.448039in}}%
\pgfpathclose%
\pgfusepath{fill}%
\end{pgfscope}%
\begin{pgfscope}%
\pgfpathrectangle{\pgfqpoint{0.539299in}{0.078740in}}{\pgfqpoint{7.842520in}{7.842520in}}%
\pgfusepath{clip}%
\pgfsetbuttcap%
\pgfsetroundjoin%
\definecolor{currentfill}{rgb}{0.282623,0.140926,0.457517}%
\pgfsetfillcolor{currentfill}%
\pgfsetlinewidth{0.000000pt}%
\definecolor{currentstroke}{rgb}{0.196571,0.711827,0.479221}%
\pgfsetstrokecolor{currentstroke}%
\pgfsetdash{}{0pt}%
\pgfpathmoveto{\pgfqpoint{6.504552in}{3.303089in}}%
\pgfpathlineto{\pgfqpoint{6.432810in}{3.275195in}}%
\pgfpathlineto{\pgfqpoint{6.570734in}{3.206547in}}%
\pgfpathclose%
\pgfusepath{fill}%
\end{pgfscope}%
\begin{pgfscope}%
\pgfpathrectangle{\pgfqpoint{0.539299in}{0.078740in}}{\pgfqpoint{7.842520in}{7.842520in}}%
\pgfusepath{clip}%
\pgfsetbuttcap%
\pgfsetroundjoin%
\definecolor{currentfill}{rgb}{0.280255,0.165693,0.476498}%
\pgfsetfillcolor{currentfill}%
\pgfsetlinewidth{0.000000pt}%
\definecolor{currentstroke}{rgb}{0.202219,0.715272,0.476084}%
\pgfsetstrokecolor{currentstroke}%
\pgfsetdash{}{0pt}%
\pgfpathmoveto{\pgfqpoint{6.222345in}{3.303711in}}%
\pgfpathlineto{\pgfqpoint{6.295060in}{3.336953in}}%
\pgfpathlineto{\pgfqpoint{6.157532in}{3.392366in}}%
\pgfpathclose%
\pgfusepath{fill}%
\end{pgfscope}%
\begin{pgfscope}%
\pgfpathrectangle{\pgfqpoint{0.539299in}{0.078740in}}{\pgfqpoint{7.842520in}{7.842520in}}%
\pgfusepath{clip}%
\pgfsetbuttcap%
\pgfsetroundjoin%
\definecolor{currentfill}{rgb}{0.231674,0.318106,0.544834}%
\pgfsetfillcolor{currentfill}%
\pgfsetlinewidth{0.000000pt}%
\definecolor{currentstroke}{rgb}{0.208030,0.718701,0.472873}%
\pgfsetstrokecolor{currentstroke}%
\pgfsetdash{}{0pt}%
\pgfpathmoveto{\pgfqpoint{5.055056in}{3.868593in}}%
\pgfpathlineto{\pgfqpoint{4.978938in}{3.922091in}}%
\pgfpathlineto{\pgfqpoint{5.114785in}{3.799326in}}%
\pgfpathclose%
\pgfusepath{fill}%
\end{pgfscope}%
\begin{pgfscope}%
\pgfpathrectangle{\pgfqpoint{0.539299in}{0.078740in}}{\pgfqpoint{7.842520in}{7.842520in}}%
\pgfusepath{clip}%
\pgfsetbuttcap%
\pgfsetroundjoin%
\definecolor{currentfill}{rgb}{0.274128,0.199721,0.498911}%
\pgfsetfillcolor{currentfill}%
\pgfsetlinewidth{0.000000pt}%
\definecolor{currentstroke}{rgb}{0.214000,0.722114,0.469588}%
\pgfsetstrokecolor{currentstroke}%
\pgfsetdash{}{0pt}%
\pgfpathmoveto{\pgfqpoint{5.809842in}{3.469040in}}%
\pgfpathlineto{\pgfqpoint{5.735793in}{3.448039in}}%
\pgfpathlineto{\pgfqpoint{5.946992in}{3.415751in}}%
\pgfpathclose%
\pgfusepath{fill}%
\end{pgfscope}%
\begin{pgfscope}%
\pgfpathrectangle{\pgfqpoint{0.539299in}{0.078740in}}{\pgfqpoint{7.842520in}{7.842520in}}%
\pgfusepath{clip}%
\pgfsetbuttcap%
\pgfsetroundjoin%
\definecolor{currentfill}{rgb}{0.269944,0.014625,0.341379}%
\pgfsetfillcolor{currentfill}%
\pgfsetlinewidth{0.000000pt}%
\definecolor{currentstroke}{rgb}{0.220124,0.725509,0.466226}%
\pgfsetstrokecolor{currentstroke}%
\pgfsetdash{}{0pt}%
\pgfpathmoveto{\pgfqpoint{7.472249in}{2.944684in}}%
\pgfpathlineto{\pgfqpoint{7.402749in}{2.891078in}}%
\pgfpathlineto{\pgfqpoint{7.541973in}{2.824378in}}%
\pgfpathclose%
\pgfusepath{fill}%
\end{pgfscope}%
\begin{pgfscope}%
\pgfpathrectangle{\pgfqpoint{0.539299in}{0.078740in}}{\pgfqpoint{7.842520in}{7.842520in}}%
\pgfusepath{clip}%
\pgfsetbuttcap%
\pgfsetroundjoin%
\definecolor{currentfill}{rgb}{0.277018,0.050344,0.375715}%
\pgfsetfillcolor{currentfill}%
\pgfsetlinewidth{0.000000pt}%
\definecolor{currentstroke}{rgb}{0.226397,0.728888,0.462789}%
\pgfsetstrokecolor{currentstroke}%
\pgfsetdash{}{0pt}%
\pgfpathmoveto{\pgfqpoint{7.125818in}{3.037569in}}%
\pgfpathlineto{\pgfqpoint{7.194201in}{2.919934in}}%
\pgfpathlineto{\pgfqpoint{7.264065in}{2.962827in}}%
\pgfpathclose%
\pgfusepath{fill}%
\end{pgfscope}%
\begin{pgfscope}%
\pgfpathrectangle{\pgfqpoint{0.539299in}{0.078740in}}{\pgfqpoint{7.842520in}{7.842520in}}%
\pgfusepath{clip}%
\pgfsetbuttcap%
\pgfsetroundjoin%
\definecolor{currentfill}{rgb}{0.280267,0.073417,0.397163}%
\pgfsetfillcolor{currentfill}%
\pgfsetlinewidth{0.000000pt}%
\definecolor{currentstroke}{rgb}{0.232815,0.732247,0.459277}%
\pgfsetstrokecolor{currentstroke}%
\pgfsetdash{}{0pt}%
\pgfpathmoveto{\pgfqpoint{6.917735in}{3.083093in}}%
\pgfpathlineto{\pgfqpoint{7.055833in}{3.001854in}}%
\pgfpathlineto{\pgfqpoint{7.125818in}{3.037569in}}%
\pgfpathclose%
\pgfusepath{fill}%
\end{pgfscope}%
\begin{pgfscope}%
\pgfpathrectangle{\pgfqpoint{0.539299in}{0.078740in}}{\pgfqpoint{7.842520in}{7.842520in}}%
\pgfusepath{clip}%
\pgfsetbuttcap%
\pgfsetroundjoin%
\definecolor{currentfill}{rgb}{0.126453,0.570633,0.549841}%
\pgfsetfillcolor{currentfill}%
\pgfsetlinewidth{0.000000pt}%
\definecolor{currentstroke}{rgb}{0.239374,0.735588,0.455688}%
\pgfsetstrokecolor{currentstroke}%
\pgfsetdash{}{0pt}%
\pgfpathmoveto{\pgfqpoint{3.121655in}{4.747134in}}%
\pgfpathlineto{\pgfqpoint{3.249818in}{4.985619in}}%
\pgfpathlineto{\pgfqpoint{3.333165in}{4.962328in}}%
\pgfpathclose%
\pgfusepath{fill}%
\end{pgfscope}%
\begin{pgfscope}%
\pgfpathrectangle{\pgfqpoint{0.539299in}{0.078740in}}{\pgfqpoint{7.842520in}{7.842520in}}%
\pgfusepath{clip}%
\pgfsetbuttcap%
\pgfsetroundjoin%
\definecolor{currentfill}{rgb}{0.283197,0.115680,0.436115}%
\pgfsetfillcolor{currentfill}%
\pgfsetlinewidth{0.000000pt}%
\definecolor{currentstroke}{rgb}{0.246070,0.738910,0.452024}%
\pgfsetstrokecolor{currentstroke}%
\pgfsetdash{}{0pt}%
\pgfpathmoveto{\pgfqpoint{6.779843in}{3.161531in}}%
\pgfpathlineto{\pgfqpoint{6.642120in}{3.235328in}}%
\pgfpathlineto{\pgfqpoint{6.708800in}{3.131236in}}%
\pgfpathclose%
\pgfusepath{fill}%
\end{pgfscope}%
\begin{pgfscope}%
\pgfpathrectangle{\pgfqpoint{0.539299in}{0.078740in}}{\pgfqpoint{7.842520in}{7.842520in}}%
\pgfusepath{clip}%
\pgfsetbuttcap%
\pgfsetroundjoin%
\definecolor{currentfill}{rgb}{0.258965,0.251537,0.524736}%
\pgfsetfillcolor{currentfill}%
\pgfsetlinewidth{0.000000pt}%
\definecolor{currentstroke}{rgb}{0.252899,0.742211,0.448284}%
\pgfsetstrokecolor{currentstroke}%
\pgfsetdash{}{0pt}%
\pgfpathmoveto{\pgfqpoint{5.387297in}{3.592738in}}%
\pgfpathlineto{\pgfqpoint{5.462306in}{3.585484in}}%
\pgfpathlineto{\pgfqpoint{5.326196in}{3.667457in}}%
\pgfpathclose%
\pgfusepath{fill}%
\end{pgfscope}%
\begin{pgfscope}%
\pgfpathrectangle{\pgfqpoint{0.539299in}{0.078740in}}{\pgfqpoint{7.842520in}{7.842520in}}%
\pgfusepath{clip}%
\pgfsetbuttcap%
\pgfsetroundjoin%
\definecolor{currentfill}{rgb}{0.278826,0.175490,0.483397}%
\pgfsetfillcolor{currentfill}%
\pgfsetlinewidth{0.000000pt}%
\definecolor{currentstroke}{rgb}{0.259857,0.745492,0.444467}%
\pgfsetstrokecolor{currentstroke}%
\pgfsetdash{}{0pt}%
\pgfpathmoveto{\pgfqpoint{6.157532in}{3.392366in}}%
\pgfpathlineto{\pgfqpoint{6.084512in}{3.361457in}}%
\pgfpathlineto{\pgfqpoint{6.222345in}{3.303711in}}%
\pgfpathclose%
\pgfusepath{fill}%
\end{pgfscope}%
\begin{pgfscope}%
\pgfpathrectangle{\pgfqpoint{0.539299in}{0.078740in}}{\pgfqpoint{7.842520in}{7.842520in}}%
\pgfusepath{clip}%
\pgfsetbuttcap%
\pgfsetroundjoin%
\definecolor{currentfill}{rgb}{0.243113,0.292092,0.538516}%
\pgfsetfillcolor{currentfill}%
\pgfsetlinewidth{0.000000pt}%
\definecolor{currentstroke}{rgb}{0.266941,0.748751,0.440573}%
\pgfsetstrokecolor{currentstroke}%
\pgfsetdash{}{0pt}%
\pgfpathmoveto{\pgfqpoint{5.190465in}{3.761377in}}%
\pgfpathlineto{\pgfqpoint{5.114785in}{3.799326in}}%
\pgfpathlineto{\pgfqpoint{5.250882in}{3.689752in}}%
\pgfpathclose%
\pgfusepath{fill}%
\end{pgfscope}%
\begin{pgfscope}%
\pgfpathrectangle{\pgfqpoint{0.539299in}{0.078740in}}{\pgfqpoint{7.842520in}{7.842520in}}%
\pgfusepath{clip}%
\pgfsetbuttcap%
\pgfsetroundjoin%
\definecolor{currentfill}{rgb}{0.119738,0.603785,0.541400}%
\pgfsetfillcolor{currentfill}%
\pgfsetlinewidth{0.000000pt}%
\definecolor{currentstroke}{rgb}{0.274149,0.751988,0.436601}%
\pgfsetstrokecolor{currentstroke}%
\pgfsetdash{}{0pt}%
\pgfpathmoveto{\pgfqpoint{3.333165in}{4.962328in}}%
\pgfpathlineto{\pgfqpoint{3.464190in}{5.097259in}}%
\pgfpathlineto{\pgfqpoint{3.547026in}{5.048670in}}%
\pgfpathclose%
\pgfusepath{fill}%
\end{pgfscope}%
\begin{pgfscope}%
\pgfpathrectangle{\pgfqpoint{0.539299in}{0.078740in}}{\pgfqpoint{7.842520in}{7.842520in}}%
\pgfusepath{clip}%
\pgfsetbuttcap%
\pgfsetroundjoin%
\definecolor{currentfill}{rgb}{0.149039,0.508051,0.557250}%
\pgfsetfillcolor{currentfill}%
\pgfsetlinewidth{0.000000pt}%
\definecolor{currentstroke}{rgb}{0.281477,0.755203,0.432552}%
\pgfsetstrokecolor{currentstroke}%
\pgfsetdash{}{0pt}%
\pgfpathmoveto{\pgfqpoint{2.954036in}{4.737238in}}%
\pgfpathlineto{\pgfqpoint{3.038103in}{4.744324in}}%
\pgfpathlineto{\pgfqpoint{2.913780in}{4.395967in}}%
\pgfpathclose%
\pgfusepath{fill}%
\end{pgfscope}%
\begin{pgfscope}%
\pgfpathrectangle{\pgfqpoint{0.539299in}{0.078740in}}{\pgfqpoint{7.842520in}{7.842520in}}%
\pgfusepath{clip}%
\pgfsetbuttcap%
\pgfsetroundjoin%
\definecolor{currentfill}{rgb}{0.278791,0.062145,0.386592}%
\pgfsetfillcolor{currentfill}%
\pgfsetlinewidth{0.000000pt}%
\definecolor{currentstroke}{rgb}{0.288921,0.758394,0.428426}%
\pgfsetstrokecolor{currentstroke}%
\pgfsetdash{}{0pt}%
\pgfpathmoveto{\pgfqpoint{7.125818in}{3.037569in}}%
\pgfpathlineto{\pgfqpoint{7.055833in}{3.001854in}}%
\pgfpathlineto{\pgfqpoint{7.194201in}{2.919934in}}%
\pgfpathclose%
\pgfusepath{fill}%
\end{pgfscope}%
\begin{pgfscope}%
\pgfpathrectangle{\pgfqpoint{0.539299in}{0.078740in}}{\pgfqpoint{7.842520in}{7.842520in}}%
\pgfusepath{clip}%
\pgfsetbuttcap%
\pgfsetroundjoin%
\definecolor{currentfill}{rgb}{0.274952,0.037752,0.364543}%
\pgfsetfillcolor{currentfill}%
\pgfsetlinewidth{0.000000pt}%
\definecolor{currentstroke}{rgb}{0.296479,0.761561,0.424223}%
\pgfsetstrokecolor{currentstroke}%
\pgfsetdash{}{0pt}%
\pgfpathmoveto{\pgfqpoint{7.264065in}{2.962827in}}%
\pgfpathlineto{\pgfqpoint{7.194201in}{2.919934in}}%
\pgfpathlineto{\pgfqpoint{7.402749in}{2.891078in}}%
\pgfpathclose%
\pgfusepath{fill}%
\end{pgfscope}%
\begin{pgfscope}%
\pgfpathrectangle{\pgfqpoint{0.539299in}{0.078740in}}{\pgfqpoint{7.842520in}{7.842520in}}%
\pgfusepath{clip}%
\pgfsetbuttcap%
\pgfsetroundjoin%
\definecolor{currentfill}{rgb}{0.131172,0.555899,0.552459}%
\pgfsetfillcolor{currentfill}%
\pgfsetlinewidth{0.000000pt}%
\definecolor{currentstroke}{rgb}{0.304148,0.764704,0.419943}%
\pgfsetstrokecolor{currentstroke}%
\pgfsetdash{}{0pt}%
\pgfpathmoveto{\pgfqpoint{3.121655in}{4.747134in}}%
\pgfpathlineto{\pgfqpoint{3.038103in}{4.744324in}}%
\pgfpathlineto{\pgfqpoint{3.249818in}{4.985619in}}%
\pgfpathclose%
\pgfusepath{fill}%
\end{pgfscope}%
\begin{pgfscope}%
\pgfpathrectangle{\pgfqpoint{0.539299in}{0.078740in}}{\pgfqpoint{7.842520in}{7.842520in}}%
\pgfusepath{clip}%
\pgfsetbuttcap%
\pgfsetroundjoin%
\definecolor{currentfill}{rgb}{0.262138,0.242286,0.520837}%
\pgfsetfillcolor{currentfill}%
\pgfsetlinewidth{0.000000pt}%
\definecolor{currentstroke}{rgb}{0.311925,0.767822,0.415586}%
\pgfsetstrokecolor{currentstroke}%
\pgfsetdash{}{0pt}%
\pgfpathmoveto{\pgfqpoint{5.387297in}{3.592738in}}%
\pgfpathlineto{\pgfqpoint{5.598833in}{3.513257in}}%
\pgfpathlineto{\pgfqpoint{5.462306in}{3.585484in}}%
\pgfpathclose%
\pgfusepath{fill}%
\end{pgfscope}%
\begin{pgfscope}%
\pgfpathrectangle{\pgfqpoint{0.539299in}{0.078740in}}{\pgfqpoint{7.842520in}{7.842520in}}%
\pgfusepath{clip}%
\pgfsetbuttcap%
\pgfsetroundjoin%
\definecolor{currentfill}{rgb}{0.283072,0.130895,0.449241}%
\pgfsetfillcolor{currentfill}%
\pgfsetlinewidth{0.000000pt}%
\definecolor{currentstroke}{rgb}{0.319809,0.770914,0.411152}%
\pgfsetstrokecolor{currentstroke}%
\pgfsetdash{}{0pt}%
\pgfpathmoveto{\pgfqpoint{6.708800in}{3.131236in}}%
\pgfpathlineto{\pgfqpoint{6.642120in}{3.235328in}}%
\pgfpathlineto{\pgfqpoint{6.570734in}{3.206547in}}%
\pgfpathclose%
\pgfusepath{fill}%
\end{pgfscope}%
\begin{pgfscope}%
\pgfpathrectangle{\pgfqpoint{0.539299in}{0.078740in}}{\pgfqpoint{7.842520in}{7.842520in}}%
\pgfusepath{clip}%
\pgfsetbuttcap%
\pgfsetroundjoin%
\definecolor{currentfill}{rgb}{0.121148,0.592739,0.544641}%
\pgfsetfillcolor{currentfill}%
\pgfsetlinewidth{0.000000pt}%
\definecolor{currentstroke}{rgb}{0.327796,0.773980,0.406640}%
\pgfsetstrokecolor{currentstroke}%
\pgfsetdash{}{0pt}%
\pgfpathmoveto{\pgfqpoint{3.949347in}{5.006539in}}%
\pgfpathlineto{\pgfqpoint{4.030569in}{4.909838in}}%
\pgfpathlineto{\pgfqpoint{3.895878in}{4.988126in}}%
\pgfpathclose%
\pgfusepath{fill}%
\end{pgfscope}%
\begin{pgfscope}%
\pgfpathrectangle{\pgfqpoint{0.539299in}{0.078740in}}{\pgfqpoint{7.842520in}{7.842520in}}%
\pgfusepath{clip}%
\pgfsetbuttcap%
\pgfsetroundjoin%
\definecolor{currentfill}{rgb}{0.280868,0.160771,0.472899}%
\pgfsetfillcolor{currentfill}%
\pgfsetlinewidth{0.000000pt}%
\definecolor{currentstroke}{rgb}{0.335885,0.777018,0.402049}%
\pgfsetstrokecolor{currentstroke}%
\pgfsetdash{}{0pt}%
\pgfpathmoveto{\pgfqpoint{6.432810in}{3.275195in}}%
\pgfpathlineto{\pgfqpoint{6.295060in}{3.336953in}}%
\pgfpathlineto{\pgfqpoint{6.360427in}{3.240693in}}%
\pgfpathclose%
\pgfusepath{fill}%
\end{pgfscope}%
\begin{pgfscope}%
\pgfpathrectangle{\pgfqpoint{0.539299in}{0.078740in}}{\pgfqpoint{7.842520in}{7.842520in}}%
\pgfusepath{clip}%
\pgfsetbuttcap%
\pgfsetroundjoin%
\definecolor{currentfill}{rgb}{0.252194,0.269783,0.531579}%
\pgfsetfillcolor{currentfill}%
\pgfsetlinewidth{0.000000pt}%
\definecolor{currentstroke}{rgb}{0.344074,0.780029,0.397381}%
\pgfsetstrokecolor{currentstroke}%
\pgfsetdash{}{0pt}%
\pgfpathmoveto{\pgfqpoint{5.326196in}{3.667457in}}%
\pgfpathlineto{\pgfqpoint{5.250882in}{3.689752in}}%
\pgfpathlineto{\pgfqpoint{5.387297in}{3.592738in}}%
\pgfpathclose%
\pgfusepath{fill}%
\end{pgfscope}%
\begin{pgfscope}%
\pgfpathrectangle{\pgfqpoint{0.539299in}{0.078740in}}{\pgfqpoint{7.842520in}{7.842520in}}%
\pgfusepath{clip}%
\pgfsetbuttcap%
\pgfsetroundjoin%
\definecolor{currentfill}{rgb}{0.119699,0.618490,0.536347}%
\pgfsetfillcolor{currentfill}%
\pgfsetlinewidth{0.000000pt}%
\definecolor{currentstroke}{rgb}{0.352360,0.783011,0.392636}%
\pgfsetstrokecolor{currentstroke}%
\pgfsetdash{}{0pt}%
\pgfpathmoveto{\pgfqpoint{3.547026in}{5.048670in}}%
\pgfpathlineto{\pgfqpoint{3.464190in}{5.097259in}}%
\pgfpathlineto{\pgfqpoint{3.679940in}{5.091602in}}%
\pgfpathclose%
\pgfusepath{fill}%
\end{pgfscope}%
\begin{pgfscope}%
\pgfpathrectangle{\pgfqpoint{0.539299in}{0.078740in}}{\pgfqpoint{7.842520in}{7.842520in}}%
\pgfusepath{clip}%
\pgfsetbuttcap%
\pgfsetroundjoin%
\definecolor{currentfill}{rgb}{0.119423,0.611141,0.538982}%
\pgfsetfillcolor{currentfill}%
\pgfsetlinewidth{0.000000pt}%
\definecolor{currentstroke}{rgb}{0.360741,0.785964,0.387814}%
\pgfsetstrokecolor{currentstroke}%
\pgfsetdash{}{0pt}%
\pgfpathmoveto{\pgfqpoint{3.895878in}{4.988126in}}%
\pgfpathlineto{\pgfqpoint{3.679940in}{5.091602in}}%
\pgfpathlineto{\pgfqpoint{3.814186in}{5.073908in}}%
\pgfpathclose%
\pgfusepath{fill}%
\end{pgfscope}%
\begin{pgfscope}%
\pgfpathrectangle{\pgfqpoint{0.539299in}{0.078740in}}{\pgfqpoint{7.842520in}{7.842520in}}%
\pgfusepath{clip}%
\pgfsetbuttcap%
\pgfsetroundjoin%
\definecolor{currentfill}{rgb}{0.282656,0.100196,0.422160}%
\pgfsetfillcolor{currentfill}%
\pgfsetlinewidth{0.000000pt}%
\definecolor{currentstroke}{rgb}{0.369214,0.788888,0.382914}%
\pgfsetstrokecolor{currentstroke}%
\pgfsetdash{}{0pt}%
\pgfpathmoveto{\pgfqpoint{6.847001in}{3.050168in}}%
\pgfpathlineto{\pgfqpoint{6.917735in}{3.083093in}}%
\pgfpathlineto{\pgfqpoint{6.779843in}{3.161531in}}%
\pgfpathclose%
\pgfusepath{fill}%
\end{pgfscope}%
\begin{pgfscope}%
\pgfpathrectangle{\pgfqpoint{0.539299in}{0.078740in}}{\pgfqpoint{7.842520in}{7.842520in}}%
\pgfusepath{clip}%
\pgfsetbuttcap%
\pgfsetroundjoin%
\definecolor{currentfill}{rgb}{0.273006,0.204520,0.501721}%
\pgfsetfillcolor{currentfill}%
\pgfsetlinewidth{0.000000pt}%
\definecolor{currentstroke}{rgb}{0.377779,0.791781,0.377939}%
\pgfsetstrokecolor{currentstroke}%
\pgfsetdash{}{0pt}%
\pgfpathmoveto{\pgfqpoint{5.946992in}{3.415751in}}%
\pgfpathlineto{\pgfqpoint{5.735793in}{3.448039in}}%
\pgfpathlineto{\pgfqpoint{5.873171in}{3.386859in}}%
\pgfpathclose%
\pgfusepath{fill}%
\end{pgfscope}%
\begin{pgfscope}%
\pgfpathrectangle{\pgfqpoint{0.539299in}{0.078740in}}{\pgfqpoint{7.842520in}{7.842520in}}%
\pgfusepath{clip}%
\pgfsetbuttcap%
\pgfsetroundjoin%
\definecolor{currentfill}{rgb}{0.159194,0.482237,0.558073}%
\pgfsetfillcolor{currentfill}%
\pgfsetlinewidth{0.000000pt}%
\definecolor{currentstroke}{rgb}{0.386433,0.794644,0.372886}%
\pgfsetstrokecolor{currentstroke}%
\pgfsetdash{}{0pt}%
\pgfpathmoveto{\pgfqpoint{2.830306in}{4.368134in}}%
\pgfpathlineto{\pgfqpoint{2.746363in}{4.336760in}}%
\pgfpathlineto{\pgfqpoint{2.954036in}{4.737238in}}%
\pgfpathclose%
\pgfusepath{fill}%
\end{pgfscope}%
\begin{pgfscope}%
\pgfpathrectangle{\pgfqpoint{0.539299in}{0.078740in}}{\pgfqpoint{7.842520in}{7.842520in}}%
\pgfusepath{clip}%
\pgfsetbuttcap%
\pgfsetroundjoin%
\definecolor{currentfill}{rgb}{0.276194,0.190074,0.493001}%
\pgfsetfillcolor{currentfill}%
\pgfsetlinewidth{0.000000pt}%
\definecolor{currentstroke}{rgb}{0.395174,0.797475,0.367757}%
\pgfsetstrokecolor{currentstroke}%
\pgfsetdash{}{0pt}%
\pgfpathmoveto{\pgfqpoint{6.010937in}{3.326788in}}%
\pgfpathlineto{\pgfqpoint{6.084512in}{3.361457in}}%
\pgfpathlineto{\pgfqpoint{5.946992in}{3.415751in}}%
\pgfpathclose%
\pgfusepath{fill}%
\end{pgfscope}%
\begin{pgfscope}%
\pgfpathrectangle{\pgfqpoint{0.539299in}{0.078740in}}{\pgfqpoint{7.842520in}{7.842520in}}%
\pgfusepath{clip}%
\pgfsetbuttcap%
\pgfsetroundjoin%
\definecolor{currentfill}{rgb}{0.280255,0.165693,0.476498}%
\pgfsetfillcolor{currentfill}%
\pgfsetlinewidth{0.000000pt}%
\definecolor{currentstroke}{rgb}{0.404001,0.800275,0.362552}%
\pgfsetstrokecolor{currentstroke}%
\pgfsetdash{}{0pt}%
\pgfpathmoveto{\pgfqpoint{6.360427in}{3.240693in}}%
\pgfpathlineto{\pgfqpoint{6.295060in}{3.336953in}}%
\pgfpathlineto{\pgfqpoint{6.222345in}{3.303711in}}%
\pgfpathclose%
\pgfusepath{fill}%
\end{pgfscope}%
\begin{pgfscope}%
\pgfpathrectangle{\pgfqpoint{0.539299in}{0.078740in}}{\pgfqpoint{7.842520in}{7.842520in}}%
\pgfusepath{clip}%
\pgfsetbuttcap%
\pgfsetroundjoin%
\definecolor{currentfill}{rgb}{0.126453,0.570633,0.549841}%
\pgfsetfillcolor{currentfill}%
\pgfsetlinewidth{0.000000pt}%
\definecolor{currentstroke}{rgb}{0.412913,0.803041,0.357269}%
\pgfsetstrokecolor{currentstroke}%
\pgfsetdash{}{0pt}%
\pgfpathmoveto{\pgfqpoint{4.165728in}{4.797221in}}%
\pgfpathlineto{\pgfqpoint{4.030569in}{4.909838in}}%
\pgfpathlineto{\pgfqpoint{4.085083in}{4.900621in}}%
\pgfpathclose%
\pgfusepath{fill}%
\end{pgfscope}%
\begin{pgfscope}%
\pgfpathrectangle{\pgfqpoint{0.539299in}{0.078740in}}{\pgfqpoint{7.842520in}{7.842520in}}%
\pgfusepath{clip}%
\pgfsetbuttcap%
\pgfsetroundjoin%
\definecolor{currentfill}{rgb}{0.281887,0.150881,0.465405}%
\pgfsetfillcolor{currentfill}%
\pgfsetlinewidth{0.000000pt}%
\definecolor{currentstroke}{rgb}{0.421908,0.805774,0.351910}%
\pgfsetstrokecolor{currentstroke}%
\pgfsetdash{}{0pt}%
\pgfpathmoveto{\pgfqpoint{6.570734in}{3.206547in}}%
\pgfpathlineto{\pgfqpoint{6.432810in}{3.275195in}}%
\pgfpathlineto{\pgfqpoint{6.360427in}{3.240693in}}%
\pgfpathclose%
\pgfusepath{fill}%
\end{pgfscope}%
\begin{pgfscope}%
\pgfpathrectangle{\pgfqpoint{0.539299in}{0.078740in}}{\pgfqpoint{7.842520in}{7.842520in}}%
\pgfusepath{clip}%
\pgfsetbuttcap%
\pgfsetroundjoin%
\definecolor{currentfill}{rgb}{0.281924,0.089666,0.412415}%
\pgfsetfillcolor{currentfill}%
\pgfsetlinewidth{0.000000pt}%
\definecolor{currentstroke}{rgb}{0.430983,0.808473,0.346476}%
\pgfsetstrokecolor{currentstroke}%
\pgfsetdash{}{0pt}%
\pgfpathmoveto{\pgfqpoint{6.847001in}{3.050168in}}%
\pgfpathlineto{\pgfqpoint{7.055833in}{3.001854in}}%
\pgfpathlineto{\pgfqpoint{6.917735in}{3.083093in}}%
\pgfpathclose%
\pgfusepath{fill}%
\end{pgfscope}%
\begin{pgfscope}%
\pgfpathrectangle{\pgfqpoint{0.539299in}{0.078740in}}{\pgfqpoint{7.842520in}{7.842520in}}%
\pgfusepath{clip}%
\pgfsetbuttcap%
\pgfsetroundjoin%
\definecolor{currentfill}{rgb}{0.283197,0.115680,0.436115}%
\pgfsetfillcolor{currentfill}%
\pgfsetlinewidth{0.000000pt}%
\definecolor{currentstroke}{rgb}{0.440137,0.811138,0.340967}%
\pgfsetstrokecolor{currentstroke}%
\pgfsetdash{}{0pt}%
\pgfpathmoveto{\pgfqpoint{6.779843in}{3.161531in}}%
\pgfpathlineto{\pgfqpoint{6.708800in}{3.131236in}}%
\pgfpathlineto{\pgfqpoint{6.847001in}{3.050168in}}%
\pgfpathclose%
\pgfusepath{fill}%
\end{pgfscope}%
\begin{pgfscope}%
\pgfpathrectangle{\pgfqpoint{0.539299in}{0.078740in}}{\pgfqpoint{7.842520in}{7.842520in}}%
\pgfusepath{clip}%
\pgfsetbuttcap%
\pgfsetroundjoin%
\definecolor{currentfill}{rgb}{0.266580,0.228262,0.514349}%
\pgfsetfillcolor{currentfill}%
\pgfsetlinewidth{0.000000pt}%
\definecolor{currentstroke}{rgb}{0.449368,0.813768,0.335384}%
\pgfsetstrokecolor{currentstroke}%
\pgfsetdash{}{0pt}%
\pgfpathmoveto{\pgfqpoint{5.735793in}{3.448039in}}%
\pgfpathlineto{\pgfqpoint{5.598833in}{3.513257in}}%
\pgfpathlineto{\pgfqpoint{5.524083in}{3.506700in}}%
\pgfpathclose%
\pgfusepath{fill}%
\end{pgfscope}%
\begin{pgfscope}%
\pgfpathrectangle{\pgfqpoint{0.539299in}{0.078740in}}{\pgfqpoint{7.842520in}{7.842520in}}%
\pgfusepath{clip}%
\pgfsetbuttcap%
\pgfsetroundjoin%
\definecolor{currentfill}{rgb}{0.131172,0.555899,0.552459}%
\pgfsetfillcolor{currentfill}%
\pgfsetlinewidth{0.000000pt}%
\definecolor{currentstroke}{rgb}{0.458674,0.816363,0.329727}%
\pgfsetstrokecolor{currentstroke}%
\pgfsetdash{}{0pt}%
\pgfpathmoveto{\pgfqpoint{4.085083in}{4.900621in}}%
\pgfpathlineto{\pgfqpoint{4.301131in}{4.661130in}}%
\pgfpathlineto{\pgfqpoint{4.165728in}{4.797221in}}%
\pgfpathclose%
\pgfusepath{fill}%
\end{pgfscope}%
\begin{pgfscope}%
\pgfpathrectangle{\pgfqpoint{0.539299in}{0.078740in}}{\pgfqpoint{7.842520in}{7.842520in}}%
\pgfusepath{clip}%
\pgfsetbuttcap%
\pgfsetroundjoin%
\definecolor{currentfill}{rgb}{0.119423,0.611141,0.538982}%
\pgfsetfillcolor{currentfill}%
\pgfsetlinewidth{0.000000pt}%
\definecolor{currentstroke}{rgb}{0.468053,0.818921,0.323998}%
\pgfsetstrokecolor{currentstroke}%
\pgfsetdash{}{0pt}%
\pgfpathmoveto{\pgfqpoint{3.814186in}{5.073908in}}%
\pgfpathlineto{\pgfqpoint{3.949347in}{5.006539in}}%
\pgfpathlineto{\pgfqpoint{3.895878in}{4.988126in}}%
\pgfpathclose%
\pgfusepath{fill}%
\end{pgfscope}%
\begin{pgfscope}%
\pgfpathrectangle{\pgfqpoint{0.539299in}{0.078740in}}{\pgfqpoint{7.842520in}{7.842520in}}%
\pgfusepath{clip}%
\pgfsetbuttcap%
\pgfsetroundjoin%
\definecolor{currentfill}{rgb}{0.273809,0.031497,0.358853}%
\pgfsetfillcolor{currentfill}%
\pgfsetlinewidth{0.000000pt}%
\definecolor{currentstroke}{rgb}{0.477504,0.821444,0.318195}%
\pgfsetstrokecolor{currentstroke}%
\pgfsetdash{}{0pt}%
\pgfpathmoveto{\pgfqpoint{7.402749in}{2.891078in}}%
\pgfpathlineto{\pgfqpoint{7.194201in}{2.919934in}}%
\pgfpathlineto{\pgfqpoint{7.332923in}{2.839586in}}%
\pgfpathclose%
\pgfusepath{fill}%
\end{pgfscope}%
\begin{pgfscope}%
\pgfpathrectangle{\pgfqpoint{0.539299in}{0.078740in}}{\pgfqpoint{7.842520in}{7.842520in}}%
\pgfusepath{clip}%
\pgfsetbuttcap%
\pgfsetroundjoin%
\definecolor{currentfill}{rgb}{0.275191,0.194905,0.496005}%
\pgfsetfillcolor{currentfill}%
\pgfsetlinewidth{0.000000pt}%
\definecolor{currentstroke}{rgb}{0.487026,0.823929,0.312321}%
\pgfsetstrokecolor{currentstroke}%
\pgfsetdash{}{0pt}%
\pgfpathmoveto{\pgfqpoint{5.946992in}{3.415751in}}%
\pgfpathlineto{\pgfqpoint{5.873171in}{3.386859in}}%
\pgfpathlineto{\pgfqpoint{6.010937in}{3.326788in}}%
\pgfpathclose%
\pgfusepath{fill}%
\end{pgfscope}%
\begin{pgfscope}%
\pgfpathrectangle{\pgfqpoint{0.539299in}{0.078740in}}{\pgfqpoint{7.842520in}{7.842520in}}%
\pgfusepath{clip}%
\pgfsetbuttcap%
\pgfsetroundjoin%
\definecolor{currentfill}{rgb}{0.147607,0.511733,0.557049}%
\pgfsetfillcolor{currentfill}%
\pgfsetlinewidth{0.000000pt}%
\definecolor{currentstroke}{rgb}{0.496615,0.826376,0.306377}%
\pgfsetstrokecolor{currentstroke}%
\pgfsetdash{}{0pt}%
\pgfpathmoveto{\pgfqpoint{4.301131in}{4.661130in}}%
\pgfpathlineto{\pgfqpoint{4.357325in}{4.615401in}}%
\pgfpathlineto{\pgfqpoint{4.436631in}{4.511408in}}%
\pgfpathclose%
\pgfusepath{fill}%
\end{pgfscope}%
\begin{pgfscope}%
\pgfpathrectangle{\pgfqpoint{0.539299in}{0.078740in}}{\pgfqpoint{7.842520in}{7.842520in}}%
\pgfusepath{clip}%
\pgfsetbuttcap%
\pgfsetroundjoin%
\definecolor{currentfill}{rgb}{0.162142,0.474838,0.558140}%
\pgfsetfillcolor{currentfill}%
\pgfsetlinewidth{0.000000pt}%
\definecolor{currentstroke}{rgb}{0.506271,0.828786,0.300362}%
\pgfsetstrokecolor{currentstroke}%
\pgfsetdash{}{0pt}%
\pgfpathmoveto{\pgfqpoint{4.493544in}{4.454843in}}%
\pgfpathlineto{\pgfqpoint{4.572157in}{4.356552in}}%
\pgfpathlineto{\pgfqpoint{4.436631in}{4.511408in}}%
\pgfpathclose%
\pgfusepath{fill}%
\end{pgfscope}%
\begin{pgfscope}%
\pgfpathrectangle{\pgfqpoint{0.539299in}{0.078740in}}{\pgfqpoint{7.842520in}{7.842520in}}%
\pgfusepath{clip}%
\pgfsetbuttcap%
\pgfsetroundjoin%
\definecolor{currentfill}{rgb}{0.171176,0.452530,0.557965}%
\pgfsetfillcolor{currentfill}%
\pgfsetlinewidth{0.000000pt}%
\definecolor{currentstroke}{rgb}{0.515992,0.831158,0.294279}%
\pgfsetstrokecolor{currentstroke}%
\pgfsetdash{}{0pt}%
\pgfpathmoveto{\pgfqpoint{4.572157in}{4.356552in}}%
\pgfpathlineto{\pgfqpoint{4.493544in}{4.454843in}}%
\pgfpathlineto{\pgfqpoint{4.707691in}{4.203491in}}%
\pgfpathclose%
\pgfusepath{fill}%
\end{pgfscope}%
\begin{pgfscope}%
\pgfpathrectangle{\pgfqpoint{0.539299in}{0.078740in}}{\pgfqpoint{7.842520in}{7.842520in}}%
\pgfusepath{clip}%
\pgfsetbuttcap%
\pgfsetroundjoin%
\definecolor{currentfill}{rgb}{0.192357,0.403199,0.555836}%
\pgfsetfillcolor{currentfill}%
\pgfsetlinewidth{0.000000pt}%
\definecolor{currentstroke}{rgb}{0.525776,0.833491,0.288127}%
\pgfsetstrokecolor{currentstroke}%
\pgfsetdash{}{0pt}%
\pgfpathmoveto{\pgfqpoint{4.707691in}{4.203491in}}%
\pgfpathlineto{\pgfqpoint{4.765949in}{4.134630in}}%
\pgfpathlineto{\pgfqpoint{4.843266in}{4.057477in}}%
\pgfpathclose%
\pgfusepath{fill}%
\end{pgfscope}%
\begin{pgfscope}%
\pgfpathrectangle{\pgfqpoint{0.539299in}{0.078740in}}{\pgfqpoint{7.842520in}{7.842520in}}%
\pgfusepath{clip}%
\pgfsetbuttcap%
\pgfsetroundjoin%
\definecolor{currentfill}{rgb}{0.269944,0.014625,0.341379}%
\pgfsetfillcolor{currentfill}%
\pgfsetlinewidth{0.000000pt}%
\definecolor{currentstroke}{rgb}{0.535621,0.835785,0.281908}%
\pgfsetstrokecolor{currentstroke}%
\pgfsetdash{}{0pt}%
\pgfpathmoveto{\pgfqpoint{7.472102in}{2.763114in}}%
\pgfpathlineto{\pgfqpoint{7.541973in}{2.824378in}}%
\pgfpathlineto{\pgfqpoint{7.402749in}{2.891078in}}%
\pgfpathclose%
\pgfusepath{fill}%
\end{pgfscope}%
\begin{pgfscope}%
\pgfpathrectangle{\pgfqpoint{0.539299in}{0.078740in}}{\pgfqpoint{7.842520in}{7.842520in}}%
\pgfusepath{clip}%
\pgfsetbuttcap%
\pgfsetroundjoin%
\definecolor{currentfill}{rgb}{0.218130,0.347432,0.550038}%
\pgfsetfillcolor{currentfill}%
\pgfsetlinewidth{0.000000pt}%
\definecolor{currentstroke}{rgb}{0.545524,0.838039,0.275626}%
\pgfsetstrokecolor{currentstroke}%
\pgfsetdash{}{0pt}%
\pgfpathmoveto{\pgfqpoint{2.577164in}{4.260977in}}%
\pgfpathlineto{\pgfqpoint{2.459954in}{3.750514in}}%
\pgfpathlineto{\pgfqpoint{2.375504in}{3.692390in}}%
\pgfpathclose%
\pgfusepath{fill}%
\end{pgfscope}%
\begin{pgfscope}%
\pgfpathrectangle{\pgfqpoint{0.539299in}{0.078740in}}{\pgfqpoint{7.842520in}{7.842520in}}%
\pgfusepath{clip}%
\pgfsetbuttcap%
\pgfsetroundjoin%
\definecolor{currentfill}{rgb}{0.260571,0.246922,0.522828}%
\pgfsetfillcolor{currentfill}%
\pgfsetlinewidth{0.000000pt}%
\definecolor{currentstroke}{rgb}{0.555484,0.840254,0.269281}%
\pgfsetstrokecolor{currentstroke}%
\pgfsetdash{}{0pt}%
\pgfpathmoveto{\pgfqpoint{5.524083in}{3.506700in}}%
\pgfpathlineto{\pgfqpoint{5.598833in}{3.513257in}}%
\pgfpathlineto{\pgfqpoint{5.387297in}{3.592738in}}%
\pgfpathclose%
\pgfusepath{fill}%
\end{pgfscope}%
\begin{pgfscope}%
\pgfpathrectangle{\pgfqpoint{0.539299in}{0.078740in}}{\pgfqpoint{7.842520in}{7.842520in}}%
\pgfusepath{clip}%
\pgfsetbuttcap%
\pgfsetroundjoin%
\definecolor{currentfill}{rgb}{0.203063,0.379716,0.553925}%
\pgfsetfillcolor{currentfill}%
\pgfsetlinewidth{0.000000pt}%
\definecolor{currentstroke}{rgb}{0.565498,0.842430,0.262877}%
\pgfsetstrokecolor{currentstroke}%
\pgfsetdash{}{0pt}%
\pgfpathmoveto{\pgfqpoint{4.765949in}{4.134630in}}%
\pgfpathlineto{\pgfqpoint{4.978938in}{3.922091in}}%
\pgfpathlineto{\pgfqpoint{4.843266in}{4.057477in}}%
\pgfpathclose%
\pgfusepath{fill}%
\end{pgfscope}%
\begin{pgfscope}%
\pgfpathrectangle{\pgfqpoint{0.539299in}{0.078740in}}{\pgfqpoint{7.842520in}{7.842520in}}%
\pgfusepath{clip}%
\pgfsetbuttcap%
\pgfsetroundjoin%
\definecolor{currentfill}{rgb}{0.218130,0.347432,0.550038}%
\pgfsetfillcolor{currentfill}%
\pgfsetlinewidth{0.000000pt}%
\definecolor{currentstroke}{rgb}{0.575563,0.844566,0.256415}%
\pgfsetstrokecolor{currentstroke}%
\pgfsetdash{}{0pt}%
\pgfpathmoveto{\pgfqpoint{4.978938in}{3.922091in}}%
\pgfpathlineto{\pgfqpoint{4.902184in}{3.985093in}}%
\pgfpathlineto{\pgfqpoint{5.114785in}{3.799326in}}%
\pgfpathclose%
\pgfusepath{fill}%
\end{pgfscope}%
\begin{pgfscope}%
\pgfpathrectangle{\pgfqpoint{0.539299in}{0.078740in}}{\pgfqpoint{7.842520in}{7.842520in}}%
\pgfusepath{clip}%
\pgfsetbuttcap%
\pgfsetroundjoin%
\definecolor{currentfill}{rgb}{0.271305,0.019942,0.347269}%
\pgfsetfillcolor{currentfill}%
\pgfsetlinewidth{0.000000pt}%
\definecolor{currentstroke}{rgb}{0.585678,0.846661,0.249897}%
\pgfsetstrokecolor{currentstroke}%
\pgfsetdash{}{0pt}%
\pgfpathmoveto{\pgfqpoint{7.402749in}{2.891078in}}%
\pgfpathlineto{\pgfqpoint{7.332923in}{2.839586in}}%
\pgfpathlineto{\pgfqpoint{7.472102in}{2.763114in}}%
\pgfpathclose%
\pgfusepath{fill}%
\end{pgfscope}%
\begin{pgfscope}%
\pgfpathrectangle{\pgfqpoint{0.539299in}{0.078740in}}{\pgfqpoint{7.842520in}{7.842520in}}%
\pgfusepath{clip}%
\pgfsetbuttcap%
\pgfsetroundjoin%
\definecolor{currentfill}{rgb}{0.278012,0.180367,0.486697}%
\pgfsetfillcolor{currentfill}%
\pgfsetlinewidth{0.000000pt}%
\definecolor{currentstroke}{rgb}{0.595839,0.848717,0.243329}%
\pgfsetstrokecolor{currentstroke}%
\pgfsetdash{}{0pt}%
\pgfpathmoveto{\pgfqpoint{6.084512in}{3.361457in}}%
\pgfpathlineto{\pgfqpoint{6.149041in}{3.265181in}}%
\pgfpathlineto{\pgfqpoint{6.222345in}{3.303711in}}%
\pgfpathclose%
\pgfusepath{fill}%
\end{pgfscope}%
\begin{pgfscope}%
\pgfpathrectangle{\pgfqpoint{0.539299in}{0.078740in}}{\pgfqpoint{7.842520in}{7.842520in}}%
\pgfusepath{clip}%
\pgfsetbuttcap%
\pgfsetroundjoin%
\definecolor{currentfill}{rgb}{0.282623,0.140926,0.457517}%
\pgfsetfillcolor{currentfill}%
\pgfsetlinewidth{0.000000pt}%
\definecolor{currentstroke}{rgb}{0.606045,0.850733,0.236712}%
\pgfsetstrokecolor{currentstroke}%
\pgfsetdash{}{0pt}%
\pgfpathmoveto{\pgfqpoint{6.570734in}{3.206547in}}%
\pgfpathlineto{\pgfqpoint{6.498702in}{3.171293in}}%
\pgfpathlineto{\pgfqpoint{6.708800in}{3.131236in}}%
\pgfpathclose%
\pgfusepath{fill}%
\end{pgfscope}%
\begin{pgfscope}%
\pgfpathrectangle{\pgfqpoint{0.539299in}{0.078740in}}{\pgfqpoint{7.842520in}{7.842520in}}%
\pgfusepath{clip}%
\pgfsetbuttcap%
\pgfsetroundjoin%
\definecolor{currentfill}{rgb}{0.120565,0.596422,0.543611}%
\pgfsetfillcolor{currentfill}%
\pgfsetlinewidth{0.000000pt}%
\definecolor{currentstroke}{rgb}{0.616293,0.852709,0.230052}%
\pgfsetstrokecolor{currentstroke}%
\pgfsetdash{}{0pt}%
\pgfpathmoveto{\pgfqpoint{4.085083in}{4.900621in}}%
\pgfpathlineto{\pgfqpoint{4.030569in}{4.909838in}}%
\pgfpathlineto{\pgfqpoint{3.949347in}{5.006539in}}%
\pgfpathclose%
\pgfusepath{fill}%
\end{pgfscope}%
\begin{pgfscope}%
\pgfpathrectangle{\pgfqpoint{0.539299in}{0.078740in}}{\pgfqpoint{7.842520in}{7.842520in}}%
\pgfusepath{clip}%
\pgfsetbuttcap%
\pgfsetroundjoin%
\definecolor{currentfill}{rgb}{0.120638,0.625828,0.533488}%
\pgfsetfillcolor{currentfill}%
\pgfsetlinewidth{0.000000pt}%
\definecolor{currentstroke}{rgb}{0.626579,0.854645,0.223353}%
\pgfsetstrokecolor{currentstroke}%
\pgfsetdash{}{0pt}%
\pgfpathmoveto{\pgfqpoint{3.464190in}{5.097259in}}%
\pgfpathlineto{\pgfqpoint{3.333165in}{4.962328in}}%
\pgfpathlineto{\pgfqpoint{3.380711in}{5.141597in}}%
\pgfpathclose%
\pgfusepath{fill}%
\end{pgfscope}%
\begin{pgfscope}%
\pgfpathrectangle{\pgfqpoint{0.539299in}{0.078740in}}{\pgfqpoint{7.842520in}{7.842520in}}%
\pgfusepath{clip}%
\pgfsetbuttcap%
\pgfsetroundjoin%
\definecolor{currentfill}{rgb}{0.278791,0.062145,0.386592}%
\pgfsetfillcolor{currentfill}%
\pgfsetlinewidth{0.000000pt}%
\definecolor{currentstroke}{rgb}{0.636902,0.856542,0.216620}%
\pgfsetstrokecolor{currentstroke}%
\pgfsetdash{}{0pt}%
\pgfpathmoveto{\pgfqpoint{7.055833in}{3.001854in}}%
\pgfpathlineto{\pgfqpoint{7.123901in}{2.876974in}}%
\pgfpathlineto{\pgfqpoint{7.194201in}{2.919934in}}%
\pgfpathclose%
\pgfusepath{fill}%
\end{pgfscope}%
\begin{pgfscope}%
\pgfpathrectangle{\pgfqpoint{0.539299in}{0.078740in}}{\pgfqpoint{7.842520in}{7.842520in}}%
\pgfusepath{clip}%
\pgfsetbuttcap%
\pgfsetroundjoin%
\definecolor{currentfill}{rgb}{0.119699,0.618490,0.536347}%
\pgfsetfillcolor{currentfill}%
\pgfsetlinewidth{0.000000pt}%
\definecolor{currentstroke}{rgb}{0.647257,0.858400,0.209861}%
\pgfsetstrokecolor{currentstroke}%
\pgfsetdash{}{0pt}%
\pgfpathmoveto{\pgfqpoint{3.380711in}{5.141597in}}%
\pgfpathlineto{\pgfqpoint{3.333165in}{4.962328in}}%
\pgfpathlineto{\pgfqpoint{3.249818in}{4.985619in}}%
\pgfpathclose%
\pgfusepath{fill}%
\end{pgfscope}%
\begin{pgfscope}%
\pgfpathrectangle{\pgfqpoint{0.539299in}{0.078740in}}{\pgfqpoint{7.842520in}{7.842520in}}%
\pgfusepath{clip}%
\pgfsetbuttcap%
\pgfsetroundjoin%
\definecolor{currentfill}{rgb}{0.281924,0.089666,0.412415}%
\pgfsetfillcolor{currentfill}%
\pgfsetlinewidth{0.000000pt}%
\definecolor{currentstroke}{rgb}{0.657642,0.860219,0.203082}%
\pgfsetstrokecolor{currentstroke}%
\pgfsetdash{}{0pt}%
\pgfpathmoveto{\pgfqpoint{6.985354in}{2.964798in}}%
\pgfpathlineto{\pgfqpoint{7.055833in}{3.001854in}}%
\pgfpathlineto{\pgfqpoint{6.847001in}{3.050168in}}%
\pgfpathclose%
\pgfusepath{fill}%
\end{pgfscope}%
\begin{pgfscope}%
\pgfpathrectangle{\pgfqpoint{0.539299in}{0.078740in}}{\pgfqpoint{7.842520in}{7.842520in}}%
\pgfusepath{clip}%
\pgfsetbuttcap%
\pgfsetroundjoin%
\definecolor{currentfill}{rgb}{0.281412,0.155834,0.469201}%
\pgfsetfillcolor{currentfill}%
\pgfsetlinewidth{0.000000pt}%
\definecolor{currentstroke}{rgb}{0.668054,0.861999,0.196293}%
\pgfsetstrokecolor{currentstroke}%
\pgfsetdash{}{0pt}%
\pgfpathmoveto{\pgfqpoint{6.360427in}{3.240693in}}%
\pgfpathlineto{\pgfqpoint{6.498702in}{3.171293in}}%
\pgfpathlineto{\pgfqpoint{6.570734in}{3.206547in}}%
\pgfpathclose%
\pgfusepath{fill}%
\end{pgfscope}%
\begin{pgfscope}%
\pgfpathrectangle{\pgfqpoint{0.539299in}{0.078740in}}{\pgfqpoint{7.842520in}{7.842520in}}%
\pgfusepath{clip}%
\pgfsetbuttcap%
\pgfsetroundjoin%
\definecolor{currentfill}{rgb}{0.235526,0.309527,0.542944}%
\pgfsetfillcolor{currentfill}%
\pgfsetlinewidth{0.000000pt}%
\definecolor{currentstroke}{rgb}{0.678489,0.863742,0.189503}%
\pgfsetstrokecolor{currentstroke}%
\pgfsetdash{}{0pt}%
\pgfpathmoveto{\pgfqpoint{5.175033in}{3.721001in}}%
\pgfpathlineto{\pgfqpoint{5.250882in}{3.689752in}}%
\pgfpathlineto{\pgfqpoint{5.114785in}{3.799326in}}%
\pgfpathclose%
\pgfusepath{fill}%
\end{pgfscope}%
\begin{pgfscope}%
\pgfpathrectangle{\pgfqpoint{0.539299in}{0.078740in}}{\pgfqpoint{7.842520in}{7.842520in}}%
\pgfusepath{clip}%
\pgfsetbuttcap%
\pgfsetroundjoin%
\definecolor{currentfill}{rgb}{0.192357,0.403199,0.555836}%
\pgfsetfillcolor{currentfill}%
\pgfsetlinewidth{0.000000pt}%
\definecolor{currentstroke}{rgb}{0.688944,0.865448,0.182725}%
\pgfsetstrokecolor{currentstroke}%
\pgfsetdash{}{0pt}%
\pgfpathmoveto{\pgfqpoint{2.459954in}{3.750514in}}%
\pgfpathlineto{\pgfqpoint{2.577164in}{4.260977in}}%
\pgfpathlineto{\pgfqpoint{2.661974in}{4.301274in}}%
\pgfpathclose%
\pgfusepath{fill}%
\end{pgfscope}%
\begin{pgfscope}%
\pgfpathrectangle{\pgfqpoint{0.539299in}{0.078740in}}{\pgfqpoint{7.842520in}{7.842520in}}%
\pgfusepath{clip}%
\pgfsetbuttcap%
\pgfsetroundjoin%
\definecolor{currentfill}{rgb}{0.265145,0.232956,0.516599}%
\pgfsetfillcolor{currentfill}%
\pgfsetlinewidth{0.000000pt}%
\definecolor{currentstroke}{rgb}{0.699415,0.867117,0.175971}%
\pgfsetstrokecolor{currentstroke}%
\pgfsetdash{}{0pt}%
\pgfpathmoveto{\pgfqpoint{5.524083in}{3.506700in}}%
\pgfpathlineto{\pgfqpoint{5.661270in}{3.429383in}}%
\pgfpathlineto{\pgfqpoint{5.735793in}{3.448039in}}%
\pgfpathclose%
\pgfusepath{fill}%
\end{pgfscope}%
\begin{pgfscope}%
\pgfpathrectangle{\pgfqpoint{0.539299in}{0.078740in}}{\pgfqpoint{7.842520in}{7.842520in}}%
\pgfusepath{clip}%
\pgfsetbuttcap%
\pgfsetroundjoin%
\definecolor{currentfill}{rgb}{0.124780,0.640461,0.527068}%
\pgfsetfillcolor{currentfill}%
\pgfsetlinewidth{0.000000pt}%
\definecolor{currentstroke}{rgb}{0.709898,0.868751,0.169257}%
\pgfsetstrokecolor{currentstroke}%
\pgfsetdash{}{0pt}%
\pgfpathmoveto{\pgfqpoint{3.679940in}{5.091602in}}%
\pgfpathlineto{\pgfqpoint{3.464190in}{5.097259in}}%
\pgfpathlineto{\pgfqpoint{3.597214in}{5.159015in}}%
\pgfpathclose%
\pgfusepath{fill}%
\end{pgfscope}%
\begin{pgfscope}%
\pgfpathrectangle{\pgfqpoint{0.539299in}{0.078740in}}{\pgfqpoint{7.842520in}{7.842520in}}%
\pgfusepath{clip}%
\pgfsetbuttcap%
\pgfsetroundjoin%
\definecolor{currentfill}{rgb}{0.270595,0.214069,0.507052}%
\pgfsetfillcolor{currentfill}%
\pgfsetlinewidth{0.000000pt}%
\definecolor{currentstroke}{rgb}{0.720391,0.870350,0.162603}%
\pgfsetstrokecolor{currentstroke}%
\pgfsetdash{}{0pt}%
\pgfpathmoveto{\pgfqpoint{5.798863in}{3.358134in}}%
\pgfpathlineto{\pgfqpoint{5.873171in}{3.386859in}}%
\pgfpathlineto{\pgfqpoint{5.735793in}{3.448039in}}%
\pgfpathclose%
\pgfusepath{fill}%
\end{pgfscope}%
\begin{pgfscope}%
\pgfpathrectangle{\pgfqpoint{0.539299in}{0.078740in}}{\pgfqpoint{7.842520in}{7.842520in}}%
\pgfusepath{clip}%
\pgfsetbuttcap%
\pgfsetroundjoin%
\definecolor{currentfill}{rgb}{0.276194,0.190074,0.493001}%
\pgfsetfillcolor{currentfill}%
\pgfsetlinewidth{0.000000pt}%
\definecolor{currentstroke}{rgb}{0.730889,0.871916,0.156029}%
\pgfsetstrokecolor{currentstroke}%
\pgfsetdash{}{0pt}%
\pgfpathmoveto{\pgfqpoint{6.010937in}{3.326788in}}%
\pgfpathlineto{\pgfqpoint{6.149041in}{3.265181in}}%
\pgfpathlineto{\pgfqpoint{6.084512in}{3.361457in}}%
\pgfpathclose%
\pgfusepath{fill}%
\end{pgfscope}%
\begin{pgfscope}%
\pgfpathrectangle{\pgfqpoint{0.539299in}{0.078740in}}{\pgfqpoint{7.842520in}{7.842520in}}%
\pgfusepath{clip}%
\pgfsetbuttcap%
\pgfsetroundjoin%
\definecolor{currentfill}{rgb}{0.128729,0.563265,0.551229}%
\pgfsetfillcolor{currentfill}%
\pgfsetlinewidth{0.000000pt}%
\definecolor{currentstroke}{rgb}{0.741388,0.873449,0.149561}%
\pgfsetstrokecolor{currentstroke}%
\pgfsetdash{}{0pt}%
\pgfpathmoveto{\pgfqpoint{4.221135in}{4.766924in}}%
\pgfpathlineto{\pgfqpoint{4.301131in}{4.661130in}}%
\pgfpathlineto{\pgfqpoint{4.085083in}{4.900621in}}%
\pgfpathclose%
\pgfusepath{fill}%
\end{pgfscope}%
\begin{pgfscope}%
\pgfpathrectangle{\pgfqpoint{0.539299in}{0.078740in}}{\pgfqpoint{7.842520in}{7.842520in}}%
\pgfusepath{clip}%
\pgfsetbuttcap%
\pgfsetroundjoin%
\definecolor{currentfill}{rgb}{0.243113,0.292092,0.538516}%
\pgfsetfillcolor{currentfill}%
\pgfsetlinewidth{0.000000pt}%
\definecolor{currentstroke}{rgb}{0.751884,0.874951,0.143228}%
\pgfsetstrokecolor{currentstroke}%
\pgfsetdash{}{0pt}%
\pgfpathmoveto{\pgfqpoint{5.250882in}{3.689752in}}%
\pgfpathlineto{\pgfqpoint{5.175033in}{3.721001in}}%
\pgfpathlineto{\pgfqpoint{5.387297in}{3.592738in}}%
\pgfpathclose%
\pgfusepath{fill}%
\end{pgfscope}%
\begin{pgfscope}%
\pgfpathrectangle{\pgfqpoint{0.539299in}{0.078740in}}{\pgfqpoint{7.842520in}{7.842520in}}%
\pgfusepath{clip}%
\pgfsetbuttcap%
\pgfsetroundjoin%
\definecolor{currentfill}{rgb}{0.278826,0.175490,0.483397}%
\pgfsetfillcolor{currentfill}%
\pgfsetlinewidth{0.000000pt}%
\definecolor{currentstroke}{rgb}{0.762373,0.876424,0.137064}%
\pgfsetstrokecolor{currentstroke}%
\pgfsetdash{}{0pt}%
\pgfpathmoveto{\pgfqpoint{6.222345in}{3.303711in}}%
\pgfpathlineto{\pgfqpoint{6.149041in}{3.265181in}}%
\pgfpathlineto{\pgfqpoint{6.360427in}{3.240693in}}%
\pgfpathclose%
\pgfusepath{fill}%
\end{pgfscope}%
\begin{pgfscope}%
\pgfpathrectangle{\pgfqpoint{0.539299in}{0.078740in}}{\pgfqpoint{7.842520in}{7.842520in}}%
\pgfusepath{clip}%
\pgfsetbuttcap%
\pgfsetroundjoin%
\definecolor{currentfill}{rgb}{0.121148,0.592739,0.544641}%
\pgfsetfillcolor{currentfill}%
\pgfsetlinewidth{0.000000pt}%
\definecolor{currentstroke}{rgb}{0.772852,0.877868,0.131109}%
\pgfsetstrokecolor{currentstroke}%
\pgfsetdash{}{0pt}%
\pgfpathmoveto{\pgfqpoint{3.249818in}{4.985619in}}%
\pgfpathlineto{\pgfqpoint{3.038103in}{4.744324in}}%
\pgfpathlineto{\pgfqpoint{3.165894in}{5.004406in}}%
\pgfpathclose%
\pgfusepath{fill}%
\end{pgfscope}%
\begin{pgfscope}%
\pgfpathrectangle{\pgfqpoint{0.539299in}{0.078740in}}{\pgfqpoint{7.842520in}{7.842520in}}%
\pgfusepath{clip}%
\pgfsetbuttcap%
\pgfsetroundjoin%
\definecolor{currentfill}{rgb}{0.136408,0.541173,0.554483}%
\pgfsetfillcolor{currentfill}%
\pgfsetlinewidth{0.000000pt}%
\definecolor{currentstroke}{rgb}{0.783315,0.879285,0.125405}%
\pgfsetstrokecolor{currentstroke}%
\pgfsetdash{}{0pt}%
\pgfpathmoveto{\pgfqpoint{4.221135in}{4.766924in}}%
\pgfpathlineto{\pgfqpoint{4.357325in}{4.615401in}}%
\pgfpathlineto{\pgfqpoint{4.301131in}{4.661130in}}%
\pgfpathclose%
\pgfusepath{fill}%
\end{pgfscope}%
\begin{pgfscope}%
\pgfpathrectangle{\pgfqpoint{0.539299in}{0.078740in}}{\pgfqpoint{7.842520in}{7.842520in}}%
\pgfusepath{clip}%
\pgfsetbuttcap%
\pgfsetroundjoin%
\definecolor{currentfill}{rgb}{0.180629,0.429975,0.557282}%
\pgfsetfillcolor{currentfill}%
\pgfsetlinewidth{0.000000pt}%
\definecolor{currentstroke}{rgb}{0.793760,0.880678,0.120005}%
\pgfsetstrokecolor{currentstroke}%
\pgfsetdash{}{0pt}%
\pgfpathmoveto{\pgfqpoint{4.629750in}{4.292637in}}%
\pgfpathlineto{\pgfqpoint{4.765949in}{4.134630in}}%
\pgfpathlineto{\pgfqpoint{4.707691in}{4.203491in}}%
\pgfpathclose%
\pgfusepath{fill}%
\end{pgfscope}%
\begin{pgfscope}%
\pgfpathrectangle{\pgfqpoint{0.539299in}{0.078740in}}{\pgfqpoint{7.842520in}{7.842520in}}%
\pgfusepath{clip}%
\pgfsetbuttcap%
\pgfsetroundjoin%
\definecolor{currentfill}{rgb}{0.169646,0.456262,0.558030}%
\pgfsetfillcolor{currentfill}%
\pgfsetlinewidth{0.000000pt}%
\definecolor{currentstroke}{rgb}{0.804182,0.882046,0.114965}%
\pgfsetstrokecolor{currentstroke}%
\pgfsetdash{}{0pt}%
\pgfpathmoveto{\pgfqpoint{4.707691in}{4.203491in}}%
\pgfpathlineto{\pgfqpoint{4.493544in}{4.454843in}}%
\pgfpathlineto{\pgfqpoint{4.629750in}{4.292637in}}%
\pgfpathclose%
\pgfusepath{fill}%
\end{pgfscope}%
\begin{pgfscope}%
\pgfpathrectangle{\pgfqpoint{0.539299in}{0.078740in}}{\pgfqpoint{7.842520in}{7.842520in}}%
\pgfusepath{clip}%
\pgfsetbuttcap%
\pgfsetroundjoin%
\definecolor{currentfill}{rgb}{0.277018,0.050344,0.375715}%
\pgfsetfillcolor{currentfill}%
\pgfsetlinewidth{0.000000pt}%
\definecolor{currentstroke}{rgb}{0.814576,0.883393,0.110347}%
\pgfsetstrokecolor{currentstroke}%
\pgfsetdash{}{0pt}%
\pgfpathmoveto{\pgfqpoint{7.332923in}{2.839586in}}%
\pgfpathlineto{\pgfqpoint{7.194201in}{2.919934in}}%
\pgfpathlineto{\pgfqpoint{7.123901in}{2.876974in}}%
\pgfpathclose%
\pgfusepath{fill}%
\end{pgfscope}%
\begin{pgfscope}%
\pgfpathrectangle{\pgfqpoint{0.539299in}{0.078740in}}{\pgfqpoint{7.842520in}{7.842520in}}%
\pgfusepath{clip}%
\pgfsetbuttcap%
\pgfsetroundjoin%
\definecolor{currentfill}{rgb}{0.150476,0.504369,0.557430}%
\pgfsetfillcolor{currentfill}%
\pgfsetlinewidth{0.000000pt}%
\definecolor{currentstroke}{rgb}{0.824940,0.884720,0.106217}%
\pgfsetstrokecolor{currentstroke}%
\pgfsetdash{}{0pt}%
\pgfpathmoveto{\pgfqpoint{4.436631in}{4.511408in}}%
\pgfpathlineto{\pgfqpoint{4.357325in}{4.615401in}}%
\pgfpathlineto{\pgfqpoint{4.493544in}{4.454843in}}%
\pgfpathclose%
\pgfusepath{fill}%
\end{pgfscope}%
\begin{pgfscope}%
\pgfpathrectangle{\pgfqpoint{0.539299in}{0.078740in}}{\pgfqpoint{7.842520in}{7.842520in}}%
\pgfusepath{clip}%
\pgfsetbuttcap%
\pgfsetroundjoin%
\definecolor{currentfill}{rgb}{0.280267,0.073417,0.397163}%
\pgfsetfillcolor{currentfill}%
\pgfsetlinewidth{0.000000pt}%
\definecolor{currentstroke}{rgb}{0.835270,0.886029,0.102646}%
\pgfsetstrokecolor{currentstroke}%
\pgfsetdash{}{0pt}%
\pgfpathmoveto{\pgfqpoint{6.985354in}{2.964798in}}%
\pgfpathlineto{\pgfqpoint{7.123901in}{2.876974in}}%
\pgfpathlineto{\pgfqpoint{7.055833in}{3.001854in}}%
\pgfpathclose%
\pgfusepath{fill}%
\end{pgfscope}%
\begin{pgfscope}%
\pgfpathrectangle{\pgfqpoint{0.539299in}{0.078740in}}{\pgfqpoint{7.842520in}{7.842520in}}%
\pgfusepath{clip}%
\pgfsetbuttcap%
\pgfsetroundjoin%
\definecolor{currentfill}{rgb}{0.201239,0.383670,0.554294}%
\pgfsetfillcolor{currentfill}%
\pgfsetlinewidth{0.000000pt}%
\definecolor{currentstroke}{rgb}{0.845561,0.887322,0.099702}%
\pgfsetstrokecolor{currentstroke}%
\pgfsetdash{}{0pt}%
\pgfpathmoveto{\pgfqpoint{4.765949in}{4.134630in}}%
\pgfpathlineto{\pgfqpoint{4.902184in}{3.985093in}}%
\pgfpathlineto{\pgfqpoint{4.978938in}{3.922091in}}%
\pgfpathclose%
\pgfusepath{fill}%
\end{pgfscope}%
\begin{pgfscope}%
\pgfpathrectangle{\pgfqpoint{0.539299in}{0.078740in}}{\pgfqpoint{7.842520in}{7.842520in}}%
\pgfusepath{clip}%
\pgfsetbuttcap%
\pgfsetroundjoin%
\definecolor{currentfill}{rgb}{0.283187,0.125848,0.444960}%
\pgfsetfillcolor{currentfill}%
\pgfsetlinewidth{0.000000pt}%
\definecolor{currentstroke}{rgb}{0.855810,0.888601,0.097452}%
\pgfsetstrokecolor{currentstroke}%
\pgfsetdash{}{0pt}%
\pgfpathmoveto{\pgfqpoint{6.847001in}{3.050168in}}%
\pgfpathlineto{\pgfqpoint{6.708800in}{3.131236in}}%
\pgfpathlineto{\pgfqpoint{6.637123in}{3.095135in}}%
\pgfpathclose%
\pgfusepath{fill}%
\end{pgfscope}%
\begin{pgfscope}%
\pgfpathrectangle{\pgfqpoint{0.539299in}{0.078740in}}{\pgfqpoint{7.842520in}{7.842520in}}%
\pgfusepath{clip}%
\pgfsetbuttcap%
\pgfsetroundjoin%
\definecolor{currentfill}{rgb}{0.267968,0.223549,0.512008}%
\pgfsetfillcolor{currentfill}%
\pgfsetlinewidth{0.000000pt}%
\definecolor{currentstroke}{rgb}{0.866013,0.889868,0.095953}%
\pgfsetstrokecolor{currentstroke}%
\pgfsetdash{}{0pt}%
\pgfpathmoveto{\pgfqpoint{5.735793in}{3.448039in}}%
\pgfpathlineto{\pgfqpoint{5.661270in}{3.429383in}}%
\pgfpathlineto{\pgfqpoint{5.798863in}{3.358134in}}%
\pgfpathclose%
\pgfusepath{fill}%
\end{pgfscope}%
\begin{pgfscope}%
\pgfpathrectangle{\pgfqpoint{0.539299in}{0.078740in}}{\pgfqpoint{7.842520in}{7.842520in}}%
\pgfusepath{clip}%
\pgfsetbuttcap%
\pgfsetroundjoin%
\definecolor{currentfill}{rgb}{0.124780,0.640461,0.527068}%
\pgfsetfillcolor{currentfill}%
\pgfsetlinewidth{0.000000pt}%
\definecolor{currentstroke}{rgb}{0.876168,0.891125,0.095250}%
\pgfsetstrokecolor{currentstroke}%
\pgfsetdash{}{0pt}%
\pgfpathmoveto{\pgfqpoint{3.679940in}{5.091602in}}%
\pgfpathlineto{\pgfqpoint{3.731742in}{5.157127in}}%
\pgfpathlineto{\pgfqpoint{3.814186in}{5.073908in}}%
\pgfpathclose%
\pgfusepath{fill}%
\end{pgfscope}%
\begin{pgfscope}%
\pgfpathrectangle{\pgfqpoint{0.539299in}{0.078740in}}{\pgfqpoint{7.842520in}{7.842520in}}%
\pgfusepath{clip}%
\pgfsetbuttcap%
\pgfsetroundjoin%
\definecolor{currentfill}{rgb}{0.216210,0.351535,0.550627}%
\pgfsetfillcolor{currentfill}%
\pgfsetlinewidth{0.000000pt}%
\definecolor{currentstroke}{rgb}{0.886271,0.892374,0.095374}%
\pgfsetstrokecolor{currentstroke}%
\pgfsetdash{}{0pt}%
\pgfpathmoveto{\pgfqpoint{5.114785in}{3.799326in}}%
\pgfpathlineto{\pgfqpoint{4.902184in}{3.985093in}}%
\pgfpathlineto{\pgfqpoint{5.038520in}{3.846770in}}%
\pgfpathclose%
\pgfusepath{fill}%
\end{pgfscope}%
\begin{pgfscope}%
\pgfpathrectangle{\pgfqpoint{0.539299in}{0.078740in}}{\pgfqpoint{7.842520in}{7.842520in}}%
\pgfusepath{clip}%
\pgfsetbuttcap%
\pgfsetroundjoin%
\definecolor{currentfill}{rgb}{0.282623,0.140926,0.457517}%
\pgfsetfillcolor{currentfill}%
\pgfsetlinewidth{0.000000pt}%
\definecolor{currentstroke}{rgb}{0.896320,0.893616,0.096335}%
\pgfsetstrokecolor{currentstroke}%
\pgfsetdash{}{0pt}%
\pgfpathmoveto{\pgfqpoint{6.708800in}{3.131236in}}%
\pgfpathlineto{\pgfqpoint{6.498702in}{3.171293in}}%
\pgfpathlineto{\pgfqpoint{6.637123in}{3.095135in}}%
\pgfpathclose%
\pgfusepath{fill}%
\end{pgfscope}%
\begin{pgfscope}%
\pgfpathrectangle{\pgfqpoint{0.539299in}{0.078740in}}{\pgfqpoint{7.842520in}{7.842520in}}%
\pgfusepath{clip}%
\pgfsetbuttcap%
\pgfsetroundjoin%
\definecolor{currentfill}{rgb}{0.124395,0.578002,0.548287}%
\pgfsetfillcolor{currentfill}%
\pgfsetlinewidth{0.000000pt}%
\definecolor{currentstroke}{rgb}{0.906311,0.894855,0.098125}%
\pgfsetstrokecolor{currentstroke}%
\pgfsetdash{}{0pt}%
\pgfpathmoveto{\pgfqpoint{3.165894in}{5.004406in}}%
\pgfpathlineto{\pgfqpoint{3.038103in}{4.744324in}}%
\pgfpathlineto{\pgfqpoint{2.954036in}{4.737238in}}%
\pgfpathclose%
\pgfusepath{fill}%
\end{pgfscope}%
\begin{pgfscope}%
\pgfpathrectangle{\pgfqpoint{0.539299in}{0.078740in}}{\pgfqpoint{7.842520in}{7.842520in}}%
\pgfusepath{clip}%
\pgfsetbuttcap%
\pgfsetroundjoin%
\definecolor{currentfill}{rgb}{0.225863,0.330805,0.547314}%
\pgfsetfillcolor{currentfill}%
\pgfsetlinewidth{0.000000pt}%
\definecolor{currentstroke}{rgb}{0.916242,0.896091,0.100717}%
\pgfsetstrokecolor{currentstroke}%
\pgfsetdash{}{0pt}%
\pgfpathmoveto{\pgfqpoint{5.175033in}{3.721001in}}%
\pgfpathlineto{\pgfqpoint{5.114785in}{3.799326in}}%
\pgfpathlineto{\pgfqpoint{5.038520in}{3.846770in}}%
\pgfpathclose%
\pgfusepath{fill}%
\end{pgfscope}%
\begin{pgfscope}%
\pgfpathrectangle{\pgfqpoint{0.539299in}{0.078740in}}{\pgfqpoint{7.842520in}{7.842520in}}%
\pgfusepath{clip}%
\pgfsetbuttcap%
\pgfsetroundjoin%
\definecolor{currentfill}{rgb}{0.273006,0.204520,0.501721}%
\pgfsetfillcolor{currentfill}%
\pgfsetlinewidth{0.000000pt}%
\definecolor{currentstroke}{rgb}{0.926106,0.897330,0.104071}%
\pgfsetstrokecolor{currentstroke}%
\pgfsetdash{}{0pt}%
\pgfpathmoveto{\pgfqpoint{5.873171in}{3.386859in}}%
\pgfpathlineto{\pgfqpoint{5.936847in}{3.290164in}}%
\pgfpathlineto{\pgfqpoint{6.010937in}{3.326788in}}%
\pgfpathclose%
\pgfusepath{fill}%
\end{pgfscope}%
\begin{pgfscope}%
\pgfpathrectangle{\pgfqpoint{0.539299in}{0.078740in}}{\pgfqpoint{7.842520in}{7.842520in}}%
\pgfusepath{clip}%
\pgfsetbuttcap%
\pgfsetroundjoin%
\definecolor{currentfill}{rgb}{0.253935,0.265254,0.529983}%
\pgfsetfillcolor{currentfill}%
\pgfsetlinewidth{0.000000pt}%
\definecolor{currentstroke}{rgb}{0.935904,0.898570,0.108131}%
\pgfsetstrokecolor{currentstroke}%
\pgfsetdash{}{0pt}%
\pgfpathmoveto{\pgfqpoint{5.387297in}{3.592738in}}%
\pgfpathlineto{\pgfqpoint{5.448865in}{3.506545in}}%
\pgfpathlineto{\pgfqpoint{5.524083in}{3.506700in}}%
\pgfpathclose%
\pgfusepath{fill}%
\end{pgfscope}%
\begin{pgfscope}%
\pgfpathrectangle{\pgfqpoint{0.539299in}{0.078740in}}{\pgfqpoint{7.842520in}{7.842520in}}%
\pgfusepath{clip}%
\pgfsetbuttcap%
\pgfsetroundjoin%
\definecolor{currentfill}{rgb}{0.141935,0.526453,0.555991}%
\pgfsetfillcolor{currentfill}%
\pgfsetlinewidth{0.000000pt}%
\definecolor{currentstroke}{rgb}{0.945636,0.899815,0.112838}%
\pgfsetstrokecolor{currentstroke}%
\pgfsetdash{}{0pt}%
\pgfpathmoveto{\pgfqpoint{2.954036in}{4.737238in}}%
\pgfpathlineto{\pgfqpoint{2.746363in}{4.336760in}}%
\pgfpathlineto{\pgfqpoint{2.869477in}{4.725189in}}%
\pgfpathclose%
\pgfusepath{fill}%
\end{pgfscope}%
\begin{pgfscope}%
\pgfpathrectangle{\pgfqpoint{0.539299in}{0.078740in}}{\pgfqpoint{7.842520in}{7.842520in}}%
\pgfusepath{clip}%
\pgfsetbuttcap%
\pgfsetroundjoin%
\definecolor{currentfill}{rgb}{0.130067,0.651384,0.521608}%
\pgfsetfillcolor{currentfill}%
\pgfsetlinewidth{0.000000pt}%
\definecolor{currentstroke}{rgb}{0.955300,0.901065,0.118128}%
\pgfsetstrokecolor{currentstroke}%
\pgfsetdash{}{0pt}%
\pgfpathmoveto{\pgfqpoint{3.597214in}{5.159015in}}%
\pgfpathlineto{\pgfqpoint{3.731742in}{5.157127in}}%
\pgfpathlineto{\pgfqpoint{3.679940in}{5.091602in}}%
\pgfpathclose%
\pgfusepath{fill}%
\end{pgfscope}%
\begin{pgfscope}%
\pgfpathrectangle{\pgfqpoint{0.539299in}{0.078740in}}{\pgfqpoint{7.842520in}{7.842520in}}%
\pgfusepath{clip}%
\pgfsetbuttcap%
\pgfsetroundjoin%
\definecolor{currentfill}{rgb}{0.271305,0.019942,0.347269}%
\pgfsetfillcolor{currentfill}%
\pgfsetlinewidth{0.000000pt}%
\definecolor{currentstroke}{rgb}{0.964894,0.902323,0.123941}%
\pgfsetstrokecolor{currentstroke}%
\pgfsetdash{}{0pt}%
\pgfpathmoveto{\pgfqpoint{7.472102in}{2.763114in}}%
\pgfpathlineto{\pgfqpoint{7.332923in}{2.839586in}}%
\pgfpathlineto{\pgfqpoint{7.401863in}{2.702549in}}%
\pgfpathclose%
\pgfusepath{fill}%
\end{pgfscope}%
\begin{pgfscope}%
\pgfpathrectangle{\pgfqpoint{0.539299in}{0.078740in}}{\pgfqpoint{7.842520in}{7.842520in}}%
\pgfusepath{clip}%
\pgfsetbuttcap%
\pgfsetroundjoin%
\definecolor{currentfill}{rgb}{0.276022,0.044167,0.370164}%
\pgfsetfillcolor{currentfill}%
\pgfsetlinewidth{0.000000pt}%
\definecolor{currentstroke}{rgb}{0.974417,0.903590,0.130215}%
\pgfsetstrokecolor{currentstroke}%
\pgfsetdash{}{0pt}%
\pgfpathmoveto{\pgfqpoint{7.262707in}{2.788802in}}%
\pgfpathlineto{\pgfqpoint{7.332923in}{2.839586in}}%
\pgfpathlineto{\pgfqpoint{7.123901in}{2.876974in}}%
\pgfpathclose%
\pgfusepath{fill}%
\end{pgfscope}%
\begin{pgfscope}%
\pgfpathrectangle{\pgfqpoint{0.539299in}{0.078740in}}{\pgfqpoint{7.842520in}{7.842520in}}%
\pgfusepath{clip}%
\pgfsetbuttcap%
\pgfsetroundjoin%
\definecolor{currentfill}{rgb}{0.278012,0.180367,0.486697}%
\pgfsetfillcolor{currentfill}%
\pgfsetlinewidth{0.000000pt}%
\definecolor{currentstroke}{rgb}{0.983868,0.904867,0.136897}%
\pgfsetstrokecolor{currentstroke}%
\pgfsetdash{}{0pt}%
\pgfpathmoveto{\pgfqpoint{6.360427in}{3.240693in}}%
\pgfpathlineto{\pgfqpoint{6.149041in}{3.265181in}}%
\pgfpathlineto{\pgfqpoint{6.287425in}{3.199853in}}%
\pgfpathclose%
\pgfusepath{fill}%
\end{pgfscope}%
\begin{pgfscope}%
\pgfpathrectangle{\pgfqpoint{0.539299in}{0.078740in}}{\pgfqpoint{7.842520in}{7.842520in}}%
\pgfusepath{clip}%
\pgfsetbuttcap%
\pgfsetroundjoin%
\definecolor{currentfill}{rgb}{0.216210,0.351535,0.550627}%
\pgfsetfillcolor{currentfill}%
\pgfsetlinewidth{0.000000pt}%
\definecolor{currentstroke}{rgb}{0.993248,0.906157,0.143936}%
\pgfsetstrokecolor{currentstroke}%
\pgfsetdash{}{0pt}%
\pgfpathmoveto{\pgfqpoint{2.375504in}{3.692390in}}%
\pgfpathlineto{\pgfqpoint{2.290658in}{3.630794in}}%
\pgfpathlineto{\pgfqpoint{2.491969in}{4.215044in}}%
\pgfpathclose%
\pgfusepath{fill}%
\end{pgfscope}%
\begin{pgfscope}%
\pgfpathrectangle{\pgfqpoint{0.539299in}{0.078740in}}{\pgfqpoint{7.842520in}{7.842520in}}%
\pgfusepath{clip}%
\pgfsetbuttcap%
\pgfsetroundjoin%
\definecolor{currentfill}{rgb}{0.241237,0.296485,0.539709}%
\pgfsetfillcolor{currentfill}%
\pgfsetlinewidth{0.000000pt}%
\definecolor{currentstroke}{rgb}{0.267004,0.004874,0.329415}%
\pgfsetstrokecolor{currentstroke}%
\pgfsetdash{}{0pt}%
\pgfpathmoveto{\pgfqpoint{5.387297in}{3.592738in}}%
\pgfpathlineto{\pgfqpoint{5.175033in}{3.721001in}}%
\pgfpathlineto{\pgfqpoint{5.311793in}{3.607895in}}%
\pgfpathclose%
\pgfusepath{fill}%
\end{pgfscope}%
\begin{pgfscope}%
\pgfpathrectangle{\pgfqpoint{0.539299in}{0.078740in}}{\pgfqpoint{7.842520in}{7.842520in}}%
\pgfusepath{clip}%
\pgfsetbuttcap%
\pgfsetroundjoin%
\definecolor{currentfill}{rgb}{0.282656,0.100196,0.422160}%
\pgfsetfillcolor{currentfill}%
\pgfsetlinewidth{0.000000pt}%
\definecolor{currentstroke}{rgb}{0.268510,0.009605,0.335427}%
\pgfsetstrokecolor{currentstroke}%
\pgfsetdash{}{0pt}%
\pgfpathmoveto{\pgfqpoint{6.847001in}{3.050168in}}%
\pgfpathlineto{\pgfqpoint{6.914313in}{2.924367in}}%
\pgfpathlineto{\pgfqpoint{6.985354in}{2.964798in}}%
\pgfpathclose%
\pgfusepath{fill}%
\end{pgfscope}%
\begin{pgfscope}%
\pgfpathrectangle{\pgfqpoint{0.539299in}{0.078740in}}{\pgfqpoint{7.842520in}{7.842520in}}%
\pgfusepath{clip}%
\pgfsetbuttcap%
\pgfsetroundjoin%
\definecolor{currentfill}{rgb}{0.280868,0.160771,0.472899}%
\pgfsetfillcolor{currentfill}%
\pgfsetlinewidth{0.000000pt}%
\definecolor{currentstroke}{rgb}{0.269944,0.014625,0.341379}%
\pgfsetstrokecolor{currentstroke}%
\pgfsetdash{}{0pt}%
\pgfpathmoveto{\pgfqpoint{6.426028in}{3.129202in}}%
\pgfpathlineto{\pgfqpoint{6.498702in}{3.171293in}}%
\pgfpathlineto{\pgfqpoint{6.360427in}{3.240693in}}%
\pgfpathclose%
\pgfusepath{fill}%
\end{pgfscope}%
\begin{pgfscope}%
\pgfpathrectangle{\pgfqpoint{0.539299in}{0.078740in}}{\pgfqpoint{7.842520in}{7.842520in}}%
\pgfusepath{clip}%
\pgfsetbuttcap%
\pgfsetroundjoin%
\definecolor{currentfill}{rgb}{0.132268,0.655014,0.519661}%
\pgfsetfillcolor{currentfill}%
\pgfsetlinewidth{0.000000pt}%
\definecolor{currentstroke}{rgb}{0.271305,0.019942,0.347269}%
\pgfsetstrokecolor{currentstroke}%
\pgfsetdash{}{0pt}%
\pgfpathmoveto{\pgfqpoint{3.597214in}{5.159015in}}%
\pgfpathlineto{\pgfqpoint{3.464190in}{5.097259in}}%
\pgfpathlineto{\pgfqpoint{3.380711in}{5.141597in}}%
\pgfpathclose%
\pgfusepath{fill}%
\end{pgfscope}%
\begin{pgfscope}%
\pgfpathrectangle{\pgfqpoint{0.539299in}{0.078740in}}{\pgfqpoint{7.842520in}{7.842520in}}%
\pgfusepath{clip}%
\pgfsetbuttcap%
\pgfsetroundjoin%
\definecolor{currentfill}{rgb}{0.270595,0.214069,0.507052}%
\pgfsetfillcolor{currentfill}%
\pgfsetlinewidth{0.000000pt}%
\definecolor{currentstroke}{rgb}{0.272594,0.025563,0.353093}%
\pgfsetstrokecolor{currentstroke}%
\pgfsetdash{}{0pt}%
\pgfpathmoveto{\pgfqpoint{5.798863in}{3.358134in}}%
\pgfpathlineto{\pgfqpoint{5.936847in}{3.290164in}}%
\pgfpathlineto{\pgfqpoint{5.873171in}{3.386859in}}%
\pgfpathclose%
\pgfusepath{fill}%
\end{pgfscope}%
\begin{pgfscope}%
\pgfpathrectangle{\pgfqpoint{0.539299in}{0.078740in}}{\pgfqpoint{7.842520in}{7.842520in}}%
\pgfusepath{clip}%
\pgfsetbuttcap%
\pgfsetroundjoin%
\definecolor{currentfill}{rgb}{0.283229,0.120777,0.440584}%
\pgfsetfillcolor{currentfill}%
\pgfsetlinewidth{0.000000pt}%
\definecolor{currentstroke}{rgb}{0.273809,0.031497,0.358853}%
\pgfsetstrokecolor{currentstroke}%
\pgfsetdash{}{0pt}%
\pgfpathmoveto{\pgfqpoint{6.637123in}{3.095135in}}%
\pgfpathlineto{\pgfqpoint{6.775662in}{3.012527in}}%
\pgfpathlineto{\pgfqpoint{6.847001in}{3.050168in}}%
\pgfpathclose%
\pgfusepath{fill}%
\end{pgfscope}%
\begin{pgfscope}%
\pgfpathrectangle{\pgfqpoint{0.539299in}{0.078740in}}{\pgfqpoint{7.842520in}{7.842520in}}%
\pgfusepath{clip}%
\pgfsetbuttcap%
\pgfsetroundjoin%
\definecolor{currentfill}{rgb}{0.258965,0.251537,0.524736}%
\pgfsetfillcolor{currentfill}%
\pgfsetlinewidth{0.000000pt}%
\definecolor{currentstroke}{rgb}{0.274952,0.037752,0.364543}%
\pgfsetstrokecolor{currentstroke}%
\pgfsetdash{}{0pt}%
\pgfpathmoveto{\pgfqpoint{5.448865in}{3.506545in}}%
\pgfpathlineto{\pgfqpoint{5.661270in}{3.429383in}}%
\pgfpathlineto{\pgfqpoint{5.524083in}{3.506700in}}%
\pgfpathclose%
\pgfusepath{fill}%
\end{pgfscope}%
\begin{pgfscope}%
\pgfpathrectangle{\pgfqpoint{0.539299in}{0.078740in}}{\pgfqpoint{7.842520in}{7.842520in}}%
\pgfusepath{clip}%
\pgfsetbuttcap%
\pgfsetroundjoin%
\definecolor{currentfill}{rgb}{0.274128,0.199721,0.498911}%
\pgfsetfillcolor{currentfill}%
\pgfsetlinewidth{0.000000pt}%
\definecolor{currentstroke}{rgb}{0.276022,0.044167,0.370164}%
\pgfsetstrokecolor{currentstroke}%
\pgfsetdash{}{0pt}%
\pgfpathmoveto{\pgfqpoint{5.936847in}{3.290164in}}%
\pgfpathlineto{\pgfqpoint{6.149041in}{3.265181in}}%
\pgfpathlineto{\pgfqpoint{6.010937in}{3.326788in}}%
\pgfpathclose%
\pgfusepath{fill}%
\end{pgfscope}%
\begin{pgfscope}%
\pgfpathrectangle{\pgfqpoint{0.539299in}{0.078740in}}{\pgfqpoint{7.842520in}{7.842520in}}%
\pgfusepath{clip}%
\pgfsetbuttcap%
\pgfsetroundjoin%
\definecolor{currentfill}{rgb}{0.124780,0.640461,0.527068}%
\pgfsetfillcolor{currentfill}%
\pgfsetlinewidth{0.000000pt}%
\definecolor{currentstroke}{rgb}{0.277018,0.050344,0.375715}%
\pgfsetstrokecolor{currentstroke}%
\pgfsetdash{}{0pt}%
\pgfpathmoveto{\pgfqpoint{3.867334in}{5.102013in}}%
\pgfpathlineto{\pgfqpoint{3.949347in}{5.006539in}}%
\pgfpathlineto{\pgfqpoint{3.814186in}{5.073908in}}%
\pgfpathclose%
\pgfusepath{fill}%
\end{pgfscope}%
\begin{pgfscope}%
\pgfpathrectangle{\pgfqpoint{0.539299in}{0.078740in}}{\pgfqpoint{7.842520in}{7.842520in}}%
\pgfusepath{clip}%
\pgfsetbuttcap%
\pgfsetroundjoin%
\definecolor{currentfill}{rgb}{0.153364,0.497000,0.557724}%
\pgfsetfillcolor{currentfill}%
\pgfsetlinewidth{0.000000pt}%
\definecolor{currentstroke}{rgb}{0.277941,0.056324,0.381191}%
\pgfsetstrokecolor{currentstroke}%
\pgfsetdash{}{0pt}%
\pgfpathmoveto{\pgfqpoint{2.746363in}{4.336760in}}%
\pgfpathlineto{\pgfqpoint{2.661974in}{4.301274in}}%
\pgfpathlineto{\pgfqpoint{2.784458in}{4.707293in}}%
\pgfpathclose%
\pgfusepath{fill}%
\end{pgfscope}%
\begin{pgfscope}%
\pgfpathrectangle{\pgfqpoint{0.539299in}{0.078740in}}{\pgfqpoint{7.842520in}{7.842520in}}%
\pgfusepath{clip}%
\pgfsetbuttcap%
\pgfsetroundjoin%
\definecolor{currentfill}{rgb}{0.123444,0.636809,0.528763}%
\pgfsetfillcolor{currentfill}%
\pgfsetlinewidth{0.000000pt}%
\definecolor{currentstroke}{rgb}{0.278791,0.062145,0.386592}%
\pgfsetstrokecolor{currentstroke}%
\pgfsetdash{}{0pt}%
\pgfpathmoveto{\pgfqpoint{3.380711in}{5.141597in}}%
\pgfpathlineto{\pgfqpoint{3.249818in}{4.985619in}}%
\pgfpathlineto{\pgfqpoint{3.165894in}{5.004406in}}%
\pgfpathclose%
\pgfusepath{fill}%
\end{pgfscope}%
\begin{pgfscope}%
\pgfpathrectangle{\pgfqpoint{0.539299in}{0.078740in}}{\pgfqpoint{7.842520in}{7.842520in}}%
\pgfusepath{clip}%
\pgfsetbuttcap%
\pgfsetroundjoin%
\definecolor{currentfill}{rgb}{0.248629,0.278775,0.534556}%
\pgfsetfillcolor{currentfill}%
\pgfsetlinewidth{0.000000pt}%
\definecolor{currentstroke}{rgb}{0.279566,0.067836,0.391917}%
\pgfsetstrokecolor{currentstroke}%
\pgfsetdash{}{0pt}%
\pgfpathmoveto{\pgfqpoint{5.387297in}{3.592738in}}%
\pgfpathlineto{\pgfqpoint{5.311793in}{3.607895in}}%
\pgfpathlineto{\pgfqpoint{5.448865in}{3.506545in}}%
\pgfpathclose%
\pgfusepath{fill}%
\end{pgfscope}%
\begin{pgfscope}%
\pgfpathrectangle{\pgfqpoint{0.539299in}{0.078740in}}{\pgfqpoint{7.842520in}{7.842520in}}%
\pgfusepath{clip}%
\pgfsetbuttcap%
\pgfsetroundjoin%
\definecolor{currentfill}{rgb}{0.273809,0.031497,0.358853}%
\pgfsetfillcolor{currentfill}%
\pgfsetlinewidth{0.000000pt}%
\definecolor{currentstroke}{rgb}{0.280267,0.073417,0.397163}%
\pgfsetstrokecolor{currentstroke}%
\pgfsetdash{}{0pt}%
\pgfpathmoveto{\pgfqpoint{7.401863in}{2.702549in}}%
\pgfpathlineto{\pgfqpoint{7.332923in}{2.839586in}}%
\pgfpathlineto{\pgfqpoint{7.262707in}{2.788802in}}%
\pgfpathclose%
\pgfusepath{fill}%
\end{pgfscope}%
\begin{pgfscope}%
\pgfpathrectangle{\pgfqpoint{0.539299in}{0.078740in}}{\pgfqpoint{7.842520in}{7.842520in}}%
\pgfusepath{clip}%
\pgfsetbuttcap%
\pgfsetroundjoin%
\definecolor{currentfill}{rgb}{0.281446,0.084320,0.407414}%
\pgfsetfillcolor{currentfill}%
\pgfsetlinewidth{0.000000pt}%
\definecolor{currentstroke}{rgb}{0.280894,0.078907,0.402329}%
\pgfsetstrokecolor{currentstroke}%
\pgfsetdash{}{0pt}%
\pgfpathmoveto{\pgfqpoint{6.914313in}{2.924367in}}%
\pgfpathlineto{\pgfqpoint{7.123901in}{2.876974in}}%
\pgfpathlineto{\pgfqpoint{6.985354in}{2.964798in}}%
\pgfpathclose%
\pgfusepath{fill}%
\end{pgfscope}%
\begin{pgfscope}%
\pgfpathrectangle{\pgfqpoint{0.539299in}{0.078740in}}{\pgfqpoint{7.842520in}{7.842520in}}%
\pgfusepath{clip}%
\pgfsetbuttcap%
\pgfsetroundjoin%
\definecolor{currentfill}{rgb}{0.279574,0.170599,0.479997}%
\pgfsetfillcolor{currentfill}%
\pgfsetlinewidth{0.000000pt}%
\definecolor{currentstroke}{rgb}{0.281446,0.084320,0.407414}%
\pgfsetstrokecolor{currentstroke}%
\pgfsetdash{}{0pt}%
\pgfpathmoveto{\pgfqpoint{6.360427in}{3.240693in}}%
\pgfpathlineto{\pgfqpoint{6.287425in}{3.199853in}}%
\pgfpathlineto{\pgfqpoint{6.426028in}{3.129202in}}%
\pgfpathclose%
\pgfusepath{fill}%
\end{pgfscope}%
\begin{pgfscope}%
\pgfpathrectangle{\pgfqpoint{0.539299in}{0.078740in}}{\pgfqpoint{7.842520in}{7.842520in}}%
\pgfusepath{clip}%
\pgfsetbuttcap%
\pgfsetroundjoin%
\definecolor{currentfill}{rgb}{0.283091,0.110553,0.431554}%
\pgfsetfillcolor{currentfill}%
\pgfsetlinewidth{0.000000pt}%
\definecolor{currentstroke}{rgb}{0.281924,0.089666,0.412415}%
\pgfsetstrokecolor{currentstroke}%
\pgfsetdash{}{0pt}%
\pgfpathmoveto{\pgfqpoint{6.775662in}{3.012527in}}%
\pgfpathlineto{\pgfqpoint{6.914313in}{2.924367in}}%
\pgfpathlineto{\pgfqpoint{6.847001in}{3.050168in}}%
\pgfpathclose%
\pgfusepath{fill}%
\end{pgfscope}%
\begin{pgfscope}%
\pgfpathrectangle{\pgfqpoint{0.539299in}{0.078740in}}{\pgfqpoint{7.842520in}{7.842520in}}%
\pgfusepath{clip}%
\pgfsetbuttcap%
\pgfsetroundjoin%
\definecolor{currentfill}{rgb}{0.121380,0.629492,0.531973}%
\pgfsetfillcolor{currentfill}%
\pgfsetlinewidth{0.000000pt}%
\definecolor{currentstroke}{rgb}{0.282327,0.094955,0.417331}%
\pgfsetstrokecolor{currentstroke}%
\pgfsetdash{}{0pt}%
\pgfpathmoveto{\pgfqpoint{3.867334in}{5.102013in}}%
\pgfpathlineto{\pgfqpoint{4.085083in}{4.900621in}}%
\pgfpathlineto{\pgfqpoint{3.949347in}{5.006539in}}%
\pgfpathclose%
\pgfusepath{fill}%
\end{pgfscope}%
\begin{pgfscope}%
\pgfpathrectangle{\pgfqpoint{0.539299in}{0.078740in}}{\pgfqpoint{7.842520in}{7.842520in}}%
\pgfusepath{clip}%
\pgfsetbuttcap%
\pgfsetroundjoin%
\definecolor{currentfill}{rgb}{0.188923,0.410910,0.556326}%
\pgfsetfillcolor{currentfill}%
\pgfsetlinewidth{0.000000pt}%
\definecolor{currentstroke}{rgb}{0.282656,0.100196,0.422160}%
\pgfsetstrokecolor{currentstroke}%
\pgfsetdash{}{0pt}%
\pgfpathmoveto{\pgfqpoint{2.375504in}{3.692390in}}%
\pgfpathlineto{\pgfqpoint{2.491969in}{4.215044in}}%
\pgfpathlineto{\pgfqpoint{2.577164in}{4.260977in}}%
\pgfpathclose%
\pgfusepath{fill}%
\end{pgfscope}%
\begin{pgfscope}%
\pgfpathrectangle{\pgfqpoint{0.539299in}{0.078740in}}{\pgfqpoint{7.842520in}{7.842520in}}%
\pgfusepath{clip}%
\pgfsetbuttcap%
\pgfsetroundjoin%
\definecolor{currentfill}{rgb}{0.281887,0.150881,0.465405}%
\pgfsetfillcolor{currentfill}%
\pgfsetlinewidth{0.000000pt}%
\definecolor{currentstroke}{rgb}{0.282910,0.105393,0.426902}%
\pgfsetstrokecolor{currentstroke}%
\pgfsetdash{}{0pt}%
\pgfpathmoveto{\pgfqpoint{6.637123in}{3.095135in}}%
\pgfpathlineto{\pgfqpoint{6.498702in}{3.171293in}}%
\pgfpathlineto{\pgfqpoint{6.564795in}{3.052267in}}%
\pgfpathclose%
\pgfusepath{fill}%
\end{pgfscope}%
\begin{pgfscope}%
\pgfpathrectangle{\pgfqpoint{0.539299in}{0.078740in}}{\pgfqpoint{7.842520in}{7.842520in}}%
\pgfusepath{clip}%
\pgfsetbuttcap%
\pgfsetroundjoin%
\definecolor{currentfill}{rgb}{0.265145,0.232956,0.516599}%
\pgfsetfillcolor{currentfill}%
\pgfsetlinewidth{0.000000pt}%
\definecolor{currentstroke}{rgb}{0.283091,0.110553,0.431554}%
\pgfsetstrokecolor{currentstroke}%
\pgfsetdash{}{0pt}%
\pgfpathmoveto{\pgfqpoint{5.661270in}{3.429383in}}%
\pgfpathlineto{\pgfqpoint{5.724096in}{3.331845in}}%
\pgfpathlineto{\pgfqpoint{5.798863in}{3.358134in}}%
\pgfpathclose%
\pgfusepath{fill}%
\end{pgfscope}%
\begin{pgfscope}%
\pgfpathrectangle{\pgfqpoint{0.539299in}{0.078740in}}{\pgfqpoint{7.842520in}{7.842520in}}%
\pgfusepath{clip}%
\pgfsetbuttcap%
\pgfsetroundjoin%
\definecolor{currentfill}{rgb}{0.134692,0.658636,0.517649}%
\pgfsetfillcolor{currentfill}%
\pgfsetlinewidth{0.000000pt}%
\definecolor{currentstroke}{rgb}{0.283197,0.115680,0.436115}%
\pgfsetstrokecolor{currentstroke}%
\pgfsetdash{}{0pt}%
\pgfpathmoveto{\pgfqpoint{3.814186in}{5.073908in}}%
\pgfpathlineto{\pgfqpoint{3.731742in}{5.157127in}}%
\pgfpathlineto{\pgfqpoint{3.867334in}{5.102013in}}%
\pgfpathclose%
\pgfusepath{fill}%
\end{pgfscope}%
\begin{pgfscope}%
\pgfpathrectangle{\pgfqpoint{0.539299in}{0.078740in}}{\pgfqpoint{7.842520in}{7.842520in}}%
\pgfusepath{clip}%
\pgfsetbuttcap%
\pgfsetroundjoin%
\definecolor{currentfill}{rgb}{0.257322,0.256130,0.526563}%
\pgfsetfillcolor{currentfill}%
\pgfsetlinewidth{0.000000pt}%
\definecolor{currentstroke}{rgb}{0.283229,0.120777,0.440584}%
\pgfsetstrokecolor{currentstroke}%
\pgfsetdash{}{0pt}%
\pgfpathmoveto{\pgfqpoint{5.586291in}{3.415267in}}%
\pgfpathlineto{\pgfqpoint{5.661270in}{3.429383in}}%
\pgfpathlineto{\pgfqpoint{5.448865in}{3.506545in}}%
\pgfpathclose%
\pgfusepath{fill}%
\end{pgfscope}%
\begin{pgfscope}%
\pgfpathrectangle{\pgfqpoint{0.539299in}{0.078740in}}{\pgfqpoint{7.842520in}{7.842520in}}%
\pgfusepath{clip}%
\pgfsetbuttcap%
\pgfsetroundjoin%
\definecolor{currentfill}{rgb}{0.187231,0.414746,0.556547}%
\pgfsetfillcolor{currentfill}%
\pgfsetlinewidth{0.000000pt}%
\definecolor{currentstroke}{rgb}{0.283187,0.125848,0.444960}%
\pgfsetstrokecolor{currentstroke}%
\pgfsetdash{}{0pt}%
\pgfpathmoveto{\pgfqpoint{4.765949in}{4.134630in}}%
\pgfpathlineto{\pgfqpoint{4.824755in}{4.057714in}}%
\pgfpathlineto{\pgfqpoint{4.902184in}{3.985093in}}%
\pgfpathclose%
\pgfusepath{fill}%
\end{pgfscope}%
\begin{pgfscope}%
\pgfpathrectangle{\pgfqpoint{0.539299in}{0.078740in}}{\pgfqpoint{7.842520in}{7.842520in}}%
\pgfusepath{clip}%
\pgfsetbuttcap%
\pgfsetroundjoin%
\definecolor{currentfill}{rgb}{0.120565,0.596422,0.543611}%
\pgfsetfillcolor{currentfill}%
\pgfsetlinewidth{0.000000pt}%
\definecolor{currentstroke}{rgb}{0.283072,0.130895,0.449241}%
\pgfsetstrokecolor{currentstroke}%
\pgfsetdash{}{0pt}%
\pgfpathmoveto{\pgfqpoint{4.085083in}{4.900621in}}%
\pgfpathlineto{\pgfqpoint{4.140312in}{4.874931in}}%
\pgfpathlineto{\pgfqpoint{4.221135in}{4.766924in}}%
\pgfpathclose%
\pgfusepath{fill}%
\end{pgfscope}%
\begin{pgfscope}%
\pgfpathrectangle{\pgfqpoint{0.539299in}{0.078740in}}{\pgfqpoint{7.842520in}{7.842520in}}%
\pgfusepath{clip}%
\pgfsetbuttcap%
\pgfsetroundjoin%
\definecolor{currentfill}{rgb}{0.166617,0.463708,0.558119}%
\pgfsetfillcolor{currentfill}%
\pgfsetlinewidth{0.000000pt}%
\definecolor{currentstroke}{rgb}{0.282884,0.135920,0.453427}%
\pgfsetstrokecolor{currentstroke}%
\pgfsetdash{}{0pt}%
\pgfpathmoveto{\pgfqpoint{4.551037in}{4.389232in}}%
\pgfpathlineto{\pgfqpoint{4.765949in}{4.134630in}}%
\pgfpathlineto{\pgfqpoint{4.629750in}{4.292637in}}%
\pgfpathclose%
\pgfusepath{fill}%
\end{pgfscope}%
\begin{pgfscope}%
\pgfpathrectangle{\pgfqpoint{0.539299in}{0.078740in}}{\pgfqpoint{7.842520in}{7.842520in}}%
\pgfusepath{clip}%
\pgfsetbuttcap%
\pgfsetroundjoin%
\definecolor{currentfill}{rgb}{0.156270,0.489624,0.557936}%
\pgfsetfillcolor{currentfill}%
\pgfsetlinewidth{0.000000pt}%
\definecolor{currentstroke}{rgb}{0.282623,0.140926,0.457517}%
\pgfsetstrokecolor{currentstroke}%
\pgfsetdash{}{0pt}%
\pgfpathmoveto{\pgfqpoint{4.629750in}{4.292637in}}%
\pgfpathlineto{\pgfqpoint{4.493544in}{4.454843in}}%
\pgfpathlineto{\pgfqpoint{4.551037in}{4.389232in}}%
\pgfpathclose%
\pgfusepath{fill}%
\end{pgfscope}%
\begin{pgfscope}%
\pgfpathrectangle{\pgfqpoint{0.539299in}{0.078740in}}{\pgfqpoint{7.842520in}{7.842520in}}%
\pgfusepath{clip}%
\pgfsetbuttcap%
\pgfsetroundjoin%
\definecolor{currentfill}{rgb}{0.137770,0.537492,0.554906}%
\pgfsetfillcolor{currentfill}%
\pgfsetlinewidth{0.000000pt}%
\definecolor{currentstroke}{rgb}{0.282290,0.145912,0.461510}%
\pgfsetstrokecolor{currentstroke}%
\pgfsetdash{}{0pt}%
\pgfpathmoveto{\pgfqpoint{2.869477in}{4.725189in}}%
\pgfpathlineto{\pgfqpoint{2.746363in}{4.336760in}}%
\pgfpathlineto{\pgfqpoint{2.784458in}{4.707293in}}%
\pgfpathclose%
\pgfusepath{fill}%
\end{pgfscope}%
\begin{pgfscope}%
\pgfpathrectangle{\pgfqpoint{0.539299in}{0.078740in}}{\pgfqpoint{7.842520in}{7.842520in}}%
\pgfusepath{clip}%
\pgfsetbuttcap%
\pgfsetroundjoin%
\definecolor{currentfill}{rgb}{0.274128,0.199721,0.498911}%
\pgfsetfillcolor{currentfill}%
\pgfsetlinewidth{0.000000pt}%
\definecolor{currentstroke}{rgb}{0.281887,0.150881,0.465405}%
\pgfsetstrokecolor{currentstroke}%
\pgfsetdash{}{0pt}%
\pgfpathmoveto{\pgfqpoint{6.075188in}{3.222781in}}%
\pgfpathlineto{\pgfqpoint{6.149041in}{3.265181in}}%
\pgfpathlineto{\pgfqpoint{5.936847in}{3.290164in}}%
\pgfpathclose%
\pgfusepath{fill}%
\end{pgfscope}%
\begin{pgfscope}%
\pgfpathrectangle{\pgfqpoint{0.539299in}{0.078740in}}{\pgfqpoint{7.842520in}{7.842520in}}%
\pgfusepath{clip}%
\pgfsetbuttcap%
\pgfsetroundjoin%
\definecolor{currentfill}{rgb}{0.197636,0.391528,0.554969}%
\pgfsetfillcolor{currentfill}%
\pgfsetlinewidth{0.000000pt}%
\definecolor{currentstroke}{rgb}{0.281412,0.155834,0.469201}%
\pgfsetstrokecolor{currentstroke}%
\pgfsetdash{}{0pt}%
\pgfpathmoveto{\pgfqpoint{5.038520in}{3.846770in}}%
\pgfpathlineto{\pgfqpoint{4.902184in}{3.985093in}}%
\pgfpathlineto{\pgfqpoint{4.824755in}{4.057714in}}%
\pgfpathclose%
\pgfusepath{fill}%
\end{pgfscope}%
\begin{pgfscope}%
\pgfpathrectangle{\pgfqpoint{0.539299in}{0.078740in}}{\pgfqpoint{7.842520in}{7.842520in}}%
\pgfusepath{clip}%
\pgfsetbuttcap%
\pgfsetroundjoin%
\definecolor{currentfill}{rgb}{0.124395,0.578002,0.548287}%
\pgfsetfillcolor{currentfill}%
\pgfsetlinewidth{0.000000pt}%
\definecolor{currentstroke}{rgb}{0.280868,0.160771,0.472899}%
\pgfsetstrokecolor{currentstroke}%
\pgfsetdash{}{0pt}%
\pgfpathmoveto{\pgfqpoint{4.140312in}{4.874931in}}%
\pgfpathlineto{\pgfqpoint{4.357325in}{4.615401in}}%
\pgfpathlineto{\pgfqpoint{4.221135in}{4.766924in}}%
\pgfpathclose%
\pgfusepath{fill}%
\end{pgfscope}%
\begin{pgfscope}%
\pgfpathrectangle{\pgfqpoint{0.539299in}{0.078740in}}{\pgfqpoint{7.842520in}{7.842520in}}%
\pgfusepath{clip}%
\pgfsetbuttcap%
\pgfsetroundjoin%
\definecolor{currentfill}{rgb}{0.281412,0.155834,0.469201}%
\pgfsetfillcolor{currentfill}%
\pgfsetlinewidth{0.000000pt}%
\definecolor{currentstroke}{rgb}{0.280255,0.165693,0.476498}%
\pgfsetstrokecolor{currentstroke}%
\pgfsetdash{}{0pt}%
\pgfpathmoveto{\pgfqpoint{6.564795in}{3.052267in}}%
\pgfpathlineto{\pgfqpoint{6.498702in}{3.171293in}}%
\pgfpathlineto{\pgfqpoint{6.426028in}{3.129202in}}%
\pgfpathclose%
\pgfusepath{fill}%
\end{pgfscope}%
\begin{pgfscope}%
\pgfpathrectangle{\pgfqpoint{0.539299in}{0.078740in}}{\pgfqpoint{7.842520in}{7.842520in}}%
\pgfusepath{clip}%
\pgfsetbuttcap%
\pgfsetroundjoin%
\definecolor{currentfill}{rgb}{0.136408,0.541173,0.554483}%
\pgfsetfillcolor{currentfill}%
\pgfsetlinewidth{0.000000pt}%
\definecolor{currentstroke}{rgb}{0.279574,0.170599,0.479997}%
\pgfsetstrokecolor{currentstroke}%
\pgfsetdash{}{0pt}%
\pgfpathmoveto{\pgfqpoint{4.493544in}{4.454843in}}%
\pgfpathlineto{\pgfqpoint{4.357325in}{4.615401in}}%
\pgfpathlineto{\pgfqpoint{4.277193in}{4.723467in}}%
\pgfpathclose%
\pgfusepath{fill}%
\end{pgfscope}%
\begin{pgfscope}%
\pgfpathrectangle{\pgfqpoint{0.539299in}{0.078740in}}{\pgfqpoint{7.842520in}{7.842520in}}%
\pgfusepath{clip}%
\pgfsetbuttcap%
\pgfsetroundjoin%
\definecolor{currentfill}{rgb}{0.277941,0.056324,0.381191}%
\pgfsetfillcolor{currentfill}%
\pgfsetlinewidth{0.000000pt}%
\definecolor{currentstroke}{rgb}{0.278826,0.175490,0.483397}%
\pgfsetstrokecolor{currentstroke}%
\pgfsetdash{}{0pt}%
\pgfpathmoveto{\pgfqpoint{7.123901in}{2.876974in}}%
\pgfpathlineto{\pgfqpoint{7.192040in}{2.737188in}}%
\pgfpathlineto{\pgfqpoint{7.262707in}{2.788802in}}%
\pgfpathclose%
\pgfusepath{fill}%
\end{pgfscope}%
\begin{pgfscope}%
\pgfpathrectangle{\pgfqpoint{0.539299in}{0.078740in}}{\pgfqpoint{7.842520in}{7.842520in}}%
\pgfusepath{clip}%
\pgfsetbuttcap%
\pgfsetroundjoin%
\definecolor{currentfill}{rgb}{0.277134,0.185228,0.489898}%
\pgfsetfillcolor{currentfill}%
\pgfsetlinewidth{0.000000pt}%
\definecolor{currentstroke}{rgb}{0.278012,0.180367,0.486697}%
\pgfsetstrokecolor{currentstroke}%
\pgfsetdash{}{0pt}%
\pgfpathmoveto{\pgfqpoint{6.287425in}{3.199853in}}%
\pgfpathlineto{\pgfqpoint{6.149041in}{3.265181in}}%
\pgfpathlineto{\pgfqpoint{6.213837in}{3.153582in}}%
\pgfpathclose%
\pgfusepath{fill}%
\end{pgfscope}%
\begin{pgfscope}%
\pgfpathrectangle{\pgfqpoint{0.539299in}{0.078740in}}{\pgfqpoint{7.842520in}{7.842520in}}%
\pgfusepath{clip}%
\pgfsetbuttcap%
\pgfsetroundjoin%
\definecolor{currentfill}{rgb}{0.281446,0.084320,0.407414}%
\pgfsetfillcolor{currentfill}%
\pgfsetlinewidth{0.000000pt}%
\definecolor{currentstroke}{rgb}{0.277134,0.185228,0.489898}%
\pgfsetstrokecolor{currentstroke}%
\pgfsetdash{}{0pt}%
\pgfpathmoveto{\pgfqpoint{7.053093in}{2.832020in}}%
\pgfpathlineto{\pgfqpoint{7.123901in}{2.876974in}}%
\pgfpathlineto{\pgfqpoint{6.914313in}{2.924367in}}%
\pgfpathclose%
\pgfusepath{fill}%
\end{pgfscope}%
\begin{pgfscope}%
\pgfpathrectangle{\pgfqpoint{0.539299in}{0.078740in}}{\pgfqpoint{7.842520in}{7.842520in}}%
\pgfusepath{clip}%
\pgfsetbuttcap%
\pgfsetroundjoin%
\definecolor{currentfill}{rgb}{0.212395,0.359683,0.551710}%
\pgfsetfillcolor{currentfill}%
\pgfsetlinewidth{0.000000pt}%
\definecolor{currentstroke}{rgb}{0.276194,0.190074,0.493001}%
\pgfsetstrokecolor{currentstroke}%
\pgfsetdash{}{0pt}%
\pgfpathmoveto{\pgfqpoint{5.175033in}{3.721001in}}%
\pgfpathlineto{\pgfqpoint{5.038520in}{3.846770in}}%
\pgfpathlineto{\pgfqpoint{4.961637in}{3.904256in}}%
\pgfpathclose%
\pgfusepath{fill}%
\end{pgfscope}%
\begin{pgfscope}%
\pgfpathrectangle{\pgfqpoint{0.539299in}{0.078740in}}{\pgfqpoint{7.842520in}{7.842520in}}%
\pgfusepath{clip}%
\pgfsetbuttcap%
\pgfsetroundjoin%
\definecolor{currentfill}{rgb}{0.119699,0.618490,0.536347}%
\pgfsetfillcolor{currentfill}%
\pgfsetlinewidth{0.000000pt}%
\definecolor{currentstroke}{rgb}{0.275191,0.194905,0.496005}%
\pgfsetstrokecolor{currentstroke}%
\pgfsetdash{}{0pt}%
\pgfpathmoveto{\pgfqpoint{3.081416in}{5.017996in}}%
\pgfpathlineto{\pgfqpoint{3.165894in}{5.004406in}}%
\pgfpathlineto{\pgfqpoint{2.954036in}{4.737238in}}%
\pgfpathclose%
\pgfusepath{fill}%
\end{pgfscope}%
\begin{pgfscope}%
\pgfpathrectangle{\pgfqpoint{0.539299in}{0.078740in}}{\pgfqpoint{7.842520in}{7.842520in}}%
\pgfusepath{clip}%
\pgfsetbuttcap%
\pgfsetroundjoin%
\definecolor{currentfill}{rgb}{0.283072,0.130895,0.449241}%
\pgfsetfillcolor{currentfill}%
\pgfsetlinewidth{0.000000pt}%
\definecolor{currentstroke}{rgb}{0.274128,0.199721,0.498911}%
\pgfsetstrokecolor{currentstroke}%
\pgfsetdash{}{0pt}%
\pgfpathmoveto{\pgfqpoint{6.703683in}{2.968722in}}%
\pgfpathlineto{\pgfqpoint{6.775662in}{3.012527in}}%
\pgfpathlineto{\pgfqpoint{6.637123in}{3.095135in}}%
\pgfpathclose%
\pgfusepath{fill}%
\end{pgfscope}%
\begin{pgfscope}%
\pgfpathrectangle{\pgfqpoint{0.539299in}{0.078740in}}{\pgfqpoint{7.842520in}{7.842520in}}%
\pgfusepath{clip}%
\pgfsetbuttcap%
\pgfsetroundjoin%
\definecolor{currentfill}{rgb}{0.262138,0.242286,0.520837}%
\pgfsetfillcolor{currentfill}%
\pgfsetlinewidth{0.000000pt}%
\definecolor{currentstroke}{rgb}{0.273006,0.204520,0.501721}%
\pgfsetstrokecolor{currentstroke}%
\pgfsetdash{}{0pt}%
\pgfpathmoveto{\pgfqpoint{5.586291in}{3.415267in}}%
\pgfpathlineto{\pgfqpoint{5.724096in}{3.331845in}}%
\pgfpathlineto{\pgfqpoint{5.661270in}{3.429383in}}%
\pgfpathclose%
\pgfusepath{fill}%
\end{pgfscope}%
\begin{pgfscope}%
\pgfpathrectangle{\pgfqpoint{0.539299in}{0.078740in}}{\pgfqpoint{7.842520in}{7.842520in}}%
\pgfusepath{clip}%
\pgfsetbuttcap%
\pgfsetroundjoin%
\definecolor{currentfill}{rgb}{0.267968,0.223549,0.512008}%
\pgfsetfillcolor{currentfill}%
\pgfsetlinewidth{0.000000pt}%
\definecolor{currentstroke}{rgb}{0.271828,0.209303,0.504434}%
\pgfsetstrokecolor{currentstroke}%
\pgfsetdash{}{0pt}%
\pgfpathmoveto{\pgfqpoint{5.798863in}{3.358134in}}%
\pgfpathlineto{\pgfqpoint{5.724096in}{3.331845in}}%
\pgfpathlineto{\pgfqpoint{5.936847in}{3.290164in}}%
\pgfpathclose%
\pgfusepath{fill}%
\end{pgfscope}%
\begin{pgfscope}%
\pgfpathrectangle{\pgfqpoint{0.539299in}{0.078740in}}{\pgfqpoint{7.842520in}{7.842520in}}%
\pgfusepath{clip}%
\pgfsetbuttcap%
\pgfsetroundjoin%
\definecolor{currentfill}{rgb}{0.150476,0.504369,0.557430}%
\pgfsetfillcolor{currentfill}%
\pgfsetlinewidth{0.000000pt}%
\definecolor{currentstroke}{rgb}{0.270595,0.214069,0.507052}%
\pgfsetstrokecolor{currentstroke}%
\pgfsetdash{}{0pt}%
\pgfpathmoveto{\pgfqpoint{2.784458in}{4.707293in}}%
\pgfpathlineto{\pgfqpoint{2.661974in}{4.301274in}}%
\pgfpathlineto{\pgfqpoint{2.577164in}{4.260977in}}%
\pgfpathclose%
\pgfusepath{fill}%
\end{pgfscope}%
\begin{pgfscope}%
\pgfpathrectangle{\pgfqpoint{0.539299in}{0.078740in}}{\pgfqpoint{7.842520in}{7.842520in}}%
\pgfusepath{clip}%
\pgfsetbuttcap%
\pgfsetroundjoin%
\definecolor{currentfill}{rgb}{0.150148,0.676631,0.506589}%
\pgfsetfillcolor{currentfill}%
\pgfsetlinewidth{0.000000pt}%
\definecolor{currentstroke}{rgb}{0.269308,0.218818,0.509577}%
\pgfsetstrokecolor{currentstroke}%
\pgfsetdash{}{0pt}%
\pgfpathmoveto{\pgfqpoint{3.380711in}{5.141597in}}%
\pgfpathlineto{\pgfqpoint{3.513793in}{5.222466in}}%
\pgfpathlineto{\pgfqpoint{3.597214in}{5.159015in}}%
\pgfpathclose%
\pgfusepath{fill}%
\end{pgfscope}%
\begin{pgfscope}%
\pgfpathrectangle{\pgfqpoint{0.539299in}{0.078740in}}{\pgfqpoint{7.842520in}{7.842520in}}%
\pgfusepath{clip}%
\pgfsetbuttcap%
\pgfsetroundjoin%
\definecolor{currentfill}{rgb}{0.275191,0.194905,0.496005}%
\pgfsetfillcolor{currentfill}%
\pgfsetlinewidth{0.000000pt}%
\definecolor{currentstroke}{rgb}{0.267968,0.223549,0.512008}%
\pgfsetstrokecolor{currentstroke}%
\pgfsetdash{}{0pt}%
\pgfpathmoveto{\pgfqpoint{6.213837in}{3.153582in}}%
\pgfpathlineto{\pgfqpoint{6.149041in}{3.265181in}}%
\pgfpathlineto{\pgfqpoint{6.075188in}{3.222781in}}%
\pgfpathclose%
\pgfusepath{fill}%
\end{pgfscope}%
\begin{pgfscope}%
\pgfpathrectangle{\pgfqpoint{0.539299in}{0.078740in}}{\pgfqpoint{7.842520in}{7.842520in}}%
\pgfusepath{clip}%
\pgfsetbuttcap%
\pgfsetroundjoin%
\definecolor{currentfill}{rgb}{0.279566,0.067836,0.391917}%
\pgfsetfillcolor{currentfill}%
\pgfsetlinewidth{0.000000pt}%
\definecolor{currentstroke}{rgb}{0.266580,0.228262,0.514349}%
\pgfsetstrokecolor{currentstroke}%
\pgfsetdash{}{0pt}%
\pgfpathmoveto{\pgfqpoint{7.123901in}{2.876974in}}%
\pgfpathlineto{\pgfqpoint{7.053093in}{2.832020in}}%
\pgfpathlineto{\pgfqpoint{7.192040in}{2.737188in}}%
\pgfpathclose%
\pgfusepath{fill}%
\end{pgfscope}%
\begin{pgfscope}%
\pgfpathrectangle{\pgfqpoint{0.539299in}{0.078740in}}{\pgfqpoint{7.842520in}{7.842520in}}%
\pgfusepath{clip}%
\pgfsetbuttcap%
\pgfsetroundjoin%
\definecolor{currentfill}{rgb}{0.231674,0.318106,0.544834}%
\pgfsetfillcolor{currentfill}%
\pgfsetlinewidth{0.000000pt}%
\definecolor{currentstroke}{rgb}{0.265145,0.232956,0.516599}%
\pgfsetstrokecolor{currentstroke}%
\pgfsetdash{}{0pt}%
\pgfpathmoveto{\pgfqpoint{5.235775in}{3.632407in}}%
\pgfpathlineto{\pgfqpoint{5.311793in}{3.607895in}}%
\pgfpathlineto{\pgfqpoint{5.175033in}{3.721001in}}%
\pgfpathclose%
\pgfusepath{fill}%
\end{pgfscope}%
\begin{pgfscope}%
\pgfpathrectangle{\pgfqpoint{0.539299in}{0.078740in}}{\pgfqpoint{7.842520in}{7.842520in}}%
\pgfusepath{clip}%
\pgfsetbuttcap%
\pgfsetroundjoin%
\definecolor{currentfill}{rgb}{0.153894,0.680203,0.504172}%
\pgfsetfillcolor{currentfill}%
\pgfsetlinewidth{0.000000pt}%
\definecolor{currentstroke}{rgb}{0.263663,0.237631,0.518762}%
\pgfsetstrokecolor{currentstroke}%
\pgfsetdash{}{0pt}%
\pgfpathmoveto{\pgfqpoint{3.513793in}{5.222466in}}%
\pgfpathlineto{\pgfqpoint{3.731742in}{5.157127in}}%
\pgfpathlineto{\pgfqpoint{3.597214in}{5.159015in}}%
\pgfpathclose%
\pgfusepath{fill}%
\end{pgfscope}%
\begin{pgfscope}%
\pgfpathrectangle{\pgfqpoint{0.539299in}{0.078740in}}{\pgfqpoint{7.842520in}{7.842520in}}%
\pgfusepath{clip}%
\pgfsetbuttcap%
\pgfsetroundjoin%
\definecolor{currentfill}{rgb}{0.272594,0.025563,0.353093}%
\pgfsetfillcolor{currentfill}%
\pgfsetlinewidth{0.000000pt}%
\definecolor{currentstroke}{rgb}{0.262138,0.242286,0.520837}%
\pgfsetstrokecolor{currentstroke}%
\pgfsetdash{}{0pt}%
\pgfpathmoveto{\pgfqpoint{7.331215in}{2.641797in}}%
\pgfpathlineto{\pgfqpoint{7.401863in}{2.702549in}}%
\pgfpathlineto{\pgfqpoint{7.262707in}{2.788802in}}%
\pgfpathclose%
\pgfusepath{fill}%
\end{pgfscope}%
\begin{pgfscope}%
\pgfpathrectangle{\pgfqpoint{0.539299in}{0.078740in}}{\pgfqpoint{7.842520in}{7.842520in}}%
\pgfusepath{clip}%
\pgfsetbuttcap%
\pgfsetroundjoin%
\definecolor{currentfill}{rgb}{0.283229,0.120777,0.440584}%
\pgfsetfillcolor{currentfill}%
\pgfsetlinewidth{0.000000pt}%
\definecolor{currentstroke}{rgb}{0.260571,0.246922,0.522828}%
\pgfsetstrokecolor{currentstroke}%
\pgfsetdash{}{0pt}%
\pgfpathmoveto{\pgfqpoint{6.703683in}{2.968722in}}%
\pgfpathlineto{\pgfqpoint{6.914313in}{2.924367in}}%
\pgfpathlineto{\pgfqpoint{6.775662in}{3.012527in}}%
\pgfpathclose%
\pgfusepath{fill}%
\end{pgfscope}%
\begin{pgfscope}%
\pgfpathrectangle{\pgfqpoint{0.539299in}{0.078740in}}{\pgfqpoint{7.842520in}{7.842520in}}%
\pgfusepath{clip}%
\pgfsetbuttcap%
\pgfsetroundjoin%
\definecolor{currentfill}{rgb}{0.124780,0.640461,0.527068}%
\pgfsetfillcolor{currentfill}%
\pgfsetlinewidth{0.000000pt}%
\definecolor{currentstroke}{rgb}{0.258965,0.251537,0.524736}%
\pgfsetstrokecolor{currentstroke}%
\pgfsetdash{}{0pt}%
\pgfpathmoveto{\pgfqpoint{4.003621in}{5.004426in}}%
\pgfpathlineto{\pgfqpoint{4.085083in}{4.900621in}}%
\pgfpathlineto{\pgfqpoint{3.867334in}{5.102013in}}%
\pgfpathclose%
\pgfusepath{fill}%
\end{pgfscope}%
\begin{pgfscope}%
\pgfpathrectangle{\pgfqpoint{0.539299in}{0.078740in}}{\pgfqpoint{7.842520in}{7.842520in}}%
\pgfusepath{clip}%
\pgfsetbuttcap%
\pgfsetroundjoin%
\definecolor{currentfill}{rgb}{0.282623,0.140926,0.457517}%
\pgfsetfillcolor{currentfill}%
\pgfsetlinewidth{0.000000pt}%
\definecolor{currentstroke}{rgb}{0.257322,0.256130,0.526563}%
\pgfsetstrokecolor{currentstroke}%
\pgfsetdash{}{0pt}%
\pgfpathmoveto{\pgfqpoint{6.637123in}{3.095135in}}%
\pgfpathlineto{\pgfqpoint{6.564795in}{3.052267in}}%
\pgfpathlineto{\pgfqpoint{6.703683in}{2.968722in}}%
\pgfpathclose%
\pgfusepath{fill}%
\end{pgfscope}%
\begin{pgfscope}%
\pgfpathrectangle{\pgfqpoint{0.539299in}{0.078740in}}{\pgfqpoint{7.842520in}{7.842520in}}%
\pgfusepath{clip}%
\pgfsetbuttcap%
\pgfsetroundjoin%
\definecolor{currentfill}{rgb}{0.278826,0.175490,0.483397}%
\pgfsetfillcolor{currentfill}%
\pgfsetlinewidth{0.000000pt}%
\definecolor{currentstroke}{rgb}{0.255645,0.260703,0.528312}%
\pgfsetstrokecolor{currentstroke}%
\pgfsetdash{}{0pt}%
\pgfpathmoveto{\pgfqpoint{6.426028in}{3.129202in}}%
\pgfpathlineto{\pgfqpoint{6.287425in}{3.199853in}}%
\pgfpathlineto{\pgfqpoint{6.352735in}{3.080591in}}%
\pgfpathclose%
\pgfusepath{fill}%
\end{pgfscope}%
\begin{pgfscope}%
\pgfpathrectangle{\pgfqpoint{0.539299in}{0.078740in}}{\pgfqpoint{7.842520in}{7.842520in}}%
\pgfusepath{clip}%
\pgfsetbuttcap%
\pgfsetroundjoin%
\definecolor{currentfill}{rgb}{0.140210,0.665859,0.513427}%
\pgfsetfillcolor{currentfill}%
\pgfsetlinewidth{0.000000pt}%
\definecolor{currentstroke}{rgb}{0.253935,0.265254,0.529983}%
\pgfsetstrokecolor{currentstroke}%
\pgfsetdash{}{0pt}%
\pgfpathmoveto{\pgfqpoint{3.165894in}{5.004406in}}%
\pgfpathlineto{\pgfqpoint{3.296607in}{5.181085in}}%
\pgfpathlineto{\pgfqpoint{3.380711in}{5.141597in}}%
\pgfpathclose%
\pgfusepath{fill}%
\end{pgfscope}%
\begin{pgfscope}%
\pgfpathrectangle{\pgfqpoint{0.539299in}{0.078740in}}{\pgfqpoint{7.842520in}{7.842520in}}%
\pgfusepath{clip}%
\pgfsetbuttcap%
\pgfsetroundjoin%
\definecolor{currentfill}{rgb}{0.243113,0.292092,0.538516}%
\pgfsetfillcolor{currentfill}%
\pgfsetlinewidth{0.000000pt}%
\definecolor{currentstroke}{rgb}{0.252194,0.269783,0.531579}%
\pgfsetstrokecolor{currentstroke}%
\pgfsetdash{}{0pt}%
\pgfpathmoveto{\pgfqpoint{5.311793in}{3.607895in}}%
\pgfpathlineto{\pgfqpoint{5.373172in}{3.514648in}}%
\pgfpathlineto{\pgfqpoint{5.448865in}{3.506545in}}%
\pgfpathclose%
\pgfusepath{fill}%
\end{pgfscope}%
\begin{pgfscope}%
\pgfpathrectangle{\pgfqpoint{0.539299in}{0.078740in}}{\pgfqpoint{7.842520in}{7.842520in}}%
\pgfusepath{clip}%
\pgfsetbuttcap%
\pgfsetroundjoin%
\definecolor{currentfill}{rgb}{0.175841,0.441290,0.557685}%
\pgfsetfillcolor{currentfill}%
\pgfsetlinewidth{0.000000pt}%
\definecolor{currentstroke}{rgb}{0.250425,0.274290,0.533103}%
\pgfsetstrokecolor{currentstroke}%
\pgfsetdash{}{0pt}%
\pgfpathmoveto{\pgfqpoint{4.687905in}{4.220515in}}%
\pgfpathlineto{\pgfqpoint{4.824755in}{4.057714in}}%
\pgfpathlineto{\pgfqpoint{4.765949in}{4.134630in}}%
\pgfpathclose%
\pgfusepath{fill}%
\end{pgfscope}%
\begin{pgfscope}%
\pgfpathrectangle{\pgfqpoint{0.539299in}{0.078740in}}{\pgfqpoint{7.842520in}{7.842520in}}%
\pgfusepath{clip}%
\pgfsetbuttcap%
\pgfsetroundjoin%
\definecolor{currentfill}{rgb}{0.120638,0.625828,0.533488}%
\pgfsetfillcolor{currentfill}%
\pgfsetlinewidth{0.000000pt}%
\definecolor{currentstroke}{rgb}{0.248629,0.278775,0.534556}%
\pgfsetstrokecolor{currentstroke}%
\pgfsetdash{}{0pt}%
\pgfpathmoveto{\pgfqpoint{4.003621in}{5.004426in}}%
\pgfpathlineto{\pgfqpoint{4.140312in}{4.874931in}}%
\pgfpathlineto{\pgfqpoint{4.085083in}{4.900621in}}%
\pgfpathclose%
\pgfusepath{fill}%
\end{pgfscope}%
\begin{pgfscope}%
\pgfpathrectangle{\pgfqpoint{0.539299in}{0.078740in}}{\pgfqpoint{7.842520in}{7.842520in}}%
\pgfusepath{clip}%
\pgfsetbuttcap%
\pgfsetroundjoin%
\definecolor{currentfill}{rgb}{0.165117,0.467423,0.558141}%
\pgfsetfillcolor{currentfill}%
\pgfsetlinewidth{0.000000pt}%
\definecolor{currentstroke}{rgb}{0.246811,0.283237,0.535941}%
\pgfsetstrokecolor{currentstroke}%
\pgfsetdash{}{0pt}%
\pgfpathmoveto{\pgfqpoint{4.687905in}{4.220515in}}%
\pgfpathlineto{\pgfqpoint{4.765949in}{4.134630in}}%
\pgfpathlineto{\pgfqpoint{4.551037in}{4.389232in}}%
\pgfpathclose%
\pgfusepath{fill}%
\end{pgfscope}%
\begin{pgfscope}%
\pgfpathrectangle{\pgfqpoint{0.539299in}{0.078740in}}{\pgfqpoint{7.842520in}{7.842520in}}%
\pgfusepath{clip}%
\pgfsetbuttcap%
\pgfsetroundjoin%
\definecolor{currentfill}{rgb}{0.195860,0.395433,0.555276}%
\pgfsetfillcolor{currentfill}%
\pgfsetlinewidth{0.000000pt}%
\definecolor{currentstroke}{rgb}{0.244972,0.287675,0.537260}%
\pgfsetstrokecolor{currentstroke}%
\pgfsetdash{}{0pt}%
\pgfpathmoveto{\pgfqpoint{4.961637in}{3.904256in}}%
\pgfpathlineto{\pgfqpoint{5.038520in}{3.846770in}}%
\pgfpathlineto{\pgfqpoint{4.824755in}{4.057714in}}%
\pgfpathclose%
\pgfusepath{fill}%
\end{pgfscope}%
\begin{pgfscope}%
\pgfpathrectangle{\pgfqpoint{0.539299in}{0.078740in}}{\pgfqpoint{7.842520in}{7.842520in}}%
\pgfusepath{clip}%
\pgfsetbuttcap%
\pgfsetroundjoin%
\definecolor{currentfill}{rgb}{0.144759,0.519093,0.556572}%
\pgfsetfillcolor{currentfill}%
\pgfsetlinewidth{0.000000pt}%
\definecolor{currentstroke}{rgb}{0.243113,0.292092,0.538516}%
\pgfsetstrokecolor{currentstroke}%
\pgfsetdash{}{0pt}%
\pgfpathmoveto{\pgfqpoint{4.551037in}{4.389232in}}%
\pgfpathlineto{\pgfqpoint{4.493544in}{4.454843in}}%
\pgfpathlineto{\pgfqpoint{4.414126in}{4.558993in}}%
\pgfpathclose%
\pgfusepath{fill}%
\end{pgfscope}%
\begin{pgfscope}%
\pgfpathrectangle{\pgfqpoint{0.539299in}{0.078740in}}{\pgfqpoint{7.842520in}{7.842520in}}%
\pgfusepath{clip}%
\pgfsetbuttcap%
\pgfsetroundjoin%
\definecolor{currentfill}{rgb}{0.274952,0.037752,0.364543}%
\pgfsetfillcolor{currentfill}%
\pgfsetlinewidth{0.000000pt}%
\definecolor{currentstroke}{rgb}{0.241237,0.296485,0.539709}%
\pgfsetstrokecolor{currentstroke}%
\pgfsetdash{}{0pt}%
\pgfpathmoveto{\pgfqpoint{7.262707in}{2.788802in}}%
\pgfpathlineto{\pgfqpoint{7.192040in}{2.737188in}}%
\pgfpathlineto{\pgfqpoint{7.331215in}{2.641797in}}%
\pgfpathclose%
\pgfusepath{fill}%
\end{pgfscope}%
\begin{pgfscope}%
\pgfpathrectangle{\pgfqpoint{0.539299in}{0.078740in}}{\pgfqpoint{7.842520in}{7.842520in}}%
\pgfusepath{clip}%
\pgfsetbuttcap%
\pgfsetroundjoin%
\definecolor{currentfill}{rgb}{0.135066,0.544853,0.554029}%
\pgfsetfillcolor{currentfill}%
\pgfsetlinewidth{0.000000pt}%
\definecolor{currentstroke}{rgb}{0.239346,0.300855,0.540844}%
\pgfsetstrokecolor{currentstroke}%
\pgfsetdash{}{0pt}%
\pgfpathmoveto{\pgfqpoint{4.277193in}{4.723467in}}%
\pgfpathlineto{\pgfqpoint{4.414126in}{4.558993in}}%
\pgfpathlineto{\pgfqpoint{4.493544in}{4.454843in}}%
\pgfpathclose%
\pgfusepath{fill}%
\end{pgfscope}%
\begin{pgfscope}%
\pgfpathrectangle{\pgfqpoint{0.539299in}{0.078740in}}{\pgfqpoint{7.842520in}{7.842520in}}%
\pgfusepath{clip}%
\pgfsetbuttcap%
\pgfsetroundjoin%
\definecolor{currentfill}{rgb}{0.123463,0.581687,0.547445}%
\pgfsetfillcolor{currentfill}%
\pgfsetlinewidth{0.000000pt}%
\definecolor{currentstroke}{rgb}{0.237441,0.305202,0.541921}%
\pgfsetstrokecolor{currentstroke}%
\pgfsetdash{}{0pt}%
\pgfpathmoveto{\pgfqpoint{4.277193in}{4.723467in}}%
\pgfpathlineto{\pgfqpoint{4.357325in}{4.615401in}}%
\pgfpathlineto{\pgfqpoint{4.140312in}{4.874931in}}%
\pgfpathclose%
\pgfusepath{fill}%
\end{pgfscope}%
\begin{pgfscope}%
\pgfpathrectangle{\pgfqpoint{0.539299in}{0.078740in}}{\pgfqpoint{7.842520in}{7.842520in}}%
\pgfusepath{clip}%
\pgfsetbuttcap%
\pgfsetroundjoin%
\definecolor{currentfill}{rgb}{0.210503,0.363727,0.552206}%
\pgfsetfillcolor{currentfill}%
\pgfsetlinewidth{0.000000pt}%
\definecolor{currentstroke}{rgb}{0.235526,0.309527,0.542944}%
\pgfsetstrokecolor{currentstroke}%
\pgfsetdash{}{0pt}%
\pgfpathmoveto{\pgfqpoint{4.961637in}{3.904256in}}%
\pgfpathlineto{\pgfqpoint{5.098620in}{3.762206in}}%
\pgfpathlineto{\pgfqpoint{5.175033in}{3.721001in}}%
\pgfpathclose%
\pgfusepath{fill}%
\end{pgfscope}%
\begin{pgfscope}%
\pgfpathrectangle{\pgfqpoint{0.539299in}{0.078740in}}{\pgfqpoint{7.842520in}{7.842520in}}%
\pgfusepath{clip}%
\pgfsetbuttcap%
\pgfsetroundjoin%
\definecolor{currentfill}{rgb}{0.214298,0.355619,0.551184}%
\pgfsetfillcolor{currentfill}%
\pgfsetlinewidth{0.000000pt}%
\definecolor{currentstroke}{rgb}{0.233603,0.313828,0.543914}%
\pgfsetstrokecolor{currentstroke}%
\pgfsetdash{}{0pt}%
\pgfpathmoveto{\pgfqpoint{2.320596in}{4.102334in}}%
\pgfpathlineto{\pgfqpoint{2.290658in}{3.630794in}}%
\pgfpathlineto{\pgfqpoint{2.205440in}{3.565187in}}%
\pgfpathclose%
\pgfusepath{fill}%
\end{pgfscope}%
\begin{pgfscope}%
\pgfpathrectangle{\pgfqpoint{0.539299in}{0.078740in}}{\pgfqpoint{7.842520in}{7.842520in}}%
\pgfusepath{clip}%
\pgfsetbuttcap%
\pgfsetroundjoin%
\definecolor{currentfill}{rgb}{0.267968,0.223549,0.512008}%
\pgfsetfillcolor{currentfill}%
\pgfsetlinewidth{0.000000pt}%
\definecolor{currentstroke}{rgb}{0.231674,0.318106,0.544834}%
\pgfsetstrokecolor{currentstroke}%
\pgfsetdash{}{0pt}%
\pgfpathmoveto{\pgfqpoint{5.724096in}{3.331845in}}%
\pgfpathlineto{\pgfqpoint{5.862282in}{3.253769in}}%
\pgfpathlineto{\pgfqpoint{5.936847in}{3.290164in}}%
\pgfpathclose%
\pgfusepath{fill}%
\end{pgfscope}%
\begin{pgfscope}%
\pgfpathrectangle{\pgfqpoint{0.539299in}{0.078740in}}{\pgfqpoint{7.842520in}{7.842520in}}%
\pgfusepath{clip}%
\pgfsetbuttcap%
\pgfsetroundjoin%
\definecolor{currentfill}{rgb}{0.278012,0.180367,0.486697}%
\pgfsetfillcolor{currentfill}%
\pgfsetlinewidth{0.000000pt}%
\definecolor{currentstroke}{rgb}{0.229739,0.322361,0.545706}%
\pgfsetstrokecolor{currentstroke}%
\pgfsetdash{}{0pt}%
\pgfpathmoveto{\pgfqpoint{6.287425in}{3.199853in}}%
\pgfpathlineto{\pgfqpoint{6.213837in}{3.153582in}}%
\pgfpathlineto{\pgfqpoint{6.352735in}{3.080591in}}%
\pgfpathclose%
\pgfusepath{fill}%
\end{pgfscope}%
\begin{pgfscope}%
\pgfpathrectangle{\pgfqpoint{0.539299in}{0.078740in}}{\pgfqpoint{7.842520in}{7.842520in}}%
\pgfusepath{clip}%
\pgfsetbuttcap%
\pgfsetroundjoin%
\definecolor{currentfill}{rgb}{0.248629,0.278775,0.534556}%
\pgfsetfillcolor{currentfill}%
\pgfsetlinewidth{0.000000pt}%
\definecolor{currentstroke}{rgb}{0.227802,0.326594,0.546532}%
\pgfsetstrokecolor{currentstroke}%
\pgfsetdash{}{0pt}%
\pgfpathmoveto{\pgfqpoint{5.586291in}{3.415267in}}%
\pgfpathlineto{\pgfqpoint{5.448865in}{3.506545in}}%
\pgfpathlineto{\pgfqpoint{5.373172in}{3.514648in}}%
\pgfpathclose%
\pgfusepath{fill}%
\end{pgfscope}%
\begin{pgfscope}%
\pgfpathrectangle{\pgfqpoint{0.539299in}{0.078740in}}{\pgfqpoint{7.842520in}{7.842520in}}%
\pgfusepath{clip}%
\pgfsetbuttcap%
\pgfsetroundjoin%
\definecolor{currentfill}{rgb}{0.282656,0.100196,0.422160}%
\pgfsetfillcolor{currentfill}%
\pgfsetlinewidth{0.000000pt}%
\definecolor{currentstroke}{rgb}{0.225863,0.330805,0.547314}%
\pgfsetstrokecolor{currentstroke}%
\pgfsetdash{}{0pt}%
\pgfpathmoveto{\pgfqpoint{7.053093in}{2.832020in}}%
\pgfpathlineto{\pgfqpoint{6.914313in}{2.924367in}}%
\pgfpathlineto{\pgfqpoint{6.842661in}{2.878816in}}%
\pgfpathclose%
\pgfusepath{fill}%
\end{pgfscope}%
\begin{pgfscope}%
\pgfpathrectangle{\pgfqpoint{0.539299in}{0.078740in}}{\pgfqpoint{7.842520in}{7.842520in}}%
\pgfusepath{clip}%
\pgfsetbuttcap%
\pgfsetroundjoin%
\definecolor{currentfill}{rgb}{0.221989,0.339161,0.548752}%
\pgfsetfillcolor{currentfill}%
\pgfsetlinewidth{0.000000pt}%
\definecolor{currentstroke}{rgb}{0.223925,0.334994,0.548053}%
\pgfsetstrokecolor{currentstroke}%
\pgfsetdash{}{0pt}%
\pgfpathmoveto{\pgfqpoint{5.098620in}{3.762206in}}%
\pgfpathlineto{\pgfqpoint{5.235775in}{3.632407in}}%
\pgfpathlineto{\pgfqpoint{5.175033in}{3.721001in}}%
\pgfpathclose%
\pgfusepath{fill}%
\end{pgfscope}%
\begin{pgfscope}%
\pgfpathrectangle{\pgfqpoint{0.539299in}{0.078740in}}{\pgfqpoint{7.842520in}{7.842520in}}%
\pgfusepath{clip}%
\pgfsetbuttcap%
\pgfsetroundjoin%
\definecolor{currentfill}{rgb}{0.271828,0.209303,0.504434}%
\pgfsetfillcolor{currentfill}%
\pgfsetlinewidth{0.000000pt}%
\definecolor{currentstroke}{rgb}{0.221989,0.339161,0.548752}%
\pgfsetstrokecolor{currentstroke}%
\pgfsetdash{}{0pt}%
\pgfpathmoveto{\pgfqpoint{6.075188in}{3.222781in}}%
\pgfpathlineto{\pgfqpoint{5.936847in}{3.290164in}}%
\pgfpathlineto{\pgfqpoint{6.000832in}{3.178458in}}%
\pgfpathclose%
\pgfusepath{fill}%
\end{pgfscope}%
\begin{pgfscope}%
\pgfpathrectangle{\pgfqpoint{0.539299in}{0.078740in}}{\pgfqpoint{7.842520in}{7.842520in}}%
\pgfusepath{clip}%
\pgfsetbuttcap%
\pgfsetroundjoin%
\definecolor{currentfill}{rgb}{0.280255,0.165693,0.476498}%
\pgfsetfillcolor{currentfill}%
\pgfsetlinewidth{0.000000pt}%
\definecolor{currentstroke}{rgb}{0.220057,0.343307,0.549413}%
\pgfsetstrokecolor{currentstroke}%
\pgfsetdash{}{0pt}%
\pgfpathmoveto{\pgfqpoint{6.352735in}{3.080591in}}%
\pgfpathlineto{\pgfqpoint{6.564795in}{3.052267in}}%
\pgfpathlineto{\pgfqpoint{6.426028in}{3.129202in}}%
\pgfpathclose%
\pgfusepath{fill}%
\end{pgfscope}%
\begin{pgfscope}%
\pgfpathrectangle{\pgfqpoint{0.539299in}{0.078740in}}{\pgfqpoint{7.842520in}{7.842520in}}%
\pgfusepath{clip}%
\pgfsetbuttcap%
\pgfsetroundjoin%
\definecolor{currentfill}{rgb}{0.119738,0.603785,0.541400}%
\pgfsetfillcolor{currentfill}%
\pgfsetlinewidth{0.000000pt}%
\definecolor{currentstroke}{rgb}{0.218130,0.347432,0.550038}%
\pgfsetstrokecolor{currentstroke}%
\pgfsetdash{}{0pt}%
\pgfpathmoveto{\pgfqpoint{2.869477in}{4.725189in}}%
\pgfpathlineto{\pgfqpoint{2.996415in}{5.025431in}}%
\pgfpathlineto{\pgfqpoint{2.954036in}{4.737238in}}%
\pgfpathclose%
\pgfusepath{fill}%
\end{pgfscope}%
\begin{pgfscope}%
\pgfpathrectangle{\pgfqpoint{0.539299in}{0.078740in}}{\pgfqpoint{7.842520in}{7.842520in}}%
\pgfusepath{clip}%
\pgfsetbuttcap%
\pgfsetroundjoin%
\definecolor{currentfill}{rgb}{0.187231,0.414746,0.556547}%
\pgfsetfillcolor{currentfill}%
\pgfsetlinewidth{0.000000pt}%
\definecolor{currentstroke}{rgb}{0.216210,0.351535,0.550627}%
\pgfsetstrokecolor{currentstroke}%
\pgfsetdash{}{0pt}%
\pgfpathmoveto{\pgfqpoint{2.491969in}{4.215044in}}%
\pgfpathlineto{\pgfqpoint{2.290658in}{3.630794in}}%
\pgfpathlineto{\pgfqpoint{2.406429in}{4.162525in}}%
\pgfpathclose%
\pgfusepath{fill}%
\end{pgfscope}%
\begin{pgfscope}%
\pgfpathrectangle{\pgfqpoint{0.539299in}{0.078740in}}{\pgfqpoint{7.842520in}{7.842520in}}%
\pgfusepath{clip}%
\pgfsetbuttcap%
\pgfsetroundjoin%
\definecolor{currentfill}{rgb}{0.283197,0.115680,0.436115}%
\pgfsetfillcolor{currentfill}%
\pgfsetlinewidth{0.000000pt}%
\definecolor{currentstroke}{rgb}{0.214298,0.355619,0.551184}%
\pgfsetstrokecolor{currentstroke}%
\pgfsetdash{}{0pt}%
\pgfpathmoveto{\pgfqpoint{6.703683in}{2.968722in}}%
\pgfpathlineto{\pgfqpoint{6.842661in}{2.878816in}}%
\pgfpathlineto{\pgfqpoint{6.914313in}{2.924367in}}%
\pgfpathclose%
\pgfusepath{fill}%
\end{pgfscope}%
\begin{pgfscope}%
\pgfpathrectangle{\pgfqpoint{0.539299in}{0.078740in}}{\pgfqpoint{7.842520in}{7.842520in}}%
\pgfusepath{clip}%
\pgfsetbuttcap%
\pgfsetroundjoin%
\definecolor{currentfill}{rgb}{0.235526,0.309527,0.542944}%
\pgfsetfillcolor{currentfill}%
\pgfsetlinewidth{0.000000pt}%
\definecolor{currentstroke}{rgb}{0.212395,0.359683,0.551710}%
\pgfsetstrokecolor{currentstroke}%
\pgfsetdash{}{0pt}%
\pgfpathmoveto{\pgfqpoint{5.235775in}{3.632407in}}%
\pgfpathlineto{\pgfqpoint{5.373172in}{3.514648in}}%
\pgfpathlineto{\pgfqpoint{5.311793in}{3.607895in}}%
\pgfpathclose%
\pgfusepath{fill}%
\end{pgfscope}%
\begin{pgfscope}%
\pgfpathrectangle{\pgfqpoint{0.539299in}{0.078740in}}{\pgfqpoint{7.842520in}{7.842520in}}%
\pgfusepath{clip}%
\pgfsetbuttcap%
\pgfsetroundjoin%
\definecolor{currentfill}{rgb}{0.137339,0.662252,0.515571}%
\pgfsetfillcolor{currentfill}%
\pgfsetlinewidth{0.000000pt}%
\definecolor{currentstroke}{rgb}{0.210503,0.363727,0.552206}%
\pgfsetstrokecolor{currentstroke}%
\pgfsetdash{}{0pt}%
\pgfpathmoveto{\pgfqpoint{3.296607in}{5.181085in}}%
\pgfpathlineto{\pgfqpoint{3.165894in}{5.004406in}}%
\pgfpathlineto{\pgfqpoint{3.081416in}{5.017996in}}%
\pgfpathclose%
\pgfusepath{fill}%
\end{pgfscope}%
\begin{pgfscope}%
\pgfpathrectangle{\pgfqpoint{0.539299in}{0.078740in}}{\pgfqpoint{7.842520in}{7.842520in}}%
\pgfusepath{clip}%
\pgfsetbuttcap%
\pgfsetroundjoin%
\definecolor{currentfill}{rgb}{0.270595,0.214069,0.507052}%
\pgfsetfillcolor{currentfill}%
\pgfsetlinewidth{0.000000pt}%
\definecolor{currentstroke}{rgb}{0.208623,0.367752,0.552675}%
\pgfsetstrokecolor{currentstroke}%
\pgfsetdash{}{0pt}%
\pgfpathmoveto{\pgfqpoint{6.000832in}{3.178458in}}%
\pgfpathlineto{\pgfqpoint{5.936847in}{3.290164in}}%
\pgfpathlineto{\pgfqpoint{5.862282in}{3.253769in}}%
\pgfpathclose%
\pgfusepath{fill}%
\end{pgfscope}%
\begin{pgfscope}%
\pgfpathrectangle{\pgfqpoint{0.539299in}{0.078740in}}{\pgfqpoint{7.842520in}{7.842520in}}%
\pgfusepath{clip}%
\pgfsetbuttcap%
\pgfsetroundjoin%
\definecolor{currentfill}{rgb}{0.255645,0.260703,0.528312}%
\pgfsetfillcolor{currentfill}%
\pgfsetlinewidth{0.000000pt}%
\definecolor{currentstroke}{rgb}{0.206756,0.371758,0.553117}%
\pgfsetstrokecolor{currentstroke}%
\pgfsetdash{}{0pt}%
\pgfpathmoveto{\pgfqpoint{5.586291in}{3.415267in}}%
\pgfpathlineto{\pgfqpoint{5.510865in}{3.407873in}}%
\pgfpathlineto{\pgfqpoint{5.724096in}{3.331845in}}%
\pgfpathclose%
\pgfusepath{fill}%
\end{pgfscope}%
\begin{pgfscope}%
\pgfpathrectangle{\pgfqpoint{0.539299in}{0.078740in}}{\pgfqpoint{7.842520in}{7.842520in}}%
\pgfusepath{clip}%
\pgfsetbuttcap%
\pgfsetroundjoin%
\definecolor{currentfill}{rgb}{0.280267,0.073417,0.397163}%
\pgfsetfillcolor{currentfill}%
\pgfsetlinewidth{0.000000pt}%
\definecolor{currentstroke}{rgb}{0.204903,0.375746,0.553533}%
\pgfsetstrokecolor{currentstroke}%
\pgfsetdash{}{0pt}%
\pgfpathmoveto{\pgfqpoint{7.192040in}{2.737188in}}%
\pgfpathlineto{\pgfqpoint{7.053093in}{2.832020in}}%
\pgfpathlineto{\pgfqpoint{6.981719in}{2.783278in}}%
\pgfpathclose%
\pgfusepath{fill}%
\end{pgfscope}%
\begin{pgfscope}%
\pgfpathrectangle{\pgfqpoint{0.539299in}{0.078740in}}{\pgfqpoint{7.842520in}{7.842520in}}%
\pgfusepath{clip}%
\pgfsetbuttcap%
\pgfsetroundjoin%
\definecolor{currentfill}{rgb}{0.274128,0.199721,0.498911}%
\pgfsetfillcolor{currentfill}%
\pgfsetlinewidth{0.000000pt}%
\definecolor{currentstroke}{rgb}{0.203063,0.379716,0.553925}%
\pgfsetstrokecolor{currentstroke}%
\pgfsetdash{}{0pt}%
\pgfpathmoveto{\pgfqpoint{6.000832in}{3.178458in}}%
\pgfpathlineto{\pgfqpoint{6.213837in}{3.153582in}}%
\pgfpathlineto{\pgfqpoint{6.075188in}{3.222781in}}%
\pgfpathclose%
\pgfusepath{fill}%
\end{pgfscope}%
\begin{pgfscope}%
\pgfpathrectangle{\pgfqpoint{0.539299in}{0.078740in}}{\pgfqpoint{7.842520in}{7.842520in}}%
\pgfusepath{clip}%
\pgfsetbuttcap%
\pgfsetroundjoin%
\definecolor{currentfill}{rgb}{0.122312,0.633153,0.530398}%
\pgfsetfillcolor{currentfill}%
\pgfsetlinewidth{0.000000pt}%
\definecolor{currentstroke}{rgb}{0.201239,0.383670,0.554294}%
\pgfsetstrokecolor{currentstroke}%
\pgfsetdash{}{0pt}%
\pgfpathmoveto{\pgfqpoint{2.954036in}{4.737238in}}%
\pgfpathlineto{\pgfqpoint{2.996415in}{5.025431in}}%
\pgfpathlineto{\pgfqpoint{3.081416in}{5.017996in}}%
\pgfpathclose%
\pgfusepath{fill}%
\end{pgfscope}%
\begin{pgfscope}%
\pgfpathrectangle{\pgfqpoint{0.539299in}{0.078740in}}{\pgfqpoint{7.842520in}{7.842520in}}%
\pgfusepath{clip}%
\pgfsetbuttcap%
\pgfsetroundjoin%
\definecolor{currentfill}{rgb}{0.281887,0.150881,0.465405}%
\pgfsetfillcolor{currentfill}%
\pgfsetlinewidth{0.000000pt}%
\definecolor{currentstroke}{rgb}{0.199430,0.387607,0.554642}%
\pgfsetstrokecolor{currentstroke}%
\pgfsetdash{}{0pt}%
\pgfpathmoveto{\pgfqpoint{6.703683in}{2.968722in}}%
\pgfpathlineto{\pgfqpoint{6.564795in}{3.052267in}}%
\pgfpathlineto{\pgfqpoint{6.491826in}{3.002343in}}%
\pgfpathclose%
\pgfusepath{fill}%
\end{pgfscope}%
\begin{pgfscope}%
\pgfpathrectangle{\pgfqpoint{0.539299in}{0.078740in}}{\pgfqpoint{7.842520in}{7.842520in}}%
\pgfusepath{clip}%
\pgfsetbuttcap%
\pgfsetroundjoin%
\definecolor{currentfill}{rgb}{0.175707,0.697900,0.491033}%
\pgfsetfillcolor{currentfill}%
\pgfsetlinewidth{0.000000pt}%
\definecolor{currentstroke}{rgb}{0.197636,0.391528,0.554969}%
\pgfsetstrokecolor{currentstroke}%
\pgfsetdash{}{0pt}%
\pgfpathmoveto{\pgfqpoint{3.648554in}{5.237108in}}%
\pgfpathlineto{\pgfqpoint{3.731742in}{5.157127in}}%
\pgfpathlineto{\pgfqpoint{3.513793in}{5.222466in}}%
\pgfpathclose%
\pgfusepath{fill}%
\end{pgfscope}%
\begin{pgfscope}%
\pgfpathrectangle{\pgfqpoint{0.539299in}{0.078740in}}{\pgfqpoint{7.842520in}{7.842520in}}%
\pgfusepath{clip}%
\pgfsetbuttcap%
\pgfsetroundjoin%
\definecolor{currentfill}{rgb}{0.282327,0.094955,0.417331}%
\pgfsetfillcolor{currentfill}%
\pgfsetlinewidth{0.000000pt}%
\definecolor{currentstroke}{rgb}{0.195860,0.395433,0.555276}%
\pgfsetstrokecolor{currentstroke}%
\pgfsetdash{}{0pt}%
\pgfpathmoveto{\pgfqpoint{6.842661in}{2.878816in}}%
\pgfpathlineto{\pgfqpoint{6.981719in}{2.783278in}}%
\pgfpathlineto{\pgfqpoint{7.053093in}{2.832020in}}%
\pgfpathclose%
\pgfusepath{fill}%
\end{pgfscope}%
\begin{pgfscope}%
\pgfpathrectangle{\pgfqpoint{0.539299in}{0.078740in}}{\pgfqpoint{7.842520in}{7.842520in}}%
\pgfusepath{clip}%
\pgfsetbuttcap%
\pgfsetroundjoin%
\definecolor{currentfill}{rgb}{0.157851,0.683765,0.501686}%
\pgfsetfillcolor{currentfill}%
\pgfsetlinewidth{0.000000pt}%
\definecolor{currentstroke}{rgb}{0.194100,0.399323,0.555565}%
\pgfsetstrokecolor{currentstroke}%
\pgfsetdash{}{0pt}%
\pgfpathmoveto{\pgfqpoint{3.784535in}{5.195377in}}%
\pgfpathlineto{\pgfqpoint{3.867334in}{5.102013in}}%
\pgfpathlineto{\pgfqpoint{3.731742in}{5.157127in}}%
\pgfpathclose%
\pgfusepath{fill}%
\end{pgfscope}%
\begin{pgfscope}%
\pgfpathrectangle{\pgfqpoint{0.539299in}{0.078740in}}{\pgfqpoint{7.842520in}{7.842520in}}%
\pgfusepath{clip}%
\pgfsetbuttcap%
\pgfsetroundjoin%
\definecolor{currentfill}{rgb}{0.248629,0.278775,0.534556}%
\pgfsetfillcolor{currentfill}%
\pgfsetlinewidth{0.000000pt}%
\definecolor{currentstroke}{rgb}{0.192357,0.403199,0.555836}%
\pgfsetstrokecolor{currentstroke}%
\pgfsetdash{}{0pt}%
\pgfpathmoveto{\pgfqpoint{5.373172in}{3.514648in}}%
\pgfpathlineto{\pgfqpoint{5.510865in}{3.407873in}}%
\pgfpathlineto{\pgfqpoint{5.586291in}{3.415267in}}%
\pgfpathclose%
\pgfusepath{fill}%
\end{pgfscope}%
\begin{pgfscope}%
\pgfpathrectangle{\pgfqpoint{0.539299in}{0.078740in}}{\pgfqpoint{7.842520in}{7.842520in}}%
\pgfusepath{clip}%
\pgfsetbuttcap%
\pgfsetroundjoin%
\definecolor{currentfill}{rgb}{0.280868,0.160771,0.472899}%
\pgfsetfillcolor{currentfill}%
\pgfsetlinewidth{0.000000pt}%
\definecolor{currentstroke}{rgb}{0.190631,0.407061,0.556089}%
\pgfsetstrokecolor{currentstroke}%
\pgfsetdash{}{0pt}%
\pgfpathmoveto{\pgfqpoint{6.491826in}{3.002343in}}%
\pgfpathlineto{\pgfqpoint{6.564795in}{3.052267in}}%
\pgfpathlineto{\pgfqpoint{6.352735in}{3.080591in}}%
\pgfpathclose%
\pgfusepath{fill}%
\end{pgfscope}%
\begin{pgfscope}%
\pgfpathrectangle{\pgfqpoint{0.539299in}{0.078740in}}{\pgfqpoint{7.842520in}{7.842520in}}%
\pgfusepath{clip}%
\pgfsetbuttcap%
\pgfsetroundjoin%
\definecolor{currentfill}{rgb}{0.180653,0.701402,0.488189}%
\pgfsetfillcolor{currentfill}%
\pgfsetlinewidth{0.000000pt}%
\definecolor{currentstroke}{rgb}{0.188923,0.410910,0.556326}%
\pgfsetstrokecolor{currentstroke}%
\pgfsetdash{}{0pt}%
\pgfpathmoveto{\pgfqpoint{3.380711in}{5.141597in}}%
\pgfpathlineto{\pgfqpoint{3.429695in}{5.281129in}}%
\pgfpathlineto{\pgfqpoint{3.513793in}{5.222466in}}%
\pgfpathclose%
\pgfusepath{fill}%
\end{pgfscope}%
\begin{pgfscope}%
\pgfpathrectangle{\pgfqpoint{0.539299in}{0.078740in}}{\pgfqpoint{7.842520in}{7.842520in}}%
\pgfusepath{clip}%
\pgfsetbuttcap%
\pgfsetroundjoin%
\definecolor{currentfill}{rgb}{0.276022,0.044167,0.370164}%
\pgfsetfillcolor{currentfill}%
\pgfsetlinewidth{0.000000pt}%
\definecolor{currentstroke}{rgb}{0.187231,0.414746,0.556547}%
\pgfsetstrokecolor{currentstroke}%
\pgfsetdash{}{0pt}%
\pgfpathmoveto{\pgfqpoint{7.331215in}{2.641797in}}%
\pgfpathlineto{\pgfqpoint{7.192040in}{2.737188in}}%
\pgfpathlineto{\pgfqpoint{7.120863in}{2.683197in}}%
\pgfpathclose%
\pgfusepath{fill}%
\end{pgfscope}%
\begin{pgfscope}%
\pgfpathrectangle{\pgfqpoint{0.539299in}{0.078740in}}{\pgfqpoint{7.842520in}{7.842520in}}%
\pgfusepath{clip}%
\pgfsetbuttcap%
\pgfsetroundjoin%
\definecolor{currentfill}{rgb}{0.262138,0.242286,0.520837}%
\pgfsetfillcolor{currentfill}%
\pgfsetlinewidth{0.000000pt}%
\definecolor{currentstroke}{rgb}{0.185556,0.418570,0.556753}%
\pgfsetstrokecolor{currentstroke}%
\pgfsetdash{}{0pt}%
\pgfpathmoveto{\pgfqpoint{5.862282in}{3.253769in}}%
\pgfpathlineto{\pgfqpoint{5.724096in}{3.331845in}}%
\pgfpathlineto{\pgfqpoint{5.648894in}{3.310389in}}%
\pgfpathclose%
\pgfusepath{fill}%
\end{pgfscope}%
\begin{pgfscope}%
\pgfpathrectangle{\pgfqpoint{0.539299in}{0.078740in}}{\pgfqpoint{7.842520in}{7.842520in}}%
\pgfusepath{clip}%
\pgfsetbuttcap%
\pgfsetroundjoin%
\definecolor{currentfill}{rgb}{0.147607,0.511733,0.557049}%
\pgfsetfillcolor{currentfill}%
\pgfsetlinewidth{0.000000pt}%
\definecolor{currentstroke}{rgb}{0.183898,0.422383,0.556944}%
\pgfsetstrokecolor{currentstroke}%
\pgfsetdash{}{0pt}%
\pgfpathmoveto{\pgfqpoint{2.699021in}{4.682479in}}%
\pgfpathlineto{\pgfqpoint{2.577164in}{4.260977in}}%
\pgfpathlineto{\pgfqpoint{2.491969in}{4.215044in}}%
\pgfpathclose%
\pgfusepath{fill}%
\end{pgfscope}%
\begin{pgfscope}%
\pgfpathrectangle{\pgfqpoint{0.539299in}{0.078740in}}{\pgfqpoint{7.842520in}{7.842520in}}%
\pgfusepath{clip}%
\pgfsetbuttcap%
\pgfsetroundjoin%
\definecolor{currentfill}{rgb}{0.175707,0.697900,0.491033}%
\pgfsetfillcolor{currentfill}%
\pgfsetlinewidth{0.000000pt}%
\definecolor{currentstroke}{rgb}{0.182256,0.426184,0.557120}%
\pgfsetstrokecolor{currentstroke}%
\pgfsetdash{}{0pt}%
\pgfpathmoveto{\pgfqpoint{3.296607in}{5.181085in}}%
\pgfpathlineto{\pgfqpoint{3.429695in}{5.281129in}}%
\pgfpathlineto{\pgfqpoint{3.380711in}{5.141597in}}%
\pgfpathclose%
\pgfusepath{fill}%
\end{pgfscope}%
\begin{pgfscope}%
\pgfpathrectangle{\pgfqpoint{0.539299in}{0.078740in}}{\pgfqpoint{7.842520in}{7.842520in}}%
\pgfusepath{clip}%
\pgfsetbuttcap%
\pgfsetroundjoin%
\definecolor{currentfill}{rgb}{0.277134,0.185228,0.489898}%
\pgfsetfillcolor{currentfill}%
\pgfsetlinewidth{0.000000pt}%
\definecolor{currentstroke}{rgb}{0.180629,0.429975,0.557282}%
\pgfsetstrokecolor{currentstroke}%
\pgfsetdash{}{0pt}%
\pgfpathmoveto{\pgfqpoint{6.352735in}{3.080591in}}%
\pgfpathlineto{\pgfqpoint{6.213837in}{3.153582in}}%
\pgfpathlineto{\pgfqpoint{6.139708in}{3.103444in}}%
\pgfpathclose%
\pgfusepath{fill}%
\end{pgfscope}%
\begin{pgfscope}%
\pgfpathrectangle{\pgfqpoint{0.539299in}{0.078740in}}{\pgfqpoint{7.842520in}{7.842520in}}%
\pgfusepath{clip}%
\pgfsetbuttcap%
\pgfsetroundjoin%
\definecolor{currentfill}{rgb}{0.185556,0.418570,0.556753}%
\pgfsetfillcolor{currentfill}%
\pgfsetlinewidth{0.000000pt}%
\definecolor{currentstroke}{rgb}{0.179019,0.433756,0.557430}%
\pgfsetstrokecolor{currentstroke}%
\pgfsetdash{}{0pt}%
\pgfpathmoveto{\pgfqpoint{2.406429in}{4.162525in}}%
\pgfpathlineto{\pgfqpoint{2.290658in}{3.630794in}}%
\pgfpathlineto{\pgfqpoint{2.320596in}{4.102334in}}%
\pgfpathclose%
\pgfusepath{fill}%
\end{pgfscope}%
\begin{pgfscope}%
\pgfpathrectangle{\pgfqpoint{0.539299in}{0.078740in}}{\pgfqpoint{7.842520in}{7.842520in}}%
\pgfusepath{clip}%
\pgfsetbuttcap%
\pgfsetroundjoin%
\definecolor{currentfill}{rgb}{0.279566,0.067836,0.391917}%
\pgfsetfillcolor{currentfill}%
\pgfsetlinewidth{0.000000pt}%
\definecolor{currentstroke}{rgb}{0.177423,0.437527,0.557565}%
\pgfsetstrokecolor{currentstroke}%
\pgfsetdash{}{0pt}%
\pgfpathmoveto{\pgfqpoint{7.120863in}{2.683197in}}%
\pgfpathlineto{\pgfqpoint{7.192040in}{2.737188in}}%
\pgfpathlineto{\pgfqpoint{6.981719in}{2.783278in}}%
\pgfpathclose%
\pgfusepath{fill}%
\end{pgfscope}%
\begin{pgfscope}%
\pgfpathrectangle{\pgfqpoint{0.539299in}{0.078740in}}{\pgfqpoint{7.842520in}{7.842520in}}%
\pgfusepath{clip}%
\pgfsetbuttcap%
\pgfsetroundjoin%
\definecolor{currentfill}{rgb}{0.274128,0.199721,0.498911}%
\pgfsetfillcolor{currentfill}%
\pgfsetlinewidth{0.000000pt}%
\definecolor{currentstroke}{rgb}{0.175841,0.441290,0.557685}%
\pgfsetstrokecolor{currentstroke}%
\pgfsetdash{}{0pt}%
\pgfpathmoveto{\pgfqpoint{6.139708in}{3.103444in}}%
\pgfpathlineto{\pgfqpoint{6.213837in}{3.153582in}}%
\pgfpathlineto{\pgfqpoint{6.000832in}{3.178458in}}%
\pgfpathclose%
\pgfusepath{fill}%
\end{pgfscope}%
\begin{pgfscope}%
\pgfpathrectangle{\pgfqpoint{0.539299in}{0.078740in}}{\pgfqpoint{7.842520in}{7.842520in}}%
\pgfusepath{clip}%
\pgfsetbuttcap%
\pgfsetroundjoin%
\definecolor{currentfill}{rgb}{0.132444,0.552216,0.553018}%
\pgfsetfillcolor{currentfill}%
\pgfsetlinewidth{0.000000pt}%
\definecolor{currentstroke}{rgb}{0.174274,0.445044,0.557792}%
\pgfsetstrokecolor{currentstroke}%
\pgfsetdash{}{0pt}%
\pgfpathmoveto{\pgfqpoint{2.577164in}{4.260977in}}%
\pgfpathlineto{\pgfqpoint{2.699021in}{4.682479in}}%
\pgfpathlineto{\pgfqpoint{2.784458in}{4.707293in}}%
\pgfpathclose%
\pgfusepath{fill}%
\end{pgfscope}%
\begin{pgfscope}%
\pgfpathrectangle{\pgfqpoint{0.539299in}{0.078740in}}{\pgfqpoint{7.842520in}{7.842520in}}%
\pgfusepath{clip}%
\pgfsetbuttcap%
\pgfsetroundjoin%
\definecolor{currentfill}{rgb}{0.255645,0.260703,0.528312}%
\pgfsetfillcolor{currentfill}%
\pgfsetlinewidth{0.000000pt}%
\definecolor{currentstroke}{rgb}{0.172719,0.448791,0.557885}%
\pgfsetstrokecolor{currentstroke}%
\pgfsetdash{}{0pt}%
\pgfpathmoveto{\pgfqpoint{5.724096in}{3.331845in}}%
\pgfpathlineto{\pgfqpoint{5.510865in}{3.407873in}}%
\pgfpathlineto{\pgfqpoint{5.648894in}{3.310389in}}%
\pgfpathclose%
\pgfusepath{fill}%
\end{pgfscope}%
\begin{pgfscope}%
\pgfpathrectangle{\pgfqpoint{0.539299in}{0.078740in}}{\pgfqpoint{7.842520in}{7.842520in}}%
\pgfusepath{clip}%
\pgfsetbuttcap%
\pgfsetroundjoin%
\definecolor{currentfill}{rgb}{0.146616,0.673050,0.508936}%
\pgfsetfillcolor{currentfill}%
\pgfsetlinewidth{0.000000pt}%
\definecolor{currentstroke}{rgb}{0.171176,0.452530,0.557965}%
\pgfsetstrokecolor{currentstroke}%
\pgfsetdash{}{0pt}%
\pgfpathmoveto{\pgfqpoint{4.003621in}{5.004426in}}%
\pgfpathlineto{\pgfqpoint{3.867334in}{5.102013in}}%
\pgfpathlineto{\pgfqpoint{3.921341in}{5.107578in}}%
\pgfpathclose%
\pgfusepath{fill}%
\end{pgfscope}%
\begin{pgfscope}%
\pgfpathrectangle{\pgfqpoint{0.539299in}{0.078740in}}{\pgfqpoint{7.842520in}{7.842520in}}%
\pgfusepath{clip}%
\pgfsetbuttcap%
\pgfsetroundjoin%
\definecolor{currentfill}{rgb}{0.119483,0.614817,0.537692}%
\pgfsetfillcolor{currentfill}%
\pgfsetlinewidth{0.000000pt}%
\definecolor{currentstroke}{rgb}{0.169646,0.456262,0.558030}%
\pgfsetstrokecolor{currentstroke}%
\pgfsetdash{}{0pt}%
\pgfpathmoveto{\pgfqpoint{2.784458in}{4.707293in}}%
\pgfpathlineto{\pgfqpoint{2.996415in}{5.025431in}}%
\pgfpathlineto{\pgfqpoint{2.869477in}{4.725189in}}%
\pgfpathclose%
\pgfusepath{fill}%
\end{pgfscope}%
\begin{pgfscope}%
\pgfpathrectangle{\pgfqpoint{0.539299in}{0.078740in}}{\pgfqpoint{7.842520in}{7.842520in}}%
\pgfusepath{clip}%
\pgfsetbuttcap%
\pgfsetroundjoin%
\definecolor{currentfill}{rgb}{0.283229,0.120777,0.440584}%
\pgfsetfillcolor{currentfill}%
\pgfsetlinewidth{0.000000pt}%
\definecolor{currentstroke}{rgb}{0.168126,0.459988,0.558082}%
\pgfsetstrokecolor{currentstroke}%
\pgfsetdash{}{0pt}%
\pgfpathmoveto{\pgfqpoint{6.770368in}{2.826882in}}%
\pgfpathlineto{\pgfqpoint{6.842661in}{2.878816in}}%
\pgfpathlineto{\pgfqpoint{6.703683in}{2.968722in}}%
\pgfpathclose%
\pgfusepath{fill}%
\end{pgfscope}%
\begin{pgfscope}%
\pgfpathrectangle{\pgfqpoint{0.539299in}{0.078740in}}{\pgfqpoint{7.842520in}{7.842520in}}%
\pgfusepath{clip}%
\pgfsetbuttcap%
\pgfsetroundjoin%
\definecolor{currentfill}{rgb}{0.180653,0.701402,0.488189}%
\pgfsetfillcolor{currentfill}%
\pgfsetlinewidth{0.000000pt}%
\definecolor{currentstroke}{rgb}{0.166617,0.463708,0.558119}%
\pgfsetstrokecolor{currentstroke}%
\pgfsetdash{}{0pt}%
\pgfpathmoveto{\pgfqpoint{3.784535in}{5.195377in}}%
\pgfpathlineto{\pgfqpoint{3.731742in}{5.157127in}}%
\pgfpathlineto{\pgfqpoint{3.648554in}{5.237108in}}%
\pgfpathclose%
\pgfusepath{fill}%
\end{pgfscope}%
\begin{pgfscope}%
\pgfpathrectangle{\pgfqpoint{0.539299in}{0.078740in}}{\pgfqpoint{7.842520in}{7.842520in}}%
\pgfusepath{clip}%
\pgfsetbuttcap%
\pgfsetroundjoin%
\definecolor{currentfill}{rgb}{0.282290,0.145912,0.461510}%
\pgfsetfillcolor{currentfill}%
\pgfsetlinewidth{0.000000pt}%
\definecolor{currentstroke}{rgb}{0.165117,0.467423,0.558141}%
\pgfsetstrokecolor{currentstroke}%
\pgfsetdash{}{0pt}%
\pgfpathmoveto{\pgfqpoint{6.491826in}{3.002343in}}%
\pgfpathlineto{\pgfqpoint{6.631052in}{2.917912in}}%
\pgfpathlineto{\pgfqpoint{6.703683in}{2.968722in}}%
\pgfpathclose%
\pgfusepath{fill}%
\end{pgfscope}%
\begin{pgfscope}%
\pgfpathrectangle{\pgfqpoint{0.539299in}{0.078740in}}{\pgfqpoint{7.842520in}{7.842520in}}%
\pgfusepath{clip}%
\pgfsetbuttcap%
\pgfsetroundjoin%
\definecolor{currentfill}{rgb}{0.183898,0.422383,0.556944}%
\pgfsetfillcolor{currentfill}%
\pgfsetlinewidth{0.000000pt}%
\definecolor{currentstroke}{rgb}{0.163625,0.471133,0.558148}%
\pgfsetstrokecolor{currentstroke}%
\pgfsetdash{}{0pt}%
\pgfpathmoveto{\pgfqpoint{4.961637in}{3.904256in}}%
\pgfpathlineto{\pgfqpoint{4.824755in}{4.057714in}}%
\pgfpathlineto{\pgfqpoint{4.884097in}{3.971926in}}%
\pgfpathclose%
\pgfusepath{fill}%
\end{pgfscope}%
\begin{pgfscope}%
\pgfpathrectangle{\pgfqpoint{0.539299in}{0.078740in}}{\pgfqpoint{7.842520in}{7.842520in}}%
\pgfusepath{clip}%
\pgfsetbuttcap%
\pgfsetroundjoin%
\definecolor{currentfill}{rgb}{0.194100,0.399323,0.555565}%
\pgfsetfillcolor{currentfill}%
\pgfsetlinewidth{0.000000pt}%
\definecolor{currentstroke}{rgb}{0.162142,0.474838,0.558140}%
\pgfsetstrokecolor{currentstroke}%
\pgfsetdash{}{0pt}%
\pgfpathmoveto{\pgfqpoint{4.884097in}{3.971926in}}%
\pgfpathlineto{\pgfqpoint{5.098620in}{3.762206in}}%
\pgfpathlineto{\pgfqpoint{4.961637in}{3.904256in}}%
\pgfpathclose%
\pgfusepath{fill}%
\end{pgfscope}%
\begin{pgfscope}%
\pgfpathrectangle{\pgfqpoint{0.539299in}{0.078740in}}{\pgfqpoint{7.842520in}{7.842520in}}%
\pgfusepath{clip}%
\pgfsetbuttcap%
\pgfsetroundjoin%
\definecolor{currentfill}{rgb}{0.162142,0.474838,0.558140}%
\pgfsetfillcolor{currentfill}%
\pgfsetlinewidth{0.000000pt}%
\definecolor{currentstroke}{rgb}{0.160665,0.478540,0.558115}%
\pgfsetstrokecolor{currentstroke}%
\pgfsetdash{}{0pt}%
\pgfpathmoveto{\pgfqpoint{4.609098in}{4.314481in}}%
\pgfpathlineto{\pgfqpoint{4.824755in}{4.057714in}}%
\pgfpathlineto{\pgfqpoint{4.687905in}{4.220515in}}%
\pgfpathclose%
\pgfusepath{fill}%
\end{pgfscope}%
\begin{pgfscope}%
\pgfpathrectangle{\pgfqpoint{0.539299in}{0.078740in}}{\pgfqpoint{7.842520in}{7.842520in}}%
\pgfusepath{clip}%
\pgfsetbuttcap%
\pgfsetroundjoin%
\definecolor{currentfill}{rgb}{0.214298,0.355619,0.551184}%
\pgfsetfillcolor{currentfill}%
\pgfsetlinewidth{0.000000pt}%
\definecolor{currentstroke}{rgb}{0.159194,0.482237,0.558073}%
\pgfsetstrokecolor{currentstroke}%
\pgfsetdash{}{0pt}%
\pgfpathmoveto{\pgfqpoint{2.205440in}{3.565187in}}%
\pgfpathlineto{\pgfqpoint{2.119879in}{3.494959in}}%
\pgfpathlineto{\pgfqpoint{2.234531in}{4.033237in}}%
\pgfpathclose%
\pgfusepath{fill}%
\end{pgfscope}%
\begin{pgfscope}%
\pgfpathrectangle{\pgfqpoint{0.539299in}{0.078740in}}{\pgfqpoint{7.842520in}{7.842520in}}%
\pgfusepath{clip}%
\pgfsetbuttcap%
\pgfsetroundjoin%
\definecolor{currentfill}{rgb}{0.151918,0.500685,0.557587}%
\pgfsetfillcolor{currentfill}%
\pgfsetlinewidth{0.000000pt}%
\definecolor{currentstroke}{rgb}{0.157729,0.485932,0.558013}%
\pgfsetstrokecolor{currentstroke}%
\pgfsetdash{}{0pt}%
\pgfpathmoveto{\pgfqpoint{4.687905in}{4.220515in}}%
\pgfpathlineto{\pgfqpoint{4.551037in}{4.389232in}}%
\pgfpathlineto{\pgfqpoint{4.609098in}{4.314481in}}%
\pgfpathclose%
\pgfusepath{fill}%
\end{pgfscope}%
\begin{pgfscope}%
\pgfpathrectangle{\pgfqpoint{0.539299in}{0.078740in}}{\pgfqpoint{7.842520in}{7.842520in}}%
\pgfusepath{clip}%
\pgfsetbuttcap%
\pgfsetroundjoin%
\definecolor{currentfill}{rgb}{0.134692,0.658636,0.517649}%
\pgfsetfillcolor{currentfill}%
\pgfsetlinewidth{0.000000pt}%
\definecolor{currentstroke}{rgb}{0.156270,0.489624,0.557936}%
\pgfsetstrokecolor{currentstroke}%
\pgfsetdash{}{0pt}%
\pgfpathmoveto{\pgfqpoint{3.921341in}{5.107578in}}%
\pgfpathlineto{\pgfqpoint{4.140312in}{4.874931in}}%
\pgfpathlineto{\pgfqpoint{4.003621in}{5.004426in}}%
\pgfpathclose%
\pgfusepath{fill}%
\end{pgfscope}%
\begin{pgfscope}%
\pgfpathrectangle{\pgfqpoint{0.539299in}{0.078740in}}{\pgfqpoint{7.842520in}{7.842520in}}%
\pgfusepath{clip}%
\pgfsetbuttcap%
\pgfsetroundjoin%
\definecolor{currentfill}{rgb}{0.267968,0.223549,0.512008}%
\pgfsetfillcolor{currentfill}%
\pgfsetlinewidth{0.000000pt}%
\definecolor{currentstroke}{rgb}{0.154815,0.493313,0.557840}%
\pgfsetstrokecolor{currentstroke}%
\pgfsetdash{}{0pt}%
\pgfpathmoveto{\pgfqpoint{6.000832in}{3.178458in}}%
\pgfpathlineto{\pgfqpoint{5.862282in}{3.253769in}}%
\pgfpathlineto{\pgfqpoint{5.787278in}{3.220081in}}%
\pgfpathclose%
\pgfusepath{fill}%
\end{pgfscope}%
\begin{pgfscope}%
\pgfpathrectangle{\pgfqpoint{0.539299in}{0.078740in}}{\pgfqpoint{7.842520in}{7.842520in}}%
\pgfusepath{clip}%
\pgfsetbuttcap%
\pgfsetroundjoin%
\definecolor{currentfill}{rgb}{0.214298,0.355619,0.551184}%
\pgfsetfillcolor{currentfill}%
\pgfsetlinewidth{0.000000pt}%
\definecolor{currentstroke}{rgb}{0.153364,0.497000,0.557724}%
\pgfsetstrokecolor{currentstroke}%
\pgfsetdash{}{0pt}%
\pgfpathmoveto{\pgfqpoint{5.159216in}{3.667393in}}%
\pgfpathlineto{\pgfqpoint{5.235775in}{3.632407in}}%
\pgfpathlineto{\pgfqpoint{5.098620in}{3.762206in}}%
\pgfpathclose%
\pgfusepath{fill}%
\end{pgfscope}%
\begin{pgfscope}%
\pgfpathrectangle{\pgfqpoint{0.539299in}{0.078740in}}{\pgfqpoint{7.842520in}{7.842520in}}%
\pgfusepath{clip}%
\pgfsetbuttcap%
\pgfsetroundjoin%
\definecolor{currentfill}{rgb}{0.274952,0.037752,0.364543}%
\pgfsetfillcolor{currentfill}%
\pgfsetlinewidth{0.000000pt}%
\definecolor{currentstroke}{rgb}{0.151918,0.500685,0.557587}%
\pgfsetstrokecolor{currentstroke}%
\pgfsetdash{}{0pt}%
\pgfpathmoveto{\pgfqpoint{7.120863in}{2.683197in}}%
\pgfpathlineto{\pgfqpoint{7.260119in}{2.579903in}}%
\pgfpathlineto{\pgfqpoint{7.331215in}{2.641797in}}%
\pgfpathclose%
\pgfusepath{fill}%
\end{pgfscope}%
\begin{pgfscope}%
\pgfpathrectangle{\pgfqpoint{0.539299in}{0.078740in}}{\pgfqpoint{7.842520in}{7.842520in}}%
\pgfusepath{clip}%
\pgfsetbuttcap%
\pgfsetroundjoin%
\definecolor{currentfill}{rgb}{0.131172,0.555899,0.552459}%
\pgfsetfillcolor{currentfill}%
\pgfsetlinewidth{0.000000pt}%
\definecolor{currentstroke}{rgb}{0.150476,0.504369,0.557430}%
\pgfsetstrokecolor{currentstroke}%
\pgfsetdash{}{0pt}%
\pgfpathmoveto{\pgfqpoint{4.333879in}{4.667896in}}%
\pgfpathlineto{\pgfqpoint{4.551037in}{4.389232in}}%
\pgfpathlineto{\pgfqpoint{4.414126in}{4.558993in}}%
\pgfpathclose%
\pgfusepath{fill}%
\end{pgfscope}%
\begin{pgfscope}%
\pgfpathrectangle{\pgfqpoint{0.539299in}{0.078740in}}{\pgfqpoint{7.842520in}{7.842520in}}%
\pgfusepath{clip}%
\pgfsetbuttcap%
\pgfsetroundjoin%
\definecolor{currentfill}{rgb}{0.123463,0.581687,0.547445}%
\pgfsetfillcolor{currentfill}%
\pgfsetlinewidth{0.000000pt}%
\definecolor{currentstroke}{rgb}{0.149039,0.508051,0.557250}%
\pgfsetstrokecolor{currentstroke}%
\pgfsetdash{}{0pt}%
\pgfpathmoveto{\pgfqpoint{4.333879in}{4.667896in}}%
\pgfpathlineto{\pgfqpoint{4.414126in}{4.558993in}}%
\pgfpathlineto{\pgfqpoint{4.277193in}{4.723467in}}%
\pgfpathclose%
\pgfusepath{fill}%
\end{pgfscope}%
\begin{pgfscope}%
\pgfpathrectangle{\pgfqpoint{0.539299in}{0.078740in}}{\pgfqpoint{7.842520in}{7.842520in}}%
\pgfusepath{clip}%
\pgfsetbuttcap%
\pgfsetroundjoin%
\definecolor{currentfill}{rgb}{0.119699,0.618490,0.536347}%
\pgfsetfillcolor{currentfill}%
\pgfsetlinewidth{0.000000pt}%
\definecolor{currentstroke}{rgb}{0.147607,0.511733,0.557049}%
\pgfsetstrokecolor{currentstroke}%
\pgfsetdash{}{0pt}%
\pgfpathmoveto{\pgfqpoint{4.140312in}{4.874931in}}%
\pgfpathlineto{\pgfqpoint{4.196221in}{4.834423in}}%
\pgfpathlineto{\pgfqpoint{4.277193in}{4.723467in}}%
\pgfpathclose%
\pgfusepath{fill}%
\end{pgfscope}%
\begin{pgfscope}%
\pgfpathrectangle{\pgfqpoint{0.539299in}{0.078740in}}{\pgfqpoint{7.842520in}{7.842520in}}%
\pgfusepath{clip}%
\pgfsetbuttcap%
\pgfsetroundjoin%
\definecolor{currentfill}{rgb}{0.283072,0.130895,0.449241}%
\pgfsetfillcolor{currentfill}%
\pgfsetlinewidth{0.000000pt}%
\definecolor{currentstroke}{rgb}{0.146180,0.515413,0.556823}%
\pgfsetstrokecolor{currentstroke}%
\pgfsetdash{}{0pt}%
\pgfpathmoveto{\pgfqpoint{6.703683in}{2.968722in}}%
\pgfpathlineto{\pgfqpoint{6.631052in}{2.917912in}}%
\pgfpathlineto{\pgfqpoint{6.770368in}{2.826882in}}%
\pgfpathclose%
\pgfusepath{fill}%
\end{pgfscope}%
\begin{pgfscope}%
\pgfpathrectangle{\pgfqpoint{0.539299in}{0.078740in}}{\pgfqpoint{7.842520in}{7.842520in}}%
\pgfusepath{clip}%
\pgfsetbuttcap%
\pgfsetroundjoin%
\definecolor{currentfill}{rgb}{0.223925,0.334994,0.548053}%
\pgfsetfillcolor{currentfill}%
\pgfsetlinewidth{0.000000pt}%
\definecolor{currentstroke}{rgb}{0.144759,0.519093,0.556572}%
\pgfsetstrokecolor{currentstroke}%
\pgfsetdash{}{0pt}%
\pgfpathmoveto{\pgfqpoint{5.235775in}{3.632407in}}%
\pgfpathlineto{\pgfqpoint{5.159216in}{3.667393in}}%
\pgfpathlineto{\pgfqpoint{5.373172in}{3.514648in}}%
\pgfpathclose%
\pgfusepath{fill}%
\end{pgfscope}%
\begin{pgfscope}%
\pgfpathrectangle{\pgfqpoint{0.539299in}{0.078740in}}{\pgfqpoint{7.842520in}{7.842520in}}%
\pgfusepath{clip}%
\pgfsetbuttcap%
\pgfsetroundjoin%
\definecolor{currentfill}{rgb}{0.262138,0.242286,0.520837}%
\pgfsetfillcolor{currentfill}%
\pgfsetlinewidth{0.000000pt}%
\definecolor{currentstroke}{rgb}{0.143343,0.522773,0.556295}%
\pgfsetstrokecolor{currentstroke}%
\pgfsetdash{}{0pt}%
\pgfpathmoveto{\pgfqpoint{5.648894in}{3.310389in}}%
\pgfpathlineto{\pgfqpoint{5.787278in}{3.220081in}}%
\pgfpathlineto{\pgfqpoint{5.862282in}{3.253769in}}%
\pgfpathclose%
\pgfusepath{fill}%
\end{pgfscope}%
\begin{pgfscope}%
\pgfpathrectangle{\pgfqpoint{0.539299in}{0.078740in}}{\pgfqpoint{7.842520in}{7.842520in}}%
\pgfusepath{clip}%
\pgfsetbuttcap%
\pgfsetroundjoin%
\definecolor{currentfill}{rgb}{0.282656,0.100196,0.422160}%
\pgfsetfillcolor{currentfill}%
\pgfsetlinewidth{0.000000pt}%
\definecolor{currentstroke}{rgb}{0.141935,0.526453,0.555991}%
\pgfsetstrokecolor{currentstroke}%
\pgfsetdash{}{0pt}%
\pgfpathmoveto{\pgfqpoint{6.842661in}{2.878816in}}%
\pgfpathlineto{\pgfqpoint{6.909736in}{2.729276in}}%
\pgfpathlineto{\pgfqpoint{6.981719in}{2.783278in}}%
\pgfpathclose%
\pgfusepath{fill}%
\end{pgfscope}%
\begin{pgfscope}%
\pgfpathrectangle{\pgfqpoint{0.539299in}{0.078740in}}{\pgfqpoint{7.842520in}{7.842520in}}%
\pgfusepath{clip}%
\pgfsetbuttcap%
\pgfsetroundjoin%
\definecolor{currentfill}{rgb}{0.170948,0.694384,0.493803}%
\pgfsetfillcolor{currentfill}%
\pgfsetlinewidth{0.000000pt}%
\definecolor{currentstroke}{rgb}{0.140536,0.530132,0.555659}%
\pgfsetstrokecolor{currentstroke}%
\pgfsetdash{}{0pt}%
\pgfpathmoveto{\pgfqpoint{3.081416in}{5.017996in}}%
\pgfpathlineto{\pgfqpoint{3.211903in}{5.214787in}}%
\pgfpathlineto{\pgfqpoint{3.296607in}{5.181085in}}%
\pgfpathclose%
\pgfusepath{fill}%
\end{pgfscope}%
\begin{pgfscope}%
\pgfpathrectangle{\pgfqpoint{0.539299in}{0.078740in}}{\pgfqpoint{7.842520in}{7.842520in}}%
\pgfusepath{clip}%
\pgfsetbuttcap%
\pgfsetroundjoin%
\definecolor{currentfill}{rgb}{0.277134,0.185228,0.489898}%
\pgfsetfillcolor{currentfill}%
\pgfsetlinewidth{0.000000pt}%
\definecolor{currentstroke}{rgb}{0.139147,0.533812,0.555298}%
\pgfsetstrokecolor{currentstroke}%
\pgfsetdash{}{0pt}%
\pgfpathmoveto{\pgfqpoint{6.352735in}{3.080591in}}%
\pgfpathlineto{\pgfqpoint{6.139708in}{3.103444in}}%
\pgfpathlineto{\pgfqpoint{6.278864in}{3.026523in}}%
\pgfpathclose%
\pgfusepath{fill}%
\end{pgfscope}%
\begin{pgfscope}%
\pgfpathrectangle{\pgfqpoint{0.539299in}{0.078740in}}{\pgfqpoint{7.842520in}{7.842520in}}%
\pgfusepath{clip}%
\pgfsetbuttcap%
\pgfsetroundjoin%
\definecolor{currentfill}{rgb}{0.208030,0.718701,0.472873}%
\pgfsetfillcolor{currentfill}%
\pgfsetlinewidth{0.000000pt}%
\definecolor{currentstroke}{rgb}{0.137770,0.537492,0.554906}%
\pgfsetstrokecolor{currentstroke}%
\pgfsetdash{}{0pt}%
\pgfpathmoveto{\pgfqpoint{3.648554in}{5.237108in}}%
\pgfpathlineto{\pgfqpoint{3.513793in}{5.222466in}}%
\pgfpathlineto{\pgfqpoint{3.429695in}{5.281129in}}%
\pgfpathclose%
\pgfusepath{fill}%
\end{pgfscope}%
\begin{pgfscope}%
\pgfpathrectangle{\pgfqpoint{0.539299in}{0.078740in}}{\pgfqpoint{7.842520in}{7.842520in}}%
\pgfusepath{clip}%
\pgfsetbuttcap%
\pgfsetroundjoin%
\definecolor{currentfill}{rgb}{0.280255,0.165693,0.476498}%
\pgfsetfillcolor{currentfill}%
\pgfsetlinewidth{0.000000pt}%
\definecolor{currentstroke}{rgb}{0.136408,0.541173,0.554483}%
\pgfsetstrokecolor{currentstroke}%
\pgfsetdash{}{0pt}%
\pgfpathmoveto{\pgfqpoint{6.491826in}{3.002343in}}%
\pgfpathlineto{\pgfqpoint{6.352735in}{3.080591in}}%
\pgfpathlineto{\pgfqpoint{6.418242in}{2.945854in}}%
\pgfpathclose%
\pgfusepath{fill}%
\end{pgfscope}%
\begin{pgfscope}%
\pgfpathrectangle{\pgfqpoint{0.539299in}{0.078740in}}{\pgfqpoint{7.842520in}{7.842520in}}%
\pgfusepath{clip}%
\pgfsetbuttcap%
\pgfsetroundjoin%
\definecolor{currentfill}{rgb}{0.170948,0.694384,0.493803}%
\pgfsetfillcolor{currentfill}%
\pgfsetlinewidth{0.000000pt}%
\definecolor{currentstroke}{rgb}{0.135066,0.544853,0.554029}%
\pgfsetstrokecolor{currentstroke}%
\pgfsetdash{}{0pt}%
\pgfpathmoveto{\pgfqpoint{3.867334in}{5.102013in}}%
\pgfpathlineto{\pgfqpoint{3.784535in}{5.195377in}}%
\pgfpathlineto{\pgfqpoint{3.921341in}{5.107578in}}%
\pgfpathclose%
\pgfusepath{fill}%
\end{pgfscope}%
\begin{pgfscope}%
\pgfpathrectangle{\pgfqpoint{0.539299in}{0.078740in}}{\pgfqpoint{7.842520in}{7.842520in}}%
\pgfusepath{clip}%
\pgfsetbuttcap%
\pgfsetroundjoin%
\definecolor{currentfill}{rgb}{0.281446,0.084320,0.407414}%
\pgfsetfillcolor{currentfill}%
\pgfsetlinewidth{0.000000pt}%
\definecolor{currentstroke}{rgb}{0.133743,0.548535,0.553541}%
\pgfsetstrokecolor{currentstroke}%
\pgfsetdash{}{0pt}%
\pgfpathmoveto{\pgfqpoint{7.120863in}{2.683197in}}%
\pgfpathlineto{\pgfqpoint{6.981719in}{2.783278in}}%
\pgfpathlineto{\pgfqpoint{6.909736in}{2.729276in}}%
\pgfpathclose%
\pgfusepath{fill}%
\end{pgfscope}%
\begin{pgfscope}%
\pgfpathrectangle{\pgfqpoint{0.539299in}{0.078740in}}{\pgfqpoint{7.842520in}{7.842520in}}%
\pgfusepath{clip}%
\pgfsetbuttcap%
\pgfsetroundjoin%
\definecolor{currentfill}{rgb}{0.241237,0.296485,0.539709}%
\pgfsetfillcolor{currentfill}%
\pgfsetlinewidth{0.000000pt}%
\definecolor{currentstroke}{rgb}{0.132444,0.552216,0.553018}%
\pgfsetstrokecolor{currentstroke}%
\pgfsetdash{}{0pt}%
\pgfpathmoveto{\pgfqpoint{5.373172in}{3.514648in}}%
\pgfpathlineto{\pgfqpoint{5.434988in}{3.409236in}}%
\pgfpathlineto{\pgfqpoint{5.510865in}{3.407873in}}%
\pgfpathclose%
\pgfusepath{fill}%
\end{pgfscope}%
\begin{pgfscope}%
\pgfpathrectangle{\pgfqpoint{0.539299in}{0.078740in}}{\pgfqpoint{7.842520in}{7.842520in}}%
\pgfusepath{clip}%
\pgfsetbuttcap%
\pgfsetroundjoin%
\definecolor{currentfill}{rgb}{0.283091,0.110553,0.431554}%
\pgfsetfillcolor{currentfill}%
\pgfsetlinewidth{0.000000pt}%
\definecolor{currentstroke}{rgb}{0.131172,0.555899,0.552459}%
\pgfsetstrokecolor{currentstroke}%
\pgfsetdash{}{0pt}%
\pgfpathmoveto{\pgfqpoint{6.770368in}{2.826882in}}%
\pgfpathlineto{\pgfqpoint{6.909736in}{2.729276in}}%
\pgfpathlineto{\pgfqpoint{6.842661in}{2.878816in}}%
\pgfpathclose%
\pgfusepath{fill}%
\end{pgfscope}%
\begin{pgfscope}%
\pgfpathrectangle{\pgfqpoint{0.539299in}{0.078740in}}{\pgfqpoint{7.842520in}{7.842520in}}%
\pgfusepath{clip}%
\pgfsetbuttcap%
\pgfsetroundjoin%
\definecolor{currentfill}{rgb}{0.278826,0.175490,0.483397}%
\pgfsetfillcolor{currentfill}%
\pgfsetlinewidth{0.000000pt}%
\definecolor{currentstroke}{rgb}{0.129933,0.559582,0.551864}%
\pgfsetstrokecolor{currentstroke}%
\pgfsetdash{}{0pt}%
\pgfpathmoveto{\pgfqpoint{6.418242in}{2.945854in}}%
\pgfpathlineto{\pgfqpoint{6.352735in}{3.080591in}}%
\pgfpathlineto{\pgfqpoint{6.278864in}{3.026523in}}%
\pgfpathclose%
\pgfusepath{fill}%
\end{pgfscope}%
\begin{pgfscope}%
\pgfpathrectangle{\pgfqpoint{0.539299in}{0.078740in}}{\pgfqpoint{7.842520in}{7.842520in}}%
\pgfusepath{clip}%
\pgfsetbuttcap%
\pgfsetroundjoin%
\definecolor{currentfill}{rgb}{0.171176,0.452530,0.557965}%
\pgfsetfillcolor{currentfill}%
\pgfsetlinewidth{0.000000pt}%
\definecolor{currentstroke}{rgb}{0.128729,0.563265,0.551229}%
\pgfsetstrokecolor{currentstroke}%
\pgfsetdash{}{0pt}%
\pgfpathmoveto{\pgfqpoint{4.884097in}{3.971926in}}%
\pgfpathlineto{\pgfqpoint{4.824755in}{4.057714in}}%
\pgfpathlineto{\pgfqpoint{4.746612in}{4.139665in}}%
\pgfpathclose%
\pgfusepath{fill}%
\end{pgfscope}%
\begin{pgfscope}%
\pgfpathrectangle{\pgfqpoint{0.539299in}{0.078740in}}{\pgfqpoint{7.842520in}{7.842520in}}%
\pgfusepath{clip}%
\pgfsetbuttcap%
\pgfsetroundjoin%
\definecolor{currentfill}{rgb}{0.281412,0.155834,0.469201}%
\pgfsetfillcolor{currentfill}%
\pgfsetlinewidth{0.000000pt}%
\definecolor{currentstroke}{rgb}{0.127568,0.566949,0.550556}%
\pgfsetstrokecolor{currentstroke}%
\pgfsetdash{}{0pt}%
\pgfpathmoveto{\pgfqpoint{6.491826in}{3.002343in}}%
\pgfpathlineto{\pgfqpoint{6.418242in}{2.945854in}}%
\pgfpathlineto{\pgfqpoint{6.631052in}{2.917912in}}%
\pgfpathclose%
\pgfusepath{fill}%
\end{pgfscope}%
\begin{pgfscope}%
\pgfpathrectangle{\pgfqpoint{0.539299in}{0.078740in}}{\pgfqpoint{7.842520in}{7.842520in}}%
\pgfusepath{clip}%
\pgfsetbuttcap%
\pgfsetroundjoin%
\definecolor{currentfill}{rgb}{0.192357,0.403199,0.555836}%
\pgfsetfillcolor{currentfill}%
\pgfsetlinewidth{0.000000pt}%
\definecolor{currentstroke}{rgb}{0.126453,0.570633,0.549841}%
\pgfsetstrokecolor{currentstroke}%
\pgfsetdash{}{0pt}%
\pgfpathmoveto{\pgfqpoint{5.021609in}{3.813991in}}%
\pgfpathlineto{\pgfqpoint{5.098620in}{3.762206in}}%
\pgfpathlineto{\pgfqpoint{4.884097in}{3.971926in}}%
\pgfpathclose%
\pgfusepath{fill}%
\end{pgfscope}%
\begin{pgfscope}%
\pgfpathrectangle{\pgfqpoint{0.539299in}{0.078740in}}{\pgfqpoint{7.842520in}{7.842520in}}%
\pgfusepath{clip}%
\pgfsetbuttcap%
\pgfsetroundjoin%
\definecolor{currentfill}{rgb}{0.160665,0.478540,0.558115}%
\pgfsetfillcolor{currentfill}%
\pgfsetlinewidth{0.000000pt}%
\definecolor{currentstroke}{rgb}{0.125394,0.574318,0.549086}%
\pgfsetstrokecolor{currentstroke}%
\pgfsetdash{}{0pt}%
\pgfpathmoveto{\pgfqpoint{4.746612in}{4.139665in}}%
\pgfpathlineto{\pgfqpoint{4.824755in}{4.057714in}}%
\pgfpathlineto{\pgfqpoint{4.609098in}{4.314481in}}%
\pgfpathclose%
\pgfusepath{fill}%
\end{pgfscope}%
\begin{pgfscope}%
\pgfpathrectangle{\pgfqpoint{0.539299in}{0.078740in}}{\pgfqpoint{7.842520in}{7.842520in}}%
\pgfusepath{clip}%
\pgfsetbuttcap%
\pgfsetroundjoin%
\definecolor{currentfill}{rgb}{0.203063,0.379716,0.553925}%
\pgfsetfillcolor{currentfill}%
\pgfsetlinewidth{0.000000pt}%
\definecolor{currentstroke}{rgb}{0.124395,0.578002,0.548287}%
\pgfsetstrokecolor{currentstroke}%
\pgfsetdash{}{0pt}%
\pgfpathmoveto{\pgfqpoint{5.159216in}{3.667393in}}%
\pgfpathlineto{\pgfqpoint{5.098620in}{3.762206in}}%
\pgfpathlineto{\pgfqpoint{5.021609in}{3.813991in}}%
\pgfpathclose%
\pgfusepath{fill}%
\end{pgfscope}%
\begin{pgfscope}%
\pgfpathrectangle{\pgfqpoint{0.539299in}{0.078740in}}{\pgfqpoint{7.842520in}{7.842520in}}%
\pgfusepath{clip}%
\pgfsetbuttcap%
\pgfsetroundjoin%
\definecolor{currentfill}{rgb}{0.267968,0.223549,0.512008}%
\pgfsetfillcolor{currentfill}%
\pgfsetlinewidth{0.000000pt}%
\definecolor{currentstroke}{rgb}{0.123463,0.581687,0.547445}%
\pgfsetstrokecolor{currentstroke}%
\pgfsetdash{}{0pt}%
\pgfpathmoveto{\pgfqpoint{5.787278in}{3.220081in}}%
\pgfpathlineto{\pgfqpoint{5.926016in}{3.134613in}}%
\pgfpathlineto{\pgfqpoint{6.000832in}{3.178458in}}%
\pgfpathclose%
\pgfusepath{fill}%
\end{pgfscope}%
\begin{pgfscope}%
\pgfpathrectangle{\pgfqpoint{0.539299in}{0.078740in}}{\pgfqpoint{7.842520in}{7.842520in}}%
\pgfusepath{clip}%
\pgfsetbuttcap%
\pgfsetroundjoin%
\definecolor{currentfill}{rgb}{0.273006,0.204520,0.501721}%
\pgfsetfillcolor{currentfill}%
\pgfsetlinewidth{0.000000pt}%
\definecolor{currentstroke}{rgb}{0.122606,0.585371,0.546557}%
\pgfsetstrokecolor{currentstroke}%
\pgfsetdash{}{0pt}%
\pgfpathmoveto{\pgfqpoint{6.065090in}{3.051608in}}%
\pgfpathlineto{\pgfqpoint{6.139708in}{3.103444in}}%
\pgfpathlineto{\pgfqpoint{6.000832in}{3.178458in}}%
\pgfpathclose%
\pgfusepath{fill}%
\end{pgfscope}%
\begin{pgfscope}%
\pgfpathrectangle{\pgfqpoint{0.539299in}{0.078740in}}{\pgfqpoint{7.842520in}{7.842520in}}%
\pgfusepath{clip}%
\pgfsetbuttcap%
\pgfsetroundjoin%
\definecolor{currentfill}{rgb}{0.162016,0.687316,0.499129}%
\pgfsetfillcolor{currentfill}%
\pgfsetlinewidth{0.000000pt}%
\definecolor{currentstroke}{rgb}{0.121831,0.589055,0.545623}%
\pgfsetstrokecolor{currentstroke}%
\pgfsetdash{}{0pt}%
\pgfpathmoveto{\pgfqpoint{2.996415in}{5.025431in}}%
\pgfpathlineto{\pgfqpoint{3.211903in}{5.214787in}}%
\pgfpathlineto{\pgfqpoint{3.081416in}{5.017996in}}%
\pgfpathclose%
\pgfusepath{fill}%
\end{pgfscope}%
\begin{pgfscope}%
\pgfpathrectangle{\pgfqpoint{0.539299in}{0.078740in}}{\pgfqpoint{7.842520in}{7.842520in}}%
\pgfusepath{clip}%
\pgfsetbuttcap%
\pgfsetroundjoin%
\definecolor{currentfill}{rgb}{0.246811,0.283237,0.535941}%
\pgfsetfillcolor{currentfill}%
\pgfsetlinewidth{0.000000pt}%
\definecolor{currentstroke}{rgb}{0.121148,0.592739,0.544641}%
\pgfsetstrokecolor{currentstroke}%
\pgfsetdash{}{0pt}%
\pgfpathmoveto{\pgfqpoint{5.648894in}{3.310389in}}%
\pgfpathlineto{\pgfqpoint{5.510865in}{3.407873in}}%
\pgfpathlineto{\pgfqpoint{5.434988in}{3.409236in}}%
\pgfpathclose%
\pgfusepath{fill}%
\end{pgfscope}%
\begin{pgfscope}%
\pgfpathrectangle{\pgfqpoint{0.539299in}{0.078740in}}{\pgfqpoint{7.842520in}{7.842520in}}%
\pgfusepath{clip}%
\pgfsetbuttcap%
\pgfsetroundjoin%
\definecolor{currentfill}{rgb}{0.140536,0.530132,0.555659}%
\pgfsetfillcolor{currentfill}%
\pgfsetlinewidth{0.000000pt}%
\definecolor{currentstroke}{rgb}{0.120565,0.596422,0.543611}%
\pgfsetstrokecolor{currentstroke}%
\pgfsetdash{}{0pt}%
\pgfpathmoveto{\pgfqpoint{4.609098in}{4.314481in}}%
\pgfpathlineto{\pgfqpoint{4.551037in}{4.389232in}}%
\pgfpathlineto{\pgfqpoint{4.471520in}{4.492348in}}%
\pgfpathclose%
\pgfusepath{fill}%
\end{pgfscope}%
\begin{pgfscope}%
\pgfpathrectangle{\pgfqpoint{0.539299in}{0.078740in}}{\pgfqpoint{7.842520in}{7.842520in}}%
\pgfusepath{clip}%
\pgfsetbuttcap%
\pgfsetroundjoin%
\definecolor{currentfill}{rgb}{0.214298,0.355619,0.551184}%
\pgfsetfillcolor{currentfill}%
\pgfsetlinewidth{0.000000pt}%
\definecolor{currentstroke}{rgb}{0.120092,0.600104,0.542530}%
\pgfsetstrokecolor{currentstroke}%
\pgfsetdash{}{0pt}%
\pgfpathmoveto{\pgfqpoint{2.234531in}{4.033237in}}%
\pgfpathlineto{\pgfqpoint{2.119879in}{3.494959in}}%
\pgfpathlineto{\pgfqpoint{2.034011in}{3.419410in}}%
\pgfpathclose%
\pgfusepath{fill}%
\end{pgfscope}%
\begin{pgfscope}%
\pgfpathrectangle{\pgfqpoint{0.539299in}{0.078740in}}{\pgfqpoint{7.842520in}{7.842520in}}%
\pgfusepath{clip}%
\pgfsetbuttcap%
\pgfsetroundjoin%
\definecolor{currentfill}{rgb}{0.140210,0.665859,0.513427}%
\pgfsetfillcolor{currentfill}%
\pgfsetlinewidth{0.000000pt}%
\definecolor{currentstroke}{rgb}{0.119738,0.603785,0.541400}%
\pgfsetstrokecolor{currentstroke}%
\pgfsetdash{}{0pt}%
\pgfpathmoveto{\pgfqpoint{4.058651in}{4.983988in}}%
\pgfpathlineto{\pgfqpoint{4.140312in}{4.874931in}}%
\pgfpathlineto{\pgfqpoint{3.921341in}{5.107578in}}%
\pgfpathclose%
\pgfusepath{fill}%
\end{pgfscope}%
\begin{pgfscope}%
\pgfpathrectangle{\pgfqpoint{0.539299in}{0.078740in}}{\pgfqpoint{7.842520in}{7.842520in}}%
\pgfusepath{clip}%
\pgfsetbuttcap%
\pgfsetroundjoin%
\definecolor{currentfill}{rgb}{0.277941,0.056324,0.381191}%
\pgfsetfillcolor{currentfill}%
\pgfsetlinewidth{0.000000pt}%
\definecolor{currentstroke}{rgb}{0.119512,0.607464,0.540218}%
\pgfsetstrokecolor{currentstroke}%
\pgfsetdash{}{0pt}%
\pgfpathmoveto{\pgfqpoint{7.049132in}{2.625447in}}%
\pgfpathlineto{\pgfqpoint{7.260119in}{2.579903in}}%
\pgfpathlineto{\pgfqpoint{7.120863in}{2.683197in}}%
\pgfpathclose%
\pgfusepath{fill}%
\end{pgfscope}%
\begin{pgfscope}%
\pgfpathrectangle{\pgfqpoint{0.539299in}{0.078740in}}{\pgfqpoint{7.842520in}{7.842520in}}%
\pgfusepath{clip}%
\pgfsetbuttcap%
\pgfsetroundjoin%
\definecolor{currentfill}{rgb}{0.129933,0.559582,0.551864}%
\pgfsetfillcolor{currentfill}%
\pgfsetlinewidth{0.000000pt}%
\definecolor{currentstroke}{rgb}{0.119423,0.611141,0.538982}%
\pgfsetstrokecolor{currentstroke}%
\pgfsetdash{}{0pt}%
\pgfpathmoveto{\pgfqpoint{4.471520in}{4.492348in}}%
\pgfpathlineto{\pgfqpoint{4.551037in}{4.389232in}}%
\pgfpathlineto{\pgfqpoint{4.333879in}{4.667896in}}%
\pgfpathclose%
\pgfusepath{fill}%
\end{pgfscope}%
\begin{pgfscope}%
\pgfpathrectangle{\pgfqpoint{0.539299in}{0.078740in}}{\pgfqpoint{7.842520in}{7.842520in}}%
\pgfusepath{clip}%
\pgfsetbuttcap%
\pgfsetroundjoin%
\definecolor{currentfill}{rgb}{0.208030,0.718701,0.472873}%
\pgfsetfillcolor{currentfill}%
\pgfsetlinewidth{0.000000pt}%
\definecolor{currentstroke}{rgb}{0.119483,0.614817,0.537692}%
\pgfsetstrokecolor{currentstroke}%
\pgfsetdash{}{0pt}%
\pgfpathmoveto{\pgfqpoint{3.429695in}{5.281129in}}%
\pgfpathlineto{\pgfqpoint{3.296607in}{5.181085in}}%
\pgfpathlineto{\pgfqpoint{3.211903in}{5.214787in}}%
\pgfpathclose%
\pgfusepath{fill}%
\end{pgfscope}%
\begin{pgfscope}%
\pgfpathrectangle{\pgfqpoint{0.539299in}{0.078740in}}{\pgfqpoint{7.842520in}{7.842520in}}%
\pgfusepath{clip}%
\pgfsetbuttcap%
\pgfsetroundjoin%
\definecolor{currentfill}{rgb}{0.223925,0.334994,0.548053}%
\pgfsetfillcolor{currentfill}%
\pgfsetlinewidth{0.000000pt}%
\definecolor{currentstroke}{rgb}{0.119699,0.618490,0.536347}%
\pgfsetstrokecolor{currentstroke}%
\pgfsetdash{}{0pt}%
\pgfpathmoveto{\pgfqpoint{5.373172in}{3.514648in}}%
\pgfpathlineto{\pgfqpoint{5.159216in}{3.667393in}}%
\pgfpathlineto{\pgfqpoint{5.296988in}{3.532612in}}%
\pgfpathclose%
\pgfusepath{fill}%
\end{pgfscope}%
\begin{pgfscope}%
\pgfpathrectangle{\pgfqpoint{0.539299in}{0.078740in}}{\pgfqpoint{7.842520in}{7.842520in}}%
\pgfusepath{clip}%
\pgfsetbuttcap%
\pgfsetroundjoin%
\definecolor{currentfill}{rgb}{0.128087,0.647749,0.523491}%
\pgfsetfillcolor{currentfill}%
\pgfsetlinewidth{0.000000pt}%
\definecolor{currentstroke}{rgb}{0.120081,0.622161,0.534946}%
\pgfsetstrokecolor{currentstroke}%
\pgfsetdash{}{0pt}%
\pgfpathmoveto{\pgfqpoint{4.140312in}{4.874931in}}%
\pgfpathlineto{\pgfqpoint{4.058651in}{4.983988in}}%
\pgfpathlineto{\pgfqpoint{4.196221in}{4.834423in}}%
\pgfpathclose%
\pgfusepath{fill}%
\end{pgfscope}%
\begin{pgfscope}%
\pgfpathrectangle{\pgfqpoint{0.539299in}{0.078740in}}{\pgfqpoint{7.842520in}{7.842520in}}%
\pgfusepath{clip}%
\pgfsetbuttcap%
\pgfsetroundjoin%
\definecolor{currentfill}{rgb}{0.119423,0.611141,0.538982}%
\pgfsetfillcolor{currentfill}%
\pgfsetlinewidth{0.000000pt}%
\definecolor{currentstroke}{rgb}{0.120638,0.625828,0.533488}%
\pgfsetstrokecolor{currentstroke}%
\pgfsetdash{}{0pt}%
\pgfpathmoveto{\pgfqpoint{4.277193in}{4.723467in}}%
\pgfpathlineto{\pgfqpoint{4.196221in}{4.834423in}}%
\pgfpathlineto{\pgfqpoint{4.333879in}{4.667896in}}%
\pgfpathclose%
\pgfusepath{fill}%
\end{pgfscope}%
\begin{pgfscope}%
\pgfpathrectangle{\pgfqpoint{0.539299in}{0.078740in}}{\pgfqpoint{7.842520in}{7.842520in}}%
\pgfusepath{clip}%
\pgfsetbuttcap%
\pgfsetroundjoin%
\definecolor{currentfill}{rgb}{0.185556,0.418570,0.556753}%
\pgfsetfillcolor{currentfill}%
\pgfsetlinewidth{0.000000pt}%
\definecolor{currentstroke}{rgb}{0.121380,0.629492,0.531973}%
\pgfsetstrokecolor{currentstroke}%
\pgfsetdash{}{0pt}%
\pgfpathmoveto{\pgfqpoint{2.234531in}{4.033237in}}%
\pgfpathlineto{\pgfqpoint{2.320596in}{4.102334in}}%
\pgfpathlineto{\pgfqpoint{2.205440in}{3.565187in}}%
\pgfpathclose%
\pgfusepath{fill}%
\end{pgfscope}%
\begin{pgfscope}%
\pgfpathrectangle{\pgfqpoint{0.539299in}{0.078740in}}{\pgfqpoint{7.842520in}{7.842520in}}%
\pgfusepath{clip}%
\pgfsetbuttcap%
\pgfsetroundjoin%
\definecolor{currentfill}{rgb}{0.270595,0.214069,0.507052}%
\pgfsetfillcolor{currentfill}%
\pgfsetlinewidth{0.000000pt}%
\definecolor{currentstroke}{rgb}{0.122312,0.633153,0.530398}%
\pgfsetstrokecolor{currentstroke}%
\pgfsetdash{}{0pt}%
\pgfpathmoveto{\pgfqpoint{6.000832in}{3.178458in}}%
\pgfpathlineto{\pgfqpoint{5.926016in}{3.134613in}}%
\pgfpathlineto{\pgfqpoint{6.065090in}{3.051608in}}%
\pgfpathclose%
\pgfusepath{fill}%
\end{pgfscope}%
\begin{pgfscope}%
\pgfpathrectangle{\pgfqpoint{0.539299in}{0.078740in}}{\pgfqpoint{7.842520in}{7.842520in}}%
\pgfusepath{clip}%
\pgfsetbuttcap%
\pgfsetroundjoin%
\definecolor{currentfill}{rgb}{0.233603,0.313828,0.543914}%
\pgfsetfillcolor{currentfill}%
\pgfsetlinewidth{0.000000pt}%
\definecolor{currentstroke}{rgb}{0.123444,0.636809,0.528763}%
\pgfsetstrokecolor{currentstroke}%
\pgfsetdash{}{0pt}%
\pgfpathmoveto{\pgfqpoint{5.296988in}{3.532612in}}%
\pgfpathlineto{\pgfqpoint{5.434988in}{3.409236in}}%
\pgfpathlineto{\pgfqpoint{5.373172in}{3.514648in}}%
\pgfpathclose%
\pgfusepath{fill}%
\end{pgfscope}%
\begin{pgfscope}%
\pgfpathrectangle{\pgfqpoint{0.539299in}{0.078740in}}{\pgfqpoint{7.842520in}{7.842520in}}%
\pgfusepath{clip}%
\pgfsetbuttcap%
\pgfsetroundjoin%
\definecolor{currentfill}{rgb}{0.280894,0.078907,0.402329}%
\pgfsetfillcolor{currentfill}%
\pgfsetlinewidth{0.000000pt}%
\definecolor{currentstroke}{rgb}{0.124780,0.640461,0.527068}%
\pgfsetstrokecolor{currentstroke}%
\pgfsetdash{}{0pt}%
\pgfpathmoveto{\pgfqpoint{6.909736in}{2.729276in}}%
\pgfpathlineto{\pgfqpoint{7.049132in}{2.625447in}}%
\pgfpathlineto{\pgfqpoint{7.120863in}{2.683197in}}%
\pgfpathclose%
\pgfusepath{fill}%
\end{pgfscope}%
\begin{pgfscope}%
\pgfpathrectangle{\pgfqpoint{0.539299in}{0.078740in}}{\pgfqpoint{7.842520in}{7.842520in}}%
\pgfusepath{clip}%
\pgfsetbuttcap%
\pgfsetroundjoin%
\definecolor{currentfill}{rgb}{0.276194,0.190074,0.493001}%
\pgfsetfillcolor{currentfill}%
\pgfsetlinewidth{0.000000pt}%
\definecolor{currentstroke}{rgb}{0.126326,0.644107,0.525311}%
\pgfsetstrokecolor{currentstroke}%
\pgfsetdash{}{0pt}%
\pgfpathmoveto{\pgfqpoint{6.278864in}{3.026523in}}%
\pgfpathlineto{\pgfqpoint{6.139708in}{3.103444in}}%
\pgfpathlineto{\pgfqpoint{6.204464in}{2.968792in}}%
\pgfpathclose%
\pgfusepath{fill}%
\end{pgfscope}%
\begin{pgfscope}%
\pgfpathrectangle{\pgfqpoint{0.539299in}{0.078740in}}{\pgfqpoint{7.842520in}{7.842520in}}%
\pgfusepath{clip}%
\pgfsetbuttcap%
\pgfsetroundjoin%
\definecolor{currentfill}{rgb}{0.283072,0.130895,0.449241}%
\pgfsetfillcolor{currentfill}%
\pgfsetlinewidth{0.000000pt}%
\definecolor{currentstroke}{rgb}{0.128087,0.647749,0.523491}%
\pgfsetstrokecolor{currentstroke}%
\pgfsetdash{}{0pt}%
\pgfpathmoveto{\pgfqpoint{6.770368in}{2.826882in}}%
\pgfpathlineto{\pgfqpoint{6.631052in}{2.917912in}}%
\pgfpathlineto{\pgfqpoint{6.697430in}{2.767965in}}%
\pgfpathclose%
\pgfusepath{fill}%
\end{pgfscope}%
\begin{pgfscope}%
\pgfpathrectangle{\pgfqpoint{0.539299in}{0.078740in}}{\pgfqpoint{7.842520in}{7.842520in}}%
\pgfusepath{clip}%
\pgfsetbuttcap%
\pgfsetroundjoin%
\definecolor{currentfill}{rgb}{0.128729,0.563265,0.551229}%
\pgfsetfillcolor{currentfill}%
\pgfsetlinewidth{0.000000pt}%
\definecolor{currentstroke}{rgb}{0.130067,0.651384,0.521608}%
\pgfsetstrokecolor{currentstroke}%
\pgfsetdash{}{0pt}%
\pgfpathmoveto{\pgfqpoint{2.699021in}{4.682479in}}%
\pgfpathlineto{\pgfqpoint{2.491969in}{4.215044in}}%
\pgfpathlineto{\pgfqpoint{2.613215in}{4.649492in}}%
\pgfpathclose%
\pgfusepath{fill}%
\end{pgfscope}%
\begin{pgfscope}%
\pgfpathrectangle{\pgfqpoint{0.539299in}{0.078740in}}{\pgfqpoint{7.842520in}{7.842520in}}%
\pgfusepath{clip}%
\pgfsetbuttcap%
\pgfsetroundjoin%
\definecolor{currentfill}{rgb}{0.258965,0.251537,0.524736}%
\pgfsetfillcolor{currentfill}%
\pgfsetlinewidth{0.000000pt}%
\definecolor{currentstroke}{rgb}{0.132268,0.655014,0.519661}%
\pgfsetstrokecolor{currentstroke}%
\pgfsetdash{}{0pt}%
\pgfpathmoveto{\pgfqpoint{5.711861in}{3.191718in}}%
\pgfpathlineto{\pgfqpoint{5.787278in}{3.220081in}}%
\pgfpathlineto{\pgfqpoint{5.648894in}{3.310389in}}%
\pgfpathclose%
\pgfusepath{fill}%
\end{pgfscope}%
\begin{pgfscope}%
\pgfpathrectangle{\pgfqpoint{0.539299in}{0.078740in}}{\pgfqpoint{7.842520in}{7.842520in}}%
\pgfusepath{clip}%
\pgfsetbuttcap%
\pgfsetroundjoin%
\definecolor{currentfill}{rgb}{0.281887,0.150881,0.465405}%
\pgfsetfillcolor{currentfill}%
\pgfsetlinewidth{0.000000pt}%
\definecolor{currentstroke}{rgb}{0.134692,0.658636,0.517649}%
\pgfsetstrokecolor{currentstroke}%
\pgfsetdash{}{0pt}%
\pgfpathmoveto{\pgfqpoint{6.631052in}{2.917912in}}%
\pgfpathlineto{\pgfqpoint{6.418242in}{2.945854in}}%
\pgfpathlineto{\pgfqpoint{6.557783in}{2.860007in}}%
\pgfpathclose%
\pgfusepath{fill}%
\end{pgfscope}%
\begin{pgfscope}%
\pgfpathrectangle{\pgfqpoint{0.539299in}{0.078740in}}{\pgfqpoint{7.842520in}{7.842520in}}%
\pgfusepath{clip}%
\pgfsetbuttcap%
\pgfsetroundjoin%
\definecolor{currentfill}{rgb}{0.132268,0.655014,0.519661}%
\pgfsetfillcolor{currentfill}%
\pgfsetlinewidth{0.000000pt}%
\definecolor{currentstroke}{rgb}{0.137339,0.662252,0.515571}%
\pgfsetstrokecolor{currentstroke}%
\pgfsetdash{}{0pt}%
\pgfpathmoveto{\pgfqpoint{2.910930in}{5.025499in}}%
\pgfpathlineto{\pgfqpoint{2.996415in}{5.025431in}}%
\pgfpathlineto{\pgfqpoint{2.784458in}{4.707293in}}%
\pgfpathclose%
\pgfusepath{fill}%
\end{pgfscope}%
\begin{pgfscope}%
\pgfpathrectangle{\pgfqpoint{0.539299in}{0.078740in}}{\pgfqpoint{7.842520in}{7.842520in}}%
\pgfusepath{clip}%
\pgfsetbuttcap%
\pgfsetroundjoin%
\definecolor{currentfill}{rgb}{0.239374,0.735588,0.455688}%
\pgfsetfillcolor{currentfill}%
\pgfsetlinewidth{0.000000pt}%
\definecolor{currentstroke}{rgb}{0.140210,0.665859,0.513427}%
\pgfsetstrokecolor{currentstroke}%
\pgfsetdash{}{0pt}%
\pgfpathmoveto{\pgfqpoint{3.429695in}{5.281129in}}%
\pgfpathlineto{\pgfqpoint{3.564639in}{5.312794in}}%
\pgfpathlineto{\pgfqpoint{3.648554in}{5.237108in}}%
\pgfpathclose%
\pgfusepath{fill}%
\end{pgfscope}%
\begin{pgfscope}%
\pgfpathrectangle{\pgfqpoint{0.539299in}{0.078740in}}{\pgfqpoint{7.842520in}{7.842520in}}%
\pgfusepath{clip}%
\pgfsetbuttcap%
\pgfsetroundjoin%
\definecolor{currentfill}{rgb}{0.276022,0.044167,0.370164}%
\pgfsetfillcolor{currentfill}%
\pgfsetlinewidth{0.000000pt}%
\definecolor{currentstroke}{rgb}{0.143303,0.669459,0.511215}%
\pgfsetstrokecolor{currentstroke}%
\pgfsetdash{}{0pt}%
\pgfpathmoveto{\pgfqpoint{7.049132in}{2.625447in}}%
\pgfpathlineto{\pgfqpoint{7.188539in}{2.515946in}}%
\pgfpathlineto{\pgfqpoint{7.260119in}{2.579903in}}%
\pgfpathclose%
\pgfusepath{fill}%
\end{pgfscope}%
\begin{pgfscope}%
\pgfpathrectangle{\pgfqpoint{0.539299in}{0.078740in}}{\pgfqpoint{7.842520in}{7.842520in}}%
\pgfusepath{clip}%
\pgfsetbuttcap%
\pgfsetroundjoin%
\definecolor{currentfill}{rgb}{0.246811,0.283237,0.535941}%
\pgfsetfillcolor{currentfill}%
\pgfsetlinewidth{0.000000pt}%
\definecolor{currentstroke}{rgb}{0.146616,0.673050,0.508936}%
\pgfsetstrokecolor{currentstroke}%
\pgfsetdash{}{0pt}%
\pgfpathmoveto{\pgfqpoint{5.434988in}{3.409236in}}%
\pgfpathlineto{\pgfqpoint{5.573268in}{3.296150in}}%
\pgfpathlineto{\pgfqpoint{5.648894in}{3.310389in}}%
\pgfpathclose%
\pgfusepath{fill}%
\end{pgfscope}%
\begin{pgfscope}%
\pgfpathrectangle{\pgfqpoint{0.539299in}{0.078740in}}{\pgfqpoint{7.842520in}{7.842520in}}%
\pgfusepath{clip}%
\pgfsetbuttcap%
\pgfsetroundjoin%
\definecolor{currentfill}{rgb}{0.143343,0.522773,0.556295}%
\pgfsetfillcolor{currentfill}%
\pgfsetlinewidth{0.000000pt}%
\definecolor{currentstroke}{rgb}{0.150148,0.676631,0.506589}%
\pgfsetstrokecolor{currentstroke}%
\pgfsetdash{}{0pt}%
\pgfpathmoveto{\pgfqpoint{2.527103in}{4.606886in}}%
\pgfpathlineto{\pgfqpoint{2.491969in}{4.215044in}}%
\pgfpathlineto{\pgfqpoint{2.406429in}{4.162525in}}%
\pgfpathclose%
\pgfusepath{fill}%
\end{pgfscope}%
\begin{pgfscope}%
\pgfpathrectangle{\pgfqpoint{0.539299in}{0.078740in}}{\pgfqpoint{7.842520in}{7.842520in}}%
\pgfusepath{clip}%
\pgfsetbuttcap%
\pgfsetroundjoin%
\definecolor{currentfill}{rgb}{0.275191,0.194905,0.496005}%
\pgfsetfillcolor{currentfill}%
\pgfsetlinewidth{0.000000pt}%
\definecolor{currentstroke}{rgb}{0.153894,0.680203,0.504172}%
\pgfsetstrokecolor{currentstroke}%
\pgfsetdash{}{0pt}%
\pgfpathmoveto{\pgfqpoint{6.204464in}{2.968792in}}%
\pgfpathlineto{\pgfqpoint{6.139708in}{3.103444in}}%
\pgfpathlineto{\pgfqpoint{6.065090in}{3.051608in}}%
\pgfpathclose%
\pgfusepath{fill}%
\end{pgfscope}%
\begin{pgfscope}%
\pgfpathrectangle{\pgfqpoint{0.539299in}{0.078740in}}{\pgfqpoint{7.842520in}{7.842520in}}%
\pgfusepath{clip}%
\pgfsetbuttcap%
\pgfsetroundjoin%
\definecolor{currentfill}{rgb}{0.283197,0.115680,0.436115}%
\pgfsetfillcolor{currentfill}%
\pgfsetlinewidth{0.000000pt}%
\definecolor{currentstroke}{rgb}{0.157851,0.683765,0.501686}%
\pgfsetstrokecolor{currentstroke}%
\pgfsetdash{}{0pt}%
\pgfpathmoveto{\pgfqpoint{6.770368in}{2.826882in}}%
\pgfpathlineto{\pgfqpoint{6.697430in}{2.767965in}}%
\pgfpathlineto{\pgfqpoint{6.909736in}{2.729276in}}%
\pgfpathclose%
\pgfusepath{fill}%
\end{pgfscope}%
\begin{pgfscope}%
\pgfpathrectangle{\pgfqpoint{0.539299in}{0.078740in}}{\pgfqpoint{7.842520in}{7.842520in}}%
\pgfusepath{clip}%
\pgfsetbuttcap%
\pgfsetroundjoin%
\definecolor{currentfill}{rgb}{0.278012,0.180367,0.486697}%
\pgfsetfillcolor{currentfill}%
\pgfsetlinewidth{0.000000pt}%
\definecolor{currentstroke}{rgb}{0.162016,0.687316,0.499129}%
\pgfsetstrokecolor{currentstroke}%
\pgfsetdash{}{0pt}%
\pgfpathmoveto{\pgfqpoint{6.204464in}{2.968792in}}%
\pgfpathlineto{\pgfqpoint{6.418242in}{2.945854in}}%
\pgfpathlineto{\pgfqpoint{6.278864in}{3.026523in}}%
\pgfpathclose%
\pgfusepath{fill}%
\end{pgfscope}%
\begin{pgfscope}%
\pgfpathrectangle{\pgfqpoint{0.539299in}{0.078740in}}{\pgfqpoint{7.842520in}{7.842520in}}%
\pgfusepath{clip}%
\pgfsetbuttcap%
\pgfsetroundjoin%
\definecolor{currentfill}{rgb}{0.263663,0.237631,0.518762}%
\pgfsetfillcolor{currentfill}%
\pgfsetlinewidth{0.000000pt}%
\definecolor{currentstroke}{rgb}{0.166383,0.690856,0.496502}%
\pgfsetstrokecolor{currentstroke}%
\pgfsetdash{}{0pt}%
\pgfpathmoveto{\pgfqpoint{5.711861in}{3.191718in}}%
\pgfpathlineto{\pgfqpoint{5.926016in}{3.134613in}}%
\pgfpathlineto{\pgfqpoint{5.787278in}{3.220081in}}%
\pgfpathclose%
\pgfusepath{fill}%
\end{pgfscope}%
\begin{pgfscope}%
\pgfpathrectangle{\pgfqpoint{0.539299in}{0.078740in}}{\pgfqpoint{7.842520in}{7.842520in}}%
\pgfusepath{clip}%
\pgfsetbuttcap%
\pgfsetroundjoin%
\definecolor{currentfill}{rgb}{0.282623,0.140926,0.457517}%
\pgfsetfillcolor{currentfill}%
\pgfsetlinewidth{0.000000pt}%
\definecolor{currentstroke}{rgb}{0.170948,0.694384,0.493803}%
\pgfsetstrokecolor{currentstroke}%
\pgfsetdash{}{0pt}%
\pgfpathmoveto{\pgfqpoint{6.697430in}{2.767965in}}%
\pgfpathlineto{\pgfqpoint{6.631052in}{2.917912in}}%
\pgfpathlineto{\pgfqpoint{6.557783in}{2.860007in}}%
\pgfpathclose%
\pgfusepath{fill}%
\end{pgfscope}%
\begin{pgfscope}%
\pgfpathrectangle{\pgfqpoint{0.539299in}{0.078740in}}{\pgfqpoint{7.842520in}{7.842520in}}%
\pgfusepath{clip}%
\pgfsetbuttcap%
\pgfsetroundjoin%
\definecolor{currentfill}{rgb}{0.226397,0.728888,0.462789}%
\pgfsetfillcolor{currentfill}%
\pgfsetlinewidth{0.000000pt}%
\definecolor{currentstroke}{rgb}{0.175707,0.697900,0.491033}%
\pgfsetstrokecolor{currentstroke}%
\pgfsetdash{}{0pt}%
\pgfpathmoveto{\pgfqpoint{3.700962in}{5.285351in}}%
\pgfpathlineto{\pgfqpoint{3.784535in}{5.195377in}}%
\pgfpathlineto{\pgfqpoint{3.648554in}{5.237108in}}%
\pgfpathclose%
\pgfusepath{fill}%
\end{pgfscope}%
\begin{pgfscope}%
\pgfpathrectangle{\pgfqpoint{0.539299in}{0.078740in}}{\pgfqpoint{7.842520in}{7.842520in}}%
\pgfusepath{clip}%
\pgfsetbuttcap%
\pgfsetroundjoin%
\definecolor{currentfill}{rgb}{0.253935,0.265254,0.529983}%
\pgfsetfillcolor{currentfill}%
\pgfsetlinewidth{0.000000pt}%
\definecolor{currentstroke}{rgb}{0.180653,0.701402,0.488189}%
\pgfsetstrokecolor{currentstroke}%
\pgfsetdash{}{0pt}%
\pgfpathmoveto{\pgfqpoint{5.648894in}{3.310389in}}%
\pgfpathlineto{\pgfqpoint{5.573268in}{3.296150in}}%
\pgfpathlineto{\pgfqpoint{5.711861in}{3.191718in}}%
\pgfpathclose%
\pgfusepath{fill}%
\end{pgfscope}%
\begin{pgfscope}%
\pgfpathrectangle{\pgfqpoint{0.539299in}{0.078740in}}{\pgfqpoint{7.842520in}{7.842520in}}%
\pgfusepath{clip}%
\pgfsetbuttcap%
\pgfsetroundjoin%
\definecolor{currentfill}{rgb}{0.220124,0.725509,0.466226}%
\pgfsetfillcolor{currentfill}%
\pgfsetlinewidth{0.000000pt}%
\definecolor{currentstroke}{rgb}{0.185783,0.704891,0.485273}%
\pgfsetstrokecolor{currentstroke}%
\pgfsetdash{}{0pt}%
\pgfpathmoveto{\pgfqpoint{3.700962in}{5.285351in}}%
\pgfpathlineto{\pgfqpoint{3.921341in}{5.107578in}}%
\pgfpathlineto{\pgfqpoint{3.784535in}{5.195377in}}%
\pgfpathclose%
\pgfusepath{fill}%
\end{pgfscope}%
\begin{pgfscope}%
\pgfpathrectangle{\pgfqpoint{0.539299in}{0.078740in}}{\pgfqpoint{7.842520in}{7.842520in}}%
\pgfusepath{clip}%
\pgfsetbuttcap%
\pgfsetroundjoin%
\definecolor{currentfill}{rgb}{0.123444,0.636809,0.528763}%
\pgfsetfillcolor{currentfill}%
\pgfsetlinewidth{0.000000pt}%
\definecolor{currentstroke}{rgb}{0.191090,0.708366,0.482284}%
\pgfsetstrokecolor{currentstroke}%
\pgfsetdash{}{0pt}%
\pgfpathmoveto{\pgfqpoint{2.825015in}{5.016747in}}%
\pgfpathlineto{\pgfqpoint{2.784458in}{4.707293in}}%
\pgfpathlineto{\pgfqpoint{2.699021in}{4.682479in}}%
\pgfpathclose%
\pgfusepath{fill}%
\end{pgfscope}%
\begin{pgfscope}%
\pgfpathrectangle{\pgfqpoint{0.539299in}{0.078740in}}{\pgfqpoint{7.842520in}{7.842520in}}%
\pgfusepath{clip}%
\pgfsetbuttcap%
\pgfsetroundjoin%
\definecolor{currentfill}{rgb}{0.280894,0.078907,0.402329}%
\pgfsetfillcolor{currentfill}%
\pgfsetlinewidth{0.000000pt}%
\definecolor{currentstroke}{rgb}{0.196571,0.711827,0.479221}%
\pgfsetstrokecolor{currentstroke}%
\pgfsetdash{}{0pt}%
\pgfpathmoveto{\pgfqpoint{7.049132in}{2.625447in}}%
\pgfpathlineto{\pgfqpoint{6.909736in}{2.729276in}}%
\pgfpathlineto{\pgfqpoint{6.976816in}{2.562893in}}%
\pgfpathclose%
\pgfusepath{fill}%
\end{pgfscope}%
\begin{pgfscope}%
\pgfpathrectangle{\pgfqpoint{0.539299in}{0.078740in}}{\pgfqpoint{7.842520in}{7.842520in}}%
\pgfusepath{clip}%
\pgfsetbuttcap%
\pgfsetroundjoin%
\definecolor{currentfill}{rgb}{0.280868,0.160771,0.472899}%
\pgfsetfillcolor{currentfill}%
\pgfsetlinewidth{0.000000pt}%
\definecolor{currentstroke}{rgb}{0.202219,0.715272,0.476084}%
\pgfsetstrokecolor{currentstroke}%
\pgfsetdash{}{0pt}%
\pgfpathmoveto{\pgfqpoint{6.557783in}{2.860007in}}%
\pgfpathlineto{\pgfqpoint{6.418242in}{2.945854in}}%
\pgfpathlineto{\pgfqpoint{6.344091in}{2.884105in}}%
\pgfpathclose%
\pgfusepath{fill}%
\end{pgfscope}%
\begin{pgfscope}%
\pgfpathrectangle{\pgfqpoint{0.539299in}{0.078740in}}{\pgfqpoint{7.842520in}{7.842520in}}%
\pgfusepath{clip}%
\pgfsetbuttcap%
\pgfsetroundjoin%
\definecolor{currentfill}{rgb}{0.216210,0.351535,0.550627}%
\pgfsetfillcolor{currentfill}%
\pgfsetlinewidth{0.000000pt}%
\definecolor{currentstroke}{rgb}{0.208030,0.718701,0.472873}%
\pgfsetstrokecolor{currentstroke}%
\pgfsetdash{}{0pt}%
\pgfpathmoveto{\pgfqpoint{2.034011in}{3.419410in}}%
\pgfpathlineto{\pgfqpoint{1.947878in}{3.337740in}}%
\pgfpathlineto{\pgfqpoint{2.148306in}{3.953837in}}%
\pgfpathclose%
\pgfusepath{fill}%
\end{pgfscope}%
\begin{pgfscope}%
\pgfpathrectangle{\pgfqpoint{0.539299in}{0.078740in}}{\pgfqpoint{7.842520in}{7.842520in}}%
\pgfusepath{clip}%
\pgfsetbuttcap%
\pgfsetroundjoin%
\definecolor{currentfill}{rgb}{0.252899,0.742211,0.448284}%
\pgfsetfillcolor{currentfill}%
\pgfsetlinewidth{0.000000pt}%
\definecolor{currentstroke}{rgb}{0.214000,0.722114,0.469588}%
\pgfsetstrokecolor{currentstroke}%
\pgfsetdash{}{0pt}%
\pgfpathmoveto{\pgfqpoint{3.648554in}{5.237108in}}%
\pgfpathlineto{\pgfqpoint{3.564639in}{5.312794in}}%
\pgfpathlineto{\pgfqpoint{3.700962in}{5.285351in}}%
\pgfpathclose%
\pgfusepath{fill}%
\end{pgfscope}%
\begin{pgfscope}%
\pgfpathrectangle{\pgfqpoint{0.539299in}{0.078740in}}{\pgfqpoint{7.842520in}{7.842520in}}%
\pgfusepath{clip}%
\pgfsetbuttcap%
\pgfsetroundjoin%
\definecolor{currentfill}{rgb}{0.190631,0.407061,0.556089}%
\pgfsetfillcolor{currentfill}%
\pgfsetlinewidth{0.000000pt}%
\definecolor{currentstroke}{rgb}{0.220124,0.725509,0.466226}%
\pgfsetstrokecolor{currentstroke}%
\pgfsetdash{}{0pt}%
\pgfpathmoveto{\pgfqpoint{5.021609in}{3.813991in}}%
\pgfpathlineto{\pgfqpoint{4.943959in}{3.876474in}}%
\pgfpathlineto{\pgfqpoint{5.159216in}{3.667393in}}%
\pgfpathclose%
\pgfusepath{fill}%
\end{pgfscope}%
\begin{pgfscope}%
\pgfpathrectangle{\pgfqpoint{0.539299in}{0.078740in}}{\pgfqpoint{7.842520in}{7.842520in}}%
\pgfusepath{clip}%
\pgfsetbuttcap%
\pgfsetroundjoin%
\definecolor{currentfill}{rgb}{0.175841,0.441290,0.557685}%
\pgfsetfillcolor{currentfill}%
\pgfsetlinewidth{0.000000pt}%
\definecolor{currentstroke}{rgb}{0.226397,0.728888,0.462789}%
\pgfsetstrokecolor{currentstroke}%
\pgfsetdash{}{0pt}%
\pgfpathmoveto{\pgfqpoint{5.021609in}{3.813991in}}%
\pgfpathlineto{\pgfqpoint{4.884097in}{3.971926in}}%
\pgfpathlineto{\pgfqpoint{4.805857in}{4.049409in}}%
\pgfpathclose%
\pgfusepath{fill}%
\end{pgfscope}%
\begin{pgfscope}%
\pgfpathrectangle{\pgfqpoint{0.539299in}{0.078740in}}{\pgfqpoint{7.842520in}{7.842520in}}%
\pgfusepath{clip}%
\pgfsetbuttcap%
\pgfsetroundjoin%
\definecolor{currentfill}{rgb}{0.252899,0.742211,0.448284}%
\pgfsetfillcolor{currentfill}%
\pgfsetlinewidth{0.000000pt}%
\definecolor{currentstroke}{rgb}{0.232815,0.732247,0.459277}%
\pgfsetstrokecolor{currentstroke}%
\pgfsetdash{}{0pt}%
\pgfpathmoveto{\pgfqpoint{3.429695in}{5.281129in}}%
\pgfpathlineto{\pgfqpoint{3.211903in}{5.214787in}}%
\pgfpathlineto{\pgfqpoint{3.344948in}{5.333814in}}%
\pgfpathclose%
\pgfusepath{fill}%
\end{pgfscope}%
\begin{pgfscope}%
\pgfpathrectangle{\pgfqpoint{0.539299in}{0.078740in}}{\pgfqpoint{7.842520in}{7.842520in}}%
\pgfusepath{clip}%
\pgfsetbuttcap%
\pgfsetroundjoin%
\definecolor{currentfill}{rgb}{0.165117,0.467423,0.558141}%
\pgfsetfillcolor{currentfill}%
\pgfsetlinewidth{0.000000pt}%
\definecolor{currentstroke}{rgb}{0.239374,0.735588,0.455688}%
\pgfsetstrokecolor{currentstroke}%
\pgfsetdash{}{0pt}%
\pgfpathmoveto{\pgfqpoint{4.805857in}{4.049409in}}%
\pgfpathlineto{\pgfqpoint{4.884097in}{3.971926in}}%
\pgfpathlineto{\pgfqpoint{4.746612in}{4.139665in}}%
\pgfpathclose%
\pgfusepath{fill}%
\end{pgfscope}%
\begin{pgfscope}%
\pgfpathrectangle{\pgfqpoint{0.539299in}{0.078740in}}{\pgfqpoint{7.842520in}{7.842520in}}%
\pgfusepath{clip}%
\pgfsetbuttcap%
\pgfsetroundjoin%
\definecolor{currentfill}{rgb}{0.208030,0.718701,0.472873}%
\pgfsetfillcolor{currentfill}%
\pgfsetlinewidth{0.000000pt}%
\definecolor{currentstroke}{rgb}{0.246070,0.738910,0.452024}%
\pgfsetstrokecolor{currentstroke}%
\pgfsetdash{}{0pt}%
\pgfpathmoveto{\pgfqpoint{3.126635in}{5.241459in}}%
\pgfpathlineto{\pgfqpoint{3.211903in}{5.214787in}}%
\pgfpathlineto{\pgfqpoint{2.996415in}{5.025431in}}%
\pgfpathclose%
\pgfusepath{fill}%
\end{pgfscope}%
\begin{pgfscope}%
\pgfpathrectangle{\pgfqpoint{0.539299in}{0.078740in}}{\pgfqpoint{7.842520in}{7.842520in}}%
\pgfusepath{clip}%
\pgfsetbuttcap%
\pgfsetroundjoin%
\definecolor{currentfill}{rgb}{0.283091,0.110553,0.431554}%
\pgfsetfillcolor{currentfill}%
\pgfsetlinewidth{0.000000pt}%
\definecolor{currentstroke}{rgb}{0.252899,0.742211,0.448284}%
\pgfsetstrokecolor{currentstroke}%
\pgfsetdash{}{0pt}%
\pgfpathmoveto{\pgfqpoint{6.909736in}{2.729276in}}%
\pgfpathlineto{\pgfqpoint{6.697430in}{2.767965in}}%
\pgfpathlineto{\pgfqpoint{6.837125in}{2.669064in}}%
\pgfpathclose%
\pgfusepath{fill}%
\end{pgfscope}%
\begin{pgfscope}%
\pgfpathrectangle{\pgfqpoint{0.539299in}{0.078740in}}{\pgfqpoint{7.842520in}{7.842520in}}%
\pgfusepath{clip}%
\pgfsetbuttcap%
\pgfsetroundjoin%
\definecolor{currentfill}{rgb}{0.270595,0.214069,0.507052}%
\pgfsetfillcolor{currentfill}%
\pgfsetlinewidth{0.000000pt}%
\definecolor{currentstroke}{rgb}{0.259857,0.745492,0.444467}%
\pgfsetstrokecolor{currentstroke}%
\pgfsetdash{}{0pt}%
\pgfpathmoveto{\pgfqpoint{6.065090in}{3.051608in}}%
\pgfpathlineto{\pgfqpoint{5.926016in}{3.134613in}}%
\pgfpathlineto{\pgfqpoint{5.990036in}{3.000736in}}%
\pgfpathclose%
\pgfusepath{fill}%
\end{pgfscope}%
\begin{pgfscope}%
\pgfpathrectangle{\pgfqpoint{0.539299in}{0.078740in}}{\pgfqpoint{7.842520in}{7.842520in}}%
\pgfusepath{clip}%
\pgfsetbuttcap%
\pgfsetroundjoin%
\definecolor{currentfill}{rgb}{0.278826,0.175490,0.483397}%
\pgfsetfillcolor{currentfill}%
\pgfsetlinewidth{0.000000pt}%
\definecolor{currentstroke}{rgb}{0.266941,0.748751,0.440573}%
\pgfsetstrokecolor{currentstroke}%
\pgfsetdash{}{0pt}%
\pgfpathmoveto{\pgfqpoint{6.204464in}{2.968792in}}%
\pgfpathlineto{\pgfqpoint{6.344091in}{2.884105in}}%
\pgfpathlineto{\pgfqpoint{6.418242in}{2.945854in}}%
\pgfpathclose%
\pgfusepath{fill}%
\end{pgfscope}%
\begin{pgfscope}%
\pgfpathrectangle{\pgfqpoint{0.539299in}{0.078740in}}{\pgfqpoint{7.842520in}{7.842520in}}%
\pgfusepath{clip}%
\pgfsetbuttcap%
\pgfsetroundjoin%
\definecolor{currentfill}{rgb}{0.212395,0.359683,0.551710}%
\pgfsetfillcolor{currentfill}%
\pgfsetlinewidth{0.000000pt}%
\definecolor{currentstroke}{rgb}{0.274149,0.751988,0.436601}%
\pgfsetstrokecolor{currentstroke}%
\pgfsetdash{}{0pt}%
\pgfpathmoveto{\pgfqpoint{5.159216in}{3.667393in}}%
\pgfpathlineto{\pgfqpoint{5.220286in}{3.561627in}}%
\pgfpathlineto{\pgfqpoint{5.296988in}{3.532612in}}%
\pgfpathclose%
\pgfusepath{fill}%
\end{pgfscope}%
\begin{pgfscope}%
\pgfpathrectangle{\pgfqpoint{0.539299in}{0.078740in}}{\pgfqpoint{7.842520in}{7.842520in}}%
\pgfusepath{clip}%
\pgfsetbuttcap%
\pgfsetroundjoin%
\definecolor{currentfill}{rgb}{0.149039,0.508051,0.557250}%
\pgfsetfillcolor{currentfill}%
\pgfsetlinewidth{0.000000pt}%
\definecolor{currentstroke}{rgb}{0.281477,0.755203,0.432552}%
\pgfsetstrokecolor{currentstroke}%
\pgfsetdash{}{0pt}%
\pgfpathmoveto{\pgfqpoint{4.746612in}{4.139665in}}%
\pgfpathlineto{\pgfqpoint{4.609098in}{4.314481in}}%
\pgfpathlineto{\pgfqpoint{4.667716in}{4.230148in}}%
\pgfpathclose%
\pgfusepath{fill}%
\end{pgfscope}%
\begin{pgfscope}%
\pgfpathrectangle{\pgfqpoint{0.539299in}{0.078740in}}{\pgfqpoint{7.842520in}{7.842520in}}%
\pgfusepath{clip}%
\pgfsetbuttcap%
\pgfsetroundjoin%
\definecolor{currentfill}{rgb}{0.263663,0.237631,0.518762}%
\pgfsetfillcolor{currentfill}%
\pgfsetlinewidth{0.000000pt}%
\definecolor{currentstroke}{rgb}{0.288921,0.758394,0.428426}%
\pgfsetstrokecolor{currentstroke}%
\pgfsetdash{}{0pt}%
\pgfpathmoveto{\pgfqpoint{5.711861in}{3.191718in}}%
\pgfpathlineto{\pgfqpoint{5.850784in}{3.093964in}}%
\pgfpathlineto{\pgfqpoint{5.926016in}{3.134613in}}%
\pgfpathclose%
\pgfusepath{fill}%
\end{pgfscope}%
\begin{pgfscope}%
\pgfpathrectangle{\pgfqpoint{0.539299in}{0.078740in}}{\pgfqpoint{7.842520in}{7.842520in}}%
\pgfusepath{clip}%
\pgfsetbuttcap%
\pgfsetroundjoin%
\definecolor{currentfill}{rgb}{0.278791,0.062145,0.386592}%
\pgfsetfillcolor{currentfill}%
\pgfsetlinewidth{0.000000pt}%
\definecolor{currentstroke}{rgb}{0.296479,0.761561,0.424223}%
\pgfsetstrokecolor{currentstroke}%
\pgfsetdash{}{0pt}%
\pgfpathmoveto{\pgfqpoint{7.188539in}{2.515946in}}%
\pgfpathlineto{\pgfqpoint{7.049132in}{2.625447in}}%
\pgfpathlineto{\pgfqpoint{6.976816in}{2.562893in}}%
\pgfpathclose%
\pgfusepath{fill}%
\end{pgfscope}%
\begin{pgfscope}%
\pgfpathrectangle{\pgfqpoint{0.539299in}{0.078740in}}{\pgfqpoint{7.842520in}{7.842520in}}%
\pgfusepath{clip}%
\pgfsetbuttcap%
\pgfsetroundjoin%
\definecolor{currentfill}{rgb}{0.126453,0.570633,0.549841}%
\pgfsetfillcolor{currentfill}%
\pgfsetlinewidth{0.000000pt}%
\definecolor{currentstroke}{rgb}{0.304148,0.764704,0.419943}%
\pgfsetstrokecolor{currentstroke}%
\pgfsetdash{}{0pt}%
\pgfpathmoveto{\pgfqpoint{2.613215in}{4.649492in}}%
\pgfpathlineto{\pgfqpoint{2.491969in}{4.215044in}}%
\pgfpathlineto{\pgfqpoint{2.527103in}{4.606886in}}%
\pgfpathclose%
\pgfusepath{fill}%
\end{pgfscope}%
\begin{pgfscope}%
\pgfpathrectangle{\pgfqpoint{0.539299in}{0.078740in}}{\pgfqpoint{7.842520in}{7.842520in}}%
\pgfusepath{clip}%
\pgfsetbuttcap%
\pgfsetroundjoin%
\definecolor{currentfill}{rgb}{0.227802,0.326594,0.546532}%
\pgfsetfillcolor{currentfill}%
\pgfsetlinewidth{0.000000pt}%
\definecolor{currentstroke}{rgb}{0.311925,0.767822,0.415586}%
\pgfsetstrokecolor{currentstroke}%
\pgfsetdash{}{0pt}%
\pgfpathmoveto{\pgfqpoint{5.358646in}{3.421072in}}%
\pgfpathlineto{\pgfqpoint{5.434988in}{3.409236in}}%
\pgfpathlineto{\pgfqpoint{5.296988in}{3.532612in}}%
\pgfpathclose%
\pgfusepath{fill}%
\end{pgfscope}%
\begin{pgfscope}%
\pgfpathrectangle{\pgfqpoint{0.539299in}{0.078740in}}{\pgfqpoint{7.842520in}{7.842520in}}%
\pgfusepath{clip}%
\pgfsetbuttcap%
\pgfsetroundjoin%
\definecolor{currentfill}{rgb}{0.191090,0.708366,0.482284}%
\pgfsetfillcolor{currentfill}%
\pgfsetlinewidth{0.000000pt}%
\definecolor{currentstroke}{rgb}{0.319809,0.770914,0.411152}%
\pgfsetstrokecolor{currentstroke}%
\pgfsetdash{}{0pt}%
\pgfpathmoveto{\pgfqpoint{4.058651in}{4.983988in}}%
\pgfpathlineto{\pgfqpoint{3.921341in}{5.107578in}}%
\pgfpathlineto{\pgfqpoint{3.838249in}{5.208623in}}%
\pgfpathclose%
\pgfusepath{fill}%
\end{pgfscope}%
\begin{pgfscope}%
\pgfpathrectangle{\pgfqpoint{0.539299in}{0.078740in}}{\pgfqpoint{7.842520in}{7.842520in}}%
\pgfusepath{clip}%
\pgfsetbuttcap%
\pgfsetroundjoin%
\definecolor{currentfill}{rgb}{0.281924,0.089666,0.412415}%
\pgfsetfillcolor{currentfill}%
\pgfsetlinewidth{0.000000pt}%
\definecolor{currentstroke}{rgb}{0.327796,0.773980,0.406640}%
\pgfsetstrokecolor{currentstroke}%
\pgfsetdash{}{0pt}%
\pgfpathmoveto{\pgfqpoint{6.976816in}{2.562893in}}%
\pgfpathlineto{\pgfqpoint{6.909736in}{2.729276in}}%
\pgfpathlineto{\pgfqpoint{6.837125in}{2.669064in}}%
\pgfpathclose%
\pgfusepath{fill}%
\end{pgfscope}%
\begin{pgfscope}%
\pgfpathrectangle{\pgfqpoint{0.539299in}{0.078740in}}{\pgfqpoint{7.842520in}{7.842520in}}%
\pgfusepath{clip}%
\pgfsetbuttcap%
\pgfsetroundjoin%
\definecolor{currentfill}{rgb}{0.127568,0.566949,0.550556}%
\pgfsetfillcolor{currentfill}%
\pgfsetlinewidth{0.000000pt}%
\definecolor{currentstroke}{rgb}{0.335885,0.777018,0.402049}%
\pgfsetstrokecolor{currentstroke}%
\pgfsetdash{}{0pt}%
\pgfpathmoveto{\pgfqpoint{4.391174in}{4.600585in}}%
\pgfpathlineto{\pgfqpoint{4.609098in}{4.314481in}}%
\pgfpathlineto{\pgfqpoint{4.471520in}{4.492348in}}%
\pgfpathclose%
\pgfusepath{fill}%
\end{pgfscope}%
\begin{pgfscope}%
\pgfpathrectangle{\pgfqpoint{0.539299in}{0.078740in}}{\pgfqpoint{7.842520in}{7.842520in}}%
\pgfusepath{clip}%
\pgfsetbuttcap%
\pgfsetroundjoin%
\definecolor{currentfill}{rgb}{0.121148,0.592739,0.544641}%
\pgfsetfillcolor{currentfill}%
\pgfsetlinewidth{0.000000pt}%
\definecolor{currentstroke}{rgb}{0.344074,0.780029,0.397381}%
\pgfsetstrokecolor{currentstroke}%
\pgfsetdash{}{0pt}%
\pgfpathmoveto{\pgfqpoint{4.471520in}{4.492348in}}%
\pgfpathlineto{\pgfqpoint{4.333879in}{4.667896in}}%
\pgfpathlineto{\pgfqpoint{4.391174in}{4.600585in}}%
\pgfpathclose%
\pgfusepath{fill}%
\end{pgfscope}%
\begin{pgfscope}%
\pgfpathrectangle{\pgfqpoint{0.539299in}{0.078740in}}{\pgfqpoint{7.842520in}{7.842520in}}%
\pgfusepath{clip}%
\pgfsetbuttcap%
\pgfsetroundjoin%
\definecolor{currentfill}{rgb}{0.143303,0.669459,0.511215}%
\pgfsetfillcolor{currentfill}%
\pgfsetlinewidth{0.000000pt}%
\definecolor{currentstroke}{rgb}{0.352360,0.783011,0.392636}%
\pgfsetstrokecolor{currentstroke}%
\pgfsetdash{}{0pt}%
\pgfpathmoveto{\pgfqpoint{2.910930in}{5.025499in}}%
\pgfpathlineto{\pgfqpoint{2.784458in}{4.707293in}}%
\pgfpathlineto{\pgfqpoint{2.825015in}{5.016747in}}%
\pgfpathclose%
\pgfusepath{fill}%
\end{pgfscope}%
\begin{pgfscope}%
\pgfpathrectangle{\pgfqpoint{0.539299in}{0.078740in}}{\pgfqpoint{7.842520in}{7.842520in}}%
\pgfusepath{clip}%
\pgfsetbuttcap%
\pgfsetroundjoin%
\definecolor{currentfill}{rgb}{0.273006,0.204520,0.501721}%
\pgfsetfillcolor{currentfill}%
\pgfsetlinewidth{0.000000pt}%
\definecolor{currentstroke}{rgb}{0.360741,0.785964,0.387814}%
\pgfsetstrokecolor{currentstroke}%
\pgfsetdash{}{0pt}%
\pgfpathmoveto{\pgfqpoint{6.065090in}{3.051608in}}%
\pgfpathlineto{\pgfqpoint{5.990036in}{3.000736in}}%
\pgfpathlineto{\pgfqpoint{6.204464in}{2.968792in}}%
\pgfpathclose%
\pgfusepath{fill}%
\end{pgfscope}%
\begin{pgfscope}%
\pgfpathrectangle{\pgfqpoint{0.539299in}{0.078740in}}{\pgfqpoint{7.842520in}{7.842520in}}%
\pgfusepath{clip}%
\pgfsetbuttcap%
\pgfsetroundjoin%
\definecolor{currentfill}{rgb}{0.267968,0.223549,0.512008}%
\pgfsetfillcolor{currentfill}%
\pgfsetlinewidth{0.000000pt}%
\definecolor{currentstroke}{rgb}{0.369214,0.788888,0.382914}%
\pgfsetstrokecolor{currentstroke}%
\pgfsetdash{}{0pt}%
\pgfpathmoveto{\pgfqpoint{5.926016in}{3.134613in}}%
\pgfpathlineto{\pgfqpoint{5.850784in}{3.093964in}}%
\pgfpathlineto{\pgfqpoint{5.990036in}{3.000736in}}%
\pgfpathclose%
\pgfusepath{fill}%
\end{pgfscope}%
\begin{pgfscope}%
\pgfpathrectangle{\pgfqpoint{0.539299in}{0.078740in}}{\pgfqpoint{7.842520in}{7.842520in}}%
\pgfusepath{clip}%
\pgfsetbuttcap%
\pgfsetroundjoin%
\definecolor{currentfill}{rgb}{0.282623,0.140926,0.457517}%
\pgfsetfillcolor{currentfill}%
\pgfsetlinewidth{0.000000pt}%
\definecolor{currentstroke}{rgb}{0.377779,0.791781,0.377939}%
\pgfsetstrokecolor{currentstroke}%
\pgfsetdash{}{0pt}%
\pgfpathmoveto{\pgfqpoint{6.557783in}{2.860007in}}%
\pgfpathlineto{\pgfqpoint{6.483913in}{2.795763in}}%
\pgfpathlineto{\pgfqpoint{6.697430in}{2.767965in}}%
\pgfpathclose%
\pgfusepath{fill}%
\end{pgfscope}%
\begin{pgfscope}%
\pgfpathrectangle{\pgfqpoint{0.539299in}{0.078740in}}{\pgfqpoint{7.842520in}{7.842520in}}%
\pgfusepath{clip}%
\pgfsetbuttcap%
\pgfsetroundjoin%
\definecolor{currentfill}{rgb}{0.122312,0.633153,0.530398}%
\pgfsetfillcolor{currentfill}%
\pgfsetlinewidth{0.000000pt}%
\definecolor{currentstroke}{rgb}{0.386433,0.794644,0.372886}%
\pgfsetstrokecolor{currentstroke}%
\pgfsetdash{}{0pt}%
\pgfpathmoveto{\pgfqpoint{4.252784in}{4.779999in}}%
\pgfpathlineto{\pgfqpoint{4.333879in}{4.667896in}}%
\pgfpathlineto{\pgfqpoint{4.196221in}{4.834423in}}%
\pgfpathclose%
\pgfusepath{fill}%
\end{pgfscope}%
\begin{pgfscope}%
\pgfpathrectangle{\pgfqpoint{0.539299in}{0.078740in}}{\pgfqpoint{7.842520in}{7.842520in}}%
\pgfusepath{clip}%
\pgfsetbuttcap%
\pgfsetroundjoin%
\definecolor{currentfill}{rgb}{0.143303,0.669459,0.511215}%
\pgfsetfillcolor{currentfill}%
\pgfsetlinewidth{0.000000pt}%
\definecolor{currentstroke}{rgb}{0.395174,0.797475,0.367757}%
\pgfsetstrokecolor{currentstroke}%
\pgfsetdash{}{0pt}%
\pgfpathmoveto{\pgfqpoint{4.196221in}{4.834423in}}%
\pgfpathlineto{\pgfqpoint{4.058651in}{4.983988in}}%
\pgfpathlineto{\pgfqpoint{4.114399in}{4.946663in}}%
\pgfpathclose%
\pgfusepath{fill}%
\end{pgfscope}%
\begin{pgfscope}%
\pgfpathrectangle{\pgfqpoint{0.539299in}{0.078740in}}{\pgfqpoint{7.842520in}{7.842520in}}%
\pgfusepath{clip}%
\pgfsetbuttcap%
\pgfsetroundjoin%
\definecolor{currentfill}{rgb}{0.241237,0.296485,0.539709}%
\pgfsetfillcolor{currentfill}%
\pgfsetlinewidth{0.000000pt}%
\definecolor{currentstroke}{rgb}{0.404001,0.800275,0.362552}%
\pgfsetstrokecolor{currentstroke}%
\pgfsetdash{}{0pt}%
\pgfpathmoveto{\pgfqpoint{5.573268in}{3.296150in}}%
\pgfpathlineto{\pgfqpoint{5.434988in}{3.409236in}}%
\pgfpathlineto{\pgfqpoint{5.497218in}{3.291318in}}%
\pgfpathclose%
\pgfusepath{fill}%
\end{pgfscope}%
\begin{pgfscope}%
\pgfpathrectangle{\pgfqpoint{0.539299in}{0.078740in}}{\pgfqpoint{7.842520in}{7.842520in}}%
\pgfusepath{clip}%
\pgfsetbuttcap%
\pgfsetroundjoin%
\definecolor{currentfill}{rgb}{0.281412,0.155834,0.469201}%
\pgfsetfillcolor{currentfill}%
\pgfsetlinewidth{0.000000pt}%
\definecolor{currentstroke}{rgb}{0.412913,0.803041,0.357269}%
\pgfsetstrokecolor{currentstroke}%
\pgfsetdash{}{0pt}%
\pgfpathmoveto{\pgfqpoint{6.344091in}{2.884105in}}%
\pgfpathlineto{\pgfqpoint{6.483913in}{2.795763in}}%
\pgfpathlineto{\pgfqpoint{6.557783in}{2.860007in}}%
\pgfpathclose%
\pgfusepath{fill}%
\end{pgfscope}%
\begin{pgfscope}%
\pgfpathrectangle{\pgfqpoint{0.539299in}{0.078740in}}{\pgfqpoint{7.842520in}{7.842520in}}%
\pgfusepath{clip}%
\pgfsetbuttcap%
\pgfsetroundjoin%
\definecolor{currentfill}{rgb}{0.185556,0.418570,0.556753}%
\pgfsetfillcolor{currentfill}%
\pgfsetlinewidth{0.000000pt}%
\definecolor{currentstroke}{rgb}{0.421908,0.805774,0.351910}%
\pgfsetstrokecolor{currentstroke}%
\pgfsetdash{}{0pt}%
\pgfpathmoveto{\pgfqpoint{2.034011in}{3.419410in}}%
\pgfpathlineto{\pgfqpoint{2.148306in}{3.953837in}}%
\pgfpathlineto{\pgfqpoint{2.234531in}{4.033237in}}%
\pgfpathclose%
\pgfusepath{fill}%
\end{pgfscope}%
\begin{pgfscope}%
\pgfpathrectangle{\pgfqpoint{0.539299in}{0.078740in}}{\pgfqpoint{7.842520in}{7.842520in}}%
\pgfusepath{clip}%
\pgfsetbuttcap%
\pgfsetroundjoin%
\definecolor{currentfill}{rgb}{0.248629,0.278775,0.534556}%
\pgfsetfillcolor{currentfill}%
\pgfsetlinewidth{0.000000pt}%
\definecolor{currentstroke}{rgb}{0.430983,0.808473,0.346476}%
\pgfsetstrokecolor{currentstroke}%
\pgfsetdash{}{0pt}%
\pgfpathmoveto{\pgfqpoint{5.573268in}{3.296150in}}%
\pgfpathlineto{\pgfqpoint{5.497218in}{3.291318in}}%
\pgfpathlineto{\pgfqpoint{5.711861in}{3.191718in}}%
\pgfpathclose%
\pgfusepath{fill}%
\end{pgfscope}%
\begin{pgfscope}%
\pgfpathrectangle{\pgfqpoint{0.539299in}{0.078740in}}{\pgfqpoint{7.842520in}{7.842520in}}%
\pgfusepath{clip}%
\pgfsetbuttcap%
\pgfsetroundjoin%
\definecolor{currentfill}{rgb}{0.277018,0.050344,0.375715}%
\pgfsetfillcolor{currentfill}%
\pgfsetlinewidth{0.000000pt}%
\definecolor{currentstroke}{rgb}{0.440137,0.811138,0.340967}%
\pgfsetstrokecolor{currentstroke}%
\pgfsetdash{}{0pt}%
\pgfpathmoveto{\pgfqpoint{7.116450in}{2.449156in}}%
\pgfpathlineto{\pgfqpoint{7.188539in}{2.515946in}}%
\pgfpathlineto{\pgfqpoint{6.976816in}{2.562893in}}%
\pgfpathclose%
\pgfusepath{fill}%
\end{pgfscope}%
\begin{pgfscope}%
\pgfpathrectangle{\pgfqpoint{0.539299in}{0.078740in}}{\pgfqpoint{7.842520in}{7.842520in}}%
\pgfusepath{clip}%
\pgfsetbuttcap%
\pgfsetroundjoin%
\definecolor{currentfill}{rgb}{0.190631,0.407061,0.556089}%
\pgfsetfillcolor{currentfill}%
\pgfsetlinewidth{0.000000pt}%
\definecolor{currentstroke}{rgb}{0.449368,0.813768,0.335384}%
\pgfsetstrokecolor{currentstroke}%
\pgfsetdash{}{0pt}%
\pgfpathmoveto{\pgfqpoint{4.943959in}{3.876474in}}%
\pgfpathlineto{\pgfqpoint{5.082079in}{3.713504in}}%
\pgfpathlineto{\pgfqpoint{5.159216in}{3.667393in}}%
\pgfpathclose%
\pgfusepath{fill}%
\end{pgfscope}%
\begin{pgfscope}%
\pgfpathrectangle{\pgfqpoint{0.539299in}{0.078740in}}{\pgfqpoint{7.842520in}{7.842520in}}%
\pgfusepath{clip}%
\pgfsetbuttcap%
\pgfsetroundjoin%
\definecolor{currentfill}{rgb}{0.203063,0.379716,0.553925}%
\pgfsetfillcolor{currentfill}%
\pgfsetlinewidth{0.000000pt}%
\definecolor{currentstroke}{rgb}{0.458674,0.816363,0.329727}%
\pgfsetstrokecolor{currentstroke}%
\pgfsetdash{}{0pt}%
\pgfpathmoveto{\pgfqpoint{5.159216in}{3.667393in}}%
\pgfpathlineto{\pgfqpoint{5.082079in}{3.713504in}}%
\pgfpathlineto{\pgfqpoint{5.220286in}{3.561627in}}%
\pgfpathclose%
\pgfusepath{fill}%
\end{pgfscope}%
\begin{pgfscope}%
\pgfpathrectangle{\pgfqpoint{0.539299in}{0.078740in}}{\pgfqpoint{7.842520in}{7.842520in}}%
\pgfusepath{clip}%
\pgfsetbuttcap%
\pgfsetroundjoin%
\definecolor{currentfill}{rgb}{0.174274,0.445044,0.557792}%
\pgfsetfillcolor{currentfill}%
\pgfsetlinewidth{0.000000pt}%
\definecolor{currentstroke}{rgb}{0.468053,0.818921,0.323998}%
\pgfsetstrokecolor{currentstroke}%
\pgfsetdash{}{0pt}%
\pgfpathmoveto{\pgfqpoint{4.805857in}{4.049409in}}%
\pgfpathlineto{\pgfqpoint{4.943959in}{3.876474in}}%
\pgfpathlineto{\pgfqpoint{5.021609in}{3.813991in}}%
\pgfpathclose%
\pgfusepath{fill}%
\end{pgfscope}%
\begin{pgfscope}%
\pgfpathrectangle{\pgfqpoint{0.539299in}{0.078740in}}{\pgfqpoint{7.842520in}{7.842520in}}%
\pgfusepath{clip}%
\pgfsetbuttcap%
\pgfsetroundjoin%
\definecolor{currentfill}{rgb}{0.288921,0.758394,0.428426}%
\pgfsetfillcolor{currentfill}%
\pgfsetlinewidth{0.000000pt}%
\definecolor{currentstroke}{rgb}{0.477504,0.821444,0.318195}%
\pgfsetstrokecolor{currentstroke}%
\pgfsetdash{}{0pt}%
\pgfpathmoveto{\pgfqpoint{3.429695in}{5.281129in}}%
\pgfpathlineto{\pgfqpoint{3.480023in}{5.382736in}}%
\pgfpathlineto{\pgfqpoint{3.564639in}{5.312794in}}%
\pgfpathclose%
\pgfusepath{fill}%
\end{pgfscope}%
\begin{pgfscope}%
\pgfpathrectangle{\pgfqpoint{0.539299in}{0.078740in}}{\pgfqpoint{7.842520in}{7.842520in}}%
\pgfusepath{clip}%
\pgfsetbuttcap%
\pgfsetroundjoin%
\definecolor{currentfill}{rgb}{0.141935,0.526453,0.555991}%
\pgfsetfillcolor{currentfill}%
\pgfsetlinewidth{0.000000pt}%
\definecolor{currentstroke}{rgb}{0.487026,0.823929,0.312321}%
\pgfsetstrokecolor{currentstroke}%
\pgfsetdash{}{0pt}%
\pgfpathmoveto{\pgfqpoint{2.406429in}{4.162525in}}%
\pgfpathlineto{\pgfqpoint{2.320596in}{4.102334in}}%
\pgfpathlineto{\pgfqpoint{2.440760in}{4.553021in}}%
\pgfpathclose%
\pgfusepath{fill}%
\end{pgfscope}%
\begin{pgfscope}%
\pgfpathrectangle{\pgfqpoint{0.539299in}{0.078740in}}{\pgfqpoint{7.842520in}{7.842520in}}%
\pgfusepath{clip}%
\pgfsetbuttcap%
\pgfsetroundjoin%
\definecolor{currentfill}{rgb}{0.239374,0.735588,0.455688}%
\pgfsetfillcolor{currentfill}%
\pgfsetlinewidth{0.000000pt}%
\definecolor{currentstroke}{rgb}{0.496615,0.826376,0.306377}%
\pgfsetstrokecolor{currentstroke}%
\pgfsetdash{}{0pt}%
\pgfpathmoveto{\pgfqpoint{3.700962in}{5.285351in}}%
\pgfpathlineto{\pgfqpoint{3.838249in}{5.208623in}}%
\pgfpathlineto{\pgfqpoint{3.921341in}{5.107578in}}%
\pgfpathclose%
\pgfusepath{fill}%
\end{pgfscope}%
\begin{pgfscope}%
\pgfpathrectangle{\pgfqpoint{0.539299in}{0.078740in}}{\pgfqpoint{7.842520in}{7.842520in}}%
\pgfusepath{clip}%
\pgfsetbuttcap%
\pgfsetroundjoin%
\definecolor{currentfill}{rgb}{0.153364,0.497000,0.557724}%
\pgfsetfillcolor{currentfill}%
\pgfsetlinewidth{0.000000pt}%
\definecolor{currentstroke}{rgb}{0.506271,0.828786,0.300362}%
\pgfsetstrokecolor{currentstroke}%
\pgfsetdash{}{0pt}%
\pgfpathmoveto{\pgfqpoint{4.746612in}{4.139665in}}%
\pgfpathlineto{\pgfqpoint{4.667716in}{4.230148in}}%
\pgfpathlineto{\pgfqpoint{4.805857in}{4.049409in}}%
\pgfpathclose%
\pgfusepath{fill}%
\end{pgfscope}%
\begin{pgfscope}%
\pgfpathrectangle{\pgfqpoint{0.539299in}{0.078740in}}{\pgfqpoint{7.842520in}{7.842520in}}%
\pgfusepath{clip}%
\pgfsetbuttcap%
\pgfsetroundjoin%
\definecolor{currentfill}{rgb}{0.218130,0.347432,0.550038}%
\pgfsetfillcolor{currentfill}%
\pgfsetlinewidth{0.000000pt}%
\definecolor{currentstroke}{rgb}{0.515992,0.831158,0.294279}%
\pgfsetstrokecolor{currentstroke}%
\pgfsetdash{}{0pt}%
\pgfpathmoveto{\pgfqpoint{5.296988in}{3.532612in}}%
\pgfpathlineto{\pgfqpoint{5.220286in}{3.561627in}}%
\pgfpathlineto{\pgfqpoint{5.358646in}{3.421072in}}%
\pgfpathclose%
\pgfusepath{fill}%
\end{pgfscope}%
\begin{pgfscope}%
\pgfpathrectangle{\pgfqpoint{0.539299in}{0.078740in}}{\pgfqpoint{7.842520in}{7.842520in}}%
\pgfusepath{clip}%
\pgfsetbuttcap%
\pgfsetroundjoin%
\definecolor{currentfill}{rgb}{0.126326,0.644107,0.525311}%
\pgfsetfillcolor{currentfill}%
\pgfsetlinewidth{0.000000pt}%
\definecolor{currentstroke}{rgb}{0.525776,0.833491,0.288127}%
\pgfsetstrokecolor{currentstroke}%
\pgfsetdash{}{0pt}%
\pgfpathmoveto{\pgfqpoint{2.825015in}{5.016747in}}%
\pgfpathlineto{\pgfqpoint{2.699021in}{4.682479in}}%
\pgfpathlineto{\pgfqpoint{2.613215in}{4.649492in}}%
\pgfpathclose%
\pgfusepath{fill}%
\end{pgfscope}%
\begin{pgfscope}%
\pgfpathrectangle{\pgfqpoint{0.539299in}{0.078740in}}{\pgfqpoint{7.842520in}{7.842520in}}%
\pgfusepath{clip}%
\pgfsetbuttcap%
\pgfsetroundjoin%
\definecolor{currentfill}{rgb}{0.283091,0.110553,0.431554}%
\pgfsetfillcolor{currentfill}%
\pgfsetlinewidth{0.000000pt}%
\definecolor{currentstroke}{rgb}{0.535621,0.835785,0.281908}%
\pgfsetstrokecolor{currentstroke}%
\pgfsetdash{}{0pt}%
\pgfpathmoveto{\pgfqpoint{6.697430in}{2.767965in}}%
\pgfpathlineto{\pgfqpoint{6.763891in}{2.602375in}}%
\pgfpathlineto{\pgfqpoint{6.837125in}{2.669064in}}%
\pgfpathclose%
\pgfusepath{fill}%
\end{pgfscope}%
\begin{pgfscope}%
\pgfpathrectangle{\pgfqpoint{0.539299in}{0.078740in}}{\pgfqpoint{7.842520in}{7.842520in}}%
\pgfusepath{clip}%
\pgfsetbuttcap%
\pgfsetroundjoin%
\definecolor{currentfill}{rgb}{0.136408,0.541173,0.554483}%
\pgfsetfillcolor{currentfill}%
\pgfsetlinewidth{0.000000pt}%
\definecolor{currentstroke}{rgb}{0.545524,0.838039,0.275626}%
\pgfsetstrokecolor{currentstroke}%
\pgfsetdash{}{0pt}%
\pgfpathmoveto{\pgfqpoint{4.667716in}{4.230148in}}%
\pgfpathlineto{\pgfqpoint{4.609098in}{4.314481in}}%
\pgfpathlineto{\pgfqpoint{4.529493in}{4.415381in}}%
\pgfpathclose%
\pgfusepath{fill}%
\end{pgfscope}%
\begin{pgfscope}%
\pgfpathrectangle{\pgfqpoint{0.539299in}{0.078740in}}{\pgfqpoint{7.842520in}{7.842520in}}%
\pgfusepath{clip}%
\pgfsetbuttcap%
\pgfsetroundjoin%
\definecolor{currentfill}{rgb}{0.282884,0.135920,0.453427}%
\pgfsetfillcolor{currentfill}%
\pgfsetlinewidth{0.000000pt}%
\definecolor{currentstroke}{rgb}{0.555484,0.840254,0.269281}%
\pgfsetstrokecolor{currentstroke}%
\pgfsetdash{}{0pt}%
\pgfpathmoveto{\pgfqpoint{6.697430in}{2.767965in}}%
\pgfpathlineto{\pgfqpoint{6.483913in}{2.795763in}}%
\pgfpathlineto{\pgfqpoint{6.623869in}{2.702268in}}%
\pgfpathclose%
\pgfusepath{fill}%
\end{pgfscope}%
\begin{pgfscope}%
\pgfpathrectangle{\pgfqpoint{0.539299in}{0.078740in}}{\pgfqpoint{7.842520in}{7.842520in}}%
\pgfusepath{clip}%
\pgfsetbuttcap%
\pgfsetroundjoin%
\definecolor{currentfill}{rgb}{0.278012,0.180367,0.486697}%
\pgfsetfillcolor{currentfill}%
\pgfsetlinewidth{0.000000pt}%
\definecolor{currentstroke}{rgb}{0.565498,0.842430,0.262877}%
\pgfsetstrokecolor{currentstroke}%
\pgfsetdash{}{0pt}%
\pgfpathmoveto{\pgfqpoint{6.269433in}{2.819189in}}%
\pgfpathlineto{\pgfqpoint{6.344091in}{2.884105in}}%
\pgfpathlineto{\pgfqpoint{6.204464in}{2.968792in}}%
\pgfpathclose%
\pgfusepath{fill}%
\end{pgfscope}%
\begin{pgfscope}%
\pgfpathrectangle{\pgfqpoint{0.539299in}{0.078740in}}{\pgfqpoint{7.842520in}{7.842520in}}%
\pgfusepath{clip}%
\pgfsetbuttcap%
\pgfsetroundjoin%
\definecolor{currentfill}{rgb}{0.233603,0.313828,0.543914}%
\pgfsetfillcolor{currentfill}%
\pgfsetlinewidth{0.000000pt}%
\definecolor{currentstroke}{rgb}{0.575563,0.844566,0.256415}%
\pgfsetstrokecolor{currentstroke}%
\pgfsetdash{}{0pt}%
\pgfpathmoveto{\pgfqpoint{5.434988in}{3.409236in}}%
\pgfpathlineto{\pgfqpoint{5.358646in}{3.421072in}}%
\pgfpathlineto{\pgfqpoint{5.497218in}{3.291318in}}%
\pgfpathclose%
\pgfusepath{fill}%
\end{pgfscope}%
\begin{pgfscope}%
\pgfpathrectangle{\pgfqpoint{0.539299in}{0.078740in}}{\pgfqpoint{7.842520in}{7.842520in}}%
\pgfusepath{clip}%
\pgfsetbuttcap%
\pgfsetroundjoin%
\definecolor{currentfill}{rgb}{0.274128,0.199721,0.498911}%
\pgfsetfillcolor{currentfill}%
\pgfsetlinewidth{0.000000pt}%
\definecolor{currentstroke}{rgb}{0.585678,0.846661,0.249897}%
\pgfsetstrokecolor{currentstroke}%
\pgfsetdash{}{0pt}%
\pgfpathmoveto{\pgfqpoint{5.990036in}{3.000736in}}%
\pgfpathlineto{\pgfqpoint{6.129597in}{2.909848in}}%
\pgfpathlineto{\pgfqpoint{6.204464in}{2.968792in}}%
\pgfpathclose%
\pgfusepath{fill}%
\end{pgfscope}%
\begin{pgfscope}%
\pgfpathrectangle{\pgfqpoint{0.539299in}{0.078740in}}{\pgfqpoint{7.842520in}{7.842520in}}%
\pgfusepath{clip}%
\pgfsetbuttcap%
\pgfsetroundjoin%
\definecolor{currentfill}{rgb}{0.304148,0.764704,0.419943}%
\pgfsetfillcolor{currentfill}%
\pgfsetlinewidth{0.000000pt}%
\definecolor{currentstroke}{rgb}{0.595839,0.848717,0.243329}%
\pgfsetstrokecolor{currentstroke}%
\pgfsetdash{}{0pt}%
\pgfpathmoveto{\pgfqpoint{3.344948in}{5.333814in}}%
\pgfpathlineto{\pgfqpoint{3.480023in}{5.382736in}}%
\pgfpathlineto{\pgfqpoint{3.429695in}{5.281129in}}%
\pgfpathclose%
\pgfusepath{fill}%
\end{pgfscope}%
\begin{pgfscope}%
\pgfpathrectangle{\pgfqpoint{0.539299in}{0.078740in}}{\pgfqpoint{7.842520in}{7.842520in}}%
\pgfusepath{clip}%
\pgfsetbuttcap%
\pgfsetroundjoin%
\definecolor{currentfill}{rgb}{0.126453,0.570633,0.549841}%
\pgfsetfillcolor{currentfill}%
\pgfsetlinewidth{0.000000pt}%
\definecolor{currentstroke}{rgb}{0.606045,0.850733,0.236712}%
\pgfsetstrokecolor{currentstroke}%
\pgfsetdash{}{0pt}%
\pgfpathmoveto{\pgfqpoint{4.529493in}{4.415381in}}%
\pgfpathlineto{\pgfqpoint{4.609098in}{4.314481in}}%
\pgfpathlineto{\pgfqpoint{4.391174in}{4.600585in}}%
\pgfpathclose%
\pgfusepath{fill}%
\end{pgfscope}%
\begin{pgfscope}%
\pgfpathrectangle{\pgfqpoint{0.539299in}{0.078740in}}{\pgfqpoint{7.842520in}{7.842520in}}%
\pgfusepath{clip}%
\pgfsetbuttcap%
\pgfsetroundjoin%
\definecolor{currentfill}{rgb}{0.202219,0.715272,0.476084}%
\pgfsetfillcolor{currentfill}%
\pgfsetlinewidth{0.000000pt}%
\definecolor{currentstroke}{rgb}{0.616293,0.852709,0.230052}%
\pgfsetstrokecolor{currentstroke}%
\pgfsetdash{}{0pt}%
\pgfpathmoveto{\pgfqpoint{2.910930in}{5.025499in}}%
\pgfpathlineto{\pgfqpoint{3.040850in}{5.259550in}}%
\pgfpathlineto{\pgfqpoint{2.996415in}{5.025431in}}%
\pgfpathclose%
\pgfusepath{fill}%
\end{pgfscope}%
\begin{pgfscope}%
\pgfpathrectangle{\pgfqpoint{0.539299in}{0.078740in}}{\pgfqpoint{7.842520in}{7.842520in}}%
\pgfusepath{clip}%
\pgfsetbuttcap%
\pgfsetroundjoin%
\definecolor{currentfill}{rgb}{0.202219,0.715272,0.476084}%
\pgfsetfillcolor{currentfill}%
\pgfsetlinewidth{0.000000pt}%
\definecolor{currentstroke}{rgb}{0.626579,0.854645,0.223353}%
\pgfsetstrokecolor{currentstroke}%
\pgfsetdash{}{0pt}%
\pgfpathmoveto{\pgfqpoint{3.976153in}{5.092524in}}%
\pgfpathlineto{\pgfqpoint{4.058651in}{4.983988in}}%
\pgfpathlineto{\pgfqpoint{3.838249in}{5.208623in}}%
\pgfpathclose%
\pgfusepath{fill}%
\end{pgfscope}%
\begin{pgfscope}%
\pgfpathrectangle{\pgfqpoint{0.539299in}{0.078740in}}{\pgfqpoint{7.842520in}{7.842520in}}%
\pgfusepath{clip}%
\pgfsetbuttcap%
\pgfsetroundjoin%
\definecolor{currentfill}{rgb}{0.260571,0.246922,0.522828}%
\pgfsetfillcolor{currentfill}%
\pgfsetlinewidth{0.000000pt}%
\definecolor{currentstroke}{rgb}{0.636902,0.856542,0.216620}%
\pgfsetstrokecolor{currentstroke}%
\pgfsetdash{}{0pt}%
\pgfpathmoveto{\pgfqpoint{5.775167in}{3.059358in}}%
\pgfpathlineto{\pgfqpoint{5.850784in}{3.093964in}}%
\pgfpathlineto{\pgfqpoint{5.711861in}{3.191718in}}%
\pgfpathclose%
\pgfusepath{fill}%
\end{pgfscope}%
\begin{pgfscope}%
\pgfpathrectangle{\pgfqpoint{0.539299in}{0.078740in}}{\pgfqpoint{7.842520in}{7.842520in}}%
\pgfusepath{clip}%
\pgfsetbuttcap%
\pgfsetroundjoin%
\definecolor{currentfill}{rgb}{0.218130,0.347432,0.550038}%
\pgfsetfillcolor{currentfill}%
\pgfsetlinewidth{0.000000pt}%
\definecolor{currentstroke}{rgb}{0.647257,0.858400,0.209861}%
\pgfsetstrokecolor{currentstroke}%
\pgfsetdash{}{0pt}%
\pgfpathmoveto{\pgfqpoint{2.062008in}{3.862537in}}%
\pgfpathlineto{\pgfqpoint{1.947878in}{3.337740in}}%
\pgfpathlineto{\pgfqpoint{1.861534in}{3.249013in}}%
\pgfpathclose%
\pgfusepath{fill}%
\end{pgfscope}%
\begin{pgfscope}%
\pgfpathrectangle{\pgfqpoint{0.539299in}{0.078740in}}{\pgfqpoint{7.842520in}{7.842520in}}%
\pgfusepath{clip}%
\pgfsetbuttcap%
\pgfsetroundjoin%
\definecolor{currentfill}{rgb}{0.282327,0.094955,0.417331}%
\pgfsetfillcolor{currentfill}%
\pgfsetlinewidth{0.000000pt}%
\definecolor{currentstroke}{rgb}{0.657642,0.860219,0.203082}%
\pgfsetstrokecolor{currentstroke}%
\pgfsetdash{}{0pt}%
\pgfpathmoveto{\pgfqpoint{6.837125in}{2.669064in}}%
\pgfpathlineto{\pgfqpoint{6.763891in}{2.602375in}}%
\pgfpathlineto{\pgfqpoint{6.976816in}{2.562893in}}%
\pgfpathclose%
\pgfusepath{fill}%
\end{pgfscope}%
\begin{pgfscope}%
\pgfpathrectangle{\pgfqpoint{0.539299in}{0.078740in}}{\pgfqpoint{7.842520in}{7.842520in}}%
\pgfusepath{clip}%
\pgfsetbuttcap%
\pgfsetroundjoin%
\definecolor{currentfill}{rgb}{0.120638,0.625828,0.533488}%
\pgfsetfillcolor{currentfill}%
\pgfsetlinewidth{0.000000pt}%
\definecolor{currentstroke}{rgb}{0.668054,0.861999,0.196293}%
\pgfsetstrokecolor{currentstroke}%
\pgfsetdash{}{0pt}%
\pgfpathmoveto{\pgfqpoint{4.333879in}{4.667896in}}%
\pgfpathlineto{\pgfqpoint{4.252784in}{4.779999in}}%
\pgfpathlineto{\pgfqpoint{4.391174in}{4.600585in}}%
\pgfpathclose%
\pgfusepath{fill}%
\end{pgfscope}%
\begin{pgfscope}%
\pgfpathrectangle{\pgfqpoint{0.539299in}{0.078740in}}{\pgfqpoint{7.842520in}{7.842520in}}%
\pgfusepath{clip}%
\pgfsetbuttcap%
\pgfsetroundjoin%
\definecolor{currentfill}{rgb}{0.175707,0.697900,0.491033}%
\pgfsetfillcolor{currentfill}%
\pgfsetlinewidth{0.000000pt}%
\definecolor{currentstroke}{rgb}{0.678489,0.863742,0.189503}%
\pgfsetstrokecolor{currentstroke}%
\pgfsetdash{}{0pt}%
\pgfpathmoveto{\pgfqpoint{4.114399in}{4.946663in}}%
\pgfpathlineto{\pgfqpoint{4.058651in}{4.983988in}}%
\pgfpathlineto{\pgfqpoint{3.976153in}{5.092524in}}%
\pgfpathclose%
\pgfusepath{fill}%
\end{pgfscope}%
\begin{pgfscope}%
\pgfpathrectangle{\pgfqpoint{0.539299in}{0.078740in}}{\pgfqpoint{7.842520in}{7.842520in}}%
\pgfusepath{clip}%
\pgfsetbuttcap%
\pgfsetroundjoin%
\definecolor{currentfill}{rgb}{0.239374,0.735588,0.455688}%
\pgfsetfillcolor{currentfill}%
\pgfsetlinewidth{0.000000pt}%
\definecolor{currentstroke}{rgb}{0.688944,0.865448,0.182725}%
\pgfsetstrokecolor{currentstroke}%
\pgfsetdash{}{0pt}%
\pgfpathmoveto{\pgfqpoint{2.996415in}{5.025431in}}%
\pgfpathlineto{\pgfqpoint{3.040850in}{5.259550in}}%
\pgfpathlineto{\pgfqpoint{3.126635in}{5.241459in}}%
\pgfpathclose%
\pgfusepath{fill}%
\end{pgfscope}%
\begin{pgfscope}%
\pgfpathrectangle{\pgfqpoint{0.539299in}{0.078740in}}{\pgfqpoint{7.842520in}{7.842520in}}%
\pgfusepath{clip}%
\pgfsetbuttcap%
\pgfsetroundjoin%
\definecolor{currentfill}{rgb}{0.283229,0.120777,0.440584}%
\pgfsetfillcolor{currentfill}%
\pgfsetlinewidth{0.000000pt}%
\definecolor{currentstroke}{rgb}{0.699415,0.867117,0.175971}%
\pgfsetstrokecolor{currentstroke}%
\pgfsetdash{}{0pt}%
\pgfpathmoveto{\pgfqpoint{6.623869in}{2.702268in}}%
\pgfpathlineto{\pgfqpoint{6.763891in}{2.602375in}}%
\pgfpathlineto{\pgfqpoint{6.697430in}{2.767965in}}%
\pgfpathclose%
\pgfusepath{fill}%
\end{pgfscope}%
\begin{pgfscope}%
\pgfpathrectangle{\pgfqpoint{0.539299in}{0.078740in}}{\pgfqpoint{7.842520in}{7.842520in}}%
\pgfusepath{clip}%
\pgfsetbuttcap%
\pgfsetroundjoin%
\definecolor{currentfill}{rgb}{0.140210,0.665859,0.513427}%
\pgfsetfillcolor{currentfill}%
\pgfsetlinewidth{0.000000pt}%
\definecolor{currentstroke}{rgb}{0.709898,0.868751,0.169257}%
\pgfsetstrokecolor{currentstroke}%
\pgfsetdash{}{0pt}%
\pgfpathmoveto{\pgfqpoint{4.196221in}{4.834423in}}%
\pgfpathlineto{\pgfqpoint{4.114399in}{4.946663in}}%
\pgfpathlineto{\pgfqpoint{4.252784in}{4.779999in}}%
\pgfpathclose%
\pgfusepath{fill}%
\end{pgfscope}%
\begin{pgfscope}%
\pgfpathrectangle{\pgfqpoint{0.539299in}{0.078740in}}{\pgfqpoint{7.842520in}{7.842520in}}%
\pgfusepath{clip}%
\pgfsetbuttcap%
\pgfsetroundjoin%
\definecolor{currentfill}{rgb}{0.277134,0.185228,0.489898}%
\pgfsetfillcolor{currentfill}%
\pgfsetlinewidth{0.000000pt}%
\definecolor{currentstroke}{rgb}{0.720391,0.870350,0.162603}%
\pgfsetstrokecolor{currentstroke}%
\pgfsetdash{}{0pt}%
\pgfpathmoveto{\pgfqpoint{6.204464in}{2.968792in}}%
\pgfpathlineto{\pgfqpoint{6.129597in}{2.909848in}}%
\pgfpathlineto{\pgfqpoint{6.269433in}{2.819189in}}%
\pgfpathclose%
\pgfusepath{fill}%
\end{pgfscope}%
\begin{pgfscope}%
\pgfpathrectangle{\pgfqpoint{0.539299in}{0.078740in}}{\pgfqpoint{7.842520in}{7.842520in}}%
\pgfusepath{clip}%
\pgfsetbuttcap%
\pgfsetroundjoin%
\definecolor{currentfill}{rgb}{0.296479,0.761561,0.424223}%
\pgfsetfillcolor{currentfill}%
\pgfsetlinewidth{0.000000pt}%
\definecolor{currentstroke}{rgb}{0.730889,0.871916,0.156029}%
\pgfsetstrokecolor{currentstroke}%
\pgfsetdash{}{0pt}%
\pgfpathmoveto{\pgfqpoint{3.344948in}{5.333814in}}%
\pgfpathlineto{\pgfqpoint{3.211903in}{5.214787in}}%
\pgfpathlineto{\pgfqpoint{3.259589in}{5.378978in}}%
\pgfpathclose%
\pgfusepath{fill}%
\end{pgfscope}%
\begin{pgfscope}%
\pgfpathrectangle{\pgfqpoint{0.539299in}{0.078740in}}{\pgfqpoint{7.842520in}{7.842520in}}%
\pgfusepath{clip}%
\pgfsetbuttcap%
\pgfsetroundjoin%
\definecolor{currentfill}{rgb}{0.248629,0.278775,0.534556}%
\pgfsetfillcolor{currentfill}%
\pgfsetlinewidth{0.000000pt}%
\definecolor{currentstroke}{rgb}{0.741388,0.873449,0.149561}%
\pgfsetstrokecolor{currentstroke}%
\pgfsetdash{}{0pt}%
\pgfpathmoveto{\pgfqpoint{5.711861in}{3.191718in}}%
\pgfpathlineto{\pgfqpoint{5.497218in}{3.291318in}}%
\pgfpathlineto{\pgfqpoint{5.636047in}{3.171258in}}%
\pgfpathclose%
\pgfusepath{fill}%
\end{pgfscope}%
\begin{pgfscope}%
\pgfpathrectangle{\pgfqpoint{0.539299in}{0.078740in}}{\pgfqpoint{7.842520in}{7.842520in}}%
\pgfusepath{clip}%
\pgfsetbuttcap%
\pgfsetroundjoin%
\definecolor{currentfill}{rgb}{0.304148,0.764704,0.419943}%
\pgfsetfillcolor{currentfill}%
\pgfsetlinewidth{0.000000pt}%
\definecolor{currentstroke}{rgb}{0.751884,0.874951,0.143228}%
\pgfsetstrokecolor{currentstroke}%
\pgfsetdash{}{0pt}%
\pgfpathmoveto{\pgfqpoint{3.564639in}{5.312794in}}%
\pgfpathlineto{\pgfqpoint{3.480023in}{5.382736in}}%
\pgfpathlineto{\pgfqpoint{3.700962in}{5.285351in}}%
\pgfpathclose%
\pgfusepath{fill}%
\end{pgfscope}%
\begin{pgfscope}%
\pgfpathrectangle{\pgfqpoint{0.539299in}{0.078740in}}{\pgfqpoint{7.842520in}{7.842520in}}%
\pgfusepath{clip}%
\pgfsetbuttcap%
\pgfsetroundjoin%
\definecolor{currentfill}{rgb}{0.281477,0.755203,0.432552}%
\pgfsetfillcolor{currentfill}%
\pgfsetlinewidth{0.000000pt}%
\definecolor{currentstroke}{rgb}{0.762373,0.876424,0.137064}%
\pgfsetstrokecolor{currentstroke}%
\pgfsetdash{}{0pt}%
\pgfpathmoveto{\pgfqpoint{3.259589in}{5.378978in}}%
\pgfpathlineto{\pgfqpoint{3.211903in}{5.214787in}}%
\pgfpathlineto{\pgfqpoint{3.126635in}{5.241459in}}%
\pgfpathclose%
\pgfusepath{fill}%
\end{pgfscope}%
\begin{pgfscope}%
\pgfpathrectangle{\pgfqpoint{0.539299in}{0.078740in}}{\pgfqpoint{7.842520in}{7.842520in}}%
\pgfusepath{clip}%
\pgfsetbuttcap%
\pgfsetroundjoin%
\definecolor{currentfill}{rgb}{0.280868,0.160771,0.472899}%
\pgfsetfillcolor{currentfill}%
\pgfsetlinewidth{0.000000pt}%
\definecolor{currentstroke}{rgb}{0.772852,0.877868,0.131109}%
\pgfsetstrokecolor{currentstroke}%
\pgfsetdash{}{0pt}%
\pgfpathmoveto{\pgfqpoint{6.409497in}{2.726792in}}%
\pgfpathlineto{\pgfqpoint{6.483913in}{2.795763in}}%
\pgfpathlineto{\pgfqpoint{6.344091in}{2.884105in}}%
\pgfpathclose%
\pgfusepath{fill}%
\end{pgfscope}%
\begin{pgfscope}%
\pgfpathrectangle{\pgfqpoint{0.539299in}{0.078740in}}{\pgfqpoint{7.842520in}{7.842520in}}%
\pgfusepath{clip}%
\pgfsetbuttcap%
\pgfsetroundjoin%
\definecolor{currentfill}{rgb}{0.267968,0.223549,0.512008}%
\pgfsetfillcolor{currentfill}%
\pgfsetlinewidth{0.000000pt}%
\definecolor{currentstroke}{rgb}{0.783315,0.879285,0.125405}%
\pgfsetstrokecolor{currentstroke}%
\pgfsetdash{}{0pt}%
\pgfpathmoveto{\pgfqpoint{5.914592in}{2.953804in}}%
\pgfpathlineto{\pgfqpoint{5.990036in}{3.000736in}}%
\pgfpathlineto{\pgfqpoint{5.850784in}{3.093964in}}%
\pgfpathclose%
\pgfusepath{fill}%
\end{pgfscope}%
\begin{pgfscope}%
\pgfpathrectangle{\pgfqpoint{0.539299in}{0.078740in}}{\pgfqpoint{7.842520in}{7.842520in}}%
\pgfusepath{clip}%
\pgfsetbuttcap%
\pgfsetroundjoin%
\definecolor{currentfill}{rgb}{0.255645,0.260703,0.528312}%
\pgfsetfillcolor{currentfill}%
\pgfsetlinewidth{0.000000pt}%
\definecolor{currentstroke}{rgb}{0.793760,0.880678,0.120005}%
\pgfsetstrokecolor{currentstroke}%
\pgfsetdash{}{0pt}%
\pgfpathmoveto{\pgfqpoint{5.636047in}{3.171258in}}%
\pgfpathlineto{\pgfqpoint{5.775167in}{3.059358in}}%
\pgfpathlineto{\pgfqpoint{5.711861in}{3.191718in}}%
\pgfpathclose%
\pgfusepath{fill}%
\end{pgfscope}%
\begin{pgfscope}%
\pgfpathrectangle{\pgfqpoint{0.539299in}{0.078740in}}{\pgfqpoint{7.842520in}{7.842520in}}%
\pgfusepath{clip}%
\pgfsetbuttcap%
\pgfsetroundjoin%
\definecolor{currentfill}{rgb}{0.125394,0.574318,0.549086}%
\pgfsetfillcolor{currentfill}%
\pgfsetlinewidth{0.000000pt}%
\definecolor{currentstroke}{rgb}{0.804182,0.882046,0.114965}%
\pgfsetstrokecolor{currentstroke}%
\pgfsetdash{}{0pt}%
\pgfpathmoveto{\pgfqpoint{2.440760in}{4.553021in}}%
\pgfpathlineto{\pgfqpoint{2.527103in}{4.606886in}}%
\pgfpathlineto{\pgfqpoint{2.406429in}{4.162525in}}%
\pgfpathclose%
\pgfusepath{fill}%
\end{pgfscope}%
\begin{pgfscope}%
\pgfpathrectangle{\pgfqpoint{0.539299in}{0.078740in}}{\pgfqpoint{7.842520in}{7.842520in}}%
\pgfusepath{clip}%
\pgfsetbuttcap%
\pgfsetroundjoin%
\definecolor{currentfill}{rgb}{0.277018,0.050344,0.375715}%
\pgfsetfillcolor{currentfill}%
\pgfsetlinewidth{0.000000pt}%
\definecolor{currentstroke}{rgb}{0.814576,0.883393,0.110347}%
\pgfsetstrokecolor{currentstroke}%
\pgfsetdash{}{0pt}%
\pgfpathmoveto{\pgfqpoint{6.976816in}{2.562893in}}%
\pgfpathlineto{\pgfqpoint{7.043838in}{2.379025in}}%
\pgfpathlineto{\pgfqpoint{7.116450in}{2.449156in}}%
\pgfpathclose%
\pgfusepath{fill}%
\end{pgfscope}%
\begin{pgfscope}%
\pgfpathrectangle{\pgfqpoint{0.539299in}{0.078740in}}{\pgfqpoint{7.842520in}{7.842520in}}%
\pgfusepath{clip}%
\pgfsetbuttcap%
\pgfsetroundjoin%
\definecolor{currentfill}{rgb}{0.141935,0.526453,0.555991}%
\pgfsetfillcolor{currentfill}%
\pgfsetlinewidth{0.000000pt}%
\definecolor{currentstroke}{rgb}{0.824940,0.884720,0.106217}%
\pgfsetstrokecolor{currentstroke}%
\pgfsetdash{}{0pt}%
\pgfpathmoveto{\pgfqpoint{2.320596in}{4.102334in}}%
\pgfpathlineto{\pgfqpoint{2.234531in}{4.033237in}}%
\pgfpathlineto{\pgfqpoint{2.440760in}{4.553021in}}%
\pgfpathclose%
\pgfusepath{fill}%
\end{pgfscope}%
\begin{pgfscope}%
\pgfpathrectangle{\pgfqpoint{0.539299in}{0.078740in}}{\pgfqpoint{7.842520in}{7.842520in}}%
\pgfusepath{clip}%
\pgfsetbuttcap%
\pgfsetroundjoin%
\definecolor{currentfill}{rgb}{0.281446,0.084320,0.407414}%
\pgfsetfillcolor{currentfill}%
\pgfsetlinewidth{0.000000pt}%
\definecolor{currentstroke}{rgb}{0.835270,0.886029,0.102646}%
\pgfsetstrokecolor{currentstroke}%
\pgfsetdash{}{0pt}%
\pgfpathmoveto{\pgfqpoint{6.976816in}{2.562893in}}%
\pgfpathlineto{\pgfqpoint{6.763891in}{2.602375in}}%
\pgfpathlineto{\pgfqpoint{6.903908in}{2.494991in}}%
\pgfpathclose%
\pgfusepath{fill}%
\end{pgfscope}%
\begin{pgfscope}%
\pgfpathrectangle{\pgfqpoint{0.539299in}{0.078740in}}{\pgfqpoint{7.842520in}{7.842520in}}%
\pgfusepath{clip}%
\pgfsetbuttcap%
\pgfsetroundjoin%
\definecolor{currentfill}{rgb}{0.280255,0.165693,0.476498}%
\pgfsetfillcolor{currentfill}%
\pgfsetlinewidth{0.000000pt}%
\definecolor{currentstroke}{rgb}{0.845561,0.887322,0.099702}%
\pgfsetstrokecolor{currentstroke}%
\pgfsetdash{}{0pt}%
\pgfpathmoveto{\pgfqpoint{6.344091in}{2.884105in}}%
\pgfpathlineto{\pgfqpoint{6.269433in}{2.819189in}}%
\pgfpathlineto{\pgfqpoint{6.409497in}{2.726792in}}%
\pgfpathclose%
\pgfusepath{fill}%
\end{pgfscope}%
\begin{pgfscope}%
\pgfpathrectangle{\pgfqpoint{0.539299in}{0.078740in}}{\pgfqpoint{7.842520in}{7.842520in}}%
\pgfusepath{clip}%
\pgfsetbuttcap%
\pgfsetroundjoin%
\definecolor{currentfill}{rgb}{0.271828,0.209303,0.504434}%
\pgfsetfillcolor{currentfill}%
\pgfsetlinewidth{0.000000pt}%
\definecolor{currentstroke}{rgb}{0.855810,0.888601,0.097452}%
\pgfsetstrokecolor{currentstroke}%
\pgfsetdash{}{0pt}%
\pgfpathmoveto{\pgfqpoint{5.990036in}{3.000736in}}%
\pgfpathlineto{\pgfqpoint{5.914592in}{2.953804in}}%
\pgfpathlineto{\pgfqpoint{6.129597in}{2.909848in}}%
\pgfpathclose%
\pgfusepath{fill}%
\end{pgfscope}%
\begin{pgfscope}%
\pgfpathrectangle{\pgfqpoint{0.539299in}{0.078740in}}{\pgfqpoint{7.842520in}{7.842520in}}%
\pgfusepath{clip}%
\pgfsetbuttcap%
\pgfsetroundjoin%
\definecolor{currentfill}{rgb}{0.188923,0.410910,0.556326}%
\pgfsetfillcolor{currentfill}%
\pgfsetlinewidth{0.000000pt}%
\definecolor{currentstroke}{rgb}{0.866013,0.889868,0.095953}%
\pgfsetstrokecolor{currentstroke}%
\pgfsetdash{}{0pt}%
\pgfpathmoveto{\pgfqpoint{1.947878in}{3.337740in}}%
\pgfpathlineto{\pgfqpoint{2.062008in}{3.862537in}}%
\pgfpathlineto{\pgfqpoint{2.148306in}{3.953837in}}%
\pgfpathclose%
\pgfusepath{fill}%
\end{pgfscope}%
\begin{pgfscope}%
\pgfpathrectangle{\pgfqpoint{0.539299in}{0.078740in}}{\pgfqpoint{7.842520in}{7.842520in}}%
\pgfusepath{clip}%
\pgfsetbuttcap%
\pgfsetroundjoin%
\definecolor{currentfill}{rgb}{0.282884,0.135920,0.453427}%
\pgfsetfillcolor{currentfill}%
\pgfsetlinewidth{0.000000pt}%
\definecolor{currentstroke}{rgb}{0.876168,0.891125,0.095250}%
\pgfsetstrokecolor{currentstroke}%
\pgfsetdash{}{0pt}%
\pgfpathmoveto{\pgfqpoint{6.549730in}{2.630860in}}%
\pgfpathlineto{\pgfqpoint{6.623869in}{2.702268in}}%
\pgfpathlineto{\pgfqpoint{6.483913in}{2.795763in}}%
\pgfpathclose%
\pgfusepath{fill}%
\end{pgfscope}%
\begin{pgfscope}%
\pgfpathrectangle{\pgfqpoint{0.539299in}{0.078740in}}{\pgfqpoint{7.842520in}{7.842520in}}%
\pgfusepath{clip}%
\pgfsetbuttcap%
\pgfsetroundjoin%
\definecolor{currentfill}{rgb}{0.263663,0.237631,0.518762}%
\pgfsetfillcolor{currentfill}%
\pgfsetlinewidth{0.000000pt}%
\definecolor{currentstroke}{rgb}{0.886271,0.892374,0.095374}%
\pgfsetstrokecolor{currentstroke}%
\pgfsetdash{}{0pt}%
\pgfpathmoveto{\pgfqpoint{5.914592in}{2.953804in}}%
\pgfpathlineto{\pgfqpoint{5.850784in}{3.093964in}}%
\pgfpathlineto{\pgfqpoint{5.775167in}{3.059358in}}%
\pgfpathclose%
\pgfusepath{fill}%
\end{pgfscope}%
\begin{pgfscope}%
\pgfpathrectangle{\pgfqpoint{0.539299in}{0.078740in}}{\pgfqpoint{7.842520in}{7.842520in}}%
\pgfusepath{clip}%
\pgfsetbuttcap%
\pgfsetroundjoin%
\definecolor{currentfill}{rgb}{0.278791,0.062145,0.386592}%
\pgfsetfillcolor{currentfill}%
\pgfsetlinewidth{0.000000pt}%
\definecolor{currentstroke}{rgb}{0.896320,0.893616,0.096335}%
\pgfsetstrokecolor{currentstroke}%
\pgfsetdash{}{0pt}%
\pgfpathmoveto{\pgfqpoint{6.903908in}{2.494991in}}%
\pgfpathlineto{\pgfqpoint{7.043838in}{2.379025in}}%
\pgfpathlineto{\pgfqpoint{6.976816in}{2.562893in}}%
\pgfpathclose%
\pgfusepath{fill}%
\end{pgfscope}%
\begin{pgfscope}%
\pgfpathrectangle{\pgfqpoint{0.539299in}{0.078740in}}{\pgfqpoint{7.842520in}{7.842520in}}%
\pgfusepath{clip}%
\pgfsetbuttcap%
\pgfsetroundjoin%
\definecolor{currentfill}{rgb}{0.195860,0.395433,0.555276}%
\pgfsetfillcolor{currentfill}%
\pgfsetlinewidth{0.000000pt}%
\definecolor{currentstroke}{rgb}{0.906311,0.894855,0.098125}%
\pgfsetstrokecolor{currentstroke}%
\pgfsetdash{}{0pt}%
\pgfpathmoveto{\pgfqpoint{5.220286in}{3.561627in}}%
\pgfpathlineto{\pgfqpoint{5.082079in}{3.713504in}}%
\pgfpathlineto{\pgfqpoint{5.143030in}{3.602315in}}%
\pgfpathclose%
\pgfusepath{fill}%
\end{pgfscope}%
\begin{pgfscope}%
\pgfpathrectangle{\pgfqpoint{0.539299in}{0.078740in}}{\pgfqpoint{7.842520in}{7.842520in}}%
\pgfusepath{clip}%
\pgfsetbuttcap%
\pgfsetroundjoin%
\definecolor{currentfill}{rgb}{0.174274,0.445044,0.557792}%
\pgfsetfillcolor{currentfill}%
\pgfsetlinewidth{0.000000pt}%
\definecolor{currentstroke}{rgb}{0.916242,0.896091,0.100717}%
\pgfsetstrokecolor{currentstroke}%
\pgfsetdash{}{0pt}%
\pgfpathmoveto{\pgfqpoint{5.082079in}{3.713504in}}%
\pgfpathlineto{\pgfqpoint{4.943959in}{3.876474in}}%
\pgfpathlineto{\pgfqpoint{4.865625in}{3.949141in}}%
\pgfpathclose%
\pgfusepath{fill}%
\end{pgfscope}%
\begin{pgfscope}%
\pgfpathrectangle{\pgfqpoint{0.539299in}{0.078740in}}{\pgfqpoint{7.842520in}{7.842520in}}%
\pgfusepath{clip}%
\pgfsetbuttcap%
\pgfsetroundjoin%
\definecolor{currentfill}{rgb}{0.206756,0.371758,0.553117}%
\pgfsetfillcolor{currentfill}%
\pgfsetlinewidth{0.000000pt}%
\definecolor{currentstroke}{rgb}{0.926106,0.897330,0.104071}%
\pgfsetstrokecolor{currentstroke}%
\pgfsetdash{}{0pt}%
\pgfpathmoveto{\pgfqpoint{5.143030in}{3.602315in}}%
\pgfpathlineto{\pgfqpoint{5.358646in}{3.421072in}}%
\pgfpathlineto{\pgfqpoint{5.220286in}{3.561627in}}%
\pgfpathclose%
\pgfusepath{fill}%
\end{pgfscope}%
\begin{pgfscope}%
\pgfpathrectangle{\pgfqpoint{0.539299in}{0.078740in}}{\pgfqpoint{7.842520in}{7.842520in}}%
\pgfusepath{clip}%
\pgfsetbuttcap%
\pgfsetroundjoin%
\definecolor{currentfill}{rgb}{0.163625,0.471133,0.558148}%
\pgfsetfillcolor{currentfill}%
\pgfsetlinewidth{0.000000pt}%
\definecolor{currentstroke}{rgb}{0.935904,0.898570,0.108131}%
\pgfsetstrokecolor{currentstroke}%
\pgfsetdash{}{0pt}%
\pgfpathmoveto{\pgfqpoint{4.865625in}{3.949141in}}%
\pgfpathlineto{\pgfqpoint{4.943959in}{3.876474in}}%
\pgfpathlineto{\pgfqpoint{4.805857in}{4.049409in}}%
\pgfpathclose%
\pgfusepath{fill}%
\end{pgfscope}%
\begin{pgfscope}%
\pgfpathrectangle{\pgfqpoint{0.539299in}{0.078740in}}{\pgfqpoint{7.842520in}{7.842520in}}%
\pgfusepath{clip}%
\pgfsetbuttcap%
\pgfsetroundjoin%
\definecolor{currentfill}{rgb}{0.226397,0.728888,0.462789}%
\pgfsetfillcolor{currentfill}%
\pgfsetlinewidth{0.000000pt}%
\definecolor{currentstroke}{rgb}{0.945636,0.899815,0.112838}%
\pgfsetstrokecolor{currentstroke}%
\pgfsetdash{}{0pt}%
\pgfpathmoveto{\pgfqpoint{2.825015in}{5.016747in}}%
\pgfpathlineto{\pgfqpoint{3.040850in}{5.259550in}}%
\pgfpathlineto{\pgfqpoint{2.910930in}{5.025499in}}%
\pgfpathclose%
\pgfusepath{fill}%
\end{pgfscope}%
\begin{pgfscope}%
\pgfpathrectangle{\pgfqpoint{0.539299in}{0.078740in}}{\pgfqpoint{7.842520in}{7.842520in}}%
\pgfusepath{clip}%
\pgfsetbuttcap%
\pgfsetroundjoin%
\definecolor{currentfill}{rgb}{0.282290,0.145912,0.461510}%
\pgfsetfillcolor{currentfill}%
\pgfsetlinewidth{0.000000pt}%
\definecolor{currentstroke}{rgb}{0.955300,0.901065,0.118128}%
\pgfsetstrokecolor{currentstroke}%
\pgfsetdash{}{0pt}%
\pgfpathmoveto{\pgfqpoint{6.483913in}{2.795763in}}%
\pgfpathlineto{\pgfqpoint{6.409497in}{2.726792in}}%
\pgfpathlineto{\pgfqpoint{6.549730in}{2.630860in}}%
\pgfpathclose%
\pgfusepath{fill}%
\end{pgfscope}%
\begin{pgfscope}%
\pgfpathrectangle{\pgfqpoint{0.539299in}{0.078740in}}{\pgfqpoint{7.842520in}{7.842520in}}%
\pgfusepath{clip}%
\pgfsetbuttcap%
\pgfsetroundjoin%
\definecolor{currentfill}{rgb}{0.223925,0.334994,0.548053}%
\pgfsetfillcolor{currentfill}%
\pgfsetlinewidth{0.000000pt}%
\definecolor{currentstroke}{rgb}{0.964894,0.902323,0.123941}%
\pgfsetstrokecolor{currentstroke}%
\pgfsetdash{}{0pt}%
\pgfpathmoveto{\pgfqpoint{5.497218in}{3.291318in}}%
\pgfpathlineto{\pgfqpoint{5.358646in}{3.421072in}}%
\pgfpathlineto{\pgfqpoint{5.281811in}{3.444596in}}%
\pgfpathclose%
\pgfusepath{fill}%
\end{pgfscope}%
\begin{pgfscope}%
\pgfpathrectangle{\pgfqpoint{0.539299in}{0.078740in}}{\pgfqpoint{7.842520in}{7.842520in}}%
\pgfusepath{clip}%
\pgfsetbuttcap%
\pgfsetroundjoin%
\definecolor{currentfill}{rgb}{0.283197,0.115680,0.436115}%
\pgfsetfillcolor{currentfill}%
\pgfsetlinewidth{0.000000pt}%
\definecolor{currentstroke}{rgb}{0.974417,0.903590,0.130215}%
\pgfsetstrokecolor{currentstroke}%
\pgfsetdash{}{0pt}%
\pgfpathmoveto{\pgfqpoint{6.690065in}{2.529737in}}%
\pgfpathlineto{\pgfqpoint{6.763891in}{2.602375in}}%
\pgfpathlineto{\pgfqpoint{6.623869in}{2.702268in}}%
\pgfpathclose%
\pgfusepath{fill}%
\end{pgfscope}%
\begin{pgfscope}%
\pgfpathrectangle{\pgfqpoint{0.539299in}{0.078740in}}{\pgfqpoint{7.842520in}{7.842520in}}%
\pgfusepath{clip}%
\pgfsetbuttcap%
\pgfsetroundjoin%
\definecolor{currentfill}{rgb}{0.223925,0.334994,0.548053}%
\pgfsetfillcolor{currentfill}%
\pgfsetlinewidth{0.000000pt}%
\definecolor{currentstroke}{rgb}{0.983868,0.904867,0.136897}%
\pgfsetstrokecolor{currentstroke}%
\pgfsetdash{}{0pt}%
\pgfpathmoveto{\pgfqpoint{1.975740in}{3.757494in}}%
\pgfpathlineto{\pgfqpoint{1.861534in}{3.249013in}}%
\pgfpathlineto{\pgfqpoint{1.775040in}{3.152116in}}%
\pgfpathclose%
\pgfusepath{fill}%
\end{pgfscope}%
\begin{pgfscope}%
\pgfpathrectangle{\pgfqpoint{0.539299in}{0.078740in}}{\pgfqpoint{7.842520in}{7.842520in}}%
\pgfusepath{clip}%
\pgfsetbuttcap%
\pgfsetroundjoin%
\definecolor{currentfill}{rgb}{0.304148,0.764704,0.419943}%
\pgfsetfillcolor{currentfill}%
\pgfsetlinewidth{0.000000pt}%
\definecolor{currentstroke}{rgb}{0.993248,0.906157,0.143936}%
\pgfsetstrokecolor{currentstroke}%
\pgfsetdash{}{0pt}%
\pgfpathmoveto{\pgfqpoint{3.754366in}{5.305678in}}%
\pgfpathlineto{\pgfqpoint{3.838249in}{5.208623in}}%
\pgfpathlineto{\pgfqpoint{3.700962in}{5.285351in}}%
\pgfpathclose%
\pgfusepath{fill}%
\end{pgfscope}%
\begin{pgfscope}%
\pgfpathrectangle{\pgfqpoint{0.539299in}{0.078740in}}{\pgfqpoint{7.842520in}{7.842520in}}%
\pgfusepath{clip}%
\pgfsetbuttcap%
\pgfsetroundjoin%
\definecolor{currentfill}{rgb}{0.140536,0.530132,0.555659}%
\pgfsetfillcolor{currentfill}%
\pgfsetlinewidth{0.000000pt}%
\definecolor{currentstroke}{rgb}{0.267004,0.004874,0.329415}%
\pgfsetstrokecolor{currentstroke}%
\pgfsetdash{}{0pt}%
\pgfpathmoveto{\pgfqpoint{4.805857in}{4.049409in}}%
\pgfpathlineto{\pgfqpoint{4.667716in}{4.230148in}}%
\pgfpathlineto{\pgfqpoint{4.588029in}{4.327753in}}%
\pgfpathclose%
\pgfusepath{fill}%
\end{pgfscope}%
\begin{pgfscope}%
\pgfpathrectangle{\pgfqpoint{0.539299in}{0.078740in}}{\pgfqpoint{7.842520in}{7.842520in}}%
\pgfusepath{clip}%
\pgfsetbuttcap%
\pgfsetroundjoin%
\definecolor{currentfill}{rgb}{0.278012,0.180367,0.486697}%
\pgfsetfillcolor{currentfill}%
\pgfsetlinewidth{0.000000pt}%
\definecolor{currentstroke}{rgb}{0.268510,0.009605,0.335427}%
\pgfsetstrokecolor{currentstroke}%
\pgfsetdash{}{0pt}%
\pgfpathmoveto{\pgfqpoint{6.194335in}{2.753858in}}%
\pgfpathlineto{\pgfqpoint{6.269433in}{2.819189in}}%
\pgfpathlineto{\pgfqpoint{6.129597in}{2.909848in}}%
\pgfpathclose%
\pgfusepath{fill}%
\end{pgfscope}%
\begin{pgfscope}%
\pgfpathrectangle{\pgfqpoint{0.539299in}{0.078740in}}{\pgfqpoint{7.842520in}{7.842520in}}%
\pgfusepath{clip}%
\pgfsetbuttcap%
\pgfsetroundjoin%
\definecolor{currentfill}{rgb}{0.352360,0.783011,0.392636}%
\pgfsetfillcolor{currentfill}%
\pgfsetlinewidth{0.000000pt}%
\definecolor{currentstroke}{rgb}{0.269944,0.014625,0.341379}%
\pgfsetstrokecolor{currentstroke}%
\pgfsetdash{}{0pt}%
\pgfpathmoveto{\pgfqpoint{3.344948in}{5.333814in}}%
\pgfpathlineto{\pgfqpoint{3.259589in}{5.378978in}}%
\pgfpathlineto{\pgfqpoint{3.480023in}{5.382736in}}%
\pgfpathclose%
\pgfusepath{fill}%
\end{pgfscope}%
\begin{pgfscope}%
\pgfpathrectangle{\pgfqpoint{0.539299in}{0.078740in}}{\pgfqpoint{7.842520in}{7.842520in}}%
\pgfusepath{clip}%
\pgfsetbuttcap%
\pgfsetroundjoin%
\definecolor{currentfill}{rgb}{0.344074,0.780029,0.397381}%
\pgfsetfillcolor{currentfill}%
\pgfsetlinewidth{0.000000pt}%
\definecolor{currentstroke}{rgb}{0.271305,0.019942,0.347269}%
\pgfsetstrokecolor{currentstroke}%
\pgfsetdash{}{0pt}%
\pgfpathmoveto{\pgfqpoint{3.700962in}{5.285351in}}%
\pgfpathlineto{\pgfqpoint{3.480023in}{5.382736in}}%
\pgfpathlineto{\pgfqpoint{3.616640in}{5.370253in}}%
\pgfpathclose%
\pgfusepath{fill}%
\end{pgfscope}%
\begin{pgfscope}%
\pgfpathrectangle{\pgfqpoint{0.539299in}{0.078740in}}{\pgfqpoint{7.842520in}{7.842520in}}%
\pgfusepath{clip}%
\pgfsetbuttcap%
\pgfsetroundjoin%
\definecolor{currentfill}{rgb}{0.273006,0.204520,0.501721}%
\pgfsetfillcolor{currentfill}%
\pgfsetlinewidth{0.000000pt}%
\definecolor{currentstroke}{rgb}{0.272594,0.025563,0.353093}%
\pgfsetstrokecolor{currentstroke}%
\pgfsetdash{}{0pt}%
\pgfpathmoveto{\pgfqpoint{6.129597in}{2.909848in}}%
\pgfpathlineto{\pgfqpoint{5.914592in}{2.953804in}}%
\pgfpathlineto{\pgfqpoint{6.054321in}{2.852638in}}%
\pgfpathclose%
\pgfusepath{fill}%
\end{pgfscope}%
\begin{pgfscope}%
\pgfpathrectangle{\pgfqpoint{0.539299in}{0.078740in}}{\pgfqpoint{7.842520in}{7.842520in}}%
\pgfusepath{clip}%
\pgfsetbuttcap%
\pgfsetroundjoin%
\definecolor{currentfill}{rgb}{0.132268,0.655014,0.519661}%
\pgfsetfillcolor{currentfill}%
\pgfsetlinewidth{0.000000pt}%
\definecolor{currentstroke}{rgb}{0.273809,0.031497,0.358853}%
\pgfsetstrokecolor{currentstroke}%
\pgfsetdash{}{0pt}%
\pgfpathmoveto{\pgfqpoint{2.613215in}{4.649492in}}%
\pgfpathlineto{\pgfqpoint{2.527103in}{4.606886in}}%
\pgfpathlineto{\pgfqpoint{2.738735in}{4.997476in}}%
\pgfpathclose%
\pgfusepath{fill}%
\end{pgfscope}%
\begin{pgfscope}%
\pgfpathrectangle{\pgfqpoint{0.539299in}{0.078740in}}{\pgfqpoint{7.842520in}{7.842520in}}%
\pgfusepath{clip}%
\pgfsetbuttcap%
\pgfsetroundjoin%
\definecolor{currentfill}{rgb}{0.243113,0.292092,0.538516}%
\pgfsetfillcolor{currentfill}%
\pgfsetlinewidth{0.000000pt}%
\definecolor{currentstroke}{rgb}{0.274952,0.037752,0.364543}%
\pgfsetstrokecolor{currentstroke}%
\pgfsetdash{}{0pt}%
\pgfpathmoveto{\pgfqpoint{5.497218in}{3.291318in}}%
\pgfpathlineto{\pgfqpoint{5.559838in}{3.161010in}}%
\pgfpathlineto{\pgfqpoint{5.636047in}{3.171258in}}%
\pgfpathclose%
\pgfusepath{fill}%
\end{pgfscope}%
\begin{pgfscope}%
\pgfpathrectangle{\pgfqpoint{0.539299in}{0.078740in}}{\pgfqpoint{7.842520in}{7.842520in}}%
\pgfusepath{clip}%
\pgfsetbuttcap%
\pgfsetroundjoin%
\definecolor{currentfill}{rgb}{0.162016,0.687316,0.499129}%
\pgfsetfillcolor{currentfill}%
\pgfsetlinewidth{0.000000pt}%
\definecolor{currentstroke}{rgb}{0.276022,0.044167,0.370164}%
\pgfsetstrokecolor{currentstroke}%
\pgfsetdash{}{0pt}%
\pgfpathmoveto{\pgfqpoint{2.613215in}{4.649492in}}%
\pgfpathlineto{\pgfqpoint{2.738735in}{4.997476in}}%
\pgfpathlineto{\pgfqpoint{2.825015in}{5.016747in}}%
\pgfpathclose%
\pgfusepath{fill}%
\end{pgfscope}%
\begin{pgfscope}%
\pgfpathrectangle{\pgfqpoint{0.539299in}{0.078740in}}{\pgfqpoint{7.842520in}{7.842520in}}%
\pgfusepath{clip}%
\pgfsetbuttcap%
\pgfsetroundjoin%
\definecolor{currentfill}{rgb}{0.282327,0.094955,0.417331}%
\pgfsetfillcolor{currentfill}%
\pgfsetlinewidth{0.000000pt}%
\definecolor{currentstroke}{rgb}{0.277018,0.050344,0.375715}%
\pgfsetstrokecolor{currentstroke}%
\pgfsetdash{}{0pt}%
\pgfpathmoveto{\pgfqpoint{6.903908in}{2.494991in}}%
\pgfpathlineto{\pgfqpoint{6.763891in}{2.602375in}}%
\pgfpathlineto{\pgfqpoint{6.690065in}{2.529737in}}%
\pgfpathclose%
\pgfusepath{fill}%
\end{pgfscope}%
\begin{pgfscope}%
\pgfpathrectangle{\pgfqpoint{0.539299in}{0.078740in}}{\pgfqpoint{7.842520in}{7.842520in}}%
\pgfusepath{clip}%
\pgfsetbuttcap%
\pgfsetroundjoin%
\definecolor{currentfill}{rgb}{0.125394,0.574318,0.549086}%
\pgfsetfillcolor{currentfill}%
\pgfsetlinewidth{0.000000pt}%
\definecolor{currentstroke}{rgb}{0.277941,0.056324,0.381191}%
\pgfsetstrokecolor{currentstroke}%
\pgfsetdash{}{0pt}%
\pgfpathmoveto{\pgfqpoint{4.667716in}{4.230148in}}%
\pgfpathlineto{\pgfqpoint{4.529493in}{4.415381in}}%
\pgfpathlineto{\pgfqpoint{4.449060in}{4.521490in}}%
\pgfpathclose%
\pgfusepath{fill}%
\end{pgfscope}%
\begin{pgfscope}%
\pgfpathrectangle{\pgfqpoint{0.539299in}{0.078740in}}{\pgfqpoint{7.842520in}{7.842520in}}%
\pgfusepath{clip}%
\pgfsetbuttcap%
\pgfsetroundjoin%
\definecolor{currentfill}{rgb}{0.283229,0.120777,0.440584}%
\pgfsetfillcolor{currentfill}%
\pgfsetlinewidth{0.000000pt}%
\definecolor{currentstroke}{rgb}{0.278791,0.062145,0.386592}%
\pgfsetstrokecolor{currentstroke}%
\pgfsetdash{}{0pt}%
\pgfpathmoveto{\pgfqpoint{6.690065in}{2.529737in}}%
\pgfpathlineto{\pgfqpoint{6.623869in}{2.702268in}}%
\pgfpathlineto{\pgfqpoint{6.549730in}{2.630860in}}%
\pgfpathclose%
\pgfusepath{fill}%
\end{pgfscope}%
\begin{pgfscope}%
\pgfpathrectangle{\pgfqpoint{0.539299in}{0.078740in}}{\pgfqpoint{7.842520in}{7.842520in}}%
\pgfusepath{clip}%
\pgfsetbuttcap%
\pgfsetroundjoin%
\definecolor{currentfill}{rgb}{0.119738,0.603785,0.541400}%
\pgfsetfillcolor{currentfill}%
\pgfsetlinewidth{0.000000pt}%
\definecolor{currentstroke}{rgb}{0.279566,0.067836,0.391917}%
\pgfsetstrokecolor{currentstroke}%
\pgfsetdash{}{0pt}%
\pgfpathmoveto{\pgfqpoint{4.449060in}{4.521490in}}%
\pgfpathlineto{\pgfqpoint{4.529493in}{4.415381in}}%
\pgfpathlineto{\pgfqpoint{4.391174in}{4.600585in}}%
\pgfpathclose%
\pgfusepath{fill}%
\end{pgfscope}%
\begin{pgfscope}%
\pgfpathrectangle{\pgfqpoint{0.539299in}{0.078740in}}{\pgfqpoint{7.842520in}{7.842520in}}%
\pgfusepath{clip}%
\pgfsetbuttcap%
\pgfsetroundjoin%
\definecolor{currentfill}{rgb}{0.319809,0.770914,0.411152}%
\pgfsetfillcolor{currentfill}%
\pgfsetlinewidth{0.000000pt}%
\definecolor{currentstroke}{rgb}{0.280267,0.073417,0.397163}%
\pgfsetstrokecolor{currentstroke}%
\pgfsetdash{}{0pt}%
\pgfpathmoveto{\pgfqpoint{3.126635in}{5.241459in}}%
\pgfpathlineto{\pgfqpoint{3.040850in}{5.259550in}}%
\pgfpathlineto{\pgfqpoint{3.259589in}{5.378978in}}%
\pgfpathclose%
\pgfusepath{fill}%
\end{pgfscope}%
\begin{pgfscope}%
\pgfpathrectangle{\pgfqpoint{0.539299in}{0.078740in}}{\pgfqpoint{7.842520in}{7.842520in}}%
\pgfusepath{clip}%
\pgfsetbuttcap%
\pgfsetroundjoin%
\definecolor{currentfill}{rgb}{0.276194,0.190074,0.493001}%
\pgfsetfillcolor{currentfill}%
\pgfsetlinewidth{0.000000pt}%
\definecolor{currentstroke}{rgb}{0.280894,0.078907,0.402329}%
\pgfsetstrokecolor{currentstroke}%
\pgfsetdash{}{0pt}%
\pgfpathmoveto{\pgfqpoint{6.054321in}{2.852638in}}%
\pgfpathlineto{\pgfqpoint{6.194335in}{2.753858in}}%
\pgfpathlineto{\pgfqpoint{6.129597in}{2.909848in}}%
\pgfpathclose%
\pgfusepath{fill}%
\end{pgfscope}%
\begin{pgfscope}%
\pgfpathrectangle{\pgfqpoint{0.539299in}{0.078740in}}{\pgfqpoint{7.842520in}{7.842520in}}%
\pgfusepath{clip}%
\pgfsetbuttcap%
\pgfsetroundjoin%
\definecolor{currentfill}{rgb}{0.266941,0.748751,0.440573}%
\pgfsetfillcolor{currentfill}%
\pgfsetlinewidth{0.000000pt}%
\definecolor{currentstroke}{rgb}{0.281446,0.084320,0.407414}%
\pgfsetstrokecolor{currentstroke}%
\pgfsetdash{}{0pt}%
\pgfpathmoveto{\pgfqpoint{3.838249in}{5.208623in}}%
\pgfpathlineto{\pgfqpoint{3.892829in}{5.198517in}}%
\pgfpathlineto{\pgfqpoint{3.976153in}{5.092524in}}%
\pgfpathclose%
\pgfusepath{fill}%
\end{pgfscope}%
\begin{pgfscope}%
\pgfpathrectangle{\pgfqpoint{0.539299in}{0.078740in}}{\pgfqpoint{7.842520in}{7.842520in}}%
\pgfusepath{clip}%
\pgfsetbuttcap%
\pgfsetroundjoin%
\definecolor{currentfill}{rgb}{0.253935,0.265254,0.529983}%
\pgfsetfillcolor{currentfill}%
\pgfsetlinewidth{0.000000pt}%
\definecolor{currentstroke}{rgb}{0.281924,0.089666,0.412415}%
\pgfsetstrokecolor{currentstroke}%
\pgfsetdash{}{0pt}%
\pgfpathmoveto{\pgfqpoint{5.699184in}{3.033538in}}%
\pgfpathlineto{\pgfqpoint{5.775167in}{3.059358in}}%
\pgfpathlineto{\pgfqpoint{5.636047in}{3.171258in}}%
\pgfpathclose%
\pgfusepath{fill}%
\end{pgfscope}%
\begin{pgfscope}%
\pgfpathrectangle{\pgfqpoint{0.539299in}{0.078740in}}{\pgfqpoint{7.842520in}{7.842520in}}%
\pgfusepath{clip}%
\pgfsetbuttcap%
\pgfsetroundjoin%
\definecolor{currentfill}{rgb}{0.185556,0.418570,0.556753}%
\pgfsetfillcolor{currentfill}%
\pgfsetlinewidth{0.000000pt}%
\definecolor{currentstroke}{rgb}{0.282327,0.094955,0.417331}%
\pgfsetstrokecolor{currentstroke}%
\pgfsetdash{}{0pt}%
\pgfpathmoveto{\pgfqpoint{5.143030in}{3.602315in}}%
\pgfpathlineto{\pgfqpoint{5.082079in}{3.713504in}}%
\pgfpathlineto{\pgfqpoint{5.004323in}{3.770773in}}%
\pgfpathclose%
\pgfusepath{fill}%
\end{pgfscope}%
\begin{pgfscope}%
\pgfpathrectangle{\pgfqpoint{0.539299in}{0.078740in}}{\pgfqpoint{7.842520in}{7.842520in}}%
\pgfusepath{clip}%
\pgfsetbuttcap%
\pgfsetroundjoin%
\definecolor{currentfill}{rgb}{0.280868,0.160771,0.472899}%
\pgfsetfillcolor{currentfill}%
\pgfsetlinewidth{0.000000pt}%
\definecolor{currentstroke}{rgb}{0.282656,0.100196,0.422160}%
\pgfsetstrokecolor{currentstroke}%
\pgfsetdash{}{0pt}%
\pgfpathmoveto{\pgfqpoint{6.269433in}{2.819189in}}%
\pgfpathlineto{\pgfqpoint{6.334603in}{2.655498in}}%
\pgfpathlineto{\pgfqpoint{6.409497in}{2.726792in}}%
\pgfpathclose%
\pgfusepath{fill}%
\end{pgfscope}%
\begin{pgfscope}%
\pgfpathrectangle{\pgfqpoint{0.539299in}{0.078740in}}{\pgfqpoint{7.842520in}{7.842520in}}%
\pgfusepath{clip}%
\pgfsetbuttcap%
\pgfsetroundjoin%
\definecolor{currentfill}{rgb}{0.206756,0.371758,0.553117}%
\pgfsetfillcolor{currentfill}%
\pgfsetlinewidth{0.000000pt}%
\definecolor{currentstroke}{rgb}{0.282910,0.105393,0.426902}%
\pgfsetstrokecolor{currentstroke}%
\pgfsetdash{}{0pt}%
\pgfpathmoveto{\pgfqpoint{5.281811in}{3.444596in}}%
\pgfpathlineto{\pgfqpoint{5.358646in}{3.421072in}}%
\pgfpathlineto{\pgfqpoint{5.143030in}{3.602315in}}%
\pgfpathclose%
\pgfusepath{fill}%
\end{pgfscope}%
\begin{pgfscope}%
\pgfpathrectangle{\pgfqpoint{0.539299in}{0.078740in}}{\pgfqpoint{7.842520in}{7.842520in}}%
\pgfusepath{clip}%
\pgfsetbuttcap%
\pgfsetroundjoin%
\definecolor{currentfill}{rgb}{0.174274,0.445044,0.557792}%
\pgfsetfillcolor{currentfill}%
\pgfsetlinewidth{0.000000pt}%
\definecolor{currentstroke}{rgb}{0.283091,0.110553,0.431554}%
\pgfsetstrokecolor{currentstroke}%
\pgfsetdash{}{0pt}%
\pgfpathmoveto{\pgfqpoint{4.865625in}{3.949141in}}%
\pgfpathlineto{\pgfqpoint{5.004323in}{3.770773in}}%
\pgfpathlineto{\pgfqpoint{5.082079in}{3.713504in}}%
\pgfpathclose%
\pgfusepath{fill}%
\end{pgfscope}%
\begin{pgfscope}%
\pgfpathrectangle{\pgfqpoint{0.539299in}{0.078740in}}{\pgfqpoint{7.842520in}{7.842520in}}%
\pgfusepath{clip}%
\pgfsetbuttcap%
\pgfsetroundjoin%
\definecolor{currentfill}{rgb}{0.279566,0.067836,0.391917}%
\pgfsetfillcolor{currentfill}%
\pgfsetlinewidth{0.000000pt}%
\definecolor{currentstroke}{rgb}{0.283197,0.115680,0.436115}%
\pgfsetstrokecolor{currentstroke}%
\pgfsetdash{}{0pt}%
\pgfpathmoveto{\pgfqpoint{7.043838in}{2.379025in}}%
\pgfpathlineto{\pgfqpoint{6.903908in}{2.494991in}}%
\pgfpathlineto{\pgfqpoint{6.830421in}{2.421820in}}%
\pgfpathclose%
\pgfusepath{fill}%
\end{pgfscope}%
\begin{pgfscope}%
\pgfpathrectangle{\pgfqpoint{0.539299in}{0.078740in}}{\pgfqpoint{7.842520in}{7.842520in}}%
\pgfusepath{clip}%
\pgfsetbuttcap%
\pgfsetroundjoin%
\definecolor{currentfill}{rgb}{0.137339,0.662252,0.515571}%
\pgfsetfillcolor{currentfill}%
\pgfsetlinewidth{0.000000pt}%
\definecolor{currentstroke}{rgb}{0.283229,0.120777,0.440584}%
\pgfsetstrokecolor{currentstroke}%
\pgfsetdash{}{0pt}%
\pgfpathmoveto{\pgfqpoint{4.391174in}{4.600585in}}%
\pgfpathlineto{\pgfqpoint{4.252784in}{4.779999in}}%
\pgfpathlineto{\pgfqpoint{4.170832in}{4.893238in}}%
\pgfpathclose%
\pgfusepath{fill}%
\end{pgfscope}%
\begin{pgfscope}%
\pgfpathrectangle{\pgfqpoint{0.539299in}{0.078740in}}{\pgfqpoint{7.842520in}{7.842520in}}%
\pgfusepath{clip}%
\pgfsetbuttcap%
\pgfsetroundjoin%
\definecolor{currentfill}{rgb}{0.232815,0.732247,0.459277}%
\pgfsetfillcolor{currentfill}%
\pgfsetlinewidth{0.000000pt}%
\definecolor{currentstroke}{rgb}{0.283187,0.125848,0.444960}%
\pgfsetstrokecolor{currentstroke}%
\pgfsetdash{}{0pt}%
\pgfpathmoveto{\pgfqpoint{3.976153in}{5.092524in}}%
\pgfpathlineto{\pgfqpoint{3.892829in}{5.198517in}}%
\pgfpathlineto{\pgfqpoint{4.114399in}{4.946663in}}%
\pgfpathclose%
\pgfusepath{fill}%
\end{pgfscope}%
\begin{pgfscope}%
\pgfpathrectangle{\pgfqpoint{0.539299in}{0.078740in}}{\pgfqpoint{7.842520in}{7.842520in}}%
\pgfusepath{clip}%
\pgfsetbuttcap%
\pgfsetroundjoin%
\definecolor{currentfill}{rgb}{0.175707,0.697900,0.491033}%
\pgfsetfillcolor{currentfill}%
\pgfsetlinewidth{0.000000pt}%
\definecolor{currentstroke}{rgb}{0.283072,0.130895,0.449241}%
\pgfsetstrokecolor{currentstroke}%
\pgfsetdash{}{0pt}%
\pgfpathmoveto{\pgfqpoint{4.252784in}{4.779999in}}%
\pgfpathlineto{\pgfqpoint{4.114399in}{4.946663in}}%
\pgfpathlineto{\pgfqpoint{4.031728in}{5.058097in}}%
\pgfpathclose%
\pgfusepath{fill}%
\end{pgfscope}%
\begin{pgfscope}%
\pgfpathrectangle{\pgfqpoint{0.539299in}{0.078740in}}{\pgfqpoint{7.842520in}{7.842520in}}%
\pgfusepath{clip}%
\pgfsetbuttcap%
\pgfsetroundjoin%
\definecolor{currentfill}{rgb}{0.223925,0.334994,0.548053}%
\pgfsetfillcolor{currentfill}%
\pgfsetlinewidth{0.000000pt}%
\definecolor{currentstroke}{rgb}{0.282884,0.135920,0.453427}%
\pgfsetstrokecolor{currentstroke}%
\pgfsetdash{}{0pt}%
\pgfpathmoveto{\pgfqpoint{5.281811in}{3.444596in}}%
\pgfpathlineto{\pgfqpoint{5.420728in}{3.297681in}}%
\pgfpathlineto{\pgfqpoint{5.497218in}{3.291318in}}%
\pgfpathclose%
\pgfusepath{fill}%
\end{pgfscope}%
\begin{pgfscope}%
\pgfpathrectangle{\pgfqpoint{0.539299in}{0.078740in}}{\pgfqpoint{7.842520in}{7.842520in}}%
\pgfusepath{clip}%
\pgfsetbuttcap%
\pgfsetroundjoin%
\definecolor{currentfill}{rgb}{0.344074,0.780029,0.397381}%
\pgfsetfillcolor{currentfill}%
\pgfsetlinewidth{0.000000pt}%
\definecolor{currentstroke}{rgb}{0.282623,0.140926,0.457517}%
\pgfsetstrokecolor{currentstroke}%
\pgfsetdash{}{0pt}%
\pgfpathmoveto{\pgfqpoint{3.616640in}{5.370253in}}%
\pgfpathlineto{\pgfqpoint{3.754366in}{5.305678in}}%
\pgfpathlineto{\pgfqpoint{3.700962in}{5.285351in}}%
\pgfpathclose%
\pgfusepath{fill}%
\end{pgfscope}%
\begin{pgfscope}%
\pgfpathrectangle{\pgfqpoint{0.539299in}{0.078740in}}{\pgfqpoint{7.842520in}{7.842520in}}%
\pgfusepath{clip}%
\pgfsetbuttcap%
\pgfsetroundjoin%
\definecolor{currentfill}{rgb}{0.260571,0.246922,0.522828}%
\pgfsetfillcolor{currentfill}%
\pgfsetlinewidth{0.000000pt}%
\definecolor{currentstroke}{rgb}{0.282290,0.145912,0.461510}%
\pgfsetstrokecolor{currentstroke}%
\pgfsetdash{}{0pt}%
\pgfpathmoveto{\pgfqpoint{5.775167in}{3.059358in}}%
\pgfpathlineto{\pgfqpoint{5.699184in}{3.033538in}}%
\pgfpathlineto{\pgfqpoint{5.914592in}{2.953804in}}%
\pgfpathclose%
\pgfusepath{fill}%
\end{pgfscope}%
\begin{pgfscope}%
\pgfpathrectangle{\pgfqpoint{0.539299in}{0.078740in}}{\pgfqpoint{7.842520in}{7.842520in}}%
\pgfusepath{clip}%
\pgfsetbuttcap%
\pgfsetroundjoin%
\definecolor{currentfill}{rgb}{0.151918,0.500685,0.557587}%
\pgfsetfillcolor{currentfill}%
\pgfsetlinewidth{0.000000pt}%
\definecolor{currentstroke}{rgb}{0.281887,0.150881,0.465405}%
\pgfsetstrokecolor{currentstroke}%
\pgfsetdash{}{0pt}%
\pgfpathmoveto{\pgfqpoint{4.805857in}{4.049409in}}%
\pgfpathlineto{\pgfqpoint{4.726876in}{4.135706in}}%
\pgfpathlineto{\pgfqpoint{4.865625in}{3.949141in}}%
\pgfpathclose%
\pgfusepath{fill}%
\end{pgfscope}%
\begin{pgfscope}%
\pgfpathrectangle{\pgfqpoint{0.539299in}{0.078740in}}{\pgfqpoint{7.842520in}{7.842520in}}%
\pgfusepath{clip}%
\pgfsetbuttcap%
\pgfsetroundjoin%
\definecolor{currentfill}{rgb}{0.192357,0.403199,0.555836}%
\pgfsetfillcolor{currentfill}%
\pgfsetlinewidth{0.000000pt}%
\definecolor{currentstroke}{rgb}{0.281412,0.155834,0.469201}%
\pgfsetstrokecolor{currentstroke}%
\pgfsetdash{}{0pt}%
\pgfpathmoveto{\pgfqpoint{2.062008in}{3.862537in}}%
\pgfpathlineto{\pgfqpoint{1.861534in}{3.249013in}}%
\pgfpathlineto{\pgfqpoint{1.975740in}{3.757494in}}%
\pgfpathclose%
\pgfusepath{fill}%
\end{pgfscope}%
\begin{pgfscope}%
\pgfpathrectangle{\pgfqpoint{0.539299in}{0.078740in}}{\pgfqpoint{7.842520in}{7.842520in}}%
\pgfusepath{clip}%
\pgfsetbuttcap%
\pgfsetroundjoin%
\definecolor{currentfill}{rgb}{0.235526,0.309527,0.542944}%
\pgfsetfillcolor{currentfill}%
\pgfsetlinewidth{0.000000pt}%
\definecolor{currentstroke}{rgb}{0.280868,0.160771,0.472899}%
\pgfsetstrokecolor{currentstroke}%
\pgfsetdash{}{0pt}%
\pgfpathmoveto{\pgfqpoint{5.497218in}{3.291318in}}%
\pgfpathlineto{\pgfqpoint{5.420728in}{3.297681in}}%
\pgfpathlineto{\pgfqpoint{5.559838in}{3.161010in}}%
\pgfpathclose%
\pgfusepath{fill}%
\end{pgfscope}%
\begin{pgfscope}%
\pgfpathrectangle{\pgfqpoint{0.539299in}{0.078740in}}{\pgfqpoint{7.842520in}{7.842520in}}%
\pgfusepath{clip}%
\pgfsetbuttcap%
\pgfsetroundjoin%
\definecolor{currentfill}{rgb}{0.281446,0.084320,0.407414}%
\pgfsetfillcolor{currentfill}%
\pgfsetlinewidth{0.000000pt}%
\definecolor{currentstroke}{rgb}{0.280255,0.165693,0.476498}%
\pgfsetstrokecolor{currentstroke}%
\pgfsetdash{}{0pt}%
\pgfpathmoveto{\pgfqpoint{6.830421in}{2.421820in}}%
\pgfpathlineto{\pgfqpoint{6.903908in}{2.494991in}}%
\pgfpathlineto{\pgfqpoint{6.690065in}{2.529737in}}%
\pgfpathclose%
\pgfusepath{fill}%
\end{pgfscope}%
\begin{pgfscope}%
\pgfpathrectangle{\pgfqpoint{0.539299in}{0.078740in}}{\pgfqpoint{7.842520in}{7.842520in}}%
\pgfusepath{clip}%
\pgfsetbuttcap%
\pgfsetroundjoin%
\definecolor{currentfill}{rgb}{0.282290,0.145912,0.461510}%
\pgfsetfillcolor{currentfill}%
\pgfsetlinewidth{0.000000pt}%
\definecolor{currentstroke}{rgb}{0.279574,0.170599,0.479997}%
\pgfsetstrokecolor{currentstroke}%
\pgfsetdash{}{0pt}%
\pgfpathmoveto{\pgfqpoint{6.549730in}{2.630860in}}%
\pgfpathlineto{\pgfqpoint{6.409497in}{2.726792in}}%
\pgfpathlineto{\pgfqpoint{6.334603in}{2.655498in}}%
\pgfpathclose%
\pgfusepath{fill}%
\end{pgfscope}%
\begin{pgfscope}%
\pgfpathrectangle{\pgfqpoint{0.539299in}{0.078740in}}{\pgfqpoint{7.842520in}{7.842520in}}%
\pgfusepath{clip}%
\pgfsetbuttcap%
\pgfsetroundjoin%
\definecolor{currentfill}{rgb}{0.140536,0.530132,0.555659}%
\pgfsetfillcolor{currentfill}%
\pgfsetlinewidth{0.000000pt}%
\definecolor{currentstroke}{rgb}{0.278826,0.175490,0.483397}%
\pgfsetstrokecolor{currentstroke}%
\pgfsetdash{}{0pt}%
\pgfpathmoveto{\pgfqpoint{4.588029in}{4.327753in}}%
\pgfpathlineto{\pgfqpoint{4.726876in}{4.135706in}}%
\pgfpathlineto{\pgfqpoint{4.805857in}{4.049409in}}%
\pgfpathclose%
\pgfusepath{fill}%
\end{pgfscope}%
\begin{pgfscope}%
\pgfpathrectangle{\pgfqpoint{0.539299in}{0.078740in}}{\pgfqpoint{7.842520in}{7.842520in}}%
\pgfusepath{clip}%
\pgfsetbuttcap%
\pgfsetroundjoin%
\definecolor{currentfill}{rgb}{0.279574,0.170599,0.479997}%
\pgfsetfillcolor{currentfill}%
\pgfsetlinewidth{0.000000pt}%
\definecolor{currentstroke}{rgb}{0.278012,0.180367,0.486697}%
\pgfsetstrokecolor{currentstroke}%
\pgfsetdash{}{0pt}%
\pgfpathmoveto{\pgfqpoint{6.269433in}{2.819189in}}%
\pgfpathlineto{\pgfqpoint{6.194335in}{2.753858in}}%
\pgfpathlineto{\pgfqpoint{6.334603in}{2.655498in}}%
\pgfpathclose%
\pgfusepath{fill}%
\end{pgfscope}%
\begin{pgfscope}%
\pgfpathrectangle{\pgfqpoint{0.539299in}{0.078740in}}{\pgfqpoint{7.842520in}{7.842520in}}%
\pgfusepath{clip}%
\pgfsetbuttcap%
\pgfsetroundjoin%
\definecolor{currentfill}{rgb}{0.248629,0.278775,0.534556}%
\pgfsetfillcolor{currentfill}%
\pgfsetlinewidth{0.000000pt}%
\definecolor{currentstroke}{rgb}{0.277134,0.185228,0.489898}%
\pgfsetstrokecolor{currentstroke}%
\pgfsetdash{}{0pt}%
\pgfpathmoveto{\pgfqpoint{5.636047in}{3.171258in}}%
\pgfpathlineto{\pgfqpoint{5.559838in}{3.161010in}}%
\pgfpathlineto{\pgfqpoint{5.699184in}{3.033538in}}%
\pgfpathclose%
\pgfusepath{fill}%
\end{pgfscope}%
\begin{pgfscope}%
\pgfpathrectangle{\pgfqpoint{0.539299in}{0.078740in}}{\pgfqpoint{7.842520in}{7.842520in}}%
\pgfusepath{clip}%
\pgfsetbuttcap%
\pgfsetroundjoin%
\definecolor{currentfill}{rgb}{0.124395,0.578002,0.548287}%
\pgfsetfillcolor{currentfill}%
\pgfsetlinewidth{0.000000pt}%
\definecolor{currentstroke}{rgb}{0.276194,0.190074,0.493001}%
\pgfsetstrokecolor{currentstroke}%
\pgfsetdash{}{0pt}%
\pgfpathmoveto{\pgfqpoint{4.449060in}{4.521490in}}%
\pgfpathlineto{\pgfqpoint{4.588029in}{4.327753in}}%
\pgfpathlineto{\pgfqpoint{4.667716in}{4.230148in}}%
\pgfpathclose%
\pgfusepath{fill}%
\end{pgfscope}%
\begin{pgfscope}%
\pgfpathrectangle{\pgfqpoint{0.539299in}{0.078740in}}{\pgfqpoint{7.842520in}{7.842520in}}%
\pgfusepath{clip}%
\pgfsetbuttcap%
\pgfsetroundjoin%
\definecolor{currentfill}{rgb}{0.141935,0.526453,0.555991}%
\pgfsetfillcolor{currentfill}%
\pgfsetlinewidth{0.000000pt}%
\definecolor{currentstroke}{rgb}{0.275191,0.194905,0.496005}%
\pgfsetstrokecolor{currentstroke}%
\pgfsetdash{}{0pt}%
\pgfpathmoveto{\pgfqpoint{2.234531in}{4.033237in}}%
\pgfpathlineto{\pgfqpoint{2.148306in}{3.953837in}}%
\pgfpathlineto{\pgfqpoint{2.267756in}{4.403873in}}%
\pgfpathclose%
\pgfusepath{fill}%
\end{pgfscope}%
\begin{pgfscope}%
\pgfpathrectangle{\pgfqpoint{0.539299in}{0.078740in}}{\pgfqpoint{7.842520in}{7.842520in}}%
\pgfusepath{clip}%
\pgfsetbuttcap%
\pgfsetroundjoin%
\definecolor{currentfill}{rgb}{0.277941,0.056324,0.381191}%
\pgfsetfillcolor{currentfill}%
\pgfsetlinewidth{0.000000pt}%
\definecolor{currentstroke}{rgb}{0.274128,0.199721,0.498911}%
\pgfsetstrokecolor{currentstroke}%
\pgfsetdash{}{0pt}%
\pgfpathmoveto{\pgfqpoint{6.830421in}{2.421820in}}%
\pgfpathlineto{\pgfqpoint{6.970702in}{2.305402in}}%
\pgfpathlineto{\pgfqpoint{7.043838in}{2.379025in}}%
\pgfpathclose%
\pgfusepath{fill}%
\end{pgfscope}%
\begin{pgfscope}%
\pgfpathrectangle{\pgfqpoint{0.539299in}{0.078740in}}{\pgfqpoint{7.842520in}{7.842520in}}%
\pgfusepath{clip}%
\pgfsetbuttcap%
\pgfsetroundjoin%
\definecolor{currentfill}{rgb}{0.124395,0.578002,0.548287}%
\pgfsetfillcolor{currentfill}%
\pgfsetlinewidth{0.000000pt}%
\definecolor{currentstroke}{rgb}{0.273006,0.204520,0.501721}%
\pgfsetstrokecolor{currentstroke}%
\pgfsetdash{}{0pt}%
\pgfpathmoveto{\pgfqpoint{2.440760in}{4.553021in}}%
\pgfpathlineto{\pgfqpoint{2.234531in}{4.033237in}}%
\pgfpathlineto{\pgfqpoint{2.354275in}{4.486045in}}%
\pgfpathclose%
\pgfusepath{fill}%
\end{pgfscope}%
\begin{pgfscope}%
\pgfpathrectangle{\pgfqpoint{0.539299in}{0.078740in}}{\pgfqpoint{7.842520in}{7.842520in}}%
\pgfusepath{clip}%
\pgfsetbuttcap%
\pgfsetroundjoin%
\definecolor{currentfill}{rgb}{0.271828,0.209303,0.504434}%
\pgfsetfillcolor{currentfill}%
\pgfsetlinewidth{0.000000pt}%
\definecolor{currentstroke}{rgb}{0.271828,0.209303,0.504434}%
\pgfsetstrokecolor{currentstroke}%
\pgfsetdash{}{0pt}%
\pgfpathmoveto{\pgfqpoint{5.914592in}{2.953804in}}%
\pgfpathlineto{\pgfqpoint{5.978689in}{2.800383in}}%
\pgfpathlineto{\pgfqpoint{6.054321in}{2.852638in}}%
\pgfpathclose%
\pgfusepath{fill}%
\end{pgfscope}%
\begin{pgfscope}%
\pgfpathrectangle{\pgfqpoint{0.539299in}{0.078740in}}{\pgfqpoint{7.842520in}{7.842520in}}%
\pgfusepath{clip}%
\pgfsetbuttcap%
\pgfsetroundjoin%
\definecolor{currentfill}{rgb}{0.311925,0.767822,0.415586}%
\pgfsetfillcolor{currentfill}%
\pgfsetlinewidth{0.000000pt}%
\definecolor{currentstroke}{rgb}{0.270595,0.214069,0.507052}%
\pgfsetstrokecolor{currentstroke}%
\pgfsetdash{}{0pt}%
\pgfpathmoveto{\pgfqpoint{3.838249in}{5.208623in}}%
\pgfpathlineto{\pgfqpoint{3.754366in}{5.305678in}}%
\pgfpathlineto{\pgfqpoint{3.892829in}{5.198517in}}%
\pgfpathclose%
\pgfusepath{fill}%
\end{pgfscope}%
\begin{pgfscope}%
\pgfpathrectangle{\pgfqpoint{0.539299in}{0.078740in}}{\pgfqpoint{7.842520in}{7.842520in}}%
\pgfusepath{clip}%
\pgfsetbuttcap%
\pgfsetroundjoin%
\definecolor{currentfill}{rgb}{0.283229,0.120777,0.440584}%
\pgfsetfillcolor{currentfill}%
\pgfsetlinewidth{0.000000pt}%
\definecolor{currentstroke}{rgb}{0.269308,0.218818,0.509577}%
\pgfsetstrokecolor{currentstroke}%
\pgfsetdash{}{0pt}%
\pgfpathmoveto{\pgfqpoint{6.549730in}{2.630860in}}%
\pgfpathlineto{\pgfqpoint{6.475077in}{2.555660in}}%
\pgfpathlineto{\pgfqpoint{6.690065in}{2.529737in}}%
\pgfpathclose%
\pgfusepath{fill}%
\end{pgfscope}%
\begin{pgfscope}%
\pgfpathrectangle{\pgfqpoint{0.539299in}{0.078740in}}{\pgfqpoint{7.842520in}{7.842520in}}%
\pgfusepath{clip}%
\pgfsetbuttcap%
\pgfsetroundjoin%
\definecolor{currentfill}{rgb}{0.123444,0.636809,0.528763}%
\pgfsetfillcolor{currentfill}%
\pgfsetlinewidth{0.000000pt}%
\definecolor{currentstroke}{rgb}{0.267968,0.223549,0.512008}%
\pgfsetstrokecolor{currentstroke}%
\pgfsetdash{}{0pt}%
\pgfpathmoveto{\pgfqpoint{4.391174in}{4.600585in}}%
\pgfpathlineto{\pgfqpoint{4.309978in}{4.711999in}}%
\pgfpathlineto{\pgfqpoint{4.449060in}{4.521490in}}%
\pgfpathclose%
\pgfusepath{fill}%
\end{pgfscope}%
\begin{pgfscope}%
\pgfpathrectangle{\pgfqpoint{0.539299in}{0.078740in}}{\pgfqpoint{7.842520in}{7.842520in}}%
\pgfusepath{clip}%
\pgfsetbuttcap%
\pgfsetroundjoin%
\definecolor{currentfill}{rgb}{0.404001,0.800275,0.362552}%
\pgfsetfillcolor{currentfill}%
\pgfsetlinewidth{0.000000pt}%
\definecolor{currentstroke}{rgb}{0.266580,0.228262,0.514349}%
\pgfsetstrokecolor{currentstroke}%
\pgfsetdash{}{0pt}%
\pgfpathmoveto{\pgfqpoint{3.480023in}{5.382736in}}%
\pgfpathlineto{\pgfqpoint{3.259589in}{5.378978in}}%
\pgfpathlineto{\pgfqpoint{3.394745in}{5.445102in}}%
\pgfpathclose%
\pgfusepath{fill}%
\end{pgfscope}%
\begin{pgfscope}%
\pgfpathrectangle{\pgfqpoint{0.539299in}{0.078740in}}{\pgfqpoint{7.842520in}{7.842520in}}%
\pgfusepath{clip}%
\pgfsetbuttcap%
\pgfsetroundjoin%
\definecolor{currentfill}{rgb}{0.262138,0.242286,0.520837}%
\pgfsetfillcolor{currentfill}%
\pgfsetlinewidth{0.000000pt}%
\definecolor{currentstroke}{rgb}{0.265145,0.232956,0.516599}%
\pgfsetstrokecolor{currentstroke}%
\pgfsetdash{}{0pt}%
\pgfpathmoveto{\pgfqpoint{5.914592in}{2.953804in}}%
\pgfpathlineto{\pgfqpoint{5.699184in}{3.033538in}}%
\pgfpathlineto{\pgfqpoint{5.838796in}{2.913871in}}%
\pgfpathclose%
\pgfusepath{fill}%
\end{pgfscope}%
\begin{pgfscope}%
\pgfpathrectangle{\pgfqpoint{0.539299in}{0.078740in}}{\pgfqpoint{7.842520in}{7.842520in}}%
\pgfusepath{clip}%
\pgfsetbuttcap%
\pgfsetroundjoin%
\definecolor{currentfill}{rgb}{0.288921,0.758394,0.428426}%
\pgfsetfillcolor{currentfill}%
\pgfsetlinewidth{0.000000pt}%
\definecolor{currentstroke}{rgb}{0.263663,0.237631,0.518762}%
\pgfsetstrokecolor{currentstroke}%
\pgfsetdash{}{0pt}%
\pgfpathmoveto{\pgfqpoint{2.825015in}{5.016747in}}%
\pgfpathlineto{\pgfqpoint{2.954612in}{5.267216in}}%
\pgfpathlineto{\pgfqpoint{3.040850in}{5.259550in}}%
\pgfpathclose%
\pgfusepath{fill}%
\end{pgfscope}%
\begin{pgfscope}%
\pgfpathrectangle{\pgfqpoint{0.539299in}{0.078740in}}{\pgfqpoint{7.842520in}{7.842520in}}%
\pgfusepath{clip}%
\pgfsetbuttcap%
\pgfsetroundjoin%
\definecolor{currentfill}{rgb}{0.140210,0.665859,0.513427}%
\pgfsetfillcolor{currentfill}%
\pgfsetlinewidth{0.000000pt}%
\definecolor{currentstroke}{rgb}{0.262138,0.242286,0.520837}%
\pgfsetstrokecolor{currentstroke}%
\pgfsetdash{}{0pt}%
\pgfpathmoveto{\pgfqpoint{4.170832in}{4.893238in}}%
\pgfpathlineto{\pgfqpoint{4.309978in}{4.711999in}}%
\pgfpathlineto{\pgfqpoint{4.391174in}{4.600585in}}%
\pgfpathclose%
\pgfusepath{fill}%
\end{pgfscope}%
\begin{pgfscope}%
\pgfpathrectangle{\pgfqpoint{0.539299in}{0.078740in}}{\pgfqpoint{7.842520in}{7.842520in}}%
\pgfusepath{clip}%
\pgfsetbuttcap%
\pgfsetroundjoin%
\definecolor{currentfill}{rgb}{0.275191,0.194905,0.496005}%
\pgfsetfillcolor{currentfill}%
\pgfsetlinewidth{0.000000pt}%
\definecolor{currentstroke}{rgb}{0.260571,0.246922,0.522828}%
\pgfsetstrokecolor{currentstroke}%
\pgfsetdash{}{0pt}%
\pgfpathmoveto{\pgfqpoint{5.978689in}{2.800383in}}%
\pgfpathlineto{\pgfqpoint{6.194335in}{2.753858in}}%
\pgfpathlineto{\pgfqpoint{6.054321in}{2.852638in}}%
\pgfpathclose%
\pgfusepath{fill}%
\end{pgfscope}%
\begin{pgfscope}%
\pgfpathrectangle{\pgfqpoint{0.539299in}{0.078740in}}{\pgfqpoint{7.842520in}{7.842520in}}%
\pgfusepath{clip}%
\pgfsetbuttcap%
\pgfsetroundjoin%
\definecolor{currentfill}{rgb}{0.282623,0.140926,0.457517}%
\pgfsetfillcolor{currentfill}%
\pgfsetlinewidth{0.000000pt}%
\definecolor{currentstroke}{rgb}{0.258965,0.251537,0.524736}%
\pgfsetstrokecolor{currentstroke}%
\pgfsetdash{}{0pt}%
\pgfpathmoveto{\pgfqpoint{6.334603in}{2.655498in}}%
\pgfpathlineto{\pgfqpoint{6.475077in}{2.555660in}}%
\pgfpathlineto{\pgfqpoint{6.549730in}{2.630860in}}%
\pgfpathclose%
\pgfusepath{fill}%
\end{pgfscope}%
\begin{pgfscope}%
\pgfpathrectangle{\pgfqpoint{0.539299in}{0.078740in}}{\pgfqpoint{7.842520in}{7.842520in}}%
\pgfusepath{clip}%
\pgfsetbuttcap%
\pgfsetroundjoin%
\definecolor{currentfill}{rgb}{0.246070,0.738910,0.452024}%
\pgfsetfillcolor{currentfill}%
\pgfsetlinewidth{0.000000pt}%
\definecolor{currentstroke}{rgb}{0.257322,0.256130,0.526563}%
\pgfsetstrokecolor{currentstroke}%
\pgfsetdash{}{0pt}%
\pgfpathmoveto{\pgfqpoint{4.114399in}{4.946663in}}%
\pgfpathlineto{\pgfqpoint{3.892829in}{5.198517in}}%
\pgfpathlineto{\pgfqpoint{4.031728in}{5.058097in}}%
\pgfpathclose%
\pgfusepath{fill}%
\end{pgfscope}%
\begin{pgfscope}%
\pgfpathrectangle{\pgfqpoint{0.539299in}{0.078740in}}{\pgfqpoint{7.842520in}{7.842520in}}%
\pgfusepath{clip}%
\pgfsetbuttcap%
\pgfsetroundjoin%
\definecolor{currentfill}{rgb}{0.231674,0.318106,0.544834}%
\pgfsetfillcolor{currentfill}%
\pgfsetlinewidth{0.000000pt}%
\definecolor{currentstroke}{rgb}{0.255645,0.260703,0.528312}%
\pgfsetstrokecolor{currentstroke}%
\pgfsetdash{}{0pt}%
\pgfpathmoveto{\pgfqpoint{1.775040in}{3.152116in}}%
\pgfpathlineto{\pgfqpoint{1.688478in}{3.045686in}}%
\pgfpathlineto{\pgfqpoint{1.803826in}{3.497087in}}%
\pgfpathclose%
\pgfusepath{fill}%
\end{pgfscope}%
\begin{pgfscope}%
\pgfpathrectangle{\pgfqpoint{0.539299in}{0.078740in}}{\pgfqpoint{7.842520in}{7.842520in}}%
\pgfusepath{clip}%
\pgfsetbuttcap%
\pgfsetroundjoin%
\definecolor{currentfill}{rgb}{0.185783,0.704891,0.485273}%
\pgfsetfillcolor{currentfill}%
\pgfsetlinewidth{0.000000pt}%
\definecolor{currentstroke}{rgb}{0.253935,0.265254,0.529983}%
\pgfsetstrokecolor{currentstroke}%
\pgfsetdash{}{0pt}%
\pgfpathmoveto{\pgfqpoint{4.031728in}{5.058097in}}%
\pgfpathlineto{\pgfqpoint{4.170832in}{4.893238in}}%
\pgfpathlineto{\pgfqpoint{4.252784in}{4.779999in}}%
\pgfpathclose%
\pgfusepath{fill}%
\end{pgfscope}%
\begin{pgfscope}%
\pgfpathrectangle{\pgfqpoint{0.539299in}{0.078740in}}{\pgfqpoint{7.842520in}{7.842520in}}%
\pgfusepath{clip}%
\pgfsetbuttcap%
\pgfsetroundjoin%
\definecolor{currentfill}{rgb}{0.267968,0.223549,0.512008}%
\pgfsetfillcolor{currentfill}%
\pgfsetlinewidth{0.000000pt}%
\definecolor{currentstroke}{rgb}{0.252194,0.269783,0.531579}%
\pgfsetstrokecolor{currentstroke}%
\pgfsetdash{}{0pt}%
\pgfpathmoveto{\pgfqpoint{5.838796in}{2.913871in}}%
\pgfpathlineto{\pgfqpoint{5.978689in}{2.800383in}}%
\pgfpathlineto{\pgfqpoint{5.914592in}{2.953804in}}%
\pgfpathclose%
\pgfusepath{fill}%
\end{pgfscope}%
\begin{pgfscope}%
\pgfpathrectangle{\pgfqpoint{0.539299in}{0.078740in}}{\pgfqpoint{7.842520in}{7.842520in}}%
\pgfusepath{clip}%
\pgfsetbuttcap%
\pgfsetroundjoin%
\definecolor{currentfill}{rgb}{0.412913,0.803041,0.357269}%
\pgfsetfillcolor{currentfill}%
\pgfsetlinewidth{0.000000pt}%
\definecolor{currentstroke}{rgb}{0.250425,0.274290,0.533103}%
\pgfsetstrokecolor{currentstroke}%
\pgfsetdash{}{0pt}%
\pgfpathmoveto{\pgfqpoint{3.480023in}{5.382736in}}%
\pgfpathlineto{\pgfqpoint{3.531606in}{5.447979in}}%
\pgfpathlineto{\pgfqpoint{3.616640in}{5.370253in}}%
\pgfpathclose%
\pgfusepath{fill}%
\end{pgfscope}%
\begin{pgfscope}%
\pgfpathrectangle{\pgfqpoint{0.539299in}{0.078740in}}{\pgfqpoint{7.842520in}{7.842520in}}%
\pgfusepath{clip}%
\pgfsetbuttcap%
\pgfsetroundjoin%
\definecolor{currentfill}{rgb}{0.386433,0.794644,0.372886}%
\pgfsetfillcolor{currentfill}%
\pgfsetlinewidth{0.000000pt}%
\definecolor{currentstroke}{rgb}{0.248629,0.278775,0.534556}%
\pgfsetstrokecolor{currentstroke}%
\pgfsetdash{}{0pt}%
\pgfpathmoveto{\pgfqpoint{3.259589in}{5.378978in}}%
\pgfpathlineto{\pgfqpoint{3.040850in}{5.259550in}}%
\pgfpathlineto{\pgfqpoint{3.173672in}{5.414731in}}%
\pgfpathclose%
\pgfusepath{fill}%
\end{pgfscope}%
\begin{pgfscope}%
\pgfpathrectangle{\pgfqpoint{0.539299in}{0.078740in}}{\pgfqpoint{7.842520in}{7.842520in}}%
\pgfusepath{clip}%
\pgfsetbuttcap%
\pgfsetroundjoin%
\definecolor{currentfill}{rgb}{0.137339,0.662252,0.515571}%
\pgfsetfillcolor{currentfill}%
\pgfsetlinewidth{0.000000pt}%
\definecolor{currentstroke}{rgb}{0.246811,0.283237,0.535941}%
\pgfsetstrokecolor{currentstroke}%
\pgfsetdash{}{0pt}%
\pgfpathmoveto{\pgfqpoint{2.652170in}{4.965747in}}%
\pgfpathlineto{\pgfqpoint{2.527103in}{4.606886in}}%
\pgfpathlineto{\pgfqpoint{2.440760in}{4.553021in}}%
\pgfpathclose%
\pgfusepath{fill}%
\end{pgfscope}%
\begin{pgfscope}%
\pgfpathrectangle{\pgfqpoint{0.539299in}{0.078740in}}{\pgfqpoint{7.842520in}{7.842520in}}%
\pgfusepath{clip}%
\pgfsetbuttcap%
\pgfsetroundjoin%
\definecolor{currentfill}{rgb}{0.281446,0.084320,0.407414}%
\pgfsetfillcolor{currentfill}%
\pgfsetlinewidth{0.000000pt}%
\definecolor{currentstroke}{rgb}{0.244972,0.287675,0.537260}%
\pgfsetstrokecolor{currentstroke}%
\pgfsetdash{}{0pt}%
\pgfpathmoveto{\pgfqpoint{6.690065in}{2.529737in}}%
\pgfpathlineto{\pgfqpoint{6.756390in}{2.344144in}}%
\pgfpathlineto{\pgfqpoint{6.830421in}{2.421820in}}%
\pgfpathclose%
\pgfusepath{fill}%
\end{pgfscope}%
\begin{pgfscope}%
\pgfpathrectangle{\pgfqpoint{0.539299in}{0.078740in}}{\pgfqpoint{7.842520in}{7.842520in}}%
\pgfusepath{clip}%
\pgfsetbuttcap%
\pgfsetroundjoin%
\definecolor{currentfill}{rgb}{0.283197,0.115680,0.436115}%
\pgfsetfillcolor{currentfill}%
\pgfsetlinewidth{0.000000pt}%
\definecolor{currentstroke}{rgb}{0.243113,0.292092,0.538516}%
\pgfsetstrokecolor{currentstroke}%
\pgfsetdash{}{0pt}%
\pgfpathmoveto{\pgfqpoint{6.475077in}{2.555660in}}%
\pgfpathlineto{\pgfqpoint{6.615697in}{2.452498in}}%
\pgfpathlineto{\pgfqpoint{6.690065in}{2.529737in}}%
\pgfpathclose%
\pgfusepath{fill}%
\end{pgfscope}%
\begin{pgfscope}%
\pgfpathrectangle{\pgfqpoint{0.539299in}{0.078740in}}{\pgfqpoint{7.842520in}{7.842520in}}%
\pgfusepath{clip}%
\pgfsetbuttcap%
\pgfsetroundjoin%
\definecolor{currentfill}{rgb}{0.175707,0.697900,0.491033}%
\pgfsetfillcolor{currentfill}%
\pgfsetlinewidth{0.000000pt}%
\definecolor{currentstroke}{rgb}{0.241237,0.296485,0.539709}%
\pgfsetstrokecolor{currentstroke}%
\pgfsetdash{}{0pt}%
\pgfpathmoveto{\pgfqpoint{2.738735in}{4.997476in}}%
\pgfpathlineto{\pgfqpoint{2.527103in}{4.606886in}}%
\pgfpathlineto{\pgfqpoint{2.652170in}{4.965747in}}%
\pgfpathclose%
\pgfusepath{fill}%
\end{pgfscope}%
\begin{pgfscope}%
\pgfpathrectangle{\pgfqpoint{0.539299in}{0.078740in}}{\pgfqpoint{7.842520in}{7.842520in}}%
\pgfusepath{clip}%
\pgfsetbuttcap%
\pgfsetroundjoin%
\definecolor{currentfill}{rgb}{0.197636,0.391528,0.554969}%
\pgfsetfillcolor{currentfill}%
\pgfsetlinewidth{0.000000pt}%
\definecolor{currentstroke}{rgb}{0.239346,0.300855,0.540844}%
\pgfsetstrokecolor{currentstroke}%
\pgfsetdash{}{0pt}%
\pgfpathmoveto{\pgfqpoint{1.775040in}{3.152116in}}%
\pgfpathlineto{\pgfqpoint{1.889628in}{3.636546in}}%
\pgfpathlineto{\pgfqpoint{1.975740in}{3.757494in}}%
\pgfpathclose%
\pgfusepath{fill}%
\end{pgfscope}%
\begin{pgfscope}%
\pgfpathrectangle{\pgfqpoint{0.539299in}{0.078740in}}{\pgfqpoint{7.842520in}{7.842520in}}%
\pgfusepath{clip}%
\pgfsetbuttcap%
\pgfsetroundjoin%
\definecolor{currentfill}{rgb}{0.197636,0.391528,0.554969}%
\pgfsetfillcolor{currentfill}%
\pgfsetlinewidth{0.000000pt}%
\definecolor{currentstroke}{rgb}{0.237441,0.305202,0.541921}%
\pgfsetstrokecolor{currentstroke}%
\pgfsetdash{}{0pt}%
\pgfpathmoveto{\pgfqpoint{5.143030in}{3.602315in}}%
\pgfpathlineto{\pgfqpoint{5.204442in}{3.480326in}}%
\pgfpathlineto{\pgfqpoint{5.281811in}{3.444596in}}%
\pgfpathclose%
\pgfusepath{fill}%
\end{pgfscope}%
\begin{pgfscope}%
\pgfpathrectangle{\pgfqpoint{0.539299in}{0.078740in}}{\pgfqpoint{7.842520in}{7.842520in}}%
\pgfusepath{clip}%
\pgfsetbuttcap%
\pgfsetroundjoin%
\definecolor{currentfill}{rgb}{0.280868,0.160771,0.472899}%
\pgfsetfillcolor{currentfill}%
\pgfsetlinewidth{0.000000pt}%
\definecolor{currentstroke}{rgb}{0.235526,0.309527,0.542944}%
\pgfsetstrokecolor{currentstroke}%
\pgfsetdash{}{0pt}%
\pgfpathmoveto{\pgfqpoint{6.259304in}{2.584902in}}%
\pgfpathlineto{\pgfqpoint{6.334603in}{2.655498in}}%
\pgfpathlineto{\pgfqpoint{6.194335in}{2.753858in}}%
\pgfpathclose%
\pgfusepath{fill}%
\end{pgfscope}%
\begin{pgfscope}%
\pgfpathrectangle{\pgfqpoint{0.539299in}{0.078740in}}{\pgfqpoint{7.842520in}{7.842520in}}%
\pgfusepath{clip}%
\pgfsetbuttcap%
\pgfsetroundjoin%
\definecolor{currentfill}{rgb}{0.208623,0.367752,0.552675}%
\pgfsetfillcolor{currentfill}%
\pgfsetlinewidth{0.000000pt}%
\definecolor{currentstroke}{rgb}{0.233603,0.313828,0.543914}%
\pgfsetstrokecolor{currentstroke}%
\pgfsetdash{}{0pt}%
\pgfpathmoveto{\pgfqpoint{5.204442in}{3.480326in}}%
\pgfpathlineto{\pgfqpoint{5.420728in}{3.297681in}}%
\pgfpathlineto{\pgfqpoint{5.281811in}{3.444596in}}%
\pgfpathclose%
\pgfusepath{fill}%
\end{pgfscope}%
\begin{pgfscope}%
\pgfpathrectangle{\pgfqpoint{0.539299in}{0.078740in}}{\pgfqpoint{7.842520in}{7.842520in}}%
\pgfusepath{clip}%
\pgfsetbuttcap%
\pgfsetroundjoin%
\definecolor{currentfill}{rgb}{0.174274,0.445044,0.557792}%
\pgfsetfillcolor{currentfill}%
\pgfsetlinewidth{0.000000pt}%
\definecolor{currentstroke}{rgb}{0.231674,0.318106,0.544834}%
\pgfsetstrokecolor{currentstroke}%
\pgfsetdash{}{0pt}%
\pgfpathmoveto{\pgfqpoint{4.925899in}{3.838472in}}%
\pgfpathlineto{\pgfqpoint{5.143030in}{3.602315in}}%
\pgfpathlineto{\pgfqpoint{5.004323in}{3.770773in}}%
\pgfpathclose%
\pgfusepath{fill}%
\end{pgfscope}%
\begin{pgfscope}%
\pgfpathrectangle{\pgfqpoint{0.539299in}{0.078740in}}{\pgfqpoint{7.842520in}{7.842520in}}%
\pgfusepath{clip}%
\pgfsetbuttcap%
\pgfsetroundjoin%
\definecolor{currentfill}{rgb}{0.125394,0.574318,0.549086}%
\pgfsetfillcolor{currentfill}%
\pgfsetlinewidth{0.000000pt}%
\definecolor{currentstroke}{rgb}{0.229739,0.322361,0.545706}%
\pgfsetstrokecolor{currentstroke}%
\pgfsetdash{}{0pt}%
\pgfpathmoveto{\pgfqpoint{2.267756in}{4.403873in}}%
\pgfpathlineto{\pgfqpoint{2.354275in}{4.486045in}}%
\pgfpathlineto{\pgfqpoint{2.234531in}{4.033237in}}%
\pgfpathclose%
\pgfusepath{fill}%
\end{pgfscope}%
\begin{pgfscope}%
\pgfpathrectangle{\pgfqpoint{0.539299in}{0.078740in}}{\pgfqpoint{7.842520in}{7.842520in}}%
\pgfusepath{clip}%
\pgfsetbuttcap%
\pgfsetroundjoin%
\definecolor{currentfill}{rgb}{0.276194,0.190074,0.493001}%
\pgfsetfillcolor{currentfill}%
\pgfsetlinewidth{0.000000pt}%
\definecolor{currentstroke}{rgb}{0.227802,0.326594,0.546532}%
\pgfsetstrokecolor{currentstroke}%
\pgfsetdash{}{0pt}%
\pgfpathmoveto{\pgfqpoint{6.118863in}{2.691325in}}%
\pgfpathlineto{\pgfqpoint{6.194335in}{2.753858in}}%
\pgfpathlineto{\pgfqpoint{5.978689in}{2.800383in}}%
\pgfpathclose%
\pgfusepath{fill}%
\end{pgfscope}%
\begin{pgfscope}%
\pgfpathrectangle{\pgfqpoint{0.539299in}{0.078740in}}{\pgfqpoint{7.842520in}{7.842520in}}%
\pgfusepath{clip}%
\pgfsetbuttcap%
\pgfsetroundjoin%
\definecolor{currentfill}{rgb}{0.277941,0.056324,0.381191}%
\pgfsetfillcolor{currentfill}%
\pgfsetlinewidth{0.000000pt}%
\definecolor{currentstroke}{rgb}{0.225863,0.330805,0.547314}%
\pgfsetstrokecolor{currentstroke}%
\pgfsetdash{}{0pt}%
\pgfpathmoveto{\pgfqpoint{6.897057in}{2.228556in}}%
\pgfpathlineto{\pgfqpoint{6.970702in}{2.305402in}}%
\pgfpathlineto{\pgfqpoint{6.830421in}{2.421820in}}%
\pgfpathclose%
\pgfusepath{fill}%
\end{pgfscope}%
\begin{pgfscope}%
\pgfpathrectangle{\pgfqpoint{0.539299in}{0.078740in}}{\pgfqpoint{7.842520in}{7.842520in}}%
\pgfusepath{clip}%
\pgfsetbuttcap%
\pgfsetroundjoin%
\definecolor{currentfill}{rgb}{0.231674,0.318106,0.544834}%
\pgfsetfillcolor{currentfill}%
\pgfsetlinewidth{0.000000pt}%
\definecolor{currentstroke}{rgb}{0.223925,0.334994,0.548053}%
\pgfsetstrokecolor{currentstroke}%
\pgfsetdash{}{0pt}%
\pgfpathmoveto{\pgfqpoint{5.483218in}{3.162772in}}%
\pgfpathlineto{\pgfqpoint{5.559838in}{3.161010in}}%
\pgfpathlineto{\pgfqpoint{5.420728in}{3.297681in}}%
\pgfpathclose%
\pgfusepath{fill}%
\end{pgfscope}%
\begin{pgfscope}%
\pgfpathrectangle{\pgfqpoint{0.539299in}{0.078740in}}{\pgfqpoint{7.842520in}{7.842520in}}%
\pgfusepath{clip}%
\pgfsetbuttcap%
\pgfsetroundjoin%
\definecolor{currentfill}{rgb}{0.282327,0.094955,0.417331}%
\pgfsetfillcolor{currentfill}%
\pgfsetlinewidth{0.000000pt}%
\definecolor{currentstroke}{rgb}{0.221989,0.339161,0.548752}%
\pgfsetstrokecolor{currentstroke}%
\pgfsetdash{}{0pt}%
\pgfpathmoveto{\pgfqpoint{6.615697in}{2.452498in}}%
\pgfpathlineto{\pgfqpoint{6.756390in}{2.344144in}}%
\pgfpathlineto{\pgfqpoint{6.690065in}{2.529737in}}%
\pgfpathclose%
\pgfusepath{fill}%
\end{pgfscope}%
\begin{pgfscope}%
\pgfpathrectangle{\pgfqpoint{0.539299in}{0.078740in}}{\pgfqpoint{7.842520in}{7.842520in}}%
\pgfusepath{clip}%
\pgfsetbuttcap%
\pgfsetroundjoin%
\definecolor{currentfill}{rgb}{0.162142,0.474838,0.558140}%
\pgfsetfillcolor{currentfill}%
\pgfsetlinewidth{0.000000pt}%
\definecolor{currentstroke}{rgb}{0.220057,0.343307,0.549413}%
\pgfsetstrokecolor{currentstroke}%
\pgfsetdash{}{0pt}%
\pgfpathmoveto{\pgfqpoint{4.925899in}{3.838472in}}%
\pgfpathlineto{\pgfqpoint{5.004323in}{3.770773in}}%
\pgfpathlineto{\pgfqpoint{4.865625in}{3.949141in}}%
\pgfpathclose%
\pgfusepath{fill}%
\end{pgfscope}%
\begin{pgfscope}%
\pgfpathrectangle{\pgfqpoint{0.539299in}{0.078740in}}{\pgfqpoint{7.842520in}{7.842520in}}%
\pgfusepath{clip}%
\pgfsetbuttcap%
\pgfsetroundjoin%
\definecolor{currentfill}{rgb}{0.430983,0.808473,0.346476}%
\pgfsetfillcolor{currentfill}%
\pgfsetlinewidth{0.000000pt}%
\definecolor{currentstroke}{rgb}{0.218130,0.347432,0.550038}%
\pgfsetstrokecolor{currentstroke}%
\pgfsetdash{}{0pt}%
\pgfpathmoveto{\pgfqpoint{3.394745in}{5.445102in}}%
\pgfpathlineto{\pgfqpoint{3.531606in}{5.447979in}}%
\pgfpathlineto{\pgfqpoint{3.480023in}{5.382736in}}%
\pgfpathclose%
\pgfusepath{fill}%
\end{pgfscope}%
\begin{pgfscope}%
\pgfpathrectangle{\pgfqpoint{0.539299in}{0.078740in}}{\pgfqpoint{7.842520in}{7.842520in}}%
\pgfusepath{clip}%
\pgfsetbuttcap%
\pgfsetroundjoin%
\definecolor{currentfill}{rgb}{0.237441,0.305202,0.541921}%
\pgfsetfillcolor{currentfill}%
\pgfsetlinewidth{0.000000pt}%
\definecolor{currentstroke}{rgb}{0.216210,0.351535,0.550627}%
\pgfsetstrokecolor{currentstroke}%
\pgfsetdash{}{0pt}%
\pgfpathmoveto{\pgfqpoint{1.803826in}{3.497087in}}%
\pgfpathlineto{\pgfqpoint{1.688478in}{3.045686in}}%
\pgfpathlineto{\pgfqpoint{1.601948in}{2.927986in}}%
\pgfpathclose%
\pgfusepath{fill}%
\end{pgfscope}%
\begin{pgfscope}%
\pgfpathrectangle{\pgfqpoint{0.539299in}{0.078740in}}{\pgfqpoint{7.842520in}{7.842520in}}%
\pgfusepath{clip}%
\pgfsetbuttcap%
\pgfsetroundjoin%
\definecolor{currentfill}{rgb}{0.246811,0.283237,0.535941}%
\pgfsetfillcolor{currentfill}%
\pgfsetlinewidth{0.000000pt}%
\definecolor{currentstroke}{rgb}{0.214298,0.355619,0.551184}%
\pgfsetstrokecolor{currentstroke}%
\pgfsetdash{}{0pt}%
\pgfpathmoveto{\pgfqpoint{5.622835in}{3.018878in}}%
\pgfpathlineto{\pgfqpoint{5.699184in}{3.033538in}}%
\pgfpathlineto{\pgfqpoint{5.559838in}{3.161010in}}%
\pgfpathclose%
\pgfusepath{fill}%
\end{pgfscope}%
\begin{pgfscope}%
\pgfpathrectangle{\pgfqpoint{0.539299in}{0.078740in}}{\pgfqpoint{7.842520in}{7.842520in}}%
\pgfusepath{clip}%
\pgfsetbuttcap%
\pgfsetroundjoin%
\definecolor{currentfill}{rgb}{0.412913,0.803041,0.357269}%
\pgfsetfillcolor{currentfill}%
\pgfsetlinewidth{0.000000pt}%
\definecolor{currentstroke}{rgb}{0.212395,0.359683,0.551710}%
\pgfsetstrokecolor{currentstroke}%
\pgfsetdash{}{0pt}%
\pgfpathmoveto{\pgfqpoint{3.616640in}{5.370253in}}%
\pgfpathlineto{\pgfqpoint{3.531606in}{5.447979in}}%
\pgfpathlineto{\pgfqpoint{3.754366in}{5.305678in}}%
\pgfpathclose%
\pgfusepath{fill}%
\end{pgfscope}%
\begin{pgfscope}%
\pgfpathrectangle{\pgfqpoint{0.539299in}{0.078740in}}{\pgfqpoint{7.842520in}{7.842520in}}%
\pgfusepath{clip}%
\pgfsetbuttcap%
\pgfsetroundjoin%
\definecolor{currentfill}{rgb}{0.144759,0.519093,0.556572}%
\pgfsetfillcolor{currentfill}%
\pgfsetlinewidth{0.000000pt}%
\definecolor{currentstroke}{rgb}{0.210503,0.363727,0.552206}%
\pgfsetstrokecolor{currentstroke}%
\pgfsetdash{}{0pt}%
\pgfpathmoveto{\pgfqpoint{4.726876in}{4.135706in}}%
\pgfpathlineto{\pgfqpoint{4.786562in}{4.030722in}}%
\pgfpathlineto{\pgfqpoint{4.865625in}{3.949141in}}%
\pgfpathclose%
\pgfusepath{fill}%
\end{pgfscope}%
\begin{pgfscope}%
\pgfpathrectangle{\pgfqpoint{0.539299in}{0.078740in}}{\pgfqpoint{7.842520in}{7.842520in}}%
\pgfusepath{clip}%
\pgfsetbuttcap%
\pgfsetroundjoin%
\definecolor{currentfill}{rgb}{0.266941,0.748751,0.440573}%
\pgfsetfillcolor{currentfill}%
\pgfsetlinewidth{0.000000pt}%
\definecolor{currentstroke}{rgb}{0.208623,0.367752,0.552675}%
\pgfsetstrokecolor{currentstroke}%
\pgfsetdash{}{0pt}%
\pgfpathmoveto{\pgfqpoint{2.738735in}{4.997476in}}%
\pgfpathlineto{\pgfqpoint{2.867996in}{5.262317in}}%
\pgfpathlineto{\pgfqpoint{2.825015in}{5.016747in}}%
\pgfpathclose%
\pgfusepath{fill}%
\end{pgfscope}%
\begin{pgfscope}%
\pgfpathrectangle{\pgfqpoint{0.539299in}{0.078740in}}{\pgfqpoint{7.842520in}{7.842520in}}%
\pgfusepath{clip}%
\pgfsetbuttcap%
\pgfsetroundjoin%
\definecolor{currentfill}{rgb}{0.279574,0.170599,0.479997}%
\pgfsetfillcolor{currentfill}%
\pgfsetlinewidth{0.000000pt}%
\definecolor{currentstroke}{rgb}{0.206756,0.371758,0.553117}%
\pgfsetstrokecolor{currentstroke}%
\pgfsetdash{}{0pt}%
\pgfpathmoveto{\pgfqpoint{6.194335in}{2.753858in}}%
\pgfpathlineto{\pgfqpoint{6.118863in}{2.691325in}}%
\pgfpathlineto{\pgfqpoint{6.259304in}{2.584902in}}%
\pgfpathclose%
\pgfusepath{fill}%
\end{pgfscope}%
\begin{pgfscope}%
\pgfpathrectangle{\pgfqpoint{0.539299in}{0.078740in}}{\pgfqpoint{7.842520in}{7.842520in}}%
\pgfusepath{clip}%
\pgfsetbuttcap%
\pgfsetroundjoin%
\definecolor{currentfill}{rgb}{0.377779,0.791781,0.377939}%
\pgfsetfillcolor{currentfill}%
\pgfsetlinewidth{0.000000pt}%
\definecolor{currentstroke}{rgb}{0.204903,0.375746,0.553533}%
\pgfsetstrokecolor{currentstroke}%
\pgfsetdash{}{0pt}%
\pgfpathmoveto{\pgfqpoint{3.173672in}{5.414731in}}%
\pgfpathlineto{\pgfqpoint{3.040850in}{5.259550in}}%
\pgfpathlineto{\pgfqpoint{2.954612in}{5.267216in}}%
\pgfpathclose%
\pgfusepath{fill}%
\end{pgfscope}%
\begin{pgfscope}%
\pgfpathrectangle{\pgfqpoint{0.539299in}{0.078740in}}{\pgfqpoint{7.842520in}{7.842520in}}%
\pgfusepath{clip}%
\pgfsetbuttcap%
\pgfsetroundjoin%
\definecolor{currentfill}{rgb}{0.282884,0.135920,0.453427}%
\pgfsetfillcolor{currentfill}%
\pgfsetlinewidth{0.000000pt}%
\definecolor{currentstroke}{rgb}{0.203063,0.379716,0.553925}%
\pgfsetstrokecolor{currentstroke}%
\pgfsetdash{}{0pt}%
\pgfpathmoveto{\pgfqpoint{6.475077in}{2.555660in}}%
\pgfpathlineto{\pgfqpoint{6.334603in}{2.655498in}}%
\pgfpathlineto{\pgfqpoint{6.399983in}{2.479315in}}%
\pgfpathclose%
\pgfusepath{fill}%
\end{pgfscope}%
\begin{pgfscope}%
\pgfpathrectangle{\pgfqpoint{0.539299in}{0.078740in}}{\pgfqpoint{7.842520in}{7.842520in}}%
\pgfusepath{clip}%
\pgfsetbuttcap%
\pgfsetroundjoin%
\definecolor{currentfill}{rgb}{0.144759,0.519093,0.556572}%
\pgfsetfillcolor{currentfill}%
\pgfsetlinewidth{0.000000pt}%
\definecolor{currentstroke}{rgb}{0.201239,0.383670,0.554294}%
\pgfsetstrokecolor{currentstroke}%
\pgfsetdash{}{0pt}%
\pgfpathmoveto{\pgfqpoint{2.181328in}{4.304148in}}%
\pgfpathlineto{\pgfqpoint{2.148306in}{3.953837in}}%
\pgfpathlineto{\pgfqpoint{2.062008in}{3.862537in}}%
\pgfpathclose%
\pgfusepath{fill}%
\end{pgfscope}%
\begin{pgfscope}%
\pgfpathrectangle{\pgfqpoint{0.539299in}{0.078740in}}{\pgfqpoint{7.842520in}{7.842520in}}%
\pgfusepath{clip}%
\pgfsetbuttcap%
\pgfsetroundjoin%
\definecolor{currentfill}{rgb}{0.253935,0.265254,0.529983}%
\pgfsetfillcolor{currentfill}%
\pgfsetlinewidth{0.000000pt}%
\definecolor{currentstroke}{rgb}{0.199430,0.387607,0.554642}%
\pgfsetstrokecolor{currentstroke}%
\pgfsetdash{}{0pt}%
\pgfpathmoveto{\pgfqpoint{5.838796in}{2.913871in}}%
\pgfpathlineto{\pgfqpoint{5.699184in}{3.033538in}}%
\pgfpathlineto{\pgfqpoint{5.622835in}{3.018878in}}%
\pgfpathclose%
\pgfusepath{fill}%
\end{pgfscope}%
\begin{pgfscope}%
\pgfpathrectangle{\pgfqpoint{0.539299in}{0.078740in}}{\pgfqpoint{7.842520in}{7.842520in}}%
\pgfusepath{clip}%
\pgfsetbuttcap%
\pgfsetroundjoin%
\definecolor{currentfill}{rgb}{0.278791,0.062145,0.386592}%
\pgfsetfillcolor{currentfill}%
\pgfsetlinewidth{0.000000pt}%
\definecolor{currentstroke}{rgb}{0.197636,0.391528,0.554969}%
\pgfsetstrokecolor{currentstroke}%
\pgfsetdash{}{0pt}%
\pgfpathmoveto{\pgfqpoint{6.830421in}{2.421820in}}%
\pgfpathlineto{\pgfqpoint{6.756390in}{2.344144in}}%
\pgfpathlineto{\pgfqpoint{6.897057in}{2.228556in}}%
\pgfpathclose%
\pgfusepath{fill}%
\end{pgfscope}%
\begin{pgfscope}%
\pgfpathrectangle{\pgfqpoint{0.539299in}{0.078740in}}{\pgfqpoint{7.842520in}{7.842520in}}%
\pgfusepath{clip}%
\pgfsetbuttcap%
\pgfsetroundjoin%
\definecolor{currentfill}{rgb}{0.203063,0.379716,0.553925}%
\pgfsetfillcolor{currentfill}%
\pgfsetlinewidth{0.000000pt}%
\definecolor{currentstroke}{rgb}{0.195860,0.395433,0.555276}%
\pgfsetstrokecolor{currentstroke}%
\pgfsetdash{}{0pt}%
\pgfpathmoveto{\pgfqpoint{1.803826in}{3.497087in}}%
\pgfpathlineto{\pgfqpoint{1.889628in}{3.636546in}}%
\pgfpathlineto{\pgfqpoint{1.775040in}{3.152116in}}%
\pgfpathclose%
\pgfusepath{fill}%
\end{pgfscope}%
\begin{pgfscope}%
\pgfpathrectangle{\pgfqpoint{0.539299in}{0.078740in}}{\pgfqpoint{7.842520in}{7.842520in}}%
\pgfusepath{clip}%
\pgfsetbuttcap%
\pgfsetroundjoin%
\definecolor{currentfill}{rgb}{0.123463,0.581687,0.547445}%
\pgfsetfillcolor{currentfill}%
\pgfsetlinewidth{0.000000pt}%
\definecolor{currentstroke}{rgb}{0.194100,0.399323,0.555565}%
\pgfsetstrokecolor{currentstroke}%
\pgfsetdash{}{0pt}%
\pgfpathmoveto{\pgfqpoint{4.726876in}{4.135706in}}%
\pgfpathlineto{\pgfqpoint{4.588029in}{4.327753in}}%
\pgfpathlineto{\pgfqpoint{4.507522in}{4.430367in}}%
\pgfpathclose%
\pgfusepath{fill}%
\end{pgfscope}%
\begin{pgfscope}%
\pgfpathrectangle{\pgfqpoint{0.539299in}{0.078740in}}{\pgfqpoint{7.842520in}{7.842520in}}%
\pgfusepath{clip}%
\pgfsetbuttcap%
\pgfsetroundjoin%
\definecolor{currentfill}{rgb}{0.319809,0.770914,0.411152}%
\pgfsetfillcolor{currentfill}%
\pgfsetlinewidth{0.000000pt}%
\definecolor{currentstroke}{rgb}{0.192357,0.403199,0.555836}%
\pgfsetstrokecolor{currentstroke}%
\pgfsetdash{}{0pt}%
\pgfpathmoveto{\pgfqpoint{2.867996in}{5.262317in}}%
\pgfpathlineto{\pgfqpoint{2.954612in}{5.267216in}}%
\pgfpathlineto{\pgfqpoint{2.825015in}{5.016747in}}%
\pgfpathclose%
\pgfusepath{fill}%
\end{pgfscope}%
\begin{pgfscope}%
\pgfpathrectangle{\pgfqpoint{0.539299in}{0.078740in}}{\pgfqpoint{7.842520in}{7.842520in}}%
\pgfusepath{clip}%
\pgfsetbuttcap%
\pgfsetroundjoin%
\definecolor{currentfill}{rgb}{0.283229,0.120777,0.440584}%
\pgfsetfillcolor{currentfill}%
\pgfsetlinewidth{0.000000pt}%
\definecolor{currentstroke}{rgb}{0.190631,0.407061,0.556089}%
\pgfsetstrokecolor{currentstroke}%
\pgfsetdash{}{0pt}%
\pgfpathmoveto{\pgfqpoint{6.399983in}{2.479315in}}%
\pgfpathlineto{\pgfqpoint{6.615697in}{2.452498in}}%
\pgfpathlineto{\pgfqpoint{6.475077in}{2.555660in}}%
\pgfpathclose%
\pgfusepath{fill}%
\end{pgfscope}%
\begin{pgfscope}%
\pgfpathrectangle{\pgfqpoint{0.539299in}{0.078740in}}{\pgfqpoint{7.842520in}{7.842520in}}%
\pgfusepath{clip}%
\pgfsetbuttcap%
\pgfsetroundjoin%
\definecolor{currentfill}{rgb}{0.265145,0.232956,0.516599}%
\pgfsetfillcolor{currentfill}%
\pgfsetlinewidth{0.000000pt}%
\definecolor{currentstroke}{rgb}{0.188923,0.410910,0.556326}%
\pgfsetstrokecolor{currentstroke}%
\pgfsetdash{}{0pt}%
\pgfpathmoveto{\pgfqpoint{5.838796in}{2.913871in}}%
\pgfpathlineto{\pgfqpoint{5.762666in}{2.883788in}}%
\pgfpathlineto{\pgfqpoint{5.978689in}{2.800383in}}%
\pgfpathclose%
\pgfusepath{fill}%
\end{pgfscope}%
\begin{pgfscope}%
\pgfpathrectangle{\pgfqpoint{0.539299in}{0.078740in}}{\pgfqpoint{7.842520in}{7.842520in}}%
\pgfusepath{clip}%
\pgfsetbuttcap%
\pgfsetroundjoin%
\definecolor{currentfill}{rgb}{0.208623,0.367752,0.552675}%
\pgfsetfillcolor{currentfill}%
\pgfsetlinewidth{0.000000pt}%
\definecolor{currentstroke}{rgb}{0.187231,0.414746,0.556547}%
\pgfsetstrokecolor{currentstroke}%
\pgfsetdash{}{0pt}%
\pgfpathmoveto{\pgfqpoint{5.343771in}{3.316408in}}%
\pgfpathlineto{\pgfqpoint{5.420728in}{3.297681in}}%
\pgfpathlineto{\pgfqpoint{5.204442in}{3.480326in}}%
\pgfpathclose%
\pgfusepath{fill}%
\end{pgfscope}%
\begin{pgfscope}%
\pgfpathrectangle{\pgfqpoint{0.539299in}{0.078740in}}{\pgfqpoint{7.842520in}{7.842520in}}%
\pgfusepath{clip}%
\pgfsetbuttcap%
\pgfsetroundjoin%
\definecolor{currentfill}{rgb}{0.185556,0.418570,0.556753}%
\pgfsetfillcolor{currentfill}%
\pgfsetlinewidth{0.000000pt}%
\definecolor{currentstroke}{rgb}{0.185556,0.418570,0.556753}%
\pgfsetstrokecolor{currentstroke}%
\pgfsetdash{}{0pt}%
\pgfpathmoveto{\pgfqpoint{5.143030in}{3.602315in}}%
\pgfpathlineto{\pgfqpoint{5.065172in}{3.654551in}}%
\pgfpathlineto{\pgfqpoint{5.204442in}{3.480326in}}%
\pgfpathclose%
\pgfusepath{fill}%
\end{pgfscope}%
\begin{pgfscope}%
\pgfpathrectangle{\pgfqpoint{0.539299in}{0.078740in}}{\pgfqpoint{7.842520in}{7.842520in}}%
\pgfusepath{clip}%
\pgfsetbuttcap%
\pgfsetroundjoin%
\definecolor{currentfill}{rgb}{0.221989,0.339161,0.548752}%
\pgfsetfillcolor{currentfill}%
\pgfsetlinewidth{0.000000pt}%
\definecolor{currentstroke}{rgb}{0.183898,0.422383,0.556944}%
\pgfsetstrokecolor{currentstroke}%
\pgfsetdash{}{0pt}%
\pgfpathmoveto{\pgfqpoint{5.420728in}{3.297681in}}%
\pgfpathlineto{\pgfqpoint{5.343771in}{3.316408in}}%
\pgfpathlineto{\pgfqpoint{5.483218in}{3.162772in}}%
\pgfpathclose%
\pgfusepath{fill}%
\end{pgfscope}%
\begin{pgfscope}%
\pgfpathrectangle{\pgfqpoint{0.539299in}{0.078740in}}{\pgfqpoint{7.842520in}{7.842520in}}%
\pgfusepath{clip}%
\pgfsetbuttcap%
\pgfsetroundjoin%
\definecolor{currentfill}{rgb}{0.282290,0.145912,0.461510}%
\pgfsetfillcolor{currentfill}%
\pgfsetlinewidth{0.000000pt}%
\definecolor{currentstroke}{rgb}{0.182256,0.426184,0.557120}%
\pgfsetstrokecolor{currentstroke}%
\pgfsetdash{}{0pt}%
\pgfpathmoveto{\pgfqpoint{6.399983in}{2.479315in}}%
\pgfpathlineto{\pgfqpoint{6.334603in}{2.655498in}}%
\pgfpathlineto{\pgfqpoint{6.259304in}{2.584902in}}%
\pgfpathclose%
\pgfusepath{fill}%
\end{pgfscope}%
\begin{pgfscope}%
\pgfpathrectangle{\pgfqpoint{0.539299in}{0.078740in}}{\pgfqpoint{7.842520in}{7.842520in}}%
\pgfusepath{clip}%
\pgfsetbuttcap%
\pgfsetroundjoin%
\definecolor{currentfill}{rgb}{0.458674,0.816363,0.329727}%
\pgfsetfillcolor{currentfill}%
\pgfsetlinewidth{0.000000pt}%
\definecolor{currentstroke}{rgb}{0.180629,0.429975,0.557282}%
\pgfsetstrokecolor{currentstroke}%
\pgfsetdash{}{0pt}%
\pgfpathmoveto{\pgfqpoint{3.308861in}{5.497668in}}%
\pgfpathlineto{\pgfqpoint{3.394745in}{5.445102in}}%
\pgfpathlineto{\pgfqpoint{3.259589in}{5.378978in}}%
\pgfpathclose%
\pgfusepath{fill}%
\end{pgfscope}%
\begin{pgfscope}%
\pgfpathrectangle{\pgfqpoint{0.539299in}{0.078740in}}{\pgfqpoint{7.842520in}{7.842520in}}%
\pgfusepath{clip}%
\pgfsetbuttcap%
\pgfsetroundjoin%
\definecolor{currentfill}{rgb}{0.174274,0.445044,0.557792}%
\pgfsetfillcolor{currentfill}%
\pgfsetlinewidth{0.000000pt}%
\definecolor{currentstroke}{rgb}{0.179019,0.433756,0.557430}%
\pgfsetstrokecolor{currentstroke}%
\pgfsetdash{}{0pt}%
\pgfpathmoveto{\pgfqpoint{5.065172in}{3.654551in}}%
\pgfpathlineto{\pgfqpoint{5.143030in}{3.602315in}}%
\pgfpathlineto{\pgfqpoint{4.925899in}{3.838472in}}%
\pgfpathclose%
\pgfusepath{fill}%
\end{pgfscope}%
\begin{pgfscope}%
\pgfpathrectangle{\pgfqpoint{0.539299in}{0.078740in}}{\pgfqpoint{7.842520in}{7.842520in}}%
\pgfusepath{clip}%
\pgfsetbuttcap%
\pgfsetroundjoin%
\definecolor{currentfill}{rgb}{0.237441,0.305202,0.541921}%
\pgfsetfillcolor{currentfill}%
\pgfsetlinewidth{0.000000pt}%
\definecolor{currentstroke}{rgb}{0.177423,0.437527,0.557565}%
\pgfsetstrokecolor{currentstroke}%
\pgfsetdash{}{0pt}%
\pgfpathmoveto{\pgfqpoint{5.622835in}{3.018878in}}%
\pgfpathlineto{\pgfqpoint{5.559838in}{3.161010in}}%
\pgfpathlineto{\pgfqpoint{5.483218in}{3.162772in}}%
\pgfpathclose%
\pgfusepath{fill}%
\end{pgfscope}%
\begin{pgfscope}%
\pgfpathrectangle{\pgfqpoint{0.539299in}{0.078740in}}{\pgfqpoint{7.842520in}{7.842520in}}%
\pgfusepath{clip}%
\pgfsetbuttcap%
\pgfsetroundjoin%
\definecolor{currentfill}{rgb}{0.120638,0.625828,0.533488}%
\pgfsetfillcolor{currentfill}%
\pgfsetlinewidth{0.000000pt}%
\definecolor{currentstroke}{rgb}{0.175841,0.441290,0.557685}%
\pgfsetstrokecolor{currentstroke}%
\pgfsetdash{}{0pt}%
\pgfpathmoveto{\pgfqpoint{4.449060in}{4.521490in}}%
\pgfpathlineto{\pgfqpoint{4.367781in}{4.630418in}}%
\pgfpathlineto{\pgfqpoint{4.588029in}{4.327753in}}%
\pgfpathclose%
\pgfusepath{fill}%
\end{pgfscope}%
\begin{pgfscope}%
\pgfpathrectangle{\pgfqpoint{0.539299in}{0.078740in}}{\pgfqpoint{7.842520in}{7.842520in}}%
\pgfusepath{clip}%
\pgfsetbuttcap%
\pgfsetroundjoin%
\definecolor{currentfill}{rgb}{0.132268,0.655014,0.519661}%
\pgfsetfillcolor{currentfill}%
\pgfsetlinewidth{0.000000pt}%
\definecolor{currentstroke}{rgb}{0.174274,0.445044,0.557792}%
\pgfsetstrokecolor{currentstroke}%
\pgfsetdash{}{0pt}%
\pgfpathmoveto{\pgfqpoint{4.309978in}{4.711999in}}%
\pgfpathlineto{\pgfqpoint{4.367781in}{4.630418in}}%
\pgfpathlineto{\pgfqpoint{4.449060in}{4.521490in}}%
\pgfpathclose%
\pgfusepath{fill}%
\end{pgfscope}%
\begin{pgfscope}%
\pgfpathrectangle{\pgfqpoint{0.539299in}{0.078740in}}{\pgfqpoint{7.842520in}{7.842520in}}%
\pgfusepath{clip}%
\pgfsetbuttcap%
\pgfsetroundjoin%
\definecolor{currentfill}{rgb}{0.150476,0.504369,0.557430}%
\pgfsetfillcolor{currentfill}%
\pgfsetlinewidth{0.000000pt}%
\definecolor{currentstroke}{rgb}{0.172719,0.448791,0.557885}%
\pgfsetstrokecolor{currentstroke}%
\pgfsetdash{}{0pt}%
\pgfpathmoveto{\pgfqpoint{4.865625in}{3.949141in}}%
\pgfpathlineto{\pgfqpoint{4.786562in}{4.030722in}}%
\pgfpathlineto{\pgfqpoint{4.925899in}{3.838472in}}%
\pgfpathclose%
\pgfusepath{fill}%
\end{pgfscope}%
\begin{pgfscope}%
\pgfpathrectangle{\pgfqpoint{0.539299in}{0.078740in}}{\pgfqpoint{7.842520in}{7.842520in}}%
\pgfusepath{clip}%
\pgfsetbuttcap%
\pgfsetroundjoin%
\definecolor{currentfill}{rgb}{0.458674,0.816363,0.329727}%
\pgfsetfillcolor{currentfill}%
\pgfsetlinewidth{0.000000pt}%
\definecolor{currentstroke}{rgb}{0.171176,0.452530,0.557965}%
\pgfsetstrokecolor{currentstroke}%
\pgfsetdash{}{0pt}%
\pgfpathmoveto{\pgfqpoint{3.173672in}{5.414731in}}%
\pgfpathlineto{\pgfqpoint{3.308861in}{5.497668in}}%
\pgfpathlineto{\pgfqpoint{3.259589in}{5.378978in}}%
\pgfpathclose%
\pgfusepath{fill}%
\end{pgfscope}%
\begin{pgfscope}%
\pgfpathrectangle{\pgfqpoint{0.539299in}{0.078740in}}{\pgfqpoint{7.842520in}{7.842520in}}%
\pgfusepath{clip}%
\pgfsetbuttcap%
\pgfsetroundjoin%
\definecolor{currentfill}{rgb}{0.369214,0.788888,0.382914}%
\pgfsetfillcolor{currentfill}%
\pgfsetlinewidth{0.000000pt}%
\definecolor{currentstroke}{rgb}{0.169646,0.456262,0.558030}%
\pgfsetstrokecolor{currentstroke}%
\pgfsetdash{}{0pt}%
\pgfpathmoveto{\pgfqpoint{3.808702in}{5.299454in}}%
\pgfpathlineto{\pgfqpoint{3.892829in}{5.198517in}}%
\pgfpathlineto{\pgfqpoint{3.754366in}{5.305678in}}%
\pgfpathclose%
\pgfusepath{fill}%
\end{pgfscope}%
\begin{pgfscope}%
\pgfpathrectangle{\pgfqpoint{0.539299in}{0.078740in}}{\pgfqpoint{7.842520in}{7.842520in}}%
\pgfusepath{clip}%
\pgfsetbuttcap%
\pgfsetroundjoin%
\definecolor{currentfill}{rgb}{0.282327,0.094955,0.417331}%
\pgfsetfillcolor{currentfill}%
\pgfsetlinewidth{0.000000pt}%
\definecolor{currentstroke}{rgb}{0.168126,0.459988,0.558082}%
\pgfsetstrokecolor{currentstroke}%
\pgfsetdash{}{0pt}%
\pgfpathmoveto{\pgfqpoint{6.756390in}{2.344144in}}%
\pgfpathlineto{\pgfqpoint{6.615697in}{2.452498in}}%
\pgfpathlineto{\pgfqpoint{6.540856in}{2.372768in}}%
\pgfpathclose%
\pgfusepath{fill}%
\end{pgfscope}%
\begin{pgfscope}%
\pgfpathrectangle{\pgfqpoint{0.539299in}{0.078740in}}{\pgfqpoint{7.842520in}{7.842520in}}%
\pgfusepath{clip}%
\pgfsetbuttcap%
\pgfsetroundjoin%
\definecolor{currentfill}{rgb}{0.275191,0.194905,0.496005}%
\pgfsetfillcolor{currentfill}%
\pgfsetlinewidth{0.000000pt}%
\definecolor{currentstroke}{rgb}{0.166617,0.463708,0.558119}%
\pgfsetstrokecolor{currentstroke}%
\pgfsetdash{}{0pt}%
\pgfpathmoveto{\pgfqpoint{6.043073in}{2.634987in}}%
\pgfpathlineto{\pgfqpoint{6.118863in}{2.691325in}}%
\pgfpathlineto{\pgfqpoint{5.978689in}{2.800383in}}%
\pgfpathclose%
\pgfusepath{fill}%
\end{pgfscope}%
\begin{pgfscope}%
\pgfpathrectangle{\pgfqpoint{0.539299in}{0.078740in}}{\pgfqpoint{7.842520in}{7.842520in}}%
\pgfusepath{clip}%
\pgfsetbuttcap%
\pgfsetroundjoin%
\definecolor{currentfill}{rgb}{0.255645,0.260703,0.528312}%
\pgfsetfillcolor{currentfill}%
\pgfsetlinewidth{0.000000pt}%
\definecolor{currentstroke}{rgb}{0.165117,0.467423,0.558141}%
\pgfsetstrokecolor{currentstroke}%
\pgfsetdash{}{0pt}%
\pgfpathmoveto{\pgfqpoint{5.622835in}{3.018878in}}%
\pgfpathlineto{\pgfqpoint{5.762666in}{2.883788in}}%
\pgfpathlineto{\pgfqpoint{5.838796in}{2.913871in}}%
\pgfpathclose%
\pgfusepath{fill}%
\end{pgfscope}%
\begin{pgfscope}%
\pgfpathrectangle{\pgfqpoint{0.539299in}{0.078740in}}{\pgfqpoint{7.842520in}{7.842520in}}%
\pgfusepath{clip}%
\pgfsetbuttcap%
\pgfsetroundjoin%
\definecolor{currentfill}{rgb}{0.133743,0.548535,0.553541}%
\pgfsetfillcolor{currentfill}%
\pgfsetlinewidth{0.000000pt}%
\definecolor{currentstroke}{rgb}{0.163625,0.471133,0.558148}%
\pgfsetstrokecolor{currentstroke}%
\pgfsetdash{}{0pt}%
\pgfpathmoveto{\pgfqpoint{4.647114in}{4.229077in}}%
\pgfpathlineto{\pgfqpoint{4.786562in}{4.030722in}}%
\pgfpathlineto{\pgfqpoint{4.726876in}{4.135706in}}%
\pgfpathclose%
\pgfusepath{fill}%
\end{pgfscope}%
\begin{pgfscope}%
\pgfpathrectangle{\pgfqpoint{0.539299in}{0.078740in}}{\pgfqpoint{7.842520in}{7.842520in}}%
\pgfusepath{clip}%
\pgfsetbuttcap%
\pgfsetroundjoin%
\definecolor{currentfill}{rgb}{0.196571,0.711827,0.479221}%
\pgfsetfillcolor{currentfill}%
\pgfsetlinewidth{0.000000pt}%
\definecolor{currentstroke}{rgb}{0.162142,0.474838,0.558140}%
\pgfsetstrokecolor{currentstroke}%
\pgfsetdash{}{0pt}%
\pgfpathmoveto{\pgfqpoint{4.309978in}{4.711999in}}%
\pgfpathlineto{\pgfqpoint{4.170832in}{4.893238in}}%
\pgfpathlineto{\pgfqpoint{4.088028in}{5.004974in}}%
\pgfpathclose%
\pgfusepath{fill}%
\end{pgfscope}%
\begin{pgfscope}%
\pgfpathrectangle{\pgfqpoint{0.539299in}{0.078740in}}{\pgfqpoint{7.842520in}{7.842520in}}%
\pgfusepath{clip}%
\pgfsetbuttcap%
\pgfsetroundjoin%
\definecolor{currentfill}{rgb}{0.311925,0.767822,0.415586}%
\pgfsetfillcolor{currentfill}%
\pgfsetlinewidth{0.000000pt}%
\definecolor{currentstroke}{rgb}{0.160665,0.478540,0.558115}%
\pgfsetstrokecolor{currentstroke}%
\pgfsetdash{}{0pt}%
\pgfpathmoveto{\pgfqpoint{4.031728in}{5.058097in}}%
\pgfpathlineto{\pgfqpoint{3.892829in}{5.198517in}}%
\pgfpathlineto{\pgfqpoint{3.948223in}{5.166109in}}%
\pgfpathclose%
\pgfusepath{fill}%
\end{pgfscope}%
\begin{pgfscope}%
\pgfpathrectangle{\pgfqpoint{0.539299in}{0.078740in}}{\pgfqpoint{7.842520in}{7.842520in}}%
\pgfusepath{clip}%
\pgfsetbuttcap%
\pgfsetroundjoin%
\definecolor{currentfill}{rgb}{0.146180,0.515413,0.556823}%
\pgfsetfillcolor{currentfill}%
\pgfsetlinewidth{0.000000pt}%
\definecolor{currentstroke}{rgb}{0.159194,0.482237,0.558073}%
\pgfsetstrokecolor{currentstroke}%
\pgfsetdash{}{0pt}%
\pgfpathmoveto{\pgfqpoint{2.181328in}{4.304148in}}%
\pgfpathlineto{\pgfqpoint{2.062008in}{3.862537in}}%
\pgfpathlineto{\pgfqpoint{1.975740in}{3.757494in}}%
\pgfpathclose%
\pgfusepath{fill}%
\end{pgfscope}%
\begin{pgfscope}%
\pgfpathrectangle{\pgfqpoint{0.539299in}{0.078740in}}{\pgfqpoint{7.842520in}{7.842520in}}%
\pgfusepath{clip}%
\pgfsetbuttcap%
\pgfsetroundjoin%
\definecolor{currentfill}{rgb}{0.283091,0.110553,0.431554}%
\pgfsetfillcolor{currentfill}%
\pgfsetlinewidth{0.000000pt}%
\definecolor{currentstroke}{rgb}{0.157729,0.485932,0.558013}%
\pgfsetstrokecolor{currentstroke}%
\pgfsetdash{}{0pt}%
\pgfpathmoveto{\pgfqpoint{6.540856in}{2.372768in}}%
\pgfpathlineto{\pgfqpoint{6.615697in}{2.452498in}}%
\pgfpathlineto{\pgfqpoint{6.399983in}{2.479315in}}%
\pgfpathclose%
\pgfusepath{fill}%
\end{pgfscope}%
\begin{pgfscope}%
\pgfpathrectangle{\pgfqpoint{0.539299in}{0.078740in}}{\pgfqpoint{7.842520in}{7.842520in}}%
\pgfusepath{clip}%
\pgfsetbuttcap%
\pgfsetroundjoin%
\definecolor{currentfill}{rgb}{0.266941,0.748751,0.440573}%
\pgfsetfillcolor{currentfill}%
\pgfsetlinewidth{0.000000pt}%
\definecolor{currentstroke}{rgb}{0.156270,0.489624,0.557936}%
\pgfsetstrokecolor{currentstroke}%
\pgfsetdash{}{0pt}%
\pgfpathmoveto{\pgfqpoint{4.031728in}{5.058097in}}%
\pgfpathlineto{\pgfqpoint{3.948223in}{5.166109in}}%
\pgfpathlineto{\pgfqpoint{4.170832in}{4.893238in}}%
\pgfpathclose%
\pgfusepath{fill}%
\end{pgfscope}%
\begin{pgfscope}%
\pgfpathrectangle{\pgfqpoint{0.539299in}{0.078740in}}{\pgfqpoint{7.842520in}{7.842520in}}%
\pgfusepath{clip}%
\pgfsetbuttcap%
\pgfsetroundjoin%
\definecolor{currentfill}{rgb}{0.440137,0.811138,0.340967}%
\pgfsetfillcolor{currentfill}%
\pgfsetlinewidth{0.000000pt}%
\definecolor{currentstroke}{rgb}{0.154815,0.493313,0.557840}%
\pgfsetstrokecolor{currentstroke}%
\pgfsetdash{}{0pt}%
\pgfpathmoveto{\pgfqpoint{3.754366in}{5.305678in}}%
\pgfpathlineto{\pgfqpoint{3.531606in}{5.447979in}}%
\pgfpathlineto{\pgfqpoint{3.669721in}{5.396407in}}%
\pgfpathclose%
\pgfusepath{fill}%
\end{pgfscope}%
\begin{pgfscope}%
\pgfpathrectangle{\pgfqpoint{0.539299in}{0.078740in}}{\pgfqpoint{7.842520in}{7.842520in}}%
\pgfusepath{clip}%
\pgfsetbuttcap%
\pgfsetroundjoin%
\definecolor{currentfill}{rgb}{0.266580,0.228262,0.514349}%
\pgfsetfillcolor{currentfill}%
\pgfsetlinewidth{0.000000pt}%
\definecolor{currentstroke}{rgb}{0.153364,0.497000,0.557724}%
\pgfsetstrokecolor{currentstroke}%
\pgfsetdash{}{0pt}%
\pgfpathmoveto{\pgfqpoint{5.762666in}{2.883788in}}%
\pgfpathlineto{\pgfqpoint{5.902739in}{2.756289in}}%
\pgfpathlineto{\pgfqpoint{5.978689in}{2.800383in}}%
\pgfpathclose%
\pgfusepath{fill}%
\end{pgfscope}%
\begin{pgfscope}%
\pgfpathrectangle{\pgfqpoint{0.539299in}{0.078740in}}{\pgfqpoint{7.842520in}{7.842520in}}%
\pgfusepath{clip}%
\pgfsetbuttcap%
\pgfsetroundjoin%
\definecolor{currentfill}{rgb}{0.123463,0.581687,0.547445}%
\pgfsetfillcolor{currentfill}%
\pgfsetlinewidth{0.000000pt}%
\definecolor{currentstroke}{rgb}{0.151918,0.500685,0.557587}%
\pgfsetstrokecolor{currentstroke}%
\pgfsetdash{}{0pt}%
\pgfpathmoveto{\pgfqpoint{4.726876in}{4.135706in}}%
\pgfpathlineto{\pgfqpoint{4.507522in}{4.430367in}}%
\pgfpathlineto{\pgfqpoint{4.647114in}{4.229077in}}%
\pgfpathclose%
\pgfusepath{fill}%
\end{pgfscope}%
\begin{pgfscope}%
\pgfpathrectangle{\pgfqpoint{0.539299in}{0.078740in}}{\pgfqpoint{7.842520in}{7.842520in}}%
\pgfusepath{clip}%
\pgfsetbuttcap%
\pgfsetroundjoin%
\definecolor{currentfill}{rgb}{0.126453,0.570633,0.549841}%
\pgfsetfillcolor{currentfill}%
\pgfsetlinewidth{0.000000pt}%
\definecolor{currentstroke}{rgb}{0.150476,0.504369,0.557430}%
\pgfsetstrokecolor{currentstroke}%
\pgfsetdash{}{0pt}%
\pgfpathmoveto{\pgfqpoint{2.267756in}{4.403873in}}%
\pgfpathlineto{\pgfqpoint{2.148306in}{3.953837in}}%
\pgfpathlineto{\pgfqpoint{2.181328in}{4.304148in}}%
\pgfpathclose%
\pgfusepath{fill}%
\end{pgfscope}%
\begin{pgfscope}%
\pgfpathrectangle{\pgfqpoint{0.539299in}{0.078740in}}{\pgfqpoint{7.842520in}{7.842520in}}%
\pgfusepath{clip}%
\pgfsetbuttcap%
\pgfsetroundjoin%
\definecolor{currentfill}{rgb}{0.278791,0.062145,0.386592}%
\pgfsetfillcolor{currentfill}%
\pgfsetlinewidth{0.000000pt}%
\definecolor{currentstroke}{rgb}{0.149039,0.508051,0.557250}%
\pgfsetstrokecolor{currentstroke}%
\pgfsetdash{}{0pt}%
\pgfpathmoveto{\pgfqpoint{6.897057in}{2.228556in}}%
\pgfpathlineto{\pgfqpoint{6.756390in}{2.344144in}}%
\pgfpathlineto{\pgfqpoint{6.681865in}{2.263405in}}%
\pgfpathclose%
\pgfusepath{fill}%
\end{pgfscope}%
\begin{pgfscope}%
\pgfpathrectangle{\pgfqpoint{0.539299in}{0.078740in}}{\pgfqpoint{7.842520in}{7.842520in}}%
\pgfusepath{clip}%
\pgfsetbuttcap%
\pgfsetroundjoin%
\definecolor{currentfill}{rgb}{0.288921,0.758394,0.428426}%
\pgfsetfillcolor{currentfill}%
\pgfsetlinewidth{0.000000pt}%
\definecolor{currentstroke}{rgb}{0.147607,0.511733,0.557049}%
\pgfsetstrokecolor{currentstroke}%
\pgfsetdash{}{0pt}%
\pgfpathmoveto{\pgfqpoint{2.867996in}{5.262317in}}%
\pgfpathlineto{\pgfqpoint{2.738735in}{4.997476in}}%
\pgfpathlineto{\pgfqpoint{2.652170in}{4.965747in}}%
\pgfpathclose%
\pgfusepath{fill}%
\end{pgfscope}%
\begin{pgfscope}%
\pgfpathrectangle{\pgfqpoint{0.539299in}{0.078740in}}{\pgfqpoint{7.842520in}{7.842520in}}%
\pgfusepath{clip}%
\pgfsetbuttcap%
\pgfsetroundjoin%
\definecolor{currentfill}{rgb}{0.252194,0.269783,0.531579}%
\pgfsetfillcolor{currentfill}%
\pgfsetlinewidth{0.000000pt}%
\definecolor{currentstroke}{rgb}{0.146180,0.515413,0.556823}%
\pgfsetstrokecolor{currentstroke}%
\pgfsetdash{}{0pt}%
\pgfpathmoveto{\pgfqpoint{1.633975in}{3.148721in}}%
\pgfpathlineto{\pgfqpoint{1.601948in}{2.927986in}}%
\pgfpathlineto{\pgfqpoint{1.515589in}{2.796677in}}%
\pgfpathclose%
\pgfusepath{fill}%
\end{pgfscope}%
\begin{pgfscope}%
\pgfpathrectangle{\pgfqpoint{0.539299in}{0.078740in}}{\pgfqpoint{7.842520in}{7.842520in}}%
\pgfusepath{clip}%
\pgfsetbuttcap%
\pgfsetroundjoin%
\definecolor{currentfill}{rgb}{0.185783,0.704891,0.485273}%
\pgfsetfillcolor{currentfill}%
\pgfsetlinewidth{0.000000pt}%
\definecolor{currentstroke}{rgb}{0.144759,0.519093,0.556572}%
\pgfsetstrokecolor{currentstroke}%
\pgfsetdash{}{0pt}%
\pgfpathmoveto{\pgfqpoint{2.440760in}{4.553021in}}%
\pgfpathlineto{\pgfqpoint{2.565417in}{4.919366in}}%
\pgfpathlineto{\pgfqpoint{2.652170in}{4.965747in}}%
\pgfpathclose%
\pgfusepath{fill}%
\end{pgfscope}%
\begin{pgfscope}%
\pgfpathrectangle{\pgfqpoint{0.539299in}{0.078740in}}{\pgfqpoint{7.842520in}{7.842520in}}%
\pgfusepath{clip}%
\pgfsetbuttcap%
\pgfsetroundjoin%
\definecolor{currentfill}{rgb}{0.280255,0.165693,0.476498}%
\pgfsetfillcolor{currentfill}%
\pgfsetlinewidth{0.000000pt}%
\definecolor{currentstroke}{rgb}{0.143343,0.522773,0.556295}%
\pgfsetstrokecolor{currentstroke}%
\pgfsetdash{}{0pt}%
\pgfpathmoveto{\pgfqpoint{6.118863in}{2.691325in}}%
\pgfpathlineto{\pgfqpoint{6.183670in}{2.518396in}}%
\pgfpathlineto{\pgfqpoint{6.259304in}{2.584902in}}%
\pgfpathclose%
\pgfusepath{fill}%
\end{pgfscope}%
\begin{pgfscope}%
\pgfpathrectangle{\pgfqpoint{0.539299in}{0.078740in}}{\pgfqpoint{7.842520in}{7.842520in}}%
\pgfusepath{clip}%
\pgfsetbuttcap%
\pgfsetroundjoin%
\definecolor{currentfill}{rgb}{0.273006,0.204520,0.501721}%
\pgfsetfillcolor{currentfill}%
\pgfsetlinewidth{0.000000pt}%
\definecolor{currentstroke}{rgb}{0.141935,0.526453,0.555991}%
\pgfsetstrokecolor{currentstroke}%
\pgfsetdash{}{0pt}%
\pgfpathmoveto{\pgfqpoint{5.978689in}{2.800383in}}%
\pgfpathlineto{\pgfqpoint{5.902739in}{2.756289in}}%
\pgfpathlineto{\pgfqpoint{6.043073in}{2.634987in}}%
\pgfpathclose%
\pgfusepath{fill}%
\end{pgfscope}%
\begin{pgfscope}%
\pgfpathrectangle{\pgfqpoint{0.539299in}{0.078740in}}{\pgfqpoint{7.842520in}{7.842520in}}%
\pgfusepath{clip}%
\pgfsetbuttcap%
\pgfsetroundjoin%
\definecolor{currentfill}{rgb}{0.281446,0.084320,0.407414}%
\pgfsetfillcolor{currentfill}%
\pgfsetlinewidth{0.000000pt}%
\definecolor{currentstroke}{rgb}{0.140536,0.530132,0.555659}%
\pgfsetstrokecolor{currentstroke}%
\pgfsetdash{}{0pt}%
\pgfpathmoveto{\pgfqpoint{6.540856in}{2.372768in}}%
\pgfpathlineto{\pgfqpoint{6.681865in}{2.263405in}}%
\pgfpathlineto{\pgfqpoint{6.756390in}{2.344144in}}%
\pgfpathclose%
\pgfusepath{fill}%
\end{pgfscope}%
\begin{pgfscope}%
\pgfpathrectangle{\pgfqpoint{0.539299in}{0.078740in}}{\pgfqpoint{7.842520in}{7.842520in}}%
\pgfusepath{clip}%
\pgfsetbuttcap%
\pgfsetroundjoin%
\definecolor{currentfill}{rgb}{0.121380,0.629492,0.531973}%
\pgfsetfillcolor{currentfill}%
\pgfsetlinewidth{0.000000pt}%
\definecolor{currentstroke}{rgb}{0.139147,0.533812,0.555298}%
\pgfsetstrokecolor{currentstroke}%
\pgfsetdash{}{0pt}%
\pgfpathmoveto{\pgfqpoint{4.588029in}{4.327753in}}%
\pgfpathlineto{\pgfqpoint{4.367781in}{4.630418in}}%
\pgfpathlineto{\pgfqpoint{4.507522in}{4.430367in}}%
\pgfpathclose%
\pgfusepath{fill}%
\end{pgfscope}%
\begin{pgfscope}%
\pgfpathrectangle{\pgfqpoint{0.539299in}{0.078740in}}{\pgfqpoint{7.842520in}{7.842520in}}%
\pgfusepath{clip}%
\pgfsetbuttcap%
\pgfsetroundjoin%
\definecolor{currentfill}{rgb}{0.496615,0.826376,0.306377}%
\pgfsetfillcolor{currentfill}%
\pgfsetlinewidth{0.000000pt}%
\definecolor{currentstroke}{rgb}{0.137770,0.537492,0.554906}%
\pgfsetstrokecolor{currentstroke}%
\pgfsetdash{}{0pt}%
\pgfpathmoveto{\pgfqpoint{3.531606in}{5.447979in}}%
\pgfpathlineto{\pgfqpoint{3.394745in}{5.445102in}}%
\pgfpathlineto{\pgfqpoint{3.308861in}{5.497668in}}%
\pgfpathclose%
\pgfusepath{fill}%
\end{pgfscope}%
\begin{pgfscope}%
\pgfpathrectangle{\pgfqpoint{0.539299in}{0.078740in}}{\pgfqpoint{7.842520in}{7.842520in}}%
\pgfusepath{clip}%
\pgfsetbuttcap%
\pgfsetroundjoin%
\definecolor{currentfill}{rgb}{0.282290,0.145912,0.461510}%
\pgfsetfillcolor{currentfill}%
\pgfsetlinewidth{0.000000pt}%
\definecolor{currentstroke}{rgb}{0.136408,0.541173,0.554483}%
\pgfsetstrokecolor{currentstroke}%
\pgfsetdash{}{0pt}%
\pgfpathmoveto{\pgfqpoint{6.259304in}{2.584902in}}%
\pgfpathlineto{\pgfqpoint{6.183670in}{2.518396in}}%
\pgfpathlineto{\pgfqpoint{6.399983in}{2.479315in}}%
\pgfpathclose%
\pgfusepath{fill}%
\end{pgfscope}%
\begin{pgfscope}%
\pgfpathrectangle{\pgfqpoint{0.539299in}{0.078740in}}{\pgfqpoint{7.842520in}{7.842520in}}%
\pgfusepath{clip}%
\pgfsetbuttcap%
\pgfsetroundjoin%
\definecolor{currentfill}{rgb}{0.421908,0.805774,0.351910}%
\pgfsetfillcolor{currentfill}%
\pgfsetlinewidth{0.000000pt}%
\definecolor{currentstroke}{rgb}{0.135066,0.544853,0.554029}%
\pgfsetstrokecolor{currentstroke}%
\pgfsetdash{}{0pt}%
\pgfpathmoveto{\pgfqpoint{3.754366in}{5.305678in}}%
\pgfpathlineto{\pgfqpoint{3.669721in}{5.396407in}}%
\pgfpathlineto{\pgfqpoint{3.808702in}{5.299454in}}%
\pgfpathclose%
\pgfusepath{fill}%
\end{pgfscope}%
\begin{pgfscope}%
\pgfpathrectangle{\pgfqpoint{0.539299in}{0.078740in}}{\pgfqpoint{7.842520in}{7.842520in}}%
\pgfusepath{clip}%
\pgfsetbuttcap%
\pgfsetroundjoin%
\definecolor{currentfill}{rgb}{0.140210,0.665859,0.513427}%
\pgfsetfillcolor{currentfill}%
\pgfsetlinewidth{0.000000pt}%
\definecolor{currentstroke}{rgb}{0.133743,0.548535,0.553541}%
\pgfsetstrokecolor{currentstroke}%
\pgfsetdash{}{0pt}%
\pgfpathmoveto{\pgfqpoint{2.440760in}{4.553021in}}%
\pgfpathlineto{\pgfqpoint{2.354275in}{4.486045in}}%
\pgfpathlineto{\pgfqpoint{2.478591in}{4.855876in}}%
\pgfpathclose%
\pgfusepath{fill}%
\end{pgfscope}%
\begin{pgfscope}%
\pgfpathrectangle{\pgfqpoint{0.539299in}{0.078740in}}{\pgfqpoint{7.842520in}{7.842520in}}%
\pgfusepath{clip}%
\pgfsetbuttcap%
\pgfsetroundjoin%
\definecolor{currentfill}{rgb}{0.449368,0.813768,0.335384}%
\pgfsetfillcolor{currentfill}%
\pgfsetlinewidth{0.000000pt}%
\definecolor{currentstroke}{rgb}{0.132444,0.552216,0.553018}%
\pgfsetstrokecolor{currentstroke}%
\pgfsetdash{}{0pt}%
\pgfpathmoveto{\pgfqpoint{2.954612in}{5.267216in}}%
\pgfpathlineto{\pgfqpoint{3.087266in}{5.438839in}}%
\pgfpathlineto{\pgfqpoint{3.173672in}{5.414731in}}%
\pgfpathclose%
\pgfusepath{fill}%
\end{pgfscope}%
\begin{pgfscope}%
\pgfpathrectangle{\pgfqpoint{0.539299in}{0.078740in}}{\pgfqpoint{7.842520in}{7.842520in}}%
\pgfusepath{clip}%
\pgfsetbuttcap%
\pgfsetroundjoin%
\definecolor{currentfill}{rgb}{0.278826,0.175490,0.483397}%
\pgfsetfillcolor{currentfill}%
\pgfsetlinewidth{0.000000pt}%
\definecolor{currentstroke}{rgb}{0.131172,0.555899,0.552459}%
\pgfsetstrokecolor{currentstroke}%
\pgfsetdash{}{0pt}%
\pgfpathmoveto{\pgfqpoint{6.043073in}{2.634987in}}%
\pgfpathlineto{\pgfqpoint{6.183670in}{2.518396in}}%
\pgfpathlineto{\pgfqpoint{6.118863in}{2.691325in}}%
\pgfpathclose%
\pgfusepath{fill}%
\end{pgfscope}%
\begin{pgfscope}%
\pgfpathrectangle{\pgfqpoint{0.539299in}{0.078740in}}{\pgfqpoint{7.842520in}{7.842520in}}%
\pgfusepath{clip}%
\pgfsetbuttcap%
\pgfsetroundjoin%
\definecolor{currentfill}{rgb}{0.214298,0.355619,0.551184}%
\pgfsetfillcolor{currentfill}%
\pgfsetlinewidth{0.000000pt}%
\definecolor{currentstroke}{rgb}{0.129933,0.559582,0.551864}%
\pgfsetstrokecolor{currentstroke}%
\pgfsetdash{}{0pt}%
\pgfpathmoveto{\pgfqpoint{1.601948in}{2.927986in}}%
\pgfpathlineto{\pgfqpoint{1.718526in}{3.335879in}}%
\pgfpathlineto{\pgfqpoint{1.803826in}{3.497087in}}%
\pgfpathclose%
\pgfusepath{fill}%
\end{pgfscope}%
\begin{pgfscope}%
\pgfpathrectangle{\pgfqpoint{0.539299in}{0.078740in}}{\pgfqpoint{7.842520in}{7.842520in}}%
\pgfusepath{clip}%
\pgfsetbuttcap%
\pgfsetroundjoin%
\definecolor{currentfill}{rgb}{0.162016,0.687316,0.499129}%
\pgfsetfillcolor{currentfill}%
\pgfsetlinewidth{0.000000pt}%
\definecolor{currentstroke}{rgb}{0.128729,0.563265,0.551229}%
\pgfsetstrokecolor{currentstroke}%
\pgfsetdash{}{0pt}%
\pgfpathmoveto{\pgfqpoint{4.227924in}{4.824025in}}%
\pgfpathlineto{\pgfqpoint{4.367781in}{4.630418in}}%
\pgfpathlineto{\pgfqpoint{4.309978in}{4.711999in}}%
\pgfpathclose%
\pgfusepath{fill}%
\end{pgfscope}%
\begin{pgfscope}%
\pgfpathrectangle{\pgfqpoint{0.539299in}{0.078740in}}{\pgfqpoint{7.842520in}{7.842520in}}%
\pgfusepath{clip}%
\pgfsetbuttcap%
\pgfsetroundjoin%
\definecolor{currentfill}{rgb}{0.277018,0.050344,0.375715}%
\pgfsetfillcolor{currentfill}%
\pgfsetlinewidth{0.000000pt}%
\definecolor{currentstroke}{rgb}{0.127568,0.566949,0.550556}%
\pgfsetstrokecolor{currentstroke}%
\pgfsetdash{}{0pt}%
\pgfpathmoveto{\pgfqpoint{6.681865in}{2.263405in}}%
\pgfpathlineto{\pgfqpoint{6.822930in}{2.149191in}}%
\pgfpathlineto{\pgfqpoint{6.897057in}{2.228556in}}%
\pgfpathclose%
\pgfusepath{fill}%
\end{pgfscope}%
\begin{pgfscope}%
\pgfpathrectangle{\pgfqpoint{0.539299in}{0.078740in}}{\pgfqpoint{7.842520in}{7.842520in}}%
\pgfusepath{clip}%
\pgfsetbuttcap%
\pgfsetroundjoin%
\definecolor{currentfill}{rgb}{0.369214,0.788888,0.382914}%
\pgfsetfillcolor{currentfill}%
\pgfsetlinewidth{0.000000pt}%
\definecolor{currentstroke}{rgb}{0.126453,0.570633,0.549841}%
\pgfsetstrokecolor{currentstroke}%
\pgfsetdash{}{0pt}%
\pgfpathmoveto{\pgfqpoint{3.948223in}{5.166109in}}%
\pgfpathlineto{\pgfqpoint{3.892829in}{5.198517in}}%
\pgfpathlineto{\pgfqpoint{3.808702in}{5.299454in}}%
\pgfpathclose%
\pgfusepath{fill}%
\end{pgfscope}%
\begin{pgfscope}%
\pgfpathrectangle{\pgfqpoint{0.539299in}{0.078740in}}{\pgfqpoint{7.842520in}{7.842520in}}%
\pgfusepath{clip}%
\pgfsetbuttcap%
\pgfsetroundjoin%
\definecolor{currentfill}{rgb}{0.202219,0.715272,0.476084}%
\pgfsetfillcolor{currentfill}%
\pgfsetlinewidth{0.000000pt}%
\definecolor{currentstroke}{rgb}{0.125394,0.574318,0.549086}%
\pgfsetstrokecolor{currentstroke}%
\pgfsetdash{}{0pt}%
\pgfpathmoveto{\pgfqpoint{4.309978in}{4.711999in}}%
\pgfpathlineto{\pgfqpoint{4.088028in}{5.004974in}}%
\pgfpathlineto{\pgfqpoint{4.227924in}{4.824025in}}%
\pgfpathclose%
\pgfusepath{fill}%
\end{pgfscope}%
\begin{pgfscope}%
\pgfpathrectangle{\pgfqpoint{0.539299in}{0.078740in}}{\pgfqpoint{7.842520in}{7.842520in}}%
\pgfusepath{clip}%
\pgfsetbuttcap%
\pgfsetroundjoin%
\definecolor{currentfill}{rgb}{0.258965,0.251537,0.524736}%
\pgfsetfillcolor{currentfill}%
\pgfsetlinewidth{0.000000pt}%
\definecolor{currentstroke}{rgb}{0.124395,0.578002,0.548287}%
\pgfsetstrokecolor{currentstroke}%
\pgfsetdash{}{0pt}%
\pgfpathmoveto{\pgfqpoint{1.515589in}{2.796677in}}%
\pgfpathlineto{\pgfqpoint{1.429595in}{2.648382in}}%
\pgfpathlineto{\pgfqpoint{1.633975in}{3.148721in}}%
\pgfpathclose%
\pgfusepath{fill}%
\end{pgfscope}%
\begin{pgfscope}%
\pgfpathrectangle{\pgfqpoint{0.539299in}{0.078740in}}{\pgfqpoint{7.842520in}{7.842520in}}%
\pgfusepath{clip}%
\pgfsetbuttcap%
\pgfsetroundjoin%
\definecolor{currentfill}{rgb}{0.199430,0.387607,0.554642}%
\pgfsetfillcolor{currentfill}%
\pgfsetlinewidth{0.000000pt}%
\definecolor{currentstroke}{rgb}{0.123463,0.581687,0.547445}%
\pgfsetstrokecolor{currentstroke}%
\pgfsetdash{}{0pt}%
\pgfpathmoveto{\pgfqpoint{5.343771in}{3.316408in}}%
\pgfpathlineto{\pgfqpoint{5.204442in}{3.480326in}}%
\pgfpathlineto{\pgfqpoint{5.266300in}{3.347822in}}%
\pgfpathclose%
\pgfusepath{fill}%
\end{pgfscope}%
\begin{pgfscope}%
\pgfpathrectangle{\pgfqpoint{0.539299in}{0.078740in}}{\pgfqpoint{7.842520in}{7.842520in}}%
\pgfusepath{clip}%
\pgfsetbuttcap%
\pgfsetroundjoin%
\definecolor{currentfill}{rgb}{0.210503,0.363727,0.552206}%
\pgfsetfillcolor{currentfill}%
\pgfsetlinewidth{0.000000pt}%
\definecolor{currentstroke}{rgb}{0.122606,0.585371,0.546557}%
\pgfsetstrokecolor{currentstroke}%
\pgfsetdash{}{0pt}%
\pgfpathmoveto{\pgfqpoint{5.483218in}{3.162772in}}%
\pgfpathlineto{\pgfqpoint{5.343771in}{3.316408in}}%
\pgfpathlineto{\pgfqpoint{5.266300in}{3.347822in}}%
\pgfpathclose%
\pgfusepath{fill}%
\end{pgfscope}%
\begin{pgfscope}%
\pgfpathrectangle{\pgfqpoint{0.539299in}{0.078740in}}{\pgfqpoint{7.842520in}{7.842520in}}%
\pgfusepath{clip}%
\pgfsetbuttcap%
\pgfsetroundjoin%
\definecolor{currentfill}{rgb}{0.274149,0.751988,0.436601}%
\pgfsetfillcolor{currentfill}%
\pgfsetlinewidth{0.000000pt}%
\definecolor{currentstroke}{rgb}{0.121831,0.589055,0.545623}%
\pgfsetstrokecolor{currentstroke}%
\pgfsetdash{}{0pt}%
\pgfpathmoveto{\pgfqpoint{4.170832in}{4.893238in}}%
\pgfpathlineto{\pgfqpoint{3.948223in}{5.166109in}}%
\pgfpathlineto{\pgfqpoint{4.088028in}{5.004974in}}%
\pgfpathclose%
\pgfusepath{fill}%
\end{pgfscope}%
\begin{pgfscope}%
\pgfpathrectangle{\pgfqpoint{0.539299in}{0.078740in}}{\pgfqpoint{7.842520in}{7.842520in}}%
\pgfusepath{clip}%
\pgfsetbuttcap%
\pgfsetroundjoin%
\definecolor{currentfill}{rgb}{0.229739,0.322361,0.545706}%
\pgfsetfillcolor{currentfill}%
\pgfsetlinewidth{0.000000pt}%
\definecolor{currentstroke}{rgb}{0.121148,0.592739,0.544641}%
\pgfsetstrokecolor{currentstroke}%
\pgfsetdash{}{0pt}%
\pgfpathmoveto{\pgfqpoint{5.483218in}{3.162772in}}%
\pgfpathlineto{\pgfqpoint{5.406152in}{3.177575in}}%
\pgfpathlineto{\pgfqpoint{5.622835in}{3.018878in}}%
\pgfpathclose%
\pgfusepath{fill}%
\end{pgfscope}%
\begin{pgfscope}%
\pgfpathrectangle{\pgfqpoint{0.539299in}{0.078740in}}{\pgfqpoint{7.842520in}{7.842520in}}%
\pgfusepath{clip}%
\pgfsetbuttcap%
\pgfsetroundjoin%
\definecolor{currentfill}{rgb}{0.283091,0.110553,0.431554}%
\pgfsetfillcolor{currentfill}%
\pgfsetlinewidth{0.000000pt}%
\definecolor{currentstroke}{rgb}{0.120565,0.596422,0.543611}%
\pgfsetstrokecolor{currentstroke}%
\pgfsetdash{}{0pt}%
\pgfpathmoveto{\pgfqpoint{6.540856in}{2.372768in}}%
\pgfpathlineto{\pgfqpoint{6.399983in}{2.479315in}}%
\pgfpathlineto{\pgfqpoint{6.465616in}{2.293258in}}%
\pgfpathclose%
\pgfusepath{fill}%
\end{pgfscope}%
\begin{pgfscope}%
\pgfpathrectangle{\pgfqpoint{0.539299in}{0.078740in}}{\pgfqpoint{7.842520in}{7.842520in}}%
\pgfusepath{clip}%
\pgfsetbuttcap%
\pgfsetroundjoin%
\definecolor{currentfill}{rgb}{0.180629,0.429975,0.557282}%
\pgfsetfillcolor{currentfill}%
\pgfsetlinewidth{0.000000pt}%
\definecolor{currentstroke}{rgb}{0.120092,0.600104,0.542530}%
\pgfsetstrokecolor{currentstroke}%
\pgfsetdash{}{0pt}%
\pgfpathmoveto{\pgfqpoint{5.126490in}{3.527887in}}%
\pgfpathlineto{\pgfqpoint{5.204442in}{3.480326in}}%
\pgfpathlineto{\pgfqpoint{5.065172in}{3.654551in}}%
\pgfpathclose%
\pgfusepath{fill}%
\end{pgfscope}%
\begin{pgfscope}%
\pgfpathrectangle{\pgfqpoint{0.539299in}{0.078740in}}{\pgfqpoint{7.842520in}{7.842520in}}%
\pgfusepath{clip}%
\pgfsetbuttcap%
\pgfsetroundjoin%
\definecolor{currentfill}{rgb}{0.252194,0.269783,0.531579}%
\pgfsetfillcolor{currentfill}%
\pgfsetlinewidth{0.000000pt}%
\definecolor{currentstroke}{rgb}{0.119738,0.603785,0.541400}%
\pgfsetstrokecolor{currentstroke}%
\pgfsetdash{}{0pt}%
\pgfpathmoveto{\pgfqpoint{5.686202in}{2.865893in}}%
\pgfpathlineto{\pgfqpoint{5.762666in}{2.883788in}}%
\pgfpathlineto{\pgfqpoint{5.622835in}{3.018878in}}%
\pgfpathclose%
\pgfusepath{fill}%
\end{pgfscope}%
\begin{pgfscope}%
\pgfpathrectangle{\pgfqpoint{0.539299in}{0.078740in}}{\pgfqpoint{7.842520in}{7.842520in}}%
\pgfusepath{clip}%
\pgfsetbuttcap%
\pgfsetroundjoin%
\definecolor{currentfill}{rgb}{0.282623,0.140926,0.457517}%
\pgfsetfillcolor{currentfill}%
\pgfsetlinewidth{0.000000pt}%
\definecolor{currentstroke}{rgb}{0.119512,0.607464,0.540218}%
\pgfsetstrokecolor{currentstroke}%
\pgfsetdash{}{0pt}%
\pgfpathmoveto{\pgfqpoint{6.399983in}{2.479315in}}%
\pgfpathlineto{\pgfqpoint{6.183670in}{2.518396in}}%
\pgfpathlineto{\pgfqpoint{6.324525in}{2.404996in}}%
\pgfpathclose%
\pgfusepath{fill}%
\end{pgfscope}%
\begin{pgfscope}%
\pgfpathrectangle{\pgfqpoint{0.539299in}{0.078740in}}{\pgfqpoint{7.842520in}{7.842520in}}%
\pgfusepath{clip}%
\pgfsetbuttcap%
\pgfsetroundjoin%
\definecolor{currentfill}{rgb}{0.163625,0.471133,0.558148}%
\pgfsetfillcolor{currentfill}%
\pgfsetlinewidth{0.000000pt}%
\definecolor{currentstroke}{rgb}{0.119423,0.611141,0.538982}%
\pgfsetstrokecolor{currentstroke}%
\pgfsetdash{}{0pt}%
\pgfpathmoveto{\pgfqpoint{4.986662in}{3.717303in}}%
\pgfpathlineto{\pgfqpoint{5.065172in}{3.654551in}}%
\pgfpathlineto{\pgfqpoint{4.925899in}{3.838472in}}%
\pgfpathclose%
\pgfusepath{fill}%
\end{pgfscope}%
\begin{pgfscope}%
\pgfpathrectangle{\pgfqpoint{0.539299in}{0.078740in}}{\pgfqpoint{7.842520in}{7.842520in}}%
\pgfusepath{clip}%
\pgfsetbuttcap%
\pgfsetroundjoin%
\definecolor{currentfill}{rgb}{0.151918,0.500685,0.557587}%
\pgfsetfillcolor{currentfill}%
\pgfsetlinewidth{0.000000pt}%
\definecolor{currentstroke}{rgb}{0.119483,0.614817,0.537692}%
\pgfsetstrokecolor{currentstroke}%
\pgfsetdash{}{0pt}%
\pgfpathmoveto{\pgfqpoint{1.975740in}{3.757494in}}%
\pgfpathlineto{\pgfqpoint{1.889628in}{3.636546in}}%
\pgfpathlineto{\pgfqpoint{2.095141in}{4.184186in}}%
\pgfpathclose%
\pgfusepath{fill}%
\end{pgfscope}%
\begin{pgfscope}%
\pgfpathrectangle{\pgfqpoint{0.539299in}{0.078740in}}{\pgfqpoint{7.842520in}{7.842520in}}%
\pgfusepath{clip}%
\pgfsetbuttcap%
\pgfsetroundjoin%
\definecolor{currentfill}{rgb}{0.506271,0.828786,0.300362}%
\pgfsetfillcolor{currentfill}%
\pgfsetlinewidth{0.000000pt}%
\definecolor{currentstroke}{rgb}{0.119699,0.618490,0.536347}%
\pgfsetstrokecolor{currentstroke}%
\pgfsetdash{}{0pt}%
\pgfpathmoveto{\pgfqpoint{3.087266in}{5.438839in}}%
\pgfpathlineto{\pgfqpoint{3.308861in}{5.497668in}}%
\pgfpathlineto{\pgfqpoint{3.173672in}{5.414731in}}%
\pgfpathclose%
\pgfusepath{fill}%
\end{pgfscope}%
\begin{pgfscope}%
\pgfpathrectangle{\pgfqpoint{0.539299in}{0.078740in}}{\pgfqpoint{7.842520in}{7.842520in}}%
\pgfusepath{clip}%
\pgfsetbuttcap%
\pgfsetroundjoin%
\definecolor{currentfill}{rgb}{0.225863,0.330805,0.547314}%
\pgfsetfillcolor{currentfill}%
\pgfsetlinewidth{0.000000pt}%
\definecolor{currentstroke}{rgb}{0.120081,0.622161,0.534946}%
\pgfsetstrokecolor{currentstroke}%
\pgfsetdash{}{0pt}%
\pgfpathmoveto{\pgfqpoint{1.718526in}{3.335879in}}%
\pgfpathlineto{\pgfqpoint{1.601948in}{2.927986in}}%
\pgfpathlineto{\pgfqpoint{1.633975in}{3.148721in}}%
\pgfpathclose%
\pgfusepath{fill}%
\end{pgfscope}%
\begin{pgfscope}%
\pgfpathrectangle{\pgfqpoint{0.539299in}{0.078740in}}{\pgfqpoint{7.842520in}{7.842520in}}%
\pgfusepath{clip}%
\pgfsetbuttcap%
\pgfsetroundjoin%
\definecolor{currentfill}{rgb}{0.258965,0.251537,0.524736}%
\pgfsetfillcolor{currentfill}%
\pgfsetlinewidth{0.000000pt}%
\definecolor{currentstroke}{rgb}{0.120638,0.625828,0.533488}%
\pgfsetstrokecolor{currentstroke}%
\pgfsetdash{}{0pt}%
\pgfpathmoveto{\pgfqpoint{5.902739in}{2.756289in}}%
\pgfpathlineto{\pgfqpoint{5.762666in}{2.883788in}}%
\pgfpathlineto{\pgfqpoint{5.686202in}{2.865893in}}%
\pgfpathclose%
\pgfusepath{fill}%
\end{pgfscope}%
\begin{pgfscope}%
\pgfpathrectangle{\pgfqpoint{0.539299in}{0.078740in}}{\pgfqpoint{7.842520in}{7.842520in}}%
\pgfusepath{clip}%
\pgfsetbuttcap%
\pgfsetroundjoin%
\definecolor{currentfill}{rgb}{0.281924,0.089666,0.412415}%
\pgfsetfillcolor{currentfill}%
\pgfsetlinewidth{0.000000pt}%
\definecolor{currentstroke}{rgb}{0.121380,0.629492,0.531973}%
\pgfsetstrokecolor{currentstroke}%
\pgfsetdash{}{0pt}%
\pgfpathmoveto{\pgfqpoint{6.465616in}{2.293258in}}%
\pgfpathlineto{\pgfqpoint{6.681865in}{2.263405in}}%
\pgfpathlineto{\pgfqpoint{6.540856in}{2.372768in}}%
\pgfpathclose%
\pgfusepath{fill}%
\end{pgfscope}%
\begin{pgfscope}%
\pgfpathrectangle{\pgfqpoint{0.539299in}{0.078740in}}{\pgfqpoint{7.842520in}{7.842520in}}%
\pgfusepath{clip}%
\pgfsetbuttcap%
\pgfsetroundjoin%
\definecolor{currentfill}{rgb}{0.185783,0.704891,0.485273}%
\pgfsetfillcolor{currentfill}%
\pgfsetlinewidth{0.000000pt}%
\definecolor{currentstroke}{rgb}{0.122312,0.633153,0.530398}%
\pgfsetstrokecolor{currentstroke}%
\pgfsetdash{}{0pt}%
\pgfpathmoveto{\pgfqpoint{2.478591in}{4.855876in}}%
\pgfpathlineto{\pgfqpoint{2.565417in}{4.919366in}}%
\pgfpathlineto{\pgfqpoint{2.440760in}{4.553021in}}%
\pgfpathclose%
\pgfusepath{fill}%
\end{pgfscope}%
\begin{pgfscope}%
\pgfpathrectangle{\pgfqpoint{0.539299in}{0.078740in}}{\pgfqpoint{7.842520in}{7.842520in}}%
\pgfusepath{clip}%
\pgfsetbuttcap%
\pgfsetroundjoin%
\definecolor{currentfill}{rgb}{0.283229,0.120777,0.440584}%
\pgfsetfillcolor{currentfill}%
\pgfsetlinewidth{0.000000pt}%
\definecolor{currentstroke}{rgb}{0.123444,0.636809,0.528763}%
\pgfsetstrokecolor{currentstroke}%
\pgfsetdash{}{0pt}%
\pgfpathmoveto{\pgfqpoint{6.465616in}{2.293258in}}%
\pgfpathlineto{\pgfqpoint{6.399983in}{2.479315in}}%
\pgfpathlineto{\pgfqpoint{6.324525in}{2.404996in}}%
\pgfpathclose%
\pgfusepath{fill}%
\end{pgfscope}%
\begin{pgfscope}%
\pgfpathrectangle{\pgfqpoint{0.539299in}{0.078740in}}{\pgfqpoint{7.842520in}{7.842520in}}%
\pgfusepath{clip}%
\pgfsetbuttcap%
\pgfsetroundjoin%
\definecolor{currentfill}{rgb}{0.139147,0.533812,0.555298}%
\pgfsetfillcolor{currentfill}%
\pgfsetlinewidth{0.000000pt}%
\definecolor{currentstroke}{rgb}{0.124780,0.640461,0.527068}%
\pgfsetstrokecolor{currentstroke}%
\pgfsetdash{}{0pt}%
\pgfpathmoveto{\pgfqpoint{4.925899in}{3.838472in}}%
\pgfpathlineto{\pgfqpoint{4.786562in}{4.030722in}}%
\pgfpathlineto{\pgfqpoint{4.706729in}{4.119068in}}%
\pgfpathclose%
\pgfusepath{fill}%
\end{pgfscope}%
\begin{pgfscope}%
\pgfpathrectangle{\pgfqpoint{0.539299in}{0.078740in}}{\pgfqpoint{7.842520in}{7.842520in}}%
\pgfusepath{clip}%
\pgfsetbuttcap%
\pgfsetroundjoin%
\definecolor{currentfill}{rgb}{0.273006,0.204520,0.501721}%
\pgfsetfillcolor{currentfill}%
\pgfsetlinewidth{0.000000pt}%
\definecolor{currentstroke}{rgb}{0.126326,0.644107,0.525311}%
\pgfsetstrokecolor{currentstroke}%
\pgfsetdash{}{0pt}%
\pgfpathmoveto{\pgfqpoint{5.902739in}{2.756289in}}%
\pgfpathlineto{\pgfqpoint{5.967002in}{2.588084in}}%
\pgfpathlineto{\pgfqpoint{6.043073in}{2.634987in}}%
\pgfpathclose%
\pgfusepath{fill}%
\end{pgfscope}%
\begin{pgfscope}%
\pgfpathrectangle{\pgfqpoint{0.539299in}{0.078740in}}{\pgfqpoint{7.842520in}{7.842520in}}%
\pgfusepath{clip}%
\pgfsetbuttcap%
\pgfsetroundjoin%
\definecolor{currentfill}{rgb}{0.271828,0.209303,0.504434}%
\pgfsetfillcolor{currentfill}%
\pgfsetlinewidth{0.000000pt}%
\definecolor{currentstroke}{rgb}{0.128087,0.647749,0.523491}%
\pgfsetstrokecolor{currentstroke}%
\pgfsetdash{}{0pt}%
\pgfpathmoveto{\pgfqpoint{1.429595in}{2.648382in}}%
\pgfpathlineto{\pgfqpoint{1.344269in}{2.477731in}}%
\pgfpathlineto{\pgfqpoint{1.550510in}{2.929884in}}%
\pgfpathclose%
\pgfusepath{fill}%
\end{pgfscope}%
\begin{pgfscope}%
\pgfpathrectangle{\pgfqpoint{0.539299in}{0.078740in}}{\pgfqpoint{7.842520in}{7.842520in}}%
\pgfusepath{clip}%
\pgfsetbuttcap%
\pgfsetroundjoin%
\definecolor{currentfill}{rgb}{0.545524,0.838039,0.275626}%
\pgfsetfillcolor{currentfill}%
\pgfsetlinewidth{0.000000pt}%
\definecolor{currentstroke}{rgb}{0.130067,0.651384,0.521608}%
\pgfsetstrokecolor{currentstroke}%
\pgfsetdash{}{0pt}%
\pgfpathmoveto{\pgfqpoint{3.531606in}{5.447979in}}%
\pgfpathlineto{\pgfqpoint{3.308861in}{5.497668in}}%
\pgfpathlineto{\pgfqpoint{3.445912in}{5.515993in}}%
\pgfpathclose%
\pgfusepath{fill}%
\end{pgfscope}%
\begin{pgfscope}%
\pgfpathrectangle{\pgfqpoint{0.539299in}{0.078740in}}{\pgfqpoint{7.842520in}{7.842520in}}%
\pgfusepath{clip}%
\pgfsetbuttcap%
\pgfsetroundjoin%
\definecolor{currentfill}{rgb}{0.212395,0.359683,0.551710}%
\pgfsetfillcolor{currentfill}%
\pgfsetlinewidth{0.000000pt}%
\definecolor{currentstroke}{rgb}{0.132268,0.655014,0.519661}%
\pgfsetstrokecolor{currentstroke}%
\pgfsetdash{}{0pt}%
\pgfpathmoveto{\pgfqpoint{5.266300in}{3.347822in}}%
\pgfpathlineto{\pgfqpoint{5.406152in}{3.177575in}}%
\pgfpathlineto{\pgfqpoint{5.483218in}{3.162772in}}%
\pgfpathclose%
\pgfusepath{fill}%
\end{pgfscope}%
\begin{pgfscope}%
\pgfpathrectangle{\pgfqpoint{0.539299in}{0.078740in}}{\pgfqpoint{7.842520in}{7.842520in}}%
\pgfusepath{clip}%
\pgfsetbuttcap%
\pgfsetroundjoin%
\definecolor{currentfill}{rgb}{0.231674,0.318106,0.544834}%
\pgfsetfillcolor{currentfill}%
\pgfsetlinewidth{0.000000pt}%
\definecolor{currentstroke}{rgb}{0.134692,0.658636,0.517649}%
\pgfsetstrokecolor{currentstroke}%
\pgfsetdash{}{0pt}%
\pgfpathmoveto{\pgfqpoint{5.622835in}{3.018878in}}%
\pgfpathlineto{\pgfqpoint{5.406152in}{3.177575in}}%
\pgfpathlineto{\pgfqpoint{5.546103in}{3.017088in}}%
\pgfpathclose%
\pgfusepath{fill}%
\end{pgfscope}%
\begin{pgfscope}%
\pgfpathrectangle{\pgfqpoint{0.539299in}{0.078740in}}{\pgfqpoint{7.842520in}{7.842520in}}%
\pgfusepath{clip}%
\pgfsetbuttcap%
\pgfsetroundjoin%
\definecolor{currentfill}{rgb}{0.243113,0.292092,0.538516}%
\pgfsetfillcolor{currentfill}%
\pgfsetlinewidth{0.000000pt}%
\definecolor{currentstroke}{rgb}{0.137339,0.662252,0.515571}%
\pgfsetstrokecolor{currentstroke}%
\pgfsetdash{}{0pt}%
\pgfpathmoveto{\pgfqpoint{5.622835in}{3.018878in}}%
\pgfpathlineto{\pgfqpoint{5.546103in}{3.017088in}}%
\pgfpathlineto{\pgfqpoint{5.686202in}{2.865893in}}%
\pgfpathclose%
\pgfusepath{fill}%
\end{pgfscope}%
\begin{pgfscope}%
\pgfpathrectangle{\pgfqpoint{0.539299in}{0.078740in}}{\pgfqpoint{7.842520in}{7.842520in}}%
\pgfusepath{clip}%
\pgfsetbuttcap%
\pgfsetroundjoin%
\definecolor{currentfill}{rgb}{0.187231,0.414746,0.556547}%
\pgfsetfillcolor{currentfill}%
\pgfsetlinewidth{0.000000pt}%
\definecolor{currentstroke}{rgb}{0.140210,0.665859,0.513427}%
\pgfsetstrokecolor{currentstroke}%
\pgfsetdash{}{0pt}%
\pgfpathmoveto{\pgfqpoint{5.266300in}{3.347822in}}%
\pgfpathlineto{\pgfqpoint{5.204442in}{3.480326in}}%
\pgfpathlineto{\pgfqpoint{5.126490in}{3.527887in}}%
\pgfpathclose%
\pgfusepath{fill}%
\end{pgfscope}%
\begin{pgfscope}%
\pgfpathrectangle{\pgfqpoint{0.539299in}{0.078740in}}{\pgfqpoint{7.842520in}{7.842520in}}%
\pgfusepath{clip}%
\pgfsetbuttcap%
\pgfsetroundjoin%
\definecolor{currentfill}{rgb}{0.277018,0.050344,0.375715}%
\pgfsetfillcolor{currentfill}%
\pgfsetlinewidth{0.000000pt}%
\definecolor{currentstroke}{rgb}{0.143303,0.669459,0.511215}%
\pgfsetstrokecolor{currentstroke}%
\pgfsetdash{}{0pt}%
\pgfpathmoveto{\pgfqpoint{6.748358in}{2.068409in}}%
\pgfpathlineto{\pgfqpoint{6.822930in}{2.149191in}}%
\pgfpathlineto{\pgfqpoint{6.681865in}{2.263405in}}%
\pgfpathclose%
\pgfusepath{fill}%
\end{pgfscope}%
\begin{pgfscope}%
\pgfpathrectangle{\pgfqpoint{0.539299in}{0.078740in}}{\pgfqpoint{7.842520in}{7.842520in}}%
\pgfusepath{clip}%
\pgfsetbuttcap%
\pgfsetroundjoin%
\definecolor{currentfill}{rgb}{0.278012,0.180367,0.486697}%
\pgfsetfillcolor{currentfill}%
\pgfsetlinewidth{0.000000pt}%
\definecolor{currentstroke}{rgb}{0.146616,0.673050,0.508936}%
\pgfsetstrokecolor{currentstroke}%
\pgfsetdash{}{0pt}%
\pgfpathmoveto{\pgfqpoint{6.043073in}{2.634987in}}%
\pgfpathlineto{\pgfqpoint{5.967002in}{2.588084in}}%
\pgfpathlineto{\pgfqpoint{6.183670in}{2.518396in}}%
\pgfpathclose%
\pgfusepath{fill}%
\end{pgfscope}%
\begin{pgfscope}%
\pgfpathrectangle{\pgfqpoint{0.539299in}{0.078740in}}{\pgfqpoint{7.842520in}{7.842520in}}%
\pgfusepath{clip}%
\pgfsetbuttcap%
\pgfsetroundjoin%
\definecolor{currentfill}{rgb}{0.123463,0.581687,0.547445}%
\pgfsetfillcolor{currentfill}%
\pgfsetlinewidth{0.000000pt}%
\definecolor{currentstroke}{rgb}{0.150148,0.676631,0.506589}%
\pgfsetstrokecolor{currentstroke}%
\pgfsetdash{}{0pt}%
\pgfpathmoveto{\pgfqpoint{4.786562in}{4.030722in}}%
\pgfpathlineto{\pgfqpoint{4.647114in}{4.229077in}}%
\pgfpathlineto{\pgfqpoint{4.566539in}{4.326950in}}%
\pgfpathclose%
\pgfusepath{fill}%
\end{pgfscope}%
\begin{pgfscope}%
\pgfpathrectangle{\pgfqpoint{0.539299in}{0.078740in}}{\pgfqpoint{7.842520in}{7.842520in}}%
\pgfusepath{clip}%
\pgfsetbuttcap%
\pgfsetroundjoin%
\definecolor{currentfill}{rgb}{0.449368,0.813768,0.335384}%
\pgfsetfillcolor{currentfill}%
\pgfsetlinewidth{0.000000pt}%
\definecolor{currentstroke}{rgb}{0.153894,0.680203,0.504172}%
\pgfsetstrokecolor{currentstroke}%
\pgfsetdash{}{0pt}%
\pgfpathmoveto{\pgfqpoint{2.954612in}{5.267216in}}%
\pgfpathlineto{\pgfqpoint{2.867996in}{5.262317in}}%
\pgfpathlineto{\pgfqpoint{3.000456in}{5.448722in}}%
\pgfpathclose%
\pgfusepath{fill}%
\end{pgfscope}%
\begin{pgfscope}%
\pgfpathrectangle{\pgfqpoint{0.539299in}{0.078740in}}{\pgfqpoint{7.842520in}{7.842520in}}%
\pgfusepath{clip}%
\pgfsetbuttcap%
\pgfsetroundjoin%
\definecolor{currentfill}{rgb}{0.515992,0.831158,0.294279}%
\pgfsetfillcolor{currentfill}%
\pgfsetlinewidth{0.000000pt}%
\definecolor{currentstroke}{rgb}{0.157851,0.683765,0.501686}%
\pgfsetstrokecolor{currentstroke}%
\pgfsetdash{}{0pt}%
\pgfpathmoveto{\pgfqpoint{3.669721in}{5.396407in}}%
\pgfpathlineto{\pgfqpoint{3.531606in}{5.447979in}}%
\pgfpathlineto{\pgfqpoint{3.584365in}{5.478002in}}%
\pgfpathclose%
\pgfusepath{fill}%
\end{pgfscope}%
\begin{pgfscope}%
\pgfpathrectangle{\pgfqpoint{0.539299in}{0.078740in}}{\pgfqpoint{7.842520in}{7.842520in}}%
\pgfusepath{clip}%
\pgfsetbuttcap%
\pgfsetroundjoin%
\definecolor{currentfill}{rgb}{0.360741,0.785964,0.387814}%
\pgfsetfillcolor{currentfill}%
\pgfsetlinewidth{0.000000pt}%
\definecolor{currentstroke}{rgb}{0.162016,0.687316,0.499129}%
\pgfsetstrokecolor{currentstroke}%
\pgfsetdash{}{0pt}%
\pgfpathmoveto{\pgfqpoint{2.867996in}{5.262317in}}%
\pgfpathlineto{\pgfqpoint{2.652170in}{4.965747in}}%
\pgfpathlineto{\pgfqpoint{2.781099in}{5.242417in}}%
\pgfpathclose%
\pgfusepath{fill}%
\end{pgfscope}%
\begin{pgfscope}%
\pgfpathrectangle{\pgfqpoint{0.539299in}{0.078740in}}{\pgfqpoint{7.842520in}{7.842520in}}%
\pgfusepath{clip}%
\pgfsetbuttcap%
\pgfsetroundjoin%
\definecolor{currentfill}{rgb}{0.131172,0.555899,0.552459}%
\pgfsetfillcolor{currentfill}%
\pgfsetlinewidth{0.000000pt}%
\definecolor{currentstroke}{rgb}{0.166383,0.690856,0.496502}%
\pgfsetstrokecolor{currentstroke}%
\pgfsetdash{}{0pt}%
\pgfpathmoveto{\pgfqpoint{1.975740in}{3.757494in}}%
\pgfpathlineto{\pgfqpoint{2.095141in}{4.184186in}}%
\pgfpathlineto{\pgfqpoint{2.181328in}{4.304148in}}%
\pgfpathclose%
\pgfusepath{fill}%
\end{pgfscope}%
\begin{pgfscope}%
\pgfpathrectangle{\pgfqpoint{0.539299in}{0.078740in}}{\pgfqpoint{7.842520in}{7.842520in}}%
\pgfusepath{clip}%
\pgfsetbuttcap%
\pgfsetroundjoin%
\definecolor{currentfill}{rgb}{0.169646,0.456262,0.558030}%
\pgfsetfillcolor{currentfill}%
\pgfsetlinewidth{0.000000pt}%
\definecolor{currentstroke}{rgb}{0.170948,0.694384,0.493803}%
\pgfsetstrokecolor{currentstroke}%
\pgfsetdash{}{0pt}%
\pgfpathmoveto{\pgfqpoint{5.065172in}{3.654551in}}%
\pgfpathlineto{\pgfqpoint{4.986662in}{3.717303in}}%
\pgfpathlineto{\pgfqpoint{5.126490in}{3.527887in}}%
\pgfpathclose%
\pgfusepath{fill}%
\end{pgfscope}%
\begin{pgfscope}%
\pgfpathrectangle{\pgfqpoint{0.539299in}{0.078740in}}{\pgfqpoint{7.842520in}{7.842520in}}%
\pgfusepath{clip}%
\pgfsetbuttcap%
\pgfsetroundjoin%
\definecolor{currentfill}{rgb}{0.280894,0.078907,0.402329}%
\pgfsetfillcolor{currentfill}%
\pgfsetlinewidth{0.000000pt}%
\definecolor{currentstroke}{rgb}{0.175707,0.697900,0.491033}%
\pgfsetstrokecolor{currentstroke}%
\pgfsetdash{}{0pt}%
\pgfpathmoveto{\pgfqpoint{6.465616in}{2.293258in}}%
\pgfpathlineto{\pgfqpoint{6.606910in}{2.181621in}}%
\pgfpathlineto{\pgfqpoint{6.681865in}{2.263405in}}%
\pgfpathclose%
\pgfusepath{fill}%
\end{pgfscope}%
\begin{pgfscope}%
\pgfpathrectangle{\pgfqpoint{0.539299in}{0.078740in}}{\pgfqpoint{7.842520in}{7.842520in}}%
\pgfusepath{clip}%
\pgfsetbuttcap%
\pgfsetroundjoin%
\definecolor{currentfill}{rgb}{0.262138,0.242286,0.520837}%
\pgfsetfillcolor{currentfill}%
\pgfsetlinewidth{0.000000pt}%
\definecolor{currentstroke}{rgb}{0.180653,0.701402,0.488189}%
\pgfsetstrokecolor{currentstroke}%
\pgfsetdash{}{0pt}%
\pgfpathmoveto{\pgfqpoint{5.686202in}{2.865893in}}%
\pgfpathlineto{\pgfqpoint{5.826491in}{2.723216in}}%
\pgfpathlineto{\pgfqpoint{5.902739in}{2.756289in}}%
\pgfpathclose%
\pgfusepath{fill}%
\end{pgfscope}%
\begin{pgfscope}%
\pgfpathrectangle{\pgfqpoint{0.539299in}{0.078740in}}{\pgfqpoint{7.842520in}{7.842520in}}%
\pgfusepath{clip}%
\pgfsetbuttcap%
\pgfsetroundjoin%
\definecolor{currentfill}{rgb}{0.487026,0.823929,0.312321}%
\pgfsetfillcolor{currentfill}%
\pgfsetlinewidth{0.000000pt}%
\definecolor{currentstroke}{rgb}{0.185783,0.704891,0.485273}%
\pgfsetstrokecolor{currentstroke}%
\pgfsetdash{}{0pt}%
\pgfpathmoveto{\pgfqpoint{3.000456in}{5.448722in}}%
\pgfpathlineto{\pgfqpoint{3.087266in}{5.438839in}}%
\pgfpathlineto{\pgfqpoint{2.954612in}{5.267216in}}%
\pgfpathclose%
\pgfusepath{fill}%
\end{pgfscope}%
\begin{pgfscope}%
\pgfpathrectangle{\pgfqpoint{0.539299in}{0.078740in}}{\pgfqpoint{7.842520in}{7.842520in}}%
\pgfusepath{clip}%
\pgfsetbuttcap%
\pgfsetroundjoin%
\definecolor{currentfill}{rgb}{0.140210,0.665859,0.513427}%
\pgfsetfillcolor{currentfill}%
\pgfsetlinewidth{0.000000pt}%
\definecolor{currentstroke}{rgb}{0.191090,0.708366,0.482284}%
\pgfsetstrokecolor{currentstroke}%
\pgfsetdash{}{0pt}%
\pgfpathmoveto{\pgfqpoint{2.267756in}{4.403873in}}%
\pgfpathlineto{\pgfqpoint{2.391827in}{4.772531in}}%
\pgfpathlineto{\pgfqpoint{2.354275in}{4.486045in}}%
\pgfpathclose%
\pgfusepath{fill}%
\end{pgfscope}%
\begin{pgfscope}%
\pgfpathrectangle{\pgfqpoint{0.539299in}{0.078740in}}{\pgfqpoint{7.842520in}{7.842520in}}%
\pgfusepath{clip}%
\pgfsetbuttcap%
\pgfsetroundjoin%
\definecolor{currentfill}{rgb}{0.269308,0.218818,0.509577}%
\pgfsetfillcolor{currentfill}%
\pgfsetlinewidth{0.000000pt}%
\definecolor{currentstroke}{rgb}{0.196571,0.711827,0.479221}%
\pgfsetstrokecolor{currentstroke}%
\pgfsetdash{}{0pt}%
\pgfpathmoveto{\pgfqpoint{5.826491in}{2.723216in}}%
\pgfpathlineto{\pgfqpoint{5.967002in}{2.588084in}}%
\pgfpathlineto{\pgfqpoint{5.902739in}{2.756289in}}%
\pgfpathclose%
\pgfusepath{fill}%
\end{pgfscope}%
\begin{pgfscope}%
\pgfpathrectangle{\pgfqpoint{0.539299in}{0.078740in}}{\pgfqpoint{7.842520in}{7.842520in}}%
\pgfusepath{clip}%
\pgfsetbuttcap%
\pgfsetroundjoin%
\definecolor{currentfill}{rgb}{0.121380,0.629492,0.531973}%
\pgfsetfillcolor{currentfill}%
\pgfsetlinewidth{0.000000pt}%
\definecolor{currentstroke}{rgb}{0.202219,0.715272,0.476084}%
\pgfsetstrokecolor{currentstroke}%
\pgfsetdash{}{0pt}%
\pgfpathmoveto{\pgfqpoint{4.647114in}{4.229077in}}%
\pgfpathlineto{\pgfqpoint{4.507522in}{4.430367in}}%
\pgfpathlineto{\pgfqpoint{4.426173in}{4.535091in}}%
\pgfpathclose%
\pgfusepath{fill}%
\end{pgfscope}%
\begin{pgfscope}%
\pgfpathrectangle{\pgfqpoint{0.539299in}{0.078740in}}{\pgfqpoint{7.842520in}{7.842520in}}%
\pgfusepath{clip}%
\pgfsetbuttcap%
\pgfsetroundjoin%
\definecolor{currentfill}{rgb}{0.283187,0.125848,0.444960}%
\pgfsetfillcolor{currentfill}%
\pgfsetlinewidth{0.000000pt}%
\definecolor{currentstroke}{rgb}{0.208030,0.718701,0.472873}%
\pgfsetstrokecolor{currentstroke}%
\pgfsetdash{}{0pt}%
\pgfpathmoveto{\pgfqpoint{1.388975in}{2.359780in}}%
\pgfpathlineto{\pgfqpoint{1.344269in}{2.477731in}}%
\pgfpathlineto{\pgfqpoint{1.260140in}{2.275047in}}%
\pgfpathclose%
\pgfusepath{fill}%
\end{pgfscope}%
\begin{pgfscope}%
\pgfpathrectangle{\pgfqpoint{0.539299in}{0.078740in}}{\pgfqpoint{7.842520in}{7.842520in}}%
\pgfusepath{clip}%
\pgfsetbuttcap%
\pgfsetroundjoin%
\definecolor{currentfill}{rgb}{0.151918,0.500685,0.557587}%
\pgfsetfillcolor{currentfill}%
\pgfsetlinewidth{0.000000pt}%
\definecolor{currentstroke}{rgb}{0.214000,0.722114,0.469588}%
\pgfsetstrokecolor{currentstroke}%
\pgfsetdash{}{0pt}%
\pgfpathmoveto{\pgfqpoint{4.846758in}{3.914972in}}%
\pgfpathlineto{\pgfqpoint{4.986662in}{3.717303in}}%
\pgfpathlineto{\pgfqpoint{4.925899in}{3.838472in}}%
\pgfpathclose%
\pgfusepath{fill}%
\end{pgfscope}%
\begin{pgfscope}%
\pgfpathrectangle{\pgfqpoint{0.539299in}{0.078740in}}{\pgfqpoint{7.842520in}{7.842520in}}%
\pgfusepath{clip}%
\pgfsetbuttcap%
\pgfsetroundjoin%
\definecolor{currentfill}{rgb}{0.278791,0.062145,0.386592}%
\pgfsetfillcolor{currentfill}%
\pgfsetlinewidth{0.000000pt}%
\definecolor{currentstroke}{rgb}{0.220124,0.725509,0.466226}%
\pgfsetstrokecolor{currentstroke}%
\pgfsetdash{}{0pt}%
\pgfpathmoveto{\pgfqpoint{6.681865in}{2.263405in}}%
\pgfpathlineto{\pgfqpoint{6.606910in}{2.181621in}}%
\pgfpathlineto{\pgfqpoint{6.748358in}{2.068409in}}%
\pgfpathclose%
\pgfusepath{fill}%
\end{pgfscope}%
\begin{pgfscope}%
\pgfpathrectangle{\pgfqpoint{0.539299in}{0.078740in}}{\pgfqpoint{7.842520in}{7.842520in}}%
\pgfusepath{clip}%
\pgfsetbuttcap%
\pgfsetroundjoin%
\definecolor{currentfill}{rgb}{0.282623,0.140926,0.457517}%
\pgfsetfillcolor{currentfill}%
\pgfsetlinewidth{0.000000pt}%
\definecolor{currentstroke}{rgb}{0.226397,0.728888,0.462789}%
\pgfsetstrokecolor{currentstroke}%
\pgfsetdash{}{0pt}%
\pgfpathmoveto{\pgfqpoint{6.324525in}{2.404996in}}%
\pgfpathlineto{\pgfqpoint{6.183670in}{2.518396in}}%
\pgfpathlineto{\pgfqpoint{6.248772in}{2.336093in}}%
\pgfpathclose%
\pgfusepath{fill}%
\end{pgfscope}%
\begin{pgfscope}%
\pgfpathrectangle{\pgfqpoint{0.539299in}{0.078740in}}{\pgfqpoint{7.842520in}{7.842520in}}%
\pgfusepath{clip}%
\pgfsetbuttcap%
\pgfsetroundjoin%
\definecolor{currentfill}{rgb}{0.496615,0.826376,0.306377}%
\pgfsetfillcolor{currentfill}%
\pgfsetlinewidth{0.000000pt}%
\definecolor{currentstroke}{rgb}{0.232815,0.732247,0.459277}%
\pgfsetstrokecolor{currentstroke}%
\pgfsetdash{}{0pt}%
\pgfpathmoveto{\pgfqpoint{3.808702in}{5.299454in}}%
\pgfpathlineto{\pgfqpoint{3.669721in}{5.396407in}}%
\pgfpathlineto{\pgfqpoint{3.584365in}{5.478002in}}%
\pgfpathclose%
\pgfusepath{fill}%
\end{pgfscope}%
\begin{pgfscope}%
\pgfpathrectangle{\pgfqpoint{0.539299in}{0.078740in}}{\pgfqpoint{7.842520in}{7.842520in}}%
\pgfusepath{clip}%
\pgfsetbuttcap%
\pgfsetroundjoin%
\definecolor{currentfill}{rgb}{0.159194,0.482237,0.558073}%
\pgfsetfillcolor{currentfill}%
\pgfsetlinewidth{0.000000pt}%
\definecolor{currentstroke}{rgb}{0.239374,0.735588,0.455688}%
\pgfsetstrokecolor{currentstroke}%
\pgfsetdash{}{0pt}%
\pgfpathmoveto{\pgfqpoint{2.009371in}{4.040892in}}%
\pgfpathlineto{\pgfqpoint{1.889628in}{3.636546in}}%
\pgfpathlineto{\pgfqpoint{1.803826in}{3.497087in}}%
\pgfpathclose%
\pgfusepath{fill}%
\end{pgfscope}%
\begin{pgfscope}%
\pgfpathrectangle{\pgfqpoint{0.539299in}{0.078740in}}{\pgfqpoint{7.842520in}{7.842520in}}%
\pgfusepath{clip}%
\pgfsetbuttcap%
\pgfsetroundjoin%
\definecolor{currentfill}{rgb}{0.137339,0.662252,0.515571}%
\pgfsetfillcolor{currentfill}%
\pgfsetlinewidth{0.000000pt}%
\definecolor{currentstroke}{rgb}{0.246070,0.738910,0.452024}%
\pgfsetstrokecolor{currentstroke}%
\pgfsetdash{}{0pt}%
\pgfpathmoveto{\pgfqpoint{4.507522in}{4.430367in}}%
\pgfpathlineto{\pgfqpoint{4.367781in}{4.630418in}}%
\pgfpathlineto{\pgfqpoint{4.426173in}{4.535091in}}%
\pgfpathclose%
\pgfusepath{fill}%
\end{pgfscope}%
\begin{pgfscope}%
\pgfpathrectangle{\pgfqpoint{0.539299in}{0.078740in}}{\pgfqpoint{7.842520in}{7.842520in}}%
\pgfusepath{clip}%
\pgfsetbuttcap%
\pgfsetroundjoin%
\definecolor{currentfill}{rgb}{0.139147,0.533812,0.555298}%
\pgfsetfillcolor{currentfill}%
\pgfsetlinewidth{0.000000pt}%
\definecolor{currentstroke}{rgb}{0.252899,0.742211,0.448284}%
\pgfsetstrokecolor{currentstroke}%
\pgfsetdash{}{0pt}%
\pgfpathmoveto{\pgfqpoint{4.925899in}{3.838472in}}%
\pgfpathlineto{\pgfqpoint{4.706729in}{4.119068in}}%
\pgfpathlineto{\pgfqpoint{4.846758in}{3.914972in}}%
\pgfpathclose%
\pgfusepath{fill}%
\end{pgfscope}%
\begin{pgfscope}%
\pgfpathrectangle{\pgfqpoint{0.539299in}{0.078740in}}{\pgfqpoint{7.842520in}{7.842520in}}%
\pgfusepath{clip}%
\pgfsetbuttcap%
\pgfsetroundjoin%
\definecolor{currentfill}{rgb}{0.246811,0.283237,0.535941}%
\pgfsetfillcolor{currentfill}%
\pgfsetlinewidth{0.000000pt}%
\definecolor{currentstroke}{rgb}{0.259857,0.745492,0.444467}%
\pgfsetstrokecolor{currentstroke}%
\pgfsetdash{}{0pt}%
\pgfpathmoveto{\pgfqpoint{1.429595in}{2.648382in}}%
\pgfpathlineto{\pgfqpoint{1.550510in}{2.929884in}}%
\pgfpathlineto{\pgfqpoint{1.633975in}{3.148721in}}%
\pgfpathclose%
\pgfusepath{fill}%
\end{pgfscope}%
\begin{pgfscope}%
\pgfpathrectangle{\pgfqpoint{0.539299in}{0.078740in}}{\pgfqpoint{7.842520in}{7.842520in}}%
\pgfusepath{clip}%
\pgfsetbuttcap%
\pgfsetroundjoin%
\definecolor{currentfill}{rgb}{0.555484,0.840254,0.269281}%
\pgfsetfillcolor{currentfill}%
\pgfsetlinewidth{0.000000pt}%
\definecolor{currentstroke}{rgb}{0.266941,0.748751,0.440573}%
\pgfsetstrokecolor{currentstroke}%
\pgfsetdash{}{0pt}%
\pgfpathmoveto{\pgfqpoint{3.584365in}{5.478002in}}%
\pgfpathlineto{\pgfqpoint{3.531606in}{5.447979in}}%
\pgfpathlineto{\pgfqpoint{3.445912in}{5.515993in}}%
\pgfpathclose%
\pgfusepath{fill}%
\end{pgfscope}%
\begin{pgfscope}%
\pgfpathrectangle{\pgfqpoint{0.539299in}{0.078740in}}{\pgfqpoint{7.842520in}{7.842520in}}%
\pgfusepath{clip}%
\pgfsetbuttcap%
\pgfsetroundjoin%
\definecolor{currentfill}{rgb}{0.283229,0.120777,0.440584}%
\pgfsetfillcolor{currentfill}%
\pgfsetlinewidth{0.000000pt}%
\definecolor{currentstroke}{rgb}{0.274149,0.751988,0.436601}%
\pgfsetstrokecolor{currentstroke}%
\pgfsetdash{}{0pt}%
\pgfpathmoveto{\pgfqpoint{6.324525in}{2.404996in}}%
\pgfpathlineto{\pgfqpoint{6.248772in}{2.336093in}}%
\pgfpathlineto{\pgfqpoint{6.465616in}{2.293258in}}%
\pgfpathclose%
\pgfusepath{fill}%
\end{pgfscope}%
\begin{pgfscope}%
\pgfpathrectangle{\pgfqpoint{0.539299in}{0.078740in}}{\pgfqpoint{7.842520in}{7.842520in}}%
\pgfusepath{clip}%
\pgfsetbuttcap%
\pgfsetroundjoin%
\definecolor{currentfill}{rgb}{0.278826,0.175490,0.483397}%
\pgfsetfillcolor{currentfill}%
\pgfsetlinewidth{0.000000pt}%
\definecolor{currentstroke}{rgb}{0.281477,0.755203,0.432552}%
\pgfsetstrokecolor{currentstroke}%
\pgfsetdash{}{0pt}%
\pgfpathmoveto{\pgfqpoint{6.183670in}{2.518396in}}%
\pgfpathlineto{\pgfqpoint{5.967002in}{2.588084in}}%
\pgfpathlineto{\pgfqpoint{6.107758in}{2.459413in}}%
\pgfpathclose%
\pgfusepath{fill}%
\end{pgfscope}%
\begin{pgfscope}%
\pgfpathrectangle{\pgfqpoint{0.539299in}{0.078740in}}{\pgfqpoint{7.842520in}{7.842520in}}%
\pgfusepath{clip}%
\pgfsetbuttcap%
\pgfsetroundjoin%
\definecolor{currentfill}{rgb}{0.565498,0.842430,0.262877}%
\pgfsetfillcolor{currentfill}%
\pgfsetlinewidth{0.000000pt}%
\definecolor{currentstroke}{rgb}{0.288921,0.758394,0.428426}%
\pgfsetstrokecolor{currentstroke}%
\pgfsetdash{}{0pt}%
\pgfpathmoveto{\pgfqpoint{3.222442in}{5.537820in}}%
\pgfpathlineto{\pgfqpoint{3.308861in}{5.497668in}}%
\pgfpathlineto{\pgfqpoint{3.087266in}{5.438839in}}%
\pgfpathclose%
\pgfusepath{fill}%
\end{pgfscope}%
\begin{pgfscope}%
\pgfpathrectangle{\pgfqpoint{0.539299in}{0.078740in}}{\pgfqpoint{7.842520in}{7.842520in}}%
\pgfusepath{clip}%
\pgfsetbuttcap%
\pgfsetroundjoin%
\definecolor{currentfill}{rgb}{0.311925,0.767822,0.415586}%
\pgfsetfillcolor{currentfill}%
\pgfsetlinewidth{0.000000pt}%
\definecolor{currentstroke}{rgb}{0.296479,0.761561,0.424223}%
\pgfsetstrokecolor{currentstroke}%
\pgfsetdash{}{0pt}%
\pgfpathmoveto{\pgfqpoint{2.652170in}{4.965747in}}%
\pgfpathlineto{\pgfqpoint{2.565417in}{4.919366in}}%
\pgfpathlineto{\pgfqpoint{2.694035in}{5.204781in}}%
\pgfpathclose%
\pgfusepath{fill}%
\end{pgfscope}%
\begin{pgfscope}%
\pgfpathrectangle{\pgfqpoint{0.539299in}{0.078740in}}{\pgfqpoint{7.842520in}{7.842520in}}%
\pgfusepath{clip}%
\pgfsetbuttcap%
\pgfsetroundjoin%
\definecolor{currentfill}{rgb}{0.123463,0.581687,0.547445}%
\pgfsetfillcolor{currentfill}%
\pgfsetlinewidth{0.000000pt}%
\definecolor{currentstroke}{rgb}{0.304148,0.764704,0.419943}%
\pgfsetstrokecolor{currentstroke}%
\pgfsetdash{}{0pt}%
\pgfpathmoveto{\pgfqpoint{4.566539in}{4.326950in}}%
\pgfpathlineto{\pgfqpoint{4.706729in}{4.119068in}}%
\pgfpathlineto{\pgfqpoint{4.786562in}{4.030722in}}%
\pgfpathclose%
\pgfusepath{fill}%
\end{pgfscope}%
\begin{pgfscope}%
\pgfpathrectangle{\pgfqpoint{0.539299in}{0.078740in}}{\pgfqpoint{7.842520in}{7.842520in}}%
\pgfusepath{clip}%
\pgfsetbuttcap%
\pgfsetroundjoin%
\definecolor{currentfill}{rgb}{0.281887,0.150881,0.465405}%
\pgfsetfillcolor{currentfill}%
\pgfsetlinewidth{0.000000pt}%
\definecolor{currentstroke}{rgb}{0.311925,0.767822,0.415586}%
\pgfsetstrokecolor{currentstroke}%
\pgfsetdash{}{0pt}%
\pgfpathmoveto{\pgfqpoint{6.248772in}{2.336093in}}%
\pgfpathlineto{\pgfqpoint{6.183670in}{2.518396in}}%
\pgfpathlineto{\pgfqpoint{6.107758in}{2.459413in}}%
\pgfpathclose%
\pgfusepath{fill}%
\end{pgfscope}%
\begin{pgfscope}%
\pgfpathrectangle{\pgfqpoint{0.539299in}{0.078740in}}{\pgfqpoint{7.842520in}{7.842520in}}%
\pgfusepath{clip}%
\pgfsetbuttcap%
\pgfsetroundjoin%
\definecolor{currentfill}{rgb}{0.214000,0.722114,0.469588}%
\pgfsetfillcolor{currentfill}%
\pgfsetlinewidth{0.000000pt}%
\definecolor{currentstroke}{rgb}{0.319809,0.770914,0.411152}%
\pgfsetstrokecolor{currentstroke}%
\pgfsetdash{}{0pt}%
\pgfpathmoveto{\pgfqpoint{4.145020in}{4.933425in}}%
\pgfpathlineto{\pgfqpoint{4.367781in}{4.630418in}}%
\pgfpathlineto{\pgfqpoint{4.227924in}{4.824025in}}%
\pgfpathclose%
\pgfusepath{fill}%
\end{pgfscope}%
\begin{pgfscope}%
\pgfpathrectangle{\pgfqpoint{0.539299in}{0.078740in}}{\pgfqpoint{7.842520in}{7.842520in}}%
\pgfusepath{clip}%
\pgfsetbuttcap%
\pgfsetroundjoin%
\definecolor{currentfill}{rgb}{0.449368,0.813768,0.335384}%
\pgfsetfillcolor{currentfill}%
\pgfsetlinewidth{0.000000pt}%
\definecolor{currentstroke}{rgb}{0.327796,0.773980,0.406640}%
\pgfsetstrokecolor{currentstroke}%
\pgfsetdash{}{0pt}%
\pgfpathmoveto{\pgfqpoint{3.723815in}{5.392306in}}%
\pgfpathlineto{\pgfqpoint{3.948223in}{5.166109in}}%
\pgfpathlineto{\pgfqpoint{3.808702in}{5.299454in}}%
\pgfpathclose%
\pgfusepath{fill}%
\end{pgfscope}%
\begin{pgfscope}%
\pgfpathrectangle{\pgfqpoint{0.539299in}{0.078740in}}{\pgfqpoint{7.842520in}{7.842520in}}%
\pgfusepath{clip}%
\pgfsetbuttcap%
\pgfsetroundjoin%
\definecolor{currentfill}{rgb}{0.278826,0.175490,0.483397}%
\pgfsetfillcolor{currentfill}%
\pgfsetlinewidth{0.000000pt}%
\definecolor{currentstroke}{rgb}{0.335885,0.777018,0.402049}%
\pgfsetstrokecolor{currentstroke}%
\pgfsetdash{}{0pt}%
\pgfpathmoveto{\pgfqpoint{1.468606in}{2.671104in}}%
\pgfpathlineto{\pgfqpoint{1.344269in}{2.477731in}}%
\pgfpathlineto{\pgfqpoint{1.388975in}{2.359780in}}%
\pgfpathclose%
\pgfusepath{fill}%
\end{pgfscope}%
\begin{pgfscope}%
\pgfpathrectangle{\pgfqpoint{0.539299in}{0.078740in}}{\pgfqpoint{7.842520in}{7.842520in}}%
\pgfusepath{clip}%
\pgfsetbuttcap%
\pgfsetroundjoin%
\definecolor{currentfill}{rgb}{0.296479,0.761561,0.424223}%
\pgfsetfillcolor{currentfill}%
\pgfsetlinewidth{0.000000pt}%
\definecolor{currentstroke}{rgb}{0.344074,0.780029,0.397381}%
\pgfsetstrokecolor{currentstroke}%
\pgfsetdash{}{0pt}%
\pgfpathmoveto{\pgfqpoint{4.227924in}{4.824025in}}%
\pgfpathlineto{\pgfqpoint{4.088028in}{5.004974in}}%
\pgfpathlineto{\pgfqpoint{4.004391in}{5.111952in}}%
\pgfpathclose%
\pgfusepath{fill}%
\end{pgfscope}%
\begin{pgfscope}%
\pgfpathrectangle{\pgfqpoint{0.539299in}{0.078740in}}{\pgfqpoint{7.842520in}{7.842520in}}%
\pgfusepath{clip}%
\pgfsetbuttcap%
\pgfsetroundjoin%
\definecolor{currentfill}{rgb}{0.266580,0.228262,0.514349}%
\pgfsetfillcolor{currentfill}%
\pgfsetlinewidth{0.000000pt}%
\definecolor{currentstroke}{rgb}{0.352360,0.783011,0.392636}%
\pgfsetstrokecolor{currentstroke}%
\pgfsetdash{}{0pt}%
\pgfpathmoveto{\pgfqpoint{1.550510in}{2.929884in}}%
\pgfpathlineto{\pgfqpoint{1.344269in}{2.477731in}}%
\pgfpathlineto{\pgfqpoint{1.468606in}{2.671104in}}%
\pgfpathclose%
\pgfusepath{fill}%
\end{pgfscope}%
\begin{pgfscope}%
\pgfpathrectangle{\pgfqpoint{0.539299in}{0.078740in}}{\pgfqpoint{7.842520in}{7.842520in}}%
\pgfusepath{clip}%
\pgfsetbuttcap%
\pgfsetroundjoin%
\definecolor{currentfill}{rgb}{0.185783,0.704891,0.485273}%
\pgfsetfillcolor{currentfill}%
\pgfsetlinewidth{0.000000pt}%
\definecolor{currentstroke}{rgb}{0.360741,0.785964,0.387814}%
\pgfsetstrokecolor{currentstroke}%
\pgfsetdash{}{0pt}%
\pgfpathmoveto{\pgfqpoint{2.354275in}{4.486045in}}%
\pgfpathlineto{\pgfqpoint{2.391827in}{4.772531in}}%
\pgfpathlineto{\pgfqpoint{2.478591in}{4.855876in}}%
\pgfpathclose%
\pgfusepath{fill}%
\end{pgfscope}%
\begin{pgfscope}%
\pgfpathrectangle{\pgfqpoint{0.539299in}{0.078740in}}{\pgfqpoint{7.842520in}{7.842520in}}%
\pgfusepath{clip}%
\pgfsetbuttcap%
\pgfsetroundjoin%
\definecolor{currentfill}{rgb}{0.352360,0.783011,0.392636}%
\pgfsetfillcolor{currentfill}%
\pgfsetlinewidth{0.000000pt}%
\definecolor{currentstroke}{rgb}{0.369214,0.788888,0.382914}%
\pgfsetstrokecolor{currentstroke}%
\pgfsetdash{}{0pt}%
\pgfpathmoveto{\pgfqpoint{4.088028in}{5.004974in}}%
\pgfpathlineto{\pgfqpoint{3.948223in}{5.166109in}}%
\pgfpathlineto{\pgfqpoint{4.004391in}{5.111952in}}%
\pgfpathclose%
\pgfusepath{fill}%
\end{pgfscope}%
\begin{pgfscope}%
\pgfpathrectangle{\pgfqpoint{0.539299in}{0.078740in}}{\pgfqpoint{7.842520in}{7.842520in}}%
\pgfusepath{clip}%
\pgfsetbuttcap%
\pgfsetroundjoin%
\definecolor{currentfill}{rgb}{0.280894,0.078907,0.402329}%
\pgfsetfillcolor{currentfill}%
\pgfsetlinewidth{0.000000pt}%
\definecolor{currentstroke}{rgb}{0.377779,0.791781,0.377939}%
\pgfsetstrokecolor{currentstroke}%
\pgfsetdash{}{0pt}%
\pgfpathmoveto{\pgfqpoint{6.465616in}{2.293258in}}%
\pgfpathlineto{\pgfqpoint{6.531589in}{2.101217in}}%
\pgfpathlineto{\pgfqpoint{6.606910in}{2.181621in}}%
\pgfpathclose%
\pgfusepath{fill}%
\end{pgfscope}%
\begin{pgfscope}%
\pgfpathrectangle{\pgfqpoint{0.539299in}{0.078740in}}{\pgfqpoint{7.842520in}{7.842520in}}%
\pgfusepath{clip}%
\pgfsetbuttcap%
\pgfsetroundjoin%
\definecolor{currentfill}{rgb}{0.121380,0.629492,0.531973}%
\pgfsetfillcolor{currentfill}%
\pgfsetlinewidth{0.000000pt}%
\definecolor{currentstroke}{rgb}{0.386433,0.794644,0.372886}%
\pgfsetstrokecolor{currentstroke}%
\pgfsetdash{}{0pt}%
\pgfpathmoveto{\pgfqpoint{4.566539in}{4.326950in}}%
\pgfpathlineto{\pgfqpoint{4.647114in}{4.229077in}}%
\pgfpathlineto{\pgfqpoint{4.426173in}{4.535091in}}%
\pgfpathclose%
\pgfusepath{fill}%
\end{pgfscope}%
\begin{pgfscope}%
\pgfpathrectangle{\pgfqpoint{0.539299in}{0.078740in}}{\pgfqpoint{7.842520in}{7.842520in}}%
\pgfusepath{clip}%
\pgfsetbuttcap%
\pgfsetroundjoin%
\definecolor{currentfill}{rgb}{0.134692,0.658636,0.517649}%
\pgfsetfillcolor{currentfill}%
\pgfsetlinewidth{0.000000pt}%
\definecolor{currentstroke}{rgb}{0.395174,0.797475,0.367757}%
\pgfsetstrokecolor{currentstroke}%
\pgfsetdash{}{0pt}%
\pgfpathmoveto{\pgfqpoint{2.181328in}{4.304148in}}%
\pgfpathlineto{\pgfqpoint{2.391827in}{4.772531in}}%
\pgfpathlineto{\pgfqpoint{2.267756in}{4.403873in}}%
\pgfpathclose%
\pgfusepath{fill}%
\end{pgfscope}%
\begin{pgfscope}%
\pgfpathrectangle{\pgfqpoint{0.539299in}{0.078740in}}{\pgfqpoint{7.842520in}{7.842520in}}%
\pgfusepath{clip}%
\pgfsetbuttcap%
\pgfsetroundjoin%
\definecolor{currentfill}{rgb}{0.136408,0.541173,0.554483}%
\pgfsetfillcolor{currentfill}%
\pgfsetlinewidth{0.000000pt}%
\definecolor{currentstroke}{rgb}{0.404001,0.800275,0.362552}%
\pgfsetstrokecolor{currentstroke}%
\pgfsetdash{}{0pt}%
\pgfpathmoveto{\pgfqpoint{1.889628in}{3.636546in}}%
\pgfpathlineto{\pgfqpoint{2.009371in}{4.040892in}}%
\pgfpathlineto{\pgfqpoint{2.095141in}{4.184186in}}%
\pgfpathclose%
\pgfusepath{fill}%
\end{pgfscope}%
\begin{pgfscope}%
\pgfpathrectangle{\pgfqpoint{0.539299in}{0.078740in}}{\pgfqpoint{7.842520in}{7.842520in}}%
\pgfusepath{clip}%
\pgfsetbuttcap%
\pgfsetroundjoin%
\definecolor{currentfill}{rgb}{0.283091,0.110553,0.431554}%
\pgfsetfillcolor{currentfill}%
\pgfsetlinewidth{0.000000pt}%
\definecolor{currentstroke}{rgb}{0.412913,0.803041,0.357269}%
\pgfsetstrokecolor{currentstroke}%
\pgfsetdash{}{0pt}%
\pgfpathmoveto{\pgfqpoint{6.465616in}{2.293258in}}%
\pgfpathlineto{\pgfqpoint{6.248772in}{2.336093in}}%
\pgfpathlineto{\pgfqpoint{6.390049in}{2.217039in}}%
\pgfpathclose%
\pgfusepath{fill}%
\end{pgfscope}%
\begin{pgfscope}%
\pgfpathrectangle{\pgfqpoint{0.539299in}{0.078740in}}{\pgfqpoint{7.842520in}{7.842520in}}%
\pgfusepath{clip}%
\pgfsetbuttcap%
\pgfsetroundjoin%
\definecolor{currentfill}{rgb}{0.377779,0.791781,0.377939}%
\pgfsetfillcolor{currentfill}%
\pgfsetlinewidth{0.000000pt}%
\definecolor{currentstroke}{rgb}{0.421908,0.805774,0.351910}%
\pgfsetstrokecolor{currentstroke}%
\pgfsetdash{}{0pt}%
\pgfpathmoveto{\pgfqpoint{2.781099in}{5.242417in}}%
\pgfpathlineto{\pgfqpoint{2.652170in}{4.965747in}}%
\pgfpathlineto{\pgfqpoint{2.694035in}{5.204781in}}%
\pgfpathclose%
\pgfusepath{fill}%
\end{pgfscope}%
\begin{pgfscope}%
\pgfpathrectangle{\pgfqpoint{0.539299in}{0.078740in}}{\pgfqpoint{7.842520in}{7.842520in}}%
\pgfusepath{clip}%
\pgfsetbuttcap%
\pgfsetroundjoin%
\definecolor{currentfill}{rgb}{0.241237,0.296485,0.539709}%
\pgfsetfillcolor{currentfill}%
\pgfsetlinewidth{0.000000pt}%
\definecolor{currentstroke}{rgb}{0.430983,0.808473,0.346476}%
\pgfsetstrokecolor{currentstroke}%
\pgfsetdash{}{0pt}%
\pgfpathmoveto{\pgfqpoint{5.686202in}{2.865893in}}%
\pgfpathlineto{\pgfqpoint{5.546103in}{3.017088in}}%
\pgfpathlineto{\pgfqpoint{5.609380in}{2.861680in}}%
\pgfpathclose%
\pgfusepath{fill}%
\end{pgfscope}%
\begin{pgfscope}%
\pgfpathrectangle{\pgfqpoint{0.539299in}{0.078740in}}{\pgfqpoint{7.842520in}{7.842520in}}%
\pgfusepath{clip}%
\pgfsetbuttcap%
\pgfsetroundjoin%
\definecolor{currentfill}{rgb}{0.277941,0.056324,0.381191}%
\pgfsetfillcolor{currentfill}%
\pgfsetlinewidth{0.000000pt}%
\definecolor{currentstroke}{rgb}{0.440137,0.811138,0.340967}%
\pgfsetstrokecolor{currentstroke}%
\pgfsetdash{}{0pt}%
\pgfpathmoveto{\pgfqpoint{6.606910in}{2.181621in}}%
\pgfpathlineto{\pgfqpoint{6.531589in}{2.101217in}}%
\pgfpathlineto{\pgfqpoint{6.748358in}{2.068409in}}%
\pgfpathclose%
\pgfusepath{fill}%
\end{pgfscope}%
\begin{pgfscope}%
\pgfpathrectangle{\pgfqpoint{0.539299in}{0.078740in}}{\pgfqpoint{7.842520in}{7.842520in}}%
\pgfusepath{clip}%
\pgfsetbuttcap%
\pgfsetroundjoin%
\definecolor{currentfill}{rgb}{0.216210,0.351535,0.550627}%
\pgfsetfillcolor{currentfill}%
\pgfsetlinewidth{0.000000pt}%
\definecolor{currentstroke}{rgb}{0.449368,0.813768,0.335384}%
\pgfsetstrokecolor{currentstroke}%
\pgfsetdash{}{0pt}%
\pgfpathmoveto{\pgfqpoint{5.546103in}{3.017088in}}%
\pgfpathlineto{\pgfqpoint{5.406152in}{3.177575in}}%
\pgfpathlineto{\pgfqpoint{5.328594in}{3.205423in}}%
\pgfpathclose%
\pgfusepath{fill}%
\end{pgfscope}%
\begin{pgfscope}%
\pgfpathrectangle{\pgfqpoint{0.539299in}{0.078740in}}{\pgfqpoint{7.842520in}{7.842520in}}%
\pgfusepath{clip}%
\pgfsetbuttcap%
\pgfsetroundjoin%
\definecolor{currentfill}{rgb}{0.203063,0.379716,0.553925}%
\pgfsetfillcolor{currentfill}%
\pgfsetlinewidth{0.000000pt}%
\definecolor{currentstroke}{rgb}{0.458674,0.816363,0.329727}%
\pgfsetstrokecolor{currentstroke}%
\pgfsetdash{}{0pt}%
\pgfpathmoveto{\pgfqpoint{5.406152in}{3.177575in}}%
\pgfpathlineto{\pgfqpoint{5.266300in}{3.347822in}}%
\pgfpathlineto{\pgfqpoint{5.328594in}{3.205423in}}%
\pgfpathclose%
\pgfusepath{fill}%
\end{pgfscope}%
\begin{pgfscope}%
\pgfpathrectangle{\pgfqpoint{0.539299in}{0.078740in}}{\pgfqpoint{7.842520in}{7.842520in}}%
\pgfusepath{clip}%
\pgfsetbuttcap%
\pgfsetroundjoin%
\definecolor{currentfill}{rgb}{0.252194,0.269783,0.531579}%
\pgfsetfillcolor{currentfill}%
\pgfsetlinewidth{0.000000pt}%
\definecolor{currentstroke}{rgb}{0.468053,0.818921,0.323998}%
\pgfsetstrokecolor{currentstroke}%
\pgfsetdash{}{0pt}%
\pgfpathmoveto{\pgfqpoint{5.609380in}{2.861680in}}%
\pgfpathlineto{\pgfqpoint{5.826491in}{2.723216in}}%
\pgfpathlineto{\pgfqpoint{5.686202in}{2.865893in}}%
\pgfpathclose%
\pgfusepath{fill}%
\end{pgfscope}%
\begin{pgfscope}%
\pgfpathrectangle{\pgfqpoint{0.539299in}{0.078740in}}{\pgfqpoint{7.842520in}{7.842520in}}%
\pgfusepath{clip}%
\pgfsetbuttcap%
\pgfsetroundjoin%
\definecolor{currentfill}{rgb}{0.281924,0.089666,0.412415}%
\pgfsetfillcolor{currentfill}%
\pgfsetlinewidth{0.000000pt}%
\definecolor{currentstroke}{rgb}{0.477504,0.821444,0.318195}%
\pgfsetstrokecolor{currentstroke}%
\pgfsetdash{}{0pt}%
\pgfpathmoveto{\pgfqpoint{6.390049in}{2.217039in}}%
\pgfpathlineto{\pgfqpoint{6.531589in}{2.101217in}}%
\pgfpathlineto{\pgfqpoint{6.465616in}{2.293258in}}%
\pgfpathclose%
\pgfusepath{fill}%
\end{pgfscope}%
\begin{pgfscope}%
\pgfpathrectangle{\pgfqpoint{0.539299in}{0.078740in}}{\pgfqpoint{7.842520in}{7.842520in}}%
\pgfusepath{clip}%
\pgfsetbuttcap%
\pgfsetroundjoin%
\definecolor{currentfill}{rgb}{0.183898,0.422383,0.556944}%
\pgfsetfillcolor{currentfill}%
\pgfsetlinewidth{0.000000pt}%
\definecolor{currentstroke}{rgb}{0.487026,0.823929,0.312321}%
\pgfsetstrokecolor{currentstroke}%
\pgfsetdash{}{0pt}%
\pgfpathmoveto{\pgfqpoint{5.188263in}{3.391183in}}%
\pgfpathlineto{\pgfqpoint{5.266300in}{3.347822in}}%
\pgfpathlineto{\pgfqpoint{5.126490in}{3.527887in}}%
\pgfpathclose%
\pgfusepath{fill}%
\end{pgfscope}%
\begin{pgfscope}%
\pgfpathrectangle{\pgfqpoint{0.539299in}{0.078740in}}{\pgfqpoint{7.842520in}{7.842520in}}%
\pgfusepath{clip}%
\pgfsetbuttcap%
\pgfsetroundjoin%
\definecolor{currentfill}{rgb}{0.525776,0.833491,0.288127}%
\pgfsetfillcolor{currentfill}%
\pgfsetlinewidth{0.000000pt}%
\definecolor{currentstroke}{rgb}{0.496615,0.826376,0.306377}%
\pgfsetstrokecolor{currentstroke}%
\pgfsetdash{}{0pt}%
\pgfpathmoveto{\pgfqpoint{3.584365in}{5.478002in}}%
\pgfpathlineto{\pgfqpoint{3.723815in}{5.392306in}}%
\pgfpathlineto{\pgfqpoint{3.808702in}{5.299454in}}%
\pgfpathclose%
\pgfusepath{fill}%
\end{pgfscope}%
\begin{pgfscope}%
\pgfpathrectangle{\pgfqpoint{0.539299in}{0.078740in}}{\pgfqpoint{7.842520in}{7.842520in}}%
\pgfusepath{clip}%
\pgfsetbuttcap%
\pgfsetroundjoin%
\definecolor{currentfill}{rgb}{0.170948,0.694384,0.493803}%
\pgfsetfillcolor{currentfill}%
\pgfsetlinewidth{0.000000pt}%
\definecolor{currentstroke}{rgb}{0.506271,0.828786,0.300362}%
\pgfsetstrokecolor{currentstroke}%
\pgfsetdash{}{0pt}%
\pgfpathmoveto{\pgfqpoint{4.426173in}{4.535091in}}%
\pgfpathlineto{\pgfqpoint{4.367781in}{4.630418in}}%
\pgfpathlineto{\pgfqpoint{4.285648in}{4.739044in}}%
\pgfpathclose%
\pgfusepath{fill}%
\end{pgfscope}%
\begin{pgfscope}%
\pgfpathrectangle{\pgfqpoint{0.539299in}{0.078740in}}{\pgfqpoint{7.842520in}{7.842520in}}%
\pgfusepath{clip}%
\pgfsetbuttcap%
\pgfsetroundjoin%
\definecolor{currentfill}{rgb}{0.606045,0.850733,0.236712}%
\pgfsetfillcolor{currentfill}%
\pgfsetlinewidth{0.000000pt}%
\definecolor{currentstroke}{rgb}{0.515992,0.831158,0.294279}%
\pgfsetstrokecolor{currentstroke}%
\pgfsetdash{}{0pt}%
\pgfpathmoveto{\pgfqpoint{3.445912in}{5.515993in}}%
\pgfpathlineto{\pgfqpoint{3.308861in}{5.497668in}}%
\pgfpathlineto{\pgfqpoint{3.359631in}{5.571327in}}%
\pgfpathclose%
\pgfusepath{fill}%
\end{pgfscope}%
\begin{pgfscope}%
\pgfpathrectangle{\pgfqpoint{0.539299in}{0.078740in}}{\pgfqpoint{7.842520in}{7.842520in}}%
\pgfusepath{clip}%
\pgfsetbuttcap%
\pgfsetroundjoin%
\definecolor{currentfill}{rgb}{0.265145,0.232956,0.516599}%
\pgfsetfillcolor{currentfill}%
\pgfsetlinewidth{0.000000pt}%
\definecolor{currentstroke}{rgb}{0.525776,0.833491,0.288127}%
\pgfsetstrokecolor{currentstroke}%
\pgfsetdash{}{0pt}%
\pgfpathmoveto{\pgfqpoint{5.826491in}{2.723216in}}%
\pgfpathlineto{\pgfqpoint{5.749939in}{2.703326in}}%
\pgfpathlineto{\pgfqpoint{5.967002in}{2.588084in}}%
\pgfpathclose%
\pgfusepath{fill}%
\end{pgfscope}%
\begin{pgfscope}%
\pgfpathrectangle{\pgfqpoint{0.539299in}{0.078740in}}{\pgfqpoint{7.842520in}{7.842520in}}%
\pgfusepath{clip}%
\pgfsetbuttcap%
\pgfsetroundjoin%
\definecolor{currentfill}{rgb}{0.175841,0.441290,0.557685}%
\pgfsetfillcolor{currentfill}%
\pgfsetlinewidth{0.000000pt}%
\definecolor{currentstroke}{rgb}{0.535621,0.835785,0.281908}%
\pgfsetstrokecolor{currentstroke}%
\pgfsetdash{}{0pt}%
\pgfpathmoveto{\pgfqpoint{1.803826in}{3.497087in}}%
\pgfpathlineto{\pgfqpoint{1.718526in}{3.335879in}}%
\pgfpathlineto{\pgfqpoint{1.839990in}{3.669004in}}%
\pgfpathclose%
\pgfusepath{fill}%
\end{pgfscope}%
\begin{pgfscope}%
\pgfpathrectangle{\pgfqpoint{0.539299in}{0.078740in}}{\pgfqpoint{7.842520in}{7.842520in}}%
\pgfusepath{clip}%
\pgfsetbuttcap%
\pgfsetroundjoin%
\definecolor{currentfill}{rgb}{0.220124,0.725509,0.466226}%
\pgfsetfillcolor{currentfill}%
\pgfsetlinewidth{0.000000pt}%
\definecolor{currentstroke}{rgb}{0.545524,0.838039,0.275626}%
\pgfsetstrokecolor{currentstroke}%
\pgfsetdash{}{0pt}%
\pgfpathmoveto{\pgfqpoint{4.285648in}{4.739044in}}%
\pgfpathlineto{\pgfqpoint{4.367781in}{4.630418in}}%
\pgfpathlineto{\pgfqpoint{4.145020in}{4.933425in}}%
\pgfpathclose%
\pgfusepath{fill}%
\end{pgfscope}%
\begin{pgfscope}%
\pgfpathrectangle{\pgfqpoint{0.539299in}{0.078740in}}{\pgfqpoint{7.842520in}{7.842520in}}%
\pgfusepath{clip}%
\pgfsetbuttcap%
\pgfsetroundjoin%
\definecolor{currentfill}{rgb}{0.278012,0.180367,0.486697}%
\pgfsetfillcolor{currentfill}%
\pgfsetlinewidth{0.000000pt}%
\definecolor{currentstroke}{rgb}{0.555484,0.840254,0.269281}%
\pgfsetstrokecolor{currentstroke}%
\pgfsetdash{}{0pt}%
\pgfpathmoveto{\pgfqpoint{5.967002in}{2.588084in}}%
\pgfpathlineto{\pgfqpoint{6.031600in}{2.411053in}}%
\pgfpathlineto{\pgfqpoint{6.107758in}{2.459413in}}%
\pgfpathclose%
\pgfusepath{fill}%
\end{pgfscope}%
\begin{pgfscope}%
\pgfpathrectangle{\pgfqpoint{0.539299in}{0.078740in}}{\pgfqpoint{7.842520in}{7.842520in}}%
\pgfusepath{clip}%
\pgfsetbuttcap%
\pgfsetroundjoin%
\definecolor{currentfill}{rgb}{0.157729,0.485932,0.558013}%
\pgfsetfillcolor{currentfill}%
\pgfsetlinewidth{0.000000pt}%
\definecolor{currentstroke}{rgb}{0.565498,0.842430,0.262877}%
\pgfsetstrokecolor{currentstroke}%
\pgfsetdash{}{0pt}%
\pgfpathmoveto{\pgfqpoint{5.126490in}{3.527887in}}%
\pgfpathlineto{\pgfqpoint{4.986662in}{3.717303in}}%
\pgfpathlineto{\pgfqpoint{4.907447in}{3.788480in}}%
\pgfpathclose%
\pgfusepath{fill}%
\end{pgfscope}%
\begin{pgfscope}%
\pgfpathrectangle{\pgfqpoint{0.539299in}{0.078740in}}{\pgfqpoint{7.842520in}{7.842520in}}%
\pgfusepath{clip}%
\pgfsetbuttcap%
\pgfsetroundjoin%
\definecolor{currentfill}{rgb}{0.468053,0.818921,0.323998}%
\pgfsetfillcolor{currentfill}%
\pgfsetlinewidth{0.000000pt}%
\definecolor{currentstroke}{rgb}{0.575563,0.844566,0.256415}%
\pgfsetstrokecolor{currentstroke}%
\pgfsetdash{}{0pt}%
\pgfpathmoveto{\pgfqpoint{3.863917in}{5.267515in}}%
\pgfpathlineto{\pgfqpoint{3.948223in}{5.166109in}}%
\pgfpathlineto{\pgfqpoint{3.723815in}{5.392306in}}%
\pgfpathclose%
\pgfusepath{fill}%
\end{pgfscope}%
\begin{pgfscope}%
\pgfpathrectangle{\pgfqpoint{0.539299in}{0.078740in}}{\pgfqpoint{7.842520in}{7.842520in}}%
\pgfusepath{clip}%
\pgfsetbuttcap%
\pgfsetroundjoin%
\definecolor{currentfill}{rgb}{0.626579,0.854645,0.223353}%
\pgfsetfillcolor{currentfill}%
\pgfsetlinewidth{0.000000pt}%
\definecolor{currentstroke}{rgb}{0.585678,0.846661,0.249897}%
\pgfsetstrokecolor{currentstroke}%
\pgfsetdash{}{0pt}%
\pgfpathmoveto{\pgfqpoint{3.359631in}{5.571327in}}%
\pgfpathlineto{\pgfqpoint{3.308861in}{5.497668in}}%
\pgfpathlineto{\pgfqpoint{3.222442in}{5.537820in}}%
\pgfpathclose%
\pgfusepath{fill}%
\end{pgfscope}%
\begin{pgfscope}%
\pgfpathrectangle{\pgfqpoint{0.539299in}{0.078740in}}{\pgfqpoint{7.842520in}{7.842520in}}%
\pgfusepath{clip}%
\pgfsetbuttcap%
\pgfsetroundjoin%
\definecolor{currentfill}{rgb}{0.319809,0.770914,0.411152}%
\pgfsetfillcolor{currentfill}%
\pgfsetlinewidth{0.000000pt}%
\definecolor{currentstroke}{rgb}{0.595839,0.848717,0.243329}%
\pgfsetstrokecolor{currentstroke}%
\pgfsetdash{}{0pt}%
\pgfpathmoveto{\pgfqpoint{2.694035in}{5.204781in}}%
\pgfpathlineto{\pgfqpoint{2.565417in}{4.919366in}}%
\pgfpathlineto{\pgfqpoint{2.478591in}{4.855876in}}%
\pgfpathclose%
\pgfusepath{fill}%
\end{pgfscope}%
\begin{pgfscope}%
\pgfpathrectangle{\pgfqpoint{0.539299in}{0.078740in}}{\pgfqpoint{7.842520in}{7.842520in}}%
\pgfusepath{clip}%
\pgfsetbuttcap%
\pgfsetroundjoin%
\definecolor{currentfill}{rgb}{0.304148,0.764704,0.419943}%
\pgfsetfillcolor{currentfill}%
\pgfsetlinewidth{0.000000pt}%
\definecolor{currentstroke}{rgb}{0.606045,0.850733,0.236712}%
\pgfsetstrokecolor{currentstroke}%
\pgfsetdash{}{0pt}%
\pgfpathmoveto{\pgfqpoint{4.004391in}{5.111952in}}%
\pgfpathlineto{\pgfqpoint{4.145020in}{4.933425in}}%
\pgfpathlineto{\pgfqpoint{4.227924in}{4.824025in}}%
\pgfpathclose%
\pgfusepath{fill}%
\end{pgfscope}%
\begin{pgfscope}%
\pgfpathrectangle{\pgfqpoint{0.539299in}{0.078740in}}{\pgfqpoint{7.842520in}{7.842520in}}%
\pgfusepath{clip}%
\pgfsetbuttcap%
\pgfsetroundjoin%
\definecolor{currentfill}{rgb}{0.281412,0.155834,0.469201}%
\pgfsetfillcolor{currentfill}%
\pgfsetlinewidth{0.000000pt}%
\definecolor{currentstroke}{rgb}{0.616293,0.852709,0.230052}%
\pgfsetstrokecolor{currentstroke}%
\pgfsetdash{}{0pt}%
\pgfpathmoveto{\pgfqpoint{6.107758in}{2.459413in}}%
\pgfpathlineto{\pgfqpoint{6.031600in}{2.411053in}}%
\pgfpathlineto{\pgfqpoint{6.248772in}{2.336093in}}%
\pgfpathclose%
\pgfusepath{fill}%
\end{pgfscope}%
\begin{pgfscope}%
\pgfpathrectangle{\pgfqpoint{0.539299in}{0.078740in}}{\pgfqpoint{7.842520in}{7.842520in}}%
\pgfusepath{clip}%
\pgfsetbuttcap%
\pgfsetroundjoin%
\definecolor{currentfill}{rgb}{0.146180,0.515413,0.556823}%
\pgfsetfillcolor{currentfill}%
\pgfsetlinewidth{0.000000pt}%
\definecolor{currentstroke}{rgb}{0.626579,0.854645,0.223353}%
\pgfsetstrokecolor{currentstroke}%
\pgfsetdash{}{0pt}%
\pgfpathmoveto{\pgfqpoint{4.907447in}{3.788480in}}%
\pgfpathlineto{\pgfqpoint{4.986662in}{3.717303in}}%
\pgfpathlineto{\pgfqpoint{4.846758in}{3.914972in}}%
\pgfpathclose%
\pgfusepath{fill}%
\end{pgfscope}%
\begin{pgfscope}%
\pgfpathrectangle{\pgfqpoint{0.539299in}{0.078740in}}{\pgfqpoint{7.842520in}{7.842520in}}%
\pgfusepath{clip}%
\pgfsetbuttcap%
\pgfsetroundjoin%
\definecolor{currentfill}{rgb}{0.421908,0.805774,0.351910}%
\pgfsetfillcolor{currentfill}%
\pgfsetlinewidth{0.000000pt}%
\definecolor{currentstroke}{rgb}{0.636902,0.856542,0.216620}%
\pgfsetstrokecolor{currentstroke}%
\pgfsetdash{}{0pt}%
\pgfpathmoveto{\pgfqpoint{4.004391in}{5.111952in}}%
\pgfpathlineto{\pgfqpoint{3.948223in}{5.166109in}}%
\pgfpathlineto{\pgfqpoint{3.863917in}{5.267515in}}%
\pgfpathclose%
\pgfusepath{fill}%
\end{pgfscope}%
\begin{pgfscope}%
\pgfpathrectangle{\pgfqpoint{0.539299in}{0.078740in}}{\pgfqpoint{7.842520in}{7.842520in}}%
\pgfusepath{clip}%
\pgfsetbuttcap%
\pgfsetroundjoin%
\definecolor{currentfill}{rgb}{0.276022,0.044167,0.370164}%
\pgfsetfillcolor{currentfill}%
\pgfsetlinewidth{0.000000pt}%
\definecolor{currentstroke}{rgb}{0.647257,0.858400,0.209861}%
\pgfsetstrokecolor{currentstroke}%
\pgfsetdash{}{0pt}%
\pgfpathmoveto{\pgfqpoint{6.748358in}{2.068409in}}%
\pgfpathlineto{\pgfqpoint{6.531589in}{2.101217in}}%
\pgfpathlineto{\pgfqpoint{6.673382in}{1.987622in}}%
\pgfpathclose%
\pgfusepath{fill}%
\end{pgfscope}%
\begin{pgfscope}%
\pgfpathrectangle{\pgfqpoint{0.539299in}{0.078740in}}{\pgfqpoint{7.842520in}{7.842520in}}%
\pgfusepath{clip}%
\pgfsetbuttcap%
\pgfsetroundjoin%
\definecolor{currentfill}{rgb}{0.496615,0.826376,0.306377}%
\pgfsetfillcolor{currentfill}%
\pgfsetlinewidth{0.000000pt}%
\definecolor{currentstroke}{rgb}{0.657642,0.860219,0.203082}%
\pgfsetstrokecolor{currentstroke}%
\pgfsetdash{}{0pt}%
\pgfpathmoveto{\pgfqpoint{2.913349in}{5.441460in}}%
\pgfpathlineto{\pgfqpoint{2.867996in}{5.262317in}}%
\pgfpathlineto{\pgfqpoint{2.781099in}{5.242417in}}%
\pgfpathclose%
\pgfusepath{fill}%
\end{pgfscope}%
\begin{pgfscope}%
\pgfpathrectangle{\pgfqpoint{0.539299in}{0.078740in}}{\pgfqpoint{7.842520in}{7.842520in}}%
\pgfusepath{clip}%
\pgfsetbuttcap%
\pgfsetroundjoin%
\definecolor{currentfill}{rgb}{0.231674,0.318106,0.544834}%
\pgfsetfillcolor{currentfill}%
\pgfsetlinewidth{0.000000pt}%
\definecolor{currentstroke}{rgb}{0.668054,0.861999,0.196293}%
\pgfsetstrokecolor{currentstroke}%
\pgfsetdash{}{0pt}%
\pgfpathmoveto{\pgfqpoint{5.609380in}{2.861680in}}%
\pgfpathlineto{\pgfqpoint{5.546103in}{3.017088in}}%
\pgfpathlineto{\pgfqpoint{5.468948in}{3.028932in}}%
\pgfpathclose%
\pgfusepath{fill}%
\end{pgfscope}%
\begin{pgfscope}%
\pgfpathrectangle{\pgfqpoint{0.539299in}{0.078740in}}{\pgfqpoint{7.842520in}{7.842520in}}%
\pgfusepath{clip}%
\pgfsetbuttcap%
\pgfsetroundjoin%
\definecolor{currentfill}{rgb}{0.218130,0.347432,0.550038}%
\pgfsetfillcolor{currentfill}%
\pgfsetlinewidth{0.000000pt}%
\definecolor{currentstroke}{rgb}{0.678489,0.863742,0.189503}%
\pgfsetstrokecolor{currentstroke}%
\pgfsetdash{}{0pt}%
\pgfpathmoveto{\pgfqpoint{5.468948in}{3.028932in}}%
\pgfpathlineto{\pgfqpoint{5.546103in}{3.017088in}}%
\pgfpathlineto{\pgfqpoint{5.328594in}{3.205423in}}%
\pgfpathclose%
\pgfusepath{fill}%
\end{pgfscope}%
\begin{pgfscope}%
\pgfpathrectangle{\pgfqpoint{0.539299in}{0.078740in}}{\pgfqpoint{7.842520in}{7.842520in}}%
\pgfusepath{clip}%
\pgfsetbuttcap%
\pgfsetroundjoin%
\definecolor{currentfill}{rgb}{0.253935,0.265254,0.529983}%
\pgfsetfillcolor{currentfill}%
\pgfsetlinewidth{0.000000pt}%
\definecolor{currentstroke}{rgb}{0.688944,0.865448,0.182725}%
\pgfsetstrokecolor{currentstroke}%
\pgfsetdash{}{0pt}%
\pgfpathmoveto{\pgfqpoint{5.609380in}{2.861680in}}%
\pgfpathlineto{\pgfqpoint{5.749939in}{2.703326in}}%
\pgfpathlineto{\pgfqpoint{5.826491in}{2.723216in}}%
\pgfpathclose%
\pgfusepath{fill}%
\end{pgfscope}%
\begin{pgfscope}%
\pgfpathrectangle{\pgfqpoint{0.539299in}{0.078740in}}{\pgfqpoint{7.842520in}{7.842520in}}%
\pgfusepath{clip}%
\pgfsetbuttcap%
\pgfsetroundjoin%
\definecolor{currentfill}{rgb}{0.146180,0.515413,0.556823}%
\pgfsetfillcolor{currentfill}%
\pgfsetlinewidth{0.000000pt}%
\definecolor{currentstroke}{rgb}{0.699415,0.867117,0.175971}%
\pgfsetstrokecolor{currentstroke}%
\pgfsetdash{}{0pt}%
\pgfpathmoveto{\pgfqpoint{1.803826in}{3.497087in}}%
\pgfpathlineto{\pgfqpoint{1.924234in}{3.870625in}}%
\pgfpathlineto{\pgfqpoint{2.009371in}{4.040892in}}%
\pgfpathclose%
\pgfusepath{fill}%
\end{pgfscope}%
\begin{pgfscope}%
\pgfpathrectangle{\pgfqpoint{0.539299in}{0.078740in}}{\pgfqpoint{7.842520in}{7.842520in}}%
\pgfusepath{clip}%
\pgfsetbuttcap%
\pgfsetroundjoin%
\definecolor{currentfill}{rgb}{0.535621,0.835785,0.281908}%
\pgfsetfillcolor{currentfill}%
\pgfsetlinewidth{0.000000pt}%
\definecolor{currentstroke}{rgb}{0.709898,0.868751,0.169257}%
\pgfsetstrokecolor{currentstroke}%
\pgfsetdash{}{0pt}%
\pgfpathmoveto{\pgfqpoint{3.000456in}{5.448722in}}%
\pgfpathlineto{\pgfqpoint{2.867996in}{5.262317in}}%
\pgfpathlineto{\pgfqpoint{2.913349in}{5.441460in}}%
\pgfpathclose%
\pgfusepath{fill}%
\end{pgfscope}%
\begin{pgfscope}%
\pgfpathrectangle{\pgfqpoint{0.539299in}{0.078740in}}{\pgfqpoint{7.842520in}{7.842520in}}%
\pgfusepath{clip}%
\pgfsetbuttcap%
\pgfsetroundjoin%
\definecolor{currentfill}{rgb}{0.192357,0.403199,0.555836}%
\pgfsetfillcolor{currentfill}%
\pgfsetlinewidth{0.000000pt}%
\definecolor{currentstroke}{rgb}{0.720391,0.870350,0.162603}%
\pgfsetstrokecolor{currentstroke}%
\pgfsetdash{}{0pt}%
\pgfpathmoveto{\pgfqpoint{1.718526in}{3.335879in}}%
\pgfpathlineto{\pgfqpoint{1.633975in}{3.148721in}}%
\pgfpathlineto{\pgfqpoint{1.756965in}{3.430597in}}%
\pgfpathclose%
\pgfusepath{fill}%
\end{pgfscope}%
\begin{pgfscope}%
\pgfpathrectangle{\pgfqpoint{0.539299in}{0.078740in}}{\pgfqpoint{7.842520in}{7.842520in}}%
\pgfusepath{clip}%
\pgfsetbuttcap%
\pgfsetroundjoin%
\definecolor{currentfill}{rgb}{0.190631,0.407061,0.556089}%
\pgfsetfillcolor{currentfill}%
\pgfsetlinewidth{0.000000pt}%
\definecolor{currentstroke}{rgb}{0.730889,0.871916,0.156029}%
\pgfsetstrokecolor{currentstroke}%
\pgfsetdash{}{0pt}%
\pgfpathmoveto{\pgfqpoint{5.266300in}{3.347822in}}%
\pgfpathlineto{\pgfqpoint{5.188263in}{3.391183in}}%
\pgfpathlineto{\pgfqpoint{5.328594in}{3.205423in}}%
\pgfpathclose%
\pgfusepath{fill}%
\end{pgfscope}%
\begin{pgfscope}%
\pgfpathrectangle{\pgfqpoint{0.539299in}{0.078740in}}{\pgfqpoint{7.842520in}{7.842520in}}%
\pgfusepath{clip}%
\pgfsetbuttcap%
\pgfsetroundjoin%
\definecolor{currentfill}{rgb}{0.267968,0.223549,0.512008}%
\pgfsetfillcolor{currentfill}%
\pgfsetlinewidth{0.000000pt}%
\definecolor{currentstroke}{rgb}{0.741388,0.873449,0.149561}%
\pgfsetstrokecolor{currentstroke}%
\pgfsetdash{}{0pt}%
\pgfpathmoveto{\pgfqpoint{5.749939in}{2.703326in}}%
\pgfpathlineto{\pgfqpoint{5.890666in}{2.553330in}}%
\pgfpathlineto{\pgfqpoint{5.967002in}{2.588084in}}%
\pgfpathclose%
\pgfusepath{fill}%
\end{pgfscope}%
\begin{pgfscope}%
\pgfpathrectangle{\pgfqpoint{0.539299in}{0.078740in}}{\pgfqpoint{7.842520in}{7.842520in}}%
\pgfusepath{clip}%
\pgfsetbuttcap%
\pgfsetroundjoin%
\definecolor{currentfill}{rgb}{0.626579,0.854645,0.223353}%
\pgfsetfillcolor{currentfill}%
\pgfsetlinewidth{0.000000pt}%
\definecolor{currentstroke}{rgb}{0.751884,0.874951,0.143228}%
\pgfsetstrokecolor{currentstroke}%
\pgfsetdash{}{0pt}%
\pgfpathmoveto{\pgfqpoint{3.445912in}{5.515993in}}%
\pgfpathlineto{\pgfqpoint{3.359631in}{5.571327in}}%
\pgfpathlineto{\pgfqpoint{3.584365in}{5.478002in}}%
\pgfpathclose%
\pgfusepath{fill}%
\end{pgfscope}%
\begin{pgfscope}%
\pgfpathrectangle{\pgfqpoint{0.539299in}{0.078740in}}{\pgfqpoint{7.842520in}{7.842520in}}%
\pgfusepath{clip}%
\pgfsetbuttcap%
\pgfsetroundjoin%
\definecolor{currentfill}{rgb}{0.616293,0.852709,0.230052}%
\pgfsetfillcolor{currentfill}%
\pgfsetlinewidth{0.000000pt}%
\definecolor{currentstroke}{rgb}{0.762373,0.876424,0.137064}%
\pgfsetstrokecolor{currentstroke}%
\pgfsetdash{}{0pt}%
\pgfpathmoveto{\pgfqpoint{3.135579in}{5.562556in}}%
\pgfpathlineto{\pgfqpoint{3.222442in}{5.537820in}}%
\pgfpathlineto{\pgfqpoint{3.087266in}{5.438839in}}%
\pgfpathclose%
\pgfusepath{fill}%
\end{pgfscope}%
\begin{pgfscope}%
\pgfpathrectangle{\pgfqpoint{0.539299in}{0.078740in}}{\pgfqpoint{7.842520in}{7.842520in}}%
\pgfusepath{clip}%
\pgfsetbuttcap%
\pgfsetroundjoin%
\definecolor{currentfill}{rgb}{0.275191,0.194905,0.496005}%
\pgfsetfillcolor{currentfill}%
\pgfsetlinewidth{0.000000pt}%
\definecolor{currentstroke}{rgb}{0.772852,0.877868,0.131109}%
\pgfsetstrokecolor{currentstroke}%
\pgfsetdash{}{0pt}%
\pgfpathmoveto{\pgfqpoint{5.890666in}{2.553330in}}%
\pgfpathlineto{\pgfqpoint{6.031600in}{2.411053in}}%
\pgfpathlineto{\pgfqpoint{5.967002in}{2.588084in}}%
\pgfpathclose%
\pgfusepath{fill}%
\end{pgfscope}%
\begin{pgfscope}%
\pgfpathrectangle{\pgfqpoint{0.539299in}{0.078740in}}{\pgfqpoint{7.842520in}{7.842520in}}%
\pgfusepath{clip}%
\pgfsetbuttcap%
\pgfsetroundjoin%
\definecolor{currentfill}{rgb}{0.283091,0.110553,0.431554}%
\pgfsetfillcolor{currentfill}%
\pgfsetlinewidth{0.000000pt}%
\definecolor{currentstroke}{rgb}{0.783315,0.879285,0.125405}%
\pgfsetstrokecolor{currentstroke}%
\pgfsetdash{}{0pt}%
\pgfpathmoveto{\pgfqpoint{6.248772in}{2.336093in}}%
\pgfpathlineto{\pgfqpoint{6.314217in}{2.147221in}}%
\pgfpathlineto{\pgfqpoint{6.390049in}{2.217039in}}%
\pgfpathclose%
\pgfusepath{fill}%
\end{pgfscope}%
\begin{pgfscope}%
\pgfpathrectangle{\pgfqpoint{0.539299in}{0.078740in}}{\pgfqpoint{7.842520in}{7.842520in}}%
\pgfusepath{clip}%
\pgfsetbuttcap%
\pgfsetroundjoin%
\definecolor{currentfill}{rgb}{0.606045,0.850733,0.236712}%
\pgfsetfillcolor{currentfill}%
\pgfsetlinewidth{0.000000pt}%
\definecolor{currentstroke}{rgb}{0.793760,0.880678,0.120005}%
\pgfsetstrokecolor{currentstroke}%
\pgfsetdash{}{0pt}%
\pgfpathmoveto{\pgfqpoint{3.087266in}{5.438839in}}%
\pgfpathlineto{\pgfqpoint{3.000456in}{5.448722in}}%
\pgfpathlineto{\pgfqpoint{3.135579in}{5.562556in}}%
\pgfpathclose%
\pgfusepath{fill}%
\end{pgfscope}%
\begin{pgfscope}%
\pgfpathrectangle{\pgfqpoint{0.539299in}{0.078740in}}{\pgfqpoint{7.842520in}{7.842520in}}%
\pgfusepath{clip}%
\pgfsetbuttcap%
\pgfsetroundjoin%
\definecolor{currentfill}{rgb}{0.123463,0.581687,0.547445}%
\pgfsetfillcolor{currentfill}%
\pgfsetlinewidth{0.000000pt}%
\definecolor{currentstroke}{rgb}{0.804182,0.882046,0.114965}%
\pgfsetstrokecolor{currentstroke}%
\pgfsetdash{}{0pt}%
\pgfpathmoveto{\pgfqpoint{4.846758in}{3.914972in}}%
\pgfpathlineto{\pgfqpoint{4.706729in}{4.119068in}}%
\pgfpathlineto{\pgfqpoint{4.626092in}{4.211058in}}%
\pgfpathclose%
\pgfusepath{fill}%
\end{pgfscope}%
\begin{pgfscope}%
\pgfpathrectangle{\pgfqpoint{0.539299in}{0.078740in}}{\pgfqpoint{7.842520in}{7.842520in}}%
\pgfusepath{clip}%
\pgfsetbuttcap%
\pgfsetroundjoin%
\definecolor{currentfill}{rgb}{0.171176,0.452530,0.557965}%
\pgfsetfillcolor{currentfill}%
\pgfsetlinewidth{0.000000pt}%
\definecolor{currentstroke}{rgb}{0.814576,0.883393,0.110347}%
\pgfsetstrokecolor{currentstroke}%
\pgfsetdash{}{0pt}%
\pgfpathmoveto{\pgfqpoint{5.126490in}{3.527887in}}%
\pgfpathlineto{\pgfqpoint{5.047899in}{3.585834in}}%
\pgfpathlineto{\pgfqpoint{5.188263in}{3.391183in}}%
\pgfpathclose%
\pgfusepath{fill}%
\end{pgfscope}%
\begin{pgfscope}%
\pgfpathrectangle{\pgfqpoint{0.539299in}{0.078740in}}{\pgfqpoint{7.842520in}{7.842520in}}%
\pgfusepath{clip}%
\pgfsetbuttcap%
\pgfsetroundjoin%
\definecolor{currentfill}{rgb}{0.159194,0.482237,0.558073}%
\pgfsetfillcolor{currentfill}%
\pgfsetlinewidth{0.000000pt}%
\definecolor{currentstroke}{rgb}{0.824940,0.884720,0.106217}%
\pgfsetstrokecolor{currentstroke}%
\pgfsetdash{}{0pt}%
\pgfpathmoveto{\pgfqpoint{4.907447in}{3.788480in}}%
\pgfpathlineto{\pgfqpoint{5.047899in}{3.585834in}}%
\pgfpathlineto{\pgfqpoint{5.126490in}{3.527887in}}%
\pgfpathclose%
\pgfusepath{fill}%
\end{pgfscope}%
\begin{pgfscope}%
\pgfpathrectangle{\pgfqpoint{0.539299in}{0.078740in}}{\pgfqpoint{7.842520in}{7.842520in}}%
\pgfusepath{clip}%
\pgfsetbuttcap%
\pgfsetroundjoin%
\definecolor{currentfill}{rgb}{0.204903,0.375746,0.553533}%
\pgfsetfillcolor{currentfill}%
\pgfsetlinewidth{0.000000pt}%
\definecolor{currentstroke}{rgb}{0.835270,0.886029,0.102646}%
\pgfsetstrokecolor{currentstroke}%
\pgfsetdash{}{0pt}%
\pgfpathmoveto{\pgfqpoint{1.756965in}{3.430597in}}%
\pgfpathlineto{\pgfqpoint{1.633975in}{3.148721in}}%
\pgfpathlineto{\pgfqpoint{1.550510in}{2.929884in}}%
\pgfpathclose%
\pgfusepath{fill}%
\end{pgfscope}%
\begin{pgfscope}%
\pgfpathrectangle{\pgfqpoint{0.539299in}{0.078740in}}{\pgfqpoint{7.842520in}{7.842520in}}%
\pgfusepath{clip}%
\pgfsetbuttcap%
\pgfsetroundjoin%
\definecolor{currentfill}{rgb}{0.154815,0.493313,0.557840}%
\pgfsetfillcolor{currentfill}%
\pgfsetlinewidth{0.000000pt}%
\definecolor{currentstroke}{rgb}{0.845561,0.887322,0.099702}%
\pgfsetstrokecolor{currentstroke}%
\pgfsetdash{}{0pt}%
\pgfpathmoveto{\pgfqpoint{1.839990in}{3.669004in}}%
\pgfpathlineto{\pgfqpoint{1.924234in}{3.870625in}}%
\pgfpathlineto{\pgfqpoint{1.803826in}{3.497087in}}%
\pgfpathclose%
\pgfusepath{fill}%
\end{pgfscope}%
\begin{pgfscope}%
\pgfpathrectangle{\pgfqpoint{0.539299in}{0.078740in}}{\pgfqpoint{7.842520in}{7.842520in}}%
\pgfusepath{clip}%
\pgfsetbuttcap%
\pgfsetroundjoin%
\definecolor{currentfill}{rgb}{0.281887,0.150881,0.465405}%
\pgfsetfillcolor{currentfill}%
\pgfsetlinewidth{0.000000pt}%
\definecolor{currentstroke}{rgb}{0.855810,0.888601,0.097452}%
\pgfsetstrokecolor{currentstroke}%
\pgfsetdash{}{0pt}%
\pgfpathmoveto{\pgfqpoint{6.248772in}{2.336093in}}%
\pgfpathlineto{\pgfqpoint{6.031600in}{2.411053in}}%
\pgfpathlineto{\pgfqpoint{6.172774in}{2.275861in}}%
\pgfpathclose%
\pgfusepath{fill}%
\end{pgfscope}%
\begin{pgfscope}%
\pgfpathrectangle{\pgfqpoint{0.539299in}{0.078740in}}{\pgfqpoint{7.842520in}{7.842520in}}%
\pgfusepath{clip}%
\pgfsetbuttcap%
\pgfsetroundjoin%
\definecolor{currentfill}{rgb}{0.266580,0.228262,0.514349}%
\pgfsetfillcolor{currentfill}%
\pgfsetlinewidth{0.000000pt}%
\definecolor{currentstroke}{rgb}{0.866013,0.889868,0.095953}%
\pgfsetstrokecolor{currentstroke}%
\pgfsetdash{}{0pt}%
\pgfpathmoveto{\pgfqpoint{1.468606in}{2.671104in}}%
\pgfpathlineto{\pgfqpoint{1.388975in}{2.359780in}}%
\pgfpathlineto{\pgfqpoint{1.596362in}{2.813436in}}%
\pgfpathclose%
\pgfusepath{fill}%
\end{pgfscope}%
\begin{pgfscope}%
\pgfpathrectangle{\pgfqpoint{0.539299in}{0.078740in}}{\pgfqpoint{7.842520in}{7.842520in}}%
\pgfusepath{clip}%
\pgfsetbuttcap%
\pgfsetroundjoin%
\definecolor{currentfill}{rgb}{0.280894,0.078907,0.402329}%
\pgfsetfillcolor{currentfill}%
\pgfsetlinewidth{0.000000pt}%
\definecolor{currentstroke}{rgb}{0.876168,0.891125,0.095250}%
\pgfsetstrokecolor{currentstroke}%
\pgfsetdash{}{0pt}%
\pgfpathmoveto{\pgfqpoint{6.455962in}{2.024776in}}%
\pgfpathlineto{\pgfqpoint{6.531589in}{2.101217in}}%
\pgfpathlineto{\pgfqpoint{6.390049in}{2.217039in}}%
\pgfpathclose%
\pgfusepath{fill}%
\end{pgfscope}%
\begin{pgfscope}%
\pgfpathrectangle{\pgfqpoint{0.539299in}{0.078740in}}{\pgfqpoint{7.842520in}{7.842520in}}%
\pgfusepath{clip}%
\pgfsetbuttcap%
\pgfsetroundjoin%
\definecolor{currentfill}{rgb}{0.121380,0.629492,0.531973}%
\pgfsetfillcolor{currentfill}%
\pgfsetlinewidth{0.000000pt}%
\definecolor{currentstroke}{rgb}{0.886271,0.892374,0.095374}%
\pgfsetstrokecolor{currentstroke}%
\pgfsetdash{}{0pt}%
\pgfpathmoveto{\pgfqpoint{4.485130in}{4.425835in}}%
\pgfpathlineto{\pgfqpoint{4.706729in}{4.119068in}}%
\pgfpathlineto{\pgfqpoint{4.566539in}{4.326950in}}%
\pgfpathclose%
\pgfusepath{fill}%
\end{pgfscope}%
\begin{pgfscope}%
\pgfpathrectangle{\pgfqpoint{0.539299in}{0.078740in}}{\pgfqpoint{7.842520in}{7.842520in}}%
\pgfusepath{clip}%
\pgfsetbuttcap%
\pgfsetroundjoin%
\definecolor{currentfill}{rgb}{0.124780,0.640461,0.527068}%
\pgfsetfillcolor{currentfill}%
\pgfsetlinewidth{0.000000pt}%
\definecolor{currentstroke}{rgb}{0.896320,0.893616,0.096335}%
\pgfsetstrokecolor{currentstroke}%
\pgfsetdash{}{0pt}%
\pgfpathmoveto{\pgfqpoint{2.095141in}{4.184186in}}%
\pgfpathlineto{\pgfqpoint{2.219156in}{4.533644in}}%
\pgfpathlineto{\pgfqpoint{2.181328in}{4.304148in}}%
\pgfpathclose%
\pgfusepath{fill}%
\end{pgfscope}%
\begin{pgfscope}%
\pgfpathrectangle{\pgfqpoint{0.539299in}{0.078740in}}{\pgfqpoint{7.842520in}{7.842520in}}%
\pgfusepath{clip}%
\pgfsetbuttcap%
\pgfsetroundjoin%
\definecolor{currentfill}{rgb}{0.283187,0.125848,0.444960}%
\pgfsetfillcolor{currentfill}%
\pgfsetlinewidth{0.000000pt}%
\definecolor{currentstroke}{rgb}{0.906311,0.894855,0.098125}%
\pgfsetstrokecolor{currentstroke}%
\pgfsetdash{}{0pt}%
\pgfpathmoveto{\pgfqpoint{6.172774in}{2.275861in}}%
\pgfpathlineto{\pgfqpoint{6.314217in}{2.147221in}}%
\pgfpathlineto{\pgfqpoint{6.248772in}{2.336093in}}%
\pgfpathclose%
\pgfusepath{fill}%
\end{pgfscope}%
\begin{pgfscope}%
\pgfpathrectangle{\pgfqpoint{0.539299in}{0.078740in}}{\pgfqpoint{7.842520in}{7.842520in}}%
\pgfusepath{clip}%
\pgfsetbuttcap%
\pgfsetroundjoin%
\definecolor{currentfill}{rgb}{0.233603,0.313828,0.543914}%
\pgfsetfillcolor{currentfill}%
\pgfsetlinewidth{0.000000pt}%
\definecolor{currentstroke}{rgb}{0.916242,0.896091,0.100717}%
\pgfsetstrokecolor{currentstroke}%
\pgfsetdash{}{0pt}%
\pgfpathmoveto{\pgfqpoint{1.675575in}{3.148455in}}%
\pgfpathlineto{\pgfqpoint{1.550510in}{2.929884in}}%
\pgfpathlineto{\pgfqpoint{1.468606in}{2.671104in}}%
\pgfpathclose%
\pgfusepath{fill}%
\end{pgfscope}%
\begin{pgfscope}%
\pgfpathrectangle{\pgfqpoint{0.539299in}{0.078740in}}{\pgfqpoint{7.842520in}{7.842520in}}%
\pgfusepath{clip}%
\pgfsetbuttcap%
\pgfsetroundjoin%
\definecolor{currentfill}{rgb}{0.170948,0.694384,0.493803}%
\pgfsetfillcolor{currentfill}%
\pgfsetlinewidth{0.000000pt}%
\definecolor{currentstroke}{rgb}{0.926106,0.897330,0.104071}%
\pgfsetstrokecolor{currentstroke}%
\pgfsetdash{}{0pt}%
\pgfpathmoveto{\pgfqpoint{2.305286in}{4.666266in}}%
\pgfpathlineto{\pgfqpoint{2.391827in}{4.772531in}}%
\pgfpathlineto{\pgfqpoint{2.181328in}{4.304148in}}%
\pgfpathclose%
\pgfusepath{fill}%
\end{pgfscope}%
\begin{pgfscope}%
\pgfpathrectangle{\pgfqpoint{0.539299in}{0.078740in}}{\pgfqpoint{7.842520in}{7.842520in}}%
\pgfusepath{clip}%
\pgfsetbuttcap%
\pgfsetroundjoin%
\definecolor{currentfill}{rgb}{0.277941,0.056324,0.381191}%
\pgfsetfillcolor{currentfill}%
\pgfsetlinewidth{0.000000pt}%
\definecolor{currentstroke}{rgb}{0.935904,0.898570,0.108131}%
\pgfsetstrokecolor{currentstroke}%
\pgfsetdash{}{0pt}%
\pgfpathmoveto{\pgfqpoint{6.673382in}{1.987622in}}%
\pgfpathlineto{\pgfqpoint{6.531589in}{2.101217in}}%
\pgfpathlineto{\pgfqpoint{6.455962in}{2.024776in}}%
\pgfpathclose%
\pgfusepath{fill}%
\end{pgfscope}%
\begin{pgfscope}%
\pgfpathrectangle{\pgfqpoint{0.539299in}{0.078740in}}{\pgfqpoint{7.842520in}{7.842520in}}%
\pgfusepath{clip}%
\pgfsetbuttcap%
\pgfsetroundjoin%
\definecolor{currentfill}{rgb}{0.135066,0.544853,0.554029}%
\pgfsetfillcolor{currentfill}%
\pgfsetlinewidth{0.000000pt}%
\definecolor{currentstroke}{rgb}{0.945636,0.899815,0.112838}%
\pgfsetstrokecolor{currentstroke}%
\pgfsetdash{}{0pt}%
\pgfpathmoveto{\pgfqpoint{4.846758in}{3.914972in}}%
\pgfpathlineto{\pgfqpoint{4.766857in}{3.997620in}}%
\pgfpathlineto{\pgfqpoint{4.907447in}{3.788480in}}%
\pgfpathclose%
\pgfusepath{fill}%
\end{pgfscope}%
\begin{pgfscope}%
\pgfpathrectangle{\pgfqpoint{0.539299in}{0.078740in}}{\pgfqpoint{7.842520in}{7.842520in}}%
\pgfusepath{clip}%
\pgfsetbuttcap%
\pgfsetroundjoin%
\definecolor{currentfill}{rgb}{0.137339,0.662252,0.515571}%
\pgfsetfillcolor{currentfill}%
\pgfsetlinewidth{0.000000pt}%
\definecolor{currentstroke}{rgb}{0.955300,0.901065,0.118128}%
\pgfsetstrokecolor{currentstroke}%
\pgfsetdash{}{0pt}%
\pgfpathmoveto{\pgfqpoint{4.426173in}{4.535091in}}%
\pgfpathlineto{\pgfqpoint{4.485130in}{4.425835in}}%
\pgfpathlineto{\pgfqpoint{4.566539in}{4.326950in}}%
\pgfpathclose%
\pgfusepath{fill}%
\end{pgfscope}%
\begin{pgfscope}%
\pgfpathrectangle{\pgfqpoint{0.539299in}{0.078740in}}{\pgfqpoint{7.842520in}{7.842520in}}%
\pgfusepath{clip}%
\pgfsetbuttcap%
\pgfsetroundjoin%
\definecolor{currentfill}{rgb}{0.281924,0.089666,0.412415}%
\pgfsetfillcolor{currentfill}%
\pgfsetlinewidth{0.000000pt}%
\definecolor{currentstroke}{rgb}{0.964894,0.902323,0.123941}%
\pgfsetstrokecolor{currentstroke}%
\pgfsetdash{}{0pt}%
\pgfpathmoveto{\pgfqpoint{6.390049in}{2.217039in}}%
\pgfpathlineto{\pgfqpoint{6.314217in}{2.147221in}}%
\pgfpathlineto{\pgfqpoint{6.455962in}{2.024776in}}%
\pgfpathclose%
\pgfusepath{fill}%
\end{pgfscope}%
\begin{pgfscope}%
\pgfpathrectangle{\pgfqpoint{0.539299in}{0.078740in}}{\pgfqpoint{7.842520in}{7.842520in}}%
\pgfusepath{clip}%
\pgfsetbuttcap%
\pgfsetroundjoin%
\definecolor{currentfill}{rgb}{0.271828,0.209303,0.504434}%
\pgfsetfillcolor{currentfill}%
\pgfsetlinewidth{0.000000pt}%
\definecolor{currentstroke}{rgb}{0.974417,0.903590,0.130215}%
\pgfsetstrokecolor{currentstroke}%
\pgfsetdash{}{0pt}%
\pgfpathmoveto{\pgfqpoint{1.596362in}{2.813436in}}%
\pgfpathlineto{\pgfqpoint{1.388975in}{2.359780in}}%
\pgfpathlineto{\pgfqpoint{1.520051in}{2.413317in}}%
\pgfpathclose%
\pgfusepath{fill}%
\end{pgfscope}%
\begin{pgfscope}%
\pgfpathrectangle{\pgfqpoint{0.539299in}{0.078740in}}{\pgfqpoint{7.842520in}{7.842520in}}%
\pgfusepath{clip}%
\pgfsetbuttcap%
\pgfsetroundjoin%
\definecolor{currentfill}{rgb}{0.123463,0.581687,0.547445}%
\pgfsetfillcolor{currentfill}%
\pgfsetlinewidth{0.000000pt}%
\definecolor{currentstroke}{rgb}{0.983868,0.904867,0.136897}%
\pgfsetstrokecolor{currentstroke}%
\pgfsetdash{}{0pt}%
\pgfpathmoveto{\pgfqpoint{4.626092in}{4.211058in}}%
\pgfpathlineto{\pgfqpoint{4.766857in}{3.997620in}}%
\pgfpathlineto{\pgfqpoint{4.846758in}{3.914972in}}%
\pgfpathclose%
\pgfusepath{fill}%
\end{pgfscope}%
\begin{pgfscope}%
\pgfpathrectangle{\pgfqpoint{0.539299in}{0.078740in}}{\pgfqpoint{7.842520in}{7.842520in}}%
\pgfusepath{clip}%
\pgfsetbuttcap%
\pgfsetroundjoin%
\definecolor{currentfill}{rgb}{0.171176,0.452530,0.557965}%
\pgfsetfillcolor{currentfill}%
\pgfsetlinewidth{0.000000pt}%
\definecolor{currentstroke}{rgb}{0.993248,0.906157,0.143936}%
\pgfsetstrokecolor{currentstroke}%
\pgfsetdash{}{0pt}%
\pgfpathmoveto{\pgfqpoint{1.756965in}{3.430597in}}%
\pgfpathlineto{\pgfqpoint{1.839990in}{3.669004in}}%
\pgfpathlineto{\pgfqpoint{1.718526in}{3.335879in}}%
\pgfpathclose%
\pgfusepath{fill}%
\end{pgfscope}%
\begin{pgfscope}%
\pgfpathrectangle{\pgfqpoint{0.539299in}{0.078740in}}{\pgfqpoint{7.842520in}{7.842520in}}%
\pgfusepath{clip}%
\pgfsetbuttcap%
\pgfsetroundjoin%
\definecolor{currentfill}{rgb}{0.515992,0.831158,0.294279}%
\pgfsetfillcolor{currentfill}%
\pgfsetlinewidth{0.000000pt}%
\definecolor{currentstroke}{rgb}{0.267004,0.004874,0.329415}%
\pgfsetstrokecolor{currentstroke}%
\pgfsetdash{}{0pt}%
\pgfpathmoveto{\pgfqpoint{2.694035in}{5.204781in}}%
\pgfpathlineto{\pgfqpoint{2.913349in}{5.441460in}}%
\pgfpathlineto{\pgfqpoint{2.781099in}{5.242417in}}%
\pgfpathclose%
\pgfusepath{fill}%
\end{pgfscope}%
\begin{pgfscope}%
\pgfpathrectangle{\pgfqpoint{0.539299in}{0.078740in}}{\pgfqpoint{7.842520in}{7.842520in}}%
\pgfusepath{clip}%
\pgfsetbuttcap%
\pgfsetroundjoin%
\definecolor{currentfill}{rgb}{0.606045,0.850733,0.236712}%
\pgfsetfillcolor{currentfill}%
\pgfsetlinewidth{0.000000pt}%
\definecolor{currentstroke}{rgb}{0.268510,0.009605,0.335427}%
\pgfsetstrokecolor{currentstroke}%
\pgfsetdash{}{0pt}%
\pgfpathmoveto{\pgfqpoint{3.584365in}{5.478002in}}%
\pgfpathlineto{\pgfqpoint{3.638229in}{5.473510in}}%
\pgfpathlineto{\pgfqpoint{3.723815in}{5.392306in}}%
\pgfpathclose%
\pgfusepath{fill}%
\end{pgfscope}%
\begin{pgfscope}%
\pgfpathrectangle{\pgfqpoint{0.539299in}{0.078740in}}{\pgfqpoint{7.842520in}{7.842520in}}%
\pgfusepath{clip}%
\pgfsetbuttcap%
\pgfsetroundjoin%
\definecolor{currentfill}{rgb}{0.678489,0.863742,0.189503}%
\pgfsetfillcolor{currentfill}%
\pgfsetlinewidth{0.000000pt}%
\definecolor{currentstroke}{rgb}{0.269944,0.014625,0.341379}%
\pgfsetstrokecolor{currentstroke}%
\pgfsetdash{}{0pt}%
\pgfpathmoveto{\pgfqpoint{3.359631in}{5.571327in}}%
\pgfpathlineto{\pgfqpoint{3.222442in}{5.537820in}}%
\pgfpathlineto{\pgfqpoint{3.135579in}{5.562556in}}%
\pgfpathclose%
\pgfusepath{fill}%
\end{pgfscope}%
\begin{pgfscope}%
\pgfpathrectangle{\pgfqpoint{0.539299in}{0.078740in}}{\pgfqpoint{7.842520in}{7.842520in}}%
\pgfusepath{clip}%
\pgfsetbuttcap%
\pgfsetroundjoin%
\definecolor{currentfill}{rgb}{0.668054,0.861999,0.196293}%
\pgfsetfillcolor{currentfill}%
\pgfsetlinewidth{0.000000pt}%
\definecolor{currentstroke}{rgb}{0.271305,0.019942,0.347269}%
\pgfsetstrokecolor{currentstroke}%
\pgfsetdash{}{0pt}%
\pgfpathmoveto{\pgfqpoint{3.359631in}{5.571327in}}%
\pgfpathlineto{\pgfqpoint{3.498366in}{5.547173in}}%
\pgfpathlineto{\pgfqpoint{3.584365in}{5.478002in}}%
\pgfpathclose%
\pgfusepath{fill}%
\end{pgfscope}%
\begin{pgfscope}%
\pgfpathrectangle{\pgfqpoint{0.539299in}{0.078740in}}{\pgfqpoint{7.842520in}{7.842520in}}%
\pgfusepath{clip}%
\pgfsetbuttcap%
\pgfsetroundjoin%
\definecolor{currentfill}{rgb}{0.220124,0.725509,0.466226}%
\pgfsetfillcolor{currentfill}%
\pgfsetlinewidth{0.000000pt}%
\definecolor{currentstroke}{rgb}{0.272594,0.025563,0.353093}%
\pgfsetstrokecolor{currentstroke}%
\pgfsetdash{}{0pt}%
\pgfpathmoveto{\pgfqpoint{4.426173in}{4.535091in}}%
\pgfpathlineto{\pgfqpoint{4.285648in}{4.739044in}}%
\pgfpathlineto{\pgfqpoint{4.202671in}{4.843466in}}%
\pgfpathclose%
\pgfusepath{fill}%
\end{pgfscope}%
\begin{pgfscope}%
\pgfpathrectangle{\pgfqpoint{0.539299in}{0.078740in}}{\pgfqpoint{7.842520in}{7.842520in}}%
\pgfusepath{clip}%
\pgfsetbuttcap%
\pgfsetroundjoin%
\definecolor{currentfill}{rgb}{0.121380,0.629492,0.531973}%
\pgfsetfillcolor{currentfill}%
\pgfsetlinewidth{0.000000pt}%
\definecolor{currentstroke}{rgb}{0.273809,0.031497,0.358853}%
\pgfsetstrokecolor{currentstroke}%
\pgfsetdash{}{0pt}%
\pgfpathmoveto{\pgfqpoint{2.219156in}{4.533644in}}%
\pgfpathlineto{\pgfqpoint{2.095141in}{4.184186in}}%
\pgfpathlineto{\pgfqpoint{2.009371in}{4.040892in}}%
\pgfpathclose%
\pgfusepath{fill}%
\end{pgfscope}%
\begin{pgfscope}%
\pgfpathrectangle{\pgfqpoint{0.539299in}{0.078740in}}{\pgfqpoint{7.842520in}{7.842520in}}%
\pgfusepath{clip}%
\pgfsetbuttcap%
\pgfsetroundjoin%
\definecolor{currentfill}{rgb}{0.395174,0.797475,0.367757}%
\pgfsetfillcolor{currentfill}%
\pgfsetlinewidth{0.000000pt}%
\definecolor{currentstroke}{rgb}{0.274952,0.037752,0.364543}%
\pgfsetstrokecolor{currentstroke}%
\pgfsetdash{}{0pt}%
\pgfpathmoveto{\pgfqpoint{2.478591in}{4.855876in}}%
\pgfpathlineto{\pgfqpoint{2.606940in}{5.146364in}}%
\pgfpathlineto{\pgfqpoint{2.694035in}{5.204781in}}%
\pgfpathclose%
\pgfusepath{fill}%
\end{pgfscope}%
\begin{pgfscope}%
\pgfpathrectangle{\pgfqpoint{0.539299in}{0.078740in}}{\pgfqpoint{7.842520in}{7.842520in}}%
\pgfusepath{clip}%
\pgfsetbuttcap%
\pgfsetroundjoin%
\definecolor{currentfill}{rgb}{0.241237,0.296485,0.539709}%
\pgfsetfillcolor{currentfill}%
\pgfsetlinewidth{0.000000pt}%
\definecolor{currentstroke}{rgb}{0.276022,0.044167,0.370164}%
\pgfsetstrokecolor{currentstroke}%
\pgfsetdash{}{0pt}%
\pgfpathmoveto{\pgfqpoint{5.749939in}{2.703326in}}%
\pgfpathlineto{\pgfqpoint{5.609380in}{2.861680in}}%
\pgfpathlineto{\pgfqpoint{5.532155in}{2.871486in}}%
\pgfpathclose%
\pgfusepath{fill}%
\end{pgfscope}%
\begin{pgfscope}%
\pgfpathrectangle{\pgfqpoint{0.539299in}{0.078740in}}{\pgfqpoint{7.842520in}{7.842520in}}%
\pgfusepath{clip}%
\pgfsetbuttcap%
\pgfsetroundjoin%
\definecolor{currentfill}{rgb}{0.229739,0.322361,0.545706}%
\pgfsetfillcolor{currentfill}%
\pgfsetlinewidth{0.000000pt}%
\definecolor{currentstroke}{rgb}{0.277018,0.050344,0.375715}%
\pgfsetstrokecolor{currentstroke}%
\pgfsetdash{}{0pt}%
\pgfpathmoveto{\pgfqpoint{5.532155in}{2.871486in}}%
\pgfpathlineto{\pgfqpoint{5.609380in}{2.861680in}}%
\pgfpathlineto{\pgfqpoint{5.468948in}{3.028932in}}%
\pgfpathclose%
\pgfusepath{fill}%
\end{pgfscope}%
\begin{pgfscope}%
\pgfpathrectangle{\pgfqpoint{0.539299in}{0.078740in}}{\pgfqpoint{7.842520in}{7.842520in}}%
\pgfusepath{clip}%
\pgfsetbuttcap%
\pgfsetroundjoin%
\definecolor{currentfill}{rgb}{0.263663,0.237631,0.518762}%
\pgfsetfillcolor{currentfill}%
\pgfsetlinewidth{0.000000pt}%
\definecolor{currentstroke}{rgb}{0.277941,0.056324,0.381191}%
\pgfsetstrokecolor{currentstroke}%
\pgfsetdash{}{0pt}%
\pgfpathmoveto{\pgfqpoint{5.814052in}{2.532529in}}%
\pgfpathlineto{\pgfqpoint{5.890666in}{2.553330in}}%
\pgfpathlineto{\pgfqpoint{5.749939in}{2.703326in}}%
\pgfpathclose%
\pgfusepath{fill}%
\end{pgfscope}%
\begin{pgfscope}%
\pgfpathrectangle{\pgfqpoint{0.539299in}{0.078740in}}{\pgfqpoint{7.842520in}{7.842520in}}%
\pgfusepath{clip}%
\pgfsetbuttcap%
\pgfsetroundjoin%
\definecolor{currentfill}{rgb}{0.276022,0.044167,0.370164}%
\pgfsetfillcolor{currentfill}%
\pgfsetlinewidth{0.000000pt}%
\definecolor{currentstroke}{rgb}{0.278791,0.062145,0.386592}%
\pgfsetstrokecolor{currentstroke}%
\pgfsetdash{}{0pt}%
\pgfpathmoveto{\pgfqpoint{6.455962in}{2.024776in}}%
\pgfpathlineto{\pgfqpoint{6.598042in}{1.908413in}}%
\pgfpathlineto{\pgfqpoint{6.673382in}{1.987622in}}%
\pgfpathclose%
\pgfusepath{fill}%
\end{pgfscope}%
\begin{pgfscope}%
\pgfpathrectangle{\pgfqpoint{0.539299in}{0.078740in}}{\pgfqpoint{7.842520in}{7.842520in}}%
\pgfusepath{clip}%
\pgfsetbuttcap%
\pgfsetroundjoin%
\definecolor{currentfill}{rgb}{0.121380,0.629492,0.531973}%
\pgfsetfillcolor{currentfill}%
\pgfsetlinewidth{0.000000pt}%
\definecolor{currentstroke}{rgb}{0.279566,0.067836,0.391917}%
\pgfsetstrokecolor{currentstroke}%
\pgfsetdash{}{0pt}%
\pgfpathmoveto{\pgfqpoint{4.485130in}{4.425835in}}%
\pgfpathlineto{\pgfqpoint{4.626092in}{4.211058in}}%
\pgfpathlineto{\pgfqpoint{4.706729in}{4.119068in}}%
\pgfpathclose%
\pgfusepath{fill}%
\end{pgfscope}%
\begin{pgfscope}%
\pgfpathrectangle{\pgfqpoint{0.539299in}{0.078740in}}{\pgfqpoint{7.842520in}{7.842520in}}%
\pgfusepath{clip}%
\pgfsetbuttcap%
\pgfsetroundjoin%
\definecolor{currentfill}{rgb}{0.281477,0.755203,0.432552}%
\pgfsetfillcolor{currentfill}%
\pgfsetlinewidth{0.000000pt}%
\definecolor{currentstroke}{rgb}{0.280267,0.073417,0.397163}%
\pgfsetstrokecolor{currentstroke}%
\pgfsetdash{}{0pt}%
\pgfpathmoveto{\pgfqpoint{4.202671in}{4.843466in}}%
\pgfpathlineto{\pgfqpoint{4.285648in}{4.739044in}}%
\pgfpathlineto{\pgfqpoint{4.145020in}{4.933425in}}%
\pgfpathclose%
\pgfusepath{fill}%
\end{pgfscope}%
\begin{pgfscope}%
\pgfpathrectangle{\pgfqpoint{0.539299in}{0.078740in}}{\pgfqpoint{7.842520in}{7.842520in}}%
\pgfusepath{clip}%
\pgfsetbuttcap%
\pgfsetroundjoin%
\definecolor{currentfill}{rgb}{0.275191,0.194905,0.496005}%
\pgfsetfillcolor{currentfill}%
\pgfsetlinewidth{0.000000pt}%
\definecolor{currentstroke}{rgb}{0.280894,0.078907,0.402329}%
\pgfsetstrokecolor{currentstroke}%
\pgfsetdash{}{0pt}%
\pgfpathmoveto{\pgfqpoint{5.955205in}{2.375680in}}%
\pgfpathlineto{\pgfqpoint{6.031600in}{2.411053in}}%
\pgfpathlineto{\pgfqpoint{5.890666in}{2.553330in}}%
\pgfpathclose%
\pgfusepath{fill}%
\end{pgfscope}%
\begin{pgfscope}%
\pgfpathrectangle{\pgfqpoint{0.539299in}{0.078740in}}{\pgfqpoint{7.842520in}{7.842520in}}%
\pgfusepath{clip}%
\pgfsetbuttcap%
\pgfsetroundjoin%
\definecolor{currentfill}{rgb}{0.233603,0.313828,0.543914}%
\pgfsetfillcolor{currentfill}%
\pgfsetlinewidth{0.000000pt}%
\definecolor{currentstroke}{rgb}{0.281446,0.084320,0.407414}%
\pgfsetstrokecolor{currentstroke}%
\pgfsetdash{}{0pt}%
\pgfpathmoveto{\pgfqpoint{1.468606in}{2.671104in}}%
\pgfpathlineto{\pgfqpoint{1.596362in}{2.813436in}}%
\pgfpathlineto{\pgfqpoint{1.675575in}{3.148455in}}%
\pgfpathclose%
\pgfusepath{fill}%
\end{pgfscope}%
\begin{pgfscope}%
\pgfpathrectangle{\pgfqpoint{0.539299in}{0.078740in}}{\pgfqpoint{7.842520in}{7.842520in}}%
\pgfusepath{clip}%
\pgfsetbuttcap%
\pgfsetroundjoin%
\definecolor{currentfill}{rgb}{0.162016,0.687316,0.499129}%
\pgfsetfillcolor{currentfill}%
\pgfsetlinewidth{0.000000pt}%
\definecolor{currentstroke}{rgb}{0.281924,0.089666,0.412415}%
\pgfsetstrokecolor{currentstroke}%
\pgfsetdash{}{0pt}%
\pgfpathmoveto{\pgfqpoint{2.181328in}{4.304148in}}%
\pgfpathlineto{\pgfqpoint{2.219156in}{4.533644in}}%
\pgfpathlineto{\pgfqpoint{2.305286in}{4.666266in}}%
\pgfpathclose%
\pgfusepath{fill}%
\end{pgfscope}%
\begin{pgfscope}%
\pgfpathrectangle{\pgfqpoint{0.539299in}{0.078740in}}{\pgfqpoint{7.842520in}{7.842520in}}%
\pgfusepath{clip}%
\pgfsetbuttcap%
\pgfsetroundjoin%
\definecolor{currentfill}{rgb}{0.201239,0.383670,0.554294}%
\pgfsetfillcolor{currentfill}%
\pgfsetlinewidth{0.000000pt}%
\definecolor{currentstroke}{rgb}{0.282327,0.094955,0.417331}%
\pgfsetstrokecolor{currentstroke}%
\pgfsetdash{}{0pt}%
\pgfpathmoveto{\pgfqpoint{5.468948in}{3.028932in}}%
\pgfpathlineto{\pgfqpoint{5.328594in}{3.205423in}}%
\pgfpathlineto{\pgfqpoint{5.250481in}{3.245064in}}%
\pgfpathclose%
\pgfusepath{fill}%
\end{pgfscope}%
\begin{pgfscope}%
\pgfpathrectangle{\pgfqpoint{0.539299in}{0.078740in}}{\pgfqpoint{7.842520in}{7.842520in}}%
\pgfusepath{clip}%
\pgfsetbuttcap%
\pgfsetroundjoin%
\definecolor{currentfill}{rgb}{0.555484,0.840254,0.269281}%
\pgfsetfillcolor{currentfill}%
\pgfsetlinewidth{0.000000pt}%
\definecolor{currentstroke}{rgb}{0.282656,0.100196,0.422160}%
\pgfsetstrokecolor{currentstroke}%
\pgfsetdash{}{0pt}%
\pgfpathmoveto{\pgfqpoint{3.863917in}{5.267515in}}%
\pgfpathlineto{\pgfqpoint{3.723815in}{5.392306in}}%
\pgfpathlineto{\pgfqpoint{3.778860in}{5.358543in}}%
\pgfpathclose%
\pgfusepath{fill}%
\end{pgfscope}%
\begin{pgfscope}%
\pgfpathrectangle{\pgfqpoint{0.539299in}{0.078740in}}{\pgfqpoint{7.842520in}{7.842520in}}%
\pgfusepath{clip}%
\pgfsetbuttcap%
\pgfsetroundjoin%
\definecolor{currentfill}{rgb}{0.199430,0.387607,0.554642}%
\pgfsetfillcolor{currentfill}%
\pgfsetlinewidth{0.000000pt}%
\definecolor{currentstroke}{rgb}{0.282910,0.105393,0.426902}%
\pgfsetstrokecolor{currentstroke}%
\pgfsetdash{}{0pt}%
\pgfpathmoveto{\pgfqpoint{1.550510in}{2.929884in}}%
\pgfpathlineto{\pgfqpoint{1.675575in}{3.148455in}}%
\pgfpathlineto{\pgfqpoint{1.756965in}{3.430597in}}%
\pgfpathclose%
\pgfusepath{fill}%
\end{pgfscope}%
\begin{pgfscope}%
\pgfpathrectangle{\pgfqpoint{0.539299in}{0.078740in}}{\pgfqpoint{7.842520in}{7.842520in}}%
\pgfusepath{clip}%
\pgfsetbuttcap%
\pgfsetroundjoin%
\definecolor{currentfill}{rgb}{0.636902,0.856542,0.216620}%
\pgfsetfillcolor{currentfill}%
\pgfsetlinewidth{0.000000pt}%
\definecolor{currentstroke}{rgb}{0.283091,0.110553,0.431554}%
\pgfsetstrokecolor{currentstroke}%
\pgfsetdash{}{0pt}%
\pgfpathmoveto{\pgfqpoint{3.135579in}{5.562556in}}%
\pgfpathlineto{\pgfqpoint{3.000456in}{5.448722in}}%
\pgfpathlineto{\pgfqpoint{2.913349in}{5.441460in}}%
\pgfpathclose%
\pgfusepath{fill}%
\end{pgfscope}%
\begin{pgfscope}%
\pgfpathrectangle{\pgfqpoint{0.539299in}{0.078740in}}{\pgfqpoint{7.842520in}{7.842520in}}%
\pgfusepath{clip}%
\pgfsetbuttcap%
\pgfsetroundjoin%
\definecolor{currentfill}{rgb}{0.281412,0.155834,0.469201}%
\pgfsetfillcolor{currentfill}%
\pgfsetlinewidth{0.000000pt}%
\definecolor{currentstroke}{rgb}{0.283197,0.115680,0.436115}%
\pgfsetstrokecolor{currentstroke}%
\pgfsetdash{}{0pt}%
\pgfpathmoveto{\pgfqpoint{6.172774in}{2.275861in}}%
\pgfpathlineto{\pgfqpoint{6.031600in}{2.411053in}}%
\pgfpathlineto{\pgfqpoint{6.096558in}{2.227023in}}%
\pgfpathclose%
\pgfusepath{fill}%
\end{pgfscope}%
\begin{pgfscope}%
\pgfpathrectangle{\pgfqpoint{0.539299in}{0.078740in}}{\pgfqpoint{7.842520in}{7.842520in}}%
\pgfusepath{clip}%
\pgfsetbuttcap%
\pgfsetroundjoin%
\definecolor{currentfill}{rgb}{0.187231,0.414746,0.556547}%
\pgfsetfillcolor{currentfill}%
\pgfsetlinewidth{0.000000pt}%
\definecolor{currentstroke}{rgb}{0.283229,0.120777,0.440584}%
\pgfsetstrokecolor{currentstroke}%
\pgfsetdash{}{0pt}%
\pgfpathmoveto{\pgfqpoint{5.250481in}{3.245064in}}%
\pgfpathlineto{\pgfqpoint{5.328594in}{3.205423in}}%
\pgfpathlineto{\pgfqpoint{5.188263in}{3.391183in}}%
\pgfpathclose%
\pgfusepath{fill}%
\end{pgfscope}%
\begin{pgfscope}%
\pgfpathrectangle{\pgfqpoint{0.539299in}{0.078740in}}{\pgfqpoint{7.842520in}{7.842520in}}%
\pgfusepath{clip}%
\pgfsetbuttcap%
\pgfsetroundjoin%
\definecolor{currentfill}{rgb}{0.377779,0.791781,0.377939}%
\pgfsetfillcolor{currentfill}%
\pgfsetlinewidth{0.000000pt}%
\definecolor{currentstroke}{rgb}{0.283187,0.125848,0.444960}%
\pgfsetstrokecolor{currentstroke}%
\pgfsetdash{}{0pt}%
\pgfpathmoveto{\pgfqpoint{4.061290in}{5.036247in}}%
\pgfpathlineto{\pgfqpoint{4.145020in}{4.933425in}}%
\pgfpathlineto{\pgfqpoint{4.004391in}{5.111952in}}%
\pgfpathclose%
\pgfusepath{fill}%
\end{pgfscope}%
\begin{pgfscope}%
\pgfpathrectangle{\pgfqpoint{0.539299in}{0.078740in}}{\pgfqpoint{7.842520in}{7.842520in}}%
\pgfusepath{clip}%
\pgfsetbuttcap%
\pgfsetroundjoin%
\definecolor{currentfill}{rgb}{0.496615,0.826376,0.306377}%
\pgfsetfillcolor{currentfill}%
\pgfsetlinewidth{0.000000pt}%
\definecolor{currentstroke}{rgb}{0.283072,0.130895,0.449241}%
\pgfsetstrokecolor{currentstroke}%
\pgfsetdash{}{0pt}%
\pgfpathmoveto{\pgfqpoint{3.778860in}{5.358543in}}%
\pgfpathlineto{\pgfqpoint{4.004391in}{5.111952in}}%
\pgfpathlineto{\pgfqpoint{3.863917in}{5.267515in}}%
\pgfpathclose%
\pgfusepath{fill}%
\end{pgfscope}%
\begin{pgfscope}%
\pgfpathrectangle{\pgfqpoint{0.539299in}{0.078740in}}{\pgfqpoint{7.842520in}{7.842520in}}%
\pgfusepath{clip}%
\pgfsetbuttcap%
\pgfsetroundjoin%
\definecolor{currentfill}{rgb}{0.311925,0.767822,0.415586}%
\pgfsetfillcolor{currentfill}%
\pgfsetlinewidth{0.000000pt}%
\definecolor{currentstroke}{rgb}{0.282884,0.135920,0.453427}%
\pgfsetstrokecolor{currentstroke}%
\pgfsetdash{}{0pt}%
\pgfpathmoveto{\pgfqpoint{2.478591in}{4.855876in}}%
\pgfpathlineto{\pgfqpoint{2.391827in}{4.772531in}}%
\pgfpathlineto{\pgfqpoint{2.519975in}{5.063797in}}%
\pgfpathclose%
\pgfusepath{fill}%
\end{pgfscope}%
\begin{pgfscope}%
\pgfpathrectangle{\pgfqpoint{0.539299in}{0.078740in}}{\pgfqpoint{7.842520in}{7.842520in}}%
\pgfusepath{clip}%
\pgfsetbuttcap%
\pgfsetroundjoin%
\definecolor{currentfill}{rgb}{0.283072,0.130895,0.449241}%
\pgfsetfillcolor{currentfill}%
\pgfsetlinewidth{0.000000pt}%
\definecolor{currentstroke}{rgb}{0.282623,0.140926,0.457517}%
\pgfsetstrokecolor{currentstroke}%
\pgfsetdash{}{0pt}%
\pgfpathmoveto{\pgfqpoint{6.096558in}{2.227023in}}%
\pgfpathlineto{\pgfqpoint{6.314217in}{2.147221in}}%
\pgfpathlineto{\pgfqpoint{6.172774in}{2.275861in}}%
\pgfpathclose%
\pgfusepath{fill}%
\end{pgfscope}%
\begin{pgfscope}%
\pgfpathrectangle{\pgfqpoint{0.539299in}{0.078740in}}{\pgfqpoint{7.842520in}{7.842520in}}%
\pgfusepath{clip}%
\pgfsetbuttcap%
\pgfsetroundjoin%
\definecolor{currentfill}{rgb}{0.657642,0.860219,0.203082}%
\pgfsetfillcolor{currentfill}%
\pgfsetlinewidth{0.000000pt}%
\definecolor{currentstroke}{rgb}{0.282290,0.145912,0.461510}%
\pgfsetstrokecolor{currentstroke}%
\pgfsetdash{}{0pt}%
\pgfpathmoveto{\pgfqpoint{3.498366in}{5.547173in}}%
\pgfpathlineto{\pgfqpoint{3.638229in}{5.473510in}}%
\pgfpathlineto{\pgfqpoint{3.584365in}{5.478002in}}%
\pgfpathclose%
\pgfusepath{fill}%
\end{pgfscope}%
\begin{pgfscope}%
\pgfpathrectangle{\pgfqpoint{0.539299in}{0.078740in}}{\pgfqpoint{7.842520in}{7.842520in}}%
\pgfusepath{clip}%
\pgfsetbuttcap%
\pgfsetroundjoin%
\definecolor{currentfill}{rgb}{0.175707,0.697900,0.491033}%
\pgfsetfillcolor{currentfill}%
\pgfsetlinewidth{0.000000pt}%
\definecolor{currentstroke}{rgb}{0.281887,0.150881,0.465405}%
\pgfsetstrokecolor{currentstroke}%
\pgfsetdash{}{0pt}%
\pgfpathmoveto{\pgfqpoint{4.343978in}{4.638176in}}%
\pgfpathlineto{\pgfqpoint{4.485130in}{4.425835in}}%
\pgfpathlineto{\pgfqpoint{4.426173in}{4.535091in}}%
\pgfpathclose%
\pgfusepath{fill}%
\end{pgfscope}%
\begin{pgfscope}%
\pgfpathrectangle{\pgfqpoint{0.539299in}{0.078740in}}{\pgfqpoint{7.842520in}{7.842520in}}%
\pgfusepath{clip}%
\pgfsetbuttcap%
\pgfsetroundjoin%
\definecolor{currentfill}{rgb}{0.160665,0.478540,0.558115}%
\pgfsetfillcolor{currentfill}%
\pgfsetlinewidth{0.000000pt}%
\definecolor{currentstroke}{rgb}{0.281412,0.155834,0.469201}%
\pgfsetstrokecolor{currentstroke}%
\pgfsetdash{}{0pt}%
\pgfpathmoveto{\pgfqpoint{4.968612in}{3.651484in}}%
\pgfpathlineto{\pgfqpoint{5.188263in}{3.391183in}}%
\pgfpathlineto{\pgfqpoint{5.047899in}{3.585834in}}%
\pgfpathclose%
\pgfusepath{fill}%
\end{pgfscope}%
\begin{pgfscope}%
\pgfpathrectangle{\pgfqpoint{0.539299in}{0.078740in}}{\pgfqpoint{7.842520in}{7.842520in}}%
\pgfusepath{clip}%
\pgfsetbuttcap%
\pgfsetroundjoin%
\definecolor{currentfill}{rgb}{0.282327,0.094955,0.417331}%
\pgfsetfillcolor{currentfill}%
\pgfsetlinewidth{0.000000pt}%
\definecolor{currentstroke}{rgb}{0.280868,0.160771,0.472899}%
\pgfsetstrokecolor{currentstroke}%
\pgfsetdash{}{0pt}%
\pgfpathmoveto{\pgfqpoint{6.314217in}{2.147221in}}%
\pgfpathlineto{\pgfqpoint{6.238160in}{2.086590in}}%
\pgfpathlineto{\pgfqpoint{6.455962in}{2.024776in}}%
\pgfpathclose%
\pgfusepath{fill}%
\end{pgfscope}%
\begin{pgfscope}%
\pgfpathrectangle{\pgfqpoint{0.539299in}{0.078740in}}{\pgfqpoint{7.842520in}{7.842520in}}%
\pgfusepath{clip}%
\pgfsetbuttcap%
\pgfsetroundjoin%
\definecolor{currentfill}{rgb}{0.255645,0.260703,0.528312}%
\pgfsetfillcolor{currentfill}%
\pgfsetlinewidth{0.000000pt}%
\definecolor{currentstroke}{rgb}{0.280255,0.165693,0.476498}%
\pgfsetstrokecolor{currentstroke}%
\pgfsetdash{}{0pt}%
\pgfpathmoveto{\pgfqpoint{5.814052in}{2.532529in}}%
\pgfpathlineto{\pgfqpoint{5.749939in}{2.703326in}}%
\pgfpathlineto{\pgfqpoint{5.673051in}{2.697726in}}%
\pgfpathclose%
\pgfusepath{fill}%
\end{pgfscope}%
\begin{pgfscope}%
\pgfpathrectangle{\pgfqpoint{0.539299in}{0.078740in}}{\pgfqpoint{7.842520in}{7.842520in}}%
\pgfusepath{clip}%
\pgfsetbuttcap%
\pgfsetroundjoin%
\definecolor{currentfill}{rgb}{0.243113,0.292092,0.538516}%
\pgfsetfillcolor{currentfill}%
\pgfsetlinewidth{0.000000pt}%
\definecolor{currentstroke}{rgb}{0.279574,0.170599,0.479997}%
\pgfsetstrokecolor{currentstroke}%
\pgfsetdash{}{0pt}%
\pgfpathmoveto{\pgfqpoint{5.673051in}{2.697726in}}%
\pgfpathlineto{\pgfqpoint{5.749939in}{2.703326in}}%
\pgfpathlineto{\pgfqpoint{5.532155in}{2.871486in}}%
\pgfpathclose%
\pgfusepath{fill}%
\end{pgfscope}%
\begin{pgfscope}%
\pgfpathrectangle{\pgfqpoint{0.539299in}{0.078740in}}{\pgfqpoint{7.842520in}{7.842520in}}%
\pgfusepath{clip}%
\pgfsetbuttcap%
\pgfsetroundjoin%
\definecolor{currentfill}{rgb}{0.270595,0.214069,0.507052}%
\pgfsetfillcolor{currentfill}%
\pgfsetlinewidth{0.000000pt}%
\definecolor{currentstroke}{rgb}{0.278826,0.175490,0.483397}%
\pgfsetstrokecolor{currentstroke}%
\pgfsetdash{}{0pt}%
\pgfpathmoveto{\pgfqpoint{5.890666in}{2.553330in}}%
\pgfpathlineto{\pgfqpoint{5.814052in}{2.532529in}}%
\pgfpathlineto{\pgfqpoint{5.955205in}{2.375680in}}%
\pgfpathclose%
\pgfusepath{fill}%
\end{pgfscope}%
\begin{pgfscope}%
\pgfpathrectangle{\pgfqpoint{0.539299in}{0.078740in}}{\pgfqpoint{7.842520in}{7.842520in}}%
\pgfusepath{clip}%
\pgfsetbuttcap%
\pgfsetroundjoin%
\definecolor{currentfill}{rgb}{0.226397,0.728888,0.462789}%
\pgfsetfillcolor{currentfill}%
\pgfsetlinewidth{0.000000pt}%
\definecolor{currentstroke}{rgb}{0.278012,0.180367,0.486697}%
\pgfsetstrokecolor{currentstroke}%
\pgfsetdash{}{0pt}%
\pgfpathmoveto{\pgfqpoint{4.202671in}{4.843466in}}%
\pgfpathlineto{\pgfqpoint{4.343978in}{4.638176in}}%
\pgfpathlineto{\pgfqpoint{4.426173in}{4.535091in}}%
\pgfpathclose%
\pgfusepath{fill}%
\end{pgfscope}%
\begin{pgfscope}%
\pgfpathrectangle{\pgfqpoint{0.539299in}{0.078740in}}{\pgfqpoint{7.842520in}{7.842520in}}%
\pgfusepath{clip}%
\pgfsetbuttcap%
\pgfsetroundjoin%
\definecolor{currentfill}{rgb}{0.216210,0.351535,0.550627}%
\pgfsetfillcolor{currentfill}%
\pgfsetlinewidth{0.000000pt}%
\definecolor{currentstroke}{rgb}{0.277134,0.185228,0.489898}%
\pgfsetstrokecolor{currentstroke}%
\pgfsetdash{}{0pt}%
\pgfpathmoveto{\pgfqpoint{5.468948in}{3.028932in}}%
\pgfpathlineto{\pgfqpoint{5.391316in}{3.053947in}}%
\pgfpathlineto{\pgfqpoint{5.532155in}{2.871486in}}%
\pgfpathclose%
\pgfusepath{fill}%
\end{pgfscope}%
\begin{pgfscope}%
\pgfpathrectangle{\pgfqpoint{0.539299in}{0.078740in}}{\pgfqpoint{7.842520in}{7.842520in}}%
\pgfusepath{clip}%
\pgfsetbuttcap%
\pgfsetroundjoin%
\definecolor{currentfill}{rgb}{0.149039,0.508051,0.557250}%
\pgfsetfillcolor{currentfill}%
\pgfsetlinewidth{0.000000pt}%
\definecolor{currentstroke}{rgb}{0.276194,0.190074,0.493001}%
\pgfsetstrokecolor{currentstroke}%
\pgfsetdash{}{0pt}%
\pgfpathmoveto{\pgfqpoint{5.047899in}{3.585834in}}%
\pgfpathlineto{\pgfqpoint{4.907447in}{3.788480in}}%
\pgfpathlineto{\pgfqpoint{4.968612in}{3.651484in}}%
\pgfpathclose%
\pgfusepath{fill}%
\end{pgfscope}%
\begin{pgfscope}%
\pgfpathrectangle{\pgfqpoint{0.539299in}{0.078740in}}{\pgfqpoint{7.842520in}{7.842520in}}%
\pgfusepath{clip}%
\pgfsetbuttcap%
\pgfsetroundjoin%
\definecolor{currentfill}{rgb}{0.279574,0.170599,0.479997}%
\pgfsetfillcolor{currentfill}%
\pgfsetlinewidth{0.000000pt}%
\definecolor{currentstroke}{rgb}{0.275191,0.194905,0.496005}%
\pgfsetstrokecolor{currentstroke}%
\pgfsetdash{}{0pt}%
\pgfpathmoveto{\pgfqpoint{6.031600in}{2.411053in}}%
\pgfpathlineto{\pgfqpoint{5.955205in}{2.375680in}}%
\pgfpathlineto{\pgfqpoint{6.096558in}{2.227023in}}%
\pgfpathclose%
\pgfusepath{fill}%
\end{pgfscope}%
\begin{pgfscope}%
\pgfpathrectangle{\pgfqpoint{0.539299in}{0.078740in}}{\pgfqpoint{7.842520in}{7.842520in}}%
\pgfusepath{clip}%
\pgfsetbuttcap%
\pgfsetroundjoin%
\definecolor{currentfill}{rgb}{0.203063,0.379716,0.553925}%
\pgfsetfillcolor{currentfill}%
\pgfsetlinewidth{0.000000pt}%
\definecolor{currentstroke}{rgb}{0.274128,0.199721,0.498911}%
\pgfsetstrokecolor{currentstroke}%
\pgfsetdash{}{0pt}%
\pgfpathmoveto{\pgfqpoint{5.250481in}{3.245064in}}%
\pgfpathlineto{\pgfqpoint{5.391316in}{3.053947in}}%
\pgfpathlineto{\pgfqpoint{5.468948in}{3.028932in}}%
\pgfpathclose%
\pgfusepath{fill}%
\end{pgfscope}%
\begin{pgfscope}%
\pgfpathrectangle{\pgfqpoint{0.539299in}{0.078740in}}{\pgfqpoint{7.842520in}{7.842520in}}%
\pgfusepath{clip}%
\pgfsetbuttcap%
\pgfsetroundjoin%
\definecolor{currentfill}{rgb}{0.616293,0.852709,0.230052}%
\pgfsetfillcolor{currentfill}%
\pgfsetlinewidth{0.000000pt}%
\definecolor{currentstroke}{rgb}{0.273006,0.204520,0.501721}%
\pgfsetstrokecolor{currentstroke}%
\pgfsetdash{}{0pt}%
\pgfpathmoveto{\pgfqpoint{3.723815in}{5.392306in}}%
\pgfpathlineto{\pgfqpoint{3.638229in}{5.473510in}}%
\pgfpathlineto{\pgfqpoint{3.778860in}{5.358543in}}%
\pgfpathclose%
\pgfusepath{fill}%
\end{pgfscope}%
\begin{pgfscope}%
\pgfpathrectangle{\pgfqpoint{0.539299in}{0.078740in}}{\pgfqpoint{7.842520in}{7.842520in}}%
\pgfusepath{clip}%
\pgfsetbuttcap%
\pgfsetroundjoin%
\definecolor{currentfill}{rgb}{0.395174,0.797475,0.367757}%
\pgfsetfillcolor{currentfill}%
\pgfsetlinewidth{0.000000pt}%
\definecolor{currentstroke}{rgb}{0.271828,0.209303,0.504434}%
\pgfsetstrokecolor{currentstroke}%
\pgfsetdash{}{0pt}%
\pgfpathmoveto{\pgfqpoint{2.519975in}{5.063797in}}%
\pgfpathlineto{\pgfqpoint{2.606940in}{5.146364in}}%
\pgfpathlineto{\pgfqpoint{2.478591in}{4.855876in}}%
\pgfpathclose%
\pgfusepath{fill}%
\end{pgfscope}%
\begin{pgfscope}%
\pgfpathrectangle{\pgfqpoint{0.539299in}{0.078740in}}{\pgfqpoint{7.842520in}{7.842520in}}%
\pgfusepath{clip}%
\pgfsetbuttcap%
\pgfsetroundjoin%
\definecolor{currentfill}{rgb}{0.276022,0.044167,0.370164}%
\pgfsetfillcolor{currentfill}%
\pgfsetlinewidth{0.000000pt}%
\definecolor{currentstroke}{rgb}{0.270595,0.214069,0.507052}%
\pgfsetstrokecolor{currentstroke}%
\pgfsetdash{}{0pt}%
\pgfpathmoveto{\pgfqpoint{6.522373in}{1.832361in}}%
\pgfpathlineto{\pgfqpoint{6.598042in}{1.908413in}}%
\pgfpathlineto{\pgfqpoint{6.455962in}{2.024776in}}%
\pgfpathclose%
\pgfusepath{fill}%
\end{pgfscope}%
\begin{pgfscope}%
\pgfpathrectangle{\pgfqpoint{0.539299in}{0.078740in}}{\pgfqpoint{7.842520in}{7.842520in}}%
\pgfusepath{clip}%
\pgfsetbuttcap%
\pgfsetroundjoin%
\definecolor{currentfill}{rgb}{0.283229,0.120777,0.440584}%
\pgfsetfillcolor{currentfill}%
\pgfsetlinewidth{0.000000pt}%
\definecolor{currentstroke}{rgb}{0.269308,0.218818,0.509577}%
\pgfsetstrokecolor{currentstroke}%
\pgfsetdash{}{0pt}%
\pgfpathmoveto{\pgfqpoint{6.096558in}{2.227023in}}%
\pgfpathlineto{\pgfqpoint{6.238160in}{2.086590in}}%
\pgfpathlineto{\pgfqpoint{6.314217in}{2.147221in}}%
\pgfpathclose%
\pgfusepath{fill}%
\end{pgfscope}%
\begin{pgfscope}%
\pgfpathrectangle{\pgfqpoint{0.539299in}{0.078740in}}{\pgfqpoint{7.842520in}{7.842520in}}%
\pgfusepath{clip}%
\pgfsetbuttcap%
\pgfsetroundjoin%
\definecolor{currentfill}{rgb}{0.352360,0.783011,0.392636}%
\pgfsetfillcolor{currentfill}%
\pgfsetlinewidth{0.000000pt}%
\definecolor{currentstroke}{rgb}{0.267968,0.223549,0.512008}%
\pgfsetstrokecolor{currentstroke}%
\pgfsetdash{}{0pt}%
\pgfpathmoveto{\pgfqpoint{4.145020in}{4.933425in}}%
\pgfpathlineto{\pgfqpoint{4.061290in}{5.036247in}}%
\pgfpathlineto{\pgfqpoint{4.202671in}{4.843466in}}%
\pgfpathclose%
\pgfusepath{fill}%
\end{pgfscope}%
\begin{pgfscope}%
\pgfpathrectangle{\pgfqpoint{0.539299in}{0.078740in}}{\pgfqpoint{7.842520in}{7.842520in}}%
\pgfusepath{clip}%
\pgfsetbuttcap%
\pgfsetroundjoin%
\definecolor{currentfill}{rgb}{0.121148,0.592739,0.544641}%
\pgfsetfillcolor{currentfill}%
\pgfsetlinewidth{0.000000pt}%
\definecolor{currentstroke}{rgb}{0.266580,0.228262,0.514349}%
\pgfsetstrokecolor{currentstroke}%
\pgfsetdash{}{0pt}%
\pgfpathmoveto{\pgfqpoint{2.049048in}{4.173228in}}%
\pgfpathlineto{\pgfqpoint{2.009371in}{4.040892in}}%
\pgfpathlineto{\pgfqpoint{1.924234in}{3.870625in}}%
\pgfpathclose%
\pgfusepath{fill}%
\end{pgfscope}%
\begin{pgfscope}%
\pgfpathrectangle{\pgfqpoint{0.539299in}{0.078740in}}{\pgfqpoint{7.842520in}{7.842520in}}%
\pgfusepath{clip}%
\pgfsetbuttcap%
\pgfsetroundjoin%
\definecolor{currentfill}{rgb}{0.246811,0.283237,0.535941}%
\pgfsetfillcolor{currentfill}%
\pgfsetlinewidth{0.000000pt}%
\definecolor{currentstroke}{rgb}{0.265145,0.232956,0.516599}%
\pgfsetstrokecolor{currentstroke}%
\pgfsetdash{}{0pt}%
\pgfpathmoveto{\pgfqpoint{1.726778in}{2.916957in}}%
\pgfpathlineto{\pgfqpoint{1.596362in}{2.813436in}}%
\pgfpathlineto{\pgfqpoint{1.520051in}{2.413317in}}%
\pgfpathclose%
\pgfusepath{fill}%
\end{pgfscope}%
\begin{pgfscope}%
\pgfpathrectangle{\pgfqpoint{0.539299in}{0.078740in}}{\pgfqpoint{7.842520in}{7.842520in}}%
\pgfusepath{clip}%
\pgfsetbuttcap%
\pgfsetroundjoin%
\definecolor{currentfill}{rgb}{0.129933,0.559582,0.551864}%
\pgfsetfillcolor{currentfill}%
\pgfsetlinewidth{0.000000pt}%
\definecolor{currentstroke}{rgb}{0.263663,0.237631,0.518762}%
\pgfsetstrokecolor{currentstroke}%
\pgfsetdash{}{0pt}%
\pgfpathmoveto{\pgfqpoint{4.827480in}{3.864804in}}%
\pgfpathlineto{\pgfqpoint{4.907447in}{3.788480in}}%
\pgfpathlineto{\pgfqpoint{4.766857in}{3.997620in}}%
\pgfpathclose%
\pgfusepath{fill}%
\end{pgfscope}%
\begin{pgfscope}%
\pgfpathrectangle{\pgfqpoint{0.539299in}{0.078740in}}{\pgfqpoint{7.842520in}{7.842520in}}%
\pgfusepath{clip}%
\pgfsetbuttcap%
\pgfsetroundjoin%
\definecolor{currentfill}{rgb}{0.730889,0.871916,0.156029}%
\pgfsetfillcolor{currentfill}%
\pgfsetlinewidth{0.000000pt}%
\definecolor{currentstroke}{rgb}{0.262138,0.242286,0.520837}%
\pgfsetstrokecolor{currentstroke}%
\pgfsetdash{}{0pt}%
\pgfpathmoveto{\pgfqpoint{3.135579in}{5.562556in}}%
\pgfpathlineto{\pgfqpoint{3.272857in}{5.610573in}}%
\pgfpathlineto{\pgfqpoint{3.359631in}{5.571327in}}%
\pgfpathclose%
\pgfusepath{fill}%
\end{pgfscope}%
\begin{pgfscope}%
\pgfpathrectangle{\pgfqpoint{0.539299in}{0.078740in}}{\pgfqpoint{7.842520in}{7.842520in}}%
\pgfusepath{clip}%
\pgfsetbuttcap%
\pgfsetroundjoin%
\definecolor{currentfill}{rgb}{0.175841,0.441290,0.557685}%
\pgfsetfillcolor{currentfill}%
\pgfsetlinewidth{0.000000pt}%
\definecolor{currentstroke}{rgb}{0.260571,0.246922,0.522828}%
\pgfsetstrokecolor{currentstroke}%
\pgfsetdash{}{0pt}%
\pgfpathmoveto{\pgfqpoint{5.188263in}{3.391183in}}%
\pgfpathlineto{\pgfqpoint{5.109597in}{3.444492in}}%
\pgfpathlineto{\pgfqpoint{5.250481in}{3.245064in}}%
\pgfpathclose%
\pgfusepath{fill}%
\end{pgfscope}%
\begin{pgfscope}%
\pgfpathrectangle{\pgfqpoint{0.539299in}{0.078740in}}{\pgfqpoint{7.842520in}{7.842520in}}%
\pgfusepath{clip}%
\pgfsetbuttcap%
\pgfsetroundjoin%
\definecolor{currentfill}{rgb}{0.730889,0.871916,0.156029}%
\pgfsetfillcolor{currentfill}%
\pgfsetlinewidth{0.000000pt}%
\definecolor{currentstroke}{rgb}{0.258965,0.251537,0.524736}%
\pgfsetstrokecolor{currentstroke}%
\pgfsetdash{}{0pt}%
\pgfpathmoveto{\pgfqpoint{3.498366in}{5.547173in}}%
\pgfpathlineto{\pgfqpoint{3.359631in}{5.571327in}}%
\pgfpathlineto{\pgfqpoint{3.272857in}{5.610573in}}%
\pgfpathclose%
\pgfusepath{fill}%
\end{pgfscope}%
\begin{pgfscope}%
\pgfpathrectangle{\pgfqpoint{0.539299in}{0.078740in}}{\pgfqpoint{7.842520in}{7.842520in}}%
\pgfusepath{clip}%
\pgfsetbuttcap%
\pgfsetroundjoin%
\definecolor{currentfill}{rgb}{0.585678,0.846661,0.249897}%
\pgfsetfillcolor{currentfill}%
\pgfsetlinewidth{0.000000pt}%
\definecolor{currentstroke}{rgb}{0.257322,0.256130,0.526563}%
\pgfsetstrokecolor{currentstroke}%
\pgfsetdash{}{0pt}%
\pgfpathmoveto{\pgfqpoint{2.694035in}{5.204781in}}%
\pgfpathlineto{\pgfqpoint{2.826074in}{5.413789in}}%
\pgfpathlineto{\pgfqpoint{2.913349in}{5.441460in}}%
\pgfpathclose%
\pgfusepath{fill}%
\end{pgfscope}%
\begin{pgfscope}%
\pgfpathrectangle{\pgfqpoint{0.539299in}{0.078740in}}{\pgfqpoint{7.842520in}{7.842520in}}%
\pgfusepath{clip}%
\pgfsetbuttcap%
\pgfsetroundjoin%
\definecolor{currentfill}{rgb}{0.506271,0.828786,0.300362}%
\pgfsetfillcolor{currentfill}%
\pgfsetlinewidth{0.000000pt}%
\definecolor{currentstroke}{rgb}{0.255645,0.260703,0.528312}%
\pgfsetstrokecolor{currentstroke}%
\pgfsetdash{}{0pt}%
\pgfpathmoveto{\pgfqpoint{3.919960in}{5.210264in}}%
\pgfpathlineto{\pgfqpoint{4.004391in}{5.111952in}}%
\pgfpathlineto{\pgfqpoint{3.778860in}{5.358543in}}%
\pgfpathclose%
\pgfusepath{fill}%
\end{pgfscope}%
\begin{pgfscope}%
\pgfpathrectangle{\pgfqpoint{0.539299in}{0.078740in}}{\pgfqpoint{7.842520in}{7.842520in}}%
\pgfusepath{clip}%
\pgfsetbuttcap%
\pgfsetroundjoin%
\definecolor{currentfill}{rgb}{0.449368,0.813768,0.335384}%
\pgfsetfillcolor{currentfill}%
\pgfsetlinewidth{0.000000pt}%
\definecolor{currentstroke}{rgb}{0.253935,0.265254,0.529983}%
\pgfsetstrokecolor{currentstroke}%
\pgfsetdash{}{0pt}%
\pgfpathmoveto{\pgfqpoint{3.919960in}{5.210264in}}%
\pgfpathlineto{\pgfqpoint{4.061290in}{5.036247in}}%
\pgfpathlineto{\pgfqpoint{4.004391in}{5.111952in}}%
\pgfpathclose%
\pgfusepath{fill}%
\end{pgfscope}%
\begin{pgfscope}%
\pgfpathrectangle{\pgfqpoint{0.539299in}{0.078740in}}{\pgfqpoint{7.842520in}{7.842520in}}%
\pgfusepath{clip}%
\pgfsetbuttcap%
\pgfsetroundjoin%
\definecolor{currentfill}{rgb}{0.281446,0.084320,0.407414}%
\pgfsetfillcolor{currentfill}%
\pgfsetlinewidth{0.000000pt}%
\definecolor{currentstroke}{rgb}{0.252194,0.269783,0.531579}%
\pgfsetstrokecolor{currentstroke}%
\pgfsetdash{}{0pt}%
\pgfpathmoveto{\pgfqpoint{6.455962in}{2.024776in}}%
\pgfpathlineto{\pgfqpoint{6.238160in}{2.086590in}}%
\pgfpathlineto{\pgfqpoint{6.380073in}{1.954750in}}%
\pgfpathclose%
\pgfusepath{fill}%
\end{pgfscope}%
\begin{pgfscope}%
\pgfpathrectangle{\pgfqpoint{0.539299in}{0.078740in}}{\pgfqpoint{7.842520in}{7.842520in}}%
\pgfusepath{clip}%
\pgfsetbuttcap%
\pgfsetroundjoin%
\definecolor{currentfill}{rgb}{0.258965,0.251537,0.524736}%
\pgfsetfillcolor{currentfill}%
\pgfsetlinewidth{0.000000pt}%
\definecolor{currentstroke}{rgb}{0.250425,0.274290,0.533103}%
\pgfsetstrokecolor{currentstroke}%
\pgfsetdash{}{0pt}%
\pgfpathmoveto{\pgfqpoint{1.520051in}{2.413317in}}%
\pgfpathlineto{\pgfqpoint{1.652813in}{2.445145in}}%
\pgfpathlineto{\pgfqpoint{1.726778in}{2.916957in}}%
\pgfpathclose%
\pgfusepath{fill}%
\end{pgfscope}%
\begin{pgfscope}%
\pgfpathrectangle{\pgfqpoint{0.539299in}{0.078740in}}{\pgfqpoint{7.842520in}{7.842520in}}%
\pgfusepath{clip}%
\pgfsetbuttcap%
\pgfsetroundjoin%
\definecolor{currentfill}{rgb}{0.304148,0.764704,0.419943}%
\pgfsetfillcolor{currentfill}%
\pgfsetlinewidth{0.000000pt}%
\definecolor{currentstroke}{rgb}{0.248629,0.278775,0.534556}%
\pgfsetstrokecolor{currentstroke}%
\pgfsetdash{}{0pt}%
\pgfpathmoveto{\pgfqpoint{2.519975in}{5.063797in}}%
\pgfpathlineto{\pgfqpoint{2.391827in}{4.772531in}}%
\pgfpathlineto{\pgfqpoint{2.305286in}{4.666266in}}%
\pgfpathclose%
\pgfusepath{fill}%
\end{pgfscope}%
\begin{pgfscope}%
\pgfpathrectangle{\pgfqpoint{0.539299in}{0.078740in}}{\pgfqpoint{7.842520in}{7.842520in}}%
\pgfusepath{clip}%
\pgfsetbuttcap%
\pgfsetroundjoin%
\definecolor{currentfill}{rgb}{0.162142,0.474838,0.558140}%
\pgfsetfillcolor{currentfill}%
\pgfsetlinewidth{0.000000pt}%
\definecolor{currentstroke}{rgb}{0.246811,0.283237,0.535941}%
\pgfsetstrokecolor{currentstroke}%
\pgfsetdash{}{0pt}%
\pgfpathmoveto{\pgfqpoint{4.968612in}{3.651484in}}%
\pgfpathlineto{\pgfqpoint{5.109597in}{3.444492in}}%
\pgfpathlineto{\pgfqpoint{5.188263in}{3.391183in}}%
\pgfpathclose%
\pgfusepath{fill}%
\end{pgfscope}%
\begin{pgfscope}%
\pgfpathrectangle{\pgfqpoint{0.539299in}{0.078740in}}{\pgfqpoint{7.842520in}{7.842520in}}%
\pgfusepath{clip}%
\pgfsetbuttcap%
\pgfsetroundjoin%
\definecolor{currentfill}{rgb}{0.277941,0.056324,0.381191}%
\pgfsetfillcolor{currentfill}%
\pgfsetlinewidth{0.000000pt}%
\definecolor{currentstroke}{rgb}{0.244972,0.287675,0.537260}%
\pgfsetstrokecolor{currentstroke}%
\pgfsetdash{}{0pt}%
\pgfpathmoveto{\pgfqpoint{6.380073in}{1.954750in}}%
\pgfpathlineto{\pgfqpoint{6.522373in}{1.832361in}}%
\pgfpathlineto{\pgfqpoint{6.455962in}{2.024776in}}%
\pgfpathclose%
\pgfusepath{fill}%
\end{pgfscope}%
\begin{pgfscope}%
\pgfpathrectangle{\pgfqpoint{0.539299in}{0.078740in}}{\pgfqpoint{7.842520in}{7.842520in}}%
\pgfusepath{clip}%
\pgfsetbuttcap%
\pgfsetroundjoin%
\definecolor{currentfill}{rgb}{0.134692,0.658636,0.517649}%
\pgfsetfillcolor{currentfill}%
\pgfsetlinewidth{0.000000pt}%
\definecolor{currentstroke}{rgb}{0.243113,0.292092,0.538516}%
\pgfsetstrokecolor{currentstroke}%
\pgfsetdash{}{0pt}%
\pgfpathmoveto{\pgfqpoint{2.219156in}{4.533644in}}%
\pgfpathlineto{\pgfqpoint{2.009371in}{4.040892in}}%
\pgfpathlineto{\pgfqpoint{2.133656in}{4.370779in}}%
\pgfpathclose%
\pgfusepath{fill}%
\end{pgfscope}%
\begin{pgfscope}%
\pgfpathrectangle{\pgfqpoint{0.539299in}{0.078740in}}{\pgfqpoint{7.842520in}{7.842520in}}%
\pgfusepath{clip}%
\pgfsetbuttcap%
\pgfsetroundjoin%
\definecolor{currentfill}{rgb}{0.216210,0.351535,0.550627}%
\pgfsetfillcolor{currentfill}%
\pgfsetlinewidth{0.000000pt}%
\definecolor{currentstroke}{rgb}{0.241237,0.296485,0.539709}%
\pgfsetstrokecolor{currentstroke}%
\pgfsetdash{}{0pt}%
\pgfpathmoveto{\pgfqpoint{1.726778in}{2.916957in}}%
\pgfpathlineto{\pgfqpoint{1.675575in}{3.148455in}}%
\pgfpathlineto{\pgfqpoint{1.596362in}{2.813436in}}%
\pgfpathclose%
\pgfusepath{fill}%
\end{pgfscope}%
\begin{pgfscope}%
\pgfpathrectangle{\pgfqpoint{0.539299in}{0.078740in}}{\pgfqpoint{7.842520in}{7.842520in}}%
\pgfusepath{clip}%
\pgfsetbuttcap%
\pgfsetroundjoin%
\definecolor{currentfill}{rgb}{0.699415,0.867117,0.175971}%
\pgfsetfillcolor{currentfill}%
\pgfsetlinewidth{0.000000pt}%
\definecolor{currentstroke}{rgb}{0.239346,0.300855,0.540844}%
\pgfsetstrokecolor{currentstroke}%
\pgfsetdash{}{0pt}%
\pgfpathmoveto{\pgfqpoint{2.913349in}{5.441460in}}%
\pgfpathlineto{\pgfqpoint{3.048388in}{5.568482in}}%
\pgfpathlineto{\pgfqpoint{3.135579in}{5.562556in}}%
\pgfpathclose%
\pgfusepath{fill}%
\end{pgfscope}%
\begin{pgfscope}%
\pgfpathrectangle{\pgfqpoint{0.539299in}{0.078740in}}{\pgfqpoint{7.842520in}{7.842520in}}%
\pgfusepath{clip}%
\pgfsetbuttcap%
\pgfsetroundjoin%
\definecolor{currentfill}{rgb}{0.127568,0.566949,0.550556}%
\pgfsetfillcolor{currentfill}%
\pgfsetlinewidth{0.000000pt}%
\definecolor{currentstroke}{rgb}{0.237441,0.305202,0.541921}%
\pgfsetstrokecolor{currentstroke}%
\pgfsetdash{}{0pt}%
\pgfpathmoveto{\pgfqpoint{1.839990in}{3.669004in}}%
\pgfpathlineto{\pgfqpoint{2.049048in}{4.173228in}}%
\pgfpathlineto{\pgfqpoint{1.924234in}{3.870625in}}%
\pgfpathclose%
\pgfusepath{fill}%
\end{pgfscope}%
\begin{pgfscope}%
\pgfpathrectangle{\pgfqpoint{0.539299in}{0.078740in}}{\pgfqpoint{7.842520in}{7.842520in}}%
\pgfusepath{clip}%
\pgfsetbuttcap%
\pgfsetroundjoin%
\definecolor{currentfill}{rgb}{0.121380,0.629492,0.531973}%
\pgfsetfillcolor{currentfill}%
\pgfsetlinewidth{0.000000pt}%
\definecolor{currentstroke}{rgb}{0.235526,0.309527,0.542944}%
\pgfsetstrokecolor{currentstroke}%
\pgfsetdash{}{0pt}%
\pgfpathmoveto{\pgfqpoint{4.766857in}{3.997620in}}%
\pgfpathlineto{\pgfqpoint{4.626092in}{4.211058in}}%
\pgfpathlineto{\pgfqpoint{4.544630in}{4.302516in}}%
\pgfpathclose%
\pgfusepath{fill}%
\end{pgfscope}%
\begin{pgfscope}%
\pgfpathrectangle{\pgfqpoint{0.539299in}{0.078740in}}{\pgfqpoint{7.842520in}{7.842520in}}%
\pgfusepath{clip}%
\pgfsetbuttcap%
\pgfsetroundjoin%
\definecolor{currentfill}{rgb}{0.136408,0.541173,0.554483}%
\pgfsetfillcolor{currentfill}%
\pgfsetlinewidth{0.000000pt}%
\definecolor{currentstroke}{rgb}{0.233603,0.313828,0.543914}%
\pgfsetstrokecolor{currentstroke}%
\pgfsetdash{}{0pt}%
\pgfpathmoveto{\pgfqpoint{4.907447in}{3.788480in}}%
\pgfpathlineto{\pgfqpoint{4.827480in}{3.864804in}}%
\pgfpathlineto{\pgfqpoint{4.968612in}{3.651484in}}%
\pgfpathclose%
\pgfusepath{fill}%
\end{pgfscope}%
\begin{pgfscope}%
\pgfpathrectangle{\pgfqpoint{0.539299in}{0.078740in}}{\pgfqpoint{7.842520in}{7.842520in}}%
\pgfusepath{clip}%
\pgfsetbuttcap%
\pgfsetroundjoin%
\definecolor{currentfill}{rgb}{0.137339,0.662252,0.515571}%
\pgfsetfillcolor{currentfill}%
\pgfsetlinewidth{0.000000pt}%
\definecolor{currentstroke}{rgb}{0.231674,0.318106,0.544834}%
\pgfsetstrokecolor{currentstroke}%
\pgfsetdash{}{0pt}%
\pgfpathmoveto{\pgfqpoint{4.626092in}{4.211058in}}%
\pgfpathlineto{\pgfqpoint{4.485130in}{4.425835in}}%
\pgfpathlineto{\pgfqpoint{4.544630in}{4.302516in}}%
\pgfpathclose%
\pgfusepath{fill}%
\end{pgfscope}%
\begin{pgfscope}%
\pgfpathrectangle{\pgfqpoint{0.539299in}{0.078740in}}{\pgfqpoint{7.842520in}{7.842520in}}%
\pgfusepath{clip}%
\pgfsetbuttcap%
\pgfsetroundjoin%
\definecolor{currentfill}{rgb}{0.177423,0.437527,0.557565}%
\pgfsetfillcolor{currentfill}%
\pgfsetlinewidth{0.000000pt}%
\definecolor{currentstroke}{rgb}{0.229739,0.322361,0.545706}%
\pgfsetstrokecolor{currentstroke}%
\pgfsetdash{}{0pt}%
\pgfpathmoveto{\pgfqpoint{1.756965in}{3.430597in}}%
\pgfpathlineto{\pgfqpoint{1.675575in}{3.148455in}}%
\pgfpathlineto{\pgfqpoint{1.804002in}{3.315816in}}%
\pgfpathclose%
\pgfusepath{fill}%
\end{pgfscope}%
\begin{pgfscope}%
\pgfpathrectangle{\pgfqpoint{0.539299in}{0.078740in}}{\pgfqpoint{7.842520in}{7.842520in}}%
\pgfusepath{clip}%
\pgfsetbuttcap%
\pgfsetroundjoin%
\definecolor{currentfill}{rgb}{0.143343,0.522773,0.556295}%
\pgfsetfillcolor{currentfill}%
\pgfsetlinewidth{0.000000pt}%
\definecolor{currentstroke}{rgb}{0.227802,0.326594,0.546532}%
\pgfsetstrokecolor{currentstroke}%
\pgfsetdash{}{0pt}%
\pgfpathmoveto{\pgfqpoint{1.965641in}{3.935827in}}%
\pgfpathlineto{\pgfqpoint{1.839990in}{3.669004in}}%
\pgfpathlineto{\pgfqpoint{1.756965in}{3.430597in}}%
\pgfpathclose%
\pgfusepath{fill}%
\end{pgfscope}%
\begin{pgfscope}%
\pgfpathrectangle{\pgfqpoint{0.539299in}{0.078740in}}{\pgfqpoint{7.842520in}{7.842520in}}%
\pgfusepath{clip}%
\pgfsetbuttcap%
\pgfsetroundjoin%
\definecolor{currentfill}{rgb}{0.535621,0.835785,0.281908}%
\pgfsetfillcolor{currentfill}%
\pgfsetlinewidth{0.000000pt}%
\definecolor{currentstroke}{rgb}{0.225863,0.330805,0.547314}%
\pgfsetstrokecolor{currentstroke}%
\pgfsetdash{}{0pt}%
\pgfpathmoveto{\pgfqpoint{2.694035in}{5.204781in}}%
\pgfpathlineto{\pgfqpoint{2.606940in}{5.146364in}}%
\pgfpathlineto{\pgfqpoint{2.738785in}{5.362101in}}%
\pgfpathclose%
\pgfusepath{fill}%
\end{pgfscope}%
\begin{pgfscope}%
\pgfpathrectangle{\pgfqpoint{0.539299in}{0.078740in}}{\pgfqpoint{7.842520in}{7.842520in}}%
\pgfusepath{clip}%
\pgfsetbuttcap%
\pgfsetroundjoin%
\definecolor{currentfill}{rgb}{0.123444,0.636809,0.528763}%
\pgfsetfillcolor{currentfill}%
\pgfsetlinewidth{0.000000pt}%
\definecolor{currentstroke}{rgb}{0.223925,0.334994,0.548053}%
\pgfsetstrokecolor{currentstroke}%
\pgfsetdash{}{0pt}%
\pgfpathmoveto{\pgfqpoint{2.133656in}{4.370779in}}%
\pgfpathlineto{\pgfqpoint{2.009371in}{4.040892in}}%
\pgfpathlineto{\pgfqpoint{2.049048in}{4.173228in}}%
\pgfpathclose%
\pgfusepath{fill}%
\end{pgfscope}%
\begin{pgfscope}%
\pgfpathrectangle{\pgfqpoint{0.539299in}{0.078740in}}{\pgfqpoint{7.842520in}{7.842520in}}%
\pgfusepath{clip}%
\pgfsetbuttcap%
\pgfsetroundjoin%
\definecolor{currentfill}{rgb}{0.255645,0.260703,0.528312}%
\pgfsetfillcolor{currentfill}%
\pgfsetlinewidth{0.000000pt}%
\definecolor{currentstroke}{rgb}{0.221989,0.339161,0.548752}%
\pgfsetstrokecolor{currentstroke}%
\pgfsetdash{}{0pt}%
\pgfpathmoveto{\pgfqpoint{5.737117in}{2.526203in}}%
\pgfpathlineto{\pgfqpoint{5.814052in}{2.532529in}}%
\pgfpathlineto{\pgfqpoint{5.673051in}{2.697726in}}%
\pgfpathclose%
\pgfusepath{fill}%
\end{pgfscope}%
\begin{pgfscope}%
\pgfpathrectangle{\pgfqpoint{0.539299in}{0.078740in}}{\pgfqpoint{7.842520in}{7.842520in}}%
\pgfusepath{clip}%
\pgfsetbuttcap%
\pgfsetroundjoin%
\definecolor{currentfill}{rgb}{0.265145,0.232956,0.516599}%
\pgfsetfillcolor{currentfill}%
\pgfsetlinewidth{0.000000pt}%
\definecolor{currentstroke}{rgb}{0.220057,0.343307,0.549413}%
\pgfsetstrokecolor{currentstroke}%
\pgfsetdash{}{0pt}%
\pgfpathmoveto{\pgfqpoint{5.737117in}{2.526203in}}%
\pgfpathlineto{\pgfqpoint{5.955205in}{2.375680in}}%
\pgfpathlineto{\pgfqpoint{5.814052in}{2.532529in}}%
\pgfpathclose%
\pgfusepath{fill}%
\end{pgfscope}%
\begin{pgfscope}%
\pgfpathrectangle{\pgfqpoint{0.539299in}{0.078740in}}{\pgfqpoint{7.842520in}{7.842520in}}%
\pgfusepath{clip}%
\pgfsetbuttcap%
\pgfsetroundjoin%
\definecolor{currentfill}{rgb}{0.751884,0.874951,0.143228}%
\pgfsetfillcolor{currentfill}%
\pgfsetlinewidth{0.000000pt}%
\definecolor{currentstroke}{rgb}{0.218130,0.347432,0.550038}%
\pgfsetstrokecolor{currentstroke}%
\pgfsetdash{}{0pt}%
\pgfpathmoveto{\pgfqpoint{3.048388in}{5.568482in}}%
\pgfpathlineto{\pgfqpoint{3.272857in}{5.610573in}}%
\pgfpathlineto{\pgfqpoint{3.135579in}{5.562556in}}%
\pgfpathclose%
\pgfusepath{fill}%
\end{pgfscope}%
\begin{pgfscope}%
\pgfpathrectangle{\pgfqpoint{0.539299in}{0.078740in}}{\pgfqpoint{7.842520in}{7.842520in}}%
\pgfusepath{clip}%
\pgfsetbuttcap%
\pgfsetroundjoin%
\definecolor{currentfill}{rgb}{0.277134,0.185228,0.489898}%
\pgfsetfillcolor{currentfill}%
\pgfsetlinewidth{0.000000pt}%
\definecolor{currentstroke}{rgb}{0.216210,0.351535,0.550627}%
\pgfsetstrokecolor{currentstroke}%
\pgfsetdash{}{0pt}%
\pgfpathmoveto{\pgfqpoint{6.096558in}{2.227023in}}%
\pgfpathlineto{\pgfqpoint{5.955205in}{2.375680in}}%
\pgfpathlineto{\pgfqpoint{5.878547in}{2.354530in}}%
\pgfpathclose%
\pgfusepath{fill}%
\end{pgfscope}%
\begin{pgfscope}%
\pgfpathrectangle{\pgfqpoint{0.539299in}{0.078740in}}{\pgfqpoint{7.842520in}{7.842520in}}%
\pgfusepath{clip}%
\pgfsetbuttcap%
\pgfsetroundjoin%
\definecolor{currentfill}{rgb}{0.730889,0.871916,0.156029}%
\pgfsetfillcolor{currentfill}%
\pgfsetlinewidth{0.000000pt}%
\definecolor{currentstroke}{rgb}{0.214298,0.355619,0.551184}%
\pgfsetstrokecolor{currentstroke}%
\pgfsetdash{}{0pt}%
\pgfpathmoveto{\pgfqpoint{3.638229in}{5.473510in}}%
\pgfpathlineto{\pgfqpoint{3.498366in}{5.547173in}}%
\pgfpathlineto{\pgfqpoint{3.411816in}{5.600141in}}%
\pgfpathclose%
\pgfusepath{fill}%
\end{pgfscope}%
\begin{pgfscope}%
\pgfpathrectangle{\pgfqpoint{0.539299in}{0.078740in}}{\pgfqpoint{7.842520in}{7.842520in}}%
\pgfusepath{clip}%
\pgfsetbuttcap%
\pgfsetroundjoin%
\definecolor{currentfill}{rgb}{0.235526,0.309527,0.542944}%
\pgfsetfillcolor{currentfill}%
\pgfsetlinewidth{0.000000pt}%
\definecolor{currentstroke}{rgb}{0.212395,0.359683,0.551710}%
\pgfsetstrokecolor{currentstroke}%
\pgfsetdash{}{0pt}%
\pgfpathmoveto{\pgfqpoint{5.532155in}{2.871486in}}%
\pgfpathlineto{\pgfqpoint{5.595772in}{2.706138in}}%
\pgfpathlineto{\pgfqpoint{5.673051in}{2.697726in}}%
\pgfpathclose%
\pgfusepath{fill}%
\end{pgfscope}%
\begin{pgfscope}%
\pgfpathrectangle{\pgfqpoint{0.539299in}{0.078740in}}{\pgfqpoint{7.842520in}{7.842520in}}%
\pgfusepath{clip}%
\pgfsetbuttcap%
\pgfsetroundjoin%
\definecolor{currentfill}{rgb}{0.121148,0.592739,0.544641}%
\pgfsetfillcolor{currentfill}%
\pgfsetlinewidth{0.000000pt}%
\definecolor{currentstroke}{rgb}{0.210503,0.363727,0.552206}%
\pgfsetstrokecolor{currentstroke}%
\pgfsetdash{}{0pt}%
\pgfpathmoveto{\pgfqpoint{4.827480in}{3.864804in}}%
\pgfpathlineto{\pgfqpoint{4.766857in}{3.997620in}}%
\pgfpathlineto{\pgfqpoint{4.686160in}{4.082636in}}%
\pgfpathclose%
\pgfusepath{fill}%
\end{pgfscope}%
\begin{pgfscope}%
\pgfpathrectangle{\pgfqpoint{0.539299in}{0.078740in}}{\pgfqpoint{7.842520in}{7.842520in}}%
\pgfusepath{clip}%
\pgfsetbuttcap%
\pgfsetroundjoin%
\definecolor{currentfill}{rgb}{0.688944,0.865448,0.182725}%
\pgfsetfillcolor{currentfill}%
\pgfsetlinewidth{0.000000pt}%
\definecolor{currentstroke}{rgb}{0.208623,0.367752,0.552675}%
\pgfsetstrokecolor{currentstroke}%
\pgfsetdash{}{0pt}%
\pgfpathmoveto{\pgfqpoint{3.048388in}{5.568482in}}%
\pgfpathlineto{\pgfqpoint{2.913349in}{5.441460in}}%
\pgfpathlineto{\pgfqpoint{2.826074in}{5.413789in}}%
\pgfpathclose%
\pgfusepath{fill}%
\end{pgfscope}%
\begin{pgfscope}%
\pgfpathrectangle{\pgfqpoint{0.539299in}{0.078740in}}{\pgfqpoint{7.842520in}{7.842520in}}%
\pgfusepath{clip}%
\pgfsetbuttcap%
\pgfsetroundjoin%
\definecolor{currentfill}{rgb}{0.283187,0.125848,0.444960}%
\pgfsetfillcolor{currentfill}%
\pgfsetlinewidth{0.000000pt}%
\definecolor{currentstroke}{rgb}{0.206756,0.371758,0.553117}%
\pgfsetstrokecolor{currentstroke}%
\pgfsetdash{}{0pt}%
\pgfpathmoveto{\pgfqpoint{6.161891in}{2.037228in}}%
\pgfpathlineto{\pgfqpoint{6.238160in}{2.086590in}}%
\pgfpathlineto{\pgfqpoint{6.096558in}{2.227023in}}%
\pgfpathclose%
\pgfusepath{fill}%
\end{pgfscope}%
\begin{pgfscope}%
\pgfpathrectangle{\pgfqpoint{0.539299in}{0.078740in}}{\pgfqpoint{7.842520in}{7.842520in}}%
\pgfusepath{clip}%
\pgfsetbuttcap%
\pgfsetroundjoin%
\definecolor{currentfill}{rgb}{0.252194,0.269783,0.531579}%
\pgfsetfillcolor{currentfill}%
\pgfsetlinewidth{0.000000pt}%
\definecolor{currentstroke}{rgb}{0.204903,0.375746,0.553533}%
\pgfsetstrokecolor{currentstroke}%
\pgfsetdash{}{0pt}%
\pgfpathmoveto{\pgfqpoint{1.726778in}{2.916957in}}%
\pgfpathlineto{\pgfqpoint{1.652813in}{2.445145in}}%
\pgfpathlineto{\pgfqpoint{1.786943in}{2.460503in}}%
\pgfpathclose%
\pgfusepath{fill}%
\end{pgfscope}%
\begin{pgfscope}%
\pgfpathrectangle{\pgfqpoint{0.539299in}{0.078740in}}{\pgfqpoint{7.842520in}{7.842520in}}%
\pgfusepath{clip}%
\pgfsetbuttcap%
\pgfsetroundjoin%
\definecolor{currentfill}{rgb}{0.762373,0.876424,0.137064}%
\pgfsetfillcolor{currentfill}%
\pgfsetlinewidth{0.000000pt}%
\definecolor{currentstroke}{rgb}{0.203063,0.379716,0.553925}%
\pgfsetstrokecolor{currentstroke}%
\pgfsetdash{}{0pt}%
\pgfpathmoveto{\pgfqpoint{3.498366in}{5.547173in}}%
\pgfpathlineto{\pgfqpoint{3.272857in}{5.610573in}}%
\pgfpathlineto{\pgfqpoint{3.411816in}{5.600141in}}%
\pgfpathclose%
\pgfusepath{fill}%
\end{pgfscope}%
\begin{pgfscope}%
\pgfpathrectangle{\pgfqpoint{0.539299in}{0.078740in}}{\pgfqpoint{7.842520in}{7.842520in}}%
\pgfusepath{clip}%
\pgfsetbuttcap%
\pgfsetroundjoin%
\definecolor{currentfill}{rgb}{0.206756,0.371758,0.553117}%
\pgfsetfillcolor{currentfill}%
\pgfsetlinewidth{0.000000pt}%
\definecolor{currentstroke}{rgb}{0.201239,0.383670,0.554294}%
\pgfsetstrokecolor{currentstroke}%
\pgfsetdash{}{0pt}%
\pgfpathmoveto{\pgfqpoint{5.532155in}{2.871486in}}%
\pgfpathlineto{\pgfqpoint{5.391316in}{3.053947in}}%
\pgfpathlineto{\pgfqpoint{5.313135in}{3.090200in}}%
\pgfpathclose%
\pgfusepath{fill}%
\end{pgfscope}%
\begin{pgfscope}%
\pgfpathrectangle{\pgfqpoint{0.539299in}{0.078740in}}{\pgfqpoint{7.842520in}{7.842520in}}%
\pgfusepath{clip}%
\pgfsetbuttcap%
\pgfsetroundjoin%
\definecolor{currentfill}{rgb}{0.606045,0.850733,0.236712}%
\pgfsetfillcolor{currentfill}%
\pgfsetlinewidth{0.000000pt}%
\definecolor{currentstroke}{rgb}{0.199430,0.387607,0.554642}%
\pgfsetstrokecolor{currentstroke}%
\pgfsetdash{}{0pt}%
\pgfpathmoveto{\pgfqpoint{2.738785in}{5.362101in}}%
\pgfpathlineto{\pgfqpoint{2.826074in}{5.413789in}}%
\pgfpathlineto{\pgfqpoint{2.694035in}{5.204781in}}%
\pgfpathclose%
\pgfusepath{fill}%
\end{pgfscope}%
\begin{pgfscope}%
\pgfpathrectangle{\pgfqpoint{0.539299in}{0.078740in}}{\pgfqpoint{7.842520in}{7.842520in}}%
\pgfusepath{clip}%
\pgfsetbuttcap%
\pgfsetroundjoin%
\definecolor{currentfill}{rgb}{0.281446,0.084320,0.407414}%
\pgfsetfillcolor{currentfill}%
\pgfsetlinewidth{0.000000pt}%
\definecolor{currentstroke}{rgb}{0.197636,0.391528,0.554969}%
\pgfsetstrokecolor{currentstroke}%
\pgfsetdash{}{0pt}%
\pgfpathmoveto{\pgfqpoint{6.238160in}{2.086590in}}%
\pgfpathlineto{\pgfqpoint{6.303947in}{1.893136in}}%
\pgfpathlineto{\pgfqpoint{6.380073in}{1.954750in}}%
\pgfpathclose%
\pgfusepath{fill}%
\end{pgfscope}%
\begin{pgfscope}%
\pgfpathrectangle{\pgfqpoint{0.539299in}{0.078740in}}{\pgfqpoint{7.842520in}{7.842520in}}%
\pgfusepath{clip}%
\pgfsetbuttcap%
\pgfsetroundjoin%
\definecolor{currentfill}{rgb}{0.226397,0.728888,0.462789}%
\pgfsetfillcolor{currentfill}%
\pgfsetlinewidth{0.000000pt}%
\definecolor{currentstroke}{rgb}{0.195860,0.395433,0.555276}%
\pgfsetstrokecolor{currentstroke}%
\pgfsetdash{}{0pt}%
\pgfpathmoveto{\pgfqpoint{4.260950in}{4.734983in}}%
\pgfpathlineto{\pgfqpoint{4.485130in}{4.425835in}}%
\pgfpathlineto{\pgfqpoint{4.343978in}{4.638176in}}%
\pgfpathclose%
\pgfusepath{fill}%
\end{pgfscope}%
\begin{pgfscope}%
\pgfpathrectangle{\pgfqpoint{0.539299in}{0.078740in}}{\pgfqpoint{7.842520in}{7.842520in}}%
\pgfusepath{clip}%
\pgfsetbuttcap%
\pgfsetroundjoin%
\definecolor{currentfill}{rgb}{0.120638,0.625828,0.533488}%
\pgfsetfillcolor{currentfill}%
\pgfsetlinewidth{0.000000pt}%
\definecolor{currentstroke}{rgb}{0.194100,0.399323,0.555565}%
\pgfsetstrokecolor{currentstroke}%
\pgfsetdash{}{0pt}%
\pgfpathmoveto{\pgfqpoint{4.686160in}{4.082636in}}%
\pgfpathlineto{\pgfqpoint{4.766857in}{3.997620in}}%
\pgfpathlineto{\pgfqpoint{4.544630in}{4.302516in}}%
\pgfpathclose%
\pgfusepath{fill}%
\end{pgfscope}%
\begin{pgfscope}%
\pgfpathrectangle{\pgfqpoint{0.539299in}{0.078740in}}{\pgfqpoint{7.842520in}{7.842520in}}%
\pgfusepath{clip}%
\pgfsetbuttcap%
\pgfsetroundjoin%
\definecolor{currentfill}{rgb}{0.192357,0.403199,0.555836}%
\pgfsetfillcolor{currentfill}%
\pgfsetlinewidth{0.000000pt}%
\definecolor{currentstroke}{rgb}{0.192357,0.403199,0.555836}%
\pgfsetstrokecolor{currentstroke}%
\pgfsetdash{}{0pt}%
\pgfpathmoveto{\pgfqpoint{5.313135in}{3.090200in}}%
\pgfpathlineto{\pgfqpoint{5.391316in}{3.053947in}}%
\pgfpathlineto{\pgfqpoint{5.250481in}{3.245064in}}%
\pgfpathclose%
\pgfusepath{fill}%
\end{pgfscope}%
\begin{pgfscope}%
\pgfpathrectangle{\pgfqpoint{0.539299in}{0.078740in}}{\pgfqpoint{7.842520in}{7.842520in}}%
\pgfusepath{clip}%
\pgfsetbuttcap%
\pgfsetroundjoin%
\definecolor{currentfill}{rgb}{0.192357,0.403199,0.555836}%
\pgfsetfillcolor{currentfill}%
\pgfsetlinewidth{0.000000pt}%
\definecolor{currentstroke}{rgb}{0.190631,0.407061,0.556089}%
\pgfsetstrokecolor{currentstroke}%
\pgfsetdash{}{0pt}%
\pgfpathmoveto{\pgfqpoint{1.804002in}{3.315816in}}%
\pgfpathlineto{\pgfqpoint{1.675575in}{3.148455in}}%
\pgfpathlineto{\pgfqpoint{1.726778in}{2.916957in}}%
\pgfpathclose%
\pgfusepath{fill}%
\end{pgfscope}%
\begin{pgfscope}%
\pgfpathrectangle{\pgfqpoint{0.539299in}{0.078740in}}{\pgfqpoint{7.842520in}{7.842520in}}%
\pgfusepath{clip}%
\pgfsetbuttcap%
\pgfsetroundjoin%
\definecolor{currentfill}{rgb}{0.288921,0.758394,0.428426}%
\pgfsetfillcolor{currentfill}%
\pgfsetlinewidth{0.000000pt}%
\definecolor{currentstroke}{rgb}{0.188923,0.410910,0.556326}%
\pgfsetstrokecolor{currentstroke}%
\pgfsetdash{}{0pt}%
\pgfpathmoveto{\pgfqpoint{4.260950in}{4.734983in}}%
\pgfpathlineto{\pgfqpoint{4.343978in}{4.638176in}}%
\pgfpathlineto{\pgfqpoint{4.202671in}{4.843466in}}%
\pgfpathclose%
\pgfusepath{fill}%
\end{pgfscope}%
\begin{pgfscope}%
\pgfpathrectangle{\pgfqpoint{0.539299in}{0.078740in}}{\pgfqpoint{7.842520in}{7.842520in}}%
\pgfusepath{clip}%
\pgfsetbuttcap%
\pgfsetroundjoin%
\definecolor{currentfill}{rgb}{0.276022,0.044167,0.370164}%
\pgfsetfillcolor{currentfill}%
\pgfsetlinewidth{0.000000pt}%
\definecolor{currentstroke}{rgb}{0.187231,0.414746,0.556547}%
\pgfsetstrokecolor{currentstroke}%
\pgfsetdash{}{0pt}%
\pgfpathmoveto{\pgfqpoint{6.446395in}{1.760841in}}%
\pgfpathlineto{\pgfqpoint{6.522373in}{1.832361in}}%
\pgfpathlineto{\pgfqpoint{6.380073in}{1.954750in}}%
\pgfpathclose%
\pgfusepath{fill}%
\end{pgfscope}%
\begin{pgfscope}%
\pgfpathrectangle{\pgfqpoint{0.539299in}{0.078740in}}{\pgfqpoint{7.842520in}{7.842520in}}%
\pgfusepath{clip}%
\pgfsetbuttcap%
\pgfsetroundjoin%
\definecolor{currentfill}{rgb}{0.688944,0.865448,0.182725}%
\pgfsetfillcolor{currentfill}%
\pgfsetlinewidth{0.000000pt}%
\definecolor{currentstroke}{rgb}{0.185556,0.418570,0.556753}%
\pgfsetstrokecolor{currentstroke}%
\pgfsetdash{}{0pt}%
\pgfpathmoveto{\pgfqpoint{3.778860in}{5.358543in}}%
\pgfpathlineto{\pgfqpoint{3.638229in}{5.473510in}}%
\pgfpathlineto{\pgfqpoint{3.552031in}{5.538964in}}%
\pgfpathclose%
\pgfusepath{fill}%
\end{pgfscope}%
\begin{pgfscope}%
\pgfpathrectangle{\pgfqpoint{0.539299in}{0.078740in}}{\pgfqpoint{7.842520in}{7.842520in}}%
\pgfusepath{clip}%
\pgfsetbuttcap%
\pgfsetroundjoin%
\definecolor{currentfill}{rgb}{0.267968,0.223549,0.512008}%
\pgfsetfillcolor{currentfill}%
\pgfsetlinewidth{0.000000pt}%
\definecolor{currentstroke}{rgb}{0.183898,0.422383,0.556944}%
\pgfsetstrokecolor{currentstroke}%
\pgfsetdash{}{0pt}%
\pgfpathmoveto{\pgfqpoint{5.878547in}{2.354530in}}%
\pgfpathlineto{\pgfqpoint{5.955205in}{2.375680in}}%
\pgfpathlineto{\pgfqpoint{5.737117in}{2.526203in}}%
\pgfpathclose%
\pgfusepath{fill}%
\end{pgfscope}%
\begin{pgfscope}%
\pgfpathrectangle{\pgfqpoint{0.539299in}{0.078740in}}{\pgfqpoint{7.842520in}{7.842520in}}%
\pgfusepath{clip}%
\pgfsetbuttcap%
\pgfsetroundjoin%
\definecolor{currentfill}{rgb}{0.278826,0.175490,0.483397}%
\pgfsetfillcolor{currentfill}%
\pgfsetlinewidth{0.000000pt}%
\definecolor{currentstroke}{rgb}{0.182256,0.426184,0.557120}%
\pgfsetstrokecolor{currentstroke}%
\pgfsetdash{}{0pt}%
\pgfpathmoveto{\pgfqpoint{6.096558in}{2.227023in}}%
\pgfpathlineto{\pgfqpoint{5.878547in}{2.354530in}}%
\pgfpathlineto{\pgfqpoint{6.020118in}{2.191364in}}%
\pgfpathclose%
\pgfusepath{fill}%
\end{pgfscope}%
\begin{pgfscope}%
\pgfpathrectangle{\pgfqpoint{0.539299in}{0.078740in}}{\pgfqpoint{7.842520in}{7.842520in}}%
\pgfusepath{clip}%
\pgfsetbuttcap%
\pgfsetroundjoin%
\definecolor{currentfill}{rgb}{0.282290,0.145912,0.461510}%
\pgfsetfillcolor{currentfill}%
\pgfsetlinewidth{0.000000pt}%
\definecolor{currentstroke}{rgb}{0.180629,0.429975,0.557282}%
\pgfsetstrokecolor{currentstroke}%
\pgfsetdash{}{0pt}%
\pgfpathmoveto{\pgfqpoint{6.020118in}{2.191364in}}%
\pgfpathlineto{\pgfqpoint{6.161891in}{2.037228in}}%
\pgfpathlineto{\pgfqpoint{6.096558in}{2.227023in}}%
\pgfpathclose%
\pgfusepath{fill}%
\end{pgfscope}%
\begin{pgfscope}%
\pgfpathrectangle{\pgfqpoint{0.539299in}{0.078740in}}{\pgfqpoint{7.842520in}{7.842520in}}%
\pgfusepath{clip}%
\pgfsetbuttcap%
\pgfsetroundjoin%
\definecolor{currentfill}{rgb}{0.244972,0.287675,0.537260}%
\pgfsetfillcolor{currentfill}%
\pgfsetlinewidth{0.000000pt}%
\definecolor{currentstroke}{rgb}{0.179019,0.433756,0.557430}%
\pgfsetstrokecolor{currentstroke}%
\pgfsetdash{}{0pt}%
\pgfpathmoveto{\pgfqpoint{5.673051in}{2.697726in}}%
\pgfpathlineto{\pgfqpoint{5.595772in}{2.706138in}}%
\pgfpathlineto{\pgfqpoint{5.737117in}{2.526203in}}%
\pgfpathclose%
\pgfusepath{fill}%
\end{pgfscope}%
\begin{pgfscope}%
\pgfpathrectangle{\pgfqpoint{0.539299in}{0.078740in}}{\pgfqpoint{7.842520in}{7.842520in}}%
\pgfusepath{clip}%
\pgfsetbuttcap%
\pgfsetroundjoin%
\definecolor{currentfill}{rgb}{0.122606,0.585371,0.546557}%
\pgfsetfillcolor{currentfill}%
\pgfsetlinewidth{0.000000pt}%
\definecolor{currentstroke}{rgb}{0.177423,0.437527,0.557565}%
\pgfsetstrokecolor{currentstroke}%
\pgfsetdash{}{0pt}%
\pgfpathmoveto{\pgfqpoint{1.965641in}{3.935827in}}%
\pgfpathlineto{\pgfqpoint{2.049048in}{4.173228in}}%
\pgfpathlineto{\pgfqpoint{1.839990in}{3.669004in}}%
\pgfpathclose%
\pgfusepath{fill}%
\end{pgfscope}%
\begin{pgfscope}%
\pgfpathrectangle{\pgfqpoint{0.539299in}{0.078740in}}{\pgfqpoint{7.842520in}{7.842520in}}%
\pgfusepath{clip}%
\pgfsetbuttcap%
\pgfsetroundjoin%
\definecolor{currentfill}{rgb}{0.165117,0.467423,0.558141}%
\pgfsetfillcolor{currentfill}%
\pgfsetlinewidth{0.000000pt}%
\definecolor{currentstroke}{rgb}{0.175841,0.441290,0.557685}%
\pgfsetstrokecolor{currentstroke}%
\pgfsetdash{}{0pt}%
\pgfpathmoveto{\pgfqpoint{5.250481in}{3.245064in}}%
\pgfpathlineto{\pgfqpoint{5.109597in}{3.444492in}}%
\pgfpathlineto{\pgfqpoint{5.030239in}{3.504331in}}%
\pgfpathclose%
\pgfusepath{fill}%
\end{pgfscope}%
\begin{pgfscope}%
\pgfpathrectangle{\pgfqpoint{0.539299in}{0.078740in}}{\pgfqpoint{7.842520in}{7.842520in}}%
\pgfusepath{clip}%
\pgfsetbuttcap%
\pgfsetroundjoin%
\definecolor{currentfill}{rgb}{0.369214,0.788888,0.382914}%
\pgfsetfillcolor{currentfill}%
\pgfsetlinewidth{0.000000pt}%
\definecolor{currentstroke}{rgb}{0.174274,0.445044,0.557792}%
\pgfsetstrokecolor{currentstroke}%
\pgfsetdash{}{0pt}%
\pgfpathmoveto{\pgfqpoint{2.305286in}{4.666266in}}%
\pgfpathlineto{\pgfqpoint{2.433328in}{4.953363in}}%
\pgfpathlineto{\pgfqpoint{2.519975in}{5.063797in}}%
\pgfpathclose%
\pgfusepath{fill}%
\end{pgfscope}%
\begin{pgfscope}%
\pgfpathrectangle{\pgfqpoint{0.539299in}{0.078740in}}{\pgfqpoint{7.842520in}{7.842520in}}%
\pgfusepath{clip}%
\pgfsetbuttcap%
\pgfsetroundjoin%
\definecolor{currentfill}{rgb}{0.259857,0.745492,0.444467}%
\pgfsetfillcolor{currentfill}%
\pgfsetlinewidth{0.000000pt}%
\definecolor{currentstroke}{rgb}{0.172719,0.448791,0.557885}%
\pgfsetstrokecolor{currentstroke}%
\pgfsetdash{}{0pt}%
\pgfpathmoveto{\pgfqpoint{2.305286in}{4.666266in}}%
\pgfpathlineto{\pgfqpoint{2.219156in}{4.533644in}}%
\pgfpathlineto{\pgfqpoint{2.347219in}{4.810962in}}%
\pgfpathclose%
\pgfusepath{fill}%
\end{pgfscope}%
\begin{pgfscope}%
\pgfpathrectangle{\pgfqpoint{0.539299in}{0.078740in}}{\pgfqpoint{7.842520in}{7.842520in}}%
\pgfusepath{clip}%
\pgfsetbuttcap%
\pgfsetroundjoin%
\definecolor{currentfill}{rgb}{0.386433,0.794644,0.372886}%
\pgfsetfillcolor{currentfill}%
\pgfsetlinewidth{0.000000pt}%
\definecolor{currentstroke}{rgb}{0.171176,0.452530,0.557965}%
\pgfsetstrokecolor{currentstroke}%
\pgfsetdash{}{0pt}%
\pgfpathmoveto{\pgfqpoint{4.061290in}{5.036247in}}%
\pgfpathlineto{\pgfqpoint{4.118884in}{4.938971in}}%
\pgfpathlineto{\pgfqpoint{4.202671in}{4.843466in}}%
\pgfpathclose%
\pgfusepath{fill}%
\end{pgfscope}%
\begin{pgfscope}%
\pgfpathrectangle{\pgfqpoint{0.539299in}{0.078740in}}{\pgfqpoint{7.842520in}{7.842520in}}%
\pgfusepath{clip}%
\pgfsetbuttcap%
\pgfsetroundjoin%
\definecolor{currentfill}{rgb}{0.282910,0.105393,0.426902}%
\pgfsetfillcolor{currentfill}%
\pgfsetlinewidth{0.000000pt}%
\definecolor{currentstroke}{rgb}{0.169646,0.456262,0.558030}%
\pgfsetstrokecolor{currentstroke}%
\pgfsetdash{}{0pt}%
\pgfpathmoveto{\pgfqpoint{6.238160in}{2.086590in}}%
\pgfpathlineto{\pgfqpoint{6.161891in}{2.037228in}}%
\pgfpathlineto{\pgfqpoint{6.303947in}{1.893136in}}%
\pgfpathclose%
\pgfusepath{fill}%
\end{pgfscope}%
\begin{pgfscope}%
\pgfpathrectangle{\pgfqpoint{0.539299in}{0.078740in}}{\pgfqpoint{7.842520in}{7.842520in}}%
\pgfusepath{clip}%
\pgfsetbuttcap%
\pgfsetroundjoin%
\definecolor{currentfill}{rgb}{0.223925,0.334994,0.548053}%
\pgfsetfillcolor{currentfill}%
\pgfsetlinewidth{0.000000pt}%
\definecolor{currentstroke}{rgb}{0.168126,0.459988,0.558082}%
\pgfsetstrokecolor{currentstroke}%
\pgfsetdash{}{0pt}%
\pgfpathmoveto{\pgfqpoint{5.532155in}{2.871486in}}%
\pgfpathlineto{\pgfqpoint{5.454461in}{2.894198in}}%
\pgfpathlineto{\pgfqpoint{5.595772in}{2.706138in}}%
\pgfpathclose%
\pgfusepath{fill}%
\end{pgfscope}%
\begin{pgfscope}%
\pgfpathrectangle{\pgfqpoint{0.539299in}{0.078740in}}{\pgfqpoint{7.842520in}{7.842520in}}%
\pgfusepath{clip}%
\pgfsetbuttcap%
\pgfsetroundjoin%
\definecolor{currentfill}{rgb}{0.157729,0.485932,0.558013}%
\pgfsetfillcolor{currentfill}%
\pgfsetlinewidth{0.000000pt}%
\definecolor{currentstroke}{rgb}{0.166617,0.463708,0.558119}%
\pgfsetstrokecolor{currentstroke}%
\pgfsetdash{}{0pt}%
\pgfpathmoveto{\pgfqpoint{1.804002in}{3.315816in}}%
\pgfpathlineto{\pgfqpoint{1.883808in}{3.652469in}}%
\pgfpathlineto{\pgfqpoint{1.756965in}{3.430597in}}%
\pgfpathclose%
\pgfusepath{fill}%
\end{pgfscope}%
\begin{pgfscope}%
\pgfpathrectangle{\pgfqpoint{0.539299in}{0.078740in}}{\pgfqpoint{7.842520in}{7.842520in}}%
\pgfusepath{clip}%
\pgfsetbuttcap%
\pgfsetroundjoin%
\definecolor{currentfill}{rgb}{0.170948,0.694384,0.493803}%
\pgfsetfillcolor{currentfill}%
\pgfsetlinewidth{0.000000pt}%
\definecolor{currentstroke}{rgb}{0.165117,0.467423,0.558141}%
\pgfsetstrokecolor{currentstroke}%
\pgfsetdash{}{0pt}%
\pgfpathmoveto{\pgfqpoint{4.544630in}{4.302516in}}%
\pgfpathlineto{\pgfqpoint{4.485130in}{4.425835in}}%
\pgfpathlineto{\pgfqpoint{4.402885in}{4.521271in}}%
\pgfpathclose%
\pgfusepath{fill}%
\end{pgfscope}%
\begin{pgfscope}%
\pgfpathrectangle{\pgfqpoint{0.539299in}{0.078740in}}{\pgfqpoint{7.842520in}{7.842520in}}%
\pgfusepath{clip}%
\pgfsetbuttcap%
\pgfsetroundjoin%
\definecolor{currentfill}{rgb}{0.151918,0.500685,0.557587}%
\pgfsetfillcolor{currentfill}%
\pgfsetlinewidth{0.000000pt}%
\definecolor{currentstroke}{rgb}{0.163625,0.471133,0.558148}%
\pgfsetstrokecolor{currentstroke}%
\pgfsetdash{}{0pt}%
\pgfpathmoveto{\pgfqpoint{5.030239in}{3.504331in}}%
\pgfpathlineto{\pgfqpoint{5.109597in}{3.444492in}}%
\pgfpathlineto{\pgfqpoint{4.968612in}{3.651484in}}%
\pgfpathclose%
\pgfusepath{fill}%
\end{pgfscope}%
\begin{pgfscope}%
\pgfpathrectangle{\pgfqpoint{0.539299in}{0.078740in}}{\pgfqpoint{7.842520in}{7.842520in}}%
\pgfusepath{clip}%
\pgfsetbuttcap%
\pgfsetroundjoin%
\definecolor{currentfill}{rgb}{0.535621,0.835785,0.281908}%
\pgfsetfillcolor{currentfill}%
\pgfsetlinewidth{0.000000pt}%
\definecolor{currentstroke}{rgb}{0.162142,0.474838,0.558140}%
\pgfsetstrokecolor{currentstroke}%
\pgfsetdash{}{0pt}%
\pgfpathmoveto{\pgfqpoint{2.738785in}{5.362101in}}%
\pgfpathlineto{\pgfqpoint{2.606940in}{5.146364in}}%
\pgfpathlineto{\pgfqpoint{2.519975in}{5.063797in}}%
\pgfpathclose%
\pgfusepath{fill}%
\end{pgfscope}%
\begin{pgfscope}%
\pgfpathrectangle{\pgfqpoint{0.539299in}{0.078740in}}{\pgfqpoint{7.842520in}{7.842520in}}%
\pgfusepath{clip}%
\pgfsetbuttcap%
\pgfsetroundjoin%
\definecolor{currentfill}{rgb}{0.595839,0.848717,0.243329}%
\pgfsetfillcolor{currentfill}%
\pgfsetlinewidth{0.000000pt}%
\definecolor{currentstroke}{rgb}{0.160665,0.478540,0.558115}%
\pgfsetstrokecolor{currentstroke}%
\pgfsetdash{}{0pt}%
\pgfpathmoveto{\pgfqpoint{3.834801in}{5.295343in}}%
\pgfpathlineto{\pgfqpoint{3.919960in}{5.210264in}}%
\pgfpathlineto{\pgfqpoint{3.778860in}{5.358543in}}%
\pgfpathclose%
\pgfusepath{fill}%
\end{pgfscope}%
\begin{pgfscope}%
\pgfpathrectangle{\pgfqpoint{0.539299in}{0.078740in}}{\pgfqpoint{7.842520in}{7.842520in}}%
\pgfusepath{clip}%
\pgfsetbuttcap%
\pgfsetroundjoin%
\definecolor{currentfill}{rgb}{0.525776,0.833491,0.288127}%
\pgfsetfillcolor{currentfill}%
\pgfsetlinewidth{0.000000pt}%
\definecolor{currentstroke}{rgb}{0.159194,0.482237,0.558073}%
\pgfsetstrokecolor{currentstroke}%
\pgfsetdash{}{0pt}%
\pgfpathmoveto{\pgfqpoint{4.061290in}{5.036247in}}%
\pgfpathlineto{\pgfqpoint{3.919960in}{5.210264in}}%
\pgfpathlineto{\pgfqpoint{3.834801in}{5.295343in}}%
\pgfpathclose%
\pgfusepath{fill}%
\end{pgfscope}%
\begin{pgfscope}%
\pgfpathrectangle{\pgfqpoint{0.539299in}{0.078740in}}{\pgfqpoint{7.842520in}{7.842520in}}%
\pgfusepath{clip}%
\pgfsetbuttcap%
\pgfsetroundjoin%
\definecolor{currentfill}{rgb}{0.208623,0.367752,0.552675}%
\pgfsetfillcolor{currentfill}%
\pgfsetlinewidth{0.000000pt}%
\definecolor{currentstroke}{rgb}{0.157729,0.485932,0.558013}%
\pgfsetstrokecolor{currentstroke}%
\pgfsetdash{}{0pt}%
\pgfpathmoveto{\pgfqpoint{5.313135in}{3.090200in}}%
\pgfpathlineto{\pgfqpoint{5.454461in}{2.894198in}}%
\pgfpathlineto{\pgfqpoint{5.532155in}{2.871486in}}%
\pgfpathclose%
\pgfusepath{fill}%
\end{pgfscope}%
\begin{pgfscope}%
\pgfpathrectangle{\pgfqpoint{0.539299in}{0.078740in}}{\pgfqpoint{7.842520in}{7.842520in}}%
\pgfusepath{clip}%
\pgfsetbuttcap%
\pgfsetroundjoin%
\definecolor{currentfill}{rgb}{0.139147,0.533812,0.555298}%
\pgfsetfillcolor{currentfill}%
\pgfsetlinewidth{0.000000pt}%
\definecolor{currentstroke}{rgb}{0.156270,0.489624,0.557936}%
\pgfsetstrokecolor{currentstroke}%
\pgfsetdash{}{0pt}%
\pgfpathmoveto{\pgfqpoint{1.756965in}{3.430597in}}%
\pgfpathlineto{\pgfqpoint{1.883808in}{3.652469in}}%
\pgfpathlineto{\pgfqpoint{1.965641in}{3.935827in}}%
\pgfpathclose%
\pgfusepath{fill}%
\end{pgfscope}%
\begin{pgfscope}%
\pgfpathrectangle{\pgfqpoint{0.539299in}{0.078740in}}{\pgfqpoint{7.842520in}{7.842520in}}%
\pgfusepath{clip}%
\pgfsetbuttcap%
\pgfsetroundjoin%
\definecolor{currentfill}{rgb}{0.278791,0.062145,0.386592}%
\pgfsetfillcolor{currentfill}%
\pgfsetlinewidth{0.000000pt}%
\definecolor{currentstroke}{rgb}{0.154815,0.493313,0.557840}%
\pgfsetstrokecolor{currentstroke}%
\pgfsetdash{}{0pt}%
\pgfpathmoveto{\pgfqpoint{6.380073in}{1.954750in}}%
\pgfpathlineto{\pgfqpoint{6.303947in}{1.893136in}}%
\pgfpathlineto{\pgfqpoint{6.446395in}{1.760841in}}%
\pgfpathclose%
\pgfusepath{fill}%
\end{pgfscope}%
\begin{pgfscope}%
\pgfpathrectangle{\pgfqpoint{0.539299in}{0.078740in}}{\pgfqpoint{7.842520in}{7.842520in}}%
\pgfusepath{clip}%
\pgfsetbuttcap%
\pgfsetroundjoin%
\definecolor{currentfill}{rgb}{0.751884,0.874951,0.143228}%
\pgfsetfillcolor{currentfill}%
\pgfsetlinewidth{0.000000pt}%
\definecolor{currentstroke}{rgb}{0.153364,0.497000,0.557724}%
\pgfsetstrokecolor{currentstroke}%
\pgfsetdash{}{0pt}%
\pgfpathmoveto{\pgfqpoint{3.411816in}{5.600141in}}%
\pgfpathlineto{\pgfqpoint{3.552031in}{5.538964in}}%
\pgfpathlineto{\pgfqpoint{3.638229in}{5.473510in}}%
\pgfpathclose%
\pgfusepath{fill}%
\end{pgfscope}%
\begin{pgfscope}%
\pgfpathrectangle{\pgfqpoint{0.539299in}{0.078740in}}{\pgfqpoint{7.842520in}{7.842520in}}%
\pgfusepath{clip}%
\pgfsetbuttcap%
\pgfsetroundjoin%
\definecolor{currentfill}{rgb}{0.220124,0.725509,0.466226}%
\pgfsetfillcolor{currentfill}%
\pgfsetlinewidth{0.000000pt}%
\definecolor{currentstroke}{rgb}{0.151918,0.500685,0.557587}%
\pgfsetstrokecolor{currentstroke}%
\pgfsetdash{}{0pt}%
\pgfpathmoveto{\pgfqpoint{4.402885in}{4.521271in}}%
\pgfpathlineto{\pgfqpoint{4.485130in}{4.425835in}}%
\pgfpathlineto{\pgfqpoint{4.260950in}{4.734983in}}%
\pgfpathclose%
\pgfusepath{fill}%
\end{pgfscope}%
\begin{pgfscope}%
\pgfpathrectangle{\pgfqpoint{0.539299in}{0.078740in}}{\pgfqpoint{7.842520in}{7.842520in}}%
\pgfusepath{clip}%
\pgfsetbuttcap%
\pgfsetroundjoin%
\definecolor{currentfill}{rgb}{0.225863,0.330805,0.547314}%
\pgfsetfillcolor{currentfill}%
\pgfsetlinewidth{0.000000pt}%
\definecolor{currentstroke}{rgb}{0.150476,0.504369,0.557430}%
\pgfsetstrokecolor{currentstroke}%
\pgfsetdash{}{0pt}%
\pgfpathmoveto{\pgfqpoint{1.786943in}{2.460503in}}%
\pgfpathlineto{\pgfqpoint{1.859345in}{2.989662in}}%
\pgfpathlineto{\pgfqpoint{1.726778in}{2.916957in}}%
\pgfpathclose%
\pgfusepath{fill}%
\end{pgfscope}%
\begin{pgfscope}%
\pgfpathrectangle{\pgfqpoint{0.539299in}{0.078740in}}{\pgfqpoint{7.842520in}{7.842520in}}%
\pgfusepath{clip}%
\pgfsetbuttcap%
\pgfsetroundjoin%
\definecolor{currentfill}{rgb}{0.804182,0.882046,0.114965}%
\pgfsetfillcolor{currentfill}%
\pgfsetlinewidth{0.000000pt}%
\definecolor{currentstroke}{rgb}{0.149039,0.508051,0.557250}%
\pgfsetstrokecolor{currentstroke}%
\pgfsetdash{}{0pt}%
\pgfpathmoveto{\pgfqpoint{3.185707in}{5.629892in}}%
\pgfpathlineto{\pgfqpoint{3.272857in}{5.610573in}}%
\pgfpathlineto{\pgfqpoint{3.048388in}{5.568482in}}%
\pgfpathclose%
\pgfusepath{fill}%
\end{pgfscope}%
\begin{pgfscope}%
\pgfpathrectangle{\pgfqpoint{0.539299in}{0.078740in}}{\pgfqpoint{7.842520in}{7.842520in}}%
\pgfusepath{clip}%
\pgfsetbuttcap%
\pgfsetroundjoin%
\definecolor{currentfill}{rgb}{0.180629,0.429975,0.557282}%
\pgfsetfillcolor{currentfill}%
\pgfsetlinewidth{0.000000pt}%
\definecolor{currentstroke}{rgb}{0.147607,0.511733,0.557049}%
\pgfsetstrokecolor{currentstroke}%
\pgfsetdash{}{0pt}%
\pgfpathmoveto{\pgfqpoint{5.171743in}{3.293790in}}%
\pgfpathlineto{\pgfqpoint{5.313135in}{3.090200in}}%
\pgfpathlineto{\pgfqpoint{5.250481in}{3.245064in}}%
\pgfpathclose%
\pgfusepath{fill}%
\end{pgfscope}%
\begin{pgfscope}%
\pgfpathrectangle{\pgfqpoint{0.539299in}{0.078740in}}{\pgfqpoint{7.842520in}{7.842520in}}%
\pgfusepath{clip}%
\pgfsetbuttcap%
\pgfsetroundjoin%
\definecolor{currentfill}{rgb}{0.751884,0.874951,0.143228}%
\pgfsetfillcolor{currentfill}%
\pgfsetlinewidth{0.000000pt}%
\definecolor{currentstroke}{rgb}{0.146180,0.515413,0.556823}%
\pgfsetstrokecolor{currentstroke}%
\pgfsetdash{}{0pt}%
\pgfpathmoveto{\pgfqpoint{2.826074in}{5.413789in}}%
\pgfpathlineto{\pgfqpoint{2.961006in}{5.551827in}}%
\pgfpathlineto{\pgfqpoint{3.048388in}{5.568482in}}%
\pgfpathclose%
\pgfusepath{fill}%
\end{pgfscope}%
\begin{pgfscope}%
\pgfpathrectangle{\pgfqpoint{0.539299in}{0.078740in}}{\pgfqpoint{7.842520in}{7.842520in}}%
\pgfusepath{clip}%
\pgfsetbuttcap%
\pgfsetroundjoin%
\definecolor{currentfill}{rgb}{0.232815,0.732247,0.459277}%
\pgfsetfillcolor{currentfill}%
\pgfsetlinewidth{0.000000pt}%
\definecolor{currentstroke}{rgb}{0.144759,0.519093,0.556572}%
\pgfsetstrokecolor{currentstroke}%
\pgfsetdash{}{0pt}%
\pgfpathmoveto{\pgfqpoint{2.347219in}{4.810962in}}%
\pgfpathlineto{\pgfqpoint{2.219156in}{4.533644in}}%
\pgfpathlineto{\pgfqpoint{2.133656in}{4.370779in}}%
\pgfpathclose%
\pgfusepath{fill}%
\end{pgfscope}%
\begin{pgfscope}%
\pgfpathrectangle{\pgfqpoint{0.539299in}{0.078740in}}{\pgfqpoint{7.842520in}{7.842520in}}%
\pgfusepath{clip}%
\pgfsetbuttcap%
\pgfsetroundjoin%
\definecolor{currentfill}{rgb}{0.126453,0.570633,0.549841}%
\pgfsetfillcolor{currentfill}%
\pgfsetlinewidth{0.000000pt}%
\definecolor{currentstroke}{rgb}{0.143343,0.522773,0.556295}%
\pgfsetstrokecolor{currentstroke}%
\pgfsetdash{}{0pt}%
\pgfpathmoveto{\pgfqpoint{4.968612in}{3.651484in}}%
\pgfpathlineto{\pgfqpoint{4.827480in}{3.864804in}}%
\pgfpathlineto{\pgfqpoint{4.746722in}{3.941721in}}%
\pgfpathclose%
\pgfusepath{fill}%
\end{pgfscope}%
\begin{pgfscope}%
\pgfpathrectangle{\pgfqpoint{0.539299in}{0.078740in}}{\pgfqpoint{7.842520in}{7.842520in}}%
\pgfusepath{clip}%
\pgfsetbuttcap%
\pgfsetroundjoin%
\definecolor{currentfill}{rgb}{0.166617,0.463708,0.558119}%
\pgfsetfillcolor{currentfill}%
\pgfsetlinewidth{0.000000pt}%
\definecolor{currentstroke}{rgb}{0.141935,0.526453,0.555991}%
\pgfsetstrokecolor{currentstroke}%
\pgfsetdash{}{0pt}%
\pgfpathmoveto{\pgfqpoint{5.250481in}{3.245064in}}%
\pgfpathlineto{\pgfqpoint{5.030239in}{3.504331in}}%
\pgfpathlineto{\pgfqpoint{5.171743in}{3.293790in}}%
\pgfpathclose%
\pgfusepath{fill}%
\end{pgfscope}%
\begin{pgfscope}%
\pgfpathrectangle{\pgfqpoint{0.539299in}{0.078740in}}{\pgfqpoint{7.842520in}{7.842520in}}%
\pgfusepath{clip}%
\pgfsetbuttcap%
\pgfsetroundjoin%
\definecolor{currentfill}{rgb}{0.344074,0.780029,0.397381}%
\pgfsetfillcolor{currentfill}%
\pgfsetlinewidth{0.000000pt}%
\definecolor{currentstroke}{rgb}{0.140536,0.530132,0.555659}%
\pgfsetstrokecolor{currentstroke}%
\pgfsetdash{}{0pt}%
\pgfpathmoveto{\pgfqpoint{2.347219in}{4.810962in}}%
\pgfpathlineto{\pgfqpoint{2.433328in}{4.953363in}}%
\pgfpathlineto{\pgfqpoint{2.305286in}{4.666266in}}%
\pgfpathclose%
\pgfusepath{fill}%
\end{pgfscope}%
\begin{pgfscope}%
\pgfpathrectangle{\pgfqpoint{0.539299in}{0.078740in}}{\pgfqpoint{7.842520in}{7.842520in}}%
\pgfusepath{clip}%
\pgfsetbuttcap%
\pgfsetroundjoin%
\definecolor{currentfill}{rgb}{0.352360,0.783011,0.392636}%
\pgfsetfillcolor{currentfill}%
\pgfsetlinewidth{0.000000pt}%
\definecolor{currentstroke}{rgb}{0.139147,0.533812,0.555298}%
\pgfsetstrokecolor{currentstroke}%
\pgfsetdash{}{0pt}%
\pgfpathmoveto{\pgfqpoint{4.202671in}{4.843466in}}%
\pgfpathlineto{\pgfqpoint{4.118884in}{4.938971in}}%
\pgfpathlineto{\pgfqpoint{4.260950in}{4.734983in}}%
\pgfpathclose%
\pgfusepath{fill}%
\end{pgfscope}%
\begin{pgfscope}%
\pgfpathrectangle{\pgfqpoint{0.539299in}{0.078740in}}{\pgfqpoint{7.842520in}{7.842520in}}%
\pgfusepath{clip}%
\pgfsetbuttcap%
\pgfsetroundjoin%
\definecolor{currentfill}{rgb}{0.709898,0.868751,0.169257}%
\pgfsetfillcolor{currentfill}%
\pgfsetlinewidth{0.000000pt}%
\definecolor{currentstroke}{rgb}{0.137770,0.537492,0.554906}%
\pgfsetstrokecolor{currentstroke}%
\pgfsetdash{}{0pt}%
\pgfpathmoveto{\pgfqpoint{3.552031in}{5.538964in}}%
\pgfpathlineto{\pgfqpoint{3.693131in}{5.434822in}}%
\pgfpathlineto{\pgfqpoint{3.778860in}{5.358543in}}%
\pgfpathclose%
\pgfusepath{fill}%
\end{pgfscope}%
\begin{pgfscope}%
\pgfpathrectangle{\pgfqpoint{0.539299in}{0.078740in}}{\pgfqpoint{7.842520in}{7.842520in}}%
\pgfusepath{clip}%
\pgfsetbuttcap%
\pgfsetroundjoin%
\definecolor{currentfill}{rgb}{0.241237,0.296485,0.539709}%
\pgfsetfillcolor{currentfill}%
\pgfsetlinewidth{0.000000pt}%
\definecolor{currentstroke}{rgb}{0.136408,0.541173,0.554483}%
\pgfsetstrokecolor{currentstroke}%
\pgfsetdash{}{0pt}%
\pgfpathmoveto{\pgfqpoint{1.786943in}{2.460503in}}%
\pgfpathlineto{\pgfqpoint{1.922231in}{2.462670in}}%
\pgfpathlineto{\pgfqpoint{1.859345in}{2.989662in}}%
\pgfpathclose%
\pgfusepath{fill}%
\end{pgfscope}%
\begin{pgfscope}%
\pgfpathrectangle{\pgfqpoint{0.539299in}{0.078740in}}{\pgfqpoint{7.842520in}{7.842520in}}%
\pgfusepath{clip}%
\pgfsetbuttcap%
\pgfsetroundjoin%
\definecolor{currentfill}{rgb}{0.468053,0.818921,0.323998}%
\pgfsetfillcolor{currentfill}%
\pgfsetlinewidth{0.000000pt}%
\definecolor{currentstroke}{rgb}{0.135066,0.544853,0.554029}%
\pgfsetstrokecolor{currentstroke}%
\pgfsetdash{}{0pt}%
\pgfpathmoveto{\pgfqpoint{4.061290in}{5.036247in}}%
\pgfpathlineto{\pgfqpoint{3.976785in}{5.127804in}}%
\pgfpathlineto{\pgfqpoint{4.118884in}{4.938971in}}%
\pgfpathclose%
\pgfusepath{fill}%
\end{pgfscope}%
\begin{pgfscope}%
\pgfpathrectangle{\pgfqpoint{0.539299in}{0.078740in}}{\pgfqpoint{7.842520in}{7.842520in}}%
\pgfusepath{clip}%
\pgfsetbuttcap%
\pgfsetroundjoin%
\definecolor{currentfill}{rgb}{0.668054,0.861999,0.196293}%
\pgfsetfillcolor{currentfill}%
\pgfsetlinewidth{0.000000pt}%
\definecolor{currentstroke}{rgb}{0.133743,0.548535,0.553541}%
\pgfsetstrokecolor{currentstroke}%
\pgfsetdash{}{0pt}%
\pgfpathmoveto{\pgfqpoint{3.778860in}{5.358543in}}%
\pgfpathlineto{\pgfqpoint{3.693131in}{5.434822in}}%
\pgfpathlineto{\pgfqpoint{3.834801in}{5.295343in}}%
\pgfpathclose%
\pgfusepath{fill}%
\end{pgfscope}%
\begin{pgfscope}%
\pgfpathrectangle{\pgfqpoint{0.539299in}{0.078740in}}{\pgfqpoint{7.842520in}{7.842520in}}%
\pgfusepath{clip}%
\pgfsetbuttcap%
\pgfsetroundjoin%
\definecolor{currentfill}{rgb}{0.824940,0.884720,0.106217}%
\pgfsetfillcolor{currentfill}%
\pgfsetlinewidth{0.000000pt}%
\definecolor{currentstroke}{rgb}{0.132444,0.552216,0.553018}%
\pgfsetstrokecolor{currentstroke}%
\pgfsetdash{}{0pt}%
\pgfpathmoveto{\pgfqpoint{3.411816in}{5.600141in}}%
\pgfpathlineto{\pgfqpoint{3.272857in}{5.610573in}}%
\pgfpathlineto{\pgfqpoint{3.324832in}{5.632651in}}%
\pgfpathclose%
\pgfusepath{fill}%
\end{pgfscope}%
\begin{pgfscope}%
\pgfpathrectangle{\pgfqpoint{0.539299in}{0.078740in}}{\pgfqpoint{7.842520in}{7.842520in}}%
\pgfusepath{clip}%
\pgfsetbuttcap%
\pgfsetroundjoin%
\definecolor{currentfill}{rgb}{0.535621,0.835785,0.281908}%
\pgfsetfillcolor{currentfill}%
\pgfsetlinewidth{0.000000pt}%
\definecolor{currentstroke}{rgb}{0.131172,0.555899,0.552459}%
\pgfsetstrokecolor{currentstroke}%
\pgfsetdash{}{0pt}%
\pgfpathmoveto{\pgfqpoint{3.834801in}{5.295343in}}%
\pgfpathlineto{\pgfqpoint{3.976785in}{5.127804in}}%
\pgfpathlineto{\pgfqpoint{4.061290in}{5.036247in}}%
\pgfpathclose%
\pgfusepath{fill}%
\end{pgfscope}%
\begin{pgfscope}%
\pgfpathrectangle{\pgfqpoint{0.539299in}{0.078740in}}{\pgfqpoint{7.842520in}{7.842520in}}%
\pgfusepath{clip}%
\pgfsetbuttcap%
\pgfsetroundjoin%
\definecolor{currentfill}{rgb}{0.120081,0.622161,0.534946}%
\pgfsetfillcolor{currentfill}%
\pgfsetlinewidth{0.000000pt}%
\definecolor{currentstroke}{rgb}{0.129933,0.559582,0.551864}%
\pgfsetstrokecolor{currentstroke}%
\pgfsetdash{}{0pt}%
\pgfpathmoveto{\pgfqpoint{4.686160in}{4.082636in}}%
\pgfpathlineto{\pgfqpoint{4.604647in}{4.165055in}}%
\pgfpathlineto{\pgfqpoint{4.827480in}{3.864804in}}%
\pgfpathclose%
\pgfusepath{fill}%
\end{pgfscope}%
\begin{pgfscope}%
\pgfpathrectangle{\pgfqpoint{0.539299in}{0.078740in}}{\pgfqpoint{7.842520in}{7.842520in}}%
\pgfusepath{clip}%
\pgfsetbuttcap%
\pgfsetroundjoin%
\definecolor{currentfill}{rgb}{0.140536,0.530132,0.555659}%
\pgfsetfillcolor{currentfill}%
\pgfsetlinewidth{0.000000pt}%
\definecolor{currentstroke}{rgb}{0.128729,0.563265,0.551229}%
\pgfsetstrokecolor{currentstroke}%
\pgfsetdash{}{0pt}%
\pgfpathmoveto{\pgfqpoint{4.888578in}{3.720803in}}%
\pgfpathlineto{\pgfqpoint{5.030239in}{3.504331in}}%
\pgfpathlineto{\pgfqpoint{4.968612in}{3.651484in}}%
\pgfpathclose%
\pgfusepath{fill}%
\end{pgfscope}%
\begin{pgfscope}%
\pgfpathrectangle{\pgfqpoint{0.539299in}{0.078740in}}{\pgfqpoint{7.842520in}{7.842520in}}%
\pgfusepath{clip}%
\pgfsetbuttcap%
\pgfsetroundjoin%
\definecolor{currentfill}{rgb}{0.175841,0.441290,0.557685}%
\pgfsetfillcolor{currentfill}%
\pgfsetlinewidth{0.000000pt}%
\definecolor{currentstroke}{rgb}{0.127568,0.566949,0.550556}%
\pgfsetstrokecolor{currentstroke}%
\pgfsetdash{}{0pt}%
\pgfpathmoveto{\pgfqpoint{1.804002in}{3.315816in}}%
\pgfpathlineto{\pgfqpoint{1.726778in}{2.916957in}}%
\pgfpathlineto{\pgfqpoint{1.935199in}{3.440713in}}%
\pgfpathclose%
\pgfusepath{fill}%
\end{pgfscope}%
\begin{pgfscope}%
\pgfpathrectangle{\pgfqpoint{0.539299in}{0.078740in}}{\pgfqpoint{7.842520in}{7.842520in}}%
\pgfusepath{clip}%
\pgfsetbuttcap%
\pgfsetroundjoin%
\definecolor{currentfill}{rgb}{0.280868,0.160771,0.472899}%
\pgfsetfillcolor{currentfill}%
\pgfsetlinewidth{0.000000pt}%
\definecolor{currentstroke}{rgb}{0.126453,0.570633,0.549841}%
\pgfsetstrokecolor{currentstroke}%
\pgfsetdash{}{0pt}%
\pgfpathmoveto{\pgfqpoint{6.020118in}{2.191364in}}%
\pgfpathlineto{\pgfqpoint{5.943414in}{2.169358in}}%
\pgfpathlineto{\pgfqpoint{6.161891in}{2.037228in}}%
\pgfpathclose%
\pgfusepath{fill}%
\end{pgfscope}%
\begin{pgfscope}%
\pgfpathrectangle{\pgfqpoint{0.539299in}{0.078740in}}{\pgfqpoint{7.842520in}{7.842520in}}%
\pgfusepath{clip}%
\pgfsetbuttcap%
\pgfsetroundjoin%
\definecolor{currentfill}{rgb}{0.271828,0.209303,0.504434}%
\pgfsetfillcolor{currentfill}%
\pgfsetlinewidth{0.000000pt}%
\definecolor{currentstroke}{rgb}{0.125394,0.574318,0.549086}%
\pgfsetstrokecolor{currentstroke}%
\pgfsetdash{}{0pt}%
\pgfpathmoveto{\pgfqpoint{6.020118in}{2.191364in}}%
\pgfpathlineto{\pgfqpoint{5.878547in}{2.354530in}}%
\pgfpathlineto{\pgfqpoint{5.801571in}{2.347348in}}%
\pgfpathclose%
\pgfusepath{fill}%
\end{pgfscope}%
\begin{pgfscope}%
\pgfpathrectangle{\pgfqpoint{0.539299in}{0.078740in}}{\pgfqpoint{7.842520in}{7.842520in}}%
\pgfusepath{clip}%
\pgfsetbuttcap%
\pgfsetroundjoin%
\definecolor{currentfill}{rgb}{0.263663,0.237631,0.518762}%
\pgfsetfillcolor{currentfill}%
\pgfsetlinewidth{0.000000pt}%
\definecolor{currentstroke}{rgb}{0.124395,0.578002,0.548287}%
\pgfsetstrokecolor{currentstroke}%
\pgfsetdash{}{0pt}%
\pgfpathmoveto{\pgfqpoint{5.737117in}{2.526203in}}%
\pgfpathlineto{\pgfqpoint{5.801571in}{2.347348in}}%
\pgfpathlineto{\pgfqpoint{5.878547in}{2.354530in}}%
\pgfpathclose%
\pgfusepath{fill}%
\end{pgfscope}%
\begin{pgfscope}%
\pgfpathrectangle{\pgfqpoint{0.539299in}{0.078740in}}{\pgfqpoint{7.842520in}{7.842520in}}%
\pgfusepath{clip}%
\pgfsetbuttcap%
\pgfsetroundjoin%
\definecolor{currentfill}{rgb}{0.282327,0.094955,0.417331}%
\pgfsetfillcolor{currentfill}%
\pgfsetlinewidth{0.000000pt}%
\definecolor{currentstroke}{rgb}{0.123463,0.581687,0.547445}%
\pgfsetstrokecolor{currentstroke}%
\pgfsetdash{}{0pt}%
\pgfpathmoveto{\pgfqpoint{6.303947in}{1.893136in}}%
\pgfpathlineto{\pgfqpoint{6.161891in}{2.037228in}}%
\pgfpathlineto{\pgfqpoint{6.227580in}{1.841168in}}%
\pgfpathclose%
\pgfusepath{fill}%
\end{pgfscope}%
\begin{pgfscope}%
\pgfpathrectangle{\pgfqpoint{0.539299in}{0.078740in}}{\pgfqpoint{7.842520in}{7.842520in}}%
\pgfusepath{clip}%
\pgfsetbuttcap%
\pgfsetroundjoin%
\definecolor{currentfill}{rgb}{0.814576,0.883393,0.110347}%
\pgfsetfillcolor{currentfill}%
\pgfsetlinewidth{0.000000pt}%
\definecolor{currentstroke}{rgb}{0.122606,0.585371,0.546557}%
\pgfsetstrokecolor{currentstroke}%
\pgfsetdash{}{0pt}%
\pgfpathmoveto{\pgfqpoint{3.185707in}{5.629892in}}%
\pgfpathlineto{\pgfqpoint{3.048388in}{5.568482in}}%
\pgfpathlineto{\pgfqpoint{2.961006in}{5.551827in}}%
\pgfpathclose%
\pgfusepath{fill}%
\end{pgfscope}%
\begin{pgfscope}%
\pgfpathrectangle{\pgfqpoint{0.539299in}{0.078740in}}{\pgfqpoint{7.842520in}{7.842520in}}%
\pgfusepath{clip}%
\pgfsetbuttcap%
\pgfsetroundjoin%
\definecolor{currentfill}{rgb}{0.132268,0.655014,0.519661}%
\pgfsetfillcolor{currentfill}%
\pgfsetlinewidth{0.000000pt}%
\definecolor{currentstroke}{rgb}{0.121831,0.589055,0.545623}%
\pgfsetstrokecolor{currentstroke}%
\pgfsetdash{}{0pt}%
\pgfpathmoveto{\pgfqpoint{4.544630in}{4.302516in}}%
\pgfpathlineto{\pgfqpoint{4.604647in}{4.165055in}}%
\pgfpathlineto{\pgfqpoint{4.686160in}{4.082636in}}%
\pgfpathclose%
\pgfusepath{fill}%
\end{pgfscope}%
\begin{pgfscope}%
\pgfpathrectangle{\pgfqpoint{0.539299in}{0.078740in}}{\pgfqpoint{7.842520in}{7.842520in}}%
\pgfusepath{clip}%
\pgfsetbuttcap%
\pgfsetroundjoin%
\definecolor{currentfill}{rgb}{0.279566,0.067836,0.391917}%
\pgfsetfillcolor{currentfill}%
\pgfsetlinewidth{0.000000pt}%
\definecolor{currentstroke}{rgb}{0.121148,0.592739,0.544641}%
\pgfsetstrokecolor{currentstroke}%
\pgfsetdash{}{0pt}%
\pgfpathmoveto{\pgfqpoint{6.227580in}{1.841168in}}%
\pgfpathlineto{\pgfqpoint{6.446395in}{1.760841in}}%
\pgfpathlineto{\pgfqpoint{6.303947in}{1.893136in}}%
\pgfpathclose%
\pgfusepath{fill}%
\end{pgfscope}%
\begin{pgfscope}%
\pgfpathrectangle{\pgfqpoint{0.539299in}{0.078740in}}{\pgfqpoint{7.842520in}{7.842520in}}%
\pgfusepath{clip}%
\pgfsetbuttcap%
\pgfsetroundjoin%
\definecolor{currentfill}{rgb}{0.835270,0.886029,0.102646}%
\pgfsetfillcolor{currentfill}%
\pgfsetlinewidth{0.000000pt}%
\definecolor{currentstroke}{rgb}{0.120565,0.596422,0.543611}%
\pgfsetstrokecolor{currentstroke}%
\pgfsetdash{}{0pt}%
\pgfpathmoveto{\pgfqpoint{3.324832in}{5.632651in}}%
\pgfpathlineto{\pgfqpoint{3.272857in}{5.610573in}}%
\pgfpathlineto{\pgfqpoint{3.185707in}{5.629892in}}%
\pgfpathclose%
\pgfusepath{fill}%
\end{pgfscope}%
\begin{pgfscope}%
\pgfpathrectangle{\pgfqpoint{0.539299in}{0.078740in}}{\pgfqpoint{7.842520in}{7.842520in}}%
\pgfusepath{clip}%
\pgfsetbuttcap%
\pgfsetroundjoin%
\definecolor{currentfill}{rgb}{0.235526,0.309527,0.542944}%
\pgfsetfillcolor{currentfill}%
\pgfsetlinewidth{0.000000pt}%
\definecolor{currentstroke}{rgb}{0.120092,0.600104,0.542530}%
\pgfsetstrokecolor{currentstroke}%
\pgfsetdash{}{0pt}%
\pgfpathmoveto{\pgfqpoint{5.595772in}{2.706138in}}%
\pgfpathlineto{\pgfqpoint{5.518023in}{2.726615in}}%
\pgfpathlineto{\pgfqpoint{5.737117in}{2.526203in}}%
\pgfpathclose%
\pgfusepath{fill}%
\end{pgfscope}%
\begin{pgfscope}%
\pgfpathrectangle{\pgfqpoint{0.539299in}{0.078740in}}{\pgfqpoint{7.842520in}{7.842520in}}%
\pgfusepath{clip}%
\pgfsetbuttcap%
\pgfsetroundjoin%
\definecolor{currentfill}{rgb}{0.127568,0.566949,0.550556}%
\pgfsetfillcolor{currentfill}%
\pgfsetlinewidth{0.000000pt}%
\definecolor{currentstroke}{rgb}{0.119738,0.603785,0.541400}%
\pgfsetstrokecolor{currentstroke}%
\pgfsetdash{}{0pt}%
\pgfpathmoveto{\pgfqpoint{4.746722in}{3.941721in}}%
\pgfpathlineto{\pgfqpoint{4.888578in}{3.720803in}}%
\pgfpathlineto{\pgfqpoint{4.968612in}{3.651484in}}%
\pgfpathclose%
\pgfusepath{fill}%
\end{pgfscope}%
\begin{pgfscope}%
\pgfpathrectangle{\pgfqpoint{0.539299in}{0.078740in}}{\pgfqpoint{7.842520in}{7.842520in}}%
\pgfusepath{clip}%
\pgfsetbuttcap%
\pgfsetroundjoin%
\definecolor{currentfill}{rgb}{0.180653,0.701402,0.488189}%
\pgfsetfillcolor{currentfill}%
\pgfsetlinewidth{0.000000pt}%
\definecolor{currentstroke}{rgb}{0.119512,0.607464,0.540218}%
\pgfsetstrokecolor{currentstroke}%
\pgfsetdash{}{0pt}%
\pgfpathmoveto{\pgfqpoint{2.133656in}{4.370779in}}%
\pgfpathlineto{\pgfqpoint{2.049048in}{4.173228in}}%
\pgfpathlineto{\pgfqpoint{2.261900in}{4.632062in}}%
\pgfpathclose%
\pgfusepath{fill}%
\end{pgfscope}%
\begin{pgfscope}%
\pgfpathrectangle{\pgfqpoint{0.539299in}{0.078740in}}{\pgfqpoint{7.842520in}{7.842520in}}%
\pgfusepath{clip}%
\pgfsetbuttcap%
\pgfsetroundjoin%
\definecolor{currentfill}{rgb}{0.183898,0.422383,0.556944}%
\pgfsetfillcolor{currentfill}%
\pgfsetlinewidth{0.000000pt}%
\definecolor{currentstroke}{rgb}{0.119423,0.611141,0.538982}%
\pgfsetstrokecolor{currentstroke}%
\pgfsetdash{}{0pt}%
\pgfpathmoveto{\pgfqpoint{1.935199in}{3.440713in}}%
\pgfpathlineto{\pgfqpoint{1.726778in}{2.916957in}}%
\pgfpathlineto{\pgfqpoint{1.859345in}{2.989662in}}%
\pgfpathclose%
\pgfusepath{fill}%
\end{pgfscope}%
\begin{pgfscope}%
\pgfpathrectangle{\pgfqpoint{0.539299in}{0.078740in}}{\pgfqpoint{7.842520in}{7.842520in}}%
\pgfusepath{clip}%
\pgfsetbuttcap%
\pgfsetroundjoin%
\definecolor{currentfill}{rgb}{0.595839,0.848717,0.243329}%
\pgfsetfillcolor{currentfill}%
\pgfsetlinewidth{0.000000pt}%
\definecolor{currentstroke}{rgb}{0.119483,0.614817,0.537692}%
\pgfsetstrokecolor{currentstroke}%
\pgfsetdash{}{0pt}%
\pgfpathmoveto{\pgfqpoint{2.519975in}{5.063797in}}%
\pgfpathlineto{\pgfqpoint{2.651662in}{5.282437in}}%
\pgfpathlineto{\pgfqpoint{2.738785in}{5.362101in}}%
\pgfpathclose%
\pgfusepath{fill}%
\end{pgfscope}%
\begin{pgfscope}%
\pgfpathrectangle{\pgfqpoint{0.539299in}{0.078740in}}{\pgfqpoint{7.842520in}{7.842520in}}%
\pgfusepath{clip}%
\pgfsetbuttcap%
\pgfsetroundjoin%
\definecolor{currentfill}{rgb}{0.151918,0.500685,0.557587}%
\pgfsetfillcolor{currentfill}%
\pgfsetlinewidth{0.000000pt}%
\definecolor{currentstroke}{rgb}{0.119699,0.618490,0.536347}%
\pgfsetstrokecolor{currentstroke}%
\pgfsetdash{}{0pt}%
\pgfpathmoveto{\pgfqpoint{1.804002in}{3.315816in}}%
\pgfpathlineto{\pgfqpoint{1.935199in}{3.440713in}}%
\pgfpathlineto{\pgfqpoint{1.883808in}{3.652469in}}%
\pgfpathclose%
\pgfusepath{fill}%
\end{pgfscope}%
\begin{pgfscope}%
\pgfpathrectangle{\pgfqpoint{0.539299in}{0.078740in}}{\pgfqpoint{7.842520in}{7.842520in}}%
\pgfusepath{clip}%
\pgfsetbuttcap%
\pgfsetroundjoin%
\definecolor{currentfill}{rgb}{0.214298,0.355619,0.551184}%
\pgfsetfillcolor{currentfill}%
\pgfsetlinewidth{0.000000pt}%
\definecolor{currentstroke}{rgb}{0.120081,0.622161,0.534946}%
\pgfsetstrokecolor{currentstroke}%
\pgfsetdash{}{0pt}%
\pgfpathmoveto{\pgfqpoint{5.595772in}{2.706138in}}%
\pgfpathlineto{\pgfqpoint{5.454461in}{2.894198in}}%
\pgfpathlineto{\pgfqpoint{5.376216in}{2.927023in}}%
\pgfpathclose%
\pgfusepath{fill}%
\end{pgfscope}%
\begin{pgfscope}%
\pgfpathrectangle{\pgfqpoint{0.539299in}{0.078740in}}{\pgfqpoint{7.842520in}{7.842520in}}%
\pgfusepath{clip}%
\pgfsetbuttcap%
\pgfsetroundjoin%
\definecolor{currentfill}{rgb}{0.824940,0.884720,0.106217}%
\pgfsetfillcolor{currentfill}%
\pgfsetlinewidth{0.000000pt}%
\definecolor{currentstroke}{rgb}{0.120638,0.625828,0.533488}%
\pgfsetstrokecolor{currentstroke}%
\pgfsetdash{}{0pt}%
\pgfpathmoveto{\pgfqpoint{3.324832in}{5.632651in}}%
\pgfpathlineto{\pgfqpoint{3.552031in}{5.538964in}}%
\pgfpathlineto{\pgfqpoint{3.411816in}{5.600141in}}%
\pgfpathclose%
\pgfusepath{fill}%
\end{pgfscope}%
\begin{pgfscope}%
\pgfpathrectangle{\pgfqpoint{0.539299in}{0.078740in}}{\pgfqpoint{7.842520in}{7.842520in}}%
\pgfusepath{clip}%
\pgfsetbuttcap%
\pgfsetroundjoin%
\definecolor{currentfill}{rgb}{0.720391,0.870350,0.162603}%
\pgfsetfillcolor{currentfill}%
\pgfsetlinewidth{0.000000pt}%
\definecolor{currentstroke}{rgb}{0.121380,0.629492,0.531973}%
\pgfsetstrokecolor{currentstroke}%
\pgfsetdash{}{0pt}%
\pgfpathmoveto{\pgfqpoint{2.826074in}{5.413789in}}%
\pgfpathlineto{\pgfqpoint{2.738785in}{5.362101in}}%
\pgfpathlineto{\pgfqpoint{2.873601in}{5.508440in}}%
\pgfpathclose%
\pgfusepath{fill}%
\end{pgfscope}%
\begin{pgfscope}%
\pgfpathrectangle{\pgfqpoint{0.539299in}{0.078740in}}{\pgfqpoint{7.842520in}{7.842520in}}%
\pgfusepath{clip}%
\pgfsetbuttcap%
\pgfsetroundjoin%
\definecolor{currentfill}{rgb}{0.197636,0.391528,0.554969}%
\pgfsetfillcolor{currentfill}%
\pgfsetlinewidth{0.000000pt}%
\definecolor{currentstroke}{rgb}{0.122312,0.633153,0.530398}%
\pgfsetstrokecolor{currentstroke}%
\pgfsetdash{}{0pt}%
\pgfpathmoveto{\pgfqpoint{5.376216in}{2.927023in}}%
\pgfpathlineto{\pgfqpoint{5.454461in}{2.894198in}}%
\pgfpathlineto{\pgfqpoint{5.313135in}{3.090200in}}%
\pgfpathclose%
\pgfusepath{fill}%
\end{pgfscope}%
\begin{pgfscope}%
\pgfpathrectangle{\pgfqpoint{0.539299in}{0.078740in}}{\pgfqpoint{7.842520in}{7.842520in}}%
\pgfusepath{clip}%
\pgfsetbuttcap%
\pgfsetroundjoin%
\definecolor{currentfill}{rgb}{0.119699,0.618490,0.536347}%
\pgfsetfillcolor{currentfill}%
\pgfsetlinewidth{0.000000pt}%
\definecolor{currentstroke}{rgb}{0.123444,0.636809,0.528763}%
\pgfsetstrokecolor{currentstroke}%
\pgfsetdash{}{0pt}%
\pgfpathmoveto{\pgfqpoint{4.604647in}{4.165055in}}%
\pgfpathlineto{\pgfqpoint{4.746722in}{3.941721in}}%
\pgfpathlineto{\pgfqpoint{4.827480in}{3.864804in}}%
\pgfpathclose%
\pgfusepath{fill}%
\end{pgfscope}%
\begin{pgfscope}%
\pgfpathrectangle{\pgfqpoint{0.539299in}{0.078740in}}{\pgfqpoint{7.842520in}{7.842520in}}%
\pgfusepath{clip}%
\pgfsetbuttcap%
\pgfsetroundjoin%
\definecolor{currentfill}{rgb}{0.283197,0.115680,0.436115}%
\pgfsetfillcolor{currentfill}%
\pgfsetlinewidth{0.000000pt}%
\definecolor{currentstroke}{rgb}{0.124780,0.640461,0.527068}%
\pgfsetstrokecolor{currentstroke}%
\pgfsetdash{}{0pt}%
\pgfpathmoveto{\pgfqpoint{6.227580in}{1.841168in}}%
\pgfpathlineto{\pgfqpoint{6.161891in}{2.037228in}}%
\pgfpathlineto{\pgfqpoint{6.085388in}{2.000146in}}%
\pgfpathclose%
\pgfusepath{fill}%
\end{pgfscope}%
\begin{pgfscope}%
\pgfpathrectangle{\pgfqpoint{0.539299in}{0.078740in}}{\pgfqpoint{7.842520in}{7.842520in}}%
\pgfusepath{clip}%
\pgfsetbuttcap%
\pgfsetroundjoin%
\definecolor{currentfill}{rgb}{0.281887,0.150881,0.465405}%
\pgfsetfillcolor{currentfill}%
\pgfsetlinewidth{0.000000pt}%
\definecolor{currentstroke}{rgb}{0.126326,0.644107,0.525311}%
\pgfsetstrokecolor{currentstroke}%
\pgfsetdash{}{0pt}%
\pgfpathmoveto{\pgfqpoint{6.161891in}{2.037228in}}%
\pgfpathlineto{\pgfqpoint{5.943414in}{2.169358in}}%
\pgfpathlineto{\pgfqpoint{6.085388in}{2.000146in}}%
\pgfpathclose%
\pgfusepath{fill}%
\end{pgfscope}%
\begin{pgfscope}%
\pgfpathrectangle{\pgfqpoint{0.539299in}{0.078740in}}{\pgfqpoint{7.842520in}{7.842520in}}%
\pgfusepath{clip}%
\pgfsetbuttcap%
\pgfsetroundjoin%
\definecolor{currentfill}{rgb}{0.274128,0.199721,0.498911}%
\pgfsetfillcolor{currentfill}%
\pgfsetlinewidth{0.000000pt}%
\definecolor{currentstroke}{rgb}{0.128087,0.647749,0.523491}%
\pgfsetstrokecolor{currentstroke}%
\pgfsetdash{}{0pt}%
\pgfpathmoveto{\pgfqpoint{5.801571in}{2.347348in}}%
\pgfpathlineto{\pgfqpoint{5.943414in}{2.169358in}}%
\pgfpathlineto{\pgfqpoint{6.020118in}{2.191364in}}%
\pgfpathclose%
\pgfusepath{fill}%
\end{pgfscope}%
\begin{pgfscope}%
\pgfpathrectangle{\pgfqpoint{0.539299in}{0.078740in}}{\pgfqpoint{7.842520in}{7.842520in}}%
\pgfusepath{clip}%
\pgfsetbuttcap%
\pgfsetroundjoin%
\definecolor{currentfill}{rgb}{0.772852,0.877868,0.131109}%
\pgfsetfillcolor{currentfill}%
\pgfsetlinewidth{0.000000pt}%
\definecolor{currentstroke}{rgb}{0.130067,0.651384,0.521608}%
\pgfsetstrokecolor{currentstroke}%
\pgfsetdash{}{0pt}%
\pgfpathmoveto{\pgfqpoint{2.873601in}{5.508440in}}%
\pgfpathlineto{\pgfqpoint{2.961006in}{5.551827in}}%
\pgfpathlineto{\pgfqpoint{2.826074in}{5.413789in}}%
\pgfpathclose%
\pgfusepath{fill}%
\end{pgfscope}%
\begin{pgfscope}%
\pgfpathrectangle{\pgfqpoint{0.539299in}{0.078740in}}{\pgfqpoint{7.842520in}{7.842520in}}%
\pgfusepath{clip}%
\pgfsetbuttcap%
\pgfsetroundjoin%
\definecolor{currentfill}{rgb}{0.220124,0.725509,0.466226}%
\pgfsetfillcolor{currentfill}%
\pgfsetlinewidth{0.000000pt}%
\definecolor{currentstroke}{rgb}{0.132268,0.655014,0.519661}%
\pgfsetstrokecolor{currentstroke}%
\pgfsetdash{}{0pt}%
\pgfpathmoveto{\pgfqpoint{4.319824in}{4.607800in}}%
\pgfpathlineto{\pgfqpoint{4.544630in}{4.302516in}}%
\pgfpathlineto{\pgfqpoint{4.402885in}{4.521271in}}%
\pgfpathclose%
\pgfusepath{fill}%
\end{pgfscope}%
\begin{pgfscope}%
\pgfpathrectangle{\pgfqpoint{0.539299in}{0.078740in}}{\pgfqpoint{7.842520in}{7.842520in}}%
\pgfusepath{clip}%
\pgfsetbuttcap%
\pgfsetroundjoin%
\definecolor{currentfill}{rgb}{0.124780,0.640461,0.527068}%
\pgfsetfillcolor{currentfill}%
\pgfsetlinewidth{0.000000pt}%
\definecolor{currentstroke}{rgb}{0.134692,0.658636,0.517649}%
\pgfsetstrokecolor{currentstroke}%
\pgfsetdash{}{0pt}%
\pgfpathmoveto{\pgfqpoint{2.094863in}{4.143974in}}%
\pgfpathlineto{\pgfqpoint{2.049048in}{4.173228in}}%
\pgfpathlineto{\pgfqpoint{1.965641in}{3.935827in}}%
\pgfpathclose%
\pgfusepath{fill}%
\end{pgfscope}%
\begin{pgfscope}%
\pgfpathrectangle{\pgfqpoint{0.539299in}{0.078740in}}{\pgfqpoint{7.842520in}{7.842520in}}%
\pgfusepath{clip}%
\pgfsetbuttcap%
\pgfsetroundjoin%
\definecolor{currentfill}{rgb}{0.253935,0.265254,0.529983}%
\pgfsetfillcolor{currentfill}%
\pgfsetlinewidth{0.000000pt}%
\definecolor{currentstroke}{rgb}{0.137339,0.662252,0.515571}%
\pgfsetstrokecolor{currentstroke}%
\pgfsetdash{}{0pt}%
\pgfpathmoveto{\pgfqpoint{5.659792in}{2.533268in}}%
\pgfpathlineto{\pgfqpoint{5.801571in}{2.347348in}}%
\pgfpathlineto{\pgfqpoint{5.737117in}{2.526203in}}%
\pgfpathclose%
\pgfusepath{fill}%
\end{pgfscope}%
\begin{pgfscope}%
\pgfpathrectangle{\pgfqpoint{0.539299in}{0.078740in}}{\pgfqpoint{7.842520in}{7.842520in}}%
\pgfusepath{clip}%
\pgfsetbuttcap%
\pgfsetroundjoin%
\definecolor{currentfill}{rgb}{0.212395,0.359683,0.551710}%
\pgfsetfillcolor{currentfill}%
\pgfsetlinewidth{0.000000pt}%
\definecolor{currentstroke}{rgb}{0.140210,0.665859,0.513427}%
\pgfsetstrokecolor{currentstroke}%
\pgfsetdash{}{0pt}%
\pgfpathmoveto{\pgfqpoint{1.859345in}{2.989662in}}%
\pgfpathlineto{\pgfqpoint{1.922231in}{2.462670in}}%
\pgfpathlineto{\pgfqpoint{1.993699in}{3.037069in}}%
\pgfpathclose%
\pgfusepath{fill}%
\end{pgfscope}%
\begin{pgfscope}%
\pgfpathrectangle{\pgfqpoint{0.539299in}{0.078740in}}{\pgfqpoint{7.842520in}{7.842520in}}%
\pgfusepath{clip}%
\pgfsetbuttcap%
\pgfsetroundjoin%
\definecolor{currentfill}{rgb}{0.506271,0.828786,0.300362}%
\pgfsetfillcolor{currentfill}%
\pgfsetlinewidth{0.000000pt}%
\definecolor{currentstroke}{rgb}{0.143303,0.669459,0.511215}%
\pgfsetstrokecolor{currentstroke}%
\pgfsetdash{}{0pt}%
\pgfpathmoveto{\pgfqpoint{2.519975in}{5.063797in}}%
\pgfpathlineto{\pgfqpoint{2.433328in}{4.953363in}}%
\pgfpathlineto{\pgfqpoint{2.564916in}{5.170485in}}%
\pgfpathclose%
\pgfusepath{fill}%
\end{pgfscope}%
\begin{pgfscope}%
\pgfpathrectangle{\pgfqpoint{0.539299in}{0.078740in}}{\pgfqpoint{7.842520in}{7.842520in}}%
\pgfusepath{clip}%
\pgfsetbuttcap%
\pgfsetroundjoin%
\definecolor{currentfill}{rgb}{0.274149,0.751988,0.436601}%
\pgfsetfillcolor{currentfill}%
\pgfsetlinewidth{0.000000pt}%
\definecolor{currentstroke}{rgb}{0.146616,0.673050,0.508936}%
\pgfsetstrokecolor{currentstroke}%
\pgfsetdash{}{0pt}%
\pgfpathmoveto{\pgfqpoint{2.133656in}{4.370779in}}%
\pgfpathlineto{\pgfqpoint{2.261900in}{4.632062in}}%
\pgfpathlineto{\pgfqpoint{2.347219in}{4.810962in}}%
\pgfpathclose%
\pgfusepath{fill}%
\end{pgfscope}%
\begin{pgfscope}%
\pgfpathrectangle{\pgfqpoint{0.539299in}{0.078740in}}{\pgfqpoint{7.842520in}{7.842520in}}%
\pgfusepath{clip}%
\pgfsetbuttcap%
\pgfsetroundjoin%
\definecolor{currentfill}{rgb}{0.277941,0.056324,0.381191}%
\pgfsetfillcolor{currentfill}%
\pgfsetlinewidth{0.000000pt}%
\definecolor{currentstroke}{rgb}{0.150148,0.676631,0.506589}%
\pgfsetstrokecolor{currentstroke}%
\pgfsetdash{}{0pt}%
\pgfpathmoveto{\pgfqpoint{6.370114in}{1.694823in}}%
\pgfpathlineto{\pgfqpoint{6.446395in}{1.760841in}}%
\pgfpathlineto{\pgfqpoint{6.227580in}{1.841168in}}%
\pgfpathclose%
\pgfusepath{fill}%
\end{pgfscope}%
\begin{pgfscope}%
\pgfpathrectangle{\pgfqpoint{0.539299in}{0.078740in}}{\pgfqpoint{7.842520in}{7.842520in}}%
\pgfusepath{clip}%
\pgfsetbuttcap%
\pgfsetroundjoin%
\definecolor{currentfill}{rgb}{0.119738,0.603785,0.541400}%
\pgfsetfillcolor{currentfill}%
\pgfsetlinewidth{0.000000pt}%
\definecolor{currentstroke}{rgb}{0.153894,0.680203,0.504172}%
\pgfsetstrokecolor{currentstroke}%
\pgfsetdash{}{0pt}%
\pgfpathmoveto{\pgfqpoint{1.883808in}{3.652469in}}%
\pgfpathlineto{\pgfqpoint{2.094863in}{4.143974in}}%
\pgfpathlineto{\pgfqpoint{1.965641in}{3.935827in}}%
\pgfpathclose%
\pgfusepath{fill}%
\end{pgfscope}%
\begin{pgfscope}%
\pgfpathrectangle{\pgfqpoint{0.539299in}{0.078740in}}{\pgfqpoint{7.842520in}{7.842520in}}%
\pgfusepath{clip}%
\pgfsetbuttcap%
\pgfsetroundjoin%
\definecolor{currentfill}{rgb}{0.239346,0.300855,0.540844}%
\pgfsetfillcolor{currentfill}%
\pgfsetlinewidth{0.000000pt}%
\definecolor{currentstroke}{rgb}{0.157851,0.683765,0.501686}%
\pgfsetstrokecolor{currentstroke}%
\pgfsetdash{}{0pt}%
\pgfpathmoveto{\pgfqpoint{5.737117in}{2.526203in}}%
\pgfpathlineto{\pgfqpoint{5.518023in}{2.726615in}}%
\pgfpathlineto{\pgfqpoint{5.659792in}{2.533268in}}%
\pgfpathclose%
\pgfusepath{fill}%
\end{pgfscope}%
\begin{pgfscope}%
\pgfpathrectangle{\pgfqpoint{0.539299in}{0.078740in}}{\pgfqpoint{7.842520in}{7.842520in}}%
\pgfusepath{clip}%
\pgfsetbuttcap%
\pgfsetroundjoin%
\definecolor{currentfill}{rgb}{0.169646,0.456262,0.558030}%
\pgfsetfillcolor{currentfill}%
\pgfsetlinewidth{0.000000pt}%
\definecolor{currentstroke}{rgb}{0.162016,0.687316,0.499129}%
\pgfsetstrokecolor{currentstroke}%
\pgfsetdash{}{0pt}%
\pgfpathmoveto{\pgfqpoint{5.313135in}{3.090200in}}%
\pgfpathlineto{\pgfqpoint{5.171743in}{3.293790in}}%
\pgfpathlineto{\pgfqpoint{5.092313in}{3.347287in}}%
\pgfpathclose%
\pgfusepath{fill}%
\end{pgfscope}%
\begin{pgfscope}%
\pgfpathrectangle{\pgfqpoint{0.539299in}{0.078740in}}{\pgfqpoint{7.842520in}{7.842520in}}%
\pgfusepath{clip}%
\pgfsetbuttcap%
\pgfsetroundjoin%
\definecolor{currentfill}{rgb}{0.281477,0.755203,0.432552}%
\pgfsetfillcolor{currentfill}%
\pgfsetlinewidth{0.000000pt}%
\definecolor{currentstroke}{rgb}{0.166383,0.690856,0.496502}%
\pgfsetstrokecolor{currentstroke}%
\pgfsetdash{}{0pt}%
\pgfpathmoveto{\pgfqpoint{4.402885in}{4.521271in}}%
\pgfpathlineto{\pgfqpoint{4.260950in}{4.734983in}}%
\pgfpathlineto{\pgfqpoint{4.319824in}{4.607800in}}%
\pgfpathclose%
\pgfusepath{fill}%
\end{pgfscope}%
\begin{pgfscope}%
\pgfpathrectangle{\pgfqpoint{0.539299in}{0.078740in}}{\pgfqpoint{7.842520in}{7.842520in}}%
\pgfusepath{clip}%
\pgfsetbuttcap%
\pgfsetroundjoin%
\definecolor{currentfill}{rgb}{0.231674,0.318106,0.544834}%
\pgfsetfillcolor{currentfill}%
\pgfsetlinewidth{0.000000pt}%
\definecolor{currentstroke}{rgb}{0.170948,0.694384,0.493803}%
\pgfsetstrokecolor{currentstroke}%
\pgfsetdash{}{0pt}%
\pgfpathmoveto{\pgfqpoint{1.993699in}{3.037069in}}%
\pgfpathlineto{\pgfqpoint{1.922231in}{2.462670in}}%
\pgfpathlineto{\pgfqpoint{2.058527in}{2.453906in}}%
\pgfpathclose%
\pgfusepath{fill}%
\end{pgfscope}%
\begin{pgfscope}%
\pgfpathrectangle{\pgfqpoint{0.539299in}{0.078740in}}{\pgfqpoint{7.842520in}{7.842520in}}%
\pgfusepath{clip}%
\pgfsetbuttcap%
\pgfsetroundjoin%
\definecolor{currentfill}{rgb}{0.216210,0.351535,0.550627}%
\pgfsetfillcolor{currentfill}%
\pgfsetlinewidth{0.000000pt}%
\definecolor{currentstroke}{rgb}{0.175707,0.697900,0.491033}%
\pgfsetstrokecolor{currentstroke}%
\pgfsetdash{}{0pt}%
\pgfpathmoveto{\pgfqpoint{5.376216in}{2.927023in}}%
\pgfpathlineto{\pgfqpoint{5.518023in}{2.726615in}}%
\pgfpathlineto{\pgfqpoint{5.595772in}{2.706138in}}%
\pgfpathclose%
\pgfusepath{fill}%
\end{pgfscope}%
\begin{pgfscope}%
\pgfpathrectangle{\pgfqpoint{0.539299in}{0.078740in}}{\pgfqpoint{7.842520in}{7.842520in}}%
\pgfusepath{clip}%
\pgfsetbuttcap%
\pgfsetroundjoin%
\definecolor{currentfill}{rgb}{0.793760,0.880678,0.120005}%
\pgfsetfillcolor{currentfill}%
\pgfsetlinewidth{0.000000pt}%
\definecolor{currentstroke}{rgb}{0.180653,0.701402,0.488189}%
\pgfsetstrokecolor{currentstroke}%
\pgfsetdash{}{0pt}%
\pgfpathmoveto{\pgfqpoint{3.693131in}{5.434822in}}%
\pgfpathlineto{\pgfqpoint{3.552031in}{5.538964in}}%
\pgfpathlineto{\pgfqpoint{3.465333in}{5.584033in}}%
\pgfpathclose%
\pgfusepath{fill}%
\end{pgfscope}%
\begin{pgfscope}%
\pgfpathrectangle{\pgfqpoint{0.539299in}{0.078740in}}{\pgfqpoint{7.842520in}{7.842520in}}%
\pgfusepath{clip}%
\pgfsetbuttcap%
\pgfsetroundjoin%
\definecolor{currentfill}{rgb}{0.156270,0.489624,0.557936}%
\pgfsetfillcolor{currentfill}%
\pgfsetlinewidth{0.000000pt}%
\definecolor{currentstroke}{rgb}{0.185783,0.704891,0.485273}%
\pgfsetstrokecolor{currentstroke}%
\pgfsetdash{}{0pt}%
\pgfpathmoveto{\pgfqpoint{5.092313in}{3.347287in}}%
\pgfpathlineto{\pgfqpoint{5.171743in}{3.293790in}}%
\pgfpathlineto{\pgfqpoint{5.030239in}{3.504331in}}%
\pgfpathclose%
\pgfusepath{fill}%
\end{pgfscope}%
\begin{pgfscope}%
\pgfpathrectangle{\pgfqpoint{0.539299in}{0.078740in}}{\pgfqpoint{7.842520in}{7.842520in}}%
\pgfusepath{clip}%
\pgfsetbuttcap%
\pgfsetroundjoin%
\definecolor{currentfill}{rgb}{0.162016,0.687316,0.499129}%
\pgfsetfillcolor{currentfill}%
\pgfsetlinewidth{0.000000pt}%
\definecolor{currentstroke}{rgb}{0.191090,0.708366,0.482284}%
\pgfsetstrokecolor{currentstroke}%
\pgfsetdash{}{0pt}%
\pgfpathmoveto{\pgfqpoint{4.544630in}{4.302516in}}%
\pgfpathlineto{\pgfqpoint{4.462343in}{4.388175in}}%
\pgfpathlineto{\pgfqpoint{4.604647in}{4.165055in}}%
\pgfpathclose%
\pgfusepath{fill}%
\end{pgfscope}%
\begin{pgfscope}%
\pgfpathrectangle{\pgfqpoint{0.539299in}{0.078740in}}{\pgfqpoint{7.842520in}{7.842520in}}%
\pgfusepath{clip}%
\pgfsetbuttcap%
\pgfsetroundjoin%
\definecolor{currentfill}{rgb}{0.855810,0.888601,0.097452}%
\pgfsetfillcolor{currentfill}%
\pgfsetlinewidth{0.000000pt}%
\definecolor{currentstroke}{rgb}{0.196571,0.711827,0.479221}%
\pgfsetstrokecolor{currentstroke}%
\pgfsetdash{}{0pt}%
\pgfpathmoveto{\pgfqpoint{2.961006in}{5.551827in}}%
\pgfpathlineto{\pgfqpoint{3.098325in}{5.625023in}}%
\pgfpathlineto{\pgfqpoint{3.185707in}{5.629892in}}%
\pgfpathclose%
\pgfusepath{fill}%
\end{pgfscope}%
\begin{pgfscope}%
\pgfpathrectangle{\pgfqpoint{0.539299in}{0.078740in}}{\pgfqpoint{7.842520in}{7.842520in}}%
\pgfusepath{clip}%
\pgfsetbuttcap%
\pgfsetroundjoin%
\definecolor{currentfill}{rgb}{0.585678,0.846661,0.249897}%
\pgfsetfillcolor{currentfill}%
\pgfsetlinewidth{0.000000pt}%
\definecolor{currentstroke}{rgb}{0.202219,0.715272,0.476084}%
\pgfsetstrokecolor{currentstroke}%
\pgfsetdash{}{0pt}%
\pgfpathmoveto{\pgfqpoint{2.564916in}{5.170485in}}%
\pgfpathlineto{\pgfqpoint{2.651662in}{5.282437in}}%
\pgfpathlineto{\pgfqpoint{2.519975in}{5.063797in}}%
\pgfpathclose%
\pgfusepath{fill}%
\end{pgfscope}%
\begin{pgfscope}%
\pgfpathrectangle{\pgfqpoint{0.539299in}{0.078740in}}{\pgfqpoint{7.842520in}{7.842520in}}%
\pgfusepath{clip}%
\pgfsetbuttcap%
\pgfsetroundjoin%
\definecolor{currentfill}{rgb}{0.133743,0.548535,0.553541}%
\pgfsetfillcolor{currentfill}%
\pgfsetlinewidth{0.000000pt}%
\definecolor{currentstroke}{rgb}{0.208030,0.718701,0.472873}%
\pgfsetstrokecolor{currentstroke}%
\pgfsetdash{}{0pt}%
\pgfpathmoveto{\pgfqpoint{1.883808in}{3.652469in}}%
\pgfpathlineto{\pgfqpoint{1.935199in}{3.440713in}}%
\pgfpathlineto{\pgfqpoint{2.013886in}{3.822753in}}%
\pgfpathclose%
\pgfusepath{fill}%
\end{pgfscope}%
\begin{pgfscope}%
\pgfpathrectangle{\pgfqpoint{0.539299in}{0.078740in}}{\pgfqpoint{7.842520in}{7.842520in}}%
\pgfusepath{clip}%
\pgfsetbuttcap%
\pgfsetroundjoin%
\definecolor{currentfill}{rgb}{0.430983,0.808473,0.346476}%
\pgfsetfillcolor{currentfill}%
\pgfsetlinewidth{0.000000pt}%
\definecolor{currentstroke}{rgb}{0.214000,0.722114,0.469588}%
\pgfsetstrokecolor{currentstroke}%
\pgfsetdash{}{0pt}%
\pgfpathmoveto{\pgfqpoint{4.034345in}{5.020034in}}%
\pgfpathlineto{\pgfqpoint{4.260950in}{4.734983in}}%
\pgfpathlineto{\pgfqpoint{4.118884in}{4.938971in}}%
\pgfpathclose%
\pgfusepath{fill}%
\end{pgfscope}%
\begin{pgfscope}%
\pgfpathrectangle{\pgfqpoint{0.539299in}{0.078740in}}{\pgfqpoint{7.842520in}{7.842520in}}%
\pgfusepath{clip}%
\pgfsetbuttcap%
\pgfsetroundjoin%
\definecolor{currentfill}{rgb}{0.175841,0.441290,0.557685}%
\pgfsetfillcolor{currentfill}%
\pgfsetlinewidth{0.000000pt}%
\definecolor{currentstroke}{rgb}{0.220124,0.725509,0.466226}%
\pgfsetstrokecolor{currentstroke}%
\pgfsetdash{}{0pt}%
\pgfpathmoveto{\pgfqpoint{1.935199in}{3.440713in}}%
\pgfpathlineto{\pgfqpoint{1.859345in}{2.989662in}}%
\pgfpathlineto{\pgfqpoint{1.993699in}{3.037069in}}%
\pgfpathclose%
\pgfusepath{fill}%
\end{pgfscope}%
\begin{pgfscope}%
\pgfpathrectangle{\pgfqpoint{0.539299in}{0.078740in}}{\pgfqpoint{7.842520in}{7.842520in}}%
\pgfusepath{clip}%
\pgfsetbuttcap%
\pgfsetroundjoin%
\definecolor{currentfill}{rgb}{0.202219,0.715272,0.476084}%
\pgfsetfillcolor{currentfill}%
\pgfsetlinewidth{0.000000pt}%
\definecolor{currentstroke}{rgb}{0.226397,0.728888,0.462789}%
\pgfsetstrokecolor{currentstroke}%
\pgfsetdash{}{0pt}%
\pgfpathmoveto{\pgfqpoint{2.261900in}{4.632062in}}%
\pgfpathlineto{\pgfqpoint{2.049048in}{4.173228in}}%
\pgfpathlineto{\pgfqpoint{2.177668in}{4.411616in}}%
\pgfpathclose%
\pgfusepath{fill}%
\end{pgfscope}%
\begin{pgfscope}%
\pgfpathrectangle{\pgfqpoint{0.539299in}{0.078740in}}{\pgfqpoint{7.842520in}{7.842520in}}%
\pgfusepath{clip}%
\pgfsetbuttcap%
\pgfsetroundjoin%
\definecolor{currentfill}{rgb}{0.185556,0.418570,0.556753}%
\pgfsetfillcolor{currentfill}%
\pgfsetlinewidth{0.000000pt}%
\definecolor{currentstroke}{rgb}{0.232815,0.732247,0.459277}%
\pgfsetstrokecolor{currentstroke}%
\pgfsetdash{}{0pt}%
\pgfpathmoveto{\pgfqpoint{5.376216in}{2.927023in}}%
\pgfpathlineto{\pgfqpoint{5.313135in}{3.090200in}}%
\pgfpathlineto{\pgfqpoint{5.234327in}{3.134099in}}%
\pgfpathclose%
\pgfusepath{fill}%
\end{pgfscope}%
\begin{pgfscope}%
\pgfpathrectangle{\pgfqpoint{0.539299in}{0.078740in}}{\pgfqpoint{7.842520in}{7.842520in}}%
\pgfusepath{clip}%
\pgfsetbuttcap%
\pgfsetroundjoin%
\definecolor{currentfill}{rgb}{0.214000,0.722114,0.469588}%
\pgfsetfillcolor{currentfill}%
\pgfsetlinewidth{0.000000pt}%
\definecolor{currentstroke}{rgb}{0.239374,0.735588,0.455688}%
\pgfsetstrokecolor{currentstroke}%
\pgfsetdash{}{0pt}%
\pgfpathmoveto{\pgfqpoint{4.462343in}{4.388175in}}%
\pgfpathlineto{\pgfqpoint{4.544630in}{4.302516in}}%
\pgfpathlineto{\pgfqpoint{4.319824in}{4.607800in}}%
\pgfpathclose%
\pgfusepath{fill}%
\end{pgfscope}%
\begin{pgfscope}%
\pgfpathrectangle{\pgfqpoint{0.539299in}{0.078740in}}{\pgfqpoint{7.842520in}{7.842520in}}%
\pgfusepath{clip}%
\pgfsetbuttcap%
\pgfsetroundjoin%
\definecolor{currentfill}{rgb}{0.845561,0.887322,0.099702}%
\pgfsetfillcolor{currentfill}%
\pgfsetlinewidth{0.000000pt}%
\definecolor{currentstroke}{rgb}{0.246070,0.738910,0.452024}%
\pgfsetstrokecolor{currentstroke}%
\pgfsetdash{}{0pt}%
\pgfpathmoveto{\pgfqpoint{3.465333in}{5.584033in}}%
\pgfpathlineto{\pgfqpoint{3.552031in}{5.538964in}}%
\pgfpathlineto{\pgfqpoint{3.324832in}{5.632651in}}%
\pgfpathclose%
\pgfusepath{fill}%
\end{pgfscope}%
\begin{pgfscope}%
\pgfpathrectangle{\pgfqpoint{0.539299in}{0.078740in}}{\pgfqpoint{7.842520in}{7.842520in}}%
\pgfusepath{clip}%
\pgfsetbuttcap%
\pgfsetroundjoin%
\definecolor{currentfill}{rgb}{0.506271,0.828786,0.300362}%
\pgfsetfillcolor{currentfill}%
\pgfsetlinewidth{0.000000pt}%
\definecolor{currentstroke}{rgb}{0.252899,0.742211,0.448284}%
\pgfsetstrokecolor{currentstroke}%
\pgfsetdash{}{0pt}%
\pgfpathmoveto{\pgfqpoint{4.118884in}{4.938971in}}%
\pgfpathlineto{\pgfqpoint{3.976785in}{5.127804in}}%
\pgfpathlineto{\pgfqpoint{4.034345in}{5.020034in}}%
\pgfpathclose%
\pgfusepath{fill}%
\end{pgfscope}%
\begin{pgfscope}%
\pgfpathrectangle{\pgfqpoint{0.539299in}{0.078740in}}{\pgfqpoint{7.842520in}{7.842520in}}%
\pgfusepath{clip}%
\pgfsetbuttcap%
\pgfsetroundjoin%
\definecolor{currentfill}{rgb}{0.477504,0.821444,0.318195}%
\pgfsetfillcolor{currentfill}%
\pgfsetlinewidth{0.000000pt}%
\definecolor{currentstroke}{rgb}{0.259857,0.745492,0.444467}%
\pgfsetstrokecolor{currentstroke}%
\pgfsetdash{}{0pt}%
\pgfpathmoveto{\pgfqpoint{2.564916in}{5.170485in}}%
\pgfpathlineto{\pgfqpoint{2.433328in}{4.953363in}}%
\pgfpathlineto{\pgfqpoint{2.347219in}{4.810962in}}%
\pgfpathclose%
\pgfusepath{fill}%
\end{pgfscope}%
\begin{pgfscope}%
\pgfpathrectangle{\pgfqpoint{0.539299in}{0.078740in}}{\pgfqpoint{7.842520in}{7.842520in}}%
\pgfusepath{clip}%
\pgfsetbuttcap%
\pgfsetroundjoin%
\definecolor{currentfill}{rgb}{0.720391,0.870350,0.162603}%
\pgfsetfillcolor{currentfill}%
\pgfsetlinewidth{0.000000pt}%
\definecolor{currentstroke}{rgb}{0.266941,0.748751,0.440573}%
\pgfsetstrokecolor{currentstroke}%
\pgfsetdash{}{0pt}%
\pgfpathmoveto{\pgfqpoint{2.873601in}{5.508440in}}%
\pgfpathlineto{\pgfqpoint{2.738785in}{5.362101in}}%
\pgfpathlineto{\pgfqpoint{2.651662in}{5.282437in}}%
\pgfpathclose%
\pgfusepath{fill}%
\end{pgfscope}%
\begin{pgfscope}%
\pgfpathrectangle{\pgfqpoint{0.539299in}{0.078740in}}{\pgfqpoint{7.842520in}{7.842520in}}%
\pgfusepath{clip}%
\pgfsetbuttcap%
\pgfsetroundjoin%
\definecolor{currentfill}{rgb}{0.709898,0.868751,0.169257}%
\pgfsetfillcolor{currentfill}%
\pgfsetlinewidth{0.000000pt}%
\definecolor{currentstroke}{rgb}{0.274149,0.751988,0.436601}%
\pgfsetstrokecolor{currentstroke}%
\pgfsetdash{}{0pt}%
\pgfpathmoveto{\pgfqpoint{3.749006in}{5.361964in}}%
\pgfpathlineto{\pgfqpoint{3.834801in}{5.295343in}}%
\pgfpathlineto{\pgfqpoint{3.693131in}{5.434822in}}%
\pgfpathclose%
\pgfusepath{fill}%
\end{pgfscope}%
\begin{pgfscope}%
\pgfpathrectangle{\pgfqpoint{0.539299in}{0.078740in}}{\pgfqpoint{7.842520in}{7.842520in}}%
\pgfusepath{clip}%
\pgfsetbuttcap%
\pgfsetroundjoin%
\definecolor{currentfill}{rgb}{0.876168,0.891125,0.095250}%
\pgfsetfillcolor{currentfill}%
\pgfsetlinewidth{0.000000pt}%
\definecolor{currentstroke}{rgb}{0.281477,0.755203,0.432552}%
\pgfsetstrokecolor{currentstroke}%
\pgfsetdash{}{0pt}%
\pgfpathmoveto{\pgfqpoint{3.185707in}{5.629892in}}%
\pgfpathlineto{\pgfqpoint{3.098325in}{5.625023in}}%
\pgfpathlineto{\pgfqpoint{3.324832in}{5.632651in}}%
\pgfpathclose%
\pgfusepath{fill}%
\end{pgfscope}%
\begin{pgfscope}%
\pgfpathrectangle{\pgfqpoint{0.539299in}{0.078740in}}{\pgfqpoint{7.842520in}{7.842520in}}%
\pgfusepath{clip}%
\pgfsetbuttcap%
\pgfsetroundjoin%
\definecolor{currentfill}{rgb}{0.157851,0.683765,0.501686}%
\pgfsetfillcolor{currentfill}%
\pgfsetlinewidth{0.000000pt}%
\definecolor{currentstroke}{rgb}{0.288921,0.758394,0.428426}%
\pgfsetstrokecolor{currentstroke}%
\pgfsetdash{}{0pt}%
\pgfpathmoveto{\pgfqpoint{2.177668in}{4.411616in}}%
\pgfpathlineto{\pgfqpoint{2.049048in}{4.173228in}}%
\pgfpathlineto{\pgfqpoint{2.094863in}{4.143974in}}%
\pgfpathclose%
\pgfusepath{fill}%
\end{pgfscope}%
\begin{pgfscope}%
\pgfpathrectangle{\pgfqpoint{0.539299in}{0.078740in}}{\pgfqpoint{7.842520in}{7.842520in}}%
\pgfusepath{clip}%
\pgfsetbuttcap%
\pgfsetroundjoin%
\definecolor{currentfill}{rgb}{0.616293,0.852709,0.230052}%
\pgfsetfillcolor{currentfill}%
\pgfsetlinewidth{0.000000pt}%
\definecolor{currentstroke}{rgb}{0.296479,0.761561,0.424223}%
\pgfsetstrokecolor{currentstroke}%
\pgfsetdash{}{0pt}%
\pgfpathmoveto{\pgfqpoint{3.891582in}{5.202680in}}%
\pgfpathlineto{\pgfqpoint{3.976785in}{5.127804in}}%
\pgfpathlineto{\pgfqpoint{3.834801in}{5.295343in}}%
\pgfpathclose%
\pgfusepath{fill}%
\end{pgfscope}%
\begin{pgfscope}%
\pgfpathrectangle{\pgfqpoint{0.539299in}{0.078740in}}{\pgfqpoint{7.842520in}{7.842520in}}%
\pgfusepath{clip}%
\pgfsetbuttcap%
\pgfsetroundjoin%
\definecolor{currentfill}{rgb}{0.171176,0.452530,0.557965}%
\pgfsetfillcolor{currentfill}%
\pgfsetlinewidth{0.000000pt}%
\definecolor{currentstroke}{rgb}{0.304148,0.764704,0.419943}%
\pgfsetstrokecolor{currentstroke}%
\pgfsetdash{}{0pt}%
\pgfpathmoveto{\pgfqpoint{5.092313in}{3.347287in}}%
\pgfpathlineto{\pgfqpoint{5.234327in}{3.134099in}}%
\pgfpathlineto{\pgfqpoint{5.313135in}{3.090200in}}%
\pgfpathclose%
\pgfusepath{fill}%
\end{pgfscope}%
\begin{pgfscope}%
\pgfpathrectangle{\pgfqpoint{0.539299in}{0.078740in}}{\pgfqpoint{7.842520in}{7.842520in}}%
\pgfusepath{clip}%
\pgfsetbuttcap%
\pgfsetroundjoin%
\definecolor{currentfill}{rgb}{0.119738,0.603785,0.541400}%
\pgfsetfillcolor{currentfill}%
\pgfsetlinewidth{0.000000pt}%
\definecolor{currentstroke}{rgb}{0.311925,0.767822,0.415586}%
\pgfsetstrokecolor{currentstroke}%
\pgfsetdash{}{0pt}%
\pgfpathmoveto{\pgfqpoint{2.013886in}{3.822753in}}%
\pgfpathlineto{\pgfqpoint{2.094863in}{4.143974in}}%
\pgfpathlineto{\pgfqpoint{1.883808in}{3.652469in}}%
\pgfpathclose%
\pgfusepath{fill}%
\end{pgfscope}%
\begin{pgfscope}%
\pgfpathrectangle{\pgfqpoint{0.539299in}{0.078740in}}{\pgfqpoint{7.842520in}{7.842520in}}%
\pgfusepath{clip}%
\pgfsetbuttcap%
\pgfsetroundjoin%
\definecolor{currentfill}{rgb}{0.129933,0.559582,0.551864}%
\pgfsetfillcolor{currentfill}%
\pgfsetlinewidth{0.000000pt}%
\definecolor{currentstroke}{rgb}{0.319809,0.770914,0.411152}%
\pgfsetstrokecolor{currentstroke}%
\pgfsetdash{}{0pt}%
\pgfpathmoveto{\pgfqpoint{5.030239in}{3.504331in}}%
\pgfpathlineto{\pgfqpoint{4.888578in}{3.720803in}}%
\pgfpathlineto{\pgfqpoint{4.807756in}{3.788327in}}%
\pgfpathclose%
\pgfusepath{fill}%
\end{pgfscope}%
\begin{pgfscope}%
\pgfpathrectangle{\pgfqpoint{0.539299in}{0.078740in}}{\pgfqpoint{7.842520in}{7.842520in}}%
\pgfusepath{clip}%
\pgfsetbuttcap%
\pgfsetroundjoin%
\definecolor{currentfill}{rgb}{0.283072,0.130895,0.449241}%
\pgfsetfillcolor{currentfill}%
\pgfsetlinewidth{0.000000pt}%
\definecolor{currentstroke}{rgb}{0.327796,0.773980,0.406640}%
\pgfsetstrokecolor{currentstroke}%
\pgfsetdash{}{0pt}%
\pgfpathmoveto{\pgfqpoint{6.085388in}{2.000146in}}%
\pgfpathlineto{\pgfqpoint{6.008600in}{1.974963in}}%
\pgfpathlineto{\pgfqpoint{6.227580in}{1.841168in}}%
\pgfpathclose%
\pgfusepath{fill}%
\end{pgfscope}%
\begin{pgfscope}%
\pgfpathrectangle{\pgfqpoint{0.539299in}{0.078740in}}{\pgfqpoint{7.842520in}{7.842520in}}%
\pgfusepath{clip}%
\pgfsetbuttcap%
\pgfsetroundjoin%
\definecolor{currentfill}{rgb}{0.280868,0.160771,0.472899}%
\pgfsetfillcolor{currentfill}%
\pgfsetlinewidth{0.000000pt}%
\definecolor{currentstroke}{rgb}{0.335885,0.777018,0.402049}%
\pgfsetstrokecolor{currentstroke}%
\pgfsetdash{}{0pt}%
\pgfpathmoveto{\pgfqpoint{5.943414in}{2.169358in}}%
\pgfpathlineto{\pgfqpoint{6.008600in}{1.974963in}}%
\pgfpathlineto{\pgfqpoint{6.085388in}{2.000146in}}%
\pgfpathclose%
\pgfusepath{fill}%
\end{pgfscope}%
\begin{pgfscope}%
\pgfpathrectangle{\pgfqpoint{0.539299in}{0.078740in}}{\pgfqpoint{7.842520in}{7.842520in}}%
\pgfusepath{clip}%
\pgfsetbuttcap%
\pgfsetroundjoin%
\definecolor{currentfill}{rgb}{0.278791,0.062145,0.386592}%
\pgfsetfillcolor{currentfill}%
\pgfsetlinewidth{0.000000pt}%
\definecolor{currentstroke}{rgb}{0.344074,0.780029,0.397381}%
\pgfsetstrokecolor{currentstroke}%
\pgfsetdash{}{0pt}%
\pgfpathmoveto{\pgfqpoint{6.227580in}{1.841168in}}%
\pgfpathlineto{\pgfqpoint{6.293516in}{1.634687in}}%
\pgfpathlineto{\pgfqpoint{6.370114in}{1.694823in}}%
\pgfpathclose%
\pgfusepath{fill}%
\end{pgfscope}%
\begin{pgfscope}%
\pgfpathrectangle{\pgfqpoint{0.539299in}{0.078740in}}{\pgfqpoint{7.842520in}{7.842520in}}%
\pgfusepath{clip}%
\pgfsetbuttcap%
\pgfsetroundjoin%
\definecolor{currentfill}{rgb}{0.352360,0.783011,0.392636}%
\pgfsetfillcolor{currentfill}%
\pgfsetlinewidth{0.000000pt}%
\definecolor{currentstroke}{rgb}{0.352360,0.783011,0.392636}%
\pgfsetstrokecolor{currentstroke}%
\pgfsetdash{}{0pt}%
\pgfpathmoveto{\pgfqpoint{4.319824in}{4.607800in}}%
\pgfpathlineto{\pgfqpoint{4.260950in}{4.734983in}}%
\pgfpathlineto{\pgfqpoint{4.177132in}{4.819968in}}%
\pgfpathclose%
\pgfusepath{fill}%
\end{pgfscope}%
\begin{pgfscope}%
\pgfpathrectangle{\pgfqpoint{0.539299in}{0.078740in}}{\pgfqpoint{7.842520in}{7.842520in}}%
\pgfusepath{clip}%
\pgfsetbuttcap%
\pgfsetroundjoin%
\definecolor{currentfill}{rgb}{0.265145,0.232956,0.516599}%
\pgfsetfillcolor{currentfill}%
\pgfsetlinewidth{0.000000pt}%
\definecolor{currentstroke}{rgb}{0.360741,0.785964,0.387814}%
\pgfsetstrokecolor{currentstroke}%
\pgfsetdash{}{0pt}%
\pgfpathmoveto{\pgfqpoint{5.943414in}{2.169358in}}%
\pgfpathlineto{\pgfqpoint{5.801571in}{2.347348in}}%
\pgfpathlineto{\pgfqpoint{5.724190in}{2.352097in}}%
\pgfpathclose%
\pgfusepath{fill}%
\end{pgfscope}%
\begin{pgfscope}%
\pgfpathrectangle{\pgfqpoint{0.539299in}{0.078740in}}{\pgfqpoint{7.842520in}{7.842520in}}%
\pgfusepath{clip}%
\pgfsetbuttcap%
\pgfsetroundjoin%
\definecolor{currentfill}{rgb}{0.814576,0.883393,0.110347}%
\pgfsetfillcolor{currentfill}%
\pgfsetlinewidth{0.000000pt}%
\definecolor{currentstroke}{rgb}{0.369214,0.788888,0.382914}%
\pgfsetstrokecolor{currentstroke}%
\pgfsetdash{}{0pt}%
\pgfpathmoveto{\pgfqpoint{3.465333in}{5.584033in}}%
\pgfpathlineto{\pgfqpoint{3.606833in}{5.491393in}}%
\pgfpathlineto{\pgfqpoint{3.693131in}{5.434822in}}%
\pgfpathclose%
\pgfusepath{fill}%
\end{pgfscope}%
\begin{pgfscope}%
\pgfpathrectangle{\pgfqpoint{0.539299in}{0.078740in}}{\pgfqpoint{7.842520in}{7.842520in}}%
\pgfusepath{clip}%
\pgfsetbuttcap%
\pgfsetroundjoin%
\definecolor{currentfill}{rgb}{0.119423,0.611141,0.538982}%
\pgfsetfillcolor{currentfill}%
\pgfsetlinewidth{0.000000pt}%
\definecolor{currentstroke}{rgb}{0.377779,0.791781,0.377939}%
\pgfsetstrokecolor{currentstroke}%
\pgfsetdash{}{0pt}%
\pgfpathmoveto{\pgfqpoint{4.888578in}{3.720803in}}%
\pgfpathlineto{\pgfqpoint{4.746722in}{3.941721in}}%
\pgfpathlineto{\pgfqpoint{4.665156in}{4.013355in}}%
\pgfpathclose%
\pgfusepath{fill}%
\end{pgfscope}%
\begin{pgfscope}%
\pgfpathrectangle{\pgfqpoint{0.539299in}{0.078740in}}{\pgfqpoint{7.842520in}{7.842520in}}%
\pgfusepath{clip}%
\pgfsetbuttcap%
\pgfsetroundjoin%
\definecolor{currentfill}{rgb}{0.144759,0.519093,0.556572}%
\pgfsetfillcolor{currentfill}%
\pgfsetlinewidth{0.000000pt}%
\definecolor{currentstroke}{rgb}{0.386433,0.794644,0.372886}%
\pgfsetstrokecolor{currentstroke}%
\pgfsetdash{}{0pt}%
\pgfpathmoveto{\pgfqpoint{5.030239in}{3.504331in}}%
\pgfpathlineto{\pgfqpoint{4.950134in}{3.565760in}}%
\pgfpathlineto{\pgfqpoint{5.092313in}{3.347287in}}%
\pgfpathclose%
\pgfusepath{fill}%
\end{pgfscope}%
\begin{pgfscope}%
\pgfpathrectangle{\pgfqpoint{0.539299in}{0.078740in}}{\pgfqpoint{7.842520in}{7.842520in}}%
\pgfusepath{clip}%
\pgfsetbuttcap%
\pgfsetroundjoin%
\definecolor{currentfill}{rgb}{0.203063,0.379716,0.553925}%
\pgfsetfillcolor{currentfill}%
\pgfsetlinewidth{0.000000pt}%
\definecolor{currentstroke}{rgb}{0.395174,0.797475,0.367757}%
\pgfsetstrokecolor{currentstroke}%
\pgfsetdash{}{0pt}%
\pgfpathmoveto{\pgfqpoint{2.058527in}{2.453906in}}%
\pgfpathlineto{\pgfqpoint{2.129568in}{3.063212in}}%
\pgfpathlineto{\pgfqpoint{1.993699in}{3.037069in}}%
\pgfpathclose%
\pgfusepath{fill}%
\end{pgfscope}%
\begin{pgfscope}%
\pgfpathrectangle{\pgfqpoint{0.539299in}{0.078740in}}{\pgfqpoint{7.842520in}{7.842520in}}%
\pgfusepath{clip}%
\pgfsetbuttcap%
\pgfsetroundjoin%
\definecolor{currentfill}{rgb}{0.421908,0.805774,0.351910}%
\pgfsetfillcolor{currentfill}%
\pgfsetlinewidth{0.000000pt}%
\definecolor{currentstroke}{rgb}{0.404001,0.800275,0.362552}%
\pgfsetstrokecolor{currentstroke}%
\pgfsetdash{}{0pt}%
\pgfpathmoveto{\pgfqpoint{4.177132in}{4.819968in}}%
\pgfpathlineto{\pgfqpoint{4.260950in}{4.734983in}}%
\pgfpathlineto{\pgfqpoint{4.034345in}{5.020034in}}%
\pgfpathclose%
\pgfusepath{fill}%
\end{pgfscope}%
\begin{pgfscope}%
\pgfpathrectangle{\pgfqpoint{0.539299in}{0.078740in}}{\pgfqpoint{7.842520in}{7.842520in}}%
\pgfusepath{clip}%
\pgfsetbuttcap%
\pgfsetroundjoin%
\definecolor{currentfill}{rgb}{0.244972,0.287675,0.537260}%
\pgfsetfillcolor{currentfill}%
\pgfsetlinewidth{0.000000pt}%
\definecolor{currentstroke}{rgb}{0.412913,0.803041,0.357269}%
\pgfsetstrokecolor{currentstroke}%
\pgfsetdash{}{0pt}%
\pgfpathmoveto{\pgfqpoint{5.581983in}{2.550780in}}%
\pgfpathlineto{\pgfqpoint{5.801571in}{2.347348in}}%
\pgfpathlineto{\pgfqpoint{5.659792in}{2.533268in}}%
\pgfpathclose%
\pgfusepath{fill}%
\end{pgfscope}%
\begin{pgfscope}%
\pgfpathrectangle{\pgfqpoint{0.539299in}{0.078740in}}{\pgfqpoint{7.842520in}{7.842520in}}%
\pgfusepath{clip}%
\pgfsetbuttcap%
\pgfsetroundjoin%
\definecolor{currentfill}{rgb}{0.772852,0.877868,0.131109}%
\pgfsetfillcolor{currentfill}%
\pgfsetlinewidth{0.000000pt}%
\definecolor{currentstroke}{rgb}{0.421908,0.805774,0.351910}%
\pgfsetstrokecolor{currentstroke}%
\pgfsetdash{}{0pt}%
\pgfpathmoveto{\pgfqpoint{3.693131in}{5.434822in}}%
\pgfpathlineto{\pgfqpoint{3.606833in}{5.491393in}}%
\pgfpathlineto{\pgfqpoint{3.749006in}{5.361964in}}%
\pgfpathclose%
\pgfusepath{fill}%
\end{pgfscope}%
\begin{pgfscope}%
\pgfpathrectangle{\pgfqpoint{0.539299in}{0.078740in}}{\pgfqpoint{7.842520in}{7.842520in}}%
\pgfusepath{clip}%
\pgfsetbuttcap%
\pgfsetroundjoin%
\definecolor{currentfill}{rgb}{0.229739,0.322361,0.545706}%
\pgfsetfillcolor{currentfill}%
\pgfsetlinewidth{0.000000pt}%
\definecolor{currentstroke}{rgb}{0.430983,0.808473,0.346476}%
\pgfsetstrokecolor{currentstroke}%
\pgfsetdash{}{0pt}%
\pgfpathmoveto{\pgfqpoint{5.659792in}{2.533268in}}%
\pgfpathlineto{\pgfqpoint{5.518023in}{2.726615in}}%
\pgfpathlineto{\pgfqpoint{5.581983in}{2.550780in}}%
\pgfpathclose%
\pgfusepath{fill}%
\end{pgfscope}%
\begin{pgfscope}%
\pgfpathrectangle{\pgfqpoint{0.539299in}{0.078740in}}{\pgfqpoint{7.842520in}{7.842520in}}%
\pgfusepath{clip}%
\pgfsetbuttcap%
\pgfsetroundjoin%
\definecolor{currentfill}{rgb}{0.131172,0.555899,0.552459}%
\pgfsetfillcolor{currentfill}%
\pgfsetlinewidth{0.000000pt}%
\definecolor{currentstroke}{rgb}{0.440137,0.811138,0.340967}%
\pgfsetstrokecolor{currentstroke}%
\pgfsetdash{}{0pt}%
\pgfpathmoveto{\pgfqpoint{5.030239in}{3.504331in}}%
\pgfpathlineto{\pgfqpoint{4.807756in}{3.788327in}}%
\pgfpathlineto{\pgfqpoint{4.950134in}{3.565760in}}%
\pgfpathclose%
\pgfusepath{fill}%
\end{pgfscope}%
\begin{pgfscope}%
\pgfpathrectangle{\pgfqpoint{0.539299in}{0.078740in}}{\pgfqpoint{7.842520in}{7.842520in}}%
\pgfusepath{clip}%
\pgfsetbuttcap%
\pgfsetroundjoin%
\definecolor{currentfill}{rgb}{0.154815,0.493313,0.557840}%
\pgfsetfillcolor{currentfill}%
\pgfsetlinewidth{0.000000pt}%
\definecolor{currentstroke}{rgb}{0.449368,0.813768,0.335384}%
\pgfsetstrokecolor{currentstroke}%
\pgfsetdash{}{0pt}%
\pgfpathmoveto{\pgfqpoint{1.993699in}{3.037069in}}%
\pgfpathlineto{\pgfqpoint{2.068708in}{3.529775in}}%
\pgfpathlineto{\pgfqpoint{1.935199in}{3.440713in}}%
\pgfpathclose%
\pgfusepath{fill}%
\end{pgfscope}%
\begin{pgfscope}%
\pgfpathrectangle{\pgfqpoint{0.539299in}{0.078740in}}{\pgfqpoint{7.842520in}{7.842520in}}%
\pgfusepath{clip}%
\pgfsetbuttcap%
\pgfsetroundjoin%
\definecolor{currentfill}{rgb}{0.585678,0.846661,0.249897}%
\pgfsetfillcolor{currentfill}%
\pgfsetlinewidth{0.000000pt}%
\definecolor{currentstroke}{rgb}{0.458674,0.816363,0.329727}%
\pgfsetstrokecolor{currentstroke}%
\pgfsetdash{}{0pt}%
\pgfpathmoveto{\pgfqpoint{3.976785in}{5.127804in}}%
\pgfpathlineto{\pgfqpoint{3.891582in}{5.202680in}}%
\pgfpathlineto{\pgfqpoint{4.034345in}{5.020034in}}%
\pgfpathclose%
\pgfusepath{fill}%
\end{pgfscope}%
\begin{pgfscope}%
\pgfpathrectangle{\pgfqpoint{0.539299in}{0.078740in}}{\pgfqpoint{7.842520in}{7.842520in}}%
\pgfusepath{clip}%
\pgfsetbuttcap%
\pgfsetroundjoin%
\definecolor{currentfill}{rgb}{0.688944,0.865448,0.182725}%
\pgfsetfillcolor{currentfill}%
\pgfsetlinewidth{0.000000pt}%
\definecolor{currentstroke}{rgb}{0.468053,0.818921,0.323998}%
\pgfsetstrokecolor{currentstroke}%
\pgfsetdash{}{0pt}%
\pgfpathmoveto{\pgfqpoint{3.891582in}{5.202680in}}%
\pgfpathlineto{\pgfqpoint{3.834801in}{5.295343in}}%
\pgfpathlineto{\pgfqpoint{3.749006in}{5.361964in}}%
\pgfpathclose%
\pgfusepath{fill}%
\end{pgfscope}%
\begin{pgfscope}%
\pgfpathrectangle{\pgfqpoint{0.539299in}{0.078740in}}{\pgfqpoint{7.842520in}{7.842520in}}%
\pgfusepath{clip}%
\pgfsetbuttcap%
\pgfsetroundjoin%
\definecolor{currentfill}{rgb}{0.281446,0.084320,0.407414}%
\pgfsetfillcolor{currentfill}%
\pgfsetlinewidth{0.000000pt}%
\definecolor{currentstroke}{rgb}{0.477504,0.821444,0.318195}%
\pgfsetstrokecolor{currentstroke}%
\pgfsetdash{}{0pt}%
\pgfpathmoveto{\pgfqpoint{6.150940in}{1.799026in}}%
\pgfpathlineto{\pgfqpoint{6.293516in}{1.634687in}}%
\pgfpathlineto{\pgfqpoint{6.227580in}{1.841168in}}%
\pgfpathclose%
\pgfusepath{fill}%
\end{pgfscope}%
\begin{pgfscope}%
\pgfpathrectangle{\pgfqpoint{0.539299in}{0.078740in}}{\pgfqpoint{7.842520in}{7.842520in}}%
\pgfusepath{clip}%
\pgfsetbuttcap%
\pgfsetroundjoin%
\definecolor{currentfill}{rgb}{0.283197,0.115680,0.436115}%
\pgfsetfillcolor{currentfill}%
\pgfsetlinewidth{0.000000pt}%
\definecolor{currentstroke}{rgb}{0.487026,0.823929,0.312321}%
\pgfsetstrokecolor{currentstroke}%
\pgfsetdash{}{0pt}%
\pgfpathmoveto{\pgfqpoint{6.227580in}{1.841168in}}%
\pgfpathlineto{\pgfqpoint{6.008600in}{1.974963in}}%
\pgfpathlineto{\pgfqpoint{6.150940in}{1.799026in}}%
\pgfpathclose%
\pgfusepath{fill}%
\end{pgfscope}%
\begin{pgfscope}%
\pgfpathrectangle{\pgfqpoint{0.539299in}{0.078740in}}{\pgfqpoint{7.842520in}{7.842520in}}%
\pgfusepath{clip}%
\pgfsetbuttcap%
\pgfsetroundjoin%
\definecolor{currentfill}{rgb}{0.140210,0.665859,0.513427}%
\pgfsetfillcolor{currentfill}%
\pgfsetlinewidth{0.000000pt}%
\definecolor{currentstroke}{rgb}{0.496615,0.826376,0.306377}%
\pgfsetstrokecolor{currentstroke}%
\pgfsetdash{}{0pt}%
\pgfpathmoveto{\pgfqpoint{4.604647in}{4.165055in}}%
\pgfpathlineto{\pgfqpoint{4.522320in}{4.238703in}}%
\pgfpathlineto{\pgfqpoint{4.746722in}{3.941721in}}%
\pgfpathclose%
\pgfusepath{fill}%
\end{pgfscope}%
\begin{pgfscope}%
\pgfpathrectangle{\pgfqpoint{0.539299in}{0.078740in}}{\pgfqpoint{7.842520in}{7.842520in}}%
\pgfusepath{clip}%
\pgfsetbuttcap%
\pgfsetroundjoin%
\definecolor{currentfill}{rgb}{0.876168,0.891125,0.095250}%
\pgfsetfillcolor{currentfill}%
\pgfsetlinewidth{0.000000pt}%
\definecolor{currentstroke}{rgb}{0.506271,0.828786,0.300362}%
\pgfsetstrokecolor{currentstroke}%
\pgfsetdash{}{0pt}%
\pgfpathmoveto{\pgfqpoint{3.010887in}{5.591305in}}%
\pgfpathlineto{\pgfqpoint{3.098325in}{5.625023in}}%
\pgfpathlineto{\pgfqpoint{2.961006in}{5.551827in}}%
\pgfpathclose%
\pgfusepath{fill}%
\end{pgfscope}%
\begin{pgfscope}%
\pgfpathrectangle{\pgfqpoint{0.539299in}{0.078740in}}{\pgfqpoint{7.842520in}{7.842520in}}%
\pgfusepath{clip}%
\pgfsetbuttcap%
\pgfsetroundjoin%
\definecolor{currentfill}{rgb}{0.855810,0.888601,0.097452}%
\pgfsetfillcolor{currentfill}%
\pgfsetlinewidth{0.000000pt}%
\definecolor{currentstroke}{rgb}{0.515992,0.831158,0.294279}%
\pgfsetstrokecolor{currentstroke}%
\pgfsetdash{}{0pt}%
\pgfpathmoveto{\pgfqpoint{2.961006in}{5.551827in}}%
\pgfpathlineto{\pgfqpoint{2.873601in}{5.508440in}}%
\pgfpathlineto{\pgfqpoint{3.010887in}{5.591305in}}%
\pgfpathclose%
\pgfusepath{fill}%
\end{pgfscope}%
\begin{pgfscope}%
\pgfpathrectangle{\pgfqpoint{0.539299in}{0.078740in}}{\pgfqpoint{7.842520in}{7.842520in}}%
\pgfusepath{clip}%
\pgfsetbuttcap%
\pgfsetroundjoin%
\definecolor{currentfill}{rgb}{0.277134,0.185228,0.489898}%
\pgfsetfillcolor{currentfill}%
\pgfsetlinewidth{0.000000pt}%
\definecolor{currentstroke}{rgb}{0.525776,0.833491,0.288127}%
\pgfsetstrokecolor{currentstroke}%
\pgfsetdash{}{0pt}%
\pgfpathmoveto{\pgfqpoint{5.866377in}{2.159848in}}%
\pgfpathlineto{\pgfqpoint{6.008600in}{1.974963in}}%
\pgfpathlineto{\pgfqpoint{5.943414in}{2.169358in}}%
\pgfpathclose%
\pgfusepath{fill}%
\end{pgfscope}%
\begin{pgfscope}%
\pgfpathrectangle{\pgfqpoint{0.539299in}{0.078740in}}{\pgfqpoint{7.842520in}{7.842520in}}%
\pgfusepath{clip}%
\pgfsetbuttcap%
\pgfsetroundjoin%
\definecolor{currentfill}{rgb}{0.223925,0.334994,0.548053}%
\pgfsetfillcolor{currentfill}%
\pgfsetlinewidth{0.000000pt}%
\definecolor{currentstroke}{rgb}{0.535621,0.835785,0.281908}%
\pgfsetstrokecolor{currentstroke}%
\pgfsetdash{}{0pt}%
\pgfpathmoveto{\pgfqpoint{2.058527in}{2.453906in}}%
\pgfpathlineto{\pgfqpoint{2.195719in}{2.435884in}}%
\pgfpathlineto{\pgfqpoint{2.266737in}{3.071192in}}%
\pgfpathclose%
\pgfusepath{fill}%
\end{pgfscope}%
\begin{pgfscope}%
\pgfpathrectangle{\pgfqpoint{0.539299in}{0.078740in}}{\pgfqpoint{7.842520in}{7.842520in}}%
\pgfusepath{clip}%
\pgfsetbuttcap%
\pgfsetroundjoin%
\definecolor{currentfill}{rgb}{0.906311,0.894855,0.098125}%
\pgfsetfillcolor{currentfill}%
\pgfsetlinewidth{0.000000pt}%
\definecolor{currentstroke}{rgb}{0.545524,0.838039,0.275626}%
\pgfsetstrokecolor{currentstroke}%
\pgfsetdash{}{0pt}%
\pgfpathmoveto{\pgfqpoint{3.324832in}{5.632651in}}%
\pgfpathlineto{\pgfqpoint{3.098325in}{5.625023in}}%
\pgfpathlineto{\pgfqpoint{3.237559in}{5.639984in}}%
\pgfpathclose%
\pgfusepath{fill}%
\end{pgfscope}%
\begin{pgfscope}%
\pgfpathrectangle{\pgfqpoint{0.539299in}{0.078740in}}{\pgfqpoint{7.842520in}{7.842520in}}%
\pgfusepath{clip}%
\pgfsetbuttcap%
\pgfsetroundjoin%
\definecolor{currentfill}{rgb}{0.386433,0.794644,0.372886}%
\pgfsetfillcolor{currentfill}%
\pgfsetlinewidth{0.000000pt}%
\definecolor{currentstroke}{rgb}{0.555484,0.840254,0.269281}%
\pgfsetstrokecolor{currentstroke}%
\pgfsetdash{}{0pt}%
\pgfpathmoveto{\pgfqpoint{2.347219in}{4.810962in}}%
\pgfpathlineto{\pgfqpoint{2.261900in}{4.632062in}}%
\pgfpathlineto{\pgfqpoint{2.393562in}{4.830606in}}%
\pgfpathclose%
\pgfusepath{fill}%
\end{pgfscope}%
\begin{pgfscope}%
\pgfpathrectangle{\pgfqpoint{0.539299in}{0.078740in}}{\pgfqpoint{7.842520in}{7.842520in}}%
\pgfusepath{clip}%
\pgfsetbuttcap%
\pgfsetroundjoin%
\definecolor{currentfill}{rgb}{0.267968,0.223549,0.512008}%
\pgfsetfillcolor{currentfill}%
\pgfsetlinewidth{0.000000pt}%
\definecolor{currentstroke}{rgb}{0.565498,0.842430,0.262877}%
\pgfsetstrokecolor{currentstroke}%
\pgfsetdash{}{0pt}%
\pgfpathmoveto{\pgfqpoint{5.866377in}{2.159848in}}%
\pgfpathlineto{\pgfqpoint{5.943414in}{2.169358in}}%
\pgfpathlineto{\pgfqpoint{5.724190in}{2.352097in}}%
\pgfpathclose%
\pgfusepath{fill}%
\end{pgfscope}%
\begin{pgfscope}%
\pgfpathrectangle{\pgfqpoint{0.539299in}{0.078740in}}{\pgfqpoint{7.842520in}{7.842520in}}%
\pgfusepath{clip}%
\pgfsetbuttcap%
\pgfsetroundjoin%
\definecolor{currentfill}{rgb}{0.197636,0.391528,0.554969}%
\pgfsetfillcolor{currentfill}%
\pgfsetlinewidth{0.000000pt}%
\definecolor{currentstroke}{rgb}{0.575563,0.844566,0.256415}%
\pgfsetstrokecolor{currentstroke}%
\pgfsetdash{}{0pt}%
\pgfpathmoveto{\pgfqpoint{5.518023in}{2.726615in}}%
\pgfpathlineto{\pgfqpoint{5.376216in}{2.927023in}}%
\pgfpathlineto{\pgfqpoint{5.297333in}{2.965317in}}%
\pgfpathclose%
\pgfusepath{fill}%
\end{pgfscope}%
\begin{pgfscope}%
\pgfpathrectangle{\pgfqpoint{0.539299in}{0.078740in}}{\pgfqpoint{7.842520in}{7.842520in}}%
\pgfusepath{clip}%
\pgfsetbuttcap%
\pgfsetroundjoin%
\definecolor{currentfill}{rgb}{0.525776,0.833491,0.288127}%
\pgfsetfillcolor{currentfill}%
\pgfsetlinewidth{0.000000pt}%
\definecolor{currentstroke}{rgb}{0.585678,0.846661,0.249897}%
\pgfsetstrokecolor{currentstroke}%
\pgfsetdash{}{0pt}%
\pgfpathmoveto{\pgfqpoint{2.347219in}{4.810962in}}%
\pgfpathlineto{\pgfqpoint{2.478789in}{5.021565in}}%
\pgfpathlineto{\pgfqpoint{2.564916in}{5.170485in}}%
\pgfpathclose%
\pgfusepath{fill}%
\end{pgfscope}%
\begin{pgfscope}%
\pgfpathrectangle{\pgfqpoint{0.539299in}{0.078740in}}{\pgfqpoint{7.842520in}{7.842520in}}%
\pgfusepath{clip}%
\pgfsetbuttcap%
\pgfsetroundjoin%
\definecolor{currentfill}{rgb}{0.772852,0.877868,0.131109}%
\pgfsetfillcolor{currentfill}%
\pgfsetlinewidth{0.000000pt}%
\definecolor{currentstroke}{rgb}{0.595839,0.848717,0.243329}%
\pgfsetstrokecolor{currentstroke}%
\pgfsetdash{}{0pt}%
\pgfpathmoveto{\pgfqpoint{2.651662in}{5.282437in}}%
\pgfpathlineto{\pgfqpoint{2.786367in}{5.433811in}}%
\pgfpathlineto{\pgfqpoint{2.873601in}{5.508440in}}%
\pgfpathclose%
\pgfusepath{fill}%
\end{pgfscope}%
\begin{pgfscope}%
\pgfpathrectangle{\pgfqpoint{0.539299in}{0.078740in}}{\pgfqpoint{7.842520in}{7.842520in}}%
\pgfusepath{clip}%
\pgfsetbuttcap%
\pgfsetroundjoin%
\definecolor{currentfill}{rgb}{0.119512,0.607464,0.540218}%
\pgfsetfillcolor{currentfill}%
\pgfsetlinewidth{0.000000pt}%
\definecolor{currentstroke}{rgb}{0.606045,0.850733,0.236712}%
\pgfsetstrokecolor{currentstroke}%
\pgfsetdash{}{0pt}%
\pgfpathmoveto{\pgfqpoint{4.665156in}{4.013355in}}%
\pgfpathlineto{\pgfqpoint{4.807756in}{3.788327in}}%
\pgfpathlineto{\pgfqpoint{4.888578in}{3.720803in}}%
\pgfpathclose%
\pgfusepath{fill}%
\end{pgfscope}%
\begin{pgfscope}%
\pgfpathrectangle{\pgfqpoint{0.539299in}{0.078740in}}{\pgfqpoint{7.842520in}{7.842520in}}%
\pgfusepath{clip}%
\pgfsetbuttcap%
\pgfsetroundjoin%
\definecolor{currentfill}{rgb}{0.122606,0.585371,0.546557}%
\pgfsetfillcolor{currentfill}%
\pgfsetlinewidth{0.000000pt}%
\definecolor{currentstroke}{rgb}{0.616293,0.852709,0.230052}%
\pgfsetstrokecolor{currentstroke}%
\pgfsetdash{}{0pt}%
\pgfpathmoveto{\pgfqpoint{2.013886in}{3.822753in}}%
\pgfpathlineto{\pgfqpoint{1.935199in}{3.440713in}}%
\pgfpathlineto{\pgfqpoint{2.146689in}{3.948369in}}%
\pgfpathclose%
\pgfusepath{fill}%
\end{pgfscope}%
\begin{pgfscope}%
\pgfpathrectangle{\pgfqpoint{0.539299in}{0.078740in}}{\pgfqpoint{7.842520in}{7.842520in}}%
\pgfusepath{clip}%
\pgfsetbuttcap%
\pgfsetroundjoin%
\definecolor{currentfill}{rgb}{0.248629,0.278775,0.534556}%
\pgfsetfillcolor{currentfill}%
\pgfsetlinewidth{0.000000pt}%
\definecolor{currentstroke}{rgb}{0.626579,0.854645,0.223353}%
\pgfsetstrokecolor{currentstroke}%
\pgfsetdash{}{0pt}%
\pgfpathmoveto{\pgfqpoint{5.581983in}{2.550780in}}%
\pgfpathlineto{\pgfqpoint{5.724190in}{2.352097in}}%
\pgfpathlineto{\pgfqpoint{5.801571in}{2.347348in}}%
\pgfpathclose%
\pgfusepath{fill}%
\end{pgfscope}%
\begin{pgfscope}%
\pgfpathrectangle{\pgfqpoint{0.539299in}{0.078740in}}{\pgfqpoint{7.842520in}{7.842520in}}%
\pgfusepath{clip}%
\pgfsetbuttcap%
\pgfsetroundjoin%
\definecolor{currentfill}{rgb}{0.182256,0.426184,0.557120}%
\pgfsetfillcolor{currentfill}%
\pgfsetlinewidth{0.000000pt}%
\definecolor{currentstroke}{rgb}{0.636902,0.856542,0.216620}%
\pgfsetstrokecolor{currentstroke}%
\pgfsetdash{}{0pt}%
\pgfpathmoveto{\pgfqpoint{5.297333in}{2.965317in}}%
\pgfpathlineto{\pgfqpoint{5.376216in}{2.927023in}}%
\pgfpathlineto{\pgfqpoint{5.234327in}{3.134099in}}%
\pgfpathclose%
\pgfusepath{fill}%
\end{pgfscope}%
\begin{pgfscope}%
\pgfpathrectangle{\pgfqpoint{0.539299in}{0.078740in}}{\pgfqpoint{7.842520in}{7.842520in}}%
\pgfusepath{clip}%
\pgfsetbuttcap%
\pgfsetroundjoin%
\definecolor{currentfill}{rgb}{0.202219,0.715272,0.476084}%
\pgfsetfillcolor{currentfill}%
\pgfsetlinewidth{0.000000pt}%
\definecolor{currentstroke}{rgb}{0.647257,0.858400,0.209861}%
\pgfsetstrokecolor{currentstroke}%
\pgfsetdash{}{0pt}%
\pgfpathmoveto{\pgfqpoint{4.604647in}{4.165055in}}%
\pgfpathlineto{\pgfqpoint{4.462343in}{4.388175in}}%
\pgfpathlineto{\pgfqpoint{4.379257in}{4.461675in}}%
\pgfpathclose%
\pgfusepath{fill}%
\end{pgfscope}%
\begin{pgfscope}%
\pgfpathrectangle{\pgfqpoint{0.539299in}{0.078740in}}{\pgfqpoint{7.842520in}{7.842520in}}%
\pgfusepath{clip}%
\pgfsetbuttcap%
\pgfsetroundjoin%
\definecolor{currentfill}{rgb}{0.163625,0.471133,0.558148}%
\pgfsetfillcolor{currentfill}%
\pgfsetlinewidth{0.000000pt}%
\definecolor{currentstroke}{rgb}{0.657642,0.860219,0.203082}%
\pgfsetstrokecolor{currentstroke}%
\pgfsetdash{}{0pt}%
\pgfpathmoveto{\pgfqpoint{1.993699in}{3.037069in}}%
\pgfpathlineto{\pgfqpoint{2.129568in}{3.063212in}}%
\pgfpathlineto{\pgfqpoint{2.068708in}{3.529775in}}%
\pgfpathclose%
\pgfusepath{fill}%
\end{pgfscope}%
\begin{pgfscope}%
\pgfpathrectangle{\pgfqpoint{0.539299in}{0.078740in}}{\pgfqpoint{7.842520in}{7.842520in}}%
\pgfusepath{clip}%
\pgfsetbuttcap%
\pgfsetroundjoin%
\definecolor{currentfill}{rgb}{0.896320,0.893616,0.096335}%
\pgfsetfillcolor{currentfill}%
\pgfsetlinewidth{0.000000pt}%
\definecolor{currentstroke}{rgb}{0.668054,0.861999,0.196293}%
\pgfsetstrokecolor{currentstroke}%
\pgfsetdash{}{0pt}%
\pgfpathmoveto{\pgfqpoint{3.465333in}{5.584033in}}%
\pgfpathlineto{\pgfqpoint{3.324832in}{5.632651in}}%
\pgfpathlineto{\pgfqpoint{3.378279in}{5.603577in}}%
\pgfpathclose%
\pgfusepath{fill}%
\end{pgfscope}%
\begin{pgfscope}%
\pgfpathrectangle{\pgfqpoint{0.539299in}{0.078740in}}{\pgfqpoint{7.842520in}{7.842520in}}%
\pgfusepath{clip}%
\pgfsetbuttcap%
\pgfsetroundjoin%
\definecolor{currentfill}{rgb}{0.197636,0.391528,0.554969}%
\pgfsetfillcolor{currentfill}%
\pgfsetlinewidth{0.000000pt}%
\definecolor{currentstroke}{rgb}{0.678489,0.863742,0.189503}%
\pgfsetstrokecolor{currentstroke}%
\pgfsetdash{}{0pt}%
\pgfpathmoveto{\pgfqpoint{2.266737in}{3.071192in}}%
\pgfpathlineto{\pgfqpoint{2.129568in}{3.063212in}}%
\pgfpathlineto{\pgfqpoint{2.058527in}{2.453906in}}%
\pgfpathclose%
\pgfusepath{fill}%
\end{pgfscope}%
\begin{pgfscope}%
\pgfpathrectangle{\pgfqpoint{0.539299in}{0.078740in}}{\pgfqpoint{7.842520in}{7.842520in}}%
\pgfusepath{clip}%
\pgfsetbuttcap%
\pgfsetroundjoin%
\definecolor{currentfill}{rgb}{0.216210,0.351535,0.550627}%
\pgfsetfillcolor{currentfill}%
\pgfsetlinewidth{0.000000pt}%
\definecolor{currentstroke}{rgb}{0.688944,0.865448,0.182725}%
\pgfsetstrokecolor{currentstroke}%
\pgfsetdash{}{0pt}%
\pgfpathmoveto{\pgfqpoint{5.518023in}{2.726615in}}%
\pgfpathlineto{\pgfqpoint{5.439710in}{2.755331in}}%
\pgfpathlineto{\pgfqpoint{5.581983in}{2.550780in}}%
\pgfpathclose%
\pgfusepath{fill}%
\end{pgfscope}%
\begin{pgfscope}%
\pgfpathrectangle{\pgfqpoint{0.539299in}{0.078740in}}{\pgfqpoint{7.842520in}{7.842520in}}%
\pgfusepath{clip}%
\pgfsetbuttcap%
\pgfsetroundjoin%
\definecolor{currentfill}{rgb}{0.296479,0.761561,0.424223}%
\pgfsetfillcolor{currentfill}%
\pgfsetlinewidth{0.000000pt}%
\definecolor{currentstroke}{rgb}{0.699415,0.867117,0.175971}%
\pgfsetstrokecolor{currentstroke}%
\pgfsetdash{}{0pt}%
\pgfpathmoveto{\pgfqpoint{2.177668in}{4.411616in}}%
\pgfpathlineto{\pgfqpoint{2.309553in}{4.592125in}}%
\pgfpathlineto{\pgfqpoint{2.261900in}{4.632062in}}%
\pgfpathclose%
\pgfusepath{fill}%
\end{pgfscope}%
\begin{pgfscope}%
\pgfpathrectangle{\pgfqpoint{0.539299in}{0.078740in}}{\pgfqpoint{7.842520in}{7.842520in}}%
\pgfusepath{clip}%
\pgfsetbuttcap%
\pgfsetroundjoin%
\definecolor{currentfill}{rgb}{0.266941,0.748751,0.440573}%
\pgfsetfillcolor{currentfill}%
\pgfsetlinewidth{0.000000pt}%
\definecolor{currentstroke}{rgb}{0.709898,0.868751,0.169257}%
\pgfsetstrokecolor{currentstroke}%
\pgfsetdash{}{0pt}%
\pgfpathmoveto{\pgfqpoint{4.319824in}{4.607800in}}%
\pgfpathlineto{\pgfqpoint{4.379257in}{4.461675in}}%
\pgfpathlineto{\pgfqpoint{4.462343in}{4.388175in}}%
\pgfpathclose%
\pgfusepath{fill}%
\end{pgfscope}%
\begin{pgfscope}%
\pgfpathrectangle{\pgfqpoint{0.539299in}{0.078740in}}{\pgfqpoint{7.842520in}{7.842520in}}%
\pgfusepath{clip}%
\pgfsetbuttcap%
\pgfsetroundjoin%
\definecolor{currentfill}{rgb}{0.137339,0.662252,0.515571}%
\pgfsetfillcolor{currentfill}%
\pgfsetlinewidth{0.000000pt}%
\definecolor{currentstroke}{rgb}{0.720391,0.870350,0.162603}%
\pgfsetstrokecolor{currentstroke}%
\pgfsetdash{}{0pt}%
\pgfpathmoveto{\pgfqpoint{4.746722in}{3.941721in}}%
\pgfpathlineto{\pgfqpoint{4.522320in}{4.238703in}}%
\pgfpathlineto{\pgfqpoint{4.665156in}{4.013355in}}%
\pgfpathclose%
\pgfusepath{fill}%
\end{pgfscope}%
\begin{pgfscope}%
\pgfpathrectangle{\pgfqpoint{0.539299in}{0.078740in}}{\pgfqpoint{7.842520in}{7.842520in}}%
\pgfusepath{clip}%
\pgfsetbuttcap%
\pgfsetroundjoin%
\definecolor{currentfill}{rgb}{0.127568,0.566949,0.550556}%
\pgfsetfillcolor{currentfill}%
\pgfsetlinewidth{0.000000pt}%
\definecolor{currentstroke}{rgb}{0.730889,0.871916,0.156029}%
\pgfsetstrokecolor{currentstroke}%
\pgfsetdash{}{0pt}%
\pgfpathmoveto{\pgfqpoint{2.146689in}{3.948369in}}%
\pgfpathlineto{\pgfqpoint{1.935199in}{3.440713in}}%
\pgfpathlineto{\pgfqpoint{2.068708in}{3.529775in}}%
\pgfpathclose%
\pgfusepath{fill}%
\end{pgfscope}%
\begin{pgfscope}%
\pgfpathrectangle{\pgfqpoint{0.539299in}{0.078740in}}{\pgfqpoint{7.842520in}{7.842520in}}%
\pgfusepath{clip}%
\pgfsetbuttcap%
\pgfsetroundjoin%
\definecolor{currentfill}{rgb}{0.162142,0.474838,0.558140}%
\pgfsetfillcolor{currentfill}%
\pgfsetlinewidth{0.000000pt}%
\definecolor{currentstroke}{rgb}{0.741388,0.873449,0.149561}%
\pgfsetstrokecolor{currentstroke}%
\pgfsetdash{}{0pt}%
\pgfpathmoveto{\pgfqpoint{5.154816in}{3.180272in}}%
\pgfpathlineto{\pgfqpoint{5.234327in}{3.134099in}}%
\pgfpathlineto{\pgfqpoint{5.092313in}{3.347287in}}%
\pgfpathclose%
\pgfusepath{fill}%
\end{pgfscope}%
\begin{pgfscope}%
\pgfpathrectangle{\pgfqpoint{0.539299in}{0.078740in}}{\pgfqpoint{7.842520in}{7.842520in}}%
\pgfusepath{clip}%
\pgfsetbuttcap%
\pgfsetroundjoin%
\definecolor{currentfill}{rgb}{0.468053,0.818921,0.323998}%
\pgfsetfillcolor{currentfill}%
\pgfsetlinewidth{0.000000pt}%
\definecolor{currentstroke}{rgb}{0.751884,0.874951,0.143228}%
\pgfsetstrokecolor{currentstroke}%
\pgfsetdash{}{0pt}%
\pgfpathmoveto{\pgfqpoint{2.393562in}{4.830606in}}%
\pgfpathlineto{\pgfqpoint{2.478789in}{5.021565in}}%
\pgfpathlineto{\pgfqpoint{2.347219in}{4.810962in}}%
\pgfpathclose%
\pgfusepath{fill}%
\end{pgfscope}%
\begin{pgfscope}%
\pgfpathrectangle{\pgfqpoint{0.539299in}{0.078740in}}{\pgfqpoint{7.842520in}{7.842520in}}%
\pgfusepath{clip}%
\pgfsetbuttcap%
\pgfsetroundjoin%
\definecolor{currentfill}{rgb}{0.143303,0.669459,0.511215}%
\pgfsetfillcolor{currentfill}%
\pgfsetlinewidth{0.000000pt}%
\definecolor{currentstroke}{rgb}{0.762373,0.876424,0.137064}%
\pgfsetstrokecolor{currentstroke}%
\pgfsetdash{}{0pt}%
\pgfpathmoveto{\pgfqpoint{2.227125in}{4.300185in}}%
\pgfpathlineto{\pgfqpoint{2.094863in}{4.143974in}}%
\pgfpathlineto{\pgfqpoint{2.013886in}{3.822753in}}%
\pgfpathclose%
\pgfusepath{fill}%
\end{pgfscope}%
\begin{pgfscope}%
\pgfpathrectangle{\pgfqpoint{0.539299in}{0.078740in}}{\pgfqpoint{7.842520in}{7.842520in}}%
\pgfusepath{clip}%
\pgfsetbuttcap%
\pgfsetroundjoin%
\definecolor{currentfill}{rgb}{0.699415,0.867117,0.175971}%
\pgfsetfillcolor{currentfill}%
\pgfsetlinewidth{0.000000pt}%
\definecolor{currentstroke}{rgb}{0.772852,0.877868,0.131109}%
\pgfsetstrokecolor{currentstroke}%
\pgfsetdash{}{0pt}%
\pgfpathmoveto{\pgfqpoint{2.699533in}{5.323077in}}%
\pgfpathlineto{\pgfqpoint{2.651662in}{5.282437in}}%
\pgfpathlineto{\pgfqpoint{2.564916in}{5.170485in}}%
\pgfpathclose%
\pgfusepath{fill}%
\end{pgfscope}%
\begin{pgfscope}%
\pgfpathrectangle{\pgfqpoint{0.539299in}{0.078740in}}{\pgfqpoint{7.842520in}{7.842520in}}%
\pgfusepath{clip}%
\pgfsetbuttcap%
\pgfsetroundjoin%
\definecolor{currentfill}{rgb}{0.191090,0.708366,0.482284}%
\pgfsetfillcolor{currentfill}%
\pgfsetlinewidth{0.000000pt}%
\definecolor{currentstroke}{rgb}{0.783315,0.879285,0.125405}%
\pgfsetstrokecolor{currentstroke}%
\pgfsetdash{}{0pt}%
\pgfpathmoveto{\pgfqpoint{2.177668in}{4.411616in}}%
\pgfpathlineto{\pgfqpoint{2.094863in}{4.143974in}}%
\pgfpathlineto{\pgfqpoint{2.227125in}{4.300185in}}%
\pgfpathclose%
\pgfusepath{fill}%
\end{pgfscope}%
\begin{pgfscope}%
\pgfpathrectangle{\pgfqpoint{0.539299in}{0.078740in}}{\pgfqpoint{7.842520in}{7.842520in}}%
\pgfusepath{clip}%
\pgfsetbuttcap%
\pgfsetroundjoin%
\definecolor{currentfill}{rgb}{0.926106,0.897330,0.104071}%
\pgfsetfillcolor{currentfill}%
\pgfsetlinewidth{0.000000pt}%
\definecolor{currentstroke}{rgb}{0.793760,0.880678,0.120005}%
\pgfsetstrokecolor{currentstroke}%
\pgfsetdash{}{0pt}%
\pgfpathmoveto{\pgfqpoint{3.378279in}{5.603577in}}%
\pgfpathlineto{\pgfqpoint{3.324832in}{5.632651in}}%
\pgfpathlineto{\pgfqpoint{3.237559in}{5.639984in}}%
\pgfpathclose%
\pgfusepath{fill}%
\end{pgfscope}%
\begin{pgfscope}%
\pgfpathrectangle{\pgfqpoint{0.539299in}{0.078740in}}{\pgfqpoint{7.842520in}{7.842520in}}%
\pgfusepath{clip}%
\pgfsetbuttcap%
\pgfsetroundjoin%
\definecolor{currentfill}{rgb}{0.199430,0.387607,0.554642}%
\pgfsetfillcolor{currentfill}%
\pgfsetlinewidth{0.000000pt}%
\definecolor{currentstroke}{rgb}{0.804182,0.882046,0.114965}%
\pgfsetstrokecolor{currentstroke}%
\pgfsetdash{}{0pt}%
\pgfpathmoveto{\pgfqpoint{5.297333in}{2.965317in}}%
\pgfpathlineto{\pgfqpoint{5.439710in}{2.755331in}}%
\pgfpathlineto{\pgfqpoint{5.518023in}{2.726615in}}%
\pgfpathclose%
\pgfusepath{fill}%
\end{pgfscope}%
\begin{pgfscope}%
\pgfpathrectangle{\pgfqpoint{0.539299in}{0.078740in}}{\pgfqpoint{7.842520in}{7.842520in}}%
\pgfusepath{clip}%
\pgfsetbuttcap%
\pgfsetroundjoin%
\definecolor{currentfill}{rgb}{0.886271,0.892374,0.095374}%
\pgfsetfillcolor{currentfill}%
\pgfsetlinewidth{0.000000pt}%
\definecolor{currentstroke}{rgb}{0.814576,0.883393,0.110347}%
\pgfsetstrokecolor{currentstroke}%
\pgfsetdash{}{0pt}%
\pgfpathmoveto{\pgfqpoint{3.378279in}{5.603577in}}%
\pgfpathlineto{\pgfqpoint{3.606833in}{5.491393in}}%
\pgfpathlineto{\pgfqpoint{3.465333in}{5.584033in}}%
\pgfpathclose%
\pgfusepath{fill}%
\end{pgfscope}%
\begin{pgfscope}%
\pgfpathrectangle{\pgfqpoint{0.539299in}{0.078740in}}{\pgfqpoint{7.842520in}{7.842520in}}%
\pgfusepath{clip}%
\pgfsetbuttcap%
\pgfsetroundjoin%
\definecolor{currentfill}{rgb}{0.855810,0.888601,0.097452}%
\pgfsetfillcolor{currentfill}%
\pgfsetlinewidth{0.000000pt}%
\definecolor{currentstroke}{rgb}{0.824940,0.884720,0.106217}%
\pgfsetstrokecolor{currentstroke}%
\pgfsetdash{}{0pt}%
\pgfpathmoveto{\pgfqpoint{3.010887in}{5.591305in}}%
\pgfpathlineto{\pgfqpoint{2.873601in}{5.508440in}}%
\pgfpathlineto{\pgfqpoint{2.786367in}{5.433811in}}%
\pgfpathclose%
\pgfusepath{fill}%
\end{pgfscope}%
\begin{pgfscope}%
\pgfpathrectangle{\pgfqpoint{0.539299in}{0.078740in}}{\pgfqpoint{7.842520in}{7.842520in}}%
\pgfusepath{clip}%
\pgfsetbuttcap%
\pgfsetroundjoin%
\definecolor{currentfill}{rgb}{0.196571,0.711827,0.479221}%
\pgfsetfillcolor{currentfill}%
\pgfsetlinewidth{0.000000pt}%
\definecolor{currentstroke}{rgb}{0.835270,0.886029,0.102646}%
\pgfsetstrokecolor{currentstroke}%
\pgfsetdash{}{0pt}%
\pgfpathmoveto{\pgfqpoint{4.379257in}{4.461675in}}%
\pgfpathlineto{\pgfqpoint{4.522320in}{4.238703in}}%
\pgfpathlineto{\pgfqpoint{4.604647in}{4.165055in}}%
\pgfpathclose%
\pgfusepath{fill}%
\end{pgfscope}%
\begin{pgfscope}%
\pgfpathrectangle{\pgfqpoint{0.539299in}{0.078740in}}{\pgfqpoint{7.842520in}{7.842520in}}%
\pgfusepath{clip}%
\pgfsetbuttcap%
\pgfsetroundjoin%
\definecolor{currentfill}{rgb}{0.412913,0.803041,0.357269}%
\pgfsetfillcolor{currentfill}%
\pgfsetlinewidth{0.000000pt}%
\definecolor{currentstroke}{rgb}{0.845561,0.887322,0.099702}%
\pgfsetstrokecolor{currentstroke}%
\pgfsetdash{}{0pt}%
\pgfpathmoveto{\pgfqpoint{4.177132in}{4.819968in}}%
\pgfpathlineto{\pgfqpoint{4.092594in}{4.886710in}}%
\pgfpathlineto{\pgfqpoint{4.319824in}{4.607800in}}%
\pgfpathclose%
\pgfusepath{fill}%
\end{pgfscope}%
\begin{pgfscope}%
\pgfpathrectangle{\pgfqpoint{0.539299in}{0.078740in}}{\pgfqpoint{7.842520in}{7.842520in}}%
\pgfusepath{clip}%
\pgfsetbuttcap%
\pgfsetroundjoin%
\definecolor{currentfill}{rgb}{0.218130,0.347432,0.550038}%
\pgfsetfillcolor{currentfill}%
\pgfsetlinewidth{0.000000pt}%
\definecolor{currentstroke}{rgb}{0.855810,0.888601,0.097452}%
\pgfsetstrokecolor{currentstroke}%
\pgfsetdash{}{0pt}%
\pgfpathmoveto{\pgfqpoint{2.195719in}{2.435884in}}%
\pgfpathlineto{\pgfqpoint{2.333716in}{2.409916in}}%
\pgfpathlineto{\pgfqpoint{2.405032in}{3.063501in}}%
\pgfpathclose%
\pgfusepath{fill}%
\end{pgfscope}%
\begin{pgfscope}%
\pgfpathrectangle{\pgfqpoint{0.539299in}{0.078740in}}{\pgfqpoint{7.842520in}{7.842520in}}%
\pgfusepath{clip}%
\pgfsetbuttcap%
\pgfsetroundjoin%
\definecolor{currentfill}{rgb}{0.762373,0.876424,0.137064}%
\pgfsetfillcolor{currentfill}%
\pgfsetlinewidth{0.000000pt}%
\definecolor{currentstroke}{rgb}{0.866013,0.889868,0.095953}%
\pgfsetstrokecolor{currentstroke}%
\pgfsetdash{}{0pt}%
\pgfpathmoveto{\pgfqpoint{2.786367in}{5.433811in}}%
\pgfpathlineto{\pgfqpoint{2.651662in}{5.282437in}}%
\pgfpathlineto{\pgfqpoint{2.699533in}{5.323077in}}%
\pgfpathclose%
\pgfusepath{fill}%
\end{pgfscope}%
\begin{pgfscope}%
\pgfpathrectangle{\pgfqpoint{0.539299in}{0.078740in}}{\pgfqpoint{7.842520in}{7.842520in}}%
\pgfusepath{clip}%
\pgfsetbuttcap%
\pgfsetroundjoin%
\definecolor{currentfill}{rgb}{0.282327,0.094955,0.417331}%
\pgfsetfillcolor{currentfill}%
\pgfsetlinewidth{0.000000pt}%
\definecolor{currentstroke}{rgb}{0.876168,0.891125,0.095250}%
\pgfsetstrokecolor{currentstroke}%
\pgfsetdash{}{0pt}%
\pgfpathmoveto{\pgfqpoint{6.073966in}{1.765621in}}%
\pgfpathlineto{\pgfqpoint{6.293516in}{1.634687in}}%
\pgfpathlineto{\pgfqpoint{6.150940in}{1.799026in}}%
\pgfpathclose%
\pgfusepath{fill}%
\end{pgfscope}%
\begin{pgfscope}%
\pgfpathrectangle{\pgfqpoint{0.539299in}{0.078740in}}{\pgfqpoint{7.842520in}{7.842520in}}%
\pgfusepath{clip}%
\pgfsetbuttcap%
\pgfsetroundjoin%
\definecolor{currentfill}{rgb}{0.283072,0.130895,0.449241}%
\pgfsetfillcolor{currentfill}%
\pgfsetlinewidth{0.000000pt}%
\definecolor{currentstroke}{rgb}{0.886271,0.892374,0.095374}%
\pgfsetstrokecolor{currentstroke}%
\pgfsetdash{}{0pt}%
\pgfpathmoveto{\pgfqpoint{6.150940in}{1.799026in}}%
\pgfpathlineto{\pgfqpoint{6.008600in}{1.974963in}}%
\pgfpathlineto{\pgfqpoint{6.073966in}{1.765621in}}%
\pgfpathclose%
\pgfusepath{fill}%
\end{pgfscope}%
\begin{pgfscope}%
\pgfpathrectangle{\pgfqpoint{0.539299in}{0.078740in}}{\pgfqpoint{7.842520in}{7.842520in}}%
\pgfusepath{clip}%
\pgfsetbuttcap%
\pgfsetroundjoin%
\definecolor{currentfill}{rgb}{0.135066,0.544853,0.554029}%
\pgfsetfillcolor{currentfill}%
\pgfsetlinewidth{0.000000pt}%
\definecolor{currentstroke}{rgb}{0.896320,0.893616,0.096335}%
\pgfsetstrokecolor{currentstroke}%
\pgfsetdash{}{0pt}%
\pgfpathmoveto{\pgfqpoint{5.092313in}{3.347287in}}%
\pgfpathlineto{\pgfqpoint{4.950134in}{3.565760in}}%
\pgfpathlineto{\pgfqpoint{4.869240in}{3.622273in}}%
\pgfpathclose%
\pgfusepath{fill}%
\end{pgfscope}%
\begin{pgfscope}%
\pgfpathrectangle{\pgfqpoint{0.539299in}{0.078740in}}{\pgfqpoint{7.842520in}{7.842520in}}%
\pgfusepath{clip}%
\pgfsetbuttcap%
\pgfsetroundjoin%
\definecolor{currentfill}{rgb}{0.171176,0.452530,0.557965}%
\pgfsetfillcolor{currentfill}%
\pgfsetlinewidth{0.000000pt}%
\definecolor{currentstroke}{rgb}{0.906311,0.894855,0.098125}%
\pgfsetstrokecolor{currentstroke}%
\pgfsetdash{}{0pt}%
\pgfpathmoveto{\pgfqpoint{5.234327in}{3.134099in}}%
\pgfpathlineto{\pgfqpoint{5.154816in}{3.180272in}}%
\pgfpathlineto{\pgfqpoint{5.297333in}{2.965317in}}%
\pgfpathclose%
\pgfusepath{fill}%
\end{pgfscope}%
\begin{pgfscope}%
\pgfpathrectangle{\pgfqpoint{0.539299in}{0.078740in}}{\pgfqpoint{7.842520in}{7.842520in}}%
\pgfusepath{clip}%
\pgfsetbuttcap%
\pgfsetroundjoin%
\definecolor{currentfill}{rgb}{0.496615,0.826376,0.306377}%
\pgfsetfillcolor{currentfill}%
\pgfsetlinewidth{0.000000pt}%
\definecolor{currentstroke}{rgb}{0.916242,0.896091,0.100717}%
\pgfsetstrokecolor{currentstroke}%
\pgfsetdash{}{0pt}%
\pgfpathmoveto{\pgfqpoint{4.034345in}{5.020034in}}%
\pgfpathlineto{\pgfqpoint{4.092594in}{4.886710in}}%
\pgfpathlineto{\pgfqpoint{4.177132in}{4.819968in}}%
\pgfpathclose%
\pgfusepath{fill}%
\end{pgfscope}%
\begin{pgfscope}%
\pgfpathrectangle{\pgfqpoint{0.539299in}{0.078740in}}{\pgfqpoint{7.842520in}{7.842520in}}%
\pgfusepath{clip}%
\pgfsetbuttcap%
\pgfsetroundjoin%
\definecolor{currentfill}{rgb}{0.277134,0.185228,0.489898}%
\pgfsetfillcolor{currentfill}%
\pgfsetlinewidth{0.000000pt}%
\definecolor{currentstroke}{rgb}{0.926106,0.897330,0.104071}%
\pgfsetstrokecolor{currentstroke}%
\pgfsetdash{}{0pt}%
\pgfpathmoveto{\pgfqpoint{5.931443in}{1.959659in}}%
\pgfpathlineto{\pgfqpoint{6.008600in}{1.974963in}}%
\pgfpathlineto{\pgfqpoint{5.866377in}{2.159848in}}%
\pgfpathclose%
\pgfusepath{fill}%
\end{pgfscope}%
\begin{pgfscope}%
\pgfpathrectangle{\pgfqpoint{0.539299in}{0.078740in}}{\pgfqpoint{7.842520in}{7.842520in}}%
\pgfusepath{clip}%
\pgfsetbuttcap%
\pgfsetroundjoin%
\definecolor{currentfill}{rgb}{0.377779,0.791781,0.377939}%
\pgfsetfillcolor{currentfill}%
\pgfsetlinewidth{0.000000pt}%
\definecolor{currentstroke}{rgb}{0.935904,0.898570,0.108131}%
\pgfsetstrokecolor{currentstroke}%
\pgfsetdash{}{0pt}%
\pgfpathmoveto{\pgfqpoint{2.261900in}{4.632062in}}%
\pgfpathlineto{\pgfqpoint{2.309553in}{4.592125in}}%
\pgfpathlineto{\pgfqpoint{2.393562in}{4.830606in}}%
\pgfpathclose%
\pgfusepath{fill}%
\end{pgfscope}%
\begin{pgfscope}%
\pgfpathrectangle{\pgfqpoint{0.539299in}{0.078740in}}{\pgfqpoint{7.842520in}{7.842520in}}%
\pgfusepath{clip}%
\pgfsetbuttcap%
\pgfsetroundjoin%
\definecolor{currentfill}{rgb}{0.134692,0.658636,0.517649}%
\pgfsetfillcolor{currentfill}%
\pgfsetlinewidth{0.000000pt}%
\definecolor{currentstroke}{rgb}{0.945636,0.899815,0.112838}%
\pgfsetstrokecolor{currentstroke}%
\pgfsetdash{}{0pt}%
\pgfpathmoveto{\pgfqpoint{2.013886in}{3.822753in}}%
\pgfpathlineto{\pgfqpoint{2.146689in}{3.948369in}}%
\pgfpathlineto{\pgfqpoint{2.227125in}{4.300185in}}%
\pgfpathclose%
\pgfusepath{fill}%
\end{pgfscope}%
\begin{pgfscope}%
\pgfpathrectangle{\pgfqpoint{0.539299in}{0.078740in}}{\pgfqpoint{7.842520in}{7.842520in}}%
\pgfusepath{clip}%
\pgfsetbuttcap%
\pgfsetroundjoin%
\definecolor{currentfill}{rgb}{0.935904,0.898570,0.108131}%
\pgfsetfillcolor{currentfill}%
\pgfsetlinewidth{0.000000pt}%
\definecolor{currentstroke}{rgb}{0.955300,0.901065,0.118128}%
\pgfsetstrokecolor{currentstroke}%
\pgfsetdash{}{0pt}%
\pgfpathmoveto{\pgfqpoint{3.237559in}{5.639984in}}%
\pgfpathlineto{\pgfqpoint{3.098325in}{5.625023in}}%
\pgfpathlineto{\pgfqpoint{3.150177in}{5.616991in}}%
\pgfpathclose%
\pgfusepath{fill}%
\end{pgfscope}%
\begin{pgfscope}%
\pgfpathrectangle{\pgfqpoint{0.539299in}{0.078740in}}{\pgfqpoint{7.842520in}{7.842520in}}%
\pgfusepath{clip}%
\pgfsetbuttcap%
\pgfsetroundjoin%
\definecolor{currentfill}{rgb}{0.814576,0.883393,0.110347}%
\pgfsetfillcolor{currentfill}%
\pgfsetlinewidth{0.000000pt}%
\definecolor{currentstroke}{rgb}{0.964894,0.902323,0.123941}%
\pgfsetstrokecolor{currentstroke}%
\pgfsetdash{}{0pt}%
\pgfpathmoveto{\pgfqpoint{3.749006in}{5.361964in}}%
\pgfpathlineto{\pgfqpoint{3.606833in}{5.491393in}}%
\pgfpathlineto{\pgfqpoint{3.662699in}{5.404278in}}%
\pgfpathclose%
\pgfusepath{fill}%
\end{pgfscope}%
\begin{pgfscope}%
\pgfpathrectangle{\pgfqpoint{0.539299in}{0.078740in}}{\pgfqpoint{7.842520in}{7.842520in}}%
\pgfusepath{clip}%
\pgfsetbuttcap%
\pgfsetroundjoin%
\definecolor{currentfill}{rgb}{0.143343,0.522773,0.556295}%
\pgfsetfillcolor{currentfill}%
\pgfsetlinewidth{0.000000pt}%
\definecolor{currentstroke}{rgb}{0.974417,0.903590,0.130215}%
\pgfsetstrokecolor{currentstroke}%
\pgfsetdash{}{0pt}%
\pgfpathmoveto{\pgfqpoint{2.068708in}{3.529775in}}%
\pgfpathlineto{\pgfqpoint{2.129568in}{3.063212in}}%
\pgfpathlineto{\pgfqpoint{2.204165in}{3.588171in}}%
\pgfpathclose%
\pgfusepath{fill}%
\end{pgfscope}%
\begin{pgfscope}%
\pgfpathrectangle{\pgfqpoint{0.539299in}{0.078740in}}{\pgfqpoint{7.842520in}{7.842520in}}%
\pgfusepath{clip}%
\pgfsetbuttcap%
\pgfsetroundjoin%
\definecolor{currentfill}{rgb}{0.751884,0.874951,0.143228}%
\pgfsetfillcolor{currentfill}%
\pgfsetlinewidth{0.000000pt}%
\definecolor{currentstroke}{rgb}{0.983868,0.904867,0.136897}%
\pgfsetstrokecolor{currentstroke}%
\pgfsetdash{}{0pt}%
\pgfpathmoveto{\pgfqpoint{3.662699in}{5.404278in}}%
\pgfpathlineto{\pgfqpoint{3.891582in}{5.202680in}}%
\pgfpathlineto{\pgfqpoint{3.749006in}{5.361964in}}%
\pgfpathclose%
\pgfusepath{fill}%
\end{pgfscope}%
\begin{pgfscope}%
\pgfpathrectangle{\pgfqpoint{0.539299in}{0.078740in}}{\pgfqpoint{7.842520in}{7.842520in}}%
\pgfusepath{clip}%
\pgfsetbuttcap%
\pgfsetroundjoin%
\definecolor{currentfill}{rgb}{0.657642,0.860219,0.203082}%
\pgfsetfillcolor{currentfill}%
\pgfsetlinewidth{0.000000pt}%
\definecolor{currentstroke}{rgb}{0.993248,0.906157,0.143936}%
\pgfsetstrokecolor{currentstroke}%
\pgfsetdash{}{0pt}%
\pgfpathmoveto{\pgfqpoint{4.034345in}{5.020034in}}%
\pgfpathlineto{\pgfqpoint{3.891582in}{5.202680in}}%
\pgfpathlineto{\pgfqpoint{3.805790in}{5.254759in}}%
\pgfpathclose%
\pgfusepath{fill}%
\end{pgfscope}%
\begin{pgfscope}%
\pgfpathrectangle{\pgfqpoint{0.539299in}{0.078740in}}{\pgfqpoint{7.842520in}{7.842520in}}%
\pgfusepath{clip}%
\pgfsetbuttcap%
\pgfsetroundjoin%
\definecolor{currentfill}{rgb}{0.262138,0.242286,0.520837}%
\pgfsetfillcolor{currentfill}%
\pgfsetlinewidth{0.000000pt}%
\definecolor{currentstroke}{rgb}{0.267004,0.004874,0.329415}%
\pgfsetstrokecolor{currentstroke}%
\pgfsetdash{}{0pt}%
\pgfpathmoveto{\pgfqpoint{5.788906in}{2.159810in}}%
\pgfpathlineto{\pgfqpoint{5.866377in}{2.159848in}}%
\pgfpathlineto{\pgfqpoint{5.724190in}{2.352097in}}%
\pgfpathclose%
\pgfusepath{fill}%
\end{pgfscope}%
\begin{pgfscope}%
\pgfpathrectangle{\pgfqpoint{0.539299in}{0.078740in}}{\pgfqpoint{7.842520in}{7.842520in}}%
\pgfusepath{clip}%
\pgfsetbuttcap%
\pgfsetroundjoin%
\definecolor{currentfill}{rgb}{0.657642,0.860219,0.203082}%
\pgfsetfillcolor{currentfill}%
\pgfsetlinewidth{0.000000pt}%
\definecolor{currentstroke}{rgb}{0.268510,0.009605,0.335427}%
\pgfsetstrokecolor{currentstroke}%
\pgfsetdash{}{0pt}%
\pgfpathmoveto{\pgfqpoint{2.564916in}{5.170485in}}%
\pgfpathlineto{\pgfqpoint{2.478789in}{5.021565in}}%
\pgfpathlineto{\pgfqpoint{2.699533in}{5.323077in}}%
\pgfpathclose%
\pgfusepath{fill}%
\end{pgfscope}%
\begin{pgfscope}%
\pgfpathrectangle{\pgfqpoint{0.539299in}{0.078740in}}{\pgfqpoint{7.842520in}{7.842520in}}%
\pgfusepath{clip}%
\pgfsetbuttcap%
\pgfsetroundjoin%
\definecolor{currentfill}{rgb}{0.935904,0.898570,0.108131}%
\pgfsetfillcolor{currentfill}%
\pgfsetlinewidth{0.000000pt}%
\definecolor{currentstroke}{rgb}{0.269944,0.014625,0.341379}%
\pgfsetstrokecolor{currentstroke}%
\pgfsetdash{}{0pt}%
\pgfpathmoveto{\pgfqpoint{3.150177in}{5.616991in}}%
\pgfpathlineto{\pgfqpoint{3.098325in}{5.625023in}}%
\pgfpathlineto{\pgfqpoint{3.010887in}{5.591305in}}%
\pgfpathclose%
\pgfusepath{fill}%
\end{pgfscope}%
\begin{pgfscope}%
\pgfpathrectangle{\pgfqpoint{0.539299in}{0.078740in}}{\pgfqpoint{7.842520in}{7.842520in}}%
\pgfusepath{clip}%
\pgfsetbuttcap%
\pgfsetroundjoin%
\definecolor{currentfill}{rgb}{0.266941,0.748751,0.440573}%
\pgfsetfillcolor{currentfill}%
\pgfsetlinewidth{0.000000pt}%
\definecolor{currentstroke}{rgb}{0.271305,0.019942,0.347269}%
\pgfsetstrokecolor{currentstroke}%
\pgfsetdash{}{0pt}%
\pgfpathmoveto{\pgfqpoint{2.227125in}{4.300185in}}%
\pgfpathlineto{\pgfqpoint{2.309553in}{4.592125in}}%
\pgfpathlineto{\pgfqpoint{2.177668in}{4.411616in}}%
\pgfpathclose%
\pgfusepath{fill}%
\end{pgfscope}%
\begin{pgfscope}%
\pgfpathrectangle{\pgfqpoint{0.539299in}{0.078740in}}{\pgfqpoint{7.842520in}{7.842520in}}%
\pgfusepath{clip}%
\pgfsetbuttcap%
\pgfsetroundjoin%
\definecolor{currentfill}{rgb}{0.327796,0.773980,0.406640}%
\pgfsetfillcolor{currentfill}%
\pgfsetlinewidth{0.000000pt}%
\definecolor{currentstroke}{rgb}{0.272594,0.025563,0.353093}%
\pgfsetstrokecolor{currentstroke}%
\pgfsetdash{}{0pt}%
\pgfpathmoveto{\pgfqpoint{4.235995in}{4.678979in}}%
\pgfpathlineto{\pgfqpoint{4.379257in}{4.461675in}}%
\pgfpathlineto{\pgfqpoint{4.319824in}{4.607800in}}%
\pgfpathclose%
\pgfusepath{fill}%
\end{pgfscope}%
\begin{pgfscope}%
\pgfpathrectangle{\pgfqpoint{0.539299in}{0.078740in}}{\pgfqpoint{7.842520in}{7.842520in}}%
\pgfusepath{clip}%
\pgfsetbuttcap%
\pgfsetroundjoin%
\definecolor{currentfill}{rgb}{0.120092,0.600104,0.542530}%
\pgfsetfillcolor{currentfill}%
\pgfsetlinewidth{0.000000pt}%
\definecolor{currentstroke}{rgb}{0.273809,0.031497,0.358853}%
\pgfsetstrokecolor{currentstroke}%
\pgfsetdash{}{0pt}%
\pgfpathmoveto{\pgfqpoint{4.950134in}{3.565760in}}%
\pgfpathlineto{\pgfqpoint{4.807756in}{3.788327in}}%
\pgfpathlineto{\pgfqpoint{4.726130in}{3.847156in}}%
\pgfpathclose%
\pgfusepath{fill}%
\end{pgfscope}%
\begin{pgfscope}%
\pgfpathrectangle{\pgfqpoint{0.539299in}{0.078740in}}{\pgfqpoint{7.842520in}{7.842520in}}%
\pgfusepath{clip}%
\pgfsetbuttcap%
\pgfsetroundjoin%
\definecolor{currentfill}{rgb}{0.192357,0.403199,0.555836}%
\pgfsetfillcolor{currentfill}%
\pgfsetlinewidth{0.000000pt}%
\definecolor{currentstroke}{rgb}{0.274952,0.037752,0.364543}%
\pgfsetstrokecolor{currentstroke}%
\pgfsetdash{}{0pt}%
\pgfpathmoveto{\pgfqpoint{2.266737in}{3.071192in}}%
\pgfpathlineto{\pgfqpoint{2.195719in}{2.435884in}}%
\pgfpathlineto{\pgfqpoint{2.405032in}{3.063501in}}%
\pgfpathclose%
\pgfusepath{fill}%
\end{pgfscope}%
\begin{pgfscope}%
\pgfpathrectangle{\pgfqpoint{0.539299in}{0.078740in}}{\pgfqpoint{7.842520in}{7.842520in}}%
\pgfusepath{clip}%
\pgfsetbuttcap%
\pgfsetroundjoin%
\definecolor{currentfill}{rgb}{0.150476,0.504369,0.557430}%
\pgfsetfillcolor{currentfill}%
\pgfsetlinewidth{0.000000pt}%
\definecolor{currentstroke}{rgb}{0.276022,0.044167,0.370164}%
\pgfsetstrokecolor{currentstroke}%
\pgfsetdash{}{0pt}%
\pgfpathmoveto{\pgfqpoint{5.092313in}{3.347287in}}%
\pgfpathlineto{\pgfqpoint{5.012128in}{3.399559in}}%
\pgfpathlineto{\pgfqpoint{5.154816in}{3.180272in}}%
\pgfpathclose%
\pgfusepath{fill}%
\end{pgfscope}%
\begin{pgfscope}%
\pgfpathrectangle{\pgfqpoint{0.539299in}{0.078740in}}{\pgfqpoint{7.842520in}{7.842520in}}%
\pgfusepath{clip}%
\pgfsetbuttcap%
\pgfsetroundjoin%
\definecolor{currentfill}{rgb}{0.280894,0.078907,0.402329}%
\pgfsetfillcolor{currentfill}%
\pgfsetlinewidth{0.000000pt}%
\definecolor{currentstroke}{rgb}{0.277018,0.050344,0.375715}%
\pgfsetstrokecolor{currentstroke}%
\pgfsetdash{}{0pt}%
\pgfpathmoveto{\pgfqpoint{6.216568in}{1.580071in}}%
\pgfpathlineto{\pgfqpoint{6.293516in}{1.634687in}}%
\pgfpathlineto{\pgfqpoint{6.073966in}{1.765621in}}%
\pgfpathclose%
\pgfusepath{fill}%
\end{pgfscope}%
\begin{pgfscope}%
\pgfpathrectangle{\pgfqpoint{0.539299in}{0.078740in}}{\pgfqpoint{7.842520in}{7.842520in}}%
\pgfusepath{clip}%
\pgfsetbuttcap%
\pgfsetroundjoin%
\definecolor{currentfill}{rgb}{0.896320,0.893616,0.096335}%
\pgfsetfillcolor{currentfill}%
\pgfsetlinewidth{0.000000pt}%
\definecolor{currentstroke}{rgb}{0.277941,0.056324,0.381191}%
\pgfsetstrokecolor{currentstroke}%
\pgfsetdash{}{0pt}%
\pgfpathmoveto{\pgfqpoint{3.520103in}{5.522735in}}%
\pgfpathlineto{\pgfqpoint{3.606833in}{5.491393in}}%
\pgfpathlineto{\pgfqpoint{3.378279in}{5.603577in}}%
\pgfpathclose%
\pgfusepath{fill}%
\end{pgfscope}%
\begin{pgfscope}%
\pgfpathrectangle{\pgfqpoint{0.539299in}{0.078740in}}{\pgfqpoint{7.842520in}{7.842520in}}%
\pgfusepath{clip}%
\pgfsetbuttcap%
\pgfsetroundjoin%
\definecolor{currentfill}{rgb}{0.239346,0.300855,0.540844}%
\pgfsetfillcolor{currentfill}%
\pgfsetlinewidth{0.000000pt}%
\definecolor{currentstroke}{rgb}{0.278791,0.062145,0.386592}%
\pgfsetstrokecolor{currentstroke}%
\pgfsetdash{}{0pt}%
\pgfpathmoveto{\pgfqpoint{5.646299in}{2.364748in}}%
\pgfpathlineto{\pgfqpoint{5.724190in}{2.352097in}}%
\pgfpathlineto{\pgfqpoint{5.581983in}{2.550780in}}%
\pgfpathclose%
\pgfusepath{fill}%
\end{pgfscope}%
\begin{pgfscope}%
\pgfpathrectangle{\pgfqpoint{0.539299in}{0.078740in}}{\pgfqpoint{7.842520in}{7.842520in}}%
\pgfusepath{clip}%
\pgfsetbuttcap%
\pgfsetroundjoin%
\definecolor{currentfill}{rgb}{0.218130,0.347432,0.550038}%
\pgfsetfillcolor{currentfill}%
\pgfsetlinewidth{0.000000pt}%
\definecolor{currentstroke}{rgb}{0.279566,0.067836,0.391917}%
\pgfsetstrokecolor{currentstroke}%
\pgfsetdash{}{0pt}%
\pgfpathmoveto{\pgfqpoint{2.405032in}{3.063501in}}%
\pgfpathlineto{\pgfqpoint{2.333716in}{2.409916in}}%
\pgfpathlineto{\pgfqpoint{2.472446in}{2.377069in}}%
\pgfpathclose%
\pgfusepath{fill}%
\end{pgfscope}%
\begin{pgfscope}%
\pgfpathrectangle{\pgfqpoint{0.539299in}{0.078740in}}{\pgfqpoint{7.842520in}{7.842520in}}%
\pgfusepath{clip}%
\pgfsetbuttcap%
\pgfsetroundjoin%
\definecolor{currentfill}{rgb}{0.121148,0.592739,0.544641}%
\pgfsetfillcolor{currentfill}%
\pgfsetlinewidth{0.000000pt}%
\definecolor{currentstroke}{rgb}{0.280267,0.073417,0.397163}%
\pgfsetstrokecolor{currentstroke}%
\pgfsetdash{}{0pt}%
\pgfpathmoveto{\pgfqpoint{2.146689in}{3.948369in}}%
\pgfpathlineto{\pgfqpoint{2.068708in}{3.529775in}}%
\pgfpathlineto{\pgfqpoint{2.204165in}{3.588171in}}%
\pgfpathclose%
\pgfusepath{fill}%
\end{pgfscope}%
\begin{pgfscope}%
\pgfpathrectangle{\pgfqpoint{0.539299in}{0.078740in}}{\pgfqpoint{7.842520in}{7.842520in}}%
\pgfusepath{clip}%
\pgfsetbuttcap%
\pgfsetroundjoin%
\definecolor{currentfill}{rgb}{0.404001,0.800275,0.362552}%
\pgfsetfillcolor{currentfill}%
\pgfsetlinewidth{0.000000pt}%
\definecolor{currentstroke}{rgb}{0.280894,0.078907,0.402329}%
\pgfsetstrokecolor{currentstroke}%
\pgfsetdash{}{0pt}%
\pgfpathmoveto{\pgfqpoint{4.319824in}{4.607800in}}%
\pgfpathlineto{\pgfqpoint{4.092594in}{4.886710in}}%
\pgfpathlineto{\pgfqpoint{4.235995in}{4.678979in}}%
\pgfpathclose%
\pgfusepath{fill}%
\end{pgfscope}%
\begin{pgfscope}%
\pgfpathrectangle{\pgfqpoint{0.539299in}{0.078740in}}{\pgfqpoint{7.842520in}{7.842520in}}%
\pgfusepath{clip}%
\pgfsetbuttcap%
\pgfsetroundjoin%
\definecolor{currentfill}{rgb}{0.281412,0.155834,0.469201}%
\pgfsetfillcolor{currentfill}%
\pgfsetlinewidth{0.000000pt}%
\definecolor{currentstroke}{rgb}{0.281446,0.084320,0.407414}%
\pgfsetstrokecolor{currentstroke}%
\pgfsetdash{}{0pt}%
\pgfpathmoveto{\pgfqpoint{6.008600in}{1.974963in}}%
\pgfpathlineto{\pgfqpoint{5.931443in}{1.959659in}}%
\pgfpathlineto{\pgfqpoint{6.073966in}{1.765621in}}%
\pgfpathclose%
\pgfusepath{fill}%
\end{pgfscope}%
\begin{pgfscope}%
\pgfpathrectangle{\pgfqpoint{0.539299in}{0.078740in}}{\pgfqpoint{7.842520in}{7.842520in}}%
\pgfusepath{clip}%
\pgfsetbuttcap%
\pgfsetroundjoin%
\definecolor{currentfill}{rgb}{0.136408,0.541173,0.554483}%
\pgfsetfillcolor{currentfill}%
\pgfsetlinewidth{0.000000pt}%
\definecolor{currentstroke}{rgb}{0.281924,0.089666,0.412415}%
\pgfsetstrokecolor{currentstroke}%
\pgfsetdash{}{0pt}%
\pgfpathmoveto{\pgfqpoint{4.869240in}{3.622273in}}%
\pgfpathlineto{\pgfqpoint{5.012128in}{3.399559in}}%
\pgfpathlineto{\pgfqpoint{5.092313in}{3.347287in}}%
\pgfpathclose%
\pgfusepath{fill}%
\end{pgfscope}%
\begin{pgfscope}%
\pgfpathrectangle{\pgfqpoint{0.539299in}{0.078740in}}{\pgfqpoint{7.842520in}{7.842520in}}%
\pgfusepath{clip}%
\pgfsetbuttcap%
\pgfsetroundjoin%
\definecolor{currentfill}{rgb}{0.122312,0.633153,0.530398}%
\pgfsetfillcolor{currentfill}%
\pgfsetlinewidth{0.000000pt}%
\definecolor{currentstroke}{rgb}{0.282327,0.094955,0.417331}%
\pgfsetstrokecolor{currentstroke}%
\pgfsetdash{}{0pt}%
\pgfpathmoveto{\pgfqpoint{4.726130in}{3.847156in}}%
\pgfpathlineto{\pgfqpoint{4.807756in}{3.788327in}}%
\pgfpathlineto{\pgfqpoint{4.665156in}{4.013355in}}%
\pgfpathclose%
\pgfusepath{fill}%
\end{pgfscope}%
\begin{pgfscope}%
\pgfpathrectangle{\pgfqpoint{0.539299in}{0.078740in}}{\pgfqpoint{7.842520in}{7.842520in}}%
\pgfusepath{clip}%
\pgfsetbuttcap%
\pgfsetroundjoin%
\definecolor{currentfill}{rgb}{0.270595,0.214069,0.507052}%
\pgfsetfillcolor{currentfill}%
\pgfsetlinewidth{0.000000pt}%
\definecolor{currentstroke}{rgb}{0.282656,0.100196,0.422160}%
\pgfsetstrokecolor{currentstroke}%
\pgfsetdash{}{0pt}%
\pgfpathmoveto{\pgfqpoint{5.931443in}{1.959659in}}%
\pgfpathlineto{\pgfqpoint{5.866377in}{2.159848in}}%
\pgfpathlineto{\pgfqpoint{5.788906in}{2.159810in}}%
\pgfpathclose%
\pgfusepath{fill}%
\end{pgfscope}%
\begin{pgfscope}%
\pgfpathrectangle{\pgfqpoint{0.539299in}{0.078740in}}{\pgfqpoint{7.842520in}{7.842520in}}%
\pgfusepath{clip}%
\pgfsetbuttcap%
\pgfsetroundjoin%
\definecolor{currentfill}{rgb}{0.866013,0.889868,0.095953}%
\pgfsetfillcolor{currentfill}%
\pgfsetlinewidth{0.000000pt}%
\definecolor{currentstroke}{rgb}{0.282910,0.105393,0.426902}%
\pgfsetstrokecolor{currentstroke}%
\pgfsetdash{}{0pt}%
\pgfpathmoveto{\pgfqpoint{3.662699in}{5.404278in}}%
\pgfpathlineto{\pgfqpoint{3.606833in}{5.491393in}}%
\pgfpathlineto{\pgfqpoint{3.520103in}{5.522735in}}%
\pgfpathclose%
\pgfusepath{fill}%
\end{pgfscope}%
\begin{pgfscope}%
\pgfpathrectangle{\pgfqpoint{0.539299in}{0.078740in}}{\pgfqpoint{7.842520in}{7.842520in}}%
\pgfusepath{clip}%
\pgfsetbuttcap%
\pgfsetroundjoin%
\definecolor{currentfill}{rgb}{0.955300,0.901065,0.118128}%
\pgfsetfillcolor{currentfill}%
\pgfsetlinewidth{0.000000pt}%
\definecolor{currentstroke}{rgb}{0.283091,0.110553,0.431554}%
\pgfsetstrokecolor{currentstroke}%
\pgfsetdash{}{0pt}%
\pgfpathmoveto{\pgfqpoint{3.237559in}{5.639984in}}%
\pgfpathlineto{\pgfqpoint{3.150177in}{5.616991in}}%
\pgfpathlineto{\pgfqpoint{3.378279in}{5.603577in}}%
\pgfpathclose%
\pgfusepath{fill}%
\end{pgfscope}%
\begin{pgfscope}%
\pgfpathrectangle{\pgfqpoint{0.539299in}{0.078740in}}{\pgfqpoint{7.842520in}{7.842520in}}%
\pgfusepath{clip}%
\pgfsetbuttcap%
\pgfsetroundjoin%
\definecolor{currentfill}{rgb}{0.896320,0.893616,0.096335}%
\pgfsetfillcolor{currentfill}%
\pgfsetlinewidth{0.000000pt}%
\definecolor{currentstroke}{rgb}{0.283197,0.115680,0.436115}%
\pgfsetstrokecolor{currentstroke}%
\pgfsetdash{}{0pt}%
\pgfpathmoveto{\pgfqpoint{2.786367in}{5.433811in}}%
\pgfpathlineto{\pgfqpoint{2.923598in}{5.523700in}}%
\pgfpathlineto{\pgfqpoint{3.010887in}{5.591305in}}%
\pgfpathclose%
\pgfusepath{fill}%
\end{pgfscope}%
\begin{pgfscope}%
\pgfpathrectangle{\pgfqpoint{0.539299in}{0.078740in}}{\pgfqpoint{7.842520in}{7.842520in}}%
\pgfusepath{clip}%
\pgfsetbuttcap%
\pgfsetroundjoin%
\definecolor{currentfill}{rgb}{0.206756,0.371758,0.553117}%
\pgfsetfillcolor{currentfill}%
\pgfsetlinewidth{0.000000pt}%
\definecolor{currentstroke}{rgb}{0.283229,0.120777,0.440584}%
\pgfsetstrokecolor{currentstroke}%
\pgfsetdash{}{0pt}%
\pgfpathmoveto{\pgfqpoint{5.439710in}{2.755331in}}%
\pgfpathlineto{\pgfqpoint{5.360735in}{2.786461in}}%
\pgfpathlineto{\pgfqpoint{5.581983in}{2.550780in}}%
\pgfpathclose%
\pgfusepath{fill}%
\end{pgfscope}%
\begin{pgfscope}%
\pgfpathrectangle{\pgfqpoint{0.539299in}{0.078740in}}{\pgfqpoint{7.842520in}{7.842520in}}%
\pgfusepath{clip}%
\pgfsetbuttcap%
\pgfsetroundjoin%
\definecolor{currentfill}{rgb}{0.575563,0.844566,0.256415}%
\pgfsetfillcolor{currentfill}%
\pgfsetlinewidth{0.000000pt}%
\definecolor{currentstroke}{rgb}{0.283187,0.125848,0.444960}%
\pgfsetstrokecolor{currentstroke}%
\pgfsetdash{}{0pt}%
\pgfpathmoveto{\pgfqpoint{3.949148in}{5.080344in}}%
\pgfpathlineto{\pgfqpoint{4.092594in}{4.886710in}}%
\pgfpathlineto{\pgfqpoint{4.034345in}{5.020034in}}%
\pgfpathclose%
\pgfusepath{fill}%
\end{pgfscope}%
\begin{pgfscope}%
\pgfpathrectangle{\pgfqpoint{0.539299in}{0.078740in}}{\pgfqpoint{7.842520in}{7.842520in}}%
\pgfusepath{clip}%
\pgfsetbuttcap%
\pgfsetroundjoin%
\definecolor{currentfill}{rgb}{0.647257,0.858400,0.209861}%
\pgfsetfillcolor{currentfill}%
\pgfsetlinewidth{0.000000pt}%
\definecolor{currentstroke}{rgb}{0.283072,0.130895,0.449241}%
\pgfsetstrokecolor{currentstroke}%
\pgfsetdash{}{0pt}%
\pgfpathmoveto{\pgfqpoint{4.034345in}{5.020034in}}%
\pgfpathlineto{\pgfqpoint{3.805790in}{5.254759in}}%
\pgfpathlineto{\pgfqpoint{3.949148in}{5.080344in}}%
\pgfpathclose%
\pgfusepath{fill}%
\end{pgfscope}%
\begin{pgfscope}%
\pgfpathrectangle{\pgfqpoint{0.539299in}{0.078740in}}{\pgfqpoint{7.842520in}{7.842520in}}%
\pgfusepath{clip}%
\pgfsetbuttcap%
\pgfsetroundjoin%
\definecolor{currentfill}{rgb}{0.121148,0.592739,0.544641}%
\pgfsetfillcolor{currentfill}%
\pgfsetlinewidth{0.000000pt}%
\definecolor{currentstroke}{rgb}{0.282884,0.135920,0.453427}%
\pgfsetstrokecolor{currentstroke}%
\pgfsetdash{}{0pt}%
\pgfpathmoveto{\pgfqpoint{4.726130in}{3.847156in}}%
\pgfpathlineto{\pgfqpoint{4.869240in}{3.622273in}}%
\pgfpathlineto{\pgfqpoint{4.950134in}{3.565760in}}%
\pgfpathclose%
\pgfusepath{fill}%
\end{pgfscope}%
\begin{pgfscope}%
\pgfpathrectangle{\pgfqpoint{0.539299in}{0.078740in}}{\pgfqpoint{7.842520in}{7.842520in}}%
\pgfusepath{clip}%
\pgfsetbuttcap%
\pgfsetroundjoin%
\definecolor{currentfill}{rgb}{0.151918,0.500685,0.557587}%
\pgfsetfillcolor{currentfill}%
\pgfsetlinewidth{0.000000pt}%
\definecolor{currentstroke}{rgb}{0.282623,0.140926,0.457517}%
\pgfsetstrokecolor{currentstroke}%
\pgfsetdash{}{0pt}%
\pgfpathmoveto{\pgfqpoint{2.341276in}{3.620065in}}%
\pgfpathlineto{\pgfqpoint{2.129568in}{3.063212in}}%
\pgfpathlineto{\pgfqpoint{2.266737in}{3.071192in}}%
\pgfpathclose%
\pgfusepath{fill}%
\end{pgfscope}%
\begin{pgfscope}%
\pgfpathrectangle{\pgfqpoint{0.539299in}{0.078740in}}{\pgfqpoint{7.842520in}{7.842520in}}%
\pgfusepath{clip}%
\pgfsetbuttcap%
\pgfsetroundjoin%
\definecolor{currentfill}{rgb}{0.751884,0.874951,0.143228}%
\pgfsetfillcolor{currentfill}%
\pgfsetlinewidth{0.000000pt}%
\definecolor{currentstroke}{rgb}{0.282290,0.145912,0.461510}%
\pgfsetstrokecolor{currentstroke}%
\pgfsetdash{}{0pt}%
\pgfpathmoveto{\pgfqpoint{3.805790in}{5.254759in}}%
\pgfpathlineto{\pgfqpoint{3.891582in}{5.202680in}}%
\pgfpathlineto{\pgfqpoint{3.662699in}{5.404278in}}%
\pgfpathclose%
\pgfusepath{fill}%
\end{pgfscope}%
\begin{pgfscope}%
\pgfpathrectangle{\pgfqpoint{0.539299in}{0.078740in}}{\pgfqpoint{7.842520in}{7.842520in}}%
\pgfusepath{clip}%
\pgfsetbuttcap%
\pgfsetroundjoin%
\definecolor{currentfill}{rgb}{0.250425,0.274290,0.533103}%
\pgfsetfillcolor{currentfill}%
\pgfsetlinewidth{0.000000pt}%
\definecolor{currentstroke}{rgb}{0.281887,0.150881,0.465405}%
\pgfsetstrokecolor{currentstroke}%
\pgfsetdash{}{0pt}%
\pgfpathmoveto{\pgfqpoint{5.724190in}{2.352097in}}%
\pgfpathlineto{\pgfqpoint{5.646299in}{2.364748in}}%
\pgfpathlineto{\pgfqpoint{5.788906in}{2.159810in}}%
\pgfpathclose%
\pgfusepath{fill}%
\end{pgfscope}%
\begin{pgfscope}%
\pgfpathrectangle{\pgfqpoint{0.539299in}{0.078740in}}{\pgfqpoint{7.842520in}{7.842520in}}%
\pgfusepath{clip}%
\pgfsetbuttcap%
\pgfsetroundjoin%
\definecolor{currentfill}{rgb}{0.190631,0.407061,0.556089}%
\pgfsetfillcolor{currentfill}%
\pgfsetlinewidth{0.000000pt}%
\definecolor{currentstroke}{rgb}{0.281412,0.155834,0.469201}%
\pgfsetstrokecolor{currentstroke}%
\pgfsetdash{}{0pt}%
\pgfpathmoveto{\pgfqpoint{5.297333in}{2.965317in}}%
\pgfpathlineto{\pgfqpoint{5.360735in}{2.786461in}}%
\pgfpathlineto{\pgfqpoint{5.439710in}{2.755331in}}%
\pgfpathclose%
\pgfusepath{fill}%
\end{pgfscope}%
\begin{pgfscope}%
\pgfpathrectangle{\pgfqpoint{0.539299in}{0.078740in}}{\pgfqpoint{7.842520in}{7.842520in}}%
\pgfusepath{clip}%
\pgfsetbuttcap%
\pgfsetroundjoin%
\definecolor{currentfill}{rgb}{0.845561,0.887322,0.099702}%
\pgfsetfillcolor{currentfill}%
\pgfsetlinewidth{0.000000pt}%
\definecolor{currentstroke}{rgb}{0.280868,0.160771,0.472899}%
\pgfsetstrokecolor{currentstroke}%
\pgfsetdash{}{0pt}%
\pgfpathmoveto{\pgfqpoint{2.923598in}{5.523700in}}%
\pgfpathlineto{\pgfqpoint{2.786367in}{5.433811in}}%
\pgfpathlineto{\pgfqpoint{2.699533in}{5.323077in}}%
\pgfpathclose%
\pgfusepath{fill}%
\end{pgfscope}%
\begin{pgfscope}%
\pgfpathrectangle{\pgfqpoint{0.539299in}{0.078740in}}{\pgfqpoint{7.842520in}{7.842520in}}%
\pgfusepath{clip}%
\pgfsetbuttcap%
\pgfsetroundjoin%
\definecolor{currentfill}{rgb}{0.136408,0.541173,0.554483}%
\pgfsetfillcolor{currentfill}%
\pgfsetlinewidth{0.000000pt}%
\definecolor{currentstroke}{rgb}{0.280255,0.165693,0.476498}%
\pgfsetstrokecolor{currentstroke}%
\pgfsetdash{}{0pt}%
\pgfpathmoveto{\pgfqpoint{2.204165in}{3.588171in}}%
\pgfpathlineto{\pgfqpoint{2.129568in}{3.063212in}}%
\pgfpathlineto{\pgfqpoint{2.341276in}{3.620065in}}%
\pgfpathclose%
\pgfusepath{fill}%
\end{pgfscope}%
\begin{pgfscope}%
\pgfpathrectangle{\pgfqpoint{0.539299in}{0.078740in}}{\pgfqpoint{7.842520in}{7.842520in}}%
\pgfusepath{clip}%
\pgfsetbuttcap%
\pgfsetroundjoin%
\definecolor{currentfill}{rgb}{0.185783,0.704891,0.485273}%
\pgfsetfillcolor{currentfill}%
\pgfsetlinewidth{0.000000pt}%
\definecolor{currentstroke}{rgb}{0.279574,0.170599,0.479997}%
\pgfsetstrokecolor{currentstroke}%
\pgfsetdash{}{0pt}%
\pgfpathmoveto{\pgfqpoint{4.665156in}{4.013355in}}%
\pgfpathlineto{\pgfqpoint{4.522320in}{4.238703in}}%
\pgfpathlineto{\pgfqpoint{4.439213in}{4.296234in}}%
\pgfpathclose%
\pgfusepath{fill}%
\end{pgfscope}%
\begin{pgfscope}%
\pgfpathrectangle{\pgfqpoint{0.539299in}{0.078740in}}{\pgfqpoint{7.842520in}{7.842520in}}%
\pgfusepath{clip}%
\pgfsetbuttcap%
\pgfsetroundjoin%
\definecolor{currentfill}{rgb}{0.688944,0.865448,0.182725}%
\pgfsetfillcolor{currentfill}%
\pgfsetlinewidth{0.000000pt}%
\definecolor{currentstroke}{rgb}{0.278826,0.175490,0.483397}%
\pgfsetstrokecolor{currentstroke}%
\pgfsetdash{}{0pt}%
\pgfpathmoveto{\pgfqpoint{2.699533in}{5.323077in}}%
\pgfpathlineto{\pgfqpoint{2.478789in}{5.021565in}}%
\pgfpathlineto{\pgfqpoint{2.613359in}{5.171039in}}%
\pgfpathclose%
\pgfusepath{fill}%
\end{pgfscope}%
\begin{pgfscope}%
\pgfpathrectangle{\pgfqpoint{0.539299in}{0.078740in}}{\pgfqpoint{7.842520in}{7.842520in}}%
\pgfusepath{clip}%
\pgfsetbuttcap%
\pgfsetroundjoin%
\definecolor{currentfill}{rgb}{0.545524,0.838039,0.275626}%
\pgfsetfillcolor{currentfill}%
\pgfsetlinewidth{0.000000pt}%
\definecolor{currentstroke}{rgb}{0.278012,0.180367,0.486697}%
\pgfsetstrokecolor{currentstroke}%
\pgfsetdash{}{0pt}%
\pgfpathmoveto{\pgfqpoint{2.393562in}{4.830606in}}%
\pgfpathlineto{\pgfqpoint{2.528143in}{4.972174in}}%
\pgfpathlineto{\pgfqpoint{2.478789in}{5.021565in}}%
\pgfpathclose%
\pgfusepath{fill}%
\end{pgfscope}%
\begin{pgfscope}%
\pgfpathrectangle{\pgfqpoint{0.539299in}{0.078740in}}{\pgfqpoint{7.842520in}{7.842520in}}%
\pgfusepath{clip}%
\pgfsetbuttcap%
\pgfsetroundjoin%
\definecolor{currentfill}{rgb}{0.225863,0.330805,0.547314}%
\pgfsetfillcolor{currentfill}%
\pgfsetlinewidth{0.000000pt}%
\definecolor{currentstroke}{rgb}{0.277134,0.185228,0.489898}%
\pgfsetstrokecolor{currentstroke}%
\pgfsetdash{}{0pt}%
\pgfpathmoveto{\pgfqpoint{5.503585in}{2.573766in}}%
\pgfpathlineto{\pgfqpoint{5.646299in}{2.364748in}}%
\pgfpathlineto{\pgfqpoint{5.581983in}{2.550780in}}%
\pgfpathclose%
\pgfusepath{fill}%
\end{pgfscope}%
\begin{pgfscope}%
\pgfpathrectangle{\pgfqpoint{0.539299in}{0.078740in}}{\pgfqpoint{7.842520in}{7.842520in}}%
\pgfusepath{clip}%
\pgfsetbuttcap%
\pgfsetroundjoin%
\definecolor{currentfill}{rgb}{0.162016,0.687316,0.499129}%
\pgfsetfillcolor{currentfill}%
\pgfsetlinewidth{0.000000pt}%
\definecolor{currentstroke}{rgb}{0.276194,0.190074,0.493001}%
\pgfsetstrokecolor{currentstroke}%
\pgfsetdash{}{0pt}%
\pgfpathmoveto{\pgfqpoint{2.227125in}{4.300185in}}%
\pgfpathlineto{\pgfqpoint{2.146689in}{3.948369in}}%
\pgfpathlineto{\pgfqpoint{2.281795in}{4.035068in}}%
\pgfpathclose%
\pgfusepath{fill}%
\end{pgfscope}%
\begin{pgfscope}%
\pgfpathrectangle{\pgfqpoint{0.539299in}{0.078740in}}{\pgfqpoint{7.842520in}{7.842520in}}%
\pgfusepath{clip}%
\pgfsetbuttcap%
\pgfsetroundjoin%
\definecolor{currentfill}{rgb}{0.945636,0.899815,0.112838}%
\pgfsetfillcolor{currentfill}%
\pgfsetlinewidth{0.000000pt}%
\definecolor{currentstroke}{rgb}{0.275191,0.194905,0.496005}%
\pgfsetstrokecolor{currentstroke}%
\pgfsetdash{}{0pt}%
\pgfpathmoveto{\pgfqpoint{3.010887in}{5.591305in}}%
\pgfpathlineto{\pgfqpoint{2.923598in}{5.523700in}}%
\pgfpathlineto{\pgfqpoint{3.150177in}{5.616991in}}%
\pgfpathclose%
\pgfusepath{fill}%
\end{pgfscope}%
\begin{pgfscope}%
\pgfpathrectangle{\pgfqpoint{0.539299in}{0.078740in}}{\pgfqpoint{7.842520in}{7.842520in}}%
\pgfusepath{clip}%
\pgfsetbuttcap%
\pgfsetroundjoin%
\definecolor{currentfill}{rgb}{0.123444,0.636809,0.528763}%
\pgfsetfillcolor{currentfill}%
\pgfsetlinewidth{0.000000pt}%
\definecolor{currentstroke}{rgb}{0.274128,0.199721,0.498911}%
\pgfsetstrokecolor{currentstroke}%
\pgfsetdash{}{0pt}%
\pgfpathmoveto{\pgfqpoint{2.204165in}{3.588171in}}%
\pgfpathlineto{\pgfqpoint{2.281795in}{4.035068in}}%
\pgfpathlineto{\pgfqpoint{2.146689in}{3.948369in}}%
\pgfpathclose%
\pgfusepath{fill}%
\end{pgfscope}%
\begin{pgfscope}%
\pgfpathrectangle{\pgfqpoint{0.539299in}{0.078740in}}{\pgfqpoint{7.842520in}{7.842520in}}%
\pgfusepath{clip}%
\pgfsetbuttcap%
\pgfsetroundjoin%
\definecolor{currentfill}{rgb}{0.246070,0.738910,0.452024}%
\pgfsetfillcolor{currentfill}%
\pgfsetlinewidth{0.000000pt}%
\definecolor{currentstroke}{rgb}{0.273006,0.204520,0.501721}%
\pgfsetstrokecolor{currentstroke}%
\pgfsetdash{}{0pt}%
\pgfpathmoveto{\pgfqpoint{4.522320in}{4.238703in}}%
\pgfpathlineto{\pgfqpoint{4.379257in}{4.461675in}}%
\pgfpathlineto{\pgfqpoint{4.439213in}{4.296234in}}%
\pgfpathclose%
\pgfusepath{fill}%
\end{pgfscope}%
\begin{pgfscope}%
\pgfpathrectangle{\pgfqpoint{0.539299in}{0.078740in}}{\pgfqpoint{7.842520in}{7.842520in}}%
\pgfusepath{clip}%
\pgfsetbuttcap%
\pgfsetroundjoin%
\definecolor{currentfill}{rgb}{0.208623,0.367752,0.552675}%
\pgfsetfillcolor{currentfill}%
\pgfsetlinewidth{0.000000pt}%
\definecolor{currentstroke}{rgb}{0.271828,0.209303,0.504434}%
\pgfsetstrokecolor{currentstroke}%
\pgfsetdash{}{0pt}%
\pgfpathmoveto{\pgfqpoint{5.581983in}{2.550780in}}%
\pgfpathlineto{\pgfqpoint{5.360735in}{2.786461in}}%
\pgfpathlineto{\pgfqpoint{5.503585in}{2.573766in}}%
\pgfpathclose%
\pgfusepath{fill}%
\end{pgfscope}%
\begin{pgfscope}%
\pgfpathrectangle{\pgfqpoint{0.539299in}{0.078740in}}{\pgfqpoint{7.842520in}{7.842520in}}%
\pgfusepath{clip}%
\pgfsetbuttcap%
\pgfsetroundjoin%
\definecolor{currentfill}{rgb}{0.160665,0.478540,0.558115}%
\pgfsetfillcolor{currentfill}%
\pgfsetlinewidth{0.000000pt}%
\definecolor{currentstroke}{rgb}{0.270595,0.214069,0.507052}%
\pgfsetstrokecolor{currentstroke}%
\pgfsetdash{}{0pt}%
\pgfpathmoveto{\pgfqpoint{5.154816in}{3.180272in}}%
\pgfpathlineto{\pgfqpoint{5.074536in}{3.221540in}}%
\pgfpathlineto{\pgfqpoint{5.297333in}{2.965317in}}%
\pgfpathclose%
\pgfusepath{fill}%
\end{pgfscope}%
\begin{pgfscope}%
\pgfpathrectangle{\pgfqpoint{0.539299in}{0.078740in}}{\pgfqpoint{7.842520in}{7.842520in}}%
\pgfusepath{clip}%
\pgfsetbuttcap%
\pgfsetroundjoin%
\definecolor{currentfill}{rgb}{0.281924,0.089666,0.412415}%
\pgfsetfillcolor{currentfill}%
\pgfsetlinewidth{0.000000pt}%
\definecolor{currentstroke}{rgb}{0.269308,0.218818,0.509577}%
\pgfsetstrokecolor{currentstroke}%
\pgfsetdash{}{0pt}%
\pgfpathmoveto{\pgfqpoint{6.073966in}{1.765621in}}%
\pgfpathlineto{\pgfqpoint{6.139221in}{1.529769in}}%
\pgfpathlineto{\pgfqpoint{6.216568in}{1.580071in}}%
\pgfpathclose%
\pgfusepath{fill}%
\end{pgfscope}%
\begin{pgfscope}%
\pgfpathrectangle{\pgfqpoint{0.539299in}{0.078740in}}{\pgfqpoint{7.842520in}{7.842520in}}%
\pgfusepath{clip}%
\pgfsetbuttcap%
\pgfsetroundjoin%
\definecolor{currentfill}{rgb}{0.140210,0.665859,0.513427}%
\pgfsetfillcolor{currentfill}%
\pgfsetlinewidth{0.000000pt}%
\definecolor{currentstroke}{rgb}{0.267968,0.223549,0.512008}%
\pgfsetstrokecolor{currentstroke}%
\pgfsetdash{}{0pt}%
\pgfpathmoveto{\pgfqpoint{4.665156in}{4.013355in}}%
\pgfpathlineto{\pgfqpoint{4.582787in}{4.072527in}}%
\pgfpathlineto{\pgfqpoint{4.726130in}{3.847156in}}%
\pgfpathclose%
\pgfusepath{fill}%
\end{pgfscope}%
\begin{pgfscope}%
\pgfpathrectangle{\pgfqpoint{0.539299in}{0.078740in}}{\pgfqpoint{7.842520in}{7.842520in}}%
\pgfusepath{clip}%
\pgfsetbuttcap%
\pgfsetroundjoin%
\definecolor{currentfill}{rgb}{0.430983,0.808473,0.346476}%
\pgfsetfillcolor{currentfill}%
\pgfsetlinewidth{0.000000pt}%
\definecolor{currentstroke}{rgb}{0.266580,0.228262,0.514349}%
\pgfsetstrokecolor{currentstroke}%
\pgfsetdash{}{0pt}%
\pgfpathmoveto{\pgfqpoint{2.393562in}{4.830606in}}%
\pgfpathlineto{\pgfqpoint{2.309553in}{4.592125in}}%
\pgfpathlineto{\pgfqpoint{2.444220in}{4.720662in}}%
\pgfpathclose%
\pgfusepath{fill}%
\end{pgfscope}%
\begin{pgfscope}%
\pgfpathrectangle{\pgfqpoint{0.539299in}{0.078740in}}{\pgfqpoint{7.842520in}{7.842520in}}%
\pgfusepath{clip}%
\pgfsetbuttcap%
\pgfsetroundjoin%
\definecolor{currentfill}{rgb}{0.187231,0.414746,0.556547}%
\pgfsetfillcolor{currentfill}%
\pgfsetlinewidth{0.000000pt}%
\definecolor{currentstroke}{rgb}{0.265145,0.232956,0.516599}%
\pgfsetstrokecolor{currentstroke}%
\pgfsetdash{}{0pt}%
\pgfpathmoveto{\pgfqpoint{2.405032in}{3.063501in}}%
\pgfpathlineto{\pgfqpoint{2.472446in}{2.377069in}}%
\pgfpathlineto{\pgfqpoint{2.544311in}{3.042210in}}%
\pgfpathclose%
\pgfusepath{fill}%
\end{pgfscope}%
\begin{pgfscope}%
\pgfpathrectangle{\pgfqpoint{0.539299in}{0.078740in}}{\pgfqpoint{7.842520in}{7.842520in}}%
\pgfusepath{clip}%
\pgfsetbuttcap%
\pgfsetroundjoin%
\definecolor{currentfill}{rgb}{0.964894,0.902323,0.123941}%
\pgfsetfillcolor{currentfill}%
\pgfsetlinewidth{0.000000pt}%
\definecolor{currentstroke}{rgb}{0.263663,0.237631,0.518762}%
\pgfsetstrokecolor{currentstroke}%
\pgfsetdash{}{0pt}%
\pgfpathmoveto{\pgfqpoint{3.378279in}{5.603577in}}%
\pgfpathlineto{\pgfqpoint{3.150177in}{5.616991in}}%
\pgfpathlineto{\pgfqpoint{3.291045in}{5.591989in}}%
\pgfpathclose%
\pgfusepath{fill}%
\end{pgfscope}%
\begin{pgfscope}%
\pgfpathrectangle{\pgfqpoint{0.539299in}{0.078740in}}{\pgfqpoint{7.842520in}{7.842520in}}%
\pgfusepath{clip}%
\pgfsetbuttcap%
\pgfsetroundjoin%
\definecolor{currentfill}{rgb}{0.214298,0.355619,0.551184}%
\pgfsetfillcolor{currentfill}%
\pgfsetlinewidth{0.000000pt}%
\definecolor{currentstroke}{rgb}{0.262138,0.242286,0.520837}%
\pgfsetstrokecolor{currentstroke}%
\pgfsetdash{}{0pt}%
\pgfpathmoveto{\pgfqpoint{2.472446in}{2.377069in}}%
\pgfpathlineto{\pgfqpoint{2.611847in}{2.338243in}}%
\pgfpathlineto{\pgfqpoint{2.684453in}{3.009085in}}%
\pgfpathclose%
\pgfusepath{fill}%
\end{pgfscope}%
\begin{pgfscope}%
\pgfpathrectangle{\pgfqpoint{0.539299in}{0.078740in}}{\pgfqpoint{7.842520in}{7.842520in}}%
\pgfusepath{clip}%
\pgfsetbuttcap%
\pgfsetroundjoin%
\definecolor{currentfill}{rgb}{0.636902,0.856542,0.216620}%
\pgfsetfillcolor{currentfill}%
\pgfsetlinewidth{0.000000pt}%
\definecolor{currentstroke}{rgb}{0.260571,0.246922,0.522828}%
\pgfsetstrokecolor{currentstroke}%
\pgfsetdash{}{0pt}%
\pgfpathmoveto{\pgfqpoint{2.478789in}{5.021565in}}%
\pgfpathlineto{\pgfqpoint{2.528143in}{4.972174in}}%
\pgfpathlineto{\pgfqpoint{2.613359in}{5.171039in}}%
\pgfpathclose%
\pgfusepath{fill}%
\end{pgfscope}%
\begin{pgfscope}%
\pgfpathrectangle{\pgfqpoint{0.539299in}{0.078740in}}{\pgfqpoint{7.842520in}{7.842520in}}%
\pgfusepath{clip}%
\pgfsetbuttcap%
\pgfsetroundjoin%
\definecolor{currentfill}{rgb}{0.149039,0.508051,0.557250}%
\pgfsetfillcolor{currentfill}%
\pgfsetlinewidth{0.000000pt}%
\definecolor{currentstroke}{rgb}{0.258965,0.251537,0.524736}%
\pgfsetstrokecolor{currentstroke}%
\pgfsetdash{}{0pt}%
\pgfpathmoveto{\pgfqpoint{2.405032in}{3.063501in}}%
\pgfpathlineto{\pgfqpoint{2.341276in}{3.620065in}}%
\pgfpathlineto{\pgfqpoint{2.266737in}{3.071192in}}%
\pgfpathclose%
\pgfusepath{fill}%
\end{pgfscope}%
\begin{pgfscope}%
\pgfpathrectangle{\pgfqpoint{0.539299in}{0.078740in}}{\pgfqpoint{7.842520in}{7.842520in}}%
\pgfusepath{clip}%
\pgfsetbuttcap%
\pgfsetroundjoin%
\definecolor{currentfill}{rgb}{0.344074,0.780029,0.397381}%
\pgfsetfillcolor{currentfill}%
\pgfsetlinewidth{0.000000pt}%
\definecolor{currentstroke}{rgb}{0.257322,0.256130,0.526563}%
\pgfsetstrokecolor{currentstroke}%
\pgfsetdash{}{0pt}%
\pgfpathmoveto{\pgfqpoint{2.227125in}{4.300185in}}%
\pgfpathlineto{\pgfqpoint{2.444220in}{4.720662in}}%
\pgfpathlineto{\pgfqpoint{2.309553in}{4.592125in}}%
\pgfpathclose%
\pgfusepath{fill}%
\end{pgfscope}%
\begin{pgfscope}%
\pgfpathrectangle{\pgfqpoint{0.539299in}{0.078740in}}{\pgfqpoint{7.842520in}{7.842520in}}%
\pgfusepath{clip}%
\pgfsetbuttcap%
\pgfsetroundjoin%
\definecolor{currentfill}{rgb}{0.175707,0.697900,0.491033}%
\pgfsetfillcolor{currentfill}%
\pgfsetlinewidth{0.000000pt}%
\definecolor{currentstroke}{rgb}{0.255645,0.260703,0.528312}%
\pgfsetstrokecolor{currentstroke}%
\pgfsetdash{}{0pt}%
\pgfpathmoveto{\pgfqpoint{4.439213in}{4.296234in}}%
\pgfpathlineto{\pgfqpoint{4.582787in}{4.072527in}}%
\pgfpathlineto{\pgfqpoint{4.665156in}{4.013355in}}%
\pgfpathclose%
\pgfusepath{fill}%
\end{pgfscope}%
\begin{pgfscope}%
\pgfpathrectangle{\pgfqpoint{0.539299in}{0.078740in}}{\pgfqpoint{7.842520in}{7.842520in}}%
\pgfusepath{clip}%
\pgfsetbuttcap%
\pgfsetroundjoin%
\definecolor{currentfill}{rgb}{0.935904,0.898570,0.108131}%
\pgfsetfillcolor{currentfill}%
\pgfsetlinewidth{0.000000pt}%
\definecolor{currentstroke}{rgb}{0.253935,0.265254,0.529983}%
\pgfsetstrokecolor{currentstroke}%
\pgfsetdash{}{0pt}%
\pgfpathmoveto{\pgfqpoint{3.433110in}{5.522817in}}%
\pgfpathlineto{\pgfqpoint{3.520103in}{5.522735in}}%
\pgfpathlineto{\pgfqpoint{3.378279in}{5.603577in}}%
\pgfpathclose%
\pgfusepath{fill}%
\end{pgfscope}%
\begin{pgfscope}%
\pgfpathrectangle{\pgfqpoint{0.539299in}{0.078740in}}{\pgfqpoint{7.842520in}{7.842520in}}%
\pgfusepath{clip}%
\pgfsetbuttcap%
\pgfsetroundjoin%
\definecolor{currentfill}{rgb}{0.278826,0.175490,0.483397}%
\pgfsetfillcolor{currentfill}%
\pgfsetlinewidth{0.000000pt}%
\definecolor{currentstroke}{rgb}{0.252194,0.269783,0.531579}%
\pgfsetstrokecolor{currentstroke}%
\pgfsetdash{}{0pt}%
\pgfpathmoveto{\pgfqpoint{6.073966in}{1.765621in}}%
\pgfpathlineto{\pgfqpoint{5.931443in}{1.959659in}}%
\pgfpathlineto{\pgfqpoint{5.853809in}{1.950429in}}%
\pgfpathclose%
\pgfusepath{fill}%
\end{pgfscope}%
\begin{pgfscope}%
\pgfpathrectangle{\pgfqpoint{0.539299in}{0.078740in}}{\pgfqpoint{7.842520in}{7.842520in}}%
\pgfusepath{clip}%
\pgfsetbuttcap%
\pgfsetroundjoin%
\definecolor{currentfill}{rgb}{0.140536,0.530132,0.555659}%
\pgfsetfillcolor{currentfill}%
\pgfsetlinewidth{0.000000pt}%
\definecolor{currentstroke}{rgb}{0.250425,0.274290,0.533103}%
\pgfsetstrokecolor{currentstroke}%
\pgfsetdash{}{0pt}%
\pgfpathmoveto{\pgfqpoint{4.931146in}{3.442939in}}%
\pgfpathlineto{\pgfqpoint{5.154816in}{3.180272in}}%
\pgfpathlineto{\pgfqpoint{5.012128in}{3.399559in}}%
\pgfpathclose%
\pgfusepath{fill}%
\end{pgfscope}%
\begin{pgfscope}%
\pgfpathrectangle{\pgfqpoint{0.539299in}{0.078740in}}{\pgfqpoint{7.842520in}{7.842520in}}%
\pgfusepath{clip}%
\pgfsetbuttcap%
\pgfsetroundjoin%
\definecolor{currentfill}{rgb}{0.386433,0.794644,0.372886}%
\pgfsetfillcolor{currentfill}%
\pgfsetlinewidth{0.000000pt}%
\definecolor{currentstroke}{rgb}{0.248629,0.278775,0.534556}%
\pgfsetstrokecolor{currentstroke}%
\pgfsetdash{}{0pt}%
\pgfpathmoveto{\pgfqpoint{4.235995in}{4.678979in}}%
\pgfpathlineto{\pgfqpoint{4.151482in}{4.727445in}}%
\pgfpathlineto{\pgfqpoint{4.379257in}{4.461675in}}%
\pgfpathclose%
\pgfusepath{fill}%
\end{pgfscope}%
\begin{pgfscope}%
\pgfpathrectangle{\pgfqpoint{0.539299in}{0.078740in}}{\pgfqpoint{7.842520in}{7.842520in}}%
\pgfusepath{clip}%
\pgfsetbuttcap%
\pgfsetroundjoin%
\definecolor{currentfill}{rgb}{0.177423,0.437527,0.557565}%
\pgfsetfillcolor{currentfill}%
\pgfsetlinewidth{0.000000pt}%
\definecolor{currentstroke}{rgb}{0.246811,0.283237,0.535941}%
\pgfsetstrokecolor{currentstroke}%
\pgfsetdash{}{0pt}%
\pgfpathmoveto{\pgfqpoint{5.217726in}{3.002518in}}%
\pgfpathlineto{\pgfqpoint{5.360735in}{2.786461in}}%
\pgfpathlineto{\pgfqpoint{5.297333in}{2.965317in}}%
\pgfpathclose%
\pgfusepath{fill}%
\end{pgfscope}%
\begin{pgfscope}%
\pgfpathrectangle{\pgfqpoint{0.539299in}{0.078740in}}{\pgfqpoint{7.842520in}{7.842520in}}%
\pgfusepath{clip}%
\pgfsetbuttcap%
\pgfsetroundjoin%
\definecolor{currentfill}{rgb}{0.120081,0.622161,0.534946}%
\pgfsetfillcolor{currentfill}%
\pgfsetlinewidth{0.000000pt}%
\definecolor{currentstroke}{rgb}{0.244972,0.287675,0.537260}%
\pgfsetstrokecolor{currentstroke}%
\pgfsetdash{}{0pt}%
\pgfpathmoveto{\pgfqpoint{2.204165in}{3.588171in}}%
\pgfpathlineto{\pgfqpoint{2.341276in}{3.620065in}}%
\pgfpathlineto{\pgfqpoint{2.281795in}{4.035068in}}%
\pgfpathclose%
\pgfusepath{fill}%
\end{pgfscope}%
\begin{pgfscope}%
\pgfpathrectangle{\pgfqpoint{0.539299in}{0.078740in}}{\pgfqpoint{7.842520in}{7.842520in}}%
\pgfusepath{clip}%
\pgfsetbuttcap%
\pgfsetroundjoin%
\definecolor{currentfill}{rgb}{0.265145,0.232956,0.516599}%
\pgfsetfillcolor{currentfill}%
\pgfsetlinewidth{0.000000pt}%
\definecolor{currentstroke}{rgb}{0.243113,0.292092,0.538516}%
\pgfsetstrokecolor{currentstroke}%
\pgfsetdash{}{0pt}%
\pgfpathmoveto{\pgfqpoint{5.788906in}{2.159810in}}%
\pgfpathlineto{\pgfqpoint{5.710883in}{2.164223in}}%
\pgfpathlineto{\pgfqpoint{5.931443in}{1.959659in}}%
\pgfpathclose%
\pgfusepath{fill}%
\end{pgfscope}%
\begin{pgfscope}%
\pgfpathrectangle{\pgfqpoint{0.539299in}{0.078740in}}{\pgfqpoint{7.842520in}{7.842520in}}%
\pgfusepath{clip}%
\pgfsetbuttcap%
\pgfsetroundjoin%
\definecolor{currentfill}{rgb}{0.866013,0.889868,0.095953}%
\pgfsetfillcolor{currentfill}%
\pgfsetlinewidth{0.000000pt}%
\definecolor{currentstroke}{rgb}{0.241237,0.296485,0.539709}%
\pgfsetstrokecolor{currentstroke}%
\pgfsetdash{}{0pt}%
\pgfpathmoveto{\pgfqpoint{2.836696in}{5.416830in}}%
\pgfpathlineto{\pgfqpoint{2.923598in}{5.523700in}}%
\pgfpathlineto{\pgfqpoint{2.699533in}{5.323077in}}%
\pgfpathclose%
\pgfusepath{fill}%
\end{pgfscope}%
\begin{pgfscope}%
\pgfpathrectangle{\pgfqpoint{0.539299in}{0.078740in}}{\pgfqpoint{7.842520in}{7.842520in}}%
\pgfusepath{clip}%
\pgfsetbuttcap%
\pgfsetroundjoin%
\definecolor{currentfill}{rgb}{0.226397,0.728888,0.462789}%
\pgfsetfillcolor{currentfill}%
\pgfsetlinewidth{0.000000pt}%
\definecolor{currentstroke}{rgb}{0.239346,0.300855,0.540844}%
\pgfsetstrokecolor{currentstroke}%
\pgfsetdash{}{0pt}%
\pgfpathmoveto{\pgfqpoint{2.281795in}{4.035068in}}%
\pgfpathlineto{\pgfqpoint{2.361965in}{4.410404in}}%
\pgfpathlineto{\pgfqpoint{2.227125in}{4.300185in}}%
\pgfpathclose%
\pgfusepath{fill}%
\end{pgfscope}%
\begin{pgfscope}%
\pgfpathrectangle{\pgfqpoint{0.539299in}{0.078740in}}{\pgfqpoint{7.842520in}{7.842520in}}%
\pgfusepath{clip}%
\pgfsetbuttcap%
\pgfsetroundjoin%
\definecolor{currentfill}{rgb}{0.906311,0.894855,0.098125}%
\pgfsetfillcolor{currentfill}%
\pgfsetlinewidth{0.000000pt}%
\definecolor{currentstroke}{rgb}{0.237441,0.305202,0.541921}%
\pgfsetstrokecolor{currentstroke}%
\pgfsetdash{}{0pt}%
\pgfpathmoveto{\pgfqpoint{3.662699in}{5.404278in}}%
\pgfpathlineto{\pgfqpoint{3.520103in}{5.522735in}}%
\pgfpathlineto{\pgfqpoint{3.433110in}{5.522817in}}%
\pgfpathclose%
\pgfusepath{fill}%
\end{pgfscope}%
\begin{pgfscope}%
\pgfpathrectangle{\pgfqpoint{0.539299in}{0.078740in}}{\pgfqpoint{7.842520in}{7.842520in}}%
\pgfusepath{clip}%
\pgfsetbuttcap%
\pgfsetroundjoin%
\definecolor{currentfill}{rgb}{0.128729,0.563265,0.551229}%
\pgfsetfillcolor{currentfill}%
\pgfsetlinewidth{0.000000pt}%
\definecolor{currentstroke}{rgb}{0.235526,0.309527,0.542944}%
\pgfsetstrokecolor{currentstroke}%
\pgfsetdash{}{0pt}%
\pgfpathmoveto{\pgfqpoint{5.012128in}{3.399559in}}%
\pgfpathlineto{\pgfqpoint{4.869240in}{3.622273in}}%
\pgfpathlineto{\pgfqpoint{4.931146in}{3.442939in}}%
\pgfpathclose%
\pgfusepath{fill}%
\end{pgfscope}%
\begin{pgfscope}%
\pgfpathrectangle{\pgfqpoint{0.539299in}{0.078740in}}{\pgfqpoint{7.842520in}{7.842520in}}%
\pgfusepath{clip}%
\pgfsetbuttcap%
\pgfsetroundjoin%
\definecolor{currentfill}{rgb}{0.163625,0.471133,0.558148}%
\pgfsetfillcolor{currentfill}%
\pgfsetlinewidth{0.000000pt}%
\definecolor{currentstroke}{rgb}{0.233603,0.313828,0.543914}%
\pgfsetstrokecolor{currentstroke}%
\pgfsetdash{}{0pt}%
\pgfpathmoveto{\pgfqpoint{5.297333in}{2.965317in}}%
\pgfpathlineto{\pgfqpoint{5.074536in}{3.221540in}}%
\pgfpathlineto{\pgfqpoint{5.217726in}{3.002518in}}%
\pgfpathclose%
\pgfusepath{fill}%
\end{pgfscope}%
\begin{pgfscope}%
\pgfpathrectangle{\pgfqpoint{0.539299in}{0.078740in}}{\pgfqpoint{7.842520in}{7.842520in}}%
\pgfusepath{clip}%
\pgfsetbuttcap%
\pgfsetroundjoin%
\definecolor{currentfill}{rgb}{0.283197,0.115680,0.436115}%
\pgfsetfillcolor{currentfill}%
\pgfsetlinewidth{0.000000pt}%
\definecolor{currentstroke}{rgb}{0.231674,0.318106,0.544834}%
\pgfsetstrokecolor{currentstroke}%
\pgfsetdash{}{0pt}%
\pgfpathmoveto{\pgfqpoint{6.073966in}{1.765621in}}%
\pgfpathlineto{\pgfqpoint{5.996574in}{1.738442in}}%
\pgfpathlineto{\pgfqpoint{6.139221in}{1.529769in}}%
\pgfpathclose%
\pgfusepath{fill}%
\end{pgfscope}%
\begin{pgfscope}%
\pgfpathrectangle{\pgfqpoint{0.539299in}{0.078740in}}{\pgfqpoint{7.842520in}{7.842520in}}%
\pgfusepath{clip}%
\pgfsetbuttcap%
\pgfsetroundjoin%
\definecolor{currentfill}{rgb}{0.506271,0.828786,0.300362}%
\pgfsetfillcolor{currentfill}%
\pgfsetlinewidth{0.000000pt}%
\definecolor{currentstroke}{rgb}{0.229739,0.322361,0.545706}%
\pgfsetstrokecolor{currentstroke}%
\pgfsetdash{}{0pt}%
\pgfpathmoveto{\pgfqpoint{4.235995in}{4.678979in}}%
\pgfpathlineto{\pgfqpoint{4.092594in}{4.886710in}}%
\pgfpathlineto{\pgfqpoint{4.007443in}{4.927974in}}%
\pgfpathclose%
\pgfusepath{fill}%
\end{pgfscope}%
\begin{pgfscope}%
\pgfpathrectangle{\pgfqpoint{0.539299in}{0.078740in}}{\pgfqpoint{7.842520in}{7.842520in}}%
\pgfusepath{clip}%
\pgfsetbuttcap%
\pgfsetroundjoin%
\definecolor{currentfill}{rgb}{0.515992,0.831158,0.294279}%
\pgfsetfillcolor{currentfill}%
\pgfsetlinewidth{0.000000pt}%
\definecolor{currentstroke}{rgb}{0.227802,0.326594,0.546532}%
\pgfsetstrokecolor{currentstroke}%
\pgfsetdash{}{0pt}%
\pgfpathmoveto{\pgfqpoint{2.444220in}{4.720662in}}%
\pgfpathlineto{\pgfqpoint{2.528143in}{4.972174in}}%
\pgfpathlineto{\pgfqpoint{2.393562in}{4.830606in}}%
\pgfpathclose%
\pgfusepath{fill}%
\end{pgfscope}%
\begin{pgfscope}%
\pgfpathrectangle{\pgfqpoint{0.539299in}{0.078740in}}{\pgfqpoint{7.842520in}{7.842520in}}%
\pgfusepath{clip}%
\pgfsetbuttcap%
\pgfsetroundjoin%
\definecolor{currentfill}{rgb}{0.964894,0.902323,0.123941}%
\pgfsetfillcolor{currentfill}%
\pgfsetlinewidth{0.000000pt}%
\definecolor{currentstroke}{rgb}{0.225863,0.330805,0.547314}%
\pgfsetstrokecolor{currentstroke}%
\pgfsetdash{}{0pt}%
\pgfpathmoveto{\pgfqpoint{3.291045in}{5.591989in}}%
\pgfpathlineto{\pgfqpoint{3.433110in}{5.522817in}}%
\pgfpathlineto{\pgfqpoint{3.378279in}{5.603577in}}%
\pgfpathclose%
\pgfusepath{fill}%
\end{pgfscope}%
\begin{pgfscope}%
\pgfpathrectangle{\pgfqpoint{0.539299in}{0.078740in}}{\pgfqpoint{7.842520in}{7.842520in}}%
\pgfusepath{clip}%
\pgfsetbuttcap%
\pgfsetroundjoin%
\definecolor{currentfill}{rgb}{0.955300,0.901065,0.118128}%
\pgfsetfillcolor{currentfill}%
\pgfsetlinewidth{0.000000pt}%
\definecolor{currentstroke}{rgb}{0.223925,0.334994,0.548053}%
\pgfsetstrokecolor{currentstroke}%
\pgfsetdash{}{0pt}%
\pgfpathmoveto{\pgfqpoint{2.923598in}{5.523700in}}%
\pgfpathlineto{\pgfqpoint{3.062895in}{5.558135in}}%
\pgfpathlineto{\pgfqpoint{3.150177in}{5.616991in}}%
\pgfpathclose%
\pgfusepath{fill}%
\end{pgfscope}%
\begin{pgfscope}%
\pgfpathrectangle{\pgfqpoint{0.539299in}{0.078740in}}{\pgfqpoint{7.842520in}{7.842520in}}%
\pgfusepath{clip}%
\pgfsetbuttcap%
\pgfsetroundjoin%
\definecolor{currentfill}{rgb}{0.214298,0.355619,0.551184}%
\pgfsetfillcolor{currentfill}%
\pgfsetlinewidth{0.000000pt}%
\definecolor{currentstroke}{rgb}{0.221989,0.339161,0.548752}%
\pgfsetstrokecolor{currentstroke}%
\pgfsetdash{}{0pt}%
\pgfpathmoveto{\pgfqpoint{2.611847in}{2.338243in}}%
\pgfpathlineto{\pgfqpoint{2.751870in}{2.294211in}}%
\pgfpathlineto{\pgfqpoint{2.684453in}{3.009085in}}%
\pgfpathclose%
\pgfusepath{fill}%
\end{pgfscope}%
\begin{pgfscope}%
\pgfpathrectangle{\pgfqpoint{0.539299in}{0.078740in}}{\pgfqpoint{7.842520in}{7.842520in}}%
\pgfusepath{clip}%
\pgfsetbuttcap%
\pgfsetroundjoin%
\definecolor{currentfill}{rgb}{0.185556,0.418570,0.556753}%
\pgfsetfillcolor{currentfill}%
\pgfsetlinewidth{0.000000pt}%
\definecolor{currentstroke}{rgb}{0.220057,0.343307,0.549413}%
\pgfsetstrokecolor{currentstroke}%
\pgfsetdash{}{0pt}%
\pgfpathmoveto{\pgfqpoint{2.544311in}{3.042210in}}%
\pgfpathlineto{\pgfqpoint{2.472446in}{2.377069in}}%
\pgfpathlineto{\pgfqpoint{2.684453in}{3.009085in}}%
\pgfpathclose%
\pgfusepath{fill}%
\end{pgfscope}%
\begin{pgfscope}%
\pgfpathrectangle{\pgfqpoint{0.539299in}{0.078740in}}{\pgfqpoint{7.842520in}{7.842520in}}%
\pgfusepath{clip}%
\pgfsetbuttcap%
\pgfsetroundjoin%
\definecolor{currentfill}{rgb}{0.304148,0.764704,0.419943}%
\pgfsetfillcolor{currentfill}%
\pgfsetlinewidth{0.000000pt}%
\definecolor{currentstroke}{rgb}{0.218130,0.347432,0.550038}%
\pgfsetstrokecolor{currentstroke}%
\pgfsetdash{}{0pt}%
\pgfpathmoveto{\pgfqpoint{4.439213in}{4.296234in}}%
\pgfpathlineto{\pgfqpoint{4.379257in}{4.461675in}}%
\pgfpathlineto{\pgfqpoint{4.295430in}{4.515609in}}%
\pgfpathclose%
\pgfusepath{fill}%
\end{pgfscope}%
\begin{pgfscope}%
\pgfpathrectangle{\pgfqpoint{0.539299in}{0.078740in}}{\pgfqpoint{7.842520in}{7.842520in}}%
\pgfusepath{clip}%
\pgfsetbuttcap%
\pgfsetroundjoin%
\definecolor{currentfill}{rgb}{0.595839,0.848717,0.243329}%
\pgfsetfillcolor{currentfill}%
\pgfsetlinewidth{0.000000pt}%
\definecolor{currentstroke}{rgb}{0.216210,0.351535,0.550627}%
\pgfsetstrokecolor{currentstroke}%
\pgfsetdash{}{0pt}%
\pgfpathmoveto{\pgfqpoint{4.007443in}{4.927974in}}%
\pgfpathlineto{\pgfqpoint{4.092594in}{4.886710in}}%
\pgfpathlineto{\pgfqpoint{3.949148in}{5.080344in}}%
\pgfpathclose%
\pgfusepath{fill}%
\end{pgfscope}%
\begin{pgfscope}%
\pgfpathrectangle{\pgfqpoint{0.539299in}{0.078740in}}{\pgfqpoint{7.842520in}{7.842520in}}%
\pgfusepath{clip}%
\pgfsetbuttcap%
\pgfsetroundjoin%
\definecolor{currentfill}{rgb}{0.327796,0.773980,0.406640}%
\pgfsetfillcolor{currentfill}%
\pgfsetlinewidth{0.000000pt}%
\definecolor{currentstroke}{rgb}{0.214298,0.355619,0.551184}%
\pgfsetstrokecolor{currentstroke}%
\pgfsetdash{}{0pt}%
\pgfpathmoveto{\pgfqpoint{2.227125in}{4.300185in}}%
\pgfpathlineto{\pgfqpoint{2.361965in}{4.410404in}}%
\pgfpathlineto{\pgfqpoint{2.444220in}{4.720662in}}%
\pgfpathclose%
\pgfusepath{fill}%
\end{pgfscope}%
\begin{pgfscope}%
\pgfpathrectangle{\pgfqpoint{0.539299in}{0.078740in}}{\pgfqpoint{7.842520in}{7.842520in}}%
\pgfusepath{clip}%
\pgfsetbuttcap%
\pgfsetroundjoin%
\definecolor{currentfill}{rgb}{0.280868,0.160771,0.472899}%
\pgfsetfillcolor{currentfill}%
\pgfsetlinewidth{0.000000pt}%
\definecolor{currentstroke}{rgb}{0.212395,0.359683,0.551710}%
\pgfsetstrokecolor{currentstroke}%
\pgfsetdash{}{0pt}%
\pgfpathmoveto{\pgfqpoint{5.853809in}{1.950429in}}%
\pgfpathlineto{\pgfqpoint{5.996574in}{1.738442in}}%
\pgfpathlineto{\pgfqpoint{6.073966in}{1.765621in}}%
\pgfpathclose%
\pgfusepath{fill}%
\end{pgfscope}%
\begin{pgfscope}%
\pgfpathrectangle{\pgfqpoint{0.539299in}{0.078740in}}{\pgfqpoint{7.842520in}{7.842520in}}%
\pgfusepath{clip}%
\pgfsetbuttcap%
\pgfsetroundjoin%
\definecolor{currentfill}{rgb}{0.241237,0.296485,0.539709}%
\pgfsetfillcolor{currentfill}%
\pgfsetlinewidth{0.000000pt}%
\definecolor{currentstroke}{rgb}{0.210503,0.363727,0.552206}%
\pgfsetstrokecolor{currentstroke}%
\pgfsetdash{}{0pt}%
\pgfpathmoveto{\pgfqpoint{5.788906in}{2.159810in}}%
\pgfpathlineto{\pgfqpoint{5.646299in}{2.364748in}}%
\pgfpathlineto{\pgfqpoint{5.567779in}{2.379196in}}%
\pgfpathclose%
\pgfusepath{fill}%
\end{pgfscope}%
\begin{pgfscope}%
\pgfpathrectangle{\pgfqpoint{0.539299in}{0.078740in}}{\pgfqpoint{7.842520in}{7.842520in}}%
\pgfusepath{clip}%
\pgfsetbuttcap%
\pgfsetroundjoin%
\definecolor{currentfill}{rgb}{0.814576,0.883393,0.110347}%
\pgfsetfillcolor{currentfill}%
\pgfsetlinewidth{0.000000pt}%
\definecolor{currentstroke}{rgb}{0.208623,0.367752,0.552675}%
\pgfsetstrokecolor{currentstroke}%
\pgfsetdash{}{0pt}%
\pgfpathmoveto{\pgfqpoint{3.719555in}{5.277291in}}%
\pgfpathlineto{\pgfqpoint{3.805790in}{5.254759in}}%
\pgfpathlineto{\pgfqpoint{3.662699in}{5.404278in}}%
\pgfpathclose%
\pgfusepath{fill}%
\end{pgfscope}%
\begin{pgfscope}%
\pgfpathrectangle{\pgfqpoint{0.539299in}{0.078740in}}{\pgfqpoint{7.842520in}{7.842520in}}%
\pgfusepath{clip}%
\pgfsetbuttcap%
\pgfsetroundjoin%
\definecolor{currentfill}{rgb}{0.709898,0.868751,0.169257}%
\pgfsetfillcolor{currentfill}%
\pgfsetlinewidth{0.000000pt}%
\definecolor{currentstroke}{rgb}{0.206756,0.371758,0.553117}%
\pgfsetstrokecolor{currentstroke}%
\pgfsetdash{}{0pt}%
\pgfpathmoveto{\pgfqpoint{3.949148in}{5.080344in}}%
\pgfpathlineto{\pgfqpoint{3.805790in}{5.254759in}}%
\pgfpathlineto{\pgfqpoint{3.863419in}{5.112875in}}%
\pgfpathclose%
\pgfusepath{fill}%
\end{pgfscope}%
\begin{pgfscope}%
\pgfpathrectangle{\pgfqpoint{0.539299in}{0.078740in}}{\pgfqpoint{7.842520in}{7.842520in}}%
\pgfusepath{clip}%
\pgfsetbuttcap%
\pgfsetroundjoin%
\definecolor{currentfill}{rgb}{0.143343,0.522773,0.556295}%
\pgfsetfillcolor{currentfill}%
\pgfsetlinewidth{0.000000pt}%
\definecolor{currentstroke}{rgb}{0.204903,0.375746,0.553533}%
\pgfsetstrokecolor{currentstroke}%
\pgfsetdash{}{0pt}%
\pgfpathmoveto{\pgfqpoint{4.931146in}{3.442939in}}%
\pgfpathlineto{\pgfqpoint{5.074536in}{3.221540in}}%
\pgfpathlineto{\pgfqpoint{5.154816in}{3.180272in}}%
\pgfpathclose%
\pgfusepath{fill}%
\end{pgfscope}%
\begin{pgfscope}%
\pgfpathrectangle{\pgfqpoint{0.539299in}{0.078740in}}{\pgfqpoint{7.842520in}{7.842520in}}%
\pgfusepath{clip}%
\pgfsetbuttcap%
\pgfsetroundjoin%
\definecolor{currentfill}{rgb}{0.119699,0.618490,0.536347}%
\pgfsetfillcolor{currentfill}%
\pgfsetlinewidth{0.000000pt}%
\definecolor{currentstroke}{rgb}{0.203063,0.379716,0.553925}%
\pgfsetstrokecolor{currentstroke}%
\pgfsetdash{}{0pt}%
\pgfpathmoveto{\pgfqpoint{4.869240in}{3.622273in}}%
\pgfpathlineto{\pgfqpoint{4.726130in}{3.847156in}}%
\pgfpathlineto{\pgfqpoint{4.787540in}{3.665841in}}%
\pgfpathclose%
\pgfusepath{fill}%
\end{pgfscope}%
\begin{pgfscope}%
\pgfpathrectangle{\pgfqpoint{0.539299in}{0.078740in}}{\pgfqpoint{7.842520in}{7.842520in}}%
\pgfusepath{clip}%
\pgfsetbuttcap%
\pgfsetroundjoin%
\definecolor{currentfill}{rgb}{0.783315,0.879285,0.125405}%
\pgfsetfillcolor{currentfill}%
\pgfsetlinewidth{0.000000pt}%
\definecolor{currentstroke}{rgb}{0.201239,0.383670,0.554294}%
\pgfsetstrokecolor{currentstroke}%
\pgfsetdash{}{0pt}%
\pgfpathmoveto{\pgfqpoint{2.699533in}{5.323077in}}%
\pgfpathlineto{\pgfqpoint{2.613359in}{5.171039in}}%
\pgfpathlineto{\pgfqpoint{2.750456in}{5.265020in}}%
\pgfpathclose%
\pgfusepath{fill}%
\end{pgfscope}%
\begin{pgfscope}%
\pgfpathrectangle{\pgfqpoint{0.539299in}{0.078740in}}{\pgfqpoint{7.842520in}{7.842520in}}%
\pgfusepath{clip}%
\pgfsetbuttcap%
\pgfsetroundjoin%
\definecolor{currentfill}{rgb}{0.377779,0.791781,0.377939}%
\pgfsetfillcolor{currentfill}%
\pgfsetlinewidth{0.000000pt}%
\definecolor{currentstroke}{rgb}{0.199430,0.387607,0.554642}%
\pgfsetstrokecolor{currentstroke}%
\pgfsetdash{}{0pt}%
\pgfpathmoveto{\pgfqpoint{4.379257in}{4.461675in}}%
\pgfpathlineto{\pgfqpoint{4.151482in}{4.727445in}}%
\pgfpathlineto{\pgfqpoint{4.295430in}{4.515609in}}%
\pgfpathclose%
\pgfusepath{fill}%
\end{pgfscope}%
\begin{pgfscope}%
\pgfpathrectangle{\pgfqpoint{0.539299in}{0.078740in}}{\pgfqpoint{7.842520in}{7.842520in}}%
\pgfusepath{clip}%
\pgfsetbuttcap%
\pgfsetroundjoin%
\definecolor{currentfill}{rgb}{0.267968,0.223549,0.512008}%
\pgfsetfillcolor{currentfill}%
\pgfsetlinewidth{0.000000pt}%
\definecolor{currentstroke}{rgb}{0.197636,0.391528,0.554969}%
\pgfsetstrokecolor{currentstroke}%
\pgfsetdash{}{0pt}%
\pgfpathmoveto{\pgfqpoint{5.710883in}{2.164223in}}%
\pgfpathlineto{\pgfqpoint{5.853809in}{1.950429in}}%
\pgfpathlineto{\pgfqpoint{5.931443in}{1.959659in}}%
\pgfpathclose%
\pgfusepath{fill}%
\end{pgfscope}%
\begin{pgfscope}%
\pgfpathrectangle{\pgfqpoint{0.539299in}{0.078740in}}{\pgfqpoint{7.842520in}{7.842520in}}%
\pgfusepath{clip}%
\pgfsetbuttcap%
\pgfsetroundjoin%
\definecolor{currentfill}{rgb}{0.216210,0.351535,0.550627}%
\pgfsetfillcolor{currentfill}%
\pgfsetlinewidth{0.000000pt}%
\definecolor{currentstroke}{rgb}{0.195860,0.395433,0.555276}%
\pgfsetstrokecolor{currentstroke}%
\pgfsetdash{}{0pt}%
\pgfpathmoveto{\pgfqpoint{5.424487in}{2.595182in}}%
\pgfpathlineto{\pgfqpoint{5.646299in}{2.364748in}}%
\pgfpathlineto{\pgfqpoint{5.503585in}{2.573766in}}%
\pgfpathclose%
\pgfusepath{fill}%
\end{pgfscope}%
\begin{pgfscope}%
\pgfpathrectangle{\pgfqpoint{0.539299in}{0.078740in}}{\pgfqpoint{7.842520in}{7.842520in}}%
\pgfusepath{clip}%
\pgfsetbuttcap%
\pgfsetroundjoin%
\definecolor{currentfill}{rgb}{0.935904,0.898570,0.108131}%
\pgfsetfillcolor{currentfill}%
\pgfsetlinewidth{0.000000pt}%
\definecolor{currentstroke}{rgb}{0.194100,0.399323,0.555565}%
\pgfsetstrokecolor{currentstroke}%
\pgfsetdash{}{0pt}%
\pgfpathmoveto{\pgfqpoint{3.062895in}{5.558135in}}%
\pgfpathlineto{\pgfqpoint{2.923598in}{5.523700in}}%
\pgfpathlineto{\pgfqpoint{2.836696in}{5.416830in}}%
\pgfpathclose%
\pgfusepath{fill}%
\end{pgfscope}%
\begin{pgfscope}%
\pgfpathrectangle{\pgfqpoint{0.539299in}{0.078740in}}{\pgfqpoint{7.842520in}{7.842520in}}%
\pgfusepath{clip}%
\pgfsetbuttcap%
\pgfsetroundjoin%
\definecolor{currentfill}{rgb}{0.129933,0.559582,0.551864}%
\pgfsetfillcolor{currentfill}%
\pgfsetlinewidth{0.000000pt}%
\definecolor{currentstroke}{rgb}{0.192357,0.403199,0.555836}%
\pgfsetstrokecolor{currentstroke}%
\pgfsetdash{}{0pt}%
\pgfpathmoveto{\pgfqpoint{2.479796in}{3.628915in}}%
\pgfpathlineto{\pgfqpoint{2.341276in}{3.620065in}}%
\pgfpathlineto{\pgfqpoint{2.405032in}{3.063501in}}%
\pgfpathclose%
\pgfusepath{fill}%
\end{pgfscope}%
\begin{pgfscope}%
\pgfpathrectangle{\pgfqpoint{0.539299in}{0.078740in}}{\pgfqpoint{7.842520in}{7.842520in}}%
\pgfusepath{clip}%
\pgfsetbuttcap%
\pgfsetroundjoin%
\definecolor{currentfill}{rgb}{0.835270,0.886029,0.102646}%
\pgfsetfillcolor{currentfill}%
\pgfsetlinewidth{0.000000pt}%
\definecolor{currentstroke}{rgb}{0.190631,0.407061,0.556089}%
\pgfsetstrokecolor{currentstroke}%
\pgfsetdash{}{0pt}%
\pgfpathmoveto{\pgfqpoint{2.699533in}{5.323077in}}%
\pgfpathlineto{\pgfqpoint{2.750456in}{5.265020in}}%
\pgfpathlineto{\pgfqpoint{2.836696in}{5.416830in}}%
\pgfpathclose%
\pgfusepath{fill}%
\end{pgfscope}%
\begin{pgfscope}%
\pgfpathrectangle{\pgfqpoint{0.539299in}{0.078740in}}{\pgfqpoint{7.842520in}{7.842520in}}%
\pgfusepath{clip}%
\pgfsetbuttcap%
\pgfsetroundjoin%
\definecolor{currentfill}{rgb}{0.496615,0.826376,0.306377}%
\pgfsetfillcolor{currentfill}%
\pgfsetlinewidth{0.000000pt}%
\definecolor{currentstroke}{rgb}{0.188923,0.410910,0.556326}%
\pgfsetstrokecolor{currentstroke}%
\pgfsetdash{}{0pt}%
\pgfpathmoveto{\pgfqpoint{4.007443in}{4.927974in}}%
\pgfpathlineto{\pgfqpoint{4.151482in}{4.727445in}}%
\pgfpathlineto{\pgfqpoint{4.235995in}{4.678979in}}%
\pgfpathclose%
\pgfusepath{fill}%
\end{pgfscope}%
\begin{pgfscope}%
\pgfpathrectangle{\pgfqpoint{0.539299in}{0.078740in}}{\pgfqpoint{7.842520in}{7.842520in}}%
\pgfusepath{clip}%
\pgfsetbuttcap%
\pgfsetroundjoin%
\definecolor{currentfill}{rgb}{0.983868,0.904867,0.136897}%
\pgfsetfillcolor{currentfill}%
\pgfsetlinewidth{0.000000pt}%
\definecolor{currentstroke}{rgb}{0.187231,0.414746,0.556547}%
\pgfsetstrokecolor{currentstroke}%
\pgfsetdash{}{0pt}%
\pgfpathmoveto{\pgfqpoint{3.150177in}{5.616991in}}%
\pgfpathlineto{\pgfqpoint{3.203841in}{5.543255in}}%
\pgfpathlineto{\pgfqpoint{3.291045in}{5.591989in}}%
\pgfpathclose%
\pgfusepath{fill}%
\end{pgfscope}%
\begin{pgfscope}%
\pgfpathrectangle{\pgfqpoint{0.539299in}{0.078740in}}{\pgfqpoint{7.842520in}{7.842520in}}%
\pgfusepath{clip}%
\pgfsetbuttcap%
\pgfsetroundjoin%
\definecolor{currentfill}{rgb}{0.906311,0.894855,0.098125}%
\pgfsetfillcolor{currentfill}%
\pgfsetlinewidth{0.000000pt}%
\definecolor{currentstroke}{rgb}{0.185556,0.418570,0.556753}%
\pgfsetstrokecolor{currentstroke}%
\pgfsetdash{}{0pt}%
\pgfpathmoveto{\pgfqpoint{3.433110in}{5.522817in}}%
\pgfpathlineto{\pgfqpoint{3.576041in}{5.415869in}}%
\pgfpathlineto{\pgfqpoint{3.662699in}{5.404278in}}%
\pgfpathclose%
\pgfusepath{fill}%
\end{pgfscope}%
\begin{pgfscope}%
\pgfpathrectangle{\pgfqpoint{0.539299in}{0.078740in}}{\pgfqpoint{7.842520in}{7.842520in}}%
\pgfusepath{clip}%
\pgfsetbuttcap%
\pgfsetroundjoin%
\definecolor{currentfill}{rgb}{0.199430,0.387607,0.554642}%
\pgfsetfillcolor{currentfill}%
\pgfsetlinewidth{0.000000pt}%
\definecolor{currentstroke}{rgb}{0.183898,0.422383,0.556944}%
\pgfsetstrokecolor{currentstroke}%
\pgfsetdash{}{0pt}%
\pgfpathmoveto{\pgfqpoint{5.503585in}{2.573766in}}%
\pgfpathlineto{\pgfqpoint{5.360735in}{2.786461in}}%
\pgfpathlineto{\pgfqpoint{5.424487in}{2.595182in}}%
\pgfpathclose%
\pgfusepath{fill}%
\end{pgfscope}%
\begin{pgfscope}%
\pgfpathrectangle{\pgfqpoint{0.539299in}{0.078740in}}{\pgfqpoint{7.842520in}{7.842520in}}%
\pgfusepath{clip}%
\pgfsetbuttcap%
\pgfsetroundjoin%
\definecolor{currentfill}{rgb}{0.140210,0.665859,0.513427}%
\pgfsetfillcolor{currentfill}%
\pgfsetlinewidth{0.000000pt}%
\definecolor{currentstroke}{rgb}{0.182256,0.426184,0.557120}%
\pgfsetstrokecolor{currentstroke}%
\pgfsetdash{}{0pt}%
\pgfpathmoveto{\pgfqpoint{2.281795in}{4.035068in}}%
\pgfpathlineto{\pgfqpoint{2.341276in}{3.620065in}}%
\pgfpathlineto{\pgfqpoint{2.418849in}{4.087714in}}%
\pgfpathclose%
\pgfusepath{fill}%
\end{pgfscope}%
\begin{pgfscope}%
\pgfpathrectangle{\pgfqpoint{0.539299in}{0.078740in}}{\pgfqpoint{7.842520in}{7.842520in}}%
\pgfusepath{clip}%
\pgfsetbuttcap%
\pgfsetroundjoin%
\definecolor{currentfill}{rgb}{0.246811,0.283237,0.535941}%
\pgfsetfillcolor{currentfill}%
\pgfsetlinewidth{0.000000pt}%
\definecolor{currentstroke}{rgb}{0.180629,0.429975,0.557282}%
\pgfsetstrokecolor{currentstroke}%
\pgfsetdash{}{0pt}%
\pgfpathmoveto{\pgfqpoint{5.567779in}{2.379196in}}%
\pgfpathlineto{\pgfqpoint{5.710883in}{2.164223in}}%
\pgfpathlineto{\pgfqpoint{5.788906in}{2.159810in}}%
\pgfpathclose%
\pgfusepath{fill}%
\end{pgfscope}%
\begin{pgfscope}%
\pgfpathrectangle{\pgfqpoint{0.539299in}{0.078740in}}{\pgfqpoint{7.842520in}{7.842520in}}%
\pgfusepath{clip}%
\pgfsetbuttcap%
\pgfsetroundjoin%
\definecolor{currentfill}{rgb}{0.121148,0.592739,0.544641}%
\pgfsetfillcolor{currentfill}%
\pgfsetlinewidth{0.000000pt}%
\definecolor{currentstroke}{rgb}{0.179019,0.433756,0.557430}%
\pgfsetstrokecolor{currentstroke}%
\pgfsetdash{}{0pt}%
\pgfpathmoveto{\pgfqpoint{4.869240in}{3.622273in}}%
\pgfpathlineto{\pgfqpoint{4.787540in}{3.665841in}}%
\pgfpathlineto{\pgfqpoint{4.931146in}{3.442939in}}%
\pgfpathclose%
\pgfusepath{fill}%
\end{pgfscope}%
\begin{pgfscope}%
\pgfpathrectangle{\pgfqpoint{0.539299in}{0.078740in}}{\pgfqpoint{7.842520in}{7.842520in}}%
\pgfusepath{clip}%
\pgfsetbuttcap%
\pgfsetroundjoin%
\definecolor{currentfill}{rgb}{0.214000,0.722114,0.469588}%
\pgfsetfillcolor{currentfill}%
\pgfsetlinewidth{0.000000pt}%
\definecolor{currentstroke}{rgb}{0.177423,0.437527,0.557565}%
\pgfsetstrokecolor{currentstroke}%
\pgfsetdash{}{0pt}%
\pgfpathmoveto{\pgfqpoint{2.361965in}{4.410404in}}%
\pgfpathlineto{\pgfqpoint{2.281795in}{4.035068in}}%
\pgfpathlineto{\pgfqpoint{2.418849in}{4.087714in}}%
\pgfpathclose%
\pgfusepath{fill}%
\end{pgfscope}%
\begin{pgfscope}%
\pgfpathrectangle{\pgfqpoint{0.539299in}{0.078740in}}{\pgfqpoint{7.842520in}{7.842520in}}%
\pgfusepath{clip}%
\pgfsetbuttcap%
\pgfsetroundjoin%
\definecolor{currentfill}{rgb}{0.720391,0.870350,0.162603}%
\pgfsetfillcolor{currentfill}%
\pgfsetlinewidth{0.000000pt}%
\definecolor{currentstroke}{rgb}{0.175841,0.441290,0.557685}%
\pgfsetstrokecolor{currentstroke}%
\pgfsetdash{}{0pt}%
\pgfpathmoveto{\pgfqpoint{2.750456in}{5.265020in}}%
\pgfpathlineto{\pgfqpoint{2.613359in}{5.171039in}}%
\pgfpathlineto{\pgfqpoint{2.528143in}{4.972174in}}%
\pgfpathclose%
\pgfusepath{fill}%
\end{pgfscope}%
\begin{pgfscope}%
\pgfpathrectangle{\pgfqpoint{0.539299in}{0.078740in}}{\pgfqpoint{7.842520in}{7.842520in}}%
\pgfusepath{clip}%
\pgfsetbuttcap%
\pgfsetroundjoin%
\definecolor{currentfill}{rgb}{0.983868,0.904867,0.136897}%
\pgfsetfillcolor{currentfill}%
\pgfsetlinewidth{0.000000pt}%
\definecolor{currentstroke}{rgb}{0.174274,0.445044,0.557792}%
\pgfsetstrokecolor{currentstroke}%
\pgfsetdash{}{0pt}%
\pgfpathmoveto{\pgfqpoint{3.062895in}{5.558135in}}%
\pgfpathlineto{\pgfqpoint{3.203841in}{5.543255in}}%
\pgfpathlineto{\pgfqpoint{3.150177in}{5.616991in}}%
\pgfpathclose%
\pgfusepath{fill}%
\end{pgfscope}%
\begin{pgfscope}%
\pgfpathrectangle{\pgfqpoint{0.539299in}{0.078740in}}{\pgfqpoint{7.842520in}{7.842520in}}%
\pgfusepath{clip}%
\pgfsetbuttcap%
\pgfsetroundjoin%
\definecolor{currentfill}{rgb}{0.162016,0.687316,0.499129}%
\pgfsetfillcolor{currentfill}%
\pgfsetlinewidth{0.000000pt}%
\definecolor{currentstroke}{rgb}{0.172719,0.448791,0.557885}%
\pgfsetstrokecolor{currentstroke}%
\pgfsetdash{}{0pt}%
\pgfpathmoveto{\pgfqpoint{4.499656in}{4.110842in}}%
\pgfpathlineto{\pgfqpoint{4.726130in}{3.847156in}}%
\pgfpathlineto{\pgfqpoint{4.582787in}{4.072527in}}%
\pgfpathclose%
\pgfusepath{fill}%
\end{pgfscope}%
\begin{pgfscope}%
\pgfpathrectangle{\pgfqpoint{0.539299in}{0.078740in}}{\pgfqpoint{7.842520in}{7.842520in}}%
\pgfusepath{clip}%
\pgfsetbuttcap%
\pgfsetroundjoin%
\definecolor{currentfill}{rgb}{0.866013,0.889868,0.095953}%
\pgfsetfillcolor{currentfill}%
\pgfsetlinewidth{0.000000pt}%
\definecolor{currentstroke}{rgb}{0.171176,0.452530,0.557965}%
\pgfsetstrokecolor{currentstroke}%
\pgfsetdash{}{0pt}%
\pgfpathmoveto{\pgfqpoint{3.662699in}{5.404278in}}%
\pgfpathlineto{\pgfqpoint{3.576041in}{5.415869in}}%
\pgfpathlineto{\pgfqpoint{3.719555in}{5.277291in}}%
\pgfpathclose%
\pgfusepath{fill}%
\end{pgfscope}%
\begin{pgfscope}%
\pgfpathrectangle{\pgfqpoint{0.539299in}{0.078740in}}{\pgfqpoint{7.842520in}{7.842520in}}%
\pgfusepath{clip}%
\pgfsetbuttcap%
\pgfsetroundjoin%
\definecolor{currentfill}{rgb}{0.668054,0.861999,0.196293}%
\pgfsetfillcolor{currentfill}%
\pgfsetlinewidth{0.000000pt}%
\definecolor{currentstroke}{rgb}{0.169646,0.456262,0.558030}%
\pgfsetstrokecolor{currentstroke}%
\pgfsetdash{}{0pt}%
\pgfpathmoveto{\pgfqpoint{3.949148in}{5.080344in}}%
\pgfpathlineto{\pgfqpoint{3.863419in}{5.112875in}}%
\pgfpathlineto{\pgfqpoint{4.007443in}{4.927974in}}%
\pgfpathclose%
\pgfusepath{fill}%
\end{pgfscope}%
\begin{pgfscope}%
\pgfpathrectangle{\pgfqpoint{0.539299in}{0.078740in}}{\pgfqpoint{7.842520in}{7.842520in}}%
\pgfusepath{clip}%
\pgfsetbuttcap%
\pgfsetroundjoin%
\definecolor{currentfill}{rgb}{0.772852,0.877868,0.131109}%
\pgfsetfillcolor{currentfill}%
\pgfsetlinewidth{0.000000pt}%
\definecolor{currentstroke}{rgb}{0.168126,0.459988,0.558082}%
\pgfsetstrokecolor{currentstroke}%
\pgfsetdash{}{0pt}%
\pgfpathmoveto{\pgfqpoint{3.863419in}{5.112875in}}%
\pgfpathlineto{\pgfqpoint{3.805790in}{5.254759in}}%
\pgfpathlineto{\pgfqpoint{3.719555in}{5.277291in}}%
\pgfpathclose%
\pgfusepath{fill}%
\end{pgfscope}%
\begin{pgfscope}%
\pgfpathrectangle{\pgfqpoint{0.539299in}{0.078740in}}{\pgfqpoint{7.842520in}{7.842520in}}%
\pgfusepath{clip}%
\pgfsetbuttcap%
\pgfsetroundjoin%
\definecolor{currentfill}{rgb}{0.143343,0.522773,0.556295}%
\pgfsetfillcolor{currentfill}%
\pgfsetlinewidth{0.000000pt}%
\definecolor{currentstroke}{rgb}{0.166617,0.463708,0.558119}%
\pgfsetstrokecolor{currentstroke}%
\pgfsetdash{}{0pt}%
\pgfpathmoveto{\pgfqpoint{2.619520in}{3.617668in}}%
\pgfpathlineto{\pgfqpoint{2.405032in}{3.063501in}}%
\pgfpathlineto{\pgfqpoint{2.544311in}{3.042210in}}%
\pgfpathclose%
\pgfusepath{fill}%
\end{pgfscope}%
\begin{pgfscope}%
\pgfpathrectangle{\pgfqpoint{0.539299in}{0.078740in}}{\pgfqpoint{7.842520in}{7.842520in}}%
\pgfusepath{clip}%
\pgfsetbuttcap%
\pgfsetroundjoin%
\definecolor{currentfill}{rgb}{0.282910,0.105393,0.426902}%
\pgfsetfillcolor{currentfill}%
\pgfsetlinewidth{0.000000pt}%
\definecolor{currentstroke}{rgb}{0.165117,0.467423,0.558141}%
\pgfsetstrokecolor{currentstroke}%
\pgfsetdash{}{0pt}%
\pgfpathmoveto{\pgfqpoint{6.061409in}{1.481685in}}%
\pgfpathlineto{\pgfqpoint{6.139221in}{1.529769in}}%
\pgfpathlineto{\pgfqpoint{5.996574in}{1.738442in}}%
\pgfpathclose%
\pgfusepath{fill}%
\end{pgfscope}%
\begin{pgfscope}%
\pgfpathrectangle{\pgfqpoint{0.539299in}{0.078740in}}{\pgfqpoint{7.842520in}{7.842520in}}%
\pgfusepath{clip}%
\pgfsetbuttcap%
\pgfsetroundjoin%
\definecolor{currentfill}{rgb}{0.220057,0.343307,0.549413}%
\pgfsetfillcolor{currentfill}%
\pgfsetlinewidth{0.000000pt}%
\definecolor{currentstroke}{rgb}{0.163625,0.471133,0.558148}%
\pgfsetstrokecolor{currentstroke}%
\pgfsetdash{}{0pt}%
\pgfpathmoveto{\pgfqpoint{5.424487in}{2.595182in}}%
\pgfpathlineto{\pgfqpoint{5.567779in}{2.379196in}}%
\pgfpathlineto{\pgfqpoint{5.646299in}{2.364748in}}%
\pgfpathclose%
\pgfusepath{fill}%
\end{pgfscope}%
\begin{pgfscope}%
\pgfpathrectangle{\pgfqpoint{0.539299in}{0.078740in}}{\pgfqpoint{7.842520in}{7.842520in}}%
\pgfusepath{clip}%
\pgfsetbuttcap%
\pgfsetroundjoin%
\definecolor{currentfill}{rgb}{0.214000,0.722114,0.469588}%
\pgfsetfillcolor{currentfill}%
\pgfsetlinewidth{0.000000pt}%
\definecolor{currentstroke}{rgb}{0.162142,0.474838,0.558140}%
\pgfsetstrokecolor{currentstroke}%
\pgfsetdash{}{0pt}%
\pgfpathmoveto{\pgfqpoint{4.582787in}{4.072527in}}%
\pgfpathlineto{\pgfqpoint{4.439213in}{4.296234in}}%
\pgfpathlineto{\pgfqpoint{4.499656in}{4.110842in}}%
\pgfpathclose%
\pgfusepath{fill}%
\end{pgfscope}%
\begin{pgfscope}%
\pgfpathrectangle{\pgfqpoint{0.539299in}{0.078740in}}{\pgfqpoint{7.842520in}{7.842520in}}%
\pgfusepath{clip}%
\pgfsetbuttcap%
\pgfsetroundjoin%
\definecolor{currentfill}{rgb}{0.168126,0.459988,0.558082}%
\pgfsetfillcolor{currentfill}%
\pgfsetlinewidth{0.000000pt}%
\definecolor{currentstroke}{rgb}{0.160665,0.478540,0.558115}%
\pgfsetstrokecolor{currentstroke}%
\pgfsetdash{}{0pt}%
\pgfpathmoveto{\pgfqpoint{5.217726in}{3.002518in}}%
\pgfpathlineto{\pgfqpoint{5.137323in}{3.030184in}}%
\pgfpathlineto{\pgfqpoint{5.360735in}{2.786461in}}%
\pgfpathclose%
\pgfusepath{fill}%
\end{pgfscope}%
\begin{pgfscope}%
\pgfpathrectangle{\pgfqpoint{0.539299in}{0.078740in}}{\pgfqpoint{7.842520in}{7.842520in}}%
\pgfusepath{clip}%
\pgfsetbuttcap%
\pgfsetroundjoin%
\definecolor{currentfill}{rgb}{0.983868,0.904867,0.136897}%
\pgfsetfillcolor{currentfill}%
\pgfsetlinewidth{0.000000pt}%
\definecolor{currentstroke}{rgb}{0.159194,0.482237,0.558073}%
\pgfsetstrokecolor{currentstroke}%
\pgfsetdash{}{0pt}%
\pgfpathmoveto{\pgfqpoint{3.203841in}{5.543255in}}%
\pgfpathlineto{\pgfqpoint{3.433110in}{5.522817in}}%
\pgfpathlineto{\pgfqpoint{3.291045in}{5.591989in}}%
\pgfpathclose%
\pgfusepath{fill}%
\end{pgfscope}%
\begin{pgfscope}%
\pgfpathrectangle{\pgfqpoint{0.539299in}{0.078740in}}{\pgfqpoint{7.842520in}{7.842520in}}%
\pgfusepath{clip}%
\pgfsetbuttcap%
\pgfsetroundjoin%
\definecolor{currentfill}{rgb}{0.214298,0.355619,0.551184}%
\pgfsetfillcolor{currentfill}%
\pgfsetlinewidth{0.000000pt}%
\definecolor{currentstroke}{rgb}{0.157729,0.485932,0.558013}%
\pgfsetstrokecolor{currentstroke}%
\pgfsetdash{}{0pt}%
\pgfpathmoveto{\pgfqpoint{2.751870in}{2.294211in}}%
\pgfpathlineto{\pgfqpoint{2.892470in}{2.245647in}}%
\pgfpathlineto{\pgfqpoint{2.966934in}{2.913287in}}%
\pgfpathclose%
\pgfusepath{fill}%
\end{pgfscope}%
\begin{pgfscope}%
\pgfpathrectangle{\pgfqpoint{0.539299in}{0.078740in}}{\pgfqpoint{7.842520in}{7.842520in}}%
\pgfusepath{clip}%
\pgfsetbuttcap%
\pgfsetroundjoin%
\definecolor{currentfill}{rgb}{0.128087,0.647749,0.523491}%
\pgfsetfillcolor{currentfill}%
\pgfsetlinewidth{0.000000pt}%
\definecolor{currentstroke}{rgb}{0.156270,0.489624,0.557936}%
\pgfsetstrokecolor{currentstroke}%
\pgfsetdash{}{0pt}%
\pgfpathmoveto{\pgfqpoint{4.787540in}{3.665841in}}%
\pgfpathlineto{\pgfqpoint{4.726130in}{3.847156in}}%
\pgfpathlineto{\pgfqpoint{4.643709in}{3.889018in}}%
\pgfpathclose%
\pgfusepath{fill}%
\end{pgfscope}%
\begin{pgfscope}%
\pgfpathrectangle{\pgfqpoint{0.539299in}{0.078740in}}{\pgfqpoint{7.842520in}{7.842520in}}%
\pgfusepath{clip}%
\pgfsetbuttcap%
\pgfsetroundjoin%
\definecolor{currentfill}{rgb}{0.185556,0.418570,0.556753}%
\pgfsetfillcolor{currentfill}%
\pgfsetlinewidth{0.000000pt}%
\definecolor{currentstroke}{rgb}{0.154815,0.493313,0.557840}%
\pgfsetstrokecolor{currentstroke}%
\pgfsetdash{}{0pt}%
\pgfpathmoveto{\pgfqpoint{2.684453in}{3.009085in}}%
\pgfpathlineto{\pgfqpoint{2.751870in}{2.294211in}}%
\pgfpathlineto{\pgfqpoint{2.825356in}{2.965660in}}%
\pgfpathclose%
\pgfusepath{fill}%
\end{pgfscope}%
\begin{pgfscope}%
\pgfpathrectangle{\pgfqpoint{0.539299in}{0.078740in}}{\pgfqpoint{7.842520in}{7.842520in}}%
\pgfusepath{clip}%
\pgfsetbuttcap%
\pgfsetroundjoin%
\definecolor{currentfill}{rgb}{0.595839,0.848717,0.243329}%
\pgfsetfillcolor{currentfill}%
\pgfsetlinewidth{0.000000pt}%
\definecolor{currentstroke}{rgb}{0.153364,0.497000,0.557724}%
\pgfsetstrokecolor{currentstroke}%
\pgfsetdash{}{0pt}%
\pgfpathmoveto{\pgfqpoint{2.665187in}{5.062348in}}%
\pgfpathlineto{\pgfqpoint{2.528143in}{4.972174in}}%
\pgfpathlineto{\pgfqpoint{2.444220in}{4.720662in}}%
\pgfpathclose%
\pgfusepath{fill}%
\end{pgfscope}%
\begin{pgfscope}%
\pgfpathrectangle{\pgfqpoint{0.539299in}{0.078740in}}{\pgfqpoint{7.842520in}{7.842520in}}%
\pgfusepath{clip}%
\pgfsetbuttcap%
\pgfsetroundjoin%
\definecolor{currentfill}{rgb}{0.127568,0.566949,0.550556}%
\pgfsetfillcolor{currentfill}%
\pgfsetlinewidth{0.000000pt}%
\definecolor{currentstroke}{rgb}{0.151918,0.500685,0.557587}%
\pgfsetstrokecolor{currentstroke}%
\pgfsetdash{}{0pt}%
\pgfpathmoveto{\pgfqpoint{2.479796in}{3.628915in}}%
\pgfpathlineto{\pgfqpoint{2.405032in}{3.063501in}}%
\pgfpathlineto{\pgfqpoint{2.619520in}{3.617668in}}%
\pgfpathclose%
\pgfusepath{fill}%
\end{pgfscope}%
\begin{pgfscope}%
\pgfpathrectangle{\pgfqpoint{0.539299in}{0.078740in}}{\pgfqpoint{7.842520in}{7.842520in}}%
\pgfusepath{clip}%
\pgfsetbuttcap%
\pgfsetroundjoin%
\definecolor{currentfill}{rgb}{0.945636,0.899815,0.112838}%
\pgfsetfillcolor{currentfill}%
\pgfsetlinewidth{0.000000pt}%
\definecolor{currentstroke}{rgb}{0.150476,0.504369,0.557430}%
\pgfsetstrokecolor{currentstroke}%
\pgfsetdash{}{0pt}%
\pgfpathmoveto{\pgfqpoint{3.062895in}{5.558135in}}%
\pgfpathlineto{\pgfqpoint{2.836696in}{5.416830in}}%
\pgfpathlineto{\pgfqpoint{2.975959in}{5.457547in}}%
\pgfpathclose%
\pgfusepath{fill}%
\end{pgfscope}%
\begin{pgfscope}%
\pgfpathrectangle{\pgfqpoint{0.539299in}{0.078740in}}{\pgfqpoint{7.842520in}{7.842520in}}%
\pgfusepath{clip}%
\pgfsetbuttcap%
\pgfsetroundjoin%
\definecolor{currentfill}{rgb}{0.440137,0.811138,0.340967}%
\pgfsetfillcolor{currentfill}%
\pgfsetlinewidth{0.000000pt}%
\definecolor{currentstroke}{rgb}{0.149039,0.508051,0.557250}%
\pgfsetstrokecolor{currentstroke}%
\pgfsetdash{}{0pt}%
\pgfpathmoveto{\pgfqpoint{2.581236in}{4.802712in}}%
\pgfpathlineto{\pgfqpoint{2.444220in}{4.720662in}}%
\pgfpathlineto{\pgfqpoint{2.361965in}{4.410404in}}%
\pgfpathclose%
\pgfusepath{fill}%
\end{pgfscope}%
\begin{pgfscope}%
\pgfpathrectangle{\pgfqpoint{0.539299in}{0.078740in}}{\pgfqpoint{7.842520in}{7.842520in}}%
\pgfusepath{clip}%
\pgfsetbuttcap%
\pgfsetroundjoin%
\definecolor{currentfill}{rgb}{0.154815,0.493313,0.557840}%
\pgfsetfillcolor{currentfill}%
\pgfsetlinewidth{0.000000pt}%
\definecolor{currentstroke}{rgb}{0.147607,0.511733,0.557049}%
\pgfsetstrokecolor{currentstroke}%
\pgfsetdash{}{0pt}%
\pgfpathmoveto{\pgfqpoint{5.217726in}{3.002518in}}%
\pgfpathlineto{\pgfqpoint{5.074536in}{3.221540in}}%
\pgfpathlineto{\pgfqpoint{5.137323in}{3.030184in}}%
\pgfpathclose%
\pgfusepath{fill}%
\end{pgfscope}%
\begin{pgfscope}%
\pgfpathrectangle{\pgfqpoint{0.539299in}{0.078740in}}{\pgfqpoint{7.842520in}{7.842520in}}%
\pgfusepath{clip}%
\pgfsetbuttcap%
\pgfsetroundjoin%
\definecolor{currentfill}{rgb}{0.153894,0.680203,0.504172}%
\pgfsetfillcolor{currentfill}%
\pgfsetlinewidth{0.000000pt}%
\definecolor{currentstroke}{rgb}{0.146180,0.515413,0.556823}%
\pgfsetstrokecolor{currentstroke}%
\pgfsetdash{}{0pt}%
\pgfpathmoveto{\pgfqpoint{4.643709in}{3.889018in}}%
\pgfpathlineto{\pgfqpoint{4.726130in}{3.847156in}}%
\pgfpathlineto{\pgfqpoint{4.499656in}{4.110842in}}%
\pgfpathclose%
\pgfusepath{fill}%
\end{pgfscope}%
\begin{pgfscope}%
\pgfpathrectangle{\pgfqpoint{0.539299in}{0.078740in}}{\pgfqpoint{7.842520in}{7.842520in}}%
\pgfusepath{clip}%
\pgfsetbuttcap%
\pgfsetroundjoin%
\definecolor{currentfill}{rgb}{0.187231,0.414746,0.556547}%
\pgfsetfillcolor{currentfill}%
\pgfsetlinewidth{0.000000pt}%
\definecolor{currentstroke}{rgb}{0.144759,0.519093,0.556572}%
\pgfsetstrokecolor{currentstroke}%
\pgfsetdash{}{0pt}%
\pgfpathmoveto{\pgfqpoint{5.424487in}{2.595182in}}%
\pgfpathlineto{\pgfqpoint{5.360735in}{2.786461in}}%
\pgfpathlineto{\pgfqpoint{5.281003in}{2.812185in}}%
\pgfpathclose%
\pgfusepath{fill}%
\end{pgfscope}%
\begin{pgfscope}%
\pgfpathrectangle{\pgfqpoint{0.539299in}{0.078740in}}{\pgfqpoint{7.842520in}{7.842520in}}%
\pgfusepath{clip}%
\pgfsetbuttcap%
\pgfsetroundjoin%
\definecolor{currentfill}{rgb}{0.274128,0.199721,0.498911}%
\pgfsetfillcolor{currentfill}%
\pgfsetlinewidth{0.000000pt}%
\definecolor{currentstroke}{rgb}{0.143343,0.522773,0.556295}%
\pgfsetstrokecolor{currentstroke}%
\pgfsetdash{}{0pt}%
\pgfpathmoveto{\pgfqpoint{5.996574in}{1.738442in}}%
\pgfpathlineto{\pgfqpoint{5.853809in}{1.950429in}}%
\pgfpathlineto{\pgfqpoint{5.775575in}{1.941647in}}%
\pgfpathclose%
\pgfusepath{fill}%
\end{pgfscope}%
\begin{pgfscope}%
\pgfpathrectangle{\pgfqpoint{0.539299in}{0.078740in}}{\pgfqpoint{7.842520in}{7.842520in}}%
\pgfusepath{clip}%
\pgfsetbuttcap%
\pgfsetroundjoin%
\definecolor{currentfill}{rgb}{0.288921,0.758394,0.428426}%
\pgfsetfillcolor{currentfill}%
\pgfsetlinewidth{0.000000pt}%
\definecolor{currentstroke}{rgb}{0.141935,0.526453,0.555991}%
\pgfsetstrokecolor{currentstroke}%
\pgfsetdash{}{0pt}%
\pgfpathmoveto{\pgfqpoint{2.418849in}{4.087714in}}%
\pgfpathlineto{\pgfqpoint{2.498985in}{4.479903in}}%
\pgfpathlineto{\pgfqpoint{2.361965in}{4.410404in}}%
\pgfpathclose%
\pgfusepath{fill}%
\end{pgfscope}%
\begin{pgfscope}%
\pgfpathrectangle{\pgfqpoint{0.539299in}{0.078740in}}{\pgfqpoint{7.842520in}{7.842520in}}%
\pgfusepath{clip}%
\pgfsetbuttcap%
\pgfsetroundjoin%
\definecolor{currentfill}{rgb}{0.352360,0.783011,0.392636}%
\pgfsetfillcolor{currentfill}%
\pgfsetlinewidth{0.000000pt}%
\definecolor{currentstroke}{rgb}{0.140536,0.530132,0.555659}%
\pgfsetstrokecolor{currentstroke}%
\pgfsetdash{}{0pt}%
\pgfpathmoveto{\pgfqpoint{4.295430in}{4.515609in}}%
\pgfpathlineto{\pgfqpoint{4.210962in}{4.541645in}}%
\pgfpathlineto{\pgfqpoint{4.439213in}{4.296234in}}%
\pgfpathclose%
\pgfusepath{fill}%
\end{pgfscope}%
\begin{pgfscope}%
\pgfpathrectangle{\pgfqpoint{0.539299in}{0.078740in}}{\pgfqpoint{7.842520in}{7.842520in}}%
\pgfusepath{clip}%
\pgfsetbuttcap%
\pgfsetroundjoin%
\definecolor{currentfill}{rgb}{0.720391,0.870350,0.162603}%
\pgfsetfillcolor{currentfill}%
\pgfsetlinewidth{0.000000pt}%
\definecolor{currentstroke}{rgb}{0.139147,0.533812,0.555298}%
\pgfsetstrokecolor{currentstroke}%
\pgfsetdash{}{0pt}%
\pgfpathmoveto{\pgfqpoint{2.528143in}{4.972174in}}%
\pgfpathlineto{\pgfqpoint{2.665187in}{5.062348in}}%
\pgfpathlineto{\pgfqpoint{2.750456in}{5.265020in}}%
\pgfpathclose%
\pgfusepath{fill}%
\end{pgfscope}%
\begin{pgfscope}%
\pgfpathrectangle{\pgfqpoint{0.539299in}{0.078740in}}{\pgfqpoint{7.842520in}{7.842520in}}%
\pgfusepath{clip}%
\pgfsetbuttcap%
\pgfsetroundjoin%
\definecolor{currentfill}{rgb}{0.128087,0.647749,0.523491}%
\pgfsetfillcolor{currentfill}%
\pgfsetlinewidth{0.000000pt}%
\definecolor{currentstroke}{rgb}{0.137770,0.537492,0.554906}%
\pgfsetstrokecolor{currentstroke}%
\pgfsetdash{}{0pt}%
\pgfpathmoveto{\pgfqpoint{2.479796in}{3.628915in}}%
\pgfpathlineto{\pgfqpoint{2.557549in}{4.110498in}}%
\pgfpathlineto{\pgfqpoint{2.341276in}{3.620065in}}%
\pgfpathclose%
\pgfusepath{fill}%
\end{pgfscope}%
\begin{pgfscope}%
\pgfpathrectangle{\pgfqpoint{0.539299in}{0.078740in}}{\pgfqpoint{7.842520in}{7.842520in}}%
\pgfusepath{clip}%
\pgfsetbuttcap%
\pgfsetroundjoin%
\definecolor{currentfill}{rgb}{0.141935,0.526453,0.555991}%
\pgfsetfillcolor{currentfill}%
\pgfsetlinewidth{0.000000pt}%
\definecolor{currentstroke}{rgb}{0.136408,0.541173,0.554483}%
\pgfsetstrokecolor{currentstroke}%
\pgfsetdash{}{0pt}%
\pgfpathmoveto{\pgfqpoint{2.684453in}{3.009085in}}%
\pgfpathlineto{\pgfqpoint{2.619520in}{3.617668in}}%
\pgfpathlineto{\pgfqpoint{2.544311in}{3.042210in}}%
\pgfpathclose%
\pgfusepath{fill}%
\end{pgfscope}%
\begin{pgfscope}%
\pgfpathrectangle{\pgfqpoint{0.539299in}{0.078740in}}{\pgfqpoint{7.842520in}{7.842520in}}%
\pgfusepath{clip}%
\pgfsetbuttcap%
\pgfsetroundjoin%
\definecolor{currentfill}{rgb}{0.153894,0.680203,0.504172}%
\pgfsetfillcolor{currentfill}%
\pgfsetlinewidth{0.000000pt}%
\definecolor{currentstroke}{rgb}{0.135066,0.544853,0.554029}%
\pgfsetstrokecolor{currentstroke}%
\pgfsetdash{}{0pt}%
\pgfpathmoveto{\pgfqpoint{2.341276in}{3.620065in}}%
\pgfpathlineto{\pgfqpoint{2.557549in}{4.110498in}}%
\pgfpathlineto{\pgfqpoint{2.418849in}{4.087714in}}%
\pgfpathclose%
\pgfusepath{fill}%
\end{pgfscope}%
\begin{pgfscope}%
\pgfpathrectangle{\pgfqpoint{0.539299in}{0.078740in}}{\pgfqpoint{7.842520in}{7.842520in}}%
\pgfusepath{clip}%
\pgfsetbuttcap%
\pgfsetroundjoin%
\definecolor{currentfill}{rgb}{0.216210,0.351535,0.550627}%
\pgfsetfillcolor{currentfill}%
\pgfsetlinewidth{0.000000pt}%
\definecolor{currentstroke}{rgb}{0.133743,0.548535,0.553541}%
\pgfsetstrokecolor{currentstroke}%
\pgfsetdash{}{0pt}%
\pgfpathmoveto{\pgfqpoint{2.966934in}{2.913287in}}%
\pgfpathlineto{\pgfqpoint{2.892470in}{2.245647in}}%
\pgfpathlineto{\pgfqpoint{3.033612in}{2.193144in}}%
\pgfpathclose%
\pgfusepath{fill}%
\end{pgfscope}%
\begin{pgfscope}%
\pgfpathrectangle{\pgfqpoint{0.539299in}{0.078740in}}{\pgfqpoint{7.842520in}{7.842520in}}%
\pgfusepath{clip}%
\pgfsetbuttcap%
\pgfsetroundjoin%
\definecolor{currentfill}{rgb}{0.282884,0.135920,0.453427}%
\pgfsetfillcolor{currentfill}%
\pgfsetlinewidth{0.000000pt}%
\definecolor{currentstroke}{rgb}{0.132444,0.552216,0.553018}%
\pgfsetstrokecolor{currentstroke}%
\pgfsetdash{}{0pt}%
\pgfpathmoveto{\pgfqpoint{5.918659in}{1.713508in}}%
\pgfpathlineto{\pgfqpoint{6.061409in}{1.481685in}}%
\pgfpathlineto{\pgfqpoint{5.996574in}{1.738442in}}%
\pgfpathclose%
\pgfusepath{fill}%
\end{pgfscope}%
\begin{pgfscope}%
\pgfpathrectangle{\pgfqpoint{0.539299in}{0.078740in}}{\pgfqpoint{7.842520in}{7.842520in}}%
\pgfusepath{clip}%
\pgfsetbuttcap%
\pgfsetroundjoin%
\definecolor{currentfill}{rgb}{0.171176,0.452530,0.557965}%
\pgfsetfillcolor{currentfill}%
\pgfsetlinewidth{0.000000pt}%
\definecolor{currentstroke}{rgb}{0.131172,0.555899,0.552459}%
\pgfsetstrokecolor{currentstroke}%
\pgfsetdash{}{0pt}%
\pgfpathmoveto{\pgfqpoint{5.360735in}{2.786461in}}%
\pgfpathlineto{\pgfqpoint{5.137323in}{3.030184in}}%
\pgfpathlineto{\pgfqpoint{5.281003in}{2.812185in}}%
\pgfpathclose%
\pgfusepath{fill}%
\end{pgfscope}%
\begin{pgfscope}%
\pgfpathrectangle{\pgfqpoint{0.539299in}{0.078740in}}{\pgfqpoint{7.842520in}{7.842520in}}%
\pgfusepath{clip}%
\pgfsetbuttcap%
\pgfsetroundjoin%
\definecolor{currentfill}{rgb}{0.185556,0.418570,0.556753}%
\pgfsetfillcolor{currentfill}%
\pgfsetlinewidth{0.000000pt}%
\definecolor{currentstroke}{rgb}{0.129933,0.559582,0.551864}%
\pgfsetstrokecolor{currentstroke}%
\pgfsetdash{}{0pt}%
\pgfpathmoveto{\pgfqpoint{2.966934in}{2.913287in}}%
\pgfpathlineto{\pgfqpoint{2.825356in}{2.965660in}}%
\pgfpathlineto{\pgfqpoint{2.751870in}{2.294211in}}%
\pgfpathclose%
\pgfusepath{fill}%
\end{pgfscope}%
\begin{pgfscope}%
\pgfpathrectangle{\pgfqpoint{0.539299in}{0.078740in}}{\pgfqpoint{7.842520in}{7.842520in}}%
\pgfusepath{clip}%
\pgfsetbuttcap%
\pgfsetroundjoin%
\definecolor{currentfill}{rgb}{0.430983,0.808473,0.346476}%
\pgfsetfillcolor{currentfill}%
\pgfsetlinewidth{0.000000pt}%
\definecolor{currentstroke}{rgb}{0.128729,0.563265,0.551229}%
\pgfsetstrokecolor{currentstroke}%
\pgfsetdash{}{0pt}%
\pgfpathmoveto{\pgfqpoint{4.295430in}{4.515609in}}%
\pgfpathlineto{\pgfqpoint{4.151482in}{4.727445in}}%
\pgfpathlineto{\pgfqpoint{4.210962in}{4.541645in}}%
\pgfpathclose%
\pgfusepath{fill}%
\end{pgfscope}%
\begin{pgfscope}%
\pgfpathrectangle{\pgfqpoint{0.539299in}{0.078740in}}{\pgfqpoint{7.842520in}{7.842520in}}%
\pgfusepath{clip}%
\pgfsetbuttcap%
\pgfsetroundjoin%
\definecolor{currentfill}{rgb}{0.412913,0.803041,0.357269}%
\pgfsetfillcolor{currentfill}%
\pgfsetlinewidth{0.000000pt}%
\definecolor{currentstroke}{rgb}{0.127568,0.566949,0.550556}%
\pgfsetstrokecolor{currentstroke}%
\pgfsetdash{}{0pt}%
\pgfpathmoveto{\pgfqpoint{2.361965in}{4.410404in}}%
\pgfpathlineto{\pgfqpoint{2.498985in}{4.479903in}}%
\pgfpathlineto{\pgfqpoint{2.581236in}{4.802712in}}%
\pgfpathclose%
\pgfusepath{fill}%
\end{pgfscope}%
\begin{pgfscope}%
\pgfpathrectangle{\pgfqpoint{0.539299in}{0.078740in}}{\pgfqpoint{7.842520in}{7.842520in}}%
\pgfusepath{clip}%
\pgfsetbuttcap%
\pgfsetroundjoin%
\definecolor{currentfill}{rgb}{0.128729,0.563265,0.551229}%
\pgfsetfillcolor{currentfill}%
\pgfsetlinewidth{0.000000pt}%
\definecolor{currentstroke}{rgb}{0.126453,0.570633,0.549841}%
\pgfsetstrokecolor{currentstroke}%
\pgfsetdash{}{0pt}%
\pgfpathmoveto{\pgfqpoint{5.074536in}{3.221540in}}%
\pgfpathlineto{\pgfqpoint{4.931146in}{3.442939in}}%
\pgfpathlineto{\pgfqpoint{4.849351in}{3.468201in}}%
\pgfpathclose%
\pgfusepath{fill}%
\end{pgfscope}%
\begin{pgfscope}%
\pgfpathrectangle{\pgfqpoint{0.539299in}{0.078740in}}{\pgfqpoint{7.842520in}{7.842520in}}%
\pgfusepath{clip}%
\pgfsetbuttcap%
\pgfsetroundjoin%
\definecolor{currentfill}{rgb}{0.253935,0.265254,0.529983}%
\pgfsetfillcolor{currentfill}%
\pgfsetlinewidth{0.000000pt}%
\definecolor{currentstroke}{rgb}{0.125394,0.574318,0.549086}%
\pgfsetstrokecolor{currentstroke}%
\pgfsetdash{}{0pt}%
\pgfpathmoveto{\pgfqpoint{5.710883in}{2.164223in}}%
\pgfpathlineto{\pgfqpoint{5.632182in}{2.166073in}}%
\pgfpathlineto{\pgfqpoint{5.853809in}{1.950429in}}%
\pgfpathclose%
\pgfusepath{fill}%
\end{pgfscope}%
\begin{pgfscope}%
\pgfpathrectangle{\pgfqpoint{0.539299in}{0.078740in}}{\pgfqpoint{7.842520in}{7.842520in}}%
\pgfusepath{clip}%
\pgfsetbuttcap%
\pgfsetroundjoin%
\definecolor{currentfill}{rgb}{0.935904,0.898570,0.108131}%
\pgfsetfillcolor{currentfill}%
\pgfsetlinewidth{0.000000pt}%
\definecolor{currentstroke}{rgb}{0.124395,0.578002,0.548287}%
\pgfsetstrokecolor{currentstroke}%
\pgfsetdash{}{0pt}%
\pgfpathmoveto{\pgfqpoint{3.489230in}{5.389845in}}%
\pgfpathlineto{\pgfqpoint{3.576041in}{5.415869in}}%
\pgfpathlineto{\pgfqpoint{3.433110in}{5.522817in}}%
\pgfpathclose%
\pgfusepath{fill}%
\end{pgfscope}%
\begin{pgfscope}%
\pgfpathrectangle{\pgfqpoint{0.539299in}{0.078740in}}{\pgfqpoint{7.842520in}{7.842520in}}%
\pgfusepath{clip}%
\pgfsetbuttcap%
\pgfsetroundjoin%
\definecolor{currentfill}{rgb}{0.585678,0.846661,0.249897}%
\pgfsetfillcolor{currentfill}%
\pgfsetlinewidth{0.000000pt}%
\definecolor{currentstroke}{rgb}{0.123463,0.581687,0.547445}%
\pgfsetstrokecolor{currentstroke}%
\pgfsetdash{}{0pt}%
\pgfpathmoveto{\pgfqpoint{2.581236in}{4.802712in}}%
\pgfpathlineto{\pgfqpoint{2.665187in}{5.062348in}}%
\pgfpathlineto{\pgfqpoint{2.444220in}{4.720662in}}%
\pgfpathclose%
\pgfusepath{fill}%
\end{pgfscope}%
\begin{pgfscope}%
\pgfpathrectangle{\pgfqpoint{0.539299in}{0.078740in}}{\pgfqpoint{7.842520in}{7.842520in}}%
\pgfusepath{clip}%
\pgfsetbuttcap%
\pgfsetroundjoin%
\definecolor{currentfill}{rgb}{0.983868,0.904867,0.136897}%
\pgfsetfillcolor{currentfill}%
\pgfsetlinewidth{0.000000pt}%
\definecolor{currentstroke}{rgb}{0.122606,0.585371,0.546557}%
\pgfsetstrokecolor{currentstroke}%
\pgfsetdash{}{0pt}%
\pgfpathmoveto{\pgfqpoint{3.346062in}{5.485171in}}%
\pgfpathlineto{\pgfqpoint{3.433110in}{5.522817in}}%
\pgfpathlineto{\pgfqpoint{3.203841in}{5.543255in}}%
\pgfpathclose%
\pgfusepath{fill}%
\end{pgfscope}%
\begin{pgfscope}%
\pgfpathrectangle{\pgfqpoint{0.539299in}{0.078740in}}{\pgfqpoint{7.842520in}{7.842520in}}%
\pgfusepath{clip}%
\pgfsetbuttcap%
\pgfsetroundjoin%
\definecolor{currentfill}{rgb}{0.876168,0.891125,0.095250}%
\pgfsetfillcolor{currentfill}%
\pgfsetlinewidth{0.000000pt}%
\definecolor{currentstroke}{rgb}{0.121831,0.589055,0.545623}%
\pgfsetstrokecolor{currentstroke}%
\pgfsetdash{}{0pt}%
\pgfpathmoveto{\pgfqpoint{2.836696in}{5.416830in}}%
\pgfpathlineto{\pgfqpoint{2.750456in}{5.265020in}}%
\pgfpathlineto{\pgfqpoint{2.889650in}{5.309100in}}%
\pgfpathclose%
\pgfusepath{fill}%
\end{pgfscope}%
\begin{pgfscope}%
\pgfpathrectangle{\pgfqpoint{0.539299in}{0.078740in}}{\pgfqpoint{7.842520in}{7.842520in}}%
\pgfusepath{clip}%
\pgfsetbuttcap%
\pgfsetroundjoin%
\definecolor{currentfill}{rgb}{0.259857,0.745492,0.444467}%
\pgfsetfillcolor{currentfill}%
\pgfsetlinewidth{0.000000pt}%
\definecolor{currentstroke}{rgb}{0.121148,0.592739,0.544641}%
\pgfsetstrokecolor{currentstroke}%
\pgfsetdash{}{0pt}%
\pgfpathmoveto{\pgfqpoint{4.499656in}{4.110842in}}%
\pgfpathlineto{\pgfqpoint{4.439213in}{4.296234in}}%
\pgfpathlineto{\pgfqpoint{4.355396in}{4.329235in}}%
\pgfpathclose%
\pgfusepath{fill}%
\end{pgfscope}%
\begin{pgfscope}%
\pgfpathrectangle{\pgfqpoint{0.539299in}{0.078740in}}{\pgfqpoint{7.842520in}{7.842520in}}%
\pgfusepath{clip}%
\pgfsetbuttcap%
\pgfsetroundjoin%
\definecolor{currentfill}{rgb}{0.277134,0.185228,0.489898}%
\pgfsetfillcolor{currentfill}%
\pgfsetlinewidth{0.000000pt}%
\definecolor{currentstroke}{rgb}{0.120565,0.596422,0.543611}%
\pgfsetstrokecolor{currentstroke}%
\pgfsetdash{}{0pt}%
\pgfpathmoveto{\pgfqpoint{5.996574in}{1.738442in}}%
\pgfpathlineto{\pgfqpoint{5.775575in}{1.941647in}}%
\pgfpathlineto{\pgfqpoint{5.918659in}{1.713508in}}%
\pgfpathclose%
\pgfusepath{fill}%
\end{pgfscope}%
\begin{pgfscope}%
\pgfpathrectangle{\pgfqpoint{0.539299in}{0.078740in}}{\pgfqpoint{7.842520in}{7.842520in}}%
\pgfusepath{clip}%
\pgfsetbuttcap%
\pgfsetroundjoin%
\definecolor{currentfill}{rgb}{0.555484,0.840254,0.269281}%
\pgfsetfillcolor{currentfill}%
\pgfsetlinewidth{0.000000pt}%
\definecolor{currentstroke}{rgb}{0.120092,0.600104,0.542530}%
\pgfsetstrokecolor{currentstroke}%
\pgfsetdash{}{0pt}%
\pgfpathmoveto{\pgfqpoint{4.066411in}{4.745025in}}%
\pgfpathlineto{\pgfqpoint{4.151482in}{4.727445in}}%
\pgfpathlineto{\pgfqpoint{4.007443in}{4.927974in}}%
\pgfpathclose%
\pgfusepath{fill}%
\end{pgfscope}%
\begin{pgfscope}%
\pgfpathrectangle{\pgfqpoint{0.539299in}{0.078740in}}{\pgfqpoint{7.842520in}{7.842520in}}%
\pgfusepath{clip}%
\pgfsetbuttcap%
\pgfsetroundjoin%
\definecolor{currentfill}{rgb}{0.916242,0.896091,0.100717}%
\pgfsetfillcolor{currentfill}%
\pgfsetlinewidth{0.000000pt}%
\definecolor{currentstroke}{rgb}{0.119738,0.603785,0.541400}%
\pgfsetstrokecolor{currentstroke}%
\pgfsetdash{}{0pt}%
\pgfpathmoveto{\pgfqpoint{2.975959in}{5.457547in}}%
\pgfpathlineto{\pgfqpoint{2.836696in}{5.416830in}}%
\pgfpathlineto{\pgfqpoint{2.889650in}{5.309100in}}%
\pgfpathclose%
\pgfusepath{fill}%
\end{pgfscope}%
\begin{pgfscope}%
\pgfpathrectangle{\pgfqpoint{0.539299in}{0.078740in}}{\pgfqpoint{7.842520in}{7.842520in}}%
\pgfusepath{clip}%
\pgfsetbuttcap%
\pgfsetroundjoin%
\definecolor{currentfill}{rgb}{0.896320,0.893616,0.096335}%
\pgfsetfillcolor{currentfill}%
\pgfsetlinewidth{0.000000pt}%
\definecolor{currentstroke}{rgb}{0.119512,0.607464,0.540218}%
\pgfsetstrokecolor{currentstroke}%
\pgfsetdash{}{0pt}%
\pgfpathmoveto{\pgfqpoint{3.719555in}{5.277291in}}%
\pgfpathlineto{\pgfqpoint{3.576041in}{5.415869in}}%
\pgfpathlineto{\pgfqpoint{3.489230in}{5.389845in}}%
\pgfpathclose%
\pgfusepath{fill}%
\end{pgfscope}%
\begin{pgfscope}%
\pgfpathrectangle{\pgfqpoint{0.539299in}{0.078740in}}{\pgfqpoint{7.842520in}{7.842520in}}%
\pgfusepath{clip}%
\pgfsetbuttcap%
\pgfsetroundjoin%
\definecolor{currentfill}{rgb}{0.983868,0.904867,0.136897}%
\pgfsetfillcolor{currentfill}%
\pgfsetlinewidth{0.000000pt}%
\definecolor{currentstroke}{rgb}{0.119423,0.611141,0.538982}%
\pgfsetstrokecolor{currentstroke}%
\pgfsetdash{}{0pt}%
\pgfpathmoveto{\pgfqpoint{3.116915in}{5.451034in}}%
\pgfpathlineto{\pgfqpoint{3.203841in}{5.543255in}}%
\pgfpathlineto{\pgfqpoint{3.062895in}{5.558135in}}%
\pgfpathclose%
\pgfusepath{fill}%
\end{pgfscope}%
\begin{pgfscope}%
\pgfpathrectangle{\pgfqpoint{0.539299in}{0.078740in}}{\pgfqpoint{7.842520in}{7.842520in}}%
\pgfusepath{clip}%
\pgfsetbuttcap%
\pgfsetroundjoin%
\definecolor{currentfill}{rgb}{0.237441,0.305202,0.541921}%
\pgfsetfillcolor{currentfill}%
\pgfsetlinewidth{0.000000pt}%
\definecolor{currentstroke}{rgb}{0.119483,0.614817,0.537692}%
\pgfsetstrokecolor{currentstroke}%
\pgfsetdash{}{0pt}%
\pgfpathmoveto{\pgfqpoint{5.567779in}{2.379196in}}%
\pgfpathlineto{\pgfqpoint{5.632182in}{2.166073in}}%
\pgfpathlineto{\pgfqpoint{5.710883in}{2.164223in}}%
\pgfpathclose%
\pgfusepath{fill}%
\end{pgfscope}%
\begin{pgfscope}%
\pgfpathrectangle{\pgfqpoint{0.539299in}{0.078740in}}{\pgfqpoint{7.842520in}{7.842520in}}%
\pgfusepath{clip}%
\pgfsetbuttcap%
\pgfsetroundjoin%
\definecolor{currentfill}{rgb}{0.120565,0.596422,0.543611}%
\pgfsetfillcolor{currentfill}%
\pgfsetlinewidth{0.000000pt}%
\definecolor{currentstroke}{rgb}{0.119699,0.618490,0.536347}%
\pgfsetstrokecolor{currentstroke}%
\pgfsetdash{}{0pt}%
\pgfpathmoveto{\pgfqpoint{4.849351in}{3.468201in}}%
\pgfpathlineto{\pgfqpoint{4.931146in}{3.442939in}}%
\pgfpathlineto{\pgfqpoint{4.787540in}{3.665841in}}%
\pgfpathclose%
\pgfusepath{fill}%
\end{pgfscope}%
\begin{pgfscope}%
\pgfpathrectangle{\pgfqpoint{0.539299in}{0.078740in}}{\pgfqpoint{7.842520in}{7.842520in}}%
\pgfusepath{clip}%
\pgfsetbuttcap%
\pgfsetroundjoin%
\definecolor{currentfill}{rgb}{0.335885,0.777018,0.402049}%
\pgfsetfillcolor{currentfill}%
\pgfsetlinewidth{0.000000pt}%
\definecolor{currentstroke}{rgb}{0.120081,0.622161,0.534946}%
\pgfsetstrokecolor{currentstroke}%
\pgfsetdash{}{0pt}%
\pgfpathmoveto{\pgfqpoint{4.439213in}{4.296234in}}%
\pgfpathlineto{\pgfqpoint{4.210962in}{4.541645in}}%
\pgfpathlineto{\pgfqpoint{4.355396in}{4.329235in}}%
\pgfpathclose%
\pgfusepath{fill}%
\end{pgfscope}%
\begin{pgfscope}%
\pgfpathrectangle{\pgfqpoint{0.539299in}{0.078740in}}{\pgfqpoint{7.842520in}{7.842520in}}%
\pgfusepath{clip}%
\pgfsetbuttcap%
\pgfsetroundjoin%
\definecolor{currentfill}{rgb}{0.974417,0.903590,0.130215}%
\pgfsetfillcolor{currentfill}%
\pgfsetlinewidth{0.000000pt}%
\definecolor{currentstroke}{rgb}{0.120638,0.625828,0.533488}%
\pgfsetstrokecolor{currentstroke}%
\pgfsetdash{}{0pt}%
\pgfpathmoveto{\pgfqpoint{3.116915in}{5.451034in}}%
\pgfpathlineto{\pgfqpoint{3.062895in}{5.558135in}}%
\pgfpathlineto{\pgfqpoint{2.975959in}{5.457547in}}%
\pgfpathclose%
\pgfusepath{fill}%
\end{pgfscope}%
\begin{pgfscope}%
\pgfpathrectangle{\pgfqpoint{0.539299in}{0.078740in}}{\pgfqpoint{7.842520in}{7.842520in}}%
\pgfusepath{clip}%
\pgfsetbuttcap%
\pgfsetroundjoin%
\definecolor{currentfill}{rgb}{0.144759,0.519093,0.556572}%
\pgfsetfillcolor{currentfill}%
\pgfsetlinewidth{0.000000pt}%
\definecolor{currentstroke}{rgb}{0.121380,0.629492,0.531973}%
\pgfsetstrokecolor{currentstroke}%
\pgfsetdash{}{0pt}%
\pgfpathmoveto{\pgfqpoint{5.137323in}{3.030184in}}%
\pgfpathlineto{\pgfqpoint{5.074536in}{3.221540in}}%
\pgfpathlineto{\pgfqpoint{4.993441in}{3.248999in}}%
\pgfpathclose%
\pgfusepath{fill}%
\end{pgfscope}%
\begin{pgfscope}%
\pgfpathrectangle{\pgfqpoint{0.539299in}{0.078740in}}{\pgfqpoint{7.842520in}{7.842520in}}%
\pgfusepath{clip}%
\pgfsetbuttcap%
\pgfsetroundjoin%
\definecolor{currentfill}{rgb}{0.709898,0.868751,0.169257}%
\pgfsetfillcolor{currentfill}%
\pgfsetlinewidth{0.000000pt}%
\definecolor{currentstroke}{rgb}{0.122312,0.633153,0.530398}%
\pgfsetstrokecolor{currentstroke}%
\pgfsetdash{}{0pt}%
\pgfpathmoveto{\pgfqpoint{3.863419in}{5.112875in}}%
\pgfpathlineto{\pgfqpoint{3.777327in}{5.109990in}}%
\pgfpathlineto{\pgfqpoint{4.007443in}{4.927974in}}%
\pgfpathclose%
\pgfusepath{fill}%
\end{pgfscope}%
\begin{pgfscope}%
\pgfpathrectangle{\pgfqpoint{0.539299in}{0.078740in}}{\pgfqpoint{7.842520in}{7.842520in}}%
\pgfusepath{clip}%
\pgfsetbuttcap%
\pgfsetroundjoin%
\definecolor{currentfill}{rgb}{0.783315,0.879285,0.125405}%
\pgfsetfillcolor{currentfill}%
\pgfsetlinewidth{0.000000pt}%
\definecolor{currentstroke}{rgb}{0.123444,0.636809,0.528763}%
\pgfsetstrokecolor{currentstroke}%
\pgfsetdash{}{0pt}%
\pgfpathmoveto{\pgfqpoint{3.777327in}{5.109990in}}%
\pgfpathlineto{\pgfqpoint{3.863419in}{5.112875in}}%
\pgfpathlineto{\pgfqpoint{3.719555in}{5.277291in}}%
\pgfpathclose%
\pgfusepath{fill}%
\end{pgfscope}%
\begin{pgfscope}%
\pgfpathrectangle{\pgfqpoint{0.539299in}{0.078740in}}{\pgfqpoint{7.842520in}{7.842520in}}%
\pgfusepath{clip}%
\pgfsetbuttcap%
\pgfsetroundjoin%
\definecolor{currentfill}{rgb}{0.964894,0.902323,0.123941}%
\pgfsetfillcolor{currentfill}%
\pgfsetlinewidth{0.000000pt}%
\definecolor{currentstroke}{rgb}{0.124780,0.640461,0.527068}%
\pgfsetstrokecolor{currentstroke}%
\pgfsetdash{}{0pt}%
\pgfpathmoveto{\pgfqpoint{3.489230in}{5.389845in}}%
\pgfpathlineto{\pgfqpoint{3.433110in}{5.522817in}}%
\pgfpathlineto{\pgfqpoint{3.346062in}{5.485171in}}%
\pgfpathclose%
\pgfusepath{fill}%
\end{pgfscope}%
\begin{pgfscope}%
\pgfpathrectangle{\pgfqpoint{0.539299in}{0.078740in}}{\pgfqpoint{7.842520in}{7.842520in}}%
\pgfusepath{clip}%
\pgfsetbuttcap%
\pgfsetroundjoin%
\definecolor{currentfill}{rgb}{0.132268,0.655014,0.519661}%
\pgfsetfillcolor{currentfill}%
\pgfsetlinewidth{0.000000pt}%
\definecolor{currentstroke}{rgb}{0.126326,0.644107,0.525311}%
\pgfsetstrokecolor{currentstroke}%
\pgfsetdash{}{0pt}%
\pgfpathmoveto{\pgfqpoint{2.619520in}{3.617668in}}%
\pgfpathlineto{\pgfqpoint{2.557549in}{4.110498in}}%
\pgfpathlineto{\pgfqpoint{2.479796in}{3.628915in}}%
\pgfpathclose%
\pgfusepath{fill}%
\end{pgfscope}%
\begin{pgfscope}%
\pgfpathrectangle{\pgfqpoint{0.539299in}{0.078740in}}{\pgfqpoint{7.842520in}{7.842520in}}%
\pgfusepath{clip}%
\pgfsetbuttcap%
\pgfsetroundjoin%
\definecolor{currentfill}{rgb}{0.258965,0.251537,0.524736}%
\pgfsetfillcolor{currentfill}%
\pgfsetlinewidth{0.000000pt}%
\definecolor{currentstroke}{rgb}{0.128087,0.647749,0.523491}%
\pgfsetstrokecolor{currentstroke}%
\pgfsetdash{}{0pt}%
\pgfpathmoveto{\pgfqpoint{5.853809in}{1.950429in}}%
\pgfpathlineto{\pgfqpoint{5.632182in}{2.166073in}}%
\pgfpathlineto{\pgfqpoint{5.775575in}{1.941647in}}%
\pgfpathclose%
\pgfusepath{fill}%
\end{pgfscope}%
\begin{pgfscope}%
\pgfpathrectangle{\pgfqpoint{0.539299in}{0.078740in}}{\pgfqpoint{7.842520in}{7.842520in}}%
\pgfusepath{clip}%
\pgfsetbuttcap%
\pgfsetroundjoin%
\definecolor{currentfill}{rgb}{0.814576,0.883393,0.110347}%
\pgfsetfillcolor{currentfill}%
\pgfsetlinewidth{0.000000pt}%
\definecolor{currentstroke}{rgb}{0.130067,0.651384,0.521608}%
\pgfsetstrokecolor{currentstroke}%
\pgfsetdash{}{0pt}%
\pgfpathmoveto{\pgfqpoint{2.750456in}{5.265020in}}%
\pgfpathlineto{\pgfqpoint{2.665187in}{5.062348in}}%
\pgfpathlineto{\pgfqpoint{2.889650in}{5.309100in}}%
\pgfpathclose%
\pgfusepath{fill}%
\end{pgfscope}%
\begin{pgfscope}%
\pgfpathrectangle{\pgfqpoint{0.539299in}{0.078740in}}{\pgfqpoint{7.842520in}{7.842520in}}%
\pgfusepath{clip}%
\pgfsetbuttcap%
\pgfsetroundjoin%
\definecolor{currentfill}{rgb}{0.210503,0.363727,0.552206}%
\pgfsetfillcolor{currentfill}%
\pgfsetlinewidth{0.000000pt}%
\definecolor{currentstroke}{rgb}{0.132268,0.655014,0.519661}%
\pgfsetstrokecolor{currentstroke}%
\pgfsetdash{}{0pt}%
\pgfpathmoveto{\pgfqpoint{5.488510in}{2.387296in}}%
\pgfpathlineto{\pgfqpoint{5.567779in}{2.379196in}}%
\pgfpathlineto{\pgfqpoint{5.424487in}{2.595182in}}%
\pgfpathclose%
\pgfusepath{fill}%
\end{pgfscope}%
\begin{pgfscope}%
\pgfpathrectangle{\pgfqpoint{0.539299in}{0.078740in}}{\pgfqpoint{7.842520in}{7.842520in}}%
\pgfusepath{clip}%
\pgfsetbuttcap%
\pgfsetroundjoin%
\definecolor{currentfill}{rgb}{0.131172,0.555899,0.552459}%
\pgfsetfillcolor{currentfill}%
\pgfsetlinewidth{0.000000pt}%
\definecolor{currentstroke}{rgb}{0.134692,0.658636,0.517649}%
\pgfsetstrokecolor{currentstroke}%
\pgfsetdash{}{0pt}%
\pgfpathmoveto{\pgfqpoint{4.993441in}{3.248999in}}%
\pgfpathlineto{\pgfqpoint{5.074536in}{3.221540in}}%
\pgfpathlineto{\pgfqpoint{4.849351in}{3.468201in}}%
\pgfpathclose%
\pgfusepath{fill}%
\end{pgfscope}%
\begin{pgfscope}%
\pgfpathrectangle{\pgfqpoint{0.539299in}{0.078740in}}{\pgfqpoint{7.842520in}{7.842520in}}%
\pgfusepath{clip}%
\pgfsetbuttcap%
\pgfsetroundjoin%
\definecolor{currentfill}{rgb}{0.130067,0.651384,0.521608}%
\pgfsetfillcolor{currentfill}%
\pgfsetlinewidth{0.000000pt}%
\definecolor{currentstroke}{rgb}{0.137339,0.662252,0.515571}%
\pgfsetstrokecolor{currentstroke}%
\pgfsetdash{}{0pt}%
\pgfpathmoveto{\pgfqpoint{4.705051in}{3.687048in}}%
\pgfpathlineto{\pgfqpoint{4.787540in}{3.665841in}}%
\pgfpathlineto{\pgfqpoint{4.643709in}{3.889018in}}%
\pgfpathclose%
\pgfusepath{fill}%
\end{pgfscope}%
\begin{pgfscope}%
\pgfpathrectangle{\pgfqpoint{0.539299in}{0.078740in}}{\pgfqpoint{7.842520in}{7.842520in}}%
\pgfusepath{clip}%
\pgfsetbuttcap%
\pgfsetroundjoin%
\definecolor{currentfill}{rgb}{0.327796,0.773980,0.406640}%
\pgfsetfillcolor{currentfill}%
\pgfsetlinewidth{0.000000pt}%
\definecolor{currentstroke}{rgb}{0.140210,0.665859,0.513427}%
\pgfsetstrokecolor{currentstroke}%
\pgfsetdash{}{0pt}%
\pgfpathmoveto{\pgfqpoint{2.637836in}{4.513404in}}%
\pgfpathlineto{\pgfqpoint{2.498985in}{4.479903in}}%
\pgfpathlineto{\pgfqpoint{2.418849in}{4.087714in}}%
\pgfpathclose%
\pgfusepath{fill}%
\end{pgfscope}%
\begin{pgfscope}%
\pgfpathrectangle{\pgfqpoint{0.539299in}{0.078740in}}{\pgfqpoint{7.842520in}{7.842520in}}%
\pgfusepath{clip}%
\pgfsetbuttcap%
\pgfsetroundjoin%
\definecolor{currentfill}{rgb}{0.487026,0.823929,0.312321}%
\pgfsetfillcolor{currentfill}%
\pgfsetlinewidth{0.000000pt}%
\definecolor{currentstroke}{rgb}{0.143303,0.669459,0.511215}%
\pgfsetstrokecolor{currentstroke}%
\pgfsetdash{}{0pt}%
\pgfpathmoveto{\pgfqpoint{4.210962in}{4.541645in}}%
\pgfpathlineto{\pgfqpoint{4.151482in}{4.727445in}}%
\pgfpathlineto{\pgfqpoint{4.066411in}{4.745025in}}%
\pgfpathclose%
\pgfusepath{fill}%
\end{pgfscope}%
\begin{pgfscope}%
\pgfpathrectangle{\pgfqpoint{0.539299in}{0.078740in}}{\pgfqpoint{7.842520in}{7.842520in}}%
\pgfusepath{clip}%
\pgfsetbuttcap%
\pgfsetroundjoin%
\definecolor{currentfill}{rgb}{0.125394,0.574318,0.549086}%
\pgfsetfillcolor{currentfill}%
\pgfsetlinewidth{0.000000pt}%
\definecolor{currentstroke}{rgb}{0.146616,0.673050,0.508936}%
\pgfsetstrokecolor{currentstroke}%
\pgfsetdash{}{0pt}%
\pgfpathmoveto{\pgfqpoint{2.760273in}{3.588885in}}%
\pgfpathlineto{\pgfqpoint{2.619520in}{3.617668in}}%
\pgfpathlineto{\pgfqpoint{2.684453in}{3.009085in}}%
\pgfpathclose%
\pgfusepath{fill}%
\end{pgfscope}%
\begin{pgfscope}%
\pgfpathrectangle{\pgfqpoint{0.539299in}{0.078740in}}{\pgfqpoint{7.842520in}{7.842520in}}%
\pgfusepath{clip}%
\pgfsetbuttcap%
\pgfsetroundjoin%
\definecolor{currentfill}{rgb}{0.218130,0.347432,0.550038}%
\pgfsetfillcolor{currentfill}%
\pgfsetlinewidth{0.000000pt}%
\definecolor{currentstroke}{rgb}{0.150148,0.676631,0.506589}%
\pgfsetstrokecolor{currentstroke}%
\pgfsetdash{}{0pt}%
\pgfpathmoveto{\pgfqpoint{3.175266in}{2.137225in}}%
\pgfpathlineto{\pgfqpoint{3.251833in}{2.786336in}}%
\pgfpathlineto{\pgfqpoint{3.033612in}{2.193144in}}%
\pgfpathclose%
\pgfusepath{fill}%
\end{pgfscope}%
\begin{pgfscope}%
\pgfpathrectangle{\pgfqpoint{0.539299in}{0.078740in}}{\pgfqpoint{7.842520in}{7.842520in}}%
\pgfusepath{clip}%
\pgfsetbuttcap%
\pgfsetroundjoin%
\definecolor{currentfill}{rgb}{0.274149,0.751988,0.436601}%
\pgfsetfillcolor{currentfill}%
\pgfsetlinewidth{0.000000pt}%
\definecolor{currentstroke}{rgb}{0.153894,0.680203,0.504172}%
\pgfsetstrokecolor{currentstroke}%
\pgfsetdash{}{0pt}%
\pgfpathmoveto{\pgfqpoint{2.418849in}{4.087714in}}%
\pgfpathlineto{\pgfqpoint{2.557549in}{4.110498in}}%
\pgfpathlineto{\pgfqpoint{2.637836in}{4.513404in}}%
\pgfpathclose%
\pgfusepath{fill}%
\end{pgfscope}%
\begin{pgfscope}%
\pgfpathrectangle{\pgfqpoint{0.539299in}{0.078740in}}{\pgfqpoint{7.842520in}{7.842520in}}%
\pgfusepath{clip}%
\pgfsetbuttcap%
\pgfsetroundjoin%
\definecolor{currentfill}{rgb}{0.283229,0.120777,0.440584}%
\pgfsetfillcolor{currentfill}%
\pgfsetlinewidth{0.000000pt}%
\definecolor{currentstroke}{rgb}{0.157851,0.683765,0.501686}%
\pgfsetstrokecolor{currentstroke}%
\pgfsetdash{}{0pt}%
\pgfpathmoveto{\pgfqpoint{5.983060in}{1.432854in}}%
\pgfpathlineto{\pgfqpoint{6.061409in}{1.481685in}}%
\pgfpathlineto{\pgfqpoint{5.918659in}{1.713508in}}%
\pgfpathclose%
\pgfusepath{fill}%
\end{pgfscope}%
\begin{pgfscope}%
\pgfpathrectangle{\pgfqpoint{0.539299in}{0.078740in}}{\pgfqpoint{7.842520in}{7.842520in}}%
\pgfusepath{clip}%
\pgfsetbuttcap%
\pgfsetroundjoin%
\definecolor{currentfill}{rgb}{0.886271,0.892374,0.095374}%
\pgfsetfillcolor{currentfill}%
\pgfsetlinewidth{0.000000pt}%
\definecolor{currentstroke}{rgb}{0.162016,0.687316,0.499129}%
\pgfsetstrokecolor{currentstroke}%
\pgfsetdash{}{0pt}%
\pgfpathmoveto{\pgfqpoint{3.489230in}{5.389845in}}%
\pgfpathlineto{\pgfqpoint{3.633064in}{5.262991in}}%
\pgfpathlineto{\pgfqpoint{3.719555in}{5.277291in}}%
\pgfpathclose%
\pgfusepath{fill}%
\end{pgfscope}%
\begin{pgfscope}%
\pgfpathrectangle{\pgfqpoint{0.539299in}{0.078740in}}{\pgfqpoint{7.842520in}{7.842520in}}%
\pgfusepath{clip}%
\pgfsetbuttcap%
\pgfsetroundjoin%
\definecolor{currentfill}{rgb}{0.141935,0.526453,0.555991}%
\pgfsetfillcolor{currentfill}%
\pgfsetlinewidth{0.000000pt}%
\definecolor{currentstroke}{rgb}{0.166383,0.690856,0.496502}%
\pgfsetstrokecolor{currentstroke}%
\pgfsetdash{}{0pt}%
\pgfpathmoveto{\pgfqpoint{2.684453in}{3.009085in}}%
\pgfpathlineto{\pgfqpoint{2.825356in}{2.965660in}}%
\pgfpathlineto{\pgfqpoint{2.901907in}{3.544823in}}%
\pgfpathclose%
\pgfusepath{fill}%
\end{pgfscope}%
\begin{pgfscope}%
\pgfpathrectangle{\pgfqpoint{0.539299in}{0.078740in}}{\pgfqpoint{7.842520in}{7.842520in}}%
\pgfusepath{clip}%
\pgfsetbuttcap%
\pgfsetroundjoin%
\definecolor{currentfill}{rgb}{0.616293,0.852709,0.230052}%
\pgfsetfillcolor{currentfill}%
\pgfsetlinewidth{0.000000pt}%
\definecolor{currentstroke}{rgb}{0.170948,0.694384,0.493803}%
\pgfsetstrokecolor{currentstroke}%
\pgfsetdash{}{0pt}%
\pgfpathmoveto{\pgfqpoint{4.007443in}{4.927974in}}%
\pgfpathlineto{\pgfqpoint{3.921827in}{4.935824in}}%
\pgfpathlineto{\pgfqpoint{4.066411in}{4.745025in}}%
\pgfpathclose%
\pgfusepath{fill}%
\end{pgfscope}%
\begin{pgfscope}%
\pgfpathrectangle{\pgfqpoint{0.539299in}{0.078740in}}{\pgfqpoint{7.842520in}{7.842520in}}%
\pgfusepath{clip}%
\pgfsetbuttcap%
\pgfsetroundjoin%
\definecolor{currentfill}{rgb}{0.187231,0.414746,0.556547}%
\pgfsetfillcolor{currentfill}%
\pgfsetlinewidth{0.000000pt}%
\definecolor{currentstroke}{rgb}{0.175707,0.697900,0.491033}%
\pgfsetstrokecolor{currentstroke}%
\pgfsetdash{}{0pt}%
\pgfpathmoveto{\pgfqpoint{3.033612in}{2.193144in}}%
\pgfpathlineto{\pgfqpoint{3.109114in}{2.853159in}}%
\pgfpathlineto{\pgfqpoint{2.966934in}{2.913287in}}%
\pgfpathclose%
\pgfusepath{fill}%
\end{pgfscope}%
\begin{pgfscope}%
\pgfpathrectangle{\pgfqpoint{0.539299in}{0.078740in}}{\pgfqpoint{7.842520in}{7.842520in}}%
\pgfusepath{clip}%
\pgfsetbuttcap%
\pgfsetroundjoin%
\definecolor{currentfill}{rgb}{0.458674,0.816363,0.329727}%
\pgfsetfillcolor{currentfill}%
\pgfsetlinewidth{0.000000pt}%
\definecolor{currentstroke}{rgb}{0.180653,0.701402,0.488189}%
\pgfsetstrokecolor{currentstroke}%
\pgfsetdash{}{0pt}%
\pgfpathmoveto{\pgfqpoint{2.581236in}{4.802712in}}%
\pgfpathlineto{\pgfqpoint{2.498985in}{4.479903in}}%
\pgfpathlineto{\pgfqpoint{2.637836in}{4.513404in}}%
\pgfpathclose%
\pgfusepath{fill}%
\end{pgfscope}%
\begin{pgfscope}%
\pgfpathrectangle{\pgfqpoint{0.539299in}{0.078740in}}{\pgfqpoint{7.842520in}{7.842520in}}%
\pgfusepath{clip}%
\pgfsetbuttcap%
\pgfsetroundjoin%
\definecolor{currentfill}{rgb}{0.688944,0.865448,0.182725}%
\pgfsetfillcolor{currentfill}%
\pgfsetlinewidth{0.000000pt}%
\definecolor{currentstroke}{rgb}{0.185783,0.704891,0.485273}%
\pgfsetstrokecolor{currentstroke}%
\pgfsetdash{}{0pt}%
\pgfpathmoveto{\pgfqpoint{3.777327in}{5.109990in}}%
\pgfpathlineto{\pgfqpoint{3.921827in}{4.935824in}}%
\pgfpathlineto{\pgfqpoint{4.007443in}{4.927974in}}%
\pgfpathclose%
\pgfusepath{fill}%
\end{pgfscope}%
\begin{pgfscope}%
\pgfpathrectangle{\pgfqpoint{0.539299in}{0.078740in}}{\pgfqpoint{7.842520in}{7.842520in}}%
\pgfusepath{clip}%
\pgfsetbuttcap%
\pgfsetroundjoin%
\definecolor{currentfill}{rgb}{0.180653,0.701402,0.488189}%
\pgfsetfillcolor{currentfill}%
\pgfsetlinewidth{0.000000pt}%
\definecolor{currentstroke}{rgb}{0.191090,0.708366,0.482284}%
\pgfsetstrokecolor{currentstroke}%
\pgfsetdash{}{0pt}%
\pgfpathmoveto{\pgfqpoint{4.499656in}{4.110842in}}%
\pgfpathlineto{\pgfqpoint{4.560545in}{3.904435in}}%
\pgfpathlineto{\pgfqpoint{4.643709in}{3.889018in}}%
\pgfpathclose%
\pgfusepath{fill}%
\end{pgfscope}%
\begin{pgfscope}%
\pgfpathrectangle{\pgfqpoint{0.539299in}{0.078740in}}{\pgfqpoint{7.842520in}{7.842520in}}%
\pgfusepath{clip}%
\pgfsetbuttcap%
\pgfsetroundjoin%
\definecolor{currentfill}{rgb}{0.835270,0.886029,0.102646}%
\pgfsetfillcolor{currentfill}%
\pgfsetlinewidth{0.000000pt}%
\definecolor{currentstroke}{rgb}{0.196571,0.711827,0.479221}%
\pgfsetstrokecolor{currentstroke}%
\pgfsetdash{}{0pt}%
\pgfpathmoveto{\pgfqpoint{3.719555in}{5.277291in}}%
\pgfpathlineto{\pgfqpoint{3.633064in}{5.262991in}}%
\pgfpathlineto{\pgfqpoint{3.777327in}{5.109990in}}%
\pgfpathclose%
\pgfusepath{fill}%
\end{pgfscope}%
\begin{pgfscope}%
\pgfpathrectangle{\pgfqpoint{0.539299in}{0.078740in}}{\pgfqpoint{7.842520in}{7.842520in}}%
\pgfusepath{clip}%
\pgfsetbuttcap%
\pgfsetroundjoin%
\definecolor{currentfill}{rgb}{0.223925,0.334994,0.548053}%
\pgfsetfillcolor{currentfill}%
\pgfsetlinewidth{0.000000pt}%
\definecolor{currentstroke}{rgb}{0.202219,0.715272,0.476084}%
\pgfsetstrokecolor{currentstroke}%
\pgfsetdash{}{0pt}%
\pgfpathmoveto{\pgfqpoint{5.567779in}{2.379196in}}%
\pgfpathlineto{\pgfqpoint{5.488510in}{2.387296in}}%
\pgfpathlineto{\pgfqpoint{5.632182in}{2.166073in}}%
\pgfpathclose%
\pgfusepath{fill}%
\end{pgfscope}%
\begin{pgfscope}%
\pgfpathrectangle{\pgfqpoint{0.539299in}{0.078740in}}{\pgfqpoint{7.842520in}{7.842520in}}%
\pgfusepath{clip}%
\pgfsetbuttcap%
\pgfsetroundjoin%
\definecolor{currentfill}{rgb}{0.945636,0.899815,0.112838}%
\pgfsetfillcolor{currentfill}%
\pgfsetlinewidth{0.000000pt}%
\definecolor{currentstroke}{rgb}{0.208030,0.718701,0.472873}%
\pgfsetstrokecolor{currentstroke}%
\pgfsetdash{}{0pt}%
\pgfpathmoveto{\pgfqpoint{2.975959in}{5.457547in}}%
\pgfpathlineto{\pgfqpoint{2.889650in}{5.309100in}}%
\pgfpathlineto{\pgfqpoint{3.116915in}{5.451034in}}%
\pgfpathclose%
\pgfusepath{fill}%
\end{pgfscope}%
\begin{pgfscope}%
\pgfpathrectangle{\pgfqpoint{0.539299in}{0.078740in}}{\pgfqpoint{7.842520in}{7.842520in}}%
\pgfusepath{clip}%
\pgfsetbuttcap%
\pgfsetroundjoin%
\definecolor{currentfill}{rgb}{0.179019,0.433756,0.557430}%
\pgfsetfillcolor{currentfill}%
\pgfsetlinewidth{0.000000pt}%
\definecolor{currentstroke}{rgb}{0.214000,0.722114,0.469588}%
\pgfsetstrokecolor{currentstroke}%
\pgfsetdash{}{0pt}%
\pgfpathmoveto{\pgfqpoint{5.281003in}{2.812185in}}%
\pgfpathlineto{\pgfqpoint{5.200440in}{2.822813in}}%
\pgfpathlineto{\pgfqpoint{5.424487in}{2.595182in}}%
\pgfpathclose%
\pgfusepath{fill}%
\end{pgfscope}%
\begin{pgfscope}%
\pgfpathrectangle{\pgfqpoint{0.539299in}{0.078740in}}{\pgfqpoint{7.842520in}{7.842520in}}%
\pgfusepath{clip}%
\pgfsetbuttcap%
\pgfsetroundjoin%
\definecolor{currentfill}{rgb}{0.636902,0.856542,0.216620}%
\pgfsetfillcolor{currentfill}%
\pgfsetlinewidth{0.000000pt}%
\definecolor{currentstroke}{rgb}{0.220124,0.725509,0.466226}%
\pgfsetstrokecolor{currentstroke}%
\pgfsetdash{}{0pt}%
\pgfpathmoveto{\pgfqpoint{2.720220in}{4.843367in}}%
\pgfpathlineto{\pgfqpoint{2.665187in}{5.062348in}}%
\pgfpathlineto{\pgfqpoint{2.581236in}{4.802712in}}%
\pgfpathclose%
\pgfusepath{fill}%
\end{pgfscope}%
\begin{pgfscope}%
\pgfpathrectangle{\pgfqpoint{0.539299in}{0.078740in}}{\pgfqpoint{7.842520in}{7.842520in}}%
\pgfusepath{clip}%
\pgfsetbuttcap%
\pgfsetroundjoin%
\definecolor{currentfill}{rgb}{0.983868,0.904867,0.136897}%
\pgfsetfillcolor{currentfill}%
\pgfsetlinewidth{0.000000pt}%
\definecolor{currentstroke}{rgb}{0.226397,0.728888,0.462789}%
\pgfsetstrokecolor{currentstroke}%
\pgfsetdash{}{0pt}%
\pgfpathmoveto{\pgfqpoint{3.346062in}{5.485171in}}%
\pgfpathlineto{\pgfqpoint{3.203841in}{5.543255in}}%
\pgfpathlineto{\pgfqpoint{3.259203in}{5.402997in}}%
\pgfpathclose%
\pgfusepath{fill}%
\end{pgfscope}%
\begin{pgfscope}%
\pgfpathrectangle{\pgfqpoint{0.539299in}{0.078740in}}{\pgfqpoint{7.842520in}{7.842520in}}%
\pgfusepath{clip}%
\pgfsetbuttcap%
\pgfsetroundjoin%
\definecolor{currentfill}{rgb}{0.120638,0.625828,0.533488}%
\pgfsetfillcolor{currentfill}%
\pgfsetlinewidth{0.000000pt}%
\definecolor{currentstroke}{rgb}{0.232815,0.732247,0.459277}%
\pgfsetstrokecolor{currentstroke}%
\pgfsetdash{}{0pt}%
\pgfpathmoveto{\pgfqpoint{4.787540in}{3.665841in}}%
\pgfpathlineto{\pgfqpoint{4.705051in}{3.687048in}}%
\pgfpathlineto{\pgfqpoint{4.849351in}{3.468201in}}%
\pgfpathclose%
\pgfusepath{fill}%
\end{pgfscope}%
\begin{pgfscope}%
\pgfpathrectangle{\pgfqpoint{0.539299in}{0.078740in}}{\pgfqpoint{7.842520in}{7.842520in}}%
\pgfusepath{clip}%
\pgfsetbuttcap%
\pgfsetroundjoin%
\definecolor{currentfill}{rgb}{0.804182,0.882046,0.114965}%
\pgfsetfillcolor{currentfill}%
\pgfsetlinewidth{0.000000pt}%
\definecolor{currentstroke}{rgb}{0.239374,0.735588,0.455688}%
\pgfsetstrokecolor{currentstroke}%
\pgfsetdash{}{0pt}%
\pgfpathmoveto{\pgfqpoint{2.889650in}{5.309100in}}%
\pgfpathlineto{\pgfqpoint{2.665187in}{5.062348in}}%
\pgfpathlineto{\pgfqpoint{2.804286in}{5.106496in}}%
\pgfpathclose%
\pgfusepath{fill}%
\end{pgfscope}%
\begin{pgfscope}%
\pgfpathrectangle{\pgfqpoint{0.539299in}{0.078740in}}{\pgfqpoint{7.842520in}{7.842520in}}%
\pgfusepath{clip}%
\pgfsetbuttcap%
\pgfsetroundjoin%
\definecolor{currentfill}{rgb}{0.983868,0.904867,0.136897}%
\pgfsetfillcolor{currentfill}%
\pgfsetlinewidth{0.000000pt}%
\definecolor{currentstroke}{rgb}{0.246070,0.738910,0.452024}%
\pgfsetstrokecolor{currentstroke}%
\pgfsetdash{}{0pt}%
\pgfpathmoveto{\pgfqpoint{3.203841in}{5.543255in}}%
\pgfpathlineto{\pgfqpoint{3.116915in}{5.451034in}}%
\pgfpathlineto{\pgfqpoint{3.259203in}{5.402997in}}%
\pgfpathclose%
\pgfusepath{fill}%
\end{pgfscope}%
\begin{pgfscope}%
\pgfpathrectangle{\pgfqpoint{0.539299in}{0.078740in}}{\pgfqpoint{7.842520in}{7.842520in}}%
\pgfusepath{clip}%
\pgfsetbuttcap%
\pgfsetroundjoin%
\definecolor{currentfill}{rgb}{0.175707,0.697900,0.491033}%
\pgfsetfillcolor{currentfill}%
\pgfsetlinewidth{0.000000pt}%
\definecolor{currentstroke}{rgb}{0.252899,0.742211,0.448284}%
\pgfsetstrokecolor{currentstroke}%
\pgfsetdash{}{0pt}%
\pgfpathmoveto{\pgfqpoint{2.619520in}{3.617668in}}%
\pgfpathlineto{\pgfqpoint{2.697635in}{4.107094in}}%
\pgfpathlineto{\pgfqpoint{2.557549in}{4.110498in}}%
\pgfpathclose%
\pgfusepath{fill}%
\end{pgfscope}%
\begin{pgfscope}%
\pgfpathrectangle{\pgfqpoint{0.539299in}{0.078740in}}{\pgfqpoint{7.842520in}{7.842520in}}%
\pgfusepath{clip}%
\pgfsetbuttcap%
\pgfsetroundjoin%
\definecolor{currentfill}{rgb}{0.163625,0.471133,0.558148}%
\pgfsetfillcolor{currentfill}%
\pgfsetlinewidth{0.000000pt}%
\definecolor{currentstroke}{rgb}{0.259857,0.745492,0.444467}%
\pgfsetstrokecolor{currentstroke}%
\pgfsetdash{}{0pt}%
\pgfpathmoveto{\pgfqpoint{5.137323in}{3.030184in}}%
\pgfpathlineto{\pgfqpoint{5.200440in}{2.822813in}}%
\pgfpathlineto{\pgfqpoint{5.281003in}{2.812185in}}%
\pgfpathclose%
\pgfusepath{fill}%
\end{pgfscope}%
\begin{pgfscope}%
\pgfpathrectangle{\pgfqpoint{0.539299in}{0.078740in}}{\pgfqpoint{7.842520in}{7.842520in}}%
\pgfusepath{clip}%
\pgfsetbuttcap%
\pgfsetroundjoin%
\definecolor{currentfill}{rgb}{0.124395,0.578002,0.548287}%
\pgfsetfillcolor{currentfill}%
\pgfsetlinewidth{0.000000pt}%
\definecolor{currentstroke}{rgb}{0.266941,0.748751,0.440573}%
\pgfsetstrokecolor{currentstroke}%
\pgfsetdash{}{0pt}%
\pgfpathmoveto{\pgfqpoint{2.684453in}{3.009085in}}%
\pgfpathlineto{\pgfqpoint{2.901907in}{3.544823in}}%
\pgfpathlineto{\pgfqpoint{2.760273in}{3.588885in}}%
\pgfpathclose%
\pgfusepath{fill}%
\end{pgfscope}%
\begin{pgfscope}%
\pgfpathrectangle{\pgfqpoint{0.539299in}{0.078740in}}{\pgfqpoint{7.842520in}{7.842520in}}%
\pgfusepath{clip}%
\pgfsetbuttcap%
\pgfsetroundjoin%
\definecolor{currentfill}{rgb}{0.197636,0.391528,0.554969}%
\pgfsetfillcolor{currentfill}%
\pgfsetlinewidth{0.000000pt}%
\definecolor{currentstroke}{rgb}{0.274149,0.751988,0.436601}%
\pgfsetstrokecolor{currentstroke}%
\pgfsetdash{}{0pt}%
\pgfpathmoveto{\pgfqpoint{5.424487in}{2.595182in}}%
\pgfpathlineto{\pgfqpoint{5.344589in}{2.606000in}}%
\pgfpathlineto{\pgfqpoint{5.488510in}{2.387296in}}%
\pgfpathclose%
\pgfusepath{fill}%
\end{pgfscope}%
\begin{pgfscope}%
\pgfpathrectangle{\pgfqpoint{0.539299in}{0.078740in}}{\pgfqpoint{7.842520in}{7.842520in}}%
\pgfusepath{clip}%
\pgfsetbuttcap%
\pgfsetroundjoin%
\definecolor{currentfill}{rgb}{0.188923,0.410910,0.556326}%
\pgfsetfillcolor{currentfill}%
\pgfsetlinewidth{0.000000pt}%
\definecolor{currentstroke}{rgb}{0.281477,0.755203,0.432552}%
\pgfsetstrokecolor{currentstroke}%
\pgfsetdash{}{0pt}%
\pgfpathmoveto{\pgfqpoint{3.033612in}{2.193144in}}%
\pgfpathlineto{\pgfqpoint{3.251833in}{2.786336in}}%
\pgfpathlineto{\pgfqpoint{3.109114in}{2.853159in}}%
\pgfpathclose%
\pgfusepath{fill}%
\end{pgfscope}%
\begin{pgfscope}%
\pgfpathrectangle{\pgfqpoint{0.539299in}{0.078740in}}{\pgfqpoint{7.842520in}{7.842520in}}%
\pgfusepath{clip}%
\pgfsetbuttcap%
\pgfsetroundjoin%
\definecolor{currentfill}{rgb}{0.296479,0.761561,0.424223}%
\pgfsetfillcolor{currentfill}%
\pgfsetlinewidth{0.000000pt}%
\definecolor{currentstroke}{rgb}{0.288921,0.758394,0.428426}%
\pgfsetstrokecolor{currentstroke}%
\pgfsetdash{}{0pt}%
\pgfpathmoveto{\pgfqpoint{4.355396in}{4.329235in}}%
\pgfpathlineto{\pgfqpoint{4.270980in}{4.328400in}}%
\pgfpathlineto{\pgfqpoint{4.499656in}{4.110842in}}%
\pgfpathclose%
\pgfusepath{fill}%
\end{pgfscope}%
\begin{pgfscope}%
\pgfpathrectangle{\pgfqpoint{0.539299in}{0.078740in}}{\pgfqpoint{7.842520in}{7.842520in}}%
\pgfusepath{clip}%
\pgfsetbuttcap%
\pgfsetroundjoin%
\definecolor{currentfill}{rgb}{0.150148,0.676631,0.506589}%
\pgfsetfillcolor{currentfill}%
\pgfsetlinewidth{0.000000pt}%
\definecolor{currentstroke}{rgb}{0.296479,0.761561,0.424223}%
\pgfsetstrokecolor{currentstroke}%
\pgfsetdash{}{0pt}%
\pgfpathmoveto{\pgfqpoint{4.643709in}{3.889018in}}%
\pgfpathlineto{\pgfqpoint{4.560545in}{3.904435in}}%
\pgfpathlineto{\pgfqpoint{4.705051in}{3.687048in}}%
\pgfpathclose%
\pgfusepath{fill}%
\end{pgfscope}%
\begin{pgfscope}%
\pgfpathrectangle{\pgfqpoint{0.539299in}{0.078740in}}{\pgfqpoint{7.842520in}{7.842520in}}%
\pgfusepath{clip}%
\pgfsetbuttcap%
\pgfsetroundjoin%
\definecolor{currentfill}{rgb}{0.709898,0.868751,0.169257}%
\pgfsetfillcolor{currentfill}%
\pgfsetlinewidth{0.000000pt}%
\definecolor{currentstroke}{rgb}{0.304148,0.764704,0.419943}%
\pgfsetstrokecolor{currentstroke}%
\pgfsetdash{}{0pt}%
\pgfpathmoveto{\pgfqpoint{2.804286in}{5.106496in}}%
\pgfpathlineto{\pgfqpoint{2.665187in}{5.062348in}}%
\pgfpathlineto{\pgfqpoint{2.720220in}{4.843367in}}%
\pgfpathclose%
\pgfusepath{fill}%
\end{pgfscope}%
\begin{pgfscope}%
\pgfpathrectangle{\pgfqpoint{0.539299in}{0.078740in}}{\pgfqpoint{7.842520in}{7.842520in}}%
\pgfusepath{clip}%
\pgfsetbuttcap%
\pgfsetroundjoin%
\definecolor{currentfill}{rgb}{0.964894,0.902323,0.123941}%
\pgfsetfillcolor{currentfill}%
\pgfsetlinewidth{0.000000pt}%
\definecolor{currentstroke}{rgb}{0.311925,0.767822,0.415586}%
\pgfsetstrokecolor{currentstroke}%
\pgfsetdash{}{0pt}%
\pgfpathmoveto{\pgfqpoint{3.259203in}{5.402997in}}%
\pgfpathlineto{\pgfqpoint{3.489230in}{5.389845in}}%
\pgfpathlineto{\pgfqpoint{3.346062in}{5.485171in}}%
\pgfpathclose%
\pgfusepath{fill}%
\end{pgfscope}%
\begin{pgfscope}%
\pgfpathrectangle{\pgfqpoint{0.539299in}{0.078740in}}{\pgfqpoint{7.842520in}{7.842520in}}%
\pgfusepath{clip}%
\pgfsetbuttcap%
\pgfsetroundjoin%
\definecolor{currentfill}{rgb}{0.545524,0.838039,0.275626}%
\pgfsetfillcolor{currentfill}%
\pgfsetlinewidth{0.000000pt}%
\definecolor{currentstroke}{rgb}{0.319809,0.770914,0.411152}%
\pgfsetstrokecolor{currentstroke}%
\pgfsetdash{}{0pt}%
\pgfpathmoveto{\pgfqpoint{2.637836in}{4.513404in}}%
\pgfpathlineto{\pgfqpoint{2.720220in}{4.843367in}}%
\pgfpathlineto{\pgfqpoint{2.581236in}{4.802712in}}%
\pgfpathclose%
\pgfusepath{fill}%
\end{pgfscope}%
\begin{pgfscope}%
\pgfpathrectangle{\pgfqpoint{0.539299in}{0.078740in}}{\pgfqpoint{7.842520in}{7.842520in}}%
\pgfusepath{clip}%
\pgfsetbuttcap%
\pgfsetroundjoin%
\definecolor{currentfill}{rgb}{0.221989,0.339161,0.548752}%
\pgfsetfillcolor{currentfill}%
\pgfsetlinewidth{0.000000pt}%
\definecolor{currentstroke}{rgb}{0.327796,0.773980,0.406640}%
\pgfsetstrokecolor{currentstroke}%
\pgfsetdash{}{0pt}%
\pgfpathmoveto{\pgfqpoint{3.395040in}{2.713747in}}%
\pgfpathlineto{\pgfqpoint{3.175266in}{2.137225in}}%
\pgfpathlineto{\pgfqpoint{3.317407in}{2.078347in}}%
\pgfpathclose%
\pgfusepath{fill}%
\end{pgfscope}%
\begin{pgfscope}%
\pgfpathrectangle{\pgfqpoint{0.539299in}{0.078740in}}{\pgfqpoint{7.842520in}{7.842520in}}%
\pgfusepath{clip}%
\pgfsetbuttcap%
\pgfsetroundjoin%
\definecolor{currentfill}{rgb}{0.267968,0.223549,0.512008}%
\pgfsetfillcolor{currentfill}%
\pgfsetlinewidth{0.000000pt}%
\definecolor{currentstroke}{rgb}{0.335885,0.777018,0.402049}%
\pgfsetstrokecolor{currentstroke}%
\pgfsetdash{}{0pt}%
\pgfpathmoveto{\pgfqpoint{5.918659in}{1.713508in}}%
\pgfpathlineto{\pgfqpoint{5.775575in}{1.941647in}}%
\pgfpathlineto{\pgfqpoint{5.696616in}{1.925940in}}%
\pgfpathclose%
\pgfusepath{fill}%
\end{pgfscope}%
\begin{pgfscope}%
\pgfpathrectangle{\pgfqpoint{0.539299in}{0.078740in}}{\pgfqpoint{7.842520in}{7.842520in}}%
\pgfusepath{clip}%
\pgfsetbuttcap%
\pgfsetroundjoin%
\definecolor{currentfill}{rgb}{0.182256,0.426184,0.557120}%
\pgfsetfillcolor{currentfill}%
\pgfsetlinewidth{0.000000pt}%
\definecolor{currentstroke}{rgb}{0.344074,0.780029,0.397381}%
\pgfsetstrokecolor{currentstroke}%
\pgfsetdash{}{0pt}%
\pgfpathmoveto{\pgfqpoint{5.200440in}{2.822813in}}%
\pgfpathlineto{\pgfqpoint{5.344589in}{2.606000in}}%
\pgfpathlineto{\pgfqpoint{5.424487in}{2.595182in}}%
\pgfpathclose%
\pgfusepath{fill}%
\end{pgfscope}%
\begin{pgfscope}%
\pgfpathrectangle{\pgfqpoint{0.539299in}{0.078740in}}{\pgfqpoint{7.842520in}{7.842520in}}%
\pgfusepath{clip}%
\pgfsetbuttcap%
\pgfsetroundjoin%
\definecolor{currentfill}{rgb}{0.296479,0.761561,0.424223}%
\pgfsetfillcolor{currentfill}%
\pgfsetlinewidth{0.000000pt}%
\definecolor{currentstroke}{rgb}{0.352360,0.783011,0.392636}%
\pgfsetstrokecolor{currentstroke}%
\pgfsetdash{}{0pt}%
\pgfpathmoveto{\pgfqpoint{2.557549in}{4.110498in}}%
\pgfpathlineto{\pgfqpoint{2.697635in}{4.107094in}}%
\pgfpathlineto{\pgfqpoint{2.637836in}{4.513404in}}%
\pgfpathclose%
\pgfusepath{fill}%
\end{pgfscope}%
\begin{pgfscope}%
\pgfpathrectangle{\pgfqpoint{0.539299in}{0.078740in}}{\pgfqpoint{7.842520in}{7.842520in}}%
\pgfusepath{clip}%
\pgfsetbuttcap%
\pgfsetroundjoin%
\definecolor{currentfill}{rgb}{0.281887,0.150881,0.465405}%
\pgfsetfillcolor{currentfill}%
\pgfsetlinewidth{0.000000pt}%
\definecolor{currentstroke}{rgb}{0.360741,0.785964,0.387814}%
\pgfsetstrokecolor{currentstroke}%
\pgfsetdash{}{0pt}%
\pgfpathmoveto{\pgfqpoint{5.918659in}{1.713508in}}%
\pgfpathlineto{\pgfqpoint{5.840111in}{1.685406in}}%
\pgfpathlineto{\pgfqpoint{5.983060in}{1.432854in}}%
\pgfpathclose%
\pgfusepath{fill}%
\end{pgfscope}%
\begin{pgfscope}%
\pgfpathrectangle{\pgfqpoint{0.539299in}{0.078740in}}{\pgfqpoint{7.842520in}{7.842520in}}%
\pgfusepath{clip}%
\pgfsetbuttcap%
\pgfsetroundjoin%
\definecolor{currentfill}{rgb}{0.369214,0.788888,0.382914}%
\pgfsetfillcolor{currentfill}%
\pgfsetlinewidth{0.000000pt}%
\definecolor{currentstroke}{rgb}{0.369214,0.788888,0.382914}%
\pgfsetstrokecolor{currentstroke}%
\pgfsetdash{}{0pt}%
\pgfpathmoveto{\pgfqpoint{4.210962in}{4.541645in}}%
\pgfpathlineto{\pgfqpoint{4.270980in}{4.328400in}}%
\pgfpathlineto{\pgfqpoint{4.355396in}{4.329235in}}%
\pgfpathclose%
\pgfusepath{fill}%
\end{pgfscope}%
\begin{pgfscope}%
\pgfpathrectangle{\pgfqpoint{0.539299in}{0.078740in}}{\pgfqpoint{7.842520in}{7.842520in}}%
\pgfusepath{clip}%
\pgfsetbuttcap%
\pgfsetroundjoin%
\definecolor{currentfill}{rgb}{0.141935,0.526453,0.555991}%
\pgfsetfillcolor{currentfill}%
\pgfsetlinewidth{0.000000pt}%
\definecolor{currentstroke}{rgb}{0.377779,0.791781,0.377939}%
\pgfsetstrokecolor{currentstroke}%
\pgfsetdash{}{0pt}%
\pgfpathmoveto{\pgfqpoint{3.044293in}{3.487491in}}%
\pgfpathlineto{\pgfqpoint{2.825356in}{2.965660in}}%
\pgfpathlineto{\pgfqpoint{2.966934in}{2.913287in}}%
\pgfpathclose%
\pgfusepath{fill}%
\end{pgfscope}%
\begin{pgfscope}%
\pgfpathrectangle{\pgfqpoint{0.539299in}{0.078740in}}{\pgfqpoint{7.842520in}{7.842520in}}%
\pgfusepath{clip}%
\pgfsetbuttcap%
\pgfsetroundjoin%
\definecolor{currentfill}{rgb}{0.935904,0.898570,0.108131}%
\pgfsetfillcolor{currentfill}%
\pgfsetlinewidth{0.000000pt}%
\definecolor{currentstroke}{rgb}{0.386433,0.794644,0.372886}%
\pgfsetstrokecolor{currentstroke}%
\pgfsetdash{}{0pt}%
\pgfpathmoveto{\pgfqpoint{3.116915in}{5.451034in}}%
\pgfpathlineto{\pgfqpoint{2.889650in}{5.309100in}}%
\pgfpathlineto{\pgfqpoint{3.030553in}{5.308756in}}%
\pgfpathclose%
\pgfusepath{fill}%
\end{pgfscope}%
\begin{pgfscope}%
\pgfpathrectangle{\pgfqpoint{0.539299in}{0.078740in}}{\pgfqpoint{7.842520in}{7.842520in}}%
\pgfusepath{clip}%
\pgfsetbuttcap%
\pgfsetroundjoin%
\definecolor{currentfill}{rgb}{0.137770,0.537492,0.554906}%
\pgfsetfillcolor{currentfill}%
\pgfsetlinewidth{0.000000pt}%
\definecolor{currentstroke}{rgb}{0.395174,0.797475,0.367757}%
\pgfsetstrokecolor{currentstroke}%
\pgfsetdash{}{0pt}%
\pgfpathmoveto{\pgfqpoint{4.993441in}{3.248999in}}%
\pgfpathlineto{\pgfqpoint{4.911519in}{3.252217in}}%
\pgfpathlineto{\pgfqpoint{5.137323in}{3.030184in}}%
\pgfpathclose%
\pgfusepath{fill}%
\end{pgfscope}%
\begin{pgfscope}%
\pgfpathrectangle{\pgfqpoint{0.539299in}{0.078740in}}{\pgfqpoint{7.842520in}{7.842520in}}%
\pgfusepath{clip}%
\pgfsetbuttcap%
\pgfsetroundjoin%
\definecolor{currentfill}{rgb}{0.220124,0.725509,0.466226}%
\pgfsetfillcolor{currentfill}%
\pgfsetlinewidth{0.000000pt}%
\definecolor{currentstroke}{rgb}{0.404001,0.800275,0.362552}%
\pgfsetstrokecolor{currentstroke}%
\pgfsetdash{}{0pt}%
\pgfpathmoveto{\pgfqpoint{4.499656in}{4.110842in}}%
\pgfpathlineto{\pgfqpoint{4.415846in}{4.118860in}}%
\pgfpathlineto{\pgfqpoint{4.560545in}{3.904435in}}%
\pgfpathclose%
\pgfusepath{fill}%
\end{pgfscope}%
\begin{pgfscope}%
\pgfpathrectangle{\pgfqpoint{0.539299in}{0.078740in}}{\pgfqpoint{7.842520in}{7.842520in}}%
\pgfusepath{clip}%
\pgfsetbuttcap%
\pgfsetroundjoin%
\definecolor{currentfill}{rgb}{0.252194,0.269783,0.531579}%
\pgfsetfillcolor{currentfill}%
\pgfsetlinewidth{0.000000pt}%
\definecolor{currentstroke}{rgb}{0.412913,0.803041,0.357269}%
\pgfsetstrokecolor{currentstroke}%
\pgfsetdash{}{0pt}%
\pgfpathmoveto{\pgfqpoint{5.775575in}{1.941647in}}%
\pgfpathlineto{\pgfqpoint{5.632182in}{2.166073in}}%
\pgfpathlineto{\pgfqpoint{5.696616in}{1.925940in}}%
\pgfpathclose%
\pgfusepath{fill}%
\end{pgfscope}%
\begin{pgfscope}%
\pgfpathrectangle{\pgfqpoint{0.539299in}{0.078740in}}{\pgfqpoint{7.842520in}{7.842520in}}%
\pgfusepath{clip}%
\pgfsetbuttcap%
\pgfsetroundjoin%
\definecolor{currentfill}{rgb}{0.140210,0.665859,0.513427}%
\pgfsetfillcolor{currentfill}%
\pgfsetlinewidth{0.000000pt}%
\definecolor{currentstroke}{rgb}{0.421908,0.805774,0.351910}%
\pgfsetstrokecolor{currentstroke}%
\pgfsetdash{}{0pt}%
\pgfpathmoveto{\pgfqpoint{2.838886in}{4.080766in}}%
\pgfpathlineto{\pgfqpoint{2.619520in}{3.617668in}}%
\pgfpathlineto{\pgfqpoint{2.760273in}{3.588885in}}%
\pgfpathclose%
\pgfusepath{fill}%
\end{pgfscope}%
\begin{pgfscope}%
\pgfpathrectangle{\pgfqpoint{0.539299in}{0.078740in}}{\pgfqpoint{7.842520in}{7.842520in}}%
\pgfusepath{clip}%
\pgfsetbuttcap%
\pgfsetroundjoin%
\definecolor{currentfill}{rgb}{0.185783,0.704891,0.485273}%
\pgfsetfillcolor{currentfill}%
\pgfsetlinewidth{0.000000pt}%
\definecolor{currentstroke}{rgb}{0.430983,0.808473,0.346476}%
\pgfsetstrokecolor{currentstroke}%
\pgfsetdash{}{0pt}%
\pgfpathmoveto{\pgfqpoint{2.697635in}{4.107094in}}%
\pgfpathlineto{\pgfqpoint{2.619520in}{3.617668in}}%
\pgfpathlineto{\pgfqpoint{2.838886in}{4.080766in}}%
\pgfpathclose%
\pgfusepath{fill}%
\end{pgfscope}%
\begin{pgfscope}%
\pgfpathrectangle{\pgfqpoint{0.539299in}{0.078740in}}{\pgfqpoint{7.842520in}{7.842520in}}%
\pgfusepath{clip}%
\pgfsetbuttcap%
\pgfsetroundjoin%
\definecolor{currentfill}{rgb}{0.126453,0.570633,0.549841}%
\pgfsetfillcolor{currentfill}%
\pgfsetlinewidth{0.000000pt}%
\definecolor{currentstroke}{rgb}{0.440137,0.811138,0.340967}%
\pgfsetstrokecolor{currentstroke}%
\pgfsetdash{}{0pt}%
\pgfpathmoveto{\pgfqpoint{4.849351in}{3.468201in}}%
\pgfpathlineto{\pgfqpoint{4.911519in}{3.252217in}}%
\pgfpathlineto{\pgfqpoint{4.993441in}{3.248999in}}%
\pgfpathclose%
\pgfusepath{fill}%
\end{pgfscope}%
\begin{pgfscope}%
\pgfpathrectangle{\pgfqpoint{0.539299in}{0.078740in}}{\pgfqpoint{7.842520in}{7.842520in}}%
\pgfusepath{clip}%
\pgfsetbuttcap%
\pgfsetroundjoin%
\definecolor{currentfill}{rgb}{0.525776,0.833491,0.288127}%
\pgfsetfillcolor{currentfill}%
\pgfsetlinewidth{0.000000pt}%
\definecolor{currentstroke}{rgb}{0.449368,0.813768,0.335384}%
\pgfsetstrokecolor{currentstroke}%
\pgfsetdash{}{0pt}%
\pgfpathmoveto{\pgfqpoint{4.066411in}{4.745025in}}%
\pgfpathlineto{\pgfqpoint{3.980947in}{4.722912in}}%
\pgfpathlineto{\pgfqpoint{4.210962in}{4.541645in}}%
\pgfpathclose%
\pgfusepath{fill}%
\end{pgfscope}%
\begin{pgfscope}%
\pgfpathrectangle{\pgfqpoint{0.539299in}{0.078740in}}{\pgfqpoint{7.842520in}{7.842520in}}%
\pgfusepath{clip}%
\pgfsetbuttcap%
\pgfsetroundjoin%
\definecolor{currentfill}{rgb}{0.274149,0.751988,0.436601}%
\pgfsetfillcolor{currentfill}%
\pgfsetlinewidth{0.000000pt}%
\definecolor{currentstroke}{rgb}{0.458674,0.816363,0.329727}%
\pgfsetstrokecolor{currentstroke}%
\pgfsetdash{}{0pt}%
\pgfpathmoveto{\pgfqpoint{4.270980in}{4.328400in}}%
\pgfpathlineto{\pgfqpoint{4.415846in}{4.118860in}}%
\pgfpathlineto{\pgfqpoint{4.499656in}{4.110842in}}%
\pgfpathclose%
\pgfusepath{fill}%
\end{pgfscope}%
\begin{pgfscope}%
\pgfpathrectangle{\pgfqpoint{0.539299in}{0.078740in}}{\pgfqpoint{7.842520in}{7.842520in}}%
\pgfusepath{clip}%
\pgfsetbuttcap%
\pgfsetroundjoin%
\definecolor{currentfill}{rgb}{0.896320,0.893616,0.096335}%
\pgfsetfillcolor{currentfill}%
\pgfsetlinewidth{0.000000pt}%
\definecolor{currentstroke}{rgb}{0.468053,0.818921,0.323998}%
\pgfsetstrokecolor{currentstroke}%
\pgfsetdash{}{0pt}%
\pgfpathmoveto{\pgfqpoint{3.489230in}{5.389845in}}%
\pgfpathlineto{\pgfqpoint{3.546541in}{5.204178in}}%
\pgfpathlineto{\pgfqpoint{3.633064in}{5.262991in}}%
\pgfpathclose%
\pgfusepath{fill}%
\end{pgfscope}%
\begin{pgfscope}%
\pgfpathrectangle{\pgfqpoint{0.539299in}{0.078740in}}{\pgfqpoint{7.842520in}{7.842520in}}%
\pgfusepath{clip}%
\pgfsetbuttcap%
\pgfsetroundjoin%
\definecolor{currentfill}{rgb}{0.273006,0.204520,0.501721}%
\pgfsetfillcolor{currentfill}%
\pgfsetlinewidth{0.000000pt}%
\definecolor{currentstroke}{rgb}{0.477504,0.821444,0.318195}%
\pgfsetstrokecolor{currentstroke}%
\pgfsetdash{}{0pt}%
\pgfpathmoveto{\pgfqpoint{5.696616in}{1.925940in}}%
\pgfpathlineto{\pgfqpoint{5.840111in}{1.685406in}}%
\pgfpathlineto{\pgfqpoint{5.918659in}{1.713508in}}%
\pgfpathclose%
\pgfusepath{fill}%
\end{pgfscope}%
\begin{pgfscope}%
\pgfpathrectangle{\pgfqpoint{0.539299in}{0.078740in}}{\pgfqpoint{7.842520in}{7.842520in}}%
\pgfusepath{clip}%
\pgfsetbuttcap%
\pgfsetroundjoin%
\definecolor{currentfill}{rgb}{0.955300,0.901065,0.118128}%
\pgfsetfillcolor{currentfill}%
\pgfsetlinewidth{0.000000pt}%
\definecolor{currentstroke}{rgb}{0.487026,0.823929,0.312321}%
\pgfsetstrokecolor{currentstroke}%
\pgfsetdash{}{0pt}%
\pgfpathmoveto{\pgfqpoint{3.259203in}{5.402997in}}%
\pgfpathlineto{\pgfqpoint{3.116915in}{5.451034in}}%
\pgfpathlineto{\pgfqpoint{3.030553in}{5.308756in}}%
\pgfpathclose%
\pgfusepath{fill}%
\end{pgfscope}%
\begin{pgfscope}%
\pgfpathrectangle{\pgfqpoint{0.539299in}{0.078740in}}{\pgfqpoint{7.842520in}{7.842520in}}%
\pgfusepath{clip}%
\pgfsetbuttcap%
\pgfsetroundjoin%
\definecolor{currentfill}{rgb}{0.223925,0.334994,0.548053}%
\pgfsetfillcolor{currentfill}%
\pgfsetlinewidth{0.000000pt}%
\definecolor{currentstroke}{rgb}{0.496615,0.826376,0.306377}%
\pgfsetstrokecolor{currentstroke}%
\pgfsetdash{}{0pt}%
\pgfpathmoveto{\pgfqpoint{3.460014in}{2.016906in}}%
\pgfpathlineto{\pgfqpoint{3.395040in}{2.713747in}}%
\pgfpathlineto{\pgfqpoint{3.317407in}{2.078347in}}%
\pgfpathclose%
\pgfusepath{fill}%
\end{pgfscope}%
\begin{pgfscope}%
\pgfpathrectangle{\pgfqpoint{0.539299in}{0.078740in}}{\pgfqpoint{7.842520in}{7.842520in}}%
\pgfusepath{clip}%
\pgfsetbuttcap%
\pgfsetroundjoin%
\definecolor{currentfill}{rgb}{0.835270,0.886029,0.102646}%
\pgfsetfillcolor{currentfill}%
\pgfsetlinewidth{0.000000pt}%
\definecolor{currentstroke}{rgb}{0.506271,0.828786,0.300362}%
\pgfsetstrokecolor{currentstroke}%
\pgfsetdash{}{0pt}%
\pgfpathmoveto{\pgfqpoint{2.945073in}{5.109745in}}%
\pgfpathlineto{\pgfqpoint{2.889650in}{5.309100in}}%
\pgfpathlineto{\pgfqpoint{2.804286in}{5.106496in}}%
\pgfpathclose%
\pgfusepath{fill}%
\end{pgfscope}%
\begin{pgfscope}%
\pgfpathrectangle{\pgfqpoint{0.539299in}{0.078740in}}{\pgfqpoint{7.842520in}{7.842520in}}%
\pgfusepath{clip}%
\pgfsetbuttcap%
\pgfsetroundjoin%
\definecolor{currentfill}{rgb}{0.153364,0.497000,0.557724}%
\pgfsetfillcolor{currentfill}%
\pgfsetlinewidth{0.000000pt}%
\definecolor{currentstroke}{rgb}{0.515992,0.831158,0.294279}%
\pgfsetstrokecolor{currentstroke}%
\pgfsetdash{}{0pt}%
\pgfpathmoveto{\pgfqpoint{5.056080in}{3.038167in}}%
\pgfpathlineto{\pgfqpoint{5.200440in}{2.822813in}}%
\pgfpathlineto{\pgfqpoint{5.137323in}{3.030184in}}%
\pgfpathclose%
\pgfusepath{fill}%
\end{pgfscope}%
\begin{pgfscope}%
\pgfpathrectangle{\pgfqpoint{0.539299in}{0.078740in}}{\pgfqpoint{7.842520in}{7.842520in}}%
\pgfusepath{clip}%
\pgfsetbuttcap%
\pgfsetroundjoin%
\definecolor{currentfill}{rgb}{0.955300,0.901065,0.118128}%
\pgfsetfillcolor{currentfill}%
\pgfsetlinewidth{0.000000pt}%
\definecolor{currentstroke}{rgb}{0.525776,0.833491,0.288127}%
\pgfsetstrokecolor{currentstroke}%
\pgfsetdash{}{0pt}%
\pgfpathmoveto{\pgfqpoint{3.402504in}{5.318957in}}%
\pgfpathlineto{\pgfqpoint{3.489230in}{5.389845in}}%
\pgfpathlineto{\pgfqpoint{3.259203in}{5.402997in}}%
\pgfpathclose%
\pgfusepath{fill}%
\end{pgfscope}%
\begin{pgfscope}%
\pgfpathrectangle{\pgfqpoint{0.539299in}{0.078740in}}{\pgfqpoint{7.842520in}{7.842520in}}%
\pgfusepath{clip}%
\pgfsetbuttcap%
\pgfsetroundjoin%
\definecolor{currentfill}{rgb}{0.125394,0.574318,0.549086}%
\pgfsetfillcolor{currentfill}%
\pgfsetlinewidth{0.000000pt}%
\definecolor{currentstroke}{rgb}{0.535621,0.835785,0.281908}%
\pgfsetstrokecolor{currentstroke}%
\pgfsetdash{}{0pt}%
\pgfpathmoveto{\pgfqpoint{2.901907in}{3.544823in}}%
\pgfpathlineto{\pgfqpoint{2.825356in}{2.965660in}}%
\pgfpathlineto{\pgfqpoint{3.044293in}{3.487491in}}%
\pgfpathclose%
\pgfusepath{fill}%
\end{pgfscope}%
\begin{pgfscope}%
\pgfpathrectangle{\pgfqpoint{0.539299in}{0.078740in}}{\pgfqpoint{7.842520in}{7.842520in}}%
\pgfusepath{clip}%
\pgfsetbuttcap%
\pgfsetroundjoin%
\definecolor{currentfill}{rgb}{0.606045,0.850733,0.236712}%
\pgfsetfillcolor{currentfill}%
\pgfsetlinewidth{0.000000pt}%
\definecolor{currentstroke}{rgb}{0.545524,0.838039,0.275626}%
\pgfsetstrokecolor{currentstroke}%
\pgfsetdash{}{0pt}%
\pgfpathmoveto{\pgfqpoint{3.921827in}{4.935824in}}%
\pgfpathlineto{\pgfqpoint{3.980947in}{4.722912in}}%
\pgfpathlineto{\pgfqpoint{4.066411in}{4.745025in}}%
\pgfpathclose%
\pgfusepath{fill}%
\end{pgfscope}%
\begin{pgfscope}%
\pgfpathrectangle{\pgfqpoint{0.539299in}{0.078740in}}{\pgfqpoint{7.842520in}{7.842520in}}%
\pgfusepath{clip}%
\pgfsetbuttcap%
\pgfsetroundjoin%
\definecolor{currentfill}{rgb}{0.192357,0.403199,0.555836}%
\pgfsetfillcolor{currentfill}%
\pgfsetlinewidth{0.000000pt}%
\definecolor{currentstroke}{rgb}{0.555484,0.840254,0.269281}%
\pgfsetstrokecolor{currentstroke}%
\pgfsetdash{}{0pt}%
\pgfpathmoveto{\pgfqpoint{3.395040in}{2.713747in}}%
\pgfpathlineto{\pgfqpoint{3.251833in}{2.786336in}}%
\pgfpathlineto{\pgfqpoint{3.175266in}{2.137225in}}%
\pgfpathclose%
\pgfusepath{fill}%
\end{pgfscope}%
\begin{pgfscope}%
\pgfpathrectangle{\pgfqpoint{0.539299in}{0.078740in}}{\pgfqpoint{7.842520in}{7.842520in}}%
\pgfusepath{clip}%
\pgfsetbuttcap%
\pgfsetroundjoin%
\definecolor{currentfill}{rgb}{0.845561,0.887322,0.099702}%
\pgfsetfillcolor{currentfill}%
\pgfsetlinewidth{0.000000pt}%
\definecolor{currentstroke}{rgb}{0.565498,0.842430,0.262877}%
\pgfsetstrokecolor{currentstroke}%
\pgfsetdash{}{0pt}%
\pgfpathmoveto{\pgfqpoint{3.633064in}{5.262991in}}%
\pgfpathlineto{\pgfqpoint{3.546541in}{5.204178in}}%
\pgfpathlineto{\pgfqpoint{3.777327in}{5.109990in}}%
\pgfpathclose%
\pgfusepath{fill}%
\end{pgfscope}%
\begin{pgfscope}%
\pgfpathrectangle{\pgfqpoint{0.539299in}{0.078740in}}{\pgfqpoint{7.842520in}{7.842520in}}%
\pgfusepath{clip}%
\pgfsetbuttcap%
\pgfsetroundjoin%
\definecolor{currentfill}{rgb}{0.751884,0.874951,0.143228}%
\pgfsetfillcolor{currentfill}%
\pgfsetlinewidth{0.000000pt}%
\definecolor{currentstroke}{rgb}{0.575563,0.844566,0.256415}%
\pgfsetstrokecolor{currentstroke}%
\pgfsetdash{}{0pt}%
\pgfpathmoveto{\pgfqpoint{3.921827in}{4.935824in}}%
\pgfpathlineto{\pgfqpoint{3.777327in}{5.109990in}}%
\pgfpathlineto{\pgfqpoint{3.691083in}{5.063599in}}%
\pgfpathclose%
\pgfusepath{fill}%
\end{pgfscope}%
\begin{pgfscope}%
\pgfpathrectangle{\pgfqpoint{0.539299in}{0.078740in}}{\pgfqpoint{7.842520in}{7.842520in}}%
\pgfusepath{clip}%
\pgfsetbuttcap%
\pgfsetroundjoin%
\definecolor{currentfill}{rgb}{0.143343,0.522773,0.556295}%
\pgfsetfillcolor{currentfill}%
\pgfsetlinewidth{0.000000pt}%
\definecolor{currentstroke}{rgb}{0.585678,0.846661,0.249897}%
\pgfsetstrokecolor{currentstroke}%
\pgfsetdash{}{0pt}%
\pgfpathmoveto{\pgfqpoint{2.966934in}{2.913287in}}%
\pgfpathlineto{\pgfqpoint{3.109114in}{2.853159in}}%
\pgfpathlineto{\pgfqpoint{3.044293in}{3.487491in}}%
\pgfpathclose%
\pgfusepath{fill}%
\end{pgfscope}%
\begin{pgfscope}%
\pgfpathrectangle{\pgfqpoint{0.539299in}{0.078740in}}{\pgfqpoint{7.842520in}{7.842520in}}%
\pgfusepath{clip}%
\pgfsetbuttcap%
\pgfsetroundjoin%
\definecolor{currentfill}{rgb}{0.225863,0.330805,0.547314}%
\pgfsetfillcolor{currentfill}%
\pgfsetlinewidth{0.000000pt}%
\definecolor{currentstroke}{rgb}{0.595839,0.848717,0.243329}%
\pgfsetstrokecolor{currentstroke}%
\pgfsetdash{}{0pt}%
\pgfpathmoveto{\pgfqpoint{5.552680in}{2.156474in}}%
\pgfpathlineto{\pgfqpoint{5.632182in}{2.166073in}}%
\pgfpathlineto{\pgfqpoint{5.488510in}{2.387296in}}%
\pgfpathclose%
\pgfusepath{fill}%
\end{pgfscope}%
\begin{pgfscope}%
\pgfpathrectangle{\pgfqpoint{0.539299in}{0.078740in}}{\pgfqpoint{7.842520in}{7.842520in}}%
\pgfusepath{clip}%
\pgfsetbuttcap%
\pgfsetroundjoin%
\definecolor{currentfill}{rgb}{0.886271,0.892374,0.095374}%
\pgfsetfillcolor{currentfill}%
\pgfsetlinewidth{0.000000pt}%
\definecolor{currentstroke}{rgb}{0.606045,0.850733,0.236712}%
\pgfsetstrokecolor{currentstroke}%
\pgfsetdash{}{0pt}%
\pgfpathmoveto{\pgfqpoint{3.030553in}{5.308756in}}%
\pgfpathlineto{\pgfqpoint{2.889650in}{5.309100in}}%
\pgfpathlineto{\pgfqpoint{2.945073in}{5.109745in}}%
\pgfpathclose%
\pgfusepath{fill}%
\end{pgfscope}%
\begin{pgfscope}%
\pgfpathrectangle{\pgfqpoint{0.539299in}{0.078740in}}{\pgfqpoint{7.842520in}{7.842520in}}%
\pgfusepath{clip}%
\pgfsetbuttcap%
\pgfsetroundjoin%
\definecolor{currentfill}{rgb}{0.377779,0.791781,0.377939}%
\pgfsetfillcolor{currentfill}%
\pgfsetlinewidth{0.000000pt}%
\definecolor{currentstroke}{rgb}{0.616293,0.852709,0.230052}%
\pgfsetstrokecolor{currentstroke}%
\pgfsetdash{}{0pt}%
\pgfpathmoveto{\pgfqpoint{2.637836in}{4.513404in}}%
\pgfpathlineto{\pgfqpoint{2.697635in}{4.107094in}}%
\pgfpathlineto{\pgfqpoint{2.778216in}{4.515166in}}%
\pgfpathclose%
\pgfusepath{fill}%
\end{pgfscope}%
\begin{pgfscope}%
\pgfpathrectangle{\pgfqpoint{0.539299in}{0.078740in}}{\pgfqpoint{7.842520in}{7.842520in}}%
\pgfusepath{clip}%
\pgfsetbuttcap%
\pgfsetroundjoin%
\definecolor{currentfill}{rgb}{0.140536,0.530132,0.555659}%
\pgfsetfillcolor{currentfill}%
\pgfsetlinewidth{0.000000pt}%
\definecolor{currentstroke}{rgb}{0.626579,0.854645,0.223353}%
\pgfsetstrokecolor{currentstroke}%
\pgfsetdash{}{0pt}%
\pgfpathmoveto{\pgfqpoint{4.911519in}{3.252217in}}%
\pgfpathlineto{\pgfqpoint{5.056080in}{3.038167in}}%
\pgfpathlineto{\pgfqpoint{5.137323in}{3.030184in}}%
\pgfpathclose%
\pgfusepath{fill}%
\end{pgfscope}%
\begin{pgfscope}%
\pgfpathrectangle{\pgfqpoint{0.539299in}{0.078740in}}{\pgfqpoint{7.842520in}{7.842520in}}%
\pgfusepath{clip}%
\pgfsetbuttcap%
\pgfsetroundjoin%
\definecolor{currentfill}{rgb}{0.575563,0.844566,0.256415}%
\pgfsetfillcolor{currentfill}%
\pgfsetlinewidth{0.000000pt}%
\definecolor{currentstroke}{rgb}{0.636902,0.856542,0.216620}%
\pgfsetstrokecolor{currentstroke}%
\pgfsetdash{}{0pt}%
\pgfpathmoveto{\pgfqpoint{2.720220in}{4.843367in}}%
\pgfpathlineto{\pgfqpoint{2.637836in}{4.513404in}}%
\pgfpathlineto{\pgfqpoint{2.860832in}{4.847363in}}%
\pgfpathclose%
\pgfusepath{fill}%
\end{pgfscope}%
\begin{pgfscope}%
\pgfpathrectangle{\pgfqpoint{0.539299in}{0.078740in}}{\pgfqpoint{7.842520in}{7.842520in}}%
\pgfusepath{clip}%
\pgfsetbuttcap%
\pgfsetroundjoin%
\definecolor{currentfill}{rgb}{0.699415,0.867117,0.175971}%
\pgfsetfillcolor{currentfill}%
\pgfsetlinewidth{0.000000pt}%
\definecolor{currentstroke}{rgb}{0.647257,0.858400,0.209861}%
\pgfsetstrokecolor{currentstroke}%
\pgfsetdash{}{0pt}%
\pgfpathmoveto{\pgfqpoint{2.804286in}{5.106496in}}%
\pgfpathlineto{\pgfqpoint{2.720220in}{4.843367in}}%
\pgfpathlineto{\pgfqpoint{2.860832in}{4.847363in}}%
\pgfpathclose%
\pgfusepath{fill}%
\end{pgfscope}%
\begin{pgfscope}%
\pgfpathrectangle{\pgfqpoint{0.539299in}{0.078740in}}{\pgfqpoint{7.842520in}{7.842520in}}%
\pgfusepath{clip}%
\pgfsetbuttcap%
\pgfsetroundjoin%
\definecolor{currentfill}{rgb}{0.421908,0.805774,0.351910}%
\pgfsetfillcolor{currentfill}%
\pgfsetlinewidth{0.000000pt}%
\definecolor{currentstroke}{rgb}{0.657642,0.860219,0.203082}%
\pgfsetstrokecolor{currentstroke}%
\pgfsetdash{}{0pt}%
\pgfpathmoveto{\pgfqpoint{4.125992in}{4.530691in}}%
\pgfpathlineto{\pgfqpoint{4.270980in}{4.328400in}}%
\pgfpathlineto{\pgfqpoint{4.210962in}{4.541645in}}%
\pgfpathclose%
\pgfusepath{fill}%
\end{pgfscope}%
\begin{pgfscope}%
\pgfpathrectangle{\pgfqpoint{0.539299in}{0.078740in}}{\pgfqpoint{7.842520in}{7.842520in}}%
\pgfusepath{clip}%
\pgfsetbuttcap%
\pgfsetroundjoin%
\definecolor{currentfill}{rgb}{0.926106,0.897330,0.104071}%
\pgfsetfillcolor{currentfill}%
\pgfsetlinewidth{0.000000pt}%
\definecolor{currentstroke}{rgb}{0.668054,0.861999,0.196293}%
\pgfsetstrokecolor{currentstroke}%
\pgfsetdash{}{0pt}%
\pgfpathmoveto{\pgfqpoint{3.402504in}{5.318957in}}%
\pgfpathlineto{\pgfqpoint{3.546541in}{5.204178in}}%
\pgfpathlineto{\pgfqpoint{3.489230in}{5.389845in}}%
\pgfpathclose%
\pgfusepath{fill}%
\end{pgfscope}%
\begin{pgfscope}%
\pgfpathrectangle{\pgfqpoint{0.539299in}{0.078740in}}{\pgfqpoint{7.842520in}{7.842520in}}%
\pgfusepath{clip}%
\pgfsetbuttcap%
\pgfsetroundjoin%
\definecolor{currentfill}{rgb}{0.124780,0.640461,0.527068}%
\pgfsetfillcolor{currentfill}%
\pgfsetlinewidth{0.000000pt}%
\definecolor{currentstroke}{rgb}{0.678489,0.863742,0.189503}%
\pgfsetstrokecolor{currentstroke}%
\pgfsetdash{}{0pt}%
\pgfpathmoveto{\pgfqpoint{4.849351in}{3.468201in}}%
\pgfpathlineto{\pgfqpoint{4.705051in}{3.687048in}}%
\pgfpathlineto{\pgfqpoint{4.621833in}{3.675341in}}%
\pgfpathclose%
\pgfusepath{fill}%
\end{pgfscope}%
\begin{pgfscope}%
\pgfpathrectangle{\pgfqpoint{0.539299in}{0.078740in}}{\pgfqpoint{7.842520in}{7.842520in}}%
\pgfusepath{clip}%
\pgfsetbuttcap%
\pgfsetroundjoin%
\definecolor{currentfill}{rgb}{0.140210,0.665859,0.513427}%
\pgfsetfillcolor{currentfill}%
\pgfsetlinewidth{0.000000pt}%
\definecolor{currentstroke}{rgb}{0.688944,0.865448,0.182725}%
\pgfsetstrokecolor{currentstroke}%
\pgfsetdash{}{0pt}%
\pgfpathmoveto{\pgfqpoint{2.760273in}{3.588885in}}%
\pgfpathlineto{\pgfqpoint{2.901907in}{3.544823in}}%
\pgfpathlineto{\pgfqpoint{2.838886in}{4.080766in}}%
\pgfpathclose%
\pgfusepath{fill}%
\end{pgfscope}%
\begin{pgfscope}%
\pgfpathrectangle{\pgfqpoint{0.539299in}{0.078740in}}{\pgfqpoint{7.842520in}{7.842520in}}%
\pgfusepath{clip}%
\pgfsetbuttcap%
\pgfsetroundjoin%
\definecolor{currentfill}{rgb}{0.496615,0.826376,0.306377}%
\pgfsetfillcolor{currentfill}%
\pgfsetlinewidth{0.000000pt}%
\definecolor{currentstroke}{rgb}{0.699415,0.867117,0.175971}%
\pgfsetstrokecolor{currentstroke}%
\pgfsetdash{}{0pt}%
\pgfpathmoveto{\pgfqpoint{4.210962in}{4.541645in}}%
\pgfpathlineto{\pgfqpoint{3.980947in}{4.722912in}}%
\pgfpathlineto{\pgfqpoint{4.125992in}{4.530691in}}%
\pgfpathclose%
\pgfusepath{fill}%
\end{pgfscope}%
\begin{pgfscope}%
\pgfpathrectangle{\pgfqpoint{0.539299in}{0.078740in}}{\pgfqpoint{7.842520in}{7.842520in}}%
\pgfusepath{clip}%
\pgfsetbuttcap%
\pgfsetroundjoin%
\definecolor{currentfill}{rgb}{0.199430,0.387607,0.554642}%
\pgfsetfillcolor{currentfill}%
\pgfsetlinewidth{0.000000pt}%
\definecolor{currentstroke}{rgb}{0.709898,0.868751,0.169257}%
\pgfsetstrokecolor{currentstroke}%
\pgfsetdash{}{0pt}%
\pgfpathmoveto{\pgfqpoint{5.408389in}{2.379042in}}%
\pgfpathlineto{\pgfqpoint{5.488510in}{2.387296in}}%
\pgfpathlineto{\pgfqpoint{5.344589in}{2.606000in}}%
\pgfpathclose%
\pgfusepath{fill}%
\end{pgfscope}%
\begin{pgfscope}%
\pgfpathrectangle{\pgfqpoint{0.539299in}{0.078740in}}{\pgfqpoint{7.842520in}{7.842520in}}%
\pgfusepath{clip}%
\pgfsetbuttcap%
\pgfsetroundjoin%
\definecolor{currentfill}{rgb}{0.241237,0.296485,0.539709}%
\pgfsetfillcolor{currentfill}%
\pgfsetlinewidth{0.000000pt}%
\definecolor{currentstroke}{rgb}{0.720391,0.870350,0.162603}%
\pgfsetstrokecolor{currentstroke}%
\pgfsetdash{}{0pt}%
\pgfpathmoveto{\pgfqpoint{5.696616in}{1.925940in}}%
\pgfpathlineto{\pgfqpoint{5.632182in}{2.166073in}}%
\pgfpathlineto{\pgfqpoint{5.552680in}{2.156474in}}%
\pgfpathclose%
\pgfusepath{fill}%
\end{pgfscope}%
\begin{pgfscope}%
\pgfpathrectangle{\pgfqpoint{0.539299in}{0.078740in}}{\pgfqpoint{7.842520in}{7.842520in}}%
\pgfusepath{clip}%
\pgfsetbuttcap%
\pgfsetroundjoin%
\definecolor{currentfill}{rgb}{0.515992,0.831158,0.294279}%
\pgfsetfillcolor{currentfill}%
\pgfsetlinewidth{0.000000pt}%
\definecolor{currentstroke}{rgb}{0.730889,0.871916,0.156029}%
\pgfsetstrokecolor{currentstroke}%
\pgfsetdash{}{0pt}%
\pgfpathmoveto{\pgfqpoint{2.860832in}{4.847363in}}%
\pgfpathlineto{\pgfqpoint{2.637836in}{4.513404in}}%
\pgfpathlineto{\pgfqpoint{2.778216in}{4.515166in}}%
\pgfpathclose%
\pgfusepath{fill}%
\end{pgfscope}%
\begin{pgfscope}%
\pgfpathrectangle{\pgfqpoint{0.539299in}{0.078740in}}{\pgfqpoint{7.842520in}{7.842520in}}%
\pgfusepath{clip}%
\pgfsetbuttcap%
\pgfsetroundjoin%
\definecolor{currentfill}{rgb}{0.945636,0.899815,0.112838}%
\pgfsetfillcolor{currentfill}%
\pgfsetlinewidth{0.000000pt}%
\definecolor{currentstroke}{rgb}{0.741388,0.873449,0.149561}%
\pgfsetstrokecolor{currentstroke}%
\pgfsetdash{}{0pt}%
\pgfpathmoveto{\pgfqpoint{3.030553in}{5.308756in}}%
\pgfpathlineto{\pgfqpoint{3.172817in}{5.269283in}}%
\pgfpathlineto{\pgfqpoint{3.259203in}{5.402997in}}%
\pgfpathclose%
\pgfusepath{fill}%
\end{pgfscope}%
\begin{pgfscope}%
\pgfpathrectangle{\pgfqpoint{0.539299in}{0.078740in}}{\pgfqpoint{7.842520in}{7.842520in}}%
\pgfusepath{clip}%
\pgfsetbuttcap%
\pgfsetroundjoin%
\definecolor{currentfill}{rgb}{0.311925,0.767822,0.415586}%
\pgfsetfillcolor{currentfill}%
\pgfsetlinewidth{0.000000pt}%
\definecolor{currentstroke}{rgb}{0.751884,0.874951,0.143228}%
\pgfsetstrokecolor{currentstroke}%
\pgfsetdash{}{0pt}%
\pgfpathmoveto{\pgfqpoint{2.778216in}{4.515166in}}%
\pgfpathlineto{\pgfqpoint{2.697635in}{4.107094in}}%
\pgfpathlineto{\pgfqpoint{2.838886in}{4.080766in}}%
\pgfpathclose%
\pgfusepath{fill}%
\end{pgfscope}%
\begin{pgfscope}%
\pgfpathrectangle{\pgfqpoint{0.539299in}{0.078740in}}{\pgfqpoint{7.842520in}{7.842520in}}%
\pgfusepath{clip}%
\pgfsetbuttcap%
\pgfsetroundjoin%
\definecolor{currentfill}{rgb}{0.762373,0.876424,0.137064}%
\pgfsetfillcolor{currentfill}%
\pgfsetlinewidth{0.000000pt}%
\definecolor{currentstroke}{rgb}{0.762373,0.876424,0.137064}%
\pgfsetstrokecolor{currentstroke}%
\pgfsetdash{}{0pt}%
\pgfpathmoveto{\pgfqpoint{2.860832in}{4.847363in}}%
\pgfpathlineto{\pgfqpoint{2.945073in}{5.109745in}}%
\pgfpathlineto{\pgfqpoint{2.804286in}{5.106496in}}%
\pgfpathclose%
\pgfusepath{fill}%
\end{pgfscope}%
\begin{pgfscope}%
\pgfpathrectangle{\pgfqpoint{0.539299in}{0.078740in}}{\pgfqpoint{7.842520in}{7.842520in}}%
\pgfusepath{clip}%
\pgfsetbuttcap%
\pgfsetroundjoin%
\definecolor{currentfill}{rgb}{0.824940,0.884720,0.106217}%
\pgfsetfillcolor{currentfill}%
\pgfsetlinewidth{0.000000pt}%
\definecolor{currentstroke}{rgb}{0.772852,0.877868,0.131109}%
\pgfsetstrokecolor{currentstroke}%
\pgfsetdash{}{0pt}%
\pgfpathmoveto{\pgfqpoint{3.777327in}{5.109990in}}%
\pgfpathlineto{\pgfqpoint{3.546541in}{5.204178in}}%
\pgfpathlineto{\pgfqpoint{3.691083in}{5.063599in}}%
\pgfpathclose%
\pgfusepath{fill}%
\end{pgfscope}%
\begin{pgfscope}%
\pgfpathrectangle{\pgfqpoint{0.539299in}{0.078740in}}{\pgfqpoint{7.842520in}{7.842520in}}%
\pgfusepath{clip}%
\pgfsetbuttcap%
\pgfsetroundjoin%
\definecolor{currentfill}{rgb}{0.162016,0.687316,0.499129}%
\pgfsetfillcolor{currentfill}%
\pgfsetlinewidth{0.000000pt}%
\definecolor{currentstroke}{rgb}{0.783315,0.879285,0.125405}%
\pgfsetstrokecolor{currentstroke}%
\pgfsetdash{}{0pt}%
\pgfpathmoveto{\pgfqpoint{4.705051in}{3.687048in}}%
\pgfpathlineto{\pgfqpoint{4.560545in}{3.904435in}}%
\pgfpathlineto{\pgfqpoint{4.476731in}{3.882966in}}%
\pgfpathclose%
\pgfusepath{fill}%
\end{pgfscope}%
\begin{pgfscope}%
\pgfpathrectangle{\pgfqpoint{0.539299in}{0.078740in}}{\pgfqpoint{7.842520in}{7.842520in}}%
\pgfusepath{clip}%
\pgfsetbuttcap%
\pgfsetroundjoin%
\definecolor{currentfill}{rgb}{0.657642,0.860219,0.203082}%
\pgfsetfillcolor{currentfill}%
\pgfsetlinewidth{0.000000pt}%
\definecolor{currentstroke}{rgb}{0.793760,0.880678,0.120005}%
\pgfsetstrokecolor{currentstroke}%
\pgfsetdash{}{0pt}%
\pgfpathmoveto{\pgfqpoint{3.835937in}{4.901790in}}%
\pgfpathlineto{\pgfqpoint{3.980947in}{4.722912in}}%
\pgfpathlineto{\pgfqpoint{3.921827in}{4.935824in}}%
\pgfpathclose%
\pgfusepath{fill}%
\end{pgfscope}%
\begin{pgfscope}%
\pgfpathrectangle{\pgfqpoint{0.539299in}{0.078740in}}{\pgfqpoint{7.842520in}{7.842520in}}%
\pgfusepath{clip}%
\pgfsetbuttcap%
\pgfsetroundjoin%
\definecolor{currentfill}{rgb}{0.120565,0.596422,0.543611}%
\pgfsetfillcolor{currentfill}%
\pgfsetlinewidth{0.000000pt}%
\definecolor{currentstroke}{rgb}{0.804182,0.882046,0.114965}%
\pgfsetstrokecolor{currentstroke}%
\pgfsetdash{}{0pt}%
\pgfpathmoveto{\pgfqpoint{4.766767in}{3.464787in}}%
\pgfpathlineto{\pgfqpoint{4.911519in}{3.252217in}}%
\pgfpathlineto{\pgfqpoint{4.849351in}{3.468201in}}%
\pgfpathclose%
\pgfusepath{fill}%
\end{pgfscope}%
\begin{pgfscope}%
\pgfpathrectangle{\pgfqpoint{0.539299in}{0.078740in}}{\pgfqpoint{7.842520in}{7.842520in}}%
\pgfusepath{clip}%
\pgfsetbuttcap%
\pgfsetroundjoin%
\definecolor{currentfill}{rgb}{0.730889,0.871916,0.156029}%
\pgfsetfillcolor{currentfill}%
\pgfsetlinewidth{0.000000pt}%
\definecolor{currentstroke}{rgb}{0.814576,0.883393,0.110347}%
\pgfsetstrokecolor{currentstroke}%
\pgfsetdash{}{0pt}%
\pgfpathmoveto{\pgfqpoint{3.691083in}{5.063599in}}%
\pgfpathlineto{\pgfqpoint{3.835937in}{4.901790in}}%
\pgfpathlineto{\pgfqpoint{3.921827in}{4.935824in}}%
\pgfpathclose%
\pgfusepath{fill}%
\end{pgfscope}%
\begin{pgfscope}%
\pgfpathrectangle{\pgfqpoint{0.539299in}{0.078740in}}{\pgfqpoint{7.842520in}{7.842520in}}%
\pgfusepath{clip}%
\pgfsetbuttcap%
\pgfsetroundjoin%
\definecolor{currentfill}{rgb}{0.280255,0.165693,0.476498}%
\pgfsetfillcolor{currentfill}%
\pgfsetlinewidth{0.000000pt}%
\definecolor{currentstroke}{rgb}{0.824940,0.884720,0.106217}%
\pgfsetstrokecolor{currentstroke}%
\pgfsetdash{}{0pt}%
\pgfpathmoveto{\pgfqpoint{5.983060in}{1.432854in}}%
\pgfpathlineto{\pgfqpoint{5.840111in}{1.685406in}}%
\pgfpathlineto{\pgfqpoint{5.760821in}{1.647418in}}%
\pgfpathclose%
\pgfusepath{fill}%
\end{pgfscope}%
\begin{pgfscope}%
\pgfpathrectangle{\pgfqpoint{0.539299in}{0.078740in}}{\pgfqpoint{7.842520in}{7.842520in}}%
\pgfusepath{clip}%
\pgfsetbuttcap%
\pgfsetroundjoin%
\definecolor{currentfill}{rgb}{0.227802,0.326594,0.546532}%
\pgfsetfillcolor{currentfill}%
\pgfsetlinewidth{0.000000pt}%
\definecolor{currentstroke}{rgb}{0.835270,0.886029,0.102646}%
\pgfsetstrokecolor{currentstroke}%
\pgfsetdash{}{0pt}%
\pgfpathmoveto{\pgfqpoint{3.460014in}{2.016906in}}%
\pgfpathlineto{\pgfqpoint{3.603070in}{1.953238in}}%
\pgfpathlineto{\pgfqpoint{3.682755in}{2.554388in}}%
\pgfpathclose%
\pgfusepath{fill}%
\end{pgfscope}%
\begin{pgfscope}%
\pgfpathrectangle{\pgfqpoint{0.539299in}{0.078740in}}{\pgfqpoint{7.842520in}{7.842520in}}%
\pgfusepath{clip}%
\pgfsetbuttcap%
\pgfsetroundjoin%
\definecolor{currentfill}{rgb}{0.121380,0.629492,0.531973}%
\pgfsetfillcolor{currentfill}%
\pgfsetlinewidth{0.000000pt}%
\definecolor{currentstroke}{rgb}{0.845561,0.887322,0.099702}%
\pgfsetstrokecolor{currentstroke}%
\pgfsetdash{}{0pt}%
\pgfpathmoveto{\pgfqpoint{4.849351in}{3.468201in}}%
\pgfpathlineto{\pgfqpoint{4.621833in}{3.675341in}}%
\pgfpathlineto{\pgfqpoint{4.766767in}{3.464787in}}%
\pgfpathclose%
\pgfusepath{fill}%
\end{pgfscope}%
\begin{pgfscope}%
\pgfpathrectangle{\pgfqpoint{0.539299in}{0.078740in}}{\pgfqpoint{7.842520in}{7.842520in}}%
\pgfusepath{clip}%
\pgfsetbuttcap%
\pgfsetroundjoin%
\definecolor{currentfill}{rgb}{0.935904,0.898570,0.108131}%
\pgfsetfillcolor{currentfill}%
\pgfsetlinewidth{0.000000pt}%
\definecolor{currentstroke}{rgb}{0.855810,0.888601,0.097452}%
\pgfsetstrokecolor{currentstroke}%
\pgfsetdash{}{0pt}%
\pgfpathmoveto{\pgfqpoint{3.259203in}{5.402997in}}%
\pgfpathlineto{\pgfqpoint{3.316139in}{5.195751in}}%
\pgfpathlineto{\pgfqpoint{3.402504in}{5.318957in}}%
\pgfpathclose%
\pgfusepath{fill}%
\end{pgfscope}%
\begin{pgfscope}%
\pgfpathrectangle{\pgfqpoint{0.539299in}{0.078740in}}{\pgfqpoint{7.842520in}{7.842520in}}%
\pgfusepath{clip}%
\pgfsetbuttcap%
\pgfsetroundjoin%
\definecolor{currentfill}{rgb}{0.208030,0.718701,0.472873}%
\pgfsetfillcolor{currentfill}%
\pgfsetlinewidth{0.000000pt}%
\definecolor{currentstroke}{rgb}{0.866013,0.889868,0.095953}%
\pgfsetstrokecolor{currentstroke}%
\pgfsetdash{}{0pt}%
\pgfpathmoveto{\pgfqpoint{4.476731in}{3.882966in}}%
\pgfpathlineto{\pgfqpoint{4.560545in}{3.904435in}}%
\pgfpathlineto{\pgfqpoint{4.415846in}{4.118860in}}%
\pgfpathclose%
\pgfusepath{fill}%
\end{pgfscope}%
\begin{pgfscope}%
\pgfpathrectangle{\pgfqpoint{0.539299in}{0.078740in}}{\pgfqpoint{7.842520in}{7.842520in}}%
\pgfusepath{clip}%
\pgfsetbuttcap%
\pgfsetroundjoin%
\definecolor{currentfill}{rgb}{0.212395,0.359683,0.551710}%
\pgfsetfillcolor{currentfill}%
\pgfsetlinewidth{0.000000pt}%
\definecolor{currentstroke}{rgb}{0.876168,0.891125,0.095250}%
\pgfsetstrokecolor{currentstroke}%
\pgfsetdash{}{0pt}%
\pgfpathmoveto{\pgfqpoint{5.488510in}{2.387296in}}%
\pgfpathlineto{\pgfqpoint{5.408389in}{2.379042in}}%
\pgfpathlineto{\pgfqpoint{5.552680in}{2.156474in}}%
\pgfpathclose%
\pgfusepath{fill}%
\end{pgfscope}%
\begin{pgfscope}%
\pgfpathrectangle{\pgfqpoint{0.539299in}{0.078740in}}{\pgfqpoint{7.842520in}{7.842520in}}%
\pgfusepath{clip}%
\pgfsetbuttcap%
\pgfsetroundjoin%
\definecolor{currentfill}{rgb}{0.195860,0.395433,0.555276}%
\pgfsetfillcolor{currentfill}%
\pgfsetlinewidth{0.000000pt}%
\definecolor{currentstroke}{rgb}{0.886271,0.892374,0.095374}%
\pgfsetstrokecolor{currentstroke}%
\pgfsetdash{}{0pt}%
\pgfpathmoveto{\pgfqpoint{3.538693in}{2.636201in}}%
\pgfpathlineto{\pgfqpoint{3.395040in}{2.713747in}}%
\pgfpathlineto{\pgfqpoint{3.460014in}{2.016906in}}%
\pgfpathclose%
\pgfusepath{fill}%
\end{pgfscope}%
\begin{pgfscope}%
\pgfpathrectangle{\pgfqpoint{0.539299in}{0.078740in}}{\pgfqpoint{7.842520in}{7.842520in}}%
\pgfusepath{clip}%
\pgfsetbuttcap%
\pgfsetroundjoin%
\definecolor{currentfill}{rgb}{0.168126,0.459988,0.558082}%
\pgfsetfillcolor{currentfill}%
\pgfsetlinewidth{0.000000pt}%
\definecolor{currentstroke}{rgb}{0.896320,0.893616,0.096335}%
\pgfsetstrokecolor{currentstroke}%
\pgfsetdash{}{0pt}%
\pgfpathmoveto{\pgfqpoint{5.344589in}{2.606000in}}%
\pgfpathlineto{\pgfqpoint{5.200440in}{2.822813in}}%
\pgfpathlineto{\pgfqpoint{5.119001in}{2.807062in}}%
\pgfpathclose%
\pgfusepath{fill}%
\end{pgfscope}%
\begin{pgfscope}%
\pgfpathrectangle{\pgfqpoint{0.539299in}{0.078740in}}{\pgfqpoint{7.842520in}{7.842520in}}%
\pgfusepath{clip}%
\pgfsetbuttcap%
\pgfsetroundjoin%
\definecolor{currentfill}{rgb}{0.126453,0.570633,0.549841}%
\pgfsetfillcolor{currentfill}%
\pgfsetlinewidth{0.000000pt}%
\definecolor{currentstroke}{rgb}{0.906311,0.894855,0.098125}%
\pgfsetstrokecolor{currentstroke}%
\pgfsetdash{}{0pt}%
\pgfpathmoveto{\pgfqpoint{3.109114in}{2.853159in}}%
\pgfpathlineto{\pgfqpoint{3.187325in}{3.418679in}}%
\pgfpathlineto{\pgfqpoint{3.044293in}{3.487491in}}%
\pgfpathclose%
\pgfusepath{fill}%
\end{pgfscope}%
\begin{pgfscope}%
\pgfpathrectangle{\pgfqpoint{0.539299in}{0.078740in}}{\pgfqpoint{7.842520in}{7.842520in}}%
\pgfusepath{clip}%
\pgfsetbuttcap%
\pgfsetroundjoin%
\definecolor{currentfill}{rgb}{0.855810,0.888601,0.097452}%
\pgfsetfillcolor{currentfill}%
\pgfsetlinewidth{0.000000pt}%
\definecolor{currentstroke}{rgb}{0.916242,0.896091,0.100717}%
\pgfsetstrokecolor{currentstroke}%
\pgfsetdash{}{0pt}%
\pgfpathmoveto{\pgfqpoint{3.030553in}{5.308756in}}%
\pgfpathlineto{\pgfqpoint{2.945073in}{5.109745in}}%
\pgfpathlineto{\pgfqpoint{3.087223in}{5.076965in}}%
\pgfpathclose%
\pgfusepath{fill}%
\end{pgfscope}%
\begin{pgfscope}%
\pgfpathrectangle{\pgfqpoint{0.539299in}{0.078740in}}{\pgfqpoint{7.842520in}{7.842520in}}%
\pgfusepath{clip}%
\pgfsetbuttcap%
\pgfsetroundjoin%
\definecolor{currentfill}{rgb}{0.147607,0.511733,0.557049}%
\pgfsetfillcolor{currentfill}%
\pgfsetlinewidth{0.000000pt}%
\definecolor{currentstroke}{rgb}{0.926106,0.897330,0.104071}%
\pgfsetstrokecolor{currentstroke}%
\pgfsetdash{}{0pt}%
\pgfpathmoveto{\pgfqpoint{3.109114in}{2.853159in}}%
\pgfpathlineto{\pgfqpoint{3.251833in}{2.786336in}}%
\pgfpathlineto{\pgfqpoint{3.330909in}{3.339983in}}%
\pgfpathclose%
\pgfusepath{fill}%
\end{pgfscope}%
\begin{pgfscope}%
\pgfpathrectangle{\pgfqpoint{0.539299in}{0.078740in}}{\pgfqpoint{7.842520in}{7.842520in}}%
\pgfusepath{clip}%
\pgfsetbuttcap%
\pgfsetroundjoin%
\definecolor{currentfill}{rgb}{0.935904,0.898570,0.108131}%
\pgfsetfillcolor{currentfill}%
\pgfsetlinewidth{0.000000pt}%
\definecolor{currentstroke}{rgb}{0.935904,0.898570,0.108131}%
\pgfsetstrokecolor{currentstroke}%
\pgfsetdash{}{0pt}%
\pgfpathmoveto{\pgfqpoint{3.172817in}{5.269283in}}%
\pgfpathlineto{\pgfqpoint{3.316139in}{5.195751in}}%
\pgfpathlineto{\pgfqpoint{3.259203in}{5.402997in}}%
\pgfpathclose%
\pgfusepath{fill}%
\end{pgfscope}%
\begin{pgfscope}%
\pgfpathrectangle{\pgfqpoint{0.539299in}{0.078740in}}{\pgfqpoint{7.842520in}{7.842520in}}%
\pgfusepath{clip}%
\pgfsetbuttcap%
\pgfsetroundjoin%
\definecolor{currentfill}{rgb}{0.185783,0.704891,0.485273}%
\pgfsetfillcolor{currentfill}%
\pgfsetlinewidth{0.000000pt}%
\definecolor{currentstroke}{rgb}{0.945636,0.899815,0.112838}%
\pgfsetstrokecolor{currentstroke}%
\pgfsetdash{}{0pt}%
\pgfpathmoveto{\pgfqpoint{2.901907in}{3.544823in}}%
\pgfpathlineto{\pgfqpoint{2.981110in}{4.034439in}}%
\pgfpathlineto{\pgfqpoint{2.838886in}{4.080766in}}%
\pgfpathclose%
\pgfusepath{fill}%
\end{pgfscope}%
\begin{pgfscope}%
\pgfpathrectangle{\pgfqpoint{0.539299in}{0.078740in}}{\pgfqpoint{7.842520in}{7.842520in}}%
\pgfusepath{clip}%
\pgfsetbuttcap%
\pgfsetroundjoin%
\definecolor{currentfill}{rgb}{0.896320,0.893616,0.096335}%
\pgfsetfillcolor{currentfill}%
\pgfsetlinewidth{0.000000pt}%
\definecolor{currentstroke}{rgb}{0.955300,0.901065,0.118128}%
\pgfsetstrokecolor{currentstroke}%
\pgfsetdash{}{0pt}%
\pgfpathmoveto{\pgfqpoint{3.087223in}{5.076965in}}%
\pgfpathlineto{\pgfqpoint{3.172817in}{5.269283in}}%
\pgfpathlineto{\pgfqpoint{3.030553in}{5.308756in}}%
\pgfpathclose%
\pgfusepath{fill}%
\end{pgfscope}%
\begin{pgfscope}%
\pgfpathrectangle{\pgfqpoint{0.539299in}{0.078740in}}{\pgfqpoint{7.842520in}{7.842520in}}%
\pgfusepath{clip}%
\pgfsetbuttcap%
\pgfsetroundjoin%
\definecolor{currentfill}{rgb}{0.282623,0.140926,0.457517}%
\pgfsetfillcolor{currentfill}%
\pgfsetlinewidth{0.000000pt}%
\definecolor{currentstroke}{rgb}{0.964894,0.902323,0.123941}%
\pgfsetstrokecolor{currentstroke}%
\pgfsetdash{}{0pt}%
\pgfpathmoveto{\pgfqpoint{5.760821in}{1.647418in}}%
\pgfpathlineto{\pgfqpoint{5.904099in}{1.379502in}}%
\pgfpathlineto{\pgfqpoint{5.983060in}{1.432854in}}%
\pgfpathclose%
\pgfusepath{fill}%
\end{pgfscope}%
\begin{pgfscope}%
\pgfpathrectangle{\pgfqpoint{0.539299in}{0.078740in}}{\pgfqpoint{7.842520in}{7.842520in}}%
\pgfusepath{clip}%
\pgfsetbuttcap%
\pgfsetroundjoin%
\definecolor{currentfill}{rgb}{0.269308,0.218818,0.509577}%
\pgfsetfillcolor{currentfill}%
\pgfsetlinewidth{0.000000pt}%
\definecolor{currentstroke}{rgb}{0.974417,0.903590,0.130215}%
\pgfsetstrokecolor{currentstroke}%
\pgfsetdash{}{0pt}%
\pgfpathmoveto{\pgfqpoint{5.760821in}{1.647418in}}%
\pgfpathlineto{\pgfqpoint{5.840111in}{1.685406in}}%
\pgfpathlineto{\pgfqpoint{5.696616in}{1.925940in}}%
\pgfpathclose%
\pgfusepath{fill}%
\end{pgfscope}%
\begin{pgfscope}%
\pgfpathrectangle{\pgfqpoint{0.539299in}{0.078740in}}{\pgfqpoint{7.842520in}{7.842520in}}%
\pgfusepath{clip}%
\pgfsetbuttcap%
\pgfsetroundjoin%
\definecolor{currentfill}{rgb}{0.140210,0.665859,0.513427}%
\pgfsetfillcolor{currentfill}%
\pgfsetlinewidth{0.000000pt}%
\definecolor{currentstroke}{rgb}{0.983868,0.904867,0.136897}%
\pgfsetstrokecolor{currentstroke}%
\pgfsetdash{}{0pt}%
\pgfpathmoveto{\pgfqpoint{3.044293in}{3.487491in}}%
\pgfpathlineto{\pgfqpoint{2.981110in}{4.034439in}}%
\pgfpathlineto{\pgfqpoint{2.901907in}{3.544823in}}%
\pgfpathclose%
\pgfusepath{fill}%
\end{pgfscope}%
\begin{pgfscope}%
\pgfpathrectangle{\pgfqpoint{0.539299in}{0.078740in}}{\pgfqpoint{7.842520in}{7.842520in}}%
\pgfusepath{clip}%
\pgfsetbuttcap%
\pgfsetroundjoin%
\definecolor{currentfill}{rgb}{0.150148,0.676631,0.506589}%
\pgfsetfillcolor{currentfill}%
\pgfsetlinewidth{0.000000pt}%
\definecolor{currentstroke}{rgb}{0.993248,0.906157,0.143936}%
\pgfsetstrokecolor{currentstroke}%
\pgfsetdash{}{0pt}%
\pgfpathmoveto{\pgfqpoint{4.476731in}{3.882966in}}%
\pgfpathlineto{\pgfqpoint{4.621833in}{3.675341in}}%
\pgfpathlineto{\pgfqpoint{4.705051in}{3.687048in}}%
\pgfpathclose%
\pgfusepath{fill}%
\end{pgfscope}%
\begin{pgfscope}%
\pgfpathrectangle{\pgfqpoint{0.539299in}{0.078740in}}{\pgfqpoint{7.842520in}{7.842520in}}%
\pgfusepath{clip}%
\pgfsetbuttcap%
\pgfsetroundjoin%
\definecolor{currentfill}{rgb}{0.906311,0.894855,0.098125}%
\pgfsetfillcolor{currentfill}%
\pgfsetlinewidth{0.000000pt}%
\definecolor{currentstroke}{rgb}{0.267004,0.004874,0.329415}%
\pgfsetstrokecolor{currentstroke}%
\pgfsetdash{}{0pt}%
\pgfpathmoveto{\pgfqpoint{3.316139in}{5.195751in}}%
\pgfpathlineto{\pgfqpoint{3.546541in}{5.204178in}}%
\pgfpathlineto{\pgfqpoint{3.402504in}{5.318957in}}%
\pgfpathclose%
\pgfusepath{fill}%
\end{pgfscope}%
\begin{pgfscope}%
\pgfpathrectangle{\pgfqpoint{0.539299in}{0.078740in}}{\pgfqpoint{7.842520in}{7.842520in}}%
\pgfusepath{clip}%
\pgfsetbuttcap%
\pgfsetroundjoin%
\definecolor{currentfill}{rgb}{0.395174,0.797475,0.367757}%
\pgfsetfillcolor{currentfill}%
\pgfsetlinewidth{0.000000pt}%
\definecolor{currentstroke}{rgb}{0.268510,0.009605,0.335427}%
\pgfsetstrokecolor{currentstroke}%
\pgfsetdash{}{0pt}%
\pgfpathmoveto{\pgfqpoint{2.838886in}{4.080766in}}%
\pgfpathlineto{\pgfqpoint{2.919862in}{4.489063in}}%
\pgfpathlineto{\pgfqpoint{2.778216in}{4.515166in}}%
\pgfpathclose%
\pgfusepath{fill}%
\end{pgfscope}%
\begin{pgfscope}%
\pgfpathrectangle{\pgfqpoint{0.539299in}{0.078740in}}{\pgfqpoint{7.842520in}{7.842520in}}%
\pgfusepath{clip}%
\pgfsetbuttcap%
\pgfsetroundjoin%
\definecolor{currentfill}{rgb}{0.335885,0.777018,0.402049}%
\pgfsetfillcolor{currentfill}%
\pgfsetlinewidth{0.000000pt}%
\definecolor{currentstroke}{rgb}{0.269944,0.014625,0.341379}%
\pgfsetstrokecolor{currentstroke}%
\pgfsetdash{}{0pt}%
\pgfpathmoveto{\pgfqpoint{4.415846in}{4.118860in}}%
\pgfpathlineto{\pgfqpoint{4.270980in}{4.328400in}}%
\pgfpathlineto{\pgfqpoint{4.186126in}{4.283787in}}%
\pgfpathclose%
\pgfusepath{fill}%
\end{pgfscope}%
\begin{pgfscope}%
\pgfpathrectangle{\pgfqpoint{0.539299in}{0.078740in}}{\pgfqpoint{7.842520in}{7.842520in}}%
\pgfusepath{clip}%
\pgfsetbuttcap%
\pgfsetroundjoin%
\definecolor{currentfill}{rgb}{0.535621,0.835785,0.281908}%
\pgfsetfillcolor{currentfill}%
\pgfsetlinewidth{0.000000pt}%
\definecolor{currentstroke}{rgb}{0.271305,0.019942,0.347269}%
\pgfsetstrokecolor{currentstroke}%
\pgfsetdash{}{0pt}%
\pgfpathmoveto{\pgfqpoint{2.778216in}{4.515166in}}%
\pgfpathlineto{\pgfqpoint{2.919862in}{4.489063in}}%
\pgfpathlineto{\pgfqpoint{2.860832in}{4.847363in}}%
\pgfpathclose%
\pgfusepath{fill}%
\end{pgfscope}%
\begin{pgfscope}%
\pgfpathrectangle{\pgfqpoint{0.539299in}{0.078740in}}{\pgfqpoint{7.842520in}{7.842520in}}%
\pgfusepath{clip}%
\pgfsetbuttcap%
\pgfsetroundjoin%
\definecolor{currentfill}{rgb}{0.772852,0.877868,0.131109}%
\pgfsetfillcolor{currentfill}%
\pgfsetlinewidth{0.000000pt}%
\definecolor{currentstroke}{rgb}{0.272594,0.025563,0.353093}%
\pgfsetstrokecolor{currentstroke}%
\pgfsetdash{}{0pt}%
\pgfpathmoveto{\pgfqpoint{2.945073in}{5.109745in}}%
\pgfpathlineto{\pgfqpoint{2.860832in}{4.847363in}}%
\pgfpathlineto{\pgfqpoint{3.087223in}{5.076965in}}%
\pgfpathclose%
\pgfusepath{fill}%
\end{pgfscope}%
\begin{pgfscope}%
\pgfpathrectangle{\pgfqpoint{0.539299in}{0.078740in}}{\pgfqpoint{7.842520in}{7.842520in}}%
\pgfusepath{clip}%
\pgfsetbuttcap%
\pgfsetroundjoin%
\definecolor{currentfill}{rgb}{0.187231,0.414746,0.556547}%
\pgfsetfillcolor{currentfill}%
\pgfsetlinewidth{0.000000pt}%
\definecolor{currentstroke}{rgb}{0.273809,0.031497,0.358853}%
\pgfsetstrokecolor{currentstroke}%
\pgfsetdash{}{0pt}%
\pgfpathmoveto{\pgfqpoint{5.263813in}{2.595433in}}%
\pgfpathlineto{\pgfqpoint{5.408389in}{2.379042in}}%
\pgfpathlineto{\pgfqpoint{5.344589in}{2.606000in}}%
\pgfpathclose%
\pgfusepath{fill}%
\end{pgfscope}%
\begin{pgfscope}%
\pgfpathrectangle{\pgfqpoint{0.539299in}{0.078740in}}{\pgfqpoint{7.842520in}{7.842520in}}%
\pgfusepath{clip}%
\pgfsetbuttcap%
\pgfsetroundjoin%
\definecolor{currentfill}{rgb}{0.149039,0.508051,0.557250}%
\pgfsetfillcolor{currentfill}%
\pgfsetlinewidth{0.000000pt}%
\definecolor{currentstroke}{rgb}{0.274952,0.037752,0.364543}%
\pgfsetstrokecolor{currentstroke}%
\pgfsetdash{}{0pt}%
\pgfpathmoveto{\pgfqpoint{5.200440in}{2.822813in}}%
\pgfpathlineto{\pgfqpoint{5.056080in}{3.038167in}}%
\pgfpathlineto{\pgfqpoint{4.973989in}{3.014923in}}%
\pgfpathclose%
\pgfusepath{fill}%
\end{pgfscope}%
\begin{pgfscope}%
\pgfpathrectangle{\pgfqpoint{0.539299in}{0.078740in}}{\pgfqpoint{7.842520in}{7.842520in}}%
\pgfusepath{clip}%
\pgfsetbuttcap%
\pgfsetroundjoin%
\definecolor{currentfill}{rgb}{0.229739,0.322361,0.545706}%
\pgfsetfillcolor{currentfill}%
\pgfsetlinewidth{0.000000pt}%
\definecolor{currentstroke}{rgb}{0.276022,0.044167,0.370164}%
\pgfsetstrokecolor{currentstroke}%
\pgfsetdash{}{0pt}%
\pgfpathmoveto{\pgfqpoint{3.682755in}{2.554388in}}%
\pgfpathlineto{\pgfqpoint{3.603070in}{1.953238in}}%
\pgfpathlineto{\pgfqpoint{3.746562in}{1.887616in}}%
\pgfpathclose%
\pgfusepath{fill}%
\end{pgfscope}%
\begin{pgfscope}%
\pgfpathrectangle{\pgfqpoint{0.539299in}{0.078740in}}{\pgfqpoint{7.842520in}{7.842520in}}%
\pgfusepath{clip}%
\pgfsetbuttcap%
\pgfsetroundjoin%
\definecolor{currentfill}{rgb}{0.199430,0.387607,0.554642}%
\pgfsetfillcolor{currentfill}%
\pgfsetlinewidth{0.000000pt}%
\definecolor{currentstroke}{rgb}{0.277018,0.050344,0.375715}%
\pgfsetstrokecolor{currentstroke}%
\pgfsetdash{}{0pt}%
\pgfpathmoveto{\pgfqpoint{3.682755in}{2.554388in}}%
\pgfpathlineto{\pgfqpoint{3.538693in}{2.636201in}}%
\pgfpathlineto{\pgfqpoint{3.460014in}{2.016906in}}%
\pgfpathclose%
\pgfusepath{fill}%
\end{pgfscope}%
\begin{pgfscope}%
\pgfpathrectangle{\pgfqpoint{0.539299in}{0.078740in}}{\pgfqpoint{7.842520in}{7.842520in}}%
\pgfusepath{clip}%
\pgfsetbuttcap%
\pgfsetroundjoin%
\definecolor{currentfill}{rgb}{0.128729,0.563265,0.551229}%
\pgfsetfillcolor{currentfill}%
\pgfsetlinewidth{0.000000pt}%
\definecolor{currentstroke}{rgb}{0.277941,0.056324,0.381191}%
\pgfsetstrokecolor{currentstroke}%
\pgfsetdash{}{0pt}%
\pgfpathmoveto{\pgfqpoint{3.330909in}{3.339983in}}%
\pgfpathlineto{\pgfqpoint{3.187325in}{3.418679in}}%
\pgfpathlineto{\pgfqpoint{3.109114in}{2.853159in}}%
\pgfpathclose%
\pgfusepath{fill}%
\end{pgfscope}%
\begin{pgfscope}%
\pgfpathrectangle{\pgfqpoint{0.539299in}{0.078740in}}{\pgfqpoint{7.842520in}{7.842520in}}%
\pgfusepath{clip}%
\pgfsetbuttcap%
\pgfsetroundjoin%
\definecolor{currentfill}{rgb}{0.440137,0.811138,0.340967}%
\pgfsetfillcolor{currentfill}%
\pgfsetlinewidth{0.000000pt}%
\definecolor{currentstroke}{rgb}{0.278791,0.062145,0.386592}%
\pgfsetstrokecolor{currentstroke}%
\pgfsetdash{}{0pt}%
\pgfpathmoveto{\pgfqpoint{4.270980in}{4.328400in}}%
\pgfpathlineto{\pgfqpoint{4.125992in}{4.530691in}}%
\pgfpathlineto{\pgfqpoint{4.040710in}{4.473147in}}%
\pgfpathclose%
\pgfusepath{fill}%
\end{pgfscope}%
\begin{pgfscope}%
\pgfpathrectangle{\pgfqpoint{0.539299in}{0.078740in}}{\pgfqpoint{7.842520in}{7.842520in}}%
\pgfusepath{clip}%
\pgfsetbuttcap%
\pgfsetroundjoin%
\definecolor{currentfill}{rgb}{0.149039,0.508051,0.557250}%
\pgfsetfillcolor{currentfill}%
\pgfsetlinewidth{0.000000pt}%
\definecolor{currentstroke}{rgb}{0.279566,0.067836,0.391917}%
\pgfsetstrokecolor{currentstroke}%
\pgfsetdash{}{0pt}%
\pgfpathmoveto{\pgfqpoint{3.251833in}{2.786336in}}%
\pgfpathlineto{\pgfqpoint{3.395040in}{2.713747in}}%
\pgfpathlineto{\pgfqpoint{3.330909in}{3.339983in}}%
\pgfpathclose%
\pgfusepath{fill}%
\end{pgfscope}%
\begin{pgfscope}%
\pgfpathrectangle{\pgfqpoint{0.539299in}{0.078740in}}{\pgfqpoint{7.842520in}{7.842520in}}%
\pgfusepath{clip}%
\pgfsetbuttcap%
\pgfsetroundjoin%
\definecolor{currentfill}{rgb}{0.172719,0.448791,0.557885}%
\pgfsetfillcolor{currentfill}%
\pgfsetlinewidth{0.000000pt}%
\definecolor{currentstroke}{rgb}{0.280267,0.073417,0.397163}%
\pgfsetstrokecolor{currentstroke}%
\pgfsetdash{}{0pt}%
\pgfpathmoveto{\pgfqpoint{5.344589in}{2.606000in}}%
\pgfpathlineto{\pgfqpoint{5.119001in}{2.807062in}}%
\pgfpathlineto{\pgfqpoint{5.263813in}{2.595433in}}%
\pgfpathclose%
\pgfusepath{fill}%
\end{pgfscope}%
\begin{pgfscope}%
\pgfpathrectangle{\pgfqpoint{0.539299in}{0.078740in}}{\pgfqpoint{7.842520in}{7.842520in}}%
\pgfusepath{clip}%
\pgfsetbuttcap%
\pgfsetroundjoin%
\definecolor{currentfill}{rgb}{0.246070,0.738910,0.452024}%
\pgfsetfillcolor{currentfill}%
\pgfsetlinewidth{0.000000pt}%
\definecolor{currentstroke}{rgb}{0.280894,0.078907,0.402329}%
\pgfsetstrokecolor{currentstroke}%
\pgfsetdash{}{0pt}%
\pgfpathmoveto{\pgfqpoint{4.415846in}{4.118860in}}%
\pgfpathlineto{\pgfqpoint{4.331483in}{4.086352in}}%
\pgfpathlineto{\pgfqpoint{4.476731in}{3.882966in}}%
\pgfpathclose%
\pgfusepath{fill}%
\end{pgfscope}%
\begin{pgfscope}%
\pgfpathrectangle{\pgfqpoint{0.539299in}{0.078740in}}{\pgfqpoint{7.842520in}{7.842520in}}%
\pgfusepath{clip}%
\pgfsetbuttcap%
\pgfsetroundjoin%
\definecolor{currentfill}{rgb}{0.824940,0.884720,0.106217}%
\pgfsetfillcolor{currentfill}%
\pgfsetlinewidth{0.000000pt}%
\definecolor{currentstroke}{rgb}{0.281446,0.084320,0.407414}%
\pgfsetstrokecolor{currentstroke}%
\pgfsetdash{}{0pt}%
\pgfpathmoveto{\pgfqpoint{3.546541in}{5.204178in}}%
\pgfpathlineto{\pgfqpoint{3.604941in}{4.965350in}}%
\pgfpathlineto{\pgfqpoint{3.691083in}{5.063599in}}%
\pgfpathclose%
\pgfusepath{fill}%
\end{pgfscope}%
\begin{pgfscope}%
\pgfpathrectangle{\pgfqpoint{0.539299in}{0.078740in}}{\pgfqpoint{7.842520in}{7.842520in}}%
\pgfusepath{clip}%
\pgfsetbuttcap%
\pgfsetroundjoin%
\definecolor{currentfill}{rgb}{0.515992,0.831158,0.294279}%
\pgfsetfillcolor{currentfill}%
\pgfsetlinewidth{0.000000pt}%
\definecolor{currentstroke}{rgb}{0.281924,0.089666,0.412415}%
\pgfsetstrokecolor{currentstroke}%
\pgfsetdash{}{0pt}%
\pgfpathmoveto{\pgfqpoint{4.040710in}{4.473147in}}%
\pgfpathlineto{\pgfqpoint{4.125992in}{4.530691in}}%
\pgfpathlineto{\pgfqpoint{3.980947in}{4.722912in}}%
\pgfpathclose%
\pgfusepath{fill}%
\end{pgfscope}%
\begin{pgfscope}%
\pgfpathrectangle{\pgfqpoint{0.539299in}{0.078740in}}{\pgfqpoint{7.842520in}{7.842520in}}%
\pgfusepath{clip}%
\pgfsetbuttcap%
\pgfsetroundjoin%
\definecolor{currentfill}{rgb}{0.129933,0.559582,0.551864}%
\pgfsetfillcolor{currentfill}%
\pgfsetlinewidth{0.000000pt}%
\definecolor{currentstroke}{rgb}{0.282327,0.094955,0.417331}%
\pgfsetstrokecolor{currentstroke}%
\pgfsetdash{}{0pt}%
\pgfpathmoveto{\pgfqpoint{4.828804in}{3.219577in}}%
\pgfpathlineto{\pgfqpoint{5.056080in}{3.038167in}}%
\pgfpathlineto{\pgfqpoint{4.911519in}{3.252217in}}%
\pgfpathclose%
\pgfusepath{fill}%
\end{pgfscope}%
\begin{pgfscope}%
\pgfpathrectangle{\pgfqpoint{0.539299in}{0.078740in}}{\pgfqpoint{7.842520in}{7.842520in}}%
\pgfusepath{clip}%
\pgfsetbuttcap%
\pgfsetroundjoin%
\definecolor{currentfill}{rgb}{0.886271,0.892374,0.095374}%
\pgfsetfillcolor{currentfill}%
\pgfsetlinewidth{0.000000pt}%
\definecolor{currentstroke}{rgb}{0.282656,0.100196,0.422160}%
\pgfsetstrokecolor{currentstroke}%
\pgfsetdash{}{0pt}%
\pgfpathmoveto{\pgfqpoint{3.087223in}{5.076965in}}%
\pgfpathlineto{\pgfqpoint{3.316139in}{5.195751in}}%
\pgfpathlineto{\pgfqpoint{3.172817in}{5.269283in}}%
\pgfpathclose%
\pgfusepath{fill}%
\end{pgfscope}%
\begin{pgfscope}%
\pgfpathrectangle{\pgfqpoint{0.539299in}{0.078740in}}{\pgfqpoint{7.842520in}{7.842520in}}%
\pgfusepath{clip}%
\pgfsetbuttcap%
\pgfsetroundjoin%
\definecolor{currentfill}{rgb}{0.606045,0.850733,0.236712}%
\pgfsetfillcolor{currentfill}%
\pgfsetlinewidth{0.000000pt}%
\definecolor{currentstroke}{rgb}{0.282910,0.105393,0.426902}%
\pgfsetstrokecolor{currentstroke}%
\pgfsetdash{}{0pt}%
\pgfpathmoveto{\pgfqpoint{2.919862in}{4.489063in}}%
\pgfpathlineto{\pgfqpoint{3.002777in}{4.819100in}}%
\pgfpathlineto{\pgfqpoint{2.860832in}{4.847363in}}%
\pgfpathclose%
\pgfusepath{fill}%
\end{pgfscope}%
\begin{pgfscope}%
\pgfpathrectangle{\pgfqpoint{0.539299in}{0.078740in}}{\pgfqpoint{7.842520in}{7.842520in}}%
\pgfusepath{clip}%
\pgfsetbuttcap%
\pgfsetroundjoin%
\definecolor{currentfill}{rgb}{0.762373,0.876424,0.137064}%
\pgfsetfillcolor{currentfill}%
\pgfsetlinewidth{0.000000pt}%
\definecolor{currentstroke}{rgb}{0.283091,0.110553,0.431554}%
\pgfsetstrokecolor{currentstroke}%
\pgfsetdash{}{0pt}%
\pgfpathmoveto{\pgfqpoint{3.604941in}{4.965350in}}%
\pgfpathlineto{\pgfqpoint{3.835937in}{4.901790in}}%
\pgfpathlineto{\pgfqpoint{3.691083in}{5.063599in}}%
\pgfpathclose%
\pgfusepath{fill}%
\end{pgfscope}%
\begin{pgfscope}%
\pgfpathrectangle{\pgfqpoint{0.539299in}{0.078740in}}{\pgfqpoint{7.842520in}{7.842520in}}%
\pgfusepath{clip}%
\pgfsetbuttcap%
\pgfsetroundjoin%
\definecolor{currentfill}{rgb}{0.720391,0.870350,0.162603}%
\pgfsetfillcolor{currentfill}%
\pgfsetlinewidth{0.000000pt}%
\definecolor{currentstroke}{rgb}{0.283197,0.115680,0.436115}%
\pgfsetstrokecolor{currentstroke}%
\pgfsetdash{}{0pt}%
\pgfpathmoveto{\pgfqpoint{3.087223in}{5.076965in}}%
\pgfpathlineto{\pgfqpoint{2.860832in}{4.847363in}}%
\pgfpathlineto{\pgfqpoint{3.002777in}{4.819100in}}%
\pgfpathclose%
\pgfusepath{fill}%
\end{pgfscope}%
\begin{pgfscope}%
\pgfpathrectangle{\pgfqpoint{0.539299in}{0.078740in}}{\pgfqpoint{7.842520in}{7.842520in}}%
\pgfusepath{clip}%
\pgfsetbuttcap%
\pgfsetroundjoin%
\definecolor{currentfill}{rgb}{0.668054,0.861999,0.196293}%
\pgfsetfillcolor{currentfill}%
\pgfsetlinewidth{0.000000pt}%
\definecolor{currentstroke}{rgb}{0.283229,0.120777,0.440584}%
\pgfsetstrokecolor{currentstroke}%
\pgfsetdash{}{0pt}%
\pgfpathmoveto{\pgfqpoint{3.980947in}{4.722912in}}%
\pgfpathlineto{\pgfqpoint{3.835937in}{4.901790in}}%
\pgfpathlineto{\pgfqpoint{3.750009in}{4.817083in}}%
\pgfpathclose%
\pgfusepath{fill}%
\end{pgfscope}%
\begin{pgfscope}%
\pgfpathrectangle{\pgfqpoint{0.539299in}{0.078740in}}{\pgfqpoint{7.842520in}{7.842520in}}%
\pgfusepath{clip}%
\pgfsetbuttcap%
\pgfsetroundjoin%
\definecolor{currentfill}{rgb}{0.876168,0.891125,0.095250}%
\pgfsetfillcolor{currentfill}%
\pgfsetlinewidth{0.000000pt}%
\definecolor{currentstroke}{rgb}{0.283187,0.125848,0.444960}%
\pgfsetstrokecolor{currentstroke}%
\pgfsetdash{}{0pt}%
\pgfpathmoveto{\pgfqpoint{3.460255in}{5.092948in}}%
\pgfpathlineto{\pgfqpoint{3.546541in}{5.204178in}}%
\pgfpathlineto{\pgfqpoint{3.316139in}{5.195751in}}%
\pgfpathclose%
\pgfusepath{fill}%
\end{pgfscope}%
\begin{pgfscope}%
\pgfpathrectangle{\pgfqpoint{0.539299in}{0.078740in}}{\pgfqpoint{7.842520in}{7.842520in}}%
\pgfusepath{clip}%
\pgfsetbuttcap%
\pgfsetroundjoin%
\definecolor{currentfill}{rgb}{0.304148,0.764704,0.419943}%
\pgfsetfillcolor{currentfill}%
\pgfsetlinewidth{0.000000pt}%
\definecolor{currentstroke}{rgb}{0.283072,0.130895,0.449241}%
\pgfsetstrokecolor{currentstroke}%
\pgfsetdash{}{0pt}%
\pgfpathmoveto{\pgfqpoint{4.186126in}{4.283787in}}%
\pgfpathlineto{\pgfqpoint{4.331483in}{4.086352in}}%
\pgfpathlineto{\pgfqpoint{4.415846in}{4.118860in}}%
\pgfpathclose%
\pgfusepath{fill}%
\end{pgfscope}%
\begin{pgfscope}%
\pgfpathrectangle{\pgfqpoint{0.539299in}{0.078740in}}{\pgfqpoint{7.842520in}{7.842520in}}%
\pgfusepath{clip}%
\pgfsetbuttcap%
\pgfsetroundjoin%
\definecolor{currentfill}{rgb}{0.395174,0.797475,0.367757}%
\pgfsetfillcolor{currentfill}%
\pgfsetlinewidth{0.000000pt}%
\definecolor{currentstroke}{rgb}{0.282884,0.135920,0.453427}%
\pgfsetstrokecolor{currentstroke}%
\pgfsetdash{}{0pt}%
\pgfpathmoveto{\pgfqpoint{2.838886in}{4.080766in}}%
\pgfpathlineto{\pgfqpoint{3.062545in}{4.438625in}}%
\pgfpathlineto{\pgfqpoint{2.919862in}{4.489063in}}%
\pgfpathclose%
\pgfusepath{fill}%
\end{pgfscope}%
\begin{pgfscope}%
\pgfpathrectangle{\pgfqpoint{0.539299in}{0.078740in}}{\pgfqpoint{7.842520in}{7.842520in}}%
\pgfusepath{clip}%
\pgfsetbuttcap%
\pgfsetroundjoin%
\definecolor{currentfill}{rgb}{0.235526,0.309527,0.542944}%
\pgfsetfillcolor{currentfill}%
\pgfsetlinewidth{0.000000pt}%
\definecolor{currentstroke}{rgb}{0.282623,0.140926,0.457517}%
\pgfsetstrokecolor{currentstroke}%
\pgfsetdash{}{0pt}%
\pgfpathmoveto{\pgfqpoint{5.472274in}{2.124924in}}%
\pgfpathlineto{\pgfqpoint{5.696616in}{1.925940in}}%
\pgfpathlineto{\pgfqpoint{5.552680in}{2.156474in}}%
\pgfpathclose%
\pgfusepath{fill}%
\end{pgfscope}%
\begin{pgfscope}%
\pgfpathrectangle{\pgfqpoint{0.539299in}{0.078740in}}{\pgfqpoint{7.842520in}{7.842520in}}%
\pgfusepath{clip}%
\pgfsetbuttcap%
\pgfsetroundjoin%
\definecolor{currentfill}{rgb}{0.319809,0.770914,0.411152}%
\pgfsetfillcolor{currentfill}%
\pgfsetlinewidth{0.000000pt}%
\definecolor{currentstroke}{rgb}{0.282290,0.145912,0.461510}%
\pgfsetstrokecolor{currentstroke}%
\pgfsetdash{}{0pt}%
\pgfpathmoveto{\pgfqpoint{2.981110in}{4.034439in}}%
\pgfpathlineto{\pgfqpoint{3.062545in}{4.438625in}}%
\pgfpathlineto{\pgfqpoint{2.838886in}{4.080766in}}%
\pgfpathclose%
\pgfusepath{fill}%
\end{pgfscope}%
\begin{pgfscope}%
\pgfpathrectangle{\pgfqpoint{0.539299in}{0.078740in}}{\pgfqpoint{7.842520in}{7.842520in}}%
\pgfusepath{clip}%
\pgfsetbuttcap%
\pgfsetroundjoin%
\definecolor{currentfill}{rgb}{0.153364,0.497000,0.557724}%
\pgfsetfillcolor{currentfill}%
\pgfsetlinewidth{0.000000pt}%
\definecolor{currentstroke}{rgb}{0.281887,0.150881,0.465405}%
\pgfsetstrokecolor{currentstroke}%
\pgfsetdash{}{0pt}%
\pgfpathmoveto{\pgfqpoint{4.973989in}{3.014923in}}%
\pgfpathlineto{\pgfqpoint{5.119001in}{2.807062in}}%
\pgfpathlineto{\pgfqpoint{5.200440in}{2.822813in}}%
\pgfpathclose%
\pgfusepath{fill}%
\end{pgfscope}%
\begin{pgfscope}%
\pgfpathrectangle{\pgfqpoint{0.539299in}{0.078740in}}{\pgfqpoint{7.842520in}{7.842520in}}%
\pgfusepath{clip}%
\pgfsetbuttcap%
\pgfsetroundjoin%
\definecolor{currentfill}{rgb}{0.180653,0.701402,0.488189}%
\pgfsetfillcolor{currentfill}%
\pgfsetlinewidth{0.000000pt}%
\definecolor{currentstroke}{rgb}{0.281412,0.155834,0.469201}%
\pgfsetstrokecolor{currentstroke}%
\pgfsetdash{}{0pt}%
\pgfpathmoveto{\pgfqpoint{3.044293in}{3.487491in}}%
\pgfpathlineto{\pgfqpoint{3.124144in}{3.970746in}}%
\pgfpathlineto{\pgfqpoint{2.981110in}{4.034439in}}%
\pgfpathclose%
\pgfusepath{fill}%
\end{pgfscope}%
\begin{pgfscope}%
\pgfpathrectangle{\pgfqpoint{0.539299in}{0.078740in}}{\pgfqpoint{7.842520in}{7.842520in}}%
\pgfusepath{clip}%
\pgfsetbuttcap%
\pgfsetroundjoin%
\definecolor{currentfill}{rgb}{0.119512,0.607464,0.540218}%
\pgfsetfillcolor{currentfill}%
\pgfsetlinewidth{0.000000pt}%
\definecolor{currentstroke}{rgb}{0.280868,0.160771,0.472899}%
\pgfsetstrokecolor{currentstroke}%
\pgfsetdash{}{0pt}%
\pgfpathmoveto{\pgfqpoint{4.683466in}{3.421156in}}%
\pgfpathlineto{\pgfqpoint{4.911519in}{3.252217in}}%
\pgfpathlineto{\pgfqpoint{4.766767in}{3.464787in}}%
\pgfpathclose%
\pgfusepath{fill}%
\end{pgfscope}%
\begin{pgfscope}%
\pgfpathrectangle{\pgfqpoint{0.539299in}{0.078740in}}{\pgfqpoint{7.842520in}{7.842520in}}%
\pgfusepath{clip}%
\pgfsetbuttcap%
\pgfsetroundjoin%
\definecolor{currentfill}{rgb}{0.260571,0.246922,0.522828}%
\pgfsetfillcolor{currentfill}%
\pgfsetlinewidth{0.000000pt}%
\definecolor{currentstroke}{rgb}{0.280255,0.165693,0.476498}%
\pgfsetstrokecolor{currentstroke}%
\pgfsetdash{}{0pt}%
\pgfpathmoveto{\pgfqpoint{5.696616in}{1.925940in}}%
\pgfpathlineto{\pgfqpoint{5.616816in}{1.894378in}}%
\pgfpathlineto{\pgfqpoint{5.760821in}{1.647418in}}%
\pgfpathclose%
\pgfusepath{fill}%
\end{pgfscope}%
\begin{pgfscope}%
\pgfpathrectangle{\pgfqpoint{0.539299in}{0.078740in}}{\pgfqpoint{7.842520in}{7.842520in}}%
\pgfusepath{clip}%
\pgfsetbuttcap%
\pgfsetroundjoin%
\definecolor{currentfill}{rgb}{0.835270,0.886029,0.102646}%
\pgfsetfillcolor{currentfill}%
\pgfsetlinewidth{0.000000pt}%
\definecolor{currentstroke}{rgb}{0.279574,0.170599,0.479997}%
\pgfsetstrokecolor{currentstroke}%
\pgfsetdash{}{0pt}%
\pgfpathmoveto{\pgfqpoint{3.460255in}{5.092948in}}%
\pgfpathlineto{\pgfqpoint{3.604941in}{4.965350in}}%
\pgfpathlineto{\pgfqpoint{3.546541in}{5.204178in}}%
\pgfpathclose%
\pgfusepath{fill}%
\end{pgfscope}%
\begin{pgfscope}%
\pgfpathrectangle{\pgfqpoint{0.539299in}{0.078740in}}{\pgfqpoint{7.842520in}{7.842520in}}%
\pgfusepath{clip}%
\pgfsetbuttcap%
\pgfsetroundjoin%
\definecolor{currentfill}{rgb}{0.412913,0.803041,0.357269}%
\pgfsetfillcolor{currentfill}%
\pgfsetlinewidth{0.000000pt}%
\definecolor{currentstroke}{rgb}{0.278826,0.175490,0.483397}%
\pgfsetstrokecolor{currentstroke}%
\pgfsetdash{}{0pt}%
\pgfpathmoveto{\pgfqpoint{4.040710in}{4.473147in}}%
\pgfpathlineto{\pgfqpoint{4.186126in}{4.283787in}}%
\pgfpathlineto{\pgfqpoint{4.270980in}{4.328400in}}%
\pgfpathclose%
\pgfusepath{fill}%
\end{pgfscope}%
\begin{pgfscope}%
\pgfpathrectangle{\pgfqpoint{0.539299in}{0.078740in}}{\pgfqpoint{7.842520in}{7.842520in}}%
\pgfusepath{clip}%
\pgfsetbuttcap%
\pgfsetroundjoin%
\definecolor{currentfill}{rgb}{0.845561,0.887322,0.099702}%
\pgfsetfillcolor{currentfill}%
\pgfsetlinewidth{0.000000pt}%
\definecolor{currentstroke}{rgb}{0.278012,0.180367,0.486697}%
\pgfsetstrokecolor{currentstroke}%
\pgfsetdash{}{0pt}%
\pgfpathmoveto{\pgfqpoint{3.230452in}{5.012748in}}%
\pgfpathlineto{\pgfqpoint{3.316139in}{5.195751in}}%
\pgfpathlineto{\pgfqpoint{3.087223in}{5.076965in}}%
\pgfpathclose%
\pgfusepath{fill}%
\end{pgfscope}%
\begin{pgfscope}%
\pgfpathrectangle{\pgfqpoint{0.539299in}{0.078740in}}{\pgfqpoint{7.842520in}{7.842520in}}%
\pgfusepath{clip}%
\pgfsetbuttcap%
\pgfsetroundjoin%
\definecolor{currentfill}{rgb}{0.216210,0.351535,0.550627}%
\pgfsetfillcolor{currentfill}%
\pgfsetlinewidth{0.000000pt}%
\definecolor{currentstroke}{rgb}{0.277134,0.185228,0.489898}%
\pgfsetstrokecolor{currentstroke}%
\pgfsetdash{}{0pt}%
\pgfpathmoveto{\pgfqpoint{5.552680in}{2.156474in}}%
\pgfpathlineto{\pgfqpoint{5.408389in}{2.379042in}}%
\pgfpathlineto{\pgfqpoint{5.472274in}{2.124924in}}%
\pgfpathclose%
\pgfusepath{fill}%
\end{pgfscope}%
\begin{pgfscope}%
\pgfpathrectangle{\pgfqpoint{0.539299in}{0.078740in}}{\pgfqpoint{7.842520in}{7.842520in}}%
\pgfusepath{clip}%
\pgfsetbuttcap%
\pgfsetroundjoin%
\definecolor{currentfill}{rgb}{0.132268,0.655014,0.519661}%
\pgfsetfillcolor{currentfill}%
\pgfsetlinewidth{0.000000pt}%
\definecolor{currentstroke}{rgb}{0.276194,0.190074,0.493001}%
\pgfsetstrokecolor{currentstroke}%
\pgfsetdash{}{0pt}%
\pgfpathmoveto{\pgfqpoint{3.267846in}{3.892059in}}%
\pgfpathlineto{\pgfqpoint{3.044293in}{3.487491in}}%
\pgfpathlineto{\pgfqpoint{3.187325in}{3.418679in}}%
\pgfpathclose%
\pgfusepath{fill}%
\end{pgfscope}%
\begin{pgfscope}%
\pgfpathrectangle{\pgfqpoint{0.539299in}{0.078740in}}{\pgfqpoint{7.842520in}{7.842520in}}%
\pgfusepath{clip}%
\pgfsetbuttcap%
\pgfsetroundjoin%
\definecolor{currentfill}{rgb}{0.135066,0.544853,0.554029}%
\pgfsetfillcolor{currentfill}%
\pgfsetlinewidth{0.000000pt}%
\definecolor{currentstroke}{rgb}{0.275191,0.194905,0.496005}%
\pgfsetstrokecolor{currentstroke}%
\pgfsetdash{}{0pt}%
\pgfpathmoveto{\pgfqpoint{4.973989in}{3.014923in}}%
\pgfpathlineto{\pgfqpoint{5.056080in}{3.038167in}}%
\pgfpathlineto{\pgfqpoint{4.828804in}{3.219577in}}%
\pgfpathclose%
\pgfusepath{fill}%
\end{pgfscope}%
\begin{pgfscope}%
\pgfpathrectangle{\pgfqpoint{0.539299in}{0.078740in}}{\pgfqpoint{7.842520in}{7.842520in}}%
\pgfusepath{clip}%
\pgfsetbuttcap%
\pgfsetroundjoin%
\definecolor{currentfill}{rgb}{0.535621,0.835785,0.281908}%
\pgfsetfillcolor{currentfill}%
\pgfsetlinewidth{0.000000pt}%
\definecolor{currentstroke}{rgb}{0.274128,0.199721,0.498911}%
\pgfsetstrokecolor{currentstroke}%
\pgfsetdash{}{0pt}%
\pgfpathmoveto{\pgfqpoint{2.919862in}{4.489063in}}%
\pgfpathlineto{\pgfqpoint{3.062545in}{4.438625in}}%
\pgfpathlineto{\pgfqpoint{3.002777in}{4.819100in}}%
\pgfpathclose%
\pgfusepath{fill}%
\end{pgfscope}%
\begin{pgfscope}%
\pgfpathrectangle{\pgfqpoint{0.539299in}{0.078740in}}{\pgfqpoint{7.842520in}{7.842520in}}%
\pgfusepath{clip}%
\pgfsetbuttcap%
\pgfsetroundjoin%
\definecolor{currentfill}{rgb}{0.730889,0.871916,0.156029}%
\pgfsetfillcolor{currentfill}%
\pgfsetlinewidth{0.000000pt}%
\definecolor{currentstroke}{rgb}{0.273006,0.204520,0.501721}%
\pgfsetstrokecolor{currentstroke}%
\pgfsetdash{}{0pt}%
\pgfpathmoveto{\pgfqpoint{3.750009in}{4.817083in}}%
\pgfpathlineto{\pgfqpoint{3.835937in}{4.901790in}}%
\pgfpathlineto{\pgfqpoint{3.604941in}{4.965350in}}%
\pgfpathclose%
\pgfusepath{fill}%
\end{pgfscope}%
\begin{pgfscope}%
\pgfpathrectangle{\pgfqpoint{0.539299in}{0.078740in}}{\pgfqpoint{7.842520in}{7.842520in}}%
\pgfusepath{clip}%
\pgfsetbuttcap%
\pgfsetroundjoin%
\definecolor{currentfill}{rgb}{0.555484,0.840254,0.269281}%
\pgfsetfillcolor{currentfill}%
\pgfsetlinewidth{0.000000pt}%
\definecolor{currentstroke}{rgb}{0.271828,0.209303,0.504434}%
\pgfsetstrokecolor{currentstroke}%
\pgfsetdash{}{0pt}%
\pgfpathmoveto{\pgfqpoint{3.895306in}{4.651896in}}%
\pgfpathlineto{\pgfqpoint{4.040710in}{4.473147in}}%
\pgfpathlineto{\pgfqpoint{3.980947in}{4.722912in}}%
\pgfpathclose%
\pgfusepath{fill}%
\end{pgfscope}%
\begin{pgfscope}%
\pgfpathrectangle{\pgfqpoint{0.539299in}{0.078740in}}{\pgfqpoint{7.842520in}{7.842520in}}%
\pgfusepath{clip}%
\pgfsetbuttcap%
\pgfsetroundjoin%
\definecolor{currentfill}{rgb}{0.132268,0.655014,0.519661}%
\pgfsetfillcolor{currentfill}%
\pgfsetlinewidth{0.000000pt}%
\definecolor{currentstroke}{rgb}{0.270595,0.214069,0.507052}%
\pgfsetstrokecolor{currentstroke}%
\pgfsetdash{}{0pt}%
\pgfpathmoveto{\pgfqpoint{4.766767in}{3.464787in}}%
\pgfpathlineto{\pgfqpoint{4.621833in}{3.675341in}}%
\pgfpathlineto{\pgfqpoint{4.537995in}{3.619381in}}%
\pgfpathclose%
\pgfusepath{fill}%
\end{pgfscope}%
\begin{pgfscope}%
\pgfpathrectangle{\pgfqpoint{0.539299in}{0.078740in}}{\pgfqpoint{7.842520in}{7.842520in}}%
\pgfusepath{clip}%
\pgfsetbuttcap%
\pgfsetroundjoin%
\definecolor{currentfill}{rgb}{0.626579,0.854645,0.223353}%
\pgfsetfillcolor{currentfill}%
\pgfsetlinewidth{0.000000pt}%
\definecolor{currentstroke}{rgb}{0.269308,0.218818,0.509577}%
\pgfsetstrokecolor{currentstroke}%
\pgfsetdash{}{0pt}%
\pgfpathmoveto{\pgfqpoint{3.980947in}{4.722912in}}%
\pgfpathlineto{\pgfqpoint{3.750009in}{4.817083in}}%
\pgfpathlineto{\pgfqpoint{3.895306in}{4.651896in}}%
\pgfpathclose%
\pgfusepath{fill}%
\end{pgfscope}%
\begin{pgfscope}%
\pgfpathrectangle{\pgfqpoint{0.539299in}{0.078740in}}{\pgfqpoint{7.842520in}{7.842520in}}%
\pgfusepath{clip}%
\pgfsetbuttcap%
\pgfsetroundjoin%
\definecolor{currentfill}{rgb}{0.237441,0.305202,0.541921}%
\pgfsetfillcolor{currentfill}%
\pgfsetlinewidth{0.000000pt}%
\definecolor{currentstroke}{rgb}{0.267968,0.223549,0.512008}%
\pgfsetstrokecolor{currentstroke}%
\pgfsetdash{}{0pt}%
\pgfpathmoveto{\pgfqpoint{3.746562in}{1.887616in}}%
\pgfpathlineto{\pgfqpoint{3.890478in}{1.820244in}}%
\pgfpathlineto{\pgfqpoint{3.971998in}{2.380078in}}%
\pgfpathclose%
\pgfusepath{fill}%
\end{pgfscope}%
\begin{pgfscope}%
\pgfpathrectangle{\pgfqpoint{0.539299in}{0.078740in}}{\pgfqpoint{7.842520in}{7.842520in}}%
\pgfusepath{clip}%
\pgfsetbuttcap%
\pgfsetroundjoin%
\definecolor{currentfill}{rgb}{0.282290,0.145912,0.461510}%
\pgfsetfillcolor{currentfill}%
\pgfsetlinewidth{0.000000pt}%
\definecolor{currentstroke}{rgb}{0.266580,0.228262,0.514349}%
\pgfsetstrokecolor{currentstroke}%
\pgfsetdash{}{0pt}%
\pgfpathmoveto{\pgfqpoint{5.824454in}{1.317143in}}%
\pgfpathlineto{\pgfqpoint{5.904099in}{1.379502in}}%
\pgfpathlineto{\pgfqpoint{5.760821in}{1.647418in}}%
\pgfpathclose%
\pgfusepath{fill}%
\end{pgfscope}%
\begin{pgfscope}%
\pgfpathrectangle{\pgfqpoint{0.539299in}{0.078740in}}{\pgfqpoint{7.842520in}{7.842520in}}%
\pgfusepath{clip}%
\pgfsetbuttcap%
\pgfsetroundjoin%
\definecolor{currentfill}{rgb}{0.720391,0.870350,0.162603}%
\pgfsetfillcolor{currentfill}%
\pgfsetlinewidth{0.000000pt}%
\definecolor{currentstroke}{rgb}{0.265145,0.232956,0.516599}%
\pgfsetstrokecolor{currentstroke}%
\pgfsetdash{}{0pt}%
\pgfpathmoveto{\pgfqpoint{3.087223in}{5.076965in}}%
\pgfpathlineto{\pgfqpoint{3.002777in}{4.819100in}}%
\pgfpathlineto{\pgfqpoint{3.145794in}{4.762662in}}%
\pgfpathclose%
\pgfusepath{fill}%
\end{pgfscope}%
\begin{pgfscope}%
\pgfpathrectangle{\pgfqpoint{0.539299in}{0.078740in}}{\pgfqpoint{7.842520in}{7.842520in}}%
\pgfusepath{clip}%
\pgfsetbuttcap%
\pgfsetroundjoin%
\definecolor{currentfill}{rgb}{0.241237,0.296485,0.539709}%
\pgfsetfillcolor{currentfill}%
\pgfsetlinewidth{0.000000pt}%
\definecolor{currentstroke}{rgb}{0.263663,0.237631,0.518762}%
\pgfsetstrokecolor{currentstroke}%
\pgfsetdash{}{0pt}%
\pgfpathmoveto{\pgfqpoint{5.616816in}{1.894378in}}%
\pgfpathlineto{\pgfqpoint{5.696616in}{1.925940in}}%
\pgfpathlineto{\pgfqpoint{5.472274in}{2.124924in}}%
\pgfpathclose%
\pgfusepath{fill}%
\end{pgfscope}%
\begin{pgfscope}%
\pgfpathrectangle{\pgfqpoint{0.539299in}{0.078740in}}{\pgfqpoint{7.842520in}{7.842520in}}%
\pgfusepath{clip}%
\pgfsetbuttcap%
\pgfsetroundjoin%
\definecolor{currentfill}{rgb}{0.204903,0.375746,0.553533}%
\pgfsetfillcolor{currentfill}%
\pgfsetlinewidth{0.000000pt}%
\definecolor{currentstroke}{rgb}{0.262138,0.242286,0.520837}%
\pgfsetstrokecolor{currentstroke}%
\pgfsetdash{}{0pt}%
\pgfpathmoveto{\pgfqpoint{3.746562in}{1.887616in}}%
\pgfpathlineto{\pgfqpoint{3.827198in}{2.468870in}}%
\pgfpathlineto{\pgfqpoint{3.682755in}{2.554388in}}%
\pgfpathclose%
\pgfusepath{fill}%
\end{pgfscope}%
\begin{pgfscope}%
\pgfpathrectangle{\pgfqpoint{0.539299in}{0.078740in}}{\pgfqpoint{7.842520in}{7.842520in}}%
\pgfusepath{clip}%
\pgfsetbuttcap%
\pgfsetroundjoin%
\definecolor{currentfill}{rgb}{0.154815,0.493313,0.557840}%
\pgfsetfillcolor{currentfill}%
\pgfsetlinewidth{0.000000pt}%
\definecolor{currentstroke}{rgb}{0.260571,0.246922,0.522828}%
\pgfsetstrokecolor{currentstroke}%
\pgfsetdash{}{0pt}%
\pgfpathmoveto{\pgfqpoint{3.619444in}{3.158415in}}%
\pgfpathlineto{\pgfqpoint{3.395040in}{2.713747in}}%
\pgfpathlineto{\pgfqpoint{3.538693in}{2.636201in}}%
\pgfpathclose%
\pgfusepath{fill}%
\end{pgfscope}%
\begin{pgfscope}%
\pgfpathrectangle{\pgfqpoint{0.539299in}{0.078740in}}{\pgfqpoint{7.842520in}{7.842520in}}%
\pgfusepath{clip}%
\pgfsetbuttcap%
\pgfsetroundjoin%
\definecolor{currentfill}{rgb}{0.175707,0.697900,0.491033}%
\pgfsetfillcolor{currentfill}%
\pgfsetlinewidth{0.000000pt}%
\definecolor{currentstroke}{rgb}{0.258965,0.251537,0.524736}%
\pgfsetstrokecolor{currentstroke}%
\pgfsetdash{}{0pt}%
\pgfpathmoveto{\pgfqpoint{3.124144in}{3.970746in}}%
\pgfpathlineto{\pgfqpoint{3.044293in}{3.487491in}}%
\pgfpathlineto{\pgfqpoint{3.267846in}{3.892059in}}%
\pgfpathclose%
\pgfusepath{fill}%
\end{pgfscope}%
\begin{pgfscope}%
\pgfpathrectangle{\pgfqpoint{0.539299in}{0.078740in}}{\pgfqpoint{7.842520in}{7.842520in}}%
\pgfusepath{clip}%
\pgfsetbuttcap%
\pgfsetroundjoin%
\definecolor{currentfill}{rgb}{0.132444,0.552216,0.553018}%
\pgfsetfillcolor{currentfill}%
\pgfsetlinewidth{0.000000pt}%
\definecolor{currentstroke}{rgb}{0.257322,0.256130,0.526563}%
\pgfsetstrokecolor{currentstroke}%
\pgfsetdash{}{0pt}%
\pgfpathmoveto{\pgfqpoint{3.330909in}{3.339983in}}%
\pgfpathlineto{\pgfqpoint{3.395040in}{2.713747in}}%
\pgfpathlineto{\pgfqpoint{3.474971in}{3.252817in}}%
\pgfpathclose%
\pgfusepath{fill}%
\end{pgfscope}%
\begin{pgfscope}%
\pgfpathrectangle{\pgfqpoint{0.539299in}{0.078740in}}{\pgfqpoint{7.842520in}{7.842520in}}%
\pgfusepath{clip}%
\pgfsetbuttcap%
\pgfsetroundjoin%
\definecolor{currentfill}{rgb}{0.311925,0.767822,0.415586}%
\pgfsetfillcolor{currentfill}%
\pgfsetlinewidth{0.000000pt}%
\definecolor{currentstroke}{rgb}{0.255645,0.260703,0.528312}%
\pgfsetstrokecolor{currentstroke}%
\pgfsetdash{}{0pt}%
\pgfpathmoveto{\pgfqpoint{3.124144in}{3.970746in}}%
\pgfpathlineto{\pgfqpoint{3.062545in}{4.438625in}}%
\pgfpathlineto{\pgfqpoint{2.981110in}{4.034439in}}%
\pgfpathclose%
\pgfusepath{fill}%
\end{pgfscope}%
\begin{pgfscope}%
\pgfpathrectangle{\pgfqpoint{0.539299in}{0.078740in}}{\pgfqpoint{7.842520in}{7.842520in}}%
\pgfusepath{clip}%
\pgfsetbuttcap%
\pgfsetroundjoin%
\definecolor{currentfill}{rgb}{0.157851,0.683765,0.501686}%
\pgfsetfillcolor{currentfill}%
\pgfsetlinewidth{0.000000pt}%
\definecolor{currentstroke}{rgb}{0.253935,0.265254,0.529983}%
\pgfsetstrokecolor{currentstroke}%
\pgfsetdash{}{0pt}%
\pgfpathmoveto{\pgfqpoint{4.537995in}{3.619381in}}%
\pgfpathlineto{\pgfqpoint{4.621833in}{3.675341in}}%
\pgfpathlineto{\pgfqpoint{4.476731in}{3.882966in}}%
\pgfpathclose%
\pgfusepath{fill}%
\end{pgfscope}%
\begin{pgfscope}%
\pgfpathrectangle{\pgfqpoint{0.539299in}{0.078740in}}{\pgfqpoint{7.842520in}{7.842520in}}%
\pgfusepath{clip}%
\pgfsetbuttcap%
\pgfsetroundjoin%
\definecolor{currentfill}{rgb}{0.835270,0.886029,0.102646}%
\pgfsetfillcolor{currentfill}%
\pgfsetlinewidth{0.000000pt}%
\definecolor{currentstroke}{rgb}{0.252194,0.269783,0.531579}%
\pgfsetstrokecolor{currentstroke}%
\pgfsetdash{}{0pt}%
\pgfpathmoveto{\pgfqpoint{3.316139in}{5.195751in}}%
\pgfpathlineto{\pgfqpoint{3.374513in}{4.921390in}}%
\pgfpathlineto{\pgfqpoint{3.460255in}{5.092948in}}%
\pgfpathclose%
\pgfusepath{fill}%
\end{pgfscope}%
\begin{pgfscope}%
\pgfpathrectangle{\pgfqpoint{0.539299in}{0.078740in}}{\pgfqpoint{7.842520in}{7.842520in}}%
\pgfusepath{clip}%
\pgfsetbuttcap%
\pgfsetroundjoin%
\definecolor{currentfill}{rgb}{0.121148,0.592739,0.544641}%
\pgfsetfillcolor{currentfill}%
\pgfsetlinewidth{0.000000pt}%
\definecolor{currentstroke}{rgb}{0.250425,0.274290,0.533103}%
\pgfsetstrokecolor{currentstroke}%
\pgfsetdash{}{0pt}%
\pgfpathmoveto{\pgfqpoint{4.683466in}{3.421156in}}%
\pgfpathlineto{\pgfqpoint{4.828804in}{3.219577in}}%
\pgfpathlineto{\pgfqpoint{4.911519in}{3.252217in}}%
\pgfpathclose%
\pgfusepath{fill}%
\end{pgfscope}%
\begin{pgfscope}%
\pgfpathrectangle{\pgfqpoint{0.539299in}{0.078740in}}{\pgfqpoint{7.842520in}{7.842520in}}%
\pgfusepath{clip}%
\pgfsetbuttcap%
\pgfsetroundjoin%
\definecolor{currentfill}{rgb}{0.772852,0.877868,0.131109}%
\pgfsetfillcolor{currentfill}%
\pgfsetlinewidth{0.000000pt}%
\definecolor{currentstroke}{rgb}{0.248629,0.278775,0.534556}%
\pgfsetstrokecolor{currentstroke}%
\pgfsetdash{}{0pt}%
\pgfpathmoveto{\pgfqpoint{3.087223in}{5.076965in}}%
\pgfpathlineto{\pgfqpoint{3.145794in}{4.762662in}}%
\pgfpathlineto{\pgfqpoint{3.230452in}{5.012748in}}%
\pgfpathclose%
\pgfusepath{fill}%
\end{pgfscope}%
\begin{pgfscope}%
\pgfpathrectangle{\pgfqpoint{0.539299in}{0.078740in}}{\pgfqpoint{7.842520in}{7.842520in}}%
\pgfusepath{clip}%
\pgfsetbuttcap%
\pgfsetroundjoin%
\definecolor{currentfill}{rgb}{0.190631,0.407061,0.556089}%
\pgfsetfillcolor{currentfill}%
\pgfsetlinewidth{0.000000pt}%
\definecolor{currentstroke}{rgb}{0.246811,0.283237,0.535941}%
\pgfsetstrokecolor{currentstroke}%
\pgfsetdash{}{0pt}%
\pgfpathmoveto{\pgfqpoint{5.327340in}{2.342890in}}%
\pgfpathlineto{\pgfqpoint{5.408389in}{2.379042in}}%
\pgfpathlineto{\pgfqpoint{5.263813in}{2.595433in}}%
\pgfpathclose%
\pgfusepath{fill}%
\end{pgfscope}%
\begin{pgfscope}%
\pgfpathrectangle{\pgfqpoint{0.539299in}{0.078740in}}{\pgfqpoint{7.842520in}{7.842520in}}%
\pgfusepath{clip}%
\pgfsetbuttcap%
\pgfsetroundjoin%
\definecolor{currentfill}{rgb}{0.824940,0.884720,0.106217}%
\pgfsetfillcolor{currentfill}%
\pgfsetlinewidth{0.000000pt}%
\definecolor{currentstroke}{rgb}{0.244972,0.287675,0.537260}%
\pgfsetstrokecolor{currentstroke}%
\pgfsetdash{}{0pt}%
\pgfpathmoveto{\pgfqpoint{3.230452in}{5.012748in}}%
\pgfpathlineto{\pgfqpoint{3.374513in}{4.921390in}}%
\pgfpathlineto{\pgfqpoint{3.316139in}{5.195751in}}%
\pgfpathclose%
\pgfusepath{fill}%
\end{pgfscope}%
\begin{pgfscope}%
\pgfpathrectangle{\pgfqpoint{0.539299in}{0.078740in}}{\pgfqpoint{7.842520in}{7.842520in}}%
\pgfusepath{clip}%
\pgfsetbuttcap%
\pgfsetroundjoin%
\definecolor{currentfill}{rgb}{0.606045,0.850733,0.236712}%
\pgfsetfillcolor{currentfill}%
\pgfsetlinewidth{0.000000pt}%
\definecolor{currentstroke}{rgb}{0.243113,0.292092,0.538516}%
\pgfsetstrokecolor{currentstroke}%
\pgfsetdash{}{0pt}%
\pgfpathmoveto{\pgfqpoint{3.002777in}{4.819100in}}%
\pgfpathlineto{\pgfqpoint{3.062545in}{4.438625in}}%
\pgfpathlineto{\pgfqpoint{3.145794in}{4.762662in}}%
\pgfpathclose%
\pgfusepath{fill}%
\end{pgfscope}%
\begin{pgfscope}%
\pgfpathrectangle{\pgfqpoint{0.539299in}{0.078740in}}{\pgfqpoint{7.842520in}{7.842520in}}%
\pgfusepath{clip}%
\pgfsetbuttcap%
\pgfsetroundjoin%
\definecolor{currentfill}{rgb}{0.226397,0.728888,0.462789}%
\pgfsetfillcolor{currentfill}%
\pgfsetlinewidth{0.000000pt}%
\definecolor{currentstroke}{rgb}{0.241237,0.296485,0.539709}%
\pgfsetstrokecolor{currentstroke}%
\pgfsetdash{}{0pt}%
\pgfpathmoveto{\pgfqpoint{4.331483in}{4.086352in}}%
\pgfpathlineto{\pgfqpoint{4.392412in}{3.813568in}}%
\pgfpathlineto{\pgfqpoint{4.476731in}{3.882966in}}%
\pgfpathclose%
\pgfusepath{fill}%
\end{pgfscope}%
\begin{pgfscope}%
\pgfpathrectangle{\pgfqpoint{0.539299in}{0.078740in}}{\pgfqpoint{7.842520in}{7.842520in}}%
\pgfusepath{clip}%
\pgfsetbuttcap%
\pgfsetroundjoin%
\definecolor{currentfill}{rgb}{0.804182,0.882046,0.114965}%
\pgfsetfillcolor{currentfill}%
\pgfsetlinewidth{0.000000pt}%
\definecolor{currentstroke}{rgb}{0.239346,0.300855,0.540844}%
\pgfsetstrokecolor{currentstroke}%
\pgfsetdash{}{0pt}%
\pgfpathmoveto{\pgfqpoint{3.460255in}{5.092948in}}%
\pgfpathlineto{\pgfqpoint{3.374513in}{4.921390in}}%
\pgfpathlineto{\pgfqpoint{3.604941in}{4.965350in}}%
\pgfpathclose%
\pgfusepath{fill}%
\end{pgfscope}%
\begin{pgfscope}%
\pgfpathrectangle{\pgfqpoint{0.539299in}{0.078740in}}{\pgfqpoint{7.842520in}{7.842520in}}%
\pgfusepath{clip}%
\pgfsetbuttcap%
\pgfsetroundjoin%
\definecolor{currentfill}{rgb}{0.210503,0.363727,0.552206}%
\pgfsetfillcolor{currentfill}%
\pgfsetlinewidth{0.000000pt}%
\definecolor{currentstroke}{rgb}{0.237441,0.305202,0.541921}%
\pgfsetstrokecolor{currentstroke}%
\pgfsetdash{}{0pt}%
\pgfpathmoveto{\pgfqpoint{3.971998in}{2.380078in}}%
\pgfpathlineto{\pgfqpoint{3.827198in}{2.468870in}}%
\pgfpathlineto{\pgfqpoint{3.746562in}{1.887616in}}%
\pgfpathclose%
\pgfusepath{fill}%
\end{pgfscope}%
\begin{pgfscope}%
\pgfpathrectangle{\pgfqpoint{0.539299in}{0.078740in}}{\pgfqpoint{7.842520in}{7.842520in}}%
\pgfusepath{clip}%
\pgfsetbuttcap%
\pgfsetroundjoin%
\definecolor{currentfill}{rgb}{0.124780,0.640461,0.527068}%
\pgfsetfillcolor{currentfill}%
\pgfsetlinewidth{0.000000pt}%
\definecolor{currentstroke}{rgb}{0.235526,0.309527,0.542944}%
\pgfsetstrokecolor{currentstroke}%
\pgfsetdash{}{0pt}%
\pgfpathmoveto{\pgfqpoint{4.537995in}{3.619381in}}%
\pgfpathlineto{\pgfqpoint{4.683466in}{3.421156in}}%
\pgfpathlineto{\pgfqpoint{4.766767in}{3.464787in}}%
\pgfpathclose%
\pgfusepath{fill}%
\end{pgfscope}%
\begin{pgfscope}%
\pgfpathrectangle{\pgfqpoint{0.539299in}{0.078740in}}{\pgfqpoint{7.842520in}{7.842520in}}%
\pgfusepath{clip}%
\pgfsetbuttcap%
\pgfsetroundjoin%
\definecolor{currentfill}{rgb}{0.136408,0.541173,0.554483}%
\pgfsetfillcolor{currentfill}%
\pgfsetlinewidth{0.000000pt}%
\definecolor{currentstroke}{rgb}{0.233603,0.313828,0.543914}%
\pgfsetstrokecolor{currentstroke}%
\pgfsetdash{}{0pt}%
\pgfpathmoveto{\pgfqpoint{3.474971in}{3.252817in}}%
\pgfpathlineto{\pgfqpoint{3.395040in}{2.713747in}}%
\pgfpathlineto{\pgfqpoint{3.619444in}{3.158415in}}%
\pgfpathclose%
\pgfusepath{fill}%
\end{pgfscope}%
\begin{pgfscope}%
\pgfpathrectangle{\pgfqpoint{0.539299in}{0.078740in}}{\pgfqpoint{7.842520in}{7.842520in}}%
\pgfusepath{clip}%
\pgfsetbuttcap%
\pgfsetroundjoin%
\definecolor{currentfill}{rgb}{0.239346,0.300855,0.540844}%
\pgfsetfillcolor{currentfill}%
\pgfsetlinewidth{0.000000pt}%
\definecolor{currentstroke}{rgb}{0.231674,0.318106,0.544834}%
\pgfsetstrokecolor{currentstroke}%
\pgfsetdash{}{0pt}%
\pgfpathmoveto{\pgfqpoint{3.971998in}{2.380078in}}%
\pgfpathlineto{\pgfqpoint{3.890478in}{1.820244in}}%
\pgfpathlineto{\pgfqpoint{4.034810in}{1.751253in}}%
\pgfpathclose%
\pgfusepath{fill}%
\end{pgfscope}%
\begin{pgfscope}%
\pgfpathrectangle{\pgfqpoint{0.539299in}{0.078740in}}{\pgfqpoint{7.842520in}{7.842520in}}%
\pgfusepath{clip}%
\pgfsetbuttcap%
\pgfsetroundjoin%
\definecolor{currentfill}{rgb}{0.126326,0.644107,0.525311}%
\pgfsetfillcolor{currentfill}%
\pgfsetlinewidth{0.000000pt}%
\definecolor{currentstroke}{rgb}{0.229739,0.322361,0.545706}%
\pgfsetstrokecolor{currentstroke}%
\pgfsetdash{}{0pt}%
\pgfpathmoveto{\pgfqpoint{3.187325in}{3.418679in}}%
\pgfpathlineto{\pgfqpoint{3.330909in}{3.339983in}}%
\pgfpathlineto{\pgfqpoint{3.412099in}{3.800510in}}%
\pgfpathclose%
\pgfusepath{fill}%
\end{pgfscope}%
\begin{pgfscope}%
\pgfpathrectangle{\pgfqpoint{0.539299in}{0.078740in}}{\pgfqpoint{7.842520in}{7.842520in}}%
\pgfusepath{clip}%
\pgfsetbuttcap%
\pgfsetroundjoin%
\definecolor{currentfill}{rgb}{0.377779,0.791781,0.377939}%
\pgfsetfillcolor{currentfill}%
\pgfsetlinewidth{0.000000pt}%
\definecolor{currentstroke}{rgb}{0.227802,0.326594,0.546532}%
\pgfsetstrokecolor{currentstroke}%
\pgfsetdash{}{0pt}%
\pgfpathmoveto{\pgfqpoint{3.206070in}{4.367077in}}%
\pgfpathlineto{\pgfqpoint{3.062545in}{4.438625in}}%
\pgfpathlineto{\pgfqpoint{3.124144in}{3.970746in}}%
\pgfpathclose%
\pgfusepath{fill}%
\end{pgfscope}%
\begin{pgfscope}%
\pgfpathrectangle{\pgfqpoint{0.539299in}{0.078740in}}{\pgfqpoint{7.842520in}{7.842520in}}%
\pgfusepath{clip}%
\pgfsetbuttcap%
\pgfsetroundjoin%
\definecolor{currentfill}{rgb}{0.204903,0.375746,0.553533}%
\pgfsetfillcolor{currentfill}%
\pgfsetlinewidth{0.000000pt}%
\definecolor{currentstroke}{rgb}{0.225863,0.330805,0.547314}%
\pgfsetstrokecolor{currentstroke}%
\pgfsetdash{}{0pt}%
\pgfpathmoveto{\pgfqpoint{5.408389in}{2.379042in}}%
\pgfpathlineto{\pgfqpoint{5.327340in}{2.342890in}}%
\pgfpathlineto{\pgfqpoint{5.472274in}{2.124924in}}%
\pgfpathclose%
\pgfusepath{fill}%
\end{pgfscope}%
\begin{pgfscope}%
\pgfpathrectangle{\pgfqpoint{0.539299in}{0.078740in}}{\pgfqpoint{7.842520in}{7.842520in}}%
\pgfusepath{clip}%
\pgfsetbuttcap%
\pgfsetroundjoin%
\definecolor{currentfill}{rgb}{0.162142,0.474838,0.558140}%
\pgfsetfillcolor{currentfill}%
\pgfsetlinewidth{0.000000pt}%
\definecolor{currentstroke}{rgb}{0.223925,0.334994,0.548053}%
\pgfsetstrokecolor{currentstroke}%
\pgfsetdash{}{0pt}%
\pgfpathmoveto{\pgfqpoint{5.263813in}{2.595433in}}%
\pgfpathlineto{\pgfqpoint{5.119001in}{2.807062in}}%
\pgfpathlineto{\pgfqpoint{5.036687in}{2.752478in}}%
\pgfpathclose%
\pgfusepath{fill}%
\end{pgfscope}%
\begin{pgfscope}%
\pgfpathrectangle{\pgfqpoint{0.539299in}{0.078740in}}{\pgfqpoint{7.842520in}{7.842520in}}%
\pgfusepath{clip}%
\pgfsetbuttcap%
\pgfsetroundjoin%
\definecolor{currentfill}{rgb}{0.344074,0.780029,0.397381}%
\pgfsetfillcolor{currentfill}%
\pgfsetlinewidth{0.000000pt}%
\definecolor{currentstroke}{rgb}{0.221989,0.339161,0.548752}%
\pgfsetstrokecolor{currentstroke}%
\pgfsetdash{}{0pt}%
\pgfpathmoveto{\pgfqpoint{4.186126in}{4.283787in}}%
\pgfpathlineto{\pgfqpoint{4.101044in}{4.185140in}}%
\pgfpathlineto{\pgfqpoint{4.331483in}{4.086352in}}%
\pgfpathclose%
\pgfusepath{fill}%
\end{pgfscope}%
\begin{pgfscope}%
\pgfpathrectangle{\pgfqpoint{0.539299in}{0.078740in}}{\pgfqpoint{7.842520in}{7.842520in}}%
\pgfusepath{clip}%
\pgfsetbuttcap%
\pgfsetroundjoin%
\definecolor{currentfill}{rgb}{0.159194,0.482237,0.558073}%
\pgfsetfillcolor{currentfill}%
\pgfsetlinewidth{0.000000pt}%
\definecolor{currentstroke}{rgb}{0.220057,0.343307,0.549413}%
\pgfsetstrokecolor{currentstroke}%
\pgfsetdash{}{0pt}%
\pgfpathmoveto{\pgfqpoint{3.764278in}{3.057833in}}%
\pgfpathlineto{\pgfqpoint{3.538693in}{2.636201in}}%
\pgfpathlineto{\pgfqpoint{3.682755in}{2.554388in}}%
\pgfpathclose%
\pgfusepath{fill}%
\end{pgfscope}%
\begin{pgfscope}%
\pgfpathrectangle{\pgfqpoint{0.539299in}{0.078740in}}{\pgfqpoint{7.842520in}{7.842520in}}%
\pgfusepath{clip}%
\pgfsetbuttcap%
\pgfsetroundjoin%
\definecolor{currentfill}{rgb}{0.162016,0.687316,0.499129}%
\pgfsetfillcolor{currentfill}%
\pgfsetlinewidth{0.000000pt}%
\definecolor{currentstroke}{rgb}{0.218130,0.347432,0.550038}%
\pgfsetstrokecolor{currentstroke}%
\pgfsetdash{}{0pt}%
\pgfpathmoveto{\pgfqpoint{3.412099in}{3.800510in}}%
\pgfpathlineto{\pgfqpoint{3.267846in}{3.892059in}}%
\pgfpathlineto{\pgfqpoint{3.187325in}{3.418679in}}%
\pgfpathclose%
\pgfusepath{fill}%
\end{pgfscope}%
\begin{pgfscope}%
\pgfpathrectangle{\pgfqpoint{0.539299in}{0.078740in}}{\pgfqpoint{7.842520in}{7.842520in}}%
\pgfusepath{clip}%
\pgfsetbuttcap%
\pgfsetroundjoin%
\definecolor{currentfill}{rgb}{0.699415,0.867117,0.175971}%
\pgfsetfillcolor{currentfill}%
\pgfsetlinewidth{0.000000pt}%
\definecolor{currentstroke}{rgb}{0.216210,0.351535,0.550627}%
\pgfsetstrokecolor{currentstroke}%
\pgfsetdash{}{0pt}%
\pgfpathmoveto{\pgfqpoint{3.230452in}{5.012748in}}%
\pgfpathlineto{\pgfqpoint{3.145794in}{4.762662in}}%
\pgfpathlineto{\pgfqpoint{3.289659in}{4.681825in}}%
\pgfpathclose%
\pgfusepath{fill}%
\end{pgfscope}%
\begin{pgfscope}%
\pgfpathrectangle{\pgfqpoint{0.539299in}{0.078740in}}{\pgfqpoint{7.842520in}{7.842520in}}%
\pgfusepath{clip}%
\pgfsetbuttcap%
\pgfsetroundjoin%
\definecolor{currentfill}{rgb}{0.585678,0.846661,0.249897}%
\pgfsetfillcolor{currentfill}%
\pgfsetlinewidth{0.000000pt}%
\definecolor{currentstroke}{rgb}{0.214298,0.355619,0.551184}%
\pgfsetstrokecolor{currentstroke}%
\pgfsetdash{}{0pt}%
\pgfpathmoveto{\pgfqpoint{3.062545in}{4.438625in}}%
\pgfpathlineto{\pgfqpoint{3.289659in}{4.681825in}}%
\pgfpathlineto{\pgfqpoint{3.145794in}{4.762662in}}%
\pgfpathclose%
\pgfusepath{fill}%
\end{pgfscope}%
\begin{pgfscope}%
\pgfpathrectangle{\pgfqpoint{0.539299in}{0.078740in}}{\pgfqpoint{7.842520in}{7.842520in}}%
\pgfusepath{clip}%
\pgfsetbuttcap%
\pgfsetroundjoin%
\definecolor{currentfill}{rgb}{0.412913,0.803041,0.357269}%
\pgfsetfillcolor{currentfill}%
\pgfsetlinewidth{0.000000pt}%
\definecolor{currentstroke}{rgb}{0.212395,0.359683,0.551710}%
\pgfsetstrokecolor{currentstroke}%
\pgfsetdash{}{0pt}%
\pgfpathmoveto{\pgfqpoint{4.101044in}{4.185140in}}%
\pgfpathlineto{\pgfqpoint{4.186126in}{4.283787in}}%
\pgfpathlineto{\pgfqpoint{4.040710in}{4.473147in}}%
\pgfpathclose%
\pgfusepath{fill}%
\end{pgfscope}%
\begin{pgfscope}%
\pgfpathrectangle{\pgfqpoint{0.539299in}{0.078740in}}{\pgfqpoint{7.842520in}{7.842520in}}%
\pgfusepath{clip}%
\pgfsetbuttcap%
\pgfsetroundjoin%
\definecolor{currentfill}{rgb}{0.699415,0.867117,0.175971}%
\pgfsetfillcolor{currentfill}%
\pgfsetlinewidth{0.000000pt}%
\definecolor{currentstroke}{rgb}{0.210503,0.363727,0.552206}%
\pgfsetstrokecolor{currentstroke}%
\pgfsetdash{}{0pt}%
\pgfpathmoveto{\pgfqpoint{3.664322in}{4.672860in}}%
\pgfpathlineto{\pgfqpoint{3.750009in}{4.817083in}}%
\pgfpathlineto{\pgfqpoint{3.604941in}{4.965350in}}%
\pgfpathclose%
\pgfusepath{fill}%
\end{pgfscope}%
\begin{pgfscope}%
\pgfpathrectangle{\pgfqpoint{0.539299in}{0.078740in}}{\pgfqpoint{7.842520in}{7.842520in}}%
\pgfusepath{clip}%
\pgfsetbuttcap%
\pgfsetroundjoin%
\definecolor{currentfill}{rgb}{0.180653,0.701402,0.488189}%
\pgfsetfillcolor{currentfill}%
\pgfsetlinewidth{0.000000pt}%
\definecolor{currentstroke}{rgb}{0.208623,0.367752,0.552675}%
\pgfsetstrokecolor{currentstroke}%
\pgfsetdash{}{0pt}%
\pgfpathmoveto{\pgfqpoint{4.476731in}{3.882966in}}%
\pgfpathlineto{\pgfqpoint{4.392412in}{3.813568in}}%
\pgfpathlineto{\pgfqpoint{4.537995in}{3.619381in}}%
\pgfpathclose%
\pgfusepath{fill}%
\end{pgfscope}%
\begin{pgfscope}%
\pgfpathrectangle{\pgfqpoint{0.539299in}{0.078740in}}{\pgfqpoint{7.842520in}{7.842520in}}%
\pgfusepath{clip}%
\pgfsetbuttcap%
\pgfsetroundjoin%
\definecolor{currentfill}{rgb}{0.751884,0.874951,0.143228}%
\pgfsetfillcolor{currentfill}%
\pgfsetlinewidth{0.000000pt}%
\definecolor{currentstroke}{rgb}{0.206756,0.371758,0.553117}%
\pgfsetstrokecolor{currentstroke}%
\pgfsetdash{}{0pt}%
\pgfpathmoveto{\pgfqpoint{3.604941in}{4.965350in}}%
\pgfpathlineto{\pgfqpoint{3.374513in}{4.921390in}}%
\pgfpathlineto{\pgfqpoint{3.519195in}{4.806876in}}%
\pgfpathclose%
\pgfusepath{fill}%
\end{pgfscope}%
\begin{pgfscope}%
\pgfpathrectangle{\pgfqpoint{0.539299in}{0.078740in}}{\pgfqpoint{7.842520in}{7.842520in}}%
\pgfusepath{clip}%
\pgfsetbuttcap%
\pgfsetroundjoin%
\definecolor{currentfill}{rgb}{0.741388,0.873449,0.149561}%
\pgfsetfillcolor{currentfill}%
\pgfsetlinewidth{0.000000pt}%
\definecolor{currentstroke}{rgb}{0.204903,0.375746,0.553533}%
\pgfsetstrokecolor{currentstroke}%
\pgfsetdash{}{0pt}%
\pgfpathmoveto{\pgfqpoint{3.289659in}{4.681825in}}%
\pgfpathlineto{\pgfqpoint{3.374513in}{4.921390in}}%
\pgfpathlineto{\pgfqpoint{3.230452in}{5.012748in}}%
\pgfpathclose%
\pgfusepath{fill}%
\end{pgfscope}%
\begin{pgfscope}%
\pgfpathrectangle{\pgfqpoint{0.539299in}{0.078740in}}{\pgfqpoint{7.842520in}{7.842520in}}%
\pgfusepath{clip}%
\pgfsetbuttcap%
\pgfsetroundjoin%
\definecolor{currentfill}{rgb}{0.278012,0.180367,0.486697}%
\pgfsetfillcolor{currentfill}%
\pgfsetlinewidth{0.000000pt}%
\definecolor{currentstroke}{rgb}{0.203063,0.379716,0.553925}%
\pgfsetstrokecolor{currentstroke}%
\pgfsetdash{}{0pt}%
\pgfpathmoveto{\pgfqpoint{5.760821in}{1.647418in}}%
\pgfpathlineto{\pgfqpoint{5.680690in}{1.591684in}}%
\pgfpathlineto{\pgfqpoint{5.824454in}{1.317143in}}%
\pgfpathclose%
\pgfusepath{fill}%
\end{pgfscope}%
\begin{pgfscope}%
\pgfpathrectangle{\pgfqpoint{0.539299in}{0.078740in}}{\pgfqpoint{7.842520in}{7.842520in}}%
\pgfusepath{clip}%
\pgfsetbuttcap%
\pgfsetroundjoin%
\definecolor{currentfill}{rgb}{0.150476,0.504369,0.557430}%
\pgfsetfillcolor{currentfill}%
\pgfsetlinewidth{0.000000pt}%
\definecolor{currentstroke}{rgb}{0.201239,0.383670,0.554294}%
\pgfsetstrokecolor{currentstroke}%
\pgfsetdash{}{0pt}%
\pgfpathmoveto{\pgfqpoint{5.036687in}{2.752478in}}%
\pgfpathlineto{\pgfqpoint{5.119001in}{2.807062in}}%
\pgfpathlineto{\pgfqpoint{4.973989in}{3.014923in}}%
\pgfpathclose%
\pgfusepath{fill}%
\end{pgfscope}%
\begin{pgfscope}%
\pgfpathrectangle{\pgfqpoint{0.539299in}{0.078740in}}{\pgfqpoint{7.842520in}{7.842520in}}%
\pgfusepath{clip}%
\pgfsetbuttcap%
\pgfsetroundjoin%
\definecolor{currentfill}{rgb}{0.545524,0.838039,0.275626}%
\pgfsetfillcolor{currentfill}%
\pgfsetlinewidth{0.000000pt}%
\definecolor{currentstroke}{rgb}{0.199430,0.387607,0.554642}%
\pgfsetstrokecolor{currentstroke}%
\pgfsetdash{}{0pt}%
\pgfpathmoveto{\pgfqpoint{3.895306in}{4.651896in}}%
\pgfpathlineto{\pgfqpoint{3.809748in}{4.522655in}}%
\pgfpathlineto{\pgfqpoint{4.040710in}{4.473147in}}%
\pgfpathclose%
\pgfusepath{fill}%
\end{pgfscope}%
\begin{pgfscope}%
\pgfpathrectangle{\pgfqpoint{0.539299in}{0.078740in}}{\pgfqpoint{7.842520in}{7.842520in}}%
\pgfusepath{clip}%
\pgfsetbuttcap%
\pgfsetroundjoin%
\definecolor{currentfill}{rgb}{0.606045,0.850733,0.236712}%
\pgfsetfillcolor{currentfill}%
\pgfsetlinewidth{0.000000pt}%
\definecolor{currentstroke}{rgb}{0.197636,0.391528,0.554969}%
\pgfsetstrokecolor{currentstroke}%
\pgfsetdash{}{0pt}%
\pgfpathmoveto{\pgfqpoint{3.895306in}{4.651896in}}%
\pgfpathlineto{\pgfqpoint{3.750009in}{4.817083in}}%
\pgfpathlineto{\pgfqpoint{3.809748in}{4.522655in}}%
\pgfpathclose%
\pgfusepath{fill}%
\end{pgfscope}%
\begin{pgfscope}%
\pgfpathrectangle{\pgfqpoint{0.539299in}{0.078740in}}{\pgfqpoint{7.842520in}{7.842520in}}%
\pgfusepath{clip}%
\pgfsetbuttcap%
\pgfsetroundjoin%
\definecolor{currentfill}{rgb}{0.123444,0.636809,0.528763}%
\pgfsetfillcolor{currentfill}%
\pgfsetlinewidth{0.000000pt}%
\definecolor{currentstroke}{rgb}{0.195860,0.395433,0.555276}%
\pgfsetstrokecolor{currentstroke}%
\pgfsetdash{}{0pt}%
\pgfpathmoveto{\pgfqpoint{3.412099in}{3.800510in}}%
\pgfpathlineto{\pgfqpoint{3.330909in}{3.339983in}}%
\pgfpathlineto{\pgfqpoint{3.474971in}{3.252817in}}%
\pgfpathclose%
\pgfusepath{fill}%
\end{pgfscope}%
\begin{pgfscope}%
\pgfpathrectangle{\pgfqpoint{0.539299in}{0.078740in}}{\pgfqpoint{7.842520in}{7.842520in}}%
\pgfusepath{clip}%
\pgfsetbuttcap%
\pgfsetroundjoin%
\definecolor{currentfill}{rgb}{0.257322,0.256130,0.526563}%
\pgfsetfillcolor{currentfill}%
\pgfsetlinewidth{0.000000pt}%
\definecolor{currentstroke}{rgb}{0.194100,0.399323,0.555565}%
\pgfsetstrokecolor{currentstroke}%
\pgfsetdash{}{0pt}%
\pgfpathmoveto{\pgfqpoint{5.760821in}{1.647418in}}%
\pgfpathlineto{\pgfqpoint{5.616816in}{1.894378in}}%
\pgfpathlineto{\pgfqpoint{5.536081in}{1.836733in}}%
\pgfpathclose%
\pgfusepath{fill}%
\end{pgfscope}%
\begin{pgfscope}%
\pgfpathrectangle{\pgfqpoint{0.539299in}{0.078740in}}{\pgfqpoint{7.842520in}{7.842520in}}%
\pgfusepath{clip}%
\pgfsetbuttcap%
\pgfsetroundjoin%
\definecolor{currentfill}{rgb}{0.515992,0.831158,0.294279}%
\pgfsetfillcolor{currentfill}%
\pgfsetlinewidth{0.000000pt}%
\definecolor{currentstroke}{rgb}{0.192357,0.403199,0.555836}%
\pgfsetstrokecolor{currentstroke}%
\pgfsetdash{}{0pt}%
\pgfpathmoveto{\pgfqpoint{3.206070in}{4.367077in}}%
\pgfpathlineto{\pgfqpoint{3.289659in}{4.681825in}}%
\pgfpathlineto{\pgfqpoint{3.062545in}{4.438625in}}%
\pgfpathclose%
\pgfusepath{fill}%
\end{pgfscope}%
\begin{pgfscope}%
\pgfpathrectangle{\pgfqpoint{0.539299in}{0.078740in}}{\pgfqpoint{7.842520in}{7.842520in}}%
\pgfusepath{clip}%
\pgfsetbuttcap%
\pgfsetroundjoin%
\definecolor{currentfill}{rgb}{0.180629,0.429975,0.557282}%
\pgfsetfillcolor{currentfill}%
\pgfsetlinewidth{0.000000pt}%
\definecolor{currentstroke}{rgb}{0.190631,0.407061,0.556089}%
\pgfsetstrokecolor{currentstroke}%
\pgfsetdash{}{0pt}%
\pgfpathmoveto{\pgfqpoint{5.263813in}{2.595433in}}%
\pgfpathlineto{\pgfqpoint{5.182120in}{2.551317in}}%
\pgfpathlineto{\pgfqpoint{5.327340in}{2.342890in}}%
\pgfpathclose%
\pgfusepath{fill}%
\end{pgfscope}%
\begin{pgfscope}%
\pgfpathrectangle{\pgfqpoint{0.539299in}{0.078740in}}{\pgfqpoint{7.842520in}{7.842520in}}%
\pgfusepath{clip}%
\pgfsetbuttcap%
\pgfsetroundjoin%
\definecolor{currentfill}{rgb}{0.288921,0.758394,0.428426}%
\pgfsetfillcolor{currentfill}%
\pgfsetlinewidth{0.000000pt}%
\definecolor{currentstroke}{rgb}{0.188923,0.410910,0.556326}%
\pgfsetstrokecolor{currentstroke}%
\pgfsetdash{}{0pt}%
\pgfpathmoveto{\pgfqpoint{3.124144in}{3.970746in}}%
\pgfpathlineto{\pgfqpoint{3.267846in}{3.892059in}}%
\pgfpathlineto{\pgfqpoint{3.350268in}{4.277351in}}%
\pgfpathclose%
\pgfusepath{fill}%
\end{pgfscope}%
\begin{pgfscope}%
\pgfpathrectangle{\pgfqpoint{0.539299in}{0.078740in}}{\pgfqpoint{7.842520in}{7.842520in}}%
\pgfusepath{clip}%
\pgfsetbuttcap%
\pgfsetroundjoin%
\definecolor{currentfill}{rgb}{0.252899,0.742211,0.448284}%
\pgfsetfillcolor{currentfill}%
\pgfsetlinewidth{0.000000pt}%
\definecolor{currentstroke}{rgb}{0.187231,0.414746,0.556547}%
\pgfsetstrokecolor{currentstroke}%
\pgfsetdash{}{0pt}%
\pgfpathmoveto{\pgfqpoint{4.246749in}{4.002642in}}%
\pgfpathlineto{\pgfqpoint{4.392412in}{3.813568in}}%
\pgfpathlineto{\pgfqpoint{4.331483in}{4.086352in}}%
\pgfpathclose%
\pgfusepath{fill}%
\end{pgfscope}%
\begin{pgfscope}%
\pgfpathrectangle{\pgfqpoint{0.539299in}{0.078740in}}{\pgfqpoint{7.842520in}{7.842520in}}%
\pgfusepath{clip}%
\pgfsetbuttcap%
\pgfsetroundjoin%
\definecolor{currentfill}{rgb}{0.141935,0.526453,0.555991}%
\pgfsetfillcolor{currentfill}%
\pgfsetlinewidth{0.000000pt}%
\definecolor{currentstroke}{rgb}{0.185556,0.418570,0.556753}%
\pgfsetstrokecolor{currentstroke}%
\pgfsetdash{}{0pt}%
\pgfpathmoveto{\pgfqpoint{3.764278in}{3.057833in}}%
\pgfpathlineto{\pgfqpoint{3.619444in}{3.158415in}}%
\pgfpathlineto{\pgfqpoint{3.538693in}{2.636201in}}%
\pgfpathclose%
\pgfusepath{fill}%
\end{pgfscope}%
\begin{pgfscope}%
\pgfpathrectangle{\pgfqpoint{0.539299in}{0.078740in}}{\pgfqpoint{7.842520in}{7.842520in}}%
\pgfusepath{clip}%
\pgfsetbuttcap%
\pgfsetroundjoin%
\definecolor{currentfill}{rgb}{0.360741,0.785964,0.387814}%
\pgfsetfillcolor{currentfill}%
\pgfsetlinewidth{0.000000pt}%
\definecolor{currentstroke}{rgb}{0.183898,0.422383,0.556944}%
\pgfsetstrokecolor{currentstroke}%
\pgfsetdash{}{0pt}%
\pgfpathmoveto{\pgfqpoint{3.206070in}{4.367077in}}%
\pgfpathlineto{\pgfqpoint{3.124144in}{3.970746in}}%
\pgfpathlineto{\pgfqpoint{3.350268in}{4.277351in}}%
\pgfpathclose%
\pgfusepath{fill}%
\end{pgfscope}%
\begin{pgfscope}%
\pgfpathrectangle{\pgfqpoint{0.539299in}{0.078740in}}{\pgfqpoint{7.842520in}{7.842520in}}%
\pgfusepath{clip}%
\pgfsetbuttcap%
\pgfsetroundjoin%
\definecolor{currentfill}{rgb}{0.709898,0.868751,0.169257}%
\pgfsetfillcolor{currentfill}%
\pgfsetlinewidth{0.000000pt}%
\definecolor{currentstroke}{rgb}{0.182256,0.426184,0.557120}%
\pgfsetstrokecolor{currentstroke}%
\pgfsetdash{}{0pt}%
\pgfpathmoveto{\pgfqpoint{3.519195in}{4.806876in}}%
\pgfpathlineto{\pgfqpoint{3.664322in}{4.672860in}}%
\pgfpathlineto{\pgfqpoint{3.604941in}{4.965350in}}%
\pgfpathclose%
\pgfusepath{fill}%
\end{pgfscope}%
\begin{pgfscope}%
\pgfpathrectangle{\pgfqpoint{0.539299in}{0.078740in}}{\pgfqpoint{7.842520in}{7.842520in}}%
\pgfusepath{clip}%
\pgfsetbuttcap%
\pgfsetroundjoin%
\definecolor{currentfill}{rgb}{0.304148,0.764704,0.419943}%
\pgfsetfillcolor{currentfill}%
\pgfsetlinewidth{0.000000pt}%
\definecolor{currentstroke}{rgb}{0.180629,0.429975,0.557282}%
\pgfsetstrokecolor{currentstroke}%
\pgfsetdash{}{0pt}%
\pgfpathmoveto{\pgfqpoint{4.331483in}{4.086352in}}%
\pgfpathlineto{\pgfqpoint{4.101044in}{4.185140in}}%
\pgfpathlineto{\pgfqpoint{4.246749in}{4.002642in}}%
\pgfpathclose%
\pgfusepath{fill}%
\end{pgfscope}%
\begin{pgfscope}%
\pgfpathrectangle{\pgfqpoint{0.539299in}{0.078740in}}{\pgfqpoint{7.842520in}{7.842520in}}%
\pgfusepath{clip}%
\pgfsetbuttcap%
\pgfsetroundjoin%
\definecolor{currentfill}{rgb}{0.132444,0.552216,0.553018}%
\pgfsetfillcolor{currentfill}%
\pgfsetlinewidth{0.000000pt}%
\definecolor{currentstroke}{rgb}{0.179019,0.433756,0.557430}%
\pgfsetstrokecolor{currentstroke}%
\pgfsetdash{}{0pt}%
\pgfpathmoveto{\pgfqpoint{4.828804in}{3.219577in}}%
\pgfpathlineto{\pgfqpoint{4.891094in}{2.947957in}}%
\pgfpathlineto{\pgfqpoint{4.973989in}{3.014923in}}%
\pgfpathclose%
\pgfusepath{fill}%
\end{pgfscope}%
\begin{pgfscope}%
\pgfpathrectangle{\pgfqpoint{0.539299in}{0.078740in}}{\pgfqpoint{7.842520in}{7.842520in}}%
\pgfusepath{clip}%
\pgfsetbuttcap%
\pgfsetroundjoin%
\definecolor{currentfill}{rgb}{0.168126,0.459988,0.558082}%
\pgfsetfillcolor{currentfill}%
\pgfsetlinewidth{0.000000pt}%
\definecolor{currentstroke}{rgb}{0.177423,0.437527,0.557565}%
\pgfsetstrokecolor{currentstroke}%
\pgfsetdash{}{0pt}%
\pgfpathmoveto{\pgfqpoint{5.036687in}{2.752478in}}%
\pgfpathlineto{\pgfqpoint{5.182120in}{2.551317in}}%
\pgfpathlineto{\pgfqpoint{5.263813in}{2.595433in}}%
\pgfpathclose%
\pgfusepath{fill}%
\end{pgfscope}%
\begin{pgfscope}%
\pgfpathrectangle{\pgfqpoint{0.539299in}{0.078740in}}{\pgfqpoint{7.842520in}{7.842520in}}%
\pgfusepath{clip}%
\pgfsetbuttcap%
\pgfsetroundjoin%
\definecolor{currentfill}{rgb}{0.699415,0.867117,0.175971}%
\pgfsetfillcolor{currentfill}%
\pgfsetlinewidth{0.000000pt}%
\definecolor{currentstroke}{rgb}{0.175841,0.441290,0.557685}%
\pgfsetstrokecolor{currentstroke}%
\pgfsetdash{}{0pt}%
\pgfpathmoveto{\pgfqpoint{3.519195in}{4.806876in}}%
\pgfpathlineto{\pgfqpoint{3.374513in}{4.921390in}}%
\pgfpathlineto{\pgfqpoint{3.289659in}{4.681825in}}%
\pgfpathclose%
\pgfusepath{fill}%
\end{pgfscope}%
\begin{pgfscope}%
\pgfpathrectangle{\pgfqpoint{0.539299in}{0.078740in}}{\pgfqpoint{7.842520in}{7.842520in}}%
\pgfusepath{clip}%
\pgfsetbuttcap%
\pgfsetroundjoin%
\definecolor{currentfill}{rgb}{0.626579,0.854645,0.223353}%
\pgfsetfillcolor{currentfill}%
\pgfsetlinewidth{0.000000pt}%
\definecolor{currentstroke}{rgb}{0.174274,0.445044,0.557792}%
\pgfsetstrokecolor{currentstroke}%
\pgfsetdash{}{0pt}%
\pgfpathmoveto{\pgfqpoint{3.809748in}{4.522655in}}%
\pgfpathlineto{\pgfqpoint{3.750009in}{4.817083in}}%
\pgfpathlineto{\pgfqpoint{3.664322in}{4.672860in}}%
\pgfpathclose%
\pgfusepath{fill}%
\end{pgfscope}%
\begin{pgfscope}%
\pgfpathrectangle{\pgfqpoint{0.539299in}{0.078740in}}{\pgfqpoint{7.842520in}{7.842520in}}%
\pgfusepath{clip}%
\pgfsetbuttcap%
\pgfsetroundjoin%
\definecolor{currentfill}{rgb}{0.165117,0.467423,0.558141}%
\pgfsetfillcolor{currentfill}%
\pgfsetlinewidth{0.000000pt}%
\definecolor{currentstroke}{rgb}{0.172719,0.448791,0.557885}%
\pgfsetstrokecolor{currentstroke}%
\pgfsetdash{}{0pt}%
\pgfpathmoveto{\pgfqpoint{3.682755in}{2.554388in}}%
\pgfpathlineto{\pgfqpoint{3.827198in}{2.468870in}}%
\pgfpathlineto{\pgfqpoint{3.909430in}{2.951944in}}%
\pgfpathclose%
\pgfusepath{fill}%
\end{pgfscope}%
\begin{pgfscope}%
\pgfpathrectangle{\pgfqpoint{0.539299in}{0.078740in}}{\pgfqpoint{7.842520in}{7.842520in}}%
\pgfusepath{clip}%
\pgfsetbuttcap%
\pgfsetroundjoin%
\definecolor{currentfill}{rgb}{0.430983,0.808473,0.346476}%
\pgfsetfillcolor{currentfill}%
\pgfsetlinewidth{0.000000pt}%
\definecolor{currentstroke}{rgb}{0.171176,0.452530,0.557965}%
\pgfsetstrokecolor{currentstroke}%
\pgfsetdash{}{0pt}%
\pgfpathmoveto{\pgfqpoint{4.101044in}{4.185140in}}%
\pgfpathlineto{\pgfqpoint{4.040710in}{4.473147in}}%
\pgfpathlineto{\pgfqpoint{3.955353in}{4.359218in}}%
\pgfpathclose%
\pgfusepath{fill}%
\end{pgfscope}%
\begin{pgfscope}%
\pgfpathrectangle{\pgfqpoint{0.539299in}{0.078740in}}{\pgfqpoint{7.842520in}{7.842520in}}%
\pgfusepath{clip}%
\pgfsetbuttcap%
\pgfsetroundjoin%
\definecolor{currentfill}{rgb}{0.496615,0.826376,0.306377}%
\pgfsetfillcolor{currentfill}%
\pgfsetlinewidth{0.000000pt}%
\definecolor{currentstroke}{rgb}{0.169646,0.456262,0.558030}%
\pgfsetstrokecolor{currentstroke}%
\pgfsetdash{}{0pt}%
\pgfpathmoveto{\pgfqpoint{4.040710in}{4.473147in}}%
\pgfpathlineto{\pgfqpoint{3.809748in}{4.522655in}}%
\pgfpathlineto{\pgfqpoint{3.955353in}{4.359218in}}%
\pgfpathclose%
\pgfusepath{fill}%
\end{pgfscope}%
\begin{pgfscope}%
\pgfpathrectangle{\pgfqpoint{0.539299in}{0.078740in}}{\pgfqpoint{7.842520in}{7.842520in}}%
\pgfusepath{clip}%
\pgfsetbuttcap%
\pgfsetroundjoin%
\definecolor{currentfill}{rgb}{0.237441,0.305202,0.541921}%
\pgfsetfillcolor{currentfill}%
\pgfsetlinewidth{0.000000pt}%
\definecolor{currentstroke}{rgb}{0.168126,0.459988,0.558082}%
\pgfsetstrokecolor{currentstroke}%
\pgfsetdash{}{0pt}%
\pgfpathmoveto{\pgfqpoint{5.472274in}{2.124924in}}%
\pgfpathlineto{\pgfqpoint{5.536081in}{1.836733in}}%
\pgfpathlineto{\pgfqpoint{5.616816in}{1.894378in}}%
\pgfpathclose%
\pgfusepath{fill}%
\end{pgfscope}%
\begin{pgfscope}%
\pgfpathrectangle{\pgfqpoint{0.539299in}{0.078740in}}{\pgfqpoint{7.842520in}{7.842520in}}%
\pgfusepath{clip}%
\pgfsetbuttcap%
\pgfsetroundjoin%
\definecolor{currentfill}{rgb}{0.265145,0.232956,0.516599}%
\pgfsetfillcolor{currentfill}%
\pgfsetlinewidth{0.000000pt}%
\definecolor{currentstroke}{rgb}{0.166617,0.463708,0.558119}%
\pgfsetstrokecolor{currentstroke}%
\pgfsetdash{}{0pt}%
\pgfpathmoveto{\pgfqpoint{5.536081in}{1.836733in}}%
\pgfpathlineto{\pgfqpoint{5.680690in}{1.591684in}}%
\pgfpathlineto{\pgfqpoint{5.760821in}{1.647418in}}%
\pgfpathclose%
\pgfusepath{fill}%
\end{pgfscope}%
\begin{pgfscope}%
\pgfpathrectangle{\pgfqpoint{0.539299in}{0.078740in}}{\pgfqpoint{7.842520in}{7.842520in}}%
\pgfusepath{clip}%
\pgfsetbuttcap%
\pgfsetroundjoin%
\definecolor{currentfill}{rgb}{0.216210,0.351535,0.550627}%
\pgfsetfillcolor{currentfill}%
\pgfsetlinewidth{0.000000pt}%
\definecolor{currentstroke}{rgb}{0.165117,0.467423,0.558141}%
\pgfsetstrokecolor{currentstroke}%
\pgfsetdash{}{0pt}%
\pgfpathmoveto{\pgfqpoint{4.034810in}{1.751253in}}%
\pgfpathlineto{\pgfqpoint{4.117136in}{2.288294in}}%
\pgfpathlineto{\pgfqpoint{3.971998in}{2.380078in}}%
\pgfpathclose%
\pgfusepath{fill}%
\end{pgfscope}%
\begin{pgfscope}%
\pgfpathrectangle{\pgfqpoint{0.539299in}{0.078740in}}{\pgfqpoint{7.842520in}{7.842520in}}%
\pgfusepath{clip}%
\pgfsetbuttcap%
\pgfsetroundjoin%
\definecolor{currentfill}{rgb}{0.246811,0.283237,0.535941}%
\pgfsetfillcolor{currentfill}%
\pgfsetlinewidth{0.000000pt}%
\definecolor{currentstroke}{rgb}{0.163625,0.471133,0.558148}%
\pgfsetstrokecolor{currentstroke}%
\pgfsetdash{}{0pt}%
\pgfpathmoveto{\pgfqpoint{4.262595in}{2.193629in}}%
\pgfpathlineto{\pgfqpoint{4.034810in}{1.751253in}}%
\pgfpathlineto{\pgfqpoint{4.179552in}{1.680684in}}%
\pgfpathclose%
\pgfusepath{fill}%
\end{pgfscope}%
\begin{pgfscope}%
\pgfpathrectangle{\pgfqpoint{0.539299in}{0.078740in}}{\pgfqpoint{7.842520in}{7.842520in}}%
\pgfusepath{clip}%
\pgfsetbuttcap%
\pgfsetroundjoin%
\definecolor{currentfill}{rgb}{0.120565,0.596422,0.543611}%
\pgfsetfillcolor{currentfill}%
\pgfsetlinewidth{0.000000pt}%
\definecolor{currentstroke}{rgb}{0.162142,0.474838,0.558140}%
\pgfsetstrokecolor{currentstroke}%
\pgfsetdash{}{0pt}%
\pgfpathmoveto{\pgfqpoint{4.745380in}{3.138731in}}%
\pgfpathlineto{\pgfqpoint{4.828804in}{3.219577in}}%
\pgfpathlineto{\pgfqpoint{4.683466in}{3.421156in}}%
\pgfpathclose%
\pgfusepath{fill}%
\end{pgfscope}%
\begin{pgfscope}%
\pgfpathrectangle{\pgfqpoint{0.539299in}{0.078740in}}{\pgfqpoint{7.842520in}{7.842520in}}%
\pgfusepath{clip}%
\pgfsetbuttcap%
\pgfsetroundjoin%
\definecolor{currentfill}{rgb}{0.496615,0.826376,0.306377}%
\pgfsetfillcolor{currentfill}%
\pgfsetlinewidth{0.000000pt}%
\definecolor{currentstroke}{rgb}{0.160665,0.478540,0.558115}%
\pgfsetstrokecolor{currentstroke}%
\pgfsetdash{}{0pt}%
\pgfpathmoveto{\pgfqpoint{3.350268in}{4.277351in}}%
\pgfpathlineto{\pgfqpoint{3.289659in}{4.681825in}}%
\pgfpathlineto{\pgfqpoint{3.206070in}{4.367077in}}%
\pgfpathclose%
\pgfusepath{fill}%
\end{pgfscope}%
\begin{pgfscope}%
\pgfpathrectangle{\pgfqpoint{0.539299in}{0.078740in}}{\pgfqpoint{7.842520in}{7.842520in}}%
\pgfusepath{clip}%
\pgfsetbuttcap%
\pgfsetroundjoin%
\definecolor{currentfill}{rgb}{0.636902,0.856542,0.216620}%
\pgfsetfillcolor{currentfill}%
\pgfsetlinewidth{0.000000pt}%
\definecolor{currentstroke}{rgb}{0.159194,0.482237,0.558073}%
\pgfsetstrokecolor{currentstroke}%
\pgfsetdash{}{0pt}%
\pgfpathmoveto{\pgfqpoint{3.289659in}{4.681825in}}%
\pgfpathlineto{\pgfqpoint{3.434182in}{4.580061in}}%
\pgfpathlineto{\pgfqpoint{3.519195in}{4.806876in}}%
\pgfpathclose%
\pgfusepath{fill}%
\end{pgfscope}%
\begin{pgfscope}%
\pgfpathrectangle{\pgfqpoint{0.539299in}{0.078740in}}{\pgfqpoint{7.842520in}{7.842520in}}%
\pgfusepath{clip}%
\pgfsetbuttcap%
\pgfsetroundjoin%
\definecolor{currentfill}{rgb}{0.143303,0.669459,0.511215}%
\pgfsetfillcolor{currentfill}%
\pgfsetlinewidth{0.000000pt}%
\definecolor{currentstroke}{rgb}{0.157729,0.485932,0.558013}%
\pgfsetstrokecolor{currentstroke}%
\pgfsetdash{}{0pt}%
\pgfpathmoveto{\pgfqpoint{3.474971in}{3.252817in}}%
\pgfpathlineto{\pgfqpoint{3.556804in}{3.697996in}}%
\pgfpathlineto{\pgfqpoint{3.412099in}{3.800510in}}%
\pgfpathclose%
\pgfusepath{fill}%
\end{pgfscope}%
\begin{pgfscope}%
\pgfpathrectangle{\pgfqpoint{0.539299in}{0.078740in}}{\pgfqpoint{7.842520in}{7.842520in}}%
\pgfusepath{clip}%
\pgfsetbuttcap%
\pgfsetroundjoin%
\definecolor{currentfill}{rgb}{0.143343,0.522773,0.556295}%
\pgfsetfillcolor{currentfill}%
\pgfsetlinewidth{0.000000pt}%
\definecolor{currentstroke}{rgb}{0.156270,0.489624,0.557936}%
\pgfsetstrokecolor{currentstroke}%
\pgfsetdash{}{0pt}%
\pgfpathmoveto{\pgfqpoint{4.973989in}{3.014923in}}%
\pgfpathlineto{\pgfqpoint{4.891094in}{2.947957in}}%
\pgfpathlineto{\pgfqpoint{5.036687in}{2.752478in}}%
\pgfpathclose%
\pgfusepath{fill}%
\end{pgfscope}%
\begin{pgfscope}%
\pgfpathrectangle{\pgfqpoint{0.539299in}{0.078740in}}{\pgfqpoint{7.842520in}{7.842520in}}%
\pgfusepath{clip}%
\pgfsetbuttcap%
\pgfsetroundjoin%
\definecolor{currentfill}{rgb}{0.124780,0.640461,0.527068}%
\pgfsetfillcolor{currentfill}%
\pgfsetlinewidth{0.000000pt}%
\definecolor{currentstroke}{rgb}{0.154815,0.493313,0.557840}%
\pgfsetstrokecolor{currentstroke}%
\pgfsetdash{}{0pt}%
\pgfpathmoveto{\pgfqpoint{4.599574in}{3.325252in}}%
\pgfpathlineto{\pgfqpoint{4.683466in}{3.421156in}}%
\pgfpathlineto{\pgfqpoint{4.537995in}{3.619381in}}%
\pgfpathclose%
\pgfusepath{fill}%
\end{pgfscope}%
\begin{pgfscope}%
\pgfpathrectangle{\pgfqpoint{0.539299in}{0.078740in}}{\pgfqpoint{7.842520in}{7.842520in}}%
\pgfusepath{clip}%
\pgfsetbuttcap%
\pgfsetroundjoin%
\definecolor{currentfill}{rgb}{0.259857,0.745492,0.444467}%
\pgfsetfillcolor{currentfill}%
\pgfsetlinewidth{0.000000pt}%
\definecolor{currentstroke}{rgb}{0.153364,0.497000,0.557724}%
\pgfsetstrokecolor{currentstroke}%
\pgfsetdash{}{0pt}%
\pgfpathmoveto{\pgfqpoint{3.267846in}{3.892059in}}%
\pgfpathlineto{\pgfqpoint{3.412099in}{3.800510in}}%
\pgfpathlineto{\pgfqpoint{3.494996in}{4.172102in}}%
\pgfpathclose%
\pgfusepath{fill}%
\end{pgfscope}%
\begin{pgfscope}%
\pgfpathrectangle{\pgfqpoint{0.539299in}{0.078740in}}{\pgfqpoint{7.842520in}{7.842520in}}%
\pgfusepath{clip}%
\pgfsetbuttcap%
\pgfsetroundjoin%
\definecolor{currentfill}{rgb}{0.147607,0.511733,0.557049}%
\pgfsetfillcolor{currentfill}%
\pgfsetlinewidth{0.000000pt}%
\definecolor{currentstroke}{rgb}{0.151918,0.500685,0.557587}%
\pgfsetstrokecolor{currentstroke}%
\pgfsetdash{}{0pt}%
\pgfpathmoveto{\pgfqpoint{3.909430in}{2.951944in}}%
\pgfpathlineto{\pgfqpoint{3.764278in}{3.057833in}}%
\pgfpathlineto{\pgfqpoint{3.682755in}{2.554388in}}%
\pgfpathclose%
\pgfusepath{fill}%
\end{pgfscope}%
\begin{pgfscope}%
\pgfpathrectangle{\pgfqpoint{0.539299in}{0.078740in}}{\pgfqpoint{7.842520in}{7.842520in}}%
\pgfusepath{clip}%
\pgfsetbuttcap%
\pgfsetroundjoin%
\definecolor{currentfill}{rgb}{0.535621,0.835785,0.281908}%
\pgfsetfillcolor{currentfill}%
\pgfsetlinewidth{0.000000pt}%
\definecolor{currentstroke}{rgb}{0.150476,0.504369,0.557430}%
\pgfsetstrokecolor{currentstroke}%
\pgfsetdash{}{0pt}%
\pgfpathmoveto{\pgfqpoint{3.434182in}{4.580061in}}%
\pgfpathlineto{\pgfqpoint{3.289659in}{4.681825in}}%
\pgfpathlineto{\pgfqpoint{3.350268in}{4.277351in}}%
\pgfpathclose%
\pgfusepath{fill}%
\end{pgfscope}%
\begin{pgfscope}%
\pgfpathrectangle{\pgfqpoint{0.539299in}{0.078740in}}{\pgfqpoint{7.842520in}{7.842520in}}%
\pgfusepath{clip}%
\pgfsetbuttcap%
\pgfsetroundjoin%
\definecolor{currentfill}{rgb}{0.119483,0.614817,0.537692}%
\pgfsetfillcolor{currentfill}%
\pgfsetlinewidth{0.000000pt}%
\definecolor{currentstroke}{rgb}{0.149039,0.508051,0.557250}%
\pgfsetstrokecolor{currentstroke}%
\pgfsetdash{}{0pt}%
\pgfpathmoveto{\pgfqpoint{3.474971in}{3.252817in}}%
\pgfpathlineto{\pgfqpoint{3.619444in}{3.158415in}}%
\pgfpathlineto{\pgfqpoint{3.701876in}{3.586183in}}%
\pgfpathclose%
\pgfusepath{fill}%
\end{pgfscope}%
\begin{pgfscope}%
\pgfpathrectangle{\pgfqpoint{0.539299in}{0.078740in}}{\pgfqpoint{7.842520in}{7.842520in}}%
\pgfusepath{clip}%
\pgfsetbuttcap%
\pgfsetroundjoin%
\definecolor{currentfill}{rgb}{0.327796,0.773980,0.406640}%
\pgfsetfillcolor{currentfill}%
\pgfsetlinewidth{0.000000pt}%
\definecolor{currentstroke}{rgb}{0.147607,0.511733,0.557049}%
\pgfsetstrokecolor{currentstroke}%
\pgfsetdash{}{0pt}%
\pgfpathmoveto{\pgfqpoint{3.494996in}{4.172102in}}%
\pgfpathlineto{\pgfqpoint{3.350268in}{4.277351in}}%
\pgfpathlineto{\pgfqpoint{3.267846in}{3.892059in}}%
\pgfpathclose%
\pgfusepath{fill}%
\end{pgfscope}%
\begin{pgfscope}%
\pgfpathrectangle{\pgfqpoint{0.539299in}{0.078740in}}{\pgfqpoint{7.842520in}{7.842520in}}%
\pgfusepath{clip}%
\pgfsetbuttcap%
\pgfsetroundjoin%
\definecolor{currentfill}{rgb}{0.616293,0.852709,0.230052}%
\pgfsetfillcolor{currentfill}%
\pgfsetlinewidth{0.000000pt}%
\definecolor{currentstroke}{rgb}{0.146180,0.515413,0.556823}%
\pgfsetstrokecolor{currentstroke}%
\pgfsetdash{}{0pt}%
\pgfpathmoveto{\pgfqpoint{3.579199in}{4.460532in}}%
\pgfpathlineto{\pgfqpoint{3.664322in}{4.672860in}}%
\pgfpathlineto{\pgfqpoint{3.519195in}{4.806876in}}%
\pgfpathclose%
\pgfusepath{fill}%
\end{pgfscope}%
\begin{pgfscope}%
\pgfpathrectangle{\pgfqpoint{0.539299in}{0.078740in}}{\pgfqpoint{7.842520in}{7.842520in}}%
\pgfusepath{clip}%
\pgfsetbuttcap%
\pgfsetroundjoin%
\definecolor{currentfill}{rgb}{0.168126,0.459988,0.558082}%
\pgfsetfillcolor{currentfill}%
\pgfsetlinewidth{0.000000pt}%
\definecolor{currentstroke}{rgb}{0.144759,0.519093,0.556572}%
\pgfsetstrokecolor{currentstroke}%
\pgfsetdash{}{0pt}%
\pgfpathmoveto{\pgfqpoint{3.909430in}{2.951944in}}%
\pgfpathlineto{\pgfqpoint{3.827198in}{2.468870in}}%
\pgfpathlineto{\pgfqpoint{3.971998in}{2.380078in}}%
\pgfpathclose%
\pgfusepath{fill}%
\end{pgfscope}%
\begin{pgfscope}%
\pgfpathrectangle{\pgfqpoint{0.539299in}{0.078740in}}{\pgfqpoint{7.842520in}{7.842520in}}%
\pgfusepath{clip}%
\pgfsetbuttcap%
\pgfsetroundjoin%
\definecolor{currentfill}{rgb}{0.221989,0.339161,0.548752}%
\pgfsetfillcolor{currentfill}%
\pgfsetlinewidth{0.000000pt}%
\definecolor{currentstroke}{rgb}{0.143343,0.522773,0.556295}%
\pgfsetstrokecolor{currentstroke}%
\pgfsetdash{}{0pt}%
\pgfpathmoveto{\pgfqpoint{4.117136in}{2.288294in}}%
\pgfpathlineto{\pgfqpoint{4.034810in}{1.751253in}}%
\pgfpathlineto{\pgfqpoint{4.262595in}{2.193629in}}%
\pgfpathclose%
\pgfusepath{fill}%
\end{pgfscope}%
\begin{pgfscope}%
\pgfpathrectangle{\pgfqpoint{0.539299in}{0.078740in}}{\pgfqpoint{7.842520in}{7.842520in}}%
\pgfusepath{clip}%
\pgfsetbuttcap%
\pgfsetroundjoin%
\definecolor{currentfill}{rgb}{0.606045,0.850733,0.236712}%
\pgfsetfillcolor{currentfill}%
\pgfsetlinewidth{0.000000pt}%
\definecolor{currentstroke}{rgb}{0.141935,0.526453,0.555991}%
\pgfsetstrokecolor{currentstroke}%
\pgfsetdash{}{0pt}%
\pgfpathmoveto{\pgfqpoint{3.519195in}{4.806876in}}%
\pgfpathlineto{\pgfqpoint{3.434182in}{4.580061in}}%
\pgfpathlineto{\pgfqpoint{3.579199in}{4.460532in}}%
\pgfpathclose%
\pgfusepath{fill}%
\end{pgfscope}%
\begin{pgfscope}%
\pgfpathrectangle{\pgfqpoint{0.539299in}{0.078740in}}{\pgfqpoint{7.842520in}{7.842520in}}%
\pgfusepath{clip}%
\pgfsetbuttcap%
\pgfsetroundjoin%
\definecolor{currentfill}{rgb}{0.127568,0.566949,0.550556}%
\pgfsetfillcolor{currentfill}%
\pgfsetlinewidth{0.000000pt}%
\definecolor{currentstroke}{rgb}{0.140536,0.530132,0.555659}%
\pgfsetstrokecolor{currentstroke}%
\pgfsetdash{}{0pt}%
\pgfpathmoveto{\pgfqpoint{4.745380in}{3.138731in}}%
\pgfpathlineto{\pgfqpoint{4.891094in}{2.947957in}}%
\pgfpathlineto{\pgfqpoint{4.828804in}{3.219577in}}%
\pgfpathclose%
\pgfusepath{fill}%
\end{pgfscope}%
\begin{pgfscope}%
\pgfpathrectangle{\pgfqpoint{0.539299in}{0.078740in}}{\pgfqpoint{7.842520in}{7.842520in}}%
\pgfusepath{clip}%
\pgfsetbuttcap%
\pgfsetroundjoin%
\definecolor{currentfill}{rgb}{0.175707,0.697900,0.491033}%
\pgfsetfillcolor{currentfill}%
\pgfsetlinewidth{0.000000pt}%
\definecolor{currentstroke}{rgb}{0.139147,0.533812,0.555298}%
\pgfsetstrokecolor{currentstroke}%
\pgfsetdash{}{0pt}%
\pgfpathmoveto{\pgfqpoint{4.537995in}{3.619381in}}%
\pgfpathlineto{\pgfqpoint{4.392412in}{3.813568in}}%
\pgfpathlineto{\pgfqpoint{4.307790in}{3.685037in}}%
\pgfpathclose%
\pgfusepath{fill}%
\end{pgfscope}%
\begin{pgfscope}%
\pgfpathrectangle{\pgfqpoint{0.539299in}{0.078740in}}{\pgfqpoint{7.842520in}{7.842520in}}%
\pgfusepath{clip}%
\pgfsetbuttcap%
\pgfsetroundjoin%
\definecolor{currentfill}{rgb}{0.565498,0.842430,0.262877}%
\pgfsetfillcolor{currentfill}%
\pgfsetlinewidth{0.000000pt}%
\definecolor{currentstroke}{rgb}{0.137770,0.537492,0.554906}%
\pgfsetstrokecolor{currentstroke}%
\pgfsetdash{}{0pt}%
\pgfpathmoveto{\pgfqpoint{3.579199in}{4.460532in}}%
\pgfpathlineto{\pgfqpoint{3.809748in}{4.522655in}}%
\pgfpathlineto{\pgfqpoint{3.664322in}{4.672860in}}%
\pgfpathclose%
\pgfusepath{fill}%
\end{pgfscope}%
\begin{pgfscope}%
\pgfpathrectangle{\pgfqpoint{0.539299in}{0.078740in}}{\pgfqpoint{7.842520in}{7.842520in}}%
\pgfusepath{clip}%
\pgfsetbuttcap%
\pgfsetroundjoin%
\definecolor{currentfill}{rgb}{0.132268,0.655014,0.519661}%
\pgfsetfillcolor{currentfill}%
\pgfsetlinewidth{0.000000pt}%
\definecolor{currentstroke}{rgb}{0.136408,0.541173,0.554483}%
\pgfsetstrokecolor{currentstroke}%
\pgfsetdash{}{0pt}%
\pgfpathmoveto{\pgfqpoint{3.701876in}{3.586183in}}%
\pgfpathlineto{\pgfqpoint{3.556804in}{3.697996in}}%
\pgfpathlineto{\pgfqpoint{3.474971in}{3.252817in}}%
\pgfpathclose%
\pgfusepath{fill}%
\end{pgfscope}%
\begin{pgfscope}%
\pgfpathrectangle{\pgfqpoint{0.539299in}{0.078740in}}{\pgfqpoint{7.842520in}{7.842520in}}%
\pgfusepath{clip}%
\pgfsetbuttcap%
\pgfsetroundjoin%
\definecolor{currentfill}{rgb}{0.203063,0.379716,0.553925}%
\pgfsetfillcolor{currentfill}%
\pgfsetlinewidth{0.000000pt}%
\definecolor{currentstroke}{rgb}{0.135066,0.544853,0.554029}%
\pgfsetstrokecolor{currentstroke}%
\pgfsetdash{}{0pt}%
\pgfpathmoveto{\pgfqpoint{5.472274in}{2.124924in}}%
\pgfpathlineto{\pgfqpoint{5.327340in}{2.342890in}}%
\pgfpathlineto{\pgfqpoint{5.245331in}{2.266200in}}%
\pgfpathclose%
\pgfusepath{fill}%
\end{pgfscope}%
\begin{pgfscope}%
\pgfpathrectangle{\pgfqpoint{0.539299in}{0.078740in}}{\pgfqpoint{7.842520in}{7.842520in}}%
\pgfusepath{clip}%
\pgfsetbuttcap%
\pgfsetroundjoin%
\definecolor{currentfill}{rgb}{0.248629,0.278775,0.534556}%
\pgfsetfillcolor{currentfill}%
\pgfsetlinewidth{0.000000pt}%
\definecolor{currentstroke}{rgb}{0.133743,0.548535,0.553541}%
\pgfsetstrokecolor{currentstroke}%
\pgfsetdash{}{0pt}%
\pgfpathmoveto{\pgfqpoint{4.179552in}{1.680684in}}%
\pgfpathlineto{\pgfqpoint{4.324695in}{1.608470in}}%
\pgfpathlineto{\pgfqpoint{4.262595in}{2.193629in}}%
\pgfpathclose%
\pgfusepath{fill}%
\end{pgfscope}%
\begin{pgfscope}%
\pgfpathrectangle{\pgfqpoint{0.539299in}{0.078740in}}{\pgfqpoint{7.842520in}{7.842520in}}%
\pgfusepath{clip}%
\pgfsetbuttcap%
\pgfsetroundjoin%
\definecolor{currentfill}{rgb}{0.449368,0.813768,0.335384}%
\pgfsetfillcolor{currentfill}%
\pgfsetlinewidth{0.000000pt}%
\definecolor{currentstroke}{rgb}{0.132444,0.552216,0.553018}%
\pgfsetstrokecolor{currentstroke}%
\pgfsetdash{}{0pt}%
\pgfpathmoveto{\pgfqpoint{3.434182in}{4.580061in}}%
\pgfpathlineto{\pgfqpoint{3.350268in}{4.277351in}}%
\pgfpathlineto{\pgfqpoint{3.494996in}{4.172102in}}%
\pgfpathclose%
\pgfusepath{fill}%
\end{pgfscope}%
\begin{pgfscope}%
\pgfpathrectangle{\pgfqpoint{0.539299in}{0.078740in}}{\pgfqpoint{7.842520in}{7.842520in}}%
\pgfusepath{clip}%
\pgfsetbuttcap%
\pgfsetroundjoin%
\definecolor{currentfill}{rgb}{0.239374,0.735588,0.455688}%
\pgfsetfillcolor{currentfill}%
\pgfsetlinewidth{0.000000pt}%
\definecolor{currentstroke}{rgb}{0.131172,0.555899,0.552459}%
\pgfsetstrokecolor{currentstroke}%
\pgfsetdash{}{0pt}%
\pgfpathmoveto{\pgfqpoint{4.161873in}{3.857033in}}%
\pgfpathlineto{\pgfqpoint{4.392412in}{3.813568in}}%
\pgfpathlineto{\pgfqpoint{4.246749in}{4.002642in}}%
\pgfpathclose%
\pgfusepath{fill}%
\end{pgfscope}%
\begin{pgfscope}%
\pgfpathrectangle{\pgfqpoint{0.539299in}{0.078740in}}{\pgfqpoint{7.842520in}{7.842520in}}%
\pgfusepath{clip}%
\pgfsetbuttcap%
\pgfsetroundjoin%
\definecolor{currentfill}{rgb}{0.119423,0.611141,0.538982}%
\pgfsetfillcolor{currentfill}%
\pgfsetlinewidth{0.000000pt}%
\definecolor{currentstroke}{rgb}{0.129933,0.559582,0.551864}%
\pgfsetstrokecolor{currentstroke}%
\pgfsetdash{}{0pt}%
\pgfpathmoveto{\pgfqpoint{4.683466in}{3.421156in}}%
\pgfpathlineto{\pgfqpoint{4.599574in}{3.325252in}}%
\pgfpathlineto{\pgfqpoint{4.745380in}{3.138731in}}%
\pgfpathclose%
\pgfusepath{fill}%
\end{pgfscope}%
\begin{pgfscope}%
\pgfpathrectangle{\pgfqpoint{0.539299in}{0.078740in}}{\pgfqpoint{7.842520in}{7.842520in}}%
\pgfusepath{clip}%
\pgfsetbuttcap%
\pgfsetroundjoin%
\definecolor{currentfill}{rgb}{0.239374,0.735588,0.455688}%
\pgfsetfillcolor{currentfill}%
\pgfsetlinewidth{0.000000pt}%
\definecolor{currentstroke}{rgb}{0.128729,0.563265,0.551229}%
\pgfsetstrokecolor{currentstroke}%
\pgfsetdash{}{0pt}%
\pgfpathmoveto{\pgfqpoint{3.494996in}{4.172102in}}%
\pgfpathlineto{\pgfqpoint{3.412099in}{3.800510in}}%
\pgfpathlineto{\pgfqpoint{3.556804in}{3.697996in}}%
\pgfpathclose%
\pgfusepath{fill}%
\end{pgfscope}%
\begin{pgfscope}%
\pgfpathrectangle{\pgfqpoint{0.539299in}{0.078740in}}{\pgfqpoint{7.842520in}{7.842520in}}%
\pgfusepath{clip}%
\pgfsetbuttcap%
\pgfsetroundjoin%
\definecolor{currentfill}{rgb}{0.311925,0.767822,0.415586}%
\pgfsetfillcolor{currentfill}%
\pgfsetlinewidth{0.000000pt}%
\definecolor{currentstroke}{rgb}{0.127568,0.566949,0.550556}%
\pgfsetstrokecolor{currentstroke}%
\pgfsetdash{}{0pt}%
\pgfpathmoveto{\pgfqpoint{4.246749in}{4.002642in}}%
\pgfpathlineto{\pgfqpoint{4.101044in}{4.185140in}}%
\pgfpathlineto{\pgfqpoint{4.015993in}{4.022293in}}%
\pgfpathclose%
\pgfusepath{fill}%
\end{pgfscope}%
\begin{pgfscope}%
\pgfpathrectangle{\pgfqpoint{0.539299in}{0.078740in}}{\pgfqpoint{7.842520in}{7.842520in}}%
\pgfusepath{clip}%
\pgfsetbuttcap%
\pgfsetroundjoin%
\definecolor{currentfill}{rgb}{0.119738,0.603785,0.541400}%
\pgfsetfillcolor{currentfill}%
\pgfsetlinewidth{0.000000pt}%
\definecolor{currentstroke}{rgb}{0.126453,0.570633,0.549841}%
\pgfsetstrokecolor{currentstroke}%
\pgfsetdash{}{0pt}%
\pgfpathmoveto{\pgfqpoint{3.619444in}{3.158415in}}%
\pgfpathlineto{\pgfqpoint{3.764278in}{3.057833in}}%
\pgfpathlineto{\pgfqpoint{3.701876in}{3.586183in}}%
\pgfpathclose%
\pgfusepath{fill}%
\end{pgfscope}%
\begin{pgfscope}%
\pgfpathrectangle{\pgfqpoint{0.539299in}{0.078740in}}{\pgfqpoint{7.842520in}{7.842520in}}%
\pgfusepath{clip}%
\pgfsetbuttcap%
\pgfsetroundjoin%
\definecolor{currentfill}{rgb}{0.458674,0.816363,0.329727}%
\pgfsetfillcolor{currentfill}%
\pgfsetlinewidth{0.000000pt}%
\definecolor{currentstroke}{rgb}{0.125394,0.574318,0.549086}%
\pgfsetstrokecolor{currentstroke}%
\pgfsetdash{}{0pt}%
\pgfpathmoveto{\pgfqpoint{3.809748in}{4.522655in}}%
\pgfpathlineto{\pgfqpoint{3.870205in}{4.179274in}}%
\pgfpathlineto{\pgfqpoint{3.955353in}{4.359218in}}%
\pgfpathclose%
\pgfusepath{fill}%
\end{pgfscope}%
\begin{pgfscope}%
\pgfpathrectangle{\pgfqpoint{0.539299in}{0.078740in}}{\pgfqpoint{7.842520in}{7.842520in}}%
\pgfusepath{clip}%
\pgfsetbuttcap%
\pgfsetroundjoin%
\definecolor{currentfill}{rgb}{0.278012,0.180367,0.486697}%
\pgfsetfillcolor{currentfill}%
\pgfsetlinewidth{0.000000pt}%
\definecolor{currentstroke}{rgb}{0.124395,0.578002,0.548287}%
\pgfsetstrokecolor{currentstroke}%
\pgfsetdash{}{0pt}%
\pgfpathmoveto{\pgfqpoint{5.680690in}{1.591684in}}%
\pgfpathlineto{\pgfqpoint{5.599643in}{1.509341in}}%
\pgfpathlineto{\pgfqpoint{5.824454in}{1.317143in}}%
\pgfpathclose%
\pgfusepath{fill}%
\end{pgfscope}%
\begin{pgfscope}%
\pgfpathrectangle{\pgfqpoint{0.539299in}{0.078740in}}{\pgfqpoint{7.842520in}{7.842520in}}%
\pgfusepath{clip}%
\pgfsetbuttcap%
\pgfsetroundjoin%
\definecolor{currentfill}{rgb}{0.377779,0.791781,0.377939}%
\pgfsetfillcolor{currentfill}%
\pgfsetlinewidth{0.000000pt}%
\definecolor{currentstroke}{rgb}{0.123463,0.581687,0.547445}%
\pgfsetstrokecolor{currentstroke}%
\pgfsetdash{}{0pt}%
\pgfpathmoveto{\pgfqpoint{4.015993in}{4.022293in}}%
\pgfpathlineto{\pgfqpoint{4.101044in}{4.185140in}}%
\pgfpathlineto{\pgfqpoint{3.955353in}{4.359218in}}%
\pgfpathclose%
\pgfusepath{fill}%
\end{pgfscope}%
\begin{pgfscope}%
\pgfpathrectangle{\pgfqpoint{0.539299in}{0.078740in}}{\pgfqpoint{7.842520in}{7.842520in}}%
\pgfusepath{clip}%
\pgfsetbuttcap%
\pgfsetroundjoin%
\definecolor{currentfill}{rgb}{0.487026,0.823929,0.312321}%
\pgfsetfillcolor{currentfill}%
\pgfsetlinewidth{0.000000pt}%
\definecolor{currentstroke}{rgb}{0.122606,0.585371,0.546557}%
\pgfsetstrokecolor{currentstroke}%
\pgfsetdash{}{0pt}%
\pgfpathmoveto{\pgfqpoint{3.494996in}{4.172102in}}%
\pgfpathlineto{\pgfqpoint{3.579199in}{4.460532in}}%
\pgfpathlineto{\pgfqpoint{3.434182in}{4.580061in}}%
\pgfpathclose%
\pgfusepath{fill}%
\end{pgfscope}%
\begin{pgfscope}%
\pgfpathrectangle{\pgfqpoint{0.539299in}{0.078740in}}{\pgfqpoint{7.842520in}{7.842520in}}%
\pgfusepath{clip}%
\pgfsetbuttcap%
\pgfsetroundjoin%
\definecolor{currentfill}{rgb}{0.227802,0.326594,0.546532}%
\pgfsetfillcolor{currentfill}%
\pgfsetlinewidth{0.000000pt}%
\definecolor{currentstroke}{rgb}{0.121831,0.589055,0.545623}%
\pgfsetstrokecolor{currentstroke}%
\pgfsetdash{}{0pt}%
\pgfpathmoveto{\pgfqpoint{5.390896in}{2.059660in}}%
\pgfpathlineto{\pgfqpoint{5.536081in}{1.836733in}}%
\pgfpathlineto{\pgfqpoint{5.472274in}{2.124924in}}%
\pgfpathclose%
\pgfusepath{fill}%
\end{pgfscope}%
\begin{pgfscope}%
\pgfpathrectangle{\pgfqpoint{0.539299in}{0.078740in}}{\pgfqpoint{7.842520in}{7.842520in}}%
\pgfusepath{clip}%
\pgfsetbuttcap%
\pgfsetroundjoin%
\definecolor{currentfill}{rgb}{0.506271,0.828786,0.300362}%
\pgfsetfillcolor{currentfill}%
\pgfsetlinewidth{0.000000pt}%
\definecolor{currentstroke}{rgb}{0.121148,0.592739,0.544641}%
\pgfsetstrokecolor{currentstroke}%
\pgfsetdash{}{0pt}%
\pgfpathmoveto{\pgfqpoint{3.724577in}{4.326091in}}%
\pgfpathlineto{\pgfqpoint{3.809748in}{4.522655in}}%
\pgfpathlineto{\pgfqpoint{3.579199in}{4.460532in}}%
\pgfpathclose%
\pgfusepath{fill}%
\end{pgfscope}%
\begin{pgfscope}%
\pgfpathrectangle{\pgfqpoint{0.539299in}{0.078740in}}{\pgfqpoint{7.842520in}{7.842520in}}%
\pgfusepath{clip}%
\pgfsetbuttcap%
\pgfsetroundjoin%
\definecolor{currentfill}{rgb}{0.130067,0.651384,0.521608}%
\pgfsetfillcolor{currentfill}%
\pgfsetlinewidth{0.000000pt}%
\definecolor{currentstroke}{rgb}{0.120565,0.596422,0.543611}%
\pgfsetstrokecolor{currentstroke}%
\pgfsetdash{}{0pt}%
\pgfpathmoveto{\pgfqpoint{4.537995in}{3.619381in}}%
\pgfpathlineto{\pgfqpoint{4.453701in}{3.507500in}}%
\pgfpathlineto{\pgfqpoint{4.599574in}{3.325252in}}%
\pgfpathclose%
\pgfusepath{fill}%
\end{pgfscope}%
\begin{pgfscope}%
\pgfpathrectangle{\pgfqpoint{0.539299in}{0.078740in}}{\pgfqpoint{7.842520in}{7.842520in}}%
\pgfusepath{clip}%
\pgfsetbuttcap%
\pgfsetroundjoin%
\definecolor{currentfill}{rgb}{0.180629,0.429975,0.557282}%
\pgfsetfillcolor{currentfill}%
\pgfsetlinewidth{0.000000pt}%
\definecolor{currentstroke}{rgb}{0.120092,0.600104,0.542530}%
\pgfsetstrokecolor{currentstroke}%
\pgfsetdash{}{0pt}%
\pgfpathmoveto{\pgfqpoint{5.327340in}{2.342890in}}%
\pgfpathlineto{\pgfqpoint{5.182120in}{2.551317in}}%
\pgfpathlineto{\pgfqpoint{5.099519in}{2.460623in}}%
\pgfpathclose%
\pgfusepath{fill}%
\end{pgfscope}%
\begin{pgfscope}%
\pgfpathrectangle{\pgfqpoint{0.539299in}{0.078740in}}{\pgfqpoint{7.842520in}{7.842520in}}%
\pgfusepath{clip}%
\pgfsetbuttcap%
\pgfsetroundjoin%
\definecolor{currentfill}{rgb}{0.153894,0.680203,0.504172}%
\pgfsetfillcolor{currentfill}%
\pgfsetlinewidth{0.000000pt}%
\definecolor{currentstroke}{rgb}{0.119738,0.603785,0.541400}%
\pgfsetstrokecolor{currentstroke}%
\pgfsetdash{}{0pt}%
\pgfpathmoveto{\pgfqpoint{4.307790in}{3.685037in}}%
\pgfpathlineto{\pgfqpoint{4.453701in}{3.507500in}}%
\pgfpathlineto{\pgfqpoint{4.537995in}{3.619381in}}%
\pgfpathclose%
\pgfusepath{fill}%
\end{pgfscope}%
\begin{pgfscope}%
\pgfpathrectangle{\pgfqpoint{0.539299in}{0.078740in}}{\pgfqpoint{7.842520in}{7.842520in}}%
\pgfusepath{clip}%
\pgfsetbuttcap%
\pgfsetroundjoin%
\definecolor{currentfill}{rgb}{0.458674,0.816363,0.329727}%
\pgfsetfillcolor{currentfill}%
\pgfsetlinewidth{0.000000pt}%
\definecolor{currentstroke}{rgb}{0.119512,0.607464,0.540218}%
\pgfsetstrokecolor{currentstroke}%
\pgfsetdash{}{0pt}%
\pgfpathmoveto{\pgfqpoint{3.724577in}{4.326091in}}%
\pgfpathlineto{\pgfqpoint{3.870205in}{4.179274in}}%
\pgfpathlineto{\pgfqpoint{3.809748in}{4.522655in}}%
\pgfpathclose%
\pgfusepath{fill}%
\end{pgfscope}%
\begin{pgfscope}%
\pgfpathrectangle{\pgfqpoint{0.539299in}{0.078740in}}{\pgfqpoint{7.842520in}{7.842520in}}%
\pgfusepath{clip}%
\pgfsetbuttcap%
\pgfsetroundjoin%
\definecolor{currentfill}{rgb}{0.449368,0.813768,0.335384}%
\pgfsetfillcolor{currentfill}%
\pgfsetlinewidth{0.000000pt}%
\definecolor{currentstroke}{rgb}{0.119423,0.611141,0.538982}%
\pgfsetstrokecolor{currentstroke}%
\pgfsetdash{}{0pt}%
\pgfpathmoveto{\pgfqpoint{3.724577in}{4.326091in}}%
\pgfpathlineto{\pgfqpoint{3.579199in}{4.460532in}}%
\pgfpathlineto{\pgfqpoint{3.494996in}{4.172102in}}%
\pgfpathclose%
\pgfusepath{fill}%
\end{pgfscope}%
\begin{pgfscope}%
\pgfpathrectangle{\pgfqpoint{0.539299in}{0.078740in}}{\pgfqpoint{7.842520in}{7.842520in}}%
\pgfusepath{clip}%
\pgfsetbuttcap%
\pgfsetroundjoin%
\definecolor{currentfill}{rgb}{0.208030,0.718701,0.472873}%
\pgfsetfillcolor{currentfill}%
\pgfsetlinewidth{0.000000pt}%
\definecolor{currentstroke}{rgb}{0.119483,0.614817,0.537692}%
\pgfsetstrokecolor{currentstroke}%
\pgfsetdash{}{0pt}%
\pgfpathmoveto{\pgfqpoint{4.307790in}{3.685037in}}%
\pgfpathlineto{\pgfqpoint{4.392412in}{3.813568in}}%
\pgfpathlineto{\pgfqpoint{4.161873in}{3.857033in}}%
\pgfpathclose%
\pgfusepath{fill}%
\end{pgfscope}%
\begin{pgfscope}%
\pgfpathrectangle{\pgfqpoint{0.539299in}{0.078740in}}{\pgfqpoint{7.842520in}{7.842520in}}%
\pgfusepath{clip}%
\pgfsetbuttcap%
\pgfsetroundjoin%
\definecolor{currentfill}{rgb}{0.281477,0.755203,0.432552}%
\pgfsetfillcolor{currentfill}%
\pgfsetlinewidth{0.000000pt}%
\definecolor{currentstroke}{rgb}{0.119699,0.618490,0.536347}%
\pgfsetstrokecolor{currentstroke}%
\pgfsetdash{}{0pt}%
\pgfpathmoveto{\pgfqpoint{3.556804in}{3.697996in}}%
\pgfpathlineto{\pgfqpoint{3.640137in}{4.053707in}}%
\pgfpathlineto{\pgfqpoint{3.494996in}{4.172102in}}%
\pgfpathclose%
\pgfusepath{fill}%
\end{pgfscope}%
\begin{pgfscope}%
\pgfpathrectangle{\pgfqpoint{0.539299in}{0.078740in}}{\pgfqpoint{7.842520in}{7.842520in}}%
\pgfusepath{clip}%
\pgfsetbuttcap%
\pgfsetroundjoin%
\definecolor{currentfill}{rgb}{0.210503,0.363727,0.552206}%
\pgfsetfillcolor{currentfill}%
\pgfsetlinewidth{0.000000pt}%
\definecolor{currentstroke}{rgb}{0.120081,0.622161,0.534946}%
\pgfsetstrokecolor{currentstroke}%
\pgfsetdash{}{0pt}%
\pgfpathmoveto{\pgfqpoint{5.245331in}{2.266200in}}%
\pgfpathlineto{\pgfqpoint{5.390896in}{2.059660in}}%
\pgfpathlineto{\pgfqpoint{5.472274in}{2.124924in}}%
\pgfpathclose%
\pgfusepath{fill}%
\end{pgfscope}%
\begin{pgfscope}%
\pgfpathrectangle{\pgfqpoint{0.539299in}{0.078740in}}{\pgfqpoint{7.842520in}{7.842520in}}%
\pgfusepath{clip}%
\pgfsetbuttcap%
\pgfsetroundjoin%
\definecolor{currentfill}{rgb}{0.281887,0.150881,0.465405}%
\pgfsetfillcolor{currentfill}%
\pgfsetlinewidth{0.000000pt}%
\definecolor{currentstroke}{rgb}{0.120638,0.625828,0.533488}%
\pgfsetstrokecolor{currentstroke}%
\pgfsetdash{}{0pt}%
\pgfpathmoveto{\pgfqpoint{5.824454in}{1.317143in}}%
\pgfpathlineto{\pgfqpoint{5.599643in}{1.509341in}}%
\pgfpathlineto{\pgfqpoint{5.744062in}{1.240624in}}%
\pgfpathclose%
\pgfusepath{fill}%
\end{pgfscope}%
\begin{pgfscope}%
\pgfpathrectangle{\pgfqpoint{0.539299in}{0.078740in}}{\pgfqpoint{7.842520in}{7.842520in}}%
\pgfusepath{clip}%
\pgfsetbuttcap%
\pgfsetroundjoin%
\definecolor{currentfill}{rgb}{0.274149,0.751988,0.436601}%
\pgfsetfillcolor{currentfill}%
\pgfsetlinewidth{0.000000pt}%
\definecolor{currentstroke}{rgb}{0.121380,0.629492,0.531973}%
\pgfsetstrokecolor{currentstroke}%
\pgfsetdash{}{0pt}%
\pgfpathmoveto{\pgfqpoint{4.015993in}{4.022293in}}%
\pgfpathlineto{\pgfqpoint{4.161873in}{3.857033in}}%
\pgfpathlineto{\pgfqpoint{4.246749in}{4.002642in}}%
\pgfpathclose%
\pgfusepath{fill}%
\end{pgfscope}%
\begin{pgfscope}%
\pgfpathrectangle{\pgfqpoint{0.539299in}{0.078740in}}{\pgfqpoint{7.842520in}{7.842520in}}%
\pgfusepath{clip}%
\pgfsetbuttcap%
\pgfsetroundjoin%
\definecolor{currentfill}{rgb}{0.154815,0.493313,0.557840}%
\pgfsetfillcolor{currentfill}%
\pgfsetlinewidth{0.000000pt}%
\definecolor{currentstroke}{rgb}{0.122312,0.633153,0.530398}%
\pgfsetstrokecolor{currentstroke}%
\pgfsetdash{}{0pt}%
\pgfpathmoveto{\pgfqpoint{3.971998in}{2.380078in}}%
\pgfpathlineto{\pgfqpoint{4.054864in}{2.841426in}}%
\pgfpathlineto{\pgfqpoint{3.909430in}{2.951944in}}%
\pgfpathclose%
\pgfusepath{fill}%
\end{pgfscope}%
\begin{pgfscope}%
\pgfpathrectangle{\pgfqpoint{0.539299in}{0.078740in}}{\pgfqpoint{7.842520in}{7.842520in}}%
\pgfusepath{clip}%
\pgfsetbuttcap%
\pgfsetroundjoin%
\definecolor{currentfill}{rgb}{0.377779,0.791781,0.377939}%
\pgfsetfillcolor{currentfill}%
\pgfsetlinewidth{0.000000pt}%
\definecolor{currentstroke}{rgb}{0.123444,0.636809,0.528763}%
\pgfsetstrokecolor{currentstroke}%
\pgfsetdash{}{0pt}%
\pgfpathmoveto{\pgfqpoint{3.955353in}{4.359218in}}%
\pgfpathlineto{\pgfqpoint{3.870205in}{4.179274in}}%
\pgfpathlineto{\pgfqpoint{4.015993in}{4.022293in}}%
\pgfpathclose%
\pgfusepath{fill}%
\end{pgfscope}%
\begin{pgfscope}%
\pgfpathrectangle{\pgfqpoint{0.539299in}{0.078740in}}{\pgfqpoint{7.842520in}{7.842520in}}%
\pgfusepath{clip}%
\pgfsetbuttcap%
\pgfsetroundjoin%
\definecolor{currentfill}{rgb}{0.162142,0.474838,0.558140}%
\pgfsetfillcolor{currentfill}%
\pgfsetlinewidth{0.000000pt}%
\definecolor{currentstroke}{rgb}{0.124780,0.640461,0.527068}%
\pgfsetstrokecolor{currentstroke}%
\pgfsetdash{}{0pt}%
\pgfpathmoveto{\pgfqpoint{5.182120in}{2.551317in}}%
\pgfpathlineto{\pgfqpoint{5.036687in}{2.752478in}}%
\pgfpathlineto{\pgfqpoint{4.953553in}{2.645992in}}%
\pgfpathclose%
\pgfusepath{fill}%
\end{pgfscope}%
\begin{pgfscope}%
\pgfpathrectangle{\pgfqpoint{0.539299in}{0.078740in}}{\pgfqpoint{7.842520in}{7.842520in}}%
\pgfusepath{clip}%
\pgfsetbuttcap%
\pgfsetroundjoin%
\definecolor{currentfill}{rgb}{0.177423,0.437527,0.557565}%
\pgfsetfillcolor{currentfill}%
\pgfsetlinewidth{0.000000pt}%
\definecolor{currentstroke}{rgb}{0.126326,0.644107,0.525311}%
\pgfsetstrokecolor{currentstroke}%
\pgfsetdash{}{0pt}%
\pgfpathmoveto{\pgfqpoint{3.971998in}{2.380078in}}%
\pgfpathlineto{\pgfqpoint{4.117136in}{2.288294in}}%
\pgfpathlineto{\pgfqpoint{4.200551in}{2.726746in}}%
\pgfpathclose%
\pgfusepath{fill}%
\end{pgfscope}%
\begin{pgfscope}%
\pgfpathrectangle{\pgfqpoint{0.539299in}{0.078740in}}{\pgfqpoint{7.842520in}{7.842520in}}%
\pgfusepath{clip}%
\pgfsetbuttcap%
\pgfsetroundjoin%
\definecolor{currentfill}{rgb}{0.202219,0.715272,0.476084}%
\pgfsetfillcolor{currentfill}%
\pgfsetlinewidth{0.000000pt}%
\definecolor{currentstroke}{rgb}{0.128087,0.647749,0.523491}%
\pgfsetstrokecolor{currentstroke}%
\pgfsetdash{}{0pt}%
\pgfpathmoveto{\pgfqpoint{3.701876in}{3.586183in}}%
\pgfpathlineto{\pgfqpoint{3.640137in}{4.053707in}}%
\pgfpathlineto{\pgfqpoint{3.556804in}{3.697996in}}%
\pgfpathclose%
\pgfusepath{fill}%
\end{pgfscope}%
\begin{pgfscope}%
\pgfpathrectangle{\pgfqpoint{0.539299in}{0.078740in}}{\pgfqpoint{7.842520in}{7.842520in}}%
\pgfusepath{clip}%
\pgfsetbuttcap%
\pgfsetroundjoin%
\definecolor{currentfill}{rgb}{0.377779,0.791781,0.377939}%
\pgfsetfillcolor{currentfill}%
\pgfsetlinewidth{0.000000pt}%
\definecolor{currentstroke}{rgb}{0.130067,0.651384,0.521608}%
\pgfsetstrokecolor{currentstroke}%
\pgfsetdash{}{0pt}%
\pgfpathmoveto{\pgfqpoint{3.494996in}{4.172102in}}%
\pgfpathlineto{\pgfqpoint{3.640137in}{4.053707in}}%
\pgfpathlineto{\pgfqpoint{3.724577in}{4.326091in}}%
\pgfpathclose%
\pgfusepath{fill}%
\end{pgfscope}%
\begin{pgfscope}%
\pgfpathrectangle{\pgfqpoint{0.539299in}{0.078740in}}{\pgfqpoint{7.842520in}{7.842520in}}%
\pgfusepath{clip}%
\pgfsetbuttcap%
\pgfsetroundjoin%
\definecolor{currentfill}{rgb}{0.122606,0.585371,0.546557}%
\pgfsetfillcolor{currentfill}%
\pgfsetlinewidth{0.000000pt}%
\definecolor{currentstroke}{rgb}{0.132268,0.655014,0.519661}%
\pgfsetstrokecolor{currentstroke}%
\pgfsetdash{}{0pt}%
\pgfpathmoveto{\pgfqpoint{3.847250in}{3.466509in}}%
\pgfpathlineto{\pgfqpoint{3.764278in}{3.057833in}}%
\pgfpathlineto{\pgfqpoint{3.909430in}{2.951944in}}%
\pgfpathclose%
\pgfusepath{fill}%
\end{pgfscope}%
\begin{pgfscope}%
\pgfpathrectangle{\pgfqpoint{0.539299in}{0.078740in}}{\pgfqpoint{7.842520in}{7.842520in}}%
\pgfusepath{clip}%
\pgfsetbuttcap%
\pgfsetroundjoin%
\definecolor{currentfill}{rgb}{0.122312,0.633153,0.530398}%
\pgfsetfillcolor{currentfill}%
\pgfsetlinewidth{0.000000pt}%
\definecolor{currentstroke}{rgb}{0.134692,0.658636,0.517649}%
\pgfsetstrokecolor{currentstroke}%
\pgfsetdash{}{0pt}%
\pgfpathmoveto{\pgfqpoint{3.701876in}{3.586183in}}%
\pgfpathlineto{\pgfqpoint{3.764278in}{3.057833in}}%
\pgfpathlineto{\pgfqpoint{3.847250in}{3.466509in}}%
\pgfpathclose%
\pgfusepath{fill}%
\end{pgfscope}%
\begin{pgfscope}%
\pgfpathrectangle{\pgfqpoint{0.539299in}{0.078740in}}{\pgfqpoint{7.842520in}{7.842520in}}%
\pgfusepath{clip}%
\pgfsetbuttcap%
\pgfsetroundjoin%
\definecolor{currentfill}{rgb}{0.263663,0.237631,0.518762}%
\pgfsetfillcolor{currentfill}%
\pgfsetlinewidth{0.000000pt}%
\definecolor{currentstroke}{rgb}{0.137339,0.662252,0.515571}%
\pgfsetstrokecolor{currentstroke}%
\pgfsetdash{}{0pt}%
\pgfpathmoveto{\pgfqpoint{5.536081in}{1.836733in}}%
\pgfpathlineto{\pgfqpoint{5.599643in}{1.509341in}}%
\pgfpathlineto{\pgfqpoint{5.680690in}{1.591684in}}%
\pgfpathclose%
\pgfusepath{fill}%
\end{pgfscope}%
\begin{pgfscope}%
\pgfpathrectangle{\pgfqpoint{0.539299in}{0.078740in}}{\pgfqpoint{7.842520in}{7.842520in}}%
\pgfusepath{clip}%
\pgfsetbuttcap%
\pgfsetroundjoin%
\definecolor{currentfill}{rgb}{0.150476,0.504369,0.557430}%
\pgfsetfillcolor{currentfill}%
\pgfsetlinewidth{0.000000pt}%
\definecolor{currentstroke}{rgb}{0.140210,0.665859,0.513427}%
\pgfsetstrokecolor{currentstroke}%
\pgfsetdash{}{0pt}%
\pgfpathmoveto{\pgfqpoint{4.953553in}{2.645992in}}%
\pgfpathlineto{\pgfqpoint{5.036687in}{2.752478in}}%
\pgfpathlineto{\pgfqpoint{4.891094in}{2.947957in}}%
\pgfpathclose%
\pgfusepath{fill}%
\end{pgfscope}%
\begin{pgfscope}%
\pgfpathrectangle{\pgfqpoint{0.539299in}{0.078740in}}{\pgfqpoint{7.842520in}{7.842520in}}%
\pgfusepath{clip}%
\pgfsetbuttcap%
\pgfsetroundjoin%
\definecolor{currentfill}{rgb}{0.386433,0.794644,0.372886}%
\pgfsetfillcolor{currentfill}%
\pgfsetlinewidth{0.000000pt}%
\definecolor{currentstroke}{rgb}{0.143303,0.669459,0.511215}%
\pgfsetstrokecolor{currentstroke}%
\pgfsetdash{}{0pt}%
\pgfpathmoveto{\pgfqpoint{3.640137in}{4.053707in}}%
\pgfpathlineto{\pgfqpoint{3.870205in}{4.179274in}}%
\pgfpathlineto{\pgfqpoint{3.724577in}{4.326091in}}%
\pgfpathclose%
\pgfusepath{fill}%
\end{pgfscope}%
\begin{pgfscope}%
\pgfpathrectangle{\pgfqpoint{0.539299in}{0.078740in}}{\pgfqpoint{7.842520in}{7.842520in}}%
\pgfusepath{clip}%
\pgfsetbuttcap%
\pgfsetroundjoin%
\definecolor{currentfill}{rgb}{0.253935,0.265254,0.529983}%
\pgfsetfillcolor{currentfill}%
\pgfsetlinewidth{0.000000pt}%
\definecolor{currentstroke}{rgb}{0.146616,0.673050,0.508936}%
\pgfsetstrokecolor{currentstroke}%
\pgfsetdash{}{0pt}%
\pgfpathmoveto{\pgfqpoint{4.408361in}{2.095987in}}%
\pgfpathlineto{\pgfqpoint{4.324695in}{1.608470in}}%
\pgfpathlineto{\pgfqpoint{4.470234in}{1.534400in}}%
\pgfpathclose%
\pgfusepath{fill}%
\end{pgfscope}%
\begin{pgfscope}%
\pgfpathrectangle{\pgfqpoint{0.539299in}{0.078740in}}{\pgfqpoint{7.842520in}{7.842520in}}%
\pgfusepath{clip}%
\pgfsetbuttcap%
\pgfsetroundjoin%
\definecolor{currentfill}{rgb}{0.187231,0.414746,0.556547}%
\pgfsetfillcolor{currentfill}%
\pgfsetlinewidth{0.000000pt}%
\definecolor{currentstroke}{rgb}{0.150148,0.676631,0.506589}%
\pgfsetstrokecolor{currentstroke}%
\pgfsetdash{}{0pt}%
\pgfpathmoveto{\pgfqpoint{5.327340in}{2.342890in}}%
\pgfpathlineto{\pgfqpoint{5.099519in}{2.460623in}}%
\pgfpathlineto{\pgfqpoint{5.245331in}{2.266200in}}%
\pgfpathclose%
\pgfusepath{fill}%
\end{pgfscope}%
\begin{pgfscope}%
\pgfpathrectangle{\pgfqpoint{0.539299in}{0.078740in}}{\pgfqpoint{7.842520in}{7.842520in}}%
\pgfusepath{clip}%
\pgfsetbuttcap%
\pgfsetroundjoin%
\definecolor{currentfill}{rgb}{0.160665,0.478540,0.558115}%
\pgfsetfillcolor{currentfill}%
\pgfsetlinewidth{0.000000pt}%
\definecolor{currentstroke}{rgb}{0.153894,0.680203,0.504172}%
\pgfsetstrokecolor{currentstroke}%
\pgfsetdash{}{0pt}%
\pgfpathmoveto{\pgfqpoint{4.200551in}{2.726746in}}%
\pgfpathlineto{\pgfqpoint{4.054864in}{2.841426in}}%
\pgfpathlineto{\pgfqpoint{3.971998in}{2.380078in}}%
\pgfpathclose%
\pgfusepath{fill}%
\end{pgfscope}%
\begin{pgfscope}%
\pgfpathrectangle{\pgfqpoint{0.539299in}{0.078740in}}{\pgfqpoint{7.842520in}{7.842520in}}%
\pgfusepath{clip}%
\pgfsetbuttcap%
\pgfsetroundjoin%
\definecolor{currentfill}{rgb}{0.229739,0.322361,0.545706}%
\pgfsetfillcolor{currentfill}%
\pgfsetlinewidth{0.000000pt}%
\definecolor{currentstroke}{rgb}{0.157851,0.683765,0.501686}%
\pgfsetstrokecolor{currentstroke}%
\pgfsetdash{}{0pt}%
\pgfpathmoveto{\pgfqpoint{4.324695in}{1.608470in}}%
\pgfpathlineto{\pgfqpoint{4.408361in}{2.095987in}}%
\pgfpathlineto{\pgfqpoint{4.262595in}{2.193629in}}%
\pgfpathclose%
\pgfusepath{fill}%
\end{pgfscope}%
\begin{pgfscope}%
\pgfpathrectangle{\pgfqpoint{0.539299in}{0.078740in}}{\pgfqpoint{7.842520in}{7.842520in}}%
\pgfusepath{clip}%
\pgfsetbuttcap%
\pgfsetroundjoin%
\definecolor{currentfill}{rgb}{0.239374,0.735588,0.455688}%
\pgfsetfillcolor{currentfill}%
\pgfsetlinewidth{0.000000pt}%
\definecolor{currentstroke}{rgb}{0.162016,0.687316,0.499129}%
\pgfsetstrokecolor{currentstroke}%
\pgfsetdash{}{0pt}%
\pgfpathmoveto{\pgfqpoint{3.785592in}{3.924272in}}%
\pgfpathlineto{\pgfqpoint{3.640137in}{4.053707in}}%
\pgfpathlineto{\pgfqpoint{3.701876in}{3.586183in}}%
\pgfpathclose%
\pgfusepath{fill}%
\end{pgfscope}%
\begin{pgfscope}%
\pgfpathrectangle{\pgfqpoint{0.539299in}{0.078740in}}{\pgfqpoint{7.842520in}{7.842520in}}%
\pgfusepath{clip}%
\pgfsetbuttcap%
\pgfsetroundjoin%
\definecolor{currentfill}{rgb}{0.319809,0.770914,0.411152}%
\pgfsetfillcolor{currentfill}%
\pgfsetlinewidth{0.000000pt}%
\definecolor{currentstroke}{rgb}{0.166383,0.690856,0.496502}%
\pgfsetstrokecolor{currentstroke}%
\pgfsetdash{}{0pt}%
\pgfpathmoveto{\pgfqpoint{3.640137in}{4.053707in}}%
\pgfpathlineto{\pgfqpoint{3.785592in}{3.924272in}}%
\pgfpathlineto{\pgfqpoint{3.870205in}{4.179274in}}%
\pgfpathclose%
\pgfusepath{fill}%
\end{pgfscope}%
\begin{pgfscope}%
\pgfpathrectangle{\pgfqpoint{0.539299in}{0.078740in}}{\pgfqpoint{7.842520in}{7.842520in}}%
\pgfusepath{clip}%
\pgfsetbuttcap%
\pgfsetroundjoin%
\definecolor{currentfill}{rgb}{0.133743,0.548535,0.553541}%
\pgfsetfillcolor{currentfill}%
\pgfsetlinewidth{0.000000pt}%
\definecolor{currentstroke}{rgb}{0.170948,0.694384,0.493803}%
\pgfsetstrokecolor{currentstroke}%
\pgfsetdash{}{0pt}%
\pgfpathmoveto{\pgfqpoint{4.807495in}{2.824405in}}%
\pgfpathlineto{\pgfqpoint{4.891094in}{2.947957in}}%
\pgfpathlineto{\pgfqpoint{4.745380in}{3.138731in}}%
\pgfpathclose%
\pgfusepath{fill}%
\end{pgfscope}%
\begin{pgfscope}%
\pgfpathrectangle{\pgfqpoint{0.539299in}{0.078740in}}{\pgfqpoint{7.842520in}{7.842520in}}%
\pgfusepath{clip}%
\pgfsetbuttcap%
\pgfsetroundjoin%
\definecolor{currentfill}{rgb}{0.319809,0.770914,0.411152}%
\pgfsetfillcolor{currentfill}%
\pgfsetlinewidth{0.000000pt}%
\definecolor{currentstroke}{rgb}{0.175707,0.697900,0.491033}%
\pgfsetstrokecolor{currentstroke}%
\pgfsetdash{}{0pt}%
\pgfpathmoveto{\pgfqpoint{3.870205in}{4.179274in}}%
\pgfpathlineto{\pgfqpoint{3.785592in}{3.924272in}}%
\pgfpathlineto{\pgfqpoint{4.015993in}{4.022293in}}%
\pgfpathclose%
\pgfusepath{fill}%
\end{pgfscope}%
\begin{pgfscope}%
\pgfpathrectangle{\pgfqpoint{0.539299in}{0.078740in}}{\pgfqpoint{7.842520in}{7.842520in}}%
\pgfusepath{clip}%
\pgfsetbuttcap%
\pgfsetroundjoin%
\definecolor{currentfill}{rgb}{0.182256,0.426184,0.557120}%
\pgfsetfillcolor{currentfill}%
\pgfsetlinewidth{0.000000pt}%
\definecolor{currentstroke}{rgb}{0.180653,0.701402,0.488189}%
\pgfsetstrokecolor{currentstroke}%
\pgfsetdash{}{0pt}%
\pgfpathmoveto{\pgfqpoint{4.117136in}{2.288294in}}%
\pgfpathlineto{\pgfqpoint{4.262595in}{2.193629in}}%
\pgfpathlineto{\pgfqpoint{4.200551in}{2.726746in}}%
\pgfpathclose%
\pgfusepath{fill}%
\end{pgfscope}%
\begin{pgfscope}%
\pgfpathrectangle{\pgfqpoint{0.539299in}{0.078740in}}{\pgfqpoint{7.842520in}{7.842520in}}%
\pgfusepath{clip}%
\pgfsetbuttcap%
\pgfsetroundjoin%
\definecolor{currentfill}{rgb}{0.121831,0.589055,0.545623}%
\pgfsetfillcolor{currentfill}%
\pgfsetlinewidth{0.000000pt}%
\definecolor{currentstroke}{rgb}{0.185783,0.704891,0.485273}%
\pgfsetstrokecolor{currentstroke}%
\pgfsetdash{}{0pt}%
\pgfpathmoveto{\pgfqpoint{4.599574in}{3.325252in}}%
\pgfpathlineto{\pgfqpoint{4.661391in}{2.997194in}}%
\pgfpathlineto{\pgfqpoint{4.745380in}{3.138731in}}%
\pgfpathclose%
\pgfusepath{fill}%
\end{pgfscope}%
\begin{pgfscope}%
\pgfpathrectangle{\pgfqpoint{0.539299in}{0.078740in}}{\pgfqpoint{7.842520in}{7.842520in}}%
\pgfusepath{clip}%
\pgfsetbuttcap%
\pgfsetroundjoin%
\definecolor{currentfill}{rgb}{0.168126,0.459988,0.558082}%
\pgfsetfillcolor{currentfill}%
\pgfsetlinewidth{0.000000pt}%
\definecolor{currentstroke}{rgb}{0.191090,0.708366,0.482284}%
\pgfsetstrokecolor{currentstroke}%
\pgfsetdash{}{0pt}%
\pgfpathmoveto{\pgfqpoint{5.099519in}{2.460623in}}%
\pgfpathlineto{\pgfqpoint{5.182120in}{2.551317in}}%
\pgfpathlineto{\pgfqpoint{4.953553in}{2.645992in}}%
\pgfpathclose%
\pgfusepath{fill}%
\end{pgfscope}%
\begin{pgfscope}%
\pgfpathrectangle{\pgfqpoint{0.539299in}{0.078740in}}{\pgfqpoint{7.842520in}{7.842520in}}%
\pgfusepath{clip}%
\pgfsetbuttcap%
\pgfsetroundjoin%
\definecolor{currentfill}{rgb}{0.208030,0.718701,0.472873}%
\pgfsetfillcolor{currentfill}%
\pgfsetlinewidth{0.000000pt}%
\definecolor{currentstroke}{rgb}{0.196571,0.711827,0.479221}%
\pgfsetstrokecolor{currentstroke}%
\pgfsetdash{}{0pt}%
\pgfpathmoveto{\pgfqpoint{3.931280in}{3.785621in}}%
\pgfpathlineto{\pgfqpoint{3.785592in}{3.924272in}}%
\pgfpathlineto{\pgfqpoint{3.701876in}{3.586183in}}%
\pgfpathclose%
\pgfusepath{fill}%
\end{pgfscope}%
\begin{pgfscope}%
\pgfpathrectangle{\pgfqpoint{0.539299in}{0.078740in}}{\pgfqpoint{7.842520in}{7.842520in}}%
\pgfusepath{clip}%
\pgfsetbuttcap%
\pgfsetroundjoin%
\definecolor{currentfill}{rgb}{0.162016,0.687316,0.499129}%
\pgfsetfillcolor{currentfill}%
\pgfsetlinewidth{0.000000pt}%
\definecolor{currentstroke}{rgb}{0.202219,0.715272,0.476084}%
\pgfsetstrokecolor{currentstroke}%
\pgfsetdash{}{0pt}%
\pgfpathmoveto{\pgfqpoint{3.701876in}{3.586183in}}%
\pgfpathlineto{\pgfqpoint{3.847250in}{3.466509in}}%
\pgfpathlineto{\pgfqpoint{3.931280in}{3.785621in}}%
\pgfpathclose%
\pgfusepath{fill}%
\end{pgfscope}%
\begin{pgfscope}%
\pgfpathrectangle{\pgfqpoint{0.539299in}{0.078740in}}{\pgfqpoint{7.842520in}{7.842520in}}%
\pgfusepath{clip}%
\pgfsetbuttcap%
\pgfsetroundjoin%
\definecolor{currentfill}{rgb}{0.119423,0.611141,0.538982}%
\pgfsetfillcolor{currentfill}%
\pgfsetlinewidth{0.000000pt}%
\definecolor{currentstroke}{rgb}{0.208030,0.718701,0.472873}%
\pgfsetstrokecolor{currentstroke}%
\pgfsetdash{}{0pt}%
\pgfpathmoveto{\pgfqpoint{3.909430in}{2.951944in}}%
\pgfpathlineto{\pgfqpoint{3.992871in}{3.340173in}}%
\pgfpathlineto{\pgfqpoint{3.847250in}{3.466509in}}%
\pgfpathclose%
\pgfusepath{fill}%
\end{pgfscope}%
\begin{pgfscope}%
\pgfpathrectangle{\pgfqpoint{0.539299in}{0.078740in}}{\pgfqpoint{7.842520in}{7.842520in}}%
\pgfusepath{clip}%
\pgfsetbuttcap%
\pgfsetroundjoin%
\definecolor{currentfill}{rgb}{0.121380,0.629492,0.531973}%
\pgfsetfillcolor{currentfill}%
\pgfsetlinewidth{0.000000pt}%
\definecolor{currentstroke}{rgb}{0.214000,0.722114,0.469588}%
\pgfsetstrokecolor{currentstroke}%
\pgfsetdash{}{0pt}%
\pgfpathmoveto{\pgfqpoint{4.515273in}{3.165074in}}%
\pgfpathlineto{\pgfqpoint{4.599574in}{3.325252in}}%
\pgfpathlineto{\pgfqpoint{4.453701in}{3.507500in}}%
\pgfpathclose%
\pgfusepath{fill}%
\end{pgfscope}%
\begin{pgfscope}%
\pgfpathrectangle{\pgfqpoint{0.539299in}{0.078740in}}{\pgfqpoint{7.842520in}{7.842520in}}%
\pgfusepath{clip}%
\pgfsetbuttcap%
\pgfsetroundjoin%
\definecolor{currentfill}{rgb}{0.259857,0.745492,0.444467}%
\pgfsetfillcolor{currentfill}%
\pgfsetlinewidth{0.000000pt}%
\definecolor{currentstroke}{rgb}{0.220124,0.725509,0.466226}%
\pgfsetstrokecolor{currentstroke}%
\pgfsetdash{}{0pt}%
\pgfpathmoveto{\pgfqpoint{4.015993in}{4.022293in}}%
\pgfpathlineto{\pgfqpoint{3.785592in}{3.924272in}}%
\pgfpathlineto{\pgfqpoint{3.931280in}{3.785621in}}%
\pgfpathclose%
\pgfusepath{fill}%
\end{pgfscope}%
\begin{pgfscope}%
\pgfpathrectangle{\pgfqpoint{0.539299in}{0.078740in}}{\pgfqpoint{7.842520in}{7.842520in}}%
\pgfusepath{clip}%
\pgfsetbuttcap%
\pgfsetroundjoin%
\definecolor{currentfill}{rgb}{0.232815,0.732247,0.459277}%
\pgfsetfillcolor{currentfill}%
\pgfsetlinewidth{0.000000pt}%
\definecolor{currentstroke}{rgb}{0.226397,0.728888,0.462789}%
\pgfsetstrokecolor{currentstroke}%
\pgfsetdash{}{0pt}%
\pgfpathmoveto{\pgfqpoint{4.077139in}{3.639292in}}%
\pgfpathlineto{\pgfqpoint{4.161873in}{3.857033in}}%
\pgfpathlineto{\pgfqpoint{4.015993in}{4.022293in}}%
\pgfpathclose%
\pgfusepath{fill}%
\end{pgfscope}%
\begin{pgfscope}%
\pgfpathrectangle{\pgfqpoint{0.539299in}{0.078740in}}{\pgfqpoint{7.842520in}{7.842520in}}%
\pgfusepath{clip}%
\pgfsetbuttcap%
\pgfsetroundjoin%
\definecolor{currentfill}{rgb}{0.150148,0.676631,0.506589}%
\pgfsetfillcolor{currentfill}%
\pgfsetlinewidth{0.000000pt}%
\definecolor{currentstroke}{rgb}{0.232815,0.732247,0.459277}%
\pgfsetstrokecolor{currentstroke}%
\pgfsetdash{}{0pt}%
\pgfpathmoveto{\pgfqpoint{4.307790in}{3.685037in}}%
\pgfpathlineto{\pgfqpoint{4.223116in}{3.486528in}}%
\pgfpathlineto{\pgfqpoint{4.453701in}{3.507500in}}%
\pgfpathclose%
\pgfusepath{fill}%
\end{pgfscope}%
\begin{pgfscope}%
\pgfpathrectangle{\pgfqpoint{0.539299in}{0.078740in}}{\pgfqpoint{7.842520in}{7.842520in}}%
\pgfusepath{clip}%
\pgfsetbuttcap%
\pgfsetroundjoin%
\definecolor{currentfill}{rgb}{0.180653,0.701402,0.488189}%
\pgfsetfillcolor{currentfill}%
\pgfsetlinewidth{0.000000pt}%
\definecolor{currentstroke}{rgb}{0.239374,0.735588,0.455688}%
\pgfsetstrokecolor{currentstroke}%
\pgfsetdash{}{0pt}%
\pgfpathmoveto{\pgfqpoint{4.161873in}{3.857033in}}%
\pgfpathlineto{\pgfqpoint{4.223116in}{3.486528in}}%
\pgfpathlineto{\pgfqpoint{4.307790in}{3.685037in}}%
\pgfpathclose%
\pgfusepath{fill}%
\end{pgfscope}%
\begin{pgfscope}%
\pgfpathrectangle{\pgfqpoint{0.539299in}{0.078740in}}{\pgfqpoint{7.842520in}{7.842520in}}%
\pgfusepath{clip}%
\pgfsetbuttcap%
\pgfsetroundjoin%
\definecolor{currentfill}{rgb}{0.131172,0.555899,0.552459}%
\pgfsetfillcolor{currentfill}%
\pgfsetlinewidth{0.000000pt}%
\definecolor{currentstroke}{rgb}{0.246070,0.738910,0.452024}%
\pgfsetstrokecolor{currentstroke}%
\pgfsetdash{}{0pt}%
\pgfpathmoveto{\pgfqpoint{3.909430in}{2.951944in}}%
\pgfpathlineto{\pgfqpoint{4.054864in}{2.841426in}}%
\pgfpathlineto{\pgfqpoint{4.138693in}{3.208126in}}%
\pgfpathclose%
\pgfusepath{fill}%
\end{pgfscope}%
\begin{pgfscope}%
\pgfpathrectangle{\pgfqpoint{0.539299in}{0.078740in}}{\pgfqpoint{7.842520in}{7.842520in}}%
\pgfusepath{clip}%
\pgfsetbuttcap%
\pgfsetroundjoin%
\definecolor{currentfill}{rgb}{0.226397,0.728888,0.462789}%
\pgfsetfillcolor{currentfill}%
\pgfsetlinewidth{0.000000pt}%
\definecolor{currentstroke}{rgb}{0.252899,0.742211,0.448284}%
\pgfsetstrokecolor{currentstroke}%
\pgfsetdash{}{0pt}%
\pgfpathmoveto{\pgfqpoint{3.931280in}{3.785621in}}%
\pgfpathlineto{\pgfqpoint{4.077139in}{3.639292in}}%
\pgfpathlineto{\pgfqpoint{4.015993in}{4.022293in}}%
\pgfpathclose%
\pgfusepath{fill}%
\end{pgfscope}%
\begin{pgfscope}%
\pgfpathrectangle{\pgfqpoint{0.539299in}{0.078740in}}{\pgfqpoint{7.842520in}{7.842520in}}%
\pgfusepath{clip}%
\pgfsetbuttcap%
\pgfsetroundjoin%
\definecolor{currentfill}{rgb}{0.146180,0.515413,0.556823}%
\pgfsetfillcolor{currentfill}%
\pgfsetlinewidth{0.000000pt}%
\definecolor{currentstroke}{rgb}{0.259857,0.745492,0.444467}%
\pgfsetstrokecolor{currentstroke}%
\pgfsetdash{}{0pt}%
\pgfpathmoveto{\pgfqpoint{4.891094in}{2.947957in}}%
\pgfpathlineto{\pgfqpoint{4.807495in}{2.824405in}}%
\pgfpathlineto{\pgfqpoint{4.953553in}{2.645992in}}%
\pgfpathclose%
\pgfusepath{fill}%
\end{pgfscope}%
\begin{pgfscope}%
\pgfpathrectangle{\pgfqpoint{0.539299in}{0.078740in}}{\pgfqpoint{7.842520in}{7.842520in}}%
\pgfusepath{clip}%
\pgfsetbuttcap%
\pgfsetroundjoin%
\definecolor{currentfill}{rgb}{0.121148,0.592739,0.544641}%
\pgfsetfillcolor{currentfill}%
\pgfsetlinewidth{0.000000pt}%
\definecolor{currentstroke}{rgb}{0.266941,0.748751,0.440573}%
\pgfsetstrokecolor{currentstroke}%
\pgfsetdash{}{0pt}%
\pgfpathmoveto{\pgfqpoint{4.138693in}{3.208126in}}%
\pgfpathlineto{\pgfqpoint{3.992871in}{3.340173in}}%
\pgfpathlineto{\pgfqpoint{3.909430in}{2.951944in}}%
\pgfpathclose%
\pgfusepath{fill}%
\end{pgfscope}%
\begin{pgfscope}%
\pgfpathrectangle{\pgfqpoint{0.539299in}{0.078740in}}{\pgfqpoint{7.842520in}{7.842520in}}%
\pgfusepath{clip}%
\pgfsetbuttcap%
\pgfsetroundjoin%
\definecolor{currentfill}{rgb}{0.146616,0.673050,0.508936}%
\pgfsetfillcolor{currentfill}%
\pgfsetlinewidth{0.000000pt}%
\definecolor{currentstroke}{rgb}{0.274149,0.751988,0.436601}%
\pgfsetstrokecolor{currentstroke}%
\pgfsetdash{}{0pt}%
\pgfpathmoveto{\pgfqpoint{3.931280in}{3.785621in}}%
\pgfpathlineto{\pgfqpoint{3.847250in}{3.466509in}}%
\pgfpathlineto{\pgfqpoint{3.992871in}{3.340173in}}%
\pgfpathclose%
\pgfusepath{fill}%
\end{pgfscope}%
\begin{pgfscope}%
\pgfpathrectangle{\pgfqpoint{0.539299in}{0.078740in}}{\pgfqpoint{7.842520in}{7.842520in}}%
\pgfusepath{clip}%
\pgfsetbuttcap%
\pgfsetroundjoin%
\definecolor{currentfill}{rgb}{0.229739,0.322361,0.545706}%
\pgfsetfillcolor{currentfill}%
\pgfsetlinewidth{0.000000pt}%
\definecolor{currentstroke}{rgb}{0.281477,0.755203,0.432552}%
\pgfsetstrokecolor{currentstroke}%
\pgfsetdash{}{0pt}%
\pgfpathmoveto{\pgfqpoint{5.390896in}{2.059660in}}%
\pgfpathlineto{\pgfqpoint{5.308523in}{1.948082in}}%
\pgfpathlineto{\pgfqpoint{5.536081in}{1.836733in}}%
\pgfpathclose%
\pgfusepath{fill}%
\end{pgfscope}%
\begin{pgfscope}%
\pgfpathrectangle{\pgfqpoint{0.539299in}{0.078740in}}{\pgfqpoint{7.842520in}{7.842520in}}%
\pgfusepath{clip}%
\pgfsetbuttcap%
\pgfsetroundjoin%
\definecolor{currentfill}{rgb}{0.175707,0.697900,0.491033}%
\pgfsetfillcolor{currentfill}%
\pgfsetlinewidth{0.000000pt}%
\definecolor{currentstroke}{rgb}{0.288921,0.758394,0.428426}%
\pgfsetstrokecolor{currentstroke}%
\pgfsetdash{}{0pt}%
\pgfpathmoveto{\pgfqpoint{4.161873in}{3.857033in}}%
\pgfpathlineto{\pgfqpoint{4.077139in}{3.639292in}}%
\pgfpathlineto{\pgfqpoint{4.223116in}{3.486528in}}%
\pgfpathclose%
\pgfusepath{fill}%
\end{pgfscope}%
\begin{pgfscope}%
\pgfpathrectangle{\pgfqpoint{0.539299in}{0.078740in}}{\pgfqpoint{7.842520in}{7.842520in}}%
\pgfusepath{clip}%
\pgfsetbuttcap%
\pgfsetroundjoin%
\definecolor{currentfill}{rgb}{0.131172,0.555899,0.552459}%
\pgfsetfillcolor{currentfill}%
\pgfsetlinewidth{0.000000pt}%
\definecolor{currentstroke}{rgb}{0.296479,0.761561,0.424223}%
\pgfsetstrokecolor{currentstroke}%
\pgfsetdash{}{0pt}%
\pgfpathmoveto{\pgfqpoint{4.745380in}{3.138731in}}%
\pgfpathlineto{\pgfqpoint{4.661391in}{2.997194in}}%
\pgfpathlineto{\pgfqpoint{4.807495in}{2.824405in}}%
\pgfpathclose%
\pgfusepath{fill}%
\end{pgfscope}%
\begin{pgfscope}%
\pgfpathrectangle{\pgfqpoint{0.539299in}{0.078740in}}{\pgfqpoint{7.842520in}{7.842520in}}%
\pgfusepath{clip}%
\pgfsetbuttcap%
\pgfsetroundjoin%
\definecolor{currentfill}{rgb}{0.132268,0.655014,0.519661}%
\pgfsetfillcolor{currentfill}%
\pgfsetlinewidth{0.000000pt}%
\definecolor{currentstroke}{rgb}{0.304148,0.764704,0.419943}%
\pgfsetstrokecolor{currentstroke}%
\pgfsetdash{}{0pt}%
\pgfpathmoveto{\pgfqpoint{4.223116in}{3.486528in}}%
\pgfpathlineto{\pgfqpoint{4.369172in}{3.328258in}}%
\pgfpathlineto{\pgfqpoint{4.453701in}{3.507500in}}%
\pgfpathclose%
\pgfusepath{fill}%
\end{pgfscope}%
\begin{pgfscope}%
\pgfpathrectangle{\pgfqpoint{0.539299in}{0.078740in}}{\pgfqpoint{7.842520in}{7.842520in}}%
\pgfusepath{clip}%
\pgfsetbuttcap%
\pgfsetroundjoin%
\definecolor{currentfill}{rgb}{0.121148,0.592739,0.544641}%
\pgfsetfillcolor{currentfill}%
\pgfsetlinewidth{0.000000pt}%
\definecolor{currentstroke}{rgb}{0.311925,0.767822,0.415586}%
\pgfsetstrokecolor{currentstroke}%
\pgfsetdash{}{0pt}%
\pgfpathmoveto{\pgfqpoint{4.515273in}{3.165074in}}%
\pgfpathlineto{\pgfqpoint{4.661391in}{2.997194in}}%
\pgfpathlineto{\pgfqpoint{4.599574in}{3.325252in}}%
\pgfpathclose%
\pgfusepath{fill}%
\end{pgfscope}%
\begin{pgfscope}%
\pgfpathrectangle{\pgfqpoint{0.539299in}{0.078740in}}{\pgfqpoint{7.842520in}{7.842520in}}%
\pgfusepath{clip}%
\pgfsetbuttcap%
\pgfsetroundjoin%
\definecolor{currentfill}{rgb}{0.157851,0.683765,0.501686}%
\pgfsetfillcolor{currentfill}%
\pgfsetlinewidth{0.000000pt}%
\definecolor{currentstroke}{rgb}{0.319809,0.770914,0.411152}%
\pgfsetstrokecolor{currentstroke}%
\pgfsetdash{}{0pt}%
\pgfpathmoveto{\pgfqpoint{3.992871in}{3.340173in}}%
\pgfpathlineto{\pgfqpoint{4.077139in}{3.639292in}}%
\pgfpathlineto{\pgfqpoint{3.931280in}{3.785621in}}%
\pgfpathclose%
\pgfusepath{fill}%
\end{pgfscope}%
\begin{pgfscope}%
\pgfpathrectangle{\pgfqpoint{0.539299in}{0.078740in}}{\pgfqpoint{7.842520in}{7.842520in}}%
\pgfusepath{clip}%
\pgfsetbuttcap%
\pgfsetroundjoin%
\definecolor{currentfill}{rgb}{0.122312,0.633153,0.530398}%
\pgfsetfillcolor{currentfill}%
\pgfsetlinewidth{0.000000pt}%
\definecolor{currentstroke}{rgb}{0.327796,0.773980,0.406640}%
\pgfsetstrokecolor{currentstroke}%
\pgfsetdash{}{0pt}%
\pgfpathmoveto{\pgfqpoint{4.453701in}{3.507500in}}%
\pgfpathlineto{\pgfqpoint{4.369172in}{3.328258in}}%
\pgfpathlineto{\pgfqpoint{4.515273in}{3.165074in}}%
\pgfpathclose%
\pgfusepath{fill}%
\end{pgfscope}%
\begin{pgfscope}%
\pgfpathrectangle{\pgfqpoint{0.539299in}{0.078740in}}{\pgfqpoint{7.842520in}{7.842520in}}%
\pgfusepath{clip}%
\pgfsetbuttcap%
\pgfsetroundjoin%
\definecolor{currentfill}{rgb}{0.258965,0.251537,0.524736}%
\pgfsetfillcolor{currentfill}%
\pgfsetlinewidth{0.000000pt}%
\definecolor{currentstroke}{rgb}{0.335885,0.777018,0.402049}%
\pgfsetstrokecolor{currentstroke}%
\pgfsetdash{}{0pt}%
\pgfpathmoveto{\pgfqpoint{4.554418in}{1.995003in}}%
\pgfpathlineto{\pgfqpoint{4.470234in}{1.534400in}}%
\pgfpathlineto{\pgfqpoint{4.616159in}{1.458069in}}%
\pgfpathclose%
\pgfusepath{fill}%
\end{pgfscope}%
\begin{pgfscope}%
\pgfpathrectangle{\pgfqpoint{0.539299in}{0.078740in}}{\pgfqpoint{7.842520in}{7.842520in}}%
\pgfusepath{clip}%
\pgfsetbuttcap%
\pgfsetroundjoin%
\definecolor{currentfill}{rgb}{0.255645,0.260703,0.528312}%
\pgfsetfillcolor{currentfill}%
\pgfsetlinewidth{0.000000pt}%
\definecolor{currentstroke}{rgb}{0.344074,0.780029,0.397381}%
\pgfsetstrokecolor{currentstroke}%
\pgfsetdash{}{0pt}%
\pgfpathmoveto{\pgfqpoint{5.536081in}{1.836733in}}%
\pgfpathlineto{\pgfqpoint{5.454351in}{1.741746in}}%
\pgfpathlineto{\pgfqpoint{5.599643in}{1.509341in}}%
\pgfpathclose%
\pgfusepath{fill}%
\end{pgfscope}%
\begin{pgfscope}%
\pgfpathrectangle{\pgfqpoint{0.539299in}{0.078740in}}{\pgfqpoint{7.842520in}{7.842520in}}%
\pgfusepath{clip}%
\pgfsetbuttcap%
\pgfsetroundjoin%
\definecolor{currentfill}{rgb}{0.237441,0.305202,0.541921}%
\pgfsetfillcolor{currentfill}%
\pgfsetlinewidth{0.000000pt}%
\definecolor{currentstroke}{rgb}{0.352360,0.783011,0.392636}%
\pgfsetstrokecolor{currentstroke}%
\pgfsetdash{}{0pt}%
\pgfpathmoveto{\pgfqpoint{4.408361in}{2.095987in}}%
\pgfpathlineto{\pgfqpoint{4.470234in}{1.534400in}}%
\pgfpathlineto{\pgfqpoint{4.554418in}{1.995003in}}%
\pgfpathclose%
\pgfusepath{fill}%
\end{pgfscope}%
\begin{pgfscope}%
\pgfpathrectangle{\pgfqpoint{0.539299in}{0.078740in}}{\pgfqpoint{7.842520in}{7.842520in}}%
\pgfusepath{clip}%
\pgfsetbuttcap%
\pgfsetroundjoin%
\definecolor{currentfill}{rgb}{0.169646,0.456262,0.558030}%
\pgfsetfillcolor{currentfill}%
\pgfsetlinewidth{0.000000pt}%
\definecolor{currentstroke}{rgb}{0.360741,0.785964,0.387814}%
\pgfsetstrokecolor{currentstroke}%
\pgfsetdash{}{0pt}%
\pgfpathmoveto{\pgfqpoint{4.200551in}{2.726746in}}%
\pgfpathlineto{\pgfqpoint{4.262595in}{2.193629in}}%
\pgfpathlineto{\pgfqpoint{4.346469in}{2.608126in}}%
\pgfpathclose%
\pgfusepath{fill}%
\end{pgfscope}%
\begin{pgfscope}%
\pgfpathrectangle{\pgfqpoint{0.539299in}{0.078740in}}{\pgfqpoint{7.842520in}{7.842520in}}%
\pgfusepath{clip}%
\pgfsetbuttcap%
\pgfsetroundjoin%
\definecolor{currentfill}{rgb}{0.140210,0.665859,0.513427}%
\pgfsetfillcolor{currentfill}%
\pgfsetlinewidth{0.000000pt}%
\definecolor{currentstroke}{rgb}{0.369214,0.788888,0.382914}%
\pgfsetstrokecolor{currentstroke}%
\pgfsetdash{}{0pt}%
\pgfpathmoveto{\pgfqpoint{4.223116in}{3.486528in}}%
\pgfpathlineto{\pgfqpoint{4.077139in}{3.639292in}}%
\pgfpathlineto{\pgfqpoint{3.992871in}{3.340173in}}%
\pgfpathclose%
\pgfusepath{fill}%
\end{pgfscope}%
\begin{pgfscope}%
\pgfpathrectangle{\pgfqpoint{0.539299in}{0.078740in}}{\pgfqpoint{7.842520in}{7.842520in}}%
\pgfusepath{clip}%
\pgfsetbuttcap%
\pgfsetroundjoin%
\definecolor{currentfill}{rgb}{0.139147,0.533812,0.555298}%
\pgfsetfillcolor{currentfill}%
\pgfsetlinewidth{0.000000pt}%
\definecolor{currentstroke}{rgb}{0.377779,0.791781,0.377939}%
\pgfsetstrokecolor{currentstroke}%
\pgfsetdash{}{0pt}%
\pgfpathmoveto{\pgfqpoint{4.054864in}{2.841426in}}%
\pgfpathlineto{\pgfqpoint{4.200551in}{2.726746in}}%
\pgfpathlineto{\pgfqpoint{4.284682in}{3.071051in}}%
\pgfpathclose%
\pgfusepath{fill}%
\end{pgfscope}%
\begin{pgfscope}%
\pgfpathrectangle{\pgfqpoint{0.539299in}{0.078740in}}{\pgfqpoint{7.842520in}{7.842520in}}%
\pgfusepath{clip}%
\pgfsetbuttcap%
\pgfsetroundjoin%
\definecolor{currentfill}{rgb}{0.212395,0.359683,0.551710}%
\pgfsetfillcolor{currentfill}%
\pgfsetlinewidth{0.000000pt}%
\definecolor{currentstroke}{rgb}{0.386433,0.794644,0.372886}%
\pgfsetstrokecolor{currentstroke}%
\pgfsetdash{}{0pt}%
\pgfpathmoveto{\pgfqpoint{5.308523in}{1.948082in}}%
\pgfpathlineto{\pgfqpoint{5.390896in}{2.059660in}}%
\pgfpathlineto{\pgfqpoint{5.245331in}{2.266200in}}%
\pgfpathclose%
\pgfusepath{fill}%
\end{pgfscope}%
\begin{pgfscope}%
\pgfpathrectangle{\pgfqpoint{0.539299in}{0.078740in}}{\pgfqpoint{7.842520in}{7.842520in}}%
\pgfusepath{clip}%
\pgfsetbuttcap%
\pgfsetroundjoin%
\definecolor{currentfill}{rgb}{0.123444,0.636809,0.528763}%
\pgfsetfillcolor{currentfill}%
\pgfsetlinewidth{0.000000pt}%
\definecolor{currentstroke}{rgb}{0.395174,0.797475,0.367757}%
\pgfsetstrokecolor{currentstroke}%
\pgfsetdash{}{0pt}%
\pgfpathmoveto{\pgfqpoint{3.992871in}{3.340173in}}%
\pgfpathlineto{\pgfqpoint{4.138693in}{3.208126in}}%
\pgfpathlineto{\pgfqpoint{4.223116in}{3.486528in}}%
\pgfpathclose%
\pgfusepath{fill}%
\end{pgfscope}%
\begin{pgfscope}%
\pgfpathrectangle{\pgfqpoint{0.539299in}{0.078740in}}{\pgfqpoint{7.842520in}{7.842520in}}%
\pgfusepath{clip}%
\pgfsetbuttcap%
\pgfsetroundjoin%
\definecolor{currentfill}{rgb}{0.126453,0.570633,0.549841}%
\pgfsetfillcolor{currentfill}%
\pgfsetlinewidth{0.000000pt}%
\definecolor{currentstroke}{rgb}{0.404001,0.800275,0.362552}%
\pgfsetstrokecolor{currentstroke}%
\pgfsetdash{}{0pt}%
\pgfpathmoveto{\pgfqpoint{4.138693in}{3.208126in}}%
\pgfpathlineto{\pgfqpoint{4.054864in}{2.841426in}}%
\pgfpathlineto{\pgfqpoint{4.284682in}{3.071051in}}%
\pgfpathclose%
\pgfusepath{fill}%
\end{pgfscope}%
\begin{pgfscope}%
\pgfpathrectangle{\pgfqpoint{0.539299in}{0.078740in}}{\pgfqpoint{7.842520in}{7.842520in}}%
\pgfusepath{clip}%
\pgfsetbuttcap%
\pgfsetroundjoin%
\definecolor{currentfill}{rgb}{0.281887,0.150881,0.465405}%
\pgfsetfillcolor{currentfill}%
\pgfsetlinewidth{0.000000pt}%
\definecolor{currentstroke}{rgb}{0.412913,0.803041,0.357269}%
\pgfsetstrokecolor{currentstroke}%
\pgfsetdash{}{0pt}%
\pgfpathmoveto{\pgfqpoint{5.599643in}{1.509341in}}%
\pgfpathlineto{\pgfqpoint{5.662874in}{1.144010in}}%
\pgfpathlineto{\pgfqpoint{5.744062in}{1.240624in}}%
\pgfpathclose%
\pgfusepath{fill}%
\end{pgfscope}%
\begin{pgfscope}%
\pgfpathrectangle{\pgfqpoint{0.539299in}{0.078740in}}{\pgfqpoint{7.842520in}{7.842520in}}%
\pgfusepath{clip}%
\pgfsetbuttcap%
\pgfsetroundjoin%
\definecolor{currentfill}{rgb}{0.194100,0.399323,0.555565}%
\pgfsetfillcolor{currentfill}%
\pgfsetlinewidth{0.000000pt}%
\definecolor{currentstroke}{rgb}{0.421908,0.805774,0.351910}%
\pgfsetstrokecolor{currentstroke}%
\pgfsetdash{}{0pt}%
\pgfpathmoveto{\pgfqpoint{4.492593in}{2.485503in}}%
\pgfpathlineto{\pgfqpoint{4.262595in}{2.193629in}}%
\pgfpathlineto{\pgfqpoint{4.408361in}{2.095987in}}%
\pgfpathclose%
\pgfusepath{fill}%
\end{pgfscope}%
\begin{pgfscope}%
\pgfpathrectangle{\pgfqpoint{0.539299in}{0.078740in}}{\pgfqpoint{7.842520in}{7.842520in}}%
\pgfusepath{clip}%
\pgfsetbuttcap%
\pgfsetroundjoin%
\definecolor{currentfill}{rgb}{0.239346,0.300855,0.540844}%
\pgfsetfillcolor{currentfill}%
\pgfsetlinewidth{0.000000pt}%
\definecolor{currentstroke}{rgb}{0.430983,0.808473,0.346476}%
\pgfsetstrokecolor{currentstroke}%
\pgfsetdash{}{0pt}%
\pgfpathmoveto{\pgfqpoint{5.536081in}{1.836733in}}%
\pgfpathlineto{\pgfqpoint{5.308523in}{1.948082in}}%
\pgfpathlineto{\pgfqpoint{5.454351in}{1.741746in}}%
\pgfpathclose%
\pgfusepath{fill}%
\end{pgfscope}%
\begin{pgfscope}%
\pgfpathrectangle{\pgfqpoint{0.539299in}{0.078740in}}{\pgfqpoint{7.842520in}{7.842520in}}%
\pgfusepath{clip}%
\pgfsetbuttcap%
\pgfsetroundjoin%
\definecolor{currentfill}{rgb}{0.175841,0.441290,0.557685}%
\pgfsetfillcolor{currentfill}%
\pgfsetlinewidth{0.000000pt}%
\definecolor{currentstroke}{rgb}{0.440137,0.811138,0.340967}%
\pgfsetstrokecolor{currentstroke}%
\pgfsetdash{}{0pt}%
\pgfpathmoveto{\pgfqpoint{4.346469in}{2.608126in}}%
\pgfpathlineto{\pgfqpoint{4.262595in}{2.193629in}}%
\pgfpathlineto{\pgfqpoint{4.492593in}{2.485503in}}%
\pgfpathclose%
\pgfusepath{fill}%
\end{pgfscope}%
\begin{pgfscope}%
\pgfpathrectangle{\pgfqpoint{0.539299in}{0.078740in}}{\pgfqpoint{7.842520in}{7.842520in}}%
\pgfusepath{clip}%
\pgfsetbuttcap%
\pgfsetroundjoin%
\definecolor{currentfill}{rgb}{0.120638,0.625828,0.533488}%
\pgfsetfillcolor{currentfill}%
\pgfsetlinewidth{0.000000pt}%
\definecolor{currentstroke}{rgb}{0.449368,0.813768,0.335384}%
\pgfsetstrokecolor{currentstroke}%
\pgfsetdash{}{0pt}%
\pgfpathmoveto{\pgfqpoint{4.284682in}{3.071051in}}%
\pgfpathlineto{\pgfqpoint{4.369172in}{3.328258in}}%
\pgfpathlineto{\pgfqpoint{4.223116in}{3.486528in}}%
\pgfpathclose%
\pgfusepath{fill}%
\end{pgfscope}%
\begin{pgfscope}%
\pgfpathrectangle{\pgfqpoint{0.539299in}{0.078740in}}{\pgfqpoint{7.842520in}{7.842520in}}%
\pgfusepath{clip}%
\pgfsetbuttcap%
\pgfsetroundjoin%
\definecolor{currentfill}{rgb}{0.119699,0.618490,0.536347}%
\pgfsetfillcolor{currentfill}%
\pgfsetlinewidth{0.000000pt}%
\definecolor{currentstroke}{rgb}{0.458674,0.816363,0.329727}%
\pgfsetstrokecolor{currentstroke}%
\pgfsetdash{}{0pt}%
\pgfpathmoveto{\pgfqpoint{4.223116in}{3.486528in}}%
\pgfpathlineto{\pgfqpoint{4.138693in}{3.208126in}}%
\pgfpathlineto{\pgfqpoint{4.284682in}{3.071051in}}%
\pgfpathclose%
\pgfusepath{fill}%
\end{pgfscope}%
\begin{pgfscope}%
\pgfpathrectangle{\pgfqpoint{0.539299in}{0.078740in}}{\pgfqpoint{7.842520in}{7.842520in}}%
\pgfusepath{clip}%
\pgfsetbuttcap%
\pgfsetroundjoin%
\definecolor{currentfill}{rgb}{0.119738,0.603785,0.541400}%
\pgfsetfillcolor{currentfill}%
\pgfsetlinewidth{0.000000pt}%
\definecolor{currentstroke}{rgb}{0.468053,0.818921,0.323998}%
\pgfsetstrokecolor{currentstroke}%
\pgfsetdash{}{0pt}%
\pgfpathmoveto{\pgfqpoint{4.515273in}{3.165074in}}%
\pgfpathlineto{\pgfqpoint{4.369172in}{3.328258in}}%
\pgfpathlineto{\pgfqpoint{4.284682in}{3.071051in}}%
\pgfpathclose%
\pgfusepath{fill}%
\end{pgfscope}%
\begin{pgfscope}%
\pgfpathrectangle{\pgfqpoint{0.539299in}{0.078740in}}{\pgfqpoint{7.842520in}{7.842520in}}%
\pgfusepath{clip}%
\pgfsetbuttcap%
\pgfsetroundjoin%
\definecolor{currentfill}{rgb}{0.183898,0.422383,0.556944}%
\pgfsetfillcolor{currentfill}%
\pgfsetlinewidth{0.000000pt}%
\definecolor{currentstroke}{rgb}{0.477504,0.821444,0.318195}%
\pgfsetstrokecolor{currentstroke}%
\pgfsetdash{}{0pt}%
\pgfpathmoveto{\pgfqpoint{5.245331in}{2.266200in}}%
\pgfpathlineto{\pgfqpoint{5.099519in}{2.460623in}}%
\pgfpathlineto{\pgfqpoint{5.016082in}{2.310084in}}%
\pgfpathclose%
\pgfusepath{fill}%
\end{pgfscope}%
\begin{pgfscope}%
\pgfpathrectangle{\pgfqpoint{0.539299in}{0.078740in}}{\pgfqpoint{7.842520in}{7.842520in}}%
\pgfusepath{clip}%
\pgfsetbuttcap%
\pgfsetroundjoin%
\definecolor{currentfill}{rgb}{0.143343,0.522773,0.556295}%
\pgfsetfillcolor{currentfill}%
\pgfsetlinewidth{0.000000pt}%
\definecolor{currentstroke}{rgb}{0.487026,0.823929,0.312321}%
\pgfsetstrokecolor{currentstroke}%
\pgfsetdash{}{0pt}%
\pgfpathmoveto{\pgfqpoint{4.284682in}{3.071051in}}%
\pgfpathlineto{\pgfqpoint{4.200551in}{2.726746in}}%
\pgfpathlineto{\pgfqpoint{4.346469in}{2.608126in}}%
\pgfpathclose%
\pgfusepath{fill}%
\end{pgfscope}%
\begin{pgfscope}%
\pgfpathrectangle{\pgfqpoint{0.539299in}{0.078740in}}{\pgfqpoint{7.842520in}{7.842520in}}%
\pgfusepath{clip}%
\pgfsetbuttcap%
\pgfsetroundjoin%
\definecolor{currentfill}{rgb}{0.171176,0.452530,0.557965}%
\pgfsetfillcolor{currentfill}%
\pgfsetlinewidth{0.000000pt}%
\definecolor{currentstroke}{rgb}{0.496615,0.826376,0.306377}%
\pgfsetstrokecolor{currentstroke}%
\pgfsetdash{}{0pt}%
\pgfpathmoveto{\pgfqpoint{5.099519in}{2.460623in}}%
\pgfpathlineto{\pgfqpoint{4.953553in}{2.645992in}}%
\pgfpathlineto{\pgfqpoint{5.016082in}{2.310084in}}%
\pgfpathclose%
\pgfusepath{fill}%
\end{pgfscope}%
\begin{pgfscope}%
\pgfpathrectangle{\pgfqpoint{0.539299in}{0.078740in}}{\pgfqpoint{7.842520in}{7.842520in}}%
\pgfusepath{clip}%
\pgfsetbuttcap%
\pgfsetroundjoin%
\definecolor{currentfill}{rgb}{0.137770,0.537492,0.554906}%
\pgfsetfillcolor{currentfill}%
\pgfsetlinewidth{0.000000pt}%
\definecolor{currentstroke}{rgb}{0.506271,0.828786,0.300362}%
\pgfsetstrokecolor{currentstroke}%
\pgfsetdash{}{0pt}%
\pgfpathmoveto{\pgfqpoint{4.661391in}{2.997194in}}%
\pgfpathlineto{\pgfqpoint{4.577041in}{2.783012in}}%
\pgfpathlineto{\pgfqpoint{4.807495in}{2.824405in}}%
\pgfpathclose%
\pgfusepath{fill}%
\end{pgfscope}%
\begin{pgfscope}%
\pgfpathrectangle{\pgfqpoint{0.539299in}{0.078740in}}{\pgfqpoint{7.842520in}{7.842520in}}%
\pgfusepath{clip}%
\pgfsetbuttcap%
\pgfsetroundjoin%
\definecolor{currentfill}{rgb}{0.129933,0.559582,0.551864}%
\pgfsetfillcolor{currentfill}%
\pgfsetlinewidth{0.000000pt}%
\definecolor{currentstroke}{rgb}{0.515992,0.831158,0.294279}%
\pgfsetstrokecolor{currentstroke}%
\pgfsetdash{}{0pt}%
\pgfpathmoveto{\pgfqpoint{4.577041in}{2.783012in}}%
\pgfpathlineto{\pgfqpoint{4.661391in}{2.997194in}}%
\pgfpathlineto{\pgfqpoint{4.515273in}{3.165074in}}%
\pgfpathclose%
\pgfusepath{fill}%
\end{pgfscope}%
\begin{pgfscope}%
\pgfpathrectangle{\pgfqpoint{0.539299in}{0.078740in}}{\pgfqpoint{7.842520in}{7.842520in}}%
\pgfusepath{clip}%
\pgfsetbuttcap%
\pgfsetroundjoin%
\definecolor{currentfill}{rgb}{0.151918,0.500685,0.557587}%
\pgfsetfillcolor{currentfill}%
\pgfsetlinewidth{0.000000pt}%
\definecolor{currentstroke}{rgb}{0.525776,0.833491,0.288127}%
\pgfsetstrokecolor{currentstroke}%
\pgfsetdash{}{0pt}%
\pgfpathmoveto{\pgfqpoint{4.953553in}{2.645992in}}%
\pgfpathlineto{\pgfqpoint{4.807495in}{2.824405in}}%
\pgfpathlineto{\pgfqpoint{4.723356in}{2.631725in}}%
\pgfpathclose%
\pgfusepath{fill}%
\end{pgfscope}%
\begin{pgfscope}%
\pgfpathrectangle{\pgfqpoint{0.539299in}{0.078740in}}{\pgfqpoint{7.842520in}{7.842520in}}%
\pgfusepath{clip}%
\pgfsetbuttcap%
\pgfsetroundjoin%
\definecolor{currentfill}{rgb}{0.124395,0.578002,0.548287}%
\pgfsetfillcolor{currentfill}%
\pgfsetlinewidth{0.000000pt}%
\definecolor{currentstroke}{rgb}{0.535621,0.835785,0.281908}%
\pgfsetstrokecolor{currentstroke}%
\pgfsetdash{}{0pt}%
\pgfpathmoveto{\pgfqpoint{4.284682in}{3.071051in}}%
\pgfpathlineto{\pgfqpoint{4.430807in}{2.929332in}}%
\pgfpathlineto{\pgfqpoint{4.515273in}{3.165074in}}%
\pgfpathclose%
\pgfusepath{fill}%
\end{pgfscope}%
\begin{pgfscope}%
\pgfpathrectangle{\pgfqpoint{0.539299in}{0.078740in}}{\pgfqpoint{7.842520in}{7.842520in}}%
\pgfusepath{clip}%
\pgfsetbuttcap%
\pgfsetroundjoin%
\definecolor{currentfill}{rgb}{0.199430,0.387607,0.554642}%
\pgfsetfillcolor{currentfill}%
\pgfsetlinewidth{0.000000pt}%
\definecolor{currentstroke}{rgb}{0.545524,0.838039,0.275626}%
\pgfsetstrokecolor{currentstroke}%
\pgfsetdash{}{0pt}%
\pgfpathmoveto{\pgfqpoint{4.492593in}{2.485503in}}%
\pgfpathlineto{\pgfqpoint{4.408361in}{2.095987in}}%
\pgfpathlineto{\pgfqpoint{4.554418in}{1.995003in}}%
\pgfpathclose%
\pgfusepath{fill}%
\end{pgfscope}%
\begin{pgfscope}%
\pgfpathrectangle{\pgfqpoint{0.539299in}{0.078740in}}{\pgfqpoint{7.842520in}{7.842520in}}%
\pgfusepath{clip}%
\pgfsetbuttcap%
\pgfsetroundjoin%
\definecolor{currentfill}{rgb}{0.137770,0.537492,0.554906}%
\pgfsetfillcolor{currentfill}%
\pgfsetlinewidth{0.000000pt}%
\definecolor{currentstroke}{rgb}{0.555484,0.840254,0.269281}%
\pgfsetstrokecolor{currentstroke}%
\pgfsetdash{}{0pt}%
\pgfpathmoveto{\pgfqpoint{4.346469in}{2.608126in}}%
\pgfpathlineto{\pgfqpoint{4.430807in}{2.929332in}}%
\pgfpathlineto{\pgfqpoint{4.284682in}{3.071051in}}%
\pgfpathclose%
\pgfusepath{fill}%
\end{pgfscope}%
\begin{pgfscope}%
\pgfpathrectangle{\pgfqpoint{0.539299in}{0.078740in}}{\pgfqpoint{7.842520in}{7.842520in}}%
\pgfusepath{clip}%
\pgfsetbuttcap%
\pgfsetroundjoin%
\definecolor{currentfill}{rgb}{0.243113,0.292092,0.538516}%
\pgfsetfillcolor{currentfill}%
\pgfsetlinewidth{0.000000pt}%
\definecolor{currentstroke}{rgb}{0.565498,0.842430,0.262877}%
\pgfsetstrokecolor{currentstroke}%
\pgfsetdash{}{0pt}%
\pgfpathmoveto{\pgfqpoint{4.616159in}{1.458069in}}%
\pgfpathlineto{\pgfqpoint{4.700745in}{1.889957in}}%
\pgfpathlineto{\pgfqpoint{4.554418in}{1.995003in}}%
\pgfpathclose%
\pgfusepath{fill}%
\end{pgfscope}%
\begin{pgfscope}%
\pgfpathrectangle{\pgfqpoint{0.539299in}{0.078740in}}{\pgfqpoint{7.842520in}{7.842520in}}%
\pgfusepath{clip}%
\pgfsetbuttcap%
\pgfsetroundjoin%
\definecolor{currentfill}{rgb}{0.131172,0.555899,0.552459}%
\pgfsetfillcolor{currentfill}%
\pgfsetlinewidth{0.000000pt}%
\definecolor{currentstroke}{rgb}{0.575563,0.844566,0.256415}%
\pgfsetstrokecolor{currentstroke}%
\pgfsetdash{}{0pt}%
\pgfpathmoveto{\pgfqpoint{4.515273in}{3.165074in}}%
\pgfpathlineto{\pgfqpoint{4.430807in}{2.929332in}}%
\pgfpathlineto{\pgfqpoint{4.577041in}{2.783012in}}%
\pgfpathclose%
\pgfusepath{fill}%
\end{pgfscope}%
\begin{pgfscope}%
\pgfpathrectangle{\pgfqpoint{0.539299in}{0.078740in}}{\pgfqpoint{7.842520in}{7.842520in}}%
\pgfusepath{clip}%
\pgfsetbuttcap%
\pgfsetroundjoin%
\definecolor{currentfill}{rgb}{0.206756,0.371758,0.553117}%
\pgfsetfillcolor{currentfill}%
\pgfsetlinewidth{0.000000pt}%
\definecolor{currentstroke}{rgb}{0.585678,0.846661,0.249897}%
\pgfsetstrokecolor{currentstroke}%
\pgfsetdash{}{0pt}%
\pgfpathmoveto{\pgfqpoint{5.245331in}{2.266200in}}%
\pgfpathlineto{\pgfqpoint{5.162383in}{2.135782in}}%
\pgfpathlineto{\pgfqpoint{5.308523in}{1.948082in}}%
\pgfpathclose%
\pgfusepath{fill}%
\end{pgfscope}%
\begin{pgfscope}%
\pgfpathrectangle{\pgfqpoint{0.539299in}{0.078740in}}{\pgfqpoint{7.842520in}{7.842520in}}%
\pgfusepath{clip}%
\pgfsetbuttcap%
\pgfsetroundjoin%
\definecolor{currentfill}{rgb}{0.153364,0.497000,0.557724}%
\pgfsetfillcolor{currentfill}%
\pgfsetlinewidth{0.000000pt}%
\definecolor{currentstroke}{rgb}{0.595839,0.848717,0.243329}%
\pgfsetstrokecolor{currentstroke}%
\pgfsetdash{}{0pt}%
\pgfpathmoveto{\pgfqpoint{4.492593in}{2.485503in}}%
\pgfpathlineto{\pgfqpoint{4.430807in}{2.929332in}}%
\pgfpathlineto{\pgfqpoint{4.346469in}{2.608126in}}%
\pgfpathclose%
\pgfusepath{fill}%
\end{pgfscope}%
\begin{pgfscope}%
\pgfpathrectangle{\pgfqpoint{0.539299in}{0.078740in}}{\pgfqpoint{7.842520in}{7.842520in}}%
\pgfusepath{clip}%
\pgfsetbuttcap%
\pgfsetroundjoin%
\definecolor{currentfill}{rgb}{0.192357,0.403199,0.555836}%
\pgfsetfillcolor{currentfill}%
\pgfsetlinewidth{0.000000pt}%
\definecolor{currentstroke}{rgb}{0.606045,0.850733,0.236712}%
\pgfsetstrokecolor{currentstroke}%
\pgfsetdash{}{0pt}%
\pgfpathmoveto{\pgfqpoint{5.016082in}{2.310084in}}%
\pgfpathlineto{\pgfqpoint{5.162383in}{2.135782in}}%
\pgfpathlineto{\pgfqpoint{5.245331in}{2.266200in}}%
\pgfpathclose%
\pgfusepath{fill}%
\end{pgfscope}%
\begin{pgfscope}%
\pgfpathrectangle{\pgfqpoint{0.539299in}{0.078740in}}{\pgfqpoint{7.842520in}{7.842520in}}%
\pgfusepath{clip}%
\pgfsetbuttcap%
\pgfsetroundjoin%
\definecolor{currentfill}{rgb}{0.147607,0.511733,0.557049}%
\pgfsetfillcolor{currentfill}%
\pgfsetlinewidth{0.000000pt}%
\definecolor{currentstroke}{rgb}{0.616293,0.852709,0.230052}%
\pgfsetstrokecolor{currentstroke}%
\pgfsetdash{}{0pt}%
\pgfpathmoveto{\pgfqpoint{4.807495in}{2.824405in}}%
\pgfpathlineto{\pgfqpoint{4.577041in}{2.783012in}}%
\pgfpathlineto{\pgfqpoint{4.723356in}{2.631725in}}%
\pgfpathclose%
\pgfusepath{fill}%
\end{pgfscope}%
\begin{pgfscope}%
\pgfpathrectangle{\pgfqpoint{0.539299in}{0.078740in}}{\pgfqpoint{7.842520in}{7.842520in}}%
\pgfusepath{clip}%
\pgfsetbuttcap%
\pgfsetroundjoin%
\definecolor{currentfill}{rgb}{0.160665,0.478540,0.558115}%
\pgfsetfillcolor{currentfill}%
\pgfsetlinewidth{0.000000pt}%
\definecolor{currentstroke}{rgb}{0.626579,0.854645,0.223353}%
\pgfsetstrokecolor{currentstroke}%
\pgfsetdash{}{0pt}%
\pgfpathmoveto{\pgfqpoint{4.723356in}{2.631725in}}%
\pgfpathlineto{\pgfqpoint{4.869718in}{2.474594in}}%
\pgfpathlineto{\pgfqpoint{4.953553in}{2.645992in}}%
\pgfpathclose%
\pgfusepath{fill}%
\end{pgfscope}%
\begin{pgfscope}%
\pgfpathrectangle{\pgfqpoint{0.539299in}{0.078740in}}{\pgfqpoint{7.842520in}{7.842520in}}%
\pgfusepath{clip}%
\pgfsetbuttcap%
\pgfsetroundjoin%
\definecolor{currentfill}{rgb}{0.266580,0.228262,0.514349}%
\pgfsetfillcolor{currentfill}%
\pgfsetlinewidth{0.000000pt}%
\definecolor{currentstroke}{rgb}{0.636902,0.856542,0.216620}%
\pgfsetstrokecolor{currentstroke}%
\pgfsetdash{}{0pt}%
\pgfpathmoveto{\pgfqpoint{4.616159in}{1.458069in}}%
\pgfpathlineto{\pgfqpoint{4.762457in}{1.378786in}}%
\pgfpathlineto{\pgfqpoint{4.847314in}{1.779621in}}%
\pgfpathclose%
\pgfusepath{fill}%
\end{pgfscope}%
\begin{pgfscope}%
\pgfpathrectangle{\pgfqpoint{0.539299in}{0.078740in}}{\pgfqpoint{7.842520in}{7.842520in}}%
\pgfusepath{clip}%
\pgfsetbuttcap%
\pgfsetroundjoin%
\definecolor{currentfill}{rgb}{0.266580,0.228262,0.514349}%
\pgfsetfillcolor{currentfill}%
\pgfsetlinewidth{0.000000pt}%
\definecolor{currentstroke}{rgb}{0.647257,0.858400,0.209861}%
\pgfsetstrokecolor{currentstroke}%
\pgfsetdash{}{0pt}%
\pgfpathmoveto{\pgfqpoint{5.517631in}{1.390496in}}%
\pgfpathlineto{\pgfqpoint{5.599643in}{1.509341in}}%
\pgfpathlineto{\pgfqpoint{5.454351in}{1.741746in}}%
\pgfpathclose%
\pgfusepath{fill}%
\end{pgfscope}%
\begin{pgfscope}%
\pgfpathrectangle{\pgfqpoint{0.539299in}{0.078740in}}{\pgfqpoint{7.842520in}{7.842520in}}%
\pgfusepath{clip}%
\pgfsetbuttcap%
\pgfsetroundjoin%
\definecolor{currentfill}{rgb}{0.169646,0.456262,0.558030}%
\pgfsetfillcolor{currentfill}%
\pgfsetlinewidth{0.000000pt}%
\definecolor{currentstroke}{rgb}{0.657642,0.860219,0.203082}%
\pgfsetstrokecolor{currentstroke}%
\pgfsetdash{}{0pt}%
\pgfpathmoveto{\pgfqpoint{5.016082in}{2.310084in}}%
\pgfpathlineto{\pgfqpoint{4.953553in}{2.645992in}}%
\pgfpathlineto{\pgfqpoint{4.869718in}{2.474594in}}%
\pgfpathclose%
\pgfusepath{fill}%
\end{pgfscope}%
\begin{pgfscope}%
\pgfpathrectangle{\pgfqpoint{0.539299in}{0.078740in}}{\pgfqpoint{7.842520in}{7.842520in}}%
\pgfusepath{clip}%
\pgfsetbuttcap%
\pgfsetroundjoin%
\definecolor{currentfill}{rgb}{0.147607,0.511733,0.557049}%
\pgfsetfillcolor{currentfill}%
\pgfsetlinewidth{0.000000pt}%
\definecolor{currentstroke}{rgb}{0.668054,0.861999,0.196293}%
\pgfsetstrokecolor{currentstroke}%
\pgfsetdash{}{0pt}%
\pgfpathmoveto{\pgfqpoint{4.492593in}{2.485503in}}%
\pgfpathlineto{\pgfqpoint{4.577041in}{2.783012in}}%
\pgfpathlineto{\pgfqpoint{4.430807in}{2.929332in}}%
\pgfpathclose%
\pgfusepath{fill}%
\end{pgfscope}%
\begin{pgfscope}%
\pgfpathrectangle{\pgfqpoint{0.539299in}{0.078740in}}{\pgfqpoint{7.842520in}{7.842520in}}%
\pgfusepath{clip}%
\pgfsetbuttcap%
\pgfsetroundjoin%
\definecolor{currentfill}{rgb}{0.278826,0.175490,0.483397}%
\pgfsetfillcolor{currentfill}%
\pgfsetlinewidth{0.000000pt}%
\definecolor{currentstroke}{rgb}{0.678489,0.863742,0.189503}%
\pgfsetstrokecolor{currentstroke}%
\pgfsetdash{}{0pt}%
\pgfpathmoveto{\pgfqpoint{5.517631in}{1.390496in}}%
\pgfpathlineto{\pgfqpoint{5.662874in}{1.144010in}}%
\pgfpathlineto{\pgfqpoint{5.599643in}{1.509341in}}%
\pgfpathclose%
\pgfusepath{fill}%
\end{pgfscope}%
\begin{pgfscope}%
\pgfpathrectangle{\pgfqpoint{0.539299in}{0.078740in}}{\pgfqpoint{7.842520in}{7.842520in}}%
\pgfusepath{clip}%
\pgfsetbuttcap%
\pgfsetroundjoin%
\definecolor{currentfill}{rgb}{0.188923,0.410910,0.556326}%
\pgfsetfillcolor{currentfill}%
\pgfsetlinewidth{0.000000pt}%
\definecolor{currentstroke}{rgb}{0.688944,0.865448,0.182725}%
\pgfsetstrokecolor{currentstroke}%
\pgfsetdash{}{0pt}%
\pgfpathmoveto{\pgfqpoint{4.554418in}{1.995003in}}%
\pgfpathlineto{\pgfqpoint{4.638902in}{2.358455in}}%
\pgfpathlineto{\pgfqpoint{4.492593in}{2.485503in}}%
\pgfpathclose%
\pgfusepath{fill}%
\end{pgfscope}%
\begin{pgfscope}%
\pgfpathrectangle{\pgfqpoint{0.539299in}{0.078740in}}{\pgfqpoint{7.842520in}{7.842520in}}%
\pgfusepath{clip}%
\pgfsetbuttcap%
\pgfsetroundjoin%
\definecolor{currentfill}{rgb}{0.250425,0.274290,0.533103}%
\pgfsetfillcolor{currentfill}%
\pgfsetlinewidth{0.000000pt}%
\definecolor{currentstroke}{rgb}{0.699415,0.867117,0.175971}%
\pgfsetstrokecolor{currentstroke}%
\pgfsetdash{}{0pt}%
\pgfpathmoveto{\pgfqpoint{4.847314in}{1.779621in}}%
\pgfpathlineto{\pgfqpoint{4.700745in}{1.889957in}}%
\pgfpathlineto{\pgfqpoint{4.616159in}{1.458069in}}%
\pgfpathclose%
\pgfusepath{fill}%
\end{pgfscope}%
\begin{pgfscope}%
\pgfpathrectangle{\pgfqpoint{0.539299in}{0.078740in}}{\pgfqpoint{7.842520in}{7.842520in}}%
\pgfusepath{clip}%
\pgfsetbuttcap%
\pgfsetroundjoin%
\definecolor{currentfill}{rgb}{0.156270,0.489624,0.557936}%
\pgfsetfillcolor{currentfill}%
\pgfsetlinewidth{0.000000pt}%
\definecolor{currentstroke}{rgb}{0.709898,0.868751,0.169257}%
\pgfsetstrokecolor{currentstroke}%
\pgfsetdash{}{0pt}%
\pgfpathmoveto{\pgfqpoint{4.723356in}{2.631725in}}%
\pgfpathlineto{\pgfqpoint{4.577041in}{2.783012in}}%
\pgfpathlineto{\pgfqpoint{4.492593in}{2.485503in}}%
\pgfpathclose%
\pgfusepath{fill}%
\end{pgfscope}%
\begin{pgfscope}%
\pgfpathrectangle{\pgfqpoint{0.539299in}{0.078740in}}{\pgfqpoint{7.842520in}{7.842520in}}%
\pgfusepath{clip}%
\pgfsetbuttcap%
\pgfsetroundjoin%
\definecolor{currentfill}{rgb}{0.168126,0.459988,0.558082}%
\pgfsetfillcolor{currentfill}%
\pgfsetlinewidth{0.000000pt}%
\definecolor{currentstroke}{rgb}{0.720391,0.870350,0.162603}%
\pgfsetstrokecolor{currentstroke}%
\pgfsetdash{}{0pt}%
\pgfpathmoveto{\pgfqpoint{4.492593in}{2.485503in}}%
\pgfpathlineto{\pgfqpoint{4.638902in}{2.358455in}}%
\pgfpathlineto{\pgfqpoint{4.723356in}{2.631725in}}%
\pgfpathclose%
\pgfusepath{fill}%
\end{pgfscope}%
\begin{pgfscope}%
\pgfpathrectangle{\pgfqpoint{0.539299in}{0.078740in}}{\pgfqpoint{7.842520in}{7.842520in}}%
\pgfusepath{clip}%
\pgfsetbuttcap%
\pgfsetroundjoin%
\definecolor{currentfill}{rgb}{0.212395,0.359683,0.551710}%
\pgfsetfillcolor{currentfill}%
\pgfsetlinewidth{0.000000pt}%
\definecolor{currentstroke}{rgb}{0.730889,0.871916,0.156029}%
\pgfsetstrokecolor{currentstroke}%
\pgfsetdash{}{0pt}%
\pgfpathmoveto{\pgfqpoint{4.554418in}{1.995003in}}%
\pgfpathlineto{\pgfqpoint{4.700745in}{1.889957in}}%
\pgfpathlineto{\pgfqpoint{4.785366in}{2.226099in}}%
\pgfpathclose%
\pgfusepath{fill}%
\end{pgfscope}%
\begin{pgfscope}%
\pgfpathrectangle{\pgfqpoint{0.539299in}{0.078740in}}{\pgfqpoint{7.842520in}{7.842520in}}%
\pgfusepath{clip}%
\pgfsetbuttcap%
\pgfsetroundjoin%
\definecolor{currentfill}{rgb}{0.195860,0.395433,0.555276}%
\pgfsetfillcolor{currentfill}%
\pgfsetlinewidth{0.000000pt}%
\definecolor{currentstroke}{rgb}{0.741388,0.873449,0.149561}%
\pgfsetstrokecolor{currentstroke}%
\pgfsetdash{}{0pt}%
\pgfpathmoveto{\pgfqpoint{4.785366in}{2.226099in}}%
\pgfpathlineto{\pgfqpoint{4.638902in}{2.358455in}}%
\pgfpathlineto{\pgfqpoint{4.554418in}{1.995003in}}%
\pgfpathclose%
\pgfusepath{fill}%
\end{pgfscope}%
\begin{pgfscope}%
\pgfpathrectangle{\pgfqpoint{0.539299in}{0.078740in}}{\pgfqpoint{7.842520in}{7.842520in}}%
\pgfusepath{clip}%
\pgfsetbuttcap%
\pgfsetroundjoin%
\definecolor{currentfill}{rgb}{0.168126,0.459988,0.558082}%
\pgfsetfillcolor{currentfill}%
\pgfsetlinewidth{0.000000pt}%
\definecolor{currentstroke}{rgb}{0.751884,0.874951,0.143228}%
\pgfsetstrokecolor{currentstroke}%
\pgfsetdash{}{0pt}%
\pgfpathmoveto{\pgfqpoint{4.723356in}{2.631725in}}%
\pgfpathlineto{\pgfqpoint{4.638902in}{2.358455in}}%
\pgfpathlineto{\pgfqpoint{4.869718in}{2.474594in}}%
\pgfpathclose%
\pgfusepath{fill}%
\end{pgfscope}%
\begin{pgfscope}%
\pgfpathrectangle{\pgfqpoint{0.539299in}{0.078740in}}{\pgfqpoint{7.842520in}{7.842520in}}%
\pgfusepath{clip}%
\pgfsetbuttcap%
\pgfsetroundjoin%
\definecolor{currentfill}{rgb}{0.243113,0.292092,0.538516}%
\pgfsetfillcolor{currentfill}%
\pgfsetlinewidth{0.000000pt}%
\definecolor{currentstroke}{rgb}{0.762373,0.876424,0.137064}%
\pgfsetstrokecolor{currentstroke}%
\pgfsetdash{}{0pt}%
\pgfpathmoveto{\pgfqpoint{5.371612in}{1.597343in}}%
\pgfpathlineto{\pgfqpoint{5.454351in}{1.741746in}}%
\pgfpathlineto{\pgfqpoint{5.308523in}{1.948082in}}%
\pgfpathclose%
\pgfusepath{fill}%
\end{pgfscope}%
\begin{pgfscope}%
\pgfpathrectangle{\pgfqpoint{0.539299in}{0.078740in}}{\pgfqpoint{7.842520in}{7.842520in}}%
\pgfusepath{clip}%
\pgfsetbuttcap%
\pgfsetroundjoin%
\definecolor{currentfill}{rgb}{0.269308,0.218818,0.509577}%
\pgfsetfillcolor{currentfill}%
\pgfsetlinewidth{0.000000pt}%
\definecolor{currentstroke}{rgb}{0.772852,0.877868,0.131109}%
\pgfsetstrokecolor{currentstroke}%
\pgfsetdash{}{0pt}%
\pgfpathmoveto{\pgfqpoint{4.847314in}{1.779621in}}%
\pgfpathlineto{\pgfqpoint{4.762457in}{1.378786in}}%
\pgfpathlineto{\pgfqpoint{4.909107in}{1.295428in}}%
\pgfpathclose%
\pgfusepath{fill}%
\end{pgfscope}%
\begin{pgfscope}%
\pgfpathrectangle{\pgfqpoint{0.539299in}{0.078740in}}{\pgfqpoint{7.842520in}{7.842520in}}%
\pgfusepath{clip}%
\pgfsetbuttcap%
\pgfsetroundjoin%
\definecolor{currentfill}{rgb}{0.180629,0.429975,0.557282}%
\pgfsetfillcolor{currentfill}%
\pgfsetlinewidth{0.000000pt}%
\definecolor{currentstroke}{rgb}{0.783315,0.879285,0.125405}%
\pgfsetstrokecolor{currentstroke}%
\pgfsetdash{}{0pt}%
\pgfpathmoveto{\pgfqpoint{5.016082in}{2.310084in}}%
\pgfpathlineto{\pgfqpoint{4.869718in}{2.474594in}}%
\pgfpathlineto{\pgfqpoint{4.785366in}{2.226099in}}%
\pgfpathclose%
\pgfusepath{fill}%
\end{pgfscope}%
\begin{pgfscope}%
\pgfpathrectangle{\pgfqpoint{0.539299in}{0.078740in}}{\pgfqpoint{7.842520in}{7.842520in}}%
\pgfusepath{clip}%
\pgfsetbuttcap%
\pgfsetroundjoin%
\definecolor{currentfill}{rgb}{0.179019,0.433756,0.557430}%
\pgfsetfillcolor{currentfill}%
\pgfsetlinewidth{0.000000pt}%
\definecolor{currentstroke}{rgb}{0.793760,0.880678,0.120005}%
\pgfsetstrokecolor{currentstroke}%
\pgfsetdash{}{0pt}%
\pgfpathmoveto{\pgfqpoint{4.638902in}{2.358455in}}%
\pgfpathlineto{\pgfqpoint{4.785366in}{2.226099in}}%
\pgfpathlineto{\pgfqpoint{4.869718in}{2.474594in}}%
\pgfpathclose%
\pgfusepath{fill}%
\end{pgfscope}%
\begin{pgfscope}%
\pgfpathrectangle{\pgfqpoint{0.539299in}{0.078740in}}{\pgfqpoint{7.842520in}{7.842520in}}%
\pgfusepath{clip}%
\pgfsetbuttcap%
\pgfsetroundjoin%
\definecolor{currentfill}{rgb}{0.214298,0.355619,0.551184}%
\pgfsetfillcolor{currentfill}%
\pgfsetlinewidth{0.000000pt}%
\definecolor{currentstroke}{rgb}{0.804182,0.882046,0.114965}%
\pgfsetstrokecolor{currentstroke}%
\pgfsetdash{}{0pt}%
\pgfpathmoveto{\pgfqpoint{5.308523in}{1.948082in}}%
\pgfpathlineto{\pgfqpoint{5.162383in}{2.135782in}}%
\pgfpathlineto{\pgfqpoint{5.078583in}{1.938515in}}%
\pgfpathclose%
\pgfusepath{fill}%
\end{pgfscope}%
\begin{pgfscope}%
\pgfpathrectangle{\pgfqpoint{0.539299in}{0.078740in}}{\pgfqpoint{7.842520in}{7.842520in}}%
\pgfusepath{clip}%
\pgfsetbuttcap%
\pgfsetroundjoin%
\definecolor{currentfill}{rgb}{0.260571,0.246922,0.522828}%
\pgfsetfillcolor{currentfill}%
\pgfsetlinewidth{0.000000pt}%
\definecolor{currentstroke}{rgb}{0.814576,0.883393,0.110347}%
\pgfsetstrokecolor{currentstroke}%
\pgfsetdash{}{0pt}%
\pgfpathmoveto{\pgfqpoint{5.454351in}{1.741746in}}%
\pgfpathlineto{\pgfqpoint{5.371612in}{1.597343in}}%
\pgfpathlineto{\pgfqpoint{5.517631in}{1.390496in}}%
\pgfpathclose%
\pgfusepath{fill}%
\end{pgfscope}%
\begin{pgfscope}%
\pgfpathrectangle{\pgfqpoint{0.539299in}{0.078740in}}{\pgfqpoint{7.842520in}{7.842520in}}%
\pgfusepath{clip}%
\pgfsetbuttcap%
\pgfsetroundjoin%
\definecolor{currentfill}{rgb}{0.201239,0.383670,0.554294}%
\pgfsetfillcolor{currentfill}%
\pgfsetlinewidth{0.000000pt}%
\definecolor{currentstroke}{rgb}{0.824940,0.884720,0.106217}%
\pgfsetstrokecolor{currentstroke}%
\pgfsetdash{}{0pt}%
\pgfpathmoveto{\pgfqpoint{5.078583in}{1.938515in}}%
\pgfpathlineto{\pgfqpoint{5.162383in}{2.135782in}}%
\pgfpathlineto{\pgfqpoint{5.016082in}{2.310084in}}%
\pgfpathclose%
\pgfusepath{fill}%
\end{pgfscope}%
\begin{pgfscope}%
\pgfpathrectangle{\pgfqpoint{0.539299in}{0.078740in}}{\pgfqpoint{7.842520in}{7.842520in}}%
\pgfusepath{clip}%
\pgfsetbuttcap%
\pgfsetroundjoin%
\definecolor{currentfill}{rgb}{0.206756,0.371758,0.553117}%
\pgfsetfillcolor{currentfill}%
\pgfsetlinewidth{0.000000pt}%
\definecolor{currentstroke}{rgb}{0.835270,0.886029,0.102646}%
\pgfsetstrokecolor{currentstroke}%
\pgfsetdash{}{0pt}%
\pgfpathmoveto{\pgfqpoint{4.931945in}{2.086920in}}%
\pgfpathlineto{\pgfqpoint{4.785366in}{2.226099in}}%
\pgfpathlineto{\pgfqpoint{4.700745in}{1.889957in}}%
\pgfpathclose%
\pgfusepath{fill}%
\end{pgfscope}%
\begin{pgfscope}%
\pgfpathrectangle{\pgfqpoint{0.539299in}{0.078740in}}{\pgfqpoint{7.842520in}{7.842520in}}%
\pgfusepath{clip}%
\pgfsetbuttcap%
\pgfsetroundjoin%
\definecolor{currentfill}{rgb}{0.192357,0.403199,0.555836}%
\pgfsetfillcolor{currentfill}%
\pgfsetlinewidth{0.000000pt}%
\definecolor{currentstroke}{rgb}{0.845561,0.887322,0.099702}%
\pgfsetstrokecolor{currentstroke}%
\pgfsetdash{}{0pt}%
\pgfpathmoveto{\pgfqpoint{4.785366in}{2.226099in}}%
\pgfpathlineto{\pgfqpoint{4.931945in}{2.086920in}}%
\pgfpathlineto{\pgfqpoint{5.016082in}{2.310084in}}%
\pgfpathclose%
\pgfusepath{fill}%
\end{pgfscope}%
\begin{pgfscope}%
\pgfpathrectangle{\pgfqpoint{0.539299in}{0.078740in}}{\pgfqpoint{7.842520in}{7.842520in}}%
\pgfusepath{clip}%
\pgfsetbuttcap%
\pgfsetroundjoin%
\definecolor{currentfill}{rgb}{0.223925,0.334994,0.548053}%
\pgfsetfillcolor{currentfill}%
\pgfsetlinewidth{0.000000pt}%
\definecolor{currentstroke}{rgb}{0.855810,0.888601,0.097452}%
\pgfsetstrokecolor{currentstroke}%
\pgfsetdash{}{0pt}%
\pgfpathmoveto{\pgfqpoint{4.931945in}{2.086920in}}%
\pgfpathlineto{\pgfqpoint{4.700745in}{1.889957in}}%
\pgfpathlineto{\pgfqpoint{4.847314in}{1.779621in}}%
\pgfpathclose%
\pgfusepath{fill}%
\end{pgfscope}%
\begin{pgfscope}%
\pgfpathrectangle{\pgfqpoint{0.539299in}{0.078740in}}{\pgfqpoint{7.842520in}{7.842520in}}%
\pgfusepath{clip}%
\pgfsetbuttcap%
\pgfsetroundjoin%
\definecolor{currentfill}{rgb}{0.201239,0.383670,0.554294}%
\pgfsetfillcolor{currentfill}%
\pgfsetlinewidth{0.000000pt}%
\definecolor{currentstroke}{rgb}{0.866013,0.889868,0.095953}%
\pgfsetstrokecolor{currentstroke}%
\pgfsetdash{}{0pt}%
\pgfpathmoveto{\pgfqpoint{5.016082in}{2.310084in}}%
\pgfpathlineto{\pgfqpoint{4.931945in}{2.086920in}}%
\pgfpathlineto{\pgfqpoint{5.078583in}{1.938515in}}%
\pgfpathclose%
\pgfusepath{fill}%
\end{pgfscope}%
\begin{pgfscope}%
\pgfpathrectangle{\pgfqpoint{0.539299in}{0.078740in}}{\pgfqpoint{7.842520in}{7.842520in}}%
\pgfusepath{clip}%
\pgfsetbuttcap%
\pgfsetroundjoin%
\definecolor{currentfill}{rgb}{0.239346,0.300855,0.540844}%
\pgfsetfillcolor{currentfill}%
\pgfsetlinewidth{0.000000pt}%
\definecolor{currentstroke}{rgb}{0.876168,0.891125,0.095250}%
\pgfsetstrokecolor{currentstroke}%
\pgfsetdash{}{0pt}%
\pgfpathmoveto{\pgfqpoint{5.225186in}{1.777193in}}%
\pgfpathlineto{\pgfqpoint{5.371612in}{1.597343in}}%
\pgfpathlineto{\pgfqpoint{5.308523in}{1.948082in}}%
\pgfpathclose%
\pgfusepath{fill}%
\end{pgfscope}%
\begin{pgfscope}%
\pgfpathrectangle{\pgfqpoint{0.539299in}{0.078740in}}{\pgfqpoint{7.842520in}{7.842520in}}%
\pgfusepath{clip}%
\pgfsetbuttcap%
\pgfsetroundjoin%
\definecolor{currentfill}{rgb}{0.225863,0.330805,0.547314}%
\pgfsetfillcolor{currentfill}%
\pgfsetlinewidth{0.000000pt}%
\definecolor{currentstroke}{rgb}{0.886271,0.892374,0.095374}%
\pgfsetstrokecolor{currentstroke}%
\pgfsetdash{}{0pt}%
\pgfpathmoveto{\pgfqpoint{5.308523in}{1.948082in}}%
\pgfpathlineto{\pgfqpoint{5.078583in}{1.938515in}}%
\pgfpathlineto{\pgfqpoint{5.225186in}{1.777193in}}%
\pgfpathclose%
\pgfusepath{fill}%
\end{pgfscope}%
\begin{pgfscope}%
\pgfpathrectangle{\pgfqpoint{0.539299in}{0.078740in}}{\pgfqpoint{7.842520in}{7.842520in}}%
\pgfusepath{clip}%
\pgfsetbuttcap%
\pgfsetroundjoin%
\definecolor{currentfill}{rgb}{0.258965,0.251537,0.524736}%
\pgfsetfillcolor{currentfill}%
\pgfsetlinewidth{0.000000pt}%
\definecolor{currentstroke}{rgb}{0.896320,0.893616,0.096335}%
\pgfsetstrokecolor{currentstroke}%
\pgfsetdash{}{0pt}%
\pgfpathmoveto{\pgfqpoint{4.909107in}{1.295428in}}%
\pgfpathlineto{\pgfqpoint{4.994083in}{1.662022in}}%
\pgfpathlineto{\pgfqpoint{4.847314in}{1.779621in}}%
\pgfpathclose%
\pgfusepath{fill}%
\end{pgfscope}%
\begin{pgfscope}%
\pgfpathrectangle{\pgfqpoint{0.539299in}{0.078740in}}{\pgfqpoint{7.842520in}{7.842520in}}%
\pgfusepath{clip}%
\pgfsetbuttcap%
\pgfsetroundjoin%
\definecolor{currentfill}{rgb}{0.280868,0.160771,0.472899}%
\pgfsetfillcolor{currentfill}%
\pgfsetlinewidth{0.000000pt}%
\definecolor{currentstroke}{rgb}{0.906311,0.894855,0.098125}%
\pgfsetstrokecolor{currentstroke}%
\pgfsetdash{}{0pt}%
\pgfpathmoveto{\pgfqpoint{5.434633in}{1.223811in}}%
\pgfpathlineto{\pgfqpoint{5.662874in}{1.144010in}}%
\pgfpathlineto{\pgfqpoint{5.517631in}{1.390496in}}%
\pgfpathclose%
\pgfusepath{fill}%
\end{pgfscope}%
\begin{pgfscope}%
\pgfpathrectangle{\pgfqpoint{0.539299in}{0.078740in}}{\pgfqpoint{7.842520in}{7.842520in}}%
\pgfusepath{clip}%
\pgfsetbuttcap%
\pgfsetroundjoin%
\definecolor{currentfill}{rgb}{0.231674,0.318106,0.544834}%
\pgfsetfillcolor{currentfill}%
\pgfsetlinewidth{0.000000pt}%
\definecolor{currentstroke}{rgb}{0.916242,0.896091,0.100717}%
\pgfsetstrokecolor{currentstroke}%
\pgfsetdash{}{0pt}%
\pgfpathmoveto{\pgfqpoint{4.847314in}{1.779621in}}%
\pgfpathlineto{\pgfqpoint{4.994083in}{1.662022in}}%
\pgfpathlineto{\pgfqpoint{4.931945in}{2.086920in}}%
\pgfpathclose%
\pgfusepath{fill}%
\end{pgfscope}%
\begin{pgfscope}%
\pgfpathrectangle{\pgfqpoint{0.539299in}{0.078740in}}{\pgfqpoint{7.842520in}{7.842520in}}%
\pgfusepath{clip}%
\pgfsetbuttcap%
\pgfsetroundjoin%
\definecolor{currentfill}{rgb}{0.223925,0.334994,0.548053}%
\pgfsetfillcolor{currentfill}%
\pgfsetlinewidth{0.000000pt}%
\definecolor{currentstroke}{rgb}{0.926106,0.897330,0.104071}%
\pgfsetstrokecolor{currentstroke}%
\pgfsetdash{}{0pt}%
\pgfpathmoveto{\pgfqpoint{4.994083in}{1.662022in}}%
\pgfpathlineto{\pgfqpoint{5.078583in}{1.938515in}}%
\pgfpathlineto{\pgfqpoint{4.931945in}{2.086920in}}%
\pgfpathclose%
\pgfusepath{fill}%
\end{pgfscope}%
\begin{pgfscope}%
\pgfpathrectangle{\pgfqpoint{0.539299in}{0.078740in}}{\pgfqpoint{7.842520in}{7.842520in}}%
\pgfusepath{clip}%
\pgfsetbuttcap%
\pgfsetroundjoin%
\definecolor{currentfill}{rgb}{0.265145,0.232956,0.516599}%
\pgfsetfillcolor{currentfill}%
\pgfsetlinewidth{0.000000pt}%
\definecolor{currentstroke}{rgb}{0.935904,0.898570,0.108131}%
\pgfsetstrokecolor{currentstroke}%
\pgfsetdash{}{0pt}%
\pgfpathmoveto{\pgfqpoint{4.909107in}{1.295428in}}%
\pgfpathlineto{\pgfqpoint{5.140982in}{1.534035in}}%
\pgfpathlineto{\pgfqpoint{4.994083in}{1.662022in}}%
\pgfpathclose%
\pgfusepath{fill}%
\end{pgfscope}%
\begin{pgfscope}%
\pgfpathrectangle{\pgfqpoint{0.539299in}{0.078740in}}{\pgfqpoint{7.842520in}{7.842520in}}%
\pgfusepath{clip}%
\pgfsetbuttcap%
\pgfsetroundjoin%
\definecolor{currentfill}{rgb}{0.283187,0.125848,0.444960}%
\pgfsetfillcolor{currentfill}%
\pgfsetlinewidth{0.000000pt}%
\definecolor{currentstroke}{rgb}{0.945636,0.899815,0.112838}%
\pgfsetstrokecolor{currentstroke}%
\pgfsetdash{}{0pt}%
\pgfpathmoveto{\pgfqpoint{5.434633in}{1.223811in}}%
\pgfpathlineto{\pgfqpoint{5.580846in}{1.019988in}}%
\pgfpathlineto{\pgfqpoint{5.662874in}{1.144010in}}%
\pgfpathclose%
\pgfusepath{fill}%
\end{pgfscope}%
\begin{pgfscope}%
\pgfpathrectangle{\pgfqpoint{0.539299in}{0.078740in}}{\pgfqpoint{7.842520in}{7.842520in}}%
\pgfusepath{clip}%
\pgfsetbuttcap%
\pgfsetroundjoin%
\definecolor{currentfill}{rgb}{0.276194,0.190074,0.493001}%
\pgfsetfillcolor{currentfill}%
\pgfsetlinewidth{0.000000pt}%
\definecolor{currentstroke}{rgb}{0.955300,0.901065,0.118128}%
\pgfsetstrokecolor{currentstroke}%
\pgfsetdash{}{0pt}%
\pgfpathmoveto{\pgfqpoint{5.056074in}{1.206164in}}%
\pgfpathlineto{\pgfqpoint{5.140982in}{1.534035in}}%
\pgfpathlineto{\pgfqpoint{4.909107in}{1.295428in}}%
\pgfpathclose%
\pgfusepath{fill}%
\end{pgfscope}%
\begin{pgfscope}%
\pgfpathrectangle{\pgfqpoint{0.539299in}{0.078740in}}{\pgfqpoint{7.842520in}{7.842520in}}%
\pgfusepath{clip}%
\pgfsetbuttcap%
\pgfsetroundjoin%
\definecolor{currentfill}{rgb}{0.267968,0.223549,0.512008}%
\pgfsetfillcolor{currentfill}%
\pgfsetlinewidth{0.000000pt}%
\definecolor{currentstroke}{rgb}{0.964894,0.902323,0.123941}%
\pgfsetstrokecolor{currentstroke}%
\pgfsetdash{}{0pt}%
\pgfpathmoveto{\pgfqpoint{5.517631in}{1.390496in}}%
\pgfpathlineto{\pgfqpoint{5.371612in}{1.597343in}}%
\pgfpathlineto{\pgfqpoint{5.287897in}{1.390668in}}%
\pgfpathclose%
\pgfusepath{fill}%
\end{pgfscope}%
\begin{pgfscope}%
\pgfpathrectangle{\pgfqpoint{0.539299in}{0.078740in}}{\pgfqpoint{7.842520in}{7.842520in}}%
\pgfusepath{clip}%
\pgfsetbuttcap%
\pgfsetroundjoin%
\definecolor{currentfill}{rgb}{0.239346,0.300855,0.540844}%
\pgfsetfillcolor{currentfill}%
\pgfsetlinewidth{0.000000pt}%
\definecolor{currentstroke}{rgb}{0.974417,0.903590,0.130215}%
\pgfsetstrokecolor{currentstroke}%
\pgfsetdash{}{0pt}%
\pgfpathmoveto{\pgfqpoint{5.078583in}{1.938515in}}%
\pgfpathlineto{\pgfqpoint{5.140982in}{1.534035in}}%
\pgfpathlineto{\pgfqpoint{5.225186in}{1.777193in}}%
\pgfpathclose%
\pgfusepath{fill}%
\end{pgfscope}%
\begin{pgfscope}%
\pgfpathrectangle{\pgfqpoint{0.539299in}{0.078740in}}{\pgfqpoint{7.842520in}{7.842520in}}%
\pgfusepath{clip}%
\pgfsetbuttcap%
\pgfsetroundjoin%
\definecolor{currentfill}{rgb}{0.243113,0.292092,0.538516}%
\pgfsetfillcolor{currentfill}%
\pgfsetlinewidth{0.000000pt}%
\definecolor{currentstroke}{rgb}{0.983868,0.904867,0.136897}%
\pgfsetstrokecolor{currentstroke}%
\pgfsetdash{}{0pt}%
\pgfpathmoveto{\pgfqpoint{4.994083in}{1.662022in}}%
\pgfpathlineto{\pgfqpoint{5.140982in}{1.534035in}}%
\pgfpathlineto{\pgfqpoint{5.078583in}{1.938515in}}%
\pgfpathclose%
\pgfusepath{fill}%
\end{pgfscope}%
\begin{pgfscope}%
\pgfpathrectangle{\pgfqpoint{0.539299in}{0.078740in}}{\pgfqpoint{7.842520in}{7.842520in}}%
\pgfusepath{clip}%
\pgfsetbuttcap%
\pgfsetroundjoin%
\definecolor{currentfill}{rgb}{0.255645,0.260703,0.528312}%
\pgfsetfillcolor{currentfill}%
\pgfsetlinewidth{0.000000pt}%
\definecolor{currentstroke}{rgb}{0.993248,0.906157,0.143936}%
\pgfsetstrokecolor{currentstroke}%
\pgfsetdash{}{0pt}%
\pgfpathmoveto{\pgfqpoint{5.287897in}{1.390668in}}%
\pgfpathlineto{\pgfqpoint{5.371612in}{1.597343in}}%
\pgfpathlineto{\pgfqpoint{5.225186in}{1.777193in}}%
\pgfpathclose%
\pgfusepath{fill}%
\end{pgfscope}%
\begin{pgfscope}%
\pgfpathrectangle{\pgfqpoint{0.539299in}{0.078740in}}{\pgfqpoint{7.842520in}{7.842520in}}%
\pgfusepath{clip}%
\pgfsetbuttcap%
\pgfsetroundjoin%
\definecolor{currentfill}{rgb}{0.276194,0.190074,0.493001}%
\pgfsetfillcolor{currentfill}%
\pgfsetlinewidth{0.000000pt}%
\definecolor{currentstroke}{rgb}{0.267004,0.004874,0.329415}%
\pgfsetstrokecolor{currentstroke}%
\pgfsetdash{}{0pt}%
\pgfpathmoveto{\pgfqpoint{5.287897in}{1.390668in}}%
\pgfpathlineto{\pgfqpoint{5.434633in}{1.223811in}}%
\pgfpathlineto{\pgfqpoint{5.517631in}{1.390496in}}%
\pgfpathclose%
\pgfusepath{fill}%
\end{pgfscope}%
\begin{pgfscope}%
\pgfpathrectangle{\pgfqpoint{0.539299in}{0.078740in}}{\pgfqpoint{7.842520in}{7.842520in}}%
\pgfusepath{clip}%
\pgfsetbuttcap%
\pgfsetroundjoin%
\definecolor{currentfill}{rgb}{0.257322,0.256130,0.526563}%
\pgfsetfillcolor{currentfill}%
\pgfsetlinewidth{0.000000pt}%
\definecolor{currentstroke}{rgb}{0.268510,0.009605,0.335427}%
\pgfsetstrokecolor{currentstroke}%
\pgfsetdash{}{0pt}%
\pgfpathmoveto{\pgfqpoint{5.225186in}{1.777193in}}%
\pgfpathlineto{\pgfqpoint{5.140982in}{1.534035in}}%
\pgfpathlineto{\pgfqpoint{5.287897in}{1.390668in}}%
\pgfpathclose%
\pgfusepath{fill}%
\end{pgfscope}%
\begin{pgfscope}%
\pgfpathrectangle{\pgfqpoint{0.539299in}{0.078740in}}{\pgfqpoint{7.842520in}{7.842520in}}%
\pgfusepath{clip}%
\pgfsetbuttcap%
\pgfsetroundjoin%
\definecolor{currentfill}{rgb}{0.273006,0.204520,0.501721}%
\pgfsetfillcolor{currentfill}%
\pgfsetlinewidth{0.000000pt}%
\definecolor{currentstroke}{rgb}{0.269944,0.014625,0.341379}%
\pgfsetstrokecolor{currentstroke}%
\pgfsetdash{}{0pt}%
\pgfpathmoveto{\pgfqpoint{5.287897in}{1.390668in}}%
\pgfpathlineto{\pgfqpoint{5.140982in}{1.534035in}}%
\pgfpathlineto{\pgfqpoint{5.056074in}{1.206164in}}%
\pgfpathclose%
\pgfusepath{fill}%
\end{pgfscope}%
\begin{pgfscope}%
\pgfpathrectangle{\pgfqpoint{0.539299in}{0.078740in}}{\pgfqpoint{7.842520in}{7.842520in}}%
\pgfusepath{clip}%
\pgfsetbuttcap%
\pgfsetroundjoin%
\definecolor{currentfill}{rgb}{0.280255,0.165693,0.476498}%
\pgfsetfillcolor{currentfill}%
\pgfsetlinewidth{0.000000pt}%
\definecolor{currentstroke}{rgb}{0.271305,0.019942,0.347269}%
\pgfsetstrokecolor{currentstroke}%
\pgfsetdash{}{0pt}%
\pgfpathmoveto{\pgfqpoint{5.056074in}{1.206164in}}%
\pgfpathlineto{\pgfqpoint{5.203294in}{1.107929in}}%
\pgfpathlineto{\pgfqpoint{5.287897in}{1.390668in}}%
\pgfpathclose%
\pgfusepath{fill}%
\end{pgfscope}%
\begin{pgfscope}%
\pgfpathrectangle{\pgfqpoint{0.539299in}{0.078740in}}{\pgfqpoint{7.842520in}{7.842520in}}%
\pgfusepath{clip}%
\pgfsetbuttcap%
\pgfsetroundjoin%
\definecolor{currentfill}{rgb}{0.280868,0.160771,0.472899}%
\pgfsetfillcolor{currentfill}%
\pgfsetlinewidth{0.000000pt}%
\definecolor{currentstroke}{rgb}{0.272594,0.025563,0.353093}%
\pgfsetstrokecolor{currentstroke}%
\pgfsetdash{}{0pt}%
\pgfpathmoveto{\pgfqpoint{5.350654in}{0.995313in}}%
\pgfpathlineto{\pgfqpoint{5.434633in}{1.223811in}}%
\pgfpathlineto{\pgfqpoint{5.287897in}{1.390668in}}%
\pgfpathclose%
\pgfusepath{fill}%
\end{pgfscope}%
\begin{pgfscope}%
\pgfpathrectangle{\pgfqpoint{0.539299in}{0.078740in}}{\pgfqpoint{7.842520in}{7.842520in}}%
\pgfusepath{clip}%
\pgfsetbuttcap%
\pgfsetroundjoin%
\definecolor{currentfill}{rgb}{0.283091,0.110553,0.431554}%
\pgfsetfillcolor{currentfill}%
\pgfsetlinewidth{0.000000pt}%
\definecolor{currentstroke}{rgb}{0.273809,0.031497,0.358853}%
\pgfsetstrokecolor{currentstroke}%
\pgfsetdash{}{0pt}%
\pgfpathmoveto{\pgfqpoint{5.497926in}{0.857909in}}%
\pgfpathlineto{\pgfqpoint{5.580846in}{1.019988in}}%
\pgfpathlineto{\pgfqpoint{5.434633in}{1.223811in}}%
\pgfpathclose%
\pgfusepath{fill}%
\end{pgfscope}%
\begin{pgfscope}%
\pgfpathrectangle{\pgfqpoint{0.539299in}{0.078740in}}{\pgfqpoint{7.842520in}{7.842520in}}%
\pgfusepath{clip}%
\pgfsetbuttcap%
\pgfsetroundjoin%
\definecolor{currentfill}{rgb}{0.281887,0.150881,0.465405}%
\pgfsetfillcolor{currentfill}%
\pgfsetlinewidth{0.000000pt}%
\definecolor{currentstroke}{rgb}{0.274952,0.037752,0.364543}%
\pgfsetstrokecolor{currentstroke}%
\pgfsetdash{}{0pt}%
\pgfpathmoveto{\pgfqpoint{5.203294in}{1.107929in}}%
\pgfpathlineto{\pgfqpoint{5.350654in}{0.995313in}}%
\pgfpathlineto{\pgfqpoint{5.287897in}{1.390668in}}%
\pgfpathclose%
\pgfusepath{fill}%
\end{pgfscope}%
\begin{pgfscope}%
\pgfpathrectangle{\pgfqpoint{0.539299in}{0.078740in}}{\pgfqpoint{7.842520in}{7.842520in}}%
\pgfusepath{clip}%
\pgfsetbuttcap%
\pgfsetroundjoin%
\definecolor{currentfill}{rgb}{0.283091,0.110553,0.431554}%
\pgfsetfillcolor{currentfill}%
\pgfsetlinewidth{0.000000pt}%
\definecolor{currentstroke}{rgb}{0.276022,0.044167,0.370164}%
\pgfsetstrokecolor{currentstroke}%
\pgfsetdash{}{0pt}%
\pgfpathmoveto{\pgfqpoint{5.434633in}{1.223811in}}%
\pgfpathlineto{\pgfqpoint{5.350654in}{0.995313in}}%
\pgfpathlineto{\pgfqpoint{5.497926in}{0.857909in}}%
\pgfpathclose%
\pgfusepath{fill}%
\end{pgfscope}%
\begin{pgfscope}%
\pgfsetbuttcap%
\pgfsetmiterjoin%
\definecolor{currentfill}{rgb}{1.000000,1.000000,1.000000}%
\pgfsetfillcolor{currentfill}%
\pgfsetlinewidth{0.000000pt}%
\definecolor{currentstroke}{rgb}{0.000000,0.000000,0.000000}%
\pgfsetstrokecolor{currentstroke}%
\pgfsetstrokeopacity{0.000000}%
\pgfsetdash{}{0pt}%
\pgfpathmoveto{\pgfqpoint{11.180860in}{0.157480in}}%
\pgfpathlineto{\pgfqpoint{11.495820in}{0.157480in}}%
\pgfpathlineto{\pgfqpoint{11.495820in}{7.842520in}}%
\pgfpathlineto{\pgfqpoint{11.180860in}{7.842520in}}%
\pgfpathclose%
\pgfusepath{fill}%
\end{pgfscope}%
\begin{pgfscope}%
\pgfpathrectangle{\pgfqpoint{11.180860in}{0.157480in}}{\pgfqpoint{0.314961in}{7.685039in}}%
\pgfusepath{clip}%
\pgfsetbuttcap%
\pgfsetmiterjoin%
\definecolor{currentfill}{rgb}{1.000000,1.000000,1.000000}%
\pgfsetfillcolor{currentfill}%
\pgfsetlinewidth{0.010037pt}%
\definecolor{currentstroke}{rgb}{1.000000,1.000000,1.000000}%
\pgfsetstrokecolor{currentstroke}%
\pgfsetdash{}{0pt}%
\pgfpathmoveto{\pgfqpoint{11.180860in}{0.157480in}}%
\pgfpathlineto{\pgfqpoint{11.180860in}{0.187500in}}%
\pgfpathlineto{\pgfqpoint{11.180860in}{7.812500in}}%
\pgfpathlineto{\pgfqpoint{11.180860in}{7.842520in}}%
\pgfpathlineto{\pgfqpoint{11.495820in}{7.842520in}}%
\pgfpathlineto{\pgfqpoint{11.495820in}{7.812500in}}%
\pgfpathlineto{\pgfqpoint{11.495820in}{0.187500in}}%
\pgfpathlineto{\pgfqpoint{11.495820in}{0.157480in}}%
\pgfpathlineto{\pgfqpoint{11.495820in}{0.157480in}}%
\pgfpathclose%
\pgfusepath{stroke,fill}%
\end{pgfscope}%
\begin{pgfscope}%
\pgfsys@transformshift{11.180000in}{0.160000in}%
\pgftext[left,bottom]{\includegraphics[interpolate=true,width=0.320000in,height=7.680000in]{unknown1-img0.png}}%
\end{pgfscope}%
\begin{pgfscope}%
\pgfsetbuttcap%
\pgfsetroundjoin%
\definecolor{currentfill}{rgb}{0.000000,0.000000,0.000000}%
\pgfsetfillcolor{currentfill}%
\pgfsetlinewidth{0.501875pt}%
\definecolor{currentstroke}{rgb}{0.000000,0.000000,0.000000}%
\pgfsetstrokecolor{currentstroke}%
\pgfsetdash{}{0pt}%
\pgfsys@defobject{currentmarker}{\pgfqpoint{-0.034722in}{0.000000in}}{\pgfqpoint{-0.000000in}{0.000000in}}{%
\pgfpathmoveto{\pgfqpoint{-0.000000in}{0.000000in}}%
\pgfpathlineto{\pgfqpoint{-0.034722in}{0.000000in}}%
\pgfusepath{stroke,fill}%
}%
\begin{pgfscope}%
\pgfsys@transformshift{11.495820in}{1.494658in}%
\pgfsys@useobject{currentmarker}{}%
\end{pgfscope}%
\end{pgfscope}%
\begin{pgfscope}%
\definecolor{textcolor}{rgb}{0.980392,0.811765,0.352941}%
\pgfsetstrokecolor{textcolor}%
\pgfsetfillcolor{textcolor}%
\pgftext[x=11.044431in, y=1.452449in, right, base]{\color{textcolor}\sffamily\fontsize{18.000000}{9.600000}\selectfont $\displaystyle 0.005$}%
\end{pgfscope}%
\begin{pgfscope}%
\pgfsetbuttcap%
\pgfsetroundjoin%
\definecolor{currentfill}{rgb}{0.000000,0.000000,0.000000}%
\pgfsetfillcolor{currentfill}%
\pgfsetlinewidth{0.501875pt}%
\definecolor{currentstroke}{rgb}{0.000000,0.000000,0.000000}%
\pgfsetstrokecolor{currentstroke}%
\pgfsetdash{}{0pt}%
\pgfsys@defobject{currentmarker}{\pgfqpoint{-0.034722in}{0.000000in}}{\pgfqpoint{-0.000000in}{0.000000in}}{%
\pgfpathmoveto{\pgfqpoint{-0.000000in}{0.000000in}}%
\pgfpathlineto{\pgfqpoint{-0.034722in}{0.000000in}}%
\pgfusepath{stroke,fill}%
}%
\begin{pgfscope}%
\pgfsys@transformshift{11.495820in}{2.832718in}%
\pgfsys@useobject{currentmarker}{}%
\end{pgfscope}%
\end{pgfscope}%
\begin{pgfscope}%
\definecolor{textcolor}{rgb}{0.980392,0.811765,0.352941}%
\pgfsetstrokecolor{textcolor}%
\pgfsetfillcolor{textcolor}%
\pgftext[x=11.044431in, y=2.790509in, right, base]{\color{textcolor}\sffamily\fontsize{18.000000}{9.600000}\selectfont $\displaystyle 0.010$}%
\end{pgfscope}%
\begin{pgfscope}%
\pgfsetbuttcap%
\pgfsetroundjoin%
\definecolor{currentfill}{rgb}{0.000000,0.000000,0.000000}%
\pgfsetfillcolor{currentfill}%
\pgfsetlinewidth{0.501875pt}%
\definecolor{currentstroke}{rgb}{0.000000,0.000000,0.000000}%
\pgfsetstrokecolor{currentstroke}%
\pgfsetdash{}{0pt}%
\pgfsys@defobject{currentmarker}{\pgfqpoint{-0.034722in}{0.000000in}}{\pgfqpoint{-0.000000in}{0.000000in}}{%
\pgfpathmoveto{\pgfqpoint{-0.000000in}{0.000000in}}%
\pgfpathlineto{\pgfqpoint{-0.034722in}{0.000000in}}%
\pgfusepath{stroke,fill}%
}%
\begin{pgfscope}%
\pgfsys@transformshift{11.495820in}{4.170779in}%
\pgfsys@useobject{currentmarker}{}%
\end{pgfscope}%
\end{pgfscope}%
\begin{pgfscope}%
\definecolor{textcolor}{rgb}{0.980392,0.811765,0.352941}%
\pgfsetstrokecolor{textcolor}%
\pgfsetfillcolor{textcolor}%
\pgftext[x=11.044431in, y=4.128569in, right, base]{\color{textcolor}\sffamily\fontsize{18.000000}{9.600000}\selectfont $\displaystyle 0.015$}%
\end{pgfscope}%
\begin{pgfscope}%
\pgfsetbuttcap%
\pgfsetroundjoin%
\definecolor{currentfill}{rgb}{0.000000,0.000000,0.000000}%
\pgfsetfillcolor{currentfill}%
\pgfsetlinewidth{0.501875pt}%
\definecolor{currentstroke}{rgb}{0.000000,0.000000,0.000000}%
\pgfsetstrokecolor{currentstroke}%
\pgfsetdash{}{0pt}%
\pgfsys@defobject{currentmarker}{\pgfqpoint{-0.034722in}{0.000000in}}{\pgfqpoint{-0.000000in}{0.000000in}}{%
\pgfpathmoveto{\pgfqpoint{-0.000000in}{0.000000in}}%
\pgfpathlineto{\pgfqpoint{-0.034722in}{0.000000in}}%
\pgfusepath{stroke,fill}%
}%
\begin{pgfscope}%
\pgfsys@transformshift{11.495820in}{5.508839in}%
\pgfsys@useobject{currentmarker}{}%
\end{pgfscope}%
\end{pgfscope}%
\begin{pgfscope}%
\definecolor{textcolor}{rgb}{0.980392,0.811765,0.352941}%
\pgfsetstrokecolor{textcolor}%
\pgfsetfillcolor{textcolor}%
\pgftext[x=11.044431in, y=5.466629in, right, base]{\color{textcolor}\sffamily\fontsize{18.000000}{9.600000}\selectfont $\displaystyle 0.020$}%
\end{pgfscope}%
\begin{pgfscope}%
\pgfsetbuttcap%
\pgfsetroundjoin%
\definecolor{currentfill}{rgb}{0.000000,0.000000,0.000000}%
\pgfsetfillcolor{currentfill}%
\pgfsetlinewidth{0.501875pt}%
\definecolor{currentstroke}{rgb}{0.000000,0.000000,0.000000}%
\pgfsetstrokecolor{currentstroke}%
\pgfsetdash{}{0pt}%
\pgfsys@defobject{currentmarker}{\pgfqpoint{-0.034722in}{0.000000in}}{\pgfqpoint{-0.000000in}{0.000000in}}{%
\pgfpathmoveto{\pgfqpoint{-0.000000in}{0.000000in}}%
\pgfpathlineto{\pgfqpoint{-0.034722in}{0.000000in}}%
\pgfusepath{stroke,fill}%
}%
\begin{pgfscope}%
\pgfsys@transformshift{11.495820in}{6.846899in}%
\pgfsys@useobject{currentmarker}{}%
\end{pgfscope}%
\end{pgfscope}%
\begin{pgfscope}%
\definecolor{textcolor}{rgb}{0.980392,0.811765,0.352941}%
\pgfsetstrokecolor{textcolor}%
\pgfsetfillcolor{textcolor}%
\pgftext[x=11.044431in, y=6.804689in, right, base]{\color{textcolor}\sffamily\fontsize{18.000000}{9.600000}\selectfont $\displaystyle 0.025$}%
\end{pgfscope}%
\begin{pgfscope}%
\pgfsetrectcap%
\pgfsetmiterjoin%
\pgfsetlinewidth{1.003750pt}%
\definecolor{currentstroke}{rgb}{0.000000,0.000000,0.000000}%
\pgfsetstrokecolor{currentstroke}%
\pgfsetdash{}{0pt}%
\pgfpathmoveto{\pgfqpoint{11.180860in}{0.157480in}}%
\pgfpathlineto{\pgfqpoint{11.180860in}{0.187500in}}%
\pgfpathlineto{\pgfqpoint{11.180860in}{7.812500in}}%
\pgfpathlineto{\pgfqpoint{11.180860in}{7.842520in}}%
\pgfpathlineto{\pgfqpoint{11.495820in}{7.842520in}}%
\pgfpathlineto{\pgfqpoint{11.495820in}{7.812500in}}%
\pgfpathlineto{\pgfqpoint{11.495820in}{0.187500in}}%
\pgfpathlineto{\pgfqpoint{11.495820in}{0.157480in}}%
\pgfpathclose%
\pgfusepath{stroke}%
\end{pgfscope}%
\end{pgfpicture}%
\makeatother%
\endgroup%
}
	\caption{$h = 2^{-5}$ 时差分逼近解}\label{fig:unknown1}
\end{figure}

为对其收敛速度进行分析, 计算 $-\log_2\lrvv{U_h - U_{h / 2}}_\infty$, 如表 \ref{tab:errorNorm2} 所示. 以 $-\log_2 h$ 为横坐标, 拟合得到
$$\log_2\lrvv{U_h - U_{h / 2}}_\infty \approx 2.009181444799865\log_2 h - 4.492800925336754,$$
拟合曲线如图 \ref{fig:log2UhUh2infty} 所示, 故差分逼近解在 $\mathbb{L}^\infty$ 范数意义下是二阶收敛的.

\begin{table}[H]\centering\heiti\zihao{-5}
	\caption{不同步长时 $-\log_2\lrvv{U_h - U_{h / 2}}_\infty$ 的值}\label{tab:errorNorm2}
	\begin{tabular}{|c|c|}\hline
		$-\log_2 h$	&	$-\log_2\lrvv{U_h - U_{h / 2}}_\infty$\\\hline
		$1$		&	$6.278314524587407$\\\hline
		$2$		&	$8.546061248379736$\\\hline
		$3$		&	$10.58435679097419$\\\hline
		$4$		&	$12.59155779257177$\\\hline
		$5$		&	$14.59260838416788$\\\hline
		$6$		&	$16.59059047122205$\\\hline
		$7$		&	$18.59039819231493$\\\hline
		$8$		&	$20.59026731037038$\\\hline
		$9$		&	$22.59027315438449$\\\hline
		$10$	&	$24.59025888822796$\\\hline
		$11$	&	$26.59000117687745$\\\hline
		$12$	&	$28.58588429968780$\\\hline
		$13$	&	$30.52135127239939$\\\hline
	\end{tabular}
\end{table}

% log2_err = [-6.278314524587407, -8.546061248379736, -10.584356790974189, -12.591557792571772, -14.592608384167875, -16.590590471222054, -18.590398192314932, -20.590267310370375, -22.590273154384494, -24.590258888227957, -26.59000117687745, -28.5858842996878, -30.52135127239939]

\begin{figure}[H]\centering
	\resizebox{0.9\linewidth}{!}{%% Creator: Matplotlib, PGF backend
%%
%% To include the figure in your LaTeX document, write
%%   \input{<filename>.pgf}
%%
%% Make sure the required packages are loaded in your preamble
%%   \usepackage{pgf}
%%
%% Figures using additional raster images can only be included by \input if
%% they are in the same directory as the main LaTeX file. For loading figures
%% from other directories you can use the `import` package
%%   \usepackage{import}
%%
%% and then include the figures with
%%   \import{<path to file>}{<filename>.pgf}
%%
%% Matplotlib used the following preamble
%%   \usepackage{fontspec}
%%   \setmainfont{DejaVuSerif.ttf}[Path=\detokenize{/Users/quejiahao/.julia/conda/3/lib/python3.9/site-packages/matplotlib/mpl-data/fonts/ttf/}]
%%   \setsansfont{DejaVuSans.ttf}[Path=\detokenize{/Users/quejiahao/.julia/conda/3/lib/python3.9/site-packages/matplotlib/mpl-data/fonts/ttf/}]
%%   \setmonofont{DejaVuSansMono.ttf}[Path=\detokenize{/Users/quejiahao/.julia/conda/3/lib/python3.9/site-packages/matplotlib/mpl-data/fonts/ttf/}]
%%
\begingroup%
\makeatletter%
\begin{pgfpicture}%
\pgfpathrectangle{\pgfpointorigin}{\pgfqpoint{12.000000in}{8.000000in}}%
\pgfusepath{use as bounding box, clip}%
\begin{pgfscope}%
\pgfsetbuttcap%
\pgfsetmiterjoin%
\definecolor{currentfill}{rgb}{1.000000,1.000000,1.000000}%
\pgfsetfillcolor{currentfill}%
\pgfsetlinewidth{0.000000pt}%
\definecolor{currentstroke}{rgb}{1.000000,1.000000,1.000000}%
\pgfsetstrokecolor{currentstroke}%
\pgfsetdash{}{0pt}%
\pgfpathmoveto{\pgfqpoint{0.000000in}{0.000000in}}%
\pgfpathlineto{\pgfqpoint{12.000000in}{0.000000in}}%
\pgfpathlineto{\pgfqpoint{12.000000in}{8.000000in}}%
\pgfpathlineto{\pgfqpoint{0.000000in}{8.000000in}}%
\pgfpathclose%
\pgfusepath{fill}%
\end{pgfscope}%
\begin{pgfscope}%
\pgfsetbuttcap%
\pgfsetmiterjoin%
\definecolor{currentfill}{rgb}{1.000000,1.000000,1.000000}%
\pgfsetfillcolor{currentfill}%
\pgfsetlinewidth{0.000000pt}%
\definecolor{currentstroke}{rgb}{0.000000,0.000000,0.000000}%
\pgfsetstrokecolor{currentstroke}%
\pgfsetstrokeopacity{0.000000}%
\pgfsetdash{}{0pt}%
\pgfpathmoveto{\pgfqpoint{0.483774in}{1.175073in}}%
\pgfpathlineto{\pgfqpoint{11.921260in}{1.175073in}}%
\pgfpathlineto{\pgfqpoint{11.921260in}{7.921260in}}%
\pgfpathlineto{\pgfqpoint{0.483774in}{7.921260in}}%
\pgfpathclose%
\pgfusepath{fill}%
\end{pgfscope}%
\begin{pgfscope}%
\pgfpathrectangle{\pgfqpoint{0.483774in}{1.175073in}}{\pgfqpoint{11.437486in}{6.746186in}}%
\pgfusepath{clip}%
\pgfsetrectcap%
\pgfsetroundjoin%
\pgfsetlinewidth{0.501875pt}%
\definecolor{currentstroke}{rgb}{0.000000,0.000000,0.000000}%
\pgfsetstrokecolor{currentstroke}%
\pgfsetstrokeopacity{0.100000}%
\pgfsetdash{}{0pt}%
\pgfpathmoveto{\pgfqpoint{2.156236in}{1.175073in}}%
\pgfpathlineto{\pgfqpoint{2.156236in}{7.921260in}}%
\pgfusepath{stroke}%
\end{pgfscope}%
\begin{pgfscope}%
\pgfsetbuttcap%
\pgfsetroundjoin%
\definecolor{currentfill}{rgb}{0.000000,0.000000,0.000000}%
\pgfsetfillcolor{currentfill}%
\pgfsetlinewidth{0.501875pt}%
\definecolor{currentstroke}{rgb}{0.000000,0.000000,0.000000}%
\pgfsetstrokecolor{currentstroke}%
\pgfsetdash{}{0pt}%
\pgfsys@defobject{currentmarker}{\pgfqpoint{0.000000in}{0.000000in}}{\pgfqpoint{0.000000in}{0.034722in}}{%
\pgfpathmoveto{\pgfqpoint{0.000000in}{0.000000in}}%
\pgfpathlineto{\pgfqpoint{0.000000in}{0.034722in}}%
\pgfusepath{stroke,fill}%
}%
\begin{pgfscope}%
\pgfsys@transformshift{2.156236in}{1.175073in}%
\pgfsys@useobject{currentmarker}{}%
\end{pgfscope}%
\end{pgfscope}%
\begin{pgfscope}%
\definecolor{textcolor}{rgb}{0.000000,0.000000,0.000000}%
\pgfsetstrokecolor{textcolor}%
\pgfsetfillcolor{textcolor}%
\pgftext[x=2.156236in,y=1.126462in,,top]{\color{textcolor}\sffamily\fontsize{18.000000}{9.600000}\selectfont $\displaystyle 2.5$}%
\end{pgfscope}%
\begin{pgfscope}%
\pgfpathrectangle{\pgfqpoint{0.483774in}{1.175073in}}{\pgfqpoint{11.437486in}{6.746186in}}%
\pgfusepath{clip}%
\pgfsetrectcap%
\pgfsetroundjoin%
\pgfsetlinewidth{0.501875pt}%
\definecolor{currentstroke}{rgb}{0.000000,0.000000,0.000000}%
\pgfsetstrokecolor{currentstroke}%
\pgfsetstrokeopacity{0.100000}%
\pgfsetdash{}{0pt}%
\pgfpathmoveto{\pgfqpoint{4.404170in}{1.175073in}}%
\pgfpathlineto{\pgfqpoint{4.404170in}{7.921260in}}%
\pgfusepath{stroke}%
\end{pgfscope}%
\begin{pgfscope}%
\pgfsetbuttcap%
\pgfsetroundjoin%
\definecolor{currentfill}{rgb}{0.000000,0.000000,0.000000}%
\pgfsetfillcolor{currentfill}%
\pgfsetlinewidth{0.501875pt}%
\definecolor{currentstroke}{rgb}{0.000000,0.000000,0.000000}%
\pgfsetstrokecolor{currentstroke}%
\pgfsetdash{}{0pt}%
\pgfsys@defobject{currentmarker}{\pgfqpoint{0.000000in}{0.000000in}}{\pgfqpoint{0.000000in}{0.034722in}}{%
\pgfpathmoveto{\pgfqpoint{0.000000in}{0.000000in}}%
\pgfpathlineto{\pgfqpoint{0.000000in}{0.034722in}}%
\pgfusepath{stroke,fill}%
}%
\begin{pgfscope}%
\pgfsys@transformshift{4.404170in}{1.175073in}%
\pgfsys@useobject{currentmarker}{}%
\end{pgfscope}%
\end{pgfscope}%
\begin{pgfscope}%
\definecolor{textcolor}{rgb}{0.000000,0.000000,0.000000}%
\pgfsetstrokecolor{textcolor}%
\pgfsetfillcolor{textcolor}%
\pgftext[x=4.404170in,y=1.126462in,,top]{\color{textcolor}\sffamily\fontsize{18.000000}{9.600000}\selectfont $\displaystyle 5.0$}%
\end{pgfscope}%
\begin{pgfscope}%
\pgfpathrectangle{\pgfqpoint{0.483774in}{1.175073in}}{\pgfqpoint{11.437486in}{6.746186in}}%
\pgfusepath{clip}%
\pgfsetrectcap%
\pgfsetroundjoin%
\pgfsetlinewidth{0.501875pt}%
\definecolor{currentstroke}{rgb}{0.000000,0.000000,0.000000}%
\pgfsetstrokecolor{currentstroke}%
\pgfsetstrokeopacity{0.100000}%
\pgfsetdash{}{0pt}%
\pgfpathmoveto{\pgfqpoint{6.652103in}{1.175073in}}%
\pgfpathlineto{\pgfqpoint{6.652103in}{7.921260in}}%
\pgfusepath{stroke}%
\end{pgfscope}%
\begin{pgfscope}%
\pgfsetbuttcap%
\pgfsetroundjoin%
\definecolor{currentfill}{rgb}{0.000000,0.000000,0.000000}%
\pgfsetfillcolor{currentfill}%
\pgfsetlinewidth{0.501875pt}%
\definecolor{currentstroke}{rgb}{0.000000,0.000000,0.000000}%
\pgfsetstrokecolor{currentstroke}%
\pgfsetdash{}{0pt}%
\pgfsys@defobject{currentmarker}{\pgfqpoint{0.000000in}{0.000000in}}{\pgfqpoint{0.000000in}{0.034722in}}{%
\pgfpathmoveto{\pgfqpoint{0.000000in}{0.000000in}}%
\pgfpathlineto{\pgfqpoint{0.000000in}{0.034722in}}%
\pgfusepath{stroke,fill}%
}%
\begin{pgfscope}%
\pgfsys@transformshift{6.652103in}{1.175073in}%
\pgfsys@useobject{currentmarker}{}%
\end{pgfscope}%
\end{pgfscope}%
\begin{pgfscope}%
\definecolor{textcolor}{rgb}{0.000000,0.000000,0.000000}%
\pgfsetstrokecolor{textcolor}%
\pgfsetfillcolor{textcolor}%
\pgftext[x=6.652103in,y=1.126462in,,top]{\color{textcolor}\sffamily\fontsize{18.000000}{9.600000}\selectfont $\displaystyle 7.5$}%
\end{pgfscope}%
\begin{pgfscope}%
\pgfpathrectangle{\pgfqpoint{0.483774in}{1.175073in}}{\pgfqpoint{11.437486in}{6.746186in}}%
\pgfusepath{clip}%
\pgfsetrectcap%
\pgfsetroundjoin%
\pgfsetlinewidth{0.501875pt}%
\definecolor{currentstroke}{rgb}{0.000000,0.000000,0.000000}%
\pgfsetstrokecolor{currentstroke}%
\pgfsetstrokeopacity{0.100000}%
\pgfsetdash{}{0pt}%
\pgfpathmoveto{\pgfqpoint{8.900037in}{1.175073in}}%
\pgfpathlineto{\pgfqpoint{8.900037in}{7.921260in}}%
\pgfusepath{stroke}%
\end{pgfscope}%
\begin{pgfscope}%
\pgfsetbuttcap%
\pgfsetroundjoin%
\definecolor{currentfill}{rgb}{0.000000,0.000000,0.000000}%
\pgfsetfillcolor{currentfill}%
\pgfsetlinewidth{0.501875pt}%
\definecolor{currentstroke}{rgb}{0.000000,0.000000,0.000000}%
\pgfsetstrokecolor{currentstroke}%
\pgfsetdash{}{0pt}%
\pgfsys@defobject{currentmarker}{\pgfqpoint{0.000000in}{0.000000in}}{\pgfqpoint{0.000000in}{0.034722in}}{%
\pgfpathmoveto{\pgfqpoint{0.000000in}{0.000000in}}%
\pgfpathlineto{\pgfqpoint{0.000000in}{0.034722in}}%
\pgfusepath{stroke,fill}%
}%
\begin{pgfscope}%
\pgfsys@transformshift{8.900037in}{1.175073in}%
\pgfsys@useobject{currentmarker}{}%
\end{pgfscope}%
\end{pgfscope}%
\begin{pgfscope}%
\definecolor{textcolor}{rgb}{0.000000,0.000000,0.000000}%
\pgfsetstrokecolor{textcolor}%
\pgfsetfillcolor{textcolor}%
\pgftext[x=8.900037in,y=1.126462in,,top]{\color{textcolor}\sffamily\fontsize{18.000000}{9.600000}\selectfont $\displaystyle 10.0$}%
\end{pgfscope}%
\begin{pgfscope}%
\pgfpathrectangle{\pgfqpoint{0.483774in}{1.175073in}}{\pgfqpoint{11.437486in}{6.746186in}}%
\pgfusepath{clip}%
\pgfsetrectcap%
\pgfsetroundjoin%
\pgfsetlinewidth{0.501875pt}%
\definecolor{currentstroke}{rgb}{0.000000,0.000000,0.000000}%
\pgfsetstrokecolor{currentstroke}%
\pgfsetstrokeopacity{0.100000}%
\pgfsetdash{}{0pt}%
\pgfpathmoveto{\pgfqpoint{11.147971in}{1.175073in}}%
\pgfpathlineto{\pgfqpoint{11.147971in}{7.921260in}}%
\pgfusepath{stroke}%
\end{pgfscope}%
\begin{pgfscope}%
\pgfsetbuttcap%
\pgfsetroundjoin%
\definecolor{currentfill}{rgb}{0.000000,0.000000,0.000000}%
\pgfsetfillcolor{currentfill}%
\pgfsetlinewidth{0.501875pt}%
\definecolor{currentstroke}{rgb}{0.000000,0.000000,0.000000}%
\pgfsetstrokecolor{currentstroke}%
\pgfsetdash{}{0pt}%
\pgfsys@defobject{currentmarker}{\pgfqpoint{0.000000in}{0.000000in}}{\pgfqpoint{0.000000in}{0.034722in}}{%
\pgfpathmoveto{\pgfqpoint{0.000000in}{0.000000in}}%
\pgfpathlineto{\pgfqpoint{0.000000in}{0.034722in}}%
\pgfusepath{stroke,fill}%
}%
\begin{pgfscope}%
\pgfsys@transformshift{11.147971in}{1.175073in}%
\pgfsys@useobject{currentmarker}{}%
\end{pgfscope}%
\end{pgfscope}%
\begin{pgfscope}%
\definecolor{textcolor}{rgb}{0.000000,0.000000,0.000000}%
\pgfsetstrokecolor{textcolor}%
\pgfsetfillcolor{textcolor}%
\pgftext[x=11.147971in,y=1.126462in,,top]{\color{textcolor}\sffamily\fontsize{18.000000}{9.600000}\selectfont $\displaystyle 12.5$}%
\end{pgfscope}%
\begin{pgfscope}%
\definecolor{textcolor}{rgb}{0.000000,0.000000,0.000000}%
\pgfsetstrokecolor{textcolor}%
\pgfsetfillcolor{textcolor}%
\pgftext[x=6.174953in, y=0.847301in, left, base]{\color{textcolor}\sffamily\fontsize{11.000000}{13.200000}\selectfont -}%
\end{pgfscope}%
\begin{pgfscope}%
\definecolor{textcolor}{rgb}{0.000000,0.000000,0.000000}%
\pgfsetstrokecolor{textcolor}%
\pgfsetfillcolor{textcolor}%
\pgftext[x=5.964771in, y=0.676232in, left, base]{\color{textcolor}\sffamily\fontsize{18.000000}{13.200000}\selectfont $-\log_2 h$}%
\end{pgfscope}%
\begin{pgfscope}%
\pgfpathrectangle{\pgfqpoint{0.483774in}{1.175073in}}{\pgfqpoint{11.437486in}{6.746186in}}%
\pgfusepath{clip}%
\pgfsetrectcap%
\pgfsetroundjoin%
\pgfsetlinewidth{0.501875pt}%
\definecolor{currentstroke}{rgb}{0.000000,0.000000,0.000000}%
\pgfsetstrokecolor{currentstroke}%
\pgfsetstrokeopacity{0.100000}%
\pgfsetdash{}{0pt}%
\pgfpathmoveto{\pgfqpoint{0.483774in}{2.339381in}}%
\pgfpathlineto{\pgfqpoint{11.921260in}{2.339381in}}%
\pgfusepath{stroke}%
\end{pgfscope}%
\begin{pgfscope}%
\pgfsetbuttcap%
\pgfsetroundjoin%
\definecolor{currentfill}{rgb}{0.000000,0.000000,0.000000}%
\pgfsetfillcolor{currentfill}%
\pgfsetlinewidth{0.501875pt}%
\definecolor{currentstroke}{rgb}{0.000000,0.000000,0.000000}%
\pgfsetstrokecolor{currentstroke}%
\pgfsetdash{}{0pt}%
\pgfsys@defobject{currentmarker}{\pgfqpoint{0.000000in}{0.000000in}}{\pgfqpoint{0.034722in}{0.000000in}}{%
\pgfpathmoveto{\pgfqpoint{0.000000in}{0.000000in}}%
\pgfpathlineto{\pgfqpoint{0.034722in}{0.000000in}}%
\pgfusepath{stroke,fill}%
}%
\begin{pgfscope}%
\pgfsys@transformshift{0.483774in}{2.339381in}%
\pgfsys@useobject{currentmarker}{}%
\end{pgfscope}%
\end{pgfscope}%
\begin{pgfscope}%
\definecolor{textcolor}{rgb}{0.000000,0.000000,0.000000}%
\pgfsetstrokecolor{textcolor}%
\pgfsetfillcolor{textcolor}%
\pgftext[x=0.217105in, y=2.297172in, left, base]{\color{textcolor}\sffamily\fontsize{18.000000}{9.600000}\selectfont $\displaystyle 10$}%
\end{pgfscope}%
\begin{pgfscope}%
\pgfpathrectangle{\pgfqpoint{0.483774in}{1.175073in}}{\pgfqpoint{11.437486in}{6.746186in}}%
\pgfusepath{clip}%
\pgfsetrectcap%
\pgfsetroundjoin%
\pgfsetlinewidth{0.501875pt}%
\definecolor{currentstroke}{rgb}{0.000000,0.000000,0.000000}%
\pgfsetstrokecolor{currentstroke}%
\pgfsetstrokeopacity{0.100000}%
\pgfsetdash{}{0pt}%
\pgfpathmoveto{\pgfqpoint{0.483774in}{3.647092in}}%
\pgfpathlineto{\pgfqpoint{11.921260in}{3.647092in}}%
\pgfusepath{stroke}%
\end{pgfscope}%
\begin{pgfscope}%
\pgfsetbuttcap%
\pgfsetroundjoin%
\definecolor{currentfill}{rgb}{0.000000,0.000000,0.000000}%
\pgfsetfillcolor{currentfill}%
\pgfsetlinewidth{0.501875pt}%
\definecolor{currentstroke}{rgb}{0.000000,0.000000,0.000000}%
\pgfsetstrokecolor{currentstroke}%
\pgfsetdash{}{0pt}%
\pgfsys@defobject{currentmarker}{\pgfqpoint{0.000000in}{0.000000in}}{\pgfqpoint{0.034722in}{0.000000in}}{%
\pgfpathmoveto{\pgfqpoint{0.000000in}{0.000000in}}%
\pgfpathlineto{\pgfqpoint{0.034722in}{0.000000in}}%
\pgfusepath{stroke,fill}%
}%
\begin{pgfscope}%
\pgfsys@transformshift{0.483774in}{3.647092in}%
\pgfsys@useobject{currentmarker}{}%
\end{pgfscope}%
\end{pgfscope}%
\begin{pgfscope}%
\definecolor{textcolor}{rgb}{0.000000,0.000000,0.000000}%
\pgfsetstrokecolor{textcolor}%
\pgfsetfillcolor{textcolor}%
\pgftext[x=0.217105in, y=3.604883in, left, base]{\color{textcolor}\sffamily\fontsize{18.000000}{9.600000}\selectfont $\displaystyle 15$}%
\end{pgfscope}%
\begin{pgfscope}%
\pgfpathrectangle{\pgfqpoint{0.483774in}{1.175073in}}{\pgfqpoint{11.437486in}{6.746186in}}%
\pgfusepath{clip}%
\pgfsetrectcap%
\pgfsetroundjoin%
\pgfsetlinewidth{0.501875pt}%
\definecolor{currentstroke}{rgb}{0.000000,0.000000,0.000000}%
\pgfsetstrokecolor{currentstroke}%
\pgfsetstrokeopacity{0.100000}%
\pgfsetdash{}{0pt}%
\pgfpathmoveto{\pgfqpoint{0.483774in}{4.954803in}}%
\pgfpathlineto{\pgfqpoint{11.921260in}{4.954803in}}%
\pgfusepath{stroke}%
\end{pgfscope}%
\begin{pgfscope}%
\pgfsetbuttcap%
\pgfsetroundjoin%
\definecolor{currentfill}{rgb}{0.000000,0.000000,0.000000}%
\pgfsetfillcolor{currentfill}%
\pgfsetlinewidth{0.501875pt}%
\definecolor{currentstroke}{rgb}{0.000000,0.000000,0.000000}%
\pgfsetstrokecolor{currentstroke}%
\pgfsetdash{}{0pt}%
\pgfsys@defobject{currentmarker}{\pgfqpoint{0.000000in}{0.000000in}}{\pgfqpoint{0.034722in}{0.000000in}}{%
\pgfpathmoveto{\pgfqpoint{0.000000in}{0.000000in}}%
\pgfpathlineto{\pgfqpoint{0.034722in}{0.000000in}}%
\pgfusepath{stroke,fill}%
}%
\begin{pgfscope}%
\pgfsys@transformshift{0.483774in}{4.954803in}%
\pgfsys@useobject{currentmarker}{}%
\end{pgfscope}%
\end{pgfscope}%
\begin{pgfscope}%
\definecolor{textcolor}{rgb}{0.000000,0.000000,0.000000}%
\pgfsetstrokecolor{textcolor}%
\pgfsetfillcolor{textcolor}%
\pgftext[x=0.217105in, y=4.912593in, left, base]{\color{textcolor}\sffamily\fontsize{18.000000}{9.600000}\selectfont $\displaystyle 20$}%
\end{pgfscope}%
\begin{pgfscope}%
\pgfpathrectangle{\pgfqpoint{0.483774in}{1.175073in}}{\pgfqpoint{11.437486in}{6.746186in}}%
\pgfusepath{clip}%
\pgfsetrectcap%
\pgfsetroundjoin%
\pgfsetlinewidth{0.501875pt}%
\definecolor{currentstroke}{rgb}{0.000000,0.000000,0.000000}%
\pgfsetstrokecolor{currentstroke}%
\pgfsetstrokeopacity{0.100000}%
\pgfsetdash{}{0pt}%
\pgfpathmoveto{\pgfqpoint{0.483774in}{6.262514in}}%
\pgfpathlineto{\pgfqpoint{11.921260in}{6.262514in}}%
\pgfusepath{stroke}%
\end{pgfscope}%
\begin{pgfscope}%
\pgfsetbuttcap%
\pgfsetroundjoin%
\definecolor{currentfill}{rgb}{0.000000,0.000000,0.000000}%
\pgfsetfillcolor{currentfill}%
\pgfsetlinewidth{0.501875pt}%
\definecolor{currentstroke}{rgb}{0.000000,0.000000,0.000000}%
\pgfsetstrokecolor{currentstroke}%
\pgfsetdash{}{0pt}%
\pgfsys@defobject{currentmarker}{\pgfqpoint{0.000000in}{0.000000in}}{\pgfqpoint{0.034722in}{0.000000in}}{%
\pgfpathmoveto{\pgfqpoint{0.000000in}{0.000000in}}%
\pgfpathlineto{\pgfqpoint{0.034722in}{0.000000in}}%
\pgfusepath{stroke,fill}%
}%
\begin{pgfscope}%
\pgfsys@transformshift{0.483774in}{6.262514in}%
\pgfsys@useobject{currentmarker}{}%
\end{pgfscope}%
\end{pgfscope}%
\begin{pgfscope}%
\definecolor{textcolor}{rgb}{0.000000,0.000000,0.000000}%
\pgfsetstrokecolor{textcolor}%
\pgfsetfillcolor{textcolor}%
\pgftext[x=0.217105in, y=6.220304in, left, base]{\color{textcolor}\sffamily\fontsize{18.000000}{9.600000}\selectfont $\displaystyle 25$}%
\end{pgfscope}%
\begin{pgfscope}%
\pgfpathrectangle{\pgfqpoint{0.483774in}{1.175073in}}{\pgfqpoint{11.437486in}{6.746186in}}%
\pgfusepath{clip}%
\pgfsetrectcap%
\pgfsetroundjoin%
\pgfsetlinewidth{0.501875pt}%
\definecolor{currentstroke}{rgb}{0.000000,0.000000,0.000000}%
\pgfsetstrokecolor{currentstroke}%
\pgfsetstrokeopacity{0.100000}%
\pgfsetdash{}{0pt}%
\pgfpathmoveto{\pgfqpoint{0.483774in}{7.570224in}}%
\pgfpathlineto{\pgfqpoint{11.921260in}{7.570224in}}%
\pgfusepath{stroke}%
\end{pgfscope}%
\begin{pgfscope}%
\pgfsetbuttcap%
\pgfsetroundjoin%
\definecolor{currentfill}{rgb}{0.000000,0.000000,0.000000}%
\pgfsetfillcolor{currentfill}%
\pgfsetlinewidth{0.501875pt}%
\definecolor{currentstroke}{rgb}{0.000000,0.000000,0.000000}%
\pgfsetstrokecolor{currentstroke}%
\pgfsetdash{}{0pt}%
\pgfsys@defobject{currentmarker}{\pgfqpoint{0.000000in}{0.000000in}}{\pgfqpoint{0.034722in}{0.000000in}}{%
\pgfpathmoveto{\pgfqpoint{0.000000in}{0.000000in}}%
\pgfpathlineto{\pgfqpoint{0.034722in}{0.000000in}}%
\pgfusepath{stroke,fill}%
}%
\begin{pgfscope}%
\pgfsys@transformshift{0.483774in}{7.570224in}%
\pgfsys@useobject{currentmarker}{}%
\end{pgfscope}%
\end{pgfscope}%
\begin{pgfscope}%
\definecolor{textcolor}{rgb}{0.000000,0.000000,0.000000}%
\pgfsetstrokecolor{textcolor}%
\pgfsetfillcolor{textcolor}%
\pgftext[x=0.217105in, y=7.528015in, left, base]{\color{textcolor}\sffamily\fontsize{18.000000}{9.600000}\selectfont $\displaystyle 30$}%
\end{pgfscope}%
\begin{pgfscope}%
\pgfpathrectangle{\pgfqpoint{0.483774in}{1.175073in}}{\pgfqpoint{11.437486in}{6.746186in}}%
\pgfusepath{clip}%
\pgfsetbuttcap%
\pgfsetroundjoin%
\definecolor{currentfill}{rgb}{0.000000,0.605603,0.978680}%
\pgfsetfillcolor{currentfill}%
\pgfsetlinewidth{1.003750pt}%
\definecolor{currentstroke}{rgb}{0.000000,0.000000,0.000000}%
\pgfsetstrokecolor{currentstroke}%
\pgfsetdash{}{0pt}%
\pgfsys@defobject{currentmarker}{\pgfqpoint{-0.055556in}{-0.055556in}}{\pgfqpoint{0.055556in}{0.055556in}}{%
\pgfpathmoveto{\pgfqpoint{0.000000in}{-0.055556in}}%
\pgfpathcurveto{\pgfqpoint{0.014734in}{-0.055556in}}{\pgfqpoint{0.028866in}{-0.049702in}}{\pgfqpoint{0.039284in}{-0.039284in}}%
\pgfpathcurveto{\pgfqpoint{0.049702in}{-0.028866in}}{\pgfqpoint{0.055556in}{-0.014734in}}{\pgfqpoint{0.055556in}{0.000000in}}%
\pgfpathcurveto{\pgfqpoint{0.055556in}{0.014734in}}{\pgfqpoint{0.049702in}{0.028866in}}{\pgfqpoint{0.039284in}{0.039284in}}%
\pgfpathcurveto{\pgfqpoint{0.028866in}{0.049702in}}{\pgfqpoint{0.014734in}{0.055556in}}{\pgfqpoint{0.000000in}{0.055556in}}%
\pgfpathcurveto{\pgfqpoint{-0.014734in}{0.055556in}}{\pgfqpoint{-0.028866in}{0.049702in}}{\pgfqpoint{-0.039284in}{0.039284in}}%
\pgfpathcurveto{\pgfqpoint{-0.049702in}{0.028866in}}{\pgfqpoint{-0.055556in}{0.014734in}}{\pgfqpoint{-0.055556in}{0.000000in}}%
\pgfpathcurveto{\pgfqpoint{-0.055556in}{-0.014734in}}{\pgfqpoint{-0.049702in}{-0.028866in}}{\pgfqpoint{-0.039284in}{-0.039284in}}%
\pgfpathcurveto{\pgfqpoint{-0.028866in}{-0.049702in}}{\pgfqpoint{-0.014734in}{-0.055556in}}{\pgfqpoint{0.000000in}{-0.055556in}}%
\pgfpathclose%
\pgfusepath{stroke,fill}%
}%
\begin{pgfscope}%
\pgfsys@transformshift{0.807476in}{1.366003in}%
\pgfsys@useobject{currentmarker}{}%
\end{pgfscope}%
\begin{pgfscope}%
\pgfsys@transformshift{1.706650in}{1.959115in}%
\pgfsys@useobject{currentmarker}{}%
\end{pgfscope}%
\begin{pgfscope}%
\pgfsys@transformshift{2.605823in}{2.492215in}%
\pgfsys@useobject{currentmarker}{}%
\end{pgfscope}%
\begin{pgfscope}%
\pgfsys@transformshift{3.504996in}{3.017183in}%
\pgfsys@useobject{currentmarker}{}%
\end{pgfscope}%
\begin{pgfscope}%
\pgfsys@transformshift{4.404170in}{3.540542in}%
\pgfsys@useobject{currentmarker}{}%
\end{pgfscope}%
\begin{pgfscope}%
\pgfsys@transformshift{5.303343in}{4.063098in}%
\pgfsys@useobject{currentmarker}{}%
\end{pgfscope}%
\begin{pgfscope}%
\pgfsys@transformshift{6.202517in}{4.586132in}%
\pgfsys@useobject{currentmarker}{}%
\end{pgfscope}%
\begin{pgfscope}%
\pgfsys@transformshift{7.101690in}{5.109183in}%
\pgfsys@useobject{currentmarker}{}%
\end{pgfscope}%
\begin{pgfscope}%
\pgfsys@transformshift{8.000864in}{5.632268in}%
\pgfsys@useobject{currentmarker}{}%
\end{pgfscope}%
\begin{pgfscope}%
\pgfsys@transformshift{8.900037in}{6.155349in}%
\pgfsys@useobject{currentmarker}{}%
\end{pgfscope}%
\begin{pgfscope}%
\pgfsys@transformshift{9.799211in}{6.678366in}%
\pgfsys@useobject{currentmarker}{}%
\end{pgfscope}%
\begin{pgfscope}%
\pgfsys@transformshift{10.698384in}{7.200374in}%
\pgfsys@useobject{currentmarker}{}%
\end{pgfscope}%
\begin{pgfscope}%
\pgfsys@transformshift{11.597557in}{7.706580in}%
\pgfsys@useobject{currentmarker}{}%
\end{pgfscope}%
\end{pgfscope}%
\begin{pgfscope}%
\pgfpathrectangle{\pgfqpoint{0.483774in}{1.175073in}}{\pgfqpoint{11.437486in}{6.746186in}}%
\pgfusepath{clip}%
\pgfsetbuttcap%
\pgfsetroundjoin%
\pgfsetlinewidth{1.003750pt}%
\definecolor{currentstroke}{rgb}{0.888874,0.435649,0.278123}%
\pgfsetstrokecolor{currentstroke}%
\pgfsetdash{}{0pt}%
\pgfpathmoveto{\pgfqpoint{0.807476in}{1.424502in}}%
\pgfpathlineto{\pgfqpoint{1.706650in}{1.949988in}}%
\pgfpathlineto{\pgfqpoint{2.605823in}{2.475473in}}%
\pgfpathlineto{\pgfqpoint{3.504996in}{3.000959in}}%
\pgfpathlineto{\pgfqpoint{4.404170in}{3.526445in}}%
\pgfpathlineto{\pgfqpoint{5.303343in}{4.051930in}}%
\pgfpathlineto{\pgfqpoint{6.202517in}{4.577416in}}%
\pgfpathlineto{\pgfqpoint{7.101690in}{5.102902in}}%
\pgfpathlineto{\pgfqpoint{8.000864in}{5.628387in}}%
\pgfpathlineto{\pgfqpoint{8.900037in}{6.153873in}}%
\pgfpathlineto{\pgfqpoint{9.799211in}{6.679359in}}%
\pgfpathlineto{\pgfqpoint{10.698384in}{7.204844in}}%
\pgfpathlineto{\pgfqpoint{11.597557in}{7.730330in}}%
\pgfusepath{stroke}%
\end{pgfscope}%
\begin{pgfscope}%
\pgfsetrectcap%
\pgfsetmiterjoin%
\pgfsetlinewidth{1.003750pt}%
\definecolor{currentstroke}{rgb}{0.000000,0.000000,0.000000}%
\pgfsetstrokecolor{currentstroke}%
\pgfsetdash{}{0pt}%
\pgfpathmoveto{\pgfqpoint{0.483774in}{1.175073in}}%
\pgfpathlineto{\pgfqpoint{0.483774in}{7.921260in}}%
\pgfusepath{stroke}%
\end{pgfscope}%
\begin{pgfscope}%
\pgfsetrectcap%
\pgfsetmiterjoin%
\pgfsetlinewidth{1.003750pt}%
\definecolor{currentstroke}{rgb}{0.000000,0.000000,0.000000}%
\pgfsetstrokecolor{currentstroke}%
\pgfsetdash{}{0pt}%
\pgfpathmoveto{\pgfqpoint{0.483774in}{1.175073in}}%
\pgfpathlineto{\pgfqpoint{11.921260in}{1.175073in}}%
\pgfusepath{stroke}%
\end{pgfscope}%
\begin{pgfscope}%
\pgfsetbuttcap%
\pgfsetmiterjoin%
\definecolor{currentfill}{rgb}{1.000000,1.000000,1.000000}%
\pgfsetfillcolor{currentfill}%
\pgfsetlinewidth{1.003750pt}%
\definecolor{currentstroke}{rgb}{0.000000,0.000000,0.000000}%
\pgfsetstrokecolor{currentstroke}%
\pgfsetdash{}{0pt}%
\pgfpathmoveto{\pgfqpoint{9.771889in}{7.292896in}}%
\pgfpathlineto{\pgfqpoint{11.865704in}{7.292896in}}%
\pgfpathlineto{\pgfqpoint{11.865704in}{7.865704in}}%
\pgfpathlineto{\pgfqpoint{9.771889in}{7.865704in}}%
\pgfpathclose%
\pgfusepath{stroke,fill}%
\end{pgfscope}%
\begin{pgfscope}%
\pgfsetbuttcap%
\pgfsetmiterjoin%
\pgfsetlinewidth{0.000000pt}%
\definecolor{currentstroke}{rgb}{0.000000,0.605603,0.978680}%
\pgfsetstrokecolor{currentstroke}%
\pgfsetdash{}{0pt}%
\pgfpathmoveto{\pgfqpoint{9.860778in}{7.657523in}}%
\pgfpathlineto{\pgfqpoint{10.083000in}{7.657523in}}%
\pgfusepath{}%
\end{pgfscope}%
\begin{pgfscope}%
\pgfsetbuttcap%
\pgfsetroundjoin%
\definecolor{currentfill}{rgb}{0.000000,0.605603,0.978680}%
\pgfsetfillcolor{currentfill}%
\pgfsetlinewidth{0.803000pt}%
\definecolor{currentstroke}{rgb}{0.000000,0.000000,0.000000}%
\pgfsetstrokecolor{currentstroke}%
\pgfsetdash{}{0pt}%
\pgfsys@defobject{currentmarker}{\pgfqpoint{-0.044444in}{-0.044444in}}{\pgfqpoint{0.044444in}{0.044444in}}{%
\pgfpathmoveto{\pgfqpoint{0.000000in}{-0.044444in}}%
\pgfpathcurveto{\pgfqpoint{0.011787in}{-0.044444in}}{\pgfqpoint{0.023092in}{-0.039761in}}{\pgfqpoint{0.031427in}{-0.031427in}}%
\pgfpathcurveto{\pgfqpoint{0.039761in}{-0.023092in}}{\pgfqpoint{0.044444in}{-0.011787in}}{\pgfqpoint{0.044444in}{0.000000in}}%
\pgfpathcurveto{\pgfqpoint{0.044444in}{0.011787in}}{\pgfqpoint{0.039761in}{0.023092in}}{\pgfqpoint{0.031427in}{0.031427in}}%
\pgfpathcurveto{\pgfqpoint{0.023092in}{0.039761in}}{\pgfqpoint{0.011787in}{0.044444in}}{\pgfqpoint{0.000000in}{0.044444in}}%
\pgfpathcurveto{\pgfqpoint{-0.011787in}{0.044444in}}{\pgfqpoint{-0.023092in}{0.039761in}}{\pgfqpoint{-0.031427in}{0.031427in}}%
\pgfpathcurveto{\pgfqpoint{-0.039761in}{0.023092in}}{\pgfqpoint{-0.044444in}{0.011787in}}{\pgfqpoint{-0.044444in}{0.000000in}}%
\pgfpathcurveto{\pgfqpoint{-0.044444in}{-0.011787in}}{\pgfqpoint{-0.039761in}{-0.023092in}}{\pgfqpoint{-0.031427in}{-0.031427in}}%
\pgfpathcurveto{\pgfqpoint{-0.023092in}{-0.039761in}}{\pgfqpoint{-0.011787in}{-0.044444in}}{\pgfqpoint{0.000000in}{-0.044444in}}%
\pgfpathclose%
\pgfusepath{stroke,fill}%
}%
\begin{pgfscope}%
\pgfsys@transformshift{9.971889in}{7.657523in}%
\pgfsys@useobject{currentmarker}{}%
\end{pgfscope}%
\end{pgfscope}%
\begin{pgfscope}%
\definecolor{textcolor}{rgb}{0.000000,0.000000,0.000000}%
\pgfsetstrokecolor{textcolor}%
\pgfsetfillcolor{textcolor}%
\pgftext[x=10.171889in, y=7.63019in, left, base]{\color{textcolor}\sffamily\fontsize{12.000000}{9.600000}\selectfont $-\log_2\lrvv{U_h - U_{h / 2}}_\infty$}%
\end{pgfscope}%
\begin{pgfscope}%
\pgfsetbuttcap%
\pgfsetmiterjoin%
\pgfsetlinewidth{1.003750pt}%
\definecolor{currentstroke}{rgb}{0.888874,0.435649,0.278123}%
\pgfsetstrokecolor{currentstroke}%
\pgfsetdash{}{0pt}%
\pgfpathmoveto{\pgfqpoint{9.860778in}{7.443786in}}%
\pgfpathlineto{\pgfqpoint{10.083000in}{7.443786in}}%
\pgfusepath{stroke}%
\end{pgfscope}%
\begin{pgfscope}%
\definecolor{textcolor}{rgb}{0.000000,0.000000,0.000000}%
\pgfsetstrokecolor{textcolor}%
\pgfsetfillcolor{textcolor}%
\pgftext[x=10.171889in,y=7.404897in,left,base]{\color{textcolor}\sffamily\fontsize{12.000000}{9.600000}\selectfont\songti 拟合曲线}%
\end{pgfscope}%
\end{pgfpicture}%
\makeatother%
\endgroup%
}
	\caption{$-\log_2\lrvv{U_h - U_{h / 2}}_\infty$ - $-\log_2 h$ 曲线}\label{fig:log2UhUh2infty}
\end{figure}

% b, k = 4.492800925336754, 2.0091814447998653

\begin{appendices}

\href{https://github.com/Quejiahao/NumericalSolutionOfPartialDifferentialEquations.jl}{数值实验源码}

\end{appendices}

\end{document}
