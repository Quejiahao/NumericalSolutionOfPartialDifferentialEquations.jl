% !TeX program	= xelatex
% !TeX encoding	= UTF-8

%-------------------- 文类 --------------------
\documentclass[UTF8, a4paper, 12pt, oneside, onecolumn]{article}

%-------------------- 宏包 --------------------
\input{../../template/usepackage}

%-------------------- 杂项 --------------------
\input{../../template/misc}

%-------------------- 字体设置 --------------------
\input{../../template/fonts}

%-------------------- 标题样式 --------------------
\input{../../template/title}

%-------------------- 自定义符号 --------------------
\input{../../template/symbols}

%-------------------- 自定义环境 --------------------
\input{../../template/environments}

%-------------------- \item 编号 --------------------
\input{../../template/itemstyle}

%-------------------- 代码样式设置 --------------------
\input{../../template/codestyle}

%-------------------- PDF 元信息, 建模时注释掉 --------------------
\input{../pdfinfo}

%-------------------- 页眉页脚 --------------------
\input{../headersfooters}

%-------------------- 正文 --------------------
\begin{document}

\thispagestyle{plain}

%\columnseprule = 1pt	% 栏线
\begin{center}
	{\zihao{-2}\heiti 常微分方程自测题~2} \\
	\vspace{1.5ex}
	{\zihao{-4}\fangsong 阙嘉豪\textsuperscript{\hyperref[auth:1]{1}}%
	\blfootnote{\zihao{6}\heiti 作者简介: \songti 阙嘉豪~(1999—), 男, 广东深圳人, 北京师范大学数学科学学院本科生.}} \\
	{\zihao{6}\songti \label{auth:1}(1. 北京师范大学 数学科学学院, 北京~~100875)}
\end{center}

\zihao{5}

%\watermark{60}{10}{\currenttime}

\part{部分}
\section{一级标题}
\subsection{二级标题}
\subsubsection{三级标题}


\begin{thebibliography}{1}
%\setlength{\parskip}{0pt}
%\setlength{\baselineskip}{16pt}
%\setlength{\leftskip}{-19pt}
%\setlength{\itemsep}{0em}
%\setlength{\itemindent}{2em}
%\setlength{\labelwidth}{0pt}
%\setlength{\labelsep}{0pt}

\addcontentsline{toc}{section}{参考文献}	% 将 "参考文献" 加入目录中

\bibitem{RongYuan} 袁荣. 常微分方程[M]. 北京: 高等教育出版社, 2012:59,62.
\end{thebibliography}
%\end{multicols}

\begin{appendices}

\section{\bfseries Some Appendix2}\label{aaa}
The contents...

\end{appendices}

\end{document}
