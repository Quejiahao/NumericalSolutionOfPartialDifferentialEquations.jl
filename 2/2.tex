% !TeX program	= xelatex
% !TeX encoding	= UTF-8

%-------------------- 文类 --------------------
\documentclass[UTF8, a4paper, 12pt, oneside, onecolumn]{article}

%-------------------- 宏包 --------------------
\input{../../template/usepackage}

%-------------------- 杂项 --------------------
\input{../../template/misc}
\def\homeworkName{抛物型偏微分方程的差分方法}
\hypersetup
{
	% 颜色
	colorlinks	= true,
	linkcolor	= black,
	urlcolor	= red,
	citecolor	= black,
	anchorcolor	= blue,
}

%-------------------- 字体设置 --------------------
\input{../../template/fonts}

%-------------------- 标题样式 --------------------
\input{../title}

%-------------------- 自定义符号 --------------------
\input{../../template/symbols}

%-------------------- 自定义环境 --------------------
\input{../../template/environments}

%-------------------- \item 编号 --------------------
\input{../../template/itemstyle}

%-------------------- 代码样式设置 --------------------
\input{../../template/codestyle}

%-------------------- PDF 元信息, 建模时注释掉 --------------------
\input{../pdfinfo}

%-------------------- 页眉页脚 --------------------
\input{../headersfooters}
\usepackage[ntheorem]{empheq}

%-------------------- 正文 --------------------
\begin{document}

\thispagestyle{plain}

%\columnseprule = 1pt	% 栏线
\begin{center}
	{\zihao{-2}\heiti \homeworkName} \\
	\vspace{1.5ex}
	{\zihao{-4}\fangsong 阙嘉豪\textsuperscript{\hyperref[auth:1]{1}}} \\
	{\zihao{6}\songti \label{auth:1}(1. 北京师范大学 数学科学学院, 北京~~100875)}
\end{center}

\zihao{5}

%\watermark{60}{10}{\currenttime}

\section{模型问题}

考虑 $\Omega := (0, 1)$ 上的热传导方程的 Dirichlet 初边值问题:
\begin{subequations}\label{equ:DirichletProblem}
	\begin{empheq}[left = {\empheqlbrace}]{align}
		u_t &= u_{xx},	&	0 < x < 1,&~t > 0,		\label{equ:DirichletProblem1}\\
		u(x, 0) &= u^0(x),	&	0 \leq x \leq 1,&	\nonumber\\
		u(0, t) &= u(1, t) = 0,	&	&~t > 0.		\label{equ:DirichletProblem3}
	\end{empheq}
\end{subequations}
由分离变量法可以证明, 方程 \eqref{equ:DirichletProblem1} 在齐次 Dirichelet 边界条件 \eqref{equ:DirichletProblem3} 下的一族相互独立的非平凡特解为
$$u_k(x, t) = \e^{-k^2\pi^2 t} \sin k\pi x,\quad k = 1, 2, \cdots.$$
若初值 $u_0$ 可以展开成 Fourier 正弦级数, 即
\begin{equation*}
	u^0\(x\) = \dsum_{k=1}^{\infty}a_k\sin k\pi x,  
\end{equation*}
其中
\begin{equation*}
	a_k = 2\int_0^1 u^0\(x\) \sin k\pi xdx, \quad k=1,2, \cdots,
\end{equation*}
则模型问题 \eqref{equ:DirichletProblem} 的解可以表示为
\begin{equation}\label{2.2.7}
	u\(x, t\) = \dsum_{k=1}^{\infty}a_k \mathrm{e}^{-k^2\pi^2 t}\sin k\pi x. 
\end{equation}

\subsection{差分逼近}

下面考虑模型问题 \eqref{equ:DirichletProblem} 的差分逼近. 首先, 在$[0,1]\times \mathbb{R}_+$ 上引入网格. 为简单起见, 我们只限于考虑均匀网格. 任给正整数$N$, 令空间步长
$$h=h_N = \Delta x=\frac{1}{N},\quad x_j = jh,\quad j=0,1,\cdots, N,$$
再令时间步长
$$\tau = \Delta t,\quad t_m = m\tau,\quad m=0,1,\cdots,$$
则平行于$t$轴的直线族$x = x_j$, $j=0,1, \cdots, N$ 和平行于$x$轴的直线族$t =t_m$, $m=0,1, \cdots$给出了$[0,1]\times \mathbb{R}_+$上的一个均匀网格, 其网格节点集为$\{(x_j, t_m)\}$. 在不会引起混淆时, 为简化记号, 常将节点$\(x_j, t_m\)$简记为$\(j,m\)$. 其次, 在网格上定义网格函数
\begin{equation*}
	U := U_{\(h, \tau\)} := \lrb{U_j^m :\quad j=0,1, \cdots, N,\quad m=0,1, \cdots}, 
\end{equation*}
模型问题 \eqref{equ:DirichletProblem} 的真解$u$在网格节点$\(x_j, t_m\)$上的取值记为$u_j^m$, 即
\begin{equation*}
	u_j^m = u\(x_j, t_m\). 
\end{equation*}
我们需要用适当的差分算子替换原问题中的微分算子, 从而导出关于网格函数$U$的差分方程初边值问题是适定的, 即其解$U$存在唯一且稳定, 并且$U$在一定的意义下逼近真解$u$. 

以下我们用三种格式: 显式格式, 隐式格式和 Crank-Nicolson 格式求解模型问题, 并分析它们的稳定性, 收敛性和误差.

\subsection{显式格式}

% \begin{figure}[H]
% 	\centering\zihao{-5}
% 	\begin{tikzpicture}[scale=1.0,bullet/.style={circle,fill=#1,inner sep=1.5pt}] 
% 		\draw[step=0.5cm] (0,0) grid (9,4.25) foreach \X in {4,4.5,5} {foreach \Y in {2}
% 			{
% 				(\X,\Y) node[bullet=black]{}
% 		}}
% 	(4.5,2.5) node[bullet=black]{}
% 	(4,-0.1) node[below]{\tiny$j-1$}
% 	(4.5,-0.1) node[below]{\tiny$j$}
% 	(5,-0.1) node[below]{\tiny$j+1$}
% 	(-0.1,2) node[left]{\tiny$m$}
% 	(-0.1,2.5) node[left]{\tiny$m+1$}
% 	(-0.3,-0) node[below]{$O$};
% 		\draw[->]	(0,-0.5) -- (0,4.5) node[above]{$t$};
% 		\draw[->] (-0.5,0) -- (9.5,0) node[right]{$x$};
% 	\end{tikzpicture}
% 	\caption{显式格式均匀网格示意图}
% 	\label{1}
% \end{figure}

% 对定义在如图 \ref{1} 所示的均匀网格上的网格函数, 
在时间方向用一阶向前差商$\frac{\Delta_{+t}}{\Delta t}$替换时间一阶微商$\frac{\partial }{\partial t}$, 在空间方向用二阶中心差商$\frac{\delta_x^2}{\(\Delta x\)^2}$替换空间二阶微商$\frac{\partial^2}{\partial x^2}$, 就得到了模型问题 \eqref{equ:DirichletProblem} 的最简单的差分格式:
\begin{subequations}\label{2.2}
	\begin{empheq}[left = {\empheqlbrace}]{align}
		\frac{U_{j}^{m+1}-U_j^m}{\tau } &= \frac{U_{j+1}^m-2U_j^m+U_{j-1}^m}{h^2}, & 1&\leq j\leq N-1,& m&\geq 0,\label{2.2.8}\\
		U_j^0 &=u_j^0, &0&\leq j\leq N,&\nonumber\\
		U_0^m &=U_N^m=0, &&&m&\geq 1.\nonumber
	\end{empheq}
\end{subequations}
由 \eqref{2.2.8} 式可得
\begin{equation}\label{2.1}
	U_j^{m+1} = \mu U_{j+1}^m +\(1-2\mu\)U_j^m +\mu U_{j-1}^m, \quad \mu=\frac{\tau}{h^2}. 
\end{equation}
如果已知第$m$个时间层$t_m$上的网格函数值$U^m = \{U_j^m\}_{j=0}^N$, 则由 \eqref{2.1} 式即可相互独立地直接计算出$U_j^{m+1}$, $j=1, \cdots, N-1$. 我们把这样的差分格式称为两时间层向前差分显式差分格式, 简称显式格式.

\subsubsection{显式格式的截断误差}

引入截断误差算子
\begin{equation}\label{2.2.11}
	T_{\(h, \tau\)} = \(\frac{\Delta_{+t}}{\tau} - \frac{\delta_x^2}{h^2}\) - \(\frac{\partial }{\partial t} - \frac{\partial^2}{\partial x^2}\).
\end{equation}
以下为记号简单起见, 我们经常会省略掉$T_{\(h, \tau\)}$及其他一些符号的下标$\(h, \tau\)$. 

设$u$是定义在$\(0, 1\)\times \mathbb{R}_+$上的充分光滑的函数, 由$u$在点$\(x, t\)$的Taylor展开式易得
\begin{align*}
	\Delta_{+t} u\(x, t\) & :=  u\(x, t+\Delta t\) - u\(x, t\) \\
	&~= u_t\(x, t\)\Delta t+\frac{1}{2}u_{tt}\(x, t\)\(\Delta t\)^2+\frac{1}{6}u_{ttt}\(x, t\)(\Delta t^3)+\cdots, \\
	\delta_x^2 u(x, t) & :=  u\(x+\Delta x, t\) - 2u\(x, t\)+u\(x-\Delta x, t\) \\
	&~= u_{xx}\(x, t\)\(\Delta x\)^2+\frac{1}{12}u_{xxxx}\(x, t\)\(\Delta x\)^4+\cdots, 
\end{align*}
则有
\begin{equation*}
	T u\(x, t\) = \frac{1}{2}u_{tt}\(x, t\)\tau-\frac{1}{12}u_{xxxx}\(x, t\)h^2+O\(\tau^2+h^4\). 
\end{equation*}
由此可知
\begin{equation*}
	\lim\limits_{h \to 0, \tau \to 0}T_{\(h, \tau\) } = 0, 
\end{equation*}
即差分格式是相容的. 由余项形式的Taylor展开
\begin{align*}
	\Delta_{+t} u\(x, t\) &= u_t\(x, t\)\Delta t+\frac{1}{2}u_{tt}\(x, \eta\)\(\Delta t\)^2, \\
	\delta_x^2 u(x, t) &= u_{xx}\(x, t\)\(\Delta x\)^2+\frac{1}{12}u_{xxxx}\(\xi, t\)\(\Delta x\)^4, 
\end{align*}
其中$\eta\in \(t, t+\tau\)$, $\xi\in \(x-h, x+h\)$, 可将截断误差表示为
\begin{equation*}
	Tu\(x, t\) = \frac{1}{2}u_{tt}\(x, \eta\)\tau -\frac{1}{12}u_{xxxx}\(\xi, t\)h^2,
\end{equation*}
即
\begin{equation*}
	Tu\(x, t\)=O\(\tau\)+O\(h^2\).
\end{equation*}
我们称显式格式 \eqref{2.2.8} 的截断误差关于时间和空间分别具有一阶和二阶精度. 

进一步, 记$\Omega_{t_{\max}} = \(0,1\)\times \(0, t_{\max}\)$, 其中 $t_{\max}$是取定的一个时刻. 当
$$M_{tt}  :=  \max\limits_{\(x, t\)\in \bar{\Omega}_{t_{\max}}}\left| u_{tt}\(x, t\)\right|,\quad M_{xxxx}  :=  \max\limits_{\(x, t\)\in \bar{\Omega}_{t_{\max}}}\left| u_{xxxx}\(x, t\)\right|$$
为有界量时, 我们有
\begin{equation}\label{2.2.19}
	\left| Tu\(x, t\)\right| \leq \frac{1}{2}M_{tt}\tau+\frac{1}{12}M_{xxxx}h^2, \quad \forall \(x, t\)\in \Omega_{t_{\max}}. 
\end{equation}

\subsubsection{显式格式的稳定性, 收敛性分析}

\begin{Conclusion}
	显式格式精度为$O\(\tau+h^2\)$, 且 $0<\mu\leq \frac{1}{2}$时, 显式格式在$\mathbb{L}^{\infty}$意义下具有稳定性和收敛性, 收敛速度为$O\(\tau\)$. 
\end{Conclusion}

\begin{Proof}
% 由 \eqref{2.1} 式可知
% \begin{align*}
% 	L_{\(h, \tau\)} U_{j}^{m+1} :=  \(\frac{\delta_x^2}{\Delta x^2}-\frac{\Delta_{+t}}{\Delta t}\) U_j^m = 0. 
% \end{align*}
% 又对于模型问题的解$u$有
% \begin{align*}
% 	L_{\(h, \tau\)} u_{j}^{m+1}+Tu_j^{m+1} = 0. 
% \end{align*}
定义逼近误差$e=U-u$的网格函数
\begin{equation*}
	e_j^m := U_j^m -u\(x_j, t_m\), \quad j=0,1, \cdots, N,\quad m=0,1, \cdots,
\end{equation*}
则由方程 \eqref{equ:DirichletProblem1}, \eqref{2.2.8} 和截断误差算子的定义 \eqref{2.2.11} 知, 逼近误差$e$是以下差分方程初边值问题的解:
\begin{subequations}
	\begin{empheq}[left = {\empheqlbrace}]{align}
		\frac{e_{j}^{m+1}-e_j^m}{\tau } &= \frac{e_{j+1}^m-2e_j^m+e_{j-1}^m}{h^2}-T_j^m, & 1&\leq j\leq N-1,& m&\geq 0,\label{2.2.213}\\
		e_j^0 &=0, &0 &\leq j\leq N,&\label{2.2.223}\\
		e_0^m &=e_N^m=0, &&&m&\geq 1,\label{2.2.233}.
	\end{empheq}
\end{subequations}
其中, $T_j^m = Tu_j^m$. 对于显式格式, 我们有
\begin{subequations}
	\begin{empheq}[left = {\empheqlbrace}]{align}
		U_j^{m+1} &= \(1-2\mu\)U_j^m+\mu\(U_{j-1}^m +U_{j+1}^m\),\nonumber\\
		e_j^{m+1} &= \(1-2\mu\)e_j^m+\mu\(e_{j-1}^m +e_{j+1}^m\)-\tau T_j^m,\label{3.15}
	\end{empheq}
\end{subequations}
因 $\mu>0$ 且$1 = \(1-2\mu\)+\mu+\mu$, 则要使得最大值原理成立, 只需满足$1-2\mu\geq 0$.  即$\mu\leq \frac{1}{2}$时, 有
\begin{equation*}
	\max_{1\leq j\leq N-1}|U_j^{m}|\leq \max\left\{
	\max_{0\leq j\leq N}| U_j^0|, \max_{1\leq l\leq m}\max\left\{|U_0^l |, |U_N^l |
	\right\}
	\right\}, \quad \forall m\geq 0. 
\end{equation*} 
由 \eqref{3.15} 式以及$\mu\leq \frac{1}{2}$的假设, $e^m$各项系数非负, 且其和不大于$e_j^{m+1}$的系数, 则
\begin{align*}
	|e_j^{m+1}| &\leq \mu|e_{j-1}^m| +\(1-2\mu\)|e_j^m|+\mu |e_{j+1}^m|+\tau |T_j^m|\\
	&\leq \mu \ep ^{m}+\(1-2\mu\)\ep ^{m}+\mu \ep ^{m}+\tau \mathcal{T}^{m}\\
	&=\ep ^{m}+\tau \mathcal{T}^{m}, & 1\leq j\leq N-1,
\end{align*}
其中 $\ep ^m= \max\limits_{0\leq j\leq N} |e_j^m|$, $\mathcal{T}^m = \max\limits_{1\leq j\leq N-1}|T_j^m|$. 由此, 可归纳得到
\begin{align*}
	\ep ^n&\leq \max\left\{\ep ^{n-1}, \max\{|e_0^n |, |e_N^n |\}\right\}+\tau \mathcal{T}^{n-1} \\
	&\leq \max\left\{\ep ^{n-2}, \max_{n-1\leq l\leq n}\max\{|e_0^l |, |e_N^l |\}\right\}+\tau \mathcal{T}^{n-1}+\tau \mathcal{T}^{n-2}\\
	&\leq \cdots \\
	&\leq \max\left\{\ep ^{0}, \max_{1\leq l\leq n}\max\{|e_0^l |, |e_N^l |\}\right\}+\tau\dsum_{k=1}^{n-1}\mathcal{T}^k, 
\end{align*}
此即
\begin{align*}
	\ep ^n &\leq \max\left\{\ep ^{0}, \max_{1\leq l\leq n}\max\{|e_0^l |, |e_N^l |\}\right\}+\(n-1\)\tau \wh{\mathcal{T}}\\
	&\leq \max\left\{\ep ^{0}, \max_{1\leq l\leq n}\max\{|e_0^l |, |e_N^l |\}\right\}+t_{\max}\wh{\mathcal{T}}, 
\end{align*}
其中$\wh{\mathcal{T}} = \max\limits_n \mathcal{T}^n =\|T\|_{\infty, \Omega_{t_{\max}}}$,  则
\begin{equation*}
		\|e\|_{\infty, \Omega_{t_{\max}}} \leq \max\left\{
		\max_{0\leq j\leq N}| e_j^0|, \max_{0< m\tau\leq t_{\max}}\{|e_0^m |+ |e_N^m |\}
		\right\}
	+ t_{\max}\|T\|_{\infty, \Omega_{t_{\max}}} . 
\end{equation*}
该式表明误差方程 \eqref{2.2.213} 在$\mathbb{L}^{\infty}\(\Omega_{t_{\max}}\)$范数意义下是稳定的. 又由于误差$e$满足齐次初边值条件 \eqref{2.2.223} 和 \eqref{2.2.233}, 则在差分格式 \eqref{2.2.8} 的 $\mathbb{L}^{\infty}\(\Omega_{t_{\max}}\)$ 稳定性和相容性条件下, 离散问题 \eqref{2.2} 的解在$\mathbb{L}^{\infty}\(\Omega_{t_{\max}}\)$的意义下收敛到真解. 
特别地, 当$M_{xxxx} $为有界量时, 由 \eqref{2.2.19} 式可得
\begin{equation*}
	\|e\|_{\infty, \Omega_{t_{\max}}} \leq  \tau \(\frac{1}{2}+\frac{1}{12\mu}\)M_{xxxx}  t_{\max}, 
\end{equation*}
但是注意到$t_{\max} \to \infty$时, 以上估计并不能保证格式的收敛性. 我们有下面更精细的估计
\begin{equation}\label{2.xxx}
	\|e\|_{\infty, \Omega_{t_{\max}}}\leq \tau\(\frac{1}{2}+\frac{1}{12\mu}\)M_{xxxx} \min\left\{t_{\max}, \frac{1}{8}\right\}. 
\end{equation}
下面证明 \eqref{2.xxx} 式. 取比较函数
$$\Phi_j^m = t_m\|T\|_{\infty, \Omega_{t_{\max}}},\quad \psi_j^m =\frac{1}{2}x_j\(1-x_j\)\|T\|_{\infty, \Omega_{t_{\max}}}.$$
可以验证
\begin{equation*}
	L_{\(h, \tau\)}\(e_j^m-\Phi_j^m\)\geq 0, \quad L_{\(h, \tau\)}\(e_j^m-\psi_j^m\)\geq 0.
\end{equation*}
事实上
\begin{equation*}
	L_{\(h, \tau\)}\Phi_j^{m}=-\|T\|_{\infty, \Omega_{t_{\max}}}, \quad L_{\(h, \tau\)}\psi_j^m = -\|T\|_{\infty, \Omega_{t_{\max}}},  
\end{equation*}
而
\begin{equation*}
	L_{\(h, \tau\)} e_j^m =T_j^m, 
\end{equation*}
故有
\begin{equation*}
	L_{\(h,\tau\)} \(e_j^m-\Phi_j^m\) = L_{\(h, \tau\)} \(e_j^m-\psi_j^m\) = \|T\|_{\infty, \Omega_{t_{\max}}}+T_j^m\geq 0. 
\end{equation*}
从而由最大值原理
\begin{align*}
	\|e\|_{\infty, \Omega_{t_{\max}}}\leq & \max\left\{
	\max_{0\leq j\leq N}| e_j^0|, \max_{1\leq l\leq m}\max\left\{|e_0^l |, |e_N^l |
	\right\}
	\right\}\\
	&+\min\left\{t_m, \frac{1}{2}x_j\(1-x_j\)\right\}\|T\|_{\infty,\Omega_{t_{\max}}},
\end{align*}
由此可得
\begin{equation*}
	\|e\|_{\infty, \Omega_{t_{\max}}}\leq \tau\(\frac{1}{2}+\frac{1}{12\mu}\)M_{xxxx}  \min\left\{t_{\max}, \frac{1}{8}\right\}
\end{equation*}
成立. 因此$\tau \to 0$时, $\|e\|_{\infty, \Omega_{t_{\max}}} \to 0$, 即显式格式收敛. 

综上所述, 显式格式精度为$O\(\tau+h^2\)$, 且 $0<\mu\leq \frac{1}{2}$时, 在$\mathbb{L}^{\infty}$意义下具有稳定性和收敛性, 收敛速度为$O\(\tau\)$. 
\end{Proof}

\begin{Conclusion}
	显式格式在$0<\mu\leq \frac{1}{2}$时, 有$\mathbb{L}^2$稳定性, 且以$O\(\tau\)$的速度收敛到真解. 
\end{Conclusion}

\begin{Proof}
	考虑离散Fourier变换
	\begin{equation*}
		\begin{cases}
			\wh{V}_k =\dfrac{1}{\sqrt{2}N}\dsum_{j=-N+1}^N V_j \mathrm{e}^{-\mathrm{i}k\pi \frac{j}{N}}, &-N+1\leq k\leq N,\\
			V_j =\dfrac{1}{\sqrt{2}}\dsum_{k=-N+1}^N \wh{V}_k \mathrm{e}^{\mathrm{i}k\pi \frac{j}{N}},&-N+1\leq j\leq N. 
		\end{cases}
	\end{equation*}
其中, $\bm V = \(V_j\)$为网格上的周期函数, 其满足Parseval等式
\begin{equation*}
	\(\frac{1}{N}\dsum_{j=-N+1}^N\left| V_j\right|^2\)^{\frac{1}{2}}=	\|\bm V\|_2 = \|\wh{\bm V}\|_2 = \(\dsum_{k=-N+1}^N\left| \wh{V}_k\right|^2\)^{\frac{1}{2}}, 
\end{equation*}
可知, 对于差分格式
\begin{equation*}
	U_j^{m+1} = \mu U_{j+1}^m +\(1-2\mu\)U_j^m +\mu U_{j-1}^m, \quad \mu=\frac{\tau}{h^2}. 
\end{equation*}
将 Fourier 逆变换代入差分格式中, 有
\begin{equation*}
	\dsum_{k=-N+1}^N \wh{V}_k^{m+1}\mathrm{e}^{\mathrm{i} k\pi \frac{j}{N}} =\dsum_{k=-N+1}^N \wh{V}_k^m \(\mu \mathrm{e}^{\mathrm{i} k\pi \frac{j-1}{N}} +\(1-2\mu\) \mathrm{e}^{\mathrm{i} k\pi \frac{j}{N}} +\mu \mathrm{e}^{\mathrm{i} k\pi \frac{j+1}{N}} \), 
\end{equation*}
对于每一个$k$都有
\begin{align*}
	\wh{V}_k^{m+1} &= \wh{V}_k^m \(\(1-2\mu\)+\mu \(\mathrm{e}^{\mathrm{i} k\pi/N} +\mathrm{e}^{-\mathrm{i} k\pi/N} \)
	\) =\lambda_k \wh{V}_k^m,
\end{align*}
其中
\begin{equation*}
	\lambda_k = 1-4\mu \sin^2\frac{k\pi h}{2}. 
\end{equation*}
要保证格式的稳定性, 需
\begin{equation*}
	\left| \lambda_k^m\right|\leq C, \quad \forall m\tau\leq t_{\max},\quad -N+1\leq k\leq N. 
\end{equation*}
则易知$\mu\leq \frac{1}{2}$时, 由
\begin{equation*}
	4\mu \sin^2\frac{k\pi }{2N}\leq 2
\end{equation*}
可知$|\lambda_k|\leq 1$, 即von Neumann条件成立. 而$\mu>\frac{1}{2}$时, 有
\begin{equation*}
	1-4\mu \sin^2\frac{N\pi}{2N}=1-4\mu<-1, 
\end{equation*}
von Neumann条件不成立. 即$0<\mu\leq \frac{1}{2}$时, 显式格式在$\mathbb{L}^2$意义下稳定, 则
\begin{equation*}
	\|e^{m+1}\|_2\leq \|e^0\|_2+\tau\dsum_{l=0}^m\|T^l\|_2. 
\end{equation*}
注意到, $\tau\dsum_{l=0}^m\|T^l\|_2$可以看作
积分$\dint_0^{\(m+1\)\tau}\|Tu\(\cdot, t\)\|_2\di t$的Riemann和, 因此
\begin{equation*}
	\lim\limits_{\tau \to 0}\dsum_{l=0}^m \|T^l\|_2 = 0, \quad \forall m.
\end{equation*}
综上所述, 显式格式在$0<\mu\leq \frac{1}{2}$时, 有$\mathbb{L}^2$稳定性且以$O\(\tau\)$的速度收敛到真解. 
\end{Proof}

\subsection{隐式格式}

% 对定义在如图$\ref{2}$所示的均匀网格上的网格函数, 
% \begin{figure}[H]
% 	\centering\zihao{-5}
% 	\begin{tikzpicture}[scale=1.0,bullet/.style={circle,fill=#1,inner sep=1.5pt}] 
% 		\draw[step=0.5cm] (0,0) grid (9,4.25) foreach \X in {4,4.5,5} {foreach \Y in {2.5}
% 			{
% 				(\X,\Y) node[bullet=black]{}
% 		}}
% 		(4.5,2) node[bullet=black]{}
% 		(4,-0.1) node[below]{\tiny$j-1$}
% 		(4.5,-0.1) node[below]{\tiny$j$}
% 		(5,-0.1) node[below]{\tiny$j+1$}
% 		(-0.1,2) node[left]{\tiny$m$}
% 		(-0.1,2.5) node[left]{\tiny$m+1$}
% 		(-0.3,-0) node[below]{$O$};
% 		\draw[->]	(0,-0.5) -- (0,4.5) node[above]{$t$};
% 		\draw[->] (-0.5,0) -- (9.5,0) node[right]{$x$};
% 	\end{tikzpicture}
% 	\caption{隐式格式均匀网格示意图}
% 	\label{2}
% \end{figure}
在时间方向用一阶向后差商$\frac{\Delta_{-t}}{\Delta t}$替换时间一阶微商$\frac{\partial }{\partial t}$, 在空间方向用二阶中心差商$\frac{\delta_x^2}{\(\Delta x\)^2}$替换空间二阶微商$\frac{\partial^2}{\partial x^2}$, 就得到了模型问题 \eqref{equ:DirichletProblem} 的最简单的隐式差分格式初边值问题:
\begin{subequations}
	\begin{empheq}[left = {\empheqlbrace}]{align}
		\frac{U_{j}^{m+1}-U_j^m}{\tau } &= \frac{U_{j+1}^{m+1}-2U_j^{m+1}+U_{j-1}^{m+1}}{h^2}, & 1&\leq j\leq N-1, &m\geq 0,\label{2.2.44}\\
		U_j^0 &=u_j^0, &0&\leq j\leq N,&\nonumber\\
		U_0^m &=U_N^m=0, &&&m\geq 1.\nonumber
	\end{empheq}
\end{subequations}
由 \eqref{2.2.44} 式可得
\begin{equation}\label{2.2.47}
	-\mu U_{j-1}^{m+1} +\(1+2\mu\) U_{j+1}^{m+1} -\mu U_{j+1}^{m+1} =U_j^m, \quad \mu=\frac{\tau}{h^2}. 
\end{equation}
这是一个关于$U_j^{m+1}$, $j=1, \cdots, N-1$ 的线性代数方程组, 
系数矩阵%
% \begin{equation}\label{matrix}
% 	A = \begin{pmatrix}
% 		1+2\mu&-\mu&&&\\
% 		-\mu&1+2\mu&-\mu &&\\
% 		&\ddots &\ddots&\ddots&\\
% 		&&-\mu &1+2\mu&-\mu \\
% 		&&&-\mu &1+2\mu
% 	\end{pmatrix}, 
% \end{equation}
% 则
% \begin{equation}\label{equa1}
% 	A\cdot U^{m+1} = U^m, \quad U^m = \(U_1^m, \cdots, U_{N-1}^m\)^T, 
% \end{equation}
% 这
是一个对角占优的三对角正定对称阵, 因此解存在唯一. 
如果已经知道第$m$个时间层$t_m$上的网格函数值$U^m =\{U_j^m\}_{j=0}^N$和边界条件$U_0^{m+1}$, $U_{N}^{m+1}$, 则通过求解线性代数方程组 \eqref{2.2.47} 就可以得到第$m+1$个时间层$t_{m+1}$上的网格函数值$U^{m+1} =\{U_j^{m+1}\}_{j=0}^N$. 
由于新时间层上的网格函数值必须通过联立求解才能从上一时间层的已知网格函数值得到, 所以我们把这样的差分格式称为两时间层向后差分隐式差分格式, 简称隐式格式. 

\subsubsection{隐式格式的截断误差}

引入截断误差算子
\begin{equation}\label{2.2.48}
	T_{\(h, \tau\)} = \(\frac{\Delta_{-t}}{\tau} - \frac{\delta_x^2}{h^2}\) - \(\frac{\partial }{\partial t} - \frac{\partial^2}{\partial x^2}\).
\end{equation}
设$u$是定义在$\(0, 1\)\times \mathbb{R}_+$上的充分光滑的函数, 由$u$在点$\(x, t\)$的Taylor展开式
\begin{align*}
	\Delta_{-t} u\(x, t\) & :=  u\(x, t\) - u\(x, t-\Delta t\) \\
	&~= u_t\(x, t\)\Delta t-\frac{1}{2}u_{tt}\(x, t\)\(\Delta t\)^2+\frac{1}{6}u_{ttt}\(x, t\)(\Delta t^3)+\cdots, \\
	\delta_x^2 u(x, t) & :=  u\(x+\Delta x, t\) - 2u\(x, t\)+u\(x-\Delta x, t\) \\
	&~= u_{xx}\(x, t\)\(\Delta x\)^2+\frac{1}{12}u_{xxxx}\(x, t\)\(\Delta x\)^4+\cdots, 
\end{align*}
有
\begin{align}\label{2.2.49}
	T u\(x, t\) = -\frac{1}{2}u_{tt}\(x, t\)\tau-\frac{1}{12}u_{xxxx}\(x, t\)h^2+O\(\tau^2+h^4\), 
\end{align}
或
\begin{align}\label{2.2.50}
	T u\(x, t\) = -\frac{1}{2}u_{tt}\(x, \tau\)\tau-\frac{1}{12}u_{xxxx}\(\xi, t\)h^2,
\end{align}
其中$\eta\in \(t-\tau, t\)$, $\xi \in \(x-\frac{h}{2}, x+\frac{h}{2}\)$. 
由此可知
\begin{equation*}
	\lim\limits_{h \to 0, \tau \to 0}T_{\(h, \tau\) } = 0, 
\end{equation*}
即差分格式是相容的, 且其局部截断误差关于时间和空间分别具有一阶和二阶精度, 即
\begin{equation*}
	Tu\(x, t\) = O\(\tau+h^2\). 
\end{equation*}

\subsubsection{隐式格式的稳定性, 收敛性分析}

\begin{Conclusion}
	隐式格式无条件$\mathbb{L}^{\infty}$稳定且收敛, 精度为 $O\(\tau+h^2\)$.
\end{Conclusion}

\begin{Proof}
	将差分格式改写为
\begin{subequations}
	\begin{empheq}[left = {\empheqlbrace}]{align}
		\(1+2\mu\)U_j^{m+1} &= \mu\(U_{j-1}^{m+1}+U_{j+1}^{m+1}\)+U_j^m\label{1111}\\
		\(1+2\mu\)e_j^{m+1} &=\mu\(e_{j-1}^{m+1}+e_{j+1}^{m+1}\) +e_j-\tau T_j^{m+1},\label{1112}
	\end{empheq}
\end{subequations}
则对任意的$\mu>0$,  \eqref{1111} 式右端各项系数大于零且其和等于左边系数, 满足最大值原理. 从而
\begin{equation*}
	\max_{1\leq j\leq N-1}|U_j^{m}|\leq \max\left\{
	\max_{0\leq j\leq N}| U_j^0|, \max_{1\leq l\leq m}\max\left\{|U_0^l |, |U_N^l |
	\right\}
	\right\}, \quad \forall m\geq 0. 
\end{equation*} 
又由 \eqref{1112} 式归纳可得
\begin{equation*}
	\max_{1\leq j\leq N-1}|e_j^{m+1}|\leq \max\left\{ \max_{0\leq j\leq N}\{|e_j^0|\}, \max_{0\leq l\leq m+1}\max\{|e_0^l|, |e+N^l|\}
	\right\}+\tau\dsum_{l=1}^{m+1}T^l, 
\end{equation*}
由此, 隐式格式无条件$\mathbb{L}^{\infty}$稳定. 进一步, 取
$$\phi_j^m = t_m\|T\|_{\infty, \Omega_{t_{\max}}},\quad \psi_j^m = \frac{1}{2}x_j\(1-x_j\)\|T\|_{\infty, \Omega_{t_{\max}}}.$$
记
\begin{equation*}
	L_{\(h, \tau\)} = \frac{\delta_x^2}{h^2} - \frac{\Delta_{-t}}{\Delta t}, 
\end{equation*}
可以验证
\begin{equation*}
	L_{\(h, \tau\)}\(e_j^{m+1}-\phi_j^{m+1}\)\geq 0, \quad L_{\(h, \tau\)}\(e_j^{m+1}-\psi_j^{m+1}\)\geq 0,
\end{equation*}
即最大值原理成立, 因此
\begin{align*}
	\|e\|_{\infty, \Omega_{t_{\max}}}\leq ~&\max\left\{
	\max_{0\leq j\leq N}| e_j^0|, \max_{1\leq l\leq m}\max\left\{|e_0^l |, |e_N^l |\right\}\right\}\\
	&+\min\left\{t_m, \frac{1}{2}x_j\(1-x_j\)\right\}\|T\|_{\infty,\Omega_{t_{\max}}},
\end{align*}
由此可得
\begin{equation*}
	\|e\|_{\infty, \Omega_{t_{\max}}}\leq \tau\(\frac{1}{2}+\frac{1}{12\mu}\)M_{xxxx} \min\left\{t_{\max}, \frac{1}{8}\right\}
\end{equation*}
成立. 于是 $\tau \to 0$时, $\|e\|_{\infty, \Omega_{t_{\max}}} \to 0$. 

综上所述, 隐式格式无条件$\mathbb{L}^{\infty}$稳定且收敛, 逼近精度为$O\(\tau+h^2\)$. 
\end{Proof}

\begin{Conclusion}
	隐式格式无条件$\mathbb{L}^2$稳定且收敛, 逼近精度为$O\(\tau+h^2\)$. 
\end{Conclusion}

\begin{Proof}
	对于差分格式
	\begin{equation*}
	-\mu U_{j-1}^{m+1}+	\(1+2\mu\)U_j^{m+1}-\mu 
	U_{j+1}^{m+1} = U_j^m,
	\end{equation*}
	将Fourier波型$U_j^m=\lambda_k^m\mathrm{e}^{\mathrm{i}k\pi jh}$代入格式中可得
	\begin{equation*}
		-\mu\lambda_k^{m+1}\mathrm{e}^{\mathrm{i}k\pi \(j-1\)h} +\(1+2\mu\)\lambda_k^{m+1}\mathrm{e}^{\mathrm{i}k\pi jh}-\mu\lambda_k^{m+1}\mathrm{e}^{\mathrm{i}k\pi \(j+1\)h} =\lambda_k^m \mathrm{e}^{\mathrm{i}k\pi jh}. 
	\end{equation*}
	% 即
	% \begin{equation*}
	% 	-\mu\lambda_k\(\cos k\pi h-\mathrm{i}\sin k\pi h\)+\(1+2\mu \)\lambda_k-\mu\lambda_k\(\cos k\pi h+\mathrm{i}\sin k\pi h\)=1. 
	% \end{equation*}
	于是增长因子为
	\begin{equation*}
		\lambda_k = \frac{1}{1+4\mu \sin^2 \frac{k\pi h}{2}}, 
	\end{equation*}
	则 $|\lambda_k|\leq 1$, 满足von Neumann条件, 即隐式格式无条件$\mathbb{L}^2$稳定. 类似地, 隐式格式收敛. 

	综上所述, 隐式格式无条件$\mathbb{L}^2$稳定且收敛, 逼近精度为$O\(\tau+h^2\)$. 
\end{Proof}

\subsection{Crank-Nicolson 格式}

% 对定义在如图$\ref{3}$所示的均匀网格上的网格函数, 
% \begin{figure}[H]
% 	\centering\zihao{-5}
% 	\begin{tikzpicture}[scale=1.0,bullet/.style={circle,fill=#1,inner sep=1.5pt}] 
% 		\draw[step=0.5cm] (0,0) grid (9,4.25) foreach \X in {4,4.5,5} {foreach \Y in {2}
% 			{
% 				(\X,\Y) node[bullet=black]{}
% 		}}
% 		(4.5,2.5) node[bullet=black]{}
% 		(4,2.5) node[bullet=black]{}
% 		(5,2.5) node[bullet=black]{}
% 		(4,-0.1) node[below]{\tiny$j-1$}
% 		(4.5,-0.1) node[below]{\tiny$j$}
% 		(5,-0.1) node[below]{\tiny$j+1$}
% 		(-0.1,2) node[left]{\tiny$m$}
% 		(-0.1,2.5) node[left]{\tiny$m+1$}
% 		(-0.3,-0) node[below]{$O$};
% 		\draw[->]	(0,-0.5) -- (0,4.5) node[above]{$t$};
% 		\draw[->] (-0.5,0) -- (9.5,0) node[right]{$x$};
% 	\end{tikzpicture}
% 	\caption{Crank-Nicolson 格式均匀网格示意图}
% 	\label{3}
% \end{figure}
在点$\(x, t+\frac{1}{2}\Delta t\)$采用关于时间的一阶中心差商$\frac{\delta_t}{\Delta t}$替换一阶微商$\frac{\partial }{\partial t}$, 用$\(x, t+\Delta t\)$和$\(x, t\)$两点的关于空间的二阶中心差商$\frac{\delta_x^2}{\(\Delta x\)^2}$的平均值替换二阶微商$\frac{\partial^2}{\partial x^2}$, 就得到了模型问题 \eqref{equ:DirichletProblem} 的 Crank-Nicolson 格式初边值问题:
\begin{equation*}
	\lb\begin{aligned}
		\frac{U_{j}^{m+1}-U_j^m}{\tau } &=\frac{1}{2} \(\frac{U_{j+1}^m-2U_j^m+U_{j-1}^m}{h^2}\rd\\
		&\quad \ld +\frac{U_{j+1}^{m+1}-2U_j^{m+1}+U_{j-1}^{m+1}}{h^2}
		\), & 1&\leq j\leq N-1, &m&\geq 0,\\
		U_j^0 &=u_j^0, &0&\leq j\leq N,&&\\
		U_0^m &=U_N^m=0, &&&m&\geq 1.
	\end{aligned}\rd
\end{equation*}

\subsubsection{Crank-Nicolson 格式的截断误差}

引入截断误差算子
\begin{equation*}
	T_{\(h, \tau\)} = \(\frac{\delta_{t}}{\tau} - \frac{\delta_x^2}{h^2}\) - \(\frac{\partial }{\partial t} - \frac{\partial^2}{\partial x^2}\).
\end{equation*}
设$u$是定义在$\(0, 1\)\times \mathbb{R}_+$上的充分光滑的函数, 由$u$在点$\(x, t+\frac{1}{2}\Delta t\)$的Taylor展开式
\begin{align*}
	\delta_{t} u\(x, t\) & :=  u(x, t+\Delta t) - u(x, t) \\
	&~= u_t\(x, t+\frac{1}{2}\Delta t\)\Delta t+\frac{1}{24}u_{ttt}\(x, t+\frac{1}{2}\Delta t\)\(\Delta t\)^3+\cdots.
\end{align*}
由$u$在点$\(x, t\) $的Taylor展开式
\begin{align*}
	\mathcal{A} u(x, t) & :=  -\frac{u\(x-\Delta x, t\) -2u\(x, t\)+u\(x+\Delta x, t\)}{\Delta x^2} \\
	&~=-\(u_{xx}\(x, t\)+\frac{1}{12}u_{xxxx}\(x, t\)\Delta x^2\)+\cdots
\end{align*}
现将右端的每一项在$\(x, t+\frac{1}{2}\Delta t\)$ 处 Taylor 展开得
\begin{align*}
	\mathcal{A} u(x, t) %=-\(u_{xx}\(x, t\)+\frac{1}{12}u_{xxxx}\(x, t\)\Delta x^2\)+\cdots \\
	=&-\(u_{xx}(x, t+\frac{1}{2}\Delta t)+\frac{1}{12}u_{xxxx}(x, t+\frac{1}{2}\Delta t)\Delta x^2\)+\cdots \\
	&+\frac{\Delta t}{2}\( u_{xxt}(x, t+\frac{1}{2}\Delta t)+\frac{1}{12}u_{xxxxt}(x, t+\frac{1}{2}\Delta t)\Delta x^2 \)+\cdots  \\
	&-\frac{1}{2}\(\frac{\Delta t}{2}\)^2 \(u_{xxtt}(x, t+\frac{1}{2}\Delta t)+\frac{1}{12} u_{xxxxtt}(x, t+\frac{1}{2}\Delta t)\Delta x^2\) +\cdots 
\end{align*}
同理可得
\begin{align*}
	\mathcal{A} u(x, t+\Delta t) %&=-[u_{xx}(x, t+\Delta t)+\frac{1}{12}u_{xxxx}(x, t+\Delta t)\Delta x^2]+\cdots \\
	=&-\( u_{xx}(x, t+\frac{1}{2}\Delta t)+\frac{1}{12}u_{xxxx}(x, t+\frac{1}{2}\Delta t)\Delta x^2 \)+\cdots \\
	&-\frac{\Delta t}{2} \(u_{xxt}(x, t+\frac{1}{2}\Delta t)+\frac{1}{12}u_{xxxxt}(x, t+\frac{1}{2}\Delta t)\Delta x^2 \)+\cdots  \\
	& -\frac{1}{2}\(\frac{\Delta t}{2}\)^2 \(u_{xxtt}(x, t+\frac{1}{2}\Delta t)+\frac{1}{12} u_{xxxxtt}(x, t+\frac{1}{2}\Delta t)\Delta x^2\) +\cdots 
\end{align*}
由$u_t = u_{xx}$可知 Crank-Nicolson 格式的局部截断误差为
\begin{equation*}
	T_j^{m+\frac{1}{2}}  :=  Tu\(x_j, t_{m+\frac{1}{2}}\) =-\frac{1}{12}\left( u_{ttt}\tau^2+u_{xxxx}h^2\right)_j^{m+\frac{1}{2}}+O\(\tau^4+h^4\), 
\end{equation*}
或
\begin{equation*}
	T_j^{m+\frac{1}{2}}  :=  Tu\(x_j, t_{m+\frac{1}{2}}\) =-\frac{1}{12}\left(u_{ttt}\(x_j, \eta\)\tau^2+u_{xxxx}\(\xi, t_{m+\frac{1}{2}}\)h^2\right). 
	\end{equation*}
其中$\eta\in \(t, t+\tau\)$, $\xi\in \(x-h, x+h\)$. 由此可知
\begin{equation*}
	\lim\limits_{h \to 0, \tau \to 0}T_{\(h, \tau\) } = 0, 
\end{equation*}
即差分格式是相容的, 且其局部截断误差关于时间和空间分别具有二阶和二阶精度, 即
\begin{equation*}
	Tu\(x, t\) = O\(\tau^2+h^2\). 
\end{equation*}

\subsubsection{Crank-Nicolson 格式的稳定性, 收敛性分析}

\begin{Conclusion}
	$0<\mu\leq 1$ 时, Crank-Nicolson 格式$\mathbb{L}^{\infty}$稳定且收敛, 精度为$O\(\tau^2+h^2\)$. 
\end{Conclusion}

\begin{Proof}
	将差分格式改写为
	\begin{equation*}
			\(1+\mu\)U_j^{m+1} =\(1-\mu\) U_j^{m}+\frac{\mu}{2}\(U_{j-1}^m+U_{j+1}^m+U_{j-1}^{m+1}+U_{j}^{m+1}\),
	\end{equation*}
	易知$1-\mu\geq 0$时, 满足最大值原理. 从而
		\begin{equation*}
			\max_{1\leq j\leq N-1}|e_j^{m+1}|\leq \max\left\{ \max_{0\leq j\leq N}\{|e_j^0|\}, \max_{0\leq l\leq m+1}\max\{|e_0^l|, |e_N^l|\}
			\right\}+\tau\dsum_{l=1}^{m+1}T^l, 
		\end{equation*}
	即$\mu\leq 1$时,  Crank-Nicolson 格式$\mathbb{L}^{\infty}$稳定. 进一步, 取
	$$\phi_j^m = t_m\|T\|_{\infty, \Omega_{t_{\max}}},\quad  \psi_j^m = \frac{1}{2}x_j\(1-x_j\)\|T\|_{\infty, \Omega_{t_{\max}}}.$$
	记
	\begin{equation*}
		L_{\(h, \tau\)} := \frac{\delta_x^2}{h^2} - \frac{\delta_{t}}{\Delta t}.
	\end{equation*}
	可以验证
	\begin{equation*}
		L_{\(h, \tau\)}\(e_j^{m+1}-\phi_j^{m+1}\)\geq 0, \quad L_{\(h, \tau\)}\(e_j^{m+1}-\psi_j^{m+1}\)\geq 0,
	\end{equation*}
	即最大值原理成立, 因此
	\begin{align*}
		\|e\|_{\infty, \Omega_{t_{\max}}}\leq &\max\left\{
		\max_{0\leq j\leq N}| e_j^0|, \max_{1\leq l\leq m}\max\left\{|e_0^l |, |e_N^l |
		\right\}\right\}\\
		&+\min\left\{t_m, \frac{1}{2}x_j\(1-x_j\)\right\}\|T\|_{\infty,\Omega_{t_{\max}}}.
	\end{align*}
	由此可得
	\begin{equation*}
		\|e\|_{\infty, \Omega_{t_{\max}}}\leq \frac{1}{12}\(M_{ttt}\tau^2+M_{xxxx}h^2\) \min\left\{t_{\max}, \frac{1}{8}\right\}
	\end{equation*}
	成立. 因此$\tau \to 0$时, $\|e\|_{\infty, \Omega_{t_{\max}}} \to 0$. 
	
	综上所述,  Crank-Nicolson 格式在$0<\mu\leq 1$时, $\mathbb{L}^{\infty}$稳定且收敛, 精度为$O\(\tau^2+h^2\)$. 
\end{Proof}

\begin{Conclusion}
	 Crank-Nicolson 格式无条件$\mathbb{L}^2$稳定且收敛, 精度为$O\(\tau^2+h^2\)$. 
\end{Conclusion}

\begin{Proof}
	将Fourier波型$U_j^m=\lambda_k^m\mathrm{e}^{\mathrm{i}k\pi jh}$代入差分格式中可得
	\begin{equation*}
	\lambda_k = \frac{1-2\mu\sin^2\frac{k\pi h}{2}}{1+2\mu \sin^2\frac{k\pi h}{2}}, 
	\end{equation*}
	则 $|\lambda_k|\leq 1$, 满足von Neumann条件, 即 Crank-Nicolson 格式无条件$\mathbb{L}^2$稳定. 进一步
	\begin{equation*}
		\|e^m\|_2 = O\(\tau^2\), 
	\end{equation*}
	二阶收敛. 

	综上所述,  Crank-Nicolson 格式无条件$\mathbb{L}^2$稳定且收敛, 逼近精度为$O\(\tau^2+h^2\)$. 
\end{Proof}

\section{数值实验}

\subsection{显式格式}

下面我们考虑$[0,1]\times \mathbb{R}_+$上的问题:
\begin{subequations}\label{solu}
	\begin{empheq}[left = {\empheqlbrace}]{align}
		u_t &= \frac{1}{2} u_{xx}, & 0&<x<1, &t>0,\label{solu1}\\
		u\(x, 0\) &= \sin\(\pi x\), &0&\leq x\leq 1,&\label{solu2}\\
		u\(0, t\) &= u\(1, t\) =0, &&&t>0\label{solu3}.
	\end{empheq}
\end{subequations}
由 \eqref{2.2.7} 式可知, 其解析解为
\begin{equation*}
	u\(x, t\) = \mathrm{e}^{-\frac{1}{2}\pi^2 t}\sin\(\pi x\). 
\end{equation*}
由显式差分格式 \eqref{2.2.8} 以及边值条件 \eqref{solu2}, \eqref{solu3} 可得
\begin{subequations}
	\begin{empheq}[left = {\empheqlbrace}]{align}
		U_{j}^{m+1} &=\(1-2\mu\)U_j^m+\mu\(U_{j-1}^m+U_{j+1}^m\),& 1 &\leq j\leq N-1, &m&\geq 0,\nonumber\\
		U_j^0 &=\sin\(jh\pi\), & 0 &\leq j\leq N,&&\nonumber\\
		U_0^m &=U_N^m=0, &&&m&\geq 1,\nonumber
	\end{empheq}
\end{subequations}
其中 $\mu = \dfrac{a\tau}{h^2}$.

\subsubsection{稳定性条件成立}

取定$t_{\max} =1$. 令$\mu = 0.25\leq 0.5$, 满足稳定性条件, 数值实验结果如表 \ref{tab:expliciterr1} 所示.
\begin{table}[H]\centering\heiti\zihao{-5}
	\caption{显式格式不同步长时的 $\mathbb{L}^2$, $\mathbb{L}^\infty$ 误差及收敛阶}\label{tab:expliciterr1}
	\begin{tabular}{|c|c|c|c|c|}\hline
		收敛阶	&	$\mathbb{L}^2$ 误差	&	$h$	&	$\mathbb{L}^\infty$ 误差		&	收敛阶\\\hline			&	$1.5871 \times 10^{-3}$	&	$2^3$	&	$2.3807 \times 10^{-3}$	&	\\\hline
		$1.9665$	&	$4.0611 \times 10^{-4}$	&	$2^4$	&	$5.9200 \times 10^{-4}$	&	$2.0077$\\\hline
		$1.9804$	&	$1.0292 \times 10^{-4}$	&	$2^5$	&	$1.4780 \times 10^{-4}$	&	$2.0019$\\\hline
		$1.9895$	&	$2.5918 \times 10^{-5}$	&	$2^6$	&	$3.6939 \times 10^{-5}$	&	$2.0005$\\\hline
		$1.9946$	&	$6.5040 \times 10^{-6}$	&	$2^7$	&	$9.2339 \times 10^{-6}$	&	$2.0001$\\\hline
		$1.9972$	&	$1.6291 \times 10^{-6}$	&	$2^8$	&	$2.3084 \times 10^{-6}$	&	$2.0000$\\\hline
		$1.9986$	&	$4.0768 \times 10^{-7}$	&	$2^9$	&	$5.7711 \times 10^{-7}$	&	$2.0000$\\\hline
		$1.9993$	&	$1.0197 \times 10^{-7}$	&	$2^{10}$	&	$1.4428 \times 10^{-7}$	&	$2.0000$\\\hline
		$1.9997$	&	$2.5498 \times 10^{-8}$	&	$2^{11}$	&	$3.6069 \times 10^{-8}$	&	$2.0000$\\\hline		
	\end{tabular}
\end{table}
\noindent 由结果可以看出解序列逐步收敛到模型问题 \eqref{solu} 的解, 且收敛阶趋于 $2$, 与理论结果相符.
%   数值解$U_{i,j}$与$u_{i,j}$在$t = t_{\max}$时刻图像如图 所示.

% \begin{figure}[H]\centering
% 	% \resizebox{0.9\linewidth}{!}{%% Creator: Matplotlib, PGF backend
%%
%% To include the figure in your LaTeX document, write
%%   \input{<filename>.pgf}
%%
%% Make sure the required packages are loaded in your preamble
%%   \usepackage{pgf}
%%
%% Figures using additional raster images can only be included by \input if
%% they are in the same directory as the main LaTeX file. For loading figures
%% from other directories you can use the `import` package
%%   \usepackage{import}
%%
%% and then include the figures with
%%   \import{<path to file>}{<filename>.pgf}
%%
%% Matplotlib used the following preamble
%%   \usepackage{fontspec}
%%   \setmainfont{DejaVuSerif.ttf}[Path=\detokenize{/Users/quejiahao/.julia/conda/3/lib/python3.9/site-packages/matplotlib/mpl-data/fonts/ttf/}]
%%   \setsansfont{DejaVuSans.ttf}[Path=\detokenize{/Users/quejiahao/.julia/conda/3/lib/python3.9/site-packages/matplotlib/mpl-data/fonts/ttf/}]
%%   \setmonofont{DejaVuSansMono.ttf}[Path=\detokenize{/Users/quejiahao/.julia/conda/3/lib/python3.9/site-packages/matplotlib/mpl-data/fonts/ttf/}]
%%
\begingroup%
\makeatletter%
\begin{pgfpicture}%
\pgfpathrectangle{\pgfpointorigin}{\pgfqpoint{12.000000in}{8.000000in}}%
\pgfusepath{use as bounding box, clip}%
\begin{pgfscope}%
\pgfsetbuttcap%
\pgfsetmiterjoin%
\definecolor{currentfill}{rgb}{0.152941,0.098039,0.141176}%
\pgfsetfillcolor{currentfill}%
\pgfsetlinewidth{0.000000pt}%
\definecolor{currentstroke}{rgb}{1.000000,1.000000,1.000000}%
\pgfsetstrokecolor{currentstroke}%
\pgfsetdash{}{0pt}%
\pgfpathmoveto{\pgfqpoint{0.000000in}{0.000000in}}%
\pgfpathlineto{\pgfqpoint{12.000000in}{0.000000in}}%
\pgfpathlineto{\pgfqpoint{12.000000in}{8.000000in}}%
\pgfpathlineto{\pgfqpoint{0.000000in}{8.000000in}}%
\pgfpathclose%
\pgfusepath{fill}%
\end{pgfscope}%
\begin{pgfscope}%
\pgfsetbuttcap%
\pgfsetmiterjoin%
\definecolor{currentfill}{rgb}{0.152941,0.098039,0.141176}%
\pgfsetfillcolor{currentfill}%
\pgfsetlinewidth{0.000000pt}%
\definecolor{currentstroke}{rgb}{0.000000,0.000000,0.000000}%
\pgfsetstrokecolor{currentstroke}%
\pgfsetstrokeopacity{0.000000}%
\pgfsetdash{}{0pt}%
\pgfpathmoveto{\pgfqpoint{0.539299in}{0.078740in}}%
\pgfpathlineto{\pgfqpoint{8.381819in}{0.078740in}}%
\pgfpathlineto{\pgfqpoint{8.381819in}{7.921260in}}%
\pgfpathlineto{\pgfqpoint{0.539299in}{7.921260in}}%
\pgfpathclose%
\pgfusepath{fill}%
\end{pgfscope}%
\begin{pgfscope}%
\pgfsetbuttcap%
\pgfsetmiterjoin%
\definecolor{currentfill}{rgb}{0.950000,0.950000,0.950000}%
\pgfsetfillcolor{currentfill}%
\pgfsetfillopacity{0.500000}%
\pgfsetlinewidth{1.003750pt}%
\definecolor{currentstroke}{rgb}{0.950000,0.950000,0.950000}%
\pgfsetstrokecolor{currentstroke}%
\pgfsetstrokeopacity{0.500000}%
\pgfsetdash{}{0pt}%
\pgfpathmoveto{\pgfqpoint{1.131463in}{2.012454in}}%
\pgfpathlineto{\pgfqpoint{3.721319in}{4.183323in}}%
\pgfpathlineto{\pgfqpoint{3.685318in}{7.314104in}}%
\pgfpathlineto{\pgfqpoint{0.971524in}{5.333700in}}%
\pgfusepath{stroke,fill}%
\end{pgfscope}%
\begin{pgfscope}%
\pgfsetbuttcap%
\pgfsetmiterjoin%
\definecolor{currentfill}{rgb}{0.900000,0.900000,0.900000}%
\pgfsetfillcolor{currentfill}%
\pgfsetfillopacity{0.500000}%
\pgfsetlinewidth{1.003750pt}%
\definecolor{currentstroke}{rgb}{0.900000,0.900000,0.900000}%
\pgfsetstrokecolor{currentstroke}%
\pgfsetstrokeopacity{0.500000}%
\pgfsetdash{}{0pt}%
\pgfpathmoveto{\pgfqpoint{3.721319in}{4.183323in}}%
\pgfpathlineto{\pgfqpoint{7.877114in}{2.975397in}}%
\pgfpathlineto{\pgfqpoint{8.025420in}{6.214014in}}%
\pgfpathlineto{\pgfqpoint{3.685318in}{7.314104in}}%
\pgfusepath{stroke,fill}%
\end{pgfscope}%
\begin{pgfscope}%
\pgfsetbuttcap%
\pgfsetmiterjoin%
\definecolor{currentfill}{rgb}{0.925000,0.925000,0.925000}%
\pgfsetfillcolor{currentfill}%
\pgfsetfillopacity{0.500000}%
\pgfsetlinewidth{1.003750pt}%
\definecolor{currentstroke}{rgb}{0.925000,0.925000,0.925000}%
\pgfsetstrokecolor{currentstroke}%
\pgfsetstrokeopacity{0.500000}%
\pgfsetdash{}{0pt}%
\pgfpathmoveto{\pgfqpoint{1.131463in}{2.012454in}}%
\pgfpathlineto{\pgfqpoint{5.536809in}{0.573668in}}%
\pgfpathlineto{\pgfqpoint{7.877114in}{2.975397in}}%
\pgfpathlineto{\pgfqpoint{3.721319in}{4.183323in}}%
\pgfusepath{stroke,fill}%
\end{pgfscope}%
\begin{pgfscope}%
\pgfsetrectcap%
\pgfsetroundjoin%
\pgfsetlinewidth{0.803000pt}%
\definecolor{currentstroke}{rgb}{0.000000,0.000000,0.000000}%
\pgfsetstrokecolor{currentstroke}%
\pgfsetdash{}{0pt}%
\pgfpathmoveto{\pgfqpoint{1.131463in}{2.012454in}}%
\pgfpathlineto{\pgfqpoint{5.536809in}{0.573668in}}%
\pgfusepath{stroke}%
\end{pgfscope}%
\begin{pgfscope}%
\definecolor{textcolor}{rgb}{0.525490,0.694118,0.356863}%
\pgfsetstrokecolor{textcolor}%
\pgfsetfillcolor{textcolor}%
\pgftext[x=3.050786in,y=0.824279in,,]{\color{textcolor}\sffamily\fontsize{24.000000}{13.200000}\bfseries\selectfont $x$}%
\end{pgfscope}%
\begin{pgfscope}%
\pgfsetbuttcap%
\pgfsetroundjoin%
\pgfsetlinewidth{0.803000pt}%
\definecolor{currentstroke}{rgb}{0.690196,0.690196,0.690196}%
\pgfsetstrokecolor{currentstroke}%
\pgfsetdash{}{0pt}%
\pgfpathmoveto{\pgfqpoint{1.945241in}{1.746674in}}%
\pgfpathlineto{\pgfqpoint{4.491766in}{3.959384in}}%
\pgfpathlineto{\pgfqpoint{4.488551in}{7.110508in}}%
\pgfusepath{stroke}%
\end{pgfscope}%
\begin{pgfscope}%
\pgfsetbuttcap%
\pgfsetroundjoin%
\pgfsetlinewidth{0.803000pt}%
\definecolor{currentstroke}{rgb}{0.690196,0.690196,0.690196}%
\pgfsetstrokecolor{currentstroke}%
\pgfsetdash{}{0pt}%
\pgfpathmoveto{\pgfqpoint{2.827694in}{1.458465in}}%
\pgfpathlineto{\pgfqpoint{5.325810in}{3.716961in}}%
\pgfpathlineto{\pgfqpoint{5.358796in}{6.889926in}}%
\pgfusepath{stroke}%
\end{pgfscope}%
\begin{pgfscope}%
\pgfsetbuttcap%
\pgfsetroundjoin%
\pgfsetlinewidth{0.803000pt}%
\definecolor{currentstroke}{rgb}{0.690196,0.690196,0.690196}%
\pgfsetstrokecolor{currentstroke}%
\pgfsetdash{}{0pt}%
\pgfpathmoveto{\pgfqpoint{3.729308in}{1.163998in}}%
\pgfpathlineto{\pgfqpoint{6.176441in}{3.469716in}}%
\pgfpathlineto{\pgfqpoint{6.247107in}{6.664765in}}%
\pgfusepath{stroke}%
\end{pgfscope}%
\begin{pgfscope}%
\pgfsetbuttcap%
\pgfsetroundjoin%
\pgfsetlinewidth{0.803000pt}%
\definecolor{currentstroke}{rgb}{0.690196,0.690196,0.690196}%
\pgfsetstrokecolor{currentstroke}%
\pgfsetdash{}{0pt}%
\pgfpathmoveto{\pgfqpoint{4.650714in}{0.863067in}}%
\pgfpathlineto{\pgfqpoint{7.044159in}{3.217505in}}%
\pgfpathlineto{\pgfqpoint{7.154054in}{6.434880in}}%
\pgfusepath{stroke}%
\end{pgfscope}%
\begin{pgfscope}%
\pgfpathrectangle{\pgfqpoint{0.539299in}{0.078740in}}{\pgfqpoint{7.842520in}{7.842520in}}%
\pgfusepath{clip}%
\pgfsetrectcap%
\pgfsetroundjoin%
\pgfsetlinewidth{0.501875pt}%
\definecolor{currentstroke}{rgb}{0.980392,0.811765,0.352941}%
\pgfsetstrokecolor{currentstroke}%
\pgfsetstrokeopacity{0.100000}%
\pgfsetdash{}{0pt}%
\pgfpathmoveto{\pgfqpoint{4.566539in}{4.105980in}}%
\pgfusepath{stroke}%
\end{pgfscope}%
\begin{pgfscope}%
\pgfsetrectcap%
\pgfsetroundjoin%
\pgfsetlinewidth{0.803000pt}%
\definecolor{currentstroke}{rgb}{0.980392,0.811765,0.352941}%
\pgfsetstrokecolor{currentstroke}%
\pgfsetdash{}{0pt}%
\pgfpathmoveto{\pgfqpoint{1.967428in}{1.765953in}}%
\pgfpathlineto{\pgfqpoint{1.900771in}{1.708034in}}%
\pgfusepath{stroke}%
\end{pgfscope}%
\begin{pgfscope}%
\definecolor{textcolor}{rgb}{0.525490,0.694118,0.356863}%
\pgfsetstrokecolor{textcolor}%
\pgfsetfillcolor{textcolor}%
\pgftext[x=1.841138in,y=1.527121in,,top]{\color{textcolor}\sffamily\fontsize{18.000000}{9.600000}\selectfont $\displaystyle 0.2$}%
\end{pgfscope}%
\begin{pgfscope}%
\pgfpathrectangle{\pgfqpoint{0.539299in}{0.078740in}}{\pgfqpoint{7.842520in}{7.842520in}}%
\pgfusepath{clip}%
\pgfsetrectcap%
\pgfsetroundjoin%
\pgfsetlinewidth{0.501875pt}%
\definecolor{currentstroke}{rgb}{0.980392,0.811765,0.352941}%
\pgfsetstrokecolor{currentstroke}%
\pgfsetstrokeopacity{0.100000}%
\pgfsetdash{}{0pt}%
\pgfpathmoveto{\pgfqpoint{4.566539in}{4.105980in}}%
\pgfusepath{stroke}%
\end{pgfscope}%
\begin{pgfscope}%
\pgfsetrectcap%
\pgfsetroundjoin%
\pgfsetlinewidth{0.803000pt}%
\definecolor{currentstroke}{rgb}{0.980392,0.811765,0.352941}%
\pgfsetstrokecolor{currentstroke}%
\pgfsetdash{}{0pt}%
\pgfpathmoveto{\pgfqpoint{2.849478in}{1.478160in}}%
\pgfpathlineto{\pgfqpoint{2.784030in}{1.418990in}}%
\pgfusepath{stroke}%
\end{pgfscope}%
\begin{pgfscope}%
\definecolor{textcolor}{rgb}{0.525490,0.694118,0.356863}%
\pgfsetstrokecolor{textcolor}%
\pgfsetfillcolor{textcolor}%
\pgftext[x=2.724041in,y=1.236037in,,top]{\color{textcolor}\sffamily\fontsize{18.000000}{9.600000}\selectfont $\displaystyle 0.4$}%
\end{pgfscope}%
\begin{pgfscope}%
\pgfpathrectangle{\pgfqpoint{0.539299in}{0.078740in}}{\pgfqpoint{7.842520in}{7.842520in}}%
\pgfusepath{clip}%
\pgfsetrectcap%
\pgfsetroundjoin%
\pgfsetlinewidth{0.501875pt}%
\definecolor{currentstroke}{rgb}{0.980392,0.811765,0.352941}%
\pgfsetstrokecolor{currentstroke}%
\pgfsetstrokeopacity{0.100000}%
\pgfsetdash{}{0pt}%
\pgfpathmoveto{\pgfqpoint{4.566539in}{4.105980in}}%
\pgfusepath{stroke}%
\end{pgfscope}%
\begin{pgfscope}%
\pgfsetrectcap%
\pgfsetroundjoin%
\pgfsetlinewidth{0.803000pt}%
\definecolor{currentstroke}{rgb}{0.980392,0.811765,0.352941}%
\pgfsetstrokecolor{currentstroke}%
\pgfsetdash{}{0pt}%
\pgfpathmoveto{\pgfqpoint{3.750667in}{1.184123in}}%
\pgfpathlineto{\pgfqpoint{3.686496in}{1.123660in}}%
\pgfusepath{stroke}%
\end{pgfscope}%
\begin{pgfscope}%
\definecolor{textcolor}{rgb}{0.525490,0.694118,0.356863}%
\pgfsetstrokecolor{textcolor}%
\pgfsetfillcolor{textcolor}%
\pgftext[x=3.626149in,y=0.938621in,,top]{\color{textcolor}\sffamily\fontsize{18.000000}{9.600000}\selectfont $\displaystyle 0.6$}%
\end{pgfscope}%
\begin{pgfscope}%
\pgfpathrectangle{\pgfqpoint{0.539299in}{0.078740in}}{\pgfqpoint{7.842520in}{7.842520in}}%
\pgfusepath{clip}%
\pgfsetrectcap%
\pgfsetroundjoin%
\pgfsetlinewidth{0.501875pt}%
\definecolor{currentstroke}{rgb}{0.980392,0.811765,0.352941}%
\pgfsetstrokecolor{currentstroke}%
\pgfsetstrokeopacity{0.100000}%
\pgfsetdash{}{0pt}%
\pgfpathmoveto{\pgfqpoint{4.566539in}{4.105980in}}%
\pgfusepath{stroke}%
\end{pgfscope}%
\begin{pgfscope}%
\pgfsetrectcap%
\pgfsetroundjoin%
\pgfsetlinewidth{0.803000pt}%
\definecolor{currentstroke}{rgb}{0.980392,0.811765,0.352941}%
\pgfsetstrokecolor{currentstroke}%
\pgfsetdash{}{0pt}%
\pgfpathmoveto{\pgfqpoint{4.671624in}{0.883635in}}%
\pgfpathlineto{\pgfqpoint{4.608802in}{0.821838in}}%
\pgfusepath{stroke}%
\end{pgfscope}%
\begin{pgfscope}%
\definecolor{textcolor}{rgb}{0.525490,0.694118,0.356863}%
\pgfsetstrokecolor{textcolor}%
\pgfsetfillcolor{textcolor}%
\pgftext[x=4.548096in,y=0.634665in,,top]{\color{textcolor}\sffamily\fontsize{18.000000}{9.600000}\selectfont $\displaystyle 0.8$}%
\end{pgfscope}%
\begin{pgfscope}%
\pgfsetrectcap%
\pgfsetroundjoin%
\pgfsetlinewidth{0.803000pt}%
\definecolor{currentstroke}{rgb}{0.000000,0.000000,0.000000}%
\pgfsetstrokecolor{currentstroke}%
\pgfsetdash{}{0pt}%
\pgfpathmoveto{\pgfqpoint{7.877114in}{2.975397in}}%
\pgfpathlineto{\pgfqpoint{5.536809in}{0.573668in}}%
\pgfusepath{stroke}%
\end{pgfscope}%
\begin{pgfscope}%
\definecolor{textcolor}{rgb}{0.525490,0.694118,0.356863}%
\pgfsetstrokecolor{textcolor}%
\pgfsetfillcolor{textcolor}%
\pgftext[x=7.133707in,y=1.439527in,,]{\color{textcolor}\sffamily\fontsize{24.000000}{13.200000}\bfseries\selectfont $y$}%
\end{pgfscope}%
\begin{pgfscope}%
\pgfsetbuttcap%
\pgfsetroundjoin%
\pgfsetlinewidth{0.803000pt}%
\definecolor{currentstroke}{rgb}{0.690196,0.690196,0.690196}%
\pgfsetstrokecolor{currentstroke}%
\pgfsetdash{}{0pt}%
\pgfpathmoveto{\pgfqpoint{1.533379in}{5.743717in}}%
\pgfpathlineto{\pgfqpoint{1.666046in}{2.460552in}}%
\pgfpathlineto{\pgfqpoint{6.021576in}{1.071158in}}%
\pgfusepath{stroke}%
\end{pgfscope}%
\begin{pgfscope}%
\pgfsetbuttcap%
\pgfsetroundjoin%
\pgfsetlinewidth{0.803000pt}%
\definecolor{currentstroke}{rgb}{0.690196,0.690196,0.690196}%
\pgfsetstrokecolor{currentstroke}%
\pgfsetdash{}{0pt}%
\pgfpathmoveto{\pgfqpoint{2.108895in}{6.163702in}}%
\pgfpathlineto{\pgfqpoint{2.214498in}{2.920275in}}%
\pgfpathlineto{\pgfqpoint{6.518001in}{1.580612in}}%
\pgfusepath{stroke}%
\end{pgfscope}%
\begin{pgfscope}%
\pgfsetbuttcap%
\pgfsetroundjoin%
\pgfsetlinewidth{0.803000pt}%
\definecolor{currentstroke}{rgb}{0.690196,0.690196,0.690196}%
\pgfsetstrokecolor{currentstroke}%
\pgfsetdash{}{0pt}%
\pgfpathmoveto{\pgfqpoint{2.663627in}{6.568520in}}%
\pgfpathlineto{\pgfqpoint{2.743979in}{3.364097in}}%
\pgfpathlineto{\pgfqpoint{6.996374in}{2.071541in}}%
\pgfusepath{stroke}%
\end{pgfscope}%
\begin{pgfscope}%
\pgfsetbuttcap%
\pgfsetroundjoin%
\pgfsetlinewidth{0.803000pt}%
\definecolor{currentstroke}{rgb}{0.690196,0.690196,0.690196}%
\pgfsetstrokecolor{currentstroke}%
\pgfsetdash{}{0pt}%
\pgfpathmoveto{\pgfqpoint{3.198678in}{6.958976in}}%
\pgfpathlineto{\pgfqpoint{3.255455in}{3.792826in}}%
\pgfpathlineto{\pgfqpoint{7.457662in}{2.544936in}}%
\pgfusepath{stroke}%
\end{pgfscope}%
\begin{pgfscope}%
\pgfpathrectangle{\pgfqpoint{0.539299in}{0.078740in}}{\pgfqpoint{7.842520in}{7.842520in}}%
\pgfusepath{clip}%
\pgfsetrectcap%
\pgfsetroundjoin%
\pgfsetlinewidth{0.501875pt}%
\definecolor{currentstroke}{rgb}{0.980392,0.811765,0.352941}%
\pgfsetstrokecolor{currentstroke}%
\pgfsetstrokeopacity{0.100000}%
\pgfsetdash{}{0pt}%
\pgfpathmoveto{\pgfqpoint{4.566539in}{4.105980in}}%
\pgfusepath{stroke}%
\end{pgfscope}%
\begin{pgfscope}%
\pgfsetrectcap%
\pgfsetroundjoin%
\pgfsetlinewidth{0.803000pt}%
\definecolor{currentstroke}{rgb}{0.980392,0.811765,0.352941}%
\pgfsetstrokecolor{currentstroke}%
\pgfsetdash{}{0pt}%
\pgfpathmoveto{\pgfqpoint{5.984893in}{1.082859in}}%
\pgfpathlineto{\pgfqpoint{6.095033in}{1.047725in}}%
\pgfusepath{stroke}%
\end{pgfscope}%
\begin{pgfscope}%
\definecolor{textcolor}{rgb}{0.525490,0.694118,0.356863}%
\pgfsetstrokecolor{textcolor}%
\pgfsetfillcolor{textcolor}%
\pgftext[x=6.198378in,y=0.887093in,,top]{\color{textcolor}\sffamily\fontsize{18.000000}{9.600000}\selectfont $\displaystyle 0.2$}%
\end{pgfscope}%
\begin{pgfscope}%
\pgfpathrectangle{\pgfqpoint{0.539299in}{0.078740in}}{\pgfqpoint{7.842520in}{7.842520in}}%
\pgfusepath{clip}%
\pgfsetrectcap%
\pgfsetroundjoin%
\pgfsetlinewidth{0.501875pt}%
\definecolor{currentstroke}{rgb}{0.980392,0.811765,0.352941}%
\pgfsetstrokecolor{currentstroke}%
\pgfsetstrokeopacity{0.100000}%
\pgfsetdash{}{0pt}%
\pgfpathmoveto{\pgfqpoint{4.566539in}{4.105980in}}%
\pgfusepath{stroke}%
\end{pgfscope}%
\begin{pgfscope}%
\pgfsetrectcap%
\pgfsetroundjoin%
\pgfsetlinewidth{0.803000pt}%
\definecolor{currentstroke}{rgb}{0.980392,0.811765,0.352941}%
\pgfsetstrokecolor{currentstroke}%
\pgfsetdash{}{0pt}%
\pgfpathmoveto{\pgfqpoint{6.481791in}{1.591885in}}%
\pgfpathlineto{\pgfqpoint{6.590512in}{1.558040in}}%
\pgfusepath{stroke}%
\end{pgfscope}%
\begin{pgfscope}%
\definecolor{textcolor}{rgb}{0.525490,0.694118,0.356863}%
\pgfsetstrokecolor{textcolor}%
\pgfsetfillcolor{textcolor}%
\pgftext[x=6.691664in,y=1.400097in,,top]{\color{textcolor}\sffamily\fontsize{18.000000}{9.600000}\selectfont $\displaystyle 0.4$}%
\end{pgfscope}%
\begin{pgfscope}%
\pgfpathrectangle{\pgfqpoint{0.539299in}{0.078740in}}{\pgfqpoint{7.842520in}{7.842520in}}%
\pgfusepath{clip}%
\pgfsetrectcap%
\pgfsetroundjoin%
\pgfsetlinewidth{0.501875pt}%
\definecolor{currentstroke}{rgb}{0.980392,0.811765,0.352941}%
\pgfsetstrokecolor{currentstroke}%
\pgfsetstrokeopacity{0.100000}%
\pgfsetdash{}{0pt}%
\pgfpathmoveto{\pgfqpoint{4.566539in}{4.105980in}}%
\pgfusepath{stroke}%
\end{pgfscope}%
\begin{pgfscope}%
\pgfsetrectcap%
\pgfsetroundjoin%
\pgfsetlinewidth{0.803000pt}%
\definecolor{currentstroke}{rgb}{0.980392,0.811765,0.352941}%
\pgfsetstrokecolor{currentstroke}%
\pgfsetdash{}{0pt}%
\pgfpathmoveto{\pgfqpoint{6.960625in}{2.082407in}}%
\pgfpathlineto{\pgfqpoint{7.067958in}{2.049782in}}%
\pgfusepath{stroke}%
\end{pgfscope}%
\begin{pgfscope}%
\definecolor{textcolor}{rgb}{0.525490,0.694118,0.356863}%
\pgfsetstrokecolor{textcolor}%
\pgfsetfillcolor{textcolor}%
\pgftext[x=7.167007in,y=1.894440in,,top]{\color{textcolor}\sffamily\fontsize{18.000000}{9.600000}\selectfont $\displaystyle 0.6$}%
\end{pgfscope}%
\begin{pgfscope}%
\pgfpathrectangle{\pgfqpoint{0.539299in}{0.078740in}}{\pgfqpoint{7.842520in}{7.842520in}}%
\pgfusepath{clip}%
\pgfsetrectcap%
\pgfsetroundjoin%
\pgfsetlinewidth{0.501875pt}%
\definecolor{currentstroke}{rgb}{0.980392,0.811765,0.352941}%
\pgfsetstrokecolor{currentstroke}%
\pgfsetstrokeopacity{0.100000}%
\pgfsetdash{}{0pt}%
\pgfpathmoveto{\pgfqpoint{4.566539in}{4.105980in}}%
\pgfusepath{stroke}%
\end{pgfscope}%
\begin{pgfscope}%
\pgfsetrectcap%
\pgfsetroundjoin%
\pgfsetlinewidth{0.803000pt}%
\definecolor{currentstroke}{rgb}{0.980392,0.811765,0.352941}%
\pgfsetstrokecolor{currentstroke}%
\pgfsetdash{}{0pt}%
\pgfpathmoveto{\pgfqpoint{7.422366in}{2.555417in}}%
\pgfpathlineto{\pgfqpoint{7.528339in}{2.523947in}}%
\pgfusepath{stroke}%
\end{pgfscope}%
\begin{pgfscope}%
\definecolor{textcolor}{rgb}{0.525490,0.694118,0.356863}%
\pgfsetstrokecolor{textcolor}%
\pgfsetfillcolor{textcolor}%
\pgftext[x=7.625368in,y=2.371123in,,top]{\color{textcolor}\sffamily\fontsize{18.000000}{9.600000}\selectfont $\displaystyle 0.8$}%
\end{pgfscope}%
\begin{pgfscope}%
\pgfsetrectcap%
\pgfsetroundjoin%
\pgfsetlinewidth{0.803000pt}%
\definecolor{currentstroke}{rgb}{0.000000,0.000000,0.000000}%
\pgfsetstrokecolor{currentstroke}%
\pgfsetdash{}{0pt}%
\pgfpathmoveto{\pgfqpoint{7.877114in}{2.975397in}}%
\pgfpathlineto{\pgfqpoint{8.025420in}{6.214014in}}%
\pgfusepath{stroke}%
\end{pgfscope}%
\begin{pgfscope}%
\definecolor{textcolor}{rgb}{0.525490,0.694118,0.356863}%
\pgfsetstrokecolor{textcolor}%
\pgfsetfillcolor{textcolor}%
\pgftext[x=8.812392in,y=4.634624in,,]{\color{textcolor}\sffamily\fontsize{24.000000}{13.200000}\bfseries\selectfont $U$}%
\end{pgfscope}%
\begin{pgfscope}%
\pgfsetbuttcap%
\pgfsetroundjoin%
\pgfsetlinewidth{0.803000pt}%
\definecolor{currentstroke}{rgb}{0.690196,0.690196,0.690196}%
\pgfsetstrokecolor{currentstroke}%
\pgfsetdash{}{0pt}%
\pgfpathmoveto{\pgfqpoint{7.903878in}{3.559855in}}%
\pgfpathlineto{\pgfqpoint{3.714811in}{4.749298in}}%
\pgfpathlineto{\pgfqpoint{1.102640in}{2.610996in}}%
\pgfusepath{stroke}%
\end{pgfscope}%
\begin{pgfscope}%
\pgfsetbuttcap%
\pgfsetroundjoin%
\pgfsetlinewidth{0.803000pt}%
\definecolor{currentstroke}{rgb}{0.690196,0.690196,0.690196}%
\pgfsetstrokecolor{currentstroke}%
\pgfsetdash{}{0pt}%
\pgfpathmoveto{\pgfqpoint{7.928165in}{4.090216in}}%
\pgfpathlineto{\pgfqpoint{3.708910in}{5.262514in}}%
\pgfpathlineto{\pgfqpoint{1.076469in}{3.154452in}}%
\pgfusepath{stroke}%
\end{pgfscope}%
\begin{pgfscope}%
\pgfsetbuttcap%
\pgfsetroundjoin%
\pgfsetlinewidth{0.803000pt}%
\definecolor{currentstroke}{rgb}{0.690196,0.690196,0.690196}%
\pgfsetstrokecolor{currentstroke}%
\pgfsetdash{}{0pt}%
\pgfpathmoveto{\pgfqpoint{7.952808in}{4.628352in}}%
\pgfpathlineto{\pgfqpoint{3.702926in}{5.782891in}}%
\pgfpathlineto{\pgfqpoint{1.049899in}{3.706182in}}%
\pgfusepath{stroke}%
\end{pgfscope}%
\begin{pgfscope}%
\pgfsetbuttcap%
\pgfsetroundjoin%
\pgfsetlinewidth{0.803000pt}%
\definecolor{currentstroke}{rgb}{0.690196,0.690196,0.690196}%
\pgfsetstrokecolor{currentstroke}%
\pgfsetdash{}{0pt}%
\pgfpathmoveto{\pgfqpoint{7.977815in}{5.174436in}}%
\pgfpathlineto{\pgfqpoint{3.696858in}{6.310580in}}%
\pgfpathlineto{\pgfqpoint{1.022922in}{4.266378in}}%
\pgfusepath{stroke}%
\end{pgfscope}%
\begin{pgfscope}%
\pgfsetbuttcap%
\pgfsetroundjoin%
\pgfsetlinewidth{0.803000pt}%
\definecolor{currentstroke}{rgb}{0.690196,0.690196,0.690196}%
\pgfsetstrokecolor{currentstroke}%
\pgfsetdash{}{0pt}%
\pgfpathmoveto{\pgfqpoint{8.003194in}{5.728645in}}%
\pgfpathlineto{\pgfqpoint{3.690704in}{6.845738in}}%
\pgfpathlineto{\pgfqpoint{0.995528in}{4.835234in}}%
\pgfusepath{stroke}%
\end{pgfscope}%
\begin{pgfscope}%
\pgfpathrectangle{\pgfqpoint{0.539299in}{0.078740in}}{\pgfqpoint{7.842520in}{7.842520in}}%
\pgfusepath{clip}%
\pgfsetrectcap%
\pgfsetroundjoin%
\pgfsetlinewidth{0.501875pt}%
\definecolor{currentstroke}{rgb}{0.980392,0.811765,0.352941}%
\pgfsetstrokecolor{currentstroke}%
\pgfsetstrokeopacity{0.100000}%
\pgfsetdash{}{0pt}%
\pgfpathmoveto{\pgfqpoint{4.566539in}{4.105980in}}%
\pgfusepath{stroke}%
\end{pgfscope}%
\begin{pgfscope}%
\pgfsetrectcap%
\pgfsetroundjoin%
\pgfsetlinewidth{0.803000pt}%
\definecolor{currentstroke}{rgb}{0.980392,0.811765,0.352941}%
\pgfsetstrokecolor{currentstroke}%
\pgfsetdash{}{0pt}%
\pgfpathmoveto{\pgfqpoint{7.868707in}{3.569842in}}%
\pgfpathlineto{\pgfqpoint{7.974305in}{3.539859in}}%
\pgfusepath{stroke}%
\end{pgfscope}%
\begin{pgfscope}%
\definecolor{textcolor}{rgb}{0.525490,0.694118,0.356863}%
\pgfsetstrokecolor{textcolor}%
\pgfsetfillcolor{textcolor}%
\pgftext[x=8.359745in,y=3.595552in,,top]{\color{textcolor}\sffamily\fontsize{18.000000}{9.600000}\selectfont $\displaystyle 0.005$}%
\end{pgfscope}%
\begin{pgfscope}%
\pgfpathrectangle{\pgfqpoint{0.539299in}{0.078740in}}{\pgfqpoint{7.842520in}{7.842520in}}%
\pgfusepath{clip}%
\pgfsetrectcap%
\pgfsetroundjoin%
\pgfsetlinewidth{0.501875pt}%
\definecolor{currentstroke}{rgb}{0.980392,0.811765,0.352941}%
\pgfsetstrokecolor{currentstroke}%
\pgfsetstrokeopacity{0.100000}%
\pgfsetdash{}{0pt}%
\pgfpathmoveto{\pgfqpoint{4.566539in}{4.105980in}}%
\pgfusepath{stroke}%
\end{pgfscope}%
\begin{pgfscope}%
\pgfsetrectcap%
\pgfsetroundjoin%
\pgfsetlinewidth{0.803000pt}%
\definecolor{currentstroke}{rgb}{0.980392,0.811765,0.352941}%
\pgfsetstrokecolor{currentstroke}%
\pgfsetdash{}{0pt}%
\pgfpathmoveto{\pgfqpoint{7.892728in}{4.100062in}}%
\pgfpathlineto{\pgfqpoint{7.999124in}{4.070500in}}%
\pgfusepath{stroke}%
\end{pgfscope}%
\begin{pgfscope}%
\definecolor{textcolor}{rgb}{0.525490,0.694118,0.356863}%
\pgfsetstrokecolor{textcolor}%
\pgfsetfillcolor{textcolor}%
\pgftext[x=8.385857in,y=4.125410in,,top]{\color{textcolor}\sffamily\fontsize{18.000000}{9.600000}\selectfont $\displaystyle 0.010$}%
\end{pgfscope}%
\begin{pgfscope}%
\pgfpathrectangle{\pgfqpoint{0.539299in}{0.078740in}}{\pgfqpoint{7.842520in}{7.842520in}}%
\pgfusepath{clip}%
\pgfsetrectcap%
\pgfsetroundjoin%
\pgfsetlinewidth{0.501875pt}%
\definecolor{currentstroke}{rgb}{0.980392,0.811765,0.352941}%
\pgfsetstrokecolor{currentstroke}%
\pgfsetstrokeopacity{0.100000}%
\pgfsetdash{}{0pt}%
\pgfpathmoveto{\pgfqpoint{4.566539in}{4.105980in}}%
\pgfusepath{stroke}%
\end{pgfscope}%
\begin{pgfscope}%
\pgfsetrectcap%
\pgfsetroundjoin%
\pgfsetlinewidth{0.803000pt}%
\definecolor{currentstroke}{rgb}{0.980392,0.811765,0.352941}%
\pgfsetstrokecolor{currentstroke}%
\pgfsetdash{}{0pt}%
\pgfpathmoveto{\pgfqpoint{7.917102in}{4.638052in}}%
\pgfpathlineto{\pgfqpoint{8.024308in}{4.608928in}}%
\pgfusepath{stroke}%
\end{pgfscope}%
\begin{pgfscope}%
\definecolor{textcolor}{rgb}{0.525490,0.694118,0.356863}%
\pgfsetstrokecolor{textcolor}%
\pgfsetfillcolor{textcolor}%
\pgftext[x=8.412350in,y=4.663024in,,top]{\color{textcolor}\sffamily\fontsize{18.000000}{9.600000}\selectfont $\displaystyle 0.015$}%
\end{pgfscope}%
\begin{pgfscope}%
\pgfpathrectangle{\pgfqpoint{0.539299in}{0.078740in}}{\pgfqpoint{7.842520in}{7.842520in}}%
\pgfusepath{clip}%
\pgfsetrectcap%
\pgfsetroundjoin%
\pgfsetlinewidth{0.501875pt}%
\definecolor{currentstroke}{rgb}{0.980392,0.811765,0.352941}%
\pgfsetstrokecolor{currentstroke}%
\pgfsetstrokeopacity{0.100000}%
\pgfsetdash{}{0pt}%
\pgfpathmoveto{\pgfqpoint{4.566539in}{4.105980in}}%
\pgfusepath{stroke}%
\end{pgfscope}%
\begin{pgfscope}%
\pgfsetrectcap%
\pgfsetroundjoin%
\pgfsetlinewidth{0.803000pt}%
\definecolor{currentstroke}{rgb}{0.980392,0.811765,0.352941}%
\pgfsetstrokecolor{currentstroke}%
\pgfsetdash{}{0pt}%
\pgfpathmoveto{\pgfqpoint{7.941835in}{5.183985in}}%
\pgfpathlineto{\pgfqpoint{8.049863in}{5.155315in}}%
\pgfusepath{stroke}%
\end{pgfscope}%
\begin{pgfscope}%
\definecolor{textcolor}{rgb}{0.525490,0.694118,0.356863}%
\pgfsetstrokecolor{textcolor}%
\pgfsetfillcolor{textcolor}%
\pgftext[x=8.439234in,y=5.208568in,,top]{\color{textcolor}\sffamily\fontsize{18.000000}{9.600000}\selectfont $\displaystyle 0.020$}%
\end{pgfscope}%
\begin{pgfscope}%
\pgfpathrectangle{\pgfqpoint{0.539299in}{0.078740in}}{\pgfqpoint{7.842520in}{7.842520in}}%
\pgfusepath{clip}%
\pgfsetrectcap%
\pgfsetroundjoin%
\pgfsetlinewidth{0.501875pt}%
\definecolor{currentstroke}{rgb}{0.980392,0.811765,0.352941}%
\pgfsetstrokecolor{currentstroke}%
\pgfsetstrokeopacity{0.100000}%
\pgfsetdash{}{0pt}%
\pgfpathmoveto{\pgfqpoint{4.566539in}{4.105980in}}%
\pgfusepath{stroke}%
\end{pgfscope}%
\begin{pgfscope}%
\pgfsetrectcap%
\pgfsetroundjoin%
\pgfsetlinewidth{0.803000pt}%
\definecolor{currentstroke}{rgb}{0.980392,0.811765,0.352941}%
\pgfsetstrokecolor{currentstroke}%
\pgfsetdash{}{0pt}%
\pgfpathmoveto{\pgfqpoint{7.966936in}{5.738038in}}%
\pgfpathlineto{\pgfqpoint{8.075799in}{5.709838in}}%
\pgfusepath{stroke}%
\end{pgfscope}%
\begin{pgfscope}%
\definecolor{textcolor}{rgb}{0.525490,0.694118,0.356863}%
\pgfsetstrokecolor{textcolor}%
\pgfsetfillcolor{textcolor}%
\pgftext[x=8.466518in,y=5.762217in,,top]{\color{textcolor}\sffamily\fontsize{18.000000}{9.600000}\selectfont $\displaystyle 0.025$}%
\end{pgfscope}%
\begin{pgfscope}%
\pgfpathrectangle{\pgfqpoint{0.539299in}{0.078740in}}{\pgfqpoint{7.842520in}{7.842520in}}%
\pgfusepath{clip}%
\pgfsetbuttcap%
\pgfsetroundjoin%
\definecolor{currentfill}{rgb}{0.283091,0.110553,0.431554}%
\pgfsetfillcolor{currentfill}%
\pgfsetlinewidth{0.000000pt}%
\definecolor{currentstroke}{rgb}{0.267004,0.004874,0.329415}%
\pgfsetstrokecolor{currentstroke}%
\pgfsetdash{}{0pt}%
\pgfpathmoveto{\pgfqpoint{3.799922in}{4.421780in}}%
\pgfpathlineto{\pgfqpoint{3.877896in}{4.341545in}}%
\pgfpathlineto{\pgfqpoint{3.750979in}{4.305549in}}%
\pgfpathclose%
\pgfusepath{fill}%
\end{pgfscope}%
\begin{pgfscope}%
\pgfpathrectangle{\pgfqpoint{0.539299in}{0.078740in}}{\pgfqpoint{7.842520in}{7.842520in}}%
\pgfusepath{clip}%
\pgfsetbuttcap%
\pgfsetroundjoin%
\definecolor{currentfill}{rgb}{0.283091,0.110553,0.431554}%
\pgfsetfillcolor{currentfill}%
\pgfsetlinewidth{0.000000pt}%
\definecolor{currentstroke}{rgb}{0.268510,0.009605,0.335427}%
\pgfsetstrokecolor{currentstroke}%
\pgfsetdash{}{0pt}%
\pgfpathmoveto{\pgfqpoint{3.750979in}{4.305549in}}%
\pgfpathlineto{\pgfqpoint{3.673363in}{4.323977in}}%
\pgfpathlineto{\pgfqpoint{3.799922in}{4.421780in}}%
\pgfpathclose%
\pgfusepath{fill}%
\end{pgfscope}%
\begin{pgfscope}%
\pgfpathrectangle{\pgfqpoint{0.539299in}{0.078740in}}{\pgfqpoint{7.842520in}{7.842520in}}%
\pgfusepath{clip}%
\pgfsetbuttcap%
\pgfsetroundjoin%
\definecolor{currentfill}{rgb}{0.283187,0.125848,0.444960}%
\pgfsetfillcolor{currentfill}%
\pgfsetlinewidth{0.000000pt}%
\definecolor{currentstroke}{rgb}{0.269944,0.014625,0.341379}%
\pgfsetstrokecolor{currentstroke}%
\pgfsetdash{}{0pt}%
\pgfpathmoveto{\pgfqpoint{4.005716in}{4.343136in}}%
\pgfpathlineto{\pgfqpoint{3.877896in}{4.341545in}}%
\pgfpathlineto{\pgfqpoint{3.799922in}{4.421780in}}%
\pgfpathclose%
\pgfusepath{fill}%
\end{pgfscope}%
\begin{pgfscope}%
\pgfpathrectangle{\pgfqpoint{0.539299in}{0.078740in}}{\pgfqpoint{7.842520in}{7.842520in}}%
\pgfusepath{clip}%
\pgfsetbuttcap%
\pgfsetroundjoin%
\definecolor{currentfill}{rgb}{0.280868,0.160771,0.472899}%
\pgfsetfillcolor{currentfill}%
\pgfsetlinewidth{0.000000pt}%
\definecolor{currentstroke}{rgb}{0.271305,0.019942,0.347269}%
\pgfsetstrokecolor{currentstroke}%
\pgfsetdash{}{0pt}%
\pgfpathmoveto{\pgfqpoint{3.927826in}{4.463428in}}%
\pgfpathlineto{\pgfqpoint{4.005716in}{4.343136in}}%
\pgfpathlineto{\pgfqpoint{3.799922in}{4.421780in}}%
\pgfpathclose%
\pgfusepath{fill}%
\end{pgfscope}%
\begin{pgfscope}%
\pgfpathrectangle{\pgfqpoint{0.539299in}{0.078740in}}{\pgfqpoint{7.842520in}{7.842520in}}%
\pgfusepath{clip}%
\pgfsetbuttcap%
\pgfsetroundjoin%
\definecolor{currentfill}{rgb}{0.280868,0.160771,0.472899}%
\pgfsetfillcolor{currentfill}%
\pgfsetlinewidth{0.000000pt}%
\definecolor{currentstroke}{rgb}{0.272594,0.025563,0.353093}%
\pgfsetstrokecolor{currentstroke}%
\pgfsetdash{}{0pt}%
\pgfpathmoveto{\pgfqpoint{3.721620in}{4.468366in}}%
\pgfpathlineto{\pgfqpoint{3.799922in}{4.421780in}}%
\pgfpathlineto{\pgfqpoint{3.673363in}{4.323977in}}%
\pgfpathclose%
\pgfusepath{fill}%
\end{pgfscope}%
\begin{pgfscope}%
\pgfpathrectangle{\pgfqpoint{0.539299in}{0.078740in}}{\pgfqpoint{7.842520in}{7.842520in}}%
\pgfusepath{clip}%
\pgfsetbuttcap%
\pgfsetroundjoin%
\definecolor{currentfill}{rgb}{0.281887,0.150881,0.465405}%
\pgfsetfillcolor{currentfill}%
\pgfsetlinewidth{0.000000pt}%
\definecolor{currentstroke}{rgb}{0.273809,0.031497,0.358853}%
\pgfsetstrokecolor{currentstroke}%
\pgfsetdash{}{0pt}%
\pgfpathmoveto{\pgfqpoint{3.721620in}{4.468366in}}%
\pgfpathlineto{\pgfqpoint{3.673363in}{4.323977in}}%
\pgfpathlineto{\pgfqpoint{3.595462in}{4.319772in}}%
\pgfpathclose%
\pgfusepath{fill}%
\end{pgfscope}%
\begin{pgfscope}%
\pgfpathrectangle{\pgfqpoint{0.539299in}{0.078740in}}{\pgfqpoint{7.842520in}{7.842520in}}%
\pgfusepath{clip}%
\pgfsetbuttcap%
\pgfsetroundjoin%
\definecolor{currentfill}{rgb}{0.281887,0.150881,0.465405}%
\pgfsetfillcolor{currentfill}%
\pgfsetlinewidth{0.000000pt}%
\definecolor{currentstroke}{rgb}{0.274952,0.037752,0.364543}%
\pgfsetstrokecolor{currentstroke}%
\pgfsetdash{}{0pt}%
\pgfpathmoveto{\pgfqpoint{4.056700in}{4.461449in}}%
\pgfpathlineto{\pgfqpoint{4.134205in}{4.319900in}}%
\pgfpathlineto{\pgfqpoint{4.005716in}{4.343136in}}%
\pgfpathclose%
\pgfusepath{fill}%
\end{pgfscope}%
\begin{pgfscope}%
\pgfpathrectangle{\pgfqpoint{0.539299in}{0.078740in}}{\pgfqpoint{7.842520in}{7.842520in}}%
\pgfusepath{clip}%
\pgfsetbuttcap%
\pgfsetroundjoin%
\definecolor{currentfill}{rgb}{0.281887,0.150881,0.465405}%
\pgfsetfillcolor{currentfill}%
\pgfsetlinewidth{0.000000pt}%
\definecolor{currentstroke}{rgb}{0.276022,0.044167,0.370164}%
\pgfsetstrokecolor{currentstroke}%
\pgfsetdash{}{0pt}%
\pgfpathmoveto{\pgfqpoint{4.263211in}{4.278418in}}%
\pgfpathlineto{\pgfqpoint{4.134205in}{4.319900in}}%
\pgfpathlineto{\pgfqpoint{4.056700in}{4.461449in}}%
\pgfpathclose%
\pgfusepath{fill}%
\end{pgfscope}%
\begin{pgfscope}%
\pgfpathrectangle{\pgfqpoint{0.539299in}{0.078740in}}{\pgfqpoint{7.842520in}{7.842520in}}%
\pgfusepath{clip}%
\pgfsetbuttcap%
\pgfsetroundjoin%
\definecolor{currentfill}{rgb}{0.278826,0.175490,0.483397}%
\pgfsetfillcolor{currentfill}%
\pgfsetlinewidth{0.000000pt}%
\definecolor{currentstroke}{rgb}{0.277018,0.050344,0.375715}%
\pgfsetstrokecolor{currentstroke}%
\pgfsetdash{}{0pt}%
\pgfpathmoveto{\pgfqpoint{4.056700in}{4.461449in}}%
\pgfpathlineto{\pgfqpoint{4.005716in}{4.343136in}}%
\pgfpathlineto{\pgfqpoint{3.927826in}{4.463428in}}%
\pgfpathclose%
\pgfusepath{fill}%
\end{pgfscope}%
\begin{pgfscope}%
\pgfpathrectangle{\pgfqpoint{0.539299in}{0.078740in}}{\pgfqpoint{7.842520in}{7.842520in}}%
\pgfusepath{clip}%
\pgfsetbuttcap%
\pgfsetroundjoin%
\definecolor{currentfill}{rgb}{0.276194,0.190074,0.493001}%
\pgfsetfillcolor{currentfill}%
\pgfsetlinewidth{0.000000pt}%
\definecolor{currentstroke}{rgb}{0.277941,0.056324,0.381191}%
\pgfsetstrokecolor{currentstroke}%
\pgfsetdash{}{0pt}%
\pgfpathmoveto{\pgfqpoint{3.927826in}{4.463428in}}%
\pgfpathlineto{\pgfqpoint{3.799922in}{4.421780in}}%
\pgfpathlineto{\pgfqpoint{3.721620in}{4.468366in}}%
\pgfpathclose%
\pgfusepath{fill}%
\end{pgfscope}%
\begin{pgfscope}%
\pgfpathrectangle{\pgfqpoint{0.539299in}{0.078740in}}{\pgfqpoint{7.842520in}{7.842520in}}%
\pgfusepath{clip}%
\pgfsetbuttcap%
\pgfsetroundjoin%
\definecolor{currentfill}{rgb}{0.280255,0.165693,0.476498}%
\pgfsetfillcolor{currentfill}%
\pgfsetlinewidth{0.000000pt}%
\definecolor{currentstroke}{rgb}{0.278791,0.062145,0.386592}%
\pgfsetstrokecolor{currentstroke}%
\pgfsetdash{}{0pt}%
\pgfpathmoveto{\pgfqpoint{3.595462in}{4.319772in}}%
\pgfpathlineto{\pgfqpoint{3.517208in}{4.302391in}}%
\pgfpathlineto{\pgfqpoint{3.721620in}{4.468366in}}%
\pgfpathclose%
\pgfusepath{fill}%
\end{pgfscope}%
\begin{pgfscope}%
\pgfpathrectangle{\pgfqpoint{0.539299in}{0.078740in}}{\pgfqpoint{7.842520in}{7.842520in}}%
\pgfusepath{clip}%
\pgfsetbuttcap%
\pgfsetroundjoin%
\definecolor{currentfill}{rgb}{0.281887,0.150881,0.465405}%
\pgfsetfillcolor{currentfill}%
\pgfsetlinewidth{0.000000pt}%
\definecolor{currentstroke}{rgb}{0.279566,0.067836,0.391917}%
\pgfsetstrokecolor{currentstroke}%
\pgfsetdash{}{0pt}%
\pgfpathmoveto{\pgfqpoint{4.186265in}{4.426031in}}%
\pgfpathlineto{\pgfqpoint{4.392627in}{4.224057in}}%
\pgfpathlineto{\pgfqpoint{4.263211in}{4.278418in}}%
\pgfpathclose%
\pgfusepath{fill}%
\end{pgfscope}%
\begin{pgfscope}%
\pgfpathrectangle{\pgfqpoint{0.539299in}{0.078740in}}{\pgfqpoint{7.842520in}{7.842520in}}%
\pgfusepath{clip}%
\pgfsetbuttcap%
\pgfsetroundjoin%
\definecolor{currentfill}{rgb}{0.278012,0.180367,0.486697}%
\pgfsetfillcolor{currentfill}%
\pgfsetlinewidth{0.000000pt}%
\definecolor{currentstroke}{rgb}{0.280267,0.073417,0.397163}%
\pgfsetstrokecolor{currentstroke}%
\pgfsetdash{}{0pt}%
\pgfpathmoveto{\pgfqpoint{4.056700in}{4.461449in}}%
\pgfpathlineto{\pgfqpoint{4.186265in}{4.426031in}}%
\pgfpathlineto{\pgfqpoint{4.263211in}{4.278418in}}%
\pgfpathclose%
\pgfusepath{fill}%
\end{pgfscope}%
\begin{pgfscope}%
\pgfpathrectangle{\pgfqpoint{0.539299in}{0.078740in}}{\pgfqpoint{7.842520in}{7.842520in}}%
\pgfusepath{clip}%
\pgfsetbuttcap%
\pgfsetroundjoin%
\definecolor{currentfill}{rgb}{0.282884,0.135920,0.453427}%
\pgfsetfillcolor{currentfill}%
\pgfsetlinewidth{0.000000pt}%
\definecolor{currentstroke}{rgb}{0.280894,0.078907,0.402329}%
\pgfsetstrokecolor{currentstroke}%
\pgfsetdash{}{0pt}%
\pgfpathmoveto{\pgfqpoint{4.522384in}{4.161514in}}%
\pgfpathlineto{\pgfqpoint{4.392627in}{4.224057in}}%
\pgfpathlineto{\pgfqpoint{4.446717in}{4.289781in}}%
\pgfpathclose%
\pgfusepath{fill}%
\end{pgfscope}%
\begin{pgfscope}%
\pgfpathrectangle{\pgfqpoint{0.539299in}{0.078740in}}{\pgfqpoint{7.842520in}{7.842520in}}%
\pgfusepath{clip}%
\pgfsetbuttcap%
\pgfsetroundjoin%
\definecolor{currentfill}{rgb}{0.267968,0.223549,0.512008}%
\pgfsetfillcolor{currentfill}%
\pgfsetlinewidth{0.000000pt}%
\definecolor{currentstroke}{rgb}{0.281446,0.084320,0.407414}%
\pgfsetstrokecolor{currentstroke}%
\pgfsetdash{}{0pt}%
\pgfpathmoveto{\pgfqpoint{3.721620in}{4.468366in}}%
\pgfpathlineto{\pgfqpoint{3.849491in}{4.547694in}}%
\pgfpathlineto{\pgfqpoint{3.927826in}{4.463428in}}%
\pgfpathclose%
\pgfusepath{fill}%
\end{pgfscope}%
\begin{pgfscope}%
\pgfpathrectangle{\pgfqpoint{0.539299in}{0.078740in}}{\pgfqpoint{7.842520in}{7.842520in}}%
\pgfusepath{clip}%
\pgfsetbuttcap%
\pgfsetroundjoin%
\definecolor{currentfill}{rgb}{0.267968,0.223549,0.512008}%
\pgfsetfillcolor{currentfill}%
\pgfsetlinewidth{0.000000pt}%
\definecolor{currentstroke}{rgb}{0.281924,0.089666,0.412415}%
\pgfsetstrokecolor{currentstroke}%
\pgfsetdash{}{0pt}%
\pgfpathmoveto{\pgfqpoint{3.927826in}{4.463428in}}%
\pgfpathlineto{\pgfqpoint{3.849491in}{4.547694in}}%
\pgfpathlineto{\pgfqpoint{4.056700in}{4.461449in}}%
\pgfpathclose%
\pgfusepath{fill}%
\end{pgfscope}%
\begin{pgfscope}%
\pgfpathrectangle{\pgfqpoint{0.539299in}{0.078740in}}{\pgfqpoint{7.842520in}{7.842520in}}%
\pgfusepath{clip}%
\pgfsetbuttcap%
\pgfsetroundjoin%
\definecolor{currentfill}{rgb}{0.278826,0.175490,0.483397}%
\pgfsetfillcolor{currentfill}%
\pgfsetlinewidth{0.000000pt}%
\definecolor{currentstroke}{rgb}{0.282327,0.094955,0.417331}%
\pgfsetstrokecolor{currentstroke}%
\pgfsetdash{}{0pt}%
\pgfpathmoveto{\pgfqpoint{4.316318in}{4.366115in}}%
\pgfpathlineto{\pgfqpoint{4.392627in}{4.224057in}}%
\pgfpathlineto{\pgfqpoint{4.186265in}{4.426031in}}%
\pgfpathclose%
\pgfusepath{fill}%
\end{pgfscope}%
\begin{pgfscope}%
\pgfpathrectangle{\pgfqpoint{0.539299in}{0.078740in}}{\pgfqpoint{7.842520in}{7.842520in}}%
\pgfusepath{clip}%
\pgfsetbuttcap%
\pgfsetroundjoin%
\definecolor{currentfill}{rgb}{0.273006,0.204520,0.501721}%
\pgfsetfillcolor{currentfill}%
\pgfsetlinewidth{0.000000pt}%
\definecolor{currentstroke}{rgb}{0.282656,0.100196,0.422160}%
\pgfsetstrokecolor{currentstroke}%
\pgfsetdash{}{0pt}%
\pgfpathmoveto{\pgfqpoint{3.517208in}{4.302391in}}%
\pgfpathlineto{\pgfqpoint{3.642937in}{4.493560in}}%
\pgfpathlineto{\pgfqpoint{3.721620in}{4.468366in}}%
\pgfpathclose%
\pgfusepath{fill}%
\end{pgfscope}%
\begin{pgfscope}%
\pgfpathrectangle{\pgfqpoint{0.539299in}{0.078740in}}{\pgfqpoint{7.842520in}{7.842520in}}%
\pgfusepath{clip}%
\pgfsetbuttcap%
\pgfsetroundjoin%
\definecolor{currentfill}{rgb}{0.283187,0.125848,0.444960}%
\pgfsetfillcolor{currentfill}%
\pgfsetlinewidth{0.000000pt}%
\definecolor{currentstroke}{rgb}{0.282910,0.105393,0.426902}%
\pgfsetstrokecolor{currentstroke}%
\pgfsetdash{}{0pt}%
\pgfpathmoveto{\pgfqpoint{4.652446in}{4.094919in}}%
\pgfpathlineto{\pgfqpoint{4.522384in}{4.161514in}}%
\pgfpathlineto{\pgfqpoint{4.446717in}{4.289781in}}%
\pgfpathclose%
\pgfusepath{fill}%
\end{pgfscope}%
\begin{pgfscope}%
\pgfpathrectangle{\pgfqpoint{0.539299in}{0.078740in}}{\pgfqpoint{7.842520in}{7.842520in}}%
\pgfusepath{clip}%
\pgfsetbuttcap%
\pgfsetroundjoin%
\definecolor{currentfill}{rgb}{0.279574,0.170599,0.479997}%
\pgfsetfillcolor{currentfill}%
\pgfsetlinewidth{0.000000pt}%
\definecolor{currentstroke}{rgb}{0.283091,0.110553,0.431554}%
\pgfsetstrokecolor{currentstroke}%
\pgfsetdash{}{0pt}%
\pgfpathmoveto{\pgfqpoint{4.446717in}{4.289781in}}%
\pgfpathlineto{\pgfqpoint{4.392627in}{4.224057in}}%
\pgfpathlineto{\pgfqpoint{4.316318in}{4.366115in}}%
\pgfpathclose%
\pgfusepath{fill}%
\end{pgfscope}%
\begin{pgfscope}%
\pgfpathrectangle{\pgfqpoint{0.539299in}{0.078740in}}{\pgfqpoint{7.842520in}{7.842520in}}%
\pgfusepath{clip}%
\pgfsetbuttcap%
\pgfsetroundjoin%
\definecolor{currentfill}{rgb}{0.276194,0.190074,0.493001}%
\pgfsetfillcolor{currentfill}%
\pgfsetlinewidth{0.000000pt}%
\definecolor{currentstroke}{rgb}{0.283197,0.115680,0.436115}%
\pgfsetstrokecolor{currentstroke}%
\pgfsetdash{}{0pt}%
\pgfpathmoveto{\pgfqpoint{3.642937in}{4.493560in}}%
\pgfpathlineto{\pgfqpoint{3.517208in}{4.302391in}}%
\pgfpathlineto{\pgfqpoint{3.438560in}{4.276751in}}%
\pgfpathclose%
\pgfusepath{fill}%
\end{pgfscope}%
\begin{pgfscope}%
\pgfpathrectangle{\pgfqpoint{0.539299in}{0.078740in}}{\pgfqpoint{7.842520in}{7.842520in}}%
\pgfusepath{clip}%
\pgfsetbuttcap%
\pgfsetroundjoin%
\definecolor{currentfill}{rgb}{0.282910,0.105393,0.426902}%
\pgfsetfillcolor{currentfill}%
\pgfsetlinewidth{0.000000pt}%
\definecolor{currentstroke}{rgb}{0.283229,0.120777,0.440584}%
\pgfsetstrokecolor{currentstroke}%
\pgfsetdash{}{0pt}%
\pgfpathmoveto{\pgfqpoint{4.708270in}{4.115796in}}%
\pgfpathlineto{\pgfqpoint{4.782803in}{4.027779in}}%
\pgfpathlineto{\pgfqpoint{4.652446in}{4.094919in}}%
\pgfpathclose%
\pgfusepath{fill}%
\end{pgfscope}%
\begin{pgfscope}%
\pgfpathrectangle{\pgfqpoint{0.539299in}{0.078740in}}{\pgfqpoint{7.842520in}{7.842520in}}%
\pgfusepath{clip}%
\pgfsetbuttcap%
\pgfsetroundjoin%
\definecolor{currentfill}{rgb}{0.258965,0.251537,0.524736}%
\pgfsetfillcolor{currentfill}%
\pgfsetlinewidth{0.000000pt}%
\definecolor{currentstroke}{rgb}{0.283187,0.125848,0.444960}%
\pgfsetstrokecolor{currentstroke}%
\pgfsetdash{}{0pt}%
\pgfpathmoveto{\pgfqpoint{4.056700in}{4.461449in}}%
\pgfpathlineto{\pgfqpoint{3.849491in}{4.547694in}}%
\pgfpathlineto{\pgfqpoint{3.978618in}{4.570065in}}%
\pgfpathclose%
\pgfusepath{fill}%
\end{pgfscope}%
\begin{pgfscope}%
\pgfpathrectangle{\pgfqpoint{0.539299in}{0.078740in}}{\pgfqpoint{7.842520in}{7.842520in}}%
\pgfusepath{clip}%
\pgfsetbuttcap%
\pgfsetroundjoin%
\definecolor{currentfill}{rgb}{0.282290,0.145912,0.461510}%
\pgfsetfillcolor{currentfill}%
\pgfsetlinewidth{0.000000pt}%
\definecolor{currentstroke}{rgb}{0.283072,0.130895,0.449241}%
\pgfsetstrokecolor{currentstroke}%
\pgfsetdash{}{0pt}%
\pgfpathmoveto{\pgfqpoint{4.577380in}{4.204259in}}%
\pgfpathlineto{\pgfqpoint{4.652446in}{4.094919in}}%
\pgfpathlineto{\pgfqpoint{4.446717in}{4.289781in}}%
\pgfpathclose%
\pgfusepath{fill}%
\end{pgfscope}%
\begin{pgfscope}%
\pgfpathrectangle{\pgfqpoint{0.539299in}{0.078740in}}{\pgfqpoint{7.842520in}{7.842520in}}%
\pgfusepath{clip}%
\pgfsetbuttcap%
\pgfsetroundjoin%
\definecolor{currentfill}{rgb}{0.282327,0.094955,0.417331}%
\pgfsetfillcolor{currentfill}%
\pgfsetlinewidth{0.000000pt}%
\definecolor{currentstroke}{rgb}{0.282884,0.135920,0.453427}%
\pgfsetstrokecolor{currentstroke}%
\pgfsetdash{}{0pt}%
\pgfpathmoveto{\pgfqpoint{4.913467in}{3.962882in}}%
\pgfpathlineto{\pgfqpoint{4.782803in}{4.027779in}}%
\pgfpathlineto{\pgfqpoint{4.708270in}{4.115796in}}%
\pgfpathclose%
\pgfusepath{fill}%
\end{pgfscope}%
\begin{pgfscope}%
\pgfpathrectangle{\pgfqpoint{0.539299in}{0.078740in}}{\pgfqpoint{7.842520in}{7.842520in}}%
\pgfusepath{clip}%
\pgfsetbuttcap%
\pgfsetroundjoin%
\definecolor{currentfill}{rgb}{0.263663,0.237631,0.518762}%
\pgfsetfillcolor{currentfill}%
\pgfsetlinewidth{0.000000pt}%
\definecolor{currentstroke}{rgb}{0.282623,0.140926,0.457517}%
\pgfsetstrokecolor{currentstroke}%
\pgfsetdash{}{0pt}%
\pgfpathmoveto{\pgfqpoint{4.108635in}{4.547005in}}%
\pgfpathlineto{\pgfqpoint{4.186265in}{4.426031in}}%
\pgfpathlineto{\pgfqpoint{4.056700in}{4.461449in}}%
\pgfpathclose%
\pgfusepath{fill}%
\end{pgfscope}%
\begin{pgfscope}%
\pgfpathrectangle{\pgfqpoint{0.539299in}{0.078740in}}{\pgfqpoint{7.842520in}{7.842520in}}%
\pgfusepath{clip}%
\pgfsetbuttcap%
\pgfsetroundjoin%
\definecolor{currentfill}{rgb}{0.255645,0.260703,0.528312}%
\pgfsetfillcolor{currentfill}%
\pgfsetlinewidth{0.000000pt}%
\definecolor{currentstroke}{rgb}{0.282290,0.145912,0.461510}%
\pgfsetstrokecolor{currentstroke}%
\pgfsetdash{}{0pt}%
\pgfpathmoveto{\pgfqpoint{3.770691in}{4.607440in}}%
\pgfpathlineto{\pgfqpoint{3.849491in}{4.547694in}}%
\pgfpathlineto{\pgfqpoint{3.721620in}{4.468366in}}%
\pgfpathclose%
\pgfusepath{fill}%
\end{pgfscope}%
\begin{pgfscope}%
\pgfpathrectangle{\pgfqpoint{0.539299in}{0.078740in}}{\pgfqpoint{7.842520in}{7.842520in}}%
\pgfusepath{clip}%
\pgfsetbuttcap%
\pgfsetroundjoin%
\definecolor{currentfill}{rgb}{0.282884,0.135920,0.453427}%
\pgfsetfillcolor{currentfill}%
\pgfsetlinewidth{0.000000pt}%
\definecolor{currentstroke}{rgb}{0.281887,0.150881,0.465405}%
\pgfsetstrokecolor{currentstroke}%
\pgfsetdash{}{0pt}%
\pgfpathmoveto{\pgfqpoint{4.708270in}{4.115796in}}%
\pgfpathlineto{\pgfqpoint{4.652446in}{4.094919in}}%
\pgfpathlineto{\pgfqpoint{4.577380in}{4.204259in}}%
\pgfpathclose%
\pgfusepath{fill}%
\end{pgfscope}%
\begin{pgfscope}%
\pgfpathrectangle{\pgfqpoint{0.539299in}{0.078740in}}{\pgfqpoint{7.842520in}{7.842520in}}%
\pgfusepath{clip}%
\pgfsetbuttcap%
\pgfsetroundjoin%
\definecolor{currentfill}{rgb}{0.257322,0.256130,0.526563}%
\pgfsetfillcolor{currentfill}%
\pgfsetlinewidth{0.000000pt}%
\definecolor{currentstroke}{rgb}{0.281412,0.155834,0.469201}%
\pgfsetstrokecolor{currentstroke}%
\pgfsetdash{}{0pt}%
\pgfpathmoveto{\pgfqpoint{3.721620in}{4.468366in}}%
\pgfpathlineto{\pgfqpoint{3.642937in}{4.493560in}}%
\pgfpathlineto{\pgfqpoint{3.770691in}{4.607440in}}%
\pgfpathclose%
\pgfusepath{fill}%
\end{pgfscope}%
\begin{pgfscope}%
\pgfpathrectangle{\pgfqpoint{0.539299in}{0.078740in}}{\pgfqpoint{7.842520in}{7.842520in}}%
\pgfusepath{clip}%
\pgfsetbuttcap%
\pgfsetroundjoin%
\definecolor{currentfill}{rgb}{0.280894,0.078907,0.402329}%
\pgfsetfillcolor{currentfill}%
\pgfsetlinewidth{0.000000pt}%
\definecolor{currentstroke}{rgb}{0.280868,0.160771,0.472899}%
\pgfsetstrokecolor{currentstroke}%
\pgfsetdash{}{0pt}%
\pgfpathmoveto{\pgfqpoint{4.839392in}{4.029475in}}%
\pgfpathlineto{\pgfqpoint{5.044466in}{3.902225in}}%
\pgfpathlineto{\pgfqpoint{4.913467in}{3.962882in}}%
\pgfpathclose%
\pgfusepath{fill}%
\end{pgfscope}%
\begin{pgfscope}%
\pgfpathrectangle{\pgfqpoint{0.539299in}{0.078740in}}{\pgfqpoint{7.842520in}{7.842520in}}%
\pgfusepath{clip}%
\pgfsetbuttcap%
\pgfsetroundjoin%
\definecolor{currentfill}{rgb}{0.255645,0.260703,0.528312}%
\pgfsetfillcolor{currentfill}%
\pgfsetlinewidth{0.000000pt}%
\definecolor{currentstroke}{rgb}{0.280255,0.165693,0.476498}%
\pgfsetstrokecolor{currentstroke}%
\pgfsetdash{}{0pt}%
\pgfpathmoveto{\pgfqpoint{3.978618in}{4.570065in}}%
\pgfpathlineto{\pgfqpoint{4.108635in}{4.547005in}}%
\pgfpathlineto{\pgfqpoint{4.056700in}{4.461449in}}%
\pgfpathclose%
\pgfusepath{fill}%
\end{pgfscope}%
\begin{pgfscope}%
\pgfpathrectangle{\pgfqpoint{0.539299in}{0.078740in}}{\pgfqpoint{7.842520in}{7.842520in}}%
\pgfusepath{clip}%
\pgfsetbuttcap%
\pgfsetroundjoin%
\definecolor{currentfill}{rgb}{0.265145,0.232956,0.516599}%
\pgfsetfillcolor{currentfill}%
\pgfsetlinewidth{0.000000pt}%
\definecolor{currentstroke}{rgb}{0.279574,0.170599,0.479997}%
\pgfsetstrokecolor{currentstroke}%
\pgfsetdash{}{0pt}%
\pgfpathmoveto{\pgfqpoint{4.186265in}{4.426031in}}%
\pgfpathlineto{\pgfqpoint{4.239257in}{4.489420in}}%
\pgfpathlineto{\pgfqpoint{4.316318in}{4.366115in}}%
\pgfpathclose%
\pgfusepath{fill}%
\end{pgfscope}%
\begin{pgfscope}%
\pgfpathrectangle{\pgfqpoint{0.539299in}{0.078740in}}{\pgfqpoint{7.842520in}{7.842520in}}%
\pgfusepath{clip}%
\pgfsetbuttcap%
\pgfsetroundjoin%
\definecolor{currentfill}{rgb}{0.265145,0.232956,0.516599}%
\pgfsetfillcolor{currentfill}%
\pgfsetlinewidth{0.000000pt}%
\definecolor{currentstroke}{rgb}{0.278826,0.175490,0.483397}%
\pgfsetstrokecolor{currentstroke}%
\pgfsetdash{}{0pt}%
\pgfpathmoveto{\pgfqpoint{3.438560in}{4.276751in}}%
\pgfpathlineto{\pgfqpoint{3.563833in}{4.504742in}}%
\pgfpathlineto{\pgfqpoint{3.642937in}{4.493560in}}%
\pgfpathclose%
\pgfusepath{fill}%
\end{pgfscope}%
\begin{pgfscope}%
\pgfpathrectangle{\pgfqpoint{0.539299in}{0.078740in}}{\pgfqpoint{7.842520in}{7.842520in}}%
\pgfusepath{clip}%
\pgfsetbuttcap%
\pgfsetroundjoin%
\definecolor{currentfill}{rgb}{0.283091,0.110553,0.431554}%
\pgfsetfillcolor{currentfill}%
\pgfsetlinewidth{0.000000pt}%
\definecolor{currentstroke}{rgb}{0.278012,0.180367,0.486697}%
\pgfsetstrokecolor{currentstroke}%
\pgfsetdash{}{0pt}%
\pgfpathmoveto{\pgfqpoint{4.708270in}{4.115796in}}%
\pgfpathlineto{\pgfqpoint{4.839392in}{4.029475in}}%
\pgfpathlineto{\pgfqpoint{4.913467in}{3.962882in}}%
\pgfpathclose%
\pgfusepath{fill}%
\end{pgfscope}%
\begin{pgfscope}%
\pgfpathrectangle{\pgfqpoint{0.539299in}{0.078740in}}{\pgfqpoint{7.842520in}{7.842520in}}%
\pgfusepath{clip}%
\pgfsetbuttcap%
\pgfsetroundjoin%
\definecolor{currentfill}{rgb}{0.271828,0.209303,0.504434}%
\pgfsetfillcolor{currentfill}%
\pgfsetlinewidth{0.000000pt}%
\definecolor{currentstroke}{rgb}{0.277134,0.185228,0.489898}%
\pgfsetstrokecolor{currentstroke}%
\pgfsetdash{}{0pt}%
\pgfpathmoveto{\pgfqpoint{3.438560in}{4.276751in}}%
\pgfpathlineto{\pgfqpoint{3.359496in}{4.245638in}}%
\pgfpathlineto{\pgfqpoint{3.563833in}{4.504742in}}%
\pgfpathclose%
\pgfusepath{fill}%
\end{pgfscope}%
\begin{pgfscope}%
\pgfpathrectangle{\pgfqpoint{0.539299in}{0.078740in}}{\pgfqpoint{7.842520in}{7.842520in}}%
\pgfusepath{clip}%
\pgfsetbuttcap%
\pgfsetroundjoin%
\definecolor{currentfill}{rgb}{0.267968,0.223549,0.512008}%
\pgfsetfillcolor{currentfill}%
\pgfsetlinewidth{0.000000pt}%
\definecolor{currentstroke}{rgb}{0.276194,0.190074,0.493001}%
\pgfsetstrokecolor{currentstroke}%
\pgfsetdash{}{0pt}%
\pgfpathmoveto{\pgfqpoint{4.239257in}{4.489420in}}%
\pgfpathlineto{\pgfqpoint{4.446717in}{4.289781in}}%
\pgfpathlineto{\pgfqpoint{4.316318in}{4.366115in}}%
\pgfpathclose%
\pgfusepath{fill}%
\end{pgfscope}%
\begin{pgfscope}%
\pgfpathrectangle{\pgfqpoint{0.539299in}{0.078740in}}{\pgfqpoint{7.842520in}{7.842520in}}%
\pgfusepath{clip}%
\pgfsetbuttcap%
\pgfsetroundjoin%
\definecolor{currentfill}{rgb}{0.278791,0.062145,0.386592}%
\pgfsetfillcolor{currentfill}%
\pgfsetlinewidth{0.000000pt}%
\definecolor{currentstroke}{rgb}{0.275191,0.194905,0.496005}%
\pgfsetstrokecolor{currentstroke}%
\pgfsetdash{}{0pt}%
\pgfpathmoveto{\pgfqpoint{5.175833in}{3.846987in}}%
\pgfpathlineto{\pgfqpoint{5.044466in}{3.902225in}}%
\pgfpathlineto{\pgfqpoint{5.102484in}{3.877072in}}%
\pgfpathclose%
\pgfusepath{fill}%
\end{pgfscope}%
\begin{pgfscope}%
\pgfpathrectangle{\pgfqpoint{0.539299in}{0.078740in}}{\pgfqpoint{7.842520in}{7.842520in}}%
\pgfusepath{clip}%
\pgfsetbuttcap%
\pgfsetroundjoin%
\definecolor{currentfill}{rgb}{0.255645,0.260703,0.528312}%
\pgfsetfillcolor{currentfill}%
\pgfsetlinewidth{0.000000pt}%
\definecolor{currentstroke}{rgb}{0.274128,0.199721,0.498911}%
\pgfsetstrokecolor{currentstroke}%
\pgfsetdash{}{0pt}%
\pgfpathmoveto{\pgfqpoint{4.108635in}{4.547005in}}%
\pgfpathlineto{\pgfqpoint{4.239257in}{4.489420in}}%
\pgfpathlineto{\pgfqpoint{4.186265in}{4.426031in}}%
\pgfpathclose%
\pgfusepath{fill}%
\end{pgfscope}%
\begin{pgfscope}%
\pgfpathrectangle{\pgfqpoint{0.539299in}{0.078740in}}{\pgfqpoint{7.842520in}{7.842520in}}%
\pgfusepath{clip}%
\pgfsetbuttcap%
\pgfsetroundjoin%
\definecolor{currentfill}{rgb}{0.282327,0.094955,0.417331}%
\pgfsetfillcolor{currentfill}%
\pgfsetlinewidth{0.000000pt}%
\definecolor{currentstroke}{rgb}{0.273006,0.204520,0.501721}%
\pgfsetstrokecolor{currentstroke}%
\pgfsetdash{}{0pt}%
\pgfpathmoveto{\pgfqpoint{4.970779in}{3.949088in}}%
\pgfpathlineto{\pgfqpoint{5.044466in}{3.902225in}}%
\pgfpathlineto{\pgfqpoint{4.839392in}{4.029475in}}%
\pgfpathclose%
\pgfusepath{fill}%
\end{pgfscope}%
\begin{pgfscope}%
\pgfpathrectangle{\pgfqpoint{0.539299in}{0.078740in}}{\pgfqpoint{7.842520in}{7.842520in}}%
\pgfusepath{clip}%
\pgfsetbuttcap%
\pgfsetroundjoin%
\definecolor{currentfill}{rgb}{0.243113,0.292092,0.538516}%
\pgfsetfillcolor{currentfill}%
\pgfsetlinewidth{0.000000pt}%
\definecolor{currentstroke}{rgb}{0.271828,0.209303,0.504434}%
\pgfsetstrokecolor{currentstroke}%
\pgfsetdash{}{0pt}%
\pgfpathmoveto{\pgfqpoint{3.978618in}{4.570065in}}%
\pgfpathlineto{\pgfqpoint{3.849491in}{4.547694in}}%
\pgfpathlineto{\pgfqpoint{3.899971in}{4.655078in}}%
\pgfpathclose%
\pgfusepath{fill}%
\end{pgfscope}%
\begin{pgfscope}%
\pgfpathrectangle{\pgfqpoint{0.539299in}{0.078740in}}{\pgfqpoint{7.842520in}{7.842520in}}%
\pgfusepath{clip}%
\pgfsetbuttcap%
\pgfsetroundjoin%
\definecolor{currentfill}{rgb}{0.275191,0.194905,0.496005}%
\pgfsetfillcolor{currentfill}%
\pgfsetlinewidth{0.000000pt}%
\definecolor{currentstroke}{rgb}{0.270595,0.214069,0.507052}%
\pgfsetstrokecolor{currentstroke}%
\pgfsetdash{}{0pt}%
\pgfpathmoveto{\pgfqpoint{4.577380in}{4.204259in}}%
\pgfpathlineto{\pgfqpoint{4.446717in}{4.289781in}}%
\pgfpathlineto{\pgfqpoint{4.501556in}{4.310839in}}%
\pgfpathclose%
\pgfusepath{fill}%
\end{pgfscope}%
\begin{pgfscope}%
\pgfpathrectangle{\pgfqpoint{0.539299in}{0.078740in}}{\pgfqpoint{7.842520in}{7.842520in}}%
\pgfusepath{clip}%
\pgfsetbuttcap%
\pgfsetroundjoin%
\definecolor{currentfill}{rgb}{0.277941,0.056324,0.381191}%
\pgfsetfillcolor{currentfill}%
\pgfsetlinewidth{0.000000pt}%
\definecolor{currentstroke}{rgb}{0.269308,0.218818,0.509577}%
\pgfsetstrokecolor{currentstroke}%
\pgfsetdash{}{0pt}%
\pgfpathmoveto{\pgfqpoint{5.102484in}{3.877072in}}%
\pgfpathlineto{\pgfqpoint{5.307605in}{3.797555in}}%
\pgfpathlineto{\pgfqpoint{5.175833in}{3.846987in}}%
\pgfpathclose%
\pgfusepath{fill}%
\end{pgfscope}%
\begin{pgfscope}%
\pgfpathrectangle{\pgfqpoint{0.539299in}{0.078740in}}{\pgfqpoint{7.842520in}{7.842520in}}%
\pgfusepath{clip}%
\pgfsetbuttcap%
\pgfsetroundjoin%
\definecolor{currentfill}{rgb}{0.281446,0.084320,0.407414}%
\pgfsetfillcolor{currentfill}%
\pgfsetlinewidth{0.000000pt}%
\definecolor{currentstroke}{rgb}{0.267968,0.223549,0.512008}%
\pgfsetstrokecolor{currentstroke}%
\pgfsetdash{}{0pt}%
\pgfpathmoveto{\pgfqpoint{5.102484in}{3.877072in}}%
\pgfpathlineto{\pgfqpoint{5.044466in}{3.902225in}}%
\pgfpathlineto{\pgfqpoint{4.970779in}{3.949088in}}%
\pgfpathclose%
\pgfusepath{fill}%
\end{pgfscope}%
\begin{pgfscope}%
\pgfpathrectangle{\pgfqpoint{0.539299in}{0.078740in}}{\pgfqpoint{7.842520in}{7.842520in}}%
\pgfusepath{clip}%
\pgfsetbuttcap%
\pgfsetroundjoin%
\definecolor{currentfill}{rgb}{0.239346,0.300855,0.540844}%
\pgfsetfillcolor{currentfill}%
\pgfsetlinewidth{0.000000pt}%
\definecolor{currentstroke}{rgb}{0.266580,0.228262,0.514349}%
\pgfsetstrokecolor{currentstroke}%
\pgfsetdash{}{0pt}%
\pgfpathmoveto{\pgfqpoint{3.899971in}{4.655078in}}%
\pgfpathlineto{\pgfqpoint{3.849491in}{4.547694in}}%
\pgfpathlineto{\pgfqpoint{3.770691in}{4.607440in}}%
\pgfpathclose%
\pgfusepath{fill}%
\end{pgfscope}%
\begin{pgfscope}%
\pgfpathrectangle{\pgfqpoint{0.539299in}{0.078740in}}{\pgfqpoint{7.842520in}{7.842520in}}%
\pgfusepath{clip}%
\pgfsetbuttcap%
\pgfsetroundjoin%
\definecolor{currentfill}{rgb}{0.277134,0.185228,0.489898}%
\pgfsetfillcolor{currentfill}%
\pgfsetlinewidth{0.000000pt}%
\definecolor{currentstroke}{rgb}{0.265145,0.232956,0.516599}%
\pgfsetstrokecolor{currentstroke}%
\pgfsetdash{}{0pt}%
\pgfpathmoveto{\pgfqpoint{4.577380in}{4.204259in}}%
\pgfpathlineto{\pgfqpoint{4.501556in}{4.310839in}}%
\pgfpathlineto{\pgfqpoint{4.708270in}{4.115796in}}%
\pgfpathclose%
\pgfusepath{fill}%
\end{pgfscope}%
\begin{pgfscope}%
\pgfpathrectangle{\pgfqpoint{0.539299in}{0.078740in}}{\pgfqpoint{7.842520in}{7.842520in}}%
\pgfusepath{clip}%
\pgfsetbuttcap%
\pgfsetroundjoin%
\definecolor{currentfill}{rgb}{0.262138,0.242286,0.520837}%
\pgfsetfillcolor{currentfill}%
\pgfsetlinewidth{0.000000pt}%
\definecolor{currentstroke}{rgb}{0.263663,0.237631,0.518762}%
\pgfsetstrokecolor{currentstroke}%
\pgfsetdash{}{0pt}%
\pgfpathmoveto{\pgfqpoint{4.370276in}{4.407562in}}%
\pgfpathlineto{\pgfqpoint{4.446717in}{4.289781in}}%
\pgfpathlineto{\pgfqpoint{4.239257in}{4.489420in}}%
\pgfpathclose%
\pgfusepath{fill}%
\end{pgfscope}%
\begin{pgfscope}%
\pgfpathrectangle{\pgfqpoint{0.539299in}{0.078740in}}{\pgfqpoint{7.842520in}{7.842520in}}%
\pgfusepath{clip}%
\pgfsetbuttcap%
\pgfsetroundjoin%
\definecolor{currentfill}{rgb}{0.239346,0.300855,0.540844}%
\pgfsetfillcolor{currentfill}%
\pgfsetlinewidth{0.000000pt}%
\definecolor{currentstroke}{rgb}{0.262138,0.242286,0.520837}%
\pgfsetstrokecolor{currentstroke}%
\pgfsetdash{}{0pt}%
\pgfpathmoveto{\pgfqpoint{3.899971in}{4.655078in}}%
\pgfpathlineto{\pgfqpoint{4.108635in}{4.547005in}}%
\pgfpathlineto{\pgfqpoint{3.978618in}{4.570065in}}%
\pgfpathclose%
\pgfusepath{fill}%
\end{pgfscope}%
\begin{pgfscope}%
\pgfpathrectangle{\pgfqpoint{0.539299in}{0.078740in}}{\pgfqpoint{7.842520in}{7.842520in}}%
\pgfusepath{clip}%
\pgfsetbuttcap%
\pgfsetroundjoin%
\definecolor{currentfill}{rgb}{0.239346,0.300855,0.540844}%
\pgfsetfillcolor{currentfill}%
\pgfsetlinewidth{0.000000pt}%
\definecolor{currentstroke}{rgb}{0.260571,0.246922,0.522828}%
\pgfsetstrokecolor{currentstroke}%
\pgfsetdash{}{0pt}%
\pgfpathmoveto{\pgfqpoint{3.642937in}{4.493560in}}%
\pgfpathlineto{\pgfqpoint{3.691406in}{4.650385in}}%
\pgfpathlineto{\pgfqpoint{3.770691in}{4.607440in}}%
\pgfpathclose%
\pgfusepath{fill}%
\end{pgfscope}%
\begin{pgfscope}%
\pgfpathrectangle{\pgfqpoint{0.539299in}{0.078740in}}{\pgfqpoint{7.842520in}{7.842520in}}%
\pgfusepath{clip}%
\pgfsetbuttcap%
\pgfsetroundjoin%
\definecolor{currentfill}{rgb}{0.243113,0.292092,0.538516}%
\pgfsetfillcolor{currentfill}%
\pgfsetlinewidth{0.000000pt}%
\definecolor{currentstroke}{rgb}{0.258965,0.251537,0.524736}%
\pgfsetstrokecolor{currentstroke}%
\pgfsetdash{}{0pt}%
\pgfpathmoveto{\pgfqpoint{3.563833in}{4.504742in}}%
\pgfpathlineto{\pgfqpoint{3.691406in}{4.650385in}}%
\pgfpathlineto{\pgfqpoint{3.642937in}{4.493560in}}%
\pgfpathclose%
\pgfusepath{fill}%
\end{pgfscope}%
\begin{pgfscope}%
\pgfpathrectangle{\pgfqpoint{0.539299in}{0.078740in}}{\pgfqpoint{7.842520in}{7.842520in}}%
\pgfusepath{clip}%
\pgfsetbuttcap%
\pgfsetroundjoin%
\definecolor{currentfill}{rgb}{0.266580,0.228262,0.514349}%
\pgfsetfillcolor{currentfill}%
\pgfsetlinewidth{0.000000pt}%
\definecolor{currentstroke}{rgb}{0.257322,0.256130,0.526563}%
\pgfsetstrokecolor{currentstroke}%
\pgfsetdash{}{0pt}%
\pgfpathmoveto{\pgfqpoint{4.501556in}{4.310839in}}%
\pgfpathlineto{\pgfqpoint{4.446717in}{4.289781in}}%
\pgfpathlineto{\pgfqpoint{4.370276in}{4.407562in}}%
\pgfpathclose%
\pgfusepath{fill}%
\end{pgfscope}%
\begin{pgfscope}%
\pgfpathrectangle{\pgfqpoint{0.539299in}{0.078740in}}{\pgfqpoint{7.842520in}{7.842520in}}%
\pgfusepath{clip}%
\pgfsetbuttcap%
\pgfsetroundjoin%
\definecolor{currentfill}{rgb}{0.257322,0.256130,0.526563}%
\pgfsetfillcolor{currentfill}%
\pgfsetlinewidth{0.000000pt}%
\definecolor{currentstroke}{rgb}{0.255645,0.260703,0.528312}%
\pgfsetstrokecolor{currentstroke}%
\pgfsetdash{}{0pt}%
\pgfpathmoveto{\pgfqpoint{3.563833in}{4.504742in}}%
\pgfpathlineto{\pgfqpoint{3.359496in}{4.245638in}}%
\pgfpathlineto{\pgfqpoint{3.484281in}{4.506457in}}%
\pgfpathclose%
\pgfusepath{fill}%
\end{pgfscope}%
\begin{pgfscope}%
\pgfpathrectangle{\pgfqpoint{0.539299in}{0.078740in}}{\pgfqpoint{7.842520in}{7.842520in}}%
\pgfusepath{clip}%
\pgfsetbuttcap%
\pgfsetroundjoin%
\definecolor{currentfill}{rgb}{0.281887,0.150881,0.465405}%
\pgfsetfillcolor{currentfill}%
\pgfsetlinewidth{0.000000pt}%
\definecolor{currentstroke}{rgb}{0.253935,0.265254,0.529983}%
\pgfsetstrokecolor{currentstroke}%
\pgfsetdash{}{0pt}%
\pgfpathmoveto{\pgfqpoint{4.839392in}{4.029475in}}%
\pgfpathlineto{\pgfqpoint{4.708270in}{4.115796in}}%
\pgfpathlineto{\pgfqpoint{4.764671in}{4.104606in}}%
\pgfpathclose%
\pgfusepath{fill}%
\end{pgfscope}%
\begin{pgfscope}%
\pgfpathrectangle{\pgfqpoint{0.539299in}{0.078740in}}{\pgfqpoint{7.842520in}{7.842520in}}%
\pgfusepath{clip}%
\pgfsetbuttcap%
\pgfsetroundjoin%
\definecolor{currentfill}{rgb}{0.266580,0.228262,0.514349}%
\pgfsetfillcolor{currentfill}%
\pgfsetlinewidth{0.000000pt}%
\definecolor{currentstroke}{rgb}{0.252194,0.269783,0.531579}%
\pgfsetstrokecolor{currentstroke}%
\pgfsetdash{}{0pt}%
\pgfpathmoveto{\pgfqpoint{3.484281in}{4.506457in}}%
\pgfpathlineto{\pgfqpoint{3.359496in}{4.245638in}}%
\pgfpathlineto{\pgfqpoint{3.279998in}{4.210712in}}%
\pgfpathclose%
\pgfusepath{fill}%
\end{pgfscope}%
\begin{pgfscope}%
\pgfpathrectangle{\pgfqpoint{0.539299in}{0.078740in}}{\pgfqpoint{7.842520in}{7.842520in}}%
\pgfusepath{clip}%
\pgfsetbuttcap%
\pgfsetroundjoin%
\definecolor{currentfill}{rgb}{0.279566,0.067836,0.391917}%
\pgfsetfillcolor{currentfill}%
\pgfsetlinewidth{0.000000pt}%
\definecolor{currentstroke}{rgb}{0.250425,0.274290,0.533103}%
\pgfsetstrokecolor{currentstroke}%
\pgfsetdash{}{0pt}%
\pgfpathmoveto{\pgfqpoint{5.234563in}{3.814537in}}%
\pgfpathlineto{\pgfqpoint{5.307605in}{3.797555in}}%
\pgfpathlineto{\pgfqpoint{5.102484in}{3.877072in}}%
\pgfpathclose%
\pgfusepath{fill}%
\end{pgfscope}%
\begin{pgfscope}%
\pgfpathrectangle{\pgfqpoint{0.539299in}{0.078740in}}{\pgfqpoint{7.842520in}{7.842520in}}%
\pgfusepath{clip}%
\pgfsetbuttcap%
\pgfsetroundjoin%
\definecolor{currentfill}{rgb}{0.276022,0.044167,0.370164}%
\pgfsetfillcolor{currentfill}%
\pgfsetlinewidth{0.000000pt}%
\definecolor{currentstroke}{rgb}{0.248629,0.278775,0.534556}%
\pgfsetstrokecolor{currentstroke}%
\pgfsetdash{}{0pt}%
\pgfpathmoveto{\pgfqpoint{5.439811in}{3.753615in}}%
\pgfpathlineto{\pgfqpoint{5.307605in}{3.797555in}}%
\pgfpathlineto{\pgfqpoint{5.367069in}{3.761390in}}%
\pgfpathclose%
\pgfusepath{fill}%
\end{pgfscope}%
\begin{pgfscope}%
\pgfpathrectangle{\pgfqpoint{0.539299in}{0.078740in}}{\pgfqpoint{7.842520in}{7.842520in}}%
\pgfusepath{clip}%
\pgfsetbuttcap%
\pgfsetroundjoin%
\definecolor{currentfill}{rgb}{0.282884,0.135920,0.453427}%
\pgfsetfillcolor{currentfill}%
\pgfsetlinewidth{0.000000pt}%
\definecolor{currentstroke}{rgb}{0.246811,0.283237,0.535941}%
\pgfsetstrokecolor{currentstroke}%
\pgfsetdash{}{0pt}%
\pgfpathmoveto{\pgfqpoint{4.970779in}{3.949088in}}%
\pgfpathlineto{\pgfqpoint{4.839392in}{4.029475in}}%
\pgfpathlineto{\pgfqpoint{4.764671in}{4.104606in}}%
\pgfpathclose%
\pgfusepath{fill}%
\end{pgfscope}%
\begin{pgfscope}%
\pgfpathrectangle{\pgfqpoint{0.539299in}{0.078740in}}{\pgfqpoint{7.842520in}{7.842520in}}%
\pgfusepath{clip}%
\pgfsetbuttcap%
\pgfsetroundjoin%
\definecolor{currentfill}{rgb}{0.274128,0.199721,0.498911}%
\pgfsetfillcolor{currentfill}%
\pgfsetlinewidth{0.000000pt}%
\definecolor{currentstroke}{rgb}{0.244972,0.287675,0.537260}%
\pgfsetstrokecolor{currentstroke}%
\pgfsetdash{}{0pt}%
\pgfpathmoveto{\pgfqpoint{4.501556in}{4.310839in}}%
\pgfpathlineto{\pgfqpoint{4.633026in}{4.207547in}}%
\pgfpathlineto{\pgfqpoint{4.708270in}{4.115796in}}%
\pgfpathclose%
\pgfusepath{fill}%
\end{pgfscope}%
\begin{pgfscope}%
\pgfpathrectangle{\pgfqpoint{0.539299in}{0.078740in}}{\pgfqpoint{7.842520in}{7.842520in}}%
\pgfusepath{clip}%
\pgfsetbuttcap%
\pgfsetroundjoin%
\definecolor{currentfill}{rgb}{0.278791,0.062145,0.386592}%
\pgfsetfillcolor{currentfill}%
\pgfsetlinewidth{0.000000pt}%
\definecolor{currentstroke}{rgb}{0.243113,0.292092,0.538516}%
\pgfsetstrokecolor{currentstroke}%
\pgfsetdash{}{0pt}%
\pgfpathmoveto{\pgfqpoint{5.367069in}{3.761390in}}%
\pgfpathlineto{\pgfqpoint{5.307605in}{3.797555in}}%
\pgfpathlineto{\pgfqpoint{5.234563in}{3.814537in}}%
\pgfpathclose%
\pgfusepath{fill}%
\end{pgfscope}%
\begin{pgfscope}%
\pgfpathrectangle{\pgfqpoint{0.539299in}{0.078740in}}{\pgfqpoint{7.842520in}{7.842520in}}%
\pgfusepath{clip}%
\pgfsetbuttcap%
\pgfsetroundjoin%
\definecolor{currentfill}{rgb}{0.278012,0.180367,0.486697}%
\pgfsetfillcolor{currentfill}%
\pgfsetlinewidth{0.000000pt}%
\definecolor{currentstroke}{rgb}{0.241237,0.296485,0.539709}%
\pgfsetstrokecolor{currentstroke}%
\pgfsetdash{}{0pt}%
\pgfpathmoveto{\pgfqpoint{4.764671in}{4.104606in}}%
\pgfpathlineto{\pgfqpoint{4.708270in}{4.115796in}}%
\pgfpathlineto{\pgfqpoint{4.633026in}{4.207547in}}%
\pgfpathclose%
\pgfusepath{fill}%
\end{pgfscope}%
\begin{pgfscope}%
\pgfpathrectangle{\pgfqpoint{0.539299in}{0.078740in}}{\pgfqpoint{7.842520in}{7.842520in}}%
\pgfusepath{clip}%
\pgfsetbuttcap%
\pgfsetroundjoin%
\definecolor{currentfill}{rgb}{0.276022,0.044167,0.370164}%
\pgfsetfillcolor{currentfill}%
\pgfsetlinewidth{0.000000pt}%
\definecolor{currentstroke}{rgb}{0.239346,0.300855,0.540844}%
\pgfsetstrokecolor{currentstroke}%
\pgfsetdash{}{0pt}%
\pgfpathmoveto{\pgfqpoint{5.367069in}{3.761390in}}%
\pgfpathlineto{\pgfqpoint{5.572472in}{3.714282in}}%
\pgfpathlineto{\pgfqpoint{5.439811in}{3.753615in}}%
\pgfpathclose%
\pgfusepath{fill}%
\end{pgfscope}%
\begin{pgfscope}%
\pgfpathrectangle{\pgfqpoint{0.539299in}{0.078740in}}{\pgfqpoint{7.842520in}{7.842520in}}%
\pgfusepath{clip}%
\pgfsetbuttcap%
\pgfsetroundjoin%
\definecolor{currentfill}{rgb}{0.237441,0.305202,0.541921}%
\pgfsetfillcolor{currentfill}%
\pgfsetlinewidth{0.000000pt}%
\definecolor{currentstroke}{rgb}{0.237441,0.305202,0.541921}%
\pgfsetstrokecolor{currentstroke}%
\pgfsetdash{}{0pt}%
\pgfpathmoveto{\pgfqpoint{4.161473in}{4.598110in}}%
\pgfpathlineto{\pgfqpoint{4.239257in}{4.489420in}}%
\pgfpathlineto{\pgfqpoint{4.108635in}{4.547005in}}%
\pgfpathclose%
\pgfusepath{fill}%
\end{pgfscope}%
\begin{pgfscope}%
\pgfpathrectangle{\pgfqpoint{0.539299in}{0.078740in}}{\pgfqpoint{7.842520in}{7.842520in}}%
\pgfusepath{clip}%
\pgfsetbuttcap%
\pgfsetroundjoin%
\definecolor{currentfill}{rgb}{0.225863,0.330805,0.547314}%
\pgfsetfillcolor{currentfill}%
\pgfsetlinewidth{0.000000pt}%
\definecolor{currentstroke}{rgb}{0.235526,0.309527,0.542944}%
\pgfsetstrokecolor{currentstroke}%
\pgfsetdash{}{0pt}%
\pgfpathmoveto{\pgfqpoint{3.770691in}{4.607440in}}%
\pgfpathlineto{\pgfqpoint{3.691406in}{4.650385in}}%
\pgfpathlineto{\pgfqpoint{3.899971in}{4.655078in}}%
\pgfpathclose%
\pgfusepath{fill}%
\end{pgfscope}%
\begin{pgfscope}%
\pgfpathrectangle{\pgfqpoint{0.539299in}{0.078740in}}{\pgfqpoint{7.842520in}{7.842520in}}%
\pgfusepath{clip}%
\pgfsetbuttcap%
\pgfsetroundjoin%
\definecolor{currentfill}{rgb}{0.229739,0.322361,0.545706}%
\pgfsetfillcolor{currentfill}%
\pgfsetlinewidth{0.000000pt}%
\definecolor{currentstroke}{rgb}{0.233603,0.313828,0.543914}%
\pgfsetstrokecolor{currentstroke}%
\pgfsetdash{}{0pt}%
\pgfpathmoveto{\pgfqpoint{3.899971in}{4.655078in}}%
\pgfpathlineto{\pgfqpoint{4.030348in}{4.648062in}}%
\pgfpathlineto{\pgfqpoint{4.108635in}{4.547005in}}%
\pgfpathclose%
\pgfusepath{fill}%
\end{pgfscope}%
\begin{pgfscope}%
\pgfpathrectangle{\pgfqpoint{0.539299in}{0.078740in}}{\pgfqpoint{7.842520in}{7.842520in}}%
\pgfusepath{clip}%
\pgfsetbuttcap%
\pgfsetroundjoin%
\definecolor{currentfill}{rgb}{0.283091,0.110553,0.431554}%
\pgfsetfillcolor{currentfill}%
\pgfsetlinewidth{0.000000pt}%
\definecolor{currentstroke}{rgb}{0.231674,0.318106,0.544834}%
\pgfsetstrokecolor{currentstroke}%
\pgfsetdash{}{0pt}%
\pgfpathmoveto{\pgfqpoint{5.028622in}{3.919505in}}%
\pgfpathlineto{\pgfqpoint{5.102484in}{3.877072in}}%
\pgfpathlineto{\pgfqpoint{4.970779in}{3.949088in}}%
\pgfpathclose%
\pgfusepath{fill}%
\end{pgfscope}%
\begin{pgfscope}%
\pgfpathrectangle{\pgfqpoint{0.539299in}{0.078740in}}{\pgfqpoint{7.842520in}{7.842520in}}%
\pgfusepath{clip}%
\pgfsetbuttcap%
\pgfsetroundjoin%
\definecolor{currentfill}{rgb}{0.281887,0.150881,0.465405}%
\pgfsetfillcolor{currentfill}%
\pgfsetlinewidth{0.000000pt}%
\definecolor{currentstroke}{rgb}{0.229739,0.322361,0.545706}%
\pgfsetstrokecolor{currentstroke}%
\pgfsetdash{}{0pt}%
\pgfpathmoveto{\pgfqpoint{4.896518in}{4.007367in}}%
\pgfpathlineto{\pgfqpoint{4.970779in}{3.949088in}}%
\pgfpathlineto{\pgfqpoint{4.764671in}{4.104606in}}%
\pgfpathclose%
\pgfusepath{fill}%
\end{pgfscope}%
\begin{pgfscope}%
\pgfpathrectangle{\pgfqpoint{0.539299in}{0.078740in}}{\pgfqpoint{7.842520in}{7.842520in}}%
\pgfusepath{clip}%
\pgfsetbuttcap%
\pgfsetroundjoin%
\definecolor{currentfill}{rgb}{0.282656,0.100196,0.422160}%
\pgfsetfillcolor{currentfill}%
\pgfsetlinewidth{0.000000pt}%
\definecolor{currentstroke}{rgb}{0.227802,0.326594,0.546532}%
\pgfsetstrokecolor{currentstroke}%
\pgfsetdash{}{0pt}%
\pgfpathmoveto{\pgfqpoint{5.234563in}{3.814537in}}%
\pgfpathlineto{\pgfqpoint{5.102484in}{3.877072in}}%
\pgfpathlineto{\pgfqpoint{5.028622in}{3.919505in}}%
\pgfpathclose%
\pgfusepath{fill}%
\end{pgfscope}%
\begin{pgfscope}%
\pgfpathrectangle{\pgfqpoint{0.539299in}{0.078740in}}{\pgfqpoint{7.842520in}{7.842520in}}%
\pgfusepath{clip}%
\pgfsetbuttcap%
\pgfsetroundjoin%
\definecolor{currentfill}{rgb}{0.244972,0.287675,0.537260}%
\pgfsetfillcolor{currentfill}%
\pgfsetlinewidth{0.000000pt}%
\definecolor{currentstroke}{rgb}{0.225863,0.330805,0.547314}%
\pgfsetstrokecolor{currentstroke}%
\pgfsetdash{}{0pt}%
\pgfpathmoveto{\pgfqpoint{4.370276in}{4.407562in}}%
\pgfpathlineto{\pgfqpoint{4.239257in}{4.489420in}}%
\pgfpathlineto{\pgfqpoint{4.293076in}{4.516692in}}%
\pgfpathclose%
\pgfusepath{fill}%
\end{pgfscope}%
\begin{pgfscope}%
\pgfpathrectangle{\pgfqpoint{0.539299in}{0.078740in}}{\pgfqpoint{7.842520in}{7.842520in}}%
\pgfusepath{clip}%
\pgfsetbuttcap%
\pgfsetroundjoin%
\definecolor{currentfill}{rgb}{0.227802,0.326594,0.546532}%
\pgfsetfillcolor{currentfill}%
\pgfsetlinewidth{0.000000pt}%
\definecolor{currentstroke}{rgb}{0.223925,0.334994,0.548053}%
\pgfsetstrokecolor{currentstroke}%
\pgfsetdash{}{0pt}%
\pgfpathmoveto{\pgfqpoint{4.108635in}{4.547005in}}%
\pgfpathlineto{\pgfqpoint{4.030348in}{4.648062in}}%
\pgfpathlineto{\pgfqpoint{4.161473in}{4.598110in}}%
\pgfpathclose%
\pgfusepath{fill}%
\end{pgfscope}%
\begin{pgfscope}%
\pgfpathrectangle{\pgfqpoint{0.539299in}{0.078740in}}{\pgfqpoint{7.842520in}{7.842520in}}%
\pgfusepath{clip}%
\pgfsetbuttcap%
\pgfsetroundjoin%
\definecolor{currentfill}{rgb}{0.282884,0.135920,0.453427}%
\pgfsetfillcolor{currentfill}%
\pgfsetlinewidth{0.000000pt}%
\definecolor{currentstroke}{rgb}{0.221989,0.339161,0.548752}%
\pgfsetstrokecolor{currentstroke}%
\pgfsetdash{}{0pt}%
\pgfpathmoveto{\pgfqpoint{5.028622in}{3.919505in}}%
\pgfpathlineto{\pgfqpoint{4.970779in}{3.949088in}}%
\pgfpathlineto{\pgfqpoint{4.896518in}{4.007367in}}%
\pgfpathclose%
\pgfusepath{fill}%
\end{pgfscope}%
\begin{pgfscope}%
\pgfpathrectangle{\pgfqpoint{0.539299in}{0.078740in}}{\pgfqpoint{7.842520in}{7.842520in}}%
\pgfusepath{clip}%
\pgfsetbuttcap%
\pgfsetroundjoin%
\definecolor{currentfill}{rgb}{0.277941,0.056324,0.381191}%
\pgfsetfillcolor{currentfill}%
\pgfsetlinewidth{0.000000pt}%
\definecolor{currentstroke}{rgb}{0.220057,0.343307,0.549413}%
\pgfsetstrokecolor{currentstroke}%
\pgfsetdash{}{0pt}%
\pgfpathmoveto{\pgfqpoint{5.500041in}{3.716527in}}%
\pgfpathlineto{\pgfqpoint{5.572472in}{3.714282in}}%
\pgfpathlineto{\pgfqpoint{5.367069in}{3.761390in}}%
\pgfpathclose%
\pgfusepath{fill}%
\end{pgfscope}%
\begin{pgfscope}%
\pgfpathrectangle{\pgfqpoint{0.539299in}{0.078740in}}{\pgfqpoint{7.842520in}{7.842520in}}%
\pgfusepath{clip}%
\pgfsetbuttcap%
\pgfsetroundjoin%
\definecolor{currentfill}{rgb}{0.250425,0.274290,0.533103}%
\pgfsetfillcolor{currentfill}%
\pgfsetlinewidth{0.000000pt}%
\definecolor{currentstroke}{rgb}{0.218130,0.347432,0.550038}%
\pgfsetstrokecolor{currentstroke}%
\pgfsetdash{}{0pt}%
\pgfpathmoveto{\pgfqpoint{4.293076in}{4.516692in}}%
\pgfpathlineto{\pgfqpoint{4.501556in}{4.310839in}}%
\pgfpathlineto{\pgfqpoint{4.370276in}{4.407562in}}%
\pgfpathclose%
\pgfusepath{fill}%
\end{pgfscope}%
\begin{pgfscope}%
\pgfpathrectangle{\pgfqpoint{0.539299in}{0.078740in}}{\pgfqpoint{7.842520in}{7.842520in}}%
\pgfusepath{clip}%
\pgfsetbuttcap%
\pgfsetroundjoin%
\definecolor{currentfill}{rgb}{0.231674,0.318106,0.544834}%
\pgfsetfillcolor{currentfill}%
\pgfsetlinewidth{0.000000pt}%
\definecolor{currentstroke}{rgb}{0.216210,0.351535,0.550627}%
\pgfsetstrokecolor{currentstroke}%
\pgfsetdash{}{0pt}%
\pgfpathmoveto{\pgfqpoint{3.611618in}{4.681644in}}%
\pgfpathlineto{\pgfqpoint{3.563833in}{4.504742in}}%
\pgfpathlineto{\pgfqpoint{3.484281in}{4.506457in}}%
\pgfpathclose%
\pgfusepath{fill}%
\end{pgfscope}%
\begin{pgfscope}%
\pgfpathrectangle{\pgfqpoint{0.539299in}{0.078740in}}{\pgfqpoint{7.842520in}{7.842520in}}%
\pgfusepath{clip}%
\pgfsetbuttcap%
\pgfsetroundjoin%
\definecolor{currentfill}{rgb}{0.250425,0.274290,0.533103}%
\pgfsetfillcolor{currentfill}%
\pgfsetlinewidth{0.000000pt}%
\definecolor{currentstroke}{rgb}{0.214298,0.355619,0.551184}%
\pgfsetstrokecolor{currentstroke}%
\pgfsetdash{}{0pt}%
\pgfpathmoveto{\pgfqpoint{3.279998in}{4.210712in}}%
\pgfpathlineto{\pgfqpoint{3.404261in}{4.501552in}}%
\pgfpathlineto{\pgfqpoint{3.484281in}{4.506457in}}%
\pgfpathclose%
\pgfusepath{fill}%
\end{pgfscope}%
\begin{pgfscope}%
\pgfpathrectangle{\pgfqpoint{0.539299in}{0.078740in}}{\pgfqpoint{7.842520in}{7.842520in}}%
\pgfusepath{clip}%
\pgfsetbuttcap%
\pgfsetroundjoin%
\definecolor{currentfill}{rgb}{0.223925,0.334994,0.548053}%
\pgfsetfillcolor{currentfill}%
\pgfsetlinewidth{0.000000pt}%
\definecolor{currentstroke}{rgb}{0.212395,0.359683,0.551710}%
\pgfsetstrokecolor{currentstroke}%
\pgfsetdash{}{0pt}%
\pgfpathmoveto{\pgfqpoint{3.611618in}{4.681644in}}%
\pgfpathlineto{\pgfqpoint{3.691406in}{4.650385in}}%
\pgfpathlineto{\pgfqpoint{3.563833in}{4.504742in}}%
\pgfpathclose%
\pgfusepath{fill}%
\end{pgfscope}%
\begin{pgfscope}%
\pgfpathrectangle{\pgfqpoint{0.539299in}{0.078740in}}{\pgfqpoint{7.842520in}{7.842520in}}%
\pgfusepath{clip}%
\pgfsetbuttcap%
\pgfsetroundjoin%
\definecolor{currentfill}{rgb}{0.276022,0.044167,0.370164}%
\pgfsetfillcolor{currentfill}%
\pgfsetlinewidth{0.000000pt}%
\definecolor{currentstroke}{rgb}{0.210503,0.363727,0.552206}%
\pgfsetstrokecolor{currentstroke}%
\pgfsetdash{}{0pt}%
\pgfpathmoveto{\pgfqpoint{5.705595in}{3.678274in}}%
\pgfpathlineto{\pgfqpoint{5.572472in}{3.714282in}}%
\pgfpathlineto{\pgfqpoint{5.633500in}{3.678097in}}%
\pgfpathclose%
\pgfusepath{fill}%
\end{pgfscope}%
\begin{pgfscope}%
\pgfpathrectangle{\pgfqpoint{0.539299in}{0.078740in}}{\pgfqpoint{7.842520in}{7.842520in}}%
\pgfusepath{clip}%
\pgfsetbuttcap%
\pgfsetroundjoin%
\definecolor{currentfill}{rgb}{0.258965,0.251537,0.524736}%
\pgfsetfillcolor{currentfill}%
\pgfsetlinewidth{0.000000pt}%
\definecolor{currentstroke}{rgb}{0.208623,0.367752,0.552675}%
\pgfsetstrokecolor{currentstroke}%
\pgfsetdash{}{0pt}%
\pgfpathmoveto{\pgfqpoint{4.501556in}{4.310839in}}%
\pgfpathlineto{\pgfqpoint{4.424973in}{4.414618in}}%
\pgfpathlineto{\pgfqpoint{4.633026in}{4.207547in}}%
\pgfpathclose%
\pgfusepath{fill}%
\end{pgfscope}%
\begin{pgfscope}%
\pgfpathrectangle{\pgfqpoint{0.539299in}{0.078740in}}{\pgfqpoint{7.842520in}{7.842520in}}%
\pgfusepath{clip}%
\pgfsetbuttcap%
\pgfsetroundjoin%
\definecolor{currentfill}{rgb}{0.214298,0.355619,0.551184}%
\pgfsetfillcolor{currentfill}%
\pgfsetlinewidth{0.000000pt}%
\definecolor{currentstroke}{rgb}{0.206756,0.371758,0.553117}%
\pgfsetstrokecolor{currentstroke}%
\pgfsetdash{}{0pt}%
\pgfpathmoveto{\pgfqpoint{3.691406in}{4.650385in}}%
\pgfpathlineto{\pgfqpoint{3.820757in}{4.723273in}}%
\pgfpathlineto{\pgfqpoint{3.899971in}{4.655078in}}%
\pgfpathclose%
\pgfusepath{fill}%
\end{pgfscope}%
\begin{pgfscope}%
\pgfpathrectangle{\pgfqpoint{0.539299in}{0.078740in}}{\pgfqpoint{7.842520in}{7.842520in}}%
\pgfusepath{clip}%
\pgfsetbuttcap%
\pgfsetroundjoin%
\definecolor{currentfill}{rgb}{0.231674,0.318106,0.544834}%
\pgfsetfillcolor{currentfill}%
\pgfsetlinewidth{0.000000pt}%
\definecolor{currentstroke}{rgb}{0.204903,0.375746,0.553533}%
\pgfsetstrokecolor{currentstroke}%
\pgfsetdash{}{0pt}%
\pgfpathmoveto{\pgfqpoint{4.239257in}{4.489420in}}%
\pgfpathlineto{\pgfqpoint{4.161473in}{4.598110in}}%
\pgfpathlineto{\pgfqpoint{4.293076in}{4.516692in}}%
\pgfpathclose%
\pgfusepath{fill}%
\end{pgfscope}%
\begin{pgfscope}%
\pgfpathrectangle{\pgfqpoint{0.539299in}{0.078740in}}{\pgfqpoint{7.842520in}{7.842520in}}%
\pgfusepath{clip}%
\pgfsetbuttcap%
\pgfsetroundjoin%
\definecolor{currentfill}{rgb}{0.277941,0.056324,0.381191}%
\pgfsetfillcolor{currentfill}%
\pgfsetlinewidth{0.000000pt}%
\definecolor{currentstroke}{rgb}{0.203063,0.379716,0.553925}%
\pgfsetstrokecolor{currentstroke}%
\pgfsetdash{}{0pt}%
\pgfpathmoveto{\pgfqpoint{5.633500in}{3.678097in}}%
\pgfpathlineto{\pgfqpoint{5.572472in}{3.714282in}}%
\pgfpathlineto{\pgfqpoint{5.500041in}{3.716527in}}%
\pgfpathclose%
\pgfusepath{fill}%
\end{pgfscope}%
\begin{pgfscope}%
\pgfpathrectangle{\pgfqpoint{0.539299in}{0.078740in}}{\pgfqpoint{7.842520in}{7.842520in}}%
\pgfusepath{clip}%
\pgfsetbuttcap%
\pgfsetroundjoin%
\definecolor{currentfill}{rgb}{0.281446,0.084320,0.407414}%
\pgfsetfillcolor{currentfill}%
\pgfsetlinewidth{0.000000pt}%
\definecolor{currentstroke}{rgb}{0.201239,0.383670,0.554294}%
\pgfsetstrokecolor{currentstroke}%
\pgfsetdash{}{0pt}%
\pgfpathmoveto{\pgfqpoint{5.293876in}{3.778414in}}%
\pgfpathlineto{\pgfqpoint{5.367069in}{3.761390in}}%
\pgfpathlineto{\pgfqpoint{5.234563in}{3.814537in}}%
\pgfpathclose%
\pgfusepath{fill}%
\end{pgfscope}%
\begin{pgfscope}%
\pgfpathrectangle{\pgfqpoint{0.539299in}{0.078740in}}{\pgfqpoint{7.842520in}{7.842520in}}%
\pgfusepath{clip}%
\pgfsetbuttcap%
\pgfsetroundjoin%
\definecolor{currentfill}{rgb}{0.283091,0.110553,0.431554}%
\pgfsetfillcolor{currentfill}%
\pgfsetlinewidth{0.000000pt}%
\definecolor{currentstroke}{rgb}{0.199430,0.387607,0.554642}%
\pgfsetstrokecolor{currentstroke}%
\pgfsetdash{}{0pt}%
\pgfpathmoveto{\pgfqpoint{5.028622in}{3.919505in}}%
\pgfpathlineto{\pgfqpoint{5.161052in}{3.843033in}}%
\pgfpathlineto{\pgfqpoint{5.234563in}{3.814537in}}%
\pgfpathclose%
\pgfusepath{fill}%
\end{pgfscope}%
\begin{pgfscope}%
\pgfpathrectangle{\pgfqpoint{0.539299in}{0.078740in}}{\pgfqpoint{7.842520in}{7.842520in}}%
\pgfusepath{clip}%
\pgfsetbuttcap%
\pgfsetroundjoin%
\definecolor{currentfill}{rgb}{0.267968,0.223549,0.512008}%
\pgfsetfillcolor{currentfill}%
\pgfsetlinewidth{0.000000pt}%
\definecolor{currentstroke}{rgb}{0.197636,0.391528,0.554969}%
\pgfsetstrokecolor{currentstroke}%
\pgfsetdash{}{0pt}%
\pgfpathmoveto{\pgfqpoint{4.764671in}{4.104606in}}%
\pgfpathlineto{\pgfqpoint{4.633026in}{4.207547in}}%
\pgfpathlineto{\pgfqpoint{4.557051in}{4.301638in}}%
\pgfpathclose%
\pgfusepath{fill}%
\end{pgfscope}%
\begin{pgfscope}%
\pgfpathrectangle{\pgfqpoint{0.539299in}{0.078740in}}{\pgfqpoint{7.842520in}{7.842520in}}%
\pgfusepath{clip}%
\pgfsetbuttcap%
\pgfsetroundjoin%
\definecolor{currentfill}{rgb}{0.258965,0.251537,0.524736}%
\pgfsetfillcolor{currentfill}%
\pgfsetlinewidth{0.000000pt}%
\definecolor{currentstroke}{rgb}{0.195860,0.395433,0.555276}%
\pgfsetstrokecolor{currentstroke}%
\pgfsetdash{}{0pt}%
\pgfpathmoveto{\pgfqpoint{3.279998in}{4.210712in}}%
\pgfpathlineto{\pgfqpoint{3.200055in}{4.172995in}}%
\pgfpathlineto{\pgfqpoint{3.323758in}{4.491822in}}%
\pgfpathclose%
\pgfusepath{fill}%
\end{pgfscope}%
\begin{pgfscope}%
\pgfpathrectangle{\pgfqpoint{0.539299in}{0.078740in}}{\pgfqpoint{7.842520in}{7.842520in}}%
\pgfusepath{clip}%
\pgfsetbuttcap%
\pgfsetroundjoin%
\definecolor{currentfill}{rgb}{0.212395,0.359683,0.551710}%
\pgfsetfillcolor{currentfill}%
\pgfsetlinewidth{0.000000pt}%
\definecolor{currentstroke}{rgb}{0.194100,0.399323,0.555565}%
\pgfsetstrokecolor{currentstroke}%
\pgfsetdash{}{0pt}%
\pgfpathmoveto{\pgfqpoint{3.951415in}{4.734438in}}%
\pgfpathlineto{\pgfqpoint{4.030348in}{4.648062in}}%
\pgfpathlineto{\pgfqpoint{3.899971in}{4.655078in}}%
\pgfpathclose%
\pgfusepath{fill}%
\end{pgfscope}%
\begin{pgfscope}%
\pgfpathrectangle{\pgfqpoint{0.539299in}{0.078740in}}{\pgfqpoint{7.842520in}{7.842520in}}%
\pgfusepath{clip}%
\pgfsetbuttcap%
\pgfsetroundjoin%
\definecolor{currentfill}{rgb}{0.276022,0.044167,0.370164}%
\pgfsetfillcolor{currentfill}%
\pgfsetlinewidth{0.000000pt}%
\definecolor{currentstroke}{rgb}{0.192357,0.403199,0.555836}%
\pgfsetstrokecolor{currentstroke}%
\pgfsetdash{}{0pt}%
\pgfpathmoveto{\pgfqpoint{5.767446in}{3.643788in}}%
\pgfpathlineto{\pgfqpoint{5.839174in}{3.644098in}}%
\pgfpathlineto{\pgfqpoint{5.705595in}{3.678274in}}%
\pgfpathclose%
\pgfusepath{fill}%
\end{pgfscope}%
\begin{pgfscope}%
\pgfpathrectangle{\pgfqpoint{0.539299in}{0.078740in}}{\pgfqpoint{7.842520in}{7.842520in}}%
\pgfusepath{clip}%
\pgfsetbuttcap%
\pgfsetroundjoin%
\definecolor{currentfill}{rgb}{0.282910,0.105393,0.426902}%
\pgfsetfillcolor{currentfill}%
\pgfsetlinewidth{0.000000pt}%
\definecolor{currentstroke}{rgb}{0.190631,0.407061,0.556089}%
\pgfsetstrokecolor{currentstroke}%
\pgfsetdash{}{0pt}%
\pgfpathmoveto{\pgfqpoint{5.234563in}{3.814537in}}%
\pgfpathlineto{\pgfqpoint{5.161052in}{3.843033in}}%
\pgfpathlineto{\pgfqpoint{5.293876in}{3.778414in}}%
\pgfpathclose%
\pgfusepath{fill}%
\end{pgfscope}%
\begin{pgfscope}%
\pgfpathrectangle{\pgfqpoint{0.539299in}{0.078740in}}{\pgfqpoint{7.842520in}{7.842520in}}%
\pgfusepath{clip}%
\pgfsetbuttcap%
\pgfsetroundjoin%
\definecolor{currentfill}{rgb}{0.243113,0.292092,0.538516}%
\pgfsetfillcolor{currentfill}%
\pgfsetlinewidth{0.000000pt}%
\definecolor{currentstroke}{rgb}{0.188923,0.410910,0.556326}%
\pgfsetstrokecolor{currentstroke}%
\pgfsetdash{}{0pt}%
\pgfpathmoveto{\pgfqpoint{4.293076in}{4.516692in}}%
\pgfpathlineto{\pgfqpoint{4.424973in}{4.414618in}}%
\pgfpathlineto{\pgfqpoint{4.501556in}{4.310839in}}%
\pgfpathclose%
\pgfusepath{fill}%
\end{pgfscope}%
\begin{pgfscope}%
\pgfpathrectangle{\pgfqpoint{0.539299in}{0.078740in}}{\pgfqpoint{7.842520in}{7.842520in}}%
\pgfusepath{clip}%
\pgfsetbuttcap%
\pgfsetroundjoin%
\definecolor{currentfill}{rgb}{0.277134,0.185228,0.489898}%
\pgfsetfillcolor{currentfill}%
\pgfsetlinewidth{0.000000pt}%
\definecolor{currentstroke}{rgb}{0.187231,0.414746,0.556547}%
\pgfsetstrokecolor{currentstroke}%
\pgfsetdash{}{0pt}%
\pgfpathmoveto{\pgfqpoint{4.764671in}{4.104606in}}%
\pgfpathlineto{\pgfqpoint{4.821630in}{4.074629in}}%
\pgfpathlineto{\pgfqpoint{4.896518in}{4.007367in}}%
\pgfpathclose%
\pgfusepath{fill}%
\end{pgfscope}%
\begin{pgfscope}%
\pgfpathrectangle{\pgfqpoint{0.539299in}{0.078740in}}{\pgfqpoint{7.842520in}{7.842520in}}%
\pgfusepath{clip}%
\pgfsetbuttcap%
\pgfsetroundjoin%
\definecolor{currentfill}{rgb}{0.280894,0.078907,0.402329}%
\pgfsetfillcolor{currentfill}%
\pgfsetlinewidth{0.000000pt}%
\definecolor{currentstroke}{rgb}{0.185556,0.418570,0.556753}%
\pgfsetstrokecolor{currentstroke}%
\pgfsetdash{}{0pt}%
\pgfpathmoveto{\pgfqpoint{5.500041in}{3.716527in}}%
\pgfpathlineto{\pgfqpoint{5.367069in}{3.761390in}}%
\pgfpathlineto{\pgfqpoint{5.427151in}{3.724780in}}%
\pgfpathclose%
\pgfusepath{fill}%
\end{pgfscope}%
\begin{pgfscope}%
\pgfpathrectangle{\pgfqpoint{0.539299in}{0.078740in}}{\pgfqpoint{7.842520in}{7.842520in}}%
\pgfusepath{clip}%
\pgfsetbuttcap%
\pgfsetroundjoin%
\definecolor{currentfill}{rgb}{0.223925,0.334994,0.548053}%
\pgfsetfillcolor{currentfill}%
\pgfsetlinewidth{0.000000pt}%
\definecolor{currentstroke}{rgb}{0.183898,0.422383,0.556944}%
\pgfsetstrokecolor{currentstroke}%
\pgfsetdash{}{0pt}%
\pgfpathmoveto{\pgfqpoint{3.484281in}{4.506457in}}%
\pgfpathlineto{\pgfqpoint{3.404261in}{4.501552in}}%
\pgfpathlineto{\pgfqpoint{3.611618in}{4.681644in}}%
\pgfpathclose%
\pgfusepath{fill}%
\end{pgfscope}%
\begin{pgfscope}%
\pgfpathrectangle{\pgfqpoint{0.539299in}{0.078740in}}{\pgfqpoint{7.842520in}{7.842520in}}%
\pgfusepath{clip}%
\pgfsetbuttcap%
\pgfsetroundjoin%
\definecolor{currentfill}{rgb}{0.243113,0.292092,0.538516}%
\pgfsetfillcolor{currentfill}%
\pgfsetlinewidth{0.000000pt}%
\definecolor{currentstroke}{rgb}{0.182256,0.426184,0.557120}%
\pgfsetstrokecolor{currentstroke}%
\pgfsetdash{}{0pt}%
\pgfpathmoveto{\pgfqpoint{3.323758in}{4.491822in}}%
\pgfpathlineto{\pgfqpoint{3.404261in}{4.501552in}}%
\pgfpathlineto{\pgfqpoint{3.279998in}{4.210712in}}%
\pgfpathclose%
\pgfusepath{fill}%
\end{pgfscope}%
\begin{pgfscope}%
\pgfpathrectangle{\pgfqpoint{0.539299in}{0.078740in}}{\pgfqpoint{7.842520in}{7.842520in}}%
\pgfusepath{clip}%
\pgfsetbuttcap%
\pgfsetroundjoin%
\definecolor{currentfill}{rgb}{0.277941,0.056324,0.381191}%
\pgfsetfillcolor{currentfill}%
\pgfsetlinewidth{0.000000pt}%
\definecolor{currentstroke}{rgb}{0.180629,0.429975,0.557282}%
\pgfsetstrokecolor{currentstroke}%
\pgfsetdash{}{0pt}%
\pgfpathmoveto{\pgfqpoint{5.705595in}{3.678274in}}%
\pgfpathlineto{\pgfqpoint{5.633500in}{3.678097in}}%
\pgfpathlineto{\pgfqpoint{5.767446in}{3.643788in}}%
\pgfpathclose%
\pgfusepath{fill}%
\end{pgfscope}%
\begin{pgfscope}%
\pgfpathrectangle{\pgfqpoint{0.539299in}{0.078740in}}{\pgfqpoint{7.842520in}{7.842520in}}%
\pgfusepath{clip}%
\pgfsetbuttcap%
\pgfsetroundjoin%
\definecolor{currentfill}{rgb}{0.253935,0.265254,0.529983}%
\pgfsetfillcolor{currentfill}%
\pgfsetlinewidth{0.000000pt}%
\definecolor{currentstroke}{rgb}{0.179019,0.433756,0.557430}%
\pgfsetstrokecolor{currentstroke}%
\pgfsetdash{}{0pt}%
\pgfpathmoveto{\pgfqpoint{4.633026in}{4.207547in}}%
\pgfpathlineto{\pgfqpoint{4.424973in}{4.414618in}}%
\pgfpathlineto{\pgfqpoint{4.557051in}{4.301638in}}%
\pgfpathclose%
\pgfusepath{fill}%
\end{pgfscope}%
\begin{pgfscope}%
\pgfpathrectangle{\pgfqpoint{0.539299in}{0.078740in}}{\pgfqpoint{7.842520in}{7.842520in}}%
\pgfusepath{clip}%
\pgfsetbuttcap%
\pgfsetroundjoin%
\definecolor{currentfill}{rgb}{0.206756,0.371758,0.553117}%
\pgfsetfillcolor{currentfill}%
\pgfsetlinewidth{0.000000pt}%
\definecolor{currentstroke}{rgb}{0.177423,0.437527,0.557565}%
\pgfsetstrokecolor{currentstroke}%
\pgfsetdash{}{0pt}%
\pgfpathmoveto{\pgfqpoint{3.899971in}{4.655078in}}%
\pgfpathlineto{\pgfqpoint{3.820757in}{4.723273in}}%
\pgfpathlineto{\pgfqpoint{3.951415in}{4.734438in}}%
\pgfpathclose%
\pgfusepath{fill}%
\end{pgfscope}%
\begin{pgfscope}%
\pgfpathrectangle{\pgfqpoint{0.539299in}{0.078740in}}{\pgfqpoint{7.842520in}{7.842520in}}%
\pgfusepath{clip}%
\pgfsetbuttcap%
\pgfsetroundjoin%
\definecolor{currentfill}{rgb}{0.208623,0.367752,0.552675}%
\pgfsetfillcolor{currentfill}%
\pgfsetlinewidth{0.000000pt}%
\definecolor{currentstroke}{rgb}{0.175841,0.441290,0.557685}%
\pgfsetstrokecolor{currentstroke}%
\pgfsetdash{}{0pt}%
\pgfpathmoveto{\pgfqpoint{3.611618in}{4.681644in}}%
\pgfpathlineto{\pgfqpoint{3.820757in}{4.723273in}}%
\pgfpathlineto{\pgfqpoint{3.691406in}{4.650385in}}%
\pgfpathclose%
\pgfusepath{fill}%
\end{pgfscope}%
\begin{pgfscope}%
\pgfpathrectangle{\pgfqpoint{0.539299in}{0.078740in}}{\pgfqpoint{7.842520in}{7.842520in}}%
\pgfusepath{clip}%
\pgfsetbuttcap%
\pgfsetroundjoin%
\definecolor{currentfill}{rgb}{0.279574,0.170599,0.479997}%
\pgfsetfillcolor{currentfill}%
\pgfsetlinewidth{0.000000pt}%
\definecolor{currentstroke}{rgb}{0.174274,0.445044,0.557792}%
\pgfsetstrokecolor{currentstroke}%
\pgfsetdash{}{0pt}%
\pgfpathmoveto{\pgfqpoint{4.821630in}{4.074629in}}%
\pgfpathlineto{\pgfqpoint{5.028622in}{3.919505in}}%
\pgfpathlineto{\pgfqpoint{4.896518in}{4.007367in}}%
\pgfpathclose%
\pgfusepath{fill}%
\end{pgfscope}%
\begin{pgfscope}%
\pgfpathrectangle{\pgfqpoint{0.539299in}{0.078740in}}{\pgfqpoint{7.842520in}{7.842520in}}%
\pgfusepath{clip}%
\pgfsetbuttcap%
\pgfsetroundjoin%
\definecolor{currentfill}{rgb}{0.277018,0.050344,0.375715}%
\pgfsetfillcolor{currentfill}%
\pgfsetlinewidth{0.000000pt}%
\definecolor{currentstroke}{rgb}{0.172719,0.448791,0.557885}%
\pgfsetstrokecolor{currentstroke}%
\pgfsetdash{}{0pt}%
\pgfpathmoveto{\pgfqpoint{5.973195in}{3.610242in}}%
\pgfpathlineto{\pgfqpoint{5.839174in}{3.644098in}}%
\pgfpathlineto{\pgfqpoint{5.767446in}{3.643788in}}%
\pgfpathclose%
\pgfusepath{fill}%
\end{pgfscope}%
\begin{pgfscope}%
\pgfpathrectangle{\pgfqpoint{0.539299in}{0.078740in}}{\pgfqpoint{7.842520in}{7.842520in}}%
\pgfusepath{clip}%
\pgfsetbuttcap%
\pgfsetroundjoin%
\definecolor{currentfill}{rgb}{0.265145,0.232956,0.516599}%
\pgfsetfillcolor{currentfill}%
\pgfsetlinewidth{0.000000pt}%
\definecolor{currentstroke}{rgb}{0.171176,0.452530,0.557965}%
\pgfsetstrokecolor{currentstroke}%
\pgfsetdash{}{0pt}%
\pgfpathmoveto{\pgfqpoint{4.557051in}{4.301638in}}%
\pgfpathlineto{\pgfqpoint{4.689266in}{4.186089in}}%
\pgfpathlineto{\pgfqpoint{4.764671in}{4.104606in}}%
\pgfpathclose%
\pgfusepath{fill}%
\end{pgfscope}%
\begin{pgfscope}%
\pgfpathrectangle{\pgfqpoint{0.539299in}{0.078740in}}{\pgfqpoint{7.842520in}{7.842520in}}%
\pgfusepath{clip}%
\pgfsetbuttcap%
\pgfsetroundjoin%
\definecolor{currentfill}{rgb}{0.282327,0.094955,0.417331}%
\pgfsetfillcolor{currentfill}%
\pgfsetlinewidth{0.000000pt}%
\definecolor{currentstroke}{rgb}{0.169646,0.456262,0.558030}%
\pgfsetstrokecolor{currentstroke}%
\pgfsetdash{}{0pt}%
\pgfpathmoveto{\pgfqpoint{5.367069in}{3.761390in}}%
\pgfpathlineto{\pgfqpoint{5.293876in}{3.778414in}}%
\pgfpathlineto{\pgfqpoint{5.427151in}{3.724780in}}%
\pgfpathclose%
\pgfusepath{fill}%
\end{pgfscope}%
\begin{pgfscope}%
\pgfpathrectangle{\pgfqpoint{0.539299in}{0.078740in}}{\pgfqpoint{7.842520in}{7.842520in}}%
\pgfusepath{clip}%
\pgfsetbuttcap%
\pgfsetroundjoin%
\definecolor{currentfill}{rgb}{0.210503,0.363727,0.552206}%
\pgfsetfillcolor{currentfill}%
\pgfsetlinewidth{0.000000pt}%
\definecolor{currentstroke}{rgb}{0.168126,0.459988,0.558082}%
\pgfsetstrokecolor{currentstroke}%
\pgfsetdash{}{0pt}%
\pgfpathmoveto{\pgfqpoint{4.082979in}{4.695694in}}%
\pgfpathlineto{\pgfqpoint{4.161473in}{4.598110in}}%
\pgfpathlineto{\pgfqpoint{4.030348in}{4.648062in}}%
\pgfpathclose%
\pgfusepath{fill}%
\end{pgfscope}%
\begin{pgfscope}%
\pgfpathrectangle{\pgfqpoint{0.539299in}{0.078740in}}{\pgfqpoint{7.842520in}{7.842520in}}%
\pgfusepath{clip}%
\pgfsetbuttcap%
\pgfsetroundjoin%
\definecolor{currentfill}{rgb}{0.280894,0.078907,0.402329}%
\pgfsetfillcolor{currentfill}%
\pgfsetlinewidth{0.000000pt}%
\definecolor{currentstroke}{rgb}{0.166617,0.463708,0.558119}%
\pgfsetstrokecolor{currentstroke}%
\pgfsetdash{}{0pt}%
\pgfpathmoveto{\pgfqpoint{5.427151in}{3.724780in}}%
\pgfpathlineto{\pgfqpoint{5.633500in}{3.678097in}}%
\pgfpathlineto{\pgfqpoint{5.500041in}{3.716527in}}%
\pgfpathclose%
\pgfusepath{fill}%
\end{pgfscope}%
\begin{pgfscope}%
\pgfpathrectangle{\pgfqpoint{0.539299in}{0.078740in}}{\pgfqpoint{7.842520in}{7.842520in}}%
\pgfusepath{clip}%
\pgfsetbuttcap%
\pgfsetroundjoin%
\definecolor{currentfill}{rgb}{0.270595,0.214069,0.507052}%
\pgfsetfillcolor{currentfill}%
\pgfsetlinewidth{0.000000pt}%
\definecolor{currentstroke}{rgb}{0.165117,0.467423,0.558141}%
\pgfsetstrokecolor{currentstroke}%
\pgfsetdash{}{0pt}%
\pgfpathmoveto{\pgfqpoint{4.689266in}{4.186089in}}%
\pgfpathlineto{\pgfqpoint{4.821630in}{4.074629in}}%
\pgfpathlineto{\pgfqpoint{4.764671in}{4.104606in}}%
\pgfpathclose%
\pgfusepath{fill}%
\end{pgfscope}%
\begin{pgfscope}%
\pgfpathrectangle{\pgfqpoint{0.539299in}{0.078740in}}{\pgfqpoint{7.842520in}{7.842520in}}%
\pgfusepath{clip}%
\pgfsetbuttcap%
\pgfsetroundjoin%
\definecolor{currentfill}{rgb}{0.282623,0.140926,0.457517}%
\pgfsetfillcolor{currentfill}%
\pgfsetlinewidth{0.000000pt}%
\definecolor{currentstroke}{rgb}{0.163625,0.471133,0.558148}%
\pgfsetstrokecolor{currentstroke}%
\pgfsetdash{}{0pt}%
\pgfpathmoveto{\pgfqpoint{5.087019in}{3.881455in}}%
\pgfpathlineto{\pgfqpoint{5.161052in}{3.843033in}}%
\pgfpathlineto{\pgfqpoint{5.028622in}{3.919505in}}%
\pgfpathclose%
\pgfusepath{fill}%
\end{pgfscope}%
\begin{pgfscope}%
\pgfpathrectangle{\pgfqpoint{0.539299in}{0.078740in}}{\pgfqpoint{7.842520in}{7.842520in}}%
\pgfusepath{clip}%
\pgfsetbuttcap%
\pgfsetroundjoin%
\definecolor{currentfill}{rgb}{0.214298,0.355619,0.551184}%
\pgfsetfillcolor{currentfill}%
\pgfsetlinewidth{0.000000pt}%
\definecolor{currentstroke}{rgb}{0.162142,0.474838,0.558140}%
\pgfsetstrokecolor{currentstroke}%
\pgfsetdash{}{0pt}%
\pgfpathmoveto{\pgfqpoint{4.082979in}{4.695694in}}%
\pgfpathlineto{\pgfqpoint{4.293076in}{4.516692in}}%
\pgfpathlineto{\pgfqpoint{4.161473in}{4.598110in}}%
\pgfpathclose%
\pgfusepath{fill}%
\end{pgfscope}%
\begin{pgfscope}%
\pgfpathrectangle{\pgfqpoint{0.539299in}{0.078740in}}{\pgfqpoint{7.842520in}{7.842520in}}%
\pgfusepath{clip}%
\pgfsetbuttcap%
\pgfsetroundjoin%
\definecolor{currentfill}{rgb}{0.278012,0.180367,0.486697}%
\pgfsetfillcolor{currentfill}%
\pgfsetlinewidth{0.000000pt}%
\definecolor{currentstroke}{rgb}{0.160665,0.478540,0.558115}%
\pgfsetstrokecolor{currentstroke}%
\pgfsetdash{}{0pt}%
\pgfpathmoveto{\pgfqpoint{4.954191in}{3.972092in}}%
\pgfpathlineto{\pgfqpoint{5.028622in}{3.919505in}}%
\pgfpathlineto{\pgfqpoint{4.821630in}{4.074629in}}%
\pgfpathclose%
\pgfusepath{fill}%
\end{pgfscope}%
\begin{pgfscope}%
\pgfpathrectangle{\pgfqpoint{0.539299in}{0.078740in}}{\pgfqpoint{7.842520in}{7.842520in}}%
\pgfusepath{clip}%
\pgfsetbuttcap%
\pgfsetroundjoin%
\definecolor{currentfill}{rgb}{0.203063,0.379716,0.553925}%
\pgfsetfillcolor{currentfill}%
\pgfsetlinewidth{0.000000pt}%
\definecolor{currentstroke}{rgb}{0.159194,0.482237,0.558073}%
\pgfsetstrokecolor{currentstroke}%
\pgfsetdash{}{0pt}%
\pgfpathmoveto{\pgfqpoint{4.082979in}{4.695694in}}%
\pgfpathlineto{\pgfqpoint{4.030348in}{4.648062in}}%
\pgfpathlineto{\pgfqpoint{3.951415in}{4.734438in}}%
\pgfpathclose%
\pgfusepath{fill}%
\end{pgfscope}%
\begin{pgfscope}%
\pgfpathrectangle{\pgfqpoint{0.539299in}{0.078740in}}{\pgfqpoint{7.842520in}{7.842520in}}%
\pgfusepath{clip}%
\pgfsetbuttcap%
\pgfsetroundjoin%
\definecolor{currentfill}{rgb}{0.277018,0.050344,0.375715}%
\pgfsetfillcolor{currentfill}%
\pgfsetlinewidth{0.000000pt}%
\definecolor{currentstroke}{rgb}{0.157729,0.485932,0.558013}%
\pgfsetstrokecolor{currentstroke}%
\pgfsetdash{}{0pt}%
\pgfpathmoveto{\pgfqpoint{6.107631in}{3.575340in}}%
\pgfpathlineto{\pgfqpoint{5.973195in}{3.610242in}}%
\pgfpathlineto{\pgfqpoint{5.901860in}{3.611134in}}%
\pgfpathclose%
\pgfusepath{fill}%
\end{pgfscope}%
\begin{pgfscope}%
\pgfpathrectangle{\pgfqpoint{0.539299in}{0.078740in}}{\pgfqpoint{7.842520in}{7.842520in}}%
\pgfusepath{clip}%
\pgfsetbuttcap%
\pgfsetroundjoin%
\definecolor{currentfill}{rgb}{0.253935,0.265254,0.529983}%
\pgfsetfillcolor{currentfill}%
\pgfsetlinewidth{0.000000pt}%
\definecolor{currentstroke}{rgb}{0.156270,0.489624,0.557936}%
\pgfsetstrokecolor{currentstroke}%
\pgfsetdash{}{0pt}%
\pgfpathmoveto{\pgfqpoint{3.200055in}{4.172995in}}%
\pgfpathlineto{\pgfqpoint{3.119658in}{4.133109in}}%
\pgfpathlineto{\pgfqpoint{3.242761in}{4.478380in}}%
\pgfpathclose%
\pgfusepath{fill}%
\end{pgfscope}%
\begin{pgfscope}%
\pgfpathrectangle{\pgfqpoint{0.539299in}{0.078740in}}{\pgfqpoint{7.842520in}{7.842520in}}%
\pgfusepath{clip}%
\pgfsetbuttcap%
\pgfsetroundjoin%
\definecolor{currentfill}{rgb}{0.278791,0.062145,0.386592}%
\pgfsetfillcolor{currentfill}%
\pgfsetlinewidth{0.000000pt}%
\definecolor{currentstroke}{rgb}{0.154815,0.493313,0.557840}%
\pgfsetstrokecolor{currentstroke}%
\pgfsetdash{}{0pt}%
\pgfpathmoveto{\pgfqpoint{5.901860in}{3.611134in}}%
\pgfpathlineto{\pgfqpoint{5.973195in}{3.610242in}}%
\pgfpathlineto{\pgfqpoint{5.767446in}{3.643788in}}%
\pgfpathclose%
\pgfusepath{fill}%
\end{pgfscope}%
\begin{pgfscope}%
\pgfpathrectangle{\pgfqpoint{0.539299in}{0.078740in}}{\pgfqpoint{7.842520in}{7.842520in}}%
\pgfusepath{clip}%
\pgfsetbuttcap%
\pgfsetroundjoin%
\definecolor{currentfill}{rgb}{0.283187,0.125848,0.444960}%
\pgfsetfillcolor{currentfill}%
\pgfsetlinewidth{0.000000pt}%
\definecolor{currentstroke}{rgb}{0.153364,0.497000,0.557724}%
\pgfsetstrokecolor{currentstroke}%
\pgfsetdash{}{0pt}%
\pgfpathmoveto{\pgfqpoint{5.220192in}{3.803938in}}%
\pgfpathlineto{\pgfqpoint{5.293876in}{3.778414in}}%
\pgfpathlineto{\pgfqpoint{5.161052in}{3.843033in}}%
\pgfpathclose%
\pgfusepath{fill}%
\end{pgfscope}%
\begin{pgfscope}%
\pgfpathrectangle{\pgfqpoint{0.539299in}{0.078740in}}{\pgfqpoint{7.842520in}{7.842520in}}%
\pgfusepath{clip}%
\pgfsetbuttcap%
\pgfsetroundjoin%
\definecolor{currentfill}{rgb}{0.280255,0.165693,0.476498}%
\pgfsetfillcolor{currentfill}%
\pgfsetlinewidth{0.000000pt}%
\definecolor{currentstroke}{rgb}{0.151918,0.500685,0.557587}%
\pgfsetstrokecolor{currentstroke}%
\pgfsetdash{}{0pt}%
\pgfpathmoveto{\pgfqpoint{5.028622in}{3.919505in}}%
\pgfpathlineto{\pgfqpoint{4.954191in}{3.972092in}}%
\pgfpathlineto{\pgfqpoint{5.087019in}{3.881455in}}%
\pgfpathclose%
\pgfusepath{fill}%
\end{pgfscope}%
\begin{pgfscope}%
\pgfpathrectangle{\pgfqpoint{0.539299in}{0.078740in}}{\pgfqpoint{7.842520in}{7.842520in}}%
\pgfusepath{clip}%
\pgfsetbuttcap%
\pgfsetroundjoin%
\definecolor{currentfill}{rgb}{0.280894,0.078907,0.402329}%
\pgfsetfillcolor{currentfill}%
\pgfsetlinewidth{0.000000pt}%
\definecolor{currentstroke}{rgb}{0.150476,0.504369,0.557430}%
\pgfsetstrokecolor{currentstroke}%
\pgfsetdash{}{0pt}%
\pgfpathmoveto{\pgfqpoint{5.560913in}{3.680214in}}%
\pgfpathlineto{\pgfqpoint{5.767446in}{3.643788in}}%
\pgfpathlineto{\pgfqpoint{5.633500in}{3.678097in}}%
\pgfpathclose%
\pgfusepath{fill}%
\end{pgfscope}%
\begin{pgfscope}%
\pgfpathrectangle{\pgfqpoint{0.539299in}{0.078740in}}{\pgfqpoint{7.842520in}{7.842520in}}%
\pgfusepath{clip}%
\pgfsetbuttcap%
\pgfsetroundjoin%
\definecolor{currentfill}{rgb}{0.195860,0.395433,0.555276}%
\pgfsetfillcolor{currentfill}%
\pgfsetlinewidth{0.000000pt}%
\definecolor{currentstroke}{rgb}{0.149039,0.508051,0.557250}%
\pgfsetstrokecolor{currentstroke}%
\pgfsetdash{}{0pt}%
\pgfpathmoveto{\pgfqpoint{3.740970in}{4.779433in}}%
\pgfpathlineto{\pgfqpoint{3.820757in}{4.723273in}}%
\pgfpathlineto{\pgfqpoint{3.611618in}{4.681644in}}%
\pgfpathclose%
\pgfusepath{fill}%
\end{pgfscope}%
\begin{pgfscope}%
\pgfpathrectangle{\pgfqpoint{0.539299in}{0.078740in}}{\pgfqpoint{7.842520in}{7.842520in}}%
\pgfusepath{clip}%
\pgfsetbuttcap%
\pgfsetroundjoin%
\definecolor{currentfill}{rgb}{0.281924,0.089666,0.412415}%
\pgfsetfillcolor{currentfill}%
\pgfsetlinewidth{0.000000pt}%
\definecolor{currentstroke}{rgb}{0.147607,0.511733,0.557049}%
\pgfsetstrokecolor{currentstroke}%
\pgfsetdash{}{0pt}%
\pgfpathmoveto{\pgfqpoint{5.427151in}{3.724780in}}%
\pgfpathlineto{\pgfqpoint{5.560913in}{3.680214in}}%
\pgfpathlineto{\pgfqpoint{5.633500in}{3.678097in}}%
\pgfpathclose%
\pgfusepath{fill}%
\end{pgfscope}%
\begin{pgfscope}%
\pgfpathrectangle{\pgfqpoint{0.539299in}{0.078740in}}{\pgfqpoint{7.842520in}{7.842520in}}%
\pgfusepath{clip}%
\pgfsetbuttcap%
\pgfsetroundjoin%
\definecolor{currentfill}{rgb}{0.225863,0.330805,0.547314}%
\pgfsetfillcolor{currentfill}%
\pgfsetlinewidth{0.000000pt}%
\definecolor{currentstroke}{rgb}{0.146180,0.515413,0.556823}%
\pgfsetstrokecolor{currentstroke}%
\pgfsetdash{}{0pt}%
\pgfpathmoveto{\pgfqpoint{4.347630in}{4.516065in}}%
\pgfpathlineto{\pgfqpoint{4.424973in}{4.414618in}}%
\pgfpathlineto{\pgfqpoint{4.293076in}{4.516692in}}%
\pgfpathclose%
\pgfusepath{fill}%
\end{pgfscope}%
\begin{pgfscope}%
\pgfpathrectangle{\pgfqpoint{0.539299in}{0.078740in}}{\pgfqpoint{7.842520in}{7.842520in}}%
\pgfusepath{clip}%
\pgfsetbuttcap%
\pgfsetroundjoin%
\definecolor{currentfill}{rgb}{0.206756,0.371758,0.553117}%
\pgfsetfillcolor{currentfill}%
\pgfsetlinewidth{0.000000pt}%
\definecolor{currentstroke}{rgb}{0.144759,0.519093,0.556572}%
\pgfsetstrokecolor{currentstroke}%
\pgfsetdash{}{0pt}%
\pgfpathmoveto{\pgfqpoint{3.404261in}{4.501552in}}%
\pgfpathlineto{\pgfqpoint{3.531309in}{4.704589in}}%
\pgfpathlineto{\pgfqpoint{3.611618in}{4.681644in}}%
\pgfpathclose%
\pgfusepath{fill}%
\end{pgfscope}%
\begin{pgfscope}%
\pgfpathrectangle{\pgfqpoint{0.539299in}{0.078740in}}{\pgfqpoint{7.842520in}{7.842520in}}%
\pgfusepath{clip}%
\pgfsetbuttcap%
\pgfsetroundjoin%
\definecolor{currentfill}{rgb}{0.283229,0.120777,0.440584}%
\pgfsetfillcolor{currentfill}%
\pgfsetlinewidth{0.000000pt}%
\definecolor{currentstroke}{rgb}{0.143343,0.522773,0.556295}%
\pgfsetstrokecolor{currentstroke}%
\pgfsetdash{}{0pt}%
\pgfpathmoveto{\pgfqpoint{5.220192in}{3.803938in}}%
\pgfpathlineto{\pgfqpoint{5.427151in}{3.724780in}}%
\pgfpathlineto{\pgfqpoint{5.293876in}{3.778414in}}%
\pgfpathclose%
\pgfusepath{fill}%
\end{pgfscope}%
\begin{pgfscope}%
\pgfpathrectangle{\pgfqpoint{0.539299in}{0.078740in}}{\pgfqpoint{7.842520in}{7.842520in}}%
\pgfusepath{clip}%
\pgfsetbuttcap%
\pgfsetroundjoin%
\definecolor{currentfill}{rgb}{0.237441,0.305202,0.541921}%
\pgfsetfillcolor{currentfill}%
\pgfsetlinewidth{0.000000pt}%
\definecolor{currentstroke}{rgb}{0.141935,0.526453,0.555991}%
\pgfsetstrokecolor{currentstroke}%
\pgfsetdash{}{0pt}%
\pgfpathmoveto{\pgfqpoint{3.323758in}{4.491822in}}%
\pgfpathlineto{\pgfqpoint{3.200055in}{4.172995in}}%
\pgfpathlineto{\pgfqpoint{3.242761in}{4.478380in}}%
\pgfpathclose%
\pgfusepath{fill}%
\end{pgfscope}%
\begin{pgfscope}%
\pgfpathrectangle{\pgfqpoint{0.539299in}{0.078740in}}{\pgfqpoint{7.842520in}{7.842520in}}%
\pgfusepath{clip}%
\pgfsetbuttcap%
\pgfsetroundjoin%
\definecolor{currentfill}{rgb}{0.237441,0.305202,0.541921}%
\pgfsetfillcolor{currentfill}%
\pgfsetlinewidth{0.000000pt}%
\definecolor{currentstroke}{rgb}{0.140536,0.530132,0.555659}%
\pgfsetstrokecolor{currentstroke}%
\pgfsetdash{}{0pt}%
\pgfpathmoveto{\pgfqpoint{4.557051in}{4.301638in}}%
\pgfpathlineto{\pgfqpoint{4.424973in}{4.414618in}}%
\pgfpathlineto{\pgfqpoint{4.480330in}{4.397499in}}%
\pgfpathclose%
\pgfusepath{fill}%
\end{pgfscope}%
\begin{pgfscope}%
\pgfpathrectangle{\pgfqpoint{0.539299in}{0.078740in}}{\pgfqpoint{7.842520in}{7.842520in}}%
\pgfusepath{clip}%
\pgfsetbuttcap%
\pgfsetroundjoin%
\definecolor{currentfill}{rgb}{0.282290,0.145912,0.461510}%
\pgfsetfillcolor{currentfill}%
\pgfsetlinewidth{0.000000pt}%
\definecolor{currentstroke}{rgb}{0.139147,0.533812,0.555298}%
\pgfsetstrokecolor{currentstroke}%
\pgfsetdash{}{0pt}%
\pgfpathmoveto{\pgfqpoint{5.220192in}{3.803938in}}%
\pgfpathlineto{\pgfqpoint{5.161052in}{3.843033in}}%
\pgfpathlineto{\pgfqpoint{5.087019in}{3.881455in}}%
\pgfpathclose%
\pgfusepath{fill}%
\end{pgfscope}%
\begin{pgfscope}%
\pgfpathrectangle{\pgfqpoint{0.539299in}{0.078740in}}{\pgfqpoint{7.842520in}{7.842520in}}%
\pgfusepath{clip}%
\pgfsetbuttcap%
\pgfsetroundjoin%
\definecolor{currentfill}{rgb}{0.212395,0.359683,0.551710}%
\pgfsetfillcolor{currentfill}%
\pgfsetlinewidth{0.000000pt}%
\definecolor{currentstroke}{rgb}{0.137770,0.537492,0.554906}%
\pgfsetstrokecolor{currentstroke}%
\pgfsetdash{}{0pt}%
\pgfpathmoveto{\pgfqpoint{3.531309in}{4.704589in}}%
\pgfpathlineto{\pgfqpoint{3.404261in}{4.501552in}}%
\pgfpathlineto{\pgfqpoint{3.323758in}{4.491822in}}%
\pgfpathclose%
\pgfusepath{fill}%
\end{pgfscope}%
\begin{pgfscope}%
\pgfpathrectangle{\pgfqpoint{0.539299in}{0.078740in}}{\pgfqpoint{7.842520in}{7.842520in}}%
\pgfusepath{clip}%
\pgfsetbuttcap%
\pgfsetroundjoin%
\definecolor{currentfill}{rgb}{0.206756,0.371758,0.553117}%
\pgfsetfillcolor{currentfill}%
\pgfsetlinewidth{0.000000pt}%
\definecolor{currentstroke}{rgb}{0.136408,0.541173,0.554483}%
\pgfsetstrokecolor{currentstroke}%
\pgfsetdash{}{0pt}%
\pgfpathmoveto{\pgfqpoint{4.215129in}{4.619034in}}%
\pgfpathlineto{\pgfqpoint{4.293076in}{4.516692in}}%
\pgfpathlineto{\pgfqpoint{4.082979in}{4.695694in}}%
\pgfpathclose%
\pgfusepath{fill}%
\end{pgfscope}%
\begin{pgfscope}%
\pgfpathrectangle{\pgfqpoint{0.539299in}{0.078740in}}{\pgfqpoint{7.842520in}{7.842520in}}%
\pgfusepath{clip}%
\pgfsetbuttcap%
\pgfsetroundjoin%
\definecolor{currentfill}{rgb}{0.246811,0.283237,0.535941}%
\pgfsetfillcolor{currentfill}%
\pgfsetlinewidth{0.000000pt}%
\definecolor{currentstroke}{rgb}{0.135066,0.544853,0.554029}%
\pgfsetstrokecolor{currentstroke}%
\pgfsetdash{}{0pt}%
\pgfpathmoveto{\pgfqpoint{4.480330in}{4.397499in}}%
\pgfpathlineto{\pgfqpoint{4.689266in}{4.186089in}}%
\pgfpathlineto{\pgfqpoint{4.557051in}{4.301638in}}%
\pgfpathclose%
\pgfusepath{fill}%
\end{pgfscope}%
\begin{pgfscope}%
\pgfpathrectangle{\pgfqpoint{0.539299in}{0.078740in}}{\pgfqpoint{7.842520in}{7.842520in}}%
\pgfusepath{clip}%
\pgfsetbuttcap%
\pgfsetroundjoin%
\definecolor{currentfill}{rgb}{0.190631,0.407061,0.556089}%
\pgfsetfillcolor{currentfill}%
\pgfsetlinewidth{0.000000pt}%
\definecolor{currentstroke}{rgb}{0.133743,0.548535,0.553541}%
\pgfsetstrokecolor{currentstroke}%
\pgfsetdash{}{0pt}%
\pgfpathmoveto{\pgfqpoint{3.951415in}{4.734438in}}%
\pgfpathlineto{\pgfqpoint{3.820757in}{4.723273in}}%
\pgfpathlineto{\pgfqpoint{3.871838in}{4.810044in}}%
\pgfpathclose%
\pgfusepath{fill}%
\end{pgfscope}%
\begin{pgfscope}%
\pgfpathrectangle{\pgfqpoint{0.539299in}{0.078740in}}{\pgfqpoint{7.842520in}{7.842520in}}%
\pgfusepath{clip}%
\pgfsetbuttcap%
\pgfsetroundjoin%
\definecolor{currentfill}{rgb}{0.278791,0.062145,0.386592}%
\pgfsetfillcolor{currentfill}%
\pgfsetlinewidth{0.000000pt}%
\definecolor{currentstroke}{rgb}{0.132444,0.552216,0.553018}%
\pgfsetstrokecolor{currentstroke}%
\pgfsetdash{}{0pt}%
\pgfpathmoveto{\pgfqpoint{5.901860in}{3.611134in}}%
\pgfpathlineto{\pgfqpoint{6.036706in}{3.577796in}}%
\pgfpathlineto{\pgfqpoint{6.107631in}{3.575340in}}%
\pgfpathclose%
\pgfusepath{fill}%
\end{pgfscope}%
\begin{pgfscope}%
\pgfpathrectangle{\pgfqpoint{0.539299in}{0.078740in}}{\pgfqpoint{7.842520in}{7.842520in}}%
\pgfusepath{clip}%
\pgfsetbuttcap%
\pgfsetroundjoin%
\definecolor{currentfill}{rgb}{0.257322,0.256130,0.526563}%
\pgfsetfillcolor{currentfill}%
\pgfsetlinewidth{0.000000pt}%
\definecolor{currentstroke}{rgb}{0.131172,0.555899,0.552459}%
\pgfsetstrokecolor{currentstroke}%
\pgfsetdash{}{0pt}%
\pgfpathmoveto{\pgfqpoint{4.821630in}{4.074629in}}%
\pgfpathlineto{\pgfqpoint{4.689266in}{4.186089in}}%
\pgfpathlineto{\pgfqpoint{4.613151in}{4.272715in}}%
\pgfpathclose%
\pgfusepath{fill}%
\end{pgfscope}%
\begin{pgfscope}%
\pgfpathrectangle{\pgfqpoint{0.539299in}{0.078740in}}{\pgfqpoint{7.842520in}{7.842520in}}%
\pgfusepath{clip}%
\pgfsetbuttcap%
\pgfsetroundjoin%
\definecolor{currentfill}{rgb}{0.212395,0.359683,0.551710}%
\pgfsetfillcolor{currentfill}%
\pgfsetlinewidth{0.000000pt}%
\definecolor{currentstroke}{rgb}{0.129933,0.559582,0.551864}%
\pgfsetstrokecolor{currentstroke}%
\pgfsetdash{}{0pt}%
\pgfpathmoveto{\pgfqpoint{4.293076in}{4.516692in}}%
\pgfpathlineto{\pgfqpoint{4.215129in}{4.619034in}}%
\pgfpathlineto{\pgfqpoint{4.347630in}{4.516065in}}%
\pgfpathclose%
\pgfusepath{fill}%
\end{pgfscope}%
\begin{pgfscope}%
\pgfpathrectangle{\pgfqpoint{0.539299in}{0.078740in}}{\pgfqpoint{7.842520in}{7.842520in}}%
\pgfusepath{clip}%
\pgfsetbuttcap%
\pgfsetroundjoin%
\definecolor{currentfill}{rgb}{0.252194,0.269783,0.531579}%
\pgfsetfillcolor{currentfill}%
\pgfsetlinewidth{0.000000pt}%
\definecolor{currentstroke}{rgb}{0.128729,0.563265,0.551229}%
\pgfsetstrokecolor{currentstroke}%
\pgfsetdash{}{0pt}%
\pgfpathmoveto{\pgfqpoint{3.242761in}{4.478380in}}%
\pgfpathlineto{\pgfqpoint{3.119658in}{4.133109in}}%
\pgfpathlineto{\pgfqpoint{3.038801in}{4.091418in}}%
\pgfpathclose%
\pgfusepath{fill}%
\end{pgfscope}%
\begin{pgfscope}%
\pgfpathrectangle{\pgfqpoint{0.539299in}{0.078740in}}{\pgfqpoint{7.842520in}{7.842520in}}%
\pgfusepath{clip}%
\pgfsetbuttcap%
\pgfsetroundjoin%
\definecolor{currentfill}{rgb}{0.281924,0.089666,0.412415}%
\pgfsetfillcolor{currentfill}%
\pgfsetlinewidth{0.000000pt}%
\definecolor{currentstroke}{rgb}{0.127568,0.566949,0.550556}%
\pgfsetstrokecolor{currentstroke}%
\pgfsetdash{}{0pt}%
\pgfpathmoveto{\pgfqpoint{5.695175in}{3.642100in}}%
\pgfpathlineto{\pgfqpoint{5.767446in}{3.643788in}}%
\pgfpathlineto{\pgfqpoint{5.560913in}{3.680214in}}%
\pgfpathclose%
\pgfusepath{fill}%
\end{pgfscope}%
\begin{pgfscope}%
\pgfpathrectangle{\pgfqpoint{0.539299in}{0.078740in}}{\pgfqpoint{7.842520in}{7.842520in}}%
\pgfusepath{clip}%
\pgfsetbuttcap%
\pgfsetroundjoin%
\definecolor{currentfill}{rgb}{0.277941,0.056324,0.381191}%
\pgfsetfillcolor{currentfill}%
\pgfsetlinewidth{0.000000pt}%
\definecolor{currentstroke}{rgb}{0.126453,0.570633,0.549841}%
\pgfsetstrokecolor{currentstroke}%
\pgfsetdash{}{0pt}%
\pgfpathmoveto{\pgfqpoint{6.107631in}{3.575340in}}%
\pgfpathlineto{\pgfqpoint{6.171939in}{3.541802in}}%
\pgfpathlineto{\pgfqpoint{6.242456in}{3.538310in}}%
\pgfpathclose%
\pgfusepath{fill}%
\end{pgfscope}%
\begin{pgfscope}%
\pgfpathrectangle{\pgfqpoint{0.539299in}{0.078740in}}{\pgfqpoint{7.842520in}{7.842520in}}%
\pgfusepath{clip}%
\pgfsetbuttcap%
\pgfsetroundjoin%
\definecolor{currentfill}{rgb}{0.187231,0.414746,0.556547}%
\pgfsetfillcolor{currentfill}%
\pgfsetlinewidth{0.000000pt}%
\definecolor{currentstroke}{rgb}{0.125394,0.574318,0.549086}%
\pgfsetstrokecolor{currentstroke}%
\pgfsetdash{}{0pt}%
\pgfpathmoveto{\pgfqpoint{3.871838in}{4.810044in}}%
\pgfpathlineto{\pgfqpoint{3.820757in}{4.723273in}}%
\pgfpathlineto{\pgfqpoint{3.740970in}{4.779433in}}%
\pgfpathclose%
\pgfusepath{fill}%
\end{pgfscope}%
\begin{pgfscope}%
\pgfpathrectangle{\pgfqpoint{0.539299in}{0.078740in}}{\pgfqpoint{7.842520in}{7.842520in}}%
\pgfusepath{clip}%
\pgfsetbuttcap%
\pgfsetroundjoin%
\definecolor{currentfill}{rgb}{0.281446,0.084320,0.407414}%
\pgfsetfillcolor{currentfill}%
\pgfsetlinewidth{0.000000pt}%
\definecolor{currentstroke}{rgb}{0.124395,0.578002,0.548287}%
\pgfsetstrokecolor{currentstroke}%
\pgfsetdash{}{0pt}%
\pgfpathmoveto{\pgfqpoint{5.767446in}{3.643788in}}%
\pgfpathlineto{\pgfqpoint{5.829924in}{3.607475in}}%
\pgfpathlineto{\pgfqpoint{5.901860in}{3.611134in}}%
\pgfpathclose%
\pgfusepath{fill}%
\end{pgfscope}%
\begin{pgfscope}%
\pgfpathrectangle{\pgfqpoint{0.539299in}{0.078740in}}{\pgfqpoint{7.842520in}{7.842520in}}%
\pgfusepath{clip}%
\pgfsetbuttcap%
\pgfsetroundjoin%
\definecolor{currentfill}{rgb}{0.223925,0.334994,0.548053}%
\pgfsetfillcolor{currentfill}%
\pgfsetlinewidth{0.000000pt}%
\definecolor{currentstroke}{rgb}{0.123463,0.581687,0.547445}%
\pgfsetstrokecolor{currentstroke}%
\pgfsetdash{}{0pt}%
\pgfpathmoveto{\pgfqpoint{4.480330in}{4.397499in}}%
\pgfpathlineto{\pgfqpoint{4.424973in}{4.414618in}}%
\pgfpathlineto{\pgfqpoint{4.347630in}{4.516065in}}%
\pgfpathclose%
\pgfusepath{fill}%
\end{pgfscope}%
\begin{pgfscope}%
\pgfpathrectangle{\pgfqpoint{0.539299in}{0.078740in}}{\pgfqpoint{7.842520in}{7.842520in}}%
\pgfusepath{clip}%
\pgfsetbuttcap%
\pgfsetroundjoin%
\definecolor{currentfill}{rgb}{0.270595,0.214069,0.507052}%
\pgfsetfillcolor{currentfill}%
\pgfsetlinewidth{0.000000pt}%
\definecolor{currentstroke}{rgb}{0.122606,0.585371,0.546557}%
\pgfsetstrokecolor{currentstroke}%
\pgfsetdash{}{0pt}%
\pgfpathmoveto{\pgfqpoint{4.879148in}{4.033520in}}%
\pgfpathlineto{\pgfqpoint{4.954191in}{3.972092in}}%
\pgfpathlineto{\pgfqpoint{4.821630in}{4.074629in}}%
\pgfpathclose%
\pgfusepath{fill}%
\end{pgfscope}%
\begin{pgfscope}%
\pgfpathrectangle{\pgfqpoint{0.539299in}{0.078740in}}{\pgfqpoint{7.842520in}{7.842520in}}%
\pgfusepath{clip}%
\pgfsetbuttcap%
\pgfsetroundjoin%
\definecolor{currentfill}{rgb}{0.278791,0.062145,0.386592}%
\pgfsetfillcolor{currentfill}%
\pgfsetlinewidth{0.000000pt}%
\definecolor{currentstroke}{rgb}{0.121831,0.589055,0.545623}%
\pgfsetstrokecolor{currentstroke}%
\pgfsetdash{}{0pt}%
\pgfpathmoveto{\pgfqpoint{6.036706in}{3.577796in}}%
\pgfpathlineto{\pgfqpoint{6.171939in}{3.541802in}}%
\pgfpathlineto{\pgfqpoint{6.107631in}{3.575340in}}%
\pgfpathclose%
\pgfusepath{fill}%
\end{pgfscope}%
\begin{pgfscope}%
\pgfpathrectangle{\pgfqpoint{0.539299in}{0.078740in}}{\pgfqpoint{7.842520in}{7.842520in}}%
\pgfusepath{clip}%
\pgfsetbuttcap%
\pgfsetroundjoin%
\definecolor{currentfill}{rgb}{0.283072,0.130895,0.449241}%
\pgfsetfillcolor{currentfill}%
\pgfsetlinewidth{0.000000pt}%
\definecolor{currentstroke}{rgb}{0.121148,0.592739,0.544641}%
\pgfsetstrokecolor{currentstroke}%
\pgfsetdash{}{0pt}%
\pgfpathmoveto{\pgfqpoint{5.353780in}{3.739197in}}%
\pgfpathlineto{\pgfqpoint{5.427151in}{3.724780in}}%
\pgfpathlineto{\pgfqpoint{5.220192in}{3.803938in}}%
\pgfpathclose%
\pgfusepath{fill}%
\end{pgfscope}%
\begin{pgfscope}%
\pgfpathrectangle{\pgfqpoint{0.539299in}{0.078740in}}{\pgfqpoint{7.842520in}{7.842520in}}%
\pgfusepath{clip}%
\pgfsetbuttcap%
\pgfsetroundjoin%
\definecolor{currentfill}{rgb}{0.187231,0.414746,0.556547}%
\pgfsetfillcolor{currentfill}%
\pgfsetlinewidth{0.000000pt}%
\definecolor{currentstroke}{rgb}{0.120565,0.596422,0.543611}%
\pgfsetstrokecolor{currentstroke}%
\pgfsetdash{}{0pt}%
\pgfpathmoveto{\pgfqpoint{3.951415in}{4.734438in}}%
\pgfpathlineto{\pgfqpoint{3.871838in}{4.810044in}}%
\pgfpathlineto{\pgfqpoint{4.082979in}{4.695694in}}%
\pgfpathclose%
\pgfusepath{fill}%
\end{pgfscope}%
\begin{pgfscope}%
\pgfpathrectangle{\pgfqpoint{0.539299in}{0.078740in}}{\pgfqpoint{7.842520in}{7.842520in}}%
\pgfusepath{clip}%
\pgfsetbuttcap%
\pgfsetroundjoin%
\definecolor{currentfill}{rgb}{0.283091,0.110553,0.431554}%
\pgfsetfillcolor{currentfill}%
\pgfsetlinewidth{0.000000pt}%
\definecolor{currentstroke}{rgb}{0.120092,0.600104,0.542530}%
\pgfsetstrokecolor{currentstroke}%
\pgfsetdash{}{0pt}%
\pgfpathmoveto{\pgfqpoint{5.487835in}{3.685615in}}%
\pgfpathlineto{\pgfqpoint{5.560913in}{3.680214in}}%
\pgfpathlineto{\pgfqpoint{5.427151in}{3.724780in}}%
\pgfpathclose%
\pgfusepath{fill}%
\end{pgfscope}%
\begin{pgfscope}%
\pgfpathrectangle{\pgfqpoint{0.539299in}{0.078740in}}{\pgfqpoint{7.842520in}{7.842520in}}%
\pgfusepath{clip}%
\pgfsetbuttcap%
\pgfsetroundjoin%
\definecolor{currentfill}{rgb}{0.281924,0.089666,0.412415}%
\pgfsetfillcolor{currentfill}%
\pgfsetlinewidth{0.000000pt}%
\definecolor{currentstroke}{rgb}{0.119738,0.603785,0.541400}%
\pgfsetstrokecolor{currentstroke}%
\pgfsetdash{}{0pt}%
\pgfpathmoveto{\pgfqpoint{5.695175in}{3.642100in}}%
\pgfpathlineto{\pgfqpoint{5.829924in}{3.607475in}}%
\pgfpathlineto{\pgfqpoint{5.767446in}{3.643788in}}%
\pgfpathclose%
\pgfusepath{fill}%
\end{pgfscope}%
\begin{pgfscope}%
\pgfpathrectangle{\pgfqpoint{0.539299in}{0.078740in}}{\pgfqpoint{7.842520in}{7.842520in}}%
\pgfusepath{clip}%
\pgfsetbuttcap%
\pgfsetroundjoin%
\definecolor{currentfill}{rgb}{0.277941,0.056324,0.381191}%
\pgfsetfillcolor{currentfill}%
\pgfsetlinewidth{0.000000pt}%
\definecolor{currentstroke}{rgb}{0.119512,0.607464,0.540218}%
\pgfsetstrokecolor{currentstroke}%
\pgfsetdash{}{0pt}%
\pgfpathmoveto{\pgfqpoint{6.242456in}{3.538310in}}%
\pgfpathlineto{\pgfqpoint{6.171939in}{3.541802in}}%
\pgfpathlineto{\pgfqpoint{6.377643in}{3.498447in}}%
\pgfpathclose%
\pgfusepath{fill}%
\end{pgfscope}%
\begin{pgfscope}%
\pgfpathrectangle{\pgfqpoint{0.539299in}{0.078740in}}{\pgfqpoint{7.842520in}{7.842520in}}%
\pgfusepath{clip}%
\pgfsetbuttcap%
\pgfsetroundjoin%
\definecolor{currentfill}{rgb}{0.241237,0.296485,0.539709}%
\pgfsetfillcolor{currentfill}%
\pgfsetlinewidth{0.000000pt}%
\definecolor{currentstroke}{rgb}{0.119423,0.611141,0.538982}%
\pgfsetstrokecolor{currentstroke}%
\pgfsetdash{}{0pt}%
\pgfpathmoveto{\pgfqpoint{4.613151in}{4.272715in}}%
\pgfpathlineto{\pgfqpoint{4.689266in}{4.186089in}}%
\pgfpathlineto{\pgfqpoint{4.480330in}{4.397499in}}%
\pgfpathclose%
\pgfusepath{fill}%
\end{pgfscope}%
\begin{pgfscope}%
\pgfpathrectangle{\pgfqpoint{0.539299in}{0.078740in}}{\pgfqpoint{7.842520in}{7.842520in}}%
\pgfusepath{clip}%
\pgfsetbuttcap%
\pgfsetroundjoin%
\definecolor{currentfill}{rgb}{0.185556,0.418570,0.556753}%
\pgfsetfillcolor{currentfill}%
\pgfsetlinewidth{0.000000pt}%
\definecolor{currentstroke}{rgb}{0.119483,0.614817,0.537692}%
\pgfsetstrokecolor{currentstroke}%
\pgfsetdash{}{0pt}%
\pgfpathmoveto{\pgfqpoint{3.611618in}{4.681644in}}%
\pgfpathlineto{\pgfqpoint{3.660600in}{4.826856in}}%
\pgfpathlineto{\pgfqpoint{3.740970in}{4.779433in}}%
\pgfpathclose%
\pgfusepath{fill}%
\end{pgfscope}%
\begin{pgfscope}%
\pgfpathrectangle{\pgfqpoint{0.539299in}{0.078740in}}{\pgfqpoint{7.842520in}{7.842520in}}%
\pgfusepath{clip}%
\pgfsetbuttcap%
\pgfsetroundjoin%
\definecolor{currentfill}{rgb}{0.187231,0.414746,0.556547}%
\pgfsetfillcolor{currentfill}%
\pgfsetlinewidth{0.000000pt}%
\definecolor{currentstroke}{rgb}{0.119699,0.618490,0.536347}%
\pgfsetstrokecolor{currentstroke}%
\pgfsetdash{}{0pt}%
\pgfpathmoveto{\pgfqpoint{3.611618in}{4.681644in}}%
\pgfpathlineto{\pgfqpoint{3.531309in}{4.704589in}}%
\pgfpathlineto{\pgfqpoint{3.660600in}{4.826856in}}%
\pgfpathclose%
\pgfusepath{fill}%
\end{pgfscope}%
\begin{pgfscope}%
\pgfpathrectangle{\pgfqpoint{0.539299in}{0.078740in}}{\pgfqpoint{7.842520in}{7.842520in}}%
\pgfusepath{clip}%
\pgfsetbuttcap%
\pgfsetroundjoin%
\definecolor{currentfill}{rgb}{0.276194,0.190074,0.493001}%
\pgfsetfillcolor{currentfill}%
\pgfsetlinewidth{0.000000pt}%
\definecolor{currentstroke}{rgb}{0.120081,0.622161,0.534946}%
\pgfsetstrokecolor{currentstroke}%
\pgfsetdash{}{0pt}%
\pgfpathmoveto{\pgfqpoint{4.954191in}{3.972092in}}%
\pgfpathlineto{\pgfqpoint{5.012423in}{3.928911in}}%
\pgfpathlineto{\pgfqpoint{5.087019in}{3.881455in}}%
\pgfpathclose%
\pgfusepath{fill}%
\end{pgfscope}%
\begin{pgfscope}%
\pgfpathrectangle{\pgfqpoint{0.539299in}{0.078740in}}{\pgfqpoint{7.842520in}{7.842520in}}%
\pgfusepath{clip}%
\pgfsetbuttcap%
\pgfsetroundjoin%
\definecolor{currentfill}{rgb}{0.253935,0.265254,0.529983}%
\pgfsetfillcolor{currentfill}%
\pgfsetlinewidth{0.000000pt}%
\definecolor{currentstroke}{rgb}{0.120638,0.625828,0.533488}%
\pgfsetstrokecolor{currentstroke}%
\pgfsetdash{}{0pt}%
\pgfpathmoveto{\pgfqpoint{4.613151in}{4.272715in}}%
\pgfpathlineto{\pgfqpoint{4.746079in}{4.149433in}}%
\pgfpathlineto{\pgfqpoint{4.821630in}{4.074629in}}%
\pgfpathclose%
\pgfusepath{fill}%
\end{pgfscope}%
\begin{pgfscope}%
\pgfpathrectangle{\pgfqpoint{0.539299in}{0.078740in}}{\pgfqpoint{7.842520in}{7.842520in}}%
\pgfusepath{clip}%
\pgfsetbuttcap%
\pgfsetroundjoin%
\definecolor{currentfill}{rgb}{0.283187,0.125848,0.444960}%
\pgfsetfillcolor{currentfill}%
\pgfsetlinewidth{0.000000pt}%
\definecolor{currentstroke}{rgb}{0.121380,0.629492,0.531973}%
\pgfsetstrokecolor{currentstroke}%
\pgfsetdash{}{0pt}%
\pgfpathmoveto{\pgfqpoint{5.353780in}{3.739197in}}%
\pgfpathlineto{\pgfqpoint{5.487835in}{3.685615in}}%
\pgfpathlineto{\pgfqpoint{5.427151in}{3.724780in}}%
\pgfpathclose%
\pgfusepath{fill}%
\end{pgfscope}%
\begin{pgfscope}%
\pgfpathrectangle{\pgfqpoint{0.539299in}{0.078740in}}{\pgfqpoint{7.842520in}{7.842520in}}%
\pgfusepath{clip}%
\pgfsetbuttcap%
\pgfsetroundjoin%
\definecolor{currentfill}{rgb}{0.262138,0.242286,0.520837}%
\pgfsetfillcolor{currentfill}%
\pgfsetlinewidth{0.000000pt}%
\definecolor{currentstroke}{rgb}{0.122312,0.633153,0.530398}%
\pgfsetstrokecolor{currentstroke}%
\pgfsetdash{}{0pt}%
\pgfpathmoveto{\pgfqpoint{4.821630in}{4.074629in}}%
\pgfpathlineto{\pgfqpoint{4.746079in}{4.149433in}}%
\pgfpathlineto{\pgfqpoint{4.879148in}{4.033520in}}%
\pgfpathclose%
\pgfusepath{fill}%
\end{pgfscope}%
\begin{pgfscope}%
\pgfpathrectangle{\pgfqpoint{0.539299in}{0.078740in}}{\pgfqpoint{7.842520in}{7.842520in}}%
\pgfusepath{clip}%
\pgfsetbuttcap%
\pgfsetroundjoin%
\definecolor{currentfill}{rgb}{0.278826,0.175490,0.483397}%
\pgfsetfillcolor{currentfill}%
\pgfsetlinewidth{0.000000pt}%
\definecolor{currentstroke}{rgb}{0.123444,0.636809,0.528763}%
\pgfsetstrokecolor{currentstroke}%
\pgfsetdash{}{0pt}%
\pgfpathmoveto{\pgfqpoint{5.087019in}{3.881455in}}%
\pgfpathlineto{\pgfqpoint{5.012423in}{3.928911in}}%
\pgfpathlineto{\pgfqpoint{5.220192in}{3.803938in}}%
\pgfpathclose%
\pgfusepath{fill}%
\end{pgfscope}%
\begin{pgfscope}%
\pgfpathrectangle{\pgfqpoint{0.539299in}{0.078740in}}{\pgfqpoint{7.842520in}{7.842520in}}%
\pgfusepath{clip}%
\pgfsetbuttcap%
\pgfsetroundjoin%
\definecolor{currentfill}{rgb}{0.281446,0.084320,0.407414}%
\pgfsetfillcolor{currentfill}%
\pgfsetlinewidth{0.000000pt}%
\definecolor{currentstroke}{rgb}{0.124780,0.640461,0.527068}%
\pgfsetstrokecolor{currentstroke}%
\pgfsetdash{}{0pt}%
\pgfpathmoveto{\pgfqpoint{5.965125in}{3.573380in}}%
\pgfpathlineto{\pgfqpoint{6.036706in}{3.577796in}}%
\pgfpathlineto{\pgfqpoint{5.901860in}{3.611134in}}%
\pgfpathclose%
\pgfusepath{fill}%
\end{pgfscope}%
\begin{pgfscope}%
\pgfpathrectangle{\pgfqpoint{0.539299in}{0.078740in}}{\pgfqpoint{7.842520in}{7.842520in}}%
\pgfusepath{clip}%
\pgfsetbuttcap%
\pgfsetroundjoin%
\definecolor{currentfill}{rgb}{0.195860,0.395433,0.555276}%
\pgfsetfillcolor{currentfill}%
\pgfsetlinewidth{0.000000pt}%
\definecolor{currentstroke}{rgb}{0.126326,0.644107,0.525311}%
\pgfsetstrokecolor{currentstroke}%
\pgfsetdash{}{0pt}%
\pgfpathmoveto{\pgfqpoint{3.323758in}{4.491822in}}%
\pgfpathlineto{\pgfqpoint{3.450467in}{4.721409in}}%
\pgfpathlineto{\pgfqpoint{3.531309in}{4.704589in}}%
\pgfpathclose%
\pgfusepath{fill}%
\end{pgfscope}%
\begin{pgfscope}%
\pgfpathrectangle{\pgfqpoint{0.539299in}{0.078740in}}{\pgfqpoint{7.842520in}{7.842520in}}%
\pgfusepath{clip}%
\pgfsetbuttcap%
\pgfsetroundjoin%
\definecolor{currentfill}{rgb}{0.283091,0.110553,0.431554}%
\pgfsetfillcolor{currentfill}%
\pgfsetlinewidth{0.000000pt}%
\definecolor{currentstroke}{rgb}{0.128087,0.647749,0.523491}%
\pgfsetstrokecolor{currentstroke}%
\pgfsetdash{}{0pt}%
\pgfpathmoveto{\pgfqpoint{5.560913in}{3.680214in}}%
\pgfpathlineto{\pgfqpoint{5.622384in}{3.640649in}}%
\pgfpathlineto{\pgfqpoint{5.695175in}{3.642100in}}%
\pgfpathclose%
\pgfusepath{fill}%
\end{pgfscope}%
\begin{pgfscope}%
\pgfpathrectangle{\pgfqpoint{0.539299in}{0.078740in}}{\pgfqpoint{7.842520in}{7.842520in}}%
\pgfusepath{clip}%
\pgfsetbuttcap%
\pgfsetroundjoin%
\definecolor{currentfill}{rgb}{0.270595,0.214069,0.507052}%
\pgfsetfillcolor{currentfill}%
\pgfsetlinewidth{0.000000pt}%
\definecolor{currentstroke}{rgb}{0.130067,0.651384,0.521608}%
\pgfsetstrokecolor{currentstroke}%
\pgfsetdash{}{0pt}%
\pgfpathmoveto{\pgfqpoint{4.879148in}{4.033520in}}%
\pgfpathlineto{\pgfqpoint{5.012423in}{3.928911in}}%
\pgfpathlineto{\pgfqpoint{4.954191in}{3.972092in}}%
\pgfpathclose%
\pgfusepath{fill}%
\end{pgfscope}%
\begin{pgfscope}%
\pgfpathrectangle{\pgfqpoint{0.539299in}{0.078740in}}{\pgfqpoint{7.842520in}{7.842520in}}%
\pgfusepath{clip}%
\pgfsetbuttcap%
\pgfsetroundjoin%
\definecolor{currentfill}{rgb}{0.190631,0.407061,0.556089}%
\pgfsetfillcolor{currentfill}%
\pgfsetlinewidth{0.000000pt}%
\definecolor{currentstroke}{rgb}{0.132268,0.655014,0.519661}%
\pgfsetstrokecolor{currentstroke}%
\pgfsetdash{}{0pt}%
\pgfpathmoveto{\pgfqpoint{4.136438in}{4.716235in}}%
\pgfpathlineto{\pgfqpoint{4.215129in}{4.619034in}}%
\pgfpathlineto{\pgfqpoint{4.082979in}{4.695694in}}%
\pgfpathclose%
\pgfusepath{fill}%
\end{pgfscope}%
\begin{pgfscope}%
\pgfpathrectangle{\pgfqpoint{0.539299in}{0.078740in}}{\pgfqpoint{7.842520in}{7.842520in}}%
\pgfusepath{clip}%
\pgfsetbuttcap%
\pgfsetroundjoin%
\definecolor{currentfill}{rgb}{0.282327,0.094955,0.417331}%
\pgfsetfillcolor{currentfill}%
\pgfsetlinewidth{0.000000pt}%
\definecolor{currentstroke}{rgb}{0.134692,0.658636,0.517649}%
\pgfsetstrokecolor{currentstroke}%
\pgfsetdash{}{0pt}%
\pgfpathmoveto{\pgfqpoint{5.901860in}{3.611134in}}%
\pgfpathlineto{\pgfqpoint{5.829924in}{3.607475in}}%
\pgfpathlineto{\pgfqpoint{5.965125in}{3.573380in}}%
\pgfpathclose%
\pgfusepath{fill}%
\end{pgfscope}%
\begin{pgfscope}%
\pgfpathrectangle{\pgfqpoint{0.539299in}{0.078740in}}{\pgfqpoint{7.842520in}{7.842520in}}%
\pgfusepath{clip}%
\pgfsetbuttcap%
\pgfsetroundjoin%
\definecolor{currentfill}{rgb}{0.279566,0.067836,0.391917}%
\pgfsetfillcolor{currentfill}%
\pgfsetlinewidth{0.000000pt}%
\definecolor{currentstroke}{rgb}{0.137339,0.662252,0.515571}%
\pgfsetstrokecolor{currentstroke}%
\pgfsetdash{}{0pt}%
\pgfpathmoveto{\pgfqpoint{6.171939in}{3.541802in}}%
\pgfpathlineto{\pgfqpoint{6.307514in}{3.501723in}}%
\pgfpathlineto{\pgfqpoint{6.377643in}{3.498447in}}%
\pgfpathclose%
\pgfusepath{fill}%
\end{pgfscope}%
\begin{pgfscope}%
\pgfpathrectangle{\pgfqpoint{0.539299in}{0.078740in}}{\pgfqpoint{7.842520in}{7.842520in}}%
\pgfusepath{clip}%
\pgfsetbuttcap%
\pgfsetroundjoin%
\definecolor{currentfill}{rgb}{0.203063,0.379716,0.553925}%
\pgfsetfillcolor{currentfill}%
\pgfsetlinewidth{0.000000pt}%
\definecolor{currentstroke}{rgb}{0.140210,0.665859,0.513427}%
\pgfsetstrokecolor{currentstroke}%
\pgfsetdash{}{0pt}%
\pgfpathmoveto{\pgfqpoint{3.242761in}{4.478380in}}%
\pgfpathlineto{\pgfqpoint{3.450467in}{4.721409in}}%
\pgfpathlineto{\pgfqpoint{3.323758in}{4.491822in}}%
\pgfpathclose%
\pgfusepath{fill}%
\end{pgfscope}%
\begin{pgfscope}%
\pgfpathrectangle{\pgfqpoint{0.539299in}{0.078740in}}{\pgfqpoint{7.842520in}{7.842520in}}%
\pgfusepath{clip}%
\pgfsetbuttcap%
\pgfsetroundjoin%
\definecolor{currentfill}{rgb}{0.180629,0.429975,0.557282}%
\pgfsetfillcolor{currentfill}%
\pgfsetlinewidth{0.000000pt}%
\definecolor{currentstroke}{rgb}{0.143303,0.669459,0.511215}%
\pgfsetstrokecolor{currentstroke}%
\pgfsetdash{}{0pt}%
\pgfpathmoveto{\pgfqpoint{4.082979in}{4.695694in}}%
\pgfpathlineto{\pgfqpoint{3.871838in}{4.810044in}}%
\pgfpathlineto{\pgfqpoint{4.003782in}{4.784962in}}%
\pgfpathclose%
\pgfusepath{fill}%
\end{pgfscope}%
\begin{pgfscope}%
\pgfpathrectangle{\pgfqpoint{0.539299in}{0.078740in}}{\pgfqpoint{7.842520in}{7.842520in}}%
\pgfusepath{clip}%
\pgfsetbuttcap%
\pgfsetroundjoin%
\definecolor{currentfill}{rgb}{0.277018,0.050344,0.375715}%
\pgfsetfillcolor{currentfill}%
\pgfsetlinewidth{0.000000pt}%
\definecolor{currentstroke}{rgb}{0.146616,0.673050,0.508936}%
\pgfsetstrokecolor{currentstroke}%
\pgfsetdash{}{0pt}%
\pgfpathmoveto{\pgfqpoint{6.377643in}{3.498447in}}%
\pgfpathlineto{\pgfqpoint{6.443391in}{3.456772in}}%
\pgfpathlineto{\pgfqpoint{6.513170in}{3.455462in}}%
\pgfpathclose%
\pgfusepath{fill}%
\end{pgfscope}%
\begin{pgfscope}%
\pgfpathrectangle{\pgfqpoint{0.539299in}{0.078740in}}{\pgfqpoint{7.842520in}{7.842520in}}%
\pgfusepath{clip}%
\pgfsetbuttcap%
\pgfsetroundjoin%
\definecolor{currentfill}{rgb}{0.283229,0.120777,0.440584}%
\pgfsetfillcolor{currentfill}%
\pgfsetlinewidth{0.000000pt}%
\definecolor{currentstroke}{rgb}{0.150148,0.676631,0.506589}%
\pgfsetstrokecolor{currentstroke}%
\pgfsetdash{}{0pt}%
\pgfpathmoveto{\pgfqpoint{5.560913in}{3.680214in}}%
\pgfpathlineto{\pgfqpoint{5.487835in}{3.685615in}}%
\pgfpathlineto{\pgfqpoint{5.622384in}{3.640649in}}%
\pgfpathclose%
\pgfusepath{fill}%
\end{pgfscope}%
\begin{pgfscope}%
\pgfpathrectangle{\pgfqpoint{0.539299in}{0.078740in}}{\pgfqpoint{7.842520in}{7.842520in}}%
\pgfusepath{clip}%
\pgfsetbuttcap%
\pgfsetroundjoin%
\definecolor{currentfill}{rgb}{0.281446,0.084320,0.407414}%
\pgfsetfillcolor{currentfill}%
\pgfsetlinewidth{0.000000pt}%
\definecolor{currentstroke}{rgb}{0.153894,0.680203,0.504172}%
\pgfsetstrokecolor{currentstroke}%
\pgfsetdash{}{0pt}%
\pgfpathmoveto{\pgfqpoint{5.965125in}{3.573380in}}%
\pgfpathlineto{\pgfqpoint{6.171939in}{3.541802in}}%
\pgfpathlineto{\pgfqpoint{6.036706in}{3.577796in}}%
\pgfpathclose%
\pgfusepath{fill}%
\end{pgfscope}%
\begin{pgfscope}%
\pgfpathrectangle{\pgfqpoint{0.539299in}{0.078740in}}{\pgfqpoint{7.842520in}{7.842520in}}%
\pgfusepath{clip}%
\pgfsetbuttcap%
\pgfsetroundjoin%
\definecolor{currentfill}{rgb}{0.246811,0.283237,0.535941}%
\pgfsetfillcolor{currentfill}%
\pgfsetlinewidth{0.000000pt}%
\definecolor{currentstroke}{rgb}{0.157851,0.683765,0.501686}%
\pgfsetstrokecolor{currentstroke}%
\pgfsetdash{}{0pt}%
\pgfpathmoveto{\pgfqpoint{2.957481in}{4.048102in}}%
\pgfpathlineto{\pgfqpoint{3.161266in}{4.461876in}}%
\pgfpathlineto{\pgfqpoint{3.038801in}{4.091418in}}%
\pgfpathclose%
\pgfusepath{fill}%
\end{pgfscope}%
\begin{pgfscope}%
\pgfpathrectangle{\pgfqpoint{0.539299in}{0.078740in}}{\pgfqpoint{7.842520in}{7.842520in}}%
\pgfusepath{clip}%
\pgfsetbuttcap%
\pgfsetroundjoin%
\definecolor{currentfill}{rgb}{0.227802,0.326594,0.546532}%
\pgfsetfillcolor{currentfill}%
\pgfsetlinewidth{0.000000pt}%
\definecolor{currentstroke}{rgb}{0.162016,0.687316,0.499129}%
\pgfsetstrokecolor{currentstroke}%
\pgfsetdash{}{0pt}%
\pgfpathmoveto{\pgfqpoint{3.038801in}{4.091418in}}%
\pgfpathlineto{\pgfqpoint{3.161266in}{4.461876in}}%
\pgfpathlineto{\pgfqpoint{3.242761in}{4.478380in}}%
\pgfpathclose%
\pgfusepath{fill}%
\end{pgfscope}%
\begin{pgfscope}%
\pgfpathrectangle{\pgfqpoint{0.539299in}{0.078740in}}{\pgfqpoint{7.842520in}{7.842520in}}%
\pgfusepath{clip}%
\pgfsetbuttcap%
\pgfsetroundjoin%
\definecolor{currentfill}{rgb}{0.281412,0.155834,0.469201}%
\pgfsetfillcolor{currentfill}%
\pgfsetlinewidth{0.000000pt}%
\definecolor{currentstroke}{rgb}{0.166383,0.690856,0.496502}%
\pgfsetstrokecolor{currentstroke}%
\pgfsetdash{}{0pt}%
\pgfpathmoveto{\pgfqpoint{5.220192in}{3.803938in}}%
\pgfpathlineto{\pgfqpoint{5.279908in}{3.760172in}}%
\pgfpathlineto{\pgfqpoint{5.353780in}{3.739197in}}%
\pgfpathclose%
\pgfusepath{fill}%
\end{pgfscope}%
\begin{pgfscope}%
\pgfpathrectangle{\pgfqpoint{0.539299in}{0.078740in}}{\pgfqpoint{7.842520in}{7.842520in}}%
\pgfusepath{clip}%
\pgfsetbuttcap%
\pgfsetroundjoin%
\definecolor{currentfill}{rgb}{0.283091,0.110553,0.431554}%
\pgfsetfillcolor{currentfill}%
\pgfsetlinewidth{0.000000pt}%
\definecolor{currentstroke}{rgb}{0.170948,0.694384,0.493803}%
\pgfsetstrokecolor{currentstroke}%
\pgfsetdash{}{0pt}%
\pgfpathmoveto{\pgfqpoint{5.622384in}{3.640649in}}%
\pgfpathlineto{\pgfqpoint{5.829924in}{3.607475in}}%
\pgfpathlineto{\pgfqpoint{5.695175in}{3.642100in}}%
\pgfpathclose%
\pgfusepath{fill}%
\end{pgfscope}%
\begin{pgfscope}%
\pgfpathrectangle{\pgfqpoint{0.539299in}{0.078740in}}{\pgfqpoint{7.842520in}{7.842520in}}%
\pgfusepath{clip}%
\pgfsetbuttcap%
\pgfsetroundjoin%
\definecolor{currentfill}{rgb}{0.199430,0.387607,0.554642}%
\pgfsetfillcolor{currentfill}%
\pgfsetlinewidth{0.000000pt}%
\definecolor{currentstroke}{rgb}{0.175707,0.697900,0.491033}%
\pgfsetstrokecolor{currentstroke}%
\pgfsetdash{}{0pt}%
\pgfpathmoveto{\pgfqpoint{4.215129in}{4.619034in}}%
\pgfpathlineto{\pgfqpoint{4.269526in}{4.615818in}}%
\pgfpathlineto{\pgfqpoint{4.347630in}{4.516065in}}%
\pgfpathclose%
\pgfusepath{fill}%
\end{pgfscope}%
\begin{pgfscope}%
\pgfpathrectangle{\pgfqpoint{0.539299in}{0.078740in}}{\pgfqpoint{7.842520in}{7.842520in}}%
\pgfusepath{clip}%
\pgfsetbuttcap%
\pgfsetroundjoin%
\definecolor{currentfill}{rgb}{0.277134,0.185228,0.489898}%
\pgfsetfillcolor{currentfill}%
\pgfsetlinewidth{0.000000pt}%
\definecolor{currentstroke}{rgb}{0.180653,0.701402,0.488189}%
\pgfsetstrokecolor{currentstroke}%
\pgfsetdash{}{0pt}%
\pgfpathmoveto{\pgfqpoint{5.012423in}{3.928911in}}%
\pgfpathlineto{\pgfqpoint{5.145984in}{3.837674in}}%
\pgfpathlineto{\pgfqpoint{5.220192in}{3.803938in}}%
\pgfpathclose%
\pgfusepath{fill}%
\end{pgfscope}%
\begin{pgfscope}%
\pgfpathrectangle{\pgfqpoint{0.539299in}{0.078740in}}{\pgfqpoint{7.842520in}{7.842520in}}%
\pgfusepath{clip}%
\pgfsetbuttcap%
\pgfsetroundjoin%
\definecolor{currentfill}{rgb}{0.278791,0.062145,0.386592}%
\pgfsetfillcolor{currentfill}%
\pgfsetlinewidth{0.000000pt}%
\definecolor{currentstroke}{rgb}{0.185783,0.704891,0.485273}%
\pgfsetstrokecolor{currentstroke}%
\pgfsetdash{}{0pt}%
\pgfpathmoveto{\pgfqpoint{6.377643in}{3.498447in}}%
\pgfpathlineto{\pgfqpoint{6.307514in}{3.501723in}}%
\pgfpathlineto{\pgfqpoint{6.443391in}{3.456772in}}%
\pgfpathclose%
\pgfusepath{fill}%
\end{pgfscope}%
\begin{pgfscope}%
\pgfpathrectangle{\pgfqpoint{0.539299in}{0.078740in}}{\pgfqpoint{7.842520in}{7.842520in}}%
\pgfusepath{clip}%
\pgfsetbuttcap%
\pgfsetroundjoin%
\definecolor{currentfill}{rgb}{0.206756,0.371758,0.553117}%
\pgfsetfillcolor{currentfill}%
\pgfsetlinewidth{0.000000pt}%
\definecolor{currentstroke}{rgb}{0.191090,0.708366,0.482284}%
\pgfsetstrokecolor{currentstroke}%
\pgfsetdash{}{0pt}%
\pgfpathmoveto{\pgfqpoint{4.347630in}{4.516065in}}%
\pgfpathlineto{\pgfqpoint{4.269526in}{4.615818in}}%
\pgfpathlineto{\pgfqpoint{4.480330in}{4.397499in}}%
\pgfpathclose%
\pgfusepath{fill}%
\end{pgfscope}%
\begin{pgfscope}%
\pgfpathrectangle{\pgfqpoint{0.539299in}{0.078740in}}{\pgfqpoint{7.842520in}{7.842520in}}%
\pgfusepath{clip}%
\pgfsetbuttcap%
\pgfsetroundjoin%
\definecolor{currentfill}{rgb}{0.180629,0.429975,0.557282}%
\pgfsetfillcolor{currentfill}%
\pgfsetlinewidth{0.000000pt}%
\definecolor{currentstroke}{rgb}{0.196571,0.711827,0.479221}%
\pgfsetstrokecolor{currentstroke}%
\pgfsetdash{}{0pt}%
\pgfpathmoveto{\pgfqpoint{4.082979in}{4.695694in}}%
\pgfpathlineto{\pgfqpoint{4.003782in}{4.784962in}}%
\pgfpathlineto{\pgfqpoint{4.136438in}{4.716235in}}%
\pgfpathclose%
\pgfusepath{fill}%
\end{pgfscope}%
\begin{pgfscope}%
\pgfpathrectangle{\pgfqpoint{0.539299in}{0.078740in}}{\pgfqpoint{7.842520in}{7.842520in}}%
\pgfusepath{clip}%
\pgfsetbuttcap%
\pgfsetroundjoin%
\definecolor{currentfill}{rgb}{0.171176,0.452530,0.557965}%
\pgfsetfillcolor{currentfill}%
\pgfsetlinewidth{0.000000pt}%
\definecolor{currentstroke}{rgb}{0.202219,0.715272,0.476084}%
\pgfsetstrokecolor{currentstroke}%
\pgfsetdash{}{0pt}%
\pgfpathmoveto{\pgfqpoint{3.791614in}{4.877687in}}%
\pgfpathlineto{\pgfqpoint{3.871838in}{4.810044in}}%
\pgfpathlineto{\pgfqpoint{3.740970in}{4.779433in}}%
\pgfpathclose%
\pgfusepath{fill}%
\end{pgfscope}%
\begin{pgfscope}%
\pgfpathrectangle{\pgfqpoint{0.539299in}{0.078740in}}{\pgfqpoint{7.842520in}{7.842520in}}%
\pgfusepath{clip}%
\pgfsetbuttcap%
\pgfsetroundjoin%
\definecolor{currentfill}{rgb}{0.281887,0.150881,0.465405}%
\pgfsetfillcolor{currentfill}%
\pgfsetlinewidth{0.000000pt}%
\definecolor{currentstroke}{rgb}{0.208030,0.718701,0.472873}%
\pgfsetstrokecolor{currentstroke}%
\pgfsetdash{}{0pt}%
\pgfpathmoveto{\pgfqpoint{5.279908in}{3.760172in}}%
\pgfpathlineto{\pgfqpoint{5.487835in}{3.685615in}}%
\pgfpathlineto{\pgfqpoint{5.353780in}{3.739197in}}%
\pgfpathclose%
\pgfusepath{fill}%
\end{pgfscope}%
\begin{pgfscope}%
\pgfpathrectangle{\pgfqpoint{0.539299in}{0.078740in}}{\pgfqpoint{7.842520in}{7.842520in}}%
\pgfusepath{clip}%
\pgfsetbuttcap%
\pgfsetroundjoin%
\definecolor{currentfill}{rgb}{0.279574,0.170599,0.479997}%
\pgfsetfillcolor{currentfill}%
\pgfsetlinewidth{0.000000pt}%
\definecolor{currentstroke}{rgb}{0.214000,0.722114,0.469588}%
\pgfsetstrokecolor{currentstroke}%
\pgfsetdash{}{0pt}%
\pgfpathmoveto{\pgfqpoint{5.220192in}{3.803938in}}%
\pgfpathlineto{\pgfqpoint{5.145984in}{3.837674in}}%
\pgfpathlineto{\pgfqpoint{5.279908in}{3.760172in}}%
\pgfpathclose%
\pgfusepath{fill}%
\end{pgfscope}%
\begin{pgfscope}%
\pgfpathrectangle{\pgfqpoint{0.539299in}{0.078740in}}{\pgfqpoint{7.842520in}{7.842520in}}%
\pgfusepath{clip}%
\pgfsetbuttcap%
\pgfsetroundjoin%
\definecolor{currentfill}{rgb}{0.225863,0.330805,0.547314}%
\pgfsetfillcolor{currentfill}%
\pgfsetlinewidth{0.000000pt}%
\definecolor{currentstroke}{rgb}{0.220124,0.725509,0.466226}%
\pgfsetstrokecolor{currentstroke}%
\pgfsetdash{}{0pt}%
\pgfpathmoveto{\pgfqpoint{4.536301in}{4.363851in}}%
\pgfpathlineto{\pgfqpoint{4.613151in}{4.272715in}}%
\pgfpathlineto{\pgfqpoint{4.480330in}{4.397499in}}%
\pgfpathclose%
\pgfusepath{fill}%
\end{pgfscope}%
\begin{pgfscope}%
\pgfpathrectangle{\pgfqpoint{0.539299in}{0.078740in}}{\pgfqpoint{7.842520in}{7.842520in}}%
\pgfusepath{clip}%
\pgfsetbuttcap%
\pgfsetroundjoin%
\definecolor{currentfill}{rgb}{0.277018,0.050344,0.375715}%
\pgfsetfillcolor{currentfill}%
\pgfsetlinewidth{0.000000pt}%
\definecolor{currentstroke}{rgb}{0.226397,0.728888,0.462789}%
\pgfsetstrokecolor{currentstroke}%
\pgfsetdash{}{0pt}%
\pgfpathmoveto{\pgfqpoint{6.579539in}{3.406827in}}%
\pgfpathlineto{\pgfqpoint{6.649027in}{3.409472in}}%
\pgfpathlineto{\pgfqpoint{6.513170in}{3.455462in}}%
\pgfpathclose%
\pgfusepath{fill}%
\end{pgfscope}%
\begin{pgfscope}%
\pgfpathrectangle{\pgfqpoint{0.539299in}{0.078740in}}{\pgfqpoint{7.842520in}{7.842520in}}%
\pgfusepath{clip}%
\pgfsetbuttcap%
\pgfsetroundjoin%
\definecolor{currentfill}{rgb}{0.169646,0.456262,0.558030}%
\pgfsetfillcolor{currentfill}%
\pgfsetlinewidth{0.000000pt}%
\definecolor{currentstroke}{rgb}{0.232815,0.732247,0.459277}%
\pgfsetstrokecolor{currentstroke}%
\pgfsetdash{}{0pt}%
\pgfpathmoveto{\pgfqpoint{3.740970in}{4.779433in}}%
\pgfpathlineto{\pgfqpoint{3.660600in}{4.826856in}}%
\pgfpathlineto{\pgfqpoint{3.791614in}{4.877687in}}%
\pgfpathclose%
\pgfusepath{fill}%
\end{pgfscope}%
\begin{pgfscope}%
\pgfpathrectangle{\pgfqpoint{0.539299in}{0.078740in}}{\pgfqpoint{7.842520in}{7.842520in}}%
\pgfusepath{clip}%
\pgfsetbuttcap%
\pgfsetroundjoin%
\definecolor{currentfill}{rgb}{0.281446,0.084320,0.407414}%
\pgfsetfillcolor{currentfill}%
\pgfsetlinewidth{0.000000pt}%
\definecolor{currentstroke}{rgb}{0.239374,0.735588,0.455688}%
\pgfsetstrokecolor{currentstroke}%
\pgfsetdash{}{0pt}%
\pgfpathmoveto{\pgfqpoint{6.307514in}{3.501723in}}%
\pgfpathlineto{\pgfqpoint{6.171939in}{3.541802in}}%
\pgfpathlineto{\pgfqpoint{6.100728in}{3.537174in}}%
\pgfpathclose%
\pgfusepath{fill}%
\end{pgfscope}%
\begin{pgfscope}%
\pgfpathrectangle{\pgfqpoint{0.539299in}{0.078740in}}{\pgfqpoint{7.842520in}{7.842520in}}%
\pgfusepath{clip}%
\pgfsetbuttcap%
\pgfsetroundjoin%
\definecolor{currentfill}{rgb}{0.239346,0.300855,0.540844}%
\pgfsetfillcolor{currentfill}%
\pgfsetlinewidth{0.000000pt}%
\definecolor{currentstroke}{rgb}{0.246070,0.738910,0.452024}%
\pgfsetstrokecolor{currentstroke}%
\pgfsetdash{}{0pt}%
\pgfpathmoveto{\pgfqpoint{4.669832in}{4.230954in}}%
\pgfpathlineto{\pgfqpoint{4.746079in}{4.149433in}}%
\pgfpathlineto{\pgfqpoint{4.613151in}{4.272715in}}%
\pgfpathclose%
\pgfusepath{fill}%
\end{pgfscope}%
\begin{pgfscope}%
\pgfpathrectangle{\pgfqpoint{0.539299in}{0.078740in}}{\pgfqpoint{7.842520in}{7.842520in}}%
\pgfusepath{clip}%
\pgfsetbuttcap%
\pgfsetroundjoin%
\definecolor{currentfill}{rgb}{0.282327,0.094955,0.417331}%
\pgfsetfillcolor{currentfill}%
\pgfsetlinewidth{0.000000pt}%
\definecolor{currentstroke}{rgb}{0.252899,0.742211,0.448284}%
\pgfsetstrokecolor{currentstroke}%
\pgfsetdash{}{0pt}%
\pgfpathmoveto{\pgfqpoint{6.100728in}{3.537174in}}%
\pgfpathlineto{\pgfqpoint{6.171939in}{3.541802in}}%
\pgfpathlineto{\pgfqpoint{5.965125in}{3.573380in}}%
\pgfpathclose%
\pgfusepath{fill}%
\end{pgfscope}%
\begin{pgfscope}%
\pgfpathrectangle{\pgfqpoint{0.539299in}{0.078740in}}{\pgfqpoint{7.842520in}{7.842520in}}%
\pgfusepath{clip}%
\pgfsetbuttcap%
\pgfsetroundjoin%
\definecolor{currentfill}{rgb}{0.187231,0.414746,0.556547}%
\pgfsetfillcolor{currentfill}%
\pgfsetlinewidth{0.000000pt}%
\definecolor{currentstroke}{rgb}{0.259857,0.745492,0.444467}%
\pgfsetstrokecolor{currentstroke}%
\pgfsetdash{}{0pt}%
\pgfpathmoveto{\pgfqpoint{4.136438in}{4.716235in}}%
\pgfpathlineto{\pgfqpoint{4.269526in}{4.615818in}}%
\pgfpathlineto{\pgfqpoint{4.215129in}{4.619034in}}%
\pgfpathclose%
\pgfusepath{fill}%
\end{pgfscope}%
\begin{pgfscope}%
\pgfpathrectangle{\pgfqpoint{0.539299in}{0.078740in}}{\pgfqpoint{7.842520in}{7.842520in}}%
\pgfusepath{clip}%
\pgfsetbuttcap%
\pgfsetroundjoin%
\definecolor{currentfill}{rgb}{0.283091,0.110553,0.431554}%
\pgfsetfillcolor{currentfill}%
\pgfsetlinewidth{0.000000pt}%
\definecolor{currentstroke}{rgb}{0.266941,0.748751,0.440573}%
\pgfsetstrokecolor{currentstroke}%
\pgfsetdash{}{0pt}%
\pgfpathmoveto{\pgfqpoint{5.965125in}{3.573380in}}%
\pgfpathlineto{\pgfqpoint{5.829924in}{3.607475in}}%
\pgfpathlineto{\pgfqpoint{5.757427in}{3.601206in}}%
\pgfpathclose%
\pgfusepath{fill}%
\end{pgfscope}%
\begin{pgfscope}%
\pgfpathrectangle{\pgfqpoint{0.539299in}{0.078740in}}{\pgfqpoint{7.842520in}{7.842520in}}%
\pgfusepath{clip}%
\pgfsetbuttcap%
\pgfsetroundjoin%
\definecolor{currentfill}{rgb}{0.248629,0.278775,0.534556}%
\pgfsetfillcolor{currentfill}%
\pgfsetlinewidth{0.000000pt}%
\definecolor{currentstroke}{rgb}{0.274149,0.751988,0.436601}%
\pgfsetstrokecolor{currentstroke}%
\pgfsetdash{}{0pt}%
\pgfpathmoveto{\pgfqpoint{4.879148in}{4.033520in}}%
\pgfpathlineto{\pgfqpoint{4.746079in}{4.149433in}}%
\pgfpathlineto{\pgfqpoint{4.669832in}{4.230954in}}%
\pgfpathclose%
\pgfusepath{fill}%
\end{pgfscope}%
\begin{pgfscope}%
\pgfpathrectangle{\pgfqpoint{0.539299in}{0.078740in}}{\pgfqpoint{7.842520in}{7.842520in}}%
\pgfusepath{clip}%
\pgfsetbuttcap%
\pgfsetroundjoin%
\definecolor{currentfill}{rgb}{0.283229,0.120777,0.440584}%
\pgfsetfillcolor{currentfill}%
\pgfsetlinewidth{0.000000pt}%
\definecolor{currentstroke}{rgb}{0.281477,0.755203,0.432552}%
\pgfsetstrokecolor{currentstroke}%
\pgfsetdash{}{0pt}%
\pgfpathmoveto{\pgfqpoint{5.757427in}{3.601206in}}%
\pgfpathlineto{\pgfqpoint{5.829924in}{3.607475in}}%
\pgfpathlineto{\pgfqpoint{5.622384in}{3.640649in}}%
\pgfpathclose%
\pgfusepath{fill}%
\end{pgfscope}%
\begin{pgfscope}%
\pgfpathrectangle{\pgfqpoint{0.539299in}{0.078740in}}{\pgfqpoint{7.842520in}{7.842520in}}%
\pgfusepath{clip}%
\pgfsetbuttcap%
\pgfsetroundjoin%
\definecolor{currentfill}{rgb}{0.171176,0.452530,0.557965}%
\pgfsetfillcolor{currentfill}%
\pgfsetlinewidth{0.000000pt}%
\definecolor{currentstroke}{rgb}{0.288921,0.758394,0.428426}%
\pgfsetstrokecolor{currentstroke}%
\pgfsetdash{}{0pt}%
\pgfpathmoveto{\pgfqpoint{3.579639in}{4.867750in}}%
\pgfpathlineto{\pgfqpoint{3.660600in}{4.826856in}}%
\pgfpathlineto{\pgfqpoint{3.531309in}{4.704589in}}%
\pgfpathclose%
\pgfusepath{fill}%
\end{pgfscope}%
\begin{pgfscope}%
\pgfpathrectangle{\pgfqpoint{0.539299in}{0.078740in}}{\pgfqpoint{7.842520in}{7.842520in}}%
\pgfusepath{clip}%
\pgfsetbuttcap%
\pgfsetroundjoin%
\definecolor{currentfill}{rgb}{0.278791,0.062145,0.386592}%
\pgfsetfillcolor{currentfill}%
\pgfsetlinewidth{0.000000pt}%
\definecolor{currentstroke}{rgb}{0.296479,0.761561,0.424223}%
\pgfsetstrokecolor{currentstroke}%
\pgfsetdash{}{0pt}%
\pgfpathmoveto{\pgfqpoint{6.513170in}{3.455462in}}%
\pgfpathlineto{\pgfqpoint{6.443391in}{3.456772in}}%
\pgfpathlineto{\pgfqpoint{6.579539in}{3.406827in}}%
\pgfpathclose%
\pgfusepath{fill}%
\end{pgfscope}%
\begin{pgfscope}%
\pgfpathrectangle{\pgfqpoint{0.539299in}{0.078740in}}{\pgfqpoint{7.842520in}{7.842520in}}%
\pgfusepath{clip}%
\pgfsetbuttcap%
\pgfsetroundjoin%
\definecolor{currentfill}{rgb}{0.174274,0.445044,0.557792}%
\pgfsetfillcolor{currentfill}%
\pgfsetlinewidth{0.000000pt}%
\definecolor{currentstroke}{rgb}{0.304148,0.764704,0.419943}%
\pgfsetstrokecolor{currentstroke}%
\pgfsetdash{}{0pt}%
\pgfpathmoveto{\pgfqpoint{3.531309in}{4.704589in}}%
\pgfpathlineto{\pgfqpoint{3.450467in}{4.721409in}}%
\pgfpathlineto{\pgfqpoint{3.579639in}{4.867750in}}%
\pgfpathclose%
\pgfusepath{fill}%
\end{pgfscope}%
\begin{pgfscope}%
\pgfpathrectangle{\pgfqpoint{0.539299in}{0.078740in}}{\pgfqpoint{7.842520in}{7.842520in}}%
\pgfusepath{clip}%
\pgfsetbuttcap%
\pgfsetroundjoin%
\definecolor{currentfill}{rgb}{0.168126,0.459988,0.558082}%
\pgfsetfillcolor{currentfill}%
\pgfsetlinewidth{0.000000pt}%
\definecolor{currentstroke}{rgb}{0.311925,0.767822,0.415586}%
\pgfsetstrokecolor{currentstroke}%
\pgfsetdash{}{0pt}%
\pgfpathmoveto{\pgfqpoint{4.003782in}{4.784962in}}%
\pgfpathlineto{\pgfqpoint{3.871838in}{4.810044in}}%
\pgfpathlineto{\pgfqpoint{3.791614in}{4.877687in}}%
\pgfpathclose%
\pgfusepath{fill}%
\end{pgfscope}%
\begin{pgfscope}%
\pgfpathrectangle{\pgfqpoint{0.539299in}{0.078740in}}{\pgfqpoint{7.842520in}{7.842520in}}%
\pgfusepath{clip}%
\pgfsetbuttcap%
\pgfsetroundjoin%
\definecolor{currentfill}{rgb}{0.276022,0.044167,0.370164}%
\pgfsetfillcolor{currentfill}%
\pgfsetlinewidth{0.000000pt}%
\definecolor{currentstroke}{rgb}{0.319809,0.770914,0.411152}%
\pgfsetstrokecolor{currentstroke}%
\pgfsetdash{}{0pt}%
\pgfpathmoveto{\pgfqpoint{6.785212in}{3.360948in}}%
\pgfpathlineto{\pgfqpoint{6.649027in}{3.409472in}}%
\pgfpathlineto{\pgfqpoint{6.579539in}{3.406827in}}%
\pgfpathclose%
\pgfusepath{fill}%
\end{pgfscope}%
\begin{pgfscope}%
\pgfpathrectangle{\pgfqpoint{0.539299in}{0.078740in}}{\pgfqpoint{7.842520in}{7.842520in}}%
\pgfusepath{clip}%
\pgfsetbuttcap%
\pgfsetroundjoin%
\definecolor{currentfill}{rgb}{0.258965,0.251537,0.524736}%
\pgfsetfillcolor{currentfill}%
\pgfsetlinewidth{0.000000pt}%
\definecolor{currentstroke}{rgb}{0.327796,0.773980,0.406640}%
\pgfsetstrokecolor{currentstroke}%
\pgfsetdash{}{0pt}%
\pgfpathmoveto{\pgfqpoint{4.803456in}{4.103048in}}%
\pgfpathlineto{\pgfqpoint{5.012423in}{3.928911in}}%
\pgfpathlineto{\pgfqpoint{4.879148in}{4.033520in}}%
\pgfpathclose%
\pgfusepath{fill}%
\end{pgfscope}%
\begin{pgfscope}%
\pgfpathrectangle{\pgfqpoint{0.539299in}{0.078740in}}{\pgfqpoint{7.842520in}{7.842520in}}%
\pgfusepath{clip}%
\pgfsetbuttcap%
\pgfsetroundjoin%
\definecolor{currentfill}{rgb}{0.201239,0.383670,0.554294}%
\pgfsetfillcolor{currentfill}%
\pgfsetlinewidth{0.000000pt}%
\definecolor{currentstroke}{rgb}{0.335885,0.777018,0.402049}%
\pgfsetstrokecolor{currentstroke}%
\pgfsetdash{}{0pt}%
\pgfpathmoveto{\pgfqpoint{4.480330in}{4.397499in}}%
\pgfpathlineto{\pgfqpoint{4.269526in}{4.615818in}}%
\pgfpathlineto{\pgfqpoint{4.402851in}{4.494989in}}%
\pgfpathclose%
\pgfusepath{fill}%
\end{pgfscope}%
\begin{pgfscope}%
\pgfpathrectangle{\pgfqpoint{0.539299in}{0.078740in}}{\pgfqpoint{7.842520in}{7.842520in}}%
\pgfusepath{clip}%
\pgfsetbuttcap%
\pgfsetroundjoin%
\definecolor{currentfill}{rgb}{0.185556,0.418570,0.556753}%
\pgfsetfillcolor{currentfill}%
\pgfsetlinewidth{0.000000pt}%
\definecolor{currentstroke}{rgb}{0.344074,0.780029,0.397381}%
\pgfsetstrokecolor{currentstroke}%
\pgfsetdash{}{0pt}%
\pgfpathmoveto{\pgfqpoint{3.369083in}{4.733483in}}%
\pgfpathlineto{\pgfqpoint{3.450467in}{4.721409in}}%
\pgfpathlineto{\pgfqpoint{3.242761in}{4.478380in}}%
\pgfpathclose%
\pgfusepath{fill}%
\end{pgfscope}%
\begin{pgfscope}%
\pgfpathrectangle{\pgfqpoint{0.539299in}{0.078740in}}{\pgfqpoint{7.842520in}{7.842520in}}%
\pgfusepath{clip}%
\pgfsetbuttcap%
\pgfsetroundjoin%
\definecolor{currentfill}{rgb}{0.210503,0.363727,0.552206}%
\pgfsetfillcolor{currentfill}%
\pgfsetlinewidth{0.000000pt}%
\definecolor{currentstroke}{rgb}{0.352360,0.783011,0.392636}%
\pgfsetstrokecolor{currentstroke}%
\pgfsetdash{}{0pt}%
\pgfpathmoveto{\pgfqpoint{4.480330in}{4.397499in}}%
\pgfpathlineto{\pgfqpoint{4.402851in}{4.494989in}}%
\pgfpathlineto{\pgfqpoint{4.536301in}{4.363851in}}%
\pgfpathclose%
\pgfusepath{fill}%
\end{pgfscope}%
\begin{pgfscope}%
\pgfpathrectangle{\pgfqpoint{0.539299in}{0.078740in}}{\pgfqpoint{7.842520in}{7.842520in}}%
\pgfusepath{clip}%
\pgfsetbuttcap%
\pgfsetroundjoin%
\definecolor{currentfill}{rgb}{0.281412,0.155834,0.469201}%
\pgfsetfillcolor{currentfill}%
\pgfsetlinewidth{0.000000pt}%
\definecolor{currentstroke}{rgb}{0.360741,0.785964,0.387814}%
\pgfsetstrokecolor{currentstroke}%
\pgfsetdash{}{0pt}%
\pgfpathmoveto{\pgfqpoint{5.414261in}{3.695329in}}%
\pgfpathlineto{\pgfqpoint{5.487835in}{3.685615in}}%
\pgfpathlineto{\pgfqpoint{5.279908in}{3.760172in}}%
\pgfpathclose%
\pgfusepath{fill}%
\end{pgfscope}%
\begin{pgfscope}%
\pgfpathrectangle{\pgfqpoint{0.539299in}{0.078740in}}{\pgfqpoint{7.842520in}{7.842520in}}%
\pgfusepath{clip}%
\pgfsetbuttcap%
\pgfsetroundjoin%
\definecolor{currentfill}{rgb}{0.282623,0.140926,0.457517}%
\pgfsetfillcolor{currentfill}%
\pgfsetlinewidth{0.000000pt}%
\definecolor{currentstroke}{rgb}{0.369214,0.788888,0.382914}%
\pgfsetstrokecolor{currentstroke}%
\pgfsetdash{}{0pt}%
\pgfpathmoveto{\pgfqpoint{5.487835in}{3.685615in}}%
\pgfpathlineto{\pgfqpoint{5.549086in}{3.640953in}}%
\pgfpathlineto{\pgfqpoint{5.622384in}{3.640649in}}%
\pgfpathclose%
\pgfusepath{fill}%
\end{pgfscope}%
\begin{pgfscope}%
\pgfpathrectangle{\pgfqpoint{0.539299in}{0.078740in}}{\pgfqpoint{7.842520in}{7.842520in}}%
\pgfusepath{clip}%
\pgfsetbuttcap%
\pgfsetroundjoin%
\definecolor{currentfill}{rgb}{0.225863,0.330805,0.547314}%
\pgfsetfillcolor{currentfill}%
\pgfsetlinewidth{0.000000pt}%
\definecolor{currentstroke}{rgb}{0.377779,0.791781,0.377939}%
\pgfsetstrokecolor{currentstroke}%
\pgfsetdash{}{0pt}%
\pgfpathmoveto{\pgfqpoint{4.669832in}{4.230954in}}%
\pgfpathlineto{\pgfqpoint{4.613151in}{4.272715in}}%
\pgfpathlineto{\pgfqpoint{4.536301in}{4.363851in}}%
\pgfpathclose%
\pgfusepath{fill}%
\end{pgfscope}%
\begin{pgfscope}%
\pgfpathrectangle{\pgfqpoint{0.539299in}{0.078740in}}{\pgfqpoint{7.842520in}{7.842520in}}%
\pgfusepath{clip}%
\pgfsetbuttcap%
\pgfsetroundjoin%
\definecolor{currentfill}{rgb}{0.267968,0.223549,0.512008}%
\pgfsetfillcolor{currentfill}%
\pgfsetlinewidth{0.000000pt}%
\definecolor{currentstroke}{rgb}{0.386433,0.794644,0.372886}%
\pgfsetstrokecolor{currentstroke}%
\pgfsetdash{}{0pt}%
\pgfpathmoveto{\pgfqpoint{5.012423in}{3.928911in}}%
\pgfpathlineto{\pgfqpoint{4.937227in}{3.984970in}}%
\pgfpathlineto{\pgfqpoint{5.145984in}{3.837674in}}%
\pgfpathclose%
\pgfusepath{fill}%
\end{pgfscope}%
\begin{pgfscope}%
\pgfpathrectangle{\pgfqpoint{0.539299in}{0.078740in}}{\pgfqpoint{7.842520in}{7.842520in}}%
\pgfusepath{clip}%
\pgfsetbuttcap%
\pgfsetroundjoin%
\definecolor{currentfill}{rgb}{0.282327,0.094955,0.417331}%
\pgfsetfillcolor{currentfill}%
\pgfsetlinewidth{0.000000pt}%
\definecolor{currentstroke}{rgb}{0.395174,0.797475,0.367757}%
\pgfsetstrokecolor{currentstroke}%
\pgfsetdash{}{0pt}%
\pgfpathmoveto{\pgfqpoint{6.100728in}{3.537174in}}%
\pgfpathlineto{\pgfqpoint{6.236676in}{3.496772in}}%
\pgfpathlineto{\pgfqpoint{6.307514in}{3.501723in}}%
\pgfpathclose%
\pgfusepath{fill}%
\end{pgfscope}%
\begin{pgfscope}%
\pgfpathrectangle{\pgfqpoint{0.539299in}{0.078740in}}{\pgfqpoint{7.842520in}{7.842520in}}%
\pgfusepath{clip}%
\pgfsetbuttcap%
\pgfsetroundjoin%
\definecolor{currentfill}{rgb}{0.221989,0.339161,0.548752}%
\pgfsetfillcolor{currentfill}%
\pgfsetlinewidth{0.000000pt}%
\definecolor{currentstroke}{rgb}{0.404001,0.800275,0.362552}%
\pgfsetstrokecolor{currentstroke}%
\pgfsetdash{}{0pt}%
\pgfpathmoveto{\pgfqpoint{2.957481in}{4.048102in}}%
\pgfpathlineto{\pgfqpoint{3.079268in}{4.442627in}}%
\pgfpathlineto{\pgfqpoint{3.161266in}{4.461876in}}%
\pgfpathclose%
\pgfusepath{fill}%
\end{pgfscope}%
\begin{pgfscope}%
\pgfpathrectangle{\pgfqpoint{0.539299in}{0.078740in}}{\pgfqpoint{7.842520in}{7.842520in}}%
\pgfusepath{clip}%
\pgfsetbuttcap%
\pgfsetroundjoin%
\definecolor{currentfill}{rgb}{0.281446,0.084320,0.407414}%
\pgfsetfillcolor{currentfill}%
\pgfsetlinewidth{0.000000pt}%
\definecolor{currentstroke}{rgb}{0.412913,0.803041,0.357269}%
\pgfsetstrokecolor{currentstroke}%
\pgfsetdash{}{0pt}%
\pgfpathmoveto{\pgfqpoint{6.372915in}{3.450815in}}%
\pgfpathlineto{\pgfqpoint{6.443391in}{3.456772in}}%
\pgfpathlineto{\pgfqpoint{6.307514in}{3.501723in}}%
\pgfpathclose%
\pgfusepath{fill}%
\end{pgfscope}%
\begin{pgfscope}%
\pgfpathrectangle{\pgfqpoint{0.539299in}{0.078740in}}{\pgfqpoint{7.842520in}{7.842520in}}%
\pgfusepath{clip}%
\pgfsetbuttcap%
\pgfsetroundjoin%
\definecolor{currentfill}{rgb}{0.244972,0.287675,0.537260}%
\pgfsetfillcolor{currentfill}%
\pgfsetlinewidth{0.000000pt}%
\definecolor{currentstroke}{rgb}{0.421908,0.805774,0.351910}%
\pgfsetstrokecolor{currentstroke}%
\pgfsetdash{}{0pt}%
\pgfpathmoveto{\pgfqpoint{4.669832in}{4.230954in}}%
\pgfpathlineto{\pgfqpoint{4.803456in}{4.103048in}}%
\pgfpathlineto{\pgfqpoint{4.879148in}{4.033520in}}%
\pgfpathclose%
\pgfusepath{fill}%
\end{pgfscope}%
\begin{pgfscope}%
\pgfpathrectangle{\pgfqpoint{0.539299in}{0.078740in}}{\pgfqpoint{7.842520in}{7.842520in}}%
\pgfusepath{clip}%
\pgfsetbuttcap%
\pgfsetroundjoin%
\definecolor{currentfill}{rgb}{0.283229,0.120777,0.440584}%
\pgfsetfillcolor{currentfill}%
\pgfsetlinewidth{0.000000pt}%
\definecolor{currentstroke}{rgb}{0.430983,0.808473,0.346476}%
\pgfsetstrokecolor{currentstroke}%
\pgfsetdash{}{0pt}%
\pgfpathmoveto{\pgfqpoint{5.757427in}{3.601206in}}%
\pgfpathlineto{\pgfqpoint{5.892938in}{3.564025in}}%
\pgfpathlineto{\pgfqpoint{5.965125in}{3.573380in}}%
\pgfpathclose%
\pgfusepath{fill}%
\end{pgfscope}%
\begin{pgfscope}%
\pgfpathrectangle{\pgfqpoint{0.539299in}{0.078740in}}{\pgfqpoint{7.842520in}{7.842520in}}%
\pgfusepath{clip}%
\pgfsetbuttcap%
\pgfsetroundjoin%
\definecolor{currentfill}{rgb}{0.281887,0.150881,0.465405}%
\pgfsetfillcolor{currentfill}%
\pgfsetlinewidth{0.000000pt}%
\definecolor{currentstroke}{rgb}{0.440137,0.811138,0.340967}%
\pgfsetstrokecolor{currentstroke}%
\pgfsetdash{}{0pt}%
\pgfpathmoveto{\pgfqpoint{5.414261in}{3.695329in}}%
\pgfpathlineto{\pgfqpoint{5.549086in}{3.640953in}}%
\pgfpathlineto{\pgfqpoint{5.487835in}{3.685615in}}%
\pgfpathclose%
\pgfusepath{fill}%
\end{pgfscope}%
\begin{pgfscope}%
\pgfpathrectangle{\pgfqpoint{0.539299in}{0.078740in}}{\pgfqpoint{7.842520in}{7.842520in}}%
\pgfusepath{clip}%
\pgfsetbuttcap%
\pgfsetroundjoin%
\definecolor{currentfill}{rgb}{0.274952,0.037752,0.364543}%
\pgfsetfillcolor{currentfill}%
\pgfsetlinewidth{0.000000pt}%
\definecolor{currentstroke}{rgb}{0.449368,0.813768,0.335384}%
\pgfsetstrokecolor{currentstroke}%
\pgfsetdash{}{0pt}%
\pgfpathmoveto{\pgfqpoint{6.921738in}{3.310623in}}%
\pgfpathlineto{\pgfqpoint{6.785212in}{3.360948in}}%
\pgfpathlineto{\pgfqpoint{6.715949in}{3.352378in}}%
\pgfpathclose%
\pgfusepath{fill}%
\end{pgfscope}%
\begin{pgfscope}%
\pgfpathrectangle{\pgfqpoint{0.539299in}{0.078740in}}{\pgfqpoint{7.842520in}{7.842520in}}%
\pgfusepath{clip}%
\pgfsetbuttcap%
\pgfsetroundjoin%
\definecolor{currentfill}{rgb}{0.277941,0.056324,0.381191}%
\pgfsetfillcolor{currentfill}%
\pgfsetlinewidth{0.000000pt}%
\definecolor{currentstroke}{rgb}{0.458674,0.816363,0.329727}%
\pgfsetstrokecolor{currentstroke}%
\pgfsetdash{}{0pt}%
\pgfpathmoveto{\pgfqpoint{6.715949in}{3.352378in}}%
\pgfpathlineto{\pgfqpoint{6.785212in}{3.360948in}}%
\pgfpathlineto{\pgfqpoint{6.579539in}{3.406827in}}%
\pgfpathclose%
\pgfusepath{fill}%
\end{pgfscope}%
\begin{pgfscope}%
\pgfpathrectangle{\pgfqpoint{0.539299in}{0.078740in}}{\pgfqpoint{7.842520in}{7.842520in}}%
\pgfusepath{clip}%
\pgfsetbuttcap%
\pgfsetroundjoin%
\definecolor{currentfill}{rgb}{0.283197,0.115680,0.436115}%
\pgfsetfillcolor{currentfill}%
\pgfsetlinewidth{0.000000pt}%
\definecolor{currentstroke}{rgb}{0.468053,0.818921,0.323998}%
\pgfsetstrokecolor{currentstroke}%
\pgfsetdash{}{0pt}%
\pgfpathmoveto{\pgfqpoint{6.100728in}{3.537174in}}%
\pgfpathlineto{\pgfqpoint{5.965125in}{3.573380in}}%
\pgfpathlineto{\pgfqpoint{6.028872in}{3.526024in}}%
\pgfpathclose%
\pgfusepath{fill}%
\end{pgfscope}%
\begin{pgfscope}%
\pgfpathrectangle{\pgfqpoint{0.539299in}{0.078740in}}{\pgfqpoint{7.842520in}{7.842520in}}%
\pgfusepath{clip}%
\pgfsetbuttcap%
\pgfsetroundjoin%
\definecolor{currentfill}{rgb}{0.257322,0.256130,0.526563}%
\pgfsetfillcolor{currentfill}%
\pgfsetlinewidth{0.000000pt}%
\definecolor{currentstroke}{rgb}{0.477504,0.821444,0.318195}%
\pgfsetstrokecolor{currentstroke}%
\pgfsetdash{}{0pt}%
\pgfpathmoveto{\pgfqpoint{4.803456in}{4.103048in}}%
\pgfpathlineto{\pgfqpoint{4.937227in}{3.984970in}}%
\pgfpathlineto{\pgfqpoint{5.012423in}{3.928911in}}%
\pgfpathclose%
\pgfusepath{fill}%
\end{pgfscope}%
\begin{pgfscope}%
\pgfpathrectangle{\pgfqpoint{0.539299in}{0.078740in}}{\pgfqpoint{7.842520in}{7.842520in}}%
\pgfusepath{clip}%
\pgfsetbuttcap%
\pgfsetroundjoin%
\definecolor{currentfill}{rgb}{0.239346,0.300855,0.540844}%
\pgfsetfillcolor{currentfill}%
\pgfsetlinewidth{0.000000pt}%
\definecolor{currentstroke}{rgb}{0.487026,0.823929,0.312321}%
\pgfsetstrokecolor{currentstroke}%
\pgfsetdash{}{0pt}%
\pgfpathmoveto{\pgfqpoint{2.875698in}{4.003213in}}%
\pgfpathlineto{\pgfqpoint{2.996771in}{4.420700in}}%
\pgfpathlineto{\pgfqpoint{2.957481in}{4.048102in}}%
\pgfpathclose%
\pgfusepath{fill}%
\end{pgfscope}%
\begin{pgfscope}%
\pgfpathrectangle{\pgfqpoint{0.539299in}{0.078740in}}{\pgfqpoint{7.842520in}{7.842520in}}%
\pgfusepath{clip}%
\pgfsetbuttcap%
\pgfsetroundjoin%
\definecolor{currentfill}{rgb}{0.275191,0.194905,0.496005}%
\pgfsetfillcolor{currentfill}%
\pgfsetlinewidth{0.000000pt}%
\definecolor{currentstroke}{rgb}{0.496615,0.826376,0.306377}%
\pgfsetstrokecolor{currentstroke}%
\pgfsetdash{}{0pt}%
\pgfpathmoveto{\pgfqpoint{5.205512in}{3.788278in}}%
\pgfpathlineto{\pgfqpoint{5.279908in}{3.760172in}}%
\pgfpathlineto{\pgfqpoint{5.145984in}{3.837674in}}%
\pgfpathclose%
\pgfusepath{fill}%
\end{pgfscope}%
\begin{pgfscope}%
\pgfpathrectangle{\pgfqpoint{0.539299in}{0.078740in}}{\pgfqpoint{7.842520in}{7.842520in}}%
\pgfusepath{clip}%
\pgfsetbuttcap%
\pgfsetroundjoin%
\definecolor{currentfill}{rgb}{0.160665,0.478540,0.558115}%
\pgfsetfillcolor{currentfill}%
\pgfsetlinewidth{0.000000pt}%
\definecolor{currentstroke}{rgb}{0.506271,0.828786,0.300362}%
\pgfsetstrokecolor{currentstroke}%
\pgfsetdash{}{0pt}%
\pgfpathmoveto{\pgfqpoint{3.791614in}{4.877687in}}%
\pgfpathlineto{\pgfqpoint{3.660600in}{4.826856in}}%
\pgfpathlineto{\pgfqpoint{3.579639in}{4.867750in}}%
\pgfpathclose%
\pgfusepath{fill}%
\end{pgfscope}%
\begin{pgfscope}%
\pgfpathrectangle{\pgfqpoint{0.539299in}{0.078740in}}{\pgfqpoint{7.842520in}{7.842520in}}%
\pgfusepath{clip}%
\pgfsetbuttcap%
\pgfsetroundjoin%
\definecolor{currentfill}{rgb}{0.282327,0.094955,0.417331}%
\pgfsetfillcolor{currentfill}%
\pgfsetlinewidth{0.000000pt}%
\definecolor{currentstroke}{rgb}{0.515992,0.831158,0.294279}%
\pgfsetstrokecolor{currentstroke}%
\pgfsetdash{}{0pt}%
\pgfpathmoveto{\pgfqpoint{6.307514in}{3.501723in}}%
\pgfpathlineto{\pgfqpoint{6.236676in}{3.496772in}}%
\pgfpathlineto{\pgfqpoint{6.372915in}{3.450815in}}%
\pgfpathclose%
\pgfusepath{fill}%
\end{pgfscope}%
\begin{pgfscope}%
\pgfpathrectangle{\pgfqpoint{0.539299in}{0.078740in}}{\pgfqpoint{7.842520in}{7.842520in}}%
\pgfusepath{clip}%
\pgfsetbuttcap%
\pgfsetroundjoin%
\definecolor{currentfill}{rgb}{0.169646,0.456262,0.558030}%
\pgfsetfillcolor{currentfill}%
\pgfsetlinewidth{0.000000pt}%
\definecolor{currentstroke}{rgb}{0.525776,0.833491,0.288127}%
\pgfsetstrokecolor{currentstroke}%
\pgfsetdash{}{0pt}%
\pgfpathmoveto{\pgfqpoint{4.003782in}{4.784962in}}%
\pgfpathlineto{\pgfqpoint{4.057001in}{4.809605in}}%
\pgfpathlineto{\pgfqpoint{4.136438in}{4.716235in}}%
\pgfpathclose%
\pgfusepath{fill}%
\end{pgfscope}%
\begin{pgfscope}%
\pgfpathrectangle{\pgfqpoint{0.539299in}{0.078740in}}{\pgfqpoint{7.842520in}{7.842520in}}%
\pgfusepath{clip}%
\pgfsetbuttcap%
\pgfsetroundjoin%
\definecolor{currentfill}{rgb}{0.282884,0.135920,0.453427}%
\pgfsetfillcolor{currentfill}%
\pgfsetlinewidth{0.000000pt}%
\definecolor{currentstroke}{rgb}{0.535621,0.835785,0.281908}%
\pgfsetstrokecolor{currentstroke}%
\pgfsetdash{}{0pt}%
\pgfpathmoveto{\pgfqpoint{5.757427in}{3.601206in}}%
\pgfpathlineto{\pgfqpoint{5.622384in}{3.640649in}}%
\pgfpathlineto{\pgfqpoint{5.684398in}{3.594113in}}%
\pgfpathclose%
\pgfusepath{fill}%
\end{pgfscope}%
\begin{pgfscope}%
\pgfpathrectangle{\pgfqpoint{0.539299in}{0.078740in}}{\pgfqpoint{7.842520in}{7.842520in}}%
\pgfusepath{clip}%
\pgfsetbuttcap%
\pgfsetroundjoin%
\definecolor{currentfill}{rgb}{0.162142,0.474838,0.558140}%
\pgfsetfillcolor{currentfill}%
\pgfsetlinewidth{0.000000pt}%
\definecolor{currentstroke}{rgb}{0.545524,0.838039,0.275626}%
\pgfsetstrokecolor{currentstroke}%
\pgfsetdash{}{0pt}%
\pgfpathmoveto{\pgfqpoint{3.923882in}{4.868004in}}%
\pgfpathlineto{\pgfqpoint{4.003782in}{4.784962in}}%
\pgfpathlineto{\pgfqpoint{3.791614in}{4.877687in}}%
\pgfpathclose%
\pgfusepath{fill}%
\end{pgfscope}%
\begin{pgfscope}%
\pgfpathrectangle{\pgfqpoint{0.539299in}{0.078740in}}{\pgfqpoint{7.842520in}{7.842520in}}%
\pgfusepath{clip}%
\pgfsetbuttcap%
\pgfsetroundjoin%
\definecolor{currentfill}{rgb}{0.190631,0.407061,0.556089}%
\pgfsetfillcolor{currentfill}%
\pgfsetlinewidth{0.000000pt}%
\definecolor{currentstroke}{rgb}{0.555484,0.840254,0.269281}%
\pgfsetstrokecolor{currentstroke}%
\pgfsetdash{}{0pt}%
\pgfpathmoveto{\pgfqpoint{3.242761in}{4.478380in}}%
\pgfpathlineto{\pgfqpoint{3.161266in}{4.461876in}}%
\pgfpathlineto{\pgfqpoint{3.287152in}{4.741607in}}%
\pgfpathclose%
\pgfusepath{fill}%
\end{pgfscope}%
\begin{pgfscope}%
\pgfpathrectangle{\pgfqpoint{0.539299in}{0.078740in}}{\pgfqpoint{7.842520in}{7.842520in}}%
\pgfusepath{clip}%
\pgfsetbuttcap%
\pgfsetroundjoin%
\definecolor{currentfill}{rgb}{0.281446,0.084320,0.407414}%
\pgfsetfillcolor{currentfill}%
\pgfsetlinewidth{0.000000pt}%
\definecolor{currentstroke}{rgb}{0.565498,0.842430,0.262877}%
\pgfsetstrokecolor{currentstroke}%
\pgfsetdash{}{0pt}%
\pgfpathmoveto{\pgfqpoint{6.579539in}{3.406827in}}%
\pgfpathlineto{\pgfqpoint{6.443391in}{3.456772in}}%
\pgfpathlineto{\pgfqpoint{6.372915in}{3.450815in}}%
\pgfpathclose%
\pgfusepath{fill}%
\end{pgfscope}%
\begin{pgfscope}%
\pgfpathrectangle{\pgfqpoint{0.539299in}{0.078740in}}{\pgfqpoint{7.842520in}{7.842520in}}%
\pgfusepath{clip}%
\pgfsetbuttcap%
\pgfsetroundjoin%
\definecolor{currentfill}{rgb}{0.283229,0.120777,0.440584}%
\pgfsetfillcolor{currentfill}%
\pgfsetlinewidth{0.000000pt}%
\definecolor{currentstroke}{rgb}{0.575563,0.844566,0.256415}%
\pgfsetstrokecolor{currentstroke}%
\pgfsetdash{}{0pt}%
\pgfpathmoveto{\pgfqpoint{6.028872in}{3.526024in}}%
\pgfpathlineto{\pgfqpoint{5.965125in}{3.573380in}}%
\pgfpathlineto{\pgfqpoint{5.892938in}{3.564025in}}%
\pgfpathclose%
\pgfusepath{fill}%
\end{pgfscope}%
\begin{pgfscope}%
\pgfpathrectangle{\pgfqpoint{0.539299in}{0.078740in}}{\pgfqpoint{7.842520in}{7.842520in}}%
\pgfusepath{clip}%
\pgfsetbuttcap%
\pgfsetroundjoin%
\definecolor{currentfill}{rgb}{0.265145,0.232956,0.516599}%
\pgfsetfillcolor{currentfill}%
\pgfsetlinewidth{0.000000pt}%
\definecolor{currentstroke}{rgb}{0.585678,0.846661,0.249897}%
\pgfsetstrokecolor{currentstroke}%
\pgfsetdash{}{0pt}%
\pgfpathmoveto{\pgfqpoint{5.145984in}{3.837674in}}%
\pgfpathlineto{\pgfqpoint{4.937227in}{3.984970in}}%
\pgfpathlineto{\pgfqpoint{5.071218in}{3.879659in}}%
\pgfpathclose%
\pgfusepath{fill}%
\end{pgfscope}%
\begin{pgfscope}%
\pgfpathrectangle{\pgfqpoint{0.539299in}{0.078740in}}{\pgfqpoint{7.842520in}{7.842520in}}%
\pgfusepath{clip}%
\pgfsetbuttcap%
\pgfsetroundjoin%
\definecolor{currentfill}{rgb}{0.172719,0.448791,0.557885}%
\pgfsetfillcolor{currentfill}%
\pgfsetlinewidth{0.000000pt}%
\definecolor{currentstroke}{rgb}{0.595839,0.848717,0.243329}%
\pgfsetstrokecolor{currentstroke}%
\pgfsetdash{}{0pt}%
\pgfpathmoveto{\pgfqpoint{4.136438in}{4.716235in}}%
\pgfpathlineto{\pgfqpoint{4.057001in}{4.809605in}}%
\pgfpathlineto{\pgfqpoint{4.269526in}{4.615818in}}%
\pgfpathclose%
\pgfusepath{fill}%
\end{pgfscope}%
\begin{pgfscope}%
\pgfpathrectangle{\pgfqpoint{0.539299in}{0.078740in}}{\pgfqpoint{7.842520in}{7.842520in}}%
\pgfusepath{clip}%
\pgfsetbuttcap%
\pgfsetroundjoin%
\definecolor{currentfill}{rgb}{0.282290,0.145912,0.461510}%
\pgfsetfillcolor{currentfill}%
\pgfsetlinewidth{0.000000pt}%
\definecolor{currentstroke}{rgb}{0.606045,0.850733,0.236712}%
\pgfsetstrokecolor{currentstroke}%
\pgfsetdash{}{0pt}%
\pgfpathmoveto{\pgfqpoint{5.622384in}{3.640649in}}%
\pgfpathlineto{\pgfqpoint{5.549086in}{3.640953in}}%
\pgfpathlineto{\pgfqpoint{5.684398in}{3.594113in}}%
\pgfpathclose%
\pgfusepath{fill}%
\end{pgfscope}%
\begin{pgfscope}%
\pgfpathrectangle{\pgfqpoint{0.539299in}{0.078740in}}{\pgfqpoint{7.842520in}{7.842520in}}%
\pgfusepath{clip}%
\pgfsetbuttcap%
\pgfsetroundjoin%
\definecolor{currentfill}{rgb}{0.179019,0.433756,0.557430}%
\pgfsetfillcolor{currentfill}%
\pgfsetlinewidth{0.000000pt}%
\definecolor{currentstroke}{rgb}{0.616293,0.852709,0.230052}%
\pgfsetstrokecolor{currentstroke}%
\pgfsetdash{}{0pt}%
\pgfpathmoveto{\pgfqpoint{3.287152in}{4.741607in}}%
\pgfpathlineto{\pgfqpoint{3.369083in}{4.733483in}}%
\pgfpathlineto{\pgfqpoint{3.242761in}{4.478380in}}%
\pgfpathclose%
\pgfusepath{fill}%
\end{pgfscope}%
\begin{pgfscope}%
\pgfpathrectangle{\pgfqpoint{0.539299in}{0.078740in}}{\pgfqpoint{7.842520in}{7.842520in}}%
\pgfusepath{clip}%
\pgfsetbuttcap%
\pgfsetroundjoin%
\definecolor{currentfill}{rgb}{0.270595,0.214069,0.507052}%
\pgfsetfillcolor{currentfill}%
\pgfsetlinewidth{0.000000pt}%
\definecolor{currentstroke}{rgb}{0.626579,0.854645,0.223353}%
\pgfsetstrokecolor{currentstroke}%
\pgfsetdash{}{0pt}%
\pgfpathmoveto{\pgfqpoint{5.071218in}{3.879659in}}%
\pgfpathlineto{\pgfqpoint{5.205512in}{3.788278in}}%
\pgfpathlineto{\pgfqpoint{5.145984in}{3.837674in}}%
\pgfpathclose%
\pgfusepath{fill}%
\end{pgfscope}%
\begin{pgfscope}%
\pgfpathrectangle{\pgfqpoint{0.539299in}{0.078740in}}{\pgfqpoint{7.842520in}{7.842520in}}%
\pgfusepath{clip}%
\pgfsetbuttcap%
\pgfsetroundjoin%
\definecolor{currentfill}{rgb}{0.278012,0.180367,0.486697}%
\pgfsetfillcolor{currentfill}%
\pgfsetlinewidth{0.000000pt}%
\definecolor{currentstroke}{rgb}{0.636902,0.856542,0.216620}%
\pgfsetstrokecolor{currentstroke}%
\pgfsetdash{}{0pt}%
\pgfpathmoveto{\pgfqpoint{5.414261in}{3.695329in}}%
\pgfpathlineto{\pgfqpoint{5.279908in}{3.760172in}}%
\pgfpathlineto{\pgfqpoint{5.340183in}{3.710444in}}%
\pgfpathclose%
\pgfusepath{fill}%
\end{pgfscope}%
\begin{pgfscope}%
\pgfpathrectangle{\pgfqpoint{0.539299in}{0.078740in}}{\pgfqpoint{7.842520in}{7.842520in}}%
\pgfusepath{clip}%
\pgfsetbuttcap%
\pgfsetroundjoin%
\definecolor{currentfill}{rgb}{0.276022,0.044167,0.370164}%
\pgfsetfillcolor{currentfill}%
\pgfsetlinewidth{0.000000pt}%
\definecolor{currentstroke}{rgb}{0.647257,0.858400,0.209861}%
\pgfsetstrokecolor{currentstroke}%
\pgfsetdash{}{0pt}%
\pgfpathmoveto{\pgfqpoint{6.715949in}{3.352378in}}%
\pgfpathlineto{\pgfqpoint{6.852629in}{3.294416in}}%
\pgfpathlineto{\pgfqpoint{6.921738in}{3.310623in}}%
\pgfpathclose%
\pgfusepath{fill}%
\end{pgfscope}%
\begin{pgfscope}%
\pgfpathrectangle{\pgfqpoint{0.539299in}{0.078740in}}{\pgfqpoint{7.842520in}{7.842520in}}%
\pgfusepath{clip}%
\pgfsetbuttcap%
\pgfsetroundjoin%
\definecolor{currentfill}{rgb}{0.160665,0.478540,0.558115}%
\pgfsetfillcolor{currentfill}%
\pgfsetlinewidth{0.000000pt}%
\definecolor{currentstroke}{rgb}{0.657642,0.860219,0.203082}%
\pgfsetstrokecolor{currentstroke}%
\pgfsetdash{}{0pt}%
\pgfpathmoveto{\pgfqpoint{3.923882in}{4.868004in}}%
\pgfpathlineto{\pgfqpoint{4.057001in}{4.809605in}}%
\pgfpathlineto{\pgfqpoint{4.003782in}{4.784962in}}%
\pgfpathclose%
\pgfusepath{fill}%
\end{pgfscope}%
\begin{pgfscope}%
\pgfpathrectangle{\pgfqpoint{0.539299in}{0.078740in}}{\pgfqpoint{7.842520in}{7.842520in}}%
\pgfusepath{clip}%
\pgfsetbuttcap%
\pgfsetroundjoin%
\definecolor{currentfill}{rgb}{0.272594,0.025563,0.353093}%
\pgfsetfillcolor{currentfill}%
\pgfsetlinewidth{0.000000pt}%
\definecolor{currentstroke}{rgb}{0.668054,0.861999,0.196293}%
\pgfsetstrokecolor{currentstroke}%
\pgfsetdash{}{0pt}%
\pgfpathmoveto{\pgfqpoint{6.989604in}{3.234274in}}%
\pgfpathlineto{\pgfqpoint{7.058628in}{3.259387in}}%
\pgfpathlineto{\pgfqpoint{6.921738in}{3.310623in}}%
\pgfpathclose%
\pgfusepath{fill}%
\end{pgfscope}%
\begin{pgfscope}%
\pgfpathrectangle{\pgfqpoint{0.539299in}{0.078740in}}{\pgfqpoint{7.842520in}{7.842520in}}%
\pgfusepath{clip}%
\pgfsetbuttcap%
\pgfsetroundjoin%
\definecolor{currentfill}{rgb}{0.283197,0.115680,0.436115}%
\pgfsetfillcolor{currentfill}%
\pgfsetlinewidth{0.000000pt}%
\definecolor{currentstroke}{rgb}{0.678489,0.863742,0.189503}%
\pgfsetstrokecolor{currentstroke}%
\pgfsetdash{}{0pt}%
\pgfpathmoveto{\pgfqpoint{6.165167in}{3.484596in}}%
\pgfpathlineto{\pgfqpoint{6.236676in}{3.496772in}}%
\pgfpathlineto{\pgfqpoint{6.100728in}{3.537174in}}%
\pgfpathclose%
\pgfusepath{fill}%
\end{pgfscope}%
\begin{pgfscope}%
\pgfpathrectangle{\pgfqpoint{0.539299in}{0.078740in}}{\pgfqpoint{7.842520in}{7.842520in}}%
\pgfusepath{clip}%
\pgfsetbuttcap%
\pgfsetroundjoin%
\definecolor{currentfill}{rgb}{0.216210,0.351535,0.550627}%
\pgfsetfillcolor{currentfill}%
\pgfsetlinewidth{0.000000pt}%
\definecolor{currentstroke}{rgb}{0.688944,0.865448,0.182725}%
\pgfsetstrokecolor{currentstroke}%
\pgfsetdash{}{0pt}%
\pgfpathmoveto{\pgfqpoint{2.996771in}{4.420700in}}%
\pgfpathlineto{\pgfqpoint{3.079268in}{4.442627in}}%
\pgfpathlineto{\pgfqpoint{2.957481in}{4.048102in}}%
\pgfpathclose%
\pgfusepath{fill}%
\end{pgfscope}%
\begin{pgfscope}%
\pgfpathrectangle{\pgfqpoint{0.539299in}{0.078740in}}{\pgfqpoint{7.842520in}{7.842520in}}%
\pgfusepath{clip}%
\pgfsetbuttcap%
\pgfsetroundjoin%
\definecolor{currentfill}{rgb}{0.185556,0.418570,0.556753}%
\pgfsetfillcolor{currentfill}%
\pgfsetlinewidth{0.000000pt}%
\definecolor{currentstroke}{rgb}{0.699415,0.867117,0.175971}%
\pgfsetstrokecolor{currentstroke}%
\pgfsetdash{}{0pt}%
\pgfpathmoveto{\pgfqpoint{4.402851in}{4.494989in}}%
\pgfpathlineto{\pgfqpoint{4.269526in}{4.615818in}}%
\pgfpathlineto{\pgfqpoint{4.324600in}{4.594132in}}%
\pgfpathclose%
\pgfusepath{fill}%
\end{pgfscope}%
\begin{pgfscope}%
\pgfpathrectangle{\pgfqpoint{0.539299in}{0.078740in}}{\pgfqpoint{7.842520in}{7.842520in}}%
\pgfusepath{clip}%
\pgfsetbuttcap%
\pgfsetroundjoin%
\definecolor{currentfill}{rgb}{0.282884,0.135920,0.453427}%
\pgfsetfillcolor{currentfill}%
\pgfsetlinewidth{0.000000pt}%
\definecolor{currentstroke}{rgb}{0.709898,0.868751,0.169257}%
\pgfsetstrokecolor{currentstroke}%
\pgfsetdash{}{0pt}%
\pgfpathmoveto{\pgfqpoint{5.820187in}{3.551515in}}%
\pgfpathlineto{\pgfqpoint{5.892938in}{3.564025in}}%
\pgfpathlineto{\pgfqpoint{5.757427in}{3.601206in}}%
\pgfpathclose%
\pgfusepath{fill}%
\end{pgfscope}%
\begin{pgfscope}%
\pgfpathrectangle{\pgfqpoint{0.539299in}{0.078740in}}{\pgfqpoint{7.842520in}{7.842520in}}%
\pgfusepath{clip}%
\pgfsetbuttcap%
\pgfsetroundjoin%
\definecolor{currentfill}{rgb}{0.163625,0.471133,0.558148}%
\pgfsetfillcolor{currentfill}%
\pgfsetlinewidth{0.000000pt}%
\definecolor{currentstroke}{rgb}{0.720391,0.870350,0.162603}%
\pgfsetstrokecolor{currentstroke}%
\pgfsetdash{}{0pt}%
\pgfpathmoveto{\pgfqpoint{3.498080in}{4.903513in}}%
\pgfpathlineto{\pgfqpoint{3.450467in}{4.721409in}}%
\pgfpathlineto{\pgfqpoint{3.369083in}{4.733483in}}%
\pgfpathclose%
\pgfusepath{fill}%
\end{pgfscope}%
\begin{pgfscope}%
\pgfpathrectangle{\pgfqpoint{0.539299in}{0.078740in}}{\pgfqpoint{7.842520in}{7.842520in}}%
\pgfusepath{clip}%
\pgfsetbuttcap%
\pgfsetroundjoin%
\definecolor{currentfill}{rgb}{0.194100,0.399323,0.555565}%
\pgfsetfillcolor{currentfill}%
\pgfsetlinewidth{0.000000pt}%
\definecolor{currentstroke}{rgb}{0.730889,0.871916,0.156029}%
\pgfsetstrokecolor{currentstroke}%
\pgfsetdash{}{0pt}%
\pgfpathmoveto{\pgfqpoint{4.536301in}{4.363851in}}%
\pgfpathlineto{\pgfqpoint{4.402851in}{4.494989in}}%
\pgfpathlineto{\pgfqpoint{4.324600in}{4.594132in}}%
\pgfpathclose%
\pgfusepath{fill}%
\end{pgfscope}%
\begin{pgfscope}%
\pgfpathrectangle{\pgfqpoint{0.539299in}{0.078740in}}{\pgfqpoint{7.842520in}{7.842520in}}%
\pgfusepath{clip}%
\pgfsetbuttcap%
\pgfsetroundjoin%
\definecolor{currentfill}{rgb}{0.159194,0.482237,0.558073}%
\pgfsetfillcolor{currentfill}%
\pgfsetlinewidth{0.000000pt}%
\definecolor{currentstroke}{rgb}{0.741388,0.873449,0.149561}%
\pgfsetstrokecolor{currentstroke}%
\pgfsetdash{}{0pt}%
\pgfpathmoveto{\pgfqpoint{3.579639in}{4.867750in}}%
\pgfpathlineto{\pgfqpoint{3.450467in}{4.721409in}}%
\pgfpathlineto{\pgfqpoint{3.498080in}{4.903513in}}%
\pgfpathclose%
\pgfusepath{fill}%
\end{pgfscope}%
\begin{pgfscope}%
\pgfpathrectangle{\pgfqpoint{0.539299in}{0.078740in}}{\pgfqpoint{7.842520in}{7.842520in}}%
\pgfusepath{clip}%
\pgfsetbuttcap%
\pgfsetroundjoin%
\definecolor{currentfill}{rgb}{0.278826,0.175490,0.483397}%
\pgfsetfillcolor{currentfill}%
\pgfsetlinewidth{0.000000pt}%
\definecolor{currentstroke}{rgb}{0.751884,0.874951,0.143228}%
\pgfsetstrokecolor{currentstroke}%
\pgfsetdash{}{0pt}%
\pgfpathmoveto{\pgfqpoint{5.340183in}{3.710444in}}%
\pgfpathlineto{\pgfqpoint{5.549086in}{3.640953in}}%
\pgfpathlineto{\pgfqpoint{5.414261in}{3.695329in}}%
\pgfpathclose%
\pgfusepath{fill}%
\end{pgfscope}%
\begin{pgfscope}%
\pgfpathrectangle{\pgfqpoint{0.539299in}{0.078740in}}{\pgfqpoint{7.842520in}{7.842520in}}%
\pgfusepath{clip}%
\pgfsetbuttcap%
\pgfsetroundjoin%
\definecolor{currentfill}{rgb}{0.237441,0.305202,0.541921}%
\pgfsetfillcolor{currentfill}%
\pgfsetlinewidth{0.000000pt}%
\definecolor{currentstroke}{rgb}{0.762373,0.876424,0.137064}%
\pgfsetstrokecolor{currentstroke}%
\pgfsetdash{}{0pt}%
\pgfpathmoveto{\pgfqpoint{2.793452in}{3.956696in}}%
\pgfpathlineto{\pgfqpoint{2.996771in}{4.420700in}}%
\pgfpathlineto{\pgfqpoint{2.875698in}{4.003213in}}%
\pgfpathclose%
\pgfusepath{fill}%
\end{pgfscope}%
\begin{pgfscope}%
\pgfpathrectangle{\pgfqpoint{0.539299in}{0.078740in}}{\pgfqpoint{7.842520in}{7.842520in}}%
\pgfusepath{clip}%
\pgfsetbuttcap%
\pgfsetroundjoin%
\definecolor{currentfill}{rgb}{0.281924,0.089666,0.412415}%
\pgfsetfillcolor{currentfill}%
\pgfsetlinewidth{0.000000pt}%
\definecolor{currentstroke}{rgb}{0.772852,0.877868,0.131109}%
\pgfsetstrokecolor{currentstroke}%
\pgfsetdash{}{0pt}%
\pgfpathmoveto{\pgfqpoint{6.372915in}{3.450815in}}%
\pgfpathlineto{\pgfqpoint{6.509400in}{3.398734in}}%
\pgfpathlineto{\pgfqpoint{6.579539in}{3.406827in}}%
\pgfpathclose%
\pgfusepath{fill}%
\end{pgfscope}%
\begin{pgfscope}%
\pgfpathrectangle{\pgfqpoint{0.539299in}{0.078740in}}{\pgfqpoint{7.842520in}{7.842520in}}%
\pgfusepath{clip}%
\pgfsetbuttcap%
\pgfsetroundjoin%
\definecolor{currentfill}{rgb}{0.275191,0.194905,0.496005}%
\pgfsetfillcolor{currentfill}%
\pgfsetlinewidth{0.000000pt}%
\definecolor{currentstroke}{rgb}{0.783315,0.879285,0.125405}%
\pgfsetstrokecolor{currentstroke}%
\pgfsetdash{}{0pt}%
\pgfpathmoveto{\pgfqpoint{5.340183in}{3.710444in}}%
\pgfpathlineto{\pgfqpoint{5.279908in}{3.760172in}}%
\pgfpathlineto{\pgfqpoint{5.205512in}{3.788278in}}%
\pgfpathclose%
\pgfusepath{fill}%
\end{pgfscope}%
\begin{pgfscope}%
\pgfpathrectangle{\pgfqpoint{0.539299in}{0.078740in}}{\pgfqpoint{7.842520in}{7.842520in}}%
\pgfusepath{clip}%
\pgfsetbuttcap%
\pgfsetroundjoin%
\definecolor{currentfill}{rgb}{0.283229,0.120777,0.440584}%
\pgfsetfillcolor{currentfill}%
\pgfsetlinewidth{0.000000pt}%
\definecolor{currentstroke}{rgb}{0.793760,0.880678,0.120005}%
\pgfsetstrokecolor{currentstroke}%
\pgfsetdash{}{0pt}%
\pgfpathmoveto{\pgfqpoint{6.100728in}{3.537174in}}%
\pgfpathlineto{\pgfqpoint{6.028872in}{3.526024in}}%
\pgfpathlineto{\pgfqpoint{6.165167in}{3.484596in}}%
\pgfpathclose%
\pgfusepath{fill}%
\end{pgfscope}%
\begin{pgfscope}%
\pgfpathrectangle{\pgfqpoint{0.539299in}{0.078740in}}{\pgfqpoint{7.842520in}{7.842520in}}%
\pgfusepath{clip}%
\pgfsetbuttcap%
\pgfsetroundjoin%
\definecolor{currentfill}{rgb}{0.274952,0.037752,0.364543}%
\pgfsetfillcolor{currentfill}%
\pgfsetlinewidth{0.000000pt}%
\definecolor{currentstroke}{rgb}{0.804182,0.882046,0.114965}%
\pgfsetstrokecolor{currentstroke}%
\pgfsetdash{}{0pt}%
\pgfpathmoveto{\pgfqpoint{6.921738in}{3.310623in}}%
\pgfpathlineto{\pgfqpoint{6.852629in}{3.294416in}}%
\pgfpathlineto{\pgfqpoint{6.989604in}{3.234274in}}%
\pgfpathclose%
\pgfusepath{fill}%
\end{pgfscope}%
\begin{pgfscope}%
\pgfpathrectangle{\pgfqpoint{0.539299in}{0.078740in}}{\pgfqpoint{7.842520in}{7.842520in}}%
\pgfusepath{clip}%
\pgfsetbuttcap%
\pgfsetroundjoin%
\definecolor{currentfill}{rgb}{0.208623,0.367752,0.552675}%
\pgfsetfillcolor{currentfill}%
\pgfsetlinewidth{0.000000pt}%
\definecolor{currentstroke}{rgb}{0.814576,0.883393,0.110347}%
\pgfsetstrokecolor{currentstroke}%
\pgfsetdash{}{0pt}%
\pgfpathmoveto{\pgfqpoint{4.536301in}{4.363851in}}%
\pgfpathlineto{\pgfqpoint{4.458695in}{4.459156in}}%
\pgfpathlineto{\pgfqpoint{4.669832in}{4.230954in}}%
\pgfpathclose%
\pgfusepath{fill}%
\end{pgfscope}%
\begin{pgfscope}%
\pgfpathrectangle{\pgfqpoint{0.539299in}{0.078740in}}{\pgfqpoint{7.842520in}{7.842520in}}%
\pgfusepath{clip}%
\pgfsetbuttcap%
\pgfsetroundjoin%
\definecolor{currentfill}{rgb}{0.280267,0.073417,0.397163}%
\pgfsetfillcolor{currentfill}%
\pgfsetlinewidth{0.000000pt}%
\definecolor{currentstroke}{rgb}{0.824940,0.884720,0.106217}%
\pgfsetstrokecolor{currentstroke}%
\pgfsetdash{}{0pt}%
\pgfpathmoveto{\pgfqpoint{6.579539in}{3.406827in}}%
\pgfpathlineto{\pgfqpoint{6.646106in}{3.340736in}}%
\pgfpathlineto{\pgfqpoint{6.715949in}{3.352378in}}%
\pgfpathclose%
\pgfusepath{fill}%
\end{pgfscope}%
\begin{pgfscope}%
\pgfpathrectangle{\pgfqpoint{0.539299in}{0.078740in}}{\pgfqpoint{7.842520in}{7.842520in}}%
\pgfusepath{clip}%
\pgfsetbuttcap%
\pgfsetroundjoin%
\definecolor{currentfill}{rgb}{0.282290,0.145912,0.461510}%
\pgfsetfillcolor{currentfill}%
\pgfsetlinewidth{0.000000pt}%
\definecolor{currentstroke}{rgb}{0.835270,0.886029,0.102646}%
\pgfsetstrokecolor{currentstroke}%
\pgfsetdash{}{0pt}%
\pgfpathmoveto{\pgfqpoint{5.757427in}{3.601206in}}%
\pgfpathlineto{\pgfqpoint{5.684398in}{3.594113in}}%
\pgfpathlineto{\pgfqpoint{5.820187in}{3.551515in}}%
\pgfpathclose%
\pgfusepath{fill}%
\end{pgfscope}%
\begin{pgfscope}%
\pgfpathrectangle{\pgfqpoint{0.539299in}{0.078740in}}{\pgfqpoint{7.842520in}{7.842520in}}%
\pgfusepath{clip}%
\pgfsetbuttcap%
\pgfsetroundjoin%
\definecolor{currentfill}{rgb}{0.283091,0.110553,0.431554}%
\pgfsetfillcolor{currentfill}%
\pgfsetlinewidth{0.000000pt}%
\definecolor{currentstroke}{rgb}{0.845561,0.887322,0.099702}%
\pgfsetstrokecolor{currentstroke}%
\pgfsetdash{}{0pt}%
\pgfpathmoveto{\pgfqpoint{6.372915in}{3.450815in}}%
\pgfpathlineto{\pgfqpoint{6.236676in}{3.496772in}}%
\pgfpathlineto{\pgfqpoint{6.165167in}{3.484596in}}%
\pgfpathclose%
\pgfusepath{fill}%
\end{pgfscope}%
\begin{pgfscope}%
\pgfpathrectangle{\pgfqpoint{0.539299in}{0.078740in}}{\pgfqpoint{7.842520in}{7.842520in}}%
\pgfusepath{clip}%
\pgfsetbuttcap%
\pgfsetroundjoin%
\definecolor{currentfill}{rgb}{0.151918,0.500685,0.557587}%
\pgfsetfillcolor{currentfill}%
\pgfsetlinewidth{0.000000pt}%
\definecolor{currentstroke}{rgb}{0.855810,0.888601,0.097452}%
\pgfsetstrokecolor{currentstroke}%
\pgfsetdash{}{0pt}%
\pgfpathmoveto{\pgfqpoint{3.579639in}{4.867750in}}%
\pgfpathlineto{\pgfqpoint{3.710739in}{4.939290in}}%
\pgfpathlineto{\pgfqpoint{3.791614in}{4.877687in}}%
\pgfpathclose%
\pgfusepath{fill}%
\end{pgfscope}%
\begin{pgfscope}%
\pgfpathrectangle{\pgfqpoint{0.539299in}{0.078740in}}{\pgfqpoint{7.842520in}{7.842520in}}%
\pgfusepath{clip}%
\pgfsetbuttcap%
\pgfsetroundjoin%
\definecolor{currentfill}{rgb}{0.229739,0.322361,0.545706}%
\pgfsetfillcolor{currentfill}%
\pgfsetlinewidth{0.000000pt}%
\definecolor{currentstroke}{rgb}{0.866013,0.889868,0.095953}%
\pgfsetstrokecolor{currentstroke}%
\pgfsetdash{}{0pt}%
\pgfpathmoveto{\pgfqpoint{4.727082in}{4.180259in}}%
\pgfpathlineto{\pgfqpoint{4.803456in}{4.103048in}}%
\pgfpathlineto{\pgfqpoint{4.669832in}{4.230954in}}%
\pgfpathclose%
\pgfusepath{fill}%
\end{pgfscope}%
\begin{pgfscope}%
\pgfpathrectangle{\pgfqpoint{0.539299in}{0.078740in}}{\pgfqpoint{7.842520in}{7.842520in}}%
\pgfusepath{clip}%
\pgfsetbuttcap%
\pgfsetroundjoin%
\definecolor{currentfill}{rgb}{0.185556,0.418570,0.556753}%
\pgfsetfillcolor{currentfill}%
\pgfsetlinewidth{0.000000pt}%
\definecolor{currentstroke}{rgb}{0.876168,0.891125,0.095250}%
\pgfsetstrokecolor{currentstroke}%
\pgfsetdash{}{0pt}%
\pgfpathmoveto{\pgfqpoint{3.287152in}{4.741607in}}%
\pgfpathlineto{\pgfqpoint{3.161266in}{4.461876in}}%
\pgfpathlineto{\pgfqpoint{3.079268in}{4.442627in}}%
\pgfpathclose%
\pgfusepath{fill}%
\end{pgfscope}%
\begin{pgfscope}%
\pgfpathrectangle{\pgfqpoint{0.539299in}{0.078740in}}{\pgfqpoint{7.842520in}{7.842520in}}%
\pgfusepath{clip}%
\pgfsetbuttcap%
\pgfsetroundjoin%
\definecolor{currentfill}{rgb}{0.168126,0.459988,0.558082}%
\pgfsetfillcolor{currentfill}%
\pgfsetlinewidth{0.000000pt}%
\definecolor{currentstroke}{rgb}{0.886271,0.892374,0.095374}%
\pgfsetstrokecolor{currentstroke}%
\pgfsetdash{}{0pt}%
\pgfpathmoveto{\pgfqpoint{4.269526in}{4.615818in}}%
\pgfpathlineto{\pgfqpoint{4.057001in}{4.809605in}}%
\pgfpathlineto{\pgfqpoint{4.190652in}{4.714468in}}%
\pgfpathclose%
\pgfusepath{fill}%
\end{pgfscope}%
\begin{pgfscope}%
\pgfpathrectangle{\pgfqpoint{0.539299in}{0.078740in}}{\pgfqpoint{7.842520in}{7.842520in}}%
\pgfusepath{clip}%
\pgfsetbuttcap%
\pgfsetroundjoin%
\definecolor{currentfill}{rgb}{0.271305,0.019942,0.347269}%
\pgfsetfillcolor{currentfill}%
\pgfsetlinewidth{0.000000pt}%
\definecolor{currentstroke}{rgb}{0.896320,0.893616,0.096335}%
\pgfsetstrokecolor{currentstroke}%
\pgfsetdash{}{0pt}%
\pgfpathmoveto{\pgfqpoint{7.126919in}{3.173464in}}%
\pgfpathlineto{\pgfqpoint{7.195913in}{3.208174in}}%
\pgfpathlineto{\pgfqpoint{7.058628in}{3.259387in}}%
\pgfpathclose%
\pgfusepath{fill}%
\end{pgfscope}%
\begin{pgfscope}%
\pgfpathrectangle{\pgfqpoint{0.539299in}{0.078740in}}{\pgfqpoint{7.842520in}{7.842520in}}%
\pgfusepath{clip}%
\pgfsetbuttcap%
\pgfsetroundjoin%
\definecolor{currentfill}{rgb}{0.282623,0.140926,0.457517}%
\pgfsetfillcolor{currentfill}%
\pgfsetlinewidth{0.000000pt}%
\definecolor{currentstroke}{rgb}{0.906311,0.894855,0.098125}%
\pgfsetstrokecolor{currentstroke}%
\pgfsetdash{}{0pt}%
\pgfpathmoveto{\pgfqpoint{5.820187in}{3.551515in}}%
\pgfpathlineto{\pgfqpoint{6.028872in}{3.526024in}}%
\pgfpathlineto{\pgfqpoint{5.892938in}{3.564025in}}%
\pgfpathclose%
\pgfusepath{fill}%
\end{pgfscope}%
\begin{pgfscope}%
\pgfpathrectangle{\pgfqpoint{0.539299in}{0.078740in}}{\pgfqpoint{7.842520in}{7.842520in}}%
\pgfusepath{clip}%
\pgfsetbuttcap%
\pgfsetroundjoin%
\definecolor{currentfill}{rgb}{0.244972,0.287675,0.537260}%
\pgfsetfillcolor{currentfill}%
\pgfsetlinewidth{0.000000pt}%
\definecolor{currentstroke}{rgb}{0.916242,0.896091,0.100717}%
\pgfsetstrokecolor{currentstroke}%
\pgfsetdash{}{0pt}%
\pgfpathmoveto{\pgfqpoint{4.803456in}{4.103048in}}%
\pgfpathlineto{\pgfqpoint{4.861395in}{4.049469in}}%
\pgfpathlineto{\pgfqpoint{4.937227in}{3.984970in}}%
\pgfpathclose%
\pgfusepath{fill}%
\end{pgfscope}%
\begin{pgfscope}%
\pgfpathrectangle{\pgfqpoint{0.539299in}{0.078740in}}{\pgfqpoint{7.842520in}{7.842520in}}%
\pgfusepath{clip}%
\pgfsetbuttcap%
\pgfsetroundjoin%
\definecolor{currentfill}{rgb}{0.281446,0.084320,0.407414}%
\pgfsetfillcolor{currentfill}%
\pgfsetlinewidth{0.000000pt}%
\definecolor{currentstroke}{rgb}{0.926106,0.897330,0.104071}%
\pgfsetstrokecolor{currentstroke}%
\pgfsetdash{}{0pt}%
\pgfpathmoveto{\pgfqpoint{6.509400in}{3.398734in}}%
\pgfpathlineto{\pgfqpoint{6.646106in}{3.340736in}}%
\pgfpathlineto{\pgfqpoint{6.579539in}{3.406827in}}%
\pgfpathclose%
\pgfusepath{fill}%
\end{pgfscope}%
\begin{pgfscope}%
\pgfpathrectangle{\pgfqpoint{0.539299in}{0.078740in}}{\pgfqpoint{7.842520in}{7.842520in}}%
\pgfusepath{clip}%
\pgfsetbuttcap%
\pgfsetroundjoin%
\definecolor{currentfill}{rgb}{0.174274,0.445044,0.557792}%
\pgfsetfillcolor{currentfill}%
\pgfsetlinewidth{0.000000pt}%
\definecolor{currentstroke}{rgb}{0.935904,0.898570,0.108131}%
\pgfsetstrokecolor{currentstroke}%
\pgfsetdash{}{0pt}%
\pgfpathmoveto{\pgfqpoint{4.324600in}{4.594132in}}%
\pgfpathlineto{\pgfqpoint{4.269526in}{4.615818in}}%
\pgfpathlineto{\pgfqpoint{4.190652in}{4.714468in}}%
\pgfpathclose%
\pgfusepath{fill}%
\end{pgfscope}%
\begin{pgfscope}%
\pgfpathrectangle{\pgfqpoint{0.539299in}{0.078740in}}{\pgfqpoint{7.842520in}{7.842520in}}%
\pgfusepath{clip}%
\pgfsetbuttcap%
\pgfsetroundjoin%
\definecolor{currentfill}{rgb}{0.252194,0.269783,0.531579}%
\pgfsetfillcolor{currentfill}%
\pgfsetlinewidth{0.000000pt}%
\definecolor{currentstroke}{rgb}{0.945636,0.899815,0.112838}%
\pgfsetstrokecolor{currentstroke}%
\pgfsetdash{}{0pt}%
\pgfpathmoveto{\pgfqpoint{5.071218in}{3.879659in}}%
\pgfpathlineto{\pgfqpoint{4.937227in}{3.984970in}}%
\pgfpathlineto{\pgfqpoint{4.861395in}{4.049469in}}%
\pgfpathclose%
\pgfusepath{fill}%
\end{pgfscope}%
\begin{pgfscope}%
\pgfpathrectangle{\pgfqpoint{0.539299in}{0.078740in}}{\pgfqpoint{7.842520in}{7.842520in}}%
\pgfusepath{clip}%
\pgfsetbuttcap%
\pgfsetroundjoin%
\definecolor{currentfill}{rgb}{0.280255,0.165693,0.476498}%
\pgfsetfillcolor{currentfill}%
\pgfsetlinewidth{0.000000pt}%
\definecolor{currentstroke}{rgb}{0.955300,0.901065,0.118128}%
\pgfsetstrokecolor{currentstroke}%
\pgfsetdash{}{0pt}%
\pgfpathmoveto{\pgfqpoint{5.684398in}{3.594113in}}%
\pgfpathlineto{\pgfqpoint{5.549086in}{3.640953in}}%
\pgfpathlineto{\pgfqpoint{5.475289in}{3.644517in}}%
\pgfpathclose%
\pgfusepath{fill}%
\end{pgfscope}%
\begin{pgfscope}%
\pgfpathrectangle{\pgfqpoint{0.539299in}{0.078740in}}{\pgfqpoint{7.842520in}{7.842520in}}%
\pgfusepath{clip}%
\pgfsetbuttcap%
\pgfsetroundjoin%
\definecolor{currentfill}{rgb}{0.150476,0.504369,0.557430}%
\pgfsetfillcolor{currentfill}%
\pgfsetlinewidth{0.000000pt}%
\definecolor{currentstroke}{rgb}{0.964894,0.902323,0.123941}%
\pgfsetstrokecolor{currentstroke}%
\pgfsetdash{}{0pt}%
\pgfpathmoveto{\pgfqpoint{3.791614in}{4.877687in}}%
\pgfpathlineto{\pgfqpoint{3.843276in}{4.946277in}}%
\pgfpathlineto{\pgfqpoint{3.923882in}{4.868004in}}%
\pgfpathclose%
\pgfusepath{fill}%
\end{pgfscope}%
\begin{pgfscope}%
\pgfpathrectangle{\pgfqpoint{0.539299in}{0.078740in}}{\pgfqpoint{7.842520in}{7.842520in}}%
\pgfusepath{clip}%
\pgfsetbuttcap%
\pgfsetroundjoin%
\definecolor{currentfill}{rgb}{0.273809,0.031497,0.358853}%
\pgfsetfillcolor{currentfill}%
\pgfsetlinewidth{0.000000pt}%
\definecolor{currentstroke}{rgb}{0.974417,0.903590,0.130215}%
\pgfsetstrokecolor{currentstroke}%
\pgfsetdash{}{0pt}%
\pgfpathmoveto{\pgfqpoint{7.126919in}{3.173464in}}%
\pgfpathlineto{\pgfqpoint{7.058628in}{3.259387in}}%
\pgfpathlineto{\pgfqpoint{6.989604in}{3.234274in}}%
\pgfpathclose%
\pgfusepath{fill}%
\end{pgfscope}%
\begin{pgfscope}%
\pgfpathrectangle{\pgfqpoint{0.539299in}{0.078740in}}{\pgfqpoint{7.842520in}{7.842520in}}%
\pgfusepath{clip}%
\pgfsetbuttcap%
\pgfsetroundjoin%
\definecolor{currentfill}{rgb}{0.278791,0.062145,0.386592}%
\pgfsetfillcolor{currentfill}%
\pgfsetlinewidth{0.000000pt}%
\definecolor{currentstroke}{rgb}{0.983868,0.904867,0.136897}%
\pgfsetstrokecolor{currentstroke}%
\pgfsetdash{}{0pt}%
\pgfpathmoveto{\pgfqpoint{6.783027in}{3.277708in}}%
\pgfpathlineto{\pgfqpoint{6.852629in}{3.294416in}}%
\pgfpathlineto{\pgfqpoint{6.715949in}{3.352378in}}%
\pgfpathclose%
\pgfusepath{fill}%
\end{pgfscope}%
\begin{pgfscope}%
\pgfpathrectangle{\pgfqpoint{0.539299in}{0.078740in}}{\pgfqpoint{7.842520in}{7.842520in}}%
\pgfusepath{clip}%
\pgfsetbuttcap%
\pgfsetroundjoin%
\definecolor{currentfill}{rgb}{0.190631,0.407061,0.556089}%
\pgfsetfillcolor{currentfill}%
\pgfsetlinewidth{0.000000pt}%
\definecolor{currentstroke}{rgb}{0.993248,0.906157,0.143936}%
\pgfsetstrokecolor{currentstroke}%
\pgfsetdash{}{0pt}%
\pgfpathmoveto{\pgfqpoint{4.324600in}{4.594132in}}%
\pgfpathlineto{\pgfqpoint{4.458695in}{4.459156in}}%
\pgfpathlineto{\pgfqpoint{4.536301in}{4.363851in}}%
\pgfpathclose%
\pgfusepath{fill}%
\end{pgfscope}%
\begin{pgfscope}%
\pgfpathrectangle{\pgfqpoint{0.539299in}{0.078740in}}{\pgfqpoint{7.842520in}{7.842520in}}%
\pgfusepath{clip}%
\pgfsetbuttcap%
\pgfsetroundjoin%
\definecolor{currentfill}{rgb}{0.278012,0.180367,0.486697}%
\pgfsetfillcolor{currentfill}%
\pgfsetlinewidth{0.000000pt}%
\definecolor{currentstroke}{rgb}{0.267004,0.004874,0.329415}%
\pgfsetstrokecolor{currentstroke}%
\pgfsetdash{}{0pt}%
\pgfpathmoveto{\pgfqpoint{5.475289in}{3.644517in}}%
\pgfpathlineto{\pgfqpoint{5.549086in}{3.640953in}}%
\pgfpathlineto{\pgfqpoint{5.340183in}{3.710444in}}%
\pgfpathclose%
\pgfusepath{fill}%
\end{pgfscope}%
\begin{pgfscope}%
\pgfpathrectangle{\pgfqpoint{0.539299in}{0.078740in}}{\pgfqpoint{7.842520in}{7.842520in}}%
\pgfusepath{clip}%
\pgfsetbuttcap%
\pgfsetroundjoin%
\definecolor{currentfill}{rgb}{0.271305,0.019942,0.347269}%
\pgfsetfillcolor{currentfill}%
\pgfsetlinewidth{0.000000pt}%
\definecolor{currentstroke}{rgb}{0.268510,0.009605,0.335427}%
\pgfsetstrokecolor{currentstroke}%
\pgfsetdash{}{0pt}%
\pgfpathmoveto{\pgfqpoint{7.333628in}{3.157861in}}%
\pgfpathlineto{\pgfqpoint{7.195913in}{3.208174in}}%
\pgfpathlineto{\pgfqpoint{7.126919in}{3.173464in}}%
\pgfpathclose%
\pgfusepath{fill}%
\end{pgfscope}%
\begin{pgfscope}%
\pgfpathrectangle{\pgfqpoint{0.539299in}{0.078740in}}{\pgfqpoint{7.842520in}{7.842520in}}%
\pgfusepath{clip}%
\pgfsetbuttcap%
\pgfsetroundjoin%
\definecolor{currentfill}{rgb}{0.206756,0.371758,0.553117}%
\pgfsetfillcolor{currentfill}%
\pgfsetlinewidth{0.000000pt}%
\definecolor{currentstroke}{rgb}{0.269944,0.014625,0.341379}%
\pgfsetstrokecolor{currentstroke}%
\pgfsetdash{}{0pt}%
\pgfpathmoveto{\pgfqpoint{4.458695in}{4.459156in}}%
\pgfpathlineto{\pgfqpoint{4.592860in}{4.318703in}}%
\pgfpathlineto{\pgfqpoint{4.669832in}{4.230954in}}%
\pgfpathclose%
\pgfusepath{fill}%
\end{pgfscope}%
\begin{pgfscope}%
\pgfpathrectangle{\pgfqpoint{0.539299in}{0.078740in}}{\pgfqpoint{7.842520in}{7.842520in}}%
\pgfusepath{clip}%
\pgfsetbuttcap%
\pgfsetroundjoin%
\definecolor{currentfill}{rgb}{0.283091,0.110553,0.431554}%
\pgfsetfillcolor{currentfill}%
\pgfsetlinewidth{0.000000pt}%
\definecolor{currentstroke}{rgb}{0.271305,0.019942,0.347269}%
\pgfsetstrokecolor{currentstroke}%
\pgfsetdash{}{0pt}%
\pgfpathmoveto{\pgfqpoint{6.301764in}{3.437824in}}%
\pgfpathlineto{\pgfqpoint{6.509400in}{3.398734in}}%
\pgfpathlineto{\pgfqpoint{6.372915in}{3.450815in}}%
\pgfpathclose%
\pgfusepath{fill}%
\end{pgfscope}%
\begin{pgfscope}%
\pgfpathrectangle{\pgfqpoint{0.539299in}{0.078740in}}{\pgfqpoint{7.842520in}{7.842520in}}%
\pgfusepath{clip}%
\pgfsetbuttcap%
\pgfsetroundjoin%
\definecolor{currentfill}{rgb}{0.216210,0.351535,0.550627}%
\pgfsetfillcolor{currentfill}%
\pgfsetlinewidth{0.000000pt}%
\definecolor{currentstroke}{rgb}{0.272594,0.025563,0.353093}%
\pgfsetstrokecolor{currentstroke}%
\pgfsetdash{}{0pt}%
\pgfpathmoveto{\pgfqpoint{4.669832in}{4.230954in}}%
\pgfpathlineto{\pgfqpoint{4.592860in}{4.318703in}}%
\pgfpathlineto{\pgfqpoint{4.727082in}{4.180259in}}%
\pgfpathclose%
\pgfusepath{fill}%
\end{pgfscope}%
\begin{pgfscope}%
\pgfpathrectangle{\pgfqpoint{0.539299in}{0.078740in}}{\pgfqpoint{7.842520in}{7.842520in}}%
\pgfusepath{clip}%
\pgfsetbuttcap%
\pgfsetroundjoin%
\definecolor{currentfill}{rgb}{0.283229,0.120777,0.440584}%
\pgfsetfillcolor{currentfill}%
\pgfsetlinewidth{0.000000pt}%
\definecolor{currentstroke}{rgb}{0.273809,0.031497,0.358853}%
\pgfsetstrokecolor{currentstroke}%
\pgfsetdash{}{0pt}%
\pgfpathmoveto{\pgfqpoint{6.165167in}{3.484596in}}%
\pgfpathlineto{\pgfqpoint{6.301764in}{3.437824in}}%
\pgfpathlineto{\pgfqpoint{6.372915in}{3.450815in}}%
\pgfpathclose%
\pgfusepath{fill}%
\end{pgfscope}%
\begin{pgfscope}%
\pgfpathrectangle{\pgfqpoint{0.539299in}{0.078740in}}{\pgfqpoint{7.842520in}{7.842520in}}%
\pgfusepath{clip}%
\pgfsetbuttcap%
\pgfsetroundjoin%
\definecolor{currentfill}{rgb}{0.263663,0.237631,0.518762}%
\pgfsetfillcolor{currentfill}%
\pgfsetlinewidth{0.000000pt}%
\definecolor{currentstroke}{rgb}{0.274952,0.037752,0.364543}%
\pgfsetstrokecolor{currentstroke}%
\pgfsetdash{}{0pt}%
\pgfpathmoveto{\pgfqpoint{5.130570in}{3.824189in}}%
\pgfpathlineto{\pgfqpoint{5.205512in}{3.788278in}}%
\pgfpathlineto{\pgfqpoint{5.071218in}{3.879659in}}%
\pgfpathclose%
\pgfusepath{fill}%
\end{pgfscope}%
\begin{pgfscope}%
\pgfpathrectangle{\pgfqpoint{0.539299in}{0.078740in}}{\pgfqpoint{7.842520in}{7.842520in}}%
\pgfusepath{clip}%
\pgfsetbuttcap%
\pgfsetroundjoin%
\definecolor{currentfill}{rgb}{0.146180,0.515413,0.556823}%
\pgfsetfillcolor{currentfill}%
\pgfsetlinewidth{0.000000pt}%
\definecolor{currentstroke}{rgb}{0.276022,0.044167,0.370164}%
\pgfsetstrokecolor{currentstroke}%
\pgfsetdash{}{0pt}%
\pgfpathmoveto{\pgfqpoint{3.710739in}{4.939290in}}%
\pgfpathlineto{\pgfqpoint{3.843276in}{4.946277in}}%
\pgfpathlineto{\pgfqpoint{3.791614in}{4.877687in}}%
\pgfpathclose%
\pgfusepath{fill}%
\end{pgfscope}%
\begin{pgfscope}%
\pgfpathrectangle{\pgfqpoint{0.539299in}{0.078740in}}{\pgfqpoint{7.842520in}{7.842520in}}%
\pgfusepath{clip}%
\pgfsetbuttcap%
\pgfsetroundjoin%
\definecolor{currentfill}{rgb}{0.280894,0.078907,0.402329}%
\pgfsetfillcolor{currentfill}%
\pgfsetlinewidth{0.000000pt}%
\definecolor{currentstroke}{rgb}{0.277018,0.050344,0.375715}%
\pgfsetstrokecolor{currentstroke}%
\pgfsetdash{}{0pt}%
\pgfpathmoveto{\pgfqpoint{6.715949in}{3.352378in}}%
\pgfpathlineto{\pgfqpoint{6.646106in}{3.340736in}}%
\pgfpathlineto{\pgfqpoint{6.783027in}{3.277708in}}%
\pgfpathclose%
\pgfusepath{fill}%
\end{pgfscope}%
\begin{pgfscope}%
\pgfpathrectangle{\pgfqpoint{0.539299in}{0.078740in}}{\pgfqpoint{7.842520in}{7.842520in}}%
\pgfusepath{clip}%
\pgfsetbuttcap%
\pgfsetroundjoin%
\definecolor{currentfill}{rgb}{0.231674,0.318106,0.544834}%
\pgfsetfillcolor{currentfill}%
\pgfsetlinewidth{0.000000pt}%
\definecolor{currentstroke}{rgb}{0.277941,0.056324,0.381191}%
\pgfsetstrokecolor{currentstroke}%
\pgfsetdash{}{0pt}%
\pgfpathmoveto{\pgfqpoint{4.727082in}{4.180259in}}%
\pgfpathlineto{\pgfqpoint{4.861395in}{4.049469in}}%
\pgfpathlineto{\pgfqpoint{4.803456in}{4.103048in}}%
\pgfpathclose%
\pgfusepath{fill}%
\end{pgfscope}%
\begin{pgfscope}%
\pgfpathrectangle{\pgfqpoint{0.539299in}{0.078740in}}{\pgfqpoint{7.842520in}{7.842520in}}%
\pgfusepath{clip}%
\pgfsetbuttcap%
\pgfsetroundjoin%
\definecolor{currentfill}{rgb}{0.147607,0.511733,0.557049}%
\pgfsetfillcolor{currentfill}%
\pgfsetlinewidth{0.000000pt}%
\definecolor{currentstroke}{rgb}{0.278791,0.062145,0.386592}%
\pgfsetstrokecolor{currentstroke}%
\pgfsetdash{}{0pt}%
\pgfpathmoveto{\pgfqpoint{3.498080in}{4.903513in}}%
\pgfpathlineto{\pgfqpoint{3.710739in}{4.939290in}}%
\pgfpathlineto{\pgfqpoint{3.579639in}{4.867750in}}%
\pgfpathclose%
\pgfusepath{fill}%
\end{pgfscope}%
\begin{pgfscope}%
\pgfpathrectangle{\pgfqpoint{0.539299in}{0.078740in}}{\pgfqpoint{7.842520in}{7.842520in}}%
\pgfusepath{clip}%
\pgfsetbuttcap%
\pgfsetroundjoin%
\definecolor{currentfill}{rgb}{0.277941,0.056324,0.381191}%
\pgfsetfillcolor{currentfill}%
\pgfsetlinewidth{0.000000pt}%
\definecolor{currentstroke}{rgb}{0.279566,0.067836,0.391917}%
\pgfsetstrokecolor{currentstroke}%
\pgfsetdash{}{0pt}%
\pgfpathmoveto{\pgfqpoint{6.989604in}{3.234274in}}%
\pgfpathlineto{\pgfqpoint{6.852629in}{3.294416in}}%
\pgfpathlineto{\pgfqpoint{6.783027in}{3.277708in}}%
\pgfpathclose%
\pgfusepath{fill}%
\end{pgfscope}%
\begin{pgfscope}%
\pgfpathrectangle{\pgfqpoint{0.539299in}{0.078740in}}{\pgfqpoint{7.842520in}{7.842520in}}%
\pgfusepath{clip}%
\pgfsetbuttcap%
\pgfsetroundjoin%
\definecolor{currentfill}{rgb}{0.282290,0.145912,0.461510}%
\pgfsetfillcolor{currentfill}%
\pgfsetlinewidth{0.000000pt}%
\definecolor{currentstroke}{rgb}{0.280267,0.073417,0.397163}%
\pgfsetstrokecolor{currentstroke}%
\pgfsetdash{}{0pt}%
\pgfpathmoveto{\pgfqpoint{5.956414in}{3.509868in}}%
\pgfpathlineto{\pgfqpoint{6.028872in}{3.526024in}}%
\pgfpathlineto{\pgfqpoint{5.820187in}{3.551515in}}%
\pgfpathclose%
\pgfusepath{fill}%
\end{pgfscope}%
\begin{pgfscope}%
\pgfpathrectangle{\pgfqpoint{0.539299in}{0.078740in}}{\pgfqpoint{7.842520in}{7.842520in}}%
\pgfusepath{clip}%
\pgfsetbuttcap%
\pgfsetroundjoin%
\definecolor{currentfill}{rgb}{0.280255,0.165693,0.476498}%
\pgfsetfillcolor{currentfill}%
\pgfsetlinewidth{0.000000pt}%
\definecolor{currentstroke}{rgb}{0.280894,0.078907,0.402329}%
\pgfsetstrokecolor{currentstroke}%
\pgfsetdash{}{0pt}%
\pgfpathmoveto{\pgfqpoint{5.820187in}{3.551515in}}%
\pgfpathlineto{\pgfqpoint{5.684398in}{3.594113in}}%
\pgfpathlineto{\pgfqpoint{5.610860in}{3.587946in}}%
\pgfpathclose%
\pgfusepath{fill}%
\end{pgfscope}%
\begin{pgfscope}%
\pgfpathrectangle{\pgfqpoint{0.539299in}{0.078740in}}{\pgfqpoint{7.842520in}{7.842520in}}%
\pgfusepath{clip}%
\pgfsetbuttcap%
\pgfsetroundjoin%
\definecolor{currentfill}{rgb}{0.267968,0.223549,0.512008}%
\pgfsetfillcolor{currentfill}%
\pgfsetlinewidth{0.000000pt}%
\definecolor{currentstroke}{rgb}{0.281446,0.084320,0.407414}%
\pgfsetstrokecolor{currentstroke}%
\pgfsetdash{}{0pt}%
\pgfpathmoveto{\pgfqpoint{5.130570in}{3.824189in}}%
\pgfpathlineto{\pgfqpoint{5.340183in}{3.710444in}}%
\pgfpathlineto{\pgfqpoint{5.205512in}{3.788278in}}%
\pgfpathclose%
\pgfusepath{fill}%
\end{pgfscope}%
\begin{pgfscope}%
\pgfpathrectangle{\pgfqpoint{0.539299in}{0.078740in}}{\pgfqpoint{7.842520in}{7.842520in}}%
\pgfusepath{clip}%
\pgfsetbuttcap%
\pgfsetroundjoin%
\definecolor{currentfill}{rgb}{0.282884,0.135920,0.453427}%
\pgfsetfillcolor{currentfill}%
\pgfsetlinewidth{0.000000pt}%
\definecolor{currentstroke}{rgb}{0.281924,0.089666,0.412415}%
\pgfsetstrokecolor{currentstroke}%
\pgfsetdash{}{0pt}%
\pgfpathmoveto{\pgfqpoint{6.093026in}{3.466206in}}%
\pgfpathlineto{\pgfqpoint{6.165167in}{3.484596in}}%
\pgfpathlineto{\pgfqpoint{6.028872in}{3.526024in}}%
\pgfpathclose%
\pgfusepath{fill}%
\end{pgfscope}%
\begin{pgfscope}%
\pgfpathrectangle{\pgfqpoint{0.539299in}{0.078740in}}{\pgfqpoint{7.842520in}{7.842520in}}%
\pgfusepath{clip}%
\pgfsetbuttcap%
\pgfsetroundjoin%
\definecolor{currentfill}{rgb}{0.248629,0.278775,0.534556}%
\pgfsetfillcolor{currentfill}%
\pgfsetlinewidth{0.000000pt}%
\definecolor{currentstroke}{rgb}{0.282327,0.094955,0.417331}%
\pgfsetstrokecolor{currentstroke}%
\pgfsetdash{}{0pt}%
\pgfpathmoveto{\pgfqpoint{4.995865in}{3.930113in}}%
\pgfpathlineto{\pgfqpoint{5.071218in}{3.879659in}}%
\pgfpathlineto{\pgfqpoint{4.861395in}{4.049469in}}%
\pgfpathclose%
\pgfusepath{fill}%
\end{pgfscope}%
\begin{pgfscope}%
\pgfpathrectangle{\pgfqpoint{0.539299in}{0.078740in}}{\pgfqpoint{7.842520in}{7.842520in}}%
\pgfusepath{clip}%
\pgfsetbuttcap%
\pgfsetroundjoin%
\definecolor{currentfill}{rgb}{0.150476,0.504369,0.557430}%
\pgfsetfillcolor{currentfill}%
\pgfsetlinewidth{0.000000pt}%
\definecolor{currentstroke}{rgb}{0.282656,0.100196,0.422160}%
\pgfsetstrokecolor{currentstroke}%
\pgfsetdash{}{0pt}%
\pgfpathmoveto{\pgfqpoint{3.976815in}{4.900063in}}%
\pgfpathlineto{\pgfqpoint{4.057001in}{4.809605in}}%
\pgfpathlineto{\pgfqpoint{3.923882in}{4.868004in}}%
\pgfpathclose%
\pgfusepath{fill}%
\end{pgfscope}%
\begin{pgfscope}%
\pgfpathrectangle{\pgfqpoint{0.539299in}{0.078740in}}{\pgfqpoint{7.842520in}{7.842520in}}%
\pgfusepath{clip}%
\pgfsetbuttcap%
\pgfsetroundjoin%
\definecolor{currentfill}{rgb}{0.278826,0.175490,0.483397}%
\pgfsetfillcolor{currentfill}%
\pgfsetlinewidth{0.000000pt}%
\definecolor{currentstroke}{rgb}{0.282910,0.105393,0.426902}%
\pgfsetstrokecolor{currentstroke}%
\pgfsetdash{}{0pt}%
\pgfpathmoveto{\pgfqpoint{5.475289in}{3.644517in}}%
\pgfpathlineto{\pgfqpoint{5.610860in}{3.587946in}}%
\pgfpathlineto{\pgfqpoint{5.684398in}{3.594113in}}%
\pgfpathclose%
\pgfusepath{fill}%
\end{pgfscope}%
\begin{pgfscope}%
\pgfpathrectangle{\pgfqpoint{0.539299in}{0.078740in}}{\pgfqpoint{7.842520in}{7.842520in}}%
\pgfusepath{clip}%
\pgfsetbuttcap%
\pgfsetroundjoin%
\definecolor{currentfill}{rgb}{0.269944,0.014625,0.341379}%
\pgfsetfillcolor{currentfill}%
\pgfsetlinewidth{0.000000pt}%
\definecolor{currentstroke}{rgb}{0.283091,0.110553,0.431554}%
\pgfsetstrokecolor{currentstroke}%
\pgfsetdash{}{0pt}%
\pgfpathmoveto{\pgfqpoint{7.264628in}{3.113510in}}%
\pgfpathlineto{\pgfqpoint{7.471813in}{3.109193in}}%
\pgfpathlineto{\pgfqpoint{7.333628in}{3.157861in}}%
\pgfpathclose%
\pgfusepath{fill}%
\end{pgfscope}%
\begin{pgfscope}%
\pgfpathrectangle{\pgfqpoint{0.539299in}{0.078740in}}{\pgfqpoint{7.842520in}{7.842520in}}%
\pgfusepath{clip}%
\pgfsetbuttcap%
\pgfsetroundjoin%
\definecolor{currentfill}{rgb}{0.271305,0.019942,0.347269}%
\pgfsetfillcolor{currentfill}%
\pgfsetlinewidth{0.000000pt}%
\definecolor{currentstroke}{rgb}{0.283197,0.115680,0.436115}%
\pgfsetstrokecolor{currentstroke}%
\pgfsetdash{}{0pt}%
\pgfpathmoveto{\pgfqpoint{7.126919in}{3.173464in}}%
\pgfpathlineto{\pgfqpoint{7.264628in}{3.113510in}}%
\pgfpathlineto{\pgfqpoint{7.333628in}{3.157861in}}%
\pgfpathclose%
\pgfusepath{fill}%
\end{pgfscope}%
\begin{pgfscope}%
\pgfpathrectangle{\pgfqpoint{0.539299in}{0.078740in}}{\pgfqpoint{7.842520in}{7.842520in}}%
\pgfusepath{clip}%
\pgfsetbuttcap%
\pgfsetroundjoin%
\definecolor{currentfill}{rgb}{0.255645,0.260703,0.528312}%
\pgfsetfillcolor{currentfill}%
\pgfsetlinewidth{0.000000pt}%
\definecolor{currentstroke}{rgb}{0.283229,0.120777,0.440584}%
\pgfsetstrokecolor{currentstroke}%
\pgfsetdash{}{0pt}%
\pgfpathmoveto{\pgfqpoint{5.130570in}{3.824189in}}%
\pgfpathlineto{\pgfqpoint{5.071218in}{3.879659in}}%
\pgfpathlineto{\pgfqpoint{4.995865in}{3.930113in}}%
\pgfpathclose%
\pgfusepath{fill}%
\end{pgfscope}%
\begin{pgfscope}%
\pgfpathrectangle{\pgfqpoint{0.539299in}{0.078740in}}{\pgfqpoint{7.842520in}{7.842520in}}%
\pgfusepath{clip}%
\pgfsetbuttcap%
\pgfsetroundjoin%
\definecolor{currentfill}{rgb}{0.283197,0.115680,0.436115}%
\pgfsetfillcolor{currentfill}%
\pgfsetlinewidth{0.000000pt}%
\definecolor{currentstroke}{rgb}{0.283187,0.125848,0.444960}%
\pgfsetstrokecolor{currentstroke}%
\pgfsetdash{}{0pt}%
\pgfpathmoveto{\pgfqpoint{6.301764in}{3.437824in}}%
\pgfpathlineto{\pgfqpoint{6.438604in}{3.384604in}}%
\pgfpathlineto{\pgfqpoint{6.509400in}{3.398734in}}%
\pgfpathclose%
\pgfusepath{fill}%
\end{pgfscope}%
\begin{pgfscope}%
\pgfpathrectangle{\pgfqpoint{0.539299in}{0.078740in}}{\pgfqpoint{7.842520in}{7.842520in}}%
\pgfusepath{clip}%
\pgfsetbuttcap%
\pgfsetroundjoin%
\definecolor{currentfill}{rgb}{0.282290,0.145912,0.461510}%
\pgfsetfillcolor{currentfill}%
\pgfsetlinewidth{0.000000pt}%
\definecolor{currentstroke}{rgb}{0.283072,0.130895,0.449241}%
\pgfsetstrokecolor{currentstroke}%
\pgfsetdash{}{0pt}%
\pgfpathmoveto{\pgfqpoint{6.093026in}{3.466206in}}%
\pgfpathlineto{\pgfqpoint{6.028872in}{3.526024in}}%
\pgfpathlineto{\pgfqpoint{5.956414in}{3.509868in}}%
\pgfpathclose%
\pgfusepath{fill}%
\end{pgfscope}%
\begin{pgfscope}%
\pgfpathrectangle{\pgfqpoint{0.539299in}{0.078740in}}{\pgfqpoint{7.842520in}{7.842520in}}%
\pgfusepath{clip}%
\pgfsetbuttcap%
\pgfsetroundjoin%
\definecolor{currentfill}{rgb}{0.153364,0.497000,0.557724}%
\pgfsetfillcolor{currentfill}%
\pgfsetlinewidth{0.000000pt}%
\definecolor{currentstroke}{rgb}{0.282884,0.135920,0.453427}%
\pgfsetstrokecolor{currentstroke}%
\pgfsetdash{}{0pt}%
\pgfpathmoveto{\pgfqpoint{4.190652in}{4.714468in}}%
\pgfpathlineto{\pgfqpoint{4.057001in}{4.809605in}}%
\pgfpathlineto{\pgfqpoint{3.976815in}{4.900063in}}%
\pgfpathclose%
\pgfusepath{fill}%
\end{pgfscope}%
\begin{pgfscope}%
\pgfpathrectangle{\pgfqpoint{0.539299in}{0.078740in}}{\pgfqpoint{7.842520in}{7.842520in}}%
\pgfusepath{clip}%
\pgfsetbuttcap%
\pgfsetroundjoin%
\definecolor{currentfill}{rgb}{0.153364,0.497000,0.557724}%
\pgfsetfillcolor{currentfill}%
\pgfsetlinewidth{0.000000pt}%
\definecolor{currentstroke}{rgb}{0.282623,0.140926,0.457517}%
\pgfsetstrokecolor{currentstroke}%
\pgfsetdash{}{0pt}%
\pgfpathmoveto{\pgfqpoint{3.415921in}{4.934934in}}%
\pgfpathlineto{\pgfqpoint{3.369083in}{4.733483in}}%
\pgfpathlineto{\pgfqpoint{3.287152in}{4.741607in}}%
\pgfpathclose%
\pgfusepath{fill}%
\end{pgfscope}%
\begin{pgfscope}%
\pgfpathrectangle{\pgfqpoint{0.539299in}{0.078740in}}{\pgfqpoint{7.842520in}{7.842520in}}%
\pgfusepath{clip}%
\pgfsetbuttcap%
\pgfsetroundjoin%
\definecolor{currentfill}{rgb}{0.210503,0.363727,0.552206}%
\pgfsetfillcolor{currentfill}%
\pgfsetlinewidth{0.000000pt}%
\definecolor{currentstroke}{rgb}{0.282290,0.145912,0.461510}%
\pgfsetstrokecolor{currentstroke}%
\pgfsetdash{}{0pt}%
\pgfpathmoveto{\pgfqpoint{2.913780in}{4.395967in}}%
\pgfpathlineto{\pgfqpoint{2.996771in}{4.420700in}}%
\pgfpathlineto{\pgfqpoint{2.793452in}{3.956696in}}%
\pgfpathclose%
\pgfusepath{fill}%
\end{pgfscope}%
\begin{pgfscope}%
\pgfpathrectangle{\pgfqpoint{0.539299in}{0.078740in}}{\pgfqpoint{7.842520in}{7.842520in}}%
\pgfusepath{clip}%
\pgfsetbuttcap%
\pgfsetroundjoin%
\definecolor{currentfill}{rgb}{0.282656,0.100196,0.422160}%
\pgfsetfillcolor{currentfill}%
\pgfsetlinewidth{0.000000pt}%
\definecolor{currentstroke}{rgb}{0.281887,0.150881,0.465405}%
\pgfsetstrokecolor{currentstroke}%
\pgfsetdash{}{0pt}%
\pgfpathmoveto{\pgfqpoint{6.575647in}{3.324679in}}%
\pgfpathlineto{\pgfqpoint{6.646106in}{3.340736in}}%
\pgfpathlineto{\pgfqpoint{6.509400in}{3.398734in}}%
\pgfpathclose%
\pgfusepath{fill}%
\end{pgfscope}%
\begin{pgfscope}%
\pgfpathrectangle{\pgfqpoint{0.539299in}{0.078740in}}{\pgfqpoint{7.842520in}{7.842520in}}%
\pgfusepath{clip}%
\pgfsetbuttcap%
\pgfsetroundjoin%
\definecolor{currentfill}{rgb}{0.168126,0.459988,0.558082}%
\pgfsetfillcolor{currentfill}%
\pgfsetlinewidth{0.000000pt}%
\definecolor{currentstroke}{rgb}{0.281412,0.155834,0.469201}%
\pgfsetstrokecolor{currentstroke}%
\pgfsetdash{}{0pt}%
\pgfpathmoveto{\pgfqpoint{3.079268in}{4.442627in}}%
\pgfpathlineto{\pgfqpoint{3.204675in}{4.746147in}}%
\pgfpathlineto{\pgfqpoint{3.287152in}{4.741607in}}%
\pgfpathclose%
\pgfusepath{fill}%
\end{pgfscope}%
\begin{pgfscope}%
\pgfpathrectangle{\pgfqpoint{0.539299in}{0.078740in}}{\pgfqpoint{7.842520in}{7.842520in}}%
\pgfusepath{clip}%
\pgfsetbuttcap%
\pgfsetroundjoin%
\definecolor{currentfill}{rgb}{0.147607,0.511733,0.557049}%
\pgfsetfillcolor{currentfill}%
\pgfsetlinewidth{0.000000pt}%
\definecolor{currentstroke}{rgb}{0.280868,0.160771,0.472899}%
\pgfsetstrokecolor{currentstroke}%
\pgfsetdash{}{0pt}%
\pgfpathmoveto{\pgfqpoint{3.415921in}{4.934934in}}%
\pgfpathlineto{\pgfqpoint{3.498080in}{4.903513in}}%
\pgfpathlineto{\pgfqpoint{3.369083in}{4.733483in}}%
\pgfpathclose%
\pgfusepath{fill}%
\end{pgfscope}%
\begin{pgfscope}%
\pgfpathrectangle{\pgfqpoint{0.539299in}{0.078740in}}{\pgfqpoint{7.842520in}{7.842520in}}%
\pgfusepath{clip}%
\pgfsetbuttcap%
\pgfsetroundjoin%
\definecolor{currentfill}{rgb}{0.282884,0.135920,0.453427}%
\pgfsetfillcolor{currentfill}%
\pgfsetlinewidth{0.000000pt}%
\definecolor{currentstroke}{rgb}{0.280255,0.165693,0.476498}%
\pgfsetstrokecolor{currentstroke}%
\pgfsetdash{}{0pt}%
\pgfpathmoveto{\pgfqpoint{6.301764in}{3.437824in}}%
\pgfpathlineto{\pgfqpoint{6.165167in}{3.484596in}}%
\pgfpathlineto{\pgfqpoint{6.093026in}{3.466206in}}%
\pgfpathclose%
\pgfusepath{fill}%
\end{pgfscope}%
\begin{pgfscope}%
\pgfpathrectangle{\pgfqpoint{0.539299in}{0.078740in}}{\pgfqpoint{7.842520in}{7.842520in}}%
\pgfusepath{clip}%
\pgfsetbuttcap%
\pgfsetroundjoin%
\definecolor{currentfill}{rgb}{0.278791,0.062145,0.386592}%
\pgfsetfillcolor{currentfill}%
\pgfsetlinewidth{0.000000pt}%
\definecolor{currentstroke}{rgb}{0.279574,0.170599,0.479997}%
\pgfsetstrokecolor{currentstroke}%
\pgfsetdash{}{0pt}%
\pgfpathmoveto{\pgfqpoint{6.783027in}{3.277708in}}%
\pgfpathlineto{\pgfqpoint{6.920184in}{3.211063in}}%
\pgfpathlineto{\pgfqpoint{6.989604in}{3.234274in}}%
\pgfpathclose%
\pgfusepath{fill}%
\end{pgfscope}%
\begin{pgfscope}%
\pgfpathrectangle{\pgfqpoint{0.539299in}{0.078740in}}{\pgfqpoint{7.842520in}{7.842520in}}%
\pgfusepath{clip}%
\pgfsetbuttcap%
\pgfsetroundjoin%
\definecolor{currentfill}{rgb}{0.276022,0.044167,0.370164}%
\pgfsetfillcolor{currentfill}%
\pgfsetlinewidth{0.000000pt}%
\definecolor{currentstroke}{rgb}{0.278826,0.175490,0.483397}%
\pgfsetstrokecolor{currentstroke}%
\pgfsetdash{}{0pt}%
\pgfpathmoveto{\pgfqpoint{7.057617in}{3.142554in}}%
\pgfpathlineto{\pgfqpoint{7.126919in}{3.173464in}}%
\pgfpathlineto{\pgfqpoint{6.989604in}{3.234274in}}%
\pgfpathclose%
\pgfusepath{fill}%
\end{pgfscope}%
\begin{pgfscope}%
\pgfpathrectangle{\pgfqpoint{0.539299in}{0.078740in}}{\pgfqpoint{7.842520in}{7.842520in}}%
\pgfusepath{clip}%
\pgfsetbuttcap%
\pgfsetroundjoin%
\definecolor{currentfill}{rgb}{0.143343,0.522773,0.556295}%
\pgfsetfillcolor{currentfill}%
\pgfsetlinewidth{0.000000pt}%
\definecolor{currentstroke}{rgb}{0.278012,0.180367,0.486697}%
\pgfsetstrokecolor{currentstroke}%
\pgfsetdash{}{0pt}%
\pgfpathmoveto{\pgfqpoint{3.923882in}{4.868004in}}%
\pgfpathlineto{\pgfqpoint{3.843276in}{4.946277in}}%
\pgfpathlineto{\pgfqpoint{3.976815in}{4.900063in}}%
\pgfpathclose%
\pgfusepath{fill}%
\end{pgfscope}%
\begin{pgfscope}%
\pgfpathrectangle{\pgfqpoint{0.539299in}{0.078740in}}{\pgfqpoint{7.842520in}{7.842520in}}%
\pgfusepath{clip}%
\pgfsetbuttcap%
\pgfsetroundjoin%
\definecolor{currentfill}{rgb}{0.273006,0.204520,0.501721}%
\pgfsetfillcolor{currentfill}%
\pgfsetlinewidth{0.000000pt}%
\definecolor{currentstroke}{rgb}{0.277134,0.185228,0.489898}%
\pgfsetstrokecolor{currentstroke}%
\pgfsetdash{}{0pt}%
\pgfpathmoveto{\pgfqpoint{5.340183in}{3.710444in}}%
\pgfpathlineto{\pgfqpoint{5.400994in}{3.652853in}}%
\pgfpathlineto{\pgfqpoint{5.475289in}{3.644517in}}%
\pgfpathclose%
\pgfusepath{fill}%
\end{pgfscope}%
\begin{pgfscope}%
\pgfpathrectangle{\pgfqpoint{0.539299in}{0.078740in}}{\pgfqpoint{7.842520in}{7.842520in}}%
\pgfusepath{clip}%
\pgfsetbuttcap%
\pgfsetroundjoin%
\definecolor{currentfill}{rgb}{0.229739,0.322361,0.545706}%
\pgfsetfillcolor{currentfill}%
\pgfsetlinewidth{0.000000pt}%
\definecolor{currentstroke}{rgb}{0.276194,0.190074,0.493001}%
\pgfsetstrokecolor{currentstroke}%
\pgfsetdash{}{0pt}%
\pgfpathmoveto{\pgfqpoint{2.793452in}{3.956696in}}%
\pgfpathlineto{\pgfqpoint{2.710747in}{3.908413in}}%
\pgfpathlineto{\pgfqpoint{2.830306in}{4.368134in}}%
\pgfpathclose%
\pgfusepath{fill}%
\end{pgfscope}%
\begin{pgfscope}%
\pgfpathrectangle{\pgfqpoint{0.539299in}{0.078740in}}{\pgfqpoint{7.842520in}{7.842520in}}%
\pgfusepath{clip}%
\pgfsetbuttcap%
\pgfsetroundjoin%
\definecolor{currentfill}{rgb}{0.160665,0.478540,0.558115}%
\pgfsetfillcolor{currentfill}%
\pgfsetlinewidth{0.000000pt}%
\definecolor{currentstroke}{rgb}{0.275191,0.194905,0.496005}%
\pgfsetstrokecolor{currentstroke}%
\pgfsetdash{}{0pt}%
\pgfpathmoveto{\pgfqpoint{4.190652in}{4.714468in}}%
\pgfpathlineto{\pgfqpoint{4.111002in}{4.812426in}}%
\pgfpathlineto{\pgfqpoint{4.324600in}{4.594132in}}%
\pgfpathclose%
\pgfusepath{fill}%
\end{pgfscope}%
\begin{pgfscope}%
\pgfpathrectangle{\pgfqpoint{0.539299in}{0.078740in}}{\pgfqpoint{7.842520in}{7.842520in}}%
\pgfusepath{clip}%
\pgfsetbuttcap%
\pgfsetroundjoin%
\definecolor{currentfill}{rgb}{0.283091,0.110553,0.431554}%
\pgfsetfillcolor{currentfill}%
\pgfsetlinewidth{0.000000pt}%
\definecolor{currentstroke}{rgb}{0.274128,0.199721,0.498911}%
\pgfsetstrokecolor{currentstroke}%
\pgfsetdash{}{0pt}%
\pgfpathmoveto{\pgfqpoint{6.509400in}{3.398734in}}%
\pgfpathlineto{\pgfqpoint{6.438604in}{3.384604in}}%
\pgfpathlineto{\pgfqpoint{6.575647in}{3.324679in}}%
\pgfpathclose%
\pgfusepath{fill}%
\end{pgfscope}%
\begin{pgfscope}%
\pgfpathrectangle{\pgfqpoint{0.539299in}{0.078740in}}{\pgfqpoint{7.842520in}{7.842520in}}%
\pgfusepath{clip}%
\pgfsetbuttcap%
\pgfsetroundjoin%
\definecolor{currentfill}{rgb}{0.265145,0.232956,0.516599}%
\pgfsetfillcolor{currentfill}%
\pgfsetlinewidth{0.000000pt}%
\definecolor{currentstroke}{rgb}{0.273006,0.204520,0.501721}%
\pgfsetstrokecolor{currentstroke}%
\pgfsetdash{}{0pt}%
\pgfpathmoveto{\pgfqpoint{5.265591in}{3.732089in}}%
\pgfpathlineto{\pgfqpoint{5.340183in}{3.710444in}}%
\pgfpathlineto{\pgfqpoint{5.130570in}{3.824189in}}%
\pgfpathclose%
\pgfusepath{fill}%
\end{pgfscope}%
\begin{pgfscope}%
\pgfpathrectangle{\pgfqpoint{0.539299in}{0.078740in}}{\pgfqpoint{7.842520in}{7.842520in}}%
\pgfusepath{clip}%
\pgfsetbuttcap%
\pgfsetroundjoin%
\definecolor{currentfill}{rgb}{0.279574,0.170599,0.479997}%
\pgfsetfillcolor{currentfill}%
\pgfsetlinewidth{0.000000pt}%
\definecolor{currentstroke}{rgb}{0.271828,0.209303,0.504434}%
\pgfsetstrokecolor{currentstroke}%
\pgfsetdash{}{0pt}%
\pgfpathmoveto{\pgfqpoint{5.610860in}{3.587946in}}%
\pgfpathlineto{\pgfqpoint{5.746902in}{3.537634in}}%
\pgfpathlineto{\pgfqpoint{5.820187in}{3.551515in}}%
\pgfpathclose%
\pgfusepath{fill}%
\end{pgfscope}%
\begin{pgfscope}%
\pgfpathrectangle{\pgfqpoint{0.539299in}{0.078740in}}{\pgfqpoint{7.842520in}{7.842520in}}%
\pgfusepath{clip}%
\pgfsetbuttcap%
\pgfsetroundjoin%
\definecolor{currentfill}{rgb}{0.282327,0.094955,0.417331}%
\pgfsetfillcolor{currentfill}%
\pgfsetlinewidth{0.000000pt}%
\definecolor{currentstroke}{rgb}{0.270595,0.214069,0.507052}%
\pgfsetstrokecolor{currentstroke}%
\pgfsetdash{}{0pt}%
\pgfpathmoveto{\pgfqpoint{6.575647in}{3.324679in}}%
\pgfpathlineto{\pgfqpoint{6.783027in}{3.277708in}}%
\pgfpathlineto{\pgfqpoint{6.646106in}{3.340736in}}%
\pgfpathclose%
\pgfusepath{fill}%
\end{pgfscope}%
\begin{pgfscope}%
\pgfpathrectangle{\pgfqpoint{0.539299in}{0.078740in}}{\pgfqpoint{7.842520in}{7.842520in}}%
\pgfusepath{clip}%
\pgfsetbuttcap%
\pgfsetroundjoin%
\definecolor{currentfill}{rgb}{0.269944,0.014625,0.341379}%
\pgfsetfillcolor{currentfill}%
\pgfsetlinewidth{0.000000pt}%
\definecolor{currentstroke}{rgb}{0.269308,0.218818,0.509577}%
\pgfsetstrokecolor{currentstroke}%
\pgfsetdash{}{0pt}%
\pgfpathmoveto{\pgfqpoint{7.402794in}{3.055818in}}%
\pgfpathlineto{\pgfqpoint{7.471813in}{3.109193in}}%
\pgfpathlineto{\pgfqpoint{7.264628in}{3.113510in}}%
\pgfpathclose%
\pgfusepath{fill}%
\end{pgfscope}%
\begin{pgfscope}%
\pgfpathrectangle{\pgfqpoint{0.539299in}{0.078740in}}{\pgfqpoint{7.842520in}{7.842520in}}%
\pgfusepath{clip}%
\pgfsetbuttcap%
\pgfsetroundjoin%
\definecolor{currentfill}{rgb}{0.277941,0.056324,0.381191}%
\pgfsetfillcolor{currentfill}%
\pgfsetlinewidth{0.000000pt}%
\definecolor{currentstroke}{rgb}{0.267968,0.223549,0.512008}%
\pgfsetstrokecolor{currentstroke}%
\pgfsetdash{}{0pt}%
\pgfpathmoveto{\pgfqpoint{6.989604in}{3.234274in}}%
\pgfpathlineto{\pgfqpoint{6.920184in}{3.211063in}}%
\pgfpathlineto{\pgfqpoint{7.057617in}{3.142554in}}%
\pgfpathclose%
\pgfusepath{fill}%
\end{pgfscope}%
\begin{pgfscope}%
\pgfpathrectangle{\pgfqpoint{0.539299in}{0.078740in}}{\pgfqpoint{7.842520in}{7.842520in}}%
\pgfusepath{clip}%
\pgfsetbuttcap%
\pgfsetroundjoin%
\definecolor{currentfill}{rgb}{0.137770,0.537492,0.554906}%
\pgfsetfillcolor{currentfill}%
\pgfsetlinewidth{0.000000pt}%
\definecolor{currentstroke}{rgb}{0.266580,0.228262,0.514349}%
\pgfsetstrokecolor{currentstroke}%
\pgfsetdash{}{0pt}%
\pgfpathmoveto{\pgfqpoint{3.629210in}{4.996070in}}%
\pgfpathlineto{\pgfqpoint{3.710739in}{4.939290in}}%
\pgfpathlineto{\pgfqpoint{3.498080in}{4.903513in}}%
\pgfpathclose%
\pgfusepath{fill}%
\end{pgfscope}%
\begin{pgfscope}%
\pgfpathrectangle{\pgfqpoint{0.539299in}{0.078740in}}{\pgfqpoint{7.842520in}{7.842520in}}%
\pgfusepath{clip}%
\pgfsetbuttcap%
\pgfsetroundjoin%
\definecolor{currentfill}{rgb}{0.280868,0.160771,0.472899}%
\pgfsetfillcolor{currentfill}%
\pgfsetlinewidth{0.000000pt}%
\definecolor{currentstroke}{rgb}{0.265145,0.232956,0.516599}%
\pgfsetstrokecolor{currentstroke}%
\pgfsetdash{}{0pt}%
\pgfpathmoveto{\pgfqpoint{5.820187in}{3.551515in}}%
\pgfpathlineto{\pgfqpoint{5.883392in}{3.490293in}}%
\pgfpathlineto{\pgfqpoint{5.956414in}{3.509868in}}%
\pgfpathclose%
\pgfusepath{fill}%
\end{pgfscope}%
\begin{pgfscope}%
\pgfpathrectangle{\pgfqpoint{0.539299in}{0.078740in}}{\pgfqpoint{7.842520in}{7.842520in}}%
\pgfusepath{clip}%
\pgfsetbuttcap%
\pgfsetroundjoin%
\definecolor{currentfill}{rgb}{0.175841,0.441290,0.557685}%
\pgfsetfillcolor{currentfill}%
\pgfsetlinewidth{0.000000pt}%
\definecolor{currentstroke}{rgb}{0.263663,0.237631,0.518762}%
\pgfsetstrokecolor{currentstroke}%
\pgfsetdash{}{0pt}%
\pgfpathmoveto{\pgfqpoint{4.324600in}{4.594132in}}%
\pgfpathlineto{\pgfqpoint{4.380311in}{4.558363in}}%
\pgfpathlineto{\pgfqpoint{4.458695in}{4.459156in}}%
\pgfpathclose%
\pgfusepath{fill}%
\end{pgfscope}%
\begin{pgfscope}%
\pgfpathrectangle{\pgfqpoint{0.539299in}{0.078740in}}{\pgfqpoint{7.842520in}{7.842520in}}%
\pgfusepath{clip}%
\pgfsetbuttcap%
\pgfsetroundjoin%
\definecolor{currentfill}{rgb}{0.273809,0.031497,0.358853}%
\pgfsetfillcolor{currentfill}%
\pgfsetlinewidth{0.000000pt}%
\definecolor{currentstroke}{rgb}{0.262138,0.242286,0.520837}%
\pgfsetstrokecolor{currentstroke}%
\pgfsetdash{}{0pt}%
\pgfpathmoveto{\pgfqpoint{7.264628in}{3.113510in}}%
\pgfpathlineto{\pgfqpoint{7.126919in}{3.173464in}}%
\pgfpathlineto{\pgfqpoint{7.057617in}{3.142554in}}%
\pgfpathclose%
\pgfusepath{fill}%
\end{pgfscope}%
\begin{pgfscope}%
\pgfpathrectangle{\pgfqpoint{0.539299in}{0.078740in}}{\pgfqpoint{7.842520in}{7.842520in}}%
\pgfusepath{clip}%
\pgfsetbuttcap%
\pgfsetroundjoin%
\definecolor{currentfill}{rgb}{0.185556,0.418570,0.556753}%
\pgfsetfillcolor{currentfill}%
\pgfsetlinewidth{0.000000pt}%
\definecolor{currentstroke}{rgb}{0.260571,0.246922,0.522828}%
\pgfsetstrokecolor{currentstroke}%
\pgfsetdash{}{0pt}%
\pgfpathmoveto{\pgfqpoint{4.380311in}{4.558363in}}%
\pgfpathlineto{\pgfqpoint{4.592860in}{4.318703in}}%
\pgfpathlineto{\pgfqpoint{4.458695in}{4.459156in}}%
\pgfpathclose%
\pgfusepath{fill}%
\end{pgfscope}%
\begin{pgfscope}%
\pgfpathrectangle{\pgfqpoint{0.539299in}{0.078740in}}{\pgfqpoint{7.842520in}{7.842520in}}%
\pgfusepath{clip}%
\pgfsetbuttcap%
\pgfsetroundjoin%
\definecolor{currentfill}{rgb}{0.267004,0.004874,0.329415}%
\pgfsetfillcolor{currentfill}%
\pgfsetlinewidth{0.000000pt}%
\definecolor{currentstroke}{rgb}{0.258965,0.251537,0.524736}%
\pgfsetstrokecolor{currentstroke}%
\pgfsetdash{}{0pt}%
\pgfpathmoveto{\pgfqpoint{7.541482in}{3.001589in}}%
\pgfpathlineto{\pgfqpoint{7.610502in}{3.062741in}}%
\pgfpathlineto{\pgfqpoint{7.471813in}{3.109193in}}%
\pgfpathclose%
\pgfusepath{fill}%
\end{pgfscope}%
\begin{pgfscope}%
\pgfpathrectangle{\pgfqpoint{0.539299in}{0.078740in}}{\pgfqpoint{7.842520in}{7.842520in}}%
\pgfusepath{clip}%
\pgfsetbuttcap%
\pgfsetroundjoin%
\definecolor{currentfill}{rgb}{0.269308,0.218818,0.509577}%
\pgfsetfillcolor{currentfill}%
\pgfsetlinewidth{0.000000pt}%
\definecolor{currentstroke}{rgb}{0.257322,0.256130,0.526563}%
\pgfsetstrokecolor{currentstroke}%
\pgfsetdash{}{0pt}%
\pgfpathmoveto{\pgfqpoint{5.265591in}{3.732089in}}%
\pgfpathlineto{\pgfqpoint{5.400994in}{3.652853in}}%
\pgfpathlineto{\pgfqpoint{5.340183in}{3.710444in}}%
\pgfpathclose%
\pgfusepath{fill}%
\end{pgfscope}%
\begin{pgfscope}%
\pgfpathrectangle{\pgfqpoint{0.539299in}{0.078740in}}{\pgfqpoint{7.842520in}{7.842520in}}%
\pgfusepath{clip}%
\pgfsetbuttcap%
\pgfsetroundjoin%
\definecolor{currentfill}{rgb}{0.136408,0.541173,0.554483}%
\pgfsetfillcolor{currentfill}%
\pgfsetlinewidth{0.000000pt}%
\definecolor{currentstroke}{rgb}{0.255645,0.260703,0.528312}%
\pgfsetstrokecolor{currentstroke}%
\pgfsetdash{}{0pt}%
\pgfpathmoveto{\pgfqpoint{3.843276in}{4.946277in}}%
\pgfpathlineto{\pgfqpoint{3.710739in}{4.939290in}}%
\pgfpathlineto{\pgfqpoint{3.629210in}{4.996070in}}%
\pgfpathclose%
\pgfusepath{fill}%
\end{pgfscope}%
\begin{pgfscope}%
\pgfpathrectangle{\pgfqpoint{0.539299in}{0.078740in}}{\pgfqpoint{7.842520in}{7.842520in}}%
\pgfusepath{clip}%
\pgfsetbuttcap%
\pgfsetroundjoin%
\definecolor{currentfill}{rgb}{0.175841,0.441290,0.557685}%
\pgfsetfillcolor{currentfill}%
\pgfsetlinewidth{0.000000pt}%
\definecolor{currentstroke}{rgb}{0.253935,0.265254,0.529983}%
\pgfsetstrokecolor{currentstroke}%
\pgfsetdash{}{0pt}%
\pgfpathmoveto{\pgfqpoint{3.079268in}{4.442627in}}%
\pgfpathlineto{\pgfqpoint{2.996771in}{4.420700in}}%
\pgfpathlineto{\pgfqpoint{3.121655in}{4.747134in}}%
\pgfpathclose%
\pgfusepath{fill}%
\end{pgfscope}%
\begin{pgfscope}%
\pgfpathrectangle{\pgfqpoint{0.539299in}{0.078740in}}{\pgfqpoint{7.842520in}{7.842520in}}%
\pgfusepath{clip}%
\pgfsetbuttcap%
\pgfsetroundjoin%
\definecolor{currentfill}{rgb}{0.204903,0.375746,0.553533}%
\pgfsetfillcolor{currentfill}%
\pgfsetlinewidth{0.000000pt}%
\definecolor{currentstroke}{rgb}{0.252194,0.269783,0.531579}%
\pgfsetstrokecolor{currentstroke}%
\pgfsetdash{}{0pt}%
\pgfpathmoveto{\pgfqpoint{4.592860in}{4.318703in}}%
\pgfpathlineto{\pgfqpoint{4.649993in}{4.264866in}}%
\pgfpathlineto{\pgfqpoint{4.727082in}{4.180259in}}%
\pgfpathclose%
\pgfusepath{fill}%
\end{pgfscope}%
\begin{pgfscope}%
\pgfpathrectangle{\pgfqpoint{0.539299in}{0.078740in}}{\pgfqpoint{7.842520in}{7.842520in}}%
\pgfusepath{clip}%
\pgfsetbuttcap%
\pgfsetroundjoin%
\definecolor{currentfill}{rgb}{0.216210,0.351535,0.550627}%
\pgfsetfillcolor{currentfill}%
\pgfsetlinewidth{0.000000pt}%
\definecolor{currentstroke}{rgb}{0.250425,0.274290,0.533103}%
\pgfsetstrokecolor{currentstroke}%
\pgfsetdash{}{0pt}%
\pgfpathmoveto{\pgfqpoint{4.727082in}{4.180259in}}%
\pgfpathlineto{\pgfqpoint{4.649993in}{4.264866in}}%
\pgfpathlineto{\pgfqpoint{4.861395in}{4.049469in}}%
\pgfpathclose%
\pgfusepath{fill}%
\end{pgfscope}%
\begin{pgfscope}%
\pgfpathrectangle{\pgfqpoint{0.539299in}{0.078740in}}{\pgfqpoint{7.842520in}{7.842520in}}%
\pgfusepath{clip}%
\pgfsetbuttcap%
\pgfsetroundjoin%
\definecolor{currentfill}{rgb}{0.275191,0.194905,0.496005}%
\pgfsetfillcolor{currentfill}%
\pgfsetlinewidth{0.000000pt}%
\definecolor{currentstroke}{rgb}{0.248629,0.278775,0.534556}%
\pgfsetstrokecolor{currentstroke}%
\pgfsetdash{}{0pt}%
\pgfpathmoveto{\pgfqpoint{5.536826in}{3.584474in}}%
\pgfpathlineto{\pgfqpoint{5.610860in}{3.587946in}}%
\pgfpathlineto{\pgfqpoint{5.475289in}{3.644517in}}%
\pgfpathclose%
\pgfusepath{fill}%
\end{pgfscope}%
\begin{pgfscope}%
\pgfpathrectangle{\pgfqpoint{0.539299in}{0.078740in}}{\pgfqpoint{7.842520in}{7.842520in}}%
\pgfusepath{clip}%
\pgfsetbuttcap%
\pgfsetroundjoin%
\definecolor{currentfill}{rgb}{0.279574,0.170599,0.479997}%
\pgfsetfillcolor{currentfill}%
\pgfsetlinewidth{0.000000pt}%
\definecolor{currentstroke}{rgb}{0.246811,0.283237,0.535941}%
\pgfsetstrokecolor{currentstroke}%
\pgfsetdash{}{0pt}%
\pgfpathmoveto{\pgfqpoint{5.820187in}{3.551515in}}%
\pgfpathlineto{\pgfqpoint{5.746902in}{3.537634in}}%
\pgfpathlineto{\pgfqpoint{5.883392in}{3.490293in}}%
\pgfpathclose%
\pgfusepath{fill}%
\end{pgfscope}%
\begin{pgfscope}%
\pgfpathrectangle{\pgfqpoint{0.539299in}{0.078740in}}{\pgfqpoint{7.842520in}{7.842520in}}%
\pgfusepath{clip}%
\pgfsetbuttcap%
\pgfsetroundjoin%
\definecolor{currentfill}{rgb}{0.268510,0.009605,0.335427}%
\pgfsetfillcolor{currentfill}%
\pgfsetlinewidth{0.000000pt}%
\definecolor{currentstroke}{rgb}{0.244972,0.287675,0.537260}%
\pgfsetstrokecolor{currentstroke}%
\pgfsetdash{}{0pt}%
\pgfpathmoveto{\pgfqpoint{7.541482in}{3.001589in}}%
\pgfpathlineto{\pgfqpoint{7.471813in}{3.109193in}}%
\pgfpathlineto{\pgfqpoint{7.402794in}{3.055818in}}%
\pgfpathclose%
\pgfusepath{fill}%
\end{pgfscope}%
\begin{pgfscope}%
\pgfpathrectangle{\pgfqpoint{0.539299in}{0.078740in}}{\pgfqpoint{7.842520in}{7.842520in}}%
\pgfusepath{clip}%
\pgfsetbuttcap%
\pgfsetroundjoin%
\definecolor{currentfill}{rgb}{0.282623,0.140926,0.457517}%
\pgfsetfillcolor{currentfill}%
\pgfsetlinewidth{0.000000pt}%
\definecolor{currentstroke}{rgb}{0.243113,0.292092,0.538516}%
\pgfsetstrokecolor{currentstroke}%
\pgfsetdash{}{0pt}%
\pgfpathmoveto{\pgfqpoint{6.229961in}{3.418141in}}%
\pgfpathlineto{\pgfqpoint{6.301764in}{3.437824in}}%
\pgfpathlineto{\pgfqpoint{6.093026in}{3.466206in}}%
\pgfpathclose%
\pgfusepath{fill}%
\end{pgfscope}%
\begin{pgfscope}%
\pgfpathrectangle{\pgfqpoint{0.539299in}{0.078740in}}{\pgfqpoint{7.842520in}{7.842520in}}%
\pgfusepath{clip}%
\pgfsetbuttcap%
\pgfsetroundjoin%
\definecolor{currentfill}{rgb}{0.149039,0.508051,0.557250}%
\pgfsetfillcolor{currentfill}%
\pgfsetlinewidth{0.000000pt}%
\definecolor{currentstroke}{rgb}{0.241237,0.296485,0.539709}%
\pgfsetstrokecolor{currentstroke}%
\pgfsetdash{}{0pt}%
\pgfpathmoveto{\pgfqpoint{3.976815in}{4.900063in}}%
\pgfpathlineto{\pgfqpoint{4.111002in}{4.812426in}}%
\pgfpathlineto{\pgfqpoint{4.190652in}{4.714468in}}%
\pgfpathclose%
\pgfusepath{fill}%
\end{pgfscope}%
\begin{pgfscope}%
\pgfpathrectangle{\pgfqpoint{0.539299in}{0.078740in}}{\pgfqpoint{7.842520in}{7.842520in}}%
\pgfusepath{clip}%
\pgfsetbuttcap%
\pgfsetroundjoin%
\definecolor{currentfill}{rgb}{0.283072,0.130895,0.449241}%
\pgfsetfillcolor{currentfill}%
\pgfsetlinewidth{0.000000pt}%
\definecolor{currentstroke}{rgb}{0.239346,0.300855,0.540844}%
\pgfsetstrokecolor{currentstroke}%
\pgfsetdash{}{0pt}%
\pgfpathmoveto{\pgfqpoint{6.438604in}{3.384604in}}%
\pgfpathlineto{\pgfqpoint{6.301764in}{3.437824in}}%
\pgfpathlineto{\pgfqpoint{6.367154in}{3.364036in}}%
\pgfpathclose%
\pgfusepath{fill}%
\end{pgfscope}%
\begin{pgfscope}%
\pgfpathrectangle{\pgfqpoint{0.539299in}{0.078740in}}{\pgfqpoint{7.842520in}{7.842520in}}%
\pgfusepath{clip}%
\pgfsetbuttcap%
\pgfsetroundjoin%
\definecolor{currentfill}{rgb}{0.235526,0.309527,0.542944}%
\pgfsetfillcolor{currentfill}%
\pgfsetlinewidth{0.000000pt}%
\definecolor{currentstroke}{rgb}{0.237441,0.305202,0.541921}%
\pgfsetstrokecolor{currentstroke}%
\pgfsetdash{}{0pt}%
\pgfpathmoveto{\pgfqpoint{4.861395in}{4.049469in}}%
\pgfpathlineto{\pgfqpoint{4.919892in}{3.989315in}}%
\pgfpathlineto{\pgfqpoint{4.995865in}{3.930113in}}%
\pgfpathclose%
\pgfusepath{fill}%
\end{pgfscope}%
\begin{pgfscope}%
\pgfpathrectangle{\pgfqpoint{0.539299in}{0.078740in}}{\pgfqpoint{7.842520in}{7.842520in}}%
\pgfusepath{clip}%
\pgfsetbuttcap%
\pgfsetroundjoin%
\definecolor{currentfill}{rgb}{0.204903,0.375746,0.553533}%
\pgfsetfillcolor{currentfill}%
\pgfsetlinewidth{0.000000pt}%
\definecolor{currentstroke}{rgb}{0.235526,0.309527,0.542944}%
\pgfsetstrokecolor{currentstroke}%
\pgfsetdash{}{0pt}%
\pgfpathmoveto{\pgfqpoint{2.830306in}{4.368134in}}%
\pgfpathlineto{\pgfqpoint{2.913780in}{4.395967in}}%
\pgfpathlineto{\pgfqpoint{2.793452in}{3.956696in}}%
\pgfpathclose%
\pgfusepath{fill}%
\end{pgfscope}%
\begin{pgfscope}%
\pgfpathrectangle{\pgfqpoint{0.539299in}{0.078740in}}{\pgfqpoint{7.842520in}{7.842520in}}%
\pgfusepath{clip}%
\pgfsetbuttcap%
\pgfsetroundjoin%
\definecolor{currentfill}{rgb}{0.267004,0.004874,0.329415}%
\pgfsetfillcolor{currentfill}%
\pgfsetlinewidth{0.000000pt}%
\definecolor{currentstroke}{rgb}{0.233603,0.313828,0.543914}%
\pgfsetstrokecolor{currentstroke}%
\pgfsetdash{}{0pt}%
\pgfpathmoveto{\pgfqpoint{7.749729in}{3.018900in}}%
\pgfpathlineto{\pgfqpoint{7.610502in}{3.062741in}}%
\pgfpathlineto{\pgfqpoint{7.541482in}{3.001589in}}%
\pgfpathclose%
\pgfusepath{fill}%
\end{pgfscope}%
\begin{pgfscope}%
\pgfpathrectangle{\pgfqpoint{0.539299in}{0.078740in}}{\pgfqpoint{7.842520in}{7.842520in}}%
\pgfusepath{clip}%
\pgfsetbuttcap%
\pgfsetroundjoin%
\definecolor{currentfill}{rgb}{0.282656,0.100196,0.422160}%
\pgfsetfillcolor{currentfill}%
\pgfsetlinewidth{0.000000pt}%
\definecolor{currentstroke}{rgb}{0.231674,0.318106,0.544834}%
\pgfsetstrokecolor{currentstroke}%
\pgfsetdash{}{0pt}%
\pgfpathmoveto{\pgfqpoint{6.712873in}{3.258583in}}%
\pgfpathlineto{\pgfqpoint{6.783027in}{3.277708in}}%
\pgfpathlineto{\pgfqpoint{6.575647in}{3.324679in}}%
\pgfpathclose%
\pgfusepath{fill}%
\end{pgfscope}%
\begin{pgfscope}%
\pgfpathrectangle{\pgfqpoint{0.539299in}{0.078740in}}{\pgfqpoint{7.842520in}{7.842520in}}%
\pgfusepath{clip}%
\pgfsetbuttcap%
\pgfsetroundjoin%
\definecolor{currentfill}{rgb}{0.162142,0.474838,0.558140}%
\pgfsetfillcolor{currentfill}%
\pgfsetlinewidth{0.000000pt}%
\definecolor{currentstroke}{rgb}{0.229739,0.322361,0.545706}%
\pgfsetstrokecolor{currentstroke}%
\pgfsetdash{}{0pt}%
\pgfpathmoveto{\pgfqpoint{3.121655in}{4.747134in}}%
\pgfpathlineto{\pgfqpoint{3.204675in}{4.746147in}}%
\pgfpathlineto{\pgfqpoint{3.079268in}{4.442627in}}%
\pgfpathclose%
\pgfusepath{fill}%
\end{pgfscope}%
\begin{pgfscope}%
\pgfpathrectangle{\pgfqpoint{0.539299in}{0.078740in}}{\pgfqpoint{7.842520in}{7.842520in}}%
\pgfusepath{clip}%
\pgfsetbuttcap%
\pgfsetroundjoin%
\definecolor{currentfill}{rgb}{0.157729,0.485932,0.558013}%
\pgfsetfillcolor{currentfill}%
\pgfsetlinewidth{0.000000pt}%
\definecolor{currentstroke}{rgb}{0.227802,0.326594,0.546532}%
\pgfsetstrokecolor{currentstroke}%
\pgfsetdash{}{0pt}%
\pgfpathmoveto{\pgfqpoint{4.324600in}{4.594132in}}%
\pgfpathlineto{\pgfqpoint{4.111002in}{4.812426in}}%
\pgfpathlineto{\pgfqpoint{4.245563in}{4.694927in}}%
\pgfpathclose%
\pgfusepath{fill}%
\end{pgfscope}%
\begin{pgfscope}%
\pgfpathrectangle{\pgfqpoint{0.539299in}{0.078740in}}{\pgfqpoint{7.842520in}{7.842520in}}%
\pgfusepath{clip}%
\pgfsetbuttcap%
\pgfsetroundjoin%
\definecolor{currentfill}{rgb}{0.280894,0.078907,0.402329}%
\pgfsetfillcolor{currentfill}%
\pgfsetlinewidth{0.000000pt}%
\definecolor{currentstroke}{rgb}{0.225863,0.330805,0.547314}%
\pgfsetstrokecolor{currentstroke}%
\pgfsetdash{}{0pt}%
\pgfpathmoveto{\pgfqpoint{6.783027in}{3.277708in}}%
\pgfpathlineto{\pgfqpoint{6.850286in}{3.187517in}}%
\pgfpathlineto{\pgfqpoint{6.920184in}{3.211063in}}%
\pgfpathclose%
\pgfusepath{fill}%
\end{pgfscope}%
\begin{pgfscope}%
\pgfpathrectangle{\pgfqpoint{0.539299in}{0.078740in}}{\pgfqpoint{7.842520in}{7.842520in}}%
\pgfusepath{clip}%
\pgfsetbuttcap%
\pgfsetroundjoin%
\definecolor{currentfill}{rgb}{0.273006,0.204520,0.501721}%
\pgfsetfillcolor{currentfill}%
\pgfsetlinewidth{0.000000pt}%
\definecolor{currentstroke}{rgb}{0.223925,0.334994,0.548053}%
\pgfsetstrokecolor{currentstroke}%
\pgfsetdash{}{0pt}%
\pgfpathmoveto{\pgfqpoint{5.475289in}{3.644517in}}%
\pgfpathlineto{\pgfqpoint{5.400994in}{3.652853in}}%
\pgfpathlineto{\pgfqpoint{5.536826in}{3.584474in}}%
\pgfpathclose%
\pgfusepath{fill}%
\end{pgfscope}%
\begin{pgfscope}%
\pgfpathrectangle{\pgfqpoint{0.539299in}{0.078740in}}{\pgfqpoint{7.842520in}{7.842520in}}%
\pgfusepath{clip}%
\pgfsetbuttcap%
\pgfsetroundjoin%
\definecolor{currentfill}{rgb}{0.243113,0.292092,0.538516}%
\pgfsetfillcolor{currentfill}%
\pgfsetlinewidth{0.000000pt}%
\definecolor{currentstroke}{rgb}{0.221989,0.339161,0.548752}%
\pgfsetstrokecolor{currentstroke}%
\pgfsetdash{}{0pt}%
\pgfpathmoveto{\pgfqpoint{4.995865in}{3.930113in}}%
\pgfpathlineto{\pgfqpoint{4.919892in}{3.989315in}}%
\pgfpathlineto{\pgfqpoint{5.130570in}{3.824189in}}%
\pgfpathclose%
\pgfusepath{fill}%
\end{pgfscope}%
\begin{pgfscope}%
\pgfpathrectangle{\pgfqpoint{0.539299in}{0.078740in}}{\pgfqpoint{7.842520in}{7.842520in}}%
\pgfusepath{clip}%
\pgfsetbuttcap%
\pgfsetroundjoin%
\definecolor{currentfill}{rgb}{0.280868,0.160771,0.472899}%
\pgfsetfillcolor{currentfill}%
\pgfsetlinewidth{0.000000pt}%
\definecolor{currentstroke}{rgb}{0.220057,0.343307,0.549413}%
\pgfsetstrokecolor{currentstroke}%
\pgfsetdash{}{0pt}%
\pgfpathmoveto{\pgfqpoint{6.020289in}{3.442779in}}%
\pgfpathlineto{\pgfqpoint{6.093026in}{3.466206in}}%
\pgfpathlineto{\pgfqpoint{5.956414in}{3.509868in}}%
\pgfpathclose%
\pgfusepath{fill}%
\end{pgfscope}%
\begin{pgfscope}%
\pgfpathrectangle{\pgfqpoint{0.539299in}{0.078740in}}{\pgfqpoint{7.842520in}{7.842520in}}%
\pgfusepath{clip}%
\pgfsetbuttcap%
\pgfsetroundjoin%
\definecolor{currentfill}{rgb}{0.276194,0.190074,0.493001}%
\pgfsetfillcolor{currentfill}%
\pgfsetlinewidth{0.000000pt}%
\definecolor{currentstroke}{rgb}{0.218130,0.347432,0.550038}%
\pgfsetstrokecolor{currentstroke}%
\pgfsetdash{}{0pt}%
\pgfpathmoveto{\pgfqpoint{5.536826in}{3.584474in}}%
\pgfpathlineto{\pgfqpoint{5.746902in}{3.537634in}}%
\pgfpathlineto{\pgfqpoint{5.610860in}{3.587946in}}%
\pgfpathclose%
\pgfusepath{fill}%
\end{pgfscope}%
\begin{pgfscope}%
\pgfpathrectangle{\pgfqpoint{0.539299in}{0.078740in}}{\pgfqpoint{7.842520in}{7.842520in}}%
\pgfusepath{clip}%
\pgfsetbuttcap%
\pgfsetroundjoin%
\definecolor{currentfill}{rgb}{0.274952,0.037752,0.364543}%
\pgfsetfillcolor{currentfill}%
\pgfsetlinewidth{0.000000pt}%
\definecolor{currentstroke}{rgb}{0.216210,0.351535,0.550627}%
\pgfsetstrokecolor{currentstroke}%
\pgfsetdash{}{0pt}%
\pgfpathmoveto{\pgfqpoint{7.195388in}{3.074084in}}%
\pgfpathlineto{\pgfqpoint{7.264628in}{3.113510in}}%
\pgfpathlineto{\pgfqpoint{7.057617in}{3.142554in}}%
\pgfpathclose%
\pgfusepath{fill}%
\end{pgfscope}%
\begin{pgfscope}%
\pgfpathrectangle{\pgfqpoint{0.539299in}{0.078740in}}{\pgfqpoint{7.842520in}{7.842520in}}%
\pgfusepath{clip}%
\pgfsetbuttcap%
\pgfsetroundjoin%
\definecolor{currentfill}{rgb}{0.165117,0.467423,0.558141}%
\pgfsetfillcolor{currentfill}%
\pgfsetlinewidth{0.000000pt}%
\definecolor{currentstroke}{rgb}{0.214298,0.355619,0.551184}%
\pgfsetstrokecolor{currentstroke}%
\pgfsetdash{}{0pt}%
\pgfpathmoveto{\pgfqpoint{4.245563in}{4.694927in}}%
\pgfpathlineto{\pgfqpoint{4.380311in}{4.558363in}}%
\pgfpathlineto{\pgfqpoint{4.324600in}{4.594132in}}%
\pgfpathclose%
\pgfusepath{fill}%
\end{pgfscope}%
\begin{pgfscope}%
\pgfpathrectangle{\pgfqpoint{0.539299in}{0.078740in}}{\pgfqpoint{7.842520in}{7.842520in}}%
\pgfusepath{clip}%
\pgfsetbuttcap%
\pgfsetroundjoin%
\definecolor{currentfill}{rgb}{0.135066,0.544853,0.554029}%
\pgfsetfillcolor{currentfill}%
\pgfsetlinewidth{0.000000pt}%
\definecolor{currentstroke}{rgb}{0.212395,0.359683,0.551710}%
\pgfsetstrokecolor{currentstroke}%
\pgfsetdash{}{0pt}%
\pgfpathmoveto{\pgfqpoint{3.629210in}{4.996070in}}%
\pgfpathlineto{\pgfqpoint{3.498080in}{4.903513in}}%
\pgfpathlineto{\pgfqpoint{3.415921in}{4.934934in}}%
\pgfpathclose%
\pgfusepath{fill}%
\end{pgfscope}%
\begin{pgfscope}%
\pgfpathrectangle{\pgfqpoint{0.539299in}{0.078740in}}{\pgfqpoint{7.842520in}{7.842520in}}%
\pgfusepath{clip}%
\pgfsetbuttcap%
\pgfsetroundjoin%
\definecolor{currentfill}{rgb}{0.282623,0.140926,0.457517}%
\pgfsetfillcolor{currentfill}%
\pgfsetlinewidth{0.000000pt}%
\definecolor{currentstroke}{rgb}{0.210503,0.363727,0.552206}%
\pgfsetstrokecolor{currentstroke}%
\pgfsetdash{}{0pt}%
\pgfpathmoveto{\pgfqpoint{6.367154in}{3.364036in}}%
\pgfpathlineto{\pgfqpoint{6.301764in}{3.437824in}}%
\pgfpathlineto{\pgfqpoint{6.229961in}{3.418141in}}%
\pgfpathclose%
\pgfusepath{fill}%
\end{pgfscope}%
\begin{pgfscope}%
\pgfpathrectangle{\pgfqpoint{0.539299in}{0.078740in}}{\pgfqpoint{7.842520in}{7.842520in}}%
\pgfusepath{clip}%
\pgfsetbuttcap%
\pgfsetroundjoin%
\definecolor{currentfill}{rgb}{0.271305,0.019942,0.347269}%
\pgfsetfillcolor{currentfill}%
\pgfsetlinewidth{0.000000pt}%
\definecolor{currentstroke}{rgb}{0.208623,0.367752,0.552675}%
\pgfsetstrokecolor{currentstroke}%
\pgfsetdash{}{0pt}%
\pgfpathmoveto{\pgfqpoint{7.264628in}{3.113510in}}%
\pgfpathlineto{\pgfqpoint{7.333570in}{3.007542in}}%
\pgfpathlineto{\pgfqpoint{7.402794in}{3.055818in}}%
\pgfpathclose%
\pgfusepath{fill}%
\end{pgfscope}%
\begin{pgfscope}%
\pgfpathrectangle{\pgfqpoint{0.539299in}{0.078740in}}{\pgfqpoint{7.842520in}{7.842520in}}%
\pgfusepath{clip}%
\pgfsetbuttcap%
\pgfsetroundjoin%
\definecolor{currentfill}{rgb}{0.182256,0.426184,0.557120}%
\pgfsetfillcolor{currentfill}%
\pgfsetlinewidth{0.000000pt}%
\definecolor{currentstroke}{rgb}{0.206756,0.371758,0.553117}%
\pgfsetstrokecolor{currentstroke}%
\pgfsetdash{}{0pt}%
\pgfpathmoveto{\pgfqpoint{4.515135in}{4.412321in}}%
\pgfpathlineto{\pgfqpoint{4.592860in}{4.318703in}}%
\pgfpathlineto{\pgfqpoint{4.380311in}{4.558363in}}%
\pgfpathclose%
\pgfusepath{fill}%
\end{pgfscope}%
\begin{pgfscope}%
\pgfpathrectangle{\pgfqpoint{0.539299in}{0.078740in}}{\pgfqpoint{7.842520in}{7.842520in}}%
\pgfusepath{clip}%
\pgfsetbuttcap%
\pgfsetroundjoin%
\definecolor{currentfill}{rgb}{0.192357,0.403199,0.555836}%
\pgfsetfillcolor{currentfill}%
\pgfsetlinewidth{0.000000pt}%
\definecolor{currentstroke}{rgb}{0.204903,0.375746,0.553533}%
\pgfsetstrokecolor{currentstroke}%
\pgfsetdash{}{0pt}%
\pgfpathmoveto{\pgfqpoint{4.515135in}{4.412321in}}%
\pgfpathlineto{\pgfqpoint{4.649993in}{4.264866in}}%
\pgfpathlineto{\pgfqpoint{4.592860in}{4.318703in}}%
\pgfpathclose%
\pgfusepath{fill}%
\end{pgfscope}%
\begin{pgfscope}%
\pgfpathrectangle{\pgfqpoint{0.539299in}{0.078740in}}{\pgfqpoint{7.842520in}{7.842520in}}%
\pgfusepath{clip}%
\pgfsetbuttcap%
\pgfsetroundjoin%
\definecolor{currentfill}{rgb}{0.281924,0.089666,0.412415}%
\pgfsetfillcolor{currentfill}%
\pgfsetlinewidth{0.000000pt}%
\definecolor{currentstroke}{rgb}{0.203063,0.379716,0.553925}%
\pgfsetstrokecolor{currentstroke}%
\pgfsetdash{}{0pt}%
\pgfpathmoveto{\pgfqpoint{6.712873in}{3.258583in}}%
\pgfpathlineto{\pgfqpoint{6.850286in}{3.187517in}}%
\pgfpathlineto{\pgfqpoint{6.783027in}{3.277708in}}%
\pgfpathclose%
\pgfusepath{fill}%
\end{pgfscope}%
\begin{pgfscope}%
\pgfpathrectangle{\pgfqpoint{0.539299in}{0.078740in}}{\pgfqpoint{7.842520in}{7.842520in}}%
\pgfusepath{clip}%
\pgfsetbuttcap%
\pgfsetroundjoin%
\definecolor{currentfill}{rgb}{0.280255,0.165693,0.476498}%
\pgfsetfillcolor{currentfill}%
\pgfsetlinewidth{0.000000pt}%
\definecolor{currentstroke}{rgb}{0.201239,0.383670,0.554294}%
\pgfsetstrokecolor{currentstroke}%
\pgfsetdash{}{0pt}%
\pgfpathmoveto{\pgfqpoint{5.956414in}{3.509868in}}%
\pgfpathlineto{\pgfqpoint{5.883392in}{3.490293in}}%
\pgfpathlineto{\pgfqpoint{6.020289in}{3.442779in}}%
\pgfpathclose%
\pgfusepath{fill}%
\end{pgfscope}%
\begin{pgfscope}%
\pgfpathrectangle{\pgfqpoint{0.539299in}{0.078740in}}{\pgfqpoint{7.842520in}{7.842520in}}%
\pgfusepath{clip}%
\pgfsetbuttcap%
\pgfsetroundjoin%
\definecolor{currentfill}{rgb}{0.279566,0.067836,0.391917}%
\pgfsetfillcolor{currentfill}%
\pgfsetlinewidth{0.000000pt}%
\definecolor{currentstroke}{rgb}{0.199430,0.387607,0.554642}%
\pgfsetstrokecolor{currentstroke}%
\pgfsetdash{}{0pt}%
\pgfpathmoveto{\pgfqpoint{6.920184in}{3.211063in}}%
\pgfpathlineto{\pgfqpoint{6.850286in}{3.187517in}}%
\pgfpathlineto{\pgfqpoint{7.057617in}{3.142554in}}%
\pgfpathclose%
\pgfusepath{fill}%
\end{pgfscope}%
\begin{pgfscope}%
\pgfpathrectangle{\pgfqpoint{0.539299in}{0.078740in}}{\pgfqpoint{7.842520in}{7.842520in}}%
\pgfusepath{clip}%
\pgfsetbuttcap%
\pgfsetroundjoin%
\definecolor{currentfill}{rgb}{0.214298,0.355619,0.551184}%
\pgfsetfillcolor{currentfill}%
\pgfsetlinewidth{0.000000pt}%
\definecolor{currentstroke}{rgb}{0.197636,0.391528,0.554969}%
\pgfsetstrokecolor{currentstroke}%
\pgfsetdash{}{0pt}%
\pgfpathmoveto{\pgfqpoint{4.861395in}{4.049469in}}%
\pgfpathlineto{\pgfqpoint{4.649993in}{4.264866in}}%
\pgfpathlineto{\pgfqpoint{4.784895in}{4.122344in}}%
\pgfpathclose%
\pgfusepath{fill}%
\end{pgfscope}%
\begin{pgfscope}%
\pgfpathrectangle{\pgfqpoint{0.539299in}{0.078740in}}{\pgfqpoint{7.842520in}{7.842520in}}%
\pgfusepath{clip}%
\pgfsetbuttcap%
\pgfsetroundjoin%
\definecolor{currentfill}{rgb}{0.257322,0.256130,0.526563}%
\pgfsetfillcolor{currentfill}%
\pgfsetlinewidth{0.000000pt}%
\definecolor{currentstroke}{rgb}{0.195860,0.395433,0.555276}%
\pgfsetstrokecolor{currentstroke}%
\pgfsetdash{}{0pt}%
\pgfpathmoveto{\pgfqpoint{5.190465in}{3.761377in}}%
\pgfpathlineto{\pgfqpoint{5.265591in}{3.732089in}}%
\pgfpathlineto{\pgfqpoint{5.130570in}{3.824189in}}%
\pgfpathclose%
\pgfusepath{fill}%
\end{pgfscope}%
\begin{pgfscope}%
\pgfpathrectangle{\pgfqpoint{0.539299in}{0.078740in}}{\pgfqpoint{7.842520in}{7.842520in}}%
\pgfusepath{clip}%
\pgfsetbuttcap%
\pgfsetroundjoin%
\definecolor{currentfill}{rgb}{0.132444,0.552216,0.553018}%
\pgfsetfillcolor{currentfill}%
\pgfsetlinewidth{0.000000pt}%
\definecolor{currentstroke}{rgb}{0.194100,0.399323,0.555565}%
\pgfsetstrokecolor{currentstroke}%
\pgfsetdash{}{0pt}%
\pgfpathmoveto{\pgfqpoint{3.976815in}{4.900063in}}%
\pgfpathlineto{\pgfqpoint{3.843276in}{4.946277in}}%
\pgfpathlineto{\pgfqpoint{3.761962in}{5.020676in}}%
\pgfpathclose%
\pgfusepath{fill}%
\end{pgfscope}%
\begin{pgfscope}%
\pgfpathrectangle{\pgfqpoint{0.539299in}{0.078740in}}{\pgfqpoint{7.842520in}{7.842520in}}%
\pgfusepath{clip}%
\pgfsetbuttcap%
\pgfsetroundjoin%
\definecolor{currentfill}{rgb}{0.283229,0.120777,0.440584}%
\pgfsetfillcolor{currentfill}%
\pgfsetlinewidth{0.000000pt}%
\definecolor{currentstroke}{rgb}{0.192357,0.403199,0.555836}%
\pgfsetstrokecolor{currentstroke}%
\pgfsetdash{}{0pt}%
\pgfpathmoveto{\pgfqpoint{6.504552in}{3.303089in}}%
\pgfpathlineto{\pgfqpoint{6.575647in}{3.324679in}}%
\pgfpathlineto{\pgfqpoint{6.438604in}{3.384604in}}%
\pgfpathclose%
\pgfusepath{fill}%
\end{pgfscope}%
\begin{pgfscope}%
\pgfpathrectangle{\pgfqpoint{0.539299in}{0.078740in}}{\pgfqpoint{7.842520in}{7.842520in}}%
\pgfusepath{clip}%
\pgfsetbuttcap%
\pgfsetroundjoin%
\definecolor{currentfill}{rgb}{0.272594,0.025563,0.353093}%
\pgfsetfillcolor{currentfill}%
\pgfsetlinewidth{0.000000pt}%
\definecolor{currentstroke}{rgb}{0.190631,0.407061,0.556089}%
\pgfsetstrokecolor{currentstroke}%
\pgfsetdash{}{0pt}%
\pgfpathmoveto{\pgfqpoint{7.195388in}{3.074084in}}%
\pgfpathlineto{\pgfqpoint{7.333570in}{3.007542in}}%
\pgfpathlineto{\pgfqpoint{7.264628in}{3.113510in}}%
\pgfpathclose%
\pgfusepath{fill}%
\end{pgfscope}%
\begin{pgfscope}%
\pgfpathrectangle{\pgfqpoint{0.539299in}{0.078740in}}{\pgfqpoint{7.842520in}{7.842520in}}%
\pgfusepath{clip}%
\pgfsetbuttcap%
\pgfsetroundjoin%
\definecolor{currentfill}{rgb}{0.223925,0.334994,0.548053}%
\pgfsetfillcolor{currentfill}%
\pgfsetlinewidth{0.000000pt}%
\definecolor{currentstroke}{rgb}{0.188923,0.410910,0.556326}%
\pgfsetstrokecolor{currentstroke}%
\pgfsetdash{}{0pt}%
\pgfpathmoveto{\pgfqpoint{4.784895in}{4.122344in}}%
\pgfpathlineto{\pgfqpoint{4.919892in}{3.989315in}}%
\pgfpathlineto{\pgfqpoint{4.861395in}{4.049469in}}%
\pgfpathclose%
\pgfusepath{fill}%
\end{pgfscope}%
\begin{pgfscope}%
\pgfpathrectangle{\pgfqpoint{0.539299in}{0.078740in}}{\pgfqpoint{7.842520in}{7.842520in}}%
\pgfusepath{clip}%
\pgfsetbuttcap%
\pgfsetroundjoin%
\definecolor{currentfill}{rgb}{0.225863,0.330805,0.547314}%
\pgfsetfillcolor{currentfill}%
\pgfsetlinewidth{0.000000pt}%
\definecolor{currentstroke}{rgb}{0.187231,0.414746,0.556547}%
\pgfsetstrokecolor{currentstroke}%
\pgfsetdash{}{0pt}%
\pgfpathmoveto{\pgfqpoint{2.710747in}{3.908413in}}%
\pgfpathlineto{\pgfqpoint{2.627589in}{3.858151in}}%
\pgfpathlineto{\pgfqpoint{2.746363in}{4.336760in}}%
\pgfpathclose%
\pgfusepath{fill}%
\end{pgfscope}%
\begin{pgfscope}%
\pgfpathrectangle{\pgfqpoint{0.539299in}{0.078740in}}{\pgfqpoint{7.842520in}{7.842520in}}%
\pgfusepath{clip}%
\pgfsetbuttcap%
\pgfsetroundjoin%
\definecolor{currentfill}{rgb}{0.129933,0.559582,0.551864}%
\pgfsetfillcolor{currentfill}%
\pgfsetlinewidth{0.000000pt}%
\definecolor{currentstroke}{rgb}{0.185556,0.418570,0.556753}%
\pgfsetstrokecolor{currentstroke}%
\pgfsetdash{}{0pt}%
\pgfpathmoveto{\pgfqpoint{3.629210in}{4.996070in}}%
\pgfpathlineto{\pgfqpoint{3.761962in}{5.020676in}}%
\pgfpathlineto{\pgfqpoint{3.843276in}{4.946277in}}%
\pgfpathclose%
\pgfusepath{fill}%
\end{pgfscope}%
\begin{pgfscope}%
\pgfpathrectangle{\pgfqpoint{0.539299in}{0.078740in}}{\pgfqpoint{7.842520in}{7.842520in}}%
\pgfusepath{clip}%
\pgfsetbuttcap%
\pgfsetroundjoin%
\definecolor{currentfill}{rgb}{0.143343,0.522773,0.556295}%
\pgfsetfillcolor{currentfill}%
\pgfsetlinewidth{0.000000pt}%
\definecolor{currentstroke}{rgb}{0.183898,0.422383,0.556944}%
\pgfsetstrokecolor{currentstroke}%
\pgfsetdash{}{0pt}%
\pgfpathmoveto{\pgfqpoint{3.287152in}{4.741607in}}%
\pgfpathlineto{\pgfqpoint{3.204675in}{4.746147in}}%
\pgfpathlineto{\pgfqpoint{3.333165in}{4.962328in}}%
\pgfpathclose%
\pgfusepath{fill}%
\end{pgfscope}%
\begin{pgfscope}%
\pgfpathrectangle{\pgfqpoint{0.539299in}{0.078740in}}{\pgfqpoint{7.842520in}{7.842520in}}%
\pgfusepath{clip}%
\pgfsetbuttcap%
\pgfsetroundjoin%
\definecolor{currentfill}{rgb}{0.267004,0.004874,0.329415}%
\pgfsetfillcolor{currentfill}%
\pgfsetlinewidth{0.000000pt}%
\definecolor{currentstroke}{rgb}{0.182256,0.426184,0.557120}%
\pgfsetstrokecolor{currentstroke}%
\pgfsetdash{}{0pt}%
\pgfpathmoveto{\pgfqpoint{7.541482in}{3.001589in}}%
\pgfpathlineto{\pgfqpoint{7.680757in}{2.951802in}}%
\pgfpathlineto{\pgfqpoint{7.749729in}{3.018900in}}%
\pgfpathclose%
\pgfusepath{fill}%
\end{pgfscope}%
\begin{pgfscope}%
\pgfpathrectangle{\pgfqpoint{0.539299in}{0.078740in}}{\pgfqpoint{7.842520in}{7.842520in}}%
\pgfusepath{clip}%
\pgfsetbuttcap%
\pgfsetroundjoin%
\definecolor{currentfill}{rgb}{0.262138,0.242286,0.520837}%
\pgfsetfillcolor{currentfill}%
\pgfsetlinewidth{0.000000pt}%
\definecolor{currentstroke}{rgb}{0.180629,0.429975,0.557282}%
\pgfsetstrokecolor{currentstroke}%
\pgfsetdash{}{0pt}%
\pgfpathmoveto{\pgfqpoint{5.400994in}{3.652853in}}%
\pgfpathlineto{\pgfqpoint{5.265591in}{3.732089in}}%
\pgfpathlineto{\pgfqpoint{5.190465in}{3.761377in}}%
\pgfpathclose%
\pgfusepath{fill}%
\end{pgfscope}%
\begin{pgfscope}%
\pgfpathrectangle{\pgfqpoint{0.539299in}{0.078740in}}{\pgfqpoint{7.842520in}{7.842520in}}%
\pgfusepath{clip}%
\pgfsetbuttcap%
\pgfsetroundjoin%
\definecolor{currentfill}{rgb}{0.281412,0.155834,0.469201}%
\pgfsetfillcolor{currentfill}%
\pgfsetlinewidth{0.000000pt}%
\definecolor{currentstroke}{rgb}{0.179019,0.433756,0.557430}%
\pgfsetstrokecolor{currentstroke}%
\pgfsetdash{}{0pt}%
\pgfpathmoveto{\pgfqpoint{6.229961in}{3.418141in}}%
\pgfpathlineto{\pgfqpoint{6.093026in}{3.466206in}}%
\pgfpathlineto{\pgfqpoint{6.157532in}{3.392366in}}%
\pgfpathclose%
\pgfusepath{fill}%
\end{pgfscope}%
\begin{pgfscope}%
\pgfpathrectangle{\pgfqpoint{0.539299in}{0.078740in}}{\pgfqpoint{7.842520in}{7.842520in}}%
\pgfusepath{clip}%
\pgfsetbuttcap%
\pgfsetroundjoin%
\definecolor{currentfill}{rgb}{0.137770,0.537492,0.554906}%
\pgfsetfillcolor{currentfill}%
\pgfsetlinewidth{0.000000pt}%
\definecolor{currentstroke}{rgb}{0.177423,0.437527,0.557565}%
\pgfsetstrokecolor{currentstroke}%
\pgfsetdash{}{0pt}%
\pgfpathmoveto{\pgfqpoint{3.333165in}{4.962328in}}%
\pgfpathlineto{\pgfqpoint{3.415921in}{4.934934in}}%
\pgfpathlineto{\pgfqpoint{3.287152in}{4.741607in}}%
\pgfpathclose%
\pgfusepath{fill}%
\end{pgfscope}%
\begin{pgfscope}%
\pgfpathrectangle{\pgfqpoint{0.539299in}{0.078740in}}{\pgfqpoint{7.842520in}{7.842520in}}%
\pgfusepath{clip}%
\pgfsetbuttcap%
\pgfsetroundjoin%
\definecolor{currentfill}{rgb}{0.283072,0.130895,0.449241}%
\pgfsetfillcolor{currentfill}%
\pgfsetlinewidth{0.000000pt}%
\definecolor{currentstroke}{rgb}{0.175841,0.441290,0.557685}%
\pgfsetstrokecolor{currentstroke}%
\pgfsetdash{}{0pt}%
\pgfpathmoveto{\pgfqpoint{6.438604in}{3.384604in}}%
\pgfpathlineto{\pgfqpoint{6.367154in}{3.364036in}}%
\pgfpathlineto{\pgfqpoint{6.504552in}{3.303089in}}%
\pgfpathclose%
\pgfusepath{fill}%
\end{pgfscope}%
\begin{pgfscope}%
\pgfpathrectangle{\pgfqpoint{0.539299in}{0.078740in}}{\pgfqpoint{7.842520in}{7.842520in}}%
\pgfusepath{clip}%
\pgfsetbuttcap%
\pgfsetroundjoin%
\definecolor{currentfill}{rgb}{0.241237,0.296485,0.539709}%
\pgfsetfillcolor{currentfill}%
\pgfsetlinewidth{0.000000pt}%
\definecolor{currentstroke}{rgb}{0.174274,0.445044,0.557792}%
\pgfsetstrokecolor{currentstroke}%
\pgfsetdash{}{0pt}%
\pgfpathmoveto{\pgfqpoint{5.130570in}{3.824189in}}%
\pgfpathlineto{\pgfqpoint{4.919892in}{3.989315in}}%
\pgfpathlineto{\pgfqpoint{5.055056in}{3.868593in}}%
\pgfpathclose%
\pgfusepath{fill}%
\end{pgfscope}%
\begin{pgfscope}%
\pgfpathrectangle{\pgfqpoint{0.539299in}{0.078740in}}{\pgfqpoint{7.842520in}{7.842520in}}%
\pgfusepath{clip}%
\pgfsetbuttcap%
\pgfsetroundjoin%
\definecolor{currentfill}{rgb}{0.268510,0.009605,0.335427}%
\pgfsetfillcolor{currentfill}%
\pgfsetlinewidth{0.000000pt}%
\definecolor{currentstroke}{rgb}{0.172719,0.448791,0.557885}%
\pgfsetstrokecolor{currentstroke}%
\pgfsetdash{}{0pt}%
\pgfpathmoveto{\pgfqpoint{7.472249in}{2.944684in}}%
\pgfpathlineto{\pgfqpoint{7.541482in}{3.001589in}}%
\pgfpathlineto{\pgfqpoint{7.402794in}{3.055818in}}%
\pgfpathclose%
\pgfusepath{fill}%
\end{pgfscope}%
\begin{pgfscope}%
\pgfpathrectangle{\pgfqpoint{0.539299in}{0.078740in}}{\pgfqpoint{7.842520in}{7.842520in}}%
\pgfusepath{clip}%
\pgfsetbuttcap%
\pgfsetroundjoin%
\definecolor{currentfill}{rgb}{0.283197,0.115680,0.436115}%
\pgfsetfillcolor{currentfill}%
\pgfsetlinewidth{0.000000pt}%
\definecolor{currentstroke}{rgb}{0.171176,0.452530,0.557965}%
\pgfsetstrokecolor{currentstroke}%
\pgfsetdash{}{0pt}%
\pgfpathmoveto{\pgfqpoint{6.712873in}{3.258583in}}%
\pgfpathlineto{\pgfqpoint{6.575647in}{3.324679in}}%
\pgfpathlineto{\pgfqpoint{6.504552in}{3.303089in}}%
\pgfpathclose%
\pgfusepath{fill}%
\end{pgfscope}%
\begin{pgfscope}%
\pgfpathrectangle{\pgfqpoint{0.539299in}{0.078740in}}{\pgfqpoint{7.842520in}{7.842520in}}%
\pgfusepath{clip}%
\pgfsetbuttcap%
\pgfsetroundjoin%
\definecolor{currentfill}{rgb}{0.275191,0.194905,0.496005}%
\pgfsetfillcolor{currentfill}%
\pgfsetlinewidth{0.000000pt}%
\definecolor{currentstroke}{rgb}{0.169646,0.456262,0.558030}%
\pgfsetstrokecolor{currentstroke}%
\pgfsetdash{}{0pt}%
\pgfpathmoveto{\pgfqpoint{5.673111in}{3.524234in}}%
\pgfpathlineto{\pgfqpoint{5.746902in}{3.537634in}}%
\pgfpathlineto{\pgfqpoint{5.536826in}{3.584474in}}%
\pgfpathclose%
\pgfusepath{fill}%
\end{pgfscope}%
\begin{pgfscope}%
\pgfpathrectangle{\pgfqpoint{0.539299in}{0.078740in}}{\pgfqpoint{7.842520in}{7.842520in}}%
\pgfusepath{clip}%
\pgfsetbuttcap%
\pgfsetroundjoin%
\definecolor{currentfill}{rgb}{0.278012,0.180367,0.486697}%
\pgfsetfillcolor{currentfill}%
\pgfsetlinewidth{0.000000pt}%
\definecolor{currentstroke}{rgb}{0.168126,0.459988,0.558082}%
\pgfsetstrokecolor{currentstroke}%
\pgfsetdash{}{0pt}%
\pgfpathmoveto{\pgfqpoint{5.809842in}{3.469040in}}%
\pgfpathlineto{\pgfqpoint{5.883392in}{3.490293in}}%
\pgfpathlineto{\pgfqpoint{5.746902in}{3.537634in}}%
\pgfpathclose%
\pgfusepath{fill}%
\end{pgfscope}%
\begin{pgfscope}%
\pgfpathrectangle{\pgfqpoint{0.539299in}{0.078740in}}{\pgfqpoint{7.842520in}{7.842520in}}%
\pgfusepath{clip}%
\pgfsetbuttcap%
\pgfsetroundjoin%
\definecolor{currentfill}{rgb}{0.248629,0.278775,0.534556}%
\pgfsetfillcolor{currentfill}%
\pgfsetlinewidth{0.000000pt}%
\definecolor{currentstroke}{rgb}{0.166617,0.463708,0.558119}%
\pgfsetstrokecolor{currentstroke}%
\pgfsetdash{}{0pt}%
\pgfpathmoveto{\pgfqpoint{5.130570in}{3.824189in}}%
\pgfpathlineto{\pgfqpoint{5.055056in}{3.868593in}}%
\pgfpathlineto{\pgfqpoint{5.190465in}{3.761377in}}%
\pgfpathclose%
\pgfusepath{fill}%
\end{pgfscope}%
\begin{pgfscope}%
\pgfpathrectangle{\pgfqpoint{0.539299in}{0.078740in}}{\pgfqpoint{7.842520in}{7.842520in}}%
\pgfusepath{clip}%
\pgfsetbuttcap%
\pgfsetroundjoin%
\definecolor{currentfill}{rgb}{0.280255,0.165693,0.476498}%
\pgfsetfillcolor{currentfill}%
\pgfsetlinewidth{0.000000pt}%
\definecolor{currentstroke}{rgb}{0.165117,0.467423,0.558141}%
\pgfsetstrokecolor{currentstroke}%
\pgfsetdash{}{0pt}%
\pgfpathmoveto{\pgfqpoint{6.093026in}{3.466206in}}%
\pgfpathlineto{\pgfqpoint{6.020289in}{3.442779in}}%
\pgfpathlineto{\pgfqpoint{6.157532in}{3.392366in}}%
\pgfpathclose%
\pgfusepath{fill}%
\end{pgfscope}%
\begin{pgfscope}%
\pgfpathrectangle{\pgfqpoint{0.539299in}{0.078740in}}{\pgfqpoint{7.842520in}{7.842520in}}%
\pgfusepath{clip}%
\pgfsetbuttcap%
\pgfsetroundjoin%
\definecolor{currentfill}{rgb}{0.280267,0.073417,0.397163}%
\pgfsetfillcolor{currentfill}%
\pgfsetlinewidth{0.000000pt}%
\definecolor{currentstroke}{rgb}{0.163625,0.471133,0.558148}%
\pgfsetstrokecolor{currentstroke}%
\pgfsetdash{}{0pt}%
\pgfpathmoveto{\pgfqpoint{6.850286in}{3.187517in}}%
\pgfpathlineto{\pgfqpoint{6.987917in}{3.113178in}}%
\pgfpathlineto{\pgfqpoint{7.057617in}{3.142554in}}%
\pgfpathclose%
\pgfusepath{fill}%
\end{pgfscope}%
\begin{pgfscope}%
\pgfpathrectangle{\pgfqpoint{0.539299in}{0.078740in}}{\pgfqpoint{7.842520in}{7.842520in}}%
\pgfusepath{clip}%
\pgfsetbuttcap%
\pgfsetroundjoin%
\definecolor{currentfill}{rgb}{0.269944,0.014625,0.341379}%
\pgfsetfillcolor{currentfill}%
\pgfsetlinewidth{0.000000pt}%
\definecolor{currentstroke}{rgb}{0.162142,0.474838,0.558140}%
\pgfsetstrokecolor{currentstroke}%
\pgfsetdash{}{0pt}%
\pgfpathmoveto{\pgfqpoint{7.402794in}{3.055818in}}%
\pgfpathlineto{\pgfqpoint{7.333570in}{3.007542in}}%
\pgfpathlineto{\pgfqpoint{7.472249in}{2.944684in}}%
\pgfpathclose%
\pgfusepath{fill}%
\end{pgfscope}%
\begin{pgfscope}%
\pgfpathrectangle{\pgfqpoint{0.539299in}{0.078740in}}{\pgfqpoint{7.842520in}{7.842520in}}%
\pgfusepath{clip}%
\pgfsetbuttcap%
\pgfsetroundjoin%
\definecolor{currentfill}{rgb}{0.277018,0.050344,0.375715}%
\pgfsetfillcolor{currentfill}%
\pgfsetlinewidth{0.000000pt}%
\definecolor{currentstroke}{rgb}{0.160665,0.478540,0.558115}%
\pgfsetstrokecolor{currentstroke}%
\pgfsetdash{}{0pt}%
\pgfpathmoveto{\pgfqpoint{7.195388in}{3.074084in}}%
\pgfpathlineto{\pgfqpoint{7.057617in}{3.142554in}}%
\pgfpathlineto{\pgfqpoint{7.125818in}{3.037569in}}%
\pgfpathclose%
\pgfusepath{fill}%
\end{pgfscope}%
\begin{pgfscope}%
\pgfpathrectangle{\pgfqpoint{0.539299in}{0.078740in}}{\pgfqpoint{7.842520in}{7.842520in}}%
\pgfusepath{clip}%
\pgfsetbuttcap%
\pgfsetroundjoin%
\definecolor{currentfill}{rgb}{0.168126,0.459988,0.558082}%
\pgfsetfillcolor{currentfill}%
\pgfsetlinewidth{0.000000pt}%
\definecolor{currentstroke}{rgb}{0.159194,0.482237,0.558073}%
\pgfsetstrokecolor{currentstroke}%
\pgfsetdash{}{0pt}%
\pgfpathmoveto{\pgfqpoint{3.038103in}{4.744324in}}%
\pgfpathlineto{\pgfqpoint{2.996771in}{4.420700in}}%
\pgfpathlineto{\pgfqpoint{2.913780in}{4.395967in}}%
\pgfpathclose%
\pgfusepath{fill}%
\end{pgfscope}%
\begin{pgfscope}%
\pgfpathrectangle{\pgfqpoint{0.539299in}{0.078740in}}{\pgfqpoint{7.842520in}{7.842520in}}%
\pgfusepath{clip}%
\pgfsetbuttcap%
\pgfsetroundjoin%
\definecolor{currentfill}{rgb}{0.281887,0.150881,0.465405}%
\pgfsetfillcolor{currentfill}%
\pgfsetlinewidth{0.000000pt}%
\definecolor{currentstroke}{rgb}{0.157729,0.485932,0.558013}%
\pgfsetstrokecolor{currentstroke}%
\pgfsetdash{}{0pt}%
\pgfpathmoveto{\pgfqpoint{6.157532in}{3.392366in}}%
\pgfpathlineto{\pgfqpoint{6.367154in}{3.364036in}}%
\pgfpathlineto{\pgfqpoint{6.229961in}{3.418141in}}%
\pgfpathclose%
\pgfusepath{fill}%
\end{pgfscope}%
\begin{pgfscope}%
\pgfpathrectangle{\pgfqpoint{0.539299in}{0.078740in}}{\pgfqpoint{7.842520in}{7.842520in}}%
\pgfusepath{clip}%
\pgfsetbuttcap%
\pgfsetroundjoin%
\definecolor{currentfill}{rgb}{0.266580,0.228262,0.514349}%
\pgfsetfillcolor{currentfill}%
\pgfsetlinewidth{0.000000pt}%
\definecolor{currentstroke}{rgb}{0.156270,0.489624,0.557936}%
\pgfsetstrokecolor{currentstroke}%
\pgfsetdash{}{0pt}%
\pgfpathmoveto{\pgfqpoint{5.536826in}{3.584474in}}%
\pgfpathlineto{\pgfqpoint{5.400994in}{3.652853in}}%
\pgfpathlineto{\pgfqpoint{5.326196in}{3.667457in}}%
\pgfpathclose%
\pgfusepath{fill}%
\end{pgfscope}%
\begin{pgfscope}%
\pgfpathrectangle{\pgfqpoint{0.539299in}{0.078740in}}{\pgfqpoint{7.842520in}{7.842520in}}%
\pgfusepath{clip}%
\pgfsetbuttcap%
\pgfsetroundjoin%
\definecolor{currentfill}{rgb}{0.136408,0.541173,0.554483}%
\pgfsetfillcolor{currentfill}%
\pgfsetlinewidth{0.000000pt}%
\definecolor{currentstroke}{rgb}{0.154815,0.493313,0.557840}%
\pgfsetstrokecolor{currentstroke}%
\pgfsetdash{}{0pt}%
\pgfpathmoveto{\pgfqpoint{3.976815in}{4.900063in}}%
\pgfpathlineto{\pgfqpoint{4.030569in}{4.909838in}}%
\pgfpathlineto{\pgfqpoint{4.111002in}{4.812426in}}%
\pgfpathclose%
\pgfusepath{fill}%
\end{pgfscope}%
\begin{pgfscope}%
\pgfpathrectangle{\pgfqpoint{0.539299in}{0.078740in}}{\pgfqpoint{7.842520in}{7.842520in}}%
\pgfusepath{clip}%
\pgfsetbuttcap%
\pgfsetroundjoin%
\definecolor{currentfill}{rgb}{0.268510,0.009605,0.335427}%
\pgfsetfillcolor{currentfill}%
\pgfsetlinewidth{0.000000pt}%
\definecolor{currentstroke}{rgb}{0.153364,0.497000,0.557724}%
\pgfsetstrokecolor{currentstroke}%
\pgfsetdash{}{0pt}%
\pgfpathmoveto{\pgfqpoint{7.680757in}{2.951802in}}%
\pgfpathlineto{\pgfqpoint{7.541482in}{3.001589in}}%
\pgfpathlineto{\pgfqpoint{7.472249in}{2.944684in}}%
\pgfpathclose%
\pgfusepath{fill}%
\end{pgfscope}%
\begin{pgfscope}%
\pgfpathrectangle{\pgfqpoint{0.539299in}{0.078740in}}{\pgfqpoint{7.842520in}{7.842520in}}%
\pgfusepath{clip}%
\pgfsetbuttcap%
\pgfsetroundjoin%
\definecolor{currentfill}{rgb}{0.276194,0.190074,0.493001}%
\pgfsetfillcolor{currentfill}%
\pgfsetlinewidth{0.000000pt}%
\definecolor{currentstroke}{rgb}{0.151918,0.500685,0.557587}%
\pgfsetstrokecolor{currentstroke}%
\pgfsetdash{}{0pt}%
\pgfpathmoveto{\pgfqpoint{5.673111in}{3.524234in}}%
\pgfpathlineto{\pgfqpoint{5.809842in}{3.469040in}}%
\pgfpathlineto{\pgfqpoint{5.746902in}{3.537634in}}%
\pgfpathclose%
\pgfusepath{fill}%
\end{pgfscope}%
\begin{pgfscope}%
\pgfpathrectangle{\pgfqpoint{0.539299in}{0.078740in}}{\pgfqpoint{7.842520in}{7.842520in}}%
\pgfusepath{clip}%
\pgfsetbuttcap%
\pgfsetroundjoin%
\definecolor{currentfill}{rgb}{0.126453,0.570633,0.549841}%
\pgfsetfillcolor{currentfill}%
\pgfsetlinewidth{0.000000pt}%
\definecolor{currentstroke}{rgb}{0.150476,0.504369,0.557430}%
\pgfsetstrokecolor{currentstroke}%
\pgfsetdash{}{0pt}%
\pgfpathmoveto{\pgfqpoint{3.547026in}{5.048670in}}%
\pgfpathlineto{\pgfqpoint{3.629210in}{4.996070in}}%
\pgfpathlineto{\pgfqpoint{3.415921in}{4.934934in}}%
\pgfpathclose%
\pgfusepath{fill}%
\end{pgfscope}%
\begin{pgfscope}%
\pgfpathrectangle{\pgfqpoint{0.539299in}{0.078740in}}{\pgfqpoint{7.842520in}{7.842520in}}%
\pgfusepath{clip}%
\pgfsetbuttcap%
\pgfsetroundjoin%
\definecolor{currentfill}{rgb}{0.201239,0.383670,0.554294}%
\pgfsetfillcolor{currentfill}%
\pgfsetlinewidth{0.000000pt}%
\definecolor{currentstroke}{rgb}{0.149039,0.508051,0.557250}%
\pgfsetstrokecolor{currentstroke}%
\pgfsetdash{}{0pt}%
\pgfpathmoveto{\pgfqpoint{2.830306in}{4.368134in}}%
\pgfpathlineto{\pgfqpoint{2.710747in}{3.908413in}}%
\pgfpathlineto{\pgfqpoint{2.746363in}{4.336760in}}%
\pgfpathclose%
\pgfusepath{fill}%
\end{pgfscope}%
\begin{pgfscope}%
\pgfpathrectangle{\pgfqpoint{0.539299in}{0.078740in}}{\pgfqpoint{7.842520in}{7.842520in}}%
\pgfusepath{clip}%
\pgfsetbuttcap%
\pgfsetroundjoin%
\definecolor{currentfill}{rgb}{0.282910,0.105393,0.426902}%
\pgfsetfillcolor{currentfill}%
\pgfsetlinewidth{0.000000pt}%
\definecolor{currentstroke}{rgb}{0.147607,0.511733,0.557049}%
\pgfsetstrokecolor{currentstroke}%
\pgfsetdash{}{0pt}%
\pgfpathmoveto{\pgfqpoint{6.712873in}{3.258583in}}%
\pgfpathlineto{\pgfqpoint{6.642120in}{3.235328in}}%
\pgfpathlineto{\pgfqpoint{6.850286in}{3.187517in}}%
\pgfpathclose%
\pgfusepath{fill}%
\end{pgfscope}%
\begin{pgfscope}%
\pgfpathrectangle{\pgfqpoint{0.539299in}{0.078740in}}{\pgfqpoint{7.842520in}{7.842520in}}%
\pgfusepath{clip}%
\pgfsetbuttcap%
\pgfsetroundjoin%
\definecolor{currentfill}{rgb}{0.278791,0.062145,0.386592}%
\pgfsetfillcolor{currentfill}%
\pgfsetlinewidth{0.000000pt}%
\definecolor{currentstroke}{rgb}{0.146180,0.515413,0.556823}%
\pgfsetstrokecolor{currentstroke}%
\pgfsetdash{}{0pt}%
\pgfpathmoveto{\pgfqpoint{7.125818in}{3.037569in}}%
\pgfpathlineto{\pgfqpoint{7.057617in}{3.142554in}}%
\pgfpathlineto{\pgfqpoint{6.987917in}{3.113178in}}%
\pgfpathclose%
\pgfusepath{fill}%
\end{pgfscope}%
\begin{pgfscope}%
\pgfpathrectangle{\pgfqpoint{0.539299in}{0.078740in}}{\pgfqpoint{7.842520in}{7.842520in}}%
\pgfusepath{clip}%
\pgfsetbuttcap%
\pgfsetroundjoin%
\definecolor{currentfill}{rgb}{0.127568,0.566949,0.550556}%
\pgfsetfillcolor{currentfill}%
\pgfsetlinewidth{0.000000pt}%
\definecolor{currentstroke}{rgb}{0.144759,0.519093,0.556572}%
\pgfsetstrokecolor{currentstroke}%
\pgfsetdash{}{0pt}%
\pgfpathmoveto{\pgfqpoint{3.761962in}{5.020676in}}%
\pgfpathlineto{\pgfqpoint{3.895878in}{4.988126in}}%
\pgfpathlineto{\pgfqpoint{3.976815in}{4.900063in}}%
\pgfpathclose%
\pgfusepath{fill}%
\end{pgfscope}%
\begin{pgfscope}%
\pgfpathrectangle{\pgfqpoint{0.539299in}{0.078740in}}{\pgfqpoint{7.842520in}{7.842520in}}%
\pgfusepath{clip}%
\pgfsetbuttcap%
\pgfsetroundjoin%
\definecolor{currentfill}{rgb}{0.140536,0.530132,0.555659}%
\pgfsetfillcolor{currentfill}%
\pgfsetlinewidth{0.000000pt}%
\definecolor{currentstroke}{rgb}{0.143343,0.522773,0.556295}%
\pgfsetstrokecolor{currentstroke}%
\pgfsetdash{}{0pt}%
\pgfpathmoveto{\pgfqpoint{4.245563in}{4.694927in}}%
\pgfpathlineto{\pgfqpoint{4.111002in}{4.812426in}}%
\pgfpathlineto{\pgfqpoint{4.030569in}{4.909838in}}%
\pgfpathclose%
\pgfusepath{fill}%
\end{pgfscope}%
\begin{pgfscope}%
\pgfpathrectangle{\pgfqpoint{0.539299in}{0.078740in}}{\pgfqpoint{7.842520in}{7.842520in}}%
\pgfusepath{clip}%
\pgfsetbuttcap%
\pgfsetroundjoin%
\definecolor{currentfill}{rgb}{0.274952,0.037752,0.364543}%
\pgfsetfillcolor{currentfill}%
\pgfsetlinewidth{0.000000pt}%
\definecolor{currentstroke}{rgb}{0.141935,0.526453,0.555991}%
\pgfsetstrokecolor{currentstroke}%
\pgfsetdash{}{0pt}%
\pgfpathmoveto{\pgfqpoint{7.125818in}{3.037569in}}%
\pgfpathlineto{\pgfqpoint{7.333570in}{3.007542in}}%
\pgfpathlineto{\pgfqpoint{7.195388in}{3.074084in}}%
\pgfpathclose%
\pgfusepath{fill}%
\end{pgfscope}%
\begin{pgfscope}%
\pgfpathrectangle{\pgfqpoint{0.539299in}{0.078740in}}{\pgfqpoint{7.842520in}{7.842520in}}%
\pgfusepath{clip}%
\pgfsetbuttcap%
\pgfsetroundjoin%
\definecolor{currentfill}{rgb}{0.258965,0.251537,0.524736}%
\pgfsetfillcolor{currentfill}%
\pgfsetlinewidth{0.000000pt}%
\definecolor{currentstroke}{rgb}{0.140536,0.530132,0.555659}%
\pgfsetstrokecolor{currentstroke}%
\pgfsetdash{}{0pt}%
\pgfpathmoveto{\pgfqpoint{5.190465in}{3.761377in}}%
\pgfpathlineto{\pgfqpoint{5.326196in}{3.667457in}}%
\pgfpathlineto{\pgfqpoint{5.400994in}{3.652853in}}%
\pgfpathclose%
\pgfusepath{fill}%
\end{pgfscope}%
\begin{pgfscope}%
\pgfpathrectangle{\pgfqpoint{0.539299in}{0.078740in}}{\pgfqpoint{7.842520in}{7.842520in}}%
\pgfusepath{clip}%
\pgfsetbuttcap%
\pgfsetroundjoin%
\definecolor{currentfill}{rgb}{0.283229,0.120777,0.440584}%
\pgfsetfillcolor{currentfill}%
\pgfsetlinewidth{0.000000pt}%
\definecolor{currentstroke}{rgb}{0.139147,0.533812,0.555298}%
\pgfsetstrokecolor{currentstroke}%
\pgfsetdash{}{0pt}%
\pgfpathmoveto{\pgfqpoint{6.504552in}{3.303089in}}%
\pgfpathlineto{\pgfqpoint{6.642120in}{3.235328in}}%
\pgfpathlineto{\pgfqpoint{6.712873in}{3.258583in}}%
\pgfpathclose%
\pgfusepath{fill}%
\end{pgfscope}%
\begin{pgfscope}%
\pgfpathrectangle{\pgfqpoint{0.539299in}{0.078740in}}{\pgfqpoint{7.842520in}{7.842520in}}%
\pgfusepath{clip}%
\pgfsetbuttcap%
\pgfsetroundjoin%
\definecolor{currentfill}{rgb}{0.278012,0.180367,0.486697}%
\pgfsetfillcolor{currentfill}%
\pgfsetlinewidth{0.000000pt}%
\definecolor{currentstroke}{rgb}{0.137770,0.537492,0.554906}%
\pgfsetstrokecolor{currentstroke}%
\pgfsetdash{}{0pt}%
\pgfpathmoveto{\pgfqpoint{5.946992in}{3.415751in}}%
\pgfpathlineto{\pgfqpoint{6.020289in}{3.442779in}}%
\pgfpathlineto{\pgfqpoint{5.883392in}{3.490293in}}%
\pgfpathclose%
\pgfusepath{fill}%
\end{pgfscope}%
\begin{pgfscope}%
\pgfpathrectangle{\pgfqpoint{0.539299in}{0.078740in}}{\pgfqpoint{7.842520in}{7.842520in}}%
\pgfusepath{clip}%
\pgfsetbuttcap%
\pgfsetroundjoin%
\definecolor{currentfill}{rgb}{0.154815,0.493313,0.557840}%
\pgfsetfillcolor{currentfill}%
\pgfsetlinewidth{0.000000pt}%
\definecolor{currentstroke}{rgb}{0.136408,0.541173,0.554483}%
\pgfsetstrokecolor{currentstroke}%
\pgfsetdash{}{0pt}%
\pgfpathmoveto{\pgfqpoint{3.121655in}{4.747134in}}%
\pgfpathlineto{\pgfqpoint{2.996771in}{4.420700in}}%
\pgfpathlineto{\pgfqpoint{3.038103in}{4.744324in}}%
\pgfpathclose%
\pgfusepath{fill}%
\end{pgfscope}%
\begin{pgfscope}%
\pgfpathrectangle{\pgfqpoint{0.539299in}{0.078740in}}{\pgfqpoint{7.842520in}{7.842520in}}%
\pgfusepath{clip}%
\pgfsetbuttcap%
\pgfsetroundjoin%
\definecolor{currentfill}{rgb}{0.150476,0.504369,0.557430}%
\pgfsetfillcolor{currentfill}%
\pgfsetlinewidth{0.000000pt}%
\definecolor{currentstroke}{rgb}{0.135066,0.544853,0.554029}%
\pgfsetstrokecolor{currentstroke}%
\pgfsetdash{}{0pt}%
\pgfpathmoveto{\pgfqpoint{4.380311in}{4.558363in}}%
\pgfpathlineto{\pgfqpoint{4.245563in}{4.694927in}}%
\pgfpathlineto{\pgfqpoint{4.165728in}{4.797221in}}%
\pgfpathclose%
\pgfusepath{fill}%
\end{pgfscope}%
\begin{pgfscope}%
\pgfpathrectangle{\pgfqpoint{0.539299in}{0.078740in}}{\pgfqpoint{7.842520in}{7.842520in}}%
\pgfusepath{clip}%
\pgfsetbuttcap%
\pgfsetroundjoin%
\definecolor{currentfill}{rgb}{0.139147,0.533812,0.555298}%
\pgfsetfillcolor{currentfill}%
\pgfsetlinewidth{0.000000pt}%
\definecolor{currentstroke}{rgb}{0.133743,0.548535,0.553541}%
\pgfsetstrokecolor{currentstroke}%
\pgfsetdash{}{0pt}%
\pgfpathmoveto{\pgfqpoint{3.333165in}{4.962328in}}%
\pgfpathlineto{\pgfqpoint{3.204675in}{4.746147in}}%
\pgfpathlineto{\pgfqpoint{3.121655in}{4.747134in}}%
\pgfpathclose%
\pgfusepath{fill}%
\end{pgfscope}%
\begin{pgfscope}%
\pgfpathrectangle{\pgfqpoint{0.539299in}{0.078740in}}{\pgfqpoint{7.842520in}{7.842520in}}%
\pgfusepath{clip}%
\pgfsetbuttcap%
\pgfsetroundjoin%
\definecolor{currentfill}{rgb}{0.168126,0.459988,0.558082}%
\pgfsetfillcolor{currentfill}%
\pgfsetlinewidth{0.000000pt}%
\definecolor{currentstroke}{rgb}{0.132444,0.552216,0.553018}%
\pgfsetstrokecolor{currentstroke}%
\pgfsetdash{}{0pt}%
\pgfpathmoveto{\pgfqpoint{4.380311in}{4.558363in}}%
\pgfpathlineto{\pgfqpoint{4.436631in}{4.511408in}}%
\pgfpathlineto{\pgfqpoint{4.515135in}{4.412321in}}%
\pgfpathclose%
\pgfusepath{fill}%
\end{pgfscope}%
\begin{pgfscope}%
\pgfpathrectangle{\pgfqpoint{0.539299in}{0.078740in}}{\pgfqpoint{7.842520in}{7.842520in}}%
\pgfusepath{clip}%
\pgfsetbuttcap%
\pgfsetroundjoin%
\definecolor{currentfill}{rgb}{0.282290,0.145912,0.461510}%
\pgfsetfillcolor{currentfill}%
\pgfsetlinewidth{0.000000pt}%
\definecolor{currentstroke}{rgb}{0.131172,0.555899,0.552459}%
\pgfsetstrokecolor{currentstroke}%
\pgfsetdash{}{0pt}%
\pgfpathmoveto{\pgfqpoint{6.504552in}{3.303089in}}%
\pgfpathlineto{\pgfqpoint{6.367154in}{3.364036in}}%
\pgfpathlineto{\pgfqpoint{6.295060in}{3.336953in}}%
\pgfpathclose%
\pgfusepath{fill}%
\end{pgfscope}%
\begin{pgfscope}%
\pgfpathrectangle{\pgfqpoint{0.539299in}{0.078740in}}{\pgfqpoint{7.842520in}{7.842520in}}%
\pgfusepath{clip}%
\pgfsetbuttcap%
\pgfsetroundjoin%
\definecolor{currentfill}{rgb}{0.177423,0.437527,0.557565}%
\pgfsetfillcolor{currentfill}%
\pgfsetlinewidth{0.000000pt}%
\definecolor{currentstroke}{rgb}{0.129933,0.559582,0.551864}%
\pgfsetstrokecolor{currentstroke}%
\pgfsetdash{}{0pt}%
\pgfpathmoveto{\pgfqpoint{4.515135in}{4.412321in}}%
\pgfpathlineto{\pgfqpoint{4.436631in}{4.511408in}}%
\pgfpathlineto{\pgfqpoint{4.649993in}{4.264866in}}%
\pgfpathclose%
\pgfusepath{fill}%
\end{pgfscope}%
\begin{pgfscope}%
\pgfpathrectangle{\pgfqpoint{0.539299in}{0.078740in}}{\pgfqpoint{7.842520in}{7.842520in}}%
\pgfusepath{clip}%
\pgfsetbuttcap%
\pgfsetroundjoin%
\definecolor{currentfill}{rgb}{0.128729,0.563265,0.551229}%
\pgfsetfillcolor{currentfill}%
\pgfsetlinewidth{0.000000pt}%
\definecolor{currentstroke}{rgb}{0.128729,0.563265,0.551229}%
\pgfsetstrokecolor{currentstroke}%
\pgfsetdash{}{0pt}%
\pgfpathmoveto{\pgfqpoint{3.976815in}{4.900063in}}%
\pgfpathlineto{\pgfqpoint{3.895878in}{4.988126in}}%
\pgfpathlineto{\pgfqpoint{4.030569in}{4.909838in}}%
\pgfpathclose%
\pgfusepath{fill}%
\end{pgfscope}%
\begin{pgfscope}%
\pgfpathrectangle{\pgfqpoint{0.539299in}{0.078740in}}{\pgfqpoint{7.842520in}{7.842520in}}%
\pgfusepath{clip}%
\pgfsetbuttcap%
\pgfsetroundjoin%
\definecolor{currentfill}{rgb}{0.197636,0.391528,0.554969}%
\pgfsetfillcolor{currentfill}%
\pgfsetlinewidth{0.000000pt}%
\definecolor{currentstroke}{rgb}{0.127568,0.566949,0.550556}%
\pgfsetstrokecolor{currentstroke}%
\pgfsetdash{}{0pt}%
\pgfpathmoveto{\pgfqpoint{4.784895in}{4.122344in}}%
\pgfpathlineto{\pgfqpoint{4.649993in}{4.264866in}}%
\pgfpathlineto{\pgfqpoint{4.707691in}{4.203491in}}%
\pgfpathclose%
\pgfusepath{fill}%
\end{pgfscope}%
\begin{pgfscope}%
\pgfpathrectangle{\pgfqpoint{0.539299in}{0.078740in}}{\pgfqpoint{7.842520in}{7.842520in}}%
\pgfusepath{clip}%
\pgfsetbuttcap%
\pgfsetroundjoin%
\definecolor{currentfill}{rgb}{0.223925,0.334994,0.548053}%
\pgfsetfillcolor{currentfill}%
\pgfsetlinewidth{0.000000pt}%
\definecolor{currentstroke}{rgb}{0.126453,0.570633,0.549841}%
\pgfsetstrokecolor{currentstroke}%
\pgfsetdash{}{0pt}%
\pgfpathmoveto{\pgfqpoint{2.661974in}{4.301274in}}%
\pgfpathlineto{\pgfqpoint{2.627589in}{3.858151in}}%
\pgfpathlineto{\pgfqpoint{2.543987in}{3.805632in}}%
\pgfpathclose%
\pgfusepath{fill}%
\end{pgfscope}%
\begin{pgfscope}%
\pgfpathrectangle{\pgfqpoint{0.539299in}{0.078740in}}{\pgfqpoint{7.842520in}{7.842520in}}%
\pgfusepath{clip}%
\pgfsetbuttcap%
\pgfsetroundjoin%
\definecolor{currentfill}{rgb}{0.281412,0.155834,0.469201}%
\pgfsetfillcolor{currentfill}%
\pgfsetlinewidth{0.000000pt}%
\definecolor{currentstroke}{rgb}{0.125394,0.574318,0.549086}%
\pgfsetstrokecolor{currentstroke}%
\pgfsetdash{}{0pt}%
\pgfpathmoveto{\pgfqpoint{6.295060in}{3.336953in}}%
\pgfpathlineto{\pgfqpoint{6.367154in}{3.364036in}}%
\pgfpathlineto{\pgfqpoint{6.157532in}{3.392366in}}%
\pgfpathclose%
\pgfusepath{fill}%
\end{pgfscope}%
\begin{pgfscope}%
\pgfpathrectangle{\pgfqpoint{0.539299in}{0.078740in}}{\pgfqpoint{7.842520in}{7.842520in}}%
\pgfusepath{clip}%
\pgfsetbuttcap%
\pgfsetroundjoin%
\definecolor{currentfill}{rgb}{0.277134,0.185228,0.489898}%
\pgfsetfillcolor{currentfill}%
\pgfsetlinewidth{0.000000pt}%
\definecolor{currentstroke}{rgb}{0.124395,0.578002,0.548287}%
\pgfsetstrokecolor{currentstroke}%
\pgfsetdash{}{0pt}%
\pgfpathmoveto{\pgfqpoint{5.946992in}{3.415751in}}%
\pgfpathlineto{\pgfqpoint{5.883392in}{3.490293in}}%
\pgfpathlineto{\pgfqpoint{5.809842in}{3.469040in}}%
\pgfpathclose%
\pgfusepath{fill}%
\end{pgfscope}%
\begin{pgfscope}%
\pgfpathrectangle{\pgfqpoint{0.539299in}{0.078740in}}{\pgfqpoint{7.842520in}{7.842520in}}%
\pgfusepath{clip}%
\pgfsetbuttcap%
\pgfsetroundjoin%
\definecolor{currentfill}{rgb}{0.268510,0.009605,0.335427}%
\pgfsetfillcolor{currentfill}%
\pgfsetlinewidth{0.000000pt}%
\definecolor{currentstroke}{rgb}{0.123463,0.581687,0.547445}%
\pgfsetstrokecolor{currentstroke}%
\pgfsetdash{}{0pt}%
\pgfpathmoveto{\pgfqpoint{7.611514in}{2.887087in}}%
\pgfpathlineto{\pgfqpoint{7.680757in}{2.951802in}}%
\pgfpathlineto{\pgfqpoint{7.472249in}{2.944684in}}%
\pgfpathclose%
\pgfusepath{fill}%
\end{pgfscope}%
\begin{pgfscope}%
\pgfpathrectangle{\pgfqpoint{0.539299in}{0.078740in}}{\pgfqpoint{7.842520in}{7.842520in}}%
\pgfusepath{clip}%
\pgfsetbuttcap%
\pgfsetroundjoin%
\definecolor{currentfill}{rgb}{0.281924,0.089666,0.412415}%
\pgfsetfillcolor{currentfill}%
\pgfsetlinewidth{0.000000pt}%
\definecolor{currentstroke}{rgb}{0.122606,0.585371,0.546557}%
\pgfsetstrokecolor{currentstroke}%
\pgfsetdash{}{0pt}%
\pgfpathmoveto{\pgfqpoint{6.987917in}{3.113178in}}%
\pgfpathlineto{\pgfqpoint{6.850286in}{3.187517in}}%
\pgfpathlineto{\pgfqpoint{6.779843in}{3.161531in}}%
\pgfpathclose%
\pgfusepath{fill}%
\end{pgfscope}%
\begin{pgfscope}%
\pgfpathrectangle{\pgfqpoint{0.539299in}{0.078740in}}{\pgfqpoint{7.842520in}{7.842520in}}%
\pgfusepath{clip}%
\pgfsetbuttcap%
\pgfsetroundjoin%
\definecolor{currentfill}{rgb}{0.271828,0.209303,0.504434}%
\pgfsetfillcolor{currentfill}%
\pgfsetlinewidth{0.000000pt}%
\definecolor{currentstroke}{rgb}{0.121831,0.589055,0.545623}%
\pgfsetstrokecolor{currentstroke}%
\pgfsetdash{}{0pt}%
\pgfpathmoveto{\pgfqpoint{5.536826in}{3.584474in}}%
\pgfpathlineto{\pgfqpoint{5.598833in}{3.513257in}}%
\pgfpathlineto{\pgfqpoint{5.673111in}{3.524234in}}%
\pgfpathclose%
\pgfusepath{fill}%
\end{pgfscope}%
\begin{pgfscope}%
\pgfpathrectangle{\pgfqpoint{0.539299in}{0.078740in}}{\pgfqpoint{7.842520in}{7.842520in}}%
\pgfusepath{clip}%
\pgfsetbuttcap%
\pgfsetroundjoin%
\definecolor{currentfill}{rgb}{0.208623,0.367752,0.552675}%
\pgfsetfillcolor{currentfill}%
\pgfsetlinewidth{0.000000pt}%
\definecolor{currentstroke}{rgb}{0.121148,0.592739,0.544641}%
\pgfsetstrokecolor{currentstroke}%
\pgfsetdash{}{0pt}%
\pgfpathmoveto{\pgfqpoint{4.707691in}{4.203491in}}%
\pgfpathlineto{\pgfqpoint{4.919892in}{3.989315in}}%
\pgfpathlineto{\pgfqpoint{4.784895in}{4.122344in}}%
\pgfpathclose%
\pgfusepath{fill}%
\end{pgfscope}%
\begin{pgfscope}%
\pgfpathrectangle{\pgfqpoint{0.539299in}{0.078740in}}{\pgfqpoint{7.842520in}{7.842520in}}%
\pgfusepath{clip}%
\pgfsetbuttcap%
\pgfsetroundjoin%
\definecolor{currentfill}{rgb}{0.265145,0.232956,0.516599}%
\pgfsetfillcolor{currentfill}%
\pgfsetlinewidth{0.000000pt}%
\definecolor{currentstroke}{rgb}{0.120565,0.596422,0.543611}%
\pgfsetstrokecolor{currentstroke}%
\pgfsetdash{}{0pt}%
\pgfpathmoveto{\pgfqpoint{5.326196in}{3.667457in}}%
\pgfpathlineto{\pgfqpoint{5.462306in}{3.585484in}}%
\pgfpathlineto{\pgfqpoint{5.536826in}{3.584474in}}%
\pgfpathclose%
\pgfusepath{fill}%
\end{pgfscope}%
\begin{pgfscope}%
\pgfpathrectangle{\pgfqpoint{0.539299in}{0.078740in}}{\pgfqpoint{7.842520in}{7.842520in}}%
\pgfusepath{clip}%
\pgfsetbuttcap%
\pgfsetroundjoin%
\definecolor{currentfill}{rgb}{0.121831,0.589055,0.545623}%
\pgfsetfillcolor{currentfill}%
\pgfsetlinewidth{0.000000pt}%
\definecolor{currentstroke}{rgb}{0.120092,0.600104,0.542530}%
\pgfsetstrokecolor{currentstroke}%
\pgfsetdash{}{0pt}%
\pgfpathmoveto{\pgfqpoint{3.679940in}{5.091602in}}%
\pgfpathlineto{\pgfqpoint{3.761962in}{5.020676in}}%
\pgfpathlineto{\pgfqpoint{3.629210in}{4.996070in}}%
\pgfpathclose%
\pgfusepath{fill}%
\end{pgfscope}%
\begin{pgfscope}%
\pgfpathrectangle{\pgfqpoint{0.539299in}{0.078740in}}{\pgfqpoint{7.842520in}{7.842520in}}%
\pgfusepath{clip}%
\pgfsetbuttcap%
\pgfsetroundjoin%
\definecolor{currentfill}{rgb}{0.272594,0.025563,0.353093}%
\pgfsetfillcolor{currentfill}%
\pgfsetlinewidth{0.000000pt}%
\definecolor{currentstroke}{rgb}{0.119738,0.603785,0.541400}%
\pgfsetstrokecolor{currentstroke}%
\pgfsetdash{}{0pt}%
\pgfpathmoveto{\pgfqpoint{7.472249in}{2.944684in}}%
\pgfpathlineto{\pgfqpoint{7.333570in}{3.007542in}}%
\pgfpathlineto{\pgfqpoint{7.264065in}{2.962827in}}%
\pgfpathclose%
\pgfusepath{fill}%
\end{pgfscope}%
\begin{pgfscope}%
\pgfpathrectangle{\pgfqpoint{0.539299in}{0.078740in}}{\pgfqpoint{7.842520in}{7.842520in}}%
\pgfusepath{clip}%
\pgfsetbuttcap%
\pgfsetroundjoin%
\definecolor{currentfill}{rgb}{0.124395,0.578002,0.548287}%
\pgfsetfillcolor{currentfill}%
\pgfsetlinewidth{0.000000pt}%
\definecolor{currentstroke}{rgb}{0.119512,0.607464,0.540218}%
\pgfsetstrokecolor{currentstroke}%
\pgfsetdash{}{0pt}%
\pgfpathmoveto{\pgfqpoint{3.547026in}{5.048670in}}%
\pgfpathlineto{\pgfqpoint{3.415921in}{4.934934in}}%
\pgfpathlineto{\pgfqpoint{3.333165in}{4.962328in}}%
\pgfpathclose%
\pgfusepath{fill}%
\end{pgfscope}%
\begin{pgfscope}%
\pgfpathrectangle{\pgfqpoint{0.539299in}{0.078740in}}{\pgfqpoint{7.842520in}{7.842520in}}%
\pgfusepath{clip}%
\pgfsetbuttcap%
\pgfsetroundjoin%
\definecolor{currentfill}{rgb}{0.278826,0.175490,0.483397}%
\pgfsetfillcolor{currentfill}%
\pgfsetlinewidth{0.000000pt}%
\definecolor{currentstroke}{rgb}{0.119423,0.611141,0.538982}%
\pgfsetstrokecolor{currentstroke}%
\pgfsetdash{}{0pt}%
\pgfpathmoveto{\pgfqpoint{5.946992in}{3.415751in}}%
\pgfpathlineto{\pgfqpoint{6.157532in}{3.392366in}}%
\pgfpathlineto{\pgfqpoint{6.020289in}{3.442779in}}%
\pgfpathclose%
\pgfusepath{fill}%
\end{pgfscope}%
\begin{pgfscope}%
\pgfpathrectangle{\pgfqpoint{0.539299in}{0.078740in}}{\pgfqpoint{7.842520in}{7.842520in}}%
\pgfusepath{clip}%
\pgfsetbuttcap%
\pgfsetroundjoin%
\definecolor{currentfill}{rgb}{0.282910,0.105393,0.426902}%
\pgfsetfillcolor{currentfill}%
\pgfsetlinewidth{0.000000pt}%
\definecolor{currentstroke}{rgb}{0.119483,0.614817,0.537692}%
\pgfsetstrokecolor{currentstroke}%
\pgfsetdash{}{0pt}%
\pgfpathmoveto{\pgfqpoint{6.850286in}{3.187517in}}%
\pgfpathlineto{\pgfqpoint{6.642120in}{3.235328in}}%
\pgfpathlineto{\pgfqpoint{6.779843in}{3.161531in}}%
\pgfpathclose%
\pgfusepath{fill}%
\end{pgfscope}%
\begin{pgfscope}%
\pgfpathrectangle{\pgfqpoint{0.539299in}{0.078740in}}{\pgfqpoint{7.842520in}{7.842520in}}%
\pgfusepath{clip}%
\pgfsetbuttcap%
\pgfsetroundjoin%
\definecolor{currentfill}{rgb}{0.274952,0.037752,0.364543}%
\pgfsetfillcolor{currentfill}%
\pgfsetlinewidth{0.000000pt}%
\definecolor{currentstroke}{rgb}{0.119699,0.618490,0.536347}%
\pgfsetstrokecolor{currentstroke}%
\pgfsetdash{}{0pt}%
\pgfpathmoveto{\pgfqpoint{7.264065in}{2.962827in}}%
\pgfpathlineto{\pgfqpoint{7.333570in}{3.007542in}}%
\pgfpathlineto{\pgfqpoint{7.125818in}{3.037569in}}%
\pgfpathclose%
\pgfusepath{fill}%
\end{pgfscope}%
\begin{pgfscope}%
\pgfpathrectangle{\pgfqpoint{0.539299in}{0.078740in}}{\pgfqpoint{7.842520in}{7.842520in}}%
\pgfusepath{clip}%
\pgfsetbuttcap%
\pgfsetroundjoin%
\definecolor{currentfill}{rgb}{0.229739,0.322361,0.545706}%
\pgfsetfillcolor{currentfill}%
\pgfsetlinewidth{0.000000pt}%
\definecolor{currentstroke}{rgb}{0.120081,0.622161,0.534946}%
\pgfsetstrokecolor{currentstroke}%
\pgfsetdash{}{0pt}%
\pgfpathmoveto{\pgfqpoint{5.055056in}{3.868593in}}%
\pgfpathlineto{\pgfqpoint{4.919892in}{3.989315in}}%
\pgfpathlineto{\pgfqpoint{4.978938in}{3.922091in}}%
\pgfpathclose%
\pgfusepath{fill}%
\end{pgfscope}%
\begin{pgfscope}%
\pgfpathrectangle{\pgfqpoint{0.539299in}{0.078740in}}{\pgfqpoint{7.842520in}{7.842520in}}%
\pgfusepath{clip}%
\pgfsetbuttcap%
\pgfsetroundjoin%
\definecolor{currentfill}{rgb}{0.121148,0.592739,0.544641}%
\pgfsetfillcolor{currentfill}%
\pgfsetlinewidth{0.000000pt}%
\definecolor{currentstroke}{rgb}{0.120638,0.625828,0.533488}%
\pgfsetstrokecolor{currentstroke}%
\pgfsetdash{}{0pt}%
\pgfpathmoveto{\pgfqpoint{3.679940in}{5.091602in}}%
\pgfpathlineto{\pgfqpoint{3.629210in}{4.996070in}}%
\pgfpathlineto{\pgfqpoint{3.547026in}{5.048670in}}%
\pgfpathclose%
\pgfusepath{fill}%
\end{pgfscope}%
\begin{pgfscope}%
\pgfpathrectangle{\pgfqpoint{0.539299in}{0.078740in}}{\pgfqpoint{7.842520in}{7.842520in}}%
\pgfusepath{clip}%
\pgfsetbuttcap%
\pgfsetroundjoin%
\definecolor{currentfill}{rgb}{0.267968,0.223549,0.512008}%
\pgfsetfillcolor{currentfill}%
\pgfsetlinewidth{0.000000pt}%
\definecolor{currentstroke}{rgb}{0.121380,0.629492,0.531973}%
\pgfsetstrokecolor{currentstroke}%
\pgfsetdash{}{0pt}%
\pgfpathmoveto{\pgfqpoint{5.462306in}{3.585484in}}%
\pgfpathlineto{\pgfqpoint{5.598833in}{3.513257in}}%
\pgfpathlineto{\pgfqpoint{5.536826in}{3.584474in}}%
\pgfpathclose%
\pgfusepath{fill}%
\end{pgfscope}%
\begin{pgfscope}%
\pgfpathrectangle{\pgfqpoint{0.539299in}{0.078740in}}{\pgfqpoint{7.842520in}{7.842520in}}%
\pgfusepath{clip}%
\pgfsetbuttcap%
\pgfsetroundjoin%
\definecolor{currentfill}{rgb}{0.137770,0.537492,0.554906}%
\pgfsetfillcolor{currentfill}%
\pgfsetlinewidth{0.000000pt}%
\definecolor{currentstroke}{rgb}{0.122312,0.633153,0.530398}%
\pgfsetstrokecolor{currentstroke}%
\pgfsetdash{}{0pt}%
\pgfpathmoveto{\pgfqpoint{4.165728in}{4.797221in}}%
\pgfpathlineto{\pgfqpoint{4.245563in}{4.694927in}}%
\pgfpathlineto{\pgfqpoint{4.030569in}{4.909838in}}%
\pgfpathclose%
\pgfusepath{fill}%
\end{pgfscope}%
\begin{pgfscope}%
\pgfpathrectangle{\pgfqpoint{0.539299in}{0.078740in}}{\pgfqpoint{7.842520in}{7.842520in}}%
\pgfusepath{clip}%
\pgfsetbuttcap%
\pgfsetroundjoin%
\definecolor{currentfill}{rgb}{0.241237,0.296485,0.539709}%
\pgfsetfillcolor{currentfill}%
\pgfsetlinewidth{0.000000pt}%
\definecolor{currentstroke}{rgb}{0.123444,0.636809,0.528763}%
\pgfsetstrokecolor{currentstroke}%
\pgfsetdash{}{0pt}%
\pgfpathmoveto{\pgfqpoint{5.190465in}{3.761377in}}%
\pgfpathlineto{\pgfqpoint{5.055056in}{3.868593in}}%
\pgfpathlineto{\pgfqpoint{5.114785in}{3.799326in}}%
\pgfpathclose%
\pgfusepath{fill}%
\end{pgfscope}%
\begin{pgfscope}%
\pgfpathrectangle{\pgfqpoint{0.539299in}{0.078740in}}{\pgfqpoint{7.842520in}{7.842520in}}%
\pgfusepath{clip}%
\pgfsetbuttcap%
\pgfsetroundjoin%
\definecolor{currentfill}{rgb}{0.280267,0.073417,0.397163}%
\pgfsetfillcolor{currentfill}%
\pgfsetlinewidth{0.000000pt}%
\definecolor{currentstroke}{rgb}{0.124780,0.640461,0.527068}%
\pgfsetstrokecolor{currentstroke}%
\pgfsetdash{}{0pt}%
\pgfpathmoveto{\pgfqpoint{7.125818in}{3.037569in}}%
\pgfpathlineto{\pgfqpoint{6.987917in}{3.113178in}}%
\pgfpathlineto{\pgfqpoint{6.917735in}{3.083093in}}%
\pgfpathclose%
\pgfusepath{fill}%
\end{pgfscope}%
\begin{pgfscope}%
\pgfpathrectangle{\pgfqpoint{0.539299in}{0.078740in}}{\pgfqpoint{7.842520in}{7.842520in}}%
\pgfusepath{clip}%
\pgfsetbuttcap%
\pgfsetroundjoin%
\definecolor{currentfill}{rgb}{0.147607,0.511733,0.557049}%
\pgfsetfillcolor{currentfill}%
\pgfsetlinewidth{0.000000pt}%
\definecolor{currentstroke}{rgb}{0.126326,0.644107,0.525311}%
\pgfsetstrokecolor{currentstroke}%
\pgfsetdash{}{0pt}%
\pgfpathmoveto{\pgfqpoint{4.165728in}{4.797221in}}%
\pgfpathlineto{\pgfqpoint{4.301131in}{4.661130in}}%
\pgfpathlineto{\pgfqpoint{4.380311in}{4.558363in}}%
\pgfpathclose%
\pgfusepath{fill}%
\end{pgfscope}%
\begin{pgfscope}%
\pgfpathrectangle{\pgfqpoint{0.539299in}{0.078740in}}{\pgfqpoint{7.842520in}{7.842520in}}%
\pgfusepath{clip}%
\pgfsetbuttcap%
\pgfsetroundjoin%
\definecolor{currentfill}{rgb}{0.281887,0.150881,0.465405}%
\pgfsetfillcolor{currentfill}%
\pgfsetlinewidth{0.000000pt}%
\definecolor{currentstroke}{rgb}{0.128087,0.647749,0.523491}%
\pgfsetstrokecolor{currentstroke}%
\pgfsetdash{}{0pt}%
\pgfpathmoveto{\pgfqpoint{6.504552in}{3.303089in}}%
\pgfpathlineto{\pgfqpoint{6.295060in}{3.336953in}}%
\pgfpathlineto{\pgfqpoint{6.432810in}{3.275195in}}%
\pgfpathclose%
\pgfusepath{fill}%
\end{pgfscope}%
\begin{pgfscope}%
\pgfpathrectangle{\pgfqpoint{0.539299in}{0.078740in}}{\pgfqpoint{7.842520in}{7.842520in}}%
\pgfusepath{clip}%
\pgfsetbuttcap%
\pgfsetroundjoin%
\definecolor{currentfill}{rgb}{0.156270,0.489624,0.557936}%
\pgfsetfillcolor{currentfill}%
\pgfsetlinewidth{0.000000pt}%
\definecolor{currentstroke}{rgb}{0.130067,0.651384,0.521608}%
\pgfsetstrokecolor{currentstroke}%
\pgfsetdash{}{0pt}%
\pgfpathmoveto{\pgfqpoint{4.380311in}{4.558363in}}%
\pgfpathlineto{\pgfqpoint{4.301131in}{4.661130in}}%
\pgfpathlineto{\pgfqpoint{4.436631in}{4.511408in}}%
\pgfpathclose%
\pgfusepath{fill}%
\end{pgfscope}%
\begin{pgfscope}%
\pgfpathrectangle{\pgfqpoint{0.539299in}{0.078740in}}{\pgfqpoint{7.842520in}{7.842520in}}%
\pgfusepath{clip}%
\pgfsetbuttcap%
\pgfsetroundjoin%
\definecolor{currentfill}{rgb}{0.273006,0.204520,0.501721}%
\pgfsetfillcolor{currentfill}%
\pgfsetlinewidth{0.000000pt}%
\definecolor{currentstroke}{rgb}{0.132268,0.655014,0.519661}%
\pgfsetstrokecolor{currentstroke}%
\pgfsetdash{}{0pt}%
\pgfpathmoveto{\pgfqpoint{5.735793in}{3.448039in}}%
\pgfpathlineto{\pgfqpoint{5.809842in}{3.469040in}}%
\pgfpathlineto{\pgfqpoint{5.673111in}{3.524234in}}%
\pgfpathclose%
\pgfusepath{fill}%
\end{pgfscope}%
\begin{pgfscope}%
\pgfpathrectangle{\pgfqpoint{0.539299in}{0.078740in}}{\pgfqpoint{7.842520in}{7.842520in}}%
\pgfusepath{clip}%
\pgfsetbuttcap%
\pgfsetroundjoin%
\definecolor{currentfill}{rgb}{0.283072,0.130895,0.449241}%
\pgfsetfillcolor{currentfill}%
\pgfsetlinewidth{0.000000pt}%
\definecolor{currentstroke}{rgb}{0.134692,0.658636,0.517649}%
\pgfsetstrokecolor{currentstroke}%
\pgfsetdash{}{0pt}%
\pgfpathmoveto{\pgfqpoint{6.570734in}{3.206547in}}%
\pgfpathlineto{\pgfqpoint{6.642120in}{3.235328in}}%
\pgfpathlineto{\pgfqpoint{6.504552in}{3.303089in}}%
\pgfpathclose%
\pgfusepath{fill}%
\end{pgfscope}%
\begin{pgfscope}%
\pgfpathrectangle{\pgfqpoint{0.539299in}{0.078740in}}{\pgfqpoint{7.842520in}{7.842520in}}%
\pgfusepath{clip}%
\pgfsetbuttcap%
\pgfsetroundjoin%
\definecolor{currentfill}{rgb}{0.175841,0.441290,0.557685}%
\pgfsetfillcolor{currentfill}%
\pgfsetlinewidth{0.000000pt}%
\definecolor{currentstroke}{rgb}{0.137339,0.662252,0.515571}%
\pgfsetstrokecolor{currentstroke}%
\pgfsetdash{}{0pt}%
\pgfpathmoveto{\pgfqpoint{4.436631in}{4.511408in}}%
\pgfpathlineto{\pgfqpoint{4.572157in}{4.356552in}}%
\pgfpathlineto{\pgfqpoint{4.649993in}{4.264866in}}%
\pgfpathclose%
\pgfusepath{fill}%
\end{pgfscope}%
\begin{pgfscope}%
\pgfpathrectangle{\pgfqpoint{0.539299in}{0.078740in}}{\pgfqpoint{7.842520in}{7.842520in}}%
\pgfusepath{clip}%
\pgfsetbuttcap%
\pgfsetroundjoin%
\definecolor{currentfill}{rgb}{0.185556,0.418570,0.556753}%
\pgfsetfillcolor{currentfill}%
\pgfsetlinewidth{0.000000pt}%
\definecolor{currentstroke}{rgb}{0.140210,0.665859,0.513427}%
\pgfsetstrokecolor{currentstroke}%
\pgfsetdash{}{0pt}%
\pgfpathmoveto{\pgfqpoint{4.707691in}{4.203491in}}%
\pgfpathlineto{\pgfqpoint{4.649993in}{4.264866in}}%
\pgfpathlineto{\pgfqpoint{4.572157in}{4.356552in}}%
\pgfpathclose%
\pgfusepath{fill}%
\end{pgfscope}%
\begin{pgfscope}%
\pgfpathrectangle{\pgfqpoint{0.539299in}{0.078740in}}{\pgfqpoint{7.842520in}{7.842520in}}%
\pgfusepath{clip}%
\pgfsetbuttcap%
\pgfsetroundjoin%
\definecolor{currentfill}{rgb}{0.281924,0.089666,0.412415}%
\pgfsetfillcolor{currentfill}%
\pgfsetlinewidth{0.000000pt}%
\definecolor{currentstroke}{rgb}{0.143303,0.669459,0.511215}%
\pgfsetstrokecolor{currentstroke}%
\pgfsetdash{}{0pt}%
\pgfpathmoveto{\pgfqpoint{6.779843in}{3.161531in}}%
\pgfpathlineto{\pgfqpoint{6.917735in}{3.083093in}}%
\pgfpathlineto{\pgfqpoint{6.987917in}{3.113178in}}%
\pgfpathclose%
\pgfusepath{fill}%
\end{pgfscope}%
\begin{pgfscope}%
\pgfpathrectangle{\pgfqpoint{0.539299in}{0.078740in}}{\pgfqpoint{7.842520in}{7.842520in}}%
\pgfusepath{clip}%
\pgfsetbuttcap%
\pgfsetroundjoin%
\definecolor{currentfill}{rgb}{0.120565,0.596422,0.543611}%
\pgfsetfillcolor{currentfill}%
\pgfsetlinewidth{0.000000pt}%
\definecolor{currentstroke}{rgb}{0.146616,0.673050,0.508936}%
\pgfsetstrokecolor{currentstroke}%
\pgfsetdash{}{0pt}%
\pgfpathmoveto{\pgfqpoint{3.761962in}{5.020676in}}%
\pgfpathlineto{\pgfqpoint{3.679940in}{5.091602in}}%
\pgfpathlineto{\pgfqpoint{3.895878in}{4.988126in}}%
\pgfpathclose%
\pgfusepath{fill}%
\end{pgfscope}%
\begin{pgfscope}%
\pgfpathrectangle{\pgfqpoint{0.539299in}{0.078740in}}{\pgfqpoint{7.842520in}{7.842520in}}%
\pgfusepath{clip}%
\pgfsetbuttcap%
\pgfsetroundjoin%
\definecolor{currentfill}{rgb}{0.272594,0.025563,0.353093}%
\pgfsetfillcolor{currentfill}%
\pgfsetlinewidth{0.000000pt}%
\definecolor{currentstroke}{rgb}{0.150148,0.676631,0.506589}%
\pgfsetstrokecolor{currentstroke}%
\pgfsetdash{}{0pt}%
\pgfpathmoveto{\pgfqpoint{7.264065in}{2.962827in}}%
\pgfpathlineto{\pgfqpoint{7.402749in}{2.891078in}}%
\pgfpathlineto{\pgfqpoint{7.472249in}{2.944684in}}%
\pgfpathclose%
\pgfusepath{fill}%
\end{pgfscope}%
\begin{pgfscope}%
\pgfpathrectangle{\pgfqpoint{0.539299in}{0.078740in}}{\pgfqpoint{7.842520in}{7.842520in}}%
\pgfusepath{clip}%
\pgfsetbuttcap%
\pgfsetroundjoin%
\definecolor{currentfill}{rgb}{0.163625,0.471133,0.558148}%
\pgfsetfillcolor{currentfill}%
\pgfsetlinewidth{0.000000pt}%
\definecolor{currentstroke}{rgb}{0.153894,0.680203,0.504172}%
\pgfsetstrokecolor{currentstroke}%
\pgfsetdash{}{0pt}%
\pgfpathmoveto{\pgfqpoint{2.954036in}{4.737238in}}%
\pgfpathlineto{\pgfqpoint{2.913780in}{4.395967in}}%
\pgfpathlineto{\pgfqpoint{2.830306in}{4.368134in}}%
\pgfpathclose%
\pgfusepath{fill}%
\end{pgfscope}%
\begin{pgfscope}%
\pgfpathrectangle{\pgfqpoint{0.539299in}{0.078740in}}{\pgfqpoint{7.842520in}{7.842520in}}%
\pgfusepath{clip}%
\pgfsetbuttcap%
\pgfsetroundjoin%
\definecolor{currentfill}{rgb}{0.206756,0.371758,0.553117}%
\pgfsetfillcolor{currentfill}%
\pgfsetlinewidth{0.000000pt}%
\definecolor{currentstroke}{rgb}{0.157851,0.683765,0.501686}%
\pgfsetstrokecolor{currentstroke}%
\pgfsetdash{}{0pt}%
\pgfpathmoveto{\pgfqpoint{4.843266in}{4.057477in}}%
\pgfpathlineto{\pgfqpoint{4.919892in}{3.989315in}}%
\pgfpathlineto{\pgfqpoint{4.707691in}{4.203491in}}%
\pgfpathclose%
\pgfusepath{fill}%
\end{pgfscope}%
\begin{pgfscope}%
\pgfpathrectangle{\pgfqpoint{0.539299in}{0.078740in}}{\pgfqpoint{7.842520in}{7.842520in}}%
\pgfusepath{clip}%
\pgfsetbuttcap%
\pgfsetroundjoin%
\definecolor{currentfill}{rgb}{0.252194,0.269783,0.531579}%
\pgfsetfillcolor{currentfill}%
\pgfsetlinewidth{0.000000pt}%
\definecolor{currentstroke}{rgb}{0.162016,0.687316,0.499129}%
\pgfsetstrokecolor{currentstroke}%
\pgfsetdash{}{0pt}%
\pgfpathmoveto{\pgfqpoint{5.250882in}{3.689752in}}%
\pgfpathlineto{\pgfqpoint{5.326196in}{3.667457in}}%
\pgfpathlineto{\pgfqpoint{5.190465in}{3.761377in}}%
\pgfpathclose%
\pgfusepath{fill}%
\end{pgfscope}%
\begin{pgfscope}%
\pgfpathrectangle{\pgfqpoint{0.539299in}{0.078740in}}{\pgfqpoint{7.842520in}{7.842520in}}%
\pgfusepath{clip}%
\pgfsetbuttcap%
\pgfsetroundjoin%
\definecolor{currentfill}{rgb}{0.216210,0.351535,0.550627}%
\pgfsetfillcolor{currentfill}%
\pgfsetlinewidth{0.000000pt}%
\definecolor{currentstroke}{rgb}{0.166383,0.690856,0.496502}%
\pgfsetstrokecolor{currentstroke}%
\pgfsetdash{}{0pt}%
\pgfpathmoveto{\pgfqpoint{4.978938in}{3.922091in}}%
\pgfpathlineto{\pgfqpoint{4.919892in}{3.989315in}}%
\pgfpathlineto{\pgfqpoint{4.843266in}{4.057477in}}%
\pgfpathclose%
\pgfusepath{fill}%
\end{pgfscope}%
\begin{pgfscope}%
\pgfpathrectangle{\pgfqpoint{0.539299in}{0.078740in}}{\pgfqpoint{7.842520in}{7.842520in}}%
\pgfusepath{clip}%
\pgfsetbuttcap%
\pgfsetroundjoin%
\definecolor{currentfill}{rgb}{0.195860,0.395433,0.555276}%
\pgfsetfillcolor{currentfill}%
\pgfsetlinewidth{0.000000pt}%
\definecolor{currentstroke}{rgb}{0.170948,0.694384,0.493803}%
\pgfsetstrokecolor{currentstroke}%
\pgfsetdash{}{0pt}%
\pgfpathmoveto{\pgfqpoint{2.661974in}{4.301274in}}%
\pgfpathlineto{\pgfqpoint{2.746363in}{4.336760in}}%
\pgfpathlineto{\pgfqpoint{2.627589in}{3.858151in}}%
\pgfpathclose%
\pgfusepath{fill}%
\end{pgfscope}%
\begin{pgfscope}%
\pgfpathrectangle{\pgfqpoint{0.539299in}{0.078740in}}{\pgfqpoint{7.842520in}{7.842520in}}%
\pgfusepath{clip}%
\pgfsetbuttcap%
\pgfsetroundjoin%
\definecolor{currentfill}{rgb}{0.278012,0.180367,0.486697}%
\pgfsetfillcolor{currentfill}%
\pgfsetlinewidth{0.000000pt}%
\definecolor{currentstroke}{rgb}{0.175707,0.697900,0.491033}%
\pgfsetstrokecolor{currentstroke}%
\pgfsetdash{}{0pt}%
\pgfpathmoveto{\pgfqpoint{6.084512in}{3.361457in}}%
\pgfpathlineto{\pgfqpoint{6.157532in}{3.392366in}}%
\pgfpathlineto{\pgfqpoint{5.946992in}{3.415751in}}%
\pgfpathclose%
\pgfusepath{fill}%
\end{pgfscope}%
\begin{pgfscope}%
\pgfpathrectangle{\pgfqpoint{0.539299in}{0.078740in}}{\pgfqpoint{7.842520in}{7.842520in}}%
\pgfusepath{clip}%
\pgfsetbuttcap%
\pgfsetroundjoin%
\definecolor{currentfill}{rgb}{0.268510,0.009605,0.335427}%
\pgfsetfillcolor{currentfill}%
\pgfsetlinewidth{0.000000pt}%
\definecolor{currentstroke}{rgb}{0.180653,0.701402,0.488189}%
\pgfsetstrokecolor{currentstroke}%
\pgfsetdash{}{0pt}%
\pgfpathmoveto{\pgfqpoint{7.472249in}{2.944684in}}%
\pgfpathlineto{\pgfqpoint{7.541973in}{2.824378in}}%
\pgfpathlineto{\pgfqpoint{7.611514in}{2.887087in}}%
\pgfpathclose%
\pgfusepath{fill}%
\end{pgfscope}%
\begin{pgfscope}%
\pgfpathrectangle{\pgfqpoint{0.539299in}{0.078740in}}{\pgfqpoint{7.842520in}{7.842520in}}%
\pgfusepath{clip}%
\pgfsetbuttcap%
\pgfsetroundjoin%
\definecolor{currentfill}{rgb}{0.221989,0.339161,0.548752}%
\pgfsetfillcolor{currentfill}%
\pgfsetlinewidth{0.000000pt}%
\definecolor{currentstroke}{rgb}{0.185783,0.704891,0.485273}%
\pgfsetstrokecolor{currentstroke}%
\pgfsetdash{}{0pt}%
\pgfpathmoveto{\pgfqpoint{2.543987in}{3.805632in}}%
\pgfpathlineto{\pgfqpoint{2.459954in}{3.750514in}}%
\pgfpathlineto{\pgfqpoint{2.661974in}{4.301274in}}%
\pgfpathclose%
\pgfusepath{fill}%
\end{pgfscope}%
\begin{pgfscope}%
\pgfpathrectangle{\pgfqpoint{0.539299in}{0.078740in}}{\pgfqpoint{7.842520in}{7.842520in}}%
\pgfusepath{clip}%
\pgfsetbuttcap%
\pgfsetroundjoin%
\definecolor{currentfill}{rgb}{0.270595,0.214069,0.507052}%
\pgfsetfillcolor{currentfill}%
\pgfsetlinewidth{0.000000pt}%
\definecolor{currentstroke}{rgb}{0.191090,0.708366,0.482284}%
\pgfsetstrokecolor{currentstroke}%
\pgfsetdash{}{0pt}%
\pgfpathmoveto{\pgfqpoint{5.673111in}{3.524234in}}%
\pgfpathlineto{\pgfqpoint{5.598833in}{3.513257in}}%
\pgfpathlineto{\pgfqpoint{5.735793in}{3.448039in}}%
\pgfpathclose%
\pgfusepath{fill}%
\end{pgfscope}%
\begin{pgfscope}%
\pgfpathrectangle{\pgfqpoint{0.539299in}{0.078740in}}{\pgfqpoint{7.842520in}{7.842520in}}%
\pgfusepath{clip}%
\pgfsetbuttcap%
\pgfsetroundjoin%
\definecolor{currentfill}{rgb}{0.282623,0.140926,0.457517}%
\pgfsetfillcolor{currentfill}%
\pgfsetlinewidth{0.000000pt}%
\definecolor{currentstroke}{rgb}{0.196571,0.711827,0.479221}%
\pgfsetstrokecolor{currentstroke}%
\pgfsetdash{}{0pt}%
\pgfpathmoveto{\pgfqpoint{6.504552in}{3.303089in}}%
\pgfpathlineto{\pgfqpoint{6.432810in}{3.275195in}}%
\pgfpathlineto{\pgfqpoint{6.570734in}{3.206547in}}%
\pgfpathclose%
\pgfusepath{fill}%
\end{pgfscope}%
\begin{pgfscope}%
\pgfpathrectangle{\pgfqpoint{0.539299in}{0.078740in}}{\pgfqpoint{7.842520in}{7.842520in}}%
\pgfusepath{clip}%
\pgfsetbuttcap%
\pgfsetroundjoin%
\definecolor{currentfill}{rgb}{0.280255,0.165693,0.476498}%
\pgfsetfillcolor{currentfill}%
\pgfsetlinewidth{0.000000pt}%
\definecolor{currentstroke}{rgb}{0.202219,0.715272,0.476084}%
\pgfsetstrokecolor{currentstroke}%
\pgfsetdash{}{0pt}%
\pgfpathmoveto{\pgfqpoint{6.222345in}{3.303711in}}%
\pgfpathlineto{\pgfqpoint{6.295060in}{3.336953in}}%
\pgfpathlineto{\pgfqpoint{6.157532in}{3.392366in}}%
\pgfpathclose%
\pgfusepath{fill}%
\end{pgfscope}%
\begin{pgfscope}%
\pgfpathrectangle{\pgfqpoint{0.539299in}{0.078740in}}{\pgfqpoint{7.842520in}{7.842520in}}%
\pgfusepath{clip}%
\pgfsetbuttcap%
\pgfsetroundjoin%
\definecolor{currentfill}{rgb}{0.231674,0.318106,0.544834}%
\pgfsetfillcolor{currentfill}%
\pgfsetlinewidth{0.000000pt}%
\definecolor{currentstroke}{rgb}{0.208030,0.718701,0.472873}%
\pgfsetstrokecolor{currentstroke}%
\pgfsetdash{}{0pt}%
\pgfpathmoveto{\pgfqpoint{5.055056in}{3.868593in}}%
\pgfpathlineto{\pgfqpoint{4.978938in}{3.922091in}}%
\pgfpathlineto{\pgfqpoint{5.114785in}{3.799326in}}%
\pgfpathclose%
\pgfusepath{fill}%
\end{pgfscope}%
\begin{pgfscope}%
\pgfpathrectangle{\pgfqpoint{0.539299in}{0.078740in}}{\pgfqpoint{7.842520in}{7.842520in}}%
\pgfusepath{clip}%
\pgfsetbuttcap%
\pgfsetroundjoin%
\definecolor{currentfill}{rgb}{0.274128,0.199721,0.498911}%
\pgfsetfillcolor{currentfill}%
\pgfsetlinewidth{0.000000pt}%
\definecolor{currentstroke}{rgb}{0.214000,0.722114,0.469588}%
\pgfsetstrokecolor{currentstroke}%
\pgfsetdash{}{0pt}%
\pgfpathmoveto{\pgfqpoint{5.809842in}{3.469040in}}%
\pgfpathlineto{\pgfqpoint{5.735793in}{3.448039in}}%
\pgfpathlineto{\pgfqpoint{5.946992in}{3.415751in}}%
\pgfpathclose%
\pgfusepath{fill}%
\end{pgfscope}%
\begin{pgfscope}%
\pgfpathrectangle{\pgfqpoint{0.539299in}{0.078740in}}{\pgfqpoint{7.842520in}{7.842520in}}%
\pgfusepath{clip}%
\pgfsetbuttcap%
\pgfsetroundjoin%
\definecolor{currentfill}{rgb}{0.269944,0.014625,0.341379}%
\pgfsetfillcolor{currentfill}%
\pgfsetlinewidth{0.000000pt}%
\definecolor{currentstroke}{rgb}{0.220124,0.725509,0.466226}%
\pgfsetstrokecolor{currentstroke}%
\pgfsetdash{}{0pt}%
\pgfpathmoveto{\pgfqpoint{7.472249in}{2.944684in}}%
\pgfpathlineto{\pgfqpoint{7.402749in}{2.891078in}}%
\pgfpathlineto{\pgfqpoint{7.541973in}{2.824378in}}%
\pgfpathclose%
\pgfusepath{fill}%
\end{pgfscope}%
\begin{pgfscope}%
\pgfpathrectangle{\pgfqpoint{0.539299in}{0.078740in}}{\pgfqpoint{7.842520in}{7.842520in}}%
\pgfusepath{clip}%
\pgfsetbuttcap%
\pgfsetroundjoin%
\definecolor{currentfill}{rgb}{0.277018,0.050344,0.375715}%
\pgfsetfillcolor{currentfill}%
\pgfsetlinewidth{0.000000pt}%
\definecolor{currentstroke}{rgb}{0.226397,0.728888,0.462789}%
\pgfsetstrokecolor{currentstroke}%
\pgfsetdash{}{0pt}%
\pgfpathmoveto{\pgfqpoint{7.125818in}{3.037569in}}%
\pgfpathlineto{\pgfqpoint{7.194201in}{2.919934in}}%
\pgfpathlineto{\pgfqpoint{7.264065in}{2.962827in}}%
\pgfpathclose%
\pgfusepath{fill}%
\end{pgfscope}%
\begin{pgfscope}%
\pgfpathrectangle{\pgfqpoint{0.539299in}{0.078740in}}{\pgfqpoint{7.842520in}{7.842520in}}%
\pgfusepath{clip}%
\pgfsetbuttcap%
\pgfsetroundjoin%
\definecolor{currentfill}{rgb}{0.280267,0.073417,0.397163}%
\pgfsetfillcolor{currentfill}%
\pgfsetlinewidth{0.000000pt}%
\definecolor{currentstroke}{rgb}{0.232815,0.732247,0.459277}%
\pgfsetstrokecolor{currentstroke}%
\pgfsetdash{}{0pt}%
\pgfpathmoveto{\pgfqpoint{6.917735in}{3.083093in}}%
\pgfpathlineto{\pgfqpoint{7.055833in}{3.001854in}}%
\pgfpathlineto{\pgfqpoint{7.125818in}{3.037569in}}%
\pgfpathclose%
\pgfusepath{fill}%
\end{pgfscope}%
\begin{pgfscope}%
\pgfpathrectangle{\pgfqpoint{0.539299in}{0.078740in}}{\pgfqpoint{7.842520in}{7.842520in}}%
\pgfusepath{clip}%
\pgfsetbuttcap%
\pgfsetroundjoin%
\definecolor{currentfill}{rgb}{0.126453,0.570633,0.549841}%
\pgfsetfillcolor{currentfill}%
\pgfsetlinewidth{0.000000pt}%
\definecolor{currentstroke}{rgb}{0.239374,0.735588,0.455688}%
\pgfsetstrokecolor{currentstroke}%
\pgfsetdash{}{0pt}%
\pgfpathmoveto{\pgfqpoint{3.121655in}{4.747134in}}%
\pgfpathlineto{\pgfqpoint{3.249818in}{4.985619in}}%
\pgfpathlineto{\pgfqpoint{3.333165in}{4.962328in}}%
\pgfpathclose%
\pgfusepath{fill}%
\end{pgfscope}%
\begin{pgfscope}%
\pgfpathrectangle{\pgfqpoint{0.539299in}{0.078740in}}{\pgfqpoint{7.842520in}{7.842520in}}%
\pgfusepath{clip}%
\pgfsetbuttcap%
\pgfsetroundjoin%
\definecolor{currentfill}{rgb}{0.283197,0.115680,0.436115}%
\pgfsetfillcolor{currentfill}%
\pgfsetlinewidth{0.000000pt}%
\definecolor{currentstroke}{rgb}{0.246070,0.738910,0.452024}%
\pgfsetstrokecolor{currentstroke}%
\pgfsetdash{}{0pt}%
\pgfpathmoveto{\pgfqpoint{6.779843in}{3.161531in}}%
\pgfpathlineto{\pgfqpoint{6.642120in}{3.235328in}}%
\pgfpathlineto{\pgfqpoint{6.708800in}{3.131236in}}%
\pgfpathclose%
\pgfusepath{fill}%
\end{pgfscope}%
\begin{pgfscope}%
\pgfpathrectangle{\pgfqpoint{0.539299in}{0.078740in}}{\pgfqpoint{7.842520in}{7.842520in}}%
\pgfusepath{clip}%
\pgfsetbuttcap%
\pgfsetroundjoin%
\definecolor{currentfill}{rgb}{0.258965,0.251537,0.524736}%
\pgfsetfillcolor{currentfill}%
\pgfsetlinewidth{0.000000pt}%
\definecolor{currentstroke}{rgb}{0.252899,0.742211,0.448284}%
\pgfsetstrokecolor{currentstroke}%
\pgfsetdash{}{0pt}%
\pgfpathmoveto{\pgfqpoint{5.387297in}{3.592738in}}%
\pgfpathlineto{\pgfqpoint{5.462306in}{3.585484in}}%
\pgfpathlineto{\pgfqpoint{5.326196in}{3.667457in}}%
\pgfpathclose%
\pgfusepath{fill}%
\end{pgfscope}%
\begin{pgfscope}%
\pgfpathrectangle{\pgfqpoint{0.539299in}{0.078740in}}{\pgfqpoint{7.842520in}{7.842520in}}%
\pgfusepath{clip}%
\pgfsetbuttcap%
\pgfsetroundjoin%
\definecolor{currentfill}{rgb}{0.278826,0.175490,0.483397}%
\pgfsetfillcolor{currentfill}%
\pgfsetlinewidth{0.000000pt}%
\definecolor{currentstroke}{rgb}{0.259857,0.745492,0.444467}%
\pgfsetstrokecolor{currentstroke}%
\pgfsetdash{}{0pt}%
\pgfpathmoveto{\pgfqpoint{6.157532in}{3.392366in}}%
\pgfpathlineto{\pgfqpoint{6.084512in}{3.361457in}}%
\pgfpathlineto{\pgfqpoint{6.222345in}{3.303711in}}%
\pgfpathclose%
\pgfusepath{fill}%
\end{pgfscope}%
\begin{pgfscope}%
\pgfpathrectangle{\pgfqpoint{0.539299in}{0.078740in}}{\pgfqpoint{7.842520in}{7.842520in}}%
\pgfusepath{clip}%
\pgfsetbuttcap%
\pgfsetroundjoin%
\definecolor{currentfill}{rgb}{0.243113,0.292092,0.538516}%
\pgfsetfillcolor{currentfill}%
\pgfsetlinewidth{0.000000pt}%
\definecolor{currentstroke}{rgb}{0.266941,0.748751,0.440573}%
\pgfsetstrokecolor{currentstroke}%
\pgfsetdash{}{0pt}%
\pgfpathmoveto{\pgfqpoint{5.190465in}{3.761377in}}%
\pgfpathlineto{\pgfqpoint{5.114785in}{3.799326in}}%
\pgfpathlineto{\pgfqpoint{5.250882in}{3.689752in}}%
\pgfpathclose%
\pgfusepath{fill}%
\end{pgfscope}%
\begin{pgfscope}%
\pgfpathrectangle{\pgfqpoint{0.539299in}{0.078740in}}{\pgfqpoint{7.842520in}{7.842520in}}%
\pgfusepath{clip}%
\pgfsetbuttcap%
\pgfsetroundjoin%
\definecolor{currentfill}{rgb}{0.119738,0.603785,0.541400}%
\pgfsetfillcolor{currentfill}%
\pgfsetlinewidth{0.000000pt}%
\definecolor{currentstroke}{rgb}{0.274149,0.751988,0.436601}%
\pgfsetstrokecolor{currentstroke}%
\pgfsetdash{}{0pt}%
\pgfpathmoveto{\pgfqpoint{3.333165in}{4.962328in}}%
\pgfpathlineto{\pgfqpoint{3.464190in}{5.097259in}}%
\pgfpathlineto{\pgfqpoint{3.547026in}{5.048670in}}%
\pgfpathclose%
\pgfusepath{fill}%
\end{pgfscope}%
\begin{pgfscope}%
\pgfpathrectangle{\pgfqpoint{0.539299in}{0.078740in}}{\pgfqpoint{7.842520in}{7.842520in}}%
\pgfusepath{clip}%
\pgfsetbuttcap%
\pgfsetroundjoin%
\definecolor{currentfill}{rgb}{0.149039,0.508051,0.557250}%
\pgfsetfillcolor{currentfill}%
\pgfsetlinewidth{0.000000pt}%
\definecolor{currentstroke}{rgb}{0.281477,0.755203,0.432552}%
\pgfsetstrokecolor{currentstroke}%
\pgfsetdash{}{0pt}%
\pgfpathmoveto{\pgfqpoint{2.954036in}{4.737238in}}%
\pgfpathlineto{\pgfqpoint{3.038103in}{4.744324in}}%
\pgfpathlineto{\pgfqpoint{2.913780in}{4.395967in}}%
\pgfpathclose%
\pgfusepath{fill}%
\end{pgfscope}%
\begin{pgfscope}%
\pgfpathrectangle{\pgfqpoint{0.539299in}{0.078740in}}{\pgfqpoint{7.842520in}{7.842520in}}%
\pgfusepath{clip}%
\pgfsetbuttcap%
\pgfsetroundjoin%
\definecolor{currentfill}{rgb}{0.278791,0.062145,0.386592}%
\pgfsetfillcolor{currentfill}%
\pgfsetlinewidth{0.000000pt}%
\definecolor{currentstroke}{rgb}{0.288921,0.758394,0.428426}%
\pgfsetstrokecolor{currentstroke}%
\pgfsetdash{}{0pt}%
\pgfpathmoveto{\pgfqpoint{7.125818in}{3.037569in}}%
\pgfpathlineto{\pgfqpoint{7.055833in}{3.001854in}}%
\pgfpathlineto{\pgfqpoint{7.194201in}{2.919934in}}%
\pgfpathclose%
\pgfusepath{fill}%
\end{pgfscope}%
\begin{pgfscope}%
\pgfpathrectangle{\pgfqpoint{0.539299in}{0.078740in}}{\pgfqpoint{7.842520in}{7.842520in}}%
\pgfusepath{clip}%
\pgfsetbuttcap%
\pgfsetroundjoin%
\definecolor{currentfill}{rgb}{0.274952,0.037752,0.364543}%
\pgfsetfillcolor{currentfill}%
\pgfsetlinewidth{0.000000pt}%
\definecolor{currentstroke}{rgb}{0.296479,0.761561,0.424223}%
\pgfsetstrokecolor{currentstroke}%
\pgfsetdash{}{0pt}%
\pgfpathmoveto{\pgfqpoint{7.264065in}{2.962827in}}%
\pgfpathlineto{\pgfqpoint{7.194201in}{2.919934in}}%
\pgfpathlineto{\pgfqpoint{7.402749in}{2.891078in}}%
\pgfpathclose%
\pgfusepath{fill}%
\end{pgfscope}%
\begin{pgfscope}%
\pgfpathrectangle{\pgfqpoint{0.539299in}{0.078740in}}{\pgfqpoint{7.842520in}{7.842520in}}%
\pgfusepath{clip}%
\pgfsetbuttcap%
\pgfsetroundjoin%
\definecolor{currentfill}{rgb}{0.131172,0.555899,0.552459}%
\pgfsetfillcolor{currentfill}%
\pgfsetlinewidth{0.000000pt}%
\definecolor{currentstroke}{rgb}{0.304148,0.764704,0.419943}%
\pgfsetstrokecolor{currentstroke}%
\pgfsetdash{}{0pt}%
\pgfpathmoveto{\pgfqpoint{3.121655in}{4.747134in}}%
\pgfpathlineto{\pgfqpoint{3.038103in}{4.744324in}}%
\pgfpathlineto{\pgfqpoint{3.249818in}{4.985619in}}%
\pgfpathclose%
\pgfusepath{fill}%
\end{pgfscope}%
\begin{pgfscope}%
\pgfpathrectangle{\pgfqpoint{0.539299in}{0.078740in}}{\pgfqpoint{7.842520in}{7.842520in}}%
\pgfusepath{clip}%
\pgfsetbuttcap%
\pgfsetroundjoin%
\definecolor{currentfill}{rgb}{0.262138,0.242286,0.520837}%
\pgfsetfillcolor{currentfill}%
\pgfsetlinewidth{0.000000pt}%
\definecolor{currentstroke}{rgb}{0.311925,0.767822,0.415586}%
\pgfsetstrokecolor{currentstroke}%
\pgfsetdash{}{0pt}%
\pgfpathmoveto{\pgfqpoint{5.387297in}{3.592738in}}%
\pgfpathlineto{\pgfqpoint{5.598833in}{3.513257in}}%
\pgfpathlineto{\pgfqpoint{5.462306in}{3.585484in}}%
\pgfpathclose%
\pgfusepath{fill}%
\end{pgfscope}%
\begin{pgfscope}%
\pgfpathrectangle{\pgfqpoint{0.539299in}{0.078740in}}{\pgfqpoint{7.842520in}{7.842520in}}%
\pgfusepath{clip}%
\pgfsetbuttcap%
\pgfsetroundjoin%
\definecolor{currentfill}{rgb}{0.283072,0.130895,0.449241}%
\pgfsetfillcolor{currentfill}%
\pgfsetlinewidth{0.000000pt}%
\definecolor{currentstroke}{rgb}{0.319809,0.770914,0.411152}%
\pgfsetstrokecolor{currentstroke}%
\pgfsetdash{}{0pt}%
\pgfpathmoveto{\pgfqpoint{6.708800in}{3.131236in}}%
\pgfpathlineto{\pgfqpoint{6.642120in}{3.235328in}}%
\pgfpathlineto{\pgfqpoint{6.570734in}{3.206547in}}%
\pgfpathclose%
\pgfusepath{fill}%
\end{pgfscope}%
\begin{pgfscope}%
\pgfpathrectangle{\pgfqpoint{0.539299in}{0.078740in}}{\pgfqpoint{7.842520in}{7.842520in}}%
\pgfusepath{clip}%
\pgfsetbuttcap%
\pgfsetroundjoin%
\definecolor{currentfill}{rgb}{0.121148,0.592739,0.544641}%
\pgfsetfillcolor{currentfill}%
\pgfsetlinewidth{0.000000pt}%
\definecolor{currentstroke}{rgb}{0.327796,0.773980,0.406640}%
\pgfsetstrokecolor{currentstroke}%
\pgfsetdash{}{0pt}%
\pgfpathmoveto{\pgfqpoint{3.949347in}{5.006539in}}%
\pgfpathlineto{\pgfqpoint{4.030569in}{4.909838in}}%
\pgfpathlineto{\pgfqpoint{3.895878in}{4.988126in}}%
\pgfpathclose%
\pgfusepath{fill}%
\end{pgfscope}%
\begin{pgfscope}%
\pgfpathrectangle{\pgfqpoint{0.539299in}{0.078740in}}{\pgfqpoint{7.842520in}{7.842520in}}%
\pgfusepath{clip}%
\pgfsetbuttcap%
\pgfsetroundjoin%
\definecolor{currentfill}{rgb}{0.280868,0.160771,0.472899}%
\pgfsetfillcolor{currentfill}%
\pgfsetlinewidth{0.000000pt}%
\definecolor{currentstroke}{rgb}{0.335885,0.777018,0.402049}%
\pgfsetstrokecolor{currentstroke}%
\pgfsetdash{}{0pt}%
\pgfpathmoveto{\pgfqpoint{6.432810in}{3.275195in}}%
\pgfpathlineto{\pgfqpoint{6.295060in}{3.336953in}}%
\pgfpathlineto{\pgfqpoint{6.360427in}{3.240693in}}%
\pgfpathclose%
\pgfusepath{fill}%
\end{pgfscope}%
\begin{pgfscope}%
\pgfpathrectangle{\pgfqpoint{0.539299in}{0.078740in}}{\pgfqpoint{7.842520in}{7.842520in}}%
\pgfusepath{clip}%
\pgfsetbuttcap%
\pgfsetroundjoin%
\definecolor{currentfill}{rgb}{0.252194,0.269783,0.531579}%
\pgfsetfillcolor{currentfill}%
\pgfsetlinewidth{0.000000pt}%
\definecolor{currentstroke}{rgb}{0.344074,0.780029,0.397381}%
\pgfsetstrokecolor{currentstroke}%
\pgfsetdash{}{0pt}%
\pgfpathmoveto{\pgfqpoint{5.326196in}{3.667457in}}%
\pgfpathlineto{\pgfqpoint{5.250882in}{3.689752in}}%
\pgfpathlineto{\pgfqpoint{5.387297in}{3.592738in}}%
\pgfpathclose%
\pgfusepath{fill}%
\end{pgfscope}%
\begin{pgfscope}%
\pgfpathrectangle{\pgfqpoint{0.539299in}{0.078740in}}{\pgfqpoint{7.842520in}{7.842520in}}%
\pgfusepath{clip}%
\pgfsetbuttcap%
\pgfsetroundjoin%
\definecolor{currentfill}{rgb}{0.119699,0.618490,0.536347}%
\pgfsetfillcolor{currentfill}%
\pgfsetlinewidth{0.000000pt}%
\definecolor{currentstroke}{rgb}{0.352360,0.783011,0.392636}%
\pgfsetstrokecolor{currentstroke}%
\pgfsetdash{}{0pt}%
\pgfpathmoveto{\pgfqpoint{3.547026in}{5.048670in}}%
\pgfpathlineto{\pgfqpoint{3.464190in}{5.097259in}}%
\pgfpathlineto{\pgfqpoint{3.679940in}{5.091602in}}%
\pgfpathclose%
\pgfusepath{fill}%
\end{pgfscope}%
\begin{pgfscope}%
\pgfpathrectangle{\pgfqpoint{0.539299in}{0.078740in}}{\pgfqpoint{7.842520in}{7.842520in}}%
\pgfusepath{clip}%
\pgfsetbuttcap%
\pgfsetroundjoin%
\definecolor{currentfill}{rgb}{0.119423,0.611141,0.538982}%
\pgfsetfillcolor{currentfill}%
\pgfsetlinewidth{0.000000pt}%
\definecolor{currentstroke}{rgb}{0.360741,0.785964,0.387814}%
\pgfsetstrokecolor{currentstroke}%
\pgfsetdash{}{0pt}%
\pgfpathmoveto{\pgfqpoint{3.895878in}{4.988126in}}%
\pgfpathlineto{\pgfqpoint{3.679940in}{5.091602in}}%
\pgfpathlineto{\pgfqpoint{3.814186in}{5.073908in}}%
\pgfpathclose%
\pgfusepath{fill}%
\end{pgfscope}%
\begin{pgfscope}%
\pgfpathrectangle{\pgfqpoint{0.539299in}{0.078740in}}{\pgfqpoint{7.842520in}{7.842520in}}%
\pgfusepath{clip}%
\pgfsetbuttcap%
\pgfsetroundjoin%
\definecolor{currentfill}{rgb}{0.282656,0.100196,0.422160}%
\pgfsetfillcolor{currentfill}%
\pgfsetlinewidth{0.000000pt}%
\definecolor{currentstroke}{rgb}{0.369214,0.788888,0.382914}%
\pgfsetstrokecolor{currentstroke}%
\pgfsetdash{}{0pt}%
\pgfpathmoveto{\pgfqpoint{6.847001in}{3.050168in}}%
\pgfpathlineto{\pgfqpoint{6.917735in}{3.083093in}}%
\pgfpathlineto{\pgfqpoint{6.779843in}{3.161531in}}%
\pgfpathclose%
\pgfusepath{fill}%
\end{pgfscope}%
\begin{pgfscope}%
\pgfpathrectangle{\pgfqpoint{0.539299in}{0.078740in}}{\pgfqpoint{7.842520in}{7.842520in}}%
\pgfusepath{clip}%
\pgfsetbuttcap%
\pgfsetroundjoin%
\definecolor{currentfill}{rgb}{0.273006,0.204520,0.501721}%
\pgfsetfillcolor{currentfill}%
\pgfsetlinewidth{0.000000pt}%
\definecolor{currentstroke}{rgb}{0.377779,0.791781,0.377939}%
\pgfsetstrokecolor{currentstroke}%
\pgfsetdash{}{0pt}%
\pgfpathmoveto{\pgfqpoint{5.946992in}{3.415751in}}%
\pgfpathlineto{\pgfqpoint{5.735793in}{3.448039in}}%
\pgfpathlineto{\pgfqpoint{5.873171in}{3.386859in}}%
\pgfpathclose%
\pgfusepath{fill}%
\end{pgfscope}%
\begin{pgfscope}%
\pgfpathrectangle{\pgfqpoint{0.539299in}{0.078740in}}{\pgfqpoint{7.842520in}{7.842520in}}%
\pgfusepath{clip}%
\pgfsetbuttcap%
\pgfsetroundjoin%
\definecolor{currentfill}{rgb}{0.159194,0.482237,0.558073}%
\pgfsetfillcolor{currentfill}%
\pgfsetlinewidth{0.000000pt}%
\definecolor{currentstroke}{rgb}{0.386433,0.794644,0.372886}%
\pgfsetstrokecolor{currentstroke}%
\pgfsetdash{}{0pt}%
\pgfpathmoveto{\pgfqpoint{2.830306in}{4.368134in}}%
\pgfpathlineto{\pgfqpoint{2.746363in}{4.336760in}}%
\pgfpathlineto{\pgfqpoint{2.954036in}{4.737238in}}%
\pgfpathclose%
\pgfusepath{fill}%
\end{pgfscope}%
\begin{pgfscope}%
\pgfpathrectangle{\pgfqpoint{0.539299in}{0.078740in}}{\pgfqpoint{7.842520in}{7.842520in}}%
\pgfusepath{clip}%
\pgfsetbuttcap%
\pgfsetroundjoin%
\definecolor{currentfill}{rgb}{0.276194,0.190074,0.493001}%
\pgfsetfillcolor{currentfill}%
\pgfsetlinewidth{0.000000pt}%
\definecolor{currentstroke}{rgb}{0.395174,0.797475,0.367757}%
\pgfsetstrokecolor{currentstroke}%
\pgfsetdash{}{0pt}%
\pgfpathmoveto{\pgfqpoint{6.010937in}{3.326788in}}%
\pgfpathlineto{\pgfqpoint{6.084512in}{3.361457in}}%
\pgfpathlineto{\pgfqpoint{5.946992in}{3.415751in}}%
\pgfpathclose%
\pgfusepath{fill}%
\end{pgfscope}%
\begin{pgfscope}%
\pgfpathrectangle{\pgfqpoint{0.539299in}{0.078740in}}{\pgfqpoint{7.842520in}{7.842520in}}%
\pgfusepath{clip}%
\pgfsetbuttcap%
\pgfsetroundjoin%
\definecolor{currentfill}{rgb}{0.280255,0.165693,0.476498}%
\pgfsetfillcolor{currentfill}%
\pgfsetlinewidth{0.000000pt}%
\definecolor{currentstroke}{rgb}{0.404001,0.800275,0.362552}%
\pgfsetstrokecolor{currentstroke}%
\pgfsetdash{}{0pt}%
\pgfpathmoveto{\pgfqpoint{6.360427in}{3.240693in}}%
\pgfpathlineto{\pgfqpoint{6.295060in}{3.336953in}}%
\pgfpathlineto{\pgfqpoint{6.222345in}{3.303711in}}%
\pgfpathclose%
\pgfusepath{fill}%
\end{pgfscope}%
\begin{pgfscope}%
\pgfpathrectangle{\pgfqpoint{0.539299in}{0.078740in}}{\pgfqpoint{7.842520in}{7.842520in}}%
\pgfusepath{clip}%
\pgfsetbuttcap%
\pgfsetroundjoin%
\definecolor{currentfill}{rgb}{0.126453,0.570633,0.549841}%
\pgfsetfillcolor{currentfill}%
\pgfsetlinewidth{0.000000pt}%
\definecolor{currentstroke}{rgb}{0.412913,0.803041,0.357269}%
\pgfsetstrokecolor{currentstroke}%
\pgfsetdash{}{0pt}%
\pgfpathmoveto{\pgfqpoint{4.165728in}{4.797221in}}%
\pgfpathlineto{\pgfqpoint{4.030569in}{4.909838in}}%
\pgfpathlineto{\pgfqpoint{4.085083in}{4.900621in}}%
\pgfpathclose%
\pgfusepath{fill}%
\end{pgfscope}%
\begin{pgfscope}%
\pgfpathrectangle{\pgfqpoint{0.539299in}{0.078740in}}{\pgfqpoint{7.842520in}{7.842520in}}%
\pgfusepath{clip}%
\pgfsetbuttcap%
\pgfsetroundjoin%
\definecolor{currentfill}{rgb}{0.281887,0.150881,0.465405}%
\pgfsetfillcolor{currentfill}%
\pgfsetlinewidth{0.000000pt}%
\definecolor{currentstroke}{rgb}{0.421908,0.805774,0.351910}%
\pgfsetstrokecolor{currentstroke}%
\pgfsetdash{}{0pt}%
\pgfpathmoveto{\pgfqpoint{6.570734in}{3.206547in}}%
\pgfpathlineto{\pgfqpoint{6.432810in}{3.275195in}}%
\pgfpathlineto{\pgfqpoint{6.360427in}{3.240693in}}%
\pgfpathclose%
\pgfusepath{fill}%
\end{pgfscope}%
\begin{pgfscope}%
\pgfpathrectangle{\pgfqpoint{0.539299in}{0.078740in}}{\pgfqpoint{7.842520in}{7.842520in}}%
\pgfusepath{clip}%
\pgfsetbuttcap%
\pgfsetroundjoin%
\definecolor{currentfill}{rgb}{0.281924,0.089666,0.412415}%
\pgfsetfillcolor{currentfill}%
\pgfsetlinewidth{0.000000pt}%
\definecolor{currentstroke}{rgb}{0.430983,0.808473,0.346476}%
\pgfsetstrokecolor{currentstroke}%
\pgfsetdash{}{0pt}%
\pgfpathmoveto{\pgfqpoint{6.847001in}{3.050168in}}%
\pgfpathlineto{\pgfqpoint{7.055833in}{3.001854in}}%
\pgfpathlineto{\pgfqpoint{6.917735in}{3.083093in}}%
\pgfpathclose%
\pgfusepath{fill}%
\end{pgfscope}%
\begin{pgfscope}%
\pgfpathrectangle{\pgfqpoint{0.539299in}{0.078740in}}{\pgfqpoint{7.842520in}{7.842520in}}%
\pgfusepath{clip}%
\pgfsetbuttcap%
\pgfsetroundjoin%
\definecolor{currentfill}{rgb}{0.283197,0.115680,0.436115}%
\pgfsetfillcolor{currentfill}%
\pgfsetlinewidth{0.000000pt}%
\definecolor{currentstroke}{rgb}{0.440137,0.811138,0.340967}%
\pgfsetstrokecolor{currentstroke}%
\pgfsetdash{}{0pt}%
\pgfpathmoveto{\pgfqpoint{6.779843in}{3.161531in}}%
\pgfpathlineto{\pgfqpoint{6.708800in}{3.131236in}}%
\pgfpathlineto{\pgfqpoint{6.847001in}{3.050168in}}%
\pgfpathclose%
\pgfusepath{fill}%
\end{pgfscope}%
\begin{pgfscope}%
\pgfpathrectangle{\pgfqpoint{0.539299in}{0.078740in}}{\pgfqpoint{7.842520in}{7.842520in}}%
\pgfusepath{clip}%
\pgfsetbuttcap%
\pgfsetroundjoin%
\definecolor{currentfill}{rgb}{0.266580,0.228262,0.514349}%
\pgfsetfillcolor{currentfill}%
\pgfsetlinewidth{0.000000pt}%
\definecolor{currentstroke}{rgb}{0.449368,0.813768,0.335384}%
\pgfsetstrokecolor{currentstroke}%
\pgfsetdash{}{0pt}%
\pgfpathmoveto{\pgfqpoint{5.735793in}{3.448039in}}%
\pgfpathlineto{\pgfqpoint{5.598833in}{3.513257in}}%
\pgfpathlineto{\pgfqpoint{5.524083in}{3.506700in}}%
\pgfpathclose%
\pgfusepath{fill}%
\end{pgfscope}%
\begin{pgfscope}%
\pgfpathrectangle{\pgfqpoint{0.539299in}{0.078740in}}{\pgfqpoint{7.842520in}{7.842520in}}%
\pgfusepath{clip}%
\pgfsetbuttcap%
\pgfsetroundjoin%
\definecolor{currentfill}{rgb}{0.131172,0.555899,0.552459}%
\pgfsetfillcolor{currentfill}%
\pgfsetlinewidth{0.000000pt}%
\definecolor{currentstroke}{rgb}{0.458674,0.816363,0.329727}%
\pgfsetstrokecolor{currentstroke}%
\pgfsetdash{}{0pt}%
\pgfpathmoveto{\pgfqpoint{4.085083in}{4.900621in}}%
\pgfpathlineto{\pgfqpoint{4.301131in}{4.661130in}}%
\pgfpathlineto{\pgfqpoint{4.165728in}{4.797221in}}%
\pgfpathclose%
\pgfusepath{fill}%
\end{pgfscope}%
\begin{pgfscope}%
\pgfpathrectangle{\pgfqpoint{0.539299in}{0.078740in}}{\pgfqpoint{7.842520in}{7.842520in}}%
\pgfusepath{clip}%
\pgfsetbuttcap%
\pgfsetroundjoin%
\definecolor{currentfill}{rgb}{0.119423,0.611141,0.538982}%
\pgfsetfillcolor{currentfill}%
\pgfsetlinewidth{0.000000pt}%
\definecolor{currentstroke}{rgb}{0.468053,0.818921,0.323998}%
\pgfsetstrokecolor{currentstroke}%
\pgfsetdash{}{0pt}%
\pgfpathmoveto{\pgfqpoint{3.814186in}{5.073908in}}%
\pgfpathlineto{\pgfqpoint{3.949347in}{5.006539in}}%
\pgfpathlineto{\pgfqpoint{3.895878in}{4.988126in}}%
\pgfpathclose%
\pgfusepath{fill}%
\end{pgfscope}%
\begin{pgfscope}%
\pgfpathrectangle{\pgfqpoint{0.539299in}{0.078740in}}{\pgfqpoint{7.842520in}{7.842520in}}%
\pgfusepath{clip}%
\pgfsetbuttcap%
\pgfsetroundjoin%
\definecolor{currentfill}{rgb}{0.273809,0.031497,0.358853}%
\pgfsetfillcolor{currentfill}%
\pgfsetlinewidth{0.000000pt}%
\definecolor{currentstroke}{rgb}{0.477504,0.821444,0.318195}%
\pgfsetstrokecolor{currentstroke}%
\pgfsetdash{}{0pt}%
\pgfpathmoveto{\pgfqpoint{7.402749in}{2.891078in}}%
\pgfpathlineto{\pgfqpoint{7.194201in}{2.919934in}}%
\pgfpathlineto{\pgfqpoint{7.332923in}{2.839586in}}%
\pgfpathclose%
\pgfusepath{fill}%
\end{pgfscope}%
\begin{pgfscope}%
\pgfpathrectangle{\pgfqpoint{0.539299in}{0.078740in}}{\pgfqpoint{7.842520in}{7.842520in}}%
\pgfusepath{clip}%
\pgfsetbuttcap%
\pgfsetroundjoin%
\definecolor{currentfill}{rgb}{0.275191,0.194905,0.496005}%
\pgfsetfillcolor{currentfill}%
\pgfsetlinewidth{0.000000pt}%
\definecolor{currentstroke}{rgb}{0.487026,0.823929,0.312321}%
\pgfsetstrokecolor{currentstroke}%
\pgfsetdash{}{0pt}%
\pgfpathmoveto{\pgfqpoint{5.946992in}{3.415751in}}%
\pgfpathlineto{\pgfqpoint{5.873171in}{3.386859in}}%
\pgfpathlineto{\pgfqpoint{6.010937in}{3.326788in}}%
\pgfpathclose%
\pgfusepath{fill}%
\end{pgfscope}%
\begin{pgfscope}%
\pgfpathrectangle{\pgfqpoint{0.539299in}{0.078740in}}{\pgfqpoint{7.842520in}{7.842520in}}%
\pgfusepath{clip}%
\pgfsetbuttcap%
\pgfsetroundjoin%
\definecolor{currentfill}{rgb}{0.147607,0.511733,0.557049}%
\pgfsetfillcolor{currentfill}%
\pgfsetlinewidth{0.000000pt}%
\definecolor{currentstroke}{rgb}{0.496615,0.826376,0.306377}%
\pgfsetstrokecolor{currentstroke}%
\pgfsetdash{}{0pt}%
\pgfpathmoveto{\pgfqpoint{4.301131in}{4.661130in}}%
\pgfpathlineto{\pgfqpoint{4.357325in}{4.615401in}}%
\pgfpathlineto{\pgfqpoint{4.436631in}{4.511408in}}%
\pgfpathclose%
\pgfusepath{fill}%
\end{pgfscope}%
\begin{pgfscope}%
\pgfpathrectangle{\pgfqpoint{0.539299in}{0.078740in}}{\pgfqpoint{7.842520in}{7.842520in}}%
\pgfusepath{clip}%
\pgfsetbuttcap%
\pgfsetroundjoin%
\definecolor{currentfill}{rgb}{0.162142,0.474838,0.558140}%
\pgfsetfillcolor{currentfill}%
\pgfsetlinewidth{0.000000pt}%
\definecolor{currentstroke}{rgb}{0.506271,0.828786,0.300362}%
\pgfsetstrokecolor{currentstroke}%
\pgfsetdash{}{0pt}%
\pgfpathmoveto{\pgfqpoint{4.493544in}{4.454843in}}%
\pgfpathlineto{\pgfqpoint{4.572157in}{4.356552in}}%
\pgfpathlineto{\pgfqpoint{4.436631in}{4.511408in}}%
\pgfpathclose%
\pgfusepath{fill}%
\end{pgfscope}%
\begin{pgfscope}%
\pgfpathrectangle{\pgfqpoint{0.539299in}{0.078740in}}{\pgfqpoint{7.842520in}{7.842520in}}%
\pgfusepath{clip}%
\pgfsetbuttcap%
\pgfsetroundjoin%
\definecolor{currentfill}{rgb}{0.171176,0.452530,0.557965}%
\pgfsetfillcolor{currentfill}%
\pgfsetlinewidth{0.000000pt}%
\definecolor{currentstroke}{rgb}{0.515992,0.831158,0.294279}%
\pgfsetstrokecolor{currentstroke}%
\pgfsetdash{}{0pt}%
\pgfpathmoveto{\pgfqpoint{4.572157in}{4.356552in}}%
\pgfpathlineto{\pgfqpoint{4.493544in}{4.454843in}}%
\pgfpathlineto{\pgfqpoint{4.707691in}{4.203491in}}%
\pgfpathclose%
\pgfusepath{fill}%
\end{pgfscope}%
\begin{pgfscope}%
\pgfpathrectangle{\pgfqpoint{0.539299in}{0.078740in}}{\pgfqpoint{7.842520in}{7.842520in}}%
\pgfusepath{clip}%
\pgfsetbuttcap%
\pgfsetroundjoin%
\definecolor{currentfill}{rgb}{0.192357,0.403199,0.555836}%
\pgfsetfillcolor{currentfill}%
\pgfsetlinewidth{0.000000pt}%
\definecolor{currentstroke}{rgb}{0.525776,0.833491,0.288127}%
\pgfsetstrokecolor{currentstroke}%
\pgfsetdash{}{0pt}%
\pgfpathmoveto{\pgfqpoint{4.707691in}{4.203491in}}%
\pgfpathlineto{\pgfqpoint{4.765949in}{4.134630in}}%
\pgfpathlineto{\pgfqpoint{4.843266in}{4.057477in}}%
\pgfpathclose%
\pgfusepath{fill}%
\end{pgfscope}%
\begin{pgfscope}%
\pgfpathrectangle{\pgfqpoint{0.539299in}{0.078740in}}{\pgfqpoint{7.842520in}{7.842520in}}%
\pgfusepath{clip}%
\pgfsetbuttcap%
\pgfsetroundjoin%
\definecolor{currentfill}{rgb}{0.269944,0.014625,0.341379}%
\pgfsetfillcolor{currentfill}%
\pgfsetlinewidth{0.000000pt}%
\definecolor{currentstroke}{rgb}{0.535621,0.835785,0.281908}%
\pgfsetstrokecolor{currentstroke}%
\pgfsetdash{}{0pt}%
\pgfpathmoveto{\pgfqpoint{7.472102in}{2.763114in}}%
\pgfpathlineto{\pgfqpoint{7.541973in}{2.824378in}}%
\pgfpathlineto{\pgfqpoint{7.402749in}{2.891078in}}%
\pgfpathclose%
\pgfusepath{fill}%
\end{pgfscope}%
\begin{pgfscope}%
\pgfpathrectangle{\pgfqpoint{0.539299in}{0.078740in}}{\pgfqpoint{7.842520in}{7.842520in}}%
\pgfusepath{clip}%
\pgfsetbuttcap%
\pgfsetroundjoin%
\definecolor{currentfill}{rgb}{0.218130,0.347432,0.550038}%
\pgfsetfillcolor{currentfill}%
\pgfsetlinewidth{0.000000pt}%
\definecolor{currentstroke}{rgb}{0.545524,0.838039,0.275626}%
\pgfsetstrokecolor{currentstroke}%
\pgfsetdash{}{0pt}%
\pgfpathmoveto{\pgfqpoint{2.577164in}{4.260977in}}%
\pgfpathlineto{\pgfqpoint{2.459954in}{3.750514in}}%
\pgfpathlineto{\pgfqpoint{2.375504in}{3.692390in}}%
\pgfpathclose%
\pgfusepath{fill}%
\end{pgfscope}%
\begin{pgfscope}%
\pgfpathrectangle{\pgfqpoint{0.539299in}{0.078740in}}{\pgfqpoint{7.842520in}{7.842520in}}%
\pgfusepath{clip}%
\pgfsetbuttcap%
\pgfsetroundjoin%
\definecolor{currentfill}{rgb}{0.260571,0.246922,0.522828}%
\pgfsetfillcolor{currentfill}%
\pgfsetlinewidth{0.000000pt}%
\definecolor{currentstroke}{rgb}{0.555484,0.840254,0.269281}%
\pgfsetstrokecolor{currentstroke}%
\pgfsetdash{}{0pt}%
\pgfpathmoveto{\pgfqpoint{5.524083in}{3.506700in}}%
\pgfpathlineto{\pgfqpoint{5.598833in}{3.513257in}}%
\pgfpathlineto{\pgfqpoint{5.387297in}{3.592738in}}%
\pgfpathclose%
\pgfusepath{fill}%
\end{pgfscope}%
\begin{pgfscope}%
\pgfpathrectangle{\pgfqpoint{0.539299in}{0.078740in}}{\pgfqpoint{7.842520in}{7.842520in}}%
\pgfusepath{clip}%
\pgfsetbuttcap%
\pgfsetroundjoin%
\definecolor{currentfill}{rgb}{0.203063,0.379716,0.553925}%
\pgfsetfillcolor{currentfill}%
\pgfsetlinewidth{0.000000pt}%
\definecolor{currentstroke}{rgb}{0.565498,0.842430,0.262877}%
\pgfsetstrokecolor{currentstroke}%
\pgfsetdash{}{0pt}%
\pgfpathmoveto{\pgfqpoint{4.765949in}{4.134630in}}%
\pgfpathlineto{\pgfqpoint{4.978938in}{3.922091in}}%
\pgfpathlineto{\pgfqpoint{4.843266in}{4.057477in}}%
\pgfpathclose%
\pgfusepath{fill}%
\end{pgfscope}%
\begin{pgfscope}%
\pgfpathrectangle{\pgfqpoint{0.539299in}{0.078740in}}{\pgfqpoint{7.842520in}{7.842520in}}%
\pgfusepath{clip}%
\pgfsetbuttcap%
\pgfsetroundjoin%
\definecolor{currentfill}{rgb}{0.218130,0.347432,0.550038}%
\pgfsetfillcolor{currentfill}%
\pgfsetlinewidth{0.000000pt}%
\definecolor{currentstroke}{rgb}{0.575563,0.844566,0.256415}%
\pgfsetstrokecolor{currentstroke}%
\pgfsetdash{}{0pt}%
\pgfpathmoveto{\pgfqpoint{4.978938in}{3.922091in}}%
\pgfpathlineto{\pgfqpoint{4.902184in}{3.985093in}}%
\pgfpathlineto{\pgfqpoint{5.114785in}{3.799326in}}%
\pgfpathclose%
\pgfusepath{fill}%
\end{pgfscope}%
\begin{pgfscope}%
\pgfpathrectangle{\pgfqpoint{0.539299in}{0.078740in}}{\pgfqpoint{7.842520in}{7.842520in}}%
\pgfusepath{clip}%
\pgfsetbuttcap%
\pgfsetroundjoin%
\definecolor{currentfill}{rgb}{0.271305,0.019942,0.347269}%
\pgfsetfillcolor{currentfill}%
\pgfsetlinewidth{0.000000pt}%
\definecolor{currentstroke}{rgb}{0.585678,0.846661,0.249897}%
\pgfsetstrokecolor{currentstroke}%
\pgfsetdash{}{0pt}%
\pgfpathmoveto{\pgfqpoint{7.402749in}{2.891078in}}%
\pgfpathlineto{\pgfqpoint{7.332923in}{2.839586in}}%
\pgfpathlineto{\pgfqpoint{7.472102in}{2.763114in}}%
\pgfpathclose%
\pgfusepath{fill}%
\end{pgfscope}%
\begin{pgfscope}%
\pgfpathrectangle{\pgfqpoint{0.539299in}{0.078740in}}{\pgfqpoint{7.842520in}{7.842520in}}%
\pgfusepath{clip}%
\pgfsetbuttcap%
\pgfsetroundjoin%
\definecolor{currentfill}{rgb}{0.278012,0.180367,0.486697}%
\pgfsetfillcolor{currentfill}%
\pgfsetlinewidth{0.000000pt}%
\definecolor{currentstroke}{rgb}{0.595839,0.848717,0.243329}%
\pgfsetstrokecolor{currentstroke}%
\pgfsetdash{}{0pt}%
\pgfpathmoveto{\pgfqpoint{6.084512in}{3.361457in}}%
\pgfpathlineto{\pgfqpoint{6.149041in}{3.265181in}}%
\pgfpathlineto{\pgfqpoint{6.222345in}{3.303711in}}%
\pgfpathclose%
\pgfusepath{fill}%
\end{pgfscope}%
\begin{pgfscope}%
\pgfpathrectangle{\pgfqpoint{0.539299in}{0.078740in}}{\pgfqpoint{7.842520in}{7.842520in}}%
\pgfusepath{clip}%
\pgfsetbuttcap%
\pgfsetroundjoin%
\definecolor{currentfill}{rgb}{0.282623,0.140926,0.457517}%
\pgfsetfillcolor{currentfill}%
\pgfsetlinewidth{0.000000pt}%
\definecolor{currentstroke}{rgb}{0.606045,0.850733,0.236712}%
\pgfsetstrokecolor{currentstroke}%
\pgfsetdash{}{0pt}%
\pgfpathmoveto{\pgfqpoint{6.570734in}{3.206547in}}%
\pgfpathlineto{\pgfqpoint{6.498702in}{3.171293in}}%
\pgfpathlineto{\pgfqpoint{6.708800in}{3.131236in}}%
\pgfpathclose%
\pgfusepath{fill}%
\end{pgfscope}%
\begin{pgfscope}%
\pgfpathrectangle{\pgfqpoint{0.539299in}{0.078740in}}{\pgfqpoint{7.842520in}{7.842520in}}%
\pgfusepath{clip}%
\pgfsetbuttcap%
\pgfsetroundjoin%
\definecolor{currentfill}{rgb}{0.120565,0.596422,0.543611}%
\pgfsetfillcolor{currentfill}%
\pgfsetlinewidth{0.000000pt}%
\definecolor{currentstroke}{rgb}{0.616293,0.852709,0.230052}%
\pgfsetstrokecolor{currentstroke}%
\pgfsetdash{}{0pt}%
\pgfpathmoveto{\pgfqpoint{4.085083in}{4.900621in}}%
\pgfpathlineto{\pgfqpoint{4.030569in}{4.909838in}}%
\pgfpathlineto{\pgfqpoint{3.949347in}{5.006539in}}%
\pgfpathclose%
\pgfusepath{fill}%
\end{pgfscope}%
\begin{pgfscope}%
\pgfpathrectangle{\pgfqpoint{0.539299in}{0.078740in}}{\pgfqpoint{7.842520in}{7.842520in}}%
\pgfusepath{clip}%
\pgfsetbuttcap%
\pgfsetroundjoin%
\definecolor{currentfill}{rgb}{0.120638,0.625828,0.533488}%
\pgfsetfillcolor{currentfill}%
\pgfsetlinewidth{0.000000pt}%
\definecolor{currentstroke}{rgb}{0.626579,0.854645,0.223353}%
\pgfsetstrokecolor{currentstroke}%
\pgfsetdash{}{0pt}%
\pgfpathmoveto{\pgfqpoint{3.464190in}{5.097259in}}%
\pgfpathlineto{\pgfqpoint{3.333165in}{4.962328in}}%
\pgfpathlineto{\pgfqpoint{3.380711in}{5.141597in}}%
\pgfpathclose%
\pgfusepath{fill}%
\end{pgfscope}%
\begin{pgfscope}%
\pgfpathrectangle{\pgfqpoint{0.539299in}{0.078740in}}{\pgfqpoint{7.842520in}{7.842520in}}%
\pgfusepath{clip}%
\pgfsetbuttcap%
\pgfsetroundjoin%
\definecolor{currentfill}{rgb}{0.278791,0.062145,0.386592}%
\pgfsetfillcolor{currentfill}%
\pgfsetlinewidth{0.000000pt}%
\definecolor{currentstroke}{rgb}{0.636902,0.856542,0.216620}%
\pgfsetstrokecolor{currentstroke}%
\pgfsetdash{}{0pt}%
\pgfpathmoveto{\pgfqpoint{7.055833in}{3.001854in}}%
\pgfpathlineto{\pgfqpoint{7.123901in}{2.876974in}}%
\pgfpathlineto{\pgfqpoint{7.194201in}{2.919934in}}%
\pgfpathclose%
\pgfusepath{fill}%
\end{pgfscope}%
\begin{pgfscope}%
\pgfpathrectangle{\pgfqpoint{0.539299in}{0.078740in}}{\pgfqpoint{7.842520in}{7.842520in}}%
\pgfusepath{clip}%
\pgfsetbuttcap%
\pgfsetroundjoin%
\definecolor{currentfill}{rgb}{0.119699,0.618490,0.536347}%
\pgfsetfillcolor{currentfill}%
\pgfsetlinewidth{0.000000pt}%
\definecolor{currentstroke}{rgb}{0.647257,0.858400,0.209861}%
\pgfsetstrokecolor{currentstroke}%
\pgfsetdash{}{0pt}%
\pgfpathmoveto{\pgfqpoint{3.380711in}{5.141597in}}%
\pgfpathlineto{\pgfqpoint{3.333165in}{4.962328in}}%
\pgfpathlineto{\pgfqpoint{3.249818in}{4.985619in}}%
\pgfpathclose%
\pgfusepath{fill}%
\end{pgfscope}%
\begin{pgfscope}%
\pgfpathrectangle{\pgfqpoint{0.539299in}{0.078740in}}{\pgfqpoint{7.842520in}{7.842520in}}%
\pgfusepath{clip}%
\pgfsetbuttcap%
\pgfsetroundjoin%
\definecolor{currentfill}{rgb}{0.281924,0.089666,0.412415}%
\pgfsetfillcolor{currentfill}%
\pgfsetlinewidth{0.000000pt}%
\definecolor{currentstroke}{rgb}{0.657642,0.860219,0.203082}%
\pgfsetstrokecolor{currentstroke}%
\pgfsetdash{}{0pt}%
\pgfpathmoveto{\pgfqpoint{6.985354in}{2.964798in}}%
\pgfpathlineto{\pgfqpoint{7.055833in}{3.001854in}}%
\pgfpathlineto{\pgfqpoint{6.847001in}{3.050168in}}%
\pgfpathclose%
\pgfusepath{fill}%
\end{pgfscope}%
\begin{pgfscope}%
\pgfpathrectangle{\pgfqpoint{0.539299in}{0.078740in}}{\pgfqpoint{7.842520in}{7.842520in}}%
\pgfusepath{clip}%
\pgfsetbuttcap%
\pgfsetroundjoin%
\definecolor{currentfill}{rgb}{0.281412,0.155834,0.469201}%
\pgfsetfillcolor{currentfill}%
\pgfsetlinewidth{0.000000pt}%
\definecolor{currentstroke}{rgb}{0.668054,0.861999,0.196293}%
\pgfsetstrokecolor{currentstroke}%
\pgfsetdash{}{0pt}%
\pgfpathmoveto{\pgfqpoint{6.360427in}{3.240693in}}%
\pgfpathlineto{\pgfqpoint{6.498702in}{3.171293in}}%
\pgfpathlineto{\pgfqpoint{6.570734in}{3.206547in}}%
\pgfpathclose%
\pgfusepath{fill}%
\end{pgfscope}%
\begin{pgfscope}%
\pgfpathrectangle{\pgfqpoint{0.539299in}{0.078740in}}{\pgfqpoint{7.842520in}{7.842520in}}%
\pgfusepath{clip}%
\pgfsetbuttcap%
\pgfsetroundjoin%
\definecolor{currentfill}{rgb}{0.235526,0.309527,0.542944}%
\pgfsetfillcolor{currentfill}%
\pgfsetlinewidth{0.000000pt}%
\definecolor{currentstroke}{rgb}{0.678489,0.863742,0.189503}%
\pgfsetstrokecolor{currentstroke}%
\pgfsetdash{}{0pt}%
\pgfpathmoveto{\pgfqpoint{5.175033in}{3.721001in}}%
\pgfpathlineto{\pgfqpoint{5.250882in}{3.689752in}}%
\pgfpathlineto{\pgfqpoint{5.114785in}{3.799326in}}%
\pgfpathclose%
\pgfusepath{fill}%
\end{pgfscope}%
\begin{pgfscope}%
\pgfpathrectangle{\pgfqpoint{0.539299in}{0.078740in}}{\pgfqpoint{7.842520in}{7.842520in}}%
\pgfusepath{clip}%
\pgfsetbuttcap%
\pgfsetroundjoin%
\definecolor{currentfill}{rgb}{0.192357,0.403199,0.555836}%
\pgfsetfillcolor{currentfill}%
\pgfsetlinewidth{0.000000pt}%
\definecolor{currentstroke}{rgb}{0.688944,0.865448,0.182725}%
\pgfsetstrokecolor{currentstroke}%
\pgfsetdash{}{0pt}%
\pgfpathmoveto{\pgfqpoint{2.459954in}{3.750514in}}%
\pgfpathlineto{\pgfqpoint{2.577164in}{4.260977in}}%
\pgfpathlineto{\pgfqpoint{2.661974in}{4.301274in}}%
\pgfpathclose%
\pgfusepath{fill}%
\end{pgfscope}%
\begin{pgfscope}%
\pgfpathrectangle{\pgfqpoint{0.539299in}{0.078740in}}{\pgfqpoint{7.842520in}{7.842520in}}%
\pgfusepath{clip}%
\pgfsetbuttcap%
\pgfsetroundjoin%
\definecolor{currentfill}{rgb}{0.265145,0.232956,0.516599}%
\pgfsetfillcolor{currentfill}%
\pgfsetlinewidth{0.000000pt}%
\definecolor{currentstroke}{rgb}{0.699415,0.867117,0.175971}%
\pgfsetstrokecolor{currentstroke}%
\pgfsetdash{}{0pt}%
\pgfpathmoveto{\pgfqpoint{5.524083in}{3.506700in}}%
\pgfpathlineto{\pgfqpoint{5.661270in}{3.429383in}}%
\pgfpathlineto{\pgfqpoint{5.735793in}{3.448039in}}%
\pgfpathclose%
\pgfusepath{fill}%
\end{pgfscope}%
\begin{pgfscope}%
\pgfpathrectangle{\pgfqpoint{0.539299in}{0.078740in}}{\pgfqpoint{7.842520in}{7.842520in}}%
\pgfusepath{clip}%
\pgfsetbuttcap%
\pgfsetroundjoin%
\definecolor{currentfill}{rgb}{0.124780,0.640461,0.527068}%
\pgfsetfillcolor{currentfill}%
\pgfsetlinewidth{0.000000pt}%
\definecolor{currentstroke}{rgb}{0.709898,0.868751,0.169257}%
\pgfsetstrokecolor{currentstroke}%
\pgfsetdash{}{0pt}%
\pgfpathmoveto{\pgfqpoint{3.679940in}{5.091602in}}%
\pgfpathlineto{\pgfqpoint{3.464190in}{5.097259in}}%
\pgfpathlineto{\pgfqpoint{3.597214in}{5.159015in}}%
\pgfpathclose%
\pgfusepath{fill}%
\end{pgfscope}%
\begin{pgfscope}%
\pgfpathrectangle{\pgfqpoint{0.539299in}{0.078740in}}{\pgfqpoint{7.842520in}{7.842520in}}%
\pgfusepath{clip}%
\pgfsetbuttcap%
\pgfsetroundjoin%
\definecolor{currentfill}{rgb}{0.270595,0.214069,0.507052}%
\pgfsetfillcolor{currentfill}%
\pgfsetlinewidth{0.000000pt}%
\definecolor{currentstroke}{rgb}{0.720391,0.870350,0.162603}%
\pgfsetstrokecolor{currentstroke}%
\pgfsetdash{}{0pt}%
\pgfpathmoveto{\pgfqpoint{5.798863in}{3.358134in}}%
\pgfpathlineto{\pgfqpoint{5.873171in}{3.386859in}}%
\pgfpathlineto{\pgfqpoint{5.735793in}{3.448039in}}%
\pgfpathclose%
\pgfusepath{fill}%
\end{pgfscope}%
\begin{pgfscope}%
\pgfpathrectangle{\pgfqpoint{0.539299in}{0.078740in}}{\pgfqpoint{7.842520in}{7.842520in}}%
\pgfusepath{clip}%
\pgfsetbuttcap%
\pgfsetroundjoin%
\definecolor{currentfill}{rgb}{0.276194,0.190074,0.493001}%
\pgfsetfillcolor{currentfill}%
\pgfsetlinewidth{0.000000pt}%
\definecolor{currentstroke}{rgb}{0.730889,0.871916,0.156029}%
\pgfsetstrokecolor{currentstroke}%
\pgfsetdash{}{0pt}%
\pgfpathmoveto{\pgfqpoint{6.010937in}{3.326788in}}%
\pgfpathlineto{\pgfqpoint{6.149041in}{3.265181in}}%
\pgfpathlineto{\pgfqpoint{6.084512in}{3.361457in}}%
\pgfpathclose%
\pgfusepath{fill}%
\end{pgfscope}%
\begin{pgfscope}%
\pgfpathrectangle{\pgfqpoint{0.539299in}{0.078740in}}{\pgfqpoint{7.842520in}{7.842520in}}%
\pgfusepath{clip}%
\pgfsetbuttcap%
\pgfsetroundjoin%
\definecolor{currentfill}{rgb}{0.128729,0.563265,0.551229}%
\pgfsetfillcolor{currentfill}%
\pgfsetlinewidth{0.000000pt}%
\definecolor{currentstroke}{rgb}{0.741388,0.873449,0.149561}%
\pgfsetstrokecolor{currentstroke}%
\pgfsetdash{}{0pt}%
\pgfpathmoveto{\pgfqpoint{4.221135in}{4.766924in}}%
\pgfpathlineto{\pgfqpoint{4.301131in}{4.661130in}}%
\pgfpathlineto{\pgfqpoint{4.085083in}{4.900621in}}%
\pgfpathclose%
\pgfusepath{fill}%
\end{pgfscope}%
\begin{pgfscope}%
\pgfpathrectangle{\pgfqpoint{0.539299in}{0.078740in}}{\pgfqpoint{7.842520in}{7.842520in}}%
\pgfusepath{clip}%
\pgfsetbuttcap%
\pgfsetroundjoin%
\definecolor{currentfill}{rgb}{0.243113,0.292092,0.538516}%
\pgfsetfillcolor{currentfill}%
\pgfsetlinewidth{0.000000pt}%
\definecolor{currentstroke}{rgb}{0.751884,0.874951,0.143228}%
\pgfsetstrokecolor{currentstroke}%
\pgfsetdash{}{0pt}%
\pgfpathmoveto{\pgfqpoint{5.250882in}{3.689752in}}%
\pgfpathlineto{\pgfqpoint{5.175033in}{3.721001in}}%
\pgfpathlineto{\pgfqpoint{5.387297in}{3.592738in}}%
\pgfpathclose%
\pgfusepath{fill}%
\end{pgfscope}%
\begin{pgfscope}%
\pgfpathrectangle{\pgfqpoint{0.539299in}{0.078740in}}{\pgfqpoint{7.842520in}{7.842520in}}%
\pgfusepath{clip}%
\pgfsetbuttcap%
\pgfsetroundjoin%
\definecolor{currentfill}{rgb}{0.278826,0.175490,0.483397}%
\pgfsetfillcolor{currentfill}%
\pgfsetlinewidth{0.000000pt}%
\definecolor{currentstroke}{rgb}{0.762373,0.876424,0.137064}%
\pgfsetstrokecolor{currentstroke}%
\pgfsetdash{}{0pt}%
\pgfpathmoveto{\pgfqpoint{6.222345in}{3.303711in}}%
\pgfpathlineto{\pgfqpoint{6.149041in}{3.265181in}}%
\pgfpathlineto{\pgfqpoint{6.360427in}{3.240693in}}%
\pgfpathclose%
\pgfusepath{fill}%
\end{pgfscope}%
\begin{pgfscope}%
\pgfpathrectangle{\pgfqpoint{0.539299in}{0.078740in}}{\pgfqpoint{7.842520in}{7.842520in}}%
\pgfusepath{clip}%
\pgfsetbuttcap%
\pgfsetroundjoin%
\definecolor{currentfill}{rgb}{0.121148,0.592739,0.544641}%
\pgfsetfillcolor{currentfill}%
\pgfsetlinewidth{0.000000pt}%
\definecolor{currentstroke}{rgb}{0.772852,0.877868,0.131109}%
\pgfsetstrokecolor{currentstroke}%
\pgfsetdash{}{0pt}%
\pgfpathmoveto{\pgfqpoint{3.249818in}{4.985619in}}%
\pgfpathlineto{\pgfqpoint{3.038103in}{4.744324in}}%
\pgfpathlineto{\pgfqpoint{3.165894in}{5.004406in}}%
\pgfpathclose%
\pgfusepath{fill}%
\end{pgfscope}%
\begin{pgfscope}%
\pgfpathrectangle{\pgfqpoint{0.539299in}{0.078740in}}{\pgfqpoint{7.842520in}{7.842520in}}%
\pgfusepath{clip}%
\pgfsetbuttcap%
\pgfsetroundjoin%
\definecolor{currentfill}{rgb}{0.136408,0.541173,0.554483}%
\pgfsetfillcolor{currentfill}%
\pgfsetlinewidth{0.000000pt}%
\definecolor{currentstroke}{rgb}{0.783315,0.879285,0.125405}%
\pgfsetstrokecolor{currentstroke}%
\pgfsetdash{}{0pt}%
\pgfpathmoveto{\pgfqpoint{4.221135in}{4.766924in}}%
\pgfpathlineto{\pgfqpoint{4.357325in}{4.615401in}}%
\pgfpathlineto{\pgfqpoint{4.301131in}{4.661130in}}%
\pgfpathclose%
\pgfusepath{fill}%
\end{pgfscope}%
\begin{pgfscope}%
\pgfpathrectangle{\pgfqpoint{0.539299in}{0.078740in}}{\pgfqpoint{7.842520in}{7.842520in}}%
\pgfusepath{clip}%
\pgfsetbuttcap%
\pgfsetroundjoin%
\definecolor{currentfill}{rgb}{0.180629,0.429975,0.557282}%
\pgfsetfillcolor{currentfill}%
\pgfsetlinewidth{0.000000pt}%
\definecolor{currentstroke}{rgb}{0.793760,0.880678,0.120005}%
\pgfsetstrokecolor{currentstroke}%
\pgfsetdash{}{0pt}%
\pgfpathmoveto{\pgfqpoint{4.629750in}{4.292637in}}%
\pgfpathlineto{\pgfqpoint{4.765949in}{4.134630in}}%
\pgfpathlineto{\pgfqpoint{4.707691in}{4.203491in}}%
\pgfpathclose%
\pgfusepath{fill}%
\end{pgfscope}%
\begin{pgfscope}%
\pgfpathrectangle{\pgfqpoint{0.539299in}{0.078740in}}{\pgfqpoint{7.842520in}{7.842520in}}%
\pgfusepath{clip}%
\pgfsetbuttcap%
\pgfsetroundjoin%
\definecolor{currentfill}{rgb}{0.169646,0.456262,0.558030}%
\pgfsetfillcolor{currentfill}%
\pgfsetlinewidth{0.000000pt}%
\definecolor{currentstroke}{rgb}{0.804182,0.882046,0.114965}%
\pgfsetstrokecolor{currentstroke}%
\pgfsetdash{}{0pt}%
\pgfpathmoveto{\pgfqpoint{4.707691in}{4.203491in}}%
\pgfpathlineto{\pgfqpoint{4.493544in}{4.454843in}}%
\pgfpathlineto{\pgfqpoint{4.629750in}{4.292637in}}%
\pgfpathclose%
\pgfusepath{fill}%
\end{pgfscope}%
\begin{pgfscope}%
\pgfpathrectangle{\pgfqpoint{0.539299in}{0.078740in}}{\pgfqpoint{7.842520in}{7.842520in}}%
\pgfusepath{clip}%
\pgfsetbuttcap%
\pgfsetroundjoin%
\definecolor{currentfill}{rgb}{0.277018,0.050344,0.375715}%
\pgfsetfillcolor{currentfill}%
\pgfsetlinewidth{0.000000pt}%
\definecolor{currentstroke}{rgb}{0.814576,0.883393,0.110347}%
\pgfsetstrokecolor{currentstroke}%
\pgfsetdash{}{0pt}%
\pgfpathmoveto{\pgfqpoint{7.332923in}{2.839586in}}%
\pgfpathlineto{\pgfqpoint{7.194201in}{2.919934in}}%
\pgfpathlineto{\pgfqpoint{7.123901in}{2.876974in}}%
\pgfpathclose%
\pgfusepath{fill}%
\end{pgfscope}%
\begin{pgfscope}%
\pgfpathrectangle{\pgfqpoint{0.539299in}{0.078740in}}{\pgfqpoint{7.842520in}{7.842520in}}%
\pgfusepath{clip}%
\pgfsetbuttcap%
\pgfsetroundjoin%
\definecolor{currentfill}{rgb}{0.150476,0.504369,0.557430}%
\pgfsetfillcolor{currentfill}%
\pgfsetlinewidth{0.000000pt}%
\definecolor{currentstroke}{rgb}{0.824940,0.884720,0.106217}%
\pgfsetstrokecolor{currentstroke}%
\pgfsetdash{}{0pt}%
\pgfpathmoveto{\pgfqpoint{4.436631in}{4.511408in}}%
\pgfpathlineto{\pgfqpoint{4.357325in}{4.615401in}}%
\pgfpathlineto{\pgfqpoint{4.493544in}{4.454843in}}%
\pgfpathclose%
\pgfusepath{fill}%
\end{pgfscope}%
\begin{pgfscope}%
\pgfpathrectangle{\pgfqpoint{0.539299in}{0.078740in}}{\pgfqpoint{7.842520in}{7.842520in}}%
\pgfusepath{clip}%
\pgfsetbuttcap%
\pgfsetroundjoin%
\definecolor{currentfill}{rgb}{0.280267,0.073417,0.397163}%
\pgfsetfillcolor{currentfill}%
\pgfsetlinewidth{0.000000pt}%
\definecolor{currentstroke}{rgb}{0.835270,0.886029,0.102646}%
\pgfsetstrokecolor{currentstroke}%
\pgfsetdash{}{0pt}%
\pgfpathmoveto{\pgfqpoint{6.985354in}{2.964798in}}%
\pgfpathlineto{\pgfqpoint{7.123901in}{2.876974in}}%
\pgfpathlineto{\pgfqpoint{7.055833in}{3.001854in}}%
\pgfpathclose%
\pgfusepath{fill}%
\end{pgfscope}%
\begin{pgfscope}%
\pgfpathrectangle{\pgfqpoint{0.539299in}{0.078740in}}{\pgfqpoint{7.842520in}{7.842520in}}%
\pgfusepath{clip}%
\pgfsetbuttcap%
\pgfsetroundjoin%
\definecolor{currentfill}{rgb}{0.201239,0.383670,0.554294}%
\pgfsetfillcolor{currentfill}%
\pgfsetlinewidth{0.000000pt}%
\definecolor{currentstroke}{rgb}{0.845561,0.887322,0.099702}%
\pgfsetstrokecolor{currentstroke}%
\pgfsetdash{}{0pt}%
\pgfpathmoveto{\pgfqpoint{4.765949in}{4.134630in}}%
\pgfpathlineto{\pgfqpoint{4.902184in}{3.985093in}}%
\pgfpathlineto{\pgfqpoint{4.978938in}{3.922091in}}%
\pgfpathclose%
\pgfusepath{fill}%
\end{pgfscope}%
\begin{pgfscope}%
\pgfpathrectangle{\pgfqpoint{0.539299in}{0.078740in}}{\pgfqpoint{7.842520in}{7.842520in}}%
\pgfusepath{clip}%
\pgfsetbuttcap%
\pgfsetroundjoin%
\definecolor{currentfill}{rgb}{0.283187,0.125848,0.444960}%
\pgfsetfillcolor{currentfill}%
\pgfsetlinewidth{0.000000pt}%
\definecolor{currentstroke}{rgb}{0.855810,0.888601,0.097452}%
\pgfsetstrokecolor{currentstroke}%
\pgfsetdash{}{0pt}%
\pgfpathmoveto{\pgfqpoint{6.847001in}{3.050168in}}%
\pgfpathlineto{\pgfqpoint{6.708800in}{3.131236in}}%
\pgfpathlineto{\pgfqpoint{6.637123in}{3.095135in}}%
\pgfpathclose%
\pgfusepath{fill}%
\end{pgfscope}%
\begin{pgfscope}%
\pgfpathrectangle{\pgfqpoint{0.539299in}{0.078740in}}{\pgfqpoint{7.842520in}{7.842520in}}%
\pgfusepath{clip}%
\pgfsetbuttcap%
\pgfsetroundjoin%
\definecolor{currentfill}{rgb}{0.267968,0.223549,0.512008}%
\pgfsetfillcolor{currentfill}%
\pgfsetlinewidth{0.000000pt}%
\definecolor{currentstroke}{rgb}{0.866013,0.889868,0.095953}%
\pgfsetstrokecolor{currentstroke}%
\pgfsetdash{}{0pt}%
\pgfpathmoveto{\pgfqpoint{5.735793in}{3.448039in}}%
\pgfpathlineto{\pgfqpoint{5.661270in}{3.429383in}}%
\pgfpathlineto{\pgfqpoint{5.798863in}{3.358134in}}%
\pgfpathclose%
\pgfusepath{fill}%
\end{pgfscope}%
\begin{pgfscope}%
\pgfpathrectangle{\pgfqpoint{0.539299in}{0.078740in}}{\pgfqpoint{7.842520in}{7.842520in}}%
\pgfusepath{clip}%
\pgfsetbuttcap%
\pgfsetroundjoin%
\definecolor{currentfill}{rgb}{0.124780,0.640461,0.527068}%
\pgfsetfillcolor{currentfill}%
\pgfsetlinewidth{0.000000pt}%
\definecolor{currentstroke}{rgb}{0.876168,0.891125,0.095250}%
\pgfsetstrokecolor{currentstroke}%
\pgfsetdash{}{0pt}%
\pgfpathmoveto{\pgfqpoint{3.679940in}{5.091602in}}%
\pgfpathlineto{\pgfqpoint{3.731742in}{5.157127in}}%
\pgfpathlineto{\pgfqpoint{3.814186in}{5.073908in}}%
\pgfpathclose%
\pgfusepath{fill}%
\end{pgfscope}%
\begin{pgfscope}%
\pgfpathrectangle{\pgfqpoint{0.539299in}{0.078740in}}{\pgfqpoint{7.842520in}{7.842520in}}%
\pgfusepath{clip}%
\pgfsetbuttcap%
\pgfsetroundjoin%
\definecolor{currentfill}{rgb}{0.216210,0.351535,0.550627}%
\pgfsetfillcolor{currentfill}%
\pgfsetlinewidth{0.000000pt}%
\definecolor{currentstroke}{rgb}{0.886271,0.892374,0.095374}%
\pgfsetstrokecolor{currentstroke}%
\pgfsetdash{}{0pt}%
\pgfpathmoveto{\pgfqpoint{5.114785in}{3.799326in}}%
\pgfpathlineto{\pgfqpoint{4.902184in}{3.985093in}}%
\pgfpathlineto{\pgfqpoint{5.038520in}{3.846770in}}%
\pgfpathclose%
\pgfusepath{fill}%
\end{pgfscope}%
\begin{pgfscope}%
\pgfpathrectangle{\pgfqpoint{0.539299in}{0.078740in}}{\pgfqpoint{7.842520in}{7.842520in}}%
\pgfusepath{clip}%
\pgfsetbuttcap%
\pgfsetroundjoin%
\definecolor{currentfill}{rgb}{0.282623,0.140926,0.457517}%
\pgfsetfillcolor{currentfill}%
\pgfsetlinewidth{0.000000pt}%
\definecolor{currentstroke}{rgb}{0.896320,0.893616,0.096335}%
\pgfsetstrokecolor{currentstroke}%
\pgfsetdash{}{0pt}%
\pgfpathmoveto{\pgfqpoint{6.708800in}{3.131236in}}%
\pgfpathlineto{\pgfqpoint{6.498702in}{3.171293in}}%
\pgfpathlineto{\pgfqpoint{6.637123in}{3.095135in}}%
\pgfpathclose%
\pgfusepath{fill}%
\end{pgfscope}%
\begin{pgfscope}%
\pgfpathrectangle{\pgfqpoint{0.539299in}{0.078740in}}{\pgfqpoint{7.842520in}{7.842520in}}%
\pgfusepath{clip}%
\pgfsetbuttcap%
\pgfsetroundjoin%
\definecolor{currentfill}{rgb}{0.124395,0.578002,0.548287}%
\pgfsetfillcolor{currentfill}%
\pgfsetlinewidth{0.000000pt}%
\definecolor{currentstroke}{rgb}{0.906311,0.894855,0.098125}%
\pgfsetstrokecolor{currentstroke}%
\pgfsetdash{}{0pt}%
\pgfpathmoveto{\pgfqpoint{3.165894in}{5.004406in}}%
\pgfpathlineto{\pgfqpoint{3.038103in}{4.744324in}}%
\pgfpathlineto{\pgfqpoint{2.954036in}{4.737238in}}%
\pgfpathclose%
\pgfusepath{fill}%
\end{pgfscope}%
\begin{pgfscope}%
\pgfpathrectangle{\pgfqpoint{0.539299in}{0.078740in}}{\pgfqpoint{7.842520in}{7.842520in}}%
\pgfusepath{clip}%
\pgfsetbuttcap%
\pgfsetroundjoin%
\definecolor{currentfill}{rgb}{0.225863,0.330805,0.547314}%
\pgfsetfillcolor{currentfill}%
\pgfsetlinewidth{0.000000pt}%
\definecolor{currentstroke}{rgb}{0.916242,0.896091,0.100717}%
\pgfsetstrokecolor{currentstroke}%
\pgfsetdash{}{0pt}%
\pgfpathmoveto{\pgfqpoint{5.175033in}{3.721001in}}%
\pgfpathlineto{\pgfqpoint{5.114785in}{3.799326in}}%
\pgfpathlineto{\pgfqpoint{5.038520in}{3.846770in}}%
\pgfpathclose%
\pgfusepath{fill}%
\end{pgfscope}%
\begin{pgfscope}%
\pgfpathrectangle{\pgfqpoint{0.539299in}{0.078740in}}{\pgfqpoint{7.842520in}{7.842520in}}%
\pgfusepath{clip}%
\pgfsetbuttcap%
\pgfsetroundjoin%
\definecolor{currentfill}{rgb}{0.273006,0.204520,0.501721}%
\pgfsetfillcolor{currentfill}%
\pgfsetlinewidth{0.000000pt}%
\definecolor{currentstroke}{rgb}{0.926106,0.897330,0.104071}%
\pgfsetstrokecolor{currentstroke}%
\pgfsetdash{}{0pt}%
\pgfpathmoveto{\pgfqpoint{5.873171in}{3.386859in}}%
\pgfpathlineto{\pgfqpoint{5.936847in}{3.290164in}}%
\pgfpathlineto{\pgfqpoint{6.010937in}{3.326788in}}%
\pgfpathclose%
\pgfusepath{fill}%
\end{pgfscope}%
\begin{pgfscope}%
\pgfpathrectangle{\pgfqpoint{0.539299in}{0.078740in}}{\pgfqpoint{7.842520in}{7.842520in}}%
\pgfusepath{clip}%
\pgfsetbuttcap%
\pgfsetroundjoin%
\definecolor{currentfill}{rgb}{0.253935,0.265254,0.529983}%
\pgfsetfillcolor{currentfill}%
\pgfsetlinewidth{0.000000pt}%
\definecolor{currentstroke}{rgb}{0.935904,0.898570,0.108131}%
\pgfsetstrokecolor{currentstroke}%
\pgfsetdash{}{0pt}%
\pgfpathmoveto{\pgfqpoint{5.387297in}{3.592738in}}%
\pgfpathlineto{\pgfqpoint{5.448865in}{3.506545in}}%
\pgfpathlineto{\pgfqpoint{5.524083in}{3.506700in}}%
\pgfpathclose%
\pgfusepath{fill}%
\end{pgfscope}%
\begin{pgfscope}%
\pgfpathrectangle{\pgfqpoint{0.539299in}{0.078740in}}{\pgfqpoint{7.842520in}{7.842520in}}%
\pgfusepath{clip}%
\pgfsetbuttcap%
\pgfsetroundjoin%
\definecolor{currentfill}{rgb}{0.141935,0.526453,0.555991}%
\pgfsetfillcolor{currentfill}%
\pgfsetlinewidth{0.000000pt}%
\definecolor{currentstroke}{rgb}{0.945636,0.899815,0.112838}%
\pgfsetstrokecolor{currentstroke}%
\pgfsetdash{}{0pt}%
\pgfpathmoveto{\pgfqpoint{2.954036in}{4.737238in}}%
\pgfpathlineto{\pgfqpoint{2.746363in}{4.336760in}}%
\pgfpathlineto{\pgfqpoint{2.869477in}{4.725189in}}%
\pgfpathclose%
\pgfusepath{fill}%
\end{pgfscope}%
\begin{pgfscope}%
\pgfpathrectangle{\pgfqpoint{0.539299in}{0.078740in}}{\pgfqpoint{7.842520in}{7.842520in}}%
\pgfusepath{clip}%
\pgfsetbuttcap%
\pgfsetroundjoin%
\definecolor{currentfill}{rgb}{0.130067,0.651384,0.521608}%
\pgfsetfillcolor{currentfill}%
\pgfsetlinewidth{0.000000pt}%
\definecolor{currentstroke}{rgb}{0.955300,0.901065,0.118128}%
\pgfsetstrokecolor{currentstroke}%
\pgfsetdash{}{0pt}%
\pgfpathmoveto{\pgfqpoint{3.597214in}{5.159015in}}%
\pgfpathlineto{\pgfqpoint{3.731742in}{5.157127in}}%
\pgfpathlineto{\pgfqpoint{3.679940in}{5.091602in}}%
\pgfpathclose%
\pgfusepath{fill}%
\end{pgfscope}%
\begin{pgfscope}%
\pgfpathrectangle{\pgfqpoint{0.539299in}{0.078740in}}{\pgfqpoint{7.842520in}{7.842520in}}%
\pgfusepath{clip}%
\pgfsetbuttcap%
\pgfsetroundjoin%
\definecolor{currentfill}{rgb}{0.271305,0.019942,0.347269}%
\pgfsetfillcolor{currentfill}%
\pgfsetlinewidth{0.000000pt}%
\definecolor{currentstroke}{rgb}{0.964894,0.902323,0.123941}%
\pgfsetstrokecolor{currentstroke}%
\pgfsetdash{}{0pt}%
\pgfpathmoveto{\pgfqpoint{7.472102in}{2.763114in}}%
\pgfpathlineto{\pgfqpoint{7.332923in}{2.839586in}}%
\pgfpathlineto{\pgfqpoint{7.401863in}{2.702549in}}%
\pgfpathclose%
\pgfusepath{fill}%
\end{pgfscope}%
\begin{pgfscope}%
\pgfpathrectangle{\pgfqpoint{0.539299in}{0.078740in}}{\pgfqpoint{7.842520in}{7.842520in}}%
\pgfusepath{clip}%
\pgfsetbuttcap%
\pgfsetroundjoin%
\definecolor{currentfill}{rgb}{0.276022,0.044167,0.370164}%
\pgfsetfillcolor{currentfill}%
\pgfsetlinewidth{0.000000pt}%
\definecolor{currentstroke}{rgb}{0.974417,0.903590,0.130215}%
\pgfsetstrokecolor{currentstroke}%
\pgfsetdash{}{0pt}%
\pgfpathmoveto{\pgfqpoint{7.262707in}{2.788802in}}%
\pgfpathlineto{\pgfqpoint{7.332923in}{2.839586in}}%
\pgfpathlineto{\pgfqpoint{7.123901in}{2.876974in}}%
\pgfpathclose%
\pgfusepath{fill}%
\end{pgfscope}%
\begin{pgfscope}%
\pgfpathrectangle{\pgfqpoint{0.539299in}{0.078740in}}{\pgfqpoint{7.842520in}{7.842520in}}%
\pgfusepath{clip}%
\pgfsetbuttcap%
\pgfsetroundjoin%
\definecolor{currentfill}{rgb}{0.278012,0.180367,0.486697}%
\pgfsetfillcolor{currentfill}%
\pgfsetlinewidth{0.000000pt}%
\definecolor{currentstroke}{rgb}{0.983868,0.904867,0.136897}%
\pgfsetstrokecolor{currentstroke}%
\pgfsetdash{}{0pt}%
\pgfpathmoveto{\pgfqpoint{6.360427in}{3.240693in}}%
\pgfpathlineto{\pgfqpoint{6.149041in}{3.265181in}}%
\pgfpathlineto{\pgfqpoint{6.287425in}{3.199853in}}%
\pgfpathclose%
\pgfusepath{fill}%
\end{pgfscope}%
\begin{pgfscope}%
\pgfpathrectangle{\pgfqpoint{0.539299in}{0.078740in}}{\pgfqpoint{7.842520in}{7.842520in}}%
\pgfusepath{clip}%
\pgfsetbuttcap%
\pgfsetroundjoin%
\definecolor{currentfill}{rgb}{0.216210,0.351535,0.550627}%
\pgfsetfillcolor{currentfill}%
\pgfsetlinewidth{0.000000pt}%
\definecolor{currentstroke}{rgb}{0.993248,0.906157,0.143936}%
\pgfsetstrokecolor{currentstroke}%
\pgfsetdash{}{0pt}%
\pgfpathmoveto{\pgfqpoint{2.375504in}{3.692390in}}%
\pgfpathlineto{\pgfqpoint{2.290658in}{3.630794in}}%
\pgfpathlineto{\pgfqpoint{2.491969in}{4.215044in}}%
\pgfpathclose%
\pgfusepath{fill}%
\end{pgfscope}%
\begin{pgfscope}%
\pgfpathrectangle{\pgfqpoint{0.539299in}{0.078740in}}{\pgfqpoint{7.842520in}{7.842520in}}%
\pgfusepath{clip}%
\pgfsetbuttcap%
\pgfsetroundjoin%
\definecolor{currentfill}{rgb}{0.241237,0.296485,0.539709}%
\pgfsetfillcolor{currentfill}%
\pgfsetlinewidth{0.000000pt}%
\definecolor{currentstroke}{rgb}{0.267004,0.004874,0.329415}%
\pgfsetstrokecolor{currentstroke}%
\pgfsetdash{}{0pt}%
\pgfpathmoveto{\pgfqpoint{5.387297in}{3.592738in}}%
\pgfpathlineto{\pgfqpoint{5.175033in}{3.721001in}}%
\pgfpathlineto{\pgfqpoint{5.311793in}{3.607895in}}%
\pgfpathclose%
\pgfusepath{fill}%
\end{pgfscope}%
\begin{pgfscope}%
\pgfpathrectangle{\pgfqpoint{0.539299in}{0.078740in}}{\pgfqpoint{7.842520in}{7.842520in}}%
\pgfusepath{clip}%
\pgfsetbuttcap%
\pgfsetroundjoin%
\definecolor{currentfill}{rgb}{0.282656,0.100196,0.422160}%
\pgfsetfillcolor{currentfill}%
\pgfsetlinewidth{0.000000pt}%
\definecolor{currentstroke}{rgb}{0.268510,0.009605,0.335427}%
\pgfsetstrokecolor{currentstroke}%
\pgfsetdash{}{0pt}%
\pgfpathmoveto{\pgfqpoint{6.847001in}{3.050168in}}%
\pgfpathlineto{\pgfqpoint{6.914313in}{2.924367in}}%
\pgfpathlineto{\pgfqpoint{6.985354in}{2.964798in}}%
\pgfpathclose%
\pgfusepath{fill}%
\end{pgfscope}%
\begin{pgfscope}%
\pgfpathrectangle{\pgfqpoint{0.539299in}{0.078740in}}{\pgfqpoint{7.842520in}{7.842520in}}%
\pgfusepath{clip}%
\pgfsetbuttcap%
\pgfsetroundjoin%
\definecolor{currentfill}{rgb}{0.280868,0.160771,0.472899}%
\pgfsetfillcolor{currentfill}%
\pgfsetlinewidth{0.000000pt}%
\definecolor{currentstroke}{rgb}{0.269944,0.014625,0.341379}%
\pgfsetstrokecolor{currentstroke}%
\pgfsetdash{}{0pt}%
\pgfpathmoveto{\pgfqpoint{6.426028in}{3.129202in}}%
\pgfpathlineto{\pgfqpoint{6.498702in}{3.171293in}}%
\pgfpathlineto{\pgfqpoint{6.360427in}{3.240693in}}%
\pgfpathclose%
\pgfusepath{fill}%
\end{pgfscope}%
\begin{pgfscope}%
\pgfpathrectangle{\pgfqpoint{0.539299in}{0.078740in}}{\pgfqpoint{7.842520in}{7.842520in}}%
\pgfusepath{clip}%
\pgfsetbuttcap%
\pgfsetroundjoin%
\definecolor{currentfill}{rgb}{0.132268,0.655014,0.519661}%
\pgfsetfillcolor{currentfill}%
\pgfsetlinewidth{0.000000pt}%
\definecolor{currentstroke}{rgb}{0.271305,0.019942,0.347269}%
\pgfsetstrokecolor{currentstroke}%
\pgfsetdash{}{0pt}%
\pgfpathmoveto{\pgfqpoint{3.597214in}{5.159015in}}%
\pgfpathlineto{\pgfqpoint{3.464190in}{5.097259in}}%
\pgfpathlineto{\pgfqpoint{3.380711in}{5.141597in}}%
\pgfpathclose%
\pgfusepath{fill}%
\end{pgfscope}%
\begin{pgfscope}%
\pgfpathrectangle{\pgfqpoint{0.539299in}{0.078740in}}{\pgfqpoint{7.842520in}{7.842520in}}%
\pgfusepath{clip}%
\pgfsetbuttcap%
\pgfsetroundjoin%
\definecolor{currentfill}{rgb}{0.270595,0.214069,0.507052}%
\pgfsetfillcolor{currentfill}%
\pgfsetlinewidth{0.000000pt}%
\definecolor{currentstroke}{rgb}{0.272594,0.025563,0.353093}%
\pgfsetstrokecolor{currentstroke}%
\pgfsetdash{}{0pt}%
\pgfpathmoveto{\pgfqpoint{5.798863in}{3.358134in}}%
\pgfpathlineto{\pgfqpoint{5.936847in}{3.290164in}}%
\pgfpathlineto{\pgfqpoint{5.873171in}{3.386859in}}%
\pgfpathclose%
\pgfusepath{fill}%
\end{pgfscope}%
\begin{pgfscope}%
\pgfpathrectangle{\pgfqpoint{0.539299in}{0.078740in}}{\pgfqpoint{7.842520in}{7.842520in}}%
\pgfusepath{clip}%
\pgfsetbuttcap%
\pgfsetroundjoin%
\definecolor{currentfill}{rgb}{0.283229,0.120777,0.440584}%
\pgfsetfillcolor{currentfill}%
\pgfsetlinewidth{0.000000pt}%
\definecolor{currentstroke}{rgb}{0.273809,0.031497,0.358853}%
\pgfsetstrokecolor{currentstroke}%
\pgfsetdash{}{0pt}%
\pgfpathmoveto{\pgfqpoint{6.637123in}{3.095135in}}%
\pgfpathlineto{\pgfqpoint{6.775662in}{3.012527in}}%
\pgfpathlineto{\pgfqpoint{6.847001in}{3.050168in}}%
\pgfpathclose%
\pgfusepath{fill}%
\end{pgfscope}%
\begin{pgfscope}%
\pgfpathrectangle{\pgfqpoint{0.539299in}{0.078740in}}{\pgfqpoint{7.842520in}{7.842520in}}%
\pgfusepath{clip}%
\pgfsetbuttcap%
\pgfsetroundjoin%
\definecolor{currentfill}{rgb}{0.258965,0.251537,0.524736}%
\pgfsetfillcolor{currentfill}%
\pgfsetlinewidth{0.000000pt}%
\definecolor{currentstroke}{rgb}{0.274952,0.037752,0.364543}%
\pgfsetstrokecolor{currentstroke}%
\pgfsetdash{}{0pt}%
\pgfpathmoveto{\pgfqpoint{5.448865in}{3.506545in}}%
\pgfpathlineto{\pgfqpoint{5.661270in}{3.429383in}}%
\pgfpathlineto{\pgfqpoint{5.524083in}{3.506700in}}%
\pgfpathclose%
\pgfusepath{fill}%
\end{pgfscope}%
\begin{pgfscope}%
\pgfpathrectangle{\pgfqpoint{0.539299in}{0.078740in}}{\pgfqpoint{7.842520in}{7.842520in}}%
\pgfusepath{clip}%
\pgfsetbuttcap%
\pgfsetroundjoin%
\definecolor{currentfill}{rgb}{0.274128,0.199721,0.498911}%
\pgfsetfillcolor{currentfill}%
\pgfsetlinewidth{0.000000pt}%
\definecolor{currentstroke}{rgb}{0.276022,0.044167,0.370164}%
\pgfsetstrokecolor{currentstroke}%
\pgfsetdash{}{0pt}%
\pgfpathmoveto{\pgfqpoint{5.936847in}{3.290164in}}%
\pgfpathlineto{\pgfqpoint{6.149041in}{3.265181in}}%
\pgfpathlineto{\pgfqpoint{6.010937in}{3.326788in}}%
\pgfpathclose%
\pgfusepath{fill}%
\end{pgfscope}%
\begin{pgfscope}%
\pgfpathrectangle{\pgfqpoint{0.539299in}{0.078740in}}{\pgfqpoint{7.842520in}{7.842520in}}%
\pgfusepath{clip}%
\pgfsetbuttcap%
\pgfsetroundjoin%
\definecolor{currentfill}{rgb}{0.124780,0.640461,0.527068}%
\pgfsetfillcolor{currentfill}%
\pgfsetlinewidth{0.000000pt}%
\definecolor{currentstroke}{rgb}{0.277018,0.050344,0.375715}%
\pgfsetstrokecolor{currentstroke}%
\pgfsetdash{}{0pt}%
\pgfpathmoveto{\pgfqpoint{3.867334in}{5.102013in}}%
\pgfpathlineto{\pgfqpoint{3.949347in}{5.006539in}}%
\pgfpathlineto{\pgfqpoint{3.814186in}{5.073908in}}%
\pgfpathclose%
\pgfusepath{fill}%
\end{pgfscope}%
\begin{pgfscope}%
\pgfpathrectangle{\pgfqpoint{0.539299in}{0.078740in}}{\pgfqpoint{7.842520in}{7.842520in}}%
\pgfusepath{clip}%
\pgfsetbuttcap%
\pgfsetroundjoin%
\definecolor{currentfill}{rgb}{0.153364,0.497000,0.557724}%
\pgfsetfillcolor{currentfill}%
\pgfsetlinewidth{0.000000pt}%
\definecolor{currentstroke}{rgb}{0.277941,0.056324,0.381191}%
\pgfsetstrokecolor{currentstroke}%
\pgfsetdash{}{0pt}%
\pgfpathmoveto{\pgfqpoint{2.746363in}{4.336760in}}%
\pgfpathlineto{\pgfqpoint{2.661974in}{4.301274in}}%
\pgfpathlineto{\pgfqpoint{2.784458in}{4.707293in}}%
\pgfpathclose%
\pgfusepath{fill}%
\end{pgfscope}%
\begin{pgfscope}%
\pgfpathrectangle{\pgfqpoint{0.539299in}{0.078740in}}{\pgfqpoint{7.842520in}{7.842520in}}%
\pgfusepath{clip}%
\pgfsetbuttcap%
\pgfsetroundjoin%
\definecolor{currentfill}{rgb}{0.123444,0.636809,0.528763}%
\pgfsetfillcolor{currentfill}%
\pgfsetlinewidth{0.000000pt}%
\definecolor{currentstroke}{rgb}{0.278791,0.062145,0.386592}%
\pgfsetstrokecolor{currentstroke}%
\pgfsetdash{}{0pt}%
\pgfpathmoveto{\pgfqpoint{3.380711in}{5.141597in}}%
\pgfpathlineto{\pgfqpoint{3.249818in}{4.985619in}}%
\pgfpathlineto{\pgfqpoint{3.165894in}{5.004406in}}%
\pgfpathclose%
\pgfusepath{fill}%
\end{pgfscope}%
\begin{pgfscope}%
\pgfpathrectangle{\pgfqpoint{0.539299in}{0.078740in}}{\pgfqpoint{7.842520in}{7.842520in}}%
\pgfusepath{clip}%
\pgfsetbuttcap%
\pgfsetroundjoin%
\definecolor{currentfill}{rgb}{0.248629,0.278775,0.534556}%
\pgfsetfillcolor{currentfill}%
\pgfsetlinewidth{0.000000pt}%
\definecolor{currentstroke}{rgb}{0.279566,0.067836,0.391917}%
\pgfsetstrokecolor{currentstroke}%
\pgfsetdash{}{0pt}%
\pgfpathmoveto{\pgfqpoint{5.387297in}{3.592738in}}%
\pgfpathlineto{\pgfqpoint{5.311793in}{3.607895in}}%
\pgfpathlineto{\pgfqpoint{5.448865in}{3.506545in}}%
\pgfpathclose%
\pgfusepath{fill}%
\end{pgfscope}%
\begin{pgfscope}%
\pgfpathrectangle{\pgfqpoint{0.539299in}{0.078740in}}{\pgfqpoint{7.842520in}{7.842520in}}%
\pgfusepath{clip}%
\pgfsetbuttcap%
\pgfsetroundjoin%
\definecolor{currentfill}{rgb}{0.273809,0.031497,0.358853}%
\pgfsetfillcolor{currentfill}%
\pgfsetlinewidth{0.000000pt}%
\definecolor{currentstroke}{rgb}{0.280267,0.073417,0.397163}%
\pgfsetstrokecolor{currentstroke}%
\pgfsetdash{}{0pt}%
\pgfpathmoveto{\pgfqpoint{7.401863in}{2.702549in}}%
\pgfpathlineto{\pgfqpoint{7.332923in}{2.839586in}}%
\pgfpathlineto{\pgfqpoint{7.262707in}{2.788802in}}%
\pgfpathclose%
\pgfusepath{fill}%
\end{pgfscope}%
\begin{pgfscope}%
\pgfpathrectangle{\pgfqpoint{0.539299in}{0.078740in}}{\pgfqpoint{7.842520in}{7.842520in}}%
\pgfusepath{clip}%
\pgfsetbuttcap%
\pgfsetroundjoin%
\definecolor{currentfill}{rgb}{0.281446,0.084320,0.407414}%
\pgfsetfillcolor{currentfill}%
\pgfsetlinewidth{0.000000pt}%
\definecolor{currentstroke}{rgb}{0.280894,0.078907,0.402329}%
\pgfsetstrokecolor{currentstroke}%
\pgfsetdash{}{0pt}%
\pgfpathmoveto{\pgfqpoint{6.914313in}{2.924367in}}%
\pgfpathlineto{\pgfqpoint{7.123901in}{2.876974in}}%
\pgfpathlineto{\pgfqpoint{6.985354in}{2.964798in}}%
\pgfpathclose%
\pgfusepath{fill}%
\end{pgfscope}%
\begin{pgfscope}%
\pgfpathrectangle{\pgfqpoint{0.539299in}{0.078740in}}{\pgfqpoint{7.842520in}{7.842520in}}%
\pgfusepath{clip}%
\pgfsetbuttcap%
\pgfsetroundjoin%
\definecolor{currentfill}{rgb}{0.279574,0.170599,0.479997}%
\pgfsetfillcolor{currentfill}%
\pgfsetlinewidth{0.000000pt}%
\definecolor{currentstroke}{rgb}{0.281446,0.084320,0.407414}%
\pgfsetstrokecolor{currentstroke}%
\pgfsetdash{}{0pt}%
\pgfpathmoveto{\pgfqpoint{6.360427in}{3.240693in}}%
\pgfpathlineto{\pgfqpoint{6.287425in}{3.199853in}}%
\pgfpathlineto{\pgfqpoint{6.426028in}{3.129202in}}%
\pgfpathclose%
\pgfusepath{fill}%
\end{pgfscope}%
\begin{pgfscope}%
\pgfpathrectangle{\pgfqpoint{0.539299in}{0.078740in}}{\pgfqpoint{7.842520in}{7.842520in}}%
\pgfusepath{clip}%
\pgfsetbuttcap%
\pgfsetroundjoin%
\definecolor{currentfill}{rgb}{0.283091,0.110553,0.431554}%
\pgfsetfillcolor{currentfill}%
\pgfsetlinewidth{0.000000pt}%
\definecolor{currentstroke}{rgb}{0.281924,0.089666,0.412415}%
\pgfsetstrokecolor{currentstroke}%
\pgfsetdash{}{0pt}%
\pgfpathmoveto{\pgfqpoint{6.775662in}{3.012527in}}%
\pgfpathlineto{\pgfqpoint{6.914313in}{2.924367in}}%
\pgfpathlineto{\pgfqpoint{6.847001in}{3.050168in}}%
\pgfpathclose%
\pgfusepath{fill}%
\end{pgfscope}%
\begin{pgfscope}%
\pgfpathrectangle{\pgfqpoint{0.539299in}{0.078740in}}{\pgfqpoint{7.842520in}{7.842520in}}%
\pgfusepath{clip}%
\pgfsetbuttcap%
\pgfsetroundjoin%
\definecolor{currentfill}{rgb}{0.121380,0.629492,0.531973}%
\pgfsetfillcolor{currentfill}%
\pgfsetlinewidth{0.000000pt}%
\definecolor{currentstroke}{rgb}{0.282327,0.094955,0.417331}%
\pgfsetstrokecolor{currentstroke}%
\pgfsetdash{}{0pt}%
\pgfpathmoveto{\pgfqpoint{3.867334in}{5.102013in}}%
\pgfpathlineto{\pgfqpoint{4.085083in}{4.900621in}}%
\pgfpathlineto{\pgfqpoint{3.949347in}{5.006539in}}%
\pgfpathclose%
\pgfusepath{fill}%
\end{pgfscope}%
\begin{pgfscope}%
\pgfpathrectangle{\pgfqpoint{0.539299in}{0.078740in}}{\pgfqpoint{7.842520in}{7.842520in}}%
\pgfusepath{clip}%
\pgfsetbuttcap%
\pgfsetroundjoin%
\definecolor{currentfill}{rgb}{0.188923,0.410910,0.556326}%
\pgfsetfillcolor{currentfill}%
\pgfsetlinewidth{0.000000pt}%
\definecolor{currentstroke}{rgb}{0.282656,0.100196,0.422160}%
\pgfsetstrokecolor{currentstroke}%
\pgfsetdash{}{0pt}%
\pgfpathmoveto{\pgfqpoint{2.375504in}{3.692390in}}%
\pgfpathlineto{\pgfqpoint{2.491969in}{4.215044in}}%
\pgfpathlineto{\pgfqpoint{2.577164in}{4.260977in}}%
\pgfpathclose%
\pgfusepath{fill}%
\end{pgfscope}%
\begin{pgfscope}%
\pgfpathrectangle{\pgfqpoint{0.539299in}{0.078740in}}{\pgfqpoint{7.842520in}{7.842520in}}%
\pgfusepath{clip}%
\pgfsetbuttcap%
\pgfsetroundjoin%
\definecolor{currentfill}{rgb}{0.281887,0.150881,0.465405}%
\pgfsetfillcolor{currentfill}%
\pgfsetlinewidth{0.000000pt}%
\definecolor{currentstroke}{rgb}{0.282910,0.105393,0.426902}%
\pgfsetstrokecolor{currentstroke}%
\pgfsetdash{}{0pt}%
\pgfpathmoveto{\pgfqpoint{6.637123in}{3.095135in}}%
\pgfpathlineto{\pgfqpoint{6.498702in}{3.171293in}}%
\pgfpathlineto{\pgfqpoint{6.564795in}{3.052267in}}%
\pgfpathclose%
\pgfusepath{fill}%
\end{pgfscope}%
\begin{pgfscope}%
\pgfpathrectangle{\pgfqpoint{0.539299in}{0.078740in}}{\pgfqpoint{7.842520in}{7.842520in}}%
\pgfusepath{clip}%
\pgfsetbuttcap%
\pgfsetroundjoin%
\definecolor{currentfill}{rgb}{0.265145,0.232956,0.516599}%
\pgfsetfillcolor{currentfill}%
\pgfsetlinewidth{0.000000pt}%
\definecolor{currentstroke}{rgb}{0.283091,0.110553,0.431554}%
\pgfsetstrokecolor{currentstroke}%
\pgfsetdash{}{0pt}%
\pgfpathmoveto{\pgfqpoint{5.661270in}{3.429383in}}%
\pgfpathlineto{\pgfqpoint{5.724096in}{3.331845in}}%
\pgfpathlineto{\pgfqpoint{5.798863in}{3.358134in}}%
\pgfpathclose%
\pgfusepath{fill}%
\end{pgfscope}%
\begin{pgfscope}%
\pgfpathrectangle{\pgfqpoint{0.539299in}{0.078740in}}{\pgfqpoint{7.842520in}{7.842520in}}%
\pgfusepath{clip}%
\pgfsetbuttcap%
\pgfsetroundjoin%
\definecolor{currentfill}{rgb}{0.134692,0.658636,0.517649}%
\pgfsetfillcolor{currentfill}%
\pgfsetlinewidth{0.000000pt}%
\definecolor{currentstroke}{rgb}{0.283197,0.115680,0.436115}%
\pgfsetstrokecolor{currentstroke}%
\pgfsetdash{}{0pt}%
\pgfpathmoveto{\pgfqpoint{3.814186in}{5.073908in}}%
\pgfpathlineto{\pgfqpoint{3.731742in}{5.157127in}}%
\pgfpathlineto{\pgfqpoint{3.867334in}{5.102013in}}%
\pgfpathclose%
\pgfusepath{fill}%
\end{pgfscope}%
\begin{pgfscope}%
\pgfpathrectangle{\pgfqpoint{0.539299in}{0.078740in}}{\pgfqpoint{7.842520in}{7.842520in}}%
\pgfusepath{clip}%
\pgfsetbuttcap%
\pgfsetroundjoin%
\definecolor{currentfill}{rgb}{0.257322,0.256130,0.526563}%
\pgfsetfillcolor{currentfill}%
\pgfsetlinewidth{0.000000pt}%
\definecolor{currentstroke}{rgb}{0.283229,0.120777,0.440584}%
\pgfsetstrokecolor{currentstroke}%
\pgfsetdash{}{0pt}%
\pgfpathmoveto{\pgfqpoint{5.586291in}{3.415267in}}%
\pgfpathlineto{\pgfqpoint{5.661270in}{3.429383in}}%
\pgfpathlineto{\pgfqpoint{5.448865in}{3.506545in}}%
\pgfpathclose%
\pgfusepath{fill}%
\end{pgfscope}%
\begin{pgfscope}%
\pgfpathrectangle{\pgfqpoint{0.539299in}{0.078740in}}{\pgfqpoint{7.842520in}{7.842520in}}%
\pgfusepath{clip}%
\pgfsetbuttcap%
\pgfsetroundjoin%
\definecolor{currentfill}{rgb}{0.187231,0.414746,0.556547}%
\pgfsetfillcolor{currentfill}%
\pgfsetlinewidth{0.000000pt}%
\definecolor{currentstroke}{rgb}{0.283187,0.125848,0.444960}%
\pgfsetstrokecolor{currentstroke}%
\pgfsetdash{}{0pt}%
\pgfpathmoveto{\pgfqpoint{4.765949in}{4.134630in}}%
\pgfpathlineto{\pgfqpoint{4.824755in}{4.057714in}}%
\pgfpathlineto{\pgfqpoint{4.902184in}{3.985093in}}%
\pgfpathclose%
\pgfusepath{fill}%
\end{pgfscope}%
\begin{pgfscope}%
\pgfpathrectangle{\pgfqpoint{0.539299in}{0.078740in}}{\pgfqpoint{7.842520in}{7.842520in}}%
\pgfusepath{clip}%
\pgfsetbuttcap%
\pgfsetroundjoin%
\definecolor{currentfill}{rgb}{0.120565,0.596422,0.543611}%
\pgfsetfillcolor{currentfill}%
\pgfsetlinewidth{0.000000pt}%
\definecolor{currentstroke}{rgb}{0.283072,0.130895,0.449241}%
\pgfsetstrokecolor{currentstroke}%
\pgfsetdash{}{0pt}%
\pgfpathmoveto{\pgfqpoint{4.085083in}{4.900621in}}%
\pgfpathlineto{\pgfqpoint{4.140312in}{4.874931in}}%
\pgfpathlineto{\pgfqpoint{4.221135in}{4.766924in}}%
\pgfpathclose%
\pgfusepath{fill}%
\end{pgfscope}%
\begin{pgfscope}%
\pgfpathrectangle{\pgfqpoint{0.539299in}{0.078740in}}{\pgfqpoint{7.842520in}{7.842520in}}%
\pgfusepath{clip}%
\pgfsetbuttcap%
\pgfsetroundjoin%
\definecolor{currentfill}{rgb}{0.166617,0.463708,0.558119}%
\pgfsetfillcolor{currentfill}%
\pgfsetlinewidth{0.000000pt}%
\definecolor{currentstroke}{rgb}{0.282884,0.135920,0.453427}%
\pgfsetstrokecolor{currentstroke}%
\pgfsetdash{}{0pt}%
\pgfpathmoveto{\pgfqpoint{4.551037in}{4.389232in}}%
\pgfpathlineto{\pgfqpoint{4.765949in}{4.134630in}}%
\pgfpathlineto{\pgfqpoint{4.629750in}{4.292637in}}%
\pgfpathclose%
\pgfusepath{fill}%
\end{pgfscope}%
\begin{pgfscope}%
\pgfpathrectangle{\pgfqpoint{0.539299in}{0.078740in}}{\pgfqpoint{7.842520in}{7.842520in}}%
\pgfusepath{clip}%
\pgfsetbuttcap%
\pgfsetroundjoin%
\definecolor{currentfill}{rgb}{0.156270,0.489624,0.557936}%
\pgfsetfillcolor{currentfill}%
\pgfsetlinewidth{0.000000pt}%
\definecolor{currentstroke}{rgb}{0.282623,0.140926,0.457517}%
\pgfsetstrokecolor{currentstroke}%
\pgfsetdash{}{0pt}%
\pgfpathmoveto{\pgfqpoint{4.629750in}{4.292637in}}%
\pgfpathlineto{\pgfqpoint{4.493544in}{4.454843in}}%
\pgfpathlineto{\pgfqpoint{4.551037in}{4.389232in}}%
\pgfpathclose%
\pgfusepath{fill}%
\end{pgfscope}%
\begin{pgfscope}%
\pgfpathrectangle{\pgfqpoint{0.539299in}{0.078740in}}{\pgfqpoint{7.842520in}{7.842520in}}%
\pgfusepath{clip}%
\pgfsetbuttcap%
\pgfsetroundjoin%
\definecolor{currentfill}{rgb}{0.137770,0.537492,0.554906}%
\pgfsetfillcolor{currentfill}%
\pgfsetlinewidth{0.000000pt}%
\definecolor{currentstroke}{rgb}{0.282290,0.145912,0.461510}%
\pgfsetstrokecolor{currentstroke}%
\pgfsetdash{}{0pt}%
\pgfpathmoveto{\pgfqpoint{2.869477in}{4.725189in}}%
\pgfpathlineto{\pgfqpoint{2.746363in}{4.336760in}}%
\pgfpathlineto{\pgfqpoint{2.784458in}{4.707293in}}%
\pgfpathclose%
\pgfusepath{fill}%
\end{pgfscope}%
\begin{pgfscope}%
\pgfpathrectangle{\pgfqpoint{0.539299in}{0.078740in}}{\pgfqpoint{7.842520in}{7.842520in}}%
\pgfusepath{clip}%
\pgfsetbuttcap%
\pgfsetroundjoin%
\definecolor{currentfill}{rgb}{0.274128,0.199721,0.498911}%
\pgfsetfillcolor{currentfill}%
\pgfsetlinewidth{0.000000pt}%
\definecolor{currentstroke}{rgb}{0.281887,0.150881,0.465405}%
\pgfsetstrokecolor{currentstroke}%
\pgfsetdash{}{0pt}%
\pgfpathmoveto{\pgfqpoint{6.075188in}{3.222781in}}%
\pgfpathlineto{\pgfqpoint{6.149041in}{3.265181in}}%
\pgfpathlineto{\pgfqpoint{5.936847in}{3.290164in}}%
\pgfpathclose%
\pgfusepath{fill}%
\end{pgfscope}%
\begin{pgfscope}%
\pgfpathrectangle{\pgfqpoint{0.539299in}{0.078740in}}{\pgfqpoint{7.842520in}{7.842520in}}%
\pgfusepath{clip}%
\pgfsetbuttcap%
\pgfsetroundjoin%
\definecolor{currentfill}{rgb}{0.197636,0.391528,0.554969}%
\pgfsetfillcolor{currentfill}%
\pgfsetlinewidth{0.000000pt}%
\definecolor{currentstroke}{rgb}{0.281412,0.155834,0.469201}%
\pgfsetstrokecolor{currentstroke}%
\pgfsetdash{}{0pt}%
\pgfpathmoveto{\pgfqpoint{5.038520in}{3.846770in}}%
\pgfpathlineto{\pgfqpoint{4.902184in}{3.985093in}}%
\pgfpathlineto{\pgfqpoint{4.824755in}{4.057714in}}%
\pgfpathclose%
\pgfusepath{fill}%
\end{pgfscope}%
\begin{pgfscope}%
\pgfpathrectangle{\pgfqpoint{0.539299in}{0.078740in}}{\pgfqpoint{7.842520in}{7.842520in}}%
\pgfusepath{clip}%
\pgfsetbuttcap%
\pgfsetroundjoin%
\definecolor{currentfill}{rgb}{0.124395,0.578002,0.548287}%
\pgfsetfillcolor{currentfill}%
\pgfsetlinewidth{0.000000pt}%
\definecolor{currentstroke}{rgb}{0.280868,0.160771,0.472899}%
\pgfsetstrokecolor{currentstroke}%
\pgfsetdash{}{0pt}%
\pgfpathmoveto{\pgfqpoint{4.140312in}{4.874931in}}%
\pgfpathlineto{\pgfqpoint{4.357325in}{4.615401in}}%
\pgfpathlineto{\pgfqpoint{4.221135in}{4.766924in}}%
\pgfpathclose%
\pgfusepath{fill}%
\end{pgfscope}%
\begin{pgfscope}%
\pgfpathrectangle{\pgfqpoint{0.539299in}{0.078740in}}{\pgfqpoint{7.842520in}{7.842520in}}%
\pgfusepath{clip}%
\pgfsetbuttcap%
\pgfsetroundjoin%
\definecolor{currentfill}{rgb}{0.281412,0.155834,0.469201}%
\pgfsetfillcolor{currentfill}%
\pgfsetlinewidth{0.000000pt}%
\definecolor{currentstroke}{rgb}{0.280255,0.165693,0.476498}%
\pgfsetstrokecolor{currentstroke}%
\pgfsetdash{}{0pt}%
\pgfpathmoveto{\pgfqpoint{6.564795in}{3.052267in}}%
\pgfpathlineto{\pgfqpoint{6.498702in}{3.171293in}}%
\pgfpathlineto{\pgfqpoint{6.426028in}{3.129202in}}%
\pgfpathclose%
\pgfusepath{fill}%
\end{pgfscope}%
\begin{pgfscope}%
\pgfpathrectangle{\pgfqpoint{0.539299in}{0.078740in}}{\pgfqpoint{7.842520in}{7.842520in}}%
\pgfusepath{clip}%
\pgfsetbuttcap%
\pgfsetroundjoin%
\definecolor{currentfill}{rgb}{0.136408,0.541173,0.554483}%
\pgfsetfillcolor{currentfill}%
\pgfsetlinewidth{0.000000pt}%
\definecolor{currentstroke}{rgb}{0.279574,0.170599,0.479997}%
\pgfsetstrokecolor{currentstroke}%
\pgfsetdash{}{0pt}%
\pgfpathmoveto{\pgfqpoint{4.493544in}{4.454843in}}%
\pgfpathlineto{\pgfqpoint{4.357325in}{4.615401in}}%
\pgfpathlineto{\pgfqpoint{4.277193in}{4.723467in}}%
\pgfpathclose%
\pgfusepath{fill}%
\end{pgfscope}%
\begin{pgfscope}%
\pgfpathrectangle{\pgfqpoint{0.539299in}{0.078740in}}{\pgfqpoint{7.842520in}{7.842520in}}%
\pgfusepath{clip}%
\pgfsetbuttcap%
\pgfsetroundjoin%
\definecolor{currentfill}{rgb}{0.277941,0.056324,0.381191}%
\pgfsetfillcolor{currentfill}%
\pgfsetlinewidth{0.000000pt}%
\definecolor{currentstroke}{rgb}{0.278826,0.175490,0.483397}%
\pgfsetstrokecolor{currentstroke}%
\pgfsetdash{}{0pt}%
\pgfpathmoveto{\pgfqpoint{7.123901in}{2.876974in}}%
\pgfpathlineto{\pgfqpoint{7.192040in}{2.737188in}}%
\pgfpathlineto{\pgfqpoint{7.262707in}{2.788802in}}%
\pgfpathclose%
\pgfusepath{fill}%
\end{pgfscope}%
\begin{pgfscope}%
\pgfpathrectangle{\pgfqpoint{0.539299in}{0.078740in}}{\pgfqpoint{7.842520in}{7.842520in}}%
\pgfusepath{clip}%
\pgfsetbuttcap%
\pgfsetroundjoin%
\definecolor{currentfill}{rgb}{0.277134,0.185228,0.489898}%
\pgfsetfillcolor{currentfill}%
\pgfsetlinewidth{0.000000pt}%
\definecolor{currentstroke}{rgb}{0.278012,0.180367,0.486697}%
\pgfsetstrokecolor{currentstroke}%
\pgfsetdash{}{0pt}%
\pgfpathmoveto{\pgfqpoint{6.287425in}{3.199853in}}%
\pgfpathlineto{\pgfqpoint{6.149041in}{3.265181in}}%
\pgfpathlineto{\pgfqpoint{6.213837in}{3.153582in}}%
\pgfpathclose%
\pgfusepath{fill}%
\end{pgfscope}%
\begin{pgfscope}%
\pgfpathrectangle{\pgfqpoint{0.539299in}{0.078740in}}{\pgfqpoint{7.842520in}{7.842520in}}%
\pgfusepath{clip}%
\pgfsetbuttcap%
\pgfsetroundjoin%
\definecolor{currentfill}{rgb}{0.281446,0.084320,0.407414}%
\pgfsetfillcolor{currentfill}%
\pgfsetlinewidth{0.000000pt}%
\definecolor{currentstroke}{rgb}{0.277134,0.185228,0.489898}%
\pgfsetstrokecolor{currentstroke}%
\pgfsetdash{}{0pt}%
\pgfpathmoveto{\pgfqpoint{7.053093in}{2.832020in}}%
\pgfpathlineto{\pgfqpoint{7.123901in}{2.876974in}}%
\pgfpathlineto{\pgfqpoint{6.914313in}{2.924367in}}%
\pgfpathclose%
\pgfusepath{fill}%
\end{pgfscope}%
\begin{pgfscope}%
\pgfpathrectangle{\pgfqpoint{0.539299in}{0.078740in}}{\pgfqpoint{7.842520in}{7.842520in}}%
\pgfusepath{clip}%
\pgfsetbuttcap%
\pgfsetroundjoin%
\definecolor{currentfill}{rgb}{0.212395,0.359683,0.551710}%
\pgfsetfillcolor{currentfill}%
\pgfsetlinewidth{0.000000pt}%
\definecolor{currentstroke}{rgb}{0.276194,0.190074,0.493001}%
\pgfsetstrokecolor{currentstroke}%
\pgfsetdash{}{0pt}%
\pgfpathmoveto{\pgfqpoint{5.175033in}{3.721001in}}%
\pgfpathlineto{\pgfqpoint{5.038520in}{3.846770in}}%
\pgfpathlineto{\pgfqpoint{4.961637in}{3.904256in}}%
\pgfpathclose%
\pgfusepath{fill}%
\end{pgfscope}%
\begin{pgfscope}%
\pgfpathrectangle{\pgfqpoint{0.539299in}{0.078740in}}{\pgfqpoint{7.842520in}{7.842520in}}%
\pgfusepath{clip}%
\pgfsetbuttcap%
\pgfsetroundjoin%
\definecolor{currentfill}{rgb}{0.119699,0.618490,0.536347}%
\pgfsetfillcolor{currentfill}%
\pgfsetlinewidth{0.000000pt}%
\definecolor{currentstroke}{rgb}{0.275191,0.194905,0.496005}%
\pgfsetstrokecolor{currentstroke}%
\pgfsetdash{}{0pt}%
\pgfpathmoveto{\pgfqpoint{3.081416in}{5.017996in}}%
\pgfpathlineto{\pgfqpoint{3.165894in}{5.004406in}}%
\pgfpathlineto{\pgfqpoint{2.954036in}{4.737238in}}%
\pgfpathclose%
\pgfusepath{fill}%
\end{pgfscope}%
\begin{pgfscope}%
\pgfpathrectangle{\pgfqpoint{0.539299in}{0.078740in}}{\pgfqpoint{7.842520in}{7.842520in}}%
\pgfusepath{clip}%
\pgfsetbuttcap%
\pgfsetroundjoin%
\definecolor{currentfill}{rgb}{0.283072,0.130895,0.449241}%
\pgfsetfillcolor{currentfill}%
\pgfsetlinewidth{0.000000pt}%
\definecolor{currentstroke}{rgb}{0.274128,0.199721,0.498911}%
\pgfsetstrokecolor{currentstroke}%
\pgfsetdash{}{0pt}%
\pgfpathmoveto{\pgfqpoint{6.703683in}{2.968722in}}%
\pgfpathlineto{\pgfqpoint{6.775662in}{3.012527in}}%
\pgfpathlineto{\pgfqpoint{6.637123in}{3.095135in}}%
\pgfpathclose%
\pgfusepath{fill}%
\end{pgfscope}%
\begin{pgfscope}%
\pgfpathrectangle{\pgfqpoint{0.539299in}{0.078740in}}{\pgfqpoint{7.842520in}{7.842520in}}%
\pgfusepath{clip}%
\pgfsetbuttcap%
\pgfsetroundjoin%
\definecolor{currentfill}{rgb}{0.262138,0.242286,0.520837}%
\pgfsetfillcolor{currentfill}%
\pgfsetlinewidth{0.000000pt}%
\definecolor{currentstroke}{rgb}{0.273006,0.204520,0.501721}%
\pgfsetstrokecolor{currentstroke}%
\pgfsetdash{}{0pt}%
\pgfpathmoveto{\pgfqpoint{5.586291in}{3.415267in}}%
\pgfpathlineto{\pgfqpoint{5.724096in}{3.331845in}}%
\pgfpathlineto{\pgfqpoint{5.661270in}{3.429383in}}%
\pgfpathclose%
\pgfusepath{fill}%
\end{pgfscope}%
\begin{pgfscope}%
\pgfpathrectangle{\pgfqpoint{0.539299in}{0.078740in}}{\pgfqpoint{7.842520in}{7.842520in}}%
\pgfusepath{clip}%
\pgfsetbuttcap%
\pgfsetroundjoin%
\definecolor{currentfill}{rgb}{0.267968,0.223549,0.512008}%
\pgfsetfillcolor{currentfill}%
\pgfsetlinewidth{0.000000pt}%
\definecolor{currentstroke}{rgb}{0.271828,0.209303,0.504434}%
\pgfsetstrokecolor{currentstroke}%
\pgfsetdash{}{0pt}%
\pgfpathmoveto{\pgfqpoint{5.798863in}{3.358134in}}%
\pgfpathlineto{\pgfqpoint{5.724096in}{3.331845in}}%
\pgfpathlineto{\pgfqpoint{5.936847in}{3.290164in}}%
\pgfpathclose%
\pgfusepath{fill}%
\end{pgfscope}%
\begin{pgfscope}%
\pgfpathrectangle{\pgfqpoint{0.539299in}{0.078740in}}{\pgfqpoint{7.842520in}{7.842520in}}%
\pgfusepath{clip}%
\pgfsetbuttcap%
\pgfsetroundjoin%
\definecolor{currentfill}{rgb}{0.150476,0.504369,0.557430}%
\pgfsetfillcolor{currentfill}%
\pgfsetlinewidth{0.000000pt}%
\definecolor{currentstroke}{rgb}{0.270595,0.214069,0.507052}%
\pgfsetstrokecolor{currentstroke}%
\pgfsetdash{}{0pt}%
\pgfpathmoveto{\pgfqpoint{2.784458in}{4.707293in}}%
\pgfpathlineto{\pgfqpoint{2.661974in}{4.301274in}}%
\pgfpathlineto{\pgfqpoint{2.577164in}{4.260977in}}%
\pgfpathclose%
\pgfusepath{fill}%
\end{pgfscope}%
\begin{pgfscope}%
\pgfpathrectangle{\pgfqpoint{0.539299in}{0.078740in}}{\pgfqpoint{7.842520in}{7.842520in}}%
\pgfusepath{clip}%
\pgfsetbuttcap%
\pgfsetroundjoin%
\definecolor{currentfill}{rgb}{0.150148,0.676631,0.506589}%
\pgfsetfillcolor{currentfill}%
\pgfsetlinewidth{0.000000pt}%
\definecolor{currentstroke}{rgb}{0.269308,0.218818,0.509577}%
\pgfsetstrokecolor{currentstroke}%
\pgfsetdash{}{0pt}%
\pgfpathmoveto{\pgfqpoint{3.380711in}{5.141597in}}%
\pgfpathlineto{\pgfqpoint{3.513793in}{5.222466in}}%
\pgfpathlineto{\pgfqpoint{3.597214in}{5.159015in}}%
\pgfpathclose%
\pgfusepath{fill}%
\end{pgfscope}%
\begin{pgfscope}%
\pgfpathrectangle{\pgfqpoint{0.539299in}{0.078740in}}{\pgfqpoint{7.842520in}{7.842520in}}%
\pgfusepath{clip}%
\pgfsetbuttcap%
\pgfsetroundjoin%
\definecolor{currentfill}{rgb}{0.275191,0.194905,0.496005}%
\pgfsetfillcolor{currentfill}%
\pgfsetlinewidth{0.000000pt}%
\definecolor{currentstroke}{rgb}{0.267968,0.223549,0.512008}%
\pgfsetstrokecolor{currentstroke}%
\pgfsetdash{}{0pt}%
\pgfpathmoveto{\pgfqpoint{6.213837in}{3.153582in}}%
\pgfpathlineto{\pgfqpoint{6.149041in}{3.265181in}}%
\pgfpathlineto{\pgfqpoint{6.075188in}{3.222781in}}%
\pgfpathclose%
\pgfusepath{fill}%
\end{pgfscope}%
\begin{pgfscope}%
\pgfpathrectangle{\pgfqpoint{0.539299in}{0.078740in}}{\pgfqpoint{7.842520in}{7.842520in}}%
\pgfusepath{clip}%
\pgfsetbuttcap%
\pgfsetroundjoin%
\definecolor{currentfill}{rgb}{0.279566,0.067836,0.391917}%
\pgfsetfillcolor{currentfill}%
\pgfsetlinewidth{0.000000pt}%
\definecolor{currentstroke}{rgb}{0.266580,0.228262,0.514349}%
\pgfsetstrokecolor{currentstroke}%
\pgfsetdash{}{0pt}%
\pgfpathmoveto{\pgfqpoint{7.123901in}{2.876974in}}%
\pgfpathlineto{\pgfqpoint{7.053093in}{2.832020in}}%
\pgfpathlineto{\pgfqpoint{7.192040in}{2.737188in}}%
\pgfpathclose%
\pgfusepath{fill}%
\end{pgfscope}%
\begin{pgfscope}%
\pgfpathrectangle{\pgfqpoint{0.539299in}{0.078740in}}{\pgfqpoint{7.842520in}{7.842520in}}%
\pgfusepath{clip}%
\pgfsetbuttcap%
\pgfsetroundjoin%
\definecolor{currentfill}{rgb}{0.231674,0.318106,0.544834}%
\pgfsetfillcolor{currentfill}%
\pgfsetlinewidth{0.000000pt}%
\definecolor{currentstroke}{rgb}{0.265145,0.232956,0.516599}%
\pgfsetstrokecolor{currentstroke}%
\pgfsetdash{}{0pt}%
\pgfpathmoveto{\pgfqpoint{5.235775in}{3.632407in}}%
\pgfpathlineto{\pgfqpoint{5.311793in}{3.607895in}}%
\pgfpathlineto{\pgfqpoint{5.175033in}{3.721001in}}%
\pgfpathclose%
\pgfusepath{fill}%
\end{pgfscope}%
\begin{pgfscope}%
\pgfpathrectangle{\pgfqpoint{0.539299in}{0.078740in}}{\pgfqpoint{7.842520in}{7.842520in}}%
\pgfusepath{clip}%
\pgfsetbuttcap%
\pgfsetroundjoin%
\definecolor{currentfill}{rgb}{0.153894,0.680203,0.504172}%
\pgfsetfillcolor{currentfill}%
\pgfsetlinewidth{0.000000pt}%
\definecolor{currentstroke}{rgb}{0.263663,0.237631,0.518762}%
\pgfsetstrokecolor{currentstroke}%
\pgfsetdash{}{0pt}%
\pgfpathmoveto{\pgfqpoint{3.513793in}{5.222466in}}%
\pgfpathlineto{\pgfqpoint{3.731742in}{5.157127in}}%
\pgfpathlineto{\pgfqpoint{3.597214in}{5.159015in}}%
\pgfpathclose%
\pgfusepath{fill}%
\end{pgfscope}%
\begin{pgfscope}%
\pgfpathrectangle{\pgfqpoint{0.539299in}{0.078740in}}{\pgfqpoint{7.842520in}{7.842520in}}%
\pgfusepath{clip}%
\pgfsetbuttcap%
\pgfsetroundjoin%
\definecolor{currentfill}{rgb}{0.272594,0.025563,0.353093}%
\pgfsetfillcolor{currentfill}%
\pgfsetlinewidth{0.000000pt}%
\definecolor{currentstroke}{rgb}{0.262138,0.242286,0.520837}%
\pgfsetstrokecolor{currentstroke}%
\pgfsetdash{}{0pt}%
\pgfpathmoveto{\pgfqpoint{7.331215in}{2.641797in}}%
\pgfpathlineto{\pgfqpoint{7.401863in}{2.702549in}}%
\pgfpathlineto{\pgfqpoint{7.262707in}{2.788802in}}%
\pgfpathclose%
\pgfusepath{fill}%
\end{pgfscope}%
\begin{pgfscope}%
\pgfpathrectangle{\pgfqpoint{0.539299in}{0.078740in}}{\pgfqpoint{7.842520in}{7.842520in}}%
\pgfusepath{clip}%
\pgfsetbuttcap%
\pgfsetroundjoin%
\definecolor{currentfill}{rgb}{0.283229,0.120777,0.440584}%
\pgfsetfillcolor{currentfill}%
\pgfsetlinewidth{0.000000pt}%
\definecolor{currentstroke}{rgb}{0.260571,0.246922,0.522828}%
\pgfsetstrokecolor{currentstroke}%
\pgfsetdash{}{0pt}%
\pgfpathmoveto{\pgfqpoint{6.703683in}{2.968722in}}%
\pgfpathlineto{\pgfqpoint{6.914313in}{2.924367in}}%
\pgfpathlineto{\pgfqpoint{6.775662in}{3.012527in}}%
\pgfpathclose%
\pgfusepath{fill}%
\end{pgfscope}%
\begin{pgfscope}%
\pgfpathrectangle{\pgfqpoint{0.539299in}{0.078740in}}{\pgfqpoint{7.842520in}{7.842520in}}%
\pgfusepath{clip}%
\pgfsetbuttcap%
\pgfsetroundjoin%
\definecolor{currentfill}{rgb}{0.124780,0.640461,0.527068}%
\pgfsetfillcolor{currentfill}%
\pgfsetlinewidth{0.000000pt}%
\definecolor{currentstroke}{rgb}{0.258965,0.251537,0.524736}%
\pgfsetstrokecolor{currentstroke}%
\pgfsetdash{}{0pt}%
\pgfpathmoveto{\pgfqpoint{4.003621in}{5.004426in}}%
\pgfpathlineto{\pgfqpoint{4.085083in}{4.900621in}}%
\pgfpathlineto{\pgfqpoint{3.867334in}{5.102013in}}%
\pgfpathclose%
\pgfusepath{fill}%
\end{pgfscope}%
\begin{pgfscope}%
\pgfpathrectangle{\pgfqpoint{0.539299in}{0.078740in}}{\pgfqpoint{7.842520in}{7.842520in}}%
\pgfusepath{clip}%
\pgfsetbuttcap%
\pgfsetroundjoin%
\definecolor{currentfill}{rgb}{0.282623,0.140926,0.457517}%
\pgfsetfillcolor{currentfill}%
\pgfsetlinewidth{0.000000pt}%
\definecolor{currentstroke}{rgb}{0.257322,0.256130,0.526563}%
\pgfsetstrokecolor{currentstroke}%
\pgfsetdash{}{0pt}%
\pgfpathmoveto{\pgfqpoint{6.637123in}{3.095135in}}%
\pgfpathlineto{\pgfqpoint{6.564795in}{3.052267in}}%
\pgfpathlineto{\pgfqpoint{6.703683in}{2.968722in}}%
\pgfpathclose%
\pgfusepath{fill}%
\end{pgfscope}%
\begin{pgfscope}%
\pgfpathrectangle{\pgfqpoint{0.539299in}{0.078740in}}{\pgfqpoint{7.842520in}{7.842520in}}%
\pgfusepath{clip}%
\pgfsetbuttcap%
\pgfsetroundjoin%
\definecolor{currentfill}{rgb}{0.278826,0.175490,0.483397}%
\pgfsetfillcolor{currentfill}%
\pgfsetlinewidth{0.000000pt}%
\definecolor{currentstroke}{rgb}{0.255645,0.260703,0.528312}%
\pgfsetstrokecolor{currentstroke}%
\pgfsetdash{}{0pt}%
\pgfpathmoveto{\pgfqpoint{6.426028in}{3.129202in}}%
\pgfpathlineto{\pgfqpoint{6.287425in}{3.199853in}}%
\pgfpathlineto{\pgfqpoint{6.352735in}{3.080591in}}%
\pgfpathclose%
\pgfusepath{fill}%
\end{pgfscope}%
\begin{pgfscope}%
\pgfpathrectangle{\pgfqpoint{0.539299in}{0.078740in}}{\pgfqpoint{7.842520in}{7.842520in}}%
\pgfusepath{clip}%
\pgfsetbuttcap%
\pgfsetroundjoin%
\definecolor{currentfill}{rgb}{0.140210,0.665859,0.513427}%
\pgfsetfillcolor{currentfill}%
\pgfsetlinewidth{0.000000pt}%
\definecolor{currentstroke}{rgb}{0.253935,0.265254,0.529983}%
\pgfsetstrokecolor{currentstroke}%
\pgfsetdash{}{0pt}%
\pgfpathmoveto{\pgfqpoint{3.165894in}{5.004406in}}%
\pgfpathlineto{\pgfqpoint{3.296607in}{5.181085in}}%
\pgfpathlineto{\pgfqpoint{3.380711in}{5.141597in}}%
\pgfpathclose%
\pgfusepath{fill}%
\end{pgfscope}%
\begin{pgfscope}%
\pgfpathrectangle{\pgfqpoint{0.539299in}{0.078740in}}{\pgfqpoint{7.842520in}{7.842520in}}%
\pgfusepath{clip}%
\pgfsetbuttcap%
\pgfsetroundjoin%
\definecolor{currentfill}{rgb}{0.243113,0.292092,0.538516}%
\pgfsetfillcolor{currentfill}%
\pgfsetlinewidth{0.000000pt}%
\definecolor{currentstroke}{rgb}{0.252194,0.269783,0.531579}%
\pgfsetstrokecolor{currentstroke}%
\pgfsetdash{}{0pt}%
\pgfpathmoveto{\pgfqpoint{5.311793in}{3.607895in}}%
\pgfpathlineto{\pgfqpoint{5.373172in}{3.514648in}}%
\pgfpathlineto{\pgfqpoint{5.448865in}{3.506545in}}%
\pgfpathclose%
\pgfusepath{fill}%
\end{pgfscope}%
\begin{pgfscope}%
\pgfpathrectangle{\pgfqpoint{0.539299in}{0.078740in}}{\pgfqpoint{7.842520in}{7.842520in}}%
\pgfusepath{clip}%
\pgfsetbuttcap%
\pgfsetroundjoin%
\definecolor{currentfill}{rgb}{0.175841,0.441290,0.557685}%
\pgfsetfillcolor{currentfill}%
\pgfsetlinewidth{0.000000pt}%
\definecolor{currentstroke}{rgb}{0.250425,0.274290,0.533103}%
\pgfsetstrokecolor{currentstroke}%
\pgfsetdash{}{0pt}%
\pgfpathmoveto{\pgfqpoint{4.687905in}{4.220515in}}%
\pgfpathlineto{\pgfqpoint{4.824755in}{4.057714in}}%
\pgfpathlineto{\pgfqpoint{4.765949in}{4.134630in}}%
\pgfpathclose%
\pgfusepath{fill}%
\end{pgfscope}%
\begin{pgfscope}%
\pgfpathrectangle{\pgfqpoint{0.539299in}{0.078740in}}{\pgfqpoint{7.842520in}{7.842520in}}%
\pgfusepath{clip}%
\pgfsetbuttcap%
\pgfsetroundjoin%
\definecolor{currentfill}{rgb}{0.120638,0.625828,0.533488}%
\pgfsetfillcolor{currentfill}%
\pgfsetlinewidth{0.000000pt}%
\definecolor{currentstroke}{rgb}{0.248629,0.278775,0.534556}%
\pgfsetstrokecolor{currentstroke}%
\pgfsetdash{}{0pt}%
\pgfpathmoveto{\pgfqpoint{4.003621in}{5.004426in}}%
\pgfpathlineto{\pgfqpoint{4.140312in}{4.874931in}}%
\pgfpathlineto{\pgfqpoint{4.085083in}{4.900621in}}%
\pgfpathclose%
\pgfusepath{fill}%
\end{pgfscope}%
\begin{pgfscope}%
\pgfpathrectangle{\pgfqpoint{0.539299in}{0.078740in}}{\pgfqpoint{7.842520in}{7.842520in}}%
\pgfusepath{clip}%
\pgfsetbuttcap%
\pgfsetroundjoin%
\definecolor{currentfill}{rgb}{0.165117,0.467423,0.558141}%
\pgfsetfillcolor{currentfill}%
\pgfsetlinewidth{0.000000pt}%
\definecolor{currentstroke}{rgb}{0.246811,0.283237,0.535941}%
\pgfsetstrokecolor{currentstroke}%
\pgfsetdash{}{0pt}%
\pgfpathmoveto{\pgfqpoint{4.687905in}{4.220515in}}%
\pgfpathlineto{\pgfqpoint{4.765949in}{4.134630in}}%
\pgfpathlineto{\pgfqpoint{4.551037in}{4.389232in}}%
\pgfpathclose%
\pgfusepath{fill}%
\end{pgfscope}%
\begin{pgfscope}%
\pgfpathrectangle{\pgfqpoint{0.539299in}{0.078740in}}{\pgfqpoint{7.842520in}{7.842520in}}%
\pgfusepath{clip}%
\pgfsetbuttcap%
\pgfsetroundjoin%
\definecolor{currentfill}{rgb}{0.195860,0.395433,0.555276}%
\pgfsetfillcolor{currentfill}%
\pgfsetlinewidth{0.000000pt}%
\definecolor{currentstroke}{rgb}{0.244972,0.287675,0.537260}%
\pgfsetstrokecolor{currentstroke}%
\pgfsetdash{}{0pt}%
\pgfpathmoveto{\pgfqpoint{4.961637in}{3.904256in}}%
\pgfpathlineto{\pgfqpoint{5.038520in}{3.846770in}}%
\pgfpathlineto{\pgfqpoint{4.824755in}{4.057714in}}%
\pgfpathclose%
\pgfusepath{fill}%
\end{pgfscope}%
\begin{pgfscope}%
\pgfpathrectangle{\pgfqpoint{0.539299in}{0.078740in}}{\pgfqpoint{7.842520in}{7.842520in}}%
\pgfusepath{clip}%
\pgfsetbuttcap%
\pgfsetroundjoin%
\definecolor{currentfill}{rgb}{0.144759,0.519093,0.556572}%
\pgfsetfillcolor{currentfill}%
\pgfsetlinewidth{0.000000pt}%
\definecolor{currentstroke}{rgb}{0.243113,0.292092,0.538516}%
\pgfsetstrokecolor{currentstroke}%
\pgfsetdash{}{0pt}%
\pgfpathmoveto{\pgfqpoint{4.551037in}{4.389232in}}%
\pgfpathlineto{\pgfqpoint{4.493544in}{4.454843in}}%
\pgfpathlineto{\pgfqpoint{4.414126in}{4.558993in}}%
\pgfpathclose%
\pgfusepath{fill}%
\end{pgfscope}%
\begin{pgfscope}%
\pgfpathrectangle{\pgfqpoint{0.539299in}{0.078740in}}{\pgfqpoint{7.842520in}{7.842520in}}%
\pgfusepath{clip}%
\pgfsetbuttcap%
\pgfsetroundjoin%
\definecolor{currentfill}{rgb}{0.274952,0.037752,0.364543}%
\pgfsetfillcolor{currentfill}%
\pgfsetlinewidth{0.000000pt}%
\definecolor{currentstroke}{rgb}{0.241237,0.296485,0.539709}%
\pgfsetstrokecolor{currentstroke}%
\pgfsetdash{}{0pt}%
\pgfpathmoveto{\pgfqpoint{7.262707in}{2.788802in}}%
\pgfpathlineto{\pgfqpoint{7.192040in}{2.737188in}}%
\pgfpathlineto{\pgfqpoint{7.331215in}{2.641797in}}%
\pgfpathclose%
\pgfusepath{fill}%
\end{pgfscope}%
\begin{pgfscope}%
\pgfpathrectangle{\pgfqpoint{0.539299in}{0.078740in}}{\pgfqpoint{7.842520in}{7.842520in}}%
\pgfusepath{clip}%
\pgfsetbuttcap%
\pgfsetroundjoin%
\definecolor{currentfill}{rgb}{0.135066,0.544853,0.554029}%
\pgfsetfillcolor{currentfill}%
\pgfsetlinewidth{0.000000pt}%
\definecolor{currentstroke}{rgb}{0.239346,0.300855,0.540844}%
\pgfsetstrokecolor{currentstroke}%
\pgfsetdash{}{0pt}%
\pgfpathmoveto{\pgfqpoint{4.277193in}{4.723467in}}%
\pgfpathlineto{\pgfqpoint{4.414126in}{4.558993in}}%
\pgfpathlineto{\pgfqpoint{4.493544in}{4.454843in}}%
\pgfpathclose%
\pgfusepath{fill}%
\end{pgfscope}%
\begin{pgfscope}%
\pgfpathrectangle{\pgfqpoint{0.539299in}{0.078740in}}{\pgfqpoint{7.842520in}{7.842520in}}%
\pgfusepath{clip}%
\pgfsetbuttcap%
\pgfsetroundjoin%
\definecolor{currentfill}{rgb}{0.123463,0.581687,0.547445}%
\pgfsetfillcolor{currentfill}%
\pgfsetlinewidth{0.000000pt}%
\definecolor{currentstroke}{rgb}{0.237441,0.305202,0.541921}%
\pgfsetstrokecolor{currentstroke}%
\pgfsetdash{}{0pt}%
\pgfpathmoveto{\pgfqpoint{4.277193in}{4.723467in}}%
\pgfpathlineto{\pgfqpoint{4.357325in}{4.615401in}}%
\pgfpathlineto{\pgfqpoint{4.140312in}{4.874931in}}%
\pgfpathclose%
\pgfusepath{fill}%
\end{pgfscope}%
\begin{pgfscope}%
\pgfpathrectangle{\pgfqpoint{0.539299in}{0.078740in}}{\pgfqpoint{7.842520in}{7.842520in}}%
\pgfusepath{clip}%
\pgfsetbuttcap%
\pgfsetroundjoin%
\definecolor{currentfill}{rgb}{0.210503,0.363727,0.552206}%
\pgfsetfillcolor{currentfill}%
\pgfsetlinewidth{0.000000pt}%
\definecolor{currentstroke}{rgb}{0.235526,0.309527,0.542944}%
\pgfsetstrokecolor{currentstroke}%
\pgfsetdash{}{0pt}%
\pgfpathmoveto{\pgfqpoint{4.961637in}{3.904256in}}%
\pgfpathlineto{\pgfqpoint{5.098620in}{3.762206in}}%
\pgfpathlineto{\pgfqpoint{5.175033in}{3.721001in}}%
\pgfpathclose%
\pgfusepath{fill}%
\end{pgfscope}%
\begin{pgfscope}%
\pgfpathrectangle{\pgfqpoint{0.539299in}{0.078740in}}{\pgfqpoint{7.842520in}{7.842520in}}%
\pgfusepath{clip}%
\pgfsetbuttcap%
\pgfsetroundjoin%
\definecolor{currentfill}{rgb}{0.214298,0.355619,0.551184}%
\pgfsetfillcolor{currentfill}%
\pgfsetlinewidth{0.000000pt}%
\definecolor{currentstroke}{rgb}{0.233603,0.313828,0.543914}%
\pgfsetstrokecolor{currentstroke}%
\pgfsetdash{}{0pt}%
\pgfpathmoveto{\pgfqpoint{2.320596in}{4.102334in}}%
\pgfpathlineto{\pgfqpoint{2.290658in}{3.630794in}}%
\pgfpathlineto{\pgfqpoint{2.205440in}{3.565187in}}%
\pgfpathclose%
\pgfusepath{fill}%
\end{pgfscope}%
\begin{pgfscope}%
\pgfpathrectangle{\pgfqpoint{0.539299in}{0.078740in}}{\pgfqpoint{7.842520in}{7.842520in}}%
\pgfusepath{clip}%
\pgfsetbuttcap%
\pgfsetroundjoin%
\definecolor{currentfill}{rgb}{0.267968,0.223549,0.512008}%
\pgfsetfillcolor{currentfill}%
\pgfsetlinewidth{0.000000pt}%
\definecolor{currentstroke}{rgb}{0.231674,0.318106,0.544834}%
\pgfsetstrokecolor{currentstroke}%
\pgfsetdash{}{0pt}%
\pgfpathmoveto{\pgfqpoint{5.724096in}{3.331845in}}%
\pgfpathlineto{\pgfqpoint{5.862282in}{3.253769in}}%
\pgfpathlineto{\pgfqpoint{5.936847in}{3.290164in}}%
\pgfpathclose%
\pgfusepath{fill}%
\end{pgfscope}%
\begin{pgfscope}%
\pgfpathrectangle{\pgfqpoint{0.539299in}{0.078740in}}{\pgfqpoint{7.842520in}{7.842520in}}%
\pgfusepath{clip}%
\pgfsetbuttcap%
\pgfsetroundjoin%
\definecolor{currentfill}{rgb}{0.278012,0.180367,0.486697}%
\pgfsetfillcolor{currentfill}%
\pgfsetlinewidth{0.000000pt}%
\definecolor{currentstroke}{rgb}{0.229739,0.322361,0.545706}%
\pgfsetstrokecolor{currentstroke}%
\pgfsetdash{}{0pt}%
\pgfpathmoveto{\pgfqpoint{6.287425in}{3.199853in}}%
\pgfpathlineto{\pgfqpoint{6.213837in}{3.153582in}}%
\pgfpathlineto{\pgfqpoint{6.352735in}{3.080591in}}%
\pgfpathclose%
\pgfusepath{fill}%
\end{pgfscope}%
\begin{pgfscope}%
\pgfpathrectangle{\pgfqpoint{0.539299in}{0.078740in}}{\pgfqpoint{7.842520in}{7.842520in}}%
\pgfusepath{clip}%
\pgfsetbuttcap%
\pgfsetroundjoin%
\definecolor{currentfill}{rgb}{0.248629,0.278775,0.534556}%
\pgfsetfillcolor{currentfill}%
\pgfsetlinewidth{0.000000pt}%
\definecolor{currentstroke}{rgb}{0.227802,0.326594,0.546532}%
\pgfsetstrokecolor{currentstroke}%
\pgfsetdash{}{0pt}%
\pgfpathmoveto{\pgfqpoint{5.586291in}{3.415267in}}%
\pgfpathlineto{\pgfqpoint{5.448865in}{3.506545in}}%
\pgfpathlineto{\pgfqpoint{5.373172in}{3.514648in}}%
\pgfpathclose%
\pgfusepath{fill}%
\end{pgfscope}%
\begin{pgfscope}%
\pgfpathrectangle{\pgfqpoint{0.539299in}{0.078740in}}{\pgfqpoint{7.842520in}{7.842520in}}%
\pgfusepath{clip}%
\pgfsetbuttcap%
\pgfsetroundjoin%
\definecolor{currentfill}{rgb}{0.282656,0.100196,0.422160}%
\pgfsetfillcolor{currentfill}%
\pgfsetlinewidth{0.000000pt}%
\definecolor{currentstroke}{rgb}{0.225863,0.330805,0.547314}%
\pgfsetstrokecolor{currentstroke}%
\pgfsetdash{}{0pt}%
\pgfpathmoveto{\pgfqpoint{7.053093in}{2.832020in}}%
\pgfpathlineto{\pgfqpoint{6.914313in}{2.924367in}}%
\pgfpathlineto{\pgfqpoint{6.842661in}{2.878816in}}%
\pgfpathclose%
\pgfusepath{fill}%
\end{pgfscope}%
\begin{pgfscope}%
\pgfpathrectangle{\pgfqpoint{0.539299in}{0.078740in}}{\pgfqpoint{7.842520in}{7.842520in}}%
\pgfusepath{clip}%
\pgfsetbuttcap%
\pgfsetroundjoin%
\definecolor{currentfill}{rgb}{0.221989,0.339161,0.548752}%
\pgfsetfillcolor{currentfill}%
\pgfsetlinewidth{0.000000pt}%
\definecolor{currentstroke}{rgb}{0.223925,0.334994,0.548053}%
\pgfsetstrokecolor{currentstroke}%
\pgfsetdash{}{0pt}%
\pgfpathmoveto{\pgfqpoint{5.098620in}{3.762206in}}%
\pgfpathlineto{\pgfqpoint{5.235775in}{3.632407in}}%
\pgfpathlineto{\pgfqpoint{5.175033in}{3.721001in}}%
\pgfpathclose%
\pgfusepath{fill}%
\end{pgfscope}%
\begin{pgfscope}%
\pgfpathrectangle{\pgfqpoint{0.539299in}{0.078740in}}{\pgfqpoint{7.842520in}{7.842520in}}%
\pgfusepath{clip}%
\pgfsetbuttcap%
\pgfsetroundjoin%
\definecolor{currentfill}{rgb}{0.271828,0.209303,0.504434}%
\pgfsetfillcolor{currentfill}%
\pgfsetlinewidth{0.000000pt}%
\definecolor{currentstroke}{rgb}{0.221989,0.339161,0.548752}%
\pgfsetstrokecolor{currentstroke}%
\pgfsetdash{}{0pt}%
\pgfpathmoveto{\pgfqpoint{6.075188in}{3.222781in}}%
\pgfpathlineto{\pgfqpoint{5.936847in}{3.290164in}}%
\pgfpathlineto{\pgfqpoint{6.000832in}{3.178458in}}%
\pgfpathclose%
\pgfusepath{fill}%
\end{pgfscope}%
\begin{pgfscope}%
\pgfpathrectangle{\pgfqpoint{0.539299in}{0.078740in}}{\pgfqpoint{7.842520in}{7.842520in}}%
\pgfusepath{clip}%
\pgfsetbuttcap%
\pgfsetroundjoin%
\definecolor{currentfill}{rgb}{0.280255,0.165693,0.476498}%
\pgfsetfillcolor{currentfill}%
\pgfsetlinewidth{0.000000pt}%
\definecolor{currentstroke}{rgb}{0.220057,0.343307,0.549413}%
\pgfsetstrokecolor{currentstroke}%
\pgfsetdash{}{0pt}%
\pgfpathmoveto{\pgfqpoint{6.352735in}{3.080591in}}%
\pgfpathlineto{\pgfqpoint{6.564795in}{3.052267in}}%
\pgfpathlineto{\pgfqpoint{6.426028in}{3.129202in}}%
\pgfpathclose%
\pgfusepath{fill}%
\end{pgfscope}%
\begin{pgfscope}%
\pgfpathrectangle{\pgfqpoint{0.539299in}{0.078740in}}{\pgfqpoint{7.842520in}{7.842520in}}%
\pgfusepath{clip}%
\pgfsetbuttcap%
\pgfsetroundjoin%
\definecolor{currentfill}{rgb}{0.119738,0.603785,0.541400}%
\pgfsetfillcolor{currentfill}%
\pgfsetlinewidth{0.000000pt}%
\definecolor{currentstroke}{rgb}{0.218130,0.347432,0.550038}%
\pgfsetstrokecolor{currentstroke}%
\pgfsetdash{}{0pt}%
\pgfpathmoveto{\pgfqpoint{2.869477in}{4.725189in}}%
\pgfpathlineto{\pgfqpoint{2.996415in}{5.025431in}}%
\pgfpathlineto{\pgfqpoint{2.954036in}{4.737238in}}%
\pgfpathclose%
\pgfusepath{fill}%
\end{pgfscope}%
\begin{pgfscope}%
\pgfpathrectangle{\pgfqpoint{0.539299in}{0.078740in}}{\pgfqpoint{7.842520in}{7.842520in}}%
\pgfusepath{clip}%
\pgfsetbuttcap%
\pgfsetroundjoin%
\definecolor{currentfill}{rgb}{0.187231,0.414746,0.556547}%
\pgfsetfillcolor{currentfill}%
\pgfsetlinewidth{0.000000pt}%
\definecolor{currentstroke}{rgb}{0.216210,0.351535,0.550627}%
\pgfsetstrokecolor{currentstroke}%
\pgfsetdash{}{0pt}%
\pgfpathmoveto{\pgfqpoint{2.491969in}{4.215044in}}%
\pgfpathlineto{\pgfqpoint{2.290658in}{3.630794in}}%
\pgfpathlineto{\pgfqpoint{2.406429in}{4.162525in}}%
\pgfpathclose%
\pgfusepath{fill}%
\end{pgfscope}%
\begin{pgfscope}%
\pgfpathrectangle{\pgfqpoint{0.539299in}{0.078740in}}{\pgfqpoint{7.842520in}{7.842520in}}%
\pgfusepath{clip}%
\pgfsetbuttcap%
\pgfsetroundjoin%
\definecolor{currentfill}{rgb}{0.283197,0.115680,0.436115}%
\pgfsetfillcolor{currentfill}%
\pgfsetlinewidth{0.000000pt}%
\definecolor{currentstroke}{rgb}{0.214298,0.355619,0.551184}%
\pgfsetstrokecolor{currentstroke}%
\pgfsetdash{}{0pt}%
\pgfpathmoveto{\pgfqpoint{6.703683in}{2.968722in}}%
\pgfpathlineto{\pgfqpoint{6.842661in}{2.878816in}}%
\pgfpathlineto{\pgfqpoint{6.914313in}{2.924367in}}%
\pgfpathclose%
\pgfusepath{fill}%
\end{pgfscope}%
\begin{pgfscope}%
\pgfpathrectangle{\pgfqpoint{0.539299in}{0.078740in}}{\pgfqpoint{7.842520in}{7.842520in}}%
\pgfusepath{clip}%
\pgfsetbuttcap%
\pgfsetroundjoin%
\definecolor{currentfill}{rgb}{0.235526,0.309527,0.542944}%
\pgfsetfillcolor{currentfill}%
\pgfsetlinewidth{0.000000pt}%
\definecolor{currentstroke}{rgb}{0.212395,0.359683,0.551710}%
\pgfsetstrokecolor{currentstroke}%
\pgfsetdash{}{0pt}%
\pgfpathmoveto{\pgfqpoint{5.235775in}{3.632407in}}%
\pgfpathlineto{\pgfqpoint{5.373172in}{3.514648in}}%
\pgfpathlineto{\pgfqpoint{5.311793in}{3.607895in}}%
\pgfpathclose%
\pgfusepath{fill}%
\end{pgfscope}%
\begin{pgfscope}%
\pgfpathrectangle{\pgfqpoint{0.539299in}{0.078740in}}{\pgfqpoint{7.842520in}{7.842520in}}%
\pgfusepath{clip}%
\pgfsetbuttcap%
\pgfsetroundjoin%
\definecolor{currentfill}{rgb}{0.137339,0.662252,0.515571}%
\pgfsetfillcolor{currentfill}%
\pgfsetlinewidth{0.000000pt}%
\definecolor{currentstroke}{rgb}{0.210503,0.363727,0.552206}%
\pgfsetstrokecolor{currentstroke}%
\pgfsetdash{}{0pt}%
\pgfpathmoveto{\pgfqpoint{3.296607in}{5.181085in}}%
\pgfpathlineto{\pgfqpoint{3.165894in}{5.004406in}}%
\pgfpathlineto{\pgfqpoint{3.081416in}{5.017996in}}%
\pgfpathclose%
\pgfusepath{fill}%
\end{pgfscope}%
\begin{pgfscope}%
\pgfpathrectangle{\pgfqpoint{0.539299in}{0.078740in}}{\pgfqpoint{7.842520in}{7.842520in}}%
\pgfusepath{clip}%
\pgfsetbuttcap%
\pgfsetroundjoin%
\definecolor{currentfill}{rgb}{0.270595,0.214069,0.507052}%
\pgfsetfillcolor{currentfill}%
\pgfsetlinewidth{0.000000pt}%
\definecolor{currentstroke}{rgb}{0.208623,0.367752,0.552675}%
\pgfsetstrokecolor{currentstroke}%
\pgfsetdash{}{0pt}%
\pgfpathmoveto{\pgfqpoint{6.000832in}{3.178458in}}%
\pgfpathlineto{\pgfqpoint{5.936847in}{3.290164in}}%
\pgfpathlineto{\pgfqpoint{5.862282in}{3.253769in}}%
\pgfpathclose%
\pgfusepath{fill}%
\end{pgfscope}%
\begin{pgfscope}%
\pgfpathrectangle{\pgfqpoint{0.539299in}{0.078740in}}{\pgfqpoint{7.842520in}{7.842520in}}%
\pgfusepath{clip}%
\pgfsetbuttcap%
\pgfsetroundjoin%
\definecolor{currentfill}{rgb}{0.255645,0.260703,0.528312}%
\pgfsetfillcolor{currentfill}%
\pgfsetlinewidth{0.000000pt}%
\definecolor{currentstroke}{rgb}{0.206756,0.371758,0.553117}%
\pgfsetstrokecolor{currentstroke}%
\pgfsetdash{}{0pt}%
\pgfpathmoveto{\pgfqpoint{5.586291in}{3.415267in}}%
\pgfpathlineto{\pgfqpoint{5.510865in}{3.407873in}}%
\pgfpathlineto{\pgfqpoint{5.724096in}{3.331845in}}%
\pgfpathclose%
\pgfusepath{fill}%
\end{pgfscope}%
\begin{pgfscope}%
\pgfpathrectangle{\pgfqpoint{0.539299in}{0.078740in}}{\pgfqpoint{7.842520in}{7.842520in}}%
\pgfusepath{clip}%
\pgfsetbuttcap%
\pgfsetroundjoin%
\definecolor{currentfill}{rgb}{0.280267,0.073417,0.397163}%
\pgfsetfillcolor{currentfill}%
\pgfsetlinewidth{0.000000pt}%
\definecolor{currentstroke}{rgb}{0.204903,0.375746,0.553533}%
\pgfsetstrokecolor{currentstroke}%
\pgfsetdash{}{0pt}%
\pgfpathmoveto{\pgfqpoint{7.192040in}{2.737188in}}%
\pgfpathlineto{\pgfqpoint{7.053093in}{2.832020in}}%
\pgfpathlineto{\pgfqpoint{6.981719in}{2.783278in}}%
\pgfpathclose%
\pgfusepath{fill}%
\end{pgfscope}%
\begin{pgfscope}%
\pgfpathrectangle{\pgfqpoint{0.539299in}{0.078740in}}{\pgfqpoint{7.842520in}{7.842520in}}%
\pgfusepath{clip}%
\pgfsetbuttcap%
\pgfsetroundjoin%
\definecolor{currentfill}{rgb}{0.274128,0.199721,0.498911}%
\pgfsetfillcolor{currentfill}%
\pgfsetlinewidth{0.000000pt}%
\definecolor{currentstroke}{rgb}{0.203063,0.379716,0.553925}%
\pgfsetstrokecolor{currentstroke}%
\pgfsetdash{}{0pt}%
\pgfpathmoveto{\pgfqpoint{6.000832in}{3.178458in}}%
\pgfpathlineto{\pgfqpoint{6.213837in}{3.153582in}}%
\pgfpathlineto{\pgfqpoint{6.075188in}{3.222781in}}%
\pgfpathclose%
\pgfusepath{fill}%
\end{pgfscope}%
\begin{pgfscope}%
\pgfpathrectangle{\pgfqpoint{0.539299in}{0.078740in}}{\pgfqpoint{7.842520in}{7.842520in}}%
\pgfusepath{clip}%
\pgfsetbuttcap%
\pgfsetroundjoin%
\definecolor{currentfill}{rgb}{0.122312,0.633153,0.530398}%
\pgfsetfillcolor{currentfill}%
\pgfsetlinewidth{0.000000pt}%
\definecolor{currentstroke}{rgb}{0.201239,0.383670,0.554294}%
\pgfsetstrokecolor{currentstroke}%
\pgfsetdash{}{0pt}%
\pgfpathmoveto{\pgfqpoint{2.954036in}{4.737238in}}%
\pgfpathlineto{\pgfqpoint{2.996415in}{5.025431in}}%
\pgfpathlineto{\pgfqpoint{3.081416in}{5.017996in}}%
\pgfpathclose%
\pgfusepath{fill}%
\end{pgfscope}%
\begin{pgfscope}%
\pgfpathrectangle{\pgfqpoint{0.539299in}{0.078740in}}{\pgfqpoint{7.842520in}{7.842520in}}%
\pgfusepath{clip}%
\pgfsetbuttcap%
\pgfsetroundjoin%
\definecolor{currentfill}{rgb}{0.281887,0.150881,0.465405}%
\pgfsetfillcolor{currentfill}%
\pgfsetlinewidth{0.000000pt}%
\definecolor{currentstroke}{rgb}{0.199430,0.387607,0.554642}%
\pgfsetstrokecolor{currentstroke}%
\pgfsetdash{}{0pt}%
\pgfpathmoveto{\pgfqpoint{6.703683in}{2.968722in}}%
\pgfpathlineto{\pgfqpoint{6.564795in}{3.052267in}}%
\pgfpathlineto{\pgfqpoint{6.491826in}{3.002343in}}%
\pgfpathclose%
\pgfusepath{fill}%
\end{pgfscope}%
\begin{pgfscope}%
\pgfpathrectangle{\pgfqpoint{0.539299in}{0.078740in}}{\pgfqpoint{7.842520in}{7.842520in}}%
\pgfusepath{clip}%
\pgfsetbuttcap%
\pgfsetroundjoin%
\definecolor{currentfill}{rgb}{0.175707,0.697900,0.491033}%
\pgfsetfillcolor{currentfill}%
\pgfsetlinewidth{0.000000pt}%
\definecolor{currentstroke}{rgb}{0.197636,0.391528,0.554969}%
\pgfsetstrokecolor{currentstroke}%
\pgfsetdash{}{0pt}%
\pgfpathmoveto{\pgfqpoint{3.648554in}{5.237108in}}%
\pgfpathlineto{\pgfqpoint{3.731742in}{5.157127in}}%
\pgfpathlineto{\pgfqpoint{3.513793in}{5.222466in}}%
\pgfpathclose%
\pgfusepath{fill}%
\end{pgfscope}%
\begin{pgfscope}%
\pgfpathrectangle{\pgfqpoint{0.539299in}{0.078740in}}{\pgfqpoint{7.842520in}{7.842520in}}%
\pgfusepath{clip}%
\pgfsetbuttcap%
\pgfsetroundjoin%
\definecolor{currentfill}{rgb}{0.282327,0.094955,0.417331}%
\pgfsetfillcolor{currentfill}%
\pgfsetlinewidth{0.000000pt}%
\definecolor{currentstroke}{rgb}{0.195860,0.395433,0.555276}%
\pgfsetstrokecolor{currentstroke}%
\pgfsetdash{}{0pt}%
\pgfpathmoveto{\pgfqpoint{6.842661in}{2.878816in}}%
\pgfpathlineto{\pgfqpoint{6.981719in}{2.783278in}}%
\pgfpathlineto{\pgfqpoint{7.053093in}{2.832020in}}%
\pgfpathclose%
\pgfusepath{fill}%
\end{pgfscope}%
\begin{pgfscope}%
\pgfpathrectangle{\pgfqpoint{0.539299in}{0.078740in}}{\pgfqpoint{7.842520in}{7.842520in}}%
\pgfusepath{clip}%
\pgfsetbuttcap%
\pgfsetroundjoin%
\definecolor{currentfill}{rgb}{0.157851,0.683765,0.501686}%
\pgfsetfillcolor{currentfill}%
\pgfsetlinewidth{0.000000pt}%
\definecolor{currentstroke}{rgb}{0.194100,0.399323,0.555565}%
\pgfsetstrokecolor{currentstroke}%
\pgfsetdash{}{0pt}%
\pgfpathmoveto{\pgfqpoint{3.784535in}{5.195377in}}%
\pgfpathlineto{\pgfqpoint{3.867334in}{5.102013in}}%
\pgfpathlineto{\pgfqpoint{3.731742in}{5.157127in}}%
\pgfpathclose%
\pgfusepath{fill}%
\end{pgfscope}%
\begin{pgfscope}%
\pgfpathrectangle{\pgfqpoint{0.539299in}{0.078740in}}{\pgfqpoint{7.842520in}{7.842520in}}%
\pgfusepath{clip}%
\pgfsetbuttcap%
\pgfsetroundjoin%
\definecolor{currentfill}{rgb}{0.248629,0.278775,0.534556}%
\pgfsetfillcolor{currentfill}%
\pgfsetlinewidth{0.000000pt}%
\definecolor{currentstroke}{rgb}{0.192357,0.403199,0.555836}%
\pgfsetstrokecolor{currentstroke}%
\pgfsetdash{}{0pt}%
\pgfpathmoveto{\pgfqpoint{5.373172in}{3.514648in}}%
\pgfpathlineto{\pgfqpoint{5.510865in}{3.407873in}}%
\pgfpathlineto{\pgfqpoint{5.586291in}{3.415267in}}%
\pgfpathclose%
\pgfusepath{fill}%
\end{pgfscope}%
\begin{pgfscope}%
\pgfpathrectangle{\pgfqpoint{0.539299in}{0.078740in}}{\pgfqpoint{7.842520in}{7.842520in}}%
\pgfusepath{clip}%
\pgfsetbuttcap%
\pgfsetroundjoin%
\definecolor{currentfill}{rgb}{0.280868,0.160771,0.472899}%
\pgfsetfillcolor{currentfill}%
\pgfsetlinewidth{0.000000pt}%
\definecolor{currentstroke}{rgb}{0.190631,0.407061,0.556089}%
\pgfsetstrokecolor{currentstroke}%
\pgfsetdash{}{0pt}%
\pgfpathmoveto{\pgfqpoint{6.491826in}{3.002343in}}%
\pgfpathlineto{\pgfqpoint{6.564795in}{3.052267in}}%
\pgfpathlineto{\pgfqpoint{6.352735in}{3.080591in}}%
\pgfpathclose%
\pgfusepath{fill}%
\end{pgfscope}%
\begin{pgfscope}%
\pgfpathrectangle{\pgfqpoint{0.539299in}{0.078740in}}{\pgfqpoint{7.842520in}{7.842520in}}%
\pgfusepath{clip}%
\pgfsetbuttcap%
\pgfsetroundjoin%
\definecolor{currentfill}{rgb}{0.180653,0.701402,0.488189}%
\pgfsetfillcolor{currentfill}%
\pgfsetlinewidth{0.000000pt}%
\definecolor{currentstroke}{rgb}{0.188923,0.410910,0.556326}%
\pgfsetstrokecolor{currentstroke}%
\pgfsetdash{}{0pt}%
\pgfpathmoveto{\pgfqpoint{3.380711in}{5.141597in}}%
\pgfpathlineto{\pgfqpoint{3.429695in}{5.281129in}}%
\pgfpathlineto{\pgfqpoint{3.513793in}{5.222466in}}%
\pgfpathclose%
\pgfusepath{fill}%
\end{pgfscope}%
\begin{pgfscope}%
\pgfpathrectangle{\pgfqpoint{0.539299in}{0.078740in}}{\pgfqpoint{7.842520in}{7.842520in}}%
\pgfusepath{clip}%
\pgfsetbuttcap%
\pgfsetroundjoin%
\definecolor{currentfill}{rgb}{0.276022,0.044167,0.370164}%
\pgfsetfillcolor{currentfill}%
\pgfsetlinewidth{0.000000pt}%
\definecolor{currentstroke}{rgb}{0.187231,0.414746,0.556547}%
\pgfsetstrokecolor{currentstroke}%
\pgfsetdash{}{0pt}%
\pgfpathmoveto{\pgfqpoint{7.331215in}{2.641797in}}%
\pgfpathlineto{\pgfqpoint{7.192040in}{2.737188in}}%
\pgfpathlineto{\pgfqpoint{7.120863in}{2.683197in}}%
\pgfpathclose%
\pgfusepath{fill}%
\end{pgfscope}%
\begin{pgfscope}%
\pgfpathrectangle{\pgfqpoint{0.539299in}{0.078740in}}{\pgfqpoint{7.842520in}{7.842520in}}%
\pgfusepath{clip}%
\pgfsetbuttcap%
\pgfsetroundjoin%
\definecolor{currentfill}{rgb}{0.262138,0.242286,0.520837}%
\pgfsetfillcolor{currentfill}%
\pgfsetlinewidth{0.000000pt}%
\definecolor{currentstroke}{rgb}{0.185556,0.418570,0.556753}%
\pgfsetstrokecolor{currentstroke}%
\pgfsetdash{}{0pt}%
\pgfpathmoveto{\pgfqpoint{5.862282in}{3.253769in}}%
\pgfpathlineto{\pgfqpoint{5.724096in}{3.331845in}}%
\pgfpathlineto{\pgfqpoint{5.648894in}{3.310389in}}%
\pgfpathclose%
\pgfusepath{fill}%
\end{pgfscope}%
\begin{pgfscope}%
\pgfpathrectangle{\pgfqpoint{0.539299in}{0.078740in}}{\pgfqpoint{7.842520in}{7.842520in}}%
\pgfusepath{clip}%
\pgfsetbuttcap%
\pgfsetroundjoin%
\definecolor{currentfill}{rgb}{0.147607,0.511733,0.557049}%
\pgfsetfillcolor{currentfill}%
\pgfsetlinewidth{0.000000pt}%
\definecolor{currentstroke}{rgb}{0.183898,0.422383,0.556944}%
\pgfsetstrokecolor{currentstroke}%
\pgfsetdash{}{0pt}%
\pgfpathmoveto{\pgfqpoint{2.699021in}{4.682479in}}%
\pgfpathlineto{\pgfqpoint{2.577164in}{4.260977in}}%
\pgfpathlineto{\pgfqpoint{2.491969in}{4.215044in}}%
\pgfpathclose%
\pgfusepath{fill}%
\end{pgfscope}%
\begin{pgfscope}%
\pgfpathrectangle{\pgfqpoint{0.539299in}{0.078740in}}{\pgfqpoint{7.842520in}{7.842520in}}%
\pgfusepath{clip}%
\pgfsetbuttcap%
\pgfsetroundjoin%
\definecolor{currentfill}{rgb}{0.175707,0.697900,0.491033}%
\pgfsetfillcolor{currentfill}%
\pgfsetlinewidth{0.000000pt}%
\definecolor{currentstroke}{rgb}{0.182256,0.426184,0.557120}%
\pgfsetstrokecolor{currentstroke}%
\pgfsetdash{}{0pt}%
\pgfpathmoveto{\pgfqpoint{3.296607in}{5.181085in}}%
\pgfpathlineto{\pgfqpoint{3.429695in}{5.281129in}}%
\pgfpathlineto{\pgfqpoint{3.380711in}{5.141597in}}%
\pgfpathclose%
\pgfusepath{fill}%
\end{pgfscope}%
\begin{pgfscope}%
\pgfpathrectangle{\pgfqpoint{0.539299in}{0.078740in}}{\pgfqpoint{7.842520in}{7.842520in}}%
\pgfusepath{clip}%
\pgfsetbuttcap%
\pgfsetroundjoin%
\definecolor{currentfill}{rgb}{0.277134,0.185228,0.489898}%
\pgfsetfillcolor{currentfill}%
\pgfsetlinewidth{0.000000pt}%
\definecolor{currentstroke}{rgb}{0.180629,0.429975,0.557282}%
\pgfsetstrokecolor{currentstroke}%
\pgfsetdash{}{0pt}%
\pgfpathmoveto{\pgfqpoint{6.352735in}{3.080591in}}%
\pgfpathlineto{\pgfqpoint{6.213837in}{3.153582in}}%
\pgfpathlineto{\pgfqpoint{6.139708in}{3.103444in}}%
\pgfpathclose%
\pgfusepath{fill}%
\end{pgfscope}%
\begin{pgfscope}%
\pgfpathrectangle{\pgfqpoint{0.539299in}{0.078740in}}{\pgfqpoint{7.842520in}{7.842520in}}%
\pgfusepath{clip}%
\pgfsetbuttcap%
\pgfsetroundjoin%
\definecolor{currentfill}{rgb}{0.185556,0.418570,0.556753}%
\pgfsetfillcolor{currentfill}%
\pgfsetlinewidth{0.000000pt}%
\definecolor{currentstroke}{rgb}{0.179019,0.433756,0.557430}%
\pgfsetstrokecolor{currentstroke}%
\pgfsetdash{}{0pt}%
\pgfpathmoveto{\pgfqpoint{2.406429in}{4.162525in}}%
\pgfpathlineto{\pgfqpoint{2.290658in}{3.630794in}}%
\pgfpathlineto{\pgfqpoint{2.320596in}{4.102334in}}%
\pgfpathclose%
\pgfusepath{fill}%
\end{pgfscope}%
\begin{pgfscope}%
\pgfpathrectangle{\pgfqpoint{0.539299in}{0.078740in}}{\pgfqpoint{7.842520in}{7.842520in}}%
\pgfusepath{clip}%
\pgfsetbuttcap%
\pgfsetroundjoin%
\definecolor{currentfill}{rgb}{0.279566,0.067836,0.391917}%
\pgfsetfillcolor{currentfill}%
\pgfsetlinewidth{0.000000pt}%
\definecolor{currentstroke}{rgb}{0.177423,0.437527,0.557565}%
\pgfsetstrokecolor{currentstroke}%
\pgfsetdash{}{0pt}%
\pgfpathmoveto{\pgfqpoint{7.120863in}{2.683197in}}%
\pgfpathlineto{\pgfqpoint{7.192040in}{2.737188in}}%
\pgfpathlineto{\pgfqpoint{6.981719in}{2.783278in}}%
\pgfpathclose%
\pgfusepath{fill}%
\end{pgfscope}%
\begin{pgfscope}%
\pgfpathrectangle{\pgfqpoint{0.539299in}{0.078740in}}{\pgfqpoint{7.842520in}{7.842520in}}%
\pgfusepath{clip}%
\pgfsetbuttcap%
\pgfsetroundjoin%
\definecolor{currentfill}{rgb}{0.274128,0.199721,0.498911}%
\pgfsetfillcolor{currentfill}%
\pgfsetlinewidth{0.000000pt}%
\definecolor{currentstroke}{rgb}{0.175841,0.441290,0.557685}%
\pgfsetstrokecolor{currentstroke}%
\pgfsetdash{}{0pt}%
\pgfpathmoveto{\pgfqpoint{6.139708in}{3.103444in}}%
\pgfpathlineto{\pgfqpoint{6.213837in}{3.153582in}}%
\pgfpathlineto{\pgfqpoint{6.000832in}{3.178458in}}%
\pgfpathclose%
\pgfusepath{fill}%
\end{pgfscope}%
\begin{pgfscope}%
\pgfpathrectangle{\pgfqpoint{0.539299in}{0.078740in}}{\pgfqpoint{7.842520in}{7.842520in}}%
\pgfusepath{clip}%
\pgfsetbuttcap%
\pgfsetroundjoin%
\definecolor{currentfill}{rgb}{0.132444,0.552216,0.553018}%
\pgfsetfillcolor{currentfill}%
\pgfsetlinewidth{0.000000pt}%
\definecolor{currentstroke}{rgb}{0.174274,0.445044,0.557792}%
\pgfsetstrokecolor{currentstroke}%
\pgfsetdash{}{0pt}%
\pgfpathmoveto{\pgfqpoint{2.577164in}{4.260977in}}%
\pgfpathlineto{\pgfqpoint{2.699021in}{4.682479in}}%
\pgfpathlineto{\pgfqpoint{2.784458in}{4.707293in}}%
\pgfpathclose%
\pgfusepath{fill}%
\end{pgfscope}%
\begin{pgfscope}%
\pgfpathrectangle{\pgfqpoint{0.539299in}{0.078740in}}{\pgfqpoint{7.842520in}{7.842520in}}%
\pgfusepath{clip}%
\pgfsetbuttcap%
\pgfsetroundjoin%
\definecolor{currentfill}{rgb}{0.255645,0.260703,0.528312}%
\pgfsetfillcolor{currentfill}%
\pgfsetlinewidth{0.000000pt}%
\definecolor{currentstroke}{rgb}{0.172719,0.448791,0.557885}%
\pgfsetstrokecolor{currentstroke}%
\pgfsetdash{}{0pt}%
\pgfpathmoveto{\pgfqpoint{5.724096in}{3.331845in}}%
\pgfpathlineto{\pgfqpoint{5.510865in}{3.407873in}}%
\pgfpathlineto{\pgfqpoint{5.648894in}{3.310389in}}%
\pgfpathclose%
\pgfusepath{fill}%
\end{pgfscope}%
\begin{pgfscope}%
\pgfpathrectangle{\pgfqpoint{0.539299in}{0.078740in}}{\pgfqpoint{7.842520in}{7.842520in}}%
\pgfusepath{clip}%
\pgfsetbuttcap%
\pgfsetroundjoin%
\definecolor{currentfill}{rgb}{0.146616,0.673050,0.508936}%
\pgfsetfillcolor{currentfill}%
\pgfsetlinewidth{0.000000pt}%
\definecolor{currentstroke}{rgb}{0.171176,0.452530,0.557965}%
\pgfsetstrokecolor{currentstroke}%
\pgfsetdash{}{0pt}%
\pgfpathmoveto{\pgfqpoint{4.003621in}{5.004426in}}%
\pgfpathlineto{\pgfqpoint{3.867334in}{5.102013in}}%
\pgfpathlineto{\pgfqpoint{3.921341in}{5.107578in}}%
\pgfpathclose%
\pgfusepath{fill}%
\end{pgfscope}%
\begin{pgfscope}%
\pgfpathrectangle{\pgfqpoint{0.539299in}{0.078740in}}{\pgfqpoint{7.842520in}{7.842520in}}%
\pgfusepath{clip}%
\pgfsetbuttcap%
\pgfsetroundjoin%
\definecolor{currentfill}{rgb}{0.119483,0.614817,0.537692}%
\pgfsetfillcolor{currentfill}%
\pgfsetlinewidth{0.000000pt}%
\definecolor{currentstroke}{rgb}{0.169646,0.456262,0.558030}%
\pgfsetstrokecolor{currentstroke}%
\pgfsetdash{}{0pt}%
\pgfpathmoveto{\pgfqpoint{2.784458in}{4.707293in}}%
\pgfpathlineto{\pgfqpoint{2.996415in}{5.025431in}}%
\pgfpathlineto{\pgfqpoint{2.869477in}{4.725189in}}%
\pgfpathclose%
\pgfusepath{fill}%
\end{pgfscope}%
\begin{pgfscope}%
\pgfpathrectangle{\pgfqpoint{0.539299in}{0.078740in}}{\pgfqpoint{7.842520in}{7.842520in}}%
\pgfusepath{clip}%
\pgfsetbuttcap%
\pgfsetroundjoin%
\definecolor{currentfill}{rgb}{0.283229,0.120777,0.440584}%
\pgfsetfillcolor{currentfill}%
\pgfsetlinewidth{0.000000pt}%
\definecolor{currentstroke}{rgb}{0.168126,0.459988,0.558082}%
\pgfsetstrokecolor{currentstroke}%
\pgfsetdash{}{0pt}%
\pgfpathmoveto{\pgfqpoint{6.770368in}{2.826882in}}%
\pgfpathlineto{\pgfqpoint{6.842661in}{2.878816in}}%
\pgfpathlineto{\pgfqpoint{6.703683in}{2.968722in}}%
\pgfpathclose%
\pgfusepath{fill}%
\end{pgfscope}%
\begin{pgfscope}%
\pgfpathrectangle{\pgfqpoint{0.539299in}{0.078740in}}{\pgfqpoint{7.842520in}{7.842520in}}%
\pgfusepath{clip}%
\pgfsetbuttcap%
\pgfsetroundjoin%
\definecolor{currentfill}{rgb}{0.180653,0.701402,0.488189}%
\pgfsetfillcolor{currentfill}%
\pgfsetlinewidth{0.000000pt}%
\definecolor{currentstroke}{rgb}{0.166617,0.463708,0.558119}%
\pgfsetstrokecolor{currentstroke}%
\pgfsetdash{}{0pt}%
\pgfpathmoveto{\pgfqpoint{3.784535in}{5.195377in}}%
\pgfpathlineto{\pgfqpoint{3.731742in}{5.157127in}}%
\pgfpathlineto{\pgfqpoint{3.648554in}{5.237108in}}%
\pgfpathclose%
\pgfusepath{fill}%
\end{pgfscope}%
\begin{pgfscope}%
\pgfpathrectangle{\pgfqpoint{0.539299in}{0.078740in}}{\pgfqpoint{7.842520in}{7.842520in}}%
\pgfusepath{clip}%
\pgfsetbuttcap%
\pgfsetroundjoin%
\definecolor{currentfill}{rgb}{0.282290,0.145912,0.461510}%
\pgfsetfillcolor{currentfill}%
\pgfsetlinewidth{0.000000pt}%
\definecolor{currentstroke}{rgb}{0.165117,0.467423,0.558141}%
\pgfsetstrokecolor{currentstroke}%
\pgfsetdash{}{0pt}%
\pgfpathmoveto{\pgfqpoint{6.491826in}{3.002343in}}%
\pgfpathlineto{\pgfqpoint{6.631052in}{2.917912in}}%
\pgfpathlineto{\pgfqpoint{6.703683in}{2.968722in}}%
\pgfpathclose%
\pgfusepath{fill}%
\end{pgfscope}%
\begin{pgfscope}%
\pgfpathrectangle{\pgfqpoint{0.539299in}{0.078740in}}{\pgfqpoint{7.842520in}{7.842520in}}%
\pgfusepath{clip}%
\pgfsetbuttcap%
\pgfsetroundjoin%
\definecolor{currentfill}{rgb}{0.183898,0.422383,0.556944}%
\pgfsetfillcolor{currentfill}%
\pgfsetlinewidth{0.000000pt}%
\definecolor{currentstroke}{rgb}{0.163625,0.471133,0.558148}%
\pgfsetstrokecolor{currentstroke}%
\pgfsetdash{}{0pt}%
\pgfpathmoveto{\pgfqpoint{4.961637in}{3.904256in}}%
\pgfpathlineto{\pgfqpoint{4.824755in}{4.057714in}}%
\pgfpathlineto{\pgfqpoint{4.884097in}{3.971926in}}%
\pgfpathclose%
\pgfusepath{fill}%
\end{pgfscope}%
\begin{pgfscope}%
\pgfpathrectangle{\pgfqpoint{0.539299in}{0.078740in}}{\pgfqpoint{7.842520in}{7.842520in}}%
\pgfusepath{clip}%
\pgfsetbuttcap%
\pgfsetroundjoin%
\definecolor{currentfill}{rgb}{0.194100,0.399323,0.555565}%
\pgfsetfillcolor{currentfill}%
\pgfsetlinewidth{0.000000pt}%
\definecolor{currentstroke}{rgb}{0.162142,0.474838,0.558140}%
\pgfsetstrokecolor{currentstroke}%
\pgfsetdash{}{0pt}%
\pgfpathmoveto{\pgfqpoint{4.884097in}{3.971926in}}%
\pgfpathlineto{\pgfqpoint{5.098620in}{3.762206in}}%
\pgfpathlineto{\pgfqpoint{4.961637in}{3.904256in}}%
\pgfpathclose%
\pgfusepath{fill}%
\end{pgfscope}%
\begin{pgfscope}%
\pgfpathrectangle{\pgfqpoint{0.539299in}{0.078740in}}{\pgfqpoint{7.842520in}{7.842520in}}%
\pgfusepath{clip}%
\pgfsetbuttcap%
\pgfsetroundjoin%
\definecolor{currentfill}{rgb}{0.162142,0.474838,0.558140}%
\pgfsetfillcolor{currentfill}%
\pgfsetlinewidth{0.000000pt}%
\definecolor{currentstroke}{rgb}{0.160665,0.478540,0.558115}%
\pgfsetstrokecolor{currentstroke}%
\pgfsetdash{}{0pt}%
\pgfpathmoveto{\pgfqpoint{4.609098in}{4.314481in}}%
\pgfpathlineto{\pgfqpoint{4.824755in}{4.057714in}}%
\pgfpathlineto{\pgfqpoint{4.687905in}{4.220515in}}%
\pgfpathclose%
\pgfusepath{fill}%
\end{pgfscope}%
\begin{pgfscope}%
\pgfpathrectangle{\pgfqpoint{0.539299in}{0.078740in}}{\pgfqpoint{7.842520in}{7.842520in}}%
\pgfusepath{clip}%
\pgfsetbuttcap%
\pgfsetroundjoin%
\definecolor{currentfill}{rgb}{0.214298,0.355619,0.551184}%
\pgfsetfillcolor{currentfill}%
\pgfsetlinewidth{0.000000pt}%
\definecolor{currentstroke}{rgb}{0.159194,0.482237,0.558073}%
\pgfsetstrokecolor{currentstroke}%
\pgfsetdash{}{0pt}%
\pgfpathmoveto{\pgfqpoint{2.205440in}{3.565187in}}%
\pgfpathlineto{\pgfqpoint{2.119879in}{3.494959in}}%
\pgfpathlineto{\pgfqpoint{2.234531in}{4.033237in}}%
\pgfpathclose%
\pgfusepath{fill}%
\end{pgfscope}%
\begin{pgfscope}%
\pgfpathrectangle{\pgfqpoint{0.539299in}{0.078740in}}{\pgfqpoint{7.842520in}{7.842520in}}%
\pgfusepath{clip}%
\pgfsetbuttcap%
\pgfsetroundjoin%
\definecolor{currentfill}{rgb}{0.151918,0.500685,0.557587}%
\pgfsetfillcolor{currentfill}%
\pgfsetlinewidth{0.000000pt}%
\definecolor{currentstroke}{rgb}{0.157729,0.485932,0.558013}%
\pgfsetstrokecolor{currentstroke}%
\pgfsetdash{}{0pt}%
\pgfpathmoveto{\pgfqpoint{4.687905in}{4.220515in}}%
\pgfpathlineto{\pgfqpoint{4.551037in}{4.389232in}}%
\pgfpathlineto{\pgfqpoint{4.609098in}{4.314481in}}%
\pgfpathclose%
\pgfusepath{fill}%
\end{pgfscope}%
\begin{pgfscope}%
\pgfpathrectangle{\pgfqpoint{0.539299in}{0.078740in}}{\pgfqpoint{7.842520in}{7.842520in}}%
\pgfusepath{clip}%
\pgfsetbuttcap%
\pgfsetroundjoin%
\definecolor{currentfill}{rgb}{0.134692,0.658636,0.517649}%
\pgfsetfillcolor{currentfill}%
\pgfsetlinewidth{0.000000pt}%
\definecolor{currentstroke}{rgb}{0.156270,0.489624,0.557936}%
\pgfsetstrokecolor{currentstroke}%
\pgfsetdash{}{0pt}%
\pgfpathmoveto{\pgfqpoint{3.921341in}{5.107578in}}%
\pgfpathlineto{\pgfqpoint{4.140312in}{4.874931in}}%
\pgfpathlineto{\pgfqpoint{4.003621in}{5.004426in}}%
\pgfpathclose%
\pgfusepath{fill}%
\end{pgfscope}%
\begin{pgfscope}%
\pgfpathrectangle{\pgfqpoint{0.539299in}{0.078740in}}{\pgfqpoint{7.842520in}{7.842520in}}%
\pgfusepath{clip}%
\pgfsetbuttcap%
\pgfsetroundjoin%
\definecolor{currentfill}{rgb}{0.267968,0.223549,0.512008}%
\pgfsetfillcolor{currentfill}%
\pgfsetlinewidth{0.000000pt}%
\definecolor{currentstroke}{rgb}{0.154815,0.493313,0.557840}%
\pgfsetstrokecolor{currentstroke}%
\pgfsetdash{}{0pt}%
\pgfpathmoveto{\pgfqpoint{6.000832in}{3.178458in}}%
\pgfpathlineto{\pgfqpoint{5.862282in}{3.253769in}}%
\pgfpathlineto{\pgfqpoint{5.787278in}{3.220081in}}%
\pgfpathclose%
\pgfusepath{fill}%
\end{pgfscope}%
\begin{pgfscope}%
\pgfpathrectangle{\pgfqpoint{0.539299in}{0.078740in}}{\pgfqpoint{7.842520in}{7.842520in}}%
\pgfusepath{clip}%
\pgfsetbuttcap%
\pgfsetroundjoin%
\definecolor{currentfill}{rgb}{0.214298,0.355619,0.551184}%
\pgfsetfillcolor{currentfill}%
\pgfsetlinewidth{0.000000pt}%
\definecolor{currentstroke}{rgb}{0.153364,0.497000,0.557724}%
\pgfsetstrokecolor{currentstroke}%
\pgfsetdash{}{0pt}%
\pgfpathmoveto{\pgfqpoint{5.159216in}{3.667393in}}%
\pgfpathlineto{\pgfqpoint{5.235775in}{3.632407in}}%
\pgfpathlineto{\pgfqpoint{5.098620in}{3.762206in}}%
\pgfpathclose%
\pgfusepath{fill}%
\end{pgfscope}%
\begin{pgfscope}%
\pgfpathrectangle{\pgfqpoint{0.539299in}{0.078740in}}{\pgfqpoint{7.842520in}{7.842520in}}%
\pgfusepath{clip}%
\pgfsetbuttcap%
\pgfsetroundjoin%
\definecolor{currentfill}{rgb}{0.274952,0.037752,0.364543}%
\pgfsetfillcolor{currentfill}%
\pgfsetlinewidth{0.000000pt}%
\definecolor{currentstroke}{rgb}{0.151918,0.500685,0.557587}%
\pgfsetstrokecolor{currentstroke}%
\pgfsetdash{}{0pt}%
\pgfpathmoveto{\pgfqpoint{7.120863in}{2.683197in}}%
\pgfpathlineto{\pgfqpoint{7.260119in}{2.579903in}}%
\pgfpathlineto{\pgfqpoint{7.331215in}{2.641797in}}%
\pgfpathclose%
\pgfusepath{fill}%
\end{pgfscope}%
\begin{pgfscope}%
\pgfpathrectangle{\pgfqpoint{0.539299in}{0.078740in}}{\pgfqpoint{7.842520in}{7.842520in}}%
\pgfusepath{clip}%
\pgfsetbuttcap%
\pgfsetroundjoin%
\definecolor{currentfill}{rgb}{0.131172,0.555899,0.552459}%
\pgfsetfillcolor{currentfill}%
\pgfsetlinewidth{0.000000pt}%
\definecolor{currentstroke}{rgb}{0.150476,0.504369,0.557430}%
\pgfsetstrokecolor{currentstroke}%
\pgfsetdash{}{0pt}%
\pgfpathmoveto{\pgfqpoint{4.333879in}{4.667896in}}%
\pgfpathlineto{\pgfqpoint{4.551037in}{4.389232in}}%
\pgfpathlineto{\pgfqpoint{4.414126in}{4.558993in}}%
\pgfpathclose%
\pgfusepath{fill}%
\end{pgfscope}%
\begin{pgfscope}%
\pgfpathrectangle{\pgfqpoint{0.539299in}{0.078740in}}{\pgfqpoint{7.842520in}{7.842520in}}%
\pgfusepath{clip}%
\pgfsetbuttcap%
\pgfsetroundjoin%
\definecolor{currentfill}{rgb}{0.123463,0.581687,0.547445}%
\pgfsetfillcolor{currentfill}%
\pgfsetlinewidth{0.000000pt}%
\definecolor{currentstroke}{rgb}{0.149039,0.508051,0.557250}%
\pgfsetstrokecolor{currentstroke}%
\pgfsetdash{}{0pt}%
\pgfpathmoveto{\pgfqpoint{4.333879in}{4.667896in}}%
\pgfpathlineto{\pgfqpoint{4.414126in}{4.558993in}}%
\pgfpathlineto{\pgfqpoint{4.277193in}{4.723467in}}%
\pgfpathclose%
\pgfusepath{fill}%
\end{pgfscope}%
\begin{pgfscope}%
\pgfpathrectangle{\pgfqpoint{0.539299in}{0.078740in}}{\pgfqpoint{7.842520in}{7.842520in}}%
\pgfusepath{clip}%
\pgfsetbuttcap%
\pgfsetroundjoin%
\definecolor{currentfill}{rgb}{0.119699,0.618490,0.536347}%
\pgfsetfillcolor{currentfill}%
\pgfsetlinewidth{0.000000pt}%
\definecolor{currentstroke}{rgb}{0.147607,0.511733,0.557049}%
\pgfsetstrokecolor{currentstroke}%
\pgfsetdash{}{0pt}%
\pgfpathmoveto{\pgfqpoint{4.140312in}{4.874931in}}%
\pgfpathlineto{\pgfqpoint{4.196221in}{4.834423in}}%
\pgfpathlineto{\pgfqpoint{4.277193in}{4.723467in}}%
\pgfpathclose%
\pgfusepath{fill}%
\end{pgfscope}%
\begin{pgfscope}%
\pgfpathrectangle{\pgfqpoint{0.539299in}{0.078740in}}{\pgfqpoint{7.842520in}{7.842520in}}%
\pgfusepath{clip}%
\pgfsetbuttcap%
\pgfsetroundjoin%
\definecolor{currentfill}{rgb}{0.283072,0.130895,0.449241}%
\pgfsetfillcolor{currentfill}%
\pgfsetlinewidth{0.000000pt}%
\definecolor{currentstroke}{rgb}{0.146180,0.515413,0.556823}%
\pgfsetstrokecolor{currentstroke}%
\pgfsetdash{}{0pt}%
\pgfpathmoveto{\pgfqpoint{6.703683in}{2.968722in}}%
\pgfpathlineto{\pgfqpoint{6.631052in}{2.917912in}}%
\pgfpathlineto{\pgfqpoint{6.770368in}{2.826882in}}%
\pgfpathclose%
\pgfusepath{fill}%
\end{pgfscope}%
\begin{pgfscope}%
\pgfpathrectangle{\pgfqpoint{0.539299in}{0.078740in}}{\pgfqpoint{7.842520in}{7.842520in}}%
\pgfusepath{clip}%
\pgfsetbuttcap%
\pgfsetroundjoin%
\definecolor{currentfill}{rgb}{0.223925,0.334994,0.548053}%
\pgfsetfillcolor{currentfill}%
\pgfsetlinewidth{0.000000pt}%
\definecolor{currentstroke}{rgb}{0.144759,0.519093,0.556572}%
\pgfsetstrokecolor{currentstroke}%
\pgfsetdash{}{0pt}%
\pgfpathmoveto{\pgfqpoint{5.235775in}{3.632407in}}%
\pgfpathlineto{\pgfqpoint{5.159216in}{3.667393in}}%
\pgfpathlineto{\pgfqpoint{5.373172in}{3.514648in}}%
\pgfpathclose%
\pgfusepath{fill}%
\end{pgfscope}%
\begin{pgfscope}%
\pgfpathrectangle{\pgfqpoint{0.539299in}{0.078740in}}{\pgfqpoint{7.842520in}{7.842520in}}%
\pgfusepath{clip}%
\pgfsetbuttcap%
\pgfsetroundjoin%
\definecolor{currentfill}{rgb}{0.262138,0.242286,0.520837}%
\pgfsetfillcolor{currentfill}%
\pgfsetlinewidth{0.000000pt}%
\definecolor{currentstroke}{rgb}{0.143343,0.522773,0.556295}%
\pgfsetstrokecolor{currentstroke}%
\pgfsetdash{}{0pt}%
\pgfpathmoveto{\pgfqpoint{5.648894in}{3.310389in}}%
\pgfpathlineto{\pgfqpoint{5.787278in}{3.220081in}}%
\pgfpathlineto{\pgfqpoint{5.862282in}{3.253769in}}%
\pgfpathclose%
\pgfusepath{fill}%
\end{pgfscope}%
\begin{pgfscope}%
\pgfpathrectangle{\pgfqpoint{0.539299in}{0.078740in}}{\pgfqpoint{7.842520in}{7.842520in}}%
\pgfusepath{clip}%
\pgfsetbuttcap%
\pgfsetroundjoin%
\definecolor{currentfill}{rgb}{0.282656,0.100196,0.422160}%
\pgfsetfillcolor{currentfill}%
\pgfsetlinewidth{0.000000pt}%
\definecolor{currentstroke}{rgb}{0.141935,0.526453,0.555991}%
\pgfsetstrokecolor{currentstroke}%
\pgfsetdash{}{0pt}%
\pgfpathmoveto{\pgfqpoint{6.842661in}{2.878816in}}%
\pgfpathlineto{\pgfqpoint{6.909736in}{2.729276in}}%
\pgfpathlineto{\pgfqpoint{6.981719in}{2.783278in}}%
\pgfpathclose%
\pgfusepath{fill}%
\end{pgfscope}%
\begin{pgfscope}%
\pgfpathrectangle{\pgfqpoint{0.539299in}{0.078740in}}{\pgfqpoint{7.842520in}{7.842520in}}%
\pgfusepath{clip}%
\pgfsetbuttcap%
\pgfsetroundjoin%
\definecolor{currentfill}{rgb}{0.170948,0.694384,0.493803}%
\pgfsetfillcolor{currentfill}%
\pgfsetlinewidth{0.000000pt}%
\definecolor{currentstroke}{rgb}{0.140536,0.530132,0.555659}%
\pgfsetstrokecolor{currentstroke}%
\pgfsetdash{}{0pt}%
\pgfpathmoveto{\pgfqpoint{3.081416in}{5.017996in}}%
\pgfpathlineto{\pgfqpoint{3.211903in}{5.214787in}}%
\pgfpathlineto{\pgfqpoint{3.296607in}{5.181085in}}%
\pgfpathclose%
\pgfusepath{fill}%
\end{pgfscope}%
\begin{pgfscope}%
\pgfpathrectangle{\pgfqpoint{0.539299in}{0.078740in}}{\pgfqpoint{7.842520in}{7.842520in}}%
\pgfusepath{clip}%
\pgfsetbuttcap%
\pgfsetroundjoin%
\definecolor{currentfill}{rgb}{0.277134,0.185228,0.489898}%
\pgfsetfillcolor{currentfill}%
\pgfsetlinewidth{0.000000pt}%
\definecolor{currentstroke}{rgb}{0.139147,0.533812,0.555298}%
\pgfsetstrokecolor{currentstroke}%
\pgfsetdash{}{0pt}%
\pgfpathmoveto{\pgfqpoint{6.352735in}{3.080591in}}%
\pgfpathlineto{\pgfqpoint{6.139708in}{3.103444in}}%
\pgfpathlineto{\pgfqpoint{6.278864in}{3.026523in}}%
\pgfpathclose%
\pgfusepath{fill}%
\end{pgfscope}%
\begin{pgfscope}%
\pgfpathrectangle{\pgfqpoint{0.539299in}{0.078740in}}{\pgfqpoint{7.842520in}{7.842520in}}%
\pgfusepath{clip}%
\pgfsetbuttcap%
\pgfsetroundjoin%
\definecolor{currentfill}{rgb}{0.208030,0.718701,0.472873}%
\pgfsetfillcolor{currentfill}%
\pgfsetlinewidth{0.000000pt}%
\definecolor{currentstroke}{rgb}{0.137770,0.537492,0.554906}%
\pgfsetstrokecolor{currentstroke}%
\pgfsetdash{}{0pt}%
\pgfpathmoveto{\pgfqpoint{3.648554in}{5.237108in}}%
\pgfpathlineto{\pgfqpoint{3.513793in}{5.222466in}}%
\pgfpathlineto{\pgfqpoint{3.429695in}{5.281129in}}%
\pgfpathclose%
\pgfusepath{fill}%
\end{pgfscope}%
\begin{pgfscope}%
\pgfpathrectangle{\pgfqpoint{0.539299in}{0.078740in}}{\pgfqpoint{7.842520in}{7.842520in}}%
\pgfusepath{clip}%
\pgfsetbuttcap%
\pgfsetroundjoin%
\definecolor{currentfill}{rgb}{0.280255,0.165693,0.476498}%
\pgfsetfillcolor{currentfill}%
\pgfsetlinewidth{0.000000pt}%
\definecolor{currentstroke}{rgb}{0.136408,0.541173,0.554483}%
\pgfsetstrokecolor{currentstroke}%
\pgfsetdash{}{0pt}%
\pgfpathmoveto{\pgfqpoint{6.491826in}{3.002343in}}%
\pgfpathlineto{\pgfqpoint{6.352735in}{3.080591in}}%
\pgfpathlineto{\pgfqpoint{6.418242in}{2.945854in}}%
\pgfpathclose%
\pgfusepath{fill}%
\end{pgfscope}%
\begin{pgfscope}%
\pgfpathrectangle{\pgfqpoint{0.539299in}{0.078740in}}{\pgfqpoint{7.842520in}{7.842520in}}%
\pgfusepath{clip}%
\pgfsetbuttcap%
\pgfsetroundjoin%
\definecolor{currentfill}{rgb}{0.170948,0.694384,0.493803}%
\pgfsetfillcolor{currentfill}%
\pgfsetlinewidth{0.000000pt}%
\definecolor{currentstroke}{rgb}{0.135066,0.544853,0.554029}%
\pgfsetstrokecolor{currentstroke}%
\pgfsetdash{}{0pt}%
\pgfpathmoveto{\pgfqpoint{3.867334in}{5.102013in}}%
\pgfpathlineto{\pgfqpoint{3.784535in}{5.195377in}}%
\pgfpathlineto{\pgfqpoint{3.921341in}{5.107578in}}%
\pgfpathclose%
\pgfusepath{fill}%
\end{pgfscope}%
\begin{pgfscope}%
\pgfpathrectangle{\pgfqpoint{0.539299in}{0.078740in}}{\pgfqpoint{7.842520in}{7.842520in}}%
\pgfusepath{clip}%
\pgfsetbuttcap%
\pgfsetroundjoin%
\definecolor{currentfill}{rgb}{0.281446,0.084320,0.407414}%
\pgfsetfillcolor{currentfill}%
\pgfsetlinewidth{0.000000pt}%
\definecolor{currentstroke}{rgb}{0.133743,0.548535,0.553541}%
\pgfsetstrokecolor{currentstroke}%
\pgfsetdash{}{0pt}%
\pgfpathmoveto{\pgfqpoint{7.120863in}{2.683197in}}%
\pgfpathlineto{\pgfqpoint{6.981719in}{2.783278in}}%
\pgfpathlineto{\pgfqpoint{6.909736in}{2.729276in}}%
\pgfpathclose%
\pgfusepath{fill}%
\end{pgfscope}%
\begin{pgfscope}%
\pgfpathrectangle{\pgfqpoint{0.539299in}{0.078740in}}{\pgfqpoint{7.842520in}{7.842520in}}%
\pgfusepath{clip}%
\pgfsetbuttcap%
\pgfsetroundjoin%
\definecolor{currentfill}{rgb}{0.241237,0.296485,0.539709}%
\pgfsetfillcolor{currentfill}%
\pgfsetlinewidth{0.000000pt}%
\definecolor{currentstroke}{rgb}{0.132444,0.552216,0.553018}%
\pgfsetstrokecolor{currentstroke}%
\pgfsetdash{}{0pt}%
\pgfpathmoveto{\pgfqpoint{5.373172in}{3.514648in}}%
\pgfpathlineto{\pgfqpoint{5.434988in}{3.409236in}}%
\pgfpathlineto{\pgfqpoint{5.510865in}{3.407873in}}%
\pgfpathclose%
\pgfusepath{fill}%
\end{pgfscope}%
\begin{pgfscope}%
\pgfpathrectangle{\pgfqpoint{0.539299in}{0.078740in}}{\pgfqpoint{7.842520in}{7.842520in}}%
\pgfusepath{clip}%
\pgfsetbuttcap%
\pgfsetroundjoin%
\definecolor{currentfill}{rgb}{0.283091,0.110553,0.431554}%
\pgfsetfillcolor{currentfill}%
\pgfsetlinewidth{0.000000pt}%
\definecolor{currentstroke}{rgb}{0.131172,0.555899,0.552459}%
\pgfsetstrokecolor{currentstroke}%
\pgfsetdash{}{0pt}%
\pgfpathmoveto{\pgfqpoint{6.770368in}{2.826882in}}%
\pgfpathlineto{\pgfqpoint{6.909736in}{2.729276in}}%
\pgfpathlineto{\pgfqpoint{6.842661in}{2.878816in}}%
\pgfpathclose%
\pgfusepath{fill}%
\end{pgfscope}%
\begin{pgfscope}%
\pgfpathrectangle{\pgfqpoint{0.539299in}{0.078740in}}{\pgfqpoint{7.842520in}{7.842520in}}%
\pgfusepath{clip}%
\pgfsetbuttcap%
\pgfsetroundjoin%
\definecolor{currentfill}{rgb}{0.278826,0.175490,0.483397}%
\pgfsetfillcolor{currentfill}%
\pgfsetlinewidth{0.000000pt}%
\definecolor{currentstroke}{rgb}{0.129933,0.559582,0.551864}%
\pgfsetstrokecolor{currentstroke}%
\pgfsetdash{}{0pt}%
\pgfpathmoveto{\pgfqpoint{6.418242in}{2.945854in}}%
\pgfpathlineto{\pgfqpoint{6.352735in}{3.080591in}}%
\pgfpathlineto{\pgfqpoint{6.278864in}{3.026523in}}%
\pgfpathclose%
\pgfusepath{fill}%
\end{pgfscope}%
\begin{pgfscope}%
\pgfpathrectangle{\pgfqpoint{0.539299in}{0.078740in}}{\pgfqpoint{7.842520in}{7.842520in}}%
\pgfusepath{clip}%
\pgfsetbuttcap%
\pgfsetroundjoin%
\definecolor{currentfill}{rgb}{0.171176,0.452530,0.557965}%
\pgfsetfillcolor{currentfill}%
\pgfsetlinewidth{0.000000pt}%
\definecolor{currentstroke}{rgb}{0.128729,0.563265,0.551229}%
\pgfsetstrokecolor{currentstroke}%
\pgfsetdash{}{0pt}%
\pgfpathmoveto{\pgfqpoint{4.884097in}{3.971926in}}%
\pgfpathlineto{\pgfqpoint{4.824755in}{4.057714in}}%
\pgfpathlineto{\pgfqpoint{4.746612in}{4.139665in}}%
\pgfpathclose%
\pgfusepath{fill}%
\end{pgfscope}%
\begin{pgfscope}%
\pgfpathrectangle{\pgfqpoint{0.539299in}{0.078740in}}{\pgfqpoint{7.842520in}{7.842520in}}%
\pgfusepath{clip}%
\pgfsetbuttcap%
\pgfsetroundjoin%
\definecolor{currentfill}{rgb}{0.281412,0.155834,0.469201}%
\pgfsetfillcolor{currentfill}%
\pgfsetlinewidth{0.000000pt}%
\definecolor{currentstroke}{rgb}{0.127568,0.566949,0.550556}%
\pgfsetstrokecolor{currentstroke}%
\pgfsetdash{}{0pt}%
\pgfpathmoveto{\pgfqpoint{6.491826in}{3.002343in}}%
\pgfpathlineto{\pgfqpoint{6.418242in}{2.945854in}}%
\pgfpathlineto{\pgfqpoint{6.631052in}{2.917912in}}%
\pgfpathclose%
\pgfusepath{fill}%
\end{pgfscope}%
\begin{pgfscope}%
\pgfpathrectangle{\pgfqpoint{0.539299in}{0.078740in}}{\pgfqpoint{7.842520in}{7.842520in}}%
\pgfusepath{clip}%
\pgfsetbuttcap%
\pgfsetroundjoin%
\definecolor{currentfill}{rgb}{0.192357,0.403199,0.555836}%
\pgfsetfillcolor{currentfill}%
\pgfsetlinewidth{0.000000pt}%
\definecolor{currentstroke}{rgb}{0.126453,0.570633,0.549841}%
\pgfsetstrokecolor{currentstroke}%
\pgfsetdash{}{0pt}%
\pgfpathmoveto{\pgfqpoint{5.021609in}{3.813991in}}%
\pgfpathlineto{\pgfqpoint{5.098620in}{3.762206in}}%
\pgfpathlineto{\pgfqpoint{4.884097in}{3.971926in}}%
\pgfpathclose%
\pgfusepath{fill}%
\end{pgfscope}%
\begin{pgfscope}%
\pgfpathrectangle{\pgfqpoint{0.539299in}{0.078740in}}{\pgfqpoint{7.842520in}{7.842520in}}%
\pgfusepath{clip}%
\pgfsetbuttcap%
\pgfsetroundjoin%
\definecolor{currentfill}{rgb}{0.160665,0.478540,0.558115}%
\pgfsetfillcolor{currentfill}%
\pgfsetlinewidth{0.000000pt}%
\definecolor{currentstroke}{rgb}{0.125394,0.574318,0.549086}%
\pgfsetstrokecolor{currentstroke}%
\pgfsetdash{}{0pt}%
\pgfpathmoveto{\pgfqpoint{4.746612in}{4.139665in}}%
\pgfpathlineto{\pgfqpoint{4.824755in}{4.057714in}}%
\pgfpathlineto{\pgfqpoint{4.609098in}{4.314481in}}%
\pgfpathclose%
\pgfusepath{fill}%
\end{pgfscope}%
\begin{pgfscope}%
\pgfpathrectangle{\pgfqpoint{0.539299in}{0.078740in}}{\pgfqpoint{7.842520in}{7.842520in}}%
\pgfusepath{clip}%
\pgfsetbuttcap%
\pgfsetroundjoin%
\definecolor{currentfill}{rgb}{0.203063,0.379716,0.553925}%
\pgfsetfillcolor{currentfill}%
\pgfsetlinewidth{0.000000pt}%
\definecolor{currentstroke}{rgb}{0.124395,0.578002,0.548287}%
\pgfsetstrokecolor{currentstroke}%
\pgfsetdash{}{0pt}%
\pgfpathmoveto{\pgfqpoint{5.159216in}{3.667393in}}%
\pgfpathlineto{\pgfqpoint{5.098620in}{3.762206in}}%
\pgfpathlineto{\pgfqpoint{5.021609in}{3.813991in}}%
\pgfpathclose%
\pgfusepath{fill}%
\end{pgfscope}%
\begin{pgfscope}%
\pgfpathrectangle{\pgfqpoint{0.539299in}{0.078740in}}{\pgfqpoint{7.842520in}{7.842520in}}%
\pgfusepath{clip}%
\pgfsetbuttcap%
\pgfsetroundjoin%
\definecolor{currentfill}{rgb}{0.267968,0.223549,0.512008}%
\pgfsetfillcolor{currentfill}%
\pgfsetlinewidth{0.000000pt}%
\definecolor{currentstroke}{rgb}{0.123463,0.581687,0.547445}%
\pgfsetstrokecolor{currentstroke}%
\pgfsetdash{}{0pt}%
\pgfpathmoveto{\pgfqpoint{5.787278in}{3.220081in}}%
\pgfpathlineto{\pgfqpoint{5.926016in}{3.134613in}}%
\pgfpathlineto{\pgfqpoint{6.000832in}{3.178458in}}%
\pgfpathclose%
\pgfusepath{fill}%
\end{pgfscope}%
\begin{pgfscope}%
\pgfpathrectangle{\pgfqpoint{0.539299in}{0.078740in}}{\pgfqpoint{7.842520in}{7.842520in}}%
\pgfusepath{clip}%
\pgfsetbuttcap%
\pgfsetroundjoin%
\definecolor{currentfill}{rgb}{0.273006,0.204520,0.501721}%
\pgfsetfillcolor{currentfill}%
\pgfsetlinewidth{0.000000pt}%
\definecolor{currentstroke}{rgb}{0.122606,0.585371,0.546557}%
\pgfsetstrokecolor{currentstroke}%
\pgfsetdash{}{0pt}%
\pgfpathmoveto{\pgfqpoint{6.065090in}{3.051608in}}%
\pgfpathlineto{\pgfqpoint{6.139708in}{3.103444in}}%
\pgfpathlineto{\pgfqpoint{6.000832in}{3.178458in}}%
\pgfpathclose%
\pgfusepath{fill}%
\end{pgfscope}%
\begin{pgfscope}%
\pgfpathrectangle{\pgfqpoint{0.539299in}{0.078740in}}{\pgfqpoint{7.842520in}{7.842520in}}%
\pgfusepath{clip}%
\pgfsetbuttcap%
\pgfsetroundjoin%
\definecolor{currentfill}{rgb}{0.162016,0.687316,0.499129}%
\pgfsetfillcolor{currentfill}%
\pgfsetlinewidth{0.000000pt}%
\definecolor{currentstroke}{rgb}{0.121831,0.589055,0.545623}%
\pgfsetstrokecolor{currentstroke}%
\pgfsetdash{}{0pt}%
\pgfpathmoveto{\pgfqpoint{2.996415in}{5.025431in}}%
\pgfpathlineto{\pgfqpoint{3.211903in}{5.214787in}}%
\pgfpathlineto{\pgfqpoint{3.081416in}{5.017996in}}%
\pgfpathclose%
\pgfusepath{fill}%
\end{pgfscope}%
\begin{pgfscope}%
\pgfpathrectangle{\pgfqpoint{0.539299in}{0.078740in}}{\pgfqpoint{7.842520in}{7.842520in}}%
\pgfusepath{clip}%
\pgfsetbuttcap%
\pgfsetroundjoin%
\definecolor{currentfill}{rgb}{0.246811,0.283237,0.535941}%
\pgfsetfillcolor{currentfill}%
\pgfsetlinewidth{0.000000pt}%
\definecolor{currentstroke}{rgb}{0.121148,0.592739,0.544641}%
\pgfsetstrokecolor{currentstroke}%
\pgfsetdash{}{0pt}%
\pgfpathmoveto{\pgfqpoint{5.648894in}{3.310389in}}%
\pgfpathlineto{\pgfqpoint{5.510865in}{3.407873in}}%
\pgfpathlineto{\pgfqpoint{5.434988in}{3.409236in}}%
\pgfpathclose%
\pgfusepath{fill}%
\end{pgfscope}%
\begin{pgfscope}%
\pgfpathrectangle{\pgfqpoint{0.539299in}{0.078740in}}{\pgfqpoint{7.842520in}{7.842520in}}%
\pgfusepath{clip}%
\pgfsetbuttcap%
\pgfsetroundjoin%
\definecolor{currentfill}{rgb}{0.140536,0.530132,0.555659}%
\pgfsetfillcolor{currentfill}%
\pgfsetlinewidth{0.000000pt}%
\definecolor{currentstroke}{rgb}{0.120565,0.596422,0.543611}%
\pgfsetstrokecolor{currentstroke}%
\pgfsetdash{}{0pt}%
\pgfpathmoveto{\pgfqpoint{4.609098in}{4.314481in}}%
\pgfpathlineto{\pgfqpoint{4.551037in}{4.389232in}}%
\pgfpathlineto{\pgfqpoint{4.471520in}{4.492348in}}%
\pgfpathclose%
\pgfusepath{fill}%
\end{pgfscope}%
\begin{pgfscope}%
\pgfpathrectangle{\pgfqpoint{0.539299in}{0.078740in}}{\pgfqpoint{7.842520in}{7.842520in}}%
\pgfusepath{clip}%
\pgfsetbuttcap%
\pgfsetroundjoin%
\definecolor{currentfill}{rgb}{0.214298,0.355619,0.551184}%
\pgfsetfillcolor{currentfill}%
\pgfsetlinewidth{0.000000pt}%
\definecolor{currentstroke}{rgb}{0.120092,0.600104,0.542530}%
\pgfsetstrokecolor{currentstroke}%
\pgfsetdash{}{0pt}%
\pgfpathmoveto{\pgfqpoint{2.234531in}{4.033237in}}%
\pgfpathlineto{\pgfqpoint{2.119879in}{3.494959in}}%
\pgfpathlineto{\pgfqpoint{2.034011in}{3.419410in}}%
\pgfpathclose%
\pgfusepath{fill}%
\end{pgfscope}%
\begin{pgfscope}%
\pgfpathrectangle{\pgfqpoint{0.539299in}{0.078740in}}{\pgfqpoint{7.842520in}{7.842520in}}%
\pgfusepath{clip}%
\pgfsetbuttcap%
\pgfsetroundjoin%
\definecolor{currentfill}{rgb}{0.140210,0.665859,0.513427}%
\pgfsetfillcolor{currentfill}%
\pgfsetlinewidth{0.000000pt}%
\definecolor{currentstroke}{rgb}{0.119738,0.603785,0.541400}%
\pgfsetstrokecolor{currentstroke}%
\pgfsetdash{}{0pt}%
\pgfpathmoveto{\pgfqpoint{4.058651in}{4.983988in}}%
\pgfpathlineto{\pgfqpoint{4.140312in}{4.874931in}}%
\pgfpathlineto{\pgfqpoint{3.921341in}{5.107578in}}%
\pgfpathclose%
\pgfusepath{fill}%
\end{pgfscope}%
\begin{pgfscope}%
\pgfpathrectangle{\pgfqpoint{0.539299in}{0.078740in}}{\pgfqpoint{7.842520in}{7.842520in}}%
\pgfusepath{clip}%
\pgfsetbuttcap%
\pgfsetroundjoin%
\definecolor{currentfill}{rgb}{0.277941,0.056324,0.381191}%
\pgfsetfillcolor{currentfill}%
\pgfsetlinewidth{0.000000pt}%
\definecolor{currentstroke}{rgb}{0.119512,0.607464,0.540218}%
\pgfsetstrokecolor{currentstroke}%
\pgfsetdash{}{0pt}%
\pgfpathmoveto{\pgfqpoint{7.049132in}{2.625447in}}%
\pgfpathlineto{\pgfqpoint{7.260119in}{2.579903in}}%
\pgfpathlineto{\pgfqpoint{7.120863in}{2.683197in}}%
\pgfpathclose%
\pgfusepath{fill}%
\end{pgfscope}%
\begin{pgfscope}%
\pgfpathrectangle{\pgfqpoint{0.539299in}{0.078740in}}{\pgfqpoint{7.842520in}{7.842520in}}%
\pgfusepath{clip}%
\pgfsetbuttcap%
\pgfsetroundjoin%
\definecolor{currentfill}{rgb}{0.129933,0.559582,0.551864}%
\pgfsetfillcolor{currentfill}%
\pgfsetlinewidth{0.000000pt}%
\definecolor{currentstroke}{rgb}{0.119423,0.611141,0.538982}%
\pgfsetstrokecolor{currentstroke}%
\pgfsetdash{}{0pt}%
\pgfpathmoveto{\pgfqpoint{4.471520in}{4.492348in}}%
\pgfpathlineto{\pgfqpoint{4.551037in}{4.389232in}}%
\pgfpathlineto{\pgfqpoint{4.333879in}{4.667896in}}%
\pgfpathclose%
\pgfusepath{fill}%
\end{pgfscope}%
\begin{pgfscope}%
\pgfpathrectangle{\pgfqpoint{0.539299in}{0.078740in}}{\pgfqpoint{7.842520in}{7.842520in}}%
\pgfusepath{clip}%
\pgfsetbuttcap%
\pgfsetroundjoin%
\definecolor{currentfill}{rgb}{0.208030,0.718701,0.472873}%
\pgfsetfillcolor{currentfill}%
\pgfsetlinewidth{0.000000pt}%
\definecolor{currentstroke}{rgb}{0.119483,0.614817,0.537692}%
\pgfsetstrokecolor{currentstroke}%
\pgfsetdash{}{0pt}%
\pgfpathmoveto{\pgfqpoint{3.429695in}{5.281129in}}%
\pgfpathlineto{\pgfqpoint{3.296607in}{5.181085in}}%
\pgfpathlineto{\pgfqpoint{3.211903in}{5.214787in}}%
\pgfpathclose%
\pgfusepath{fill}%
\end{pgfscope}%
\begin{pgfscope}%
\pgfpathrectangle{\pgfqpoint{0.539299in}{0.078740in}}{\pgfqpoint{7.842520in}{7.842520in}}%
\pgfusepath{clip}%
\pgfsetbuttcap%
\pgfsetroundjoin%
\definecolor{currentfill}{rgb}{0.223925,0.334994,0.548053}%
\pgfsetfillcolor{currentfill}%
\pgfsetlinewidth{0.000000pt}%
\definecolor{currentstroke}{rgb}{0.119699,0.618490,0.536347}%
\pgfsetstrokecolor{currentstroke}%
\pgfsetdash{}{0pt}%
\pgfpathmoveto{\pgfqpoint{5.373172in}{3.514648in}}%
\pgfpathlineto{\pgfqpoint{5.159216in}{3.667393in}}%
\pgfpathlineto{\pgfqpoint{5.296988in}{3.532612in}}%
\pgfpathclose%
\pgfusepath{fill}%
\end{pgfscope}%
\begin{pgfscope}%
\pgfpathrectangle{\pgfqpoint{0.539299in}{0.078740in}}{\pgfqpoint{7.842520in}{7.842520in}}%
\pgfusepath{clip}%
\pgfsetbuttcap%
\pgfsetroundjoin%
\definecolor{currentfill}{rgb}{0.128087,0.647749,0.523491}%
\pgfsetfillcolor{currentfill}%
\pgfsetlinewidth{0.000000pt}%
\definecolor{currentstroke}{rgb}{0.120081,0.622161,0.534946}%
\pgfsetstrokecolor{currentstroke}%
\pgfsetdash{}{0pt}%
\pgfpathmoveto{\pgfqpoint{4.140312in}{4.874931in}}%
\pgfpathlineto{\pgfqpoint{4.058651in}{4.983988in}}%
\pgfpathlineto{\pgfqpoint{4.196221in}{4.834423in}}%
\pgfpathclose%
\pgfusepath{fill}%
\end{pgfscope}%
\begin{pgfscope}%
\pgfpathrectangle{\pgfqpoint{0.539299in}{0.078740in}}{\pgfqpoint{7.842520in}{7.842520in}}%
\pgfusepath{clip}%
\pgfsetbuttcap%
\pgfsetroundjoin%
\definecolor{currentfill}{rgb}{0.119423,0.611141,0.538982}%
\pgfsetfillcolor{currentfill}%
\pgfsetlinewidth{0.000000pt}%
\definecolor{currentstroke}{rgb}{0.120638,0.625828,0.533488}%
\pgfsetstrokecolor{currentstroke}%
\pgfsetdash{}{0pt}%
\pgfpathmoveto{\pgfqpoint{4.277193in}{4.723467in}}%
\pgfpathlineto{\pgfqpoint{4.196221in}{4.834423in}}%
\pgfpathlineto{\pgfqpoint{4.333879in}{4.667896in}}%
\pgfpathclose%
\pgfusepath{fill}%
\end{pgfscope}%
\begin{pgfscope}%
\pgfpathrectangle{\pgfqpoint{0.539299in}{0.078740in}}{\pgfqpoint{7.842520in}{7.842520in}}%
\pgfusepath{clip}%
\pgfsetbuttcap%
\pgfsetroundjoin%
\definecolor{currentfill}{rgb}{0.185556,0.418570,0.556753}%
\pgfsetfillcolor{currentfill}%
\pgfsetlinewidth{0.000000pt}%
\definecolor{currentstroke}{rgb}{0.121380,0.629492,0.531973}%
\pgfsetstrokecolor{currentstroke}%
\pgfsetdash{}{0pt}%
\pgfpathmoveto{\pgfqpoint{2.234531in}{4.033237in}}%
\pgfpathlineto{\pgfqpoint{2.320596in}{4.102334in}}%
\pgfpathlineto{\pgfqpoint{2.205440in}{3.565187in}}%
\pgfpathclose%
\pgfusepath{fill}%
\end{pgfscope}%
\begin{pgfscope}%
\pgfpathrectangle{\pgfqpoint{0.539299in}{0.078740in}}{\pgfqpoint{7.842520in}{7.842520in}}%
\pgfusepath{clip}%
\pgfsetbuttcap%
\pgfsetroundjoin%
\definecolor{currentfill}{rgb}{0.270595,0.214069,0.507052}%
\pgfsetfillcolor{currentfill}%
\pgfsetlinewidth{0.000000pt}%
\definecolor{currentstroke}{rgb}{0.122312,0.633153,0.530398}%
\pgfsetstrokecolor{currentstroke}%
\pgfsetdash{}{0pt}%
\pgfpathmoveto{\pgfqpoint{6.000832in}{3.178458in}}%
\pgfpathlineto{\pgfqpoint{5.926016in}{3.134613in}}%
\pgfpathlineto{\pgfqpoint{6.065090in}{3.051608in}}%
\pgfpathclose%
\pgfusepath{fill}%
\end{pgfscope}%
\begin{pgfscope}%
\pgfpathrectangle{\pgfqpoint{0.539299in}{0.078740in}}{\pgfqpoint{7.842520in}{7.842520in}}%
\pgfusepath{clip}%
\pgfsetbuttcap%
\pgfsetroundjoin%
\definecolor{currentfill}{rgb}{0.233603,0.313828,0.543914}%
\pgfsetfillcolor{currentfill}%
\pgfsetlinewidth{0.000000pt}%
\definecolor{currentstroke}{rgb}{0.123444,0.636809,0.528763}%
\pgfsetstrokecolor{currentstroke}%
\pgfsetdash{}{0pt}%
\pgfpathmoveto{\pgfqpoint{5.296988in}{3.532612in}}%
\pgfpathlineto{\pgfqpoint{5.434988in}{3.409236in}}%
\pgfpathlineto{\pgfqpoint{5.373172in}{3.514648in}}%
\pgfpathclose%
\pgfusepath{fill}%
\end{pgfscope}%
\begin{pgfscope}%
\pgfpathrectangle{\pgfqpoint{0.539299in}{0.078740in}}{\pgfqpoint{7.842520in}{7.842520in}}%
\pgfusepath{clip}%
\pgfsetbuttcap%
\pgfsetroundjoin%
\definecolor{currentfill}{rgb}{0.280894,0.078907,0.402329}%
\pgfsetfillcolor{currentfill}%
\pgfsetlinewidth{0.000000pt}%
\definecolor{currentstroke}{rgb}{0.124780,0.640461,0.527068}%
\pgfsetstrokecolor{currentstroke}%
\pgfsetdash{}{0pt}%
\pgfpathmoveto{\pgfqpoint{6.909736in}{2.729276in}}%
\pgfpathlineto{\pgfqpoint{7.049132in}{2.625447in}}%
\pgfpathlineto{\pgfqpoint{7.120863in}{2.683197in}}%
\pgfpathclose%
\pgfusepath{fill}%
\end{pgfscope}%
\begin{pgfscope}%
\pgfpathrectangle{\pgfqpoint{0.539299in}{0.078740in}}{\pgfqpoint{7.842520in}{7.842520in}}%
\pgfusepath{clip}%
\pgfsetbuttcap%
\pgfsetroundjoin%
\definecolor{currentfill}{rgb}{0.276194,0.190074,0.493001}%
\pgfsetfillcolor{currentfill}%
\pgfsetlinewidth{0.000000pt}%
\definecolor{currentstroke}{rgb}{0.126326,0.644107,0.525311}%
\pgfsetstrokecolor{currentstroke}%
\pgfsetdash{}{0pt}%
\pgfpathmoveto{\pgfqpoint{6.278864in}{3.026523in}}%
\pgfpathlineto{\pgfqpoint{6.139708in}{3.103444in}}%
\pgfpathlineto{\pgfqpoint{6.204464in}{2.968792in}}%
\pgfpathclose%
\pgfusepath{fill}%
\end{pgfscope}%
\begin{pgfscope}%
\pgfpathrectangle{\pgfqpoint{0.539299in}{0.078740in}}{\pgfqpoint{7.842520in}{7.842520in}}%
\pgfusepath{clip}%
\pgfsetbuttcap%
\pgfsetroundjoin%
\definecolor{currentfill}{rgb}{0.283072,0.130895,0.449241}%
\pgfsetfillcolor{currentfill}%
\pgfsetlinewidth{0.000000pt}%
\definecolor{currentstroke}{rgb}{0.128087,0.647749,0.523491}%
\pgfsetstrokecolor{currentstroke}%
\pgfsetdash{}{0pt}%
\pgfpathmoveto{\pgfqpoint{6.770368in}{2.826882in}}%
\pgfpathlineto{\pgfqpoint{6.631052in}{2.917912in}}%
\pgfpathlineto{\pgfqpoint{6.697430in}{2.767965in}}%
\pgfpathclose%
\pgfusepath{fill}%
\end{pgfscope}%
\begin{pgfscope}%
\pgfpathrectangle{\pgfqpoint{0.539299in}{0.078740in}}{\pgfqpoint{7.842520in}{7.842520in}}%
\pgfusepath{clip}%
\pgfsetbuttcap%
\pgfsetroundjoin%
\definecolor{currentfill}{rgb}{0.128729,0.563265,0.551229}%
\pgfsetfillcolor{currentfill}%
\pgfsetlinewidth{0.000000pt}%
\definecolor{currentstroke}{rgb}{0.130067,0.651384,0.521608}%
\pgfsetstrokecolor{currentstroke}%
\pgfsetdash{}{0pt}%
\pgfpathmoveto{\pgfqpoint{2.699021in}{4.682479in}}%
\pgfpathlineto{\pgfqpoint{2.491969in}{4.215044in}}%
\pgfpathlineto{\pgfqpoint{2.613215in}{4.649492in}}%
\pgfpathclose%
\pgfusepath{fill}%
\end{pgfscope}%
\begin{pgfscope}%
\pgfpathrectangle{\pgfqpoint{0.539299in}{0.078740in}}{\pgfqpoint{7.842520in}{7.842520in}}%
\pgfusepath{clip}%
\pgfsetbuttcap%
\pgfsetroundjoin%
\definecolor{currentfill}{rgb}{0.258965,0.251537,0.524736}%
\pgfsetfillcolor{currentfill}%
\pgfsetlinewidth{0.000000pt}%
\definecolor{currentstroke}{rgb}{0.132268,0.655014,0.519661}%
\pgfsetstrokecolor{currentstroke}%
\pgfsetdash{}{0pt}%
\pgfpathmoveto{\pgfqpoint{5.711861in}{3.191718in}}%
\pgfpathlineto{\pgfqpoint{5.787278in}{3.220081in}}%
\pgfpathlineto{\pgfqpoint{5.648894in}{3.310389in}}%
\pgfpathclose%
\pgfusepath{fill}%
\end{pgfscope}%
\begin{pgfscope}%
\pgfpathrectangle{\pgfqpoint{0.539299in}{0.078740in}}{\pgfqpoint{7.842520in}{7.842520in}}%
\pgfusepath{clip}%
\pgfsetbuttcap%
\pgfsetroundjoin%
\definecolor{currentfill}{rgb}{0.281887,0.150881,0.465405}%
\pgfsetfillcolor{currentfill}%
\pgfsetlinewidth{0.000000pt}%
\definecolor{currentstroke}{rgb}{0.134692,0.658636,0.517649}%
\pgfsetstrokecolor{currentstroke}%
\pgfsetdash{}{0pt}%
\pgfpathmoveto{\pgfqpoint{6.631052in}{2.917912in}}%
\pgfpathlineto{\pgfqpoint{6.418242in}{2.945854in}}%
\pgfpathlineto{\pgfqpoint{6.557783in}{2.860007in}}%
\pgfpathclose%
\pgfusepath{fill}%
\end{pgfscope}%
\begin{pgfscope}%
\pgfpathrectangle{\pgfqpoint{0.539299in}{0.078740in}}{\pgfqpoint{7.842520in}{7.842520in}}%
\pgfusepath{clip}%
\pgfsetbuttcap%
\pgfsetroundjoin%
\definecolor{currentfill}{rgb}{0.132268,0.655014,0.519661}%
\pgfsetfillcolor{currentfill}%
\pgfsetlinewidth{0.000000pt}%
\definecolor{currentstroke}{rgb}{0.137339,0.662252,0.515571}%
\pgfsetstrokecolor{currentstroke}%
\pgfsetdash{}{0pt}%
\pgfpathmoveto{\pgfqpoint{2.910930in}{5.025499in}}%
\pgfpathlineto{\pgfqpoint{2.996415in}{5.025431in}}%
\pgfpathlineto{\pgfqpoint{2.784458in}{4.707293in}}%
\pgfpathclose%
\pgfusepath{fill}%
\end{pgfscope}%
\begin{pgfscope}%
\pgfpathrectangle{\pgfqpoint{0.539299in}{0.078740in}}{\pgfqpoint{7.842520in}{7.842520in}}%
\pgfusepath{clip}%
\pgfsetbuttcap%
\pgfsetroundjoin%
\definecolor{currentfill}{rgb}{0.239374,0.735588,0.455688}%
\pgfsetfillcolor{currentfill}%
\pgfsetlinewidth{0.000000pt}%
\definecolor{currentstroke}{rgb}{0.140210,0.665859,0.513427}%
\pgfsetstrokecolor{currentstroke}%
\pgfsetdash{}{0pt}%
\pgfpathmoveto{\pgfqpoint{3.429695in}{5.281129in}}%
\pgfpathlineto{\pgfqpoint{3.564639in}{5.312794in}}%
\pgfpathlineto{\pgfqpoint{3.648554in}{5.237108in}}%
\pgfpathclose%
\pgfusepath{fill}%
\end{pgfscope}%
\begin{pgfscope}%
\pgfpathrectangle{\pgfqpoint{0.539299in}{0.078740in}}{\pgfqpoint{7.842520in}{7.842520in}}%
\pgfusepath{clip}%
\pgfsetbuttcap%
\pgfsetroundjoin%
\definecolor{currentfill}{rgb}{0.276022,0.044167,0.370164}%
\pgfsetfillcolor{currentfill}%
\pgfsetlinewidth{0.000000pt}%
\definecolor{currentstroke}{rgb}{0.143303,0.669459,0.511215}%
\pgfsetstrokecolor{currentstroke}%
\pgfsetdash{}{0pt}%
\pgfpathmoveto{\pgfqpoint{7.049132in}{2.625447in}}%
\pgfpathlineto{\pgfqpoint{7.188539in}{2.515946in}}%
\pgfpathlineto{\pgfqpoint{7.260119in}{2.579903in}}%
\pgfpathclose%
\pgfusepath{fill}%
\end{pgfscope}%
\begin{pgfscope}%
\pgfpathrectangle{\pgfqpoint{0.539299in}{0.078740in}}{\pgfqpoint{7.842520in}{7.842520in}}%
\pgfusepath{clip}%
\pgfsetbuttcap%
\pgfsetroundjoin%
\definecolor{currentfill}{rgb}{0.246811,0.283237,0.535941}%
\pgfsetfillcolor{currentfill}%
\pgfsetlinewidth{0.000000pt}%
\definecolor{currentstroke}{rgb}{0.146616,0.673050,0.508936}%
\pgfsetstrokecolor{currentstroke}%
\pgfsetdash{}{0pt}%
\pgfpathmoveto{\pgfqpoint{5.434988in}{3.409236in}}%
\pgfpathlineto{\pgfqpoint{5.573268in}{3.296150in}}%
\pgfpathlineto{\pgfqpoint{5.648894in}{3.310389in}}%
\pgfpathclose%
\pgfusepath{fill}%
\end{pgfscope}%
\begin{pgfscope}%
\pgfpathrectangle{\pgfqpoint{0.539299in}{0.078740in}}{\pgfqpoint{7.842520in}{7.842520in}}%
\pgfusepath{clip}%
\pgfsetbuttcap%
\pgfsetroundjoin%
\definecolor{currentfill}{rgb}{0.143343,0.522773,0.556295}%
\pgfsetfillcolor{currentfill}%
\pgfsetlinewidth{0.000000pt}%
\definecolor{currentstroke}{rgb}{0.150148,0.676631,0.506589}%
\pgfsetstrokecolor{currentstroke}%
\pgfsetdash{}{0pt}%
\pgfpathmoveto{\pgfqpoint{2.527103in}{4.606886in}}%
\pgfpathlineto{\pgfqpoint{2.491969in}{4.215044in}}%
\pgfpathlineto{\pgfqpoint{2.406429in}{4.162525in}}%
\pgfpathclose%
\pgfusepath{fill}%
\end{pgfscope}%
\begin{pgfscope}%
\pgfpathrectangle{\pgfqpoint{0.539299in}{0.078740in}}{\pgfqpoint{7.842520in}{7.842520in}}%
\pgfusepath{clip}%
\pgfsetbuttcap%
\pgfsetroundjoin%
\definecolor{currentfill}{rgb}{0.275191,0.194905,0.496005}%
\pgfsetfillcolor{currentfill}%
\pgfsetlinewidth{0.000000pt}%
\definecolor{currentstroke}{rgb}{0.153894,0.680203,0.504172}%
\pgfsetstrokecolor{currentstroke}%
\pgfsetdash{}{0pt}%
\pgfpathmoveto{\pgfqpoint{6.204464in}{2.968792in}}%
\pgfpathlineto{\pgfqpoint{6.139708in}{3.103444in}}%
\pgfpathlineto{\pgfqpoint{6.065090in}{3.051608in}}%
\pgfpathclose%
\pgfusepath{fill}%
\end{pgfscope}%
\begin{pgfscope}%
\pgfpathrectangle{\pgfqpoint{0.539299in}{0.078740in}}{\pgfqpoint{7.842520in}{7.842520in}}%
\pgfusepath{clip}%
\pgfsetbuttcap%
\pgfsetroundjoin%
\definecolor{currentfill}{rgb}{0.283197,0.115680,0.436115}%
\pgfsetfillcolor{currentfill}%
\pgfsetlinewidth{0.000000pt}%
\definecolor{currentstroke}{rgb}{0.157851,0.683765,0.501686}%
\pgfsetstrokecolor{currentstroke}%
\pgfsetdash{}{0pt}%
\pgfpathmoveto{\pgfqpoint{6.770368in}{2.826882in}}%
\pgfpathlineto{\pgfqpoint{6.697430in}{2.767965in}}%
\pgfpathlineto{\pgfqpoint{6.909736in}{2.729276in}}%
\pgfpathclose%
\pgfusepath{fill}%
\end{pgfscope}%
\begin{pgfscope}%
\pgfpathrectangle{\pgfqpoint{0.539299in}{0.078740in}}{\pgfqpoint{7.842520in}{7.842520in}}%
\pgfusepath{clip}%
\pgfsetbuttcap%
\pgfsetroundjoin%
\definecolor{currentfill}{rgb}{0.278012,0.180367,0.486697}%
\pgfsetfillcolor{currentfill}%
\pgfsetlinewidth{0.000000pt}%
\definecolor{currentstroke}{rgb}{0.162016,0.687316,0.499129}%
\pgfsetstrokecolor{currentstroke}%
\pgfsetdash{}{0pt}%
\pgfpathmoveto{\pgfqpoint{6.204464in}{2.968792in}}%
\pgfpathlineto{\pgfqpoint{6.418242in}{2.945854in}}%
\pgfpathlineto{\pgfqpoint{6.278864in}{3.026523in}}%
\pgfpathclose%
\pgfusepath{fill}%
\end{pgfscope}%
\begin{pgfscope}%
\pgfpathrectangle{\pgfqpoint{0.539299in}{0.078740in}}{\pgfqpoint{7.842520in}{7.842520in}}%
\pgfusepath{clip}%
\pgfsetbuttcap%
\pgfsetroundjoin%
\definecolor{currentfill}{rgb}{0.263663,0.237631,0.518762}%
\pgfsetfillcolor{currentfill}%
\pgfsetlinewidth{0.000000pt}%
\definecolor{currentstroke}{rgb}{0.166383,0.690856,0.496502}%
\pgfsetstrokecolor{currentstroke}%
\pgfsetdash{}{0pt}%
\pgfpathmoveto{\pgfqpoint{5.711861in}{3.191718in}}%
\pgfpathlineto{\pgfqpoint{5.926016in}{3.134613in}}%
\pgfpathlineto{\pgfqpoint{5.787278in}{3.220081in}}%
\pgfpathclose%
\pgfusepath{fill}%
\end{pgfscope}%
\begin{pgfscope}%
\pgfpathrectangle{\pgfqpoint{0.539299in}{0.078740in}}{\pgfqpoint{7.842520in}{7.842520in}}%
\pgfusepath{clip}%
\pgfsetbuttcap%
\pgfsetroundjoin%
\definecolor{currentfill}{rgb}{0.282623,0.140926,0.457517}%
\pgfsetfillcolor{currentfill}%
\pgfsetlinewidth{0.000000pt}%
\definecolor{currentstroke}{rgb}{0.170948,0.694384,0.493803}%
\pgfsetstrokecolor{currentstroke}%
\pgfsetdash{}{0pt}%
\pgfpathmoveto{\pgfqpoint{6.697430in}{2.767965in}}%
\pgfpathlineto{\pgfqpoint{6.631052in}{2.917912in}}%
\pgfpathlineto{\pgfqpoint{6.557783in}{2.860007in}}%
\pgfpathclose%
\pgfusepath{fill}%
\end{pgfscope}%
\begin{pgfscope}%
\pgfpathrectangle{\pgfqpoint{0.539299in}{0.078740in}}{\pgfqpoint{7.842520in}{7.842520in}}%
\pgfusepath{clip}%
\pgfsetbuttcap%
\pgfsetroundjoin%
\definecolor{currentfill}{rgb}{0.226397,0.728888,0.462789}%
\pgfsetfillcolor{currentfill}%
\pgfsetlinewidth{0.000000pt}%
\definecolor{currentstroke}{rgb}{0.175707,0.697900,0.491033}%
\pgfsetstrokecolor{currentstroke}%
\pgfsetdash{}{0pt}%
\pgfpathmoveto{\pgfqpoint{3.700962in}{5.285351in}}%
\pgfpathlineto{\pgfqpoint{3.784535in}{5.195377in}}%
\pgfpathlineto{\pgfqpoint{3.648554in}{5.237108in}}%
\pgfpathclose%
\pgfusepath{fill}%
\end{pgfscope}%
\begin{pgfscope}%
\pgfpathrectangle{\pgfqpoint{0.539299in}{0.078740in}}{\pgfqpoint{7.842520in}{7.842520in}}%
\pgfusepath{clip}%
\pgfsetbuttcap%
\pgfsetroundjoin%
\definecolor{currentfill}{rgb}{0.253935,0.265254,0.529983}%
\pgfsetfillcolor{currentfill}%
\pgfsetlinewidth{0.000000pt}%
\definecolor{currentstroke}{rgb}{0.180653,0.701402,0.488189}%
\pgfsetstrokecolor{currentstroke}%
\pgfsetdash{}{0pt}%
\pgfpathmoveto{\pgfqpoint{5.648894in}{3.310389in}}%
\pgfpathlineto{\pgfqpoint{5.573268in}{3.296150in}}%
\pgfpathlineto{\pgfqpoint{5.711861in}{3.191718in}}%
\pgfpathclose%
\pgfusepath{fill}%
\end{pgfscope}%
\begin{pgfscope}%
\pgfpathrectangle{\pgfqpoint{0.539299in}{0.078740in}}{\pgfqpoint{7.842520in}{7.842520in}}%
\pgfusepath{clip}%
\pgfsetbuttcap%
\pgfsetroundjoin%
\definecolor{currentfill}{rgb}{0.220124,0.725509,0.466226}%
\pgfsetfillcolor{currentfill}%
\pgfsetlinewidth{0.000000pt}%
\definecolor{currentstroke}{rgb}{0.185783,0.704891,0.485273}%
\pgfsetstrokecolor{currentstroke}%
\pgfsetdash{}{0pt}%
\pgfpathmoveto{\pgfqpoint{3.700962in}{5.285351in}}%
\pgfpathlineto{\pgfqpoint{3.921341in}{5.107578in}}%
\pgfpathlineto{\pgfqpoint{3.784535in}{5.195377in}}%
\pgfpathclose%
\pgfusepath{fill}%
\end{pgfscope}%
\begin{pgfscope}%
\pgfpathrectangle{\pgfqpoint{0.539299in}{0.078740in}}{\pgfqpoint{7.842520in}{7.842520in}}%
\pgfusepath{clip}%
\pgfsetbuttcap%
\pgfsetroundjoin%
\definecolor{currentfill}{rgb}{0.123444,0.636809,0.528763}%
\pgfsetfillcolor{currentfill}%
\pgfsetlinewidth{0.000000pt}%
\definecolor{currentstroke}{rgb}{0.191090,0.708366,0.482284}%
\pgfsetstrokecolor{currentstroke}%
\pgfsetdash{}{0pt}%
\pgfpathmoveto{\pgfqpoint{2.825015in}{5.016747in}}%
\pgfpathlineto{\pgfqpoint{2.784458in}{4.707293in}}%
\pgfpathlineto{\pgfqpoint{2.699021in}{4.682479in}}%
\pgfpathclose%
\pgfusepath{fill}%
\end{pgfscope}%
\begin{pgfscope}%
\pgfpathrectangle{\pgfqpoint{0.539299in}{0.078740in}}{\pgfqpoint{7.842520in}{7.842520in}}%
\pgfusepath{clip}%
\pgfsetbuttcap%
\pgfsetroundjoin%
\definecolor{currentfill}{rgb}{0.280894,0.078907,0.402329}%
\pgfsetfillcolor{currentfill}%
\pgfsetlinewidth{0.000000pt}%
\definecolor{currentstroke}{rgb}{0.196571,0.711827,0.479221}%
\pgfsetstrokecolor{currentstroke}%
\pgfsetdash{}{0pt}%
\pgfpathmoveto{\pgfqpoint{7.049132in}{2.625447in}}%
\pgfpathlineto{\pgfqpoint{6.909736in}{2.729276in}}%
\pgfpathlineto{\pgfqpoint{6.976816in}{2.562893in}}%
\pgfpathclose%
\pgfusepath{fill}%
\end{pgfscope}%
\begin{pgfscope}%
\pgfpathrectangle{\pgfqpoint{0.539299in}{0.078740in}}{\pgfqpoint{7.842520in}{7.842520in}}%
\pgfusepath{clip}%
\pgfsetbuttcap%
\pgfsetroundjoin%
\definecolor{currentfill}{rgb}{0.280868,0.160771,0.472899}%
\pgfsetfillcolor{currentfill}%
\pgfsetlinewidth{0.000000pt}%
\definecolor{currentstroke}{rgb}{0.202219,0.715272,0.476084}%
\pgfsetstrokecolor{currentstroke}%
\pgfsetdash{}{0pt}%
\pgfpathmoveto{\pgfqpoint{6.557783in}{2.860007in}}%
\pgfpathlineto{\pgfqpoint{6.418242in}{2.945854in}}%
\pgfpathlineto{\pgfqpoint{6.344091in}{2.884105in}}%
\pgfpathclose%
\pgfusepath{fill}%
\end{pgfscope}%
\begin{pgfscope}%
\pgfpathrectangle{\pgfqpoint{0.539299in}{0.078740in}}{\pgfqpoint{7.842520in}{7.842520in}}%
\pgfusepath{clip}%
\pgfsetbuttcap%
\pgfsetroundjoin%
\definecolor{currentfill}{rgb}{0.216210,0.351535,0.550627}%
\pgfsetfillcolor{currentfill}%
\pgfsetlinewidth{0.000000pt}%
\definecolor{currentstroke}{rgb}{0.208030,0.718701,0.472873}%
\pgfsetstrokecolor{currentstroke}%
\pgfsetdash{}{0pt}%
\pgfpathmoveto{\pgfqpoint{2.034011in}{3.419410in}}%
\pgfpathlineto{\pgfqpoint{1.947878in}{3.337740in}}%
\pgfpathlineto{\pgfqpoint{2.148306in}{3.953837in}}%
\pgfpathclose%
\pgfusepath{fill}%
\end{pgfscope}%
\begin{pgfscope}%
\pgfpathrectangle{\pgfqpoint{0.539299in}{0.078740in}}{\pgfqpoint{7.842520in}{7.842520in}}%
\pgfusepath{clip}%
\pgfsetbuttcap%
\pgfsetroundjoin%
\definecolor{currentfill}{rgb}{0.252899,0.742211,0.448284}%
\pgfsetfillcolor{currentfill}%
\pgfsetlinewidth{0.000000pt}%
\definecolor{currentstroke}{rgb}{0.214000,0.722114,0.469588}%
\pgfsetstrokecolor{currentstroke}%
\pgfsetdash{}{0pt}%
\pgfpathmoveto{\pgfqpoint{3.648554in}{5.237108in}}%
\pgfpathlineto{\pgfqpoint{3.564639in}{5.312794in}}%
\pgfpathlineto{\pgfqpoint{3.700962in}{5.285351in}}%
\pgfpathclose%
\pgfusepath{fill}%
\end{pgfscope}%
\begin{pgfscope}%
\pgfpathrectangle{\pgfqpoint{0.539299in}{0.078740in}}{\pgfqpoint{7.842520in}{7.842520in}}%
\pgfusepath{clip}%
\pgfsetbuttcap%
\pgfsetroundjoin%
\definecolor{currentfill}{rgb}{0.190631,0.407061,0.556089}%
\pgfsetfillcolor{currentfill}%
\pgfsetlinewidth{0.000000pt}%
\definecolor{currentstroke}{rgb}{0.220124,0.725509,0.466226}%
\pgfsetstrokecolor{currentstroke}%
\pgfsetdash{}{0pt}%
\pgfpathmoveto{\pgfqpoint{5.021609in}{3.813991in}}%
\pgfpathlineto{\pgfqpoint{4.943959in}{3.876474in}}%
\pgfpathlineto{\pgfqpoint{5.159216in}{3.667393in}}%
\pgfpathclose%
\pgfusepath{fill}%
\end{pgfscope}%
\begin{pgfscope}%
\pgfpathrectangle{\pgfqpoint{0.539299in}{0.078740in}}{\pgfqpoint{7.842520in}{7.842520in}}%
\pgfusepath{clip}%
\pgfsetbuttcap%
\pgfsetroundjoin%
\definecolor{currentfill}{rgb}{0.175841,0.441290,0.557685}%
\pgfsetfillcolor{currentfill}%
\pgfsetlinewidth{0.000000pt}%
\definecolor{currentstroke}{rgb}{0.226397,0.728888,0.462789}%
\pgfsetstrokecolor{currentstroke}%
\pgfsetdash{}{0pt}%
\pgfpathmoveto{\pgfqpoint{5.021609in}{3.813991in}}%
\pgfpathlineto{\pgfqpoint{4.884097in}{3.971926in}}%
\pgfpathlineto{\pgfqpoint{4.805857in}{4.049409in}}%
\pgfpathclose%
\pgfusepath{fill}%
\end{pgfscope}%
\begin{pgfscope}%
\pgfpathrectangle{\pgfqpoint{0.539299in}{0.078740in}}{\pgfqpoint{7.842520in}{7.842520in}}%
\pgfusepath{clip}%
\pgfsetbuttcap%
\pgfsetroundjoin%
\definecolor{currentfill}{rgb}{0.252899,0.742211,0.448284}%
\pgfsetfillcolor{currentfill}%
\pgfsetlinewidth{0.000000pt}%
\definecolor{currentstroke}{rgb}{0.232815,0.732247,0.459277}%
\pgfsetstrokecolor{currentstroke}%
\pgfsetdash{}{0pt}%
\pgfpathmoveto{\pgfqpoint{3.429695in}{5.281129in}}%
\pgfpathlineto{\pgfqpoint{3.211903in}{5.214787in}}%
\pgfpathlineto{\pgfqpoint{3.344948in}{5.333814in}}%
\pgfpathclose%
\pgfusepath{fill}%
\end{pgfscope}%
\begin{pgfscope}%
\pgfpathrectangle{\pgfqpoint{0.539299in}{0.078740in}}{\pgfqpoint{7.842520in}{7.842520in}}%
\pgfusepath{clip}%
\pgfsetbuttcap%
\pgfsetroundjoin%
\definecolor{currentfill}{rgb}{0.165117,0.467423,0.558141}%
\pgfsetfillcolor{currentfill}%
\pgfsetlinewidth{0.000000pt}%
\definecolor{currentstroke}{rgb}{0.239374,0.735588,0.455688}%
\pgfsetstrokecolor{currentstroke}%
\pgfsetdash{}{0pt}%
\pgfpathmoveto{\pgfqpoint{4.805857in}{4.049409in}}%
\pgfpathlineto{\pgfqpoint{4.884097in}{3.971926in}}%
\pgfpathlineto{\pgfqpoint{4.746612in}{4.139665in}}%
\pgfpathclose%
\pgfusepath{fill}%
\end{pgfscope}%
\begin{pgfscope}%
\pgfpathrectangle{\pgfqpoint{0.539299in}{0.078740in}}{\pgfqpoint{7.842520in}{7.842520in}}%
\pgfusepath{clip}%
\pgfsetbuttcap%
\pgfsetroundjoin%
\definecolor{currentfill}{rgb}{0.208030,0.718701,0.472873}%
\pgfsetfillcolor{currentfill}%
\pgfsetlinewidth{0.000000pt}%
\definecolor{currentstroke}{rgb}{0.246070,0.738910,0.452024}%
\pgfsetstrokecolor{currentstroke}%
\pgfsetdash{}{0pt}%
\pgfpathmoveto{\pgfqpoint{3.126635in}{5.241459in}}%
\pgfpathlineto{\pgfqpoint{3.211903in}{5.214787in}}%
\pgfpathlineto{\pgfqpoint{2.996415in}{5.025431in}}%
\pgfpathclose%
\pgfusepath{fill}%
\end{pgfscope}%
\begin{pgfscope}%
\pgfpathrectangle{\pgfqpoint{0.539299in}{0.078740in}}{\pgfqpoint{7.842520in}{7.842520in}}%
\pgfusepath{clip}%
\pgfsetbuttcap%
\pgfsetroundjoin%
\definecolor{currentfill}{rgb}{0.283091,0.110553,0.431554}%
\pgfsetfillcolor{currentfill}%
\pgfsetlinewidth{0.000000pt}%
\definecolor{currentstroke}{rgb}{0.252899,0.742211,0.448284}%
\pgfsetstrokecolor{currentstroke}%
\pgfsetdash{}{0pt}%
\pgfpathmoveto{\pgfqpoint{6.909736in}{2.729276in}}%
\pgfpathlineto{\pgfqpoint{6.697430in}{2.767965in}}%
\pgfpathlineto{\pgfqpoint{6.837125in}{2.669064in}}%
\pgfpathclose%
\pgfusepath{fill}%
\end{pgfscope}%
\begin{pgfscope}%
\pgfpathrectangle{\pgfqpoint{0.539299in}{0.078740in}}{\pgfqpoint{7.842520in}{7.842520in}}%
\pgfusepath{clip}%
\pgfsetbuttcap%
\pgfsetroundjoin%
\definecolor{currentfill}{rgb}{0.270595,0.214069,0.507052}%
\pgfsetfillcolor{currentfill}%
\pgfsetlinewidth{0.000000pt}%
\definecolor{currentstroke}{rgb}{0.259857,0.745492,0.444467}%
\pgfsetstrokecolor{currentstroke}%
\pgfsetdash{}{0pt}%
\pgfpathmoveto{\pgfqpoint{6.065090in}{3.051608in}}%
\pgfpathlineto{\pgfqpoint{5.926016in}{3.134613in}}%
\pgfpathlineto{\pgfqpoint{5.990036in}{3.000736in}}%
\pgfpathclose%
\pgfusepath{fill}%
\end{pgfscope}%
\begin{pgfscope}%
\pgfpathrectangle{\pgfqpoint{0.539299in}{0.078740in}}{\pgfqpoint{7.842520in}{7.842520in}}%
\pgfusepath{clip}%
\pgfsetbuttcap%
\pgfsetroundjoin%
\definecolor{currentfill}{rgb}{0.278826,0.175490,0.483397}%
\pgfsetfillcolor{currentfill}%
\pgfsetlinewidth{0.000000pt}%
\definecolor{currentstroke}{rgb}{0.266941,0.748751,0.440573}%
\pgfsetstrokecolor{currentstroke}%
\pgfsetdash{}{0pt}%
\pgfpathmoveto{\pgfqpoint{6.204464in}{2.968792in}}%
\pgfpathlineto{\pgfqpoint{6.344091in}{2.884105in}}%
\pgfpathlineto{\pgfqpoint{6.418242in}{2.945854in}}%
\pgfpathclose%
\pgfusepath{fill}%
\end{pgfscope}%
\begin{pgfscope}%
\pgfpathrectangle{\pgfqpoint{0.539299in}{0.078740in}}{\pgfqpoint{7.842520in}{7.842520in}}%
\pgfusepath{clip}%
\pgfsetbuttcap%
\pgfsetroundjoin%
\definecolor{currentfill}{rgb}{0.212395,0.359683,0.551710}%
\pgfsetfillcolor{currentfill}%
\pgfsetlinewidth{0.000000pt}%
\definecolor{currentstroke}{rgb}{0.274149,0.751988,0.436601}%
\pgfsetstrokecolor{currentstroke}%
\pgfsetdash{}{0pt}%
\pgfpathmoveto{\pgfqpoint{5.159216in}{3.667393in}}%
\pgfpathlineto{\pgfqpoint{5.220286in}{3.561627in}}%
\pgfpathlineto{\pgfqpoint{5.296988in}{3.532612in}}%
\pgfpathclose%
\pgfusepath{fill}%
\end{pgfscope}%
\begin{pgfscope}%
\pgfpathrectangle{\pgfqpoint{0.539299in}{0.078740in}}{\pgfqpoint{7.842520in}{7.842520in}}%
\pgfusepath{clip}%
\pgfsetbuttcap%
\pgfsetroundjoin%
\definecolor{currentfill}{rgb}{0.149039,0.508051,0.557250}%
\pgfsetfillcolor{currentfill}%
\pgfsetlinewidth{0.000000pt}%
\definecolor{currentstroke}{rgb}{0.281477,0.755203,0.432552}%
\pgfsetstrokecolor{currentstroke}%
\pgfsetdash{}{0pt}%
\pgfpathmoveto{\pgfqpoint{4.746612in}{4.139665in}}%
\pgfpathlineto{\pgfqpoint{4.609098in}{4.314481in}}%
\pgfpathlineto{\pgfqpoint{4.667716in}{4.230148in}}%
\pgfpathclose%
\pgfusepath{fill}%
\end{pgfscope}%
\begin{pgfscope}%
\pgfpathrectangle{\pgfqpoint{0.539299in}{0.078740in}}{\pgfqpoint{7.842520in}{7.842520in}}%
\pgfusepath{clip}%
\pgfsetbuttcap%
\pgfsetroundjoin%
\definecolor{currentfill}{rgb}{0.263663,0.237631,0.518762}%
\pgfsetfillcolor{currentfill}%
\pgfsetlinewidth{0.000000pt}%
\definecolor{currentstroke}{rgb}{0.288921,0.758394,0.428426}%
\pgfsetstrokecolor{currentstroke}%
\pgfsetdash{}{0pt}%
\pgfpathmoveto{\pgfqpoint{5.711861in}{3.191718in}}%
\pgfpathlineto{\pgfqpoint{5.850784in}{3.093964in}}%
\pgfpathlineto{\pgfqpoint{5.926016in}{3.134613in}}%
\pgfpathclose%
\pgfusepath{fill}%
\end{pgfscope}%
\begin{pgfscope}%
\pgfpathrectangle{\pgfqpoint{0.539299in}{0.078740in}}{\pgfqpoint{7.842520in}{7.842520in}}%
\pgfusepath{clip}%
\pgfsetbuttcap%
\pgfsetroundjoin%
\definecolor{currentfill}{rgb}{0.278791,0.062145,0.386592}%
\pgfsetfillcolor{currentfill}%
\pgfsetlinewidth{0.000000pt}%
\definecolor{currentstroke}{rgb}{0.296479,0.761561,0.424223}%
\pgfsetstrokecolor{currentstroke}%
\pgfsetdash{}{0pt}%
\pgfpathmoveto{\pgfqpoint{7.188539in}{2.515946in}}%
\pgfpathlineto{\pgfqpoint{7.049132in}{2.625447in}}%
\pgfpathlineto{\pgfqpoint{6.976816in}{2.562893in}}%
\pgfpathclose%
\pgfusepath{fill}%
\end{pgfscope}%
\begin{pgfscope}%
\pgfpathrectangle{\pgfqpoint{0.539299in}{0.078740in}}{\pgfqpoint{7.842520in}{7.842520in}}%
\pgfusepath{clip}%
\pgfsetbuttcap%
\pgfsetroundjoin%
\definecolor{currentfill}{rgb}{0.126453,0.570633,0.549841}%
\pgfsetfillcolor{currentfill}%
\pgfsetlinewidth{0.000000pt}%
\definecolor{currentstroke}{rgb}{0.304148,0.764704,0.419943}%
\pgfsetstrokecolor{currentstroke}%
\pgfsetdash{}{0pt}%
\pgfpathmoveto{\pgfqpoint{2.613215in}{4.649492in}}%
\pgfpathlineto{\pgfqpoint{2.491969in}{4.215044in}}%
\pgfpathlineto{\pgfqpoint{2.527103in}{4.606886in}}%
\pgfpathclose%
\pgfusepath{fill}%
\end{pgfscope}%
\begin{pgfscope}%
\pgfpathrectangle{\pgfqpoint{0.539299in}{0.078740in}}{\pgfqpoint{7.842520in}{7.842520in}}%
\pgfusepath{clip}%
\pgfsetbuttcap%
\pgfsetroundjoin%
\definecolor{currentfill}{rgb}{0.227802,0.326594,0.546532}%
\pgfsetfillcolor{currentfill}%
\pgfsetlinewidth{0.000000pt}%
\definecolor{currentstroke}{rgb}{0.311925,0.767822,0.415586}%
\pgfsetstrokecolor{currentstroke}%
\pgfsetdash{}{0pt}%
\pgfpathmoveto{\pgfqpoint{5.358646in}{3.421072in}}%
\pgfpathlineto{\pgfqpoint{5.434988in}{3.409236in}}%
\pgfpathlineto{\pgfqpoint{5.296988in}{3.532612in}}%
\pgfpathclose%
\pgfusepath{fill}%
\end{pgfscope}%
\begin{pgfscope}%
\pgfpathrectangle{\pgfqpoint{0.539299in}{0.078740in}}{\pgfqpoint{7.842520in}{7.842520in}}%
\pgfusepath{clip}%
\pgfsetbuttcap%
\pgfsetroundjoin%
\definecolor{currentfill}{rgb}{0.191090,0.708366,0.482284}%
\pgfsetfillcolor{currentfill}%
\pgfsetlinewidth{0.000000pt}%
\definecolor{currentstroke}{rgb}{0.319809,0.770914,0.411152}%
\pgfsetstrokecolor{currentstroke}%
\pgfsetdash{}{0pt}%
\pgfpathmoveto{\pgfqpoint{4.058651in}{4.983988in}}%
\pgfpathlineto{\pgfqpoint{3.921341in}{5.107578in}}%
\pgfpathlineto{\pgfqpoint{3.838249in}{5.208623in}}%
\pgfpathclose%
\pgfusepath{fill}%
\end{pgfscope}%
\begin{pgfscope}%
\pgfpathrectangle{\pgfqpoint{0.539299in}{0.078740in}}{\pgfqpoint{7.842520in}{7.842520in}}%
\pgfusepath{clip}%
\pgfsetbuttcap%
\pgfsetroundjoin%
\definecolor{currentfill}{rgb}{0.281924,0.089666,0.412415}%
\pgfsetfillcolor{currentfill}%
\pgfsetlinewidth{0.000000pt}%
\definecolor{currentstroke}{rgb}{0.327796,0.773980,0.406640}%
\pgfsetstrokecolor{currentstroke}%
\pgfsetdash{}{0pt}%
\pgfpathmoveto{\pgfqpoint{6.976816in}{2.562893in}}%
\pgfpathlineto{\pgfqpoint{6.909736in}{2.729276in}}%
\pgfpathlineto{\pgfqpoint{6.837125in}{2.669064in}}%
\pgfpathclose%
\pgfusepath{fill}%
\end{pgfscope}%
\begin{pgfscope}%
\pgfpathrectangle{\pgfqpoint{0.539299in}{0.078740in}}{\pgfqpoint{7.842520in}{7.842520in}}%
\pgfusepath{clip}%
\pgfsetbuttcap%
\pgfsetroundjoin%
\definecolor{currentfill}{rgb}{0.127568,0.566949,0.550556}%
\pgfsetfillcolor{currentfill}%
\pgfsetlinewidth{0.000000pt}%
\definecolor{currentstroke}{rgb}{0.335885,0.777018,0.402049}%
\pgfsetstrokecolor{currentstroke}%
\pgfsetdash{}{0pt}%
\pgfpathmoveto{\pgfqpoint{4.391174in}{4.600585in}}%
\pgfpathlineto{\pgfqpoint{4.609098in}{4.314481in}}%
\pgfpathlineto{\pgfqpoint{4.471520in}{4.492348in}}%
\pgfpathclose%
\pgfusepath{fill}%
\end{pgfscope}%
\begin{pgfscope}%
\pgfpathrectangle{\pgfqpoint{0.539299in}{0.078740in}}{\pgfqpoint{7.842520in}{7.842520in}}%
\pgfusepath{clip}%
\pgfsetbuttcap%
\pgfsetroundjoin%
\definecolor{currentfill}{rgb}{0.121148,0.592739,0.544641}%
\pgfsetfillcolor{currentfill}%
\pgfsetlinewidth{0.000000pt}%
\definecolor{currentstroke}{rgb}{0.344074,0.780029,0.397381}%
\pgfsetstrokecolor{currentstroke}%
\pgfsetdash{}{0pt}%
\pgfpathmoveto{\pgfqpoint{4.471520in}{4.492348in}}%
\pgfpathlineto{\pgfqpoint{4.333879in}{4.667896in}}%
\pgfpathlineto{\pgfqpoint{4.391174in}{4.600585in}}%
\pgfpathclose%
\pgfusepath{fill}%
\end{pgfscope}%
\begin{pgfscope}%
\pgfpathrectangle{\pgfqpoint{0.539299in}{0.078740in}}{\pgfqpoint{7.842520in}{7.842520in}}%
\pgfusepath{clip}%
\pgfsetbuttcap%
\pgfsetroundjoin%
\definecolor{currentfill}{rgb}{0.143303,0.669459,0.511215}%
\pgfsetfillcolor{currentfill}%
\pgfsetlinewidth{0.000000pt}%
\definecolor{currentstroke}{rgb}{0.352360,0.783011,0.392636}%
\pgfsetstrokecolor{currentstroke}%
\pgfsetdash{}{0pt}%
\pgfpathmoveto{\pgfqpoint{2.910930in}{5.025499in}}%
\pgfpathlineto{\pgfqpoint{2.784458in}{4.707293in}}%
\pgfpathlineto{\pgfqpoint{2.825015in}{5.016747in}}%
\pgfpathclose%
\pgfusepath{fill}%
\end{pgfscope}%
\begin{pgfscope}%
\pgfpathrectangle{\pgfqpoint{0.539299in}{0.078740in}}{\pgfqpoint{7.842520in}{7.842520in}}%
\pgfusepath{clip}%
\pgfsetbuttcap%
\pgfsetroundjoin%
\definecolor{currentfill}{rgb}{0.273006,0.204520,0.501721}%
\pgfsetfillcolor{currentfill}%
\pgfsetlinewidth{0.000000pt}%
\definecolor{currentstroke}{rgb}{0.360741,0.785964,0.387814}%
\pgfsetstrokecolor{currentstroke}%
\pgfsetdash{}{0pt}%
\pgfpathmoveto{\pgfqpoint{6.065090in}{3.051608in}}%
\pgfpathlineto{\pgfqpoint{5.990036in}{3.000736in}}%
\pgfpathlineto{\pgfqpoint{6.204464in}{2.968792in}}%
\pgfpathclose%
\pgfusepath{fill}%
\end{pgfscope}%
\begin{pgfscope}%
\pgfpathrectangle{\pgfqpoint{0.539299in}{0.078740in}}{\pgfqpoint{7.842520in}{7.842520in}}%
\pgfusepath{clip}%
\pgfsetbuttcap%
\pgfsetroundjoin%
\definecolor{currentfill}{rgb}{0.267968,0.223549,0.512008}%
\pgfsetfillcolor{currentfill}%
\pgfsetlinewidth{0.000000pt}%
\definecolor{currentstroke}{rgb}{0.369214,0.788888,0.382914}%
\pgfsetstrokecolor{currentstroke}%
\pgfsetdash{}{0pt}%
\pgfpathmoveto{\pgfqpoint{5.926016in}{3.134613in}}%
\pgfpathlineto{\pgfqpoint{5.850784in}{3.093964in}}%
\pgfpathlineto{\pgfqpoint{5.990036in}{3.000736in}}%
\pgfpathclose%
\pgfusepath{fill}%
\end{pgfscope}%
\begin{pgfscope}%
\pgfpathrectangle{\pgfqpoint{0.539299in}{0.078740in}}{\pgfqpoint{7.842520in}{7.842520in}}%
\pgfusepath{clip}%
\pgfsetbuttcap%
\pgfsetroundjoin%
\definecolor{currentfill}{rgb}{0.282623,0.140926,0.457517}%
\pgfsetfillcolor{currentfill}%
\pgfsetlinewidth{0.000000pt}%
\definecolor{currentstroke}{rgb}{0.377779,0.791781,0.377939}%
\pgfsetstrokecolor{currentstroke}%
\pgfsetdash{}{0pt}%
\pgfpathmoveto{\pgfqpoint{6.557783in}{2.860007in}}%
\pgfpathlineto{\pgfqpoint{6.483913in}{2.795763in}}%
\pgfpathlineto{\pgfqpoint{6.697430in}{2.767965in}}%
\pgfpathclose%
\pgfusepath{fill}%
\end{pgfscope}%
\begin{pgfscope}%
\pgfpathrectangle{\pgfqpoint{0.539299in}{0.078740in}}{\pgfqpoint{7.842520in}{7.842520in}}%
\pgfusepath{clip}%
\pgfsetbuttcap%
\pgfsetroundjoin%
\definecolor{currentfill}{rgb}{0.122312,0.633153,0.530398}%
\pgfsetfillcolor{currentfill}%
\pgfsetlinewidth{0.000000pt}%
\definecolor{currentstroke}{rgb}{0.386433,0.794644,0.372886}%
\pgfsetstrokecolor{currentstroke}%
\pgfsetdash{}{0pt}%
\pgfpathmoveto{\pgfqpoint{4.252784in}{4.779999in}}%
\pgfpathlineto{\pgfqpoint{4.333879in}{4.667896in}}%
\pgfpathlineto{\pgfqpoint{4.196221in}{4.834423in}}%
\pgfpathclose%
\pgfusepath{fill}%
\end{pgfscope}%
\begin{pgfscope}%
\pgfpathrectangle{\pgfqpoint{0.539299in}{0.078740in}}{\pgfqpoint{7.842520in}{7.842520in}}%
\pgfusepath{clip}%
\pgfsetbuttcap%
\pgfsetroundjoin%
\definecolor{currentfill}{rgb}{0.143303,0.669459,0.511215}%
\pgfsetfillcolor{currentfill}%
\pgfsetlinewidth{0.000000pt}%
\definecolor{currentstroke}{rgb}{0.395174,0.797475,0.367757}%
\pgfsetstrokecolor{currentstroke}%
\pgfsetdash{}{0pt}%
\pgfpathmoveto{\pgfqpoint{4.196221in}{4.834423in}}%
\pgfpathlineto{\pgfqpoint{4.058651in}{4.983988in}}%
\pgfpathlineto{\pgfqpoint{4.114399in}{4.946663in}}%
\pgfpathclose%
\pgfusepath{fill}%
\end{pgfscope}%
\begin{pgfscope}%
\pgfpathrectangle{\pgfqpoint{0.539299in}{0.078740in}}{\pgfqpoint{7.842520in}{7.842520in}}%
\pgfusepath{clip}%
\pgfsetbuttcap%
\pgfsetroundjoin%
\definecolor{currentfill}{rgb}{0.241237,0.296485,0.539709}%
\pgfsetfillcolor{currentfill}%
\pgfsetlinewidth{0.000000pt}%
\definecolor{currentstroke}{rgb}{0.404001,0.800275,0.362552}%
\pgfsetstrokecolor{currentstroke}%
\pgfsetdash{}{0pt}%
\pgfpathmoveto{\pgfqpoint{5.573268in}{3.296150in}}%
\pgfpathlineto{\pgfqpoint{5.434988in}{3.409236in}}%
\pgfpathlineto{\pgfqpoint{5.497218in}{3.291318in}}%
\pgfpathclose%
\pgfusepath{fill}%
\end{pgfscope}%
\begin{pgfscope}%
\pgfpathrectangle{\pgfqpoint{0.539299in}{0.078740in}}{\pgfqpoint{7.842520in}{7.842520in}}%
\pgfusepath{clip}%
\pgfsetbuttcap%
\pgfsetroundjoin%
\definecolor{currentfill}{rgb}{0.281412,0.155834,0.469201}%
\pgfsetfillcolor{currentfill}%
\pgfsetlinewidth{0.000000pt}%
\definecolor{currentstroke}{rgb}{0.412913,0.803041,0.357269}%
\pgfsetstrokecolor{currentstroke}%
\pgfsetdash{}{0pt}%
\pgfpathmoveto{\pgfqpoint{6.344091in}{2.884105in}}%
\pgfpathlineto{\pgfqpoint{6.483913in}{2.795763in}}%
\pgfpathlineto{\pgfqpoint{6.557783in}{2.860007in}}%
\pgfpathclose%
\pgfusepath{fill}%
\end{pgfscope}%
\begin{pgfscope}%
\pgfpathrectangle{\pgfqpoint{0.539299in}{0.078740in}}{\pgfqpoint{7.842520in}{7.842520in}}%
\pgfusepath{clip}%
\pgfsetbuttcap%
\pgfsetroundjoin%
\definecolor{currentfill}{rgb}{0.185556,0.418570,0.556753}%
\pgfsetfillcolor{currentfill}%
\pgfsetlinewidth{0.000000pt}%
\definecolor{currentstroke}{rgb}{0.421908,0.805774,0.351910}%
\pgfsetstrokecolor{currentstroke}%
\pgfsetdash{}{0pt}%
\pgfpathmoveto{\pgfqpoint{2.034011in}{3.419410in}}%
\pgfpathlineto{\pgfqpoint{2.148306in}{3.953837in}}%
\pgfpathlineto{\pgfqpoint{2.234531in}{4.033237in}}%
\pgfpathclose%
\pgfusepath{fill}%
\end{pgfscope}%
\begin{pgfscope}%
\pgfpathrectangle{\pgfqpoint{0.539299in}{0.078740in}}{\pgfqpoint{7.842520in}{7.842520in}}%
\pgfusepath{clip}%
\pgfsetbuttcap%
\pgfsetroundjoin%
\definecolor{currentfill}{rgb}{0.248629,0.278775,0.534556}%
\pgfsetfillcolor{currentfill}%
\pgfsetlinewidth{0.000000pt}%
\definecolor{currentstroke}{rgb}{0.430983,0.808473,0.346476}%
\pgfsetstrokecolor{currentstroke}%
\pgfsetdash{}{0pt}%
\pgfpathmoveto{\pgfqpoint{5.573268in}{3.296150in}}%
\pgfpathlineto{\pgfqpoint{5.497218in}{3.291318in}}%
\pgfpathlineto{\pgfqpoint{5.711861in}{3.191718in}}%
\pgfpathclose%
\pgfusepath{fill}%
\end{pgfscope}%
\begin{pgfscope}%
\pgfpathrectangle{\pgfqpoint{0.539299in}{0.078740in}}{\pgfqpoint{7.842520in}{7.842520in}}%
\pgfusepath{clip}%
\pgfsetbuttcap%
\pgfsetroundjoin%
\definecolor{currentfill}{rgb}{0.277018,0.050344,0.375715}%
\pgfsetfillcolor{currentfill}%
\pgfsetlinewidth{0.000000pt}%
\definecolor{currentstroke}{rgb}{0.440137,0.811138,0.340967}%
\pgfsetstrokecolor{currentstroke}%
\pgfsetdash{}{0pt}%
\pgfpathmoveto{\pgfqpoint{7.116450in}{2.449156in}}%
\pgfpathlineto{\pgfqpoint{7.188539in}{2.515946in}}%
\pgfpathlineto{\pgfqpoint{6.976816in}{2.562893in}}%
\pgfpathclose%
\pgfusepath{fill}%
\end{pgfscope}%
\begin{pgfscope}%
\pgfpathrectangle{\pgfqpoint{0.539299in}{0.078740in}}{\pgfqpoint{7.842520in}{7.842520in}}%
\pgfusepath{clip}%
\pgfsetbuttcap%
\pgfsetroundjoin%
\definecolor{currentfill}{rgb}{0.190631,0.407061,0.556089}%
\pgfsetfillcolor{currentfill}%
\pgfsetlinewidth{0.000000pt}%
\definecolor{currentstroke}{rgb}{0.449368,0.813768,0.335384}%
\pgfsetstrokecolor{currentstroke}%
\pgfsetdash{}{0pt}%
\pgfpathmoveto{\pgfqpoint{4.943959in}{3.876474in}}%
\pgfpathlineto{\pgfqpoint{5.082079in}{3.713504in}}%
\pgfpathlineto{\pgfqpoint{5.159216in}{3.667393in}}%
\pgfpathclose%
\pgfusepath{fill}%
\end{pgfscope}%
\begin{pgfscope}%
\pgfpathrectangle{\pgfqpoint{0.539299in}{0.078740in}}{\pgfqpoint{7.842520in}{7.842520in}}%
\pgfusepath{clip}%
\pgfsetbuttcap%
\pgfsetroundjoin%
\definecolor{currentfill}{rgb}{0.203063,0.379716,0.553925}%
\pgfsetfillcolor{currentfill}%
\pgfsetlinewidth{0.000000pt}%
\definecolor{currentstroke}{rgb}{0.458674,0.816363,0.329727}%
\pgfsetstrokecolor{currentstroke}%
\pgfsetdash{}{0pt}%
\pgfpathmoveto{\pgfqpoint{5.159216in}{3.667393in}}%
\pgfpathlineto{\pgfqpoint{5.082079in}{3.713504in}}%
\pgfpathlineto{\pgfqpoint{5.220286in}{3.561627in}}%
\pgfpathclose%
\pgfusepath{fill}%
\end{pgfscope}%
\begin{pgfscope}%
\pgfpathrectangle{\pgfqpoint{0.539299in}{0.078740in}}{\pgfqpoint{7.842520in}{7.842520in}}%
\pgfusepath{clip}%
\pgfsetbuttcap%
\pgfsetroundjoin%
\definecolor{currentfill}{rgb}{0.174274,0.445044,0.557792}%
\pgfsetfillcolor{currentfill}%
\pgfsetlinewidth{0.000000pt}%
\definecolor{currentstroke}{rgb}{0.468053,0.818921,0.323998}%
\pgfsetstrokecolor{currentstroke}%
\pgfsetdash{}{0pt}%
\pgfpathmoveto{\pgfqpoint{4.805857in}{4.049409in}}%
\pgfpathlineto{\pgfqpoint{4.943959in}{3.876474in}}%
\pgfpathlineto{\pgfqpoint{5.021609in}{3.813991in}}%
\pgfpathclose%
\pgfusepath{fill}%
\end{pgfscope}%
\begin{pgfscope}%
\pgfpathrectangle{\pgfqpoint{0.539299in}{0.078740in}}{\pgfqpoint{7.842520in}{7.842520in}}%
\pgfusepath{clip}%
\pgfsetbuttcap%
\pgfsetroundjoin%
\definecolor{currentfill}{rgb}{0.288921,0.758394,0.428426}%
\pgfsetfillcolor{currentfill}%
\pgfsetlinewidth{0.000000pt}%
\definecolor{currentstroke}{rgb}{0.477504,0.821444,0.318195}%
\pgfsetstrokecolor{currentstroke}%
\pgfsetdash{}{0pt}%
\pgfpathmoveto{\pgfqpoint{3.429695in}{5.281129in}}%
\pgfpathlineto{\pgfqpoint{3.480023in}{5.382736in}}%
\pgfpathlineto{\pgfqpoint{3.564639in}{5.312794in}}%
\pgfpathclose%
\pgfusepath{fill}%
\end{pgfscope}%
\begin{pgfscope}%
\pgfpathrectangle{\pgfqpoint{0.539299in}{0.078740in}}{\pgfqpoint{7.842520in}{7.842520in}}%
\pgfusepath{clip}%
\pgfsetbuttcap%
\pgfsetroundjoin%
\definecolor{currentfill}{rgb}{0.141935,0.526453,0.555991}%
\pgfsetfillcolor{currentfill}%
\pgfsetlinewidth{0.000000pt}%
\definecolor{currentstroke}{rgb}{0.487026,0.823929,0.312321}%
\pgfsetstrokecolor{currentstroke}%
\pgfsetdash{}{0pt}%
\pgfpathmoveto{\pgfqpoint{2.406429in}{4.162525in}}%
\pgfpathlineto{\pgfqpoint{2.320596in}{4.102334in}}%
\pgfpathlineto{\pgfqpoint{2.440760in}{4.553021in}}%
\pgfpathclose%
\pgfusepath{fill}%
\end{pgfscope}%
\begin{pgfscope}%
\pgfpathrectangle{\pgfqpoint{0.539299in}{0.078740in}}{\pgfqpoint{7.842520in}{7.842520in}}%
\pgfusepath{clip}%
\pgfsetbuttcap%
\pgfsetroundjoin%
\definecolor{currentfill}{rgb}{0.239374,0.735588,0.455688}%
\pgfsetfillcolor{currentfill}%
\pgfsetlinewidth{0.000000pt}%
\definecolor{currentstroke}{rgb}{0.496615,0.826376,0.306377}%
\pgfsetstrokecolor{currentstroke}%
\pgfsetdash{}{0pt}%
\pgfpathmoveto{\pgfqpoint{3.700962in}{5.285351in}}%
\pgfpathlineto{\pgfqpoint{3.838249in}{5.208623in}}%
\pgfpathlineto{\pgfqpoint{3.921341in}{5.107578in}}%
\pgfpathclose%
\pgfusepath{fill}%
\end{pgfscope}%
\begin{pgfscope}%
\pgfpathrectangle{\pgfqpoint{0.539299in}{0.078740in}}{\pgfqpoint{7.842520in}{7.842520in}}%
\pgfusepath{clip}%
\pgfsetbuttcap%
\pgfsetroundjoin%
\definecolor{currentfill}{rgb}{0.153364,0.497000,0.557724}%
\pgfsetfillcolor{currentfill}%
\pgfsetlinewidth{0.000000pt}%
\definecolor{currentstroke}{rgb}{0.506271,0.828786,0.300362}%
\pgfsetstrokecolor{currentstroke}%
\pgfsetdash{}{0pt}%
\pgfpathmoveto{\pgfqpoint{4.746612in}{4.139665in}}%
\pgfpathlineto{\pgfqpoint{4.667716in}{4.230148in}}%
\pgfpathlineto{\pgfqpoint{4.805857in}{4.049409in}}%
\pgfpathclose%
\pgfusepath{fill}%
\end{pgfscope}%
\begin{pgfscope}%
\pgfpathrectangle{\pgfqpoint{0.539299in}{0.078740in}}{\pgfqpoint{7.842520in}{7.842520in}}%
\pgfusepath{clip}%
\pgfsetbuttcap%
\pgfsetroundjoin%
\definecolor{currentfill}{rgb}{0.218130,0.347432,0.550038}%
\pgfsetfillcolor{currentfill}%
\pgfsetlinewidth{0.000000pt}%
\definecolor{currentstroke}{rgb}{0.515992,0.831158,0.294279}%
\pgfsetstrokecolor{currentstroke}%
\pgfsetdash{}{0pt}%
\pgfpathmoveto{\pgfqpoint{5.296988in}{3.532612in}}%
\pgfpathlineto{\pgfqpoint{5.220286in}{3.561627in}}%
\pgfpathlineto{\pgfqpoint{5.358646in}{3.421072in}}%
\pgfpathclose%
\pgfusepath{fill}%
\end{pgfscope}%
\begin{pgfscope}%
\pgfpathrectangle{\pgfqpoint{0.539299in}{0.078740in}}{\pgfqpoint{7.842520in}{7.842520in}}%
\pgfusepath{clip}%
\pgfsetbuttcap%
\pgfsetroundjoin%
\definecolor{currentfill}{rgb}{0.126326,0.644107,0.525311}%
\pgfsetfillcolor{currentfill}%
\pgfsetlinewidth{0.000000pt}%
\definecolor{currentstroke}{rgb}{0.525776,0.833491,0.288127}%
\pgfsetstrokecolor{currentstroke}%
\pgfsetdash{}{0pt}%
\pgfpathmoveto{\pgfqpoint{2.825015in}{5.016747in}}%
\pgfpathlineto{\pgfqpoint{2.699021in}{4.682479in}}%
\pgfpathlineto{\pgfqpoint{2.613215in}{4.649492in}}%
\pgfpathclose%
\pgfusepath{fill}%
\end{pgfscope}%
\begin{pgfscope}%
\pgfpathrectangle{\pgfqpoint{0.539299in}{0.078740in}}{\pgfqpoint{7.842520in}{7.842520in}}%
\pgfusepath{clip}%
\pgfsetbuttcap%
\pgfsetroundjoin%
\definecolor{currentfill}{rgb}{0.283091,0.110553,0.431554}%
\pgfsetfillcolor{currentfill}%
\pgfsetlinewidth{0.000000pt}%
\definecolor{currentstroke}{rgb}{0.535621,0.835785,0.281908}%
\pgfsetstrokecolor{currentstroke}%
\pgfsetdash{}{0pt}%
\pgfpathmoveto{\pgfqpoint{6.697430in}{2.767965in}}%
\pgfpathlineto{\pgfqpoint{6.763891in}{2.602375in}}%
\pgfpathlineto{\pgfqpoint{6.837125in}{2.669064in}}%
\pgfpathclose%
\pgfusepath{fill}%
\end{pgfscope}%
\begin{pgfscope}%
\pgfpathrectangle{\pgfqpoint{0.539299in}{0.078740in}}{\pgfqpoint{7.842520in}{7.842520in}}%
\pgfusepath{clip}%
\pgfsetbuttcap%
\pgfsetroundjoin%
\definecolor{currentfill}{rgb}{0.136408,0.541173,0.554483}%
\pgfsetfillcolor{currentfill}%
\pgfsetlinewidth{0.000000pt}%
\definecolor{currentstroke}{rgb}{0.545524,0.838039,0.275626}%
\pgfsetstrokecolor{currentstroke}%
\pgfsetdash{}{0pt}%
\pgfpathmoveto{\pgfqpoint{4.667716in}{4.230148in}}%
\pgfpathlineto{\pgfqpoint{4.609098in}{4.314481in}}%
\pgfpathlineto{\pgfqpoint{4.529493in}{4.415381in}}%
\pgfpathclose%
\pgfusepath{fill}%
\end{pgfscope}%
\begin{pgfscope}%
\pgfpathrectangle{\pgfqpoint{0.539299in}{0.078740in}}{\pgfqpoint{7.842520in}{7.842520in}}%
\pgfusepath{clip}%
\pgfsetbuttcap%
\pgfsetroundjoin%
\definecolor{currentfill}{rgb}{0.282884,0.135920,0.453427}%
\pgfsetfillcolor{currentfill}%
\pgfsetlinewidth{0.000000pt}%
\definecolor{currentstroke}{rgb}{0.555484,0.840254,0.269281}%
\pgfsetstrokecolor{currentstroke}%
\pgfsetdash{}{0pt}%
\pgfpathmoveto{\pgfqpoint{6.697430in}{2.767965in}}%
\pgfpathlineto{\pgfqpoint{6.483913in}{2.795763in}}%
\pgfpathlineto{\pgfqpoint{6.623869in}{2.702268in}}%
\pgfpathclose%
\pgfusepath{fill}%
\end{pgfscope}%
\begin{pgfscope}%
\pgfpathrectangle{\pgfqpoint{0.539299in}{0.078740in}}{\pgfqpoint{7.842520in}{7.842520in}}%
\pgfusepath{clip}%
\pgfsetbuttcap%
\pgfsetroundjoin%
\definecolor{currentfill}{rgb}{0.278012,0.180367,0.486697}%
\pgfsetfillcolor{currentfill}%
\pgfsetlinewidth{0.000000pt}%
\definecolor{currentstroke}{rgb}{0.565498,0.842430,0.262877}%
\pgfsetstrokecolor{currentstroke}%
\pgfsetdash{}{0pt}%
\pgfpathmoveto{\pgfqpoint{6.269433in}{2.819189in}}%
\pgfpathlineto{\pgfqpoint{6.344091in}{2.884105in}}%
\pgfpathlineto{\pgfqpoint{6.204464in}{2.968792in}}%
\pgfpathclose%
\pgfusepath{fill}%
\end{pgfscope}%
\begin{pgfscope}%
\pgfpathrectangle{\pgfqpoint{0.539299in}{0.078740in}}{\pgfqpoint{7.842520in}{7.842520in}}%
\pgfusepath{clip}%
\pgfsetbuttcap%
\pgfsetroundjoin%
\definecolor{currentfill}{rgb}{0.233603,0.313828,0.543914}%
\pgfsetfillcolor{currentfill}%
\pgfsetlinewidth{0.000000pt}%
\definecolor{currentstroke}{rgb}{0.575563,0.844566,0.256415}%
\pgfsetstrokecolor{currentstroke}%
\pgfsetdash{}{0pt}%
\pgfpathmoveto{\pgfqpoint{5.434988in}{3.409236in}}%
\pgfpathlineto{\pgfqpoint{5.358646in}{3.421072in}}%
\pgfpathlineto{\pgfqpoint{5.497218in}{3.291318in}}%
\pgfpathclose%
\pgfusepath{fill}%
\end{pgfscope}%
\begin{pgfscope}%
\pgfpathrectangle{\pgfqpoint{0.539299in}{0.078740in}}{\pgfqpoint{7.842520in}{7.842520in}}%
\pgfusepath{clip}%
\pgfsetbuttcap%
\pgfsetroundjoin%
\definecolor{currentfill}{rgb}{0.274128,0.199721,0.498911}%
\pgfsetfillcolor{currentfill}%
\pgfsetlinewidth{0.000000pt}%
\definecolor{currentstroke}{rgb}{0.585678,0.846661,0.249897}%
\pgfsetstrokecolor{currentstroke}%
\pgfsetdash{}{0pt}%
\pgfpathmoveto{\pgfqpoint{5.990036in}{3.000736in}}%
\pgfpathlineto{\pgfqpoint{6.129597in}{2.909848in}}%
\pgfpathlineto{\pgfqpoint{6.204464in}{2.968792in}}%
\pgfpathclose%
\pgfusepath{fill}%
\end{pgfscope}%
\begin{pgfscope}%
\pgfpathrectangle{\pgfqpoint{0.539299in}{0.078740in}}{\pgfqpoint{7.842520in}{7.842520in}}%
\pgfusepath{clip}%
\pgfsetbuttcap%
\pgfsetroundjoin%
\definecolor{currentfill}{rgb}{0.304148,0.764704,0.419943}%
\pgfsetfillcolor{currentfill}%
\pgfsetlinewidth{0.000000pt}%
\definecolor{currentstroke}{rgb}{0.595839,0.848717,0.243329}%
\pgfsetstrokecolor{currentstroke}%
\pgfsetdash{}{0pt}%
\pgfpathmoveto{\pgfqpoint{3.344948in}{5.333814in}}%
\pgfpathlineto{\pgfqpoint{3.480023in}{5.382736in}}%
\pgfpathlineto{\pgfqpoint{3.429695in}{5.281129in}}%
\pgfpathclose%
\pgfusepath{fill}%
\end{pgfscope}%
\begin{pgfscope}%
\pgfpathrectangle{\pgfqpoint{0.539299in}{0.078740in}}{\pgfqpoint{7.842520in}{7.842520in}}%
\pgfusepath{clip}%
\pgfsetbuttcap%
\pgfsetroundjoin%
\definecolor{currentfill}{rgb}{0.126453,0.570633,0.549841}%
\pgfsetfillcolor{currentfill}%
\pgfsetlinewidth{0.000000pt}%
\definecolor{currentstroke}{rgb}{0.606045,0.850733,0.236712}%
\pgfsetstrokecolor{currentstroke}%
\pgfsetdash{}{0pt}%
\pgfpathmoveto{\pgfqpoint{4.529493in}{4.415381in}}%
\pgfpathlineto{\pgfqpoint{4.609098in}{4.314481in}}%
\pgfpathlineto{\pgfqpoint{4.391174in}{4.600585in}}%
\pgfpathclose%
\pgfusepath{fill}%
\end{pgfscope}%
\begin{pgfscope}%
\pgfpathrectangle{\pgfqpoint{0.539299in}{0.078740in}}{\pgfqpoint{7.842520in}{7.842520in}}%
\pgfusepath{clip}%
\pgfsetbuttcap%
\pgfsetroundjoin%
\definecolor{currentfill}{rgb}{0.202219,0.715272,0.476084}%
\pgfsetfillcolor{currentfill}%
\pgfsetlinewidth{0.000000pt}%
\definecolor{currentstroke}{rgb}{0.616293,0.852709,0.230052}%
\pgfsetstrokecolor{currentstroke}%
\pgfsetdash{}{0pt}%
\pgfpathmoveto{\pgfqpoint{2.910930in}{5.025499in}}%
\pgfpathlineto{\pgfqpoint{3.040850in}{5.259550in}}%
\pgfpathlineto{\pgfqpoint{2.996415in}{5.025431in}}%
\pgfpathclose%
\pgfusepath{fill}%
\end{pgfscope}%
\begin{pgfscope}%
\pgfpathrectangle{\pgfqpoint{0.539299in}{0.078740in}}{\pgfqpoint{7.842520in}{7.842520in}}%
\pgfusepath{clip}%
\pgfsetbuttcap%
\pgfsetroundjoin%
\definecolor{currentfill}{rgb}{0.202219,0.715272,0.476084}%
\pgfsetfillcolor{currentfill}%
\pgfsetlinewidth{0.000000pt}%
\definecolor{currentstroke}{rgb}{0.626579,0.854645,0.223353}%
\pgfsetstrokecolor{currentstroke}%
\pgfsetdash{}{0pt}%
\pgfpathmoveto{\pgfqpoint{3.976153in}{5.092524in}}%
\pgfpathlineto{\pgfqpoint{4.058651in}{4.983988in}}%
\pgfpathlineto{\pgfqpoint{3.838249in}{5.208623in}}%
\pgfpathclose%
\pgfusepath{fill}%
\end{pgfscope}%
\begin{pgfscope}%
\pgfpathrectangle{\pgfqpoint{0.539299in}{0.078740in}}{\pgfqpoint{7.842520in}{7.842520in}}%
\pgfusepath{clip}%
\pgfsetbuttcap%
\pgfsetroundjoin%
\definecolor{currentfill}{rgb}{0.260571,0.246922,0.522828}%
\pgfsetfillcolor{currentfill}%
\pgfsetlinewidth{0.000000pt}%
\definecolor{currentstroke}{rgb}{0.636902,0.856542,0.216620}%
\pgfsetstrokecolor{currentstroke}%
\pgfsetdash{}{0pt}%
\pgfpathmoveto{\pgfqpoint{5.775167in}{3.059358in}}%
\pgfpathlineto{\pgfqpoint{5.850784in}{3.093964in}}%
\pgfpathlineto{\pgfqpoint{5.711861in}{3.191718in}}%
\pgfpathclose%
\pgfusepath{fill}%
\end{pgfscope}%
\begin{pgfscope}%
\pgfpathrectangle{\pgfqpoint{0.539299in}{0.078740in}}{\pgfqpoint{7.842520in}{7.842520in}}%
\pgfusepath{clip}%
\pgfsetbuttcap%
\pgfsetroundjoin%
\definecolor{currentfill}{rgb}{0.218130,0.347432,0.550038}%
\pgfsetfillcolor{currentfill}%
\pgfsetlinewidth{0.000000pt}%
\definecolor{currentstroke}{rgb}{0.647257,0.858400,0.209861}%
\pgfsetstrokecolor{currentstroke}%
\pgfsetdash{}{0pt}%
\pgfpathmoveto{\pgfqpoint{2.062008in}{3.862537in}}%
\pgfpathlineto{\pgfqpoint{1.947878in}{3.337740in}}%
\pgfpathlineto{\pgfqpoint{1.861534in}{3.249013in}}%
\pgfpathclose%
\pgfusepath{fill}%
\end{pgfscope}%
\begin{pgfscope}%
\pgfpathrectangle{\pgfqpoint{0.539299in}{0.078740in}}{\pgfqpoint{7.842520in}{7.842520in}}%
\pgfusepath{clip}%
\pgfsetbuttcap%
\pgfsetroundjoin%
\definecolor{currentfill}{rgb}{0.282327,0.094955,0.417331}%
\pgfsetfillcolor{currentfill}%
\pgfsetlinewidth{0.000000pt}%
\definecolor{currentstroke}{rgb}{0.657642,0.860219,0.203082}%
\pgfsetstrokecolor{currentstroke}%
\pgfsetdash{}{0pt}%
\pgfpathmoveto{\pgfqpoint{6.837125in}{2.669064in}}%
\pgfpathlineto{\pgfqpoint{6.763891in}{2.602375in}}%
\pgfpathlineto{\pgfqpoint{6.976816in}{2.562893in}}%
\pgfpathclose%
\pgfusepath{fill}%
\end{pgfscope}%
\begin{pgfscope}%
\pgfpathrectangle{\pgfqpoint{0.539299in}{0.078740in}}{\pgfqpoint{7.842520in}{7.842520in}}%
\pgfusepath{clip}%
\pgfsetbuttcap%
\pgfsetroundjoin%
\definecolor{currentfill}{rgb}{0.120638,0.625828,0.533488}%
\pgfsetfillcolor{currentfill}%
\pgfsetlinewidth{0.000000pt}%
\definecolor{currentstroke}{rgb}{0.668054,0.861999,0.196293}%
\pgfsetstrokecolor{currentstroke}%
\pgfsetdash{}{0pt}%
\pgfpathmoveto{\pgfqpoint{4.333879in}{4.667896in}}%
\pgfpathlineto{\pgfqpoint{4.252784in}{4.779999in}}%
\pgfpathlineto{\pgfqpoint{4.391174in}{4.600585in}}%
\pgfpathclose%
\pgfusepath{fill}%
\end{pgfscope}%
\begin{pgfscope}%
\pgfpathrectangle{\pgfqpoint{0.539299in}{0.078740in}}{\pgfqpoint{7.842520in}{7.842520in}}%
\pgfusepath{clip}%
\pgfsetbuttcap%
\pgfsetroundjoin%
\definecolor{currentfill}{rgb}{0.175707,0.697900,0.491033}%
\pgfsetfillcolor{currentfill}%
\pgfsetlinewidth{0.000000pt}%
\definecolor{currentstroke}{rgb}{0.678489,0.863742,0.189503}%
\pgfsetstrokecolor{currentstroke}%
\pgfsetdash{}{0pt}%
\pgfpathmoveto{\pgfqpoint{4.114399in}{4.946663in}}%
\pgfpathlineto{\pgfqpoint{4.058651in}{4.983988in}}%
\pgfpathlineto{\pgfqpoint{3.976153in}{5.092524in}}%
\pgfpathclose%
\pgfusepath{fill}%
\end{pgfscope}%
\begin{pgfscope}%
\pgfpathrectangle{\pgfqpoint{0.539299in}{0.078740in}}{\pgfqpoint{7.842520in}{7.842520in}}%
\pgfusepath{clip}%
\pgfsetbuttcap%
\pgfsetroundjoin%
\definecolor{currentfill}{rgb}{0.239374,0.735588,0.455688}%
\pgfsetfillcolor{currentfill}%
\pgfsetlinewidth{0.000000pt}%
\definecolor{currentstroke}{rgb}{0.688944,0.865448,0.182725}%
\pgfsetstrokecolor{currentstroke}%
\pgfsetdash{}{0pt}%
\pgfpathmoveto{\pgfqpoint{2.996415in}{5.025431in}}%
\pgfpathlineto{\pgfqpoint{3.040850in}{5.259550in}}%
\pgfpathlineto{\pgfqpoint{3.126635in}{5.241459in}}%
\pgfpathclose%
\pgfusepath{fill}%
\end{pgfscope}%
\begin{pgfscope}%
\pgfpathrectangle{\pgfqpoint{0.539299in}{0.078740in}}{\pgfqpoint{7.842520in}{7.842520in}}%
\pgfusepath{clip}%
\pgfsetbuttcap%
\pgfsetroundjoin%
\definecolor{currentfill}{rgb}{0.283229,0.120777,0.440584}%
\pgfsetfillcolor{currentfill}%
\pgfsetlinewidth{0.000000pt}%
\definecolor{currentstroke}{rgb}{0.699415,0.867117,0.175971}%
\pgfsetstrokecolor{currentstroke}%
\pgfsetdash{}{0pt}%
\pgfpathmoveto{\pgfqpoint{6.623869in}{2.702268in}}%
\pgfpathlineto{\pgfqpoint{6.763891in}{2.602375in}}%
\pgfpathlineto{\pgfqpoint{6.697430in}{2.767965in}}%
\pgfpathclose%
\pgfusepath{fill}%
\end{pgfscope}%
\begin{pgfscope}%
\pgfpathrectangle{\pgfqpoint{0.539299in}{0.078740in}}{\pgfqpoint{7.842520in}{7.842520in}}%
\pgfusepath{clip}%
\pgfsetbuttcap%
\pgfsetroundjoin%
\definecolor{currentfill}{rgb}{0.140210,0.665859,0.513427}%
\pgfsetfillcolor{currentfill}%
\pgfsetlinewidth{0.000000pt}%
\definecolor{currentstroke}{rgb}{0.709898,0.868751,0.169257}%
\pgfsetstrokecolor{currentstroke}%
\pgfsetdash{}{0pt}%
\pgfpathmoveto{\pgfqpoint{4.196221in}{4.834423in}}%
\pgfpathlineto{\pgfqpoint{4.114399in}{4.946663in}}%
\pgfpathlineto{\pgfqpoint{4.252784in}{4.779999in}}%
\pgfpathclose%
\pgfusepath{fill}%
\end{pgfscope}%
\begin{pgfscope}%
\pgfpathrectangle{\pgfqpoint{0.539299in}{0.078740in}}{\pgfqpoint{7.842520in}{7.842520in}}%
\pgfusepath{clip}%
\pgfsetbuttcap%
\pgfsetroundjoin%
\definecolor{currentfill}{rgb}{0.277134,0.185228,0.489898}%
\pgfsetfillcolor{currentfill}%
\pgfsetlinewidth{0.000000pt}%
\definecolor{currentstroke}{rgb}{0.720391,0.870350,0.162603}%
\pgfsetstrokecolor{currentstroke}%
\pgfsetdash{}{0pt}%
\pgfpathmoveto{\pgfqpoint{6.204464in}{2.968792in}}%
\pgfpathlineto{\pgfqpoint{6.129597in}{2.909848in}}%
\pgfpathlineto{\pgfqpoint{6.269433in}{2.819189in}}%
\pgfpathclose%
\pgfusepath{fill}%
\end{pgfscope}%
\begin{pgfscope}%
\pgfpathrectangle{\pgfqpoint{0.539299in}{0.078740in}}{\pgfqpoint{7.842520in}{7.842520in}}%
\pgfusepath{clip}%
\pgfsetbuttcap%
\pgfsetroundjoin%
\definecolor{currentfill}{rgb}{0.296479,0.761561,0.424223}%
\pgfsetfillcolor{currentfill}%
\pgfsetlinewidth{0.000000pt}%
\definecolor{currentstroke}{rgb}{0.730889,0.871916,0.156029}%
\pgfsetstrokecolor{currentstroke}%
\pgfsetdash{}{0pt}%
\pgfpathmoveto{\pgfqpoint{3.344948in}{5.333814in}}%
\pgfpathlineto{\pgfqpoint{3.211903in}{5.214787in}}%
\pgfpathlineto{\pgfqpoint{3.259589in}{5.378978in}}%
\pgfpathclose%
\pgfusepath{fill}%
\end{pgfscope}%
\begin{pgfscope}%
\pgfpathrectangle{\pgfqpoint{0.539299in}{0.078740in}}{\pgfqpoint{7.842520in}{7.842520in}}%
\pgfusepath{clip}%
\pgfsetbuttcap%
\pgfsetroundjoin%
\definecolor{currentfill}{rgb}{0.248629,0.278775,0.534556}%
\pgfsetfillcolor{currentfill}%
\pgfsetlinewidth{0.000000pt}%
\definecolor{currentstroke}{rgb}{0.741388,0.873449,0.149561}%
\pgfsetstrokecolor{currentstroke}%
\pgfsetdash{}{0pt}%
\pgfpathmoveto{\pgfqpoint{5.711861in}{3.191718in}}%
\pgfpathlineto{\pgfqpoint{5.497218in}{3.291318in}}%
\pgfpathlineto{\pgfqpoint{5.636047in}{3.171258in}}%
\pgfpathclose%
\pgfusepath{fill}%
\end{pgfscope}%
\begin{pgfscope}%
\pgfpathrectangle{\pgfqpoint{0.539299in}{0.078740in}}{\pgfqpoint{7.842520in}{7.842520in}}%
\pgfusepath{clip}%
\pgfsetbuttcap%
\pgfsetroundjoin%
\definecolor{currentfill}{rgb}{0.304148,0.764704,0.419943}%
\pgfsetfillcolor{currentfill}%
\pgfsetlinewidth{0.000000pt}%
\definecolor{currentstroke}{rgb}{0.751884,0.874951,0.143228}%
\pgfsetstrokecolor{currentstroke}%
\pgfsetdash{}{0pt}%
\pgfpathmoveto{\pgfqpoint{3.564639in}{5.312794in}}%
\pgfpathlineto{\pgfqpoint{3.480023in}{5.382736in}}%
\pgfpathlineto{\pgfqpoint{3.700962in}{5.285351in}}%
\pgfpathclose%
\pgfusepath{fill}%
\end{pgfscope}%
\begin{pgfscope}%
\pgfpathrectangle{\pgfqpoint{0.539299in}{0.078740in}}{\pgfqpoint{7.842520in}{7.842520in}}%
\pgfusepath{clip}%
\pgfsetbuttcap%
\pgfsetroundjoin%
\definecolor{currentfill}{rgb}{0.281477,0.755203,0.432552}%
\pgfsetfillcolor{currentfill}%
\pgfsetlinewidth{0.000000pt}%
\definecolor{currentstroke}{rgb}{0.762373,0.876424,0.137064}%
\pgfsetstrokecolor{currentstroke}%
\pgfsetdash{}{0pt}%
\pgfpathmoveto{\pgfqpoint{3.259589in}{5.378978in}}%
\pgfpathlineto{\pgfqpoint{3.211903in}{5.214787in}}%
\pgfpathlineto{\pgfqpoint{3.126635in}{5.241459in}}%
\pgfpathclose%
\pgfusepath{fill}%
\end{pgfscope}%
\begin{pgfscope}%
\pgfpathrectangle{\pgfqpoint{0.539299in}{0.078740in}}{\pgfqpoint{7.842520in}{7.842520in}}%
\pgfusepath{clip}%
\pgfsetbuttcap%
\pgfsetroundjoin%
\definecolor{currentfill}{rgb}{0.280868,0.160771,0.472899}%
\pgfsetfillcolor{currentfill}%
\pgfsetlinewidth{0.000000pt}%
\definecolor{currentstroke}{rgb}{0.772852,0.877868,0.131109}%
\pgfsetstrokecolor{currentstroke}%
\pgfsetdash{}{0pt}%
\pgfpathmoveto{\pgfqpoint{6.409497in}{2.726792in}}%
\pgfpathlineto{\pgfqpoint{6.483913in}{2.795763in}}%
\pgfpathlineto{\pgfqpoint{6.344091in}{2.884105in}}%
\pgfpathclose%
\pgfusepath{fill}%
\end{pgfscope}%
\begin{pgfscope}%
\pgfpathrectangle{\pgfqpoint{0.539299in}{0.078740in}}{\pgfqpoint{7.842520in}{7.842520in}}%
\pgfusepath{clip}%
\pgfsetbuttcap%
\pgfsetroundjoin%
\definecolor{currentfill}{rgb}{0.267968,0.223549,0.512008}%
\pgfsetfillcolor{currentfill}%
\pgfsetlinewidth{0.000000pt}%
\definecolor{currentstroke}{rgb}{0.783315,0.879285,0.125405}%
\pgfsetstrokecolor{currentstroke}%
\pgfsetdash{}{0pt}%
\pgfpathmoveto{\pgfqpoint{5.914592in}{2.953804in}}%
\pgfpathlineto{\pgfqpoint{5.990036in}{3.000736in}}%
\pgfpathlineto{\pgfqpoint{5.850784in}{3.093964in}}%
\pgfpathclose%
\pgfusepath{fill}%
\end{pgfscope}%
\begin{pgfscope}%
\pgfpathrectangle{\pgfqpoint{0.539299in}{0.078740in}}{\pgfqpoint{7.842520in}{7.842520in}}%
\pgfusepath{clip}%
\pgfsetbuttcap%
\pgfsetroundjoin%
\definecolor{currentfill}{rgb}{0.255645,0.260703,0.528312}%
\pgfsetfillcolor{currentfill}%
\pgfsetlinewidth{0.000000pt}%
\definecolor{currentstroke}{rgb}{0.793760,0.880678,0.120005}%
\pgfsetstrokecolor{currentstroke}%
\pgfsetdash{}{0pt}%
\pgfpathmoveto{\pgfqpoint{5.636047in}{3.171258in}}%
\pgfpathlineto{\pgfqpoint{5.775167in}{3.059358in}}%
\pgfpathlineto{\pgfqpoint{5.711861in}{3.191718in}}%
\pgfpathclose%
\pgfusepath{fill}%
\end{pgfscope}%
\begin{pgfscope}%
\pgfpathrectangle{\pgfqpoint{0.539299in}{0.078740in}}{\pgfqpoint{7.842520in}{7.842520in}}%
\pgfusepath{clip}%
\pgfsetbuttcap%
\pgfsetroundjoin%
\definecolor{currentfill}{rgb}{0.125394,0.574318,0.549086}%
\pgfsetfillcolor{currentfill}%
\pgfsetlinewidth{0.000000pt}%
\definecolor{currentstroke}{rgb}{0.804182,0.882046,0.114965}%
\pgfsetstrokecolor{currentstroke}%
\pgfsetdash{}{0pt}%
\pgfpathmoveto{\pgfqpoint{2.440760in}{4.553021in}}%
\pgfpathlineto{\pgfqpoint{2.527103in}{4.606886in}}%
\pgfpathlineto{\pgfqpoint{2.406429in}{4.162525in}}%
\pgfpathclose%
\pgfusepath{fill}%
\end{pgfscope}%
\begin{pgfscope}%
\pgfpathrectangle{\pgfqpoint{0.539299in}{0.078740in}}{\pgfqpoint{7.842520in}{7.842520in}}%
\pgfusepath{clip}%
\pgfsetbuttcap%
\pgfsetroundjoin%
\definecolor{currentfill}{rgb}{0.277018,0.050344,0.375715}%
\pgfsetfillcolor{currentfill}%
\pgfsetlinewidth{0.000000pt}%
\definecolor{currentstroke}{rgb}{0.814576,0.883393,0.110347}%
\pgfsetstrokecolor{currentstroke}%
\pgfsetdash{}{0pt}%
\pgfpathmoveto{\pgfqpoint{6.976816in}{2.562893in}}%
\pgfpathlineto{\pgfqpoint{7.043838in}{2.379025in}}%
\pgfpathlineto{\pgfqpoint{7.116450in}{2.449156in}}%
\pgfpathclose%
\pgfusepath{fill}%
\end{pgfscope}%
\begin{pgfscope}%
\pgfpathrectangle{\pgfqpoint{0.539299in}{0.078740in}}{\pgfqpoint{7.842520in}{7.842520in}}%
\pgfusepath{clip}%
\pgfsetbuttcap%
\pgfsetroundjoin%
\definecolor{currentfill}{rgb}{0.141935,0.526453,0.555991}%
\pgfsetfillcolor{currentfill}%
\pgfsetlinewidth{0.000000pt}%
\definecolor{currentstroke}{rgb}{0.824940,0.884720,0.106217}%
\pgfsetstrokecolor{currentstroke}%
\pgfsetdash{}{0pt}%
\pgfpathmoveto{\pgfqpoint{2.320596in}{4.102334in}}%
\pgfpathlineto{\pgfqpoint{2.234531in}{4.033237in}}%
\pgfpathlineto{\pgfqpoint{2.440760in}{4.553021in}}%
\pgfpathclose%
\pgfusepath{fill}%
\end{pgfscope}%
\begin{pgfscope}%
\pgfpathrectangle{\pgfqpoint{0.539299in}{0.078740in}}{\pgfqpoint{7.842520in}{7.842520in}}%
\pgfusepath{clip}%
\pgfsetbuttcap%
\pgfsetroundjoin%
\definecolor{currentfill}{rgb}{0.281446,0.084320,0.407414}%
\pgfsetfillcolor{currentfill}%
\pgfsetlinewidth{0.000000pt}%
\definecolor{currentstroke}{rgb}{0.835270,0.886029,0.102646}%
\pgfsetstrokecolor{currentstroke}%
\pgfsetdash{}{0pt}%
\pgfpathmoveto{\pgfqpoint{6.976816in}{2.562893in}}%
\pgfpathlineto{\pgfqpoint{6.763891in}{2.602375in}}%
\pgfpathlineto{\pgfqpoint{6.903908in}{2.494991in}}%
\pgfpathclose%
\pgfusepath{fill}%
\end{pgfscope}%
\begin{pgfscope}%
\pgfpathrectangle{\pgfqpoint{0.539299in}{0.078740in}}{\pgfqpoint{7.842520in}{7.842520in}}%
\pgfusepath{clip}%
\pgfsetbuttcap%
\pgfsetroundjoin%
\definecolor{currentfill}{rgb}{0.280255,0.165693,0.476498}%
\pgfsetfillcolor{currentfill}%
\pgfsetlinewidth{0.000000pt}%
\definecolor{currentstroke}{rgb}{0.845561,0.887322,0.099702}%
\pgfsetstrokecolor{currentstroke}%
\pgfsetdash{}{0pt}%
\pgfpathmoveto{\pgfqpoint{6.344091in}{2.884105in}}%
\pgfpathlineto{\pgfqpoint{6.269433in}{2.819189in}}%
\pgfpathlineto{\pgfqpoint{6.409497in}{2.726792in}}%
\pgfpathclose%
\pgfusepath{fill}%
\end{pgfscope}%
\begin{pgfscope}%
\pgfpathrectangle{\pgfqpoint{0.539299in}{0.078740in}}{\pgfqpoint{7.842520in}{7.842520in}}%
\pgfusepath{clip}%
\pgfsetbuttcap%
\pgfsetroundjoin%
\definecolor{currentfill}{rgb}{0.271828,0.209303,0.504434}%
\pgfsetfillcolor{currentfill}%
\pgfsetlinewidth{0.000000pt}%
\definecolor{currentstroke}{rgb}{0.855810,0.888601,0.097452}%
\pgfsetstrokecolor{currentstroke}%
\pgfsetdash{}{0pt}%
\pgfpathmoveto{\pgfqpoint{5.990036in}{3.000736in}}%
\pgfpathlineto{\pgfqpoint{5.914592in}{2.953804in}}%
\pgfpathlineto{\pgfqpoint{6.129597in}{2.909848in}}%
\pgfpathclose%
\pgfusepath{fill}%
\end{pgfscope}%
\begin{pgfscope}%
\pgfpathrectangle{\pgfqpoint{0.539299in}{0.078740in}}{\pgfqpoint{7.842520in}{7.842520in}}%
\pgfusepath{clip}%
\pgfsetbuttcap%
\pgfsetroundjoin%
\definecolor{currentfill}{rgb}{0.188923,0.410910,0.556326}%
\pgfsetfillcolor{currentfill}%
\pgfsetlinewidth{0.000000pt}%
\definecolor{currentstroke}{rgb}{0.866013,0.889868,0.095953}%
\pgfsetstrokecolor{currentstroke}%
\pgfsetdash{}{0pt}%
\pgfpathmoveto{\pgfqpoint{1.947878in}{3.337740in}}%
\pgfpathlineto{\pgfqpoint{2.062008in}{3.862537in}}%
\pgfpathlineto{\pgfqpoint{2.148306in}{3.953837in}}%
\pgfpathclose%
\pgfusepath{fill}%
\end{pgfscope}%
\begin{pgfscope}%
\pgfpathrectangle{\pgfqpoint{0.539299in}{0.078740in}}{\pgfqpoint{7.842520in}{7.842520in}}%
\pgfusepath{clip}%
\pgfsetbuttcap%
\pgfsetroundjoin%
\definecolor{currentfill}{rgb}{0.282884,0.135920,0.453427}%
\pgfsetfillcolor{currentfill}%
\pgfsetlinewidth{0.000000pt}%
\definecolor{currentstroke}{rgb}{0.876168,0.891125,0.095250}%
\pgfsetstrokecolor{currentstroke}%
\pgfsetdash{}{0pt}%
\pgfpathmoveto{\pgfqpoint{6.549730in}{2.630860in}}%
\pgfpathlineto{\pgfqpoint{6.623869in}{2.702268in}}%
\pgfpathlineto{\pgfqpoint{6.483913in}{2.795763in}}%
\pgfpathclose%
\pgfusepath{fill}%
\end{pgfscope}%
\begin{pgfscope}%
\pgfpathrectangle{\pgfqpoint{0.539299in}{0.078740in}}{\pgfqpoint{7.842520in}{7.842520in}}%
\pgfusepath{clip}%
\pgfsetbuttcap%
\pgfsetroundjoin%
\definecolor{currentfill}{rgb}{0.263663,0.237631,0.518762}%
\pgfsetfillcolor{currentfill}%
\pgfsetlinewidth{0.000000pt}%
\definecolor{currentstroke}{rgb}{0.886271,0.892374,0.095374}%
\pgfsetstrokecolor{currentstroke}%
\pgfsetdash{}{0pt}%
\pgfpathmoveto{\pgfqpoint{5.914592in}{2.953804in}}%
\pgfpathlineto{\pgfqpoint{5.850784in}{3.093964in}}%
\pgfpathlineto{\pgfqpoint{5.775167in}{3.059358in}}%
\pgfpathclose%
\pgfusepath{fill}%
\end{pgfscope}%
\begin{pgfscope}%
\pgfpathrectangle{\pgfqpoint{0.539299in}{0.078740in}}{\pgfqpoint{7.842520in}{7.842520in}}%
\pgfusepath{clip}%
\pgfsetbuttcap%
\pgfsetroundjoin%
\definecolor{currentfill}{rgb}{0.278791,0.062145,0.386592}%
\pgfsetfillcolor{currentfill}%
\pgfsetlinewidth{0.000000pt}%
\definecolor{currentstroke}{rgb}{0.896320,0.893616,0.096335}%
\pgfsetstrokecolor{currentstroke}%
\pgfsetdash{}{0pt}%
\pgfpathmoveto{\pgfqpoint{6.903908in}{2.494991in}}%
\pgfpathlineto{\pgfqpoint{7.043838in}{2.379025in}}%
\pgfpathlineto{\pgfqpoint{6.976816in}{2.562893in}}%
\pgfpathclose%
\pgfusepath{fill}%
\end{pgfscope}%
\begin{pgfscope}%
\pgfpathrectangle{\pgfqpoint{0.539299in}{0.078740in}}{\pgfqpoint{7.842520in}{7.842520in}}%
\pgfusepath{clip}%
\pgfsetbuttcap%
\pgfsetroundjoin%
\definecolor{currentfill}{rgb}{0.195860,0.395433,0.555276}%
\pgfsetfillcolor{currentfill}%
\pgfsetlinewidth{0.000000pt}%
\definecolor{currentstroke}{rgb}{0.906311,0.894855,0.098125}%
\pgfsetstrokecolor{currentstroke}%
\pgfsetdash{}{0pt}%
\pgfpathmoveto{\pgfqpoint{5.220286in}{3.561627in}}%
\pgfpathlineto{\pgfqpoint{5.082079in}{3.713504in}}%
\pgfpathlineto{\pgfqpoint{5.143030in}{3.602315in}}%
\pgfpathclose%
\pgfusepath{fill}%
\end{pgfscope}%
\begin{pgfscope}%
\pgfpathrectangle{\pgfqpoint{0.539299in}{0.078740in}}{\pgfqpoint{7.842520in}{7.842520in}}%
\pgfusepath{clip}%
\pgfsetbuttcap%
\pgfsetroundjoin%
\definecolor{currentfill}{rgb}{0.174274,0.445044,0.557792}%
\pgfsetfillcolor{currentfill}%
\pgfsetlinewidth{0.000000pt}%
\definecolor{currentstroke}{rgb}{0.916242,0.896091,0.100717}%
\pgfsetstrokecolor{currentstroke}%
\pgfsetdash{}{0pt}%
\pgfpathmoveto{\pgfqpoint{5.082079in}{3.713504in}}%
\pgfpathlineto{\pgfqpoint{4.943959in}{3.876474in}}%
\pgfpathlineto{\pgfqpoint{4.865625in}{3.949141in}}%
\pgfpathclose%
\pgfusepath{fill}%
\end{pgfscope}%
\begin{pgfscope}%
\pgfpathrectangle{\pgfqpoint{0.539299in}{0.078740in}}{\pgfqpoint{7.842520in}{7.842520in}}%
\pgfusepath{clip}%
\pgfsetbuttcap%
\pgfsetroundjoin%
\definecolor{currentfill}{rgb}{0.206756,0.371758,0.553117}%
\pgfsetfillcolor{currentfill}%
\pgfsetlinewidth{0.000000pt}%
\definecolor{currentstroke}{rgb}{0.926106,0.897330,0.104071}%
\pgfsetstrokecolor{currentstroke}%
\pgfsetdash{}{0pt}%
\pgfpathmoveto{\pgfqpoint{5.143030in}{3.602315in}}%
\pgfpathlineto{\pgfqpoint{5.358646in}{3.421072in}}%
\pgfpathlineto{\pgfqpoint{5.220286in}{3.561627in}}%
\pgfpathclose%
\pgfusepath{fill}%
\end{pgfscope}%
\begin{pgfscope}%
\pgfpathrectangle{\pgfqpoint{0.539299in}{0.078740in}}{\pgfqpoint{7.842520in}{7.842520in}}%
\pgfusepath{clip}%
\pgfsetbuttcap%
\pgfsetroundjoin%
\definecolor{currentfill}{rgb}{0.163625,0.471133,0.558148}%
\pgfsetfillcolor{currentfill}%
\pgfsetlinewidth{0.000000pt}%
\definecolor{currentstroke}{rgb}{0.935904,0.898570,0.108131}%
\pgfsetstrokecolor{currentstroke}%
\pgfsetdash{}{0pt}%
\pgfpathmoveto{\pgfqpoint{4.865625in}{3.949141in}}%
\pgfpathlineto{\pgfqpoint{4.943959in}{3.876474in}}%
\pgfpathlineto{\pgfqpoint{4.805857in}{4.049409in}}%
\pgfpathclose%
\pgfusepath{fill}%
\end{pgfscope}%
\begin{pgfscope}%
\pgfpathrectangle{\pgfqpoint{0.539299in}{0.078740in}}{\pgfqpoint{7.842520in}{7.842520in}}%
\pgfusepath{clip}%
\pgfsetbuttcap%
\pgfsetroundjoin%
\definecolor{currentfill}{rgb}{0.226397,0.728888,0.462789}%
\pgfsetfillcolor{currentfill}%
\pgfsetlinewidth{0.000000pt}%
\definecolor{currentstroke}{rgb}{0.945636,0.899815,0.112838}%
\pgfsetstrokecolor{currentstroke}%
\pgfsetdash{}{0pt}%
\pgfpathmoveto{\pgfqpoint{2.825015in}{5.016747in}}%
\pgfpathlineto{\pgfqpoint{3.040850in}{5.259550in}}%
\pgfpathlineto{\pgfqpoint{2.910930in}{5.025499in}}%
\pgfpathclose%
\pgfusepath{fill}%
\end{pgfscope}%
\begin{pgfscope}%
\pgfpathrectangle{\pgfqpoint{0.539299in}{0.078740in}}{\pgfqpoint{7.842520in}{7.842520in}}%
\pgfusepath{clip}%
\pgfsetbuttcap%
\pgfsetroundjoin%
\definecolor{currentfill}{rgb}{0.282290,0.145912,0.461510}%
\pgfsetfillcolor{currentfill}%
\pgfsetlinewidth{0.000000pt}%
\definecolor{currentstroke}{rgb}{0.955300,0.901065,0.118128}%
\pgfsetstrokecolor{currentstroke}%
\pgfsetdash{}{0pt}%
\pgfpathmoveto{\pgfqpoint{6.483913in}{2.795763in}}%
\pgfpathlineto{\pgfqpoint{6.409497in}{2.726792in}}%
\pgfpathlineto{\pgfqpoint{6.549730in}{2.630860in}}%
\pgfpathclose%
\pgfusepath{fill}%
\end{pgfscope}%
\begin{pgfscope}%
\pgfpathrectangle{\pgfqpoint{0.539299in}{0.078740in}}{\pgfqpoint{7.842520in}{7.842520in}}%
\pgfusepath{clip}%
\pgfsetbuttcap%
\pgfsetroundjoin%
\definecolor{currentfill}{rgb}{0.223925,0.334994,0.548053}%
\pgfsetfillcolor{currentfill}%
\pgfsetlinewidth{0.000000pt}%
\definecolor{currentstroke}{rgb}{0.964894,0.902323,0.123941}%
\pgfsetstrokecolor{currentstroke}%
\pgfsetdash{}{0pt}%
\pgfpathmoveto{\pgfqpoint{5.497218in}{3.291318in}}%
\pgfpathlineto{\pgfqpoint{5.358646in}{3.421072in}}%
\pgfpathlineto{\pgfqpoint{5.281811in}{3.444596in}}%
\pgfpathclose%
\pgfusepath{fill}%
\end{pgfscope}%
\begin{pgfscope}%
\pgfpathrectangle{\pgfqpoint{0.539299in}{0.078740in}}{\pgfqpoint{7.842520in}{7.842520in}}%
\pgfusepath{clip}%
\pgfsetbuttcap%
\pgfsetroundjoin%
\definecolor{currentfill}{rgb}{0.283197,0.115680,0.436115}%
\pgfsetfillcolor{currentfill}%
\pgfsetlinewidth{0.000000pt}%
\definecolor{currentstroke}{rgb}{0.974417,0.903590,0.130215}%
\pgfsetstrokecolor{currentstroke}%
\pgfsetdash{}{0pt}%
\pgfpathmoveto{\pgfqpoint{6.690065in}{2.529737in}}%
\pgfpathlineto{\pgfqpoint{6.763891in}{2.602375in}}%
\pgfpathlineto{\pgfqpoint{6.623869in}{2.702268in}}%
\pgfpathclose%
\pgfusepath{fill}%
\end{pgfscope}%
\begin{pgfscope}%
\pgfpathrectangle{\pgfqpoint{0.539299in}{0.078740in}}{\pgfqpoint{7.842520in}{7.842520in}}%
\pgfusepath{clip}%
\pgfsetbuttcap%
\pgfsetroundjoin%
\definecolor{currentfill}{rgb}{0.223925,0.334994,0.548053}%
\pgfsetfillcolor{currentfill}%
\pgfsetlinewidth{0.000000pt}%
\definecolor{currentstroke}{rgb}{0.983868,0.904867,0.136897}%
\pgfsetstrokecolor{currentstroke}%
\pgfsetdash{}{0pt}%
\pgfpathmoveto{\pgfqpoint{1.975740in}{3.757494in}}%
\pgfpathlineto{\pgfqpoint{1.861534in}{3.249013in}}%
\pgfpathlineto{\pgfqpoint{1.775040in}{3.152116in}}%
\pgfpathclose%
\pgfusepath{fill}%
\end{pgfscope}%
\begin{pgfscope}%
\pgfpathrectangle{\pgfqpoint{0.539299in}{0.078740in}}{\pgfqpoint{7.842520in}{7.842520in}}%
\pgfusepath{clip}%
\pgfsetbuttcap%
\pgfsetroundjoin%
\definecolor{currentfill}{rgb}{0.304148,0.764704,0.419943}%
\pgfsetfillcolor{currentfill}%
\pgfsetlinewidth{0.000000pt}%
\definecolor{currentstroke}{rgb}{0.993248,0.906157,0.143936}%
\pgfsetstrokecolor{currentstroke}%
\pgfsetdash{}{0pt}%
\pgfpathmoveto{\pgfqpoint{3.754366in}{5.305678in}}%
\pgfpathlineto{\pgfqpoint{3.838249in}{5.208623in}}%
\pgfpathlineto{\pgfqpoint{3.700962in}{5.285351in}}%
\pgfpathclose%
\pgfusepath{fill}%
\end{pgfscope}%
\begin{pgfscope}%
\pgfpathrectangle{\pgfqpoint{0.539299in}{0.078740in}}{\pgfqpoint{7.842520in}{7.842520in}}%
\pgfusepath{clip}%
\pgfsetbuttcap%
\pgfsetroundjoin%
\definecolor{currentfill}{rgb}{0.140536,0.530132,0.555659}%
\pgfsetfillcolor{currentfill}%
\pgfsetlinewidth{0.000000pt}%
\definecolor{currentstroke}{rgb}{0.267004,0.004874,0.329415}%
\pgfsetstrokecolor{currentstroke}%
\pgfsetdash{}{0pt}%
\pgfpathmoveto{\pgfqpoint{4.805857in}{4.049409in}}%
\pgfpathlineto{\pgfqpoint{4.667716in}{4.230148in}}%
\pgfpathlineto{\pgfqpoint{4.588029in}{4.327753in}}%
\pgfpathclose%
\pgfusepath{fill}%
\end{pgfscope}%
\begin{pgfscope}%
\pgfpathrectangle{\pgfqpoint{0.539299in}{0.078740in}}{\pgfqpoint{7.842520in}{7.842520in}}%
\pgfusepath{clip}%
\pgfsetbuttcap%
\pgfsetroundjoin%
\definecolor{currentfill}{rgb}{0.278012,0.180367,0.486697}%
\pgfsetfillcolor{currentfill}%
\pgfsetlinewidth{0.000000pt}%
\definecolor{currentstroke}{rgb}{0.268510,0.009605,0.335427}%
\pgfsetstrokecolor{currentstroke}%
\pgfsetdash{}{0pt}%
\pgfpathmoveto{\pgfqpoint{6.194335in}{2.753858in}}%
\pgfpathlineto{\pgfqpoint{6.269433in}{2.819189in}}%
\pgfpathlineto{\pgfqpoint{6.129597in}{2.909848in}}%
\pgfpathclose%
\pgfusepath{fill}%
\end{pgfscope}%
\begin{pgfscope}%
\pgfpathrectangle{\pgfqpoint{0.539299in}{0.078740in}}{\pgfqpoint{7.842520in}{7.842520in}}%
\pgfusepath{clip}%
\pgfsetbuttcap%
\pgfsetroundjoin%
\definecolor{currentfill}{rgb}{0.352360,0.783011,0.392636}%
\pgfsetfillcolor{currentfill}%
\pgfsetlinewidth{0.000000pt}%
\definecolor{currentstroke}{rgb}{0.269944,0.014625,0.341379}%
\pgfsetstrokecolor{currentstroke}%
\pgfsetdash{}{0pt}%
\pgfpathmoveto{\pgfqpoint{3.344948in}{5.333814in}}%
\pgfpathlineto{\pgfqpoint{3.259589in}{5.378978in}}%
\pgfpathlineto{\pgfqpoint{3.480023in}{5.382736in}}%
\pgfpathclose%
\pgfusepath{fill}%
\end{pgfscope}%
\begin{pgfscope}%
\pgfpathrectangle{\pgfqpoint{0.539299in}{0.078740in}}{\pgfqpoint{7.842520in}{7.842520in}}%
\pgfusepath{clip}%
\pgfsetbuttcap%
\pgfsetroundjoin%
\definecolor{currentfill}{rgb}{0.344074,0.780029,0.397381}%
\pgfsetfillcolor{currentfill}%
\pgfsetlinewidth{0.000000pt}%
\definecolor{currentstroke}{rgb}{0.271305,0.019942,0.347269}%
\pgfsetstrokecolor{currentstroke}%
\pgfsetdash{}{0pt}%
\pgfpathmoveto{\pgfqpoint{3.700962in}{5.285351in}}%
\pgfpathlineto{\pgfqpoint{3.480023in}{5.382736in}}%
\pgfpathlineto{\pgfqpoint{3.616640in}{5.370253in}}%
\pgfpathclose%
\pgfusepath{fill}%
\end{pgfscope}%
\begin{pgfscope}%
\pgfpathrectangle{\pgfqpoint{0.539299in}{0.078740in}}{\pgfqpoint{7.842520in}{7.842520in}}%
\pgfusepath{clip}%
\pgfsetbuttcap%
\pgfsetroundjoin%
\definecolor{currentfill}{rgb}{0.273006,0.204520,0.501721}%
\pgfsetfillcolor{currentfill}%
\pgfsetlinewidth{0.000000pt}%
\definecolor{currentstroke}{rgb}{0.272594,0.025563,0.353093}%
\pgfsetstrokecolor{currentstroke}%
\pgfsetdash{}{0pt}%
\pgfpathmoveto{\pgfqpoint{6.129597in}{2.909848in}}%
\pgfpathlineto{\pgfqpoint{5.914592in}{2.953804in}}%
\pgfpathlineto{\pgfqpoint{6.054321in}{2.852638in}}%
\pgfpathclose%
\pgfusepath{fill}%
\end{pgfscope}%
\begin{pgfscope}%
\pgfpathrectangle{\pgfqpoint{0.539299in}{0.078740in}}{\pgfqpoint{7.842520in}{7.842520in}}%
\pgfusepath{clip}%
\pgfsetbuttcap%
\pgfsetroundjoin%
\definecolor{currentfill}{rgb}{0.132268,0.655014,0.519661}%
\pgfsetfillcolor{currentfill}%
\pgfsetlinewidth{0.000000pt}%
\definecolor{currentstroke}{rgb}{0.273809,0.031497,0.358853}%
\pgfsetstrokecolor{currentstroke}%
\pgfsetdash{}{0pt}%
\pgfpathmoveto{\pgfqpoint{2.613215in}{4.649492in}}%
\pgfpathlineto{\pgfqpoint{2.527103in}{4.606886in}}%
\pgfpathlineto{\pgfqpoint{2.738735in}{4.997476in}}%
\pgfpathclose%
\pgfusepath{fill}%
\end{pgfscope}%
\begin{pgfscope}%
\pgfpathrectangle{\pgfqpoint{0.539299in}{0.078740in}}{\pgfqpoint{7.842520in}{7.842520in}}%
\pgfusepath{clip}%
\pgfsetbuttcap%
\pgfsetroundjoin%
\definecolor{currentfill}{rgb}{0.243113,0.292092,0.538516}%
\pgfsetfillcolor{currentfill}%
\pgfsetlinewidth{0.000000pt}%
\definecolor{currentstroke}{rgb}{0.274952,0.037752,0.364543}%
\pgfsetstrokecolor{currentstroke}%
\pgfsetdash{}{0pt}%
\pgfpathmoveto{\pgfqpoint{5.497218in}{3.291318in}}%
\pgfpathlineto{\pgfqpoint{5.559838in}{3.161010in}}%
\pgfpathlineto{\pgfqpoint{5.636047in}{3.171258in}}%
\pgfpathclose%
\pgfusepath{fill}%
\end{pgfscope}%
\begin{pgfscope}%
\pgfpathrectangle{\pgfqpoint{0.539299in}{0.078740in}}{\pgfqpoint{7.842520in}{7.842520in}}%
\pgfusepath{clip}%
\pgfsetbuttcap%
\pgfsetroundjoin%
\definecolor{currentfill}{rgb}{0.162016,0.687316,0.499129}%
\pgfsetfillcolor{currentfill}%
\pgfsetlinewidth{0.000000pt}%
\definecolor{currentstroke}{rgb}{0.276022,0.044167,0.370164}%
\pgfsetstrokecolor{currentstroke}%
\pgfsetdash{}{0pt}%
\pgfpathmoveto{\pgfqpoint{2.613215in}{4.649492in}}%
\pgfpathlineto{\pgfqpoint{2.738735in}{4.997476in}}%
\pgfpathlineto{\pgfqpoint{2.825015in}{5.016747in}}%
\pgfpathclose%
\pgfusepath{fill}%
\end{pgfscope}%
\begin{pgfscope}%
\pgfpathrectangle{\pgfqpoint{0.539299in}{0.078740in}}{\pgfqpoint{7.842520in}{7.842520in}}%
\pgfusepath{clip}%
\pgfsetbuttcap%
\pgfsetroundjoin%
\definecolor{currentfill}{rgb}{0.282327,0.094955,0.417331}%
\pgfsetfillcolor{currentfill}%
\pgfsetlinewidth{0.000000pt}%
\definecolor{currentstroke}{rgb}{0.277018,0.050344,0.375715}%
\pgfsetstrokecolor{currentstroke}%
\pgfsetdash{}{0pt}%
\pgfpathmoveto{\pgfqpoint{6.903908in}{2.494991in}}%
\pgfpathlineto{\pgfqpoint{6.763891in}{2.602375in}}%
\pgfpathlineto{\pgfqpoint{6.690065in}{2.529737in}}%
\pgfpathclose%
\pgfusepath{fill}%
\end{pgfscope}%
\begin{pgfscope}%
\pgfpathrectangle{\pgfqpoint{0.539299in}{0.078740in}}{\pgfqpoint{7.842520in}{7.842520in}}%
\pgfusepath{clip}%
\pgfsetbuttcap%
\pgfsetroundjoin%
\definecolor{currentfill}{rgb}{0.125394,0.574318,0.549086}%
\pgfsetfillcolor{currentfill}%
\pgfsetlinewidth{0.000000pt}%
\definecolor{currentstroke}{rgb}{0.277941,0.056324,0.381191}%
\pgfsetstrokecolor{currentstroke}%
\pgfsetdash{}{0pt}%
\pgfpathmoveto{\pgfqpoint{4.667716in}{4.230148in}}%
\pgfpathlineto{\pgfqpoint{4.529493in}{4.415381in}}%
\pgfpathlineto{\pgfqpoint{4.449060in}{4.521490in}}%
\pgfpathclose%
\pgfusepath{fill}%
\end{pgfscope}%
\begin{pgfscope}%
\pgfpathrectangle{\pgfqpoint{0.539299in}{0.078740in}}{\pgfqpoint{7.842520in}{7.842520in}}%
\pgfusepath{clip}%
\pgfsetbuttcap%
\pgfsetroundjoin%
\definecolor{currentfill}{rgb}{0.283229,0.120777,0.440584}%
\pgfsetfillcolor{currentfill}%
\pgfsetlinewidth{0.000000pt}%
\definecolor{currentstroke}{rgb}{0.278791,0.062145,0.386592}%
\pgfsetstrokecolor{currentstroke}%
\pgfsetdash{}{0pt}%
\pgfpathmoveto{\pgfqpoint{6.690065in}{2.529737in}}%
\pgfpathlineto{\pgfqpoint{6.623869in}{2.702268in}}%
\pgfpathlineto{\pgfqpoint{6.549730in}{2.630860in}}%
\pgfpathclose%
\pgfusepath{fill}%
\end{pgfscope}%
\begin{pgfscope}%
\pgfpathrectangle{\pgfqpoint{0.539299in}{0.078740in}}{\pgfqpoint{7.842520in}{7.842520in}}%
\pgfusepath{clip}%
\pgfsetbuttcap%
\pgfsetroundjoin%
\definecolor{currentfill}{rgb}{0.119738,0.603785,0.541400}%
\pgfsetfillcolor{currentfill}%
\pgfsetlinewidth{0.000000pt}%
\definecolor{currentstroke}{rgb}{0.279566,0.067836,0.391917}%
\pgfsetstrokecolor{currentstroke}%
\pgfsetdash{}{0pt}%
\pgfpathmoveto{\pgfqpoint{4.449060in}{4.521490in}}%
\pgfpathlineto{\pgfqpoint{4.529493in}{4.415381in}}%
\pgfpathlineto{\pgfqpoint{4.391174in}{4.600585in}}%
\pgfpathclose%
\pgfusepath{fill}%
\end{pgfscope}%
\begin{pgfscope}%
\pgfpathrectangle{\pgfqpoint{0.539299in}{0.078740in}}{\pgfqpoint{7.842520in}{7.842520in}}%
\pgfusepath{clip}%
\pgfsetbuttcap%
\pgfsetroundjoin%
\definecolor{currentfill}{rgb}{0.319809,0.770914,0.411152}%
\pgfsetfillcolor{currentfill}%
\pgfsetlinewidth{0.000000pt}%
\definecolor{currentstroke}{rgb}{0.280267,0.073417,0.397163}%
\pgfsetstrokecolor{currentstroke}%
\pgfsetdash{}{0pt}%
\pgfpathmoveto{\pgfqpoint{3.126635in}{5.241459in}}%
\pgfpathlineto{\pgfqpoint{3.040850in}{5.259550in}}%
\pgfpathlineto{\pgfqpoint{3.259589in}{5.378978in}}%
\pgfpathclose%
\pgfusepath{fill}%
\end{pgfscope}%
\begin{pgfscope}%
\pgfpathrectangle{\pgfqpoint{0.539299in}{0.078740in}}{\pgfqpoint{7.842520in}{7.842520in}}%
\pgfusepath{clip}%
\pgfsetbuttcap%
\pgfsetroundjoin%
\definecolor{currentfill}{rgb}{0.276194,0.190074,0.493001}%
\pgfsetfillcolor{currentfill}%
\pgfsetlinewidth{0.000000pt}%
\definecolor{currentstroke}{rgb}{0.280894,0.078907,0.402329}%
\pgfsetstrokecolor{currentstroke}%
\pgfsetdash{}{0pt}%
\pgfpathmoveto{\pgfqpoint{6.054321in}{2.852638in}}%
\pgfpathlineto{\pgfqpoint{6.194335in}{2.753858in}}%
\pgfpathlineto{\pgfqpoint{6.129597in}{2.909848in}}%
\pgfpathclose%
\pgfusepath{fill}%
\end{pgfscope}%
\begin{pgfscope}%
\pgfpathrectangle{\pgfqpoint{0.539299in}{0.078740in}}{\pgfqpoint{7.842520in}{7.842520in}}%
\pgfusepath{clip}%
\pgfsetbuttcap%
\pgfsetroundjoin%
\definecolor{currentfill}{rgb}{0.266941,0.748751,0.440573}%
\pgfsetfillcolor{currentfill}%
\pgfsetlinewidth{0.000000pt}%
\definecolor{currentstroke}{rgb}{0.281446,0.084320,0.407414}%
\pgfsetstrokecolor{currentstroke}%
\pgfsetdash{}{0pt}%
\pgfpathmoveto{\pgfqpoint{3.838249in}{5.208623in}}%
\pgfpathlineto{\pgfqpoint{3.892829in}{5.198517in}}%
\pgfpathlineto{\pgfqpoint{3.976153in}{5.092524in}}%
\pgfpathclose%
\pgfusepath{fill}%
\end{pgfscope}%
\begin{pgfscope}%
\pgfpathrectangle{\pgfqpoint{0.539299in}{0.078740in}}{\pgfqpoint{7.842520in}{7.842520in}}%
\pgfusepath{clip}%
\pgfsetbuttcap%
\pgfsetroundjoin%
\definecolor{currentfill}{rgb}{0.253935,0.265254,0.529983}%
\pgfsetfillcolor{currentfill}%
\pgfsetlinewidth{0.000000pt}%
\definecolor{currentstroke}{rgb}{0.281924,0.089666,0.412415}%
\pgfsetstrokecolor{currentstroke}%
\pgfsetdash{}{0pt}%
\pgfpathmoveto{\pgfqpoint{5.699184in}{3.033538in}}%
\pgfpathlineto{\pgfqpoint{5.775167in}{3.059358in}}%
\pgfpathlineto{\pgfqpoint{5.636047in}{3.171258in}}%
\pgfpathclose%
\pgfusepath{fill}%
\end{pgfscope}%
\begin{pgfscope}%
\pgfpathrectangle{\pgfqpoint{0.539299in}{0.078740in}}{\pgfqpoint{7.842520in}{7.842520in}}%
\pgfusepath{clip}%
\pgfsetbuttcap%
\pgfsetroundjoin%
\definecolor{currentfill}{rgb}{0.185556,0.418570,0.556753}%
\pgfsetfillcolor{currentfill}%
\pgfsetlinewidth{0.000000pt}%
\definecolor{currentstroke}{rgb}{0.282327,0.094955,0.417331}%
\pgfsetstrokecolor{currentstroke}%
\pgfsetdash{}{0pt}%
\pgfpathmoveto{\pgfqpoint{5.143030in}{3.602315in}}%
\pgfpathlineto{\pgfqpoint{5.082079in}{3.713504in}}%
\pgfpathlineto{\pgfqpoint{5.004323in}{3.770773in}}%
\pgfpathclose%
\pgfusepath{fill}%
\end{pgfscope}%
\begin{pgfscope}%
\pgfpathrectangle{\pgfqpoint{0.539299in}{0.078740in}}{\pgfqpoint{7.842520in}{7.842520in}}%
\pgfusepath{clip}%
\pgfsetbuttcap%
\pgfsetroundjoin%
\definecolor{currentfill}{rgb}{0.280868,0.160771,0.472899}%
\pgfsetfillcolor{currentfill}%
\pgfsetlinewidth{0.000000pt}%
\definecolor{currentstroke}{rgb}{0.282656,0.100196,0.422160}%
\pgfsetstrokecolor{currentstroke}%
\pgfsetdash{}{0pt}%
\pgfpathmoveto{\pgfqpoint{6.269433in}{2.819189in}}%
\pgfpathlineto{\pgfqpoint{6.334603in}{2.655498in}}%
\pgfpathlineto{\pgfqpoint{6.409497in}{2.726792in}}%
\pgfpathclose%
\pgfusepath{fill}%
\end{pgfscope}%
\begin{pgfscope}%
\pgfpathrectangle{\pgfqpoint{0.539299in}{0.078740in}}{\pgfqpoint{7.842520in}{7.842520in}}%
\pgfusepath{clip}%
\pgfsetbuttcap%
\pgfsetroundjoin%
\definecolor{currentfill}{rgb}{0.206756,0.371758,0.553117}%
\pgfsetfillcolor{currentfill}%
\pgfsetlinewidth{0.000000pt}%
\definecolor{currentstroke}{rgb}{0.282910,0.105393,0.426902}%
\pgfsetstrokecolor{currentstroke}%
\pgfsetdash{}{0pt}%
\pgfpathmoveto{\pgfqpoint{5.281811in}{3.444596in}}%
\pgfpathlineto{\pgfqpoint{5.358646in}{3.421072in}}%
\pgfpathlineto{\pgfqpoint{5.143030in}{3.602315in}}%
\pgfpathclose%
\pgfusepath{fill}%
\end{pgfscope}%
\begin{pgfscope}%
\pgfpathrectangle{\pgfqpoint{0.539299in}{0.078740in}}{\pgfqpoint{7.842520in}{7.842520in}}%
\pgfusepath{clip}%
\pgfsetbuttcap%
\pgfsetroundjoin%
\definecolor{currentfill}{rgb}{0.174274,0.445044,0.557792}%
\pgfsetfillcolor{currentfill}%
\pgfsetlinewidth{0.000000pt}%
\definecolor{currentstroke}{rgb}{0.283091,0.110553,0.431554}%
\pgfsetstrokecolor{currentstroke}%
\pgfsetdash{}{0pt}%
\pgfpathmoveto{\pgfqpoint{4.865625in}{3.949141in}}%
\pgfpathlineto{\pgfqpoint{5.004323in}{3.770773in}}%
\pgfpathlineto{\pgfqpoint{5.082079in}{3.713504in}}%
\pgfpathclose%
\pgfusepath{fill}%
\end{pgfscope}%
\begin{pgfscope}%
\pgfpathrectangle{\pgfqpoint{0.539299in}{0.078740in}}{\pgfqpoint{7.842520in}{7.842520in}}%
\pgfusepath{clip}%
\pgfsetbuttcap%
\pgfsetroundjoin%
\definecolor{currentfill}{rgb}{0.279566,0.067836,0.391917}%
\pgfsetfillcolor{currentfill}%
\pgfsetlinewidth{0.000000pt}%
\definecolor{currentstroke}{rgb}{0.283197,0.115680,0.436115}%
\pgfsetstrokecolor{currentstroke}%
\pgfsetdash{}{0pt}%
\pgfpathmoveto{\pgfqpoint{7.043838in}{2.379025in}}%
\pgfpathlineto{\pgfqpoint{6.903908in}{2.494991in}}%
\pgfpathlineto{\pgfqpoint{6.830421in}{2.421820in}}%
\pgfpathclose%
\pgfusepath{fill}%
\end{pgfscope}%
\begin{pgfscope}%
\pgfpathrectangle{\pgfqpoint{0.539299in}{0.078740in}}{\pgfqpoint{7.842520in}{7.842520in}}%
\pgfusepath{clip}%
\pgfsetbuttcap%
\pgfsetroundjoin%
\definecolor{currentfill}{rgb}{0.137339,0.662252,0.515571}%
\pgfsetfillcolor{currentfill}%
\pgfsetlinewidth{0.000000pt}%
\definecolor{currentstroke}{rgb}{0.283229,0.120777,0.440584}%
\pgfsetstrokecolor{currentstroke}%
\pgfsetdash{}{0pt}%
\pgfpathmoveto{\pgfqpoint{4.391174in}{4.600585in}}%
\pgfpathlineto{\pgfqpoint{4.252784in}{4.779999in}}%
\pgfpathlineto{\pgfqpoint{4.170832in}{4.893238in}}%
\pgfpathclose%
\pgfusepath{fill}%
\end{pgfscope}%
\begin{pgfscope}%
\pgfpathrectangle{\pgfqpoint{0.539299in}{0.078740in}}{\pgfqpoint{7.842520in}{7.842520in}}%
\pgfusepath{clip}%
\pgfsetbuttcap%
\pgfsetroundjoin%
\definecolor{currentfill}{rgb}{0.232815,0.732247,0.459277}%
\pgfsetfillcolor{currentfill}%
\pgfsetlinewidth{0.000000pt}%
\definecolor{currentstroke}{rgb}{0.283187,0.125848,0.444960}%
\pgfsetstrokecolor{currentstroke}%
\pgfsetdash{}{0pt}%
\pgfpathmoveto{\pgfqpoint{3.976153in}{5.092524in}}%
\pgfpathlineto{\pgfqpoint{3.892829in}{5.198517in}}%
\pgfpathlineto{\pgfqpoint{4.114399in}{4.946663in}}%
\pgfpathclose%
\pgfusepath{fill}%
\end{pgfscope}%
\begin{pgfscope}%
\pgfpathrectangle{\pgfqpoint{0.539299in}{0.078740in}}{\pgfqpoint{7.842520in}{7.842520in}}%
\pgfusepath{clip}%
\pgfsetbuttcap%
\pgfsetroundjoin%
\definecolor{currentfill}{rgb}{0.175707,0.697900,0.491033}%
\pgfsetfillcolor{currentfill}%
\pgfsetlinewidth{0.000000pt}%
\definecolor{currentstroke}{rgb}{0.283072,0.130895,0.449241}%
\pgfsetstrokecolor{currentstroke}%
\pgfsetdash{}{0pt}%
\pgfpathmoveto{\pgfqpoint{4.252784in}{4.779999in}}%
\pgfpathlineto{\pgfqpoint{4.114399in}{4.946663in}}%
\pgfpathlineto{\pgfqpoint{4.031728in}{5.058097in}}%
\pgfpathclose%
\pgfusepath{fill}%
\end{pgfscope}%
\begin{pgfscope}%
\pgfpathrectangle{\pgfqpoint{0.539299in}{0.078740in}}{\pgfqpoint{7.842520in}{7.842520in}}%
\pgfusepath{clip}%
\pgfsetbuttcap%
\pgfsetroundjoin%
\definecolor{currentfill}{rgb}{0.223925,0.334994,0.548053}%
\pgfsetfillcolor{currentfill}%
\pgfsetlinewidth{0.000000pt}%
\definecolor{currentstroke}{rgb}{0.282884,0.135920,0.453427}%
\pgfsetstrokecolor{currentstroke}%
\pgfsetdash{}{0pt}%
\pgfpathmoveto{\pgfqpoint{5.281811in}{3.444596in}}%
\pgfpathlineto{\pgfqpoint{5.420728in}{3.297681in}}%
\pgfpathlineto{\pgfqpoint{5.497218in}{3.291318in}}%
\pgfpathclose%
\pgfusepath{fill}%
\end{pgfscope}%
\begin{pgfscope}%
\pgfpathrectangle{\pgfqpoint{0.539299in}{0.078740in}}{\pgfqpoint{7.842520in}{7.842520in}}%
\pgfusepath{clip}%
\pgfsetbuttcap%
\pgfsetroundjoin%
\definecolor{currentfill}{rgb}{0.344074,0.780029,0.397381}%
\pgfsetfillcolor{currentfill}%
\pgfsetlinewidth{0.000000pt}%
\definecolor{currentstroke}{rgb}{0.282623,0.140926,0.457517}%
\pgfsetstrokecolor{currentstroke}%
\pgfsetdash{}{0pt}%
\pgfpathmoveto{\pgfqpoint{3.616640in}{5.370253in}}%
\pgfpathlineto{\pgfqpoint{3.754366in}{5.305678in}}%
\pgfpathlineto{\pgfqpoint{3.700962in}{5.285351in}}%
\pgfpathclose%
\pgfusepath{fill}%
\end{pgfscope}%
\begin{pgfscope}%
\pgfpathrectangle{\pgfqpoint{0.539299in}{0.078740in}}{\pgfqpoint{7.842520in}{7.842520in}}%
\pgfusepath{clip}%
\pgfsetbuttcap%
\pgfsetroundjoin%
\definecolor{currentfill}{rgb}{0.260571,0.246922,0.522828}%
\pgfsetfillcolor{currentfill}%
\pgfsetlinewidth{0.000000pt}%
\definecolor{currentstroke}{rgb}{0.282290,0.145912,0.461510}%
\pgfsetstrokecolor{currentstroke}%
\pgfsetdash{}{0pt}%
\pgfpathmoveto{\pgfqpoint{5.775167in}{3.059358in}}%
\pgfpathlineto{\pgfqpoint{5.699184in}{3.033538in}}%
\pgfpathlineto{\pgfqpoint{5.914592in}{2.953804in}}%
\pgfpathclose%
\pgfusepath{fill}%
\end{pgfscope}%
\begin{pgfscope}%
\pgfpathrectangle{\pgfqpoint{0.539299in}{0.078740in}}{\pgfqpoint{7.842520in}{7.842520in}}%
\pgfusepath{clip}%
\pgfsetbuttcap%
\pgfsetroundjoin%
\definecolor{currentfill}{rgb}{0.151918,0.500685,0.557587}%
\pgfsetfillcolor{currentfill}%
\pgfsetlinewidth{0.000000pt}%
\definecolor{currentstroke}{rgb}{0.281887,0.150881,0.465405}%
\pgfsetstrokecolor{currentstroke}%
\pgfsetdash{}{0pt}%
\pgfpathmoveto{\pgfqpoint{4.805857in}{4.049409in}}%
\pgfpathlineto{\pgfqpoint{4.726876in}{4.135706in}}%
\pgfpathlineto{\pgfqpoint{4.865625in}{3.949141in}}%
\pgfpathclose%
\pgfusepath{fill}%
\end{pgfscope}%
\begin{pgfscope}%
\pgfpathrectangle{\pgfqpoint{0.539299in}{0.078740in}}{\pgfqpoint{7.842520in}{7.842520in}}%
\pgfusepath{clip}%
\pgfsetbuttcap%
\pgfsetroundjoin%
\definecolor{currentfill}{rgb}{0.192357,0.403199,0.555836}%
\pgfsetfillcolor{currentfill}%
\pgfsetlinewidth{0.000000pt}%
\definecolor{currentstroke}{rgb}{0.281412,0.155834,0.469201}%
\pgfsetstrokecolor{currentstroke}%
\pgfsetdash{}{0pt}%
\pgfpathmoveto{\pgfqpoint{2.062008in}{3.862537in}}%
\pgfpathlineto{\pgfqpoint{1.861534in}{3.249013in}}%
\pgfpathlineto{\pgfqpoint{1.975740in}{3.757494in}}%
\pgfpathclose%
\pgfusepath{fill}%
\end{pgfscope}%
\begin{pgfscope}%
\pgfpathrectangle{\pgfqpoint{0.539299in}{0.078740in}}{\pgfqpoint{7.842520in}{7.842520in}}%
\pgfusepath{clip}%
\pgfsetbuttcap%
\pgfsetroundjoin%
\definecolor{currentfill}{rgb}{0.235526,0.309527,0.542944}%
\pgfsetfillcolor{currentfill}%
\pgfsetlinewidth{0.000000pt}%
\definecolor{currentstroke}{rgb}{0.280868,0.160771,0.472899}%
\pgfsetstrokecolor{currentstroke}%
\pgfsetdash{}{0pt}%
\pgfpathmoveto{\pgfqpoint{5.497218in}{3.291318in}}%
\pgfpathlineto{\pgfqpoint{5.420728in}{3.297681in}}%
\pgfpathlineto{\pgfqpoint{5.559838in}{3.161010in}}%
\pgfpathclose%
\pgfusepath{fill}%
\end{pgfscope}%
\begin{pgfscope}%
\pgfpathrectangle{\pgfqpoint{0.539299in}{0.078740in}}{\pgfqpoint{7.842520in}{7.842520in}}%
\pgfusepath{clip}%
\pgfsetbuttcap%
\pgfsetroundjoin%
\definecolor{currentfill}{rgb}{0.281446,0.084320,0.407414}%
\pgfsetfillcolor{currentfill}%
\pgfsetlinewidth{0.000000pt}%
\definecolor{currentstroke}{rgb}{0.280255,0.165693,0.476498}%
\pgfsetstrokecolor{currentstroke}%
\pgfsetdash{}{0pt}%
\pgfpathmoveto{\pgfqpoint{6.830421in}{2.421820in}}%
\pgfpathlineto{\pgfqpoint{6.903908in}{2.494991in}}%
\pgfpathlineto{\pgfqpoint{6.690065in}{2.529737in}}%
\pgfpathclose%
\pgfusepath{fill}%
\end{pgfscope}%
\begin{pgfscope}%
\pgfpathrectangle{\pgfqpoint{0.539299in}{0.078740in}}{\pgfqpoint{7.842520in}{7.842520in}}%
\pgfusepath{clip}%
\pgfsetbuttcap%
\pgfsetroundjoin%
\definecolor{currentfill}{rgb}{0.282290,0.145912,0.461510}%
\pgfsetfillcolor{currentfill}%
\pgfsetlinewidth{0.000000pt}%
\definecolor{currentstroke}{rgb}{0.279574,0.170599,0.479997}%
\pgfsetstrokecolor{currentstroke}%
\pgfsetdash{}{0pt}%
\pgfpathmoveto{\pgfqpoint{6.549730in}{2.630860in}}%
\pgfpathlineto{\pgfqpoint{6.409497in}{2.726792in}}%
\pgfpathlineto{\pgfqpoint{6.334603in}{2.655498in}}%
\pgfpathclose%
\pgfusepath{fill}%
\end{pgfscope}%
\begin{pgfscope}%
\pgfpathrectangle{\pgfqpoint{0.539299in}{0.078740in}}{\pgfqpoint{7.842520in}{7.842520in}}%
\pgfusepath{clip}%
\pgfsetbuttcap%
\pgfsetroundjoin%
\definecolor{currentfill}{rgb}{0.140536,0.530132,0.555659}%
\pgfsetfillcolor{currentfill}%
\pgfsetlinewidth{0.000000pt}%
\definecolor{currentstroke}{rgb}{0.278826,0.175490,0.483397}%
\pgfsetstrokecolor{currentstroke}%
\pgfsetdash{}{0pt}%
\pgfpathmoveto{\pgfqpoint{4.588029in}{4.327753in}}%
\pgfpathlineto{\pgfqpoint{4.726876in}{4.135706in}}%
\pgfpathlineto{\pgfqpoint{4.805857in}{4.049409in}}%
\pgfpathclose%
\pgfusepath{fill}%
\end{pgfscope}%
\begin{pgfscope}%
\pgfpathrectangle{\pgfqpoint{0.539299in}{0.078740in}}{\pgfqpoint{7.842520in}{7.842520in}}%
\pgfusepath{clip}%
\pgfsetbuttcap%
\pgfsetroundjoin%
\definecolor{currentfill}{rgb}{0.279574,0.170599,0.479997}%
\pgfsetfillcolor{currentfill}%
\pgfsetlinewidth{0.000000pt}%
\definecolor{currentstroke}{rgb}{0.278012,0.180367,0.486697}%
\pgfsetstrokecolor{currentstroke}%
\pgfsetdash{}{0pt}%
\pgfpathmoveto{\pgfqpoint{6.269433in}{2.819189in}}%
\pgfpathlineto{\pgfqpoint{6.194335in}{2.753858in}}%
\pgfpathlineto{\pgfqpoint{6.334603in}{2.655498in}}%
\pgfpathclose%
\pgfusepath{fill}%
\end{pgfscope}%
\begin{pgfscope}%
\pgfpathrectangle{\pgfqpoint{0.539299in}{0.078740in}}{\pgfqpoint{7.842520in}{7.842520in}}%
\pgfusepath{clip}%
\pgfsetbuttcap%
\pgfsetroundjoin%
\definecolor{currentfill}{rgb}{0.248629,0.278775,0.534556}%
\pgfsetfillcolor{currentfill}%
\pgfsetlinewidth{0.000000pt}%
\definecolor{currentstroke}{rgb}{0.277134,0.185228,0.489898}%
\pgfsetstrokecolor{currentstroke}%
\pgfsetdash{}{0pt}%
\pgfpathmoveto{\pgfqpoint{5.636047in}{3.171258in}}%
\pgfpathlineto{\pgfqpoint{5.559838in}{3.161010in}}%
\pgfpathlineto{\pgfqpoint{5.699184in}{3.033538in}}%
\pgfpathclose%
\pgfusepath{fill}%
\end{pgfscope}%
\begin{pgfscope}%
\pgfpathrectangle{\pgfqpoint{0.539299in}{0.078740in}}{\pgfqpoint{7.842520in}{7.842520in}}%
\pgfusepath{clip}%
\pgfsetbuttcap%
\pgfsetroundjoin%
\definecolor{currentfill}{rgb}{0.124395,0.578002,0.548287}%
\pgfsetfillcolor{currentfill}%
\pgfsetlinewidth{0.000000pt}%
\definecolor{currentstroke}{rgb}{0.276194,0.190074,0.493001}%
\pgfsetstrokecolor{currentstroke}%
\pgfsetdash{}{0pt}%
\pgfpathmoveto{\pgfqpoint{4.449060in}{4.521490in}}%
\pgfpathlineto{\pgfqpoint{4.588029in}{4.327753in}}%
\pgfpathlineto{\pgfqpoint{4.667716in}{4.230148in}}%
\pgfpathclose%
\pgfusepath{fill}%
\end{pgfscope}%
\begin{pgfscope}%
\pgfpathrectangle{\pgfqpoint{0.539299in}{0.078740in}}{\pgfqpoint{7.842520in}{7.842520in}}%
\pgfusepath{clip}%
\pgfsetbuttcap%
\pgfsetroundjoin%
\definecolor{currentfill}{rgb}{0.141935,0.526453,0.555991}%
\pgfsetfillcolor{currentfill}%
\pgfsetlinewidth{0.000000pt}%
\definecolor{currentstroke}{rgb}{0.275191,0.194905,0.496005}%
\pgfsetstrokecolor{currentstroke}%
\pgfsetdash{}{0pt}%
\pgfpathmoveto{\pgfqpoint{2.234531in}{4.033237in}}%
\pgfpathlineto{\pgfqpoint{2.148306in}{3.953837in}}%
\pgfpathlineto{\pgfqpoint{2.267756in}{4.403873in}}%
\pgfpathclose%
\pgfusepath{fill}%
\end{pgfscope}%
\begin{pgfscope}%
\pgfpathrectangle{\pgfqpoint{0.539299in}{0.078740in}}{\pgfqpoint{7.842520in}{7.842520in}}%
\pgfusepath{clip}%
\pgfsetbuttcap%
\pgfsetroundjoin%
\definecolor{currentfill}{rgb}{0.277941,0.056324,0.381191}%
\pgfsetfillcolor{currentfill}%
\pgfsetlinewidth{0.000000pt}%
\definecolor{currentstroke}{rgb}{0.274128,0.199721,0.498911}%
\pgfsetstrokecolor{currentstroke}%
\pgfsetdash{}{0pt}%
\pgfpathmoveto{\pgfqpoint{6.830421in}{2.421820in}}%
\pgfpathlineto{\pgfqpoint{6.970702in}{2.305402in}}%
\pgfpathlineto{\pgfqpoint{7.043838in}{2.379025in}}%
\pgfpathclose%
\pgfusepath{fill}%
\end{pgfscope}%
\begin{pgfscope}%
\pgfpathrectangle{\pgfqpoint{0.539299in}{0.078740in}}{\pgfqpoint{7.842520in}{7.842520in}}%
\pgfusepath{clip}%
\pgfsetbuttcap%
\pgfsetroundjoin%
\definecolor{currentfill}{rgb}{0.124395,0.578002,0.548287}%
\pgfsetfillcolor{currentfill}%
\pgfsetlinewidth{0.000000pt}%
\definecolor{currentstroke}{rgb}{0.273006,0.204520,0.501721}%
\pgfsetstrokecolor{currentstroke}%
\pgfsetdash{}{0pt}%
\pgfpathmoveto{\pgfqpoint{2.440760in}{4.553021in}}%
\pgfpathlineto{\pgfqpoint{2.234531in}{4.033237in}}%
\pgfpathlineto{\pgfqpoint{2.354275in}{4.486045in}}%
\pgfpathclose%
\pgfusepath{fill}%
\end{pgfscope}%
\begin{pgfscope}%
\pgfpathrectangle{\pgfqpoint{0.539299in}{0.078740in}}{\pgfqpoint{7.842520in}{7.842520in}}%
\pgfusepath{clip}%
\pgfsetbuttcap%
\pgfsetroundjoin%
\definecolor{currentfill}{rgb}{0.271828,0.209303,0.504434}%
\pgfsetfillcolor{currentfill}%
\pgfsetlinewidth{0.000000pt}%
\definecolor{currentstroke}{rgb}{0.271828,0.209303,0.504434}%
\pgfsetstrokecolor{currentstroke}%
\pgfsetdash{}{0pt}%
\pgfpathmoveto{\pgfqpoint{5.914592in}{2.953804in}}%
\pgfpathlineto{\pgfqpoint{5.978689in}{2.800383in}}%
\pgfpathlineto{\pgfqpoint{6.054321in}{2.852638in}}%
\pgfpathclose%
\pgfusepath{fill}%
\end{pgfscope}%
\begin{pgfscope}%
\pgfpathrectangle{\pgfqpoint{0.539299in}{0.078740in}}{\pgfqpoint{7.842520in}{7.842520in}}%
\pgfusepath{clip}%
\pgfsetbuttcap%
\pgfsetroundjoin%
\definecolor{currentfill}{rgb}{0.311925,0.767822,0.415586}%
\pgfsetfillcolor{currentfill}%
\pgfsetlinewidth{0.000000pt}%
\definecolor{currentstroke}{rgb}{0.270595,0.214069,0.507052}%
\pgfsetstrokecolor{currentstroke}%
\pgfsetdash{}{0pt}%
\pgfpathmoveto{\pgfqpoint{3.838249in}{5.208623in}}%
\pgfpathlineto{\pgfqpoint{3.754366in}{5.305678in}}%
\pgfpathlineto{\pgfqpoint{3.892829in}{5.198517in}}%
\pgfpathclose%
\pgfusepath{fill}%
\end{pgfscope}%
\begin{pgfscope}%
\pgfpathrectangle{\pgfqpoint{0.539299in}{0.078740in}}{\pgfqpoint{7.842520in}{7.842520in}}%
\pgfusepath{clip}%
\pgfsetbuttcap%
\pgfsetroundjoin%
\definecolor{currentfill}{rgb}{0.283229,0.120777,0.440584}%
\pgfsetfillcolor{currentfill}%
\pgfsetlinewidth{0.000000pt}%
\definecolor{currentstroke}{rgb}{0.269308,0.218818,0.509577}%
\pgfsetstrokecolor{currentstroke}%
\pgfsetdash{}{0pt}%
\pgfpathmoveto{\pgfqpoint{6.549730in}{2.630860in}}%
\pgfpathlineto{\pgfqpoint{6.475077in}{2.555660in}}%
\pgfpathlineto{\pgfqpoint{6.690065in}{2.529737in}}%
\pgfpathclose%
\pgfusepath{fill}%
\end{pgfscope}%
\begin{pgfscope}%
\pgfpathrectangle{\pgfqpoint{0.539299in}{0.078740in}}{\pgfqpoint{7.842520in}{7.842520in}}%
\pgfusepath{clip}%
\pgfsetbuttcap%
\pgfsetroundjoin%
\definecolor{currentfill}{rgb}{0.123444,0.636809,0.528763}%
\pgfsetfillcolor{currentfill}%
\pgfsetlinewidth{0.000000pt}%
\definecolor{currentstroke}{rgb}{0.267968,0.223549,0.512008}%
\pgfsetstrokecolor{currentstroke}%
\pgfsetdash{}{0pt}%
\pgfpathmoveto{\pgfqpoint{4.391174in}{4.600585in}}%
\pgfpathlineto{\pgfqpoint{4.309978in}{4.711999in}}%
\pgfpathlineto{\pgfqpoint{4.449060in}{4.521490in}}%
\pgfpathclose%
\pgfusepath{fill}%
\end{pgfscope}%
\begin{pgfscope}%
\pgfpathrectangle{\pgfqpoint{0.539299in}{0.078740in}}{\pgfqpoint{7.842520in}{7.842520in}}%
\pgfusepath{clip}%
\pgfsetbuttcap%
\pgfsetroundjoin%
\definecolor{currentfill}{rgb}{0.404001,0.800275,0.362552}%
\pgfsetfillcolor{currentfill}%
\pgfsetlinewidth{0.000000pt}%
\definecolor{currentstroke}{rgb}{0.266580,0.228262,0.514349}%
\pgfsetstrokecolor{currentstroke}%
\pgfsetdash{}{0pt}%
\pgfpathmoveto{\pgfqpoint{3.480023in}{5.382736in}}%
\pgfpathlineto{\pgfqpoint{3.259589in}{5.378978in}}%
\pgfpathlineto{\pgfqpoint{3.394745in}{5.445102in}}%
\pgfpathclose%
\pgfusepath{fill}%
\end{pgfscope}%
\begin{pgfscope}%
\pgfpathrectangle{\pgfqpoint{0.539299in}{0.078740in}}{\pgfqpoint{7.842520in}{7.842520in}}%
\pgfusepath{clip}%
\pgfsetbuttcap%
\pgfsetroundjoin%
\definecolor{currentfill}{rgb}{0.262138,0.242286,0.520837}%
\pgfsetfillcolor{currentfill}%
\pgfsetlinewidth{0.000000pt}%
\definecolor{currentstroke}{rgb}{0.265145,0.232956,0.516599}%
\pgfsetstrokecolor{currentstroke}%
\pgfsetdash{}{0pt}%
\pgfpathmoveto{\pgfqpoint{5.914592in}{2.953804in}}%
\pgfpathlineto{\pgfqpoint{5.699184in}{3.033538in}}%
\pgfpathlineto{\pgfqpoint{5.838796in}{2.913871in}}%
\pgfpathclose%
\pgfusepath{fill}%
\end{pgfscope}%
\begin{pgfscope}%
\pgfpathrectangle{\pgfqpoint{0.539299in}{0.078740in}}{\pgfqpoint{7.842520in}{7.842520in}}%
\pgfusepath{clip}%
\pgfsetbuttcap%
\pgfsetroundjoin%
\definecolor{currentfill}{rgb}{0.288921,0.758394,0.428426}%
\pgfsetfillcolor{currentfill}%
\pgfsetlinewidth{0.000000pt}%
\definecolor{currentstroke}{rgb}{0.263663,0.237631,0.518762}%
\pgfsetstrokecolor{currentstroke}%
\pgfsetdash{}{0pt}%
\pgfpathmoveto{\pgfqpoint{2.825015in}{5.016747in}}%
\pgfpathlineto{\pgfqpoint{2.954612in}{5.267216in}}%
\pgfpathlineto{\pgfqpoint{3.040850in}{5.259550in}}%
\pgfpathclose%
\pgfusepath{fill}%
\end{pgfscope}%
\begin{pgfscope}%
\pgfpathrectangle{\pgfqpoint{0.539299in}{0.078740in}}{\pgfqpoint{7.842520in}{7.842520in}}%
\pgfusepath{clip}%
\pgfsetbuttcap%
\pgfsetroundjoin%
\definecolor{currentfill}{rgb}{0.140210,0.665859,0.513427}%
\pgfsetfillcolor{currentfill}%
\pgfsetlinewidth{0.000000pt}%
\definecolor{currentstroke}{rgb}{0.262138,0.242286,0.520837}%
\pgfsetstrokecolor{currentstroke}%
\pgfsetdash{}{0pt}%
\pgfpathmoveto{\pgfqpoint{4.170832in}{4.893238in}}%
\pgfpathlineto{\pgfqpoint{4.309978in}{4.711999in}}%
\pgfpathlineto{\pgfqpoint{4.391174in}{4.600585in}}%
\pgfpathclose%
\pgfusepath{fill}%
\end{pgfscope}%
\begin{pgfscope}%
\pgfpathrectangle{\pgfqpoint{0.539299in}{0.078740in}}{\pgfqpoint{7.842520in}{7.842520in}}%
\pgfusepath{clip}%
\pgfsetbuttcap%
\pgfsetroundjoin%
\definecolor{currentfill}{rgb}{0.275191,0.194905,0.496005}%
\pgfsetfillcolor{currentfill}%
\pgfsetlinewidth{0.000000pt}%
\definecolor{currentstroke}{rgb}{0.260571,0.246922,0.522828}%
\pgfsetstrokecolor{currentstroke}%
\pgfsetdash{}{0pt}%
\pgfpathmoveto{\pgfqpoint{5.978689in}{2.800383in}}%
\pgfpathlineto{\pgfqpoint{6.194335in}{2.753858in}}%
\pgfpathlineto{\pgfqpoint{6.054321in}{2.852638in}}%
\pgfpathclose%
\pgfusepath{fill}%
\end{pgfscope}%
\begin{pgfscope}%
\pgfpathrectangle{\pgfqpoint{0.539299in}{0.078740in}}{\pgfqpoint{7.842520in}{7.842520in}}%
\pgfusepath{clip}%
\pgfsetbuttcap%
\pgfsetroundjoin%
\definecolor{currentfill}{rgb}{0.282623,0.140926,0.457517}%
\pgfsetfillcolor{currentfill}%
\pgfsetlinewidth{0.000000pt}%
\definecolor{currentstroke}{rgb}{0.258965,0.251537,0.524736}%
\pgfsetstrokecolor{currentstroke}%
\pgfsetdash{}{0pt}%
\pgfpathmoveto{\pgfqpoint{6.334603in}{2.655498in}}%
\pgfpathlineto{\pgfqpoint{6.475077in}{2.555660in}}%
\pgfpathlineto{\pgfqpoint{6.549730in}{2.630860in}}%
\pgfpathclose%
\pgfusepath{fill}%
\end{pgfscope}%
\begin{pgfscope}%
\pgfpathrectangle{\pgfqpoint{0.539299in}{0.078740in}}{\pgfqpoint{7.842520in}{7.842520in}}%
\pgfusepath{clip}%
\pgfsetbuttcap%
\pgfsetroundjoin%
\definecolor{currentfill}{rgb}{0.246070,0.738910,0.452024}%
\pgfsetfillcolor{currentfill}%
\pgfsetlinewidth{0.000000pt}%
\definecolor{currentstroke}{rgb}{0.257322,0.256130,0.526563}%
\pgfsetstrokecolor{currentstroke}%
\pgfsetdash{}{0pt}%
\pgfpathmoveto{\pgfqpoint{4.114399in}{4.946663in}}%
\pgfpathlineto{\pgfqpoint{3.892829in}{5.198517in}}%
\pgfpathlineto{\pgfqpoint{4.031728in}{5.058097in}}%
\pgfpathclose%
\pgfusepath{fill}%
\end{pgfscope}%
\begin{pgfscope}%
\pgfpathrectangle{\pgfqpoint{0.539299in}{0.078740in}}{\pgfqpoint{7.842520in}{7.842520in}}%
\pgfusepath{clip}%
\pgfsetbuttcap%
\pgfsetroundjoin%
\definecolor{currentfill}{rgb}{0.231674,0.318106,0.544834}%
\pgfsetfillcolor{currentfill}%
\pgfsetlinewidth{0.000000pt}%
\definecolor{currentstroke}{rgb}{0.255645,0.260703,0.528312}%
\pgfsetstrokecolor{currentstroke}%
\pgfsetdash{}{0pt}%
\pgfpathmoveto{\pgfqpoint{1.775040in}{3.152116in}}%
\pgfpathlineto{\pgfqpoint{1.688478in}{3.045686in}}%
\pgfpathlineto{\pgfqpoint{1.803826in}{3.497087in}}%
\pgfpathclose%
\pgfusepath{fill}%
\end{pgfscope}%
\begin{pgfscope}%
\pgfpathrectangle{\pgfqpoint{0.539299in}{0.078740in}}{\pgfqpoint{7.842520in}{7.842520in}}%
\pgfusepath{clip}%
\pgfsetbuttcap%
\pgfsetroundjoin%
\definecolor{currentfill}{rgb}{0.185783,0.704891,0.485273}%
\pgfsetfillcolor{currentfill}%
\pgfsetlinewidth{0.000000pt}%
\definecolor{currentstroke}{rgb}{0.253935,0.265254,0.529983}%
\pgfsetstrokecolor{currentstroke}%
\pgfsetdash{}{0pt}%
\pgfpathmoveto{\pgfqpoint{4.031728in}{5.058097in}}%
\pgfpathlineto{\pgfqpoint{4.170832in}{4.893238in}}%
\pgfpathlineto{\pgfqpoint{4.252784in}{4.779999in}}%
\pgfpathclose%
\pgfusepath{fill}%
\end{pgfscope}%
\begin{pgfscope}%
\pgfpathrectangle{\pgfqpoint{0.539299in}{0.078740in}}{\pgfqpoint{7.842520in}{7.842520in}}%
\pgfusepath{clip}%
\pgfsetbuttcap%
\pgfsetroundjoin%
\definecolor{currentfill}{rgb}{0.267968,0.223549,0.512008}%
\pgfsetfillcolor{currentfill}%
\pgfsetlinewidth{0.000000pt}%
\definecolor{currentstroke}{rgb}{0.252194,0.269783,0.531579}%
\pgfsetstrokecolor{currentstroke}%
\pgfsetdash{}{0pt}%
\pgfpathmoveto{\pgfqpoint{5.838796in}{2.913871in}}%
\pgfpathlineto{\pgfqpoint{5.978689in}{2.800383in}}%
\pgfpathlineto{\pgfqpoint{5.914592in}{2.953804in}}%
\pgfpathclose%
\pgfusepath{fill}%
\end{pgfscope}%
\begin{pgfscope}%
\pgfpathrectangle{\pgfqpoint{0.539299in}{0.078740in}}{\pgfqpoint{7.842520in}{7.842520in}}%
\pgfusepath{clip}%
\pgfsetbuttcap%
\pgfsetroundjoin%
\definecolor{currentfill}{rgb}{0.412913,0.803041,0.357269}%
\pgfsetfillcolor{currentfill}%
\pgfsetlinewidth{0.000000pt}%
\definecolor{currentstroke}{rgb}{0.250425,0.274290,0.533103}%
\pgfsetstrokecolor{currentstroke}%
\pgfsetdash{}{0pt}%
\pgfpathmoveto{\pgfqpoint{3.480023in}{5.382736in}}%
\pgfpathlineto{\pgfqpoint{3.531606in}{5.447979in}}%
\pgfpathlineto{\pgfqpoint{3.616640in}{5.370253in}}%
\pgfpathclose%
\pgfusepath{fill}%
\end{pgfscope}%
\begin{pgfscope}%
\pgfpathrectangle{\pgfqpoint{0.539299in}{0.078740in}}{\pgfqpoint{7.842520in}{7.842520in}}%
\pgfusepath{clip}%
\pgfsetbuttcap%
\pgfsetroundjoin%
\definecolor{currentfill}{rgb}{0.386433,0.794644,0.372886}%
\pgfsetfillcolor{currentfill}%
\pgfsetlinewidth{0.000000pt}%
\definecolor{currentstroke}{rgb}{0.248629,0.278775,0.534556}%
\pgfsetstrokecolor{currentstroke}%
\pgfsetdash{}{0pt}%
\pgfpathmoveto{\pgfqpoint{3.259589in}{5.378978in}}%
\pgfpathlineto{\pgfqpoint{3.040850in}{5.259550in}}%
\pgfpathlineto{\pgfqpoint{3.173672in}{5.414731in}}%
\pgfpathclose%
\pgfusepath{fill}%
\end{pgfscope}%
\begin{pgfscope}%
\pgfpathrectangle{\pgfqpoint{0.539299in}{0.078740in}}{\pgfqpoint{7.842520in}{7.842520in}}%
\pgfusepath{clip}%
\pgfsetbuttcap%
\pgfsetroundjoin%
\definecolor{currentfill}{rgb}{0.137339,0.662252,0.515571}%
\pgfsetfillcolor{currentfill}%
\pgfsetlinewidth{0.000000pt}%
\definecolor{currentstroke}{rgb}{0.246811,0.283237,0.535941}%
\pgfsetstrokecolor{currentstroke}%
\pgfsetdash{}{0pt}%
\pgfpathmoveto{\pgfqpoint{2.652170in}{4.965747in}}%
\pgfpathlineto{\pgfqpoint{2.527103in}{4.606886in}}%
\pgfpathlineto{\pgfqpoint{2.440760in}{4.553021in}}%
\pgfpathclose%
\pgfusepath{fill}%
\end{pgfscope}%
\begin{pgfscope}%
\pgfpathrectangle{\pgfqpoint{0.539299in}{0.078740in}}{\pgfqpoint{7.842520in}{7.842520in}}%
\pgfusepath{clip}%
\pgfsetbuttcap%
\pgfsetroundjoin%
\definecolor{currentfill}{rgb}{0.281446,0.084320,0.407414}%
\pgfsetfillcolor{currentfill}%
\pgfsetlinewidth{0.000000pt}%
\definecolor{currentstroke}{rgb}{0.244972,0.287675,0.537260}%
\pgfsetstrokecolor{currentstroke}%
\pgfsetdash{}{0pt}%
\pgfpathmoveto{\pgfqpoint{6.690065in}{2.529737in}}%
\pgfpathlineto{\pgfqpoint{6.756390in}{2.344144in}}%
\pgfpathlineto{\pgfqpoint{6.830421in}{2.421820in}}%
\pgfpathclose%
\pgfusepath{fill}%
\end{pgfscope}%
\begin{pgfscope}%
\pgfpathrectangle{\pgfqpoint{0.539299in}{0.078740in}}{\pgfqpoint{7.842520in}{7.842520in}}%
\pgfusepath{clip}%
\pgfsetbuttcap%
\pgfsetroundjoin%
\definecolor{currentfill}{rgb}{0.283197,0.115680,0.436115}%
\pgfsetfillcolor{currentfill}%
\pgfsetlinewidth{0.000000pt}%
\definecolor{currentstroke}{rgb}{0.243113,0.292092,0.538516}%
\pgfsetstrokecolor{currentstroke}%
\pgfsetdash{}{0pt}%
\pgfpathmoveto{\pgfqpoint{6.475077in}{2.555660in}}%
\pgfpathlineto{\pgfqpoint{6.615697in}{2.452498in}}%
\pgfpathlineto{\pgfqpoint{6.690065in}{2.529737in}}%
\pgfpathclose%
\pgfusepath{fill}%
\end{pgfscope}%
\begin{pgfscope}%
\pgfpathrectangle{\pgfqpoint{0.539299in}{0.078740in}}{\pgfqpoint{7.842520in}{7.842520in}}%
\pgfusepath{clip}%
\pgfsetbuttcap%
\pgfsetroundjoin%
\definecolor{currentfill}{rgb}{0.175707,0.697900,0.491033}%
\pgfsetfillcolor{currentfill}%
\pgfsetlinewidth{0.000000pt}%
\definecolor{currentstroke}{rgb}{0.241237,0.296485,0.539709}%
\pgfsetstrokecolor{currentstroke}%
\pgfsetdash{}{0pt}%
\pgfpathmoveto{\pgfqpoint{2.738735in}{4.997476in}}%
\pgfpathlineto{\pgfqpoint{2.527103in}{4.606886in}}%
\pgfpathlineto{\pgfqpoint{2.652170in}{4.965747in}}%
\pgfpathclose%
\pgfusepath{fill}%
\end{pgfscope}%
\begin{pgfscope}%
\pgfpathrectangle{\pgfqpoint{0.539299in}{0.078740in}}{\pgfqpoint{7.842520in}{7.842520in}}%
\pgfusepath{clip}%
\pgfsetbuttcap%
\pgfsetroundjoin%
\definecolor{currentfill}{rgb}{0.197636,0.391528,0.554969}%
\pgfsetfillcolor{currentfill}%
\pgfsetlinewidth{0.000000pt}%
\definecolor{currentstroke}{rgb}{0.239346,0.300855,0.540844}%
\pgfsetstrokecolor{currentstroke}%
\pgfsetdash{}{0pt}%
\pgfpathmoveto{\pgfqpoint{1.775040in}{3.152116in}}%
\pgfpathlineto{\pgfqpoint{1.889628in}{3.636546in}}%
\pgfpathlineto{\pgfqpoint{1.975740in}{3.757494in}}%
\pgfpathclose%
\pgfusepath{fill}%
\end{pgfscope}%
\begin{pgfscope}%
\pgfpathrectangle{\pgfqpoint{0.539299in}{0.078740in}}{\pgfqpoint{7.842520in}{7.842520in}}%
\pgfusepath{clip}%
\pgfsetbuttcap%
\pgfsetroundjoin%
\definecolor{currentfill}{rgb}{0.197636,0.391528,0.554969}%
\pgfsetfillcolor{currentfill}%
\pgfsetlinewidth{0.000000pt}%
\definecolor{currentstroke}{rgb}{0.237441,0.305202,0.541921}%
\pgfsetstrokecolor{currentstroke}%
\pgfsetdash{}{0pt}%
\pgfpathmoveto{\pgfqpoint{5.143030in}{3.602315in}}%
\pgfpathlineto{\pgfqpoint{5.204442in}{3.480326in}}%
\pgfpathlineto{\pgfqpoint{5.281811in}{3.444596in}}%
\pgfpathclose%
\pgfusepath{fill}%
\end{pgfscope}%
\begin{pgfscope}%
\pgfpathrectangle{\pgfqpoint{0.539299in}{0.078740in}}{\pgfqpoint{7.842520in}{7.842520in}}%
\pgfusepath{clip}%
\pgfsetbuttcap%
\pgfsetroundjoin%
\definecolor{currentfill}{rgb}{0.280868,0.160771,0.472899}%
\pgfsetfillcolor{currentfill}%
\pgfsetlinewidth{0.000000pt}%
\definecolor{currentstroke}{rgb}{0.235526,0.309527,0.542944}%
\pgfsetstrokecolor{currentstroke}%
\pgfsetdash{}{0pt}%
\pgfpathmoveto{\pgfqpoint{6.259304in}{2.584902in}}%
\pgfpathlineto{\pgfqpoint{6.334603in}{2.655498in}}%
\pgfpathlineto{\pgfqpoint{6.194335in}{2.753858in}}%
\pgfpathclose%
\pgfusepath{fill}%
\end{pgfscope}%
\begin{pgfscope}%
\pgfpathrectangle{\pgfqpoint{0.539299in}{0.078740in}}{\pgfqpoint{7.842520in}{7.842520in}}%
\pgfusepath{clip}%
\pgfsetbuttcap%
\pgfsetroundjoin%
\definecolor{currentfill}{rgb}{0.208623,0.367752,0.552675}%
\pgfsetfillcolor{currentfill}%
\pgfsetlinewidth{0.000000pt}%
\definecolor{currentstroke}{rgb}{0.233603,0.313828,0.543914}%
\pgfsetstrokecolor{currentstroke}%
\pgfsetdash{}{0pt}%
\pgfpathmoveto{\pgfqpoint{5.204442in}{3.480326in}}%
\pgfpathlineto{\pgfqpoint{5.420728in}{3.297681in}}%
\pgfpathlineto{\pgfqpoint{5.281811in}{3.444596in}}%
\pgfpathclose%
\pgfusepath{fill}%
\end{pgfscope}%
\begin{pgfscope}%
\pgfpathrectangle{\pgfqpoint{0.539299in}{0.078740in}}{\pgfqpoint{7.842520in}{7.842520in}}%
\pgfusepath{clip}%
\pgfsetbuttcap%
\pgfsetroundjoin%
\definecolor{currentfill}{rgb}{0.174274,0.445044,0.557792}%
\pgfsetfillcolor{currentfill}%
\pgfsetlinewidth{0.000000pt}%
\definecolor{currentstroke}{rgb}{0.231674,0.318106,0.544834}%
\pgfsetstrokecolor{currentstroke}%
\pgfsetdash{}{0pt}%
\pgfpathmoveto{\pgfqpoint{4.925899in}{3.838472in}}%
\pgfpathlineto{\pgfqpoint{5.143030in}{3.602315in}}%
\pgfpathlineto{\pgfqpoint{5.004323in}{3.770773in}}%
\pgfpathclose%
\pgfusepath{fill}%
\end{pgfscope}%
\begin{pgfscope}%
\pgfpathrectangle{\pgfqpoint{0.539299in}{0.078740in}}{\pgfqpoint{7.842520in}{7.842520in}}%
\pgfusepath{clip}%
\pgfsetbuttcap%
\pgfsetroundjoin%
\definecolor{currentfill}{rgb}{0.125394,0.574318,0.549086}%
\pgfsetfillcolor{currentfill}%
\pgfsetlinewidth{0.000000pt}%
\definecolor{currentstroke}{rgb}{0.229739,0.322361,0.545706}%
\pgfsetstrokecolor{currentstroke}%
\pgfsetdash{}{0pt}%
\pgfpathmoveto{\pgfqpoint{2.267756in}{4.403873in}}%
\pgfpathlineto{\pgfqpoint{2.354275in}{4.486045in}}%
\pgfpathlineto{\pgfqpoint{2.234531in}{4.033237in}}%
\pgfpathclose%
\pgfusepath{fill}%
\end{pgfscope}%
\begin{pgfscope}%
\pgfpathrectangle{\pgfqpoint{0.539299in}{0.078740in}}{\pgfqpoint{7.842520in}{7.842520in}}%
\pgfusepath{clip}%
\pgfsetbuttcap%
\pgfsetroundjoin%
\definecolor{currentfill}{rgb}{0.276194,0.190074,0.493001}%
\pgfsetfillcolor{currentfill}%
\pgfsetlinewidth{0.000000pt}%
\definecolor{currentstroke}{rgb}{0.227802,0.326594,0.546532}%
\pgfsetstrokecolor{currentstroke}%
\pgfsetdash{}{0pt}%
\pgfpathmoveto{\pgfqpoint{6.118863in}{2.691325in}}%
\pgfpathlineto{\pgfqpoint{6.194335in}{2.753858in}}%
\pgfpathlineto{\pgfqpoint{5.978689in}{2.800383in}}%
\pgfpathclose%
\pgfusepath{fill}%
\end{pgfscope}%
\begin{pgfscope}%
\pgfpathrectangle{\pgfqpoint{0.539299in}{0.078740in}}{\pgfqpoint{7.842520in}{7.842520in}}%
\pgfusepath{clip}%
\pgfsetbuttcap%
\pgfsetroundjoin%
\definecolor{currentfill}{rgb}{0.277941,0.056324,0.381191}%
\pgfsetfillcolor{currentfill}%
\pgfsetlinewidth{0.000000pt}%
\definecolor{currentstroke}{rgb}{0.225863,0.330805,0.547314}%
\pgfsetstrokecolor{currentstroke}%
\pgfsetdash{}{0pt}%
\pgfpathmoveto{\pgfqpoint{6.897057in}{2.228556in}}%
\pgfpathlineto{\pgfqpoint{6.970702in}{2.305402in}}%
\pgfpathlineto{\pgfqpoint{6.830421in}{2.421820in}}%
\pgfpathclose%
\pgfusepath{fill}%
\end{pgfscope}%
\begin{pgfscope}%
\pgfpathrectangle{\pgfqpoint{0.539299in}{0.078740in}}{\pgfqpoint{7.842520in}{7.842520in}}%
\pgfusepath{clip}%
\pgfsetbuttcap%
\pgfsetroundjoin%
\definecolor{currentfill}{rgb}{0.231674,0.318106,0.544834}%
\pgfsetfillcolor{currentfill}%
\pgfsetlinewidth{0.000000pt}%
\definecolor{currentstroke}{rgb}{0.223925,0.334994,0.548053}%
\pgfsetstrokecolor{currentstroke}%
\pgfsetdash{}{0pt}%
\pgfpathmoveto{\pgfqpoint{5.483218in}{3.162772in}}%
\pgfpathlineto{\pgfqpoint{5.559838in}{3.161010in}}%
\pgfpathlineto{\pgfqpoint{5.420728in}{3.297681in}}%
\pgfpathclose%
\pgfusepath{fill}%
\end{pgfscope}%
\begin{pgfscope}%
\pgfpathrectangle{\pgfqpoint{0.539299in}{0.078740in}}{\pgfqpoint{7.842520in}{7.842520in}}%
\pgfusepath{clip}%
\pgfsetbuttcap%
\pgfsetroundjoin%
\definecolor{currentfill}{rgb}{0.282327,0.094955,0.417331}%
\pgfsetfillcolor{currentfill}%
\pgfsetlinewidth{0.000000pt}%
\definecolor{currentstroke}{rgb}{0.221989,0.339161,0.548752}%
\pgfsetstrokecolor{currentstroke}%
\pgfsetdash{}{0pt}%
\pgfpathmoveto{\pgfqpoint{6.615697in}{2.452498in}}%
\pgfpathlineto{\pgfqpoint{6.756390in}{2.344144in}}%
\pgfpathlineto{\pgfqpoint{6.690065in}{2.529737in}}%
\pgfpathclose%
\pgfusepath{fill}%
\end{pgfscope}%
\begin{pgfscope}%
\pgfpathrectangle{\pgfqpoint{0.539299in}{0.078740in}}{\pgfqpoint{7.842520in}{7.842520in}}%
\pgfusepath{clip}%
\pgfsetbuttcap%
\pgfsetroundjoin%
\definecolor{currentfill}{rgb}{0.162142,0.474838,0.558140}%
\pgfsetfillcolor{currentfill}%
\pgfsetlinewidth{0.000000pt}%
\definecolor{currentstroke}{rgb}{0.220057,0.343307,0.549413}%
\pgfsetstrokecolor{currentstroke}%
\pgfsetdash{}{0pt}%
\pgfpathmoveto{\pgfqpoint{4.925899in}{3.838472in}}%
\pgfpathlineto{\pgfqpoint{5.004323in}{3.770773in}}%
\pgfpathlineto{\pgfqpoint{4.865625in}{3.949141in}}%
\pgfpathclose%
\pgfusepath{fill}%
\end{pgfscope}%
\begin{pgfscope}%
\pgfpathrectangle{\pgfqpoint{0.539299in}{0.078740in}}{\pgfqpoint{7.842520in}{7.842520in}}%
\pgfusepath{clip}%
\pgfsetbuttcap%
\pgfsetroundjoin%
\definecolor{currentfill}{rgb}{0.430983,0.808473,0.346476}%
\pgfsetfillcolor{currentfill}%
\pgfsetlinewidth{0.000000pt}%
\definecolor{currentstroke}{rgb}{0.218130,0.347432,0.550038}%
\pgfsetstrokecolor{currentstroke}%
\pgfsetdash{}{0pt}%
\pgfpathmoveto{\pgfqpoint{3.394745in}{5.445102in}}%
\pgfpathlineto{\pgfqpoint{3.531606in}{5.447979in}}%
\pgfpathlineto{\pgfqpoint{3.480023in}{5.382736in}}%
\pgfpathclose%
\pgfusepath{fill}%
\end{pgfscope}%
\begin{pgfscope}%
\pgfpathrectangle{\pgfqpoint{0.539299in}{0.078740in}}{\pgfqpoint{7.842520in}{7.842520in}}%
\pgfusepath{clip}%
\pgfsetbuttcap%
\pgfsetroundjoin%
\definecolor{currentfill}{rgb}{0.237441,0.305202,0.541921}%
\pgfsetfillcolor{currentfill}%
\pgfsetlinewidth{0.000000pt}%
\definecolor{currentstroke}{rgb}{0.216210,0.351535,0.550627}%
\pgfsetstrokecolor{currentstroke}%
\pgfsetdash{}{0pt}%
\pgfpathmoveto{\pgfqpoint{1.803826in}{3.497087in}}%
\pgfpathlineto{\pgfqpoint{1.688478in}{3.045686in}}%
\pgfpathlineto{\pgfqpoint{1.601948in}{2.927986in}}%
\pgfpathclose%
\pgfusepath{fill}%
\end{pgfscope}%
\begin{pgfscope}%
\pgfpathrectangle{\pgfqpoint{0.539299in}{0.078740in}}{\pgfqpoint{7.842520in}{7.842520in}}%
\pgfusepath{clip}%
\pgfsetbuttcap%
\pgfsetroundjoin%
\definecolor{currentfill}{rgb}{0.246811,0.283237,0.535941}%
\pgfsetfillcolor{currentfill}%
\pgfsetlinewidth{0.000000pt}%
\definecolor{currentstroke}{rgb}{0.214298,0.355619,0.551184}%
\pgfsetstrokecolor{currentstroke}%
\pgfsetdash{}{0pt}%
\pgfpathmoveto{\pgfqpoint{5.622835in}{3.018878in}}%
\pgfpathlineto{\pgfqpoint{5.699184in}{3.033538in}}%
\pgfpathlineto{\pgfqpoint{5.559838in}{3.161010in}}%
\pgfpathclose%
\pgfusepath{fill}%
\end{pgfscope}%
\begin{pgfscope}%
\pgfpathrectangle{\pgfqpoint{0.539299in}{0.078740in}}{\pgfqpoint{7.842520in}{7.842520in}}%
\pgfusepath{clip}%
\pgfsetbuttcap%
\pgfsetroundjoin%
\definecolor{currentfill}{rgb}{0.412913,0.803041,0.357269}%
\pgfsetfillcolor{currentfill}%
\pgfsetlinewidth{0.000000pt}%
\definecolor{currentstroke}{rgb}{0.212395,0.359683,0.551710}%
\pgfsetstrokecolor{currentstroke}%
\pgfsetdash{}{0pt}%
\pgfpathmoveto{\pgfqpoint{3.616640in}{5.370253in}}%
\pgfpathlineto{\pgfqpoint{3.531606in}{5.447979in}}%
\pgfpathlineto{\pgfqpoint{3.754366in}{5.305678in}}%
\pgfpathclose%
\pgfusepath{fill}%
\end{pgfscope}%
\begin{pgfscope}%
\pgfpathrectangle{\pgfqpoint{0.539299in}{0.078740in}}{\pgfqpoint{7.842520in}{7.842520in}}%
\pgfusepath{clip}%
\pgfsetbuttcap%
\pgfsetroundjoin%
\definecolor{currentfill}{rgb}{0.144759,0.519093,0.556572}%
\pgfsetfillcolor{currentfill}%
\pgfsetlinewidth{0.000000pt}%
\definecolor{currentstroke}{rgb}{0.210503,0.363727,0.552206}%
\pgfsetstrokecolor{currentstroke}%
\pgfsetdash{}{0pt}%
\pgfpathmoveto{\pgfqpoint{4.726876in}{4.135706in}}%
\pgfpathlineto{\pgfqpoint{4.786562in}{4.030722in}}%
\pgfpathlineto{\pgfqpoint{4.865625in}{3.949141in}}%
\pgfpathclose%
\pgfusepath{fill}%
\end{pgfscope}%
\begin{pgfscope}%
\pgfpathrectangle{\pgfqpoint{0.539299in}{0.078740in}}{\pgfqpoint{7.842520in}{7.842520in}}%
\pgfusepath{clip}%
\pgfsetbuttcap%
\pgfsetroundjoin%
\definecolor{currentfill}{rgb}{0.266941,0.748751,0.440573}%
\pgfsetfillcolor{currentfill}%
\pgfsetlinewidth{0.000000pt}%
\definecolor{currentstroke}{rgb}{0.208623,0.367752,0.552675}%
\pgfsetstrokecolor{currentstroke}%
\pgfsetdash{}{0pt}%
\pgfpathmoveto{\pgfqpoint{2.738735in}{4.997476in}}%
\pgfpathlineto{\pgfqpoint{2.867996in}{5.262317in}}%
\pgfpathlineto{\pgfqpoint{2.825015in}{5.016747in}}%
\pgfpathclose%
\pgfusepath{fill}%
\end{pgfscope}%
\begin{pgfscope}%
\pgfpathrectangle{\pgfqpoint{0.539299in}{0.078740in}}{\pgfqpoint{7.842520in}{7.842520in}}%
\pgfusepath{clip}%
\pgfsetbuttcap%
\pgfsetroundjoin%
\definecolor{currentfill}{rgb}{0.279574,0.170599,0.479997}%
\pgfsetfillcolor{currentfill}%
\pgfsetlinewidth{0.000000pt}%
\definecolor{currentstroke}{rgb}{0.206756,0.371758,0.553117}%
\pgfsetstrokecolor{currentstroke}%
\pgfsetdash{}{0pt}%
\pgfpathmoveto{\pgfqpoint{6.194335in}{2.753858in}}%
\pgfpathlineto{\pgfqpoint{6.118863in}{2.691325in}}%
\pgfpathlineto{\pgfqpoint{6.259304in}{2.584902in}}%
\pgfpathclose%
\pgfusepath{fill}%
\end{pgfscope}%
\begin{pgfscope}%
\pgfpathrectangle{\pgfqpoint{0.539299in}{0.078740in}}{\pgfqpoint{7.842520in}{7.842520in}}%
\pgfusepath{clip}%
\pgfsetbuttcap%
\pgfsetroundjoin%
\definecolor{currentfill}{rgb}{0.377779,0.791781,0.377939}%
\pgfsetfillcolor{currentfill}%
\pgfsetlinewidth{0.000000pt}%
\definecolor{currentstroke}{rgb}{0.204903,0.375746,0.553533}%
\pgfsetstrokecolor{currentstroke}%
\pgfsetdash{}{0pt}%
\pgfpathmoveto{\pgfqpoint{3.173672in}{5.414731in}}%
\pgfpathlineto{\pgfqpoint{3.040850in}{5.259550in}}%
\pgfpathlineto{\pgfqpoint{2.954612in}{5.267216in}}%
\pgfpathclose%
\pgfusepath{fill}%
\end{pgfscope}%
\begin{pgfscope}%
\pgfpathrectangle{\pgfqpoint{0.539299in}{0.078740in}}{\pgfqpoint{7.842520in}{7.842520in}}%
\pgfusepath{clip}%
\pgfsetbuttcap%
\pgfsetroundjoin%
\definecolor{currentfill}{rgb}{0.282884,0.135920,0.453427}%
\pgfsetfillcolor{currentfill}%
\pgfsetlinewidth{0.000000pt}%
\definecolor{currentstroke}{rgb}{0.203063,0.379716,0.553925}%
\pgfsetstrokecolor{currentstroke}%
\pgfsetdash{}{0pt}%
\pgfpathmoveto{\pgfqpoint{6.475077in}{2.555660in}}%
\pgfpathlineto{\pgfqpoint{6.334603in}{2.655498in}}%
\pgfpathlineto{\pgfqpoint{6.399983in}{2.479315in}}%
\pgfpathclose%
\pgfusepath{fill}%
\end{pgfscope}%
\begin{pgfscope}%
\pgfpathrectangle{\pgfqpoint{0.539299in}{0.078740in}}{\pgfqpoint{7.842520in}{7.842520in}}%
\pgfusepath{clip}%
\pgfsetbuttcap%
\pgfsetroundjoin%
\definecolor{currentfill}{rgb}{0.144759,0.519093,0.556572}%
\pgfsetfillcolor{currentfill}%
\pgfsetlinewidth{0.000000pt}%
\definecolor{currentstroke}{rgb}{0.201239,0.383670,0.554294}%
\pgfsetstrokecolor{currentstroke}%
\pgfsetdash{}{0pt}%
\pgfpathmoveto{\pgfqpoint{2.181328in}{4.304148in}}%
\pgfpathlineto{\pgfqpoint{2.148306in}{3.953837in}}%
\pgfpathlineto{\pgfqpoint{2.062008in}{3.862537in}}%
\pgfpathclose%
\pgfusepath{fill}%
\end{pgfscope}%
\begin{pgfscope}%
\pgfpathrectangle{\pgfqpoint{0.539299in}{0.078740in}}{\pgfqpoint{7.842520in}{7.842520in}}%
\pgfusepath{clip}%
\pgfsetbuttcap%
\pgfsetroundjoin%
\definecolor{currentfill}{rgb}{0.253935,0.265254,0.529983}%
\pgfsetfillcolor{currentfill}%
\pgfsetlinewidth{0.000000pt}%
\definecolor{currentstroke}{rgb}{0.199430,0.387607,0.554642}%
\pgfsetstrokecolor{currentstroke}%
\pgfsetdash{}{0pt}%
\pgfpathmoveto{\pgfqpoint{5.838796in}{2.913871in}}%
\pgfpathlineto{\pgfqpoint{5.699184in}{3.033538in}}%
\pgfpathlineto{\pgfqpoint{5.622835in}{3.018878in}}%
\pgfpathclose%
\pgfusepath{fill}%
\end{pgfscope}%
\begin{pgfscope}%
\pgfpathrectangle{\pgfqpoint{0.539299in}{0.078740in}}{\pgfqpoint{7.842520in}{7.842520in}}%
\pgfusepath{clip}%
\pgfsetbuttcap%
\pgfsetroundjoin%
\definecolor{currentfill}{rgb}{0.278791,0.062145,0.386592}%
\pgfsetfillcolor{currentfill}%
\pgfsetlinewidth{0.000000pt}%
\definecolor{currentstroke}{rgb}{0.197636,0.391528,0.554969}%
\pgfsetstrokecolor{currentstroke}%
\pgfsetdash{}{0pt}%
\pgfpathmoveto{\pgfqpoint{6.830421in}{2.421820in}}%
\pgfpathlineto{\pgfqpoint{6.756390in}{2.344144in}}%
\pgfpathlineto{\pgfqpoint{6.897057in}{2.228556in}}%
\pgfpathclose%
\pgfusepath{fill}%
\end{pgfscope}%
\begin{pgfscope}%
\pgfpathrectangle{\pgfqpoint{0.539299in}{0.078740in}}{\pgfqpoint{7.842520in}{7.842520in}}%
\pgfusepath{clip}%
\pgfsetbuttcap%
\pgfsetroundjoin%
\definecolor{currentfill}{rgb}{0.203063,0.379716,0.553925}%
\pgfsetfillcolor{currentfill}%
\pgfsetlinewidth{0.000000pt}%
\definecolor{currentstroke}{rgb}{0.195860,0.395433,0.555276}%
\pgfsetstrokecolor{currentstroke}%
\pgfsetdash{}{0pt}%
\pgfpathmoveto{\pgfqpoint{1.803826in}{3.497087in}}%
\pgfpathlineto{\pgfqpoint{1.889628in}{3.636546in}}%
\pgfpathlineto{\pgfqpoint{1.775040in}{3.152116in}}%
\pgfpathclose%
\pgfusepath{fill}%
\end{pgfscope}%
\begin{pgfscope}%
\pgfpathrectangle{\pgfqpoint{0.539299in}{0.078740in}}{\pgfqpoint{7.842520in}{7.842520in}}%
\pgfusepath{clip}%
\pgfsetbuttcap%
\pgfsetroundjoin%
\definecolor{currentfill}{rgb}{0.123463,0.581687,0.547445}%
\pgfsetfillcolor{currentfill}%
\pgfsetlinewidth{0.000000pt}%
\definecolor{currentstroke}{rgb}{0.194100,0.399323,0.555565}%
\pgfsetstrokecolor{currentstroke}%
\pgfsetdash{}{0pt}%
\pgfpathmoveto{\pgfqpoint{4.726876in}{4.135706in}}%
\pgfpathlineto{\pgfqpoint{4.588029in}{4.327753in}}%
\pgfpathlineto{\pgfqpoint{4.507522in}{4.430367in}}%
\pgfpathclose%
\pgfusepath{fill}%
\end{pgfscope}%
\begin{pgfscope}%
\pgfpathrectangle{\pgfqpoint{0.539299in}{0.078740in}}{\pgfqpoint{7.842520in}{7.842520in}}%
\pgfusepath{clip}%
\pgfsetbuttcap%
\pgfsetroundjoin%
\definecolor{currentfill}{rgb}{0.319809,0.770914,0.411152}%
\pgfsetfillcolor{currentfill}%
\pgfsetlinewidth{0.000000pt}%
\definecolor{currentstroke}{rgb}{0.192357,0.403199,0.555836}%
\pgfsetstrokecolor{currentstroke}%
\pgfsetdash{}{0pt}%
\pgfpathmoveto{\pgfqpoint{2.867996in}{5.262317in}}%
\pgfpathlineto{\pgfqpoint{2.954612in}{5.267216in}}%
\pgfpathlineto{\pgfqpoint{2.825015in}{5.016747in}}%
\pgfpathclose%
\pgfusepath{fill}%
\end{pgfscope}%
\begin{pgfscope}%
\pgfpathrectangle{\pgfqpoint{0.539299in}{0.078740in}}{\pgfqpoint{7.842520in}{7.842520in}}%
\pgfusepath{clip}%
\pgfsetbuttcap%
\pgfsetroundjoin%
\definecolor{currentfill}{rgb}{0.283229,0.120777,0.440584}%
\pgfsetfillcolor{currentfill}%
\pgfsetlinewidth{0.000000pt}%
\definecolor{currentstroke}{rgb}{0.190631,0.407061,0.556089}%
\pgfsetstrokecolor{currentstroke}%
\pgfsetdash{}{0pt}%
\pgfpathmoveto{\pgfqpoint{6.399983in}{2.479315in}}%
\pgfpathlineto{\pgfqpoint{6.615697in}{2.452498in}}%
\pgfpathlineto{\pgfqpoint{6.475077in}{2.555660in}}%
\pgfpathclose%
\pgfusepath{fill}%
\end{pgfscope}%
\begin{pgfscope}%
\pgfpathrectangle{\pgfqpoint{0.539299in}{0.078740in}}{\pgfqpoint{7.842520in}{7.842520in}}%
\pgfusepath{clip}%
\pgfsetbuttcap%
\pgfsetroundjoin%
\definecolor{currentfill}{rgb}{0.265145,0.232956,0.516599}%
\pgfsetfillcolor{currentfill}%
\pgfsetlinewidth{0.000000pt}%
\definecolor{currentstroke}{rgb}{0.188923,0.410910,0.556326}%
\pgfsetstrokecolor{currentstroke}%
\pgfsetdash{}{0pt}%
\pgfpathmoveto{\pgfqpoint{5.838796in}{2.913871in}}%
\pgfpathlineto{\pgfqpoint{5.762666in}{2.883788in}}%
\pgfpathlineto{\pgfqpoint{5.978689in}{2.800383in}}%
\pgfpathclose%
\pgfusepath{fill}%
\end{pgfscope}%
\begin{pgfscope}%
\pgfpathrectangle{\pgfqpoint{0.539299in}{0.078740in}}{\pgfqpoint{7.842520in}{7.842520in}}%
\pgfusepath{clip}%
\pgfsetbuttcap%
\pgfsetroundjoin%
\definecolor{currentfill}{rgb}{0.208623,0.367752,0.552675}%
\pgfsetfillcolor{currentfill}%
\pgfsetlinewidth{0.000000pt}%
\definecolor{currentstroke}{rgb}{0.187231,0.414746,0.556547}%
\pgfsetstrokecolor{currentstroke}%
\pgfsetdash{}{0pt}%
\pgfpathmoveto{\pgfqpoint{5.343771in}{3.316408in}}%
\pgfpathlineto{\pgfqpoint{5.420728in}{3.297681in}}%
\pgfpathlineto{\pgfqpoint{5.204442in}{3.480326in}}%
\pgfpathclose%
\pgfusepath{fill}%
\end{pgfscope}%
\begin{pgfscope}%
\pgfpathrectangle{\pgfqpoint{0.539299in}{0.078740in}}{\pgfqpoint{7.842520in}{7.842520in}}%
\pgfusepath{clip}%
\pgfsetbuttcap%
\pgfsetroundjoin%
\definecolor{currentfill}{rgb}{0.185556,0.418570,0.556753}%
\pgfsetfillcolor{currentfill}%
\pgfsetlinewidth{0.000000pt}%
\definecolor{currentstroke}{rgb}{0.185556,0.418570,0.556753}%
\pgfsetstrokecolor{currentstroke}%
\pgfsetdash{}{0pt}%
\pgfpathmoveto{\pgfqpoint{5.143030in}{3.602315in}}%
\pgfpathlineto{\pgfqpoint{5.065172in}{3.654551in}}%
\pgfpathlineto{\pgfqpoint{5.204442in}{3.480326in}}%
\pgfpathclose%
\pgfusepath{fill}%
\end{pgfscope}%
\begin{pgfscope}%
\pgfpathrectangle{\pgfqpoint{0.539299in}{0.078740in}}{\pgfqpoint{7.842520in}{7.842520in}}%
\pgfusepath{clip}%
\pgfsetbuttcap%
\pgfsetroundjoin%
\definecolor{currentfill}{rgb}{0.221989,0.339161,0.548752}%
\pgfsetfillcolor{currentfill}%
\pgfsetlinewidth{0.000000pt}%
\definecolor{currentstroke}{rgb}{0.183898,0.422383,0.556944}%
\pgfsetstrokecolor{currentstroke}%
\pgfsetdash{}{0pt}%
\pgfpathmoveto{\pgfqpoint{5.420728in}{3.297681in}}%
\pgfpathlineto{\pgfqpoint{5.343771in}{3.316408in}}%
\pgfpathlineto{\pgfqpoint{5.483218in}{3.162772in}}%
\pgfpathclose%
\pgfusepath{fill}%
\end{pgfscope}%
\begin{pgfscope}%
\pgfpathrectangle{\pgfqpoint{0.539299in}{0.078740in}}{\pgfqpoint{7.842520in}{7.842520in}}%
\pgfusepath{clip}%
\pgfsetbuttcap%
\pgfsetroundjoin%
\definecolor{currentfill}{rgb}{0.282290,0.145912,0.461510}%
\pgfsetfillcolor{currentfill}%
\pgfsetlinewidth{0.000000pt}%
\definecolor{currentstroke}{rgb}{0.182256,0.426184,0.557120}%
\pgfsetstrokecolor{currentstroke}%
\pgfsetdash{}{0pt}%
\pgfpathmoveto{\pgfqpoint{6.399983in}{2.479315in}}%
\pgfpathlineto{\pgfqpoint{6.334603in}{2.655498in}}%
\pgfpathlineto{\pgfqpoint{6.259304in}{2.584902in}}%
\pgfpathclose%
\pgfusepath{fill}%
\end{pgfscope}%
\begin{pgfscope}%
\pgfpathrectangle{\pgfqpoint{0.539299in}{0.078740in}}{\pgfqpoint{7.842520in}{7.842520in}}%
\pgfusepath{clip}%
\pgfsetbuttcap%
\pgfsetroundjoin%
\definecolor{currentfill}{rgb}{0.458674,0.816363,0.329727}%
\pgfsetfillcolor{currentfill}%
\pgfsetlinewidth{0.000000pt}%
\definecolor{currentstroke}{rgb}{0.180629,0.429975,0.557282}%
\pgfsetstrokecolor{currentstroke}%
\pgfsetdash{}{0pt}%
\pgfpathmoveto{\pgfqpoint{3.308861in}{5.497668in}}%
\pgfpathlineto{\pgfqpoint{3.394745in}{5.445102in}}%
\pgfpathlineto{\pgfqpoint{3.259589in}{5.378978in}}%
\pgfpathclose%
\pgfusepath{fill}%
\end{pgfscope}%
\begin{pgfscope}%
\pgfpathrectangle{\pgfqpoint{0.539299in}{0.078740in}}{\pgfqpoint{7.842520in}{7.842520in}}%
\pgfusepath{clip}%
\pgfsetbuttcap%
\pgfsetroundjoin%
\definecolor{currentfill}{rgb}{0.174274,0.445044,0.557792}%
\pgfsetfillcolor{currentfill}%
\pgfsetlinewidth{0.000000pt}%
\definecolor{currentstroke}{rgb}{0.179019,0.433756,0.557430}%
\pgfsetstrokecolor{currentstroke}%
\pgfsetdash{}{0pt}%
\pgfpathmoveto{\pgfqpoint{5.065172in}{3.654551in}}%
\pgfpathlineto{\pgfqpoint{5.143030in}{3.602315in}}%
\pgfpathlineto{\pgfqpoint{4.925899in}{3.838472in}}%
\pgfpathclose%
\pgfusepath{fill}%
\end{pgfscope}%
\begin{pgfscope}%
\pgfpathrectangle{\pgfqpoint{0.539299in}{0.078740in}}{\pgfqpoint{7.842520in}{7.842520in}}%
\pgfusepath{clip}%
\pgfsetbuttcap%
\pgfsetroundjoin%
\definecolor{currentfill}{rgb}{0.237441,0.305202,0.541921}%
\pgfsetfillcolor{currentfill}%
\pgfsetlinewidth{0.000000pt}%
\definecolor{currentstroke}{rgb}{0.177423,0.437527,0.557565}%
\pgfsetstrokecolor{currentstroke}%
\pgfsetdash{}{0pt}%
\pgfpathmoveto{\pgfqpoint{5.622835in}{3.018878in}}%
\pgfpathlineto{\pgfqpoint{5.559838in}{3.161010in}}%
\pgfpathlineto{\pgfqpoint{5.483218in}{3.162772in}}%
\pgfpathclose%
\pgfusepath{fill}%
\end{pgfscope}%
\begin{pgfscope}%
\pgfpathrectangle{\pgfqpoint{0.539299in}{0.078740in}}{\pgfqpoint{7.842520in}{7.842520in}}%
\pgfusepath{clip}%
\pgfsetbuttcap%
\pgfsetroundjoin%
\definecolor{currentfill}{rgb}{0.120638,0.625828,0.533488}%
\pgfsetfillcolor{currentfill}%
\pgfsetlinewidth{0.000000pt}%
\definecolor{currentstroke}{rgb}{0.175841,0.441290,0.557685}%
\pgfsetstrokecolor{currentstroke}%
\pgfsetdash{}{0pt}%
\pgfpathmoveto{\pgfqpoint{4.449060in}{4.521490in}}%
\pgfpathlineto{\pgfqpoint{4.367781in}{4.630418in}}%
\pgfpathlineto{\pgfqpoint{4.588029in}{4.327753in}}%
\pgfpathclose%
\pgfusepath{fill}%
\end{pgfscope}%
\begin{pgfscope}%
\pgfpathrectangle{\pgfqpoint{0.539299in}{0.078740in}}{\pgfqpoint{7.842520in}{7.842520in}}%
\pgfusepath{clip}%
\pgfsetbuttcap%
\pgfsetroundjoin%
\definecolor{currentfill}{rgb}{0.132268,0.655014,0.519661}%
\pgfsetfillcolor{currentfill}%
\pgfsetlinewidth{0.000000pt}%
\definecolor{currentstroke}{rgb}{0.174274,0.445044,0.557792}%
\pgfsetstrokecolor{currentstroke}%
\pgfsetdash{}{0pt}%
\pgfpathmoveto{\pgfqpoint{4.309978in}{4.711999in}}%
\pgfpathlineto{\pgfqpoint{4.367781in}{4.630418in}}%
\pgfpathlineto{\pgfqpoint{4.449060in}{4.521490in}}%
\pgfpathclose%
\pgfusepath{fill}%
\end{pgfscope}%
\begin{pgfscope}%
\pgfpathrectangle{\pgfqpoint{0.539299in}{0.078740in}}{\pgfqpoint{7.842520in}{7.842520in}}%
\pgfusepath{clip}%
\pgfsetbuttcap%
\pgfsetroundjoin%
\definecolor{currentfill}{rgb}{0.150476,0.504369,0.557430}%
\pgfsetfillcolor{currentfill}%
\pgfsetlinewidth{0.000000pt}%
\definecolor{currentstroke}{rgb}{0.172719,0.448791,0.557885}%
\pgfsetstrokecolor{currentstroke}%
\pgfsetdash{}{0pt}%
\pgfpathmoveto{\pgfqpoint{4.865625in}{3.949141in}}%
\pgfpathlineto{\pgfqpoint{4.786562in}{4.030722in}}%
\pgfpathlineto{\pgfqpoint{4.925899in}{3.838472in}}%
\pgfpathclose%
\pgfusepath{fill}%
\end{pgfscope}%
\begin{pgfscope}%
\pgfpathrectangle{\pgfqpoint{0.539299in}{0.078740in}}{\pgfqpoint{7.842520in}{7.842520in}}%
\pgfusepath{clip}%
\pgfsetbuttcap%
\pgfsetroundjoin%
\definecolor{currentfill}{rgb}{0.458674,0.816363,0.329727}%
\pgfsetfillcolor{currentfill}%
\pgfsetlinewidth{0.000000pt}%
\definecolor{currentstroke}{rgb}{0.171176,0.452530,0.557965}%
\pgfsetstrokecolor{currentstroke}%
\pgfsetdash{}{0pt}%
\pgfpathmoveto{\pgfqpoint{3.173672in}{5.414731in}}%
\pgfpathlineto{\pgfqpoint{3.308861in}{5.497668in}}%
\pgfpathlineto{\pgfqpoint{3.259589in}{5.378978in}}%
\pgfpathclose%
\pgfusepath{fill}%
\end{pgfscope}%
\begin{pgfscope}%
\pgfpathrectangle{\pgfqpoint{0.539299in}{0.078740in}}{\pgfqpoint{7.842520in}{7.842520in}}%
\pgfusepath{clip}%
\pgfsetbuttcap%
\pgfsetroundjoin%
\definecolor{currentfill}{rgb}{0.369214,0.788888,0.382914}%
\pgfsetfillcolor{currentfill}%
\pgfsetlinewidth{0.000000pt}%
\definecolor{currentstroke}{rgb}{0.169646,0.456262,0.558030}%
\pgfsetstrokecolor{currentstroke}%
\pgfsetdash{}{0pt}%
\pgfpathmoveto{\pgfqpoint{3.808702in}{5.299454in}}%
\pgfpathlineto{\pgfqpoint{3.892829in}{5.198517in}}%
\pgfpathlineto{\pgfqpoint{3.754366in}{5.305678in}}%
\pgfpathclose%
\pgfusepath{fill}%
\end{pgfscope}%
\begin{pgfscope}%
\pgfpathrectangle{\pgfqpoint{0.539299in}{0.078740in}}{\pgfqpoint{7.842520in}{7.842520in}}%
\pgfusepath{clip}%
\pgfsetbuttcap%
\pgfsetroundjoin%
\definecolor{currentfill}{rgb}{0.282327,0.094955,0.417331}%
\pgfsetfillcolor{currentfill}%
\pgfsetlinewidth{0.000000pt}%
\definecolor{currentstroke}{rgb}{0.168126,0.459988,0.558082}%
\pgfsetstrokecolor{currentstroke}%
\pgfsetdash{}{0pt}%
\pgfpathmoveto{\pgfqpoint{6.756390in}{2.344144in}}%
\pgfpathlineto{\pgfqpoint{6.615697in}{2.452498in}}%
\pgfpathlineto{\pgfqpoint{6.540856in}{2.372768in}}%
\pgfpathclose%
\pgfusepath{fill}%
\end{pgfscope}%
\begin{pgfscope}%
\pgfpathrectangle{\pgfqpoint{0.539299in}{0.078740in}}{\pgfqpoint{7.842520in}{7.842520in}}%
\pgfusepath{clip}%
\pgfsetbuttcap%
\pgfsetroundjoin%
\definecolor{currentfill}{rgb}{0.275191,0.194905,0.496005}%
\pgfsetfillcolor{currentfill}%
\pgfsetlinewidth{0.000000pt}%
\definecolor{currentstroke}{rgb}{0.166617,0.463708,0.558119}%
\pgfsetstrokecolor{currentstroke}%
\pgfsetdash{}{0pt}%
\pgfpathmoveto{\pgfqpoint{6.043073in}{2.634987in}}%
\pgfpathlineto{\pgfqpoint{6.118863in}{2.691325in}}%
\pgfpathlineto{\pgfqpoint{5.978689in}{2.800383in}}%
\pgfpathclose%
\pgfusepath{fill}%
\end{pgfscope}%
\begin{pgfscope}%
\pgfpathrectangle{\pgfqpoint{0.539299in}{0.078740in}}{\pgfqpoint{7.842520in}{7.842520in}}%
\pgfusepath{clip}%
\pgfsetbuttcap%
\pgfsetroundjoin%
\definecolor{currentfill}{rgb}{0.255645,0.260703,0.528312}%
\pgfsetfillcolor{currentfill}%
\pgfsetlinewidth{0.000000pt}%
\definecolor{currentstroke}{rgb}{0.165117,0.467423,0.558141}%
\pgfsetstrokecolor{currentstroke}%
\pgfsetdash{}{0pt}%
\pgfpathmoveto{\pgfqpoint{5.622835in}{3.018878in}}%
\pgfpathlineto{\pgfqpoint{5.762666in}{2.883788in}}%
\pgfpathlineto{\pgfqpoint{5.838796in}{2.913871in}}%
\pgfpathclose%
\pgfusepath{fill}%
\end{pgfscope}%
\begin{pgfscope}%
\pgfpathrectangle{\pgfqpoint{0.539299in}{0.078740in}}{\pgfqpoint{7.842520in}{7.842520in}}%
\pgfusepath{clip}%
\pgfsetbuttcap%
\pgfsetroundjoin%
\definecolor{currentfill}{rgb}{0.133743,0.548535,0.553541}%
\pgfsetfillcolor{currentfill}%
\pgfsetlinewidth{0.000000pt}%
\definecolor{currentstroke}{rgb}{0.163625,0.471133,0.558148}%
\pgfsetstrokecolor{currentstroke}%
\pgfsetdash{}{0pt}%
\pgfpathmoveto{\pgfqpoint{4.647114in}{4.229077in}}%
\pgfpathlineto{\pgfqpoint{4.786562in}{4.030722in}}%
\pgfpathlineto{\pgfqpoint{4.726876in}{4.135706in}}%
\pgfpathclose%
\pgfusepath{fill}%
\end{pgfscope}%
\begin{pgfscope}%
\pgfpathrectangle{\pgfqpoint{0.539299in}{0.078740in}}{\pgfqpoint{7.842520in}{7.842520in}}%
\pgfusepath{clip}%
\pgfsetbuttcap%
\pgfsetroundjoin%
\definecolor{currentfill}{rgb}{0.196571,0.711827,0.479221}%
\pgfsetfillcolor{currentfill}%
\pgfsetlinewidth{0.000000pt}%
\definecolor{currentstroke}{rgb}{0.162142,0.474838,0.558140}%
\pgfsetstrokecolor{currentstroke}%
\pgfsetdash{}{0pt}%
\pgfpathmoveto{\pgfqpoint{4.309978in}{4.711999in}}%
\pgfpathlineto{\pgfqpoint{4.170832in}{4.893238in}}%
\pgfpathlineto{\pgfqpoint{4.088028in}{5.004974in}}%
\pgfpathclose%
\pgfusepath{fill}%
\end{pgfscope}%
\begin{pgfscope}%
\pgfpathrectangle{\pgfqpoint{0.539299in}{0.078740in}}{\pgfqpoint{7.842520in}{7.842520in}}%
\pgfusepath{clip}%
\pgfsetbuttcap%
\pgfsetroundjoin%
\definecolor{currentfill}{rgb}{0.311925,0.767822,0.415586}%
\pgfsetfillcolor{currentfill}%
\pgfsetlinewidth{0.000000pt}%
\definecolor{currentstroke}{rgb}{0.160665,0.478540,0.558115}%
\pgfsetstrokecolor{currentstroke}%
\pgfsetdash{}{0pt}%
\pgfpathmoveto{\pgfqpoint{4.031728in}{5.058097in}}%
\pgfpathlineto{\pgfqpoint{3.892829in}{5.198517in}}%
\pgfpathlineto{\pgfqpoint{3.948223in}{5.166109in}}%
\pgfpathclose%
\pgfusepath{fill}%
\end{pgfscope}%
\begin{pgfscope}%
\pgfpathrectangle{\pgfqpoint{0.539299in}{0.078740in}}{\pgfqpoint{7.842520in}{7.842520in}}%
\pgfusepath{clip}%
\pgfsetbuttcap%
\pgfsetroundjoin%
\definecolor{currentfill}{rgb}{0.146180,0.515413,0.556823}%
\pgfsetfillcolor{currentfill}%
\pgfsetlinewidth{0.000000pt}%
\definecolor{currentstroke}{rgb}{0.159194,0.482237,0.558073}%
\pgfsetstrokecolor{currentstroke}%
\pgfsetdash{}{0pt}%
\pgfpathmoveto{\pgfqpoint{2.181328in}{4.304148in}}%
\pgfpathlineto{\pgfqpoint{2.062008in}{3.862537in}}%
\pgfpathlineto{\pgfqpoint{1.975740in}{3.757494in}}%
\pgfpathclose%
\pgfusepath{fill}%
\end{pgfscope}%
\begin{pgfscope}%
\pgfpathrectangle{\pgfqpoint{0.539299in}{0.078740in}}{\pgfqpoint{7.842520in}{7.842520in}}%
\pgfusepath{clip}%
\pgfsetbuttcap%
\pgfsetroundjoin%
\definecolor{currentfill}{rgb}{0.283091,0.110553,0.431554}%
\pgfsetfillcolor{currentfill}%
\pgfsetlinewidth{0.000000pt}%
\definecolor{currentstroke}{rgb}{0.157729,0.485932,0.558013}%
\pgfsetstrokecolor{currentstroke}%
\pgfsetdash{}{0pt}%
\pgfpathmoveto{\pgfqpoint{6.540856in}{2.372768in}}%
\pgfpathlineto{\pgfqpoint{6.615697in}{2.452498in}}%
\pgfpathlineto{\pgfqpoint{6.399983in}{2.479315in}}%
\pgfpathclose%
\pgfusepath{fill}%
\end{pgfscope}%
\begin{pgfscope}%
\pgfpathrectangle{\pgfqpoint{0.539299in}{0.078740in}}{\pgfqpoint{7.842520in}{7.842520in}}%
\pgfusepath{clip}%
\pgfsetbuttcap%
\pgfsetroundjoin%
\definecolor{currentfill}{rgb}{0.266941,0.748751,0.440573}%
\pgfsetfillcolor{currentfill}%
\pgfsetlinewidth{0.000000pt}%
\definecolor{currentstroke}{rgb}{0.156270,0.489624,0.557936}%
\pgfsetstrokecolor{currentstroke}%
\pgfsetdash{}{0pt}%
\pgfpathmoveto{\pgfqpoint{4.031728in}{5.058097in}}%
\pgfpathlineto{\pgfqpoint{3.948223in}{5.166109in}}%
\pgfpathlineto{\pgfqpoint{4.170832in}{4.893238in}}%
\pgfpathclose%
\pgfusepath{fill}%
\end{pgfscope}%
\begin{pgfscope}%
\pgfpathrectangle{\pgfqpoint{0.539299in}{0.078740in}}{\pgfqpoint{7.842520in}{7.842520in}}%
\pgfusepath{clip}%
\pgfsetbuttcap%
\pgfsetroundjoin%
\definecolor{currentfill}{rgb}{0.440137,0.811138,0.340967}%
\pgfsetfillcolor{currentfill}%
\pgfsetlinewidth{0.000000pt}%
\definecolor{currentstroke}{rgb}{0.154815,0.493313,0.557840}%
\pgfsetstrokecolor{currentstroke}%
\pgfsetdash{}{0pt}%
\pgfpathmoveto{\pgfqpoint{3.754366in}{5.305678in}}%
\pgfpathlineto{\pgfqpoint{3.531606in}{5.447979in}}%
\pgfpathlineto{\pgfqpoint{3.669721in}{5.396407in}}%
\pgfpathclose%
\pgfusepath{fill}%
\end{pgfscope}%
\begin{pgfscope}%
\pgfpathrectangle{\pgfqpoint{0.539299in}{0.078740in}}{\pgfqpoint{7.842520in}{7.842520in}}%
\pgfusepath{clip}%
\pgfsetbuttcap%
\pgfsetroundjoin%
\definecolor{currentfill}{rgb}{0.266580,0.228262,0.514349}%
\pgfsetfillcolor{currentfill}%
\pgfsetlinewidth{0.000000pt}%
\definecolor{currentstroke}{rgb}{0.153364,0.497000,0.557724}%
\pgfsetstrokecolor{currentstroke}%
\pgfsetdash{}{0pt}%
\pgfpathmoveto{\pgfqpoint{5.762666in}{2.883788in}}%
\pgfpathlineto{\pgfqpoint{5.902739in}{2.756289in}}%
\pgfpathlineto{\pgfqpoint{5.978689in}{2.800383in}}%
\pgfpathclose%
\pgfusepath{fill}%
\end{pgfscope}%
\begin{pgfscope}%
\pgfpathrectangle{\pgfqpoint{0.539299in}{0.078740in}}{\pgfqpoint{7.842520in}{7.842520in}}%
\pgfusepath{clip}%
\pgfsetbuttcap%
\pgfsetroundjoin%
\definecolor{currentfill}{rgb}{0.123463,0.581687,0.547445}%
\pgfsetfillcolor{currentfill}%
\pgfsetlinewidth{0.000000pt}%
\definecolor{currentstroke}{rgb}{0.151918,0.500685,0.557587}%
\pgfsetstrokecolor{currentstroke}%
\pgfsetdash{}{0pt}%
\pgfpathmoveto{\pgfqpoint{4.726876in}{4.135706in}}%
\pgfpathlineto{\pgfqpoint{4.507522in}{4.430367in}}%
\pgfpathlineto{\pgfqpoint{4.647114in}{4.229077in}}%
\pgfpathclose%
\pgfusepath{fill}%
\end{pgfscope}%
\begin{pgfscope}%
\pgfpathrectangle{\pgfqpoint{0.539299in}{0.078740in}}{\pgfqpoint{7.842520in}{7.842520in}}%
\pgfusepath{clip}%
\pgfsetbuttcap%
\pgfsetroundjoin%
\definecolor{currentfill}{rgb}{0.126453,0.570633,0.549841}%
\pgfsetfillcolor{currentfill}%
\pgfsetlinewidth{0.000000pt}%
\definecolor{currentstroke}{rgb}{0.150476,0.504369,0.557430}%
\pgfsetstrokecolor{currentstroke}%
\pgfsetdash{}{0pt}%
\pgfpathmoveto{\pgfqpoint{2.267756in}{4.403873in}}%
\pgfpathlineto{\pgfqpoint{2.148306in}{3.953837in}}%
\pgfpathlineto{\pgfqpoint{2.181328in}{4.304148in}}%
\pgfpathclose%
\pgfusepath{fill}%
\end{pgfscope}%
\begin{pgfscope}%
\pgfpathrectangle{\pgfqpoint{0.539299in}{0.078740in}}{\pgfqpoint{7.842520in}{7.842520in}}%
\pgfusepath{clip}%
\pgfsetbuttcap%
\pgfsetroundjoin%
\definecolor{currentfill}{rgb}{0.278791,0.062145,0.386592}%
\pgfsetfillcolor{currentfill}%
\pgfsetlinewidth{0.000000pt}%
\definecolor{currentstroke}{rgb}{0.149039,0.508051,0.557250}%
\pgfsetstrokecolor{currentstroke}%
\pgfsetdash{}{0pt}%
\pgfpathmoveto{\pgfqpoint{6.897057in}{2.228556in}}%
\pgfpathlineto{\pgfqpoint{6.756390in}{2.344144in}}%
\pgfpathlineto{\pgfqpoint{6.681865in}{2.263405in}}%
\pgfpathclose%
\pgfusepath{fill}%
\end{pgfscope}%
\begin{pgfscope}%
\pgfpathrectangle{\pgfqpoint{0.539299in}{0.078740in}}{\pgfqpoint{7.842520in}{7.842520in}}%
\pgfusepath{clip}%
\pgfsetbuttcap%
\pgfsetroundjoin%
\definecolor{currentfill}{rgb}{0.288921,0.758394,0.428426}%
\pgfsetfillcolor{currentfill}%
\pgfsetlinewidth{0.000000pt}%
\definecolor{currentstroke}{rgb}{0.147607,0.511733,0.557049}%
\pgfsetstrokecolor{currentstroke}%
\pgfsetdash{}{0pt}%
\pgfpathmoveto{\pgfqpoint{2.867996in}{5.262317in}}%
\pgfpathlineto{\pgfqpoint{2.738735in}{4.997476in}}%
\pgfpathlineto{\pgfqpoint{2.652170in}{4.965747in}}%
\pgfpathclose%
\pgfusepath{fill}%
\end{pgfscope}%
\begin{pgfscope}%
\pgfpathrectangle{\pgfqpoint{0.539299in}{0.078740in}}{\pgfqpoint{7.842520in}{7.842520in}}%
\pgfusepath{clip}%
\pgfsetbuttcap%
\pgfsetroundjoin%
\definecolor{currentfill}{rgb}{0.252194,0.269783,0.531579}%
\pgfsetfillcolor{currentfill}%
\pgfsetlinewidth{0.000000pt}%
\definecolor{currentstroke}{rgb}{0.146180,0.515413,0.556823}%
\pgfsetstrokecolor{currentstroke}%
\pgfsetdash{}{0pt}%
\pgfpathmoveto{\pgfqpoint{1.633975in}{3.148721in}}%
\pgfpathlineto{\pgfqpoint{1.601948in}{2.927986in}}%
\pgfpathlineto{\pgfqpoint{1.515589in}{2.796677in}}%
\pgfpathclose%
\pgfusepath{fill}%
\end{pgfscope}%
\begin{pgfscope}%
\pgfpathrectangle{\pgfqpoint{0.539299in}{0.078740in}}{\pgfqpoint{7.842520in}{7.842520in}}%
\pgfusepath{clip}%
\pgfsetbuttcap%
\pgfsetroundjoin%
\definecolor{currentfill}{rgb}{0.185783,0.704891,0.485273}%
\pgfsetfillcolor{currentfill}%
\pgfsetlinewidth{0.000000pt}%
\definecolor{currentstroke}{rgb}{0.144759,0.519093,0.556572}%
\pgfsetstrokecolor{currentstroke}%
\pgfsetdash{}{0pt}%
\pgfpathmoveto{\pgfqpoint{2.440760in}{4.553021in}}%
\pgfpathlineto{\pgfqpoint{2.565417in}{4.919366in}}%
\pgfpathlineto{\pgfqpoint{2.652170in}{4.965747in}}%
\pgfpathclose%
\pgfusepath{fill}%
\end{pgfscope}%
\begin{pgfscope}%
\pgfpathrectangle{\pgfqpoint{0.539299in}{0.078740in}}{\pgfqpoint{7.842520in}{7.842520in}}%
\pgfusepath{clip}%
\pgfsetbuttcap%
\pgfsetroundjoin%
\definecolor{currentfill}{rgb}{0.280255,0.165693,0.476498}%
\pgfsetfillcolor{currentfill}%
\pgfsetlinewidth{0.000000pt}%
\definecolor{currentstroke}{rgb}{0.143343,0.522773,0.556295}%
\pgfsetstrokecolor{currentstroke}%
\pgfsetdash{}{0pt}%
\pgfpathmoveto{\pgfqpoint{6.118863in}{2.691325in}}%
\pgfpathlineto{\pgfqpoint{6.183670in}{2.518396in}}%
\pgfpathlineto{\pgfqpoint{6.259304in}{2.584902in}}%
\pgfpathclose%
\pgfusepath{fill}%
\end{pgfscope}%
\begin{pgfscope}%
\pgfpathrectangle{\pgfqpoint{0.539299in}{0.078740in}}{\pgfqpoint{7.842520in}{7.842520in}}%
\pgfusepath{clip}%
\pgfsetbuttcap%
\pgfsetroundjoin%
\definecolor{currentfill}{rgb}{0.273006,0.204520,0.501721}%
\pgfsetfillcolor{currentfill}%
\pgfsetlinewidth{0.000000pt}%
\definecolor{currentstroke}{rgb}{0.141935,0.526453,0.555991}%
\pgfsetstrokecolor{currentstroke}%
\pgfsetdash{}{0pt}%
\pgfpathmoveto{\pgfqpoint{5.978689in}{2.800383in}}%
\pgfpathlineto{\pgfqpoint{5.902739in}{2.756289in}}%
\pgfpathlineto{\pgfqpoint{6.043073in}{2.634987in}}%
\pgfpathclose%
\pgfusepath{fill}%
\end{pgfscope}%
\begin{pgfscope}%
\pgfpathrectangle{\pgfqpoint{0.539299in}{0.078740in}}{\pgfqpoint{7.842520in}{7.842520in}}%
\pgfusepath{clip}%
\pgfsetbuttcap%
\pgfsetroundjoin%
\definecolor{currentfill}{rgb}{0.281446,0.084320,0.407414}%
\pgfsetfillcolor{currentfill}%
\pgfsetlinewidth{0.000000pt}%
\definecolor{currentstroke}{rgb}{0.140536,0.530132,0.555659}%
\pgfsetstrokecolor{currentstroke}%
\pgfsetdash{}{0pt}%
\pgfpathmoveto{\pgfqpoint{6.540856in}{2.372768in}}%
\pgfpathlineto{\pgfqpoint{6.681865in}{2.263405in}}%
\pgfpathlineto{\pgfqpoint{6.756390in}{2.344144in}}%
\pgfpathclose%
\pgfusepath{fill}%
\end{pgfscope}%
\begin{pgfscope}%
\pgfpathrectangle{\pgfqpoint{0.539299in}{0.078740in}}{\pgfqpoint{7.842520in}{7.842520in}}%
\pgfusepath{clip}%
\pgfsetbuttcap%
\pgfsetroundjoin%
\definecolor{currentfill}{rgb}{0.121380,0.629492,0.531973}%
\pgfsetfillcolor{currentfill}%
\pgfsetlinewidth{0.000000pt}%
\definecolor{currentstroke}{rgb}{0.139147,0.533812,0.555298}%
\pgfsetstrokecolor{currentstroke}%
\pgfsetdash{}{0pt}%
\pgfpathmoveto{\pgfqpoint{4.588029in}{4.327753in}}%
\pgfpathlineto{\pgfqpoint{4.367781in}{4.630418in}}%
\pgfpathlineto{\pgfqpoint{4.507522in}{4.430367in}}%
\pgfpathclose%
\pgfusepath{fill}%
\end{pgfscope}%
\begin{pgfscope}%
\pgfpathrectangle{\pgfqpoint{0.539299in}{0.078740in}}{\pgfqpoint{7.842520in}{7.842520in}}%
\pgfusepath{clip}%
\pgfsetbuttcap%
\pgfsetroundjoin%
\definecolor{currentfill}{rgb}{0.496615,0.826376,0.306377}%
\pgfsetfillcolor{currentfill}%
\pgfsetlinewidth{0.000000pt}%
\definecolor{currentstroke}{rgb}{0.137770,0.537492,0.554906}%
\pgfsetstrokecolor{currentstroke}%
\pgfsetdash{}{0pt}%
\pgfpathmoveto{\pgfqpoint{3.531606in}{5.447979in}}%
\pgfpathlineto{\pgfqpoint{3.394745in}{5.445102in}}%
\pgfpathlineto{\pgfqpoint{3.308861in}{5.497668in}}%
\pgfpathclose%
\pgfusepath{fill}%
\end{pgfscope}%
\begin{pgfscope}%
\pgfpathrectangle{\pgfqpoint{0.539299in}{0.078740in}}{\pgfqpoint{7.842520in}{7.842520in}}%
\pgfusepath{clip}%
\pgfsetbuttcap%
\pgfsetroundjoin%
\definecolor{currentfill}{rgb}{0.282290,0.145912,0.461510}%
\pgfsetfillcolor{currentfill}%
\pgfsetlinewidth{0.000000pt}%
\definecolor{currentstroke}{rgb}{0.136408,0.541173,0.554483}%
\pgfsetstrokecolor{currentstroke}%
\pgfsetdash{}{0pt}%
\pgfpathmoveto{\pgfqpoint{6.259304in}{2.584902in}}%
\pgfpathlineto{\pgfqpoint{6.183670in}{2.518396in}}%
\pgfpathlineto{\pgfqpoint{6.399983in}{2.479315in}}%
\pgfpathclose%
\pgfusepath{fill}%
\end{pgfscope}%
\begin{pgfscope}%
\pgfpathrectangle{\pgfqpoint{0.539299in}{0.078740in}}{\pgfqpoint{7.842520in}{7.842520in}}%
\pgfusepath{clip}%
\pgfsetbuttcap%
\pgfsetroundjoin%
\definecolor{currentfill}{rgb}{0.421908,0.805774,0.351910}%
\pgfsetfillcolor{currentfill}%
\pgfsetlinewidth{0.000000pt}%
\definecolor{currentstroke}{rgb}{0.135066,0.544853,0.554029}%
\pgfsetstrokecolor{currentstroke}%
\pgfsetdash{}{0pt}%
\pgfpathmoveto{\pgfqpoint{3.754366in}{5.305678in}}%
\pgfpathlineto{\pgfqpoint{3.669721in}{5.396407in}}%
\pgfpathlineto{\pgfqpoint{3.808702in}{5.299454in}}%
\pgfpathclose%
\pgfusepath{fill}%
\end{pgfscope}%
\begin{pgfscope}%
\pgfpathrectangle{\pgfqpoint{0.539299in}{0.078740in}}{\pgfqpoint{7.842520in}{7.842520in}}%
\pgfusepath{clip}%
\pgfsetbuttcap%
\pgfsetroundjoin%
\definecolor{currentfill}{rgb}{0.140210,0.665859,0.513427}%
\pgfsetfillcolor{currentfill}%
\pgfsetlinewidth{0.000000pt}%
\definecolor{currentstroke}{rgb}{0.133743,0.548535,0.553541}%
\pgfsetstrokecolor{currentstroke}%
\pgfsetdash{}{0pt}%
\pgfpathmoveto{\pgfqpoint{2.440760in}{4.553021in}}%
\pgfpathlineto{\pgfqpoint{2.354275in}{4.486045in}}%
\pgfpathlineto{\pgfqpoint{2.478591in}{4.855876in}}%
\pgfpathclose%
\pgfusepath{fill}%
\end{pgfscope}%
\begin{pgfscope}%
\pgfpathrectangle{\pgfqpoint{0.539299in}{0.078740in}}{\pgfqpoint{7.842520in}{7.842520in}}%
\pgfusepath{clip}%
\pgfsetbuttcap%
\pgfsetroundjoin%
\definecolor{currentfill}{rgb}{0.449368,0.813768,0.335384}%
\pgfsetfillcolor{currentfill}%
\pgfsetlinewidth{0.000000pt}%
\definecolor{currentstroke}{rgb}{0.132444,0.552216,0.553018}%
\pgfsetstrokecolor{currentstroke}%
\pgfsetdash{}{0pt}%
\pgfpathmoveto{\pgfqpoint{2.954612in}{5.267216in}}%
\pgfpathlineto{\pgfqpoint{3.087266in}{5.438839in}}%
\pgfpathlineto{\pgfqpoint{3.173672in}{5.414731in}}%
\pgfpathclose%
\pgfusepath{fill}%
\end{pgfscope}%
\begin{pgfscope}%
\pgfpathrectangle{\pgfqpoint{0.539299in}{0.078740in}}{\pgfqpoint{7.842520in}{7.842520in}}%
\pgfusepath{clip}%
\pgfsetbuttcap%
\pgfsetroundjoin%
\definecolor{currentfill}{rgb}{0.278826,0.175490,0.483397}%
\pgfsetfillcolor{currentfill}%
\pgfsetlinewidth{0.000000pt}%
\definecolor{currentstroke}{rgb}{0.131172,0.555899,0.552459}%
\pgfsetstrokecolor{currentstroke}%
\pgfsetdash{}{0pt}%
\pgfpathmoveto{\pgfqpoint{6.043073in}{2.634987in}}%
\pgfpathlineto{\pgfqpoint{6.183670in}{2.518396in}}%
\pgfpathlineto{\pgfqpoint{6.118863in}{2.691325in}}%
\pgfpathclose%
\pgfusepath{fill}%
\end{pgfscope}%
\begin{pgfscope}%
\pgfpathrectangle{\pgfqpoint{0.539299in}{0.078740in}}{\pgfqpoint{7.842520in}{7.842520in}}%
\pgfusepath{clip}%
\pgfsetbuttcap%
\pgfsetroundjoin%
\definecolor{currentfill}{rgb}{0.214298,0.355619,0.551184}%
\pgfsetfillcolor{currentfill}%
\pgfsetlinewidth{0.000000pt}%
\definecolor{currentstroke}{rgb}{0.129933,0.559582,0.551864}%
\pgfsetstrokecolor{currentstroke}%
\pgfsetdash{}{0pt}%
\pgfpathmoveto{\pgfqpoint{1.601948in}{2.927986in}}%
\pgfpathlineto{\pgfqpoint{1.718526in}{3.335879in}}%
\pgfpathlineto{\pgfqpoint{1.803826in}{3.497087in}}%
\pgfpathclose%
\pgfusepath{fill}%
\end{pgfscope}%
\begin{pgfscope}%
\pgfpathrectangle{\pgfqpoint{0.539299in}{0.078740in}}{\pgfqpoint{7.842520in}{7.842520in}}%
\pgfusepath{clip}%
\pgfsetbuttcap%
\pgfsetroundjoin%
\definecolor{currentfill}{rgb}{0.162016,0.687316,0.499129}%
\pgfsetfillcolor{currentfill}%
\pgfsetlinewidth{0.000000pt}%
\definecolor{currentstroke}{rgb}{0.128729,0.563265,0.551229}%
\pgfsetstrokecolor{currentstroke}%
\pgfsetdash{}{0pt}%
\pgfpathmoveto{\pgfqpoint{4.227924in}{4.824025in}}%
\pgfpathlineto{\pgfqpoint{4.367781in}{4.630418in}}%
\pgfpathlineto{\pgfqpoint{4.309978in}{4.711999in}}%
\pgfpathclose%
\pgfusepath{fill}%
\end{pgfscope}%
\begin{pgfscope}%
\pgfpathrectangle{\pgfqpoint{0.539299in}{0.078740in}}{\pgfqpoint{7.842520in}{7.842520in}}%
\pgfusepath{clip}%
\pgfsetbuttcap%
\pgfsetroundjoin%
\definecolor{currentfill}{rgb}{0.277018,0.050344,0.375715}%
\pgfsetfillcolor{currentfill}%
\pgfsetlinewidth{0.000000pt}%
\definecolor{currentstroke}{rgb}{0.127568,0.566949,0.550556}%
\pgfsetstrokecolor{currentstroke}%
\pgfsetdash{}{0pt}%
\pgfpathmoveto{\pgfqpoint{6.681865in}{2.263405in}}%
\pgfpathlineto{\pgfqpoint{6.822930in}{2.149191in}}%
\pgfpathlineto{\pgfqpoint{6.897057in}{2.228556in}}%
\pgfpathclose%
\pgfusepath{fill}%
\end{pgfscope}%
\begin{pgfscope}%
\pgfpathrectangle{\pgfqpoint{0.539299in}{0.078740in}}{\pgfqpoint{7.842520in}{7.842520in}}%
\pgfusepath{clip}%
\pgfsetbuttcap%
\pgfsetroundjoin%
\definecolor{currentfill}{rgb}{0.369214,0.788888,0.382914}%
\pgfsetfillcolor{currentfill}%
\pgfsetlinewidth{0.000000pt}%
\definecolor{currentstroke}{rgb}{0.126453,0.570633,0.549841}%
\pgfsetstrokecolor{currentstroke}%
\pgfsetdash{}{0pt}%
\pgfpathmoveto{\pgfqpoint{3.948223in}{5.166109in}}%
\pgfpathlineto{\pgfqpoint{3.892829in}{5.198517in}}%
\pgfpathlineto{\pgfqpoint{3.808702in}{5.299454in}}%
\pgfpathclose%
\pgfusepath{fill}%
\end{pgfscope}%
\begin{pgfscope}%
\pgfpathrectangle{\pgfqpoint{0.539299in}{0.078740in}}{\pgfqpoint{7.842520in}{7.842520in}}%
\pgfusepath{clip}%
\pgfsetbuttcap%
\pgfsetroundjoin%
\definecolor{currentfill}{rgb}{0.202219,0.715272,0.476084}%
\pgfsetfillcolor{currentfill}%
\pgfsetlinewidth{0.000000pt}%
\definecolor{currentstroke}{rgb}{0.125394,0.574318,0.549086}%
\pgfsetstrokecolor{currentstroke}%
\pgfsetdash{}{0pt}%
\pgfpathmoveto{\pgfqpoint{4.309978in}{4.711999in}}%
\pgfpathlineto{\pgfqpoint{4.088028in}{5.004974in}}%
\pgfpathlineto{\pgfqpoint{4.227924in}{4.824025in}}%
\pgfpathclose%
\pgfusepath{fill}%
\end{pgfscope}%
\begin{pgfscope}%
\pgfpathrectangle{\pgfqpoint{0.539299in}{0.078740in}}{\pgfqpoint{7.842520in}{7.842520in}}%
\pgfusepath{clip}%
\pgfsetbuttcap%
\pgfsetroundjoin%
\definecolor{currentfill}{rgb}{0.258965,0.251537,0.524736}%
\pgfsetfillcolor{currentfill}%
\pgfsetlinewidth{0.000000pt}%
\definecolor{currentstroke}{rgb}{0.124395,0.578002,0.548287}%
\pgfsetstrokecolor{currentstroke}%
\pgfsetdash{}{0pt}%
\pgfpathmoveto{\pgfqpoint{1.515589in}{2.796677in}}%
\pgfpathlineto{\pgfqpoint{1.429595in}{2.648382in}}%
\pgfpathlineto{\pgfqpoint{1.633975in}{3.148721in}}%
\pgfpathclose%
\pgfusepath{fill}%
\end{pgfscope}%
\begin{pgfscope}%
\pgfpathrectangle{\pgfqpoint{0.539299in}{0.078740in}}{\pgfqpoint{7.842520in}{7.842520in}}%
\pgfusepath{clip}%
\pgfsetbuttcap%
\pgfsetroundjoin%
\definecolor{currentfill}{rgb}{0.199430,0.387607,0.554642}%
\pgfsetfillcolor{currentfill}%
\pgfsetlinewidth{0.000000pt}%
\definecolor{currentstroke}{rgb}{0.123463,0.581687,0.547445}%
\pgfsetstrokecolor{currentstroke}%
\pgfsetdash{}{0pt}%
\pgfpathmoveto{\pgfqpoint{5.343771in}{3.316408in}}%
\pgfpathlineto{\pgfqpoint{5.204442in}{3.480326in}}%
\pgfpathlineto{\pgfqpoint{5.266300in}{3.347822in}}%
\pgfpathclose%
\pgfusepath{fill}%
\end{pgfscope}%
\begin{pgfscope}%
\pgfpathrectangle{\pgfqpoint{0.539299in}{0.078740in}}{\pgfqpoint{7.842520in}{7.842520in}}%
\pgfusepath{clip}%
\pgfsetbuttcap%
\pgfsetroundjoin%
\definecolor{currentfill}{rgb}{0.210503,0.363727,0.552206}%
\pgfsetfillcolor{currentfill}%
\pgfsetlinewidth{0.000000pt}%
\definecolor{currentstroke}{rgb}{0.122606,0.585371,0.546557}%
\pgfsetstrokecolor{currentstroke}%
\pgfsetdash{}{0pt}%
\pgfpathmoveto{\pgfqpoint{5.483218in}{3.162772in}}%
\pgfpathlineto{\pgfqpoint{5.343771in}{3.316408in}}%
\pgfpathlineto{\pgfqpoint{5.266300in}{3.347822in}}%
\pgfpathclose%
\pgfusepath{fill}%
\end{pgfscope}%
\begin{pgfscope}%
\pgfpathrectangle{\pgfqpoint{0.539299in}{0.078740in}}{\pgfqpoint{7.842520in}{7.842520in}}%
\pgfusepath{clip}%
\pgfsetbuttcap%
\pgfsetroundjoin%
\definecolor{currentfill}{rgb}{0.274149,0.751988,0.436601}%
\pgfsetfillcolor{currentfill}%
\pgfsetlinewidth{0.000000pt}%
\definecolor{currentstroke}{rgb}{0.121831,0.589055,0.545623}%
\pgfsetstrokecolor{currentstroke}%
\pgfsetdash{}{0pt}%
\pgfpathmoveto{\pgfqpoint{4.170832in}{4.893238in}}%
\pgfpathlineto{\pgfqpoint{3.948223in}{5.166109in}}%
\pgfpathlineto{\pgfqpoint{4.088028in}{5.004974in}}%
\pgfpathclose%
\pgfusepath{fill}%
\end{pgfscope}%
\begin{pgfscope}%
\pgfpathrectangle{\pgfqpoint{0.539299in}{0.078740in}}{\pgfqpoint{7.842520in}{7.842520in}}%
\pgfusepath{clip}%
\pgfsetbuttcap%
\pgfsetroundjoin%
\definecolor{currentfill}{rgb}{0.229739,0.322361,0.545706}%
\pgfsetfillcolor{currentfill}%
\pgfsetlinewidth{0.000000pt}%
\definecolor{currentstroke}{rgb}{0.121148,0.592739,0.544641}%
\pgfsetstrokecolor{currentstroke}%
\pgfsetdash{}{0pt}%
\pgfpathmoveto{\pgfqpoint{5.483218in}{3.162772in}}%
\pgfpathlineto{\pgfqpoint{5.406152in}{3.177575in}}%
\pgfpathlineto{\pgfqpoint{5.622835in}{3.018878in}}%
\pgfpathclose%
\pgfusepath{fill}%
\end{pgfscope}%
\begin{pgfscope}%
\pgfpathrectangle{\pgfqpoint{0.539299in}{0.078740in}}{\pgfqpoint{7.842520in}{7.842520in}}%
\pgfusepath{clip}%
\pgfsetbuttcap%
\pgfsetroundjoin%
\definecolor{currentfill}{rgb}{0.283091,0.110553,0.431554}%
\pgfsetfillcolor{currentfill}%
\pgfsetlinewidth{0.000000pt}%
\definecolor{currentstroke}{rgb}{0.120565,0.596422,0.543611}%
\pgfsetstrokecolor{currentstroke}%
\pgfsetdash{}{0pt}%
\pgfpathmoveto{\pgfqpoint{6.540856in}{2.372768in}}%
\pgfpathlineto{\pgfqpoint{6.399983in}{2.479315in}}%
\pgfpathlineto{\pgfqpoint{6.465616in}{2.293258in}}%
\pgfpathclose%
\pgfusepath{fill}%
\end{pgfscope}%
\begin{pgfscope}%
\pgfpathrectangle{\pgfqpoint{0.539299in}{0.078740in}}{\pgfqpoint{7.842520in}{7.842520in}}%
\pgfusepath{clip}%
\pgfsetbuttcap%
\pgfsetroundjoin%
\definecolor{currentfill}{rgb}{0.180629,0.429975,0.557282}%
\pgfsetfillcolor{currentfill}%
\pgfsetlinewidth{0.000000pt}%
\definecolor{currentstroke}{rgb}{0.120092,0.600104,0.542530}%
\pgfsetstrokecolor{currentstroke}%
\pgfsetdash{}{0pt}%
\pgfpathmoveto{\pgfqpoint{5.126490in}{3.527887in}}%
\pgfpathlineto{\pgfqpoint{5.204442in}{3.480326in}}%
\pgfpathlineto{\pgfqpoint{5.065172in}{3.654551in}}%
\pgfpathclose%
\pgfusepath{fill}%
\end{pgfscope}%
\begin{pgfscope}%
\pgfpathrectangle{\pgfqpoint{0.539299in}{0.078740in}}{\pgfqpoint{7.842520in}{7.842520in}}%
\pgfusepath{clip}%
\pgfsetbuttcap%
\pgfsetroundjoin%
\definecolor{currentfill}{rgb}{0.252194,0.269783,0.531579}%
\pgfsetfillcolor{currentfill}%
\pgfsetlinewidth{0.000000pt}%
\definecolor{currentstroke}{rgb}{0.119738,0.603785,0.541400}%
\pgfsetstrokecolor{currentstroke}%
\pgfsetdash{}{0pt}%
\pgfpathmoveto{\pgfqpoint{5.686202in}{2.865893in}}%
\pgfpathlineto{\pgfqpoint{5.762666in}{2.883788in}}%
\pgfpathlineto{\pgfqpoint{5.622835in}{3.018878in}}%
\pgfpathclose%
\pgfusepath{fill}%
\end{pgfscope}%
\begin{pgfscope}%
\pgfpathrectangle{\pgfqpoint{0.539299in}{0.078740in}}{\pgfqpoint{7.842520in}{7.842520in}}%
\pgfusepath{clip}%
\pgfsetbuttcap%
\pgfsetroundjoin%
\definecolor{currentfill}{rgb}{0.282623,0.140926,0.457517}%
\pgfsetfillcolor{currentfill}%
\pgfsetlinewidth{0.000000pt}%
\definecolor{currentstroke}{rgb}{0.119512,0.607464,0.540218}%
\pgfsetstrokecolor{currentstroke}%
\pgfsetdash{}{0pt}%
\pgfpathmoveto{\pgfqpoint{6.399983in}{2.479315in}}%
\pgfpathlineto{\pgfqpoint{6.183670in}{2.518396in}}%
\pgfpathlineto{\pgfqpoint{6.324525in}{2.404996in}}%
\pgfpathclose%
\pgfusepath{fill}%
\end{pgfscope}%
\begin{pgfscope}%
\pgfpathrectangle{\pgfqpoint{0.539299in}{0.078740in}}{\pgfqpoint{7.842520in}{7.842520in}}%
\pgfusepath{clip}%
\pgfsetbuttcap%
\pgfsetroundjoin%
\definecolor{currentfill}{rgb}{0.163625,0.471133,0.558148}%
\pgfsetfillcolor{currentfill}%
\pgfsetlinewidth{0.000000pt}%
\definecolor{currentstroke}{rgb}{0.119423,0.611141,0.538982}%
\pgfsetstrokecolor{currentstroke}%
\pgfsetdash{}{0pt}%
\pgfpathmoveto{\pgfqpoint{4.986662in}{3.717303in}}%
\pgfpathlineto{\pgfqpoint{5.065172in}{3.654551in}}%
\pgfpathlineto{\pgfqpoint{4.925899in}{3.838472in}}%
\pgfpathclose%
\pgfusepath{fill}%
\end{pgfscope}%
\begin{pgfscope}%
\pgfpathrectangle{\pgfqpoint{0.539299in}{0.078740in}}{\pgfqpoint{7.842520in}{7.842520in}}%
\pgfusepath{clip}%
\pgfsetbuttcap%
\pgfsetroundjoin%
\definecolor{currentfill}{rgb}{0.151918,0.500685,0.557587}%
\pgfsetfillcolor{currentfill}%
\pgfsetlinewidth{0.000000pt}%
\definecolor{currentstroke}{rgb}{0.119483,0.614817,0.537692}%
\pgfsetstrokecolor{currentstroke}%
\pgfsetdash{}{0pt}%
\pgfpathmoveto{\pgfqpoint{1.975740in}{3.757494in}}%
\pgfpathlineto{\pgfqpoint{1.889628in}{3.636546in}}%
\pgfpathlineto{\pgfqpoint{2.095141in}{4.184186in}}%
\pgfpathclose%
\pgfusepath{fill}%
\end{pgfscope}%
\begin{pgfscope}%
\pgfpathrectangle{\pgfqpoint{0.539299in}{0.078740in}}{\pgfqpoint{7.842520in}{7.842520in}}%
\pgfusepath{clip}%
\pgfsetbuttcap%
\pgfsetroundjoin%
\definecolor{currentfill}{rgb}{0.506271,0.828786,0.300362}%
\pgfsetfillcolor{currentfill}%
\pgfsetlinewidth{0.000000pt}%
\definecolor{currentstroke}{rgb}{0.119699,0.618490,0.536347}%
\pgfsetstrokecolor{currentstroke}%
\pgfsetdash{}{0pt}%
\pgfpathmoveto{\pgfqpoint{3.087266in}{5.438839in}}%
\pgfpathlineto{\pgfqpoint{3.308861in}{5.497668in}}%
\pgfpathlineto{\pgfqpoint{3.173672in}{5.414731in}}%
\pgfpathclose%
\pgfusepath{fill}%
\end{pgfscope}%
\begin{pgfscope}%
\pgfpathrectangle{\pgfqpoint{0.539299in}{0.078740in}}{\pgfqpoint{7.842520in}{7.842520in}}%
\pgfusepath{clip}%
\pgfsetbuttcap%
\pgfsetroundjoin%
\definecolor{currentfill}{rgb}{0.225863,0.330805,0.547314}%
\pgfsetfillcolor{currentfill}%
\pgfsetlinewidth{0.000000pt}%
\definecolor{currentstroke}{rgb}{0.120081,0.622161,0.534946}%
\pgfsetstrokecolor{currentstroke}%
\pgfsetdash{}{0pt}%
\pgfpathmoveto{\pgfqpoint{1.718526in}{3.335879in}}%
\pgfpathlineto{\pgfqpoint{1.601948in}{2.927986in}}%
\pgfpathlineto{\pgfqpoint{1.633975in}{3.148721in}}%
\pgfpathclose%
\pgfusepath{fill}%
\end{pgfscope}%
\begin{pgfscope}%
\pgfpathrectangle{\pgfqpoint{0.539299in}{0.078740in}}{\pgfqpoint{7.842520in}{7.842520in}}%
\pgfusepath{clip}%
\pgfsetbuttcap%
\pgfsetroundjoin%
\definecolor{currentfill}{rgb}{0.258965,0.251537,0.524736}%
\pgfsetfillcolor{currentfill}%
\pgfsetlinewidth{0.000000pt}%
\definecolor{currentstroke}{rgb}{0.120638,0.625828,0.533488}%
\pgfsetstrokecolor{currentstroke}%
\pgfsetdash{}{0pt}%
\pgfpathmoveto{\pgfqpoint{5.902739in}{2.756289in}}%
\pgfpathlineto{\pgfqpoint{5.762666in}{2.883788in}}%
\pgfpathlineto{\pgfqpoint{5.686202in}{2.865893in}}%
\pgfpathclose%
\pgfusepath{fill}%
\end{pgfscope}%
\begin{pgfscope}%
\pgfpathrectangle{\pgfqpoint{0.539299in}{0.078740in}}{\pgfqpoint{7.842520in}{7.842520in}}%
\pgfusepath{clip}%
\pgfsetbuttcap%
\pgfsetroundjoin%
\definecolor{currentfill}{rgb}{0.281924,0.089666,0.412415}%
\pgfsetfillcolor{currentfill}%
\pgfsetlinewidth{0.000000pt}%
\definecolor{currentstroke}{rgb}{0.121380,0.629492,0.531973}%
\pgfsetstrokecolor{currentstroke}%
\pgfsetdash{}{0pt}%
\pgfpathmoveto{\pgfqpoint{6.465616in}{2.293258in}}%
\pgfpathlineto{\pgfqpoint{6.681865in}{2.263405in}}%
\pgfpathlineto{\pgfqpoint{6.540856in}{2.372768in}}%
\pgfpathclose%
\pgfusepath{fill}%
\end{pgfscope}%
\begin{pgfscope}%
\pgfpathrectangle{\pgfqpoint{0.539299in}{0.078740in}}{\pgfqpoint{7.842520in}{7.842520in}}%
\pgfusepath{clip}%
\pgfsetbuttcap%
\pgfsetroundjoin%
\definecolor{currentfill}{rgb}{0.185783,0.704891,0.485273}%
\pgfsetfillcolor{currentfill}%
\pgfsetlinewidth{0.000000pt}%
\definecolor{currentstroke}{rgb}{0.122312,0.633153,0.530398}%
\pgfsetstrokecolor{currentstroke}%
\pgfsetdash{}{0pt}%
\pgfpathmoveto{\pgfqpoint{2.478591in}{4.855876in}}%
\pgfpathlineto{\pgfqpoint{2.565417in}{4.919366in}}%
\pgfpathlineto{\pgfqpoint{2.440760in}{4.553021in}}%
\pgfpathclose%
\pgfusepath{fill}%
\end{pgfscope}%
\begin{pgfscope}%
\pgfpathrectangle{\pgfqpoint{0.539299in}{0.078740in}}{\pgfqpoint{7.842520in}{7.842520in}}%
\pgfusepath{clip}%
\pgfsetbuttcap%
\pgfsetroundjoin%
\definecolor{currentfill}{rgb}{0.283229,0.120777,0.440584}%
\pgfsetfillcolor{currentfill}%
\pgfsetlinewidth{0.000000pt}%
\definecolor{currentstroke}{rgb}{0.123444,0.636809,0.528763}%
\pgfsetstrokecolor{currentstroke}%
\pgfsetdash{}{0pt}%
\pgfpathmoveto{\pgfqpoint{6.465616in}{2.293258in}}%
\pgfpathlineto{\pgfqpoint{6.399983in}{2.479315in}}%
\pgfpathlineto{\pgfqpoint{6.324525in}{2.404996in}}%
\pgfpathclose%
\pgfusepath{fill}%
\end{pgfscope}%
\begin{pgfscope}%
\pgfpathrectangle{\pgfqpoint{0.539299in}{0.078740in}}{\pgfqpoint{7.842520in}{7.842520in}}%
\pgfusepath{clip}%
\pgfsetbuttcap%
\pgfsetroundjoin%
\definecolor{currentfill}{rgb}{0.139147,0.533812,0.555298}%
\pgfsetfillcolor{currentfill}%
\pgfsetlinewidth{0.000000pt}%
\definecolor{currentstroke}{rgb}{0.124780,0.640461,0.527068}%
\pgfsetstrokecolor{currentstroke}%
\pgfsetdash{}{0pt}%
\pgfpathmoveto{\pgfqpoint{4.925899in}{3.838472in}}%
\pgfpathlineto{\pgfqpoint{4.786562in}{4.030722in}}%
\pgfpathlineto{\pgfqpoint{4.706729in}{4.119068in}}%
\pgfpathclose%
\pgfusepath{fill}%
\end{pgfscope}%
\begin{pgfscope}%
\pgfpathrectangle{\pgfqpoint{0.539299in}{0.078740in}}{\pgfqpoint{7.842520in}{7.842520in}}%
\pgfusepath{clip}%
\pgfsetbuttcap%
\pgfsetroundjoin%
\definecolor{currentfill}{rgb}{0.273006,0.204520,0.501721}%
\pgfsetfillcolor{currentfill}%
\pgfsetlinewidth{0.000000pt}%
\definecolor{currentstroke}{rgb}{0.126326,0.644107,0.525311}%
\pgfsetstrokecolor{currentstroke}%
\pgfsetdash{}{0pt}%
\pgfpathmoveto{\pgfqpoint{5.902739in}{2.756289in}}%
\pgfpathlineto{\pgfqpoint{5.967002in}{2.588084in}}%
\pgfpathlineto{\pgfqpoint{6.043073in}{2.634987in}}%
\pgfpathclose%
\pgfusepath{fill}%
\end{pgfscope}%
\begin{pgfscope}%
\pgfpathrectangle{\pgfqpoint{0.539299in}{0.078740in}}{\pgfqpoint{7.842520in}{7.842520in}}%
\pgfusepath{clip}%
\pgfsetbuttcap%
\pgfsetroundjoin%
\definecolor{currentfill}{rgb}{0.271828,0.209303,0.504434}%
\pgfsetfillcolor{currentfill}%
\pgfsetlinewidth{0.000000pt}%
\definecolor{currentstroke}{rgb}{0.128087,0.647749,0.523491}%
\pgfsetstrokecolor{currentstroke}%
\pgfsetdash{}{0pt}%
\pgfpathmoveto{\pgfqpoint{1.429595in}{2.648382in}}%
\pgfpathlineto{\pgfqpoint{1.344269in}{2.477731in}}%
\pgfpathlineto{\pgfqpoint{1.550510in}{2.929884in}}%
\pgfpathclose%
\pgfusepath{fill}%
\end{pgfscope}%
\begin{pgfscope}%
\pgfpathrectangle{\pgfqpoint{0.539299in}{0.078740in}}{\pgfqpoint{7.842520in}{7.842520in}}%
\pgfusepath{clip}%
\pgfsetbuttcap%
\pgfsetroundjoin%
\definecolor{currentfill}{rgb}{0.545524,0.838039,0.275626}%
\pgfsetfillcolor{currentfill}%
\pgfsetlinewidth{0.000000pt}%
\definecolor{currentstroke}{rgb}{0.130067,0.651384,0.521608}%
\pgfsetstrokecolor{currentstroke}%
\pgfsetdash{}{0pt}%
\pgfpathmoveto{\pgfqpoint{3.531606in}{5.447979in}}%
\pgfpathlineto{\pgfqpoint{3.308861in}{5.497668in}}%
\pgfpathlineto{\pgfqpoint{3.445912in}{5.515993in}}%
\pgfpathclose%
\pgfusepath{fill}%
\end{pgfscope}%
\begin{pgfscope}%
\pgfpathrectangle{\pgfqpoint{0.539299in}{0.078740in}}{\pgfqpoint{7.842520in}{7.842520in}}%
\pgfusepath{clip}%
\pgfsetbuttcap%
\pgfsetroundjoin%
\definecolor{currentfill}{rgb}{0.212395,0.359683,0.551710}%
\pgfsetfillcolor{currentfill}%
\pgfsetlinewidth{0.000000pt}%
\definecolor{currentstroke}{rgb}{0.132268,0.655014,0.519661}%
\pgfsetstrokecolor{currentstroke}%
\pgfsetdash{}{0pt}%
\pgfpathmoveto{\pgfqpoint{5.266300in}{3.347822in}}%
\pgfpathlineto{\pgfqpoint{5.406152in}{3.177575in}}%
\pgfpathlineto{\pgfqpoint{5.483218in}{3.162772in}}%
\pgfpathclose%
\pgfusepath{fill}%
\end{pgfscope}%
\begin{pgfscope}%
\pgfpathrectangle{\pgfqpoint{0.539299in}{0.078740in}}{\pgfqpoint{7.842520in}{7.842520in}}%
\pgfusepath{clip}%
\pgfsetbuttcap%
\pgfsetroundjoin%
\definecolor{currentfill}{rgb}{0.231674,0.318106,0.544834}%
\pgfsetfillcolor{currentfill}%
\pgfsetlinewidth{0.000000pt}%
\definecolor{currentstroke}{rgb}{0.134692,0.658636,0.517649}%
\pgfsetstrokecolor{currentstroke}%
\pgfsetdash{}{0pt}%
\pgfpathmoveto{\pgfqpoint{5.622835in}{3.018878in}}%
\pgfpathlineto{\pgfqpoint{5.406152in}{3.177575in}}%
\pgfpathlineto{\pgfqpoint{5.546103in}{3.017088in}}%
\pgfpathclose%
\pgfusepath{fill}%
\end{pgfscope}%
\begin{pgfscope}%
\pgfpathrectangle{\pgfqpoint{0.539299in}{0.078740in}}{\pgfqpoint{7.842520in}{7.842520in}}%
\pgfusepath{clip}%
\pgfsetbuttcap%
\pgfsetroundjoin%
\definecolor{currentfill}{rgb}{0.243113,0.292092,0.538516}%
\pgfsetfillcolor{currentfill}%
\pgfsetlinewidth{0.000000pt}%
\definecolor{currentstroke}{rgb}{0.137339,0.662252,0.515571}%
\pgfsetstrokecolor{currentstroke}%
\pgfsetdash{}{0pt}%
\pgfpathmoveto{\pgfqpoint{5.622835in}{3.018878in}}%
\pgfpathlineto{\pgfqpoint{5.546103in}{3.017088in}}%
\pgfpathlineto{\pgfqpoint{5.686202in}{2.865893in}}%
\pgfpathclose%
\pgfusepath{fill}%
\end{pgfscope}%
\begin{pgfscope}%
\pgfpathrectangle{\pgfqpoint{0.539299in}{0.078740in}}{\pgfqpoint{7.842520in}{7.842520in}}%
\pgfusepath{clip}%
\pgfsetbuttcap%
\pgfsetroundjoin%
\definecolor{currentfill}{rgb}{0.187231,0.414746,0.556547}%
\pgfsetfillcolor{currentfill}%
\pgfsetlinewidth{0.000000pt}%
\definecolor{currentstroke}{rgb}{0.140210,0.665859,0.513427}%
\pgfsetstrokecolor{currentstroke}%
\pgfsetdash{}{0pt}%
\pgfpathmoveto{\pgfqpoint{5.266300in}{3.347822in}}%
\pgfpathlineto{\pgfqpoint{5.204442in}{3.480326in}}%
\pgfpathlineto{\pgfqpoint{5.126490in}{3.527887in}}%
\pgfpathclose%
\pgfusepath{fill}%
\end{pgfscope}%
\begin{pgfscope}%
\pgfpathrectangle{\pgfqpoint{0.539299in}{0.078740in}}{\pgfqpoint{7.842520in}{7.842520in}}%
\pgfusepath{clip}%
\pgfsetbuttcap%
\pgfsetroundjoin%
\definecolor{currentfill}{rgb}{0.277018,0.050344,0.375715}%
\pgfsetfillcolor{currentfill}%
\pgfsetlinewidth{0.000000pt}%
\definecolor{currentstroke}{rgb}{0.143303,0.669459,0.511215}%
\pgfsetstrokecolor{currentstroke}%
\pgfsetdash{}{0pt}%
\pgfpathmoveto{\pgfqpoint{6.748358in}{2.068409in}}%
\pgfpathlineto{\pgfqpoint{6.822930in}{2.149191in}}%
\pgfpathlineto{\pgfqpoint{6.681865in}{2.263405in}}%
\pgfpathclose%
\pgfusepath{fill}%
\end{pgfscope}%
\begin{pgfscope}%
\pgfpathrectangle{\pgfqpoint{0.539299in}{0.078740in}}{\pgfqpoint{7.842520in}{7.842520in}}%
\pgfusepath{clip}%
\pgfsetbuttcap%
\pgfsetroundjoin%
\definecolor{currentfill}{rgb}{0.278012,0.180367,0.486697}%
\pgfsetfillcolor{currentfill}%
\pgfsetlinewidth{0.000000pt}%
\definecolor{currentstroke}{rgb}{0.146616,0.673050,0.508936}%
\pgfsetstrokecolor{currentstroke}%
\pgfsetdash{}{0pt}%
\pgfpathmoveto{\pgfqpoint{6.043073in}{2.634987in}}%
\pgfpathlineto{\pgfqpoint{5.967002in}{2.588084in}}%
\pgfpathlineto{\pgfqpoint{6.183670in}{2.518396in}}%
\pgfpathclose%
\pgfusepath{fill}%
\end{pgfscope}%
\begin{pgfscope}%
\pgfpathrectangle{\pgfqpoint{0.539299in}{0.078740in}}{\pgfqpoint{7.842520in}{7.842520in}}%
\pgfusepath{clip}%
\pgfsetbuttcap%
\pgfsetroundjoin%
\definecolor{currentfill}{rgb}{0.123463,0.581687,0.547445}%
\pgfsetfillcolor{currentfill}%
\pgfsetlinewidth{0.000000pt}%
\definecolor{currentstroke}{rgb}{0.150148,0.676631,0.506589}%
\pgfsetstrokecolor{currentstroke}%
\pgfsetdash{}{0pt}%
\pgfpathmoveto{\pgfqpoint{4.786562in}{4.030722in}}%
\pgfpathlineto{\pgfqpoint{4.647114in}{4.229077in}}%
\pgfpathlineto{\pgfqpoint{4.566539in}{4.326950in}}%
\pgfpathclose%
\pgfusepath{fill}%
\end{pgfscope}%
\begin{pgfscope}%
\pgfpathrectangle{\pgfqpoint{0.539299in}{0.078740in}}{\pgfqpoint{7.842520in}{7.842520in}}%
\pgfusepath{clip}%
\pgfsetbuttcap%
\pgfsetroundjoin%
\definecolor{currentfill}{rgb}{0.449368,0.813768,0.335384}%
\pgfsetfillcolor{currentfill}%
\pgfsetlinewidth{0.000000pt}%
\definecolor{currentstroke}{rgb}{0.153894,0.680203,0.504172}%
\pgfsetstrokecolor{currentstroke}%
\pgfsetdash{}{0pt}%
\pgfpathmoveto{\pgfqpoint{2.954612in}{5.267216in}}%
\pgfpathlineto{\pgfqpoint{2.867996in}{5.262317in}}%
\pgfpathlineto{\pgfqpoint{3.000456in}{5.448722in}}%
\pgfpathclose%
\pgfusepath{fill}%
\end{pgfscope}%
\begin{pgfscope}%
\pgfpathrectangle{\pgfqpoint{0.539299in}{0.078740in}}{\pgfqpoint{7.842520in}{7.842520in}}%
\pgfusepath{clip}%
\pgfsetbuttcap%
\pgfsetroundjoin%
\definecolor{currentfill}{rgb}{0.515992,0.831158,0.294279}%
\pgfsetfillcolor{currentfill}%
\pgfsetlinewidth{0.000000pt}%
\definecolor{currentstroke}{rgb}{0.157851,0.683765,0.501686}%
\pgfsetstrokecolor{currentstroke}%
\pgfsetdash{}{0pt}%
\pgfpathmoveto{\pgfqpoint{3.669721in}{5.396407in}}%
\pgfpathlineto{\pgfqpoint{3.531606in}{5.447979in}}%
\pgfpathlineto{\pgfqpoint{3.584365in}{5.478002in}}%
\pgfpathclose%
\pgfusepath{fill}%
\end{pgfscope}%
\begin{pgfscope}%
\pgfpathrectangle{\pgfqpoint{0.539299in}{0.078740in}}{\pgfqpoint{7.842520in}{7.842520in}}%
\pgfusepath{clip}%
\pgfsetbuttcap%
\pgfsetroundjoin%
\definecolor{currentfill}{rgb}{0.360741,0.785964,0.387814}%
\pgfsetfillcolor{currentfill}%
\pgfsetlinewidth{0.000000pt}%
\definecolor{currentstroke}{rgb}{0.162016,0.687316,0.499129}%
\pgfsetstrokecolor{currentstroke}%
\pgfsetdash{}{0pt}%
\pgfpathmoveto{\pgfqpoint{2.867996in}{5.262317in}}%
\pgfpathlineto{\pgfqpoint{2.652170in}{4.965747in}}%
\pgfpathlineto{\pgfqpoint{2.781099in}{5.242417in}}%
\pgfpathclose%
\pgfusepath{fill}%
\end{pgfscope}%
\begin{pgfscope}%
\pgfpathrectangle{\pgfqpoint{0.539299in}{0.078740in}}{\pgfqpoint{7.842520in}{7.842520in}}%
\pgfusepath{clip}%
\pgfsetbuttcap%
\pgfsetroundjoin%
\definecolor{currentfill}{rgb}{0.131172,0.555899,0.552459}%
\pgfsetfillcolor{currentfill}%
\pgfsetlinewidth{0.000000pt}%
\definecolor{currentstroke}{rgb}{0.166383,0.690856,0.496502}%
\pgfsetstrokecolor{currentstroke}%
\pgfsetdash{}{0pt}%
\pgfpathmoveto{\pgfqpoint{1.975740in}{3.757494in}}%
\pgfpathlineto{\pgfqpoint{2.095141in}{4.184186in}}%
\pgfpathlineto{\pgfqpoint{2.181328in}{4.304148in}}%
\pgfpathclose%
\pgfusepath{fill}%
\end{pgfscope}%
\begin{pgfscope}%
\pgfpathrectangle{\pgfqpoint{0.539299in}{0.078740in}}{\pgfqpoint{7.842520in}{7.842520in}}%
\pgfusepath{clip}%
\pgfsetbuttcap%
\pgfsetroundjoin%
\definecolor{currentfill}{rgb}{0.169646,0.456262,0.558030}%
\pgfsetfillcolor{currentfill}%
\pgfsetlinewidth{0.000000pt}%
\definecolor{currentstroke}{rgb}{0.170948,0.694384,0.493803}%
\pgfsetstrokecolor{currentstroke}%
\pgfsetdash{}{0pt}%
\pgfpathmoveto{\pgfqpoint{5.065172in}{3.654551in}}%
\pgfpathlineto{\pgfqpoint{4.986662in}{3.717303in}}%
\pgfpathlineto{\pgfqpoint{5.126490in}{3.527887in}}%
\pgfpathclose%
\pgfusepath{fill}%
\end{pgfscope}%
\begin{pgfscope}%
\pgfpathrectangle{\pgfqpoint{0.539299in}{0.078740in}}{\pgfqpoint{7.842520in}{7.842520in}}%
\pgfusepath{clip}%
\pgfsetbuttcap%
\pgfsetroundjoin%
\definecolor{currentfill}{rgb}{0.280894,0.078907,0.402329}%
\pgfsetfillcolor{currentfill}%
\pgfsetlinewidth{0.000000pt}%
\definecolor{currentstroke}{rgb}{0.175707,0.697900,0.491033}%
\pgfsetstrokecolor{currentstroke}%
\pgfsetdash{}{0pt}%
\pgfpathmoveto{\pgfqpoint{6.465616in}{2.293258in}}%
\pgfpathlineto{\pgfqpoint{6.606910in}{2.181621in}}%
\pgfpathlineto{\pgfqpoint{6.681865in}{2.263405in}}%
\pgfpathclose%
\pgfusepath{fill}%
\end{pgfscope}%
\begin{pgfscope}%
\pgfpathrectangle{\pgfqpoint{0.539299in}{0.078740in}}{\pgfqpoint{7.842520in}{7.842520in}}%
\pgfusepath{clip}%
\pgfsetbuttcap%
\pgfsetroundjoin%
\definecolor{currentfill}{rgb}{0.262138,0.242286,0.520837}%
\pgfsetfillcolor{currentfill}%
\pgfsetlinewidth{0.000000pt}%
\definecolor{currentstroke}{rgb}{0.180653,0.701402,0.488189}%
\pgfsetstrokecolor{currentstroke}%
\pgfsetdash{}{0pt}%
\pgfpathmoveto{\pgfqpoint{5.686202in}{2.865893in}}%
\pgfpathlineto{\pgfqpoint{5.826491in}{2.723216in}}%
\pgfpathlineto{\pgfqpoint{5.902739in}{2.756289in}}%
\pgfpathclose%
\pgfusepath{fill}%
\end{pgfscope}%
\begin{pgfscope}%
\pgfpathrectangle{\pgfqpoint{0.539299in}{0.078740in}}{\pgfqpoint{7.842520in}{7.842520in}}%
\pgfusepath{clip}%
\pgfsetbuttcap%
\pgfsetroundjoin%
\definecolor{currentfill}{rgb}{0.487026,0.823929,0.312321}%
\pgfsetfillcolor{currentfill}%
\pgfsetlinewidth{0.000000pt}%
\definecolor{currentstroke}{rgb}{0.185783,0.704891,0.485273}%
\pgfsetstrokecolor{currentstroke}%
\pgfsetdash{}{0pt}%
\pgfpathmoveto{\pgfqpoint{3.000456in}{5.448722in}}%
\pgfpathlineto{\pgfqpoint{3.087266in}{5.438839in}}%
\pgfpathlineto{\pgfqpoint{2.954612in}{5.267216in}}%
\pgfpathclose%
\pgfusepath{fill}%
\end{pgfscope}%
\begin{pgfscope}%
\pgfpathrectangle{\pgfqpoint{0.539299in}{0.078740in}}{\pgfqpoint{7.842520in}{7.842520in}}%
\pgfusepath{clip}%
\pgfsetbuttcap%
\pgfsetroundjoin%
\definecolor{currentfill}{rgb}{0.140210,0.665859,0.513427}%
\pgfsetfillcolor{currentfill}%
\pgfsetlinewidth{0.000000pt}%
\definecolor{currentstroke}{rgb}{0.191090,0.708366,0.482284}%
\pgfsetstrokecolor{currentstroke}%
\pgfsetdash{}{0pt}%
\pgfpathmoveto{\pgfqpoint{2.267756in}{4.403873in}}%
\pgfpathlineto{\pgfqpoint{2.391827in}{4.772531in}}%
\pgfpathlineto{\pgfqpoint{2.354275in}{4.486045in}}%
\pgfpathclose%
\pgfusepath{fill}%
\end{pgfscope}%
\begin{pgfscope}%
\pgfpathrectangle{\pgfqpoint{0.539299in}{0.078740in}}{\pgfqpoint{7.842520in}{7.842520in}}%
\pgfusepath{clip}%
\pgfsetbuttcap%
\pgfsetroundjoin%
\definecolor{currentfill}{rgb}{0.269308,0.218818,0.509577}%
\pgfsetfillcolor{currentfill}%
\pgfsetlinewidth{0.000000pt}%
\definecolor{currentstroke}{rgb}{0.196571,0.711827,0.479221}%
\pgfsetstrokecolor{currentstroke}%
\pgfsetdash{}{0pt}%
\pgfpathmoveto{\pgfqpoint{5.826491in}{2.723216in}}%
\pgfpathlineto{\pgfqpoint{5.967002in}{2.588084in}}%
\pgfpathlineto{\pgfqpoint{5.902739in}{2.756289in}}%
\pgfpathclose%
\pgfusepath{fill}%
\end{pgfscope}%
\begin{pgfscope}%
\pgfpathrectangle{\pgfqpoint{0.539299in}{0.078740in}}{\pgfqpoint{7.842520in}{7.842520in}}%
\pgfusepath{clip}%
\pgfsetbuttcap%
\pgfsetroundjoin%
\definecolor{currentfill}{rgb}{0.121380,0.629492,0.531973}%
\pgfsetfillcolor{currentfill}%
\pgfsetlinewidth{0.000000pt}%
\definecolor{currentstroke}{rgb}{0.202219,0.715272,0.476084}%
\pgfsetstrokecolor{currentstroke}%
\pgfsetdash{}{0pt}%
\pgfpathmoveto{\pgfqpoint{4.647114in}{4.229077in}}%
\pgfpathlineto{\pgfqpoint{4.507522in}{4.430367in}}%
\pgfpathlineto{\pgfqpoint{4.426173in}{4.535091in}}%
\pgfpathclose%
\pgfusepath{fill}%
\end{pgfscope}%
\begin{pgfscope}%
\pgfpathrectangle{\pgfqpoint{0.539299in}{0.078740in}}{\pgfqpoint{7.842520in}{7.842520in}}%
\pgfusepath{clip}%
\pgfsetbuttcap%
\pgfsetroundjoin%
\definecolor{currentfill}{rgb}{0.283187,0.125848,0.444960}%
\pgfsetfillcolor{currentfill}%
\pgfsetlinewidth{0.000000pt}%
\definecolor{currentstroke}{rgb}{0.208030,0.718701,0.472873}%
\pgfsetstrokecolor{currentstroke}%
\pgfsetdash{}{0pt}%
\pgfpathmoveto{\pgfqpoint{1.388975in}{2.359780in}}%
\pgfpathlineto{\pgfqpoint{1.344269in}{2.477731in}}%
\pgfpathlineto{\pgfqpoint{1.260140in}{2.275047in}}%
\pgfpathclose%
\pgfusepath{fill}%
\end{pgfscope}%
\begin{pgfscope}%
\pgfpathrectangle{\pgfqpoint{0.539299in}{0.078740in}}{\pgfqpoint{7.842520in}{7.842520in}}%
\pgfusepath{clip}%
\pgfsetbuttcap%
\pgfsetroundjoin%
\definecolor{currentfill}{rgb}{0.151918,0.500685,0.557587}%
\pgfsetfillcolor{currentfill}%
\pgfsetlinewidth{0.000000pt}%
\definecolor{currentstroke}{rgb}{0.214000,0.722114,0.469588}%
\pgfsetstrokecolor{currentstroke}%
\pgfsetdash{}{0pt}%
\pgfpathmoveto{\pgfqpoint{4.846758in}{3.914972in}}%
\pgfpathlineto{\pgfqpoint{4.986662in}{3.717303in}}%
\pgfpathlineto{\pgfqpoint{4.925899in}{3.838472in}}%
\pgfpathclose%
\pgfusepath{fill}%
\end{pgfscope}%
\begin{pgfscope}%
\pgfpathrectangle{\pgfqpoint{0.539299in}{0.078740in}}{\pgfqpoint{7.842520in}{7.842520in}}%
\pgfusepath{clip}%
\pgfsetbuttcap%
\pgfsetroundjoin%
\definecolor{currentfill}{rgb}{0.278791,0.062145,0.386592}%
\pgfsetfillcolor{currentfill}%
\pgfsetlinewidth{0.000000pt}%
\definecolor{currentstroke}{rgb}{0.220124,0.725509,0.466226}%
\pgfsetstrokecolor{currentstroke}%
\pgfsetdash{}{0pt}%
\pgfpathmoveto{\pgfqpoint{6.681865in}{2.263405in}}%
\pgfpathlineto{\pgfqpoint{6.606910in}{2.181621in}}%
\pgfpathlineto{\pgfqpoint{6.748358in}{2.068409in}}%
\pgfpathclose%
\pgfusepath{fill}%
\end{pgfscope}%
\begin{pgfscope}%
\pgfpathrectangle{\pgfqpoint{0.539299in}{0.078740in}}{\pgfqpoint{7.842520in}{7.842520in}}%
\pgfusepath{clip}%
\pgfsetbuttcap%
\pgfsetroundjoin%
\definecolor{currentfill}{rgb}{0.282623,0.140926,0.457517}%
\pgfsetfillcolor{currentfill}%
\pgfsetlinewidth{0.000000pt}%
\definecolor{currentstroke}{rgb}{0.226397,0.728888,0.462789}%
\pgfsetstrokecolor{currentstroke}%
\pgfsetdash{}{0pt}%
\pgfpathmoveto{\pgfqpoint{6.324525in}{2.404996in}}%
\pgfpathlineto{\pgfqpoint{6.183670in}{2.518396in}}%
\pgfpathlineto{\pgfqpoint{6.248772in}{2.336093in}}%
\pgfpathclose%
\pgfusepath{fill}%
\end{pgfscope}%
\begin{pgfscope}%
\pgfpathrectangle{\pgfqpoint{0.539299in}{0.078740in}}{\pgfqpoint{7.842520in}{7.842520in}}%
\pgfusepath{clip}%
\pgfsetbuttcap%
\pgfsetroundjoin%
\definecolor{currentfill}{rgb}{0.496615,0.826376,0.306377}%
\pgfsetfillcolor{currentfill}%
\pgfsetlinewidth{0.000000pt}%
\definecolor{currentstroke}{rgb}{0.232815,0.732247,0.459277}%
\pgfsetstrokecolor{currentstroke}%
\pgfsetdash{}{0pt}%
\pgfpathmoveto{\pgfqpoint{3.808702in}{5.299454in}}%
\pgfpathlineto{\pgfqpoint{3.669721in}{5.396407in}}%
\pgfpathlineto{\pgfqpoint{3.584365in}{5.478002in}}%
\pgfpathclose%
\pgfusepath{fill}%
\end{pgfscope}%
\begin{pgfscope}%
\pgfpathrectangle{\pgfqpoint{0.539299in}{0.078740in}}{\pgfqpoint{7.842520in}{7.842520in}}%
\pgfusepath{clip}%
\pgfsetbuttcap%
\pgfsetroundjoin%
\definecolor{currentfill}{rgb}{0.159194,0.482237,0.558073}%
\pgfsetfillcolor{currentfill}%
\pgfsetlinewidth{0.000000pt}%
\definecolor{currentstroke}{rgb}{0.239374,0.735588,0.455688}%
\pgfsetstrokecolor{currentstroke}%
\pgfsetdash{}{0pt}%
\pgfpathmoveto{\pgfqpoint{2.009371in}{4.040892in}}%
\pgfpathlineto{\pgfqpoint{1.889628in}{3.636546in}}%
\pgfpathlineto{\pgfqpoint{1.803826in}{3.497087in}}%
\pgfpathclose%
\pgfusepath{fill}%
\end{pgfscope}%
\begin{pgfscope}%
\pgfpathrectangle{\pgfqpoint{0.539299in}{0.078740in}}{\pgfqpoint{7.842520in}{7.842520in}}%
\pgfusepath{clip}%
\pgfsetbuttcap%
\pgfsetroundjoin%
\definecolor{currentfill}{rgb}{0.137339,0.662252,0.515571}%
\pgfsetfillcolor{currentfill}%
\pgfsetlinewidth{0.000000pt}%
\definecolor{currentstroke}{rgb}{0.246070,0.738910,0.452024}%
\pgfsetstrokecolor{currentstroke}%
\pgfsetdash{}{0pt}%
\pgfpathmoveto{\pgfqpoint{4.507522in}{4.430367in}}%
\pgfpathlineto{\pgfqpoint{4.367781in}{4.630418in}}%
\pgfpathlineto{\pgfqpoint{4.426173in}{4.535091in}}%
\pgfpathclose%
\pgfusepath{fill}%
\end{pgfscope}%
\begin{pgfscope}%
\pgfpathrectangle{\pgfqpoint{0.539299in}{0.078740in}}{\pgfqpoint{7.842520in}{7.842520in}}%
\pgfusepath{clip}%
\pgfsetbuttcap%
\pgfsetroundjoin%
\definecolor{currentfill}{rgb}{0.139147,0.533812,0.555298}%
\pgfsetfillcolor{currentfill}%
\pgfsetlinewidth{0.000000pt}%
\definecolor{currentstroke}{rgb}{0.252899,0.742211,0.448284}%
\pgfsetstrokecolor{currentstroke}%
\pgfsetdash{}{0pt}%
\pgfpathmoveto{\pgfqpoint{4.925899in}{3.838472in}}%
\pgfpathlineto{\pgfqpoint{4.706729in}{4.119068in}}%
\pgfpathlineto{\pgfqpoint{4.846758in}{3.914972in}}%
\pgfpathclose%
\pgfusepath{fill}%
\end{pgfscope}%
\begin{pgfscope}%
\pgfpathrectangle{\pgfqpoint{0.539299in}{0.078740in}}{\pgfqpoint{7.842520in}{7.842520in}}%
\pgfusepath{clip}%
\pgfsetbuttcap%
\pgfsetroundjoin%
\definecolor{currentfill}{rgb}{0.246811,0.283237,0.535941}%
\pgfsetfillcolor{currentfill}%
\pgfsetlinewidth{0.000000pt}%
\definecolor{currentstroke}{rgb}{0.259857,0.745492,0.444467}%
\pgfsetstrokecolor{currentstroke}%
\pgfsetdash{}{0pt}%
\pgfpathmoveto{\pgfqpoint{1.429595in}{2.648382in}}%
\pgfpathlineto{\pgfqpoint{1.550510in}{2.929884in}}%
\pgfpathlineto{\pgfqpoint{1.633975in}{3.148721in}}%
\pgfpathclose%
\pgfusepath{fill}%
\end{pgfscope}%
\begin{pgfscope}%
\pgfpathrectangle{\pgfqpoint{0.539299in}{0.078740in}}{\pgfqpoint{7.842520in}{7.842520in}}%
\pgfusepath{clip}%
\pgfsetbuttcap%
\pgfsetroundjoin%
\definecolor{currentfill}{rgb}{0.555484,0.840254,0.269281}%
\pgfsetfillcolor{currentfill}%
\pgfsetlinewidth{0.000000pt}%
\definecolor{currentstroke}{rgb}{0.266941,0.748751,0.440573}%
\pgfsetstrokecolor{currentstroke}%
\pgfsetdash{}{0pt}%
\pgfpathmoveto{\pgfqpoint{3.584365in}{5.478002in}}%
\pgfpathlineto{\pgfqpoint{3.531606in}{5.447979in}}%
\pgfpathlineto{\pgfqpoint{3.445912in}{5.515993in}}%
\pgfpathclose%
\pgfusepath{fill}%
\end{pgfscope}%
\begin{pgfscope}%
\pgfpathrectangle{\pgfqpoint{0.539299in}{0.078740in}}{\pgfqpoint{7.842520in}{7.842520in}}%
\pgfusepath{clip}%
\pgfsetbuttcap%
\pgfsetroundjoin%
\definecolor{currentfill}{rgb}{0.283229,0.120777,0.440584}%
\pgfsetfillcolor{currentfill}%
\pgfsetlinewidth{0.000000pt}%
\definecolor{currentstroke}{rgb}{0.274149,0.751988,0.436601}%
\pgfsetstrokecolor{currentstroke}%
\pgfsetdash{}{0pt}%
\pgfpathmoveto{\pgfqpoint{6.324525in}{2.404996in}}%
\pgfpathlineto{\pgfqpoint{6.248772in}{2.336093in}}%
\pgfpathlineto{\pgfqpoint{6.465616in}{2.293258in}}%
\pgfpathclose%
\pgfusepath{fill}%
\end{pgfscope}%
\begin{pgfscope}%
\pgfpathrectangle{\pgfqpoint{0.539299in}{0.078740in}}{\pgfqpoint{7.842520in}{7.842520in}}%
\pgfusepath{clip}%
\pgfsetbuttcap%
\pgfsetroundjoin%
\definecolor{currentfill}{rgb}{0.278826,0.175490,0.483397}%
\pgfsetfillcolor{currentfill}%
\pgfsetlinewidth{0.000000pt}%
\definecolor{currentstroke}{rgb}{0.281477,0.755203,0.432552}%
\pgfsetstrokecolor{currentstroke}%
\pgfsetdash{}{0pt}%
\pgfpathmoveto{\pgfqpoint{6.183670in}{2.518396in}}%
\pgfpathlineto{\pgfqpoint{5.967002in}{2.588084in}}%
\pgfpathlineto{\pgfqpoint{6.107758in}{2.459413in}}%
\pgfpathclose%
\pgfusepath{fill}%
\end{pgfscope}%
\begin{pgfscope}%
\pgfpathrectangle{\pgfqpoint{0.539299in}{0.078740in}}{\pgfqpoint{7.842520in}{7.842520in}}%
\pgfusepath{clip}%
\pgfsetbuttcap%
\pgfsetroundjoin%
\definecolor{currentfill}{rgb}{0.565498,0.842430,0.262877}%
\pgfsetfillcolor{currentfill}%
\pgfsetlinewidth{0.000000pt}%
\definecolor{currentstroke}{rgb}{0.288921,0.758394,0.428426}%
\pgfsetstrokecolor{currentstroke}%
\pgfsetdash{}{0pt}%
\pgfpathmoveto{\pgfqpoint{3.222442in}{5.537820in}}%
\pgfpathlineto{\pgfqpoint{3.308861in}{5.497668in}}%
\pgfpathlineto{\pgfqpoint{3.087266in}{5.438839in}}%
\pgfpathclose%
\pgfusepath{fill}%
\end{pgfscope}%
\begin{pgfscope}%
\pgfpathrectangle{\pgfqpoint{0.539299in}{0.078740in}}{\pgfqpoint{7.842520in}{7.842520in}}%
\pgfusepath{clip}%
\pgfsetbuttcap%
\pgfsetroundjoin%
\definecolor{currentfill}{rgb}{0.311925,0.767822,0.415586}%
\pgfsetfillcolor{currentfill}%
\pgfsetlinewidth{0.000000pt}%
\definecolor{currentstroke}{rgb}{0.296479,0.761561,0.424223}%
\pgfsetstrokecolor{currentstroke}%
\pgfsetdash{}{0pt}%
\pgfpathmoveto{\pgfqpoint{2.652170in}{4.965747in}}%
\pgfpathlineto{\pgfqpoint{2.565417in}{4.919366in}}%
\pgfpathlineto{\pgfqpoint{2.694035in}{5.204781in}}%
\pgfpathclose%
\pgfusepath{fill}%
\end{pgfscope}%
\begin{pgfscope}%
\pgfpathrectangle{\pgfqpoint{0.539299in}{0.078740in}}{\pgfqpoint{7.842520in}{7.842520in}}%
\pgfusepath{clip}%
\pgfsetbuttcap%
\pgfsetroundjoin%
\definecolor{currentfill}{rgb}{0.123463,0.581687,0.547445}%
\pgfsetfillcolor{currentfill}%
\pgfsetlinewidth{0.000000pt}%
\definecolor{currentstroke}{rgb}{0.304148,0.764704,0.419943}%
\pgfsetstrokecolor{currentstroke}%
\pgfsetdash{}{0pt}%
\pgfpathmoveto{\pgfqpoint{4.566539in}{4.326950in}}%
\pgfpathlineto{\pgfqpoint{4.706729in}{4.119068in}}%
\pgfpathlineto{\pgfqpoint{4.786562in}{4.030722in}}%
\pgfpathclose%
\pgfusepath{fill}%
\end{pgfscope}%
\begin{pgfscope}%
\pgfpathrectangle{\pgfqpoint{0.539299in}{0.078740in}}{\pgfqpoint{7.842520in}{7.842520in}}%
\pgfusepath{clip}%
\pgfsetbuttcap%
\pgfsetroundjoin%
\definecolor{currentfill}{rgb}{0.281887,0.150881,0.465405}%
\pgfsetfillcolor{currentfill}%
\pgfsetlinewidth{0.000000pt}%
\definecolor{currentstroke}{rgb}{0.311925,0.767822,0.415586}%
\pgfsetstrokecolor{currentstroke}%
\pgfsetdash{}{0pt}%
\pgfpathmoveto{\pgfqpoint{6.248772in}{2.336093in}}%
\pgfpathlineto{\pgfqpoint{6.183670in}{2.518396in}}%
\pgfpathlineto{\pgfqpoint{6.107758in}{2.459413in}}%
\pgfpathclose%
\pgfusepath{fill}%
\end{pgfscope}%
\begin{pgfscope}%
\pgfpathrectangle{\pgfqpoint{0.539299in}{0.078740in}}{\pgfqpoint{7.842520in}{7.842520in}}%
\pgfusepath{clip}%
\pgfsetbuttcap%
\pgfsetroundjoin%
\definecolor{currentfill}{rgb}{0.214000,0.722114,0.469588}%
\pgfsetfillcolor{currentfill}%
\pgfsetlinewidth{0.000000pt}%
\definecolor{currentstroke}{rgb}{0.319809,0.770914,0.411152}%
\pgfsetstrokecolor{currentstroke}%
\pgfsetdash{}{0pt}%
\pgfpathmoveto{\pgfqpoint{4.145020in}{4.933425in}}%
\pgfpathlineto{\pgfqpoint{4.367781in}{4.630418in}}%
\pgfpathlineto{\pgfqpoint{4.227924in}{4.824025in}}%
\pgfpathclose%
\pgfusepath{fill}%
\end{pgfscope}%
\begin{pgfscope}%
\pgfpathrectangle{\pgfqpoint{0.539299in}{0.078740in}}{\pgfqpoint{7.842520in}{7.842520in}}%
\pgfusepath{clip}%
\pgfsetbuttcap%
\pgfsetroundjoin%
\definecolor{currentfill}{rgb}{0.449368,0.813768,0.335384}%
\pgfsetfillcolor{currentfill}%
\pgfsetlinewidth{0.000000pt}%
\definecolor{currentstroke}{rgb}{0.327796,0.773980,0.406640}%
\pgfsetstrokecolor{currentstroke}%
\pgfsetdash{}{0pt}%
\pgfpathmoveto{\pgfqpoint{3.723815in}{5.392306in}}%
\pgfpathlineto{\pgfqpoint{3.948223in}{5.166109in}}%
\pgfpathlineto{\pgfqpoint{3.808702in}{5.299454in}}%
\pgfpathclose%
\pgfusepath{fill}%
\end{pgfscope}%
\begin{pgfscope}%
\pgfpathrectangle{\pgfqpoint{0.539299in}{0.078740in}}{\pgfqpoint{7.842520in}{7.842520in}}%
\pgfusepath{clip}%
\pgfsetbuttcap%
\pgfsetroundjoin%
\definecolor{currentfill}{rgb}{0.278826,0.175490,0.483397}%
\pgfsetfillcolor{currentfill}%
\pgfsetlinewidth{0.000000pt}%
\definecolor{currentstroke}{rgb}{0.335885,0.777018,0.402049}%
\pgfsetstrokecolor{currentstroke}%
\pgfsetdash{}{0pt}%
\pgfpathmoveto{\pgfqpoint{1.468606in}{2.671104in}}%
\pgfpathlineto{\pgfqpoint{1.344269in}{2.477731in}}%
\pgfpathlineto{\pgfqpoint{1.388975in}{2.359780in}}%
\pgfpathclose%
\pgfusepath{fill}%
\end{pgfscope}%
\begin{pgfscope}%
\pgfpathrectangle{\pgfqpoint{0.539299in}{0.078740in}}{\pgfqpoint{7.842520in}{7.842520in}}%
\pgfusepath{clip}%
\pgfsetbuttcap%
\pgfsetroundjoin%
\definecolor{currentfill}{rgb}{0.296479,0.761561,0.424223}%
\pgfsetfillcolor{currentfill}%
\pgfsetlinewidth{0.000000pt}%
\definecolor{currentstroke}{rgb}{0.344074,0.780029,0.397381}%
\pgfsetstrokecolor{currentstroke}%
\pgfsetdash{}{0pt}%
\pgfpathmoveto{\pgfqpoint{4.227924in}{4.824025in}}%
\pgfpathlineto{\pgfqpoint{4.088028in}{5.004974in}}%
\pgfpathlineto{\pgfqpoint{4.004391in}{5.111952in}}%
\pgfpathclose%
\pgfusepath{fill}%
\end{pgfscope}%
\begin{pgfscope}%
\pgfpathrectangle{\pgfqpoint{0.539299in}{0.078740in}}{\pgfqpoint{7.842520in}{7.842520in}}%
\pgfusepath{clip}%
\pgfsetbuttcap%
\pgfsetroundjoin%
\definecolor{currentfill}{rgb}{0.266580,0.228262,0.514349}%
\pgfsetfillcolor{currentfill}%
\pgfsetlinewidth{0.000000pt}%
\definecolor{currentstroke}{rgb}{0.352360,0.783011,0.392636}%
\pgfsetstrokecolor{currentstroke}%
\pgfsetdash{}{0pt}%
\pgfpathmoveto{\pgfqpoint{1.550510in}{2.929884in}}%
\pgfpathlineto{\pgfqpoint{1.344269in}{2.477731in}}%
\pgfpathlineto{\pgfqpoint{1.468606in}{2.671104in}}%
\pgfpathclose%
\pgfusepath{fill}%
\end{pgfscope}%
\begin{pgfscope}%
\pgfpathrectangle{\pgfqpoint{0.539299in}{0.078740in}}{\pgfqpoint{7.842520in}{7.842520in}}%
\pgfusepath{clip}%
\pgfsetbuttcap%
\pgfsetroundjoin%
\definecolor{currentfill}{rgb}{0.185783,0.704891,0.485273}%
\pgfsetfillcolor{currentfill}%
\pgfsetlinewidth{0.000000pt}%
\definecolor{currentstroke}{rgb}{0.360741,0.785964,0.387814}%
\pgfsetstrokecolor{currentstroke}%
\pgfsetdash{}{0pt}%
\pgfpathmoveto{\pgfqpoint{2.354275in}{4.486045in}}%
\pgfpathlineto{\pgfqpoint{2.391827in}{4.772531in}}%
\pgfpathlineto{\pgfqpoint{2.478591in}{4.855876in}}%
\pgfpathclose%
\pgfusepath{fill}%
\end{pgfscope}%
\begin{pgfscope}%
\pgfpathrectangle{\pgfqpoint{0.539299in}{0.078740in}}{\pgfqpoint{7.842520in}{7.842520in}}%
\pgfusepath{clip}%
\pgfsetbuttcap%
\pgfsetroundjoin%
\definecolor{currentfill}{rgb}{0.352360,0.783011,0.392636}%
\pgfsetfillcolor{currentfill}%
\pgfsetlinewidth{0.000000pt}%
\definecolor{currentstroke}{rgb}{0.369214,0.788888,0.382914}%
\pgfsetstrokecolor{currentstroke}%
\pgfsetdash{}{0pt}%
\pgfpathmoveto{\pgfqpoint{4.088028in}{5.004974in}}%
\pgfpathlineto{\pgfqpoint{3.948223in}{5.166109in}}%
\pgfpathlineto{\pgfqpoint{4.004391in}{5.111952in}}%
\pgfpathclose%
\pgfusepath{fill}%
\end{pgfscope}%
\begin{pgfscope}%
\pgfpathrectangle{\pgfqpoint{0.539299in}{0.078740in}}{\pgfqpoint{7.842520in}{7.842520in}}%
\pgfusepath{clip}%
\pgfsetbuttcap%
\pgfsetroundjoin%
\definecolor{currentfill}{rgb}{0.280894,0.078907,0.402329}%
\pgfsetfillcolor{currentfill}%
\pgfsetlinewidth{0.000000pt}%
\definecolor{currentstroke}{rgb}{0.377779,0.791781,0.377939}%
\pgfsetstrokecolor{currentstroke}%
\pgfsetdash{}{0pt}%
\pgfpathmoveto{\pgfqpoint{6.465616in}{2.293258in}}%
\pgfpathlineto{\pgfqpoint{6.531589in}{2.101217in}}%
\pgfpathlineto{\pgfqpoint{6.606910in}{2.181621in}}%
\pgfpathclose%
\pgfusepath{fill}%
\end{pgfscope}%
\begin{pgfscope}%
\pgfpathrectangle{\pgfqpoint{0.539299in}{0.078740in}}{\pgfqpoint{7.842520in}{7.842520in}}%
\pgfusepath{clip}%
\pgfsetbuttcap%
\pgfsetroundjoin%
\definecolor{currentfill}{rgb}{0.121380,0.629492,0.531973}%
\pgfsetfillcolor{currentfill}%
\pgfsetlinewidth{0.000000pt}%
\definecolor{currentstroke}{rgb}{0.386433,0.794644,0.372886}%
\pgfsetstrokecolor{currentstroke}%
\pgfsetdash{}{0pt}%
\pgfpathmoveto{\pgfqpoint{4.566539in}{4.326950in}}%
\pgfpathlineto{\pgfqpoint{4.647114in}{4.229077in}}%
\pgfpathlineto{\pgfqpoint{4.426173in}{4.535091in}}%
\pgfpathclose%
\pgfusepath{fill}%
\end{pgfscope}%
\begin{pgfscope}%
\pgfpathrectangle{\pgfqpoint{0.539299in}{0.078740in}}{\pgfqpoint{7.842520in}{7.842520in}}%
\pgfusepath{clip}%
\pgfsetbuttcap%
\pgfsetroundjoin%
\definecolor{currentfill}{rgb}{0.134692,0.658636,0.517649}%
\pgfsetfillcolor{currentfill}%
\pgfsetlinewidth{0.000000pt}%
\definecolor{currentstroke}{rgb}{0.395174,0.797475,0.367757}%
\pgfsetstrokecolor{currentstroke}%
\pgfsetdash{}{0pt}%
\pgfpathmoveto{\pgfqpoint{2.181328in}{4.304148in}}%
\pgfpathlineto{\pgfqpoint{2.391827in}{4.772531in}}%
\pgfpathlineto{\pgfqpoint{2.267756in}{4.403873in}}%
\pgfpathclose%
\pgfusepath{fill}%
\end{pgfscope}%
\begin{pgfscope}%
\pgfpathrectangle{\pgfqpoint{0.539299in}{0.078740in}}{\pgfqpoint{7.842520in}{7.842520in}}%
\pgfusepath{clip}%
\pgfsetbuttcap%
\pgfsetroundjoin%
\definecolor{currentfill}{rgb}{0.136408,0.541173,0.554483}%
\pgfsetfillcolor{currentfill}%
\pgfsetlinewidth{0.000000pt}%
\definecolor{currentstroke}{rgb}{0.404001,0.800275,0.362552}%
\pgfsetstrokecolor{currentstroke}%
\pgfsetdash{}{0pt}%
\pgfpathmoveto{\pgfqpoint{1.889628in}{3.636546in}}%
\pgfpathlineto{\pgfqpoint{2.009371in}{4.040892in}}%
\pgfpathlineto{\pgfqpoint{2.095141in}{4.184186in}}%
\pgfpathclose%
\pgfusepath{fill}%
\end{pgfscope}%
\begin{pgfscope}%
\pgfpathrectangle{\pgfqpoint{0.539299in}{0.078740in}}{\pgfqpoint{7.842520in}{7.842520in}}%
\pgfusepath{clip}%
\pgfsetbuttcap%
\pgfsetroundjoin%
\definecolor{currentfill}{rgb}{0.283091,0.110553,0.431554}%
\pgfsetfillcolor{currentfill}%
\pgfsetlinewidth{0.000000pt}%
\definecolor{currentstroke}{rgb}{0.412913,0.803041,0.357269}%
\pgfsetstrokecolor{currentstroke}%
\pgfsetdash{}{0pt}%
\pgfpathmoveto{\pgfqpoint{6.465616in}{2.293258in}}%
\pgfpathlineto{\pgfqpoint{6.248772in}{2.336093in}}%
\pgfpathlineto{\pgfqpoint{6.390049in}{2.217039in}}%
\pgfpathclose%
\pgfusepath{fill}%
\end{pgfscope}%
\begin{pgfscope}%
\pgfpathrectangle{\pgfqpoint{0.539299in}{0.078740in}}{\pgfqpoint{7.842520in}{7.842520in}}%
\pgfusepath{clip}%
\pgfsetbuttcap%
\pgfsetroundjoin%
\definecolor{currentfill}{rgb}{0.377779,0.791781,0.377939}%
\pgfsetfillcolor{currentfill}%
\pgfsetlinewidth{0.000000pt}%
\definecolor{currentstroke}{rgb}{0.421908,0.805774,0.351910}%
\pgfsetstrokecolor{currentstroke}%
\pgfsetdash{}{0pt}%
\pgfpathmoveto{\pgfqpoint{2.781099in}{5.242417in}}%
\pgfpathlineto{\pgfqpoint{2.652170in}{4.965747in}}%
\pgfpathlineto{\pgfqpoint{2.694035in}{5.204781in}}%
\pgfpathclose%
\pgfusepath{fill}%
\end{pgfscope}%
\begin{pgfscope}%
\pgfpathrectangle{\pgfqpoint{0.539299in}{0.078740in}}{\pgfqpoint{7.842520in}{7.842520in}}%
\pgfusepath{clip}%
\pgfsetbuttcap%
\pgfsetroundjoin%
\definecolor{currentfill}{rgb}{0.241237,0.296485,0.539709}%
\pgfsetfillcolor{currentfill}%
\pgfsetlinewidth{0.000000pt}%
\definecolor{currentstroke}{rgb}{0.430983,0.808473,0.346476}%
\pgfsetstrokecolor{currentstroke}%
\pgfsetdash{}{0pt}%
\pgfpathmoveto{\pgfqpoint{5.686202in}{2.865893in}}%
\pgfpathlineto{\pgfqpoint{5.546103in}{3.017088in}}%
\pgfpathlineto{\pgfqpoint{5.609380in}{2.861680in}}%
\pgfpathclose%
\pgfusepath{fill}%
\end{pgfscope}%
\begin{pgfscope}%
\pgfpathrectangle{\pgfqpoint{0.539299in}{0.078740in}}{\pgfqpoint{7.842520in}{7.842520in}}%
\pgfusepath{clip}%
\pgfsetbuttcap%
\pgfsetroundjoin%
\definecolor{currentfill}{rgb}{0.277941,0.056324,0.381191}%
\pgfsetfillcolor{currentfill}%
\pgfsetlinewidth{0.000000pt}%
\definecolor{currentstroke}{rgb}{0.440137,0.811138,0.340967}%
\pgfsetstrokecolor{currentstroke}%
\pgfsetdash{}{0pt}%
\pgfpathmoveto{\pgfqpoint{6.606910in}{2.181621in}}%
\pgfpathlineto{\pgfqpoint{6.531589in}{2.101217in}}%
\pgfpathlineto{\pgfqpoint{6.748358in}{2.068409in}}%
\pgfpathclose%
\pgfusepath{fill}%
\end{pgfscope}%
\begin{pgfscope}%
\pgfpathrectangle{\pgfqpoint{0.539299in}{0.078740in}}{\pgfqpoint{7.842520in}{7.842520in}}%
\pgfusepath{clip}%
\pgfsetbuttcap%
\pgfsetroundjoin%
\definecolor{currentfill}{rgb}{0.216210,0.351535,0.550627}%
\pgfsetfillcolor{currentfill}%
\pgfsetlinewidth{0.000000pt}%
\definecolor{currentstroke}{rgb}{0.449368,0.813768,0.335384}%
\pgfsetstrokecolor{currentstroke}%
\pgfsetdash{}{0pt}%
\pgfpathmoveto{\pgfqpoint{5.546103in}{3.017088in}}%
\pgfpathlineto{\pgfqpoint{5.406152in}{3.177575in}}%
\pgfpathlineto{\pgfqpoint{5.328594in}{3.205423in}}%
\pgfpathclose%
\pgfusepath{fill}%
\end{pgfscope}%
\begin{pgfscope}%
\pgfpathrectangle{\pgfqpoint{0.539299in}{0.078740in}}{\pgfqpoint{7.842520in}{7.842520in}}%
\pgfusepath{clip}%
\pgfsetbuttcap%
\pgfsetroundjoin%
\definecolor{currentfill}{rgb}{0.203063,0.379716,0.553925}%
\pgfsetfillcolor{currentfill}%
\pgfsetlinewidth{0.000000pt}%
\definecolor{currentstroke}{rgb}{0.458674,0.816363,0.329727}%
\pgfsetstrokecolor{currentstroke}%
\pgfsetdash{}{0pt}%
\pgfpathmoveto{\pgfqpoint{5.406152in}{3.177575in}}%
\pgfpathlineto{\pgfqpoint{5.266300in}{3.347822in}}%
\pgfpathlineto{\pgfqpoint{5.328594in}{3.205423in}}%
\pgfpathclose%
\pgfusepath{fill}%
\end{pgfscope}%
\begin{pgfscope}%
\pgfpathrectangle{\pgfqpoint{0.539299in}{0.078740in}}{\pgfqpoint{7.842520in}{7.842520in}}%
\pgfusepath{clip}%
\pgfsetbuttcap%
\pgfsetroundjoin%
\definecolor{currentfill}{rgb}{0.252194,0.269783,0.531579}%
\pgfsetfillcolor{currentfill}%
\pgfsetlinewidth{0.000000pt}%
\definecolor{currentstroke}{rgb}{0.468053,0.818921,0.323998}%
\pgfsetstrokecolor{currentstroke}%
\pgfsetdash{}{0pt}%
\pgfpathmoveto{\pgfqpoint{5.609380in}{2.861680in}}%
\pgfpathlineto{\pgfqpoint{5.826491in}{2.723216in}}%
\pgfpathlineto{\pgfqpoint{5.686202in}{2.865893in}}%
\pgfpathclose%
\pgfusepath{fill}%
\end{pgfscope}%
\begin{pgfscope}%
\pgfpathrectangle{\pgfqpoint{0.539299in}{0.078740in}}{\pgfqpoint{7.842520in}{7.842520in}}%
\pgfusepath{clip}%
\pgfsetbuttcap%
\pgfsetroundjoin%
\definecolor{currentfill}{rgb}{0.281924,0.089666,0.412415}%
\pgfsetfillcolor{currentfill}%
\pgfsetlinewidth{0.000000pt}%
\definecolor{currentstroke}{rgb}{0.477504,0.821444,0.318195}%
\pgfsetstrokecolor{currentstroke}%
\pgfsetdash{}{0pt}%
\pgfpathmoveto{\pgfqpoint{6.390049in}{2.217039in}}%
\pgfpathlineto{\pgfqpoint{6.531589in}{2.101217in}}%
\pgfpathlineto{\pgfqpoint{6.465616in}{2.293258in}}%
\pgfpathclose%
\pgfusepath{fill}%
\end{pgfscope}%
\begin{pgfscope}%
\pgfpathrectangle{\pgfqpoint{0.539299in}{0.078740in}}{\pgfqpoint{7.842520in}{7.842520in}}%
\pgfusepath{clip}%
\pgfsetbuttcap%
\pgfsetroundjoin%
\definecolor{currentfill}{rgb}{0.183898,0.422383,0.556944}%
\pgfsetfillcolor{currentfill}%
\pgfsetlinewidth{0.000000pt}%
\definecolor{currentstroke}{rgb}{0.487026,0.823929,0.312321}%
\pgfsetstrokecolor{currentstroke}%
\pgfsetdash{}{0pt}%
\pgfpathmoveto{\pgfqpoint{5.188263in}{3.391183in}}%
\pgfpathlineto{\pgfqpoint{5.266300in}{3.347822in}}%
\pgfpathlineto{\pgfqpoint{5.126490in}{3.527887in}}%
\pgfpathclose%
\pgfusepath{fill}%
\end{pgfscope}%
\begin{pgfscope}%
\pgfpathrectangle{\pgfqpoint{0.539299in}{0.078740in}}{\pgfqpoint{7.842520in}{7.842520in}}%
\pgfusepath{clip}%
\pgfsetbuttcap%
\pgfsetroundjoin%
\definecolor{currentfill}{rgb}{0.525776,0.833491,0.288127}%
\pgfsetfillcolor{currentfill}%
\pgfsetlinewidth{0.000000pt}%
\definecolor{currentstroke}{rgb}{0.496615,0.826376,0.306377}%
\pgfsetstrokecolor{currentstroke}%
\pgfsetdash{}{0pt}%
\pgfpathmoveto{\pgfqpoint{3.584365in}{5.478002in}}%
\pgfpathlineto{\pgfqpoint{3.723815in}{5.392306in}}%
\pgfpathlineto{\pgfqpoint{3.808702in}{5.299454in}}%
\pgfpathclose%
\pgfusepath{fill}%
\end{pgfscope}%
\begin{pgfscope}%
\pgfpathrectangle{\pgfqpoint{0.539299in}{0.078740in}}{\pgfqpoint{7.842520in}{7.842520in}}%
\pgfusepath{clip}%
\pgfsetbuttcap%
\pgfsetroundjoin%
\definecolor{currentfill}{rgb}{0.170948,0.694384,0.493803}%
\pgfsetfillcolor{currentfill}%
\pgfsetlinewidth{0.000000pt}%
\definecolor{currentstroke}{rgb}{0.506271,0.828786,0.300362}%
\pgfsetstrokecolor{currentstroke}%
\pgfsetdash{}{0pt}%
\pgfpathmoveto{\pgfqpoint{4.426173in}{4.535091in}}%
\pgfpathlineto{\pgfqpoint{4.367781in}{4.630418in}}%
\pgfpathlineto{\pgfqpoint{4.285648in}{4.739044in}}%
\pgfpathclose%
\pgfusepath{fill}%
\end{pgfscope}%
\begin{pgfscope}%
\pgfpathrectangle{\pgfqpoint{0.539299in}{0.078740in}}{\pgfqpoint{7.842520in}{7.842520in}}%
\pgfusepath{clip}%
\pgfsetbuttcap%
\pgfsetroundjoin%
\definecolor{currentfill}{rgb}{0.606045,0.850733,0.236712}%
\pgfsetfillcolor{currentfill}%
\pgfsetlinewidth{0.000000pt}%
\definecolor{currentstroke}{rgb}{0.515992,0.831158,0.294279}%
\pgfsetstrokecolor{currentstroke}%
\pgfsetdash{}{0pt}%
\pgfpathmoveto{\pgfqpoint{3.445912in}{5.515993in}}%
\pgfpathlineto{\pgfqpoint{3.308861in}{5.497668in}}%
\pgfpathlineto{\pgfqpoint{3.359631in}{5.571327in}}%
\pgfpathclose%
\pgfusepath{fill}%
\end{pgfscope}%
\begin{pgfscope}%
\pgfpathrectangle{\pgfqpoint{0.539299in}{0.078740in}}{\pgfqpoint{7.842520in}{7.842520in}}%
\pgfusepath{clip}%
\pgfsetbuttcap%
\pgfsetroundjoin%
\definecolor{currentfill}{rgb}{0.265145,0.232956,0.516599}%
\pgfsetfillcolor{currentfill}%
\pgfsetlinewidth{0.000000pt}%
\definecolor{currentstroke}{rgb}{0.525776,0.833491,0.288127}%
\pgfsetstrokecolor{currentstroke}%
\pgfsetdash{}{0pt}%
\pgfpathmoveto{\pgfqpoint{5.826491in}{2.723216in}}%
\pgfpathlineto{\pgfqpoint{5.749939in}{2.703326in}}%
\pgfpathlineto{\pgfqpoint{5.967002in}{2.588084in}}%
\pgfpathclose%
\pgfusepath{fill}%
\end{pgfscope}%
\begin{pgfscope}%
\pgfpathrectangle{\pgfqpoint{0.539299in}{0.078740in}}{\pgfqpoint{7.842520in}{7.842520in}}%
\pgfusepath{clip}%
\pgfsetbuttcap%
\pgfsetroundjoin%
\definecolor{currentfill}{rgb}{0.175841,0.441290,0.557685}%
\pgfsetfillcolor{currentfill}%
\pgfsetlinewidth{0.000000pt}%
\definecolor{currentstroke}{rgb}{0.535621,0.835785,0.281908}%
\pgfsetstrokecolor{currentstroke}%
\pgfsetdash{}{0pt}%
\pgfpathmoveto{\pgfqpoint{1.803826in}{3.497087in}}%
\pgfpathlineto{\pgfqpoint{1.718526in}{3.335879in}}%
\pgfpathlineto{\pgfqpoint{1.839990in}{3.669004in}}%
\pgfpathclose%
\pgfusepath{fill}%
\end{pgfscope}%
\begin{pgfscope}%
\pgfpathrectangle{\pgfqpoint{0.539299in}{0.078740in}}{\pgfqpoint{7.842520in}{7.842520in}}%
\pgfusepath{clip}%
\pgfsetbuttcap%
\pgfsetroundjoin%
\definecolor{currentfill}{rgb}{0.220124,0.725509,0.466226}%
\pgfsetfillcolor{currentfill}%
\pgfsetlinewidth{0.000000pt}%
\definecolor{currentstroke}{rgb}{0.545524,0.838039,0.275626}%
\pgfsetstrokecolor{currentstroke}%
\pgfsetdash{}{0pt}%
\pgfpathmoveto{\pgfqpoint{4.285648in}{4.739044in}}%
\pgfpathlineto{\pgfqpoint{4.367781in}{4.630418in}}%
\pgfpathlineto{\pgfqpoint{4.145020in}{4.933425in}}%
\pgfpathclose%
\pgfusepath{fill}%
\end{pgfscope}%
\begin{pgfscope}%
\pgfpathrectangle{\pgfqpoint{0.539299in}{0.078740in}}{\pgfqpoint{7.842520in}{7.842520in}}%
\pgfusepath{clip}%
\pgfsetbuttcap%
\pgfsetroundjoin%
\definecolor{currentfill}{rgb}{0.278012,0.180367,0.486697}%
\pgfsetfillcolor{currentfill}%
\pgfsetlinewidth{0.000000pt}%
\definecolor{currentstroke}{rgb}{0.555484,0.840254,0.269281}%
\pgfsetstrokecolor{currentstroke}%
\pgfsetdash{}{0pt}%
\pgfpathmoveto{\pgfqpoint{5.967002in}{2.588084in}}%
\pgfpathlineto{\pgfqpoint{6.031600in}{2.411053in}}%
\pgfpathlineto{\pgfqpoint{6.107758in}{2.459413in}}%
\pgfpathclose%
\pgfusepath{fill}%
\end{pgfscope}%
\begin{pgfscope}%
\pgfpathrectangle{\pgfqpoint{0.539299in}{0.078740in}}{\pgfqpoint{7.842520in}{7.842520in}}%
\pgfusepath{clip}%
\pgfsetbuttcap%
\pgfsetroundjoin%
\definecolor{currentfill}{rgb}{0.157729,0.485932,0.558013}%
\pgfsetfillcolor{currentfill}%
\pgfsetlinewidth{0.000000pt}%
\definecolor{currentstroke}{rgb}{0.565498,0.842430,0.262877}%
\pgfsetstrokecolor{currentstroke}%
\pgfsetdash{}{0pt}%
\pgfpathmoveto{\pgfqpoint{5.126490in}{3.527887in}}%
\pgfpathlineto{\pgfqpoint{4.986662in}{3.717303in}}%
\pgfpathlineto{\pgfqpoint{4.907447in}{3.788480in}}%
\pgfpathclose%
\pgfusepath{fill}%
\end{pgfscope}%
\begin{pgfscope}%
\pgfpathrectangle{\pgfqpoint{0.539299in}{0.078740in}}{\pgfqpoint{7.842520in}{7.842520in}}%
\pgfusepath{clip}%
\pgfsetbuttcap%
\pgfsetroundjoin%
\definecolor{currentfill}{rgb}{0.468053,0.818921,0.323998}%
\pgfsetfillcolor{currentfill}%
\pgfsetlinewidth{0.000000pt}%
\definecolor{currentstroke}{rgb}{0.575563,0.844566,0.256415}%
\pgfsetstrokecolor{currentstroke}%
\pgfsetdash{}{0pt}%
\pgfpathmoveto{\pgfqpoint{3.863917in}{5.267515in}}%
\pgfpathlineto{\pgfqpoint{3.948223in}{5.166109in}}%
\pgfpathlineto{\pgfqpoint{3.723815in}{5.392306in}}%
\pgfpathclose%
\pgfusepath{fill}%
\end{pgfscope}%
\begin{pgfscope}%
\pgfpathrectangle{\pgfqpoint{0.539299in}{0.078740in}}{\pgfqpoint{7.842520in}{7.842520in}}%
\pgfusepath{clip}%
\pgfsetbuttcap%
\pgfsetroundjoin%
\definecolor{currentfill}{rgb}{0.626579,0.854645,0.223353}%
\pgfsetfillcolor{currentfill}%
\pgfsetlinewidth{0.000000pt}%
\definecolor{currentstroke}{rgb}{0.585678,0.846661,0.249897}%
\pgfsetstrokecolor{currentstroke}%
\pgfsetdash{}{0pt}%
\pgfpathmoveto{\pgfqpoint{3.359631in}{5.571327in}}%
\pgfpathlineto{\pgfqpoint{3.308861in}{5.497668in}}%
\pgfpathlineto{\pgfqpoint{3.222442in}{5.537820in}}%
\pgfpathclose%
\pgfusepath{fill}%
\end{pgfscope}%
\begin{pgfscope}%
\pgfpathrectangle{\pgfqpoint{0.539299in}{0.078740in}}{\pgfqpoint{7.842520in}{7.842520in}}%
\pgfusepath{clip}%
\pgfsetbuttcap%
\pgfsetroundjoin%
\definecolor{currentfill}{rgb}{0.319809,0.770914,0.411152}%
\pgfsetfillcolor{currentfill}%
\pgfsetlinewidth{0.000000pt}%
\definecolor{currentstroke}{rgb}{0.595839,0.848717,0.243329}%
\pgfsetstrokecolor{currentstroke}%
\pgfsetdash{}{0pt}%
\pgfpathmoveto{\pgfqpoint{2.694035in}{5.204781in}}%
\pgfpathlineto{\pgfqpoint{2.565417in}{4.919366in}}%
\pgfpathlineto{\pgfqpoint{2.478591in}{4.855876in}}%
\pgfpathclose%
\pgfusepath{fill}%
\end{pgfscope}%
\begin{pgfscope}%
\pgfpathrectangle{\pgfqpoint{0.539299in}{0.078740in}}{\pgfqpoint{7.842520in}{7.842520in}}%
\pgfusepath{clip}%
\pgfsetbuttcap%
\pgfsetroundjoin%
\definecolor{currentfill}{rgb}{0.304148,0.764704,0.419943}%
\pgfsetfillcolor{currentfill}%
\pgfsetlinewidth{0.000000pt}%
\definecolor{currentstroke}{rgb}{0.606045,0.850733,0.236712}%
\pgfsetstrokecolor{currentstroke}%
\pgfsetdash{}{0pt}%
\pgfpathmoveto{\pgfqpoint{4.004391in}{5.111952in}}%
\pgfpathlineto{\pgfqpoint{4.145020in}{4.933425in}}%
\pgfpathlineto{\pgfqpoint{4.227924in}{4.824025in}}%
\pgfpathclose%
\pgfusepath{fill}%
\end{pgfscope}%
\begin{pgfscope}%
\pgfpathrectangle{\pgfqpoint{0.539299in}{0.078740in}}{\pgfqpoint{7.842520in}{7.842520in}}%
\pgfusepath{clip}%
\pgfsetbuttcap%
\pgfsetroundjoin%
\definecolor{currentfill}{rgb}{0.281412,0.155834,0.469201}%
\pgfsetfillcolor{currentfill}%
\pgfsetlinewidth{0.000000pt}%
\definecolor{currentstroke}{rgb}{0.616293,0.852709,0.230052}%
\pgfsetstrokecolor{currentstroke}%
\pgfsetdash{}{0pt}%
\pgfpathmoveto{\pgfqpoint{6.107758in}{2.459413in}}%
\pgfpathlineto{\pgfqpoint{6.031600in}{2.411053in}}%
\pgfpathlineto{\pgfqpoint{6.248772in}{2.336093in}}%
\pgfpathclose%
\pgfusepath{fill}%
\end{pgfscope}%
\begin{pgfscope}%
\pgfpathrectangle{\pgfqpoint{0.539299in}{0.078740in}}{\pgfqpoint{7.842520in}{7.842520in}}%
\pgfusepath{clip}%
\pgfsetbuttcap%
\pgfsetroundjoin%
\definecolor{currentfill}{rgb}{0.146180,0.515413,0.556823}%
\pgfsetfillcolor{currentfill}%
\pgfsetlinewidth{0.000000pt}%
\definecolor{currentstroke}{rgb}{0.626579,0.854645,0.223353}%
\pgfsetstrokecolor{currentstroke}%
\pgfsetdash{}{0pt}%
\pgfpathmoveto{\pgfqpoint{4.907447in}{3.788480in}}%
\pgfpathlineto{\pgfqpoint{4.986662in}{3.717303in}}%
\pgfpathlineto{\pgfqpoint{4.846758in}{3.914972in}}%
\pgfpathclose%
\pgfusepath{fill}%
\end{pgfscope}%
\begin{pgfscope}%
\pgfpathrectangle{\pgfqpoint{0.539299in}{0.078740in}}{\pgfqpoint{7.842520in}{7.842520in}}%
\pgfusepath{clip}%
\pgfsetbuttcap%
\pgfsetroundjoin%
\definecolor{currentfill}{rgb}{0.421908,0.805774,0.351910}%
\pgfsetfillcolor{currentfill}%
\pgfsetlinewidth{0.000000pt}%
\definecolor{currentstroke}{rgb}{0.636902,0.856542,0.216620}%
\pgfsetstrokecolor{currentstroke}%
\pgfsetdash{}{0pt}%
\pgfpathmoveto{\pgfqpoint{4.004391in}{5.111952in}}%
\pgfpathlineto{\pgfqpoint{3.948223in}{5.166109in}}%
\pgfpathlineto{\pgfqpoint{3.863917in}{5.267515in}}%
\pgfpathclose%
\pgfusepath{fill}%
\end{pgfscope}%
\begin{pgfscope}%
\pgfpathrectangle{\pgfqpoint{0.539299in}{0.078740in}}{\pgfqpoint{7.842520in}{7.842520in}}%
\pgfusepath{clip}%
\pgfsetbuttcap%
\pgfsetroundjoin%
\definecolor{currentfill}{rgb}{0.276022,0.044167,0.370164}%
\pgfsetfillcolor{currentfill}%
\pgfsetlinewidth{0.000000pt}%
\definecolor{currentstroke}{rgb}{0.647257,0.858400,0.209861}%
\pgfsetstrokecolor{currentstroke}%
\pgfsetdash{}{0pt}%
\pgfpathmoveto{\pgfqpoint{6.748358in}{2.068409in}}%
\pgfpathlineto{\pgfqpoint{6.531589in}{2.101217in}}%
\pgfpathlineto{\pgfqpoint{6.673382in}{1.987622in}}%
\pgfpathclose%
\pgfusepath{fill}%
\end{pgfscope}%
\begin{pgfscope}%
\pgfpathrectangle{\pgfqpoint{0.539299in}{0.078740in}}{\pgfqpoint{7.842520in}{7.842520in}}%
\pgfusepath{clip}%
\pgfsetbuttcap%
\pgfsetroundjoin%
\definecolor{currentfill}{rgb}{0.496615,0.826376,0.306377}%
\pgfsetfillcolor{currentfill}%
\pgfsetlinewidth{0.000000pt}%
\definecolor{currentstroke}{rgb}{0.657642,0.860219,0.203082}%
\pgfsetstrokecolor{currentstroke}%
\pgfsetdash{}{0pt}%
\pgfpathmoveto{\pgfqpoint{2.913349in}{5.441460in}}%
\pgfpathlineto{\pgfqpoint{2.867996in}{5.262317in}}%
\pgfpathlineto{\pgfqpoint{2.781099in}{5.242417in}}%
\pgfpathclose%
\pgfusepath{fill}%
\end{pgfscope}%
\begin{pgfscope}%
\pgfpathrectangle{\pgfqpoint{0.539299in}{0.078740in}}{\pgfqpoint{7.842520in}{7.842520in}}%
\pgfusepath{clip}%
\pgfsetbuttcap%
\pgfsetroundjoin%
\definecolor{currentfill}{rgb}{0.231674,0.318106,0.544834}%
\pgfsetfillcolor{currentfill}%
\pgfsetlinewidth{0.000000pt}%
\definecolor{currentstroke}{rgb}{0.668054,0.861999,0.196293}%
\pgfsetstrokecolor{currentstroke}%
\pgfsetdash{}{0pt}%
\pgfpathmoveto{\pgfqpoint{5.609380in}{2.861680in}}%
\pgfpathlineto{\pgfqpoint{5.546103in}{3.017088in}}%
\pgfpathlineto{\pgfqpoint{5.468948in}{3.028932in}}%
\pgfpathclose%
\pgfusepath{fill}%
\end{pgfscope}%
\begin{pgfscope}%
\pgfpathrectangle{\pgfqpoint{0.539299in}{0.078740in}}{\pgfqpoint{7.842520in}{7.842520in}}%
\pgfusepath{clip}%
\pgfsetbuttcap%
\pgfsetroundjoin%
\definecolor{currentfill}{rgb}{0.218130,0.347432,0.550038}%
\pgfsetfillcolor{currentfill}%
\pgfsetlinewidth{0.000000pt}%
\definecolor{currentstroke}{rgb}{0.678489,0.863742,0.189503}%
\pgfsetstrokecolor{currentstroke}%
\pgfsetdash{}{0pt}%
\pgfpathmoveto{\pgfqpoint{5.468948in}{3.028932in}}%
\pgfpathlineto{\pgfqpoint{5.546103in}{3.017088in}}%
\pgfpathlineto{\pgfqpoint{5.328594in}{3.205423in}}%
\pgfpathclose%
\pgfusepath{fill}%
\end{pgfscope}%
\begin{pgfscope}%
\pgfpathrectangle{\pgfqpoint{0.539299in}{0.078740in}}{\pgfqpoint{7.842520in}{7.842520in}}%
\pgfusepath{clip}%
\pgfsetbuttcap%
\pgfsetroundjoin%
\definecolor{currentfill}{rgb}{0.253935,0.265254,0.529983}%
\pgfsetfillcolor{currentfill}%
\pgfsetlinewidth{0.000000pt}%
\definecolor{currentstroke}{rgb}{0.688944,0.865448,0.182725}%
\pgfsetstrokecolor{currentstroke}%
\pgfsetdash{}{0pt}%
\pgfpathmoveto{\pgfqpoint{5.609380in}{2.861680in}}%
\pgfpathlineto{\pgfqpoint{5.749939in}{2.703326in}}%
\pgfpathlineto{\pgfqpoint{5.826491in}{2.723216in}}%
\pgfpathclose%
\pgfusepath{fill}%
\end{pgfscope}%
\begin{pgfscope}%
\pgfpathrectangle{\pgfqpoint{0.539299in}{0.078740in}}{\pgfqpoint{7.842520in}{7.842520in}}%
\pgfusepath{clip}%
\pgfsetbuttcap%
\pgfsetroundjoin%
\definecolor{currentfill}{rgb}{0.146180,0.515413,0.556823}%
\pgfsetfillcolor{currentfill}%
\pgfsetlinewidth{0.000000pt}%
\definecolor{currentstroke}{rgb}{0.699415,0.867117,0.175971}%
\pgfsetstrokecolor{currentstroke}%
\pgfsetdash{}{0pt}%
\pgfpathmoveto{\pgfqpoint{1.803826in}{3.497087in}}%
\pgfpathlineto{\pgfqpoint{1.924234in}{3.870625in}}%
\pgfpathlineto{\pgfqpoint{2.009371in}{4.040892in}}%
\pgfpathclose%
\pgfusepath{fill}%
\end{pgfscope}%
\begin{pgfscope}%
\pgfpathrectangle{\pgfqpoint{0.539299in}{0.078740in}}{\pgfqpoint{7.842520in}{7.842520in}}%
\pgfusepath{clip}%
\pgfsetbuttcap%
\pgfsetroundjoin%
\definecolor{currentfill}{rgb}{0.535621,0.835785,0.281908}%
\pgfsetfillcolor{currentfill}%
\pgfsetlinewidth{0.000000pt}%
\definecolor{currentstroke}{rgb}{0.709898,0.868751,0.169257}%
\pgfsetstrokecolor{currentstroke}%
\pgfsetdash{}{0pt}%
\pgfpathmoveto{\pgfqpoint{3.000456in}{5.448722in}}%
\pgfpathlineto{\pgfqpoint{2.867996in}{5.262317in}}%
\pgfpathlineto{\pgfqpoint{2.913349in}{5.441460in}}%
\pgfpathclose%
\pgfusepath{fill}%
\end{pgfscope}%
\begin{pgfscope}%
\pgfpathrectangle{\pgfqpoint{0.539299in}{0.078740in}}{\pgfqpoint{7.842520in}{7.842520in}}%
\pgfusepath{clip}%
\pgfsetbuttcap%
\pgfsetroundjoin%
\definecolor{currentfill}{rgb}{0.192357,0.403199,0.555836}%
\pgfsetfillcolor{currentfill}%
\pgfsetlinewidth{0.000000pt}%
\definecolor{currentstroke}{rgb}{0.720391,0.870350,0.162603}%
\pgfsetstrokecolor{currentstroke}%
\pgfsetdash{}{0pt}%
\pgfpathmoveto{\pgfqpoint{1.718526in}{3.335879in}}%
\pgfpathlineto{\pgfqpoint{1.633975in}{3.148721in}}%
\pgfpathlineto{\pgfqpoint{1.756965in}{3.430597in}}%
\pgfpathclose%
\pgfusepath{fill}%
\end{pgfscope}%
\begin{pgfscope}%
\pgfpathrectangle{\pgfqpoint{0.539299in}{0.078740in}}{\pgfqpoint{7.842520in}{7.842520in}}%
\pgfusepath{clip}%
\pgfsetbuttcap%
\pgfsetroundjoin%
\definecolor{currentfill}{rgb}{0.190631,0.407061,0.556089}%
\pgfsetfillcolor{currentfill}%
\pgfsetlinewidth{0.000000pt}%
\definecolor{currentstroke}{rgb}{0.730889,0.871916,0.156029}%
\pgfsetstrokecolor{currentstroke}%
\pgfsetdash{}{0pt}%
\pgfpathmoveto{\pgfqpoint{5.266300in}{3.347822in}}%
\pgfpathlineto{\pgfqpoint{5.188263in}{3.391183in}}%
\pgfpathlineto{\pgfqpoint{5.328594in}{3.205423in}}%
\pgfpathclose%
\pgfusepath{fill}%
\end{pgfscope}%
\begin{pgfscope}%
\pgfpathrectangle{\pgfqpoint{0.539299in}{0.078740in}}{\pgfqpoint{7.842520in}{7.842520in}}%
\pgfusepath{clip}%
\pgfsetbuttcap%
\pgfsetroundjoin%
\definecolor{currentfill}{rgb}{0.267968,0.223549,0.512008}%
\pgfsetfillcolor{currentfill}%
\pgfsetlinewidth{0.000000pt}%
\definecolor{currentstroke}{rgb}{0.741388,0.873449,0.149561}%
\pgfsetstrokecolor{currentstroke}%
\pgfsetdash{}{0pt}%
\pgfpathmoveto{\pgfqpoint{5.749939in}{2.703326in}}%
\pgfpathlineto{\pgfqpoint{5.890666in}{2.553330in}}%
\pgfpathlineto{\pgfqpoint{5.967002in}{2.588084in}}%
\pgfpathclose%
\pgfusepath{fill}%
\end{pgfscope}%
\begin{pgfscope}%
\pgfpathrectangle{\pgfqpoint{0.539299in}{0.078740in}}{\pgfqpoint{7.842520in}{7.842520in}}%
\pgfusepath{clip}%
\pgfsetbuttcap%
\pgfsetroundjoin%
\definecolor{currentfill}{rgb}{0.626579,0.854645,0.223353}%
\pgfsetfillcolor{currentfill}%
\pgfsetlinewidth{0.000000pt}%
\definecolor{currentstroke}{rgb}{0.751884,0.874951,0.143228}%
\pgfsetstrokecolor{currentstroke}%
\pgfsetdash{}{0pt}%
\pgfpathmoveto{\pgfqpoint{3.445912in}{5.515993in}}%
\pgfpathlineto{\pgfqpoint{3.359631in}{5.571327in}}%
\pgfpathlineto{\pgfqpoint{3.584365in}{5.478002in}}%
\pgfpathclose%
\pgfusepath{fill}%
\end{pgfscope}%
\begin{pgfscope}%
\pgfpathrectangle{\pgfqpoint{0.539299in}{0.078740in}}{\pgfqpoint{7.842520in}{7.842520in}}%
\pgfusepath{clip}%
\pgfsetbuttcap%
\pgfsetroundjoin%
\definecolor{currentfill}{rgb}{0.616293,0.852709,0.230052}%
\pgfsetfillcolor{currentfill}%
\pgfsetlinewidth{0.000000pt}%
\definecolor{currentstroke}{rgb}{0.762373,0.876424,0.137064}%
\pgfsetstrokecolor{currentstroke}%
\pgfsetdash{}{0pt}%
\pgfpathmoveto{\pgfqpoint{3.135579in}{5.562556in}}%
\pgfpathlineto{\pgfqpoint{3.222442in}{5.537820in}}%
\pgfpathlineto{\pgfqpoint{3.087266in}{5.438839in}}%
\pgfpathclose%
\pgfusepath{fill}%
\end{pgfscope}%
\begin{pgfscope}%
\pgfpathrectangle{\pgfqpoint{0.539299in}{0.078740in}}{\pgfqpoint{7.842520in}{7.842520in}}%
\pgfusepath{clip}%
\pgfsetbuttcap%
\pgfsetroundjoin%
\definecolor{currentfill}{rgb}{0.275191,0.194905,0.496005}%
\pgfsetfillcolor{currentfill}%
\pgfsetlinewidth{0.000000pt}%
\definecolor{currentstroke}{rgb}{0.772852,0.877868,0.131109}%
\pgfsetstrokecolor{currentstroke}%
\pgfsetdash{}{0pt}%
\pgfpathmoveto{\pgfqpoint{5.890666in}{2.553330in}}%
\pgfpathlineto{\pgfqpoint{6.031600in}{2.411053in}}%
\pgfpathlineto{\pgfqpoint{5.967002in}{2.588084in}}%
\pgfpathclose%
\pgfusepath{fill}%
\end{pgfscope}%
\begin{pgfscope}%
\pgfpathrectangle{\pgfqpoint{0.539299in}{0.078740in}}{\pgfqpoint{7.842520in}{7.842520in}}%
\pgfusepath{clip}%
\pgfsetbuttcap%
\pgfsetroundjoin%
\definecolor{currentfill}{rgb}{0.283091,0.110553,0.431554}%
\pgfsetfillcolor{currentfill}%
\pgfsetlinewidth{0.000000pt}%
\definecolor{currentstroke}{rgb}{0.783315,0.879285,0.125405}%
\pgfsetstrokecolor{currentstroke}%
\pgfsetdash{}{0pt}%
\pgfpathmoveto{\pgfqpoint{6.248772in}{2.336093in}}%
\pgfpathlineto{\pgfqpoint{6.314217in}{2.147221in}}%
\pgfpathlineto{\pgfqpoint{6.390049in}{2.217039in}}%
\pgfpathclose%
\pgfusepath{fill}%
\end{pgfscope}%
\begin{pgfscope}%
\pgfpathrectangle{\pgfqpoint{0.539299in}{0.078740in}}{\pgfqpoint{7.842520in}{7.842520in}}%
\pgfusepath{clip}%
\pgfsetbuttcap%
\pgfsetroundjoin%
\definecolor{currentfill}{rgb}{0.606045,0.850733,0.236712}%
\pgfsetfillcolor{currentfill}%
\pgfsetlinewidth{0.000000pt}%
\definecolor{currentstroke}{rgb}{0.793760,0.880678,0.120005}%
\pgfsetstrokecolor{currentstroke}%
\pgfsetdash{}{0pt}%
\pgfpathmoveto{\pgfqpoint{3.087266in}{5.438839in}}%
\pgfpathlineto{\pgfqpoint{3.000456in}{5.448722in}}%
\pgfpathlineto{\pgfqpoint{3.135579in}{5.562556in}}%
\pgfpathclose%
\pgfusepath{fill}%
\end{pgfscope}%
\begin{pgfscope}%
\pgfpathrectangle{\pgfqpoint{0.539299in}{0.078740in}}{\pgfqpoint{7.842520in}{7.842520in}}%
\pgfusepath{clip}%
\pgfsetbuttcap%
\pgfsetroundjoin%
\definecolor{currentfill}{rgb}{0.123463,0.581687,0.547445}%
\pgfsetfillcolor{currentfill}%
\pgfsetlinewidth{0.000000pt}%
\definecolor{currentstroke}{rgb}{0.804182,0.882046,0.114965}%
\pgfsetstrokecolor{currentstroke}%
\pgfsetdash{}{0pt}%
\pgfpathmoveto{\pgfqpoint{4.846758in}{3.914972in}}%
\pgfpathlineto{\pgfqpoint{4.706729in}{4.119068in}}%
\pgfpathlineto{\pgfqpoint{4.626092in}{4.211058in}}%
\pgfpathclose%
\pgfusepath{fill}%
\end{pgfscope}%
\begin{pgfscope}%
\pgfpathrectangle{\pgfqpoint{0.539299in}{0.078740in}}{\pgfqpoint{7.842520in}{7.842520in}}%
\pgfusepath{clip}%
\pgfsetbuttcap%
\pgfsetroundjoin%
\definecolor{currentfill}{rgb}{0.171176,0.452530,0.557965}%
\pgfsetfillcolor{currentfill}%
\pgfsetlinewidth{0.000000pt}%
\definecolor{currentstroke}{rgb}{0.814576,0.883393,0.110347}%
\pgfsetstrokecolor{currentstroke}%
\pgfsetdash{}{0pt}%
\pgfpathmoveto{\pgfqpoint{5.126490in}{3.527887in}}%
\pgfpathlineto{\pgfqpoint{5.047899in}{3.585834in}}%
\pgfpathlineto{\pgfqpoint{5.188263in}{3.391183in}}%
\pgfpathclose%
\pgfusepath{fill}%
\end{pgfscope}%
\begin{pgfscope}%
\pgfpathrectangle{\pgfqpoint{0.539299in}{0.078740in}}{\pgfqpoint{7.842520in}{7.842520in}}%
\pgfusepath{clip}%
\pgfsetbuttcap%
\pgfsetroundjoin%
\definecolor{currentfill}{rgb}{0.159194,0.482237,0.558073}%
\pgfsetfillcolor{currentfill}%
\pgfsetlinewidth{0.000000pt}%
\definecolor{currentstroke}{rgb}{0.824940,0.884720,0.106217}%
\pgfsetstrokecolor{currentstroke}%
\pgfsetdash{}{0pt}%
\pgfpathmoveto{\pgfqpoint{4.907447in}{3.788480in}}%
\pgfpathlineto{\pgfqpoint{5.047899in}{3.585834in}}%
\pgfpathlineto{\pgfqpoint{5.126490in}{3.527887in}}%
\pgfpathclose%
\pgfusepath{fill}%
\end{pgfscope}%
\begin{pgfscope}%
\pgfpathrectangle{\pgfqpoint{0.539299in}{0.078740in}}{\pgfqpoint{7.842520in}{7.842520in}}%
\pgfusepath{clip}%
\pgfsetbuttcap%
\pgfsetroundjoin%
\definecolor{currentfill}{rgb}{0.204903,0.375746,0.553533}%
\pgfsetfillcolor{currentfill}%
\pgfsetlinewidth{0.000000pt}%
\definecolor{currentstroke}{rgb}{0.835270,0.886029,0.102646}%
\pgfsetstrokecolor{currentstroke}%
\pgfsetdash{}{0pt}%
\pgfpathmoveto{\pgfqpoint{1.756965in}{3.430597in}}%
\pgfpathlineto{\pgfqpoint{1.633975in}{3.148721in}}%
\pgfpathlineto{\pgfqpoint{1.550510in}{2.929884in}}%
\pgfpathclose%
\pgfusepath{fill}%
\end{pgfscope}%
\begin{pgfscope}%
\pgfpathrectangle{\pgfqpoint{0.539299in}{0.078740in}}{\pgfqpoint{7.842520in}{7.842520in}}%
\pgfusepath{clip}%
\pgfsetbuttcap%
\pgfsetroundjoin%
\definecolor{currentfill}{rgb}{0.154815,0.493313,0.557840}%
\pgfsetfillcolor{currentfill}%
\pgfsetlinewidth{0.000000pt}%
\definecolor{currentstroke}{rgb}{0.845561,0.887322,0.099702}%
\pgfsetstrokecolor{currentstroke}%
\pgfsetdash{}{0pt}%
\pgfpathmoveto{\pgfqpoint{1.839990in}{3.669004in}}%
\pgfpathlineto{\pgfqpoint{1.924234in}{3.870625in}}%
\pgfpathlineto{\pgfqpoint{1.803826in}{3.497087in}}%
\pgfpathclose%
\pgfusepath{fill}%
\end{pgfscope}%
\begin{pgfscope}%
\pgfpathrectangle{\pgfqpoint{0.539299in}{0.078740in}}{\pgfqpoint{7.842520in}{7.842520in}}%
\pgfusepath{clip}%
\pgfsetbuttcap%
\pgfsetroundjoin%
\definecolor{currentfill}{rgb}{0.281887,0.150881,0.465405}%
\pgfsetfillcolor{currentfill}%
\pgfsetlinewidth{0.000000pt}%
\definecolor{currentstroke}{rgb}{0.855810,0.888601,0.097452}%
\pgfsetstrokecolor{currentstroke}%
\pgfsetdash{}{0pt}%
\pgfpathmoveto{\pgfqpoint{6.248772in}{2.336093in}}%
\pgfpathlineto{\pgfqpoint{6.031600in}{2.411053in}}%
\pgfpathlineto{\pgfqpoint{6.172774in}{2.275861in}}%
\pgfpathclose%
\pgfusepath{fill}%
\end{pgfscope}%
\begin{pgfscope}%
\pgfpathrectangle{\pgfqpoint{0.539299in}{0.078740in}}{\pgfqpoint{7.842520in}{7.842520in}}%
\pgfusepath{clip}%
\pgfsetbuttcap%
\pgfsetroundjoin%
\definecolor{currentfill}{rgb}{0.266580,0.228262,0.514349}%
\pgfsetfillcolor{currentfill}%
\pgfsetlinewidth{0.000000pt}%
\definecolor{currentstroke}{rgb}{0.866013,0.889868,0.095953}%
\pgfsetstrokecolor{currentstroke}%
\pgfsetdash{}{0pt}%
\pgfpathmoveto{\pgfqpoint{1.468606in}{2.671104in}}%
\pgfpathlineto{\pgfqpoint{1.388975in}{2.359780in}}%
\pgfpathlineto{\pgfqpoint{1.596362in}{2.813436in}}%
\pgfpathclose%
\pgfusepath{fill}%
\end{pgfscope}%
\begin{pgfscope}%
\pgfpathrectangle{\pgfqpoint{0.539299in}{0.078740in}}{\pgfqpoint{7.842520in}{7.842520in}}%
\pgfusepath{clip}%
\pgfsetbuttcap%
\pgfsetroundjoin%
\definecolor{currentfill}{rgb}{0.280894,0.078907,0.402329}%
\pgfsetfillcolor{currentfill}%
\pgfsetlinewidth{0.000000pt}%
\definecolor{currentstroke}{rgb}{0.876168,0.891125,0.095250}%
\pgfsetstrokecolor{currentstroke}%
\pgfsetdash{}{0pt}%
\pgfpathmoveto{\pgfqpoint{6.455962in}{2.024776in}}%
\pgfpathlineto{\pgfqpoint{6.531589in}{2.101217in}}%
\pgfpathlineto{\pgfqpoint{6.390049in}{2.217039in}}%
\pgfpathclose%
\pgfusepath{fill}%
\end{pgfscope}%
\begin{pgfscope}%
\pgfpathrectangle{\pgfqpoint{0.539299in}{0.078740in}}{\pgfqpoint{7.842520in}{7.842520in}}%
\pgfusepath{clip}%
\pgfsetbuttcap%
\pgfsetroundjoin%
\definecolor{currentfill}{rgb}{0.121380,0.629492,0.531973}%
\pgfsetfillcolor{currentfill}%
\pgfsetlinewidth{0.000000pt}%
\definecolor{currentstroke}{rgb}{0.886271,0.892374,0.095374}%
\pgfsetstrokecolor{currentstroke}%
\pgfsetdash{}{0pt}%
\pgfpathmoveto{\pgfqpoint{4.485130in}{4.425835in}}%
\pgfpathlineto{\pgfqpoint{4.706729in}{4.119068in}}%
\pgfpathlineto{\pgfqpoint{4.566539in}{4.326950in}}%
\pgfpathclose%
\pgfusepath{fill}%
\end{pgfscope}%
\begin{pgfscope}%
\pgfpathrectangle{\pgfqpoint{0.539299in}{0.078740in}}{\pgfqpoint{7.842520in}{7.842520in}}%
\pgfusepath{clip}%
\pgfsetbuttcap%
\pgfsetroundjoin%
\definecolor{currentfill}{rgb}{0.124780,0.640461,0.527068}%
\pgfsetfillcolor{currentfill}%
\pgfsetlinewidth{0.000000pt}%
\definecolor{currentstroke}{rgb}{0.896320,0.893616,0.096335}%
\pgfsetstrokecolor{currentstroke}%
\pgfsetdash{}{0pt}%
\pgfpathmoveto{\pgfqpoint{2.095141in}{4.184186in}}%
\pgfpathlineto{\pgfqpoint{2.219156in}{4.533644in}}%
\pgfpathlineto{\pgfqpoint{2.181328in}{4.304148in}}%
\pgfpathclose%
\pgfusepath{fill}%
\end{pgfscope}%
\begin{pgfscope}%
\pgfpathrectangle{\pgfqpoint{0.539299in}{0.078740in}}{\pgfqpoint{7.842520in}{7.842520in}}%
\pgfusepath{clip}%
\pgfsetbuttcap%
\pgfsetroundjoin%
\definecolor{currentfill}{rgb}{0.283187,0.125848,0.444960}%
\pgfsetfillcolor{currentfill}%
\pgfsetlinewidth{0.000000pt}%
\definecolor{currentstroke}{rgb}{0.906311,0.894855,0.098125}%
\pgfsetstrokecolor{currentstroke}%
\pgfsetdash{}{0pt}%
\pgfpathmoveto{\pgfqpoint{6.172774in}{2.275861in}}%
\pgfpathlineto{\pgfqpoint{6.314217in}{2.147221in}}%
\pgfpathlineto{\pgfqpoint{6.248772in}{2.336093in}}%
\pgfpathclose%
\pgfusepath{fill}%
\end{pgfscope}%
\begin{pgfscope}%
\pgfpathrectangle{\pgfqpoint{0.539299in}{0.078740in}}{\pgfqpoint{7.842520in}{7.842520in}}%
\pgfusepath{clip}%
\pgfsetbuttcap%
\pgfsetroundjoin%
\definecolor{currentfill}{rgb}{0.233603,0.313828,0.543914}%
\pgfsetfillcolor{currentfill}%
\pgfsetlinewidth{0.000000pt}%
\definecolor{currentstroke}{rgb}{0.916242,0.896091,0.100717}%
\pgfsetstrokecolor{currentstroke}%
\pgfsetdash{}{0pt}%
\pgfpathmoveto{\pgfqpoint{1.675575in}{3.148455in}}%
\pgfpathlineto{\pgfqpoint{1.550510in}{2.929884in}}%
\pgfpathlineto{\pgfqpoint{1.468606in}{2.671104in}}%
\pgfpathclose%
\pgfusepath{fill}%
\end{pgfscope}%
\begin{pgfscope}%
\pgfpathrectangle{\pgfqpoint{0.539299in}{0.078740in}}{\pgfqpoint{7.842520in}{7.842520in}}%
\pgfusepath{clip}%
\pgfsetbuttcap%
\pgfsetroundjoin%
\definecolor{currentfill}{rgb}{0.170948,0.694384,0.493803}%
\pgfsetfillcolor{currentfill}%
\pgfsetlinewidth{0.000000pt}%
\definecolor{currentstroke}{rgb}{0.926106,0.897330,0.104071}%
\pgfsetstrokecolor{currentstroke}%
\pgfsetdash{}{0pt}%
\pgfpathmoveto{\pgfqpoint{2.305286in}{4.666266in}}%
\pgfpathlineto{\pgfqpoint{2.391827in}{4.772531in}}%
\pgfpathlineto{\pgfqpoint{2.181328in}{4.304148in}}%
\pgfpathclose%
\pgfusepath{fill}%
\end{pgfscope}%
\begin{pgfscope}%
\pgfpathrectangle{\pgfqpoint{0.539299in}{0.078740in}}{\pgfqpoint{7.842520in}{7.842520in}}%
\pgfusepath{clip}%
\pgfsetbuttcap%
\pgfsetroundjoin%
\definecolor{currentfill}{rgb}{0.277941,0.056324,0.381191}%
\pgfsetfillcolor{currentfill}%
\pgfsetlinewidth{0.000000pt}%
\definecolor{currentstroke}{rgb}{0.935904,0.898570,0.108131}%
\pgfsetstrokecolor{currentstroke}%
\pgfsetdash{}{0pt}%
\pgfpathmoveto{\pgfqpoint{6.673382in}{1.987622in}}%
\pgfpathlineto{\pgfqpoint{6.531589in}{2.101217in}}%
\pgfpathlineto{\pgfqpoint{6.455962in}{2.024776in}}%
\pgfpathclose%
\pgfusepath{fill}%
\end{pgfscope}%
\begin{pgfscope}%
\pgfpathrectangle{\pgfqpoint{0.539299in}{0.078740in}}{\pgfqpoint{7.842520in}{7.842520in}}%
\pgfusepath{clip}%
\pgfsetbuttcap%
\pgfsetroundjoin%
\definecolor{currentfill}{rgb}{0.135066,0.544853,0.554029}%
\pgfsetfillcolor{currentfill}%
\pgfsetlinewidth{0.000000pt}%
\definecolor{currentstroke}{rgb}{0.945636,0.899815,0.112838}%
\pgfsetstrokecolor{currentstroke}%
\pgfsetdash{}{0pt}%
\pgfpathmoveto{\pgfqpoint{4.846758in}{3.914972in}}%
\pgfpathlineto{\pgfqpoint{4.766857in}{3.997620in}}%
\pgfpathlineto{\pgfqpoint{4.907447in}{3.788480in}}%
\pgfpathclose%
\pgfusepath{fill}%
\end{pgfscope}%
\begin{pgfscope}%
\pgfpathrectangle{\pgfqpoint{0.539299in}{0.078740in}}{\pgfqpoint{7.842520in}{7.842520in}}%
\pgfusepath{clip}%
\pgfsetbuttcap%
\pgfsetroundjoin%
\definecolor{currentfill}{rgb}{0.137339,0.662252,0.515571}%
\pgfsetfillcolor{currentfill}%
\pgfsetlinewidth{0.000000pt}%
\definecolor{currentstroke}{rgb}{0.955300,0.901065,0.118128}%
\pgfsetstrokecolor{currentstroke}%
\pgfsetdash{}{0pt}%
\pgfpathmoveto{\pgfqpoint{4.426173in}{4.535091in}}%
\pgfpathlineto{\pgfqpoint{4.485130in}{4.425835in}}%
\pgfpathlineto{\pgfqpoint{4.566539in}{4.326950in}}%
\pgfpathclose%
\pgfusepath{fill}%
\end{pgfscope}%
\begin{pgfscope}%
\pgfpathrectangle{\pgfqpoint{0.539299in}{0.078740in}}{\pgfqpoint{7.842520in}{7.842520in}}%
\pgfusepath{clip}%
\pgfsetbuttcap%
\pgfsetroundjoin%
\definecolor{currentfill}{rgb}{0.281924,0.089666,0.412415}%
\pgfsetfillcolor{currentfill}%
\pgfsetlinewidth{0.000000pt}%
\definecolor{currentstroke}{rgb}{0.964894,0.902323,0.123941}%
\pgfsetstrokecolor{currentstroke}%
\pgfsetdash{}{0pt}%
\pgfpathmoveto{\pgfqpoint{6.390049in}{2.217039in}}%
\pgfpathlineto{\pgfqpoint{6.314217in}{2.147221in}}%
\pgfpathlineto{\pgfqpoint{6.455962in}{2.024776in}}%
\pgfpathclose%
\pgfusepath{fill}%
\end{pgfscope}%
\begin{pgfscope}%
\pgfpathrectangle{\pgfqpoint{0.539299in}{0.078740in}}{\pgfqpoint{7.842520in}{7.842520in}}%
\pgfusepath{clip}%
\pgfsetbuttcap%
\pgfsetroundjoin%
\definecolor{currentfill}{rgb}{0.271828,0.209303,0.504434}%
\pgfsetfillcolor{currentfill}%
\pgfsetlinewidth{0.000000pt}%
\definecolor{currentstroke}{rgb}{0.974417,0.903590,0.130215}%
\pgfsetstrokecolor{currentstroke}%
\pgfsetdash{}{0pt}%
\pgfpathmoveto{\pgfqpoint{1.596362in}{2.813436in}}%
\pgfpathlineto{\pgfqpoint{1.388975in}{2.359780in}}%
\pgfpathlineto{\pgfqpoint{1.520051in}{2.413317in}}%
\pgfpathclose%
\pgfusepath{fill}%
\end{pgfscope}%
\begin{pgfscope}%
\pgfpathrectangle{\pgfqpoint{0.539299in}{0.078740in}}{\pgfqpoint{7.842520in}{7.842520in}}%
\pgfusepath{clip}%
\pgfsetbuttcap%
\pgfsetroundjoin%
\definecolor{currentfill}{rgb}{0.123463,0.581687,0.547445}%
\pgfsetfillcolor{currentfill}%
\pgfsetlinewidth{0.000000pt}%
\definecolor{currentstroke}{rgb}{0.983868,0.904867,0.136897}%
\pgfsetstrokecolor{currentstroke}%
\pgfsetdash{}{0pt}%
\pgfpathmoveto{\pgfqpoint{4.626092in}{4.211058in}}%
\pgfpathlineto{\pgfqpoint{4.766857in}{3.997620in}}%
\pgfpathlineto{\pgfqpoint{4.846758in}{3.914972in}}%
\pgfpathclose%
\pgfusepath{fill}%
\end{pgfscope}%
\begin{pgfscope}%
\pgfpathrectangle{\pgfqpoint{0.539299in}{0.078740in}}{\pgfqpoint{7.842520in}{7.842520in}}%
\pgfusepath{clip}%
\pgfsetbuttcap%
\pgfsetroundjoin%
\definecolor{currentfill}{rgb}{0.171176,0.452530,0.557965}%
\pgfsetfillcolor{currentfill}%
\pgfsetlinewidth{0.000000pt}%
\definecolor{currentstroke}{rgb}{0.993248,0.906157,0.143936}%
\pgfsetstrokecolor{currentstroke}%
\pgfsetdash{}{0pt}%
\pgfpathmoveto{\pgfqpoint{1.756965in}{3.430597in}}%
\pgfpathlineto{\pgfqpoint{1.839990in}{3.669004in}}%
\pgfpathlineto{\pgfqpoint{1.718526in}{3.335879in}}%
\pgfpathclose%
\pgfusepath{fill}%
\end{pgfscope}%
\begin{pgfscope}%
\pgfpathrectangle{\pgfqpoint{0.539299in}{0.078740in}}{\pgfqpoint{7.842520in}{7.842520in}}%
\pgfusepath{clip}%
\pgfsetbuttcap%
\pgfsetroundjoin%
\definecolor{currentfill}{rgb}{0.515992,0.831158,0.294279}%
\pgfsetfillcolor{currentfill}%
\pgfsetlinewidth{0.000000pt}%
\definecolor{currentstroke}{rgb}{0.267004,0.004874,0.329415}%
\pgfsetstrokecolor{currentstroke}%
\pgfsetdash{}{0pt}%
\pgfpathmoveto{\pgfqpoint{2.694035in}{5.204781in}}%
\pgfpathlineto{\pgfqpoint{2.913349in}{5.441460in}}%
\pgfpathlineto{\pgfqpoint{2.781099in}{5.242417in}}%
\pgfpathclose%
\pgfusepath{fill}%
\end{pgfscope}%
\begin{pgfscope}%
\pgfpathrectangle{\pgfqpoint{0.539299in}{0.078740in}}{\pgfqpoint{7.842520in}{7.842520in}}%
\pgfusepath{clip}%
\pgfsetbuttcap%
\pgfsetroundjoin%
\definecolor{currentfill}{rgb}{0.606045,0.850733,0.236712}%
\pgfsetfillcolor{currentfill}%
\pgfsetlinewidth{0.000000pt}%
\definecolor{currentstroke}{rgb}{0.268510,0.009605,0.335427}%
\pgfsetstrokecolor{currentstroke}%
\pgfsetdash{}{0pt}%
\pgfpathmoveto{\pgfqpoint{3.584365in}{5.478002in}}%
\pgfpathlineto{\pgfqpoint{3.638229in}{5.473510in}}%
\pgfpathlineto{\pgfqpoint{3.723815in}{5.392306in}}%
\pgfpathclose%
\pgfusepath{fill}%
\end{pgfscope}%
\begin{pgfscope}%
\pgfpathrectangle{\pgfqpoint{0.539299in}{0.078740in}}{\pgfqpoint{7.842520in}{7.842520in}}%
\pgfusepath{clip}%
\pgfsetbuttcap%
\pgfsetroundjoin%
\definecolor{currentfill}{rgb}{0.678489,0.863742,0.189503}%
\pgfsetfillcolor{currentfill}%
\pgfsetlinewidth{0.000000pt}%
\definecolor{currentstroke}{rgb}{0.269944,0.014625,0.341379}%
\pgfsetstrokecolor{currentstroke}%
\pgfsetdash{}{0pt}%
\pgfpathmoveto{\pgfqpoint{3.359631in}{5.571327in}}%
\pgfpathlineto{\pgfqpoint{3.222442in}{5.537820in}}%
\pgfpathlineto{\pgfqpoint{3.135579in}{5.562556in}}%
\pgfpathclose%
\pgfusepath{fill}%
\end{pgfscope}%
\begin{pgfscope}%
\pgfpathrectangle{\pgfqpoint{0.539299in}{0.078740in}}{\pgfqpoint{7.842520in}{7.842520in}}%
\pgfusepath{clip}%
\pgfsetbuttcap%
\pgfsetroundjoin%
\definecolor{currentfill}{rgb}{0.668054,0.861999,0.196293}%
\pgfsetfillcolor{currentfill}%
\pgfsetlinewidth{0.000000pt}%
\definecolor{currentstroke}{rgb}{0.271305,0.019942,0.347269}%
\pgfsetstrokecolor{currentstroke}%
\pgfsetdash{}{0pt}%
\pgfpathmoveto{\pgfqpoint{3.359631in}{5.571327in}}%
\pgfpathlineto{\pgfqpoint{3.498366in}{5.547173in}}%
\pgfpathlineto{\pgfqpoint{3.584365in}{5.478002in}}%
\pgfpathclose%
\pgfusepath{fill}%
\end{pgfscope}%
\begin{pgfscope}%
\pgfpathrectangle{\pgfqpoint{0.539299in}{0.078740in}}{\pgfqpoint{7.842520in}{7.842520in}}%
\pgfusepath{clip}%
\pgfsetbuttcap%
\pgfsetroundjoin%
\definecolor{currentfill}{rgb}{0.220124,0.725509,0.466226}%
\pgfsetfillcolor{currentfill}%
\pgfsetlinewidth{0.000000pt}%
\definecolor{currentstroke}{rgb}{0.272594,0.025563,0.353093}%
\pgfsetstrokecolor{currentstroke}%
\pgfsetdash{}{0pt}%
\pgfpathmoveto{\pgfqpoint{4.426173in}{4.535091in}}%
\pgfpathlineto{\pgfqpoint{4.285648in}{4.739044in}}%
\pgfpathlineto{\pgfqpoint{4.202671in}{4.843466in}}%
\pgfpathclose%
\pgfusepath{fill}%
\end{pgfscope}%
\begin{pgfscope}%
\pgfpathrectangle{\pgfqpoint{0.539299in}{0.078740in}}{\pgfqpoint{7.842520in}{7.842520in}}%
\pgfusepath{clip}%
\pgfsetbuttcap%
\pgfsetroundjoin%
\definecolor{currentfill}{rgb}{0.121380,0.629492,0.531973}%
\pgfsetfillcolor{currentfill}%
\pgfsetlinewidth{0.000000pt}%
\definecolor{currentstroke}{rgb}{0.273809,0.031497,0.358853}%
\pgfsetstrokecolor{currentstroke}%
\pgfsetdash{}{0pt}%
\pgfpathmoveto{\pgfqpoint{2.219156in}{4.533644in}}%
\pgfpathlineto{\pgfqpoint{2.095141in}{4.184186in}}%
\pgfpathlineto{\pgfqpoint{2.009371in}{4.040892in}}%
\pgfpathclose%
\pgfusepath{fill}%
\end{pgfscope}%
\begin{pgfscope}%
\pgfpathrectangle{\pgfqpoint{0.539299in}{0.078740in}}{\pgfqpoint{7.842520in}{7.842520in}}%
\pgfusepath{clip}%
\pgfsetbuttcap%
\pgfsetroundjoin%
\definecolor{currentfill}{rgb}{0.395174,0.797475,0.367757}%
\pgfsetfillcolor{currentfill}%
\pgfsetlinewidth{0.000000pt}%
\definecolor{currentstroke}{rgb}{0.274952,0.037752,0.364543}%
\pgfsetstrokecolor{currentstroke}%
\pgfsetdash{}{0pt}%
\pgfpathmoveto{\pgfqpoint{2.478591in}{4.855876in}}%
\pgfpathlineto{\pgfqpoint{2.606940in}{5.146364in}}%
\pgfpathlineto{\pgfqpoint{2.694035in}{5.204781in}}%
\pgfpathclose%
\pgfusepath{fill}%
\end{pgfscope}%
\begin{pgfscope}%
\pgfpathrectangle{\pgfqpoint{0.539299in}{0.078740in}}{\pgfqpoint{7.842520in}{7.842520in}}%
\pgfusepath{clip}%
\pgfsetbuttcap%
\pgfsetroundjoin%
\definecolor{currentfill}{rgb}{0.241237,0.296485,0.539709}%
\pgfsetfillcolor{currentfill}%
\pgfsetlinewidth{0.000000pt}%
\definecolor{currentstroke}{rgb}{0.276022,0.044167,0.370164}%
\pgfsetstrokecolor{currentstroke}%
\pgfsetdash{}{0pt}%
\pgfpathmoveto{\pgfqpoint{5.749939in}{2.703326in}}%
\pgfpathlineto{\pgfqpoint{5.609380in}{2.861680in}}%
\pgfpathlineto{\pgfqpoint{5.532155in}{2.871486in}}%
\pgfpathclose%
\pgfusepath{fill}%
\end{pgfscope}%
\begin{pgfscope}%
\pgfpathrectangle{\pgfqpoint{0.539299in}{0.078740in}}{\pgfqpoint{7.842520in}{7.842520in}}%
\pgfusepath{clip}%
\pgfsetbuttcap%
\pgfsetroundjoin%
\definecolor{currentfill}{rgb}{0.229739,0.322361,0.545706}%
\pgfsetfillcolor{currentfill}%
\pgfsetlinewidth{0.000000pt}%
\definecolor{currentstroke}{rgb}{0.277018,0.050344,0.375715}%
\pgfsetstrokecolor{currentstroke}%
\pgfsetdash{}{0pt}%
\pgfpathmoveto{\pgfqpoint{5.532155in}{2.871486in}}%
\pgfpathlineto{\pgfqpoint{5.609380in}{2.861680in}}%
\pgfpathlineto{\pgfqpoint{5.468948in}{3.028932in}}%
\pgfpathclose%
\pgfusepath{fill}%
\end{pgfscope}%
\begin{pgfscope}%
\pgfpathrectangle{\pgfqpoint{0.539299in}{0.078740in}}{\pgfqpoint{7.842520in}{7.842520in}}%
\pgfusepath{clip}%
\pgfsetbuttcap%
\pgfsetroundjoin%
\definecolor{currentfill}{rgb}{0.263663,0.237631,0.518762}%
\pgfsetfillcolor{currentfill}%
\pgfsetlinewidth{0.000000pt}%
\definecolor{currentstroke}{rgb}{0.277941,0.056324,0.381191}%
\pgfsetstrokecolor{currentstroke}%
\pgfsetdash{}{0pt}%
\pgfpathmoveto{\pgfqpoint{5.814052in}{2.532529in}}%
\pgfpathlineto{\pgfqpoint{5.890666in}{2.553330in}}%
\pgfpathlineto{\pgfqpoint{5.749939in}{2.703326in}}%
\pgfpathclose%
\pgfusepath{fill}%
\end{pgfscope}%
\begin{pgfscope}%
\pgfpathrectangle{\pgfqpoint{0.539299in}{0.078740in}}{\pgfqpoint{7.842520in}{7.842520in}}%
\pgfusepath{clip}%
\pgfsetbuttcap%
\pgfsetroundjoin%
\definecolor{currentfill}{rgb}{0.276022,0.044167,0.370164}%
\pgfsetfillcolor{currentfill}%
\pgfsetlinewidth{0.000000pt}%
\definecolor{currentstroke}{rgb}{0.278791,0.062145,0.386592}%
\pgfsetstrokecolor{currentstroke}%
\pgfsetdash{}{0pt}%
\pgfpathmoveto{\pgfqpoint{6.455962in}{2.024776in}}%
\pgfpathlineto{\pgfqpoint{6.598042in}{1.908413in}}%
\pgfpathlineto{\pgfqpoint{6.673382in}{1.987622in}}%
\pgfpathclose%
\pgfusepath{fill}%
\end{pgfscope}%
\begin{pgfscope}%
\pgfpathrectangle{\pgfqpoint{0.539299in}{0.078740in}}{\pgfqpoint{7.842520in}{7.842520in}}%
\pgfusepath{clip}%
\pgfsetbuttcap%
\pgfsetroundjoin%
\definecolor{currentfill}{rgb}{0.121380,0.629492,0.531973}%
\pgfsetfillcolor{currentfill}%
\pgfsetlinewidth{0.000000pt}%
\definecolor{currentstroke}{rgb}{0.279566,0.067836,0.391917}%
\pgfsetstrokecolor{currentstroke}%
\pgfsetdash{}{0pt}%
\pgfpathmoveto{\pgfqpoint{4.485130in}{4.425835in}}%
\pgfpathlineto{\pgfqpoint{4.626092in}{4.211058in}}%
\pgfpathlineto{\pgfqpoint{4.706729in}{4.119068in}}%
\pgfpathclose%
\pgfusepath{fill}%
\end{pgfscope}%
\begin{pgfscope}%
\pgfpathrectangle{\pgfqpoint{0.539299in}{0.078740in}}{\pgfqpoint{7.842520in}{7.842520in}}%
\pgfusepath{clip}%
\pgfsetbuttcap%
\pgfsetroundjoin%
\definecolor{currentfill}{rgb}{0.281477,0.755203,0.432552}%
\pgfsetfillcolor{currentfill}%
\pgfsetlinewidth{0.000000pt}%
\definecolor{currentstroke}{rgb}{0.280267,0.073417,0.397163}%
\pgfsetstrokecolor{currentstroke}%
\pgfsetdash{}{0pt}%
\pgfpathmoveto{\pgfqpoint{4.202671in}{4.843466in}}%
\pgfpathlineto{\pgfqpoint{4.285648in}{4.739044in}}%
\pgfpathlineto{\pgfqpoint{4.145020in}{4.933425in}}%
\pgfpathclose%
\pgfusepath{fill}%
\end{pgfscope}%
\begin{pgfscope}%
\pgfpathrectangle{\pgfqpoint{0.539299in}{0.078740in}}{\pgfqpoint{7.842520in}{7.842520in}}%
\pgfusepath{clip}%
\pgfsetbuttcap%
\pgfsetroundjoin%
\definecolor{currentfill}{rgb}{0.275191,0.194905,0.496005}%
\pgfsetfillcolor{currentfill}%
\pgfsetlinewidth{0.000000pt}%
\definecolor{currentstroke}{rgb}{0.280894,0.078907,0.402329}%
\pgfsetstrokecolor{currentstroke}%
\pgfsetdash{}{0pt}%
\pgfpathmoveto{\pgfqpoint{5.955205in}{2.375680in}}%
\pgfpathlineto{\pgfqpoint{6.031600in}{2.411053in}}%
\pgfpathlineto{\pgfqpoint{5.890666in}{2.553330in}}%
\pgfpathclose%
\pgfusepath{fill}%
\end{pgfscope}%
\begin{pgfscope}%
\pgfpathrectangle{\pgfqpoint{0.539299in}{0.078740in}}{\pgfqpoint{7.842520in}{7.842520in}}%
\pgfusepath{clip}%
\pgfsetbuttcap%
\pgfsetroundjoin%
\definecolor{currentfill}{rgb}{0.233603,0.313828,0.543914}%
\pgfsetfillcolor{currentfill}%
\pgfsetlinewidth{0.000000pt}%
\definecolor{currentstroke}{rgb}{0.281446,0.084320,0.407414}%
\pgfsetstrokecolor{currentstroke}%
\pgfsetdash{}{0pt}%
\pgfpathmoveto{\pgfqpoint{1.468606in}{2.671104in}}%
\pgfpathlineto{\pgfqpoint{1.596362in}{2.813436in}}%
\pgfpathlineto{\pgfqpoint{1.675575in}{3.148455in}}%
\pgfpathclose%
\pgfusepath{fill}%
\end{pgfscope}%
\begin{pgfscope}%
\pgfpathrectangle{\pgfqpoint{0.539299in}{0.078740in}}{\pgfqpoint{7.842520in}{7.842520in}}%
\pgfusepath{clip}%
\pgfsetbuttcap%
\pgfsetroundjoin%
\definecolor{currentfill}{rgb}{0.162016,0.687316,0.499129}%
\pgfsetfillcolor{currentfill}%
\pgfsetlinewidth{0.000000pt}%
\definecolor{currentstroke}{rgb}{0.281924,0.089666,0.412415}%
\pgfsetstrokecolor{currentstroke}%
\pgfsetdash{}{0pt}%
\pgfpathmoveto{\pgfqpoint{2.181328in}{4.304148in}}%
\pgfpathlineto{\pgfqpoint{2.219156in}{4.533644in}}%
\pgfpathlineto{\pgfqpoint{2.305286in}{4.666266in}}%
\pgfpathclose%
\pgfusepath{fill}%
\end{pgfscope}%
\begin{pgfscope}%
\pgfpathrectangle{\pgfqpoint{0.539299in}{0.078740in}}{\pgfqpoint{7.842520in}{7.842520in}}%
\pgfusepath{clip}%
\pgfsetbuttcap%
\pgfsetroundjoin%
\definecolor{currentfill}{rgb}{0.201239,0.383670,0.554294}%
\pgfsetfillcolor{currentfill}%
\pgfsetlinewidth{0.000000pt}%
\definecolor{currentstroke}{rgb}{0.282327,0.094955,0.417331}%
\pgfsetstrokecolor{currentstroke}%
\pgfsetdash{}{0pt}%
\pgfpathmoveto{\pgfqpoint{5.468948in}{3.028932in}}%
\pgfpathlineto{\pgfqpoint{5.328594in}{3.205423in}}%
\pgfpathlineto{\pgfqpoint{5.250481in}{3.245064in}}%
\pgfpathclose%
\pgfusepath{fill}%
\end{pgfscope}%
\begin{pgfscope}%
\pgfpathrectangle{\pgfqpoint{0.539299in}{0.078740in}}{\pgfqpoint{7.842520in}{7.842520in}}%
\pgfusepath{clip}%
\pgfsetbuttcap%
\pgfsetroundjoin%
\definecolor{currentfill}{rgb}{0.555484,0.840254,0.269281}%
\pgfsetfillcolor{currentfill}%
\pgfsetlinewidth{0.000000pt}%
\definecolor{currentstroke}{rgb}{0.282656,0.100196,0.422160}%
\pgfsetstrokecolor{currentstroke}%
\pgfsetdash{}{0pt}%
\pgfpathmoveto{\pgfqpoint{3.863917in}{5.267515in}}%
\pgfpathlineto{\pgfqpoint{3.723815in}{5.392306in}}%
\pgfpathlineto{\pgfqpoint{3.778860in}{5.358543in}}%
\pgfpathclose%
\pgfusepath{fill}%
\end{pgfscope}%
\begin{pgfscope}%
\pgfpathrectangle{\pgfqpoint{0.539299in}{0.078740in}}{\pgfqpoint{7.842520in}{7.842520in}}%
\pgfusepath{clip}%
\pgfsetbuttcap%
\pgfsetroundjoin%
\definecolor{currentfill}{rgb}{0.199430,0.387607,0.554642}%
\pgfsetfillcolor{currentfill}%
\pgfsetlinewidth{0.000000pt}%
\definecolor{currentstroke}{rgb}{0.282910,0.105393,0.426902}%
\pgfsetstrokecolor{currentstroke}%
\pgfsetdash{}{0pt}%
\pgfpathmoveto{\pgfqpoint{1.550510in}{2.929884in}}%
\pgfpathlineto{\pgfqpoint{1.675575in}{3.148455in}}%
\pgfpathlineto{\pgfqpoint{1.756965in}{3.430597in}}%
\pgfpathclose%
\pgfusepath{fill}%
\end{pgfscope}%
\begin{pgfscope}%
\pgfpathrectangle{\pgfqpoint{0.539299in}{0.078740in}}{\pgfqpoint{7.842520in}{7.842520in}}%
\pgfusepath{clip}%
\pgfsetbuttcap%
\pgfsetroundjoin%
\definecolor{currentfill}{rgb}{0.636902,0.856542,0.216620}%
\pgfsetfillcolor{currentfill}%
\pgfsetlinewidth{0.000000pt}%
\definecolor{currentstroke}{rgb}{0.283091,0.110553,0.431554}%
\pgfsetstrokecolor{currentstroke}%
\pgfsetdash{}{0pt}%
\pgfpathmoveto{\pgfqpoint{3.135579in}{5.562556in}}%
\pgfpathlineto{\pgfqpoint{3.000456in}{5.448722in}}%
\pgfpathlineto{\pgfqpoint{2.913349in}{5.441460in}}%
\pgfpathclose%
\pgfusepath{fill}%
\end{pgfscope}%
\begin{pgfscope}%
\pgfpathrectangle{\pgfqpoint{0.539299in}{0.078740in}}{\pgfqpoint{7.842520in}{7.842520in}}%
\pgfusepath{clip}%
\pgfsetbuttcap%
\pgfsetroundjoin%
\definecolor{currentfill}{rgb}{0.281412,0.155834,0.469201}%
\pgfsetfillcolor{currentfill}%
\pgfsetlinewidth{0.000000pt}%
\definecolor{currentstroke}{rgb}{0.283197,0.115680,0.436115}%
\pgfsetstrokecolor{currentstroke}%
\pgfsetdash{}{0pt}%
\pgfpathmoveto{\pgfqpoint{6.172774in}{2.275861in}}%
\pgfpathlineto{\pgfqpoint{6.031600in}{2.411053in}}%
\pgfpathlineto{\pgfqpoint{6.096558in}{2.227023in}}%
\pgfpathclose%
\pgfusepath{fill}%
\end{pgfscope}%
\begin{pgfscope}%
\pgfpathrectangle{\pgfqpoint{0.539299in}{0.078740in}}{\pgfqpoint{7.842520in}{7.842520in}}%
\pgfusepath{clip}%
\pgfsetbuttcap%
\pgfsetroundjoin%
\definecolor{currentfill}{rgb}{0.187231,0.414746,0.556547}%
\pgfsetfillcolor{currentfill}%
\pgfsetlinewidth{0.000000pt}%
\definecolor{currentstroke}{rgb}{0.283229,0.120777,0.440584}%
\pgfsetstrokecolor{currentstroke}%
\pgfsetdash{}{0pt}%
\pgfpathmoveto{\pgfqpoint{5.250481in}{3.245064in}}%
\pgfpathlineto{\pgfqpoint{5.328594in}{3.205423in}}%
\pgfpathlineto{\pgfqpoint{5.188263in}{3.391183in}}%
\pgfpathclose%
\pgfusepath{fill}%
\end{pgfscope}%
\begin{pgfscope}%
\pgfpathrectangle{\pgfqpoint{0.539299in}{0.078740in}}{\pgfqpoint{7.842520in}{7.842520in}}%
\pgfusepath{clip}%
\pgfsetbuttcap%
\pgfsetroundjoin%
\definecolor{currentfill}{rgb}{0.377779,0.791781,0.377939}%
\pgfsetfillcolor{currentfill}%
\pgfsetlinewidth{0.000000pt}%
\definecolor{currentstroke}{rgb}{0.283187,0.125848,0.444960}%
\pgfsetstrokecolor{currentstroke}%
\pgfsetdash{}{0pt}%
\pgfpathmoveto{\pgfqpoint{4.061290in}{5.036247in}}%
\pgfpathlineto{\pgfqpoint{4.145020in}{4.933425in}}%
\pgfpathlineto{\pgfqpoint{4.004391in}{5.111952in}}%
\pgfpathclose%
\pgfusepath{fill}%
\end{pgfscope}%
\begin{pgfscope}%
\pgfpathrectangle{\pgfqpoint{0.539299in}{0.078740in}}{\pgfqpoint{7.842520in}{7.842520in}}%
\pgfusepath{clip}%
\pgfsetbuttcap%
\pgfsetroundjoin%
\definecolor{currentfill}{rgb}{0.496615,0.826376,0.306377}%
\pgfsetfillcolor{currentfill}%
\pgfsetlinewidth{0.000000pt}%
\definecolor{currentstroke}{rgb}{0.283072,0.130895,0.449241}%
\pgfsetstrokecolor{currentstroke}%
\pgfsetdash{}{0pt}%
\pgfpathmoveto{\pgfqpoint{3.778860in}{5.358543in}}%
\pgfpathlineto{\pgfqpoint{4.004391in}{5.111952in}}%
\pgfpathlineto{\pgfqpoint{3.863917in}{5.267515in}}%
\pgfpathclose%
\pgfusepath{fill}%
\end{pgfscope}%
\begin{pgfscope}%
\pgfpathrectangle{\pgfqpoint{0.539299in}{0.078740in}}{\pgfqpoint{7.842520in}{7.842520in}}%
\pgfusepath{clip}%
\pgfsetbuttcap%
\pgfsetroundjoin%
\definecolor{currentfill}{rgb}{0.311925,0.767822,0.415586}%
\pgfsetfillcolor{currentfill}%
\pgfsetlinewidth{0.000000pt}%
\definecolor{currentstroke}{rgb}{0.282884,0.135920,0.453427}%
\pgfsetstrokecolor{currentstroke}%
\pgfsetdash{}{0pt}%
\pgfpathmoveto{\pgfqpoint{2.478591in}{4.855876in}}%
\pgfpathlineto{\pgfqpoint{2.391827in}{4.772531in}}%
\pgfpathlineto{\pgfqpoint{2.519975in}{5.063797in}}%
\pgfpathclose%
\pgfusepath{fill}%
\end{pgfscope}%
\begin{pgfscope}%
\pgfpathrectangle{\pgfqpoint{0.539299in}{0.078740in}}{\pgfqpoint{7.842520in}{7.842520in}}%
\pgfusepath{clip}%
\pgfsetbuttcap%
\pgfsetroundjoin%
\definecolor{currentfill}{rgb}{0.283072,0.130895,0.449241}%
\pgfsetfillcolor{currentfill}%
\pgfsetlinewidth{0.000000pt}%
\definecolor{currentstroke}{rgb}{0.282623,0.140926,0.457517}%
\pgfsetstrokecolor{currentstroke}%
\pgfsetdash{}{0pt}%
\pgfpathmoveto{\pgfqpoint{6.096558in}{2.227023in}}%
\pgfpathlineto{\pgfqpoint{6.314217in}{2.147221in}}%
\pgfpathlineto{\pgfqpoint{6.172774in}{2.275861in}}%
\pgfpathclose%
\pgfusepath{fill}%
\end{pgfscope}%
\begin{pgfscope}%
\pgfpathrectangle{\pgfqpoint{0.539299in}{0.078740in}}{\pgfqpoint{7.842520in}{7.842520in}}%
\pgfusepath{clip}%
\pgfsetbuttcap%
\pgfsetroundjoin%
\definecolor{currentfill}{rgb}{0.657642,0.860219,0.203082}%
\pgfsetfillcolor{currentfill}%
\pgfsetlinewidth{0.000000pt}%
\definecolor{currentstroke}{rgb}{0.282290,0.145912,0.461510}%
\pgfsetstrokecolor{currentstroke}%
\pgfsetdash{}{0pt}%
\pgfpathmoveto{\pgfqpoint{3.498366in}{5.547173in}}%
\pgfpathlineto{\pgfqpoint{3.638229in}{5.473510in}}%
\pgfpathlineto{\pgfqpoint{3.584365in}{5.478002in}}%
\pgfpathclose%
\pgfusepath{fill}%
\end{pgfscope}%
\begin{pgfscope}%
\pgfpathrectangle{\pgfqpoint{0.539299in}{0.078740in}}{\pgfqpoint{7.842520in}{7.842520in}}%
\pgfusepath{clip}%
\pgfsetbuttcap%
\pgfsetroundjoin%
\definecolor{currentfill}{rgb}{0.175707,0.697900,0.491033}%
\pgfsetfillcolor{currentfill}%
\pgfsetlinewidth{0.000000pt}%
\definecolor{currentstroke}{rgb}{0.281887,0.150881,0.465405}%
\pgfsetstrokecolor{currentstroke}%
\pgfsetdash{}{0pt}%
\pgfpathmoveto{\pgfqpoint{4.343978in}{4.638176in}}%
\pgfpathlineto{\pgfqpoint{4.485130in}{4.425835in}}%
\pgfpathlineto{\pgfqpoint{4.426173in}{4.535091in}}%
\pgfpathclose%
\pgfusepath{fill}%
\end{pgfscope}%
\begin{pgfscope}%
\pgfpathrectangle{\pgfqpoint{0.539299in}{0.078740in}}{\pgfqpoint{7.842520in}{7.842520in}}%
\pgfusepath{clip}%
\pgfsetbuttcap%
\pgfsetroundjoin%
\definecolor{currentfill}{rgb}{0.160665,0.478540,0.558115}%
\pgfsetfillcolor{currentfill}%
\pgfsetlinewidth{0.000000pt}%
\definecolor{currentstroke}{rgb}{0.281412,0.155834,0.469201}%
\pgfsetstrokecolor{currentstroke}%
\pgfsetdash{}{0pt}%
\pgfpathmoveto{\pgfqpoint{4.968612in}{3.651484in}}%
\pgfpathlineto{\pgfqpoint{5.188263in}{3.391183in}}%
\pgfpathlineto{\pgfqpoint{5.047899in}{3.585834in}}%
\pgfpathclose%
\pgfusepath{fill}%
\end{pgfscope}%
\begin{pgfscope}%
\pgfpathrectangle{\pgfqpoint{0.539299in}{0.078740in}}{\pgfqpoint{7.842520in}{7.842520in}}%
\pgfusepath{clip}%
\pgfsetbuttcap%
\pgfsetroundjoin%
\definecolor{currentfill}{rgb}{0.282327,0.094955,0.417331}%
\pgfsetfillcolor{currentfill}%
\pgfsetlinewidth{0.000000pt}%
\definecolor{currentstroke}{rgb}{0.280868,0.160771,0.472899}%
\pgfsetstrokecolor{currentstroke}%
\pgfsetdash{}{0pt}%
\pgfpathmoveto{\pgfqpoint{6.314217in}{2.147221in}}%
\pgfpathlineto{\pgfqpoint{6.238160in}{2.086590in}}%
\pgfpathlineto{\pgfqpoint{6.455962in}{2.024776in}}%
\pgfpathclose%
\pgfusepath{fill}%
\end{pgfscope}%
\begin{pgfscope}%
\pgfpathrectangle{\pgfqpoint{0.539299in}{0.078740in}}{\pgfqpoint{7.842520in}{7.842520in}}%
\pgfusepath{clip}%
\pgfsetbuttcap%
\pgfsetroundjoin%
\definecolor{currentfill}{rgb}{0.255645,0.260703,0.528312}%
\pgfsetfillcolor{currentfill}%
\pgfsetlinewidth{0.000000pt}%
\definecolor{currentstroke}{rgb}{0.280255,0.165693,0.476498}%
\pgfsetstrokecolor{currentstroke}%
\pgfsetdash{}{0pt}%
\pgfpathmoveto{\pgfqpoint{5.814052in}{2.532529in}}%
\pgfpathlineto{\pgfqpoint{5.749939in}{2.703326in}}%
\pgfpathlineto{\pgfqpoint{5.673051in}{2.697726in}}%
\pgfpathclose%
\pgfusepath{fill}%
\end{pgfscope}%
\begin{pgfscope}%
\pgfpathrectangle{\pgfqpoint{0.539299in}{0.078740in}}{\pgfqpoint{7.842520in}{7.842520in}}%
\pgfusepath{clip}%
\pgfsetbuttcap%
\pgfsetroundjoin%
\definecolor{currentfill}{rgb}{0.243113,0.292092,0.538516}%
\pgfsetfillcolor{currentfill}%
\pgfsetlinewidth{0.000000pt}%
\definecolor{currentstroke}{rgb}{0.279574,0.170599,0.479997}%
\pgfsetstrokecolor{currentstroke}%
\pgfsetdash{}{0pt}%
\pgfpathmoveto{\pgfqpoint{5.673051in}{2.697726in}}%
\pgfpathlineto{\pgfqpoint{5.749939in}{2.703326in}}%
\pgfpathlineto{\pgfqpoint{5.532155in}{2.871486in}}%
\pgfpathclose%
\pgfusepath{fill}%
\end{pgfscope}%
\begin{pgfscope}%
\pgfpathrectangle{\pgfqpoint{0.539299in}{0.078740in}}{\pgfqpoint{7.842520in}{7.842520in}}%
\pgfusepath{clip}%
\pgfsetbuttcap%
\pgfsetroundjoin%
\definecolor{currentfill}{rgb}{0.270595,0.214069,0.507052}%
\pgfsetfillcolor{currentfill}%
\pgfsetlinewidth{0.000000pt}%
\definecolor{currentstroke}{rgb}{0.278826,0.175490,0.483397}%
\pgfsetstrokecolor{currentstroke}%
\pgfsetdash{}{0pt}%
\pgfpathmoveto{\pgfqpoint{5.890666in}{2.553330in}}%
\pgfpathlineto{\pgfqpoint{5.814052in}{2.532529in}}%
\pgfpathlineto{\pgfqpoint{5.955205in}{2.375680in}}%
\pgfpathclose%
\pgfusepath{fill}%
\end{pgfscope}%
\begin{pgfscope}%
\pgfpathrectangle{\pgfqpoint{0.539299in}{0.078740in}}{\pgfqpoint{7.842520in}{7.842520in}}%
\pgfusepath{clip}%
\pgfsetbuttcap%
\pgfsetroundjoin%
\definecolor{currentfill}{rgb}{0.226397,0.728888,0.462789}%
\pgfsetfillcolor{currentfill}%
\pgfsetlinewidth{0.000000pt}%
\definecolor{currentstroke}{rgb}{0.278012,0.180367,0.486697}%
\pgfsetstrokecolor{currentstroke}%
\pgfsetdash{}{0pt}%
\pgfpathmoveto{\pgfqpoint{4.202671in}{4.843466in}}%
\pgfpathlineto{\pgfqpoint{4.343978in}{4.638176in}}%
\pgfpathlineto{\pgfqpoint{4.426173in}{4.535091in}}%
\pgfpathclose%
\pgfusepath{fill}%
\end{pgfscope}%
\begin{pgfscope}%
\pgfpathrectangle{\pgfqpoint{0.539299in}{0.078740in}}{\pgfqpoint{7.842520in}{7.842520in}}%
\pgfusepath{clip}%
\pgfsetbuttcap%
\pgfsetroundjoin%
\definecolor{currentfill}{rgb}{0.216210,0.351535,0.550627}%
\pgfsetfillcolor{currentfill}%
\pgfsetlinewidth{0.000000pt}%
\definecolor{currentstroke}{rgb}{0.277134,0.185228,0.489898}%
\pgfsetstrokecolor{currentstroke}%
\pgfsetdash{}{0pt}%
\pgfpathmoveto{\pgfqpoint{5.468948in}{3.028932in}}%
\pgfpathlineto{\pgfqpoint{5.391316in}{3.053947in}}%
\pgfpathlineto{\pgfqpoint{5.532155in}{2.871486in}}%
\pgfpathclose%
\pgfusepath{fill}%
\end{pgfscope}%
\begin{pgfscope}%
\pgfpathrectangle{\pgfqpoint{0.539299in}{0.078740in}}{\pgfqpoint{7.842520in}{7.842520in}}%
\pgfusepath{clip}%
\pgfsetbuttcap%
\pgfsetroundjoin%
\definecolor{currentfill}{rgb}{0.149039,0.508051,0.557250}%
\pgfsetfillcolor{currentfill}%
\pgfsetlinewidth{0.000000pt}%
\definecolor{currentstroke}{rgb}{0.276194,0.190074,0.493001}%
\pgfsetstrokecolor{currentstroke}%
\pgfsetdash{}{0pt}%
\pgfpathmoveto{\pgfqpoint{5.047899in}{3.585834in}}%
\pgfpathlineto{\pgfqpoint{4.907447in}{3.788480in}}%
\pgfpathlineto{\pgfqpoint{4.968612in}{3.651484in}}%
\pgfpathclose%
\pgfusepath{fill}%
\end{pgfscope}%
\begin{pgfscope}%
\pgfpathrectangle{\pgfqpoint{0.539299in}{0.078740in}}{\pgfqpoint{7.842520in}{7.842520in}}%
\pgfusepath{clip}%
\pgfsetbuttcap%
\pgfsetroundjoin%
\definecolor{currentfill}{rgb}{0.279574,0.170599,0.479997}%
\pgfsetfillcolor{currentfill}%
\pgfsetlinewidth{0.000000pt}%
\definecolor{currentstroke}{rgb}{0.275191,0.194905,0.496005}%
\pgfsetstrokecolor{currentstroke}%
\pgfsetdash{}{0pt}%
\pgfpathmoveto{\pgfqpoint{6.031600in}{2.411053in}}%
\pgfpathlineto{\pgfqpoint{5.955205in}{2.375680in}}%
\pgfpathlineto{\pgfqpoint{6.096558in}{2.227023in}}%
\pgfpathclose%
\pgfusepath{fill}%
\end{pgfscope}%
\begin{pgfscope}%
\pgfpathrectangle{\pgfqpoint{0.539299in}{0.078740in}}{\pgfqpoint{7.842520in}{7.842520in}}%
\pgfusepath{clip}%
\pgfsetbuttcap%
\pgfsetroundjoin%
\definecolor{currentfill}{rgb}{0.203063,0.379716,0.553925}%
\pgfsetfillcolor{currentfill}%
\pgfsetlinewidth{0.000000pt}%
\definecolor{currentstroke}{rgb}{0.274128,0.199721,0.498911}%
\pgfsetstrokecolor{currentstroke}%
\pgfsetdash{}{0pt}%
\pgfpathmoveto{\pgfqpoint{5.250481in}{3.245064in}}%
\pgfpathlineto{\pgfqpoint{5.391316in}{3.053947in}}%
\pgfpathlineto{\pgfqpoint{5.468948in}{3.028932in}}%
\pgfpathclose%
\pgfusepath{fill}%
\end{pgfscope}%
\begin{pgfscope}%
\pgfpathrectangle{\pgfqpoint{0.539299in}{0.078740in}}{\pgfqpoint{7.842520in}{7.842520in}}%
\pgfusepath{clip}%
\pgfsetbuttcap%
\pgfsetroundjoin%
\definecolor{currentfill}{rgb}{0.616293,0.852709,0.230052}%
\pgfsetfillcolor{currentfill}%
\pgfsetlinewidth{0.000000pt}%
\definecolor{currentstroke}{rgb}{0.273006,0.204520,0.501721}%
\pgfsetstrokecolor{currentstroke}%
\pgfsetdash{}{0pt}%
\pgfpathmoveto{\pgfqpoint{3.723815in}{5.392306in}}%
\pgfpathlineto{\pgfqpoint{3.638229in}{5.473510in}}%
\pgfpathlineto{\pgfqpoint{3.778860in}{5.358543in}}%
\pgfpathclose%
\pgfusepath{fill}%
\end{pgfscope}%
\begin{pgfscope}%
\pgfpathrectangle{\pgfqpoint{0.539299in}{0.078740in}}{\pgfqpoint{7.842520in}{7.842520in}}%
\pgfusepath{clip}%
\pgfsetbuttcap%
\pgfsetroundjoin%
\definecolor{currentfill}{rgb}{0.395174,0.797475,0.367757}%
\pgfsetfillcolor{currentfill}%
\pgfsetlinewidth{0.000000pt}%
\definecolor{currentstroke}{rgb}{0.271828,0.209303,0.504434}%
\pgfsetstrokecolor{currentstroke}%
\pgfsetdash{}{0pt}%
\pgfpathmoveto{\pgfqpoint{2.519975in}{5.063797in}}%
\pgfpathlineto{\pgfqpoint{2.606940in}{5.146364in}}%
\pgfpathlineto{\pgfqpoint{2.478591in}{4.855876in}}%
\pgfpathclose%
\pgfusepath{fill}%
\end{pgfscope}%
\begin{pgfscope}%
\pgfpathrectangle{\pgfqpoint{0.539299in}{0.078740in}}{\pgfqpoint{7.842520in}{7.842520in}}%
\pgfusepath{clip}%
\pgfsetbuttcap%
\pgfsetroundjoin%
\definecolor{currentfill}{rgb}{0.276022,0.044167,0.370164}%
\pgfsetfillcolor{currentfill}%
\pgfsetlinewidth{0.000000pt}%
\definecolor{currentstroke}{rgb}{0.270595,0.214069,0.507052}%
\pgfsetstrokecolor{currentstroke}%
\pgfsetdash{}{0pt}%
\pgfpathmoveto{\pgfqpoint{6.522373in}{1.832361in}}%
\pgfpathlineto{\pgfqpoint{6.598042in}{1.908413in}}%
\pgfpathlineto{\pgfqpoint{6.455962in}{2.024776in}}%
\pgfpathclose%
\pgfusepath{fill}%
\end{pgfscope}%
\begin{pgfscope}%
\pgfpathrectangle{\pgfqpoint{0.539299in}{0.078740in}}{\pgfqpoint{7.842520in}{7.842520in}}%
\pgfusepath{clip}%
\pgfsetbuttcap%
\pgfsetroundjoin%
\definecolor{currentfill}{rgb}{0.283229,0.120777,0.440584}%
\pgfsetfillcolor{currentfill}%
\pgfsetlinewidth{0.000000pt}%
\definecolor{currentstroke}{rgb}{0.269308,0.218818,0.509577}%
\pgfsetstrokecolor{currentstroke}%
\pgfsetdash{}{0pt}%
\pgfpathmoveto{\pgfqpoint{6.096558in}{2.227023in}}%
\pgfpathlineto{\pgfqpoint{6.238160in}{2.086590in}}%
\pgfpathlineto{\pgfqpoint{6.314217in}{2.147221in}}%
\pgfpathclose%
\pgfusepath{fill}%
\end{pgfscope}%
\begin{pgfscope}%
\pgfpathrectangle{\pgfqpoint{0.539299in}{0.078740in}}{\pgfqpoint{7.842520in}{7.842520in}}%
\pgfusepath{clip}%
\pgfsetbuttcap%
\pgfsetroundjoin%
\definecolor{currentfill}{rgb}{0.352360,0.783011,0.392636}%
\pgfsetfillcolor{currentfill}%
\pgfsetlinewidth{0.000000pt}%
\definecolor{currentstroke}{rgb}{0.267968,0.223549,0.512008}%
\pgfsetstrokecolor{currentstroke}%
\pgfsetdash{}{0pt}%
\pgfpathmoveto{\pgfqpoint{4.145020in}{4.933425in}}%
\pgfpathlineto{\pgfqpoint{4.061290in}{5.036247in}}%
\pgfpathlineto{\pgfqpoint{4.202671in}{4.843466in}}%
\pgfpathclose%
\pgfusepath{fill}%
\end{pgfscope}%
\begin{pgfscope}%
\pgfpathrectangle{\pgfqpoint{0.539299in}{0.078740in}}{\pgfqpoint{7.842520in}{7.842520in}}%
\pgfusepath{clip}%
\pgfsetbuttcap%
\pgfsetroundjoin%
\definecolor{currentfill}{rgb}{0.121148,0.592739,0.544641}%
\pgfsetfillcolor{currentfill}%
\pgfsetlinewidth{0.000000pt}%
\definecolor{currentstroke}{rgb}{0.266580,0.228262,0.514349}%
\pgfsetstrokecolor{currentstroke}%
\pgfsetdash{}{0pt}%
\pgfpathmoveto{\pgfqpoint{2.049048in}{4.173228in}}%
\pgfpathlineto{\pgfqpoint{2.009371in}{4.040892in}}%
\pgfpathlineto{\pgfqpoint{1.924234in}{3.870625in}}%
\pgfpathclose%
\pgfusepath{fill}%
\end{pgfscope}%
\begin{pgfscope}%
\pgfpathrectangle{\pgfqpoint{0.539299in}{0.078740in}}{\pgfqpoint{7.842520in}{7.842520in}}%
\pgfusepath{clip}%
\pgfsetbuttcap%
\pgfsetroundjoin%
\definecolor{currentfill}{rgb}{0.246811,0.283237,0.535941}%
\pgfsetfillcolor{currentfill}%
\pgfsetlinewidth{0.000000pt}%
\definecolor{currentstroke}{rgb}{0.265145,0.232956,0.516599}%
\pgfsetstrokecolor{currentstroke}%
\pgfsetdash{}{0pt}%
\pgfpathmoveto{\pgfqpoint{1.726778in}{2.916957in}}%
\pgfpathlineto{\pgfqpoint{1.596362in}{2.813436in}}%
\pgfpathlineto{\pgfqpoint{1.520051in}{2.413317in}}%
\pgfpathclose%
\pgfusepath{fill}%
\end{pgfscope}%
\begin{pgfscope}%
\pgfpathrectangle{\pgfqpoint{0.539299in}{0.078740in}}{\pgfqpoint{7.842520in}{7.842520in}}%
\pgfusepath{clip}%
\pgfsetbuttcap%
\pgfsetroundjoin%
\definecolor{currentfill}{rgb}{0.129933,0.559582,0.551864}%
\pgfsetfillcolor{currentfill}%
\pgfsetlinewidth{0.000000pt}%
\definecolor{currentstroke}{rgb}{0.263663,0.237631,0.518762}%
\pgfsetstrokecolor{currentstroke}%
\pgfsetdash{}{0pt}%
\pgfpathmoveto{\pgfqpoint{4.827480in}{3.864804in}}%
\pgfpathlineto{\pgfqpoint{4.907447in}{3.788480in}}%
\pgfpathlineto{\pgfqpoint{4.766857in}{3.997620in}}%
\pgfpathclose%
\pgfusepath{fill}%
\end{pgfscope}%
\begin{pgfscope}%
\pgfpathrectangle{\pgfqpoint{0.539299in}{0.078740in}}{\pgfqpoint{7.842520in}{7.842520in}}%
\pgfusepath{clip}%
\pgfsetbuttcap%
\pgfsetroundjoin%
\definecolor{currentfill}{rgb}{0.730889,0.871916,0.156029}%
\pgfsetfillcolor{currentfill}%
\pgfsetlinewidth{0.000000pt}%
\definecolor{currentstroke}{rgb}{0.262138,0.242286,0.520837}%
\pgfsetstrokecolor{currentstroke}%
\pgfsetdash{}{0pt}%
\pgfpathmoveto{\pgfqpoint{3.135579in}{5.562556in}}%
\pgfpathlineto{\pgfqpoint{3.272857in}{5.610573in}}%
\pgfpathlineto{\pgfqpoint{3.359631in}{5.571327in}}%
\pgfpathclose%
\pgfusepath{fill}%
\end{pgfscope}%
\begin{pgfscope}%
\pgfpathrectangle{\pgfqpoint{0.539299in}{0.078740in}}{\pgfqpoint{7.842520in}{7.842520in}}%
\pgfusepath{clip}%
\pgfsetbuttcap%
\pgfsetroundjoin%
\definecolor{currentfill}{rgb}{0.175841,0.441290,0.557685}%
\pgfsetfillcolor{currentfill}%
\pgfsetlinewidth{0.000000pt}%
\definecolor{currentstroke}{rgb}{0.260571,0.246922,0.522828}%
\pgfsetstrokecolor{currentstroke}%
\pgfsetdash{}{0pt}%
\pgfpathmoveto{\pgfqpoint{5.188263in}{3.391183in}}%
\pgfpathlineto{\pgfqpoint{5.109597in}{3.444492in}}%
\pgfpathlineto{\pgfqpoint{5.250481in}{3.245064in}}%
\pgfpathclose%
\pgfusepath{fill}%
\end{pgfscope}%
\begin{pgfscope}%
\pgfpathrectangle{\pgfqpoint{0.539299in}{0.078740in}}{\pgfqpoint{7.842520in}{7.842520in}}%
\pgfusepath{clip}%
\pgfsetbuttcap%
\pgfsetroundjoin%
\definecolor{currentfill}{rgb}{0.730889,0.871916,0.156029}%
\pgfsetfillcolor{currentfill}%
\pgfsetlinewidth{0.000000pt}%
\definecolor{currentstroke}{rgb}{0.258965,0.251537,0.524736}%
\pgfsetstrokecolor{currentstroke}%
\pgfsetdash{}{0pt}%
\pgfpathmoveto{\pgfqpoint{3.498366in}{5.547173in}}%
\pgfpathlineto{\pgfqpoint{3.359631in}{5.571327in}}%
\pgfpathlineto{\pgfqpoint{3.272857in}{5.610573in}}%
\pgfpathclose%
\pgfusepath{fill}%
\end{pgfscope}%
\begin{pgfscope}%
\pgfpathrectangle{\pgfqpoint{0.539299in}{0.078740in}}{\pgfqpoint{7.842520in}{7.842520in}}%
\pgfusepath{clip}%
\pgfsetbuttcap%
\pgfsetroundjoin%
\definecolor{currentfill}{rgb}{0.585678,0.846661,0.249897}%
\pgfsetfillcolor{currentfill}%
\pgfsetlinewidth{0.000000pt}%
\definecolor{currentstroke}{rgb}{0.257322,0.256130,0.526563}%
\pgfsetstrokecolor{currentstroke}%
\pgfsetdash{}{0pt}%
\pgfpathmoveto{\pgfqpoint{2.694035in}{5.204781in}}%
\pgfpathlineto{\pgfqpoint{2.826074in}{5.413789in}}%
\pgfpathlineto{\pgfqpoint{2.913349in}{5.441460in}}%
\pgfpathclose%
\pgfusepath{fill}%
\end{pgfscope}%
\begin{pgfscope}%
\pgfpathrectangle{\pgfqpoint{0.539299in}{0.078740in}}{\pgfqpoint{7.842520in}{7.842520in}}%
\pgfusepath{clip}%
\pgfsetbuttcap%
\pgfsetroundjoin%
\definecolor{currentfill}{rgb}{0.506271,0.828786,0.300362}%
\pgfsetfillcolor{currentfill}%
\pgfsetlinewidth{0.000000pt}%
\definecolor{currentstroke}{rgb}{0.255645,0.260703,0.528312}%
\pgfsetstrokecolor{currentstroke}%
\pgfsetdash{}{0pt}%
\pgfpathmoveto{\pgfqpoint{3.919960in}{5.210264in}}%
\pgfpathlineto{\pgfqpoint{4.004391in}{5.111952in}}%
\pgfpathlineto{\pgfqpoint{3.778860in}{5.358543in}}%
\pgfpathclose%
\pgfusepath{fill}%
\end{pgfscope}%
\begin{pgfscope}%
\pgfpathrectangle{\pgfqpoint{0.539299in}{0.078740in}}{\pgfqpoint{7.842520in}{7.842520in}}%
\pgfusepath{clip}%
\pgfsetbuttcap%
\pgfsetroundjoin%
\definecolor{currentfill}{rgb}{0.449368,0.813768,0.335384}%
\pgfsetfillcolor{currentfill}%
\pgfsetlinewidth{0.000000pt}%
\definecolor{currentstroke}{rgb}{0.253935,0.265254,0.529983}%
\pgfsetstrokecolor{currentstroke}%
\pgfsetdash{}{0pt}%
\pgfpathmoveto{\pgfqpoint{3.919960in}{5.210264in}}%
\pgfpathlineto{\pgfqpoint{4.061290in}{5.036247in}}%
\pgfpathlineto{\pgfqpoint{4.004391in}{5.111952in}}%
\pgfpathclose%
\pgfusepath{fill}%
\end{pgfscope}%
\begin{pgfscope}%
\pgfpathrectangle{\pgfqpoint{0.539299in}{0.078740in}}{\pgfqpoint{7.842520in}{7.842520in}}%
\pgfusepath{clip}%
\pgfsetbuttcap%
\pgfsetroundjoin%
\definecolor{currentfill}{rgb}{0.281446,0.084320,0.407414}%
\pgfsetfillcolor{currentfill}%
\pgfsetlinewidth{0.000000pt}%
\definecolor{currentstroke}{rgb}{0.252194,0.269783,0.531579}%
\pgfsetstrokecolor{currentstroke}%
\pgfsetdash{}{0pt}%
\pgfpathmoveto{\pgfqpoint{6.455962in}{2.024776in}}%
\pgfpathlineto{\pgfqpoint{6.238160in}{2.086590in}}%
\pgfpathlineto{\pgfqpoint{6.380073in}{1.954750in}}%
\pgfpathclose%
\pgfusepath{fill}%
\end{pgfscope}%
\begin{pgfscope}%
\pgfpathrectangle{\pgfqpoint{0.539299in}{0.078740in}}{\pgfqpoint{7.842520in}{7.842520in}}%
\pgfusepath{clip}%
\pgfsetbuttcap%
\pgfsetroundjoin%
\definecolor{currentfill}{rgb}{0.258965,0.251537,0.524736}%
\pgfsetfillcolor{currentfill}%
\pgfsetlinewidth{0.000000pt}%
\definecolor{currentstroke}{rgb}{0.250425,0.274290,0.533103}%
\pgfsetstrokecolor{currentstroke}%
\pgfsetdash{}{0pt}%
\pgfpathmoveto{\pgfqpoint{1.520051in}{2.413317in}}%
\pgfpathlineto{\pgfqpoint{1.652813in}{2.445145in}}%
\pgfpathlineto{\pgfqpoint{1.726778in}{2.916957in}}%
\pgfpathclose%
\pgfusepath{fill}%
\end{pgfscope}%
\begin{pgfscope}%
\pgfpathrectangle{\pgfqpoint{0.539299in}{0.078740in}}{\pgfqpoint{7.842520in}{7.842520in}}%
\pgfusepath{clip}%
\pgfsetbuttcap%
\pgfsetroundjoin%
\definecolor{currentfill}{rgb}{0.304148,0.764704,0.419943}%
\pgfsetfillcolor{currentfill}%
\pgfsetlinewidth{0.000000pt}%
\definecolor{currentstroke}{rgb}{0.248629,0.278775,0.534556}%
\pgfsetstrokecolor{currentstroke}%
\pgfsetdash{}{0pt}%
\pgfpathmoveto{\pgfqpoint{2.519975in}{5.063797in}}%
\pgfpathlineto{\pgfqpoint{2.391827in}{4.772531in}}%
\pgfpathlineto{\pgfqpoint{2.305286in}{4.666266in}}%
\pgfpathclose%
\pgfusepath{fill}%
\end{pgfscope}%
\begin{pgfscope}%
\pgfpathrectangle{\pgfqpoint{0.539299in}{0.078740in}}{\pgfqpoint{7.842520in}{7.842520in}}%
\pgfusepath{clip}%
\pgfsetbuttcap%
\pgfsetroundjoin%
\definecolor{currentfill}{rgb}{0.162142,0.474838,0.558140}%
\pgfsetfillcolor{currentfill}%
\pgfsetlinewidth{0.000000pt}%
\definecolor{currentstroke}{rgb}{0.246811,0.283237,0.535941}%
\pgfsetstrokecolor{currentstroke}%
\pgfsetdash{}{0pt}%
\pgfpathmoveto{\pgfqpoint{4.968612in}{3.651484in}}%
\pgfpathlineto{\pgfqpoint{5.109597in}{3.444492in}}%
\pgfpathlineto{\pgfqpoint{5.188263in}{3.391183in}}%
\pgfpathclose%
\pgfusepath{fill}%
\end{pgfscope}%
\begin{pgfscope}%
\pgfpathrectangle{\pgfqpoint{0.539299in}{0.078740in}}{\pgfqpoint{7.842520in}{7.842520in}}%
\pgfusepath{clip}%
\pgfsetbuttcap%
\pgfsetroundjoin%
\definecolor{currentfill}{rgb}{0.277941,0.056324,0.381191}%
\pgfsetfillcolor{currentfill}%
\pgfsetlinewidth{0.000000pt}%
\definecolor{currentstroke}{rgb}{0.244972,0.287675,0.537260}%
\pgfsetstrokecolor{currentstroke}%
\pgfsetdash{}{0pt}%
\pgfpathmoveto{\pgfqpoint{6.380073in}{1.954750in}}%
\pgfpathlineto{\pgfqpoint{6.522373in}{1.832361in}}%
\pgfpathlineto{\pgfqpoint{6.455962in}{2.024776in}}%
\pgfpathclose%
\pgfusepath{fill}%
\end{pgfscope}%
\begin{pgfscope}%
\pgfpathrectangle{\pgfqpoint{0.539299in}{0.078740in}}{\pgfqpoint{7.842520in}{7.842520in}}%
\pgfusepath{clip}%
\pgfsetbuttcap%
\pgfsetroundjoin%
\definecolor{currentfill}{rgb}{0.134692,0.658636,0.517649}%
\pgfsetfillcolor{currentfill}%
\pgfsetlinewidth{0.000000pt}%
\definecolor{currentstroke}{rgb}{0.243113,0.292092,0.538516}%
\pgfsetstrokecolor{currentstroke}%
\pgfsetdash{}{0pt}%
\pgfpathmoveto{\pgfqpoint{2.219156in}{4.533644in}}%
\pgfpathlineto{\pgfqpoint{2.009371in}{4.040892in}}%
\pgfpathlineto{\pgfqpoint{2.133656in}{4.370779in}}%
\pgfpathclose%
\pgfusepath{fill}%
\end{pgfscope}%
\begin{pgfscope}%
\pgfpathrectangle{\pgfqpoint{0.539299in}{0.078740in}}{\pgfqpoint{7.842520in}{7.842520in}}%
\pgfusepath{clip}%
\pgfsetbuttcap%
\pgfsetroundjoin%
\definecolor{currentfill}{rgb}{0.216210,0.351535,0.550627}%
\pgfsetfillcolor{currentfill}%
\pgfsetlinewidth{0.000000pt}%
\definecolor{currentstroke}{rgb}{0.241237,0.296485,0.539709}%
\pgfsetstrokecolor{currentstroke}%
\pgfsetdash{}{0pt}%
\pgfpathmoveto{\pgfqpoint{1.726778in}{2.916957in}}%
\pgfpathlineto{\pgfqpoint{1.675575in}{3.148455in}}%
\pgfpathlineto{\pgfqpoint{1.596362in}{2.813436in}}%
\pgfpathclose%
\pgfusepath{fill}%
\end{pgfscope}%
\begin{pgfscope}%
\pgfpathrectangle{\pgfqpoint{0.539299in}{0.078740in}}{\pgfqpoint{7.842520in}{7.842520in}}%
\pgfusepath{clip}%
\pgfsetbuttcap%
\pgfsetroundjoin%
\definecolor{currentfill}{rgb}{0.699415,0.867117,0.175971}%
\pgfsetfillcolor{currentfill}%
\pgfsetlinewidth{0.000000pt}%
\definecolor{currentstroke}{rgb}{0.239346,0.300855,0.540844}%
\pgfsetstrokecolor{currentstroke}%
\pgfsetdash{}{0pt}%
\pgfpathmoveto{\pgfqpoint{2.913349in}{5.441460in}}%
\pgfpathlineto{\pgfqpoint{3.048388in}{5.568482in}}%
\pgfpathlineto{\pgfqpoint{3.135579in}{5.562556in}}%
\pgfpathclose%
\pgfusepath{fill}%
\end{pgfscope}%
\begin{pgfscope}%
\pgfpathrectangle{\pgfqpoint{0.539299in}{0.078740in}}{\pgfqpoint{7.842520in}{7.842520in}}%
\pgfusepath{clip}%
\pgfsetbuttcap%
\pgfsetroundjoin%
\definecolor{currentfill}{rgb}{0.127568,0.566949,0.550556}%
\pgfsetfillcolor{currentfill}%
\pgfsetlinewidth{0.000000pt}%
\definecolor{currentstroke}{rgb}{0.237441,0.305202,0.541921}%
\pgfsetstrokecolor{currentstroke}%
\pgfsetdash{}{0pt}%
\pgfpathmoveto{\pgfqpoint{1.839990in}{3.669004in}}%
\pgfpathlineto{\pgfqpoint{2.049048in}{4.173228in}}%
\pgfpathlineto{\pgfqpoint{1.924234in}{3.870625in}}%
\pgfpathclose%
\pgfusepath{fill}%
\end{pgfscope}%
\begin{pgfscope}%
\pgfpathrectangle{\pgfqpoint{0.539299in}{0.078740in}}{\pgfqpoint{7.842520in}{7.842520in}}%
\pgfusepath{clip}%
\pgfsetbuttcap%
\pgfsetroundjoin%
\definecolor{currentfill}{rgb}{0.121380,0.629492,0.531973}%
\pgfsetfillcolor{currentfill}%
\pgfsetlinewidth{0.000000pt}%
\definecolor{currentstroke}{rgb}{0.235526,0.309527,0.542944}%
\pgfsetstrokecolor{currentstroke}%
\pgfsetdash{}{0pt}%
\pgfpathmoveto{\pgfqpoint{4.766857in}{3.997620in}}%
\pgfpathlineto{\pgfqpoint{4.626092in}{4.211058in}}%
\pgfpathlineto{\pgfqpoint{4.544630in}{4.302516in}}%
\pgfpathclose%
\pgfusepath{fill}%
\end{pgfscope}%
\begin{pgfscope}%
\pgfpathrectangle{\pgfqpoint{0.539299in}{0.078740in}}{\pgfqpoint{7.842520in}{7.842520in}}%
\pgfusepath{clip}%
\pgfsetbuttcap%
\pgfsetroundjoin%
\definecolor{currentfill}{rgb}{0.136408,0.541173,0.554483}%
\pgfsetfillcolor{currentfill}%
\pgfsetlinewidth{0.000000pt}%
\definecolor{currentstroke}{rgb}{0.233603,0.313828,0.543914}%
\pgfsetstrokecolor{currentstroke}%
\pgfsetdash{}{0pt}%
\pgfpathmoveto{\pgfqpoint{4.907447in}{3.788480in}}%
\pgfpathlineto{\pgfqpoint{4.827480in}{3.864804in}}%
\pgfpathlineto{\pgfqpoint{4.968612in}{3.651484in}}%
\pgfpathclose%
\pgfusepath{fill}%
\end{pgfscope}%
\begin{pgfscope}%
\pgfpathrectangle{\pgfqpoint{0.539299in}{0.078740in}}{\pgfqpoint{7.842520in}{7.842520in}}%
\pgfusepath{clip}%
\pgfsetbuttcap%
\pgfsetroundjoin%
\definecolor{currentfill}{rgb}{0.137339,0.662252,0.515571}%
\pgfsetfillcolor{currentfill}%
\pgfsetlinewidth{0.000000pt}%
\definecolor{currentstroke}{rgb}{0.231674,0.318106,0.544834}%
\pgfsetstrokecolor{currentstroke}%
\pgfsetdash{}{0pt}%
\pgfpathmoveto{\pgfqpoint{4.626092in}{4.211058in}}%
\pgfpathlineto{\pgfqpoint{4.485130in}{4.425835in}}%
\pgfpathlineto{\pgfqpoint{4.544630in}{4.302516in}}%
\pgfpathclose%
\pgfusepath{fill}%
\end{pgfscope}%
\begin{pgfscope}%
\pgfpathrectangle{\pgfqpoint{0.539299in}{0.078740in}}{\pgfqpoint{7.842520in}{7.842520in}}%
\pgfusepath{clip}%
\pgfsetbuttcap%
\pgfsetroundjoin%
\definecolor{currentfill}{rgb}{0.177423,0.437527,0.557565}%
\pgfsetfillcolor{currentfill}%
\pgfsetlinewidth{0.000000pt}%
\definecolor{currentstroke}{rgb}{0.229739,0.322361,0.545706}%
\pgfsetstrokecolor{currentstroke}%
\pgfsetdash{}{0pt}%
\pgfpathmoveto{\pgfqpoint{1.756965in}{3.430597in}}%
\pgfpathlineto{\pgfqpoint{1.675575in}{3.148455in}}%
\pgfpathlineto{\pgfqpoint{1.804002in}{3.315816in}}%
\pgfpathclose%
\pgfusepath{fill}%
\end{pgfscope}%
\begin{pgfscope}%
\pgfpathrectangle{\pgfqpoint{0.539299in}{0.078740in}}{\pgfqpoint{7.842520in}{7.842520in}}%
\pgfusepath{clip}%
\pgfsetbuttcap%
\pgfsetroundjoin%
\definecolor{currentfill}{rgb}{0.143343,0.522773,0.556295}%
\pgfsetfillcolor{currentfill}%
\pgfsetlinewidth{0.000000pt}%
\definecolor{currentstroke}{rgb}{0.227802,0.326594,0.546532}%
\pgfsetstrokecolor{currentstroke}%
\pgfsetdash{}{0pt}%
\pgfpathmoveto{\pgfqpoint{1.965641in}{3.935827in}}%
\pgfpathlineto{\pgfqpoint{1.839990in}{3.669004in}}%
\pgfpathlineto{\pgfqpoint{1.756965in}{3.430597in}}%
\pgfpathclose%
\pgfusepath{fill}%
\end{pgfscope}%
\begin{pgfscope}%
\pgfpathrectangle{\pgfqpoint{0.539299in}{0.078740in}}{\pgfqpoint{7.842520in}{7.842520in}}%
\pgfusepath{clip}%
\pgfsetbuttcap%
\pgfsetroundjoin%
\definecolor{currentfill}{rgb}{0.535621,0.835785,0.281908}%
\pgfsetfillcolor{currentfill}%
\pgfsetlinewidth{0.000000pt}%
\definecolor{currentstroke}{rgb}{0.225863,0.330805,0.547314}%
\pgfsetstrokecolor{currentstroke}%
\pgfsetdash{}{0pt}%
\pgfpathmoveto{\pgfqpoint{2.694035in}{5.204781in}}%
\pgfpathlineto{\pgfqpoint{2.606940in}{5.146364in}}%
\pgfpathlineto{\pgfqpoint{2.738785in}{5.362101in}}%
\pgfpathclose%
\pgfusepath{fill}%
\end{pgfscope}%
\begin{pgfscope}%
\pgfpathrectangle{\pgfqpoint{0.539299in}{0.078740in}}{\pgfqpoint{7.842520in}{7.842520in}}%
\pgfusepath{clip}%
\pgfsetbuttcap%
\pgfsetroundjoin%
\definecolor{currentfill}{rgb}{0.123444,0.636809,0.528763}%
\pgfsetfillcolor{currentfill}%
\pgfsetlinewidth{0.000000pt}%
\definecolor{currentstroke}{rgb}{0.223925,0.334994,0.548053}%
\pgfsetstrokecolor{currentstroke}%
\pgfsetdash{}{0pt}%
\pgfpathmoveto{\pgfqpoint{2.133656in}{4.370779in}}%
\pgfpathlineto{\pgfqpoint{2.009371in}{4.040892in}}%
\pgfpathlineto{\pgfqpoint{2.049048in}{4.173228in}}%
\pgfpathclose%
\pgfusepath{fill}%
\end{pgfscope}%
\begin{pgfscope}%
\pgfpathrectangle{\pgfqpoint{0.539299in}{0.078740in}}{\pgfqpoint{7.842520in}{7.842520in}}%
\pgfusepath{clip}%
\pgfsetbuttcap%
\pgfsetroundjoin%
\definecolor{currentfill}{rgb}{0.255645,0.260703,0.528312}%
\pgfsetfillcolor{currentfill}%
\pgfsetlinewidth{0.000000pt}%
\definecolor{currentstroke}{rgb}{0.221989,0.339161,0.548752}%
\pgfsetstrokecolor{currentstroke}%
\pgfsetdash{}{0pt}%
\pgfpathmoveto{\pgfqpoint{5.737117in}{2.526203in}}%
\pgfpathlineto{\pgfqpoint{5.814052in}{2.532529in}}%
\pgfpathlineto{\pgfqpoint{5.673051in}{2.697726in}}%
\pgfpathclose%
\pgfusepath{fill}%
\end{pgfscope}%
\begin{pgfscope}%
\pgfpathrectangle{\pgfqpoint{0.539299in}{0.078740in}}{\pgfqpoint{7.842520in}{7.842520in}}%
\pgfusepath{clip}%
\pgfsetbuttcap%
\pgfsetroundjoin%
\definecolor{currentfill}{rgb}{0.265145,0.232956,0.516599}%
\pgfsetfillcolor{currentfill}%
\pgfsetlinewidth{0.000000pt}%
\definecolor{currentstroke}{rgb}{0.220057,0.343307,0.549413}%
\pgfsetstrokecolor{currentstroke}%
\pgfsetdash{}{0pt}%
\pgfpathmoveto{\pgfqpoint{5.737117in}{2.526203in}}%
\pgfpathlineto{\pgfqpoint{5.955205in}{2.375680in}}%
\pgfpathlineto{\pgfqpoint{5.814052in}{2.532529in}}%
\pgfpathclose%
\pgfusepath{fill}%
\end{pgfscope}%
\begin{pgfscope}%
\pgfpathrectangle{\pgfqpoint{0.539299in}{0.078740in}}{\pgfqpoint{7.842520in}{7.842520in}}%
\pgfusepath{clip}%
\pgfsetbuttcap%
\pgfsetroundjoin%
\definecolor{currentfill}{rgb}{0.751884,0.874951,0.143228}%
\pgfsetfillcolor{currentfill}%
\pgfsetlinewidth{0.000000pt}%
\definecolor{currentstroke}{rgb}{0.218130,0.347432,0.550038}%
\pgfsetstrokecolor{currentstroke}%
\pgfsetdash{}{0pt}%
\pgfpathmoveto{\pgfqpoint{3.048388in}{5.568482in}}%
\pgfpathlineto{\pgfqpoint{3.272857in}{5.610573in}}%
\pgfpathlineto{\pgfqpoint{3.135579in}{5.562556in}}%
\pgfpathclose%
\pgfusepath{fill}%
\end{pgfscope}%
\begin{pgfscope}%
\pgfpathrectangle{\pgfqpoint{0.539299in}{0.078740in}}{\pgfqpoint{7.842520in}{7.842520in}}%
\pgfusepath{clip}%
\pgfsetbuttcap%
\pgfsetroundjoin%
\definecolor{currentfill}{rgb}{0.277134,0.185228,0.489898}%
\pgfsetfillcolor{currentfill}%
\pgfsetlinewidth{0.000000pt}%
\definecolor{currentstroke}{rgb}{0.216210,0.351535,0.550627}%
\pgfsetstrokecolor{currentstroke}%
\pgfsetdash{}{0pt}%
\pgfpathmoveto{\pgfqpoint{6.096558in}{2.227023in}}%
\pgfpathlineto{\pgfqpoint{5.955205in}{2.375680in}}%
\pgfpathlineto{\pgfqpoint{5.878547in}{2.354530in}}%
\pgfpathclose%
\pgfusepath{fill}%
\end{pgfscope}%
\begin{pgfscope}%
\pgfpathrectangle{\pgfqpoint{0.539299in}{0.078740in}}{\pgfqpoint{7.842520in}{7.842520in}}%
\pgfusepath{clip}%
\pgfsetbuttcap%
\pgfsetroundjoin%
\definecolor{currentfill}{rgb}{0.730889,0.871916,0.156029}%
\pgfsetfillcolor{currentfill}%
\pgfsetlinewidth{0.000000pt}%
\definecolor{currentstroke}{rgb}{0.214298,0.355619,0.551184}%
\pgfsetstrokecolor{currentstroke}%
\pgfsetdash{}{0pt}%
\pgfpathmoveto{\pgfqpoint{3.638229in}{5.473510in}}%
\pgfpathlineto{\pgfqpoint{3.498366in}{5.547173in}}%
\pgfpathlineto{\pgfqpoint{3.411816in}{5.600141in}}%
\pgfpathclose%
\pgfusepath{fill}%
\end{pgfscope}%
\begin{pgfscope}%
\pgfpathrectangle{\pgfqpoint{0.539299in}{0.078740in}}{\pgfqpoint{7.842520in}{7.842520in}}%
\pgfusepath{clip}%
\pgfsetbuttcap%
\pgfsetroundjoin%
\definecolor{currentfill}{rgb}{0.235526,0.309527,0.542944}%
\pgfsetfillcolor{currentfill}%
\pgfsetlinewidth{0.000000pt}%
\definecolor{currentstroke}{rgb}{0.212395,0.359683,0.551710}%
\pgfsetstrokecolor{currentstroke}%
\pgfsetdash{}{0pt}%
\pgfpathmoveto{\pgfqpoint{5.532155in}{2.871486in}}%
\pgfpathlineto{\pgfqpoint{5.595772in}{2.706138in}}%
\pgfpathlineto{\pgfqpoint{5.673051in}{2.697726in}}%
\pgfpathclose%
\pgfusepath{fill}%
\end{pgfscope}%
\begin{pgfscope}%
\pgfpathrectangle{\pgfqpoint{0.539299in}{0.078740in}}{\pgfqpoint{7.842520in}{7.842520in}}%
\pgfusepath{clip}%
\pgfsetbuttcap%
\pgfsetroundjoin%
\definecolor{currentfill}{rgb}{0.121148,0.592739,0.544641}%
\pgfsetfillcolor{currentfill}%
\pgfsetlinewidth{0.000000pt}%
\definecolor{currentstroke}{rgb}{0.210503,0.363727,0.552206}%
\pgfsetstrokecolor{currentstroke}%
\pgfsetdash{}{0pt}%
\pgfpathmoveto{\pgfqpoint{4.827480in}{3.864804in}}%
\pgfpathlineto{\pgfqpoint{4.766857in}{3.997620in}}%
\pgfpathlineto{\pgfqpoint{4.686160in}{4.082636in}}%
\pgfpathclose%
\pgfusepath{fill}%
\end{pgfscope}%
\begin{pgfscope}%
\pgfpathrectangle{\pgfqpoint{0.539299in}{0.078740in}}{\pgfqpoint{7.842520in}{7.842520in}}%
\pgfusepath{clip}%
\pgfsetbuttcap%
\pgfsetroundjoin%
\definecolor{currentfill}{rgb}{0.688944,0.865448,0.182725}%
\pgfsetfillcolor{currentfill}%
\pgfsetlinewidth{0.000000pt}%
\definecolor{currentstroke}{rgb}{0.208623,0.367752,0.552675}%
\pgfsetstrokecolor{currentstroke}%
\pgfsetdash{}{0pt}%
\pgfpathmoveto{\pgfqpoint{3.048388in}{5.568482in}}%
\pgfpathlineto{\pgfqpoint{2.913349in}{5.441460in}}%
\pgfpathlineto{\pgfqpoint{2.826074in}{5.413789in}}%
\pgfpathclose%
\pgfusepath{fill}%
\end{pgfscope}%
\begin{pgfscope}%
\pgfpathrectangle{\pgfqpoint{0.539299in}{0.078740in}}{\pgfqpoint{7.842520in}{7.842520in}}%
\pgfusepath{clip}%
\pgfsetbuttcap%
\pgfsetroundjoin%
\definecolor{currentfill}{rgb}{0.283187,0.125848,0.444960}%
\pgfsetfillcolor{currentfill}%
\pgfsetlinewidth{0.000000pt}%
\definecolor{currentstroke}{rgb}{0.206756,0.371758,0.553117}%
\pgfsetstrokecolor{currentstroke}%
\pgfsetdash{}{0pt}%
\pgfpathmoveto{\pgfqpoint{6.161891in}{2.037228in}}%
\pgfpathlineto{\pgfqpoint{6.238160in}{2.086590in}}%
\pgfpathlineto{\pgfqpoint{6.096558in}{2.227023in}}%
\pgfpathclose%
\pgfusepath{fill}%
\end{pgfscope}%
\begin{pgfscope}%
\pgfpathrectangle{\pgfqpoint{0.539299in}{0.078740in}}{\pgfqpoint{7.842520in}{7.842520in}}%
\pgfusepath{clip}%
\pgfsetbuttcap%
\pgfsetroundjoin%
\definecolor{currentfill}{rgb}{0.252194,0.269783,0.531579}%
\pgfsetfillcolor{currentfill}%
\pgfsetlinewidth{0.000000pt}%
\definecolor{currentstroke}{rgb}{0.204903,0.375746,0.553533}%
\pgfsetstrokecolor{currentstroke}%
\pgfsetdash{}{0pt}%
\pgfpathmoveto{\pgfqpoint{1.726778in}{2.916957in}}%
\pgfpathlineto{\pgfqpoint{1.652813in}{2.445145in}}%
\pgfpathlineto{\pgfqpoint{1.786943in}{2.460503in}}%
\pgfpathclose%
\pgfusepath{fill}%
\end{pgfscope}%
\begin{pgfscope}%
\pgfpathrectangle{\pgfqpoint{0.539299in}{0.078740in}}{\pgfqpoint{7.842520in}{7.842520in}}%
\pgfusepath{clip}%
\pgfsetbuttcap%
\pgfsetroundjoin%
\definecolor{currentfill}{rgb}{0.762373,0.876424,0.137064}%
\pgfsetfillcolor{currentfill}%
\pgfsetlinewidth{0.000000pt}%
\definecolor{currentstroke}{rgb}{0.203063,0.379716,0.553925}%
\pgfsetstrokecolor{currentstroke}%
\pgfsetdash{}{0pt}%
\pgfpathmoveto{\pgfqpoint{3.498366in}{5.547173in}}%
\pgfpathlineto{\pgfqpoint{3.272857in}{5.610573in}}%
\pgfpathlineto{\pgfqpoint{3.411816in}{5.600141in}}%
\pgfpathclose%
\pgfusepath{fill}%
\end{pgfscope}%
\begin{pgfscope}%
\pgfpathrectangle{\pgfqpoint{0.539299in}{0.078740in}}{\pgfqpoint{7.842520in}{7.842520in}}%
\pgfusepath{clip}%
\pgfsetbuttcap%
\pgfsetroundjoin%
\definecolor{currentfill}{rgb}{0.206756,0.371758,0.553117}%
\pgfsetfillcolor{currentfill}%
\pgfsetlinewidth{0.000000pt}%
\definecolor{currentstroke}{rgb}{0.201239,0.383670,0.554294}%
\pgfsetstrokecolor{currentstroke}%
\pgfsetdash{}{0pt}%
\pgfpathmoveto{\pgfqpoint{5.532155in}{2.871486in}}%
\pgfpathlineto{\pgfqpoint{5.391316in}{3.053947in}}%
\pgfpathlineto{\pgfqpoint{5.313135in}{3.090200in}}%
\pgfpathclose%
\pgfusepath{fill}%
\end{pgfscope}%
\begin{pgfscope}%
\pgfpathrectangle{\pgfqpoint{0.539299in}{0.078740in}}{\pgfqpoint{7.842520in}{7.842520in}}%
\pgfusepath{clip}%
\pgfsetbuttcap%
\pgfsetroundjoin%
\definecolor{currentfill}{rgb}{0.606045,0.850733,0.236712}%
\pgfsetfillcolor{currentfill}%
\pgfsetlinewidth{0.000000pt}%
\definecolor{currentstroke}{rgb}{0.199430,0.387607,0.554642}%
\pgfsetstrokecolor{currentstroke}%
\pgfsetdash{}{0pt}%
\pgfpathmoveto{\pgfqpoint{2.738785in}{5.362101in}}%
\pgfpathlineto{\pgfqpoint{2.826074in}{5.413789in}}%
\pgfpathlineto{\pgfqpoint{2.694035in}{5.204781in}}%
\pgfpathclose%
\pgfusepath{fill}%
\end{pgfscope}%
\begin{pgfscope}%
\pgfpathrectangle{\pgfqpoint{0.539299in}{0.078740in}}{\pgfqpoint{7.842520in}{7.842520in}}%
\pgfusepath{clip}%
\pgfsetbuttcap%
\pgfsetroundjoin%
\definecolor{currentfill}{rgb}{0.281446,0.084320,0.407414}%
\pgfsetfillcolor{currentfill}%
\pgfsetlinewidth{0.000000pt}%
\definecolor{currentstroke}{rgb}{0.197636,0.391528,0.554969}%
\pgfsetstrokecolor{currentstroke}%
\pgfsetdash{}{0pt}%
\pgfpathmoveto{\pgfqpoint{6.238160in}{2.086590in}}%
\pgfpathlineto{\pgfqpoint{6.303947in}{1.893136in}}%
\pgfpathlineto{\pgfqpoint{6.380073in}{1.954750in}}%
\pgfpathclose%
\pgfusepath{fill}%
\end{pgfscope}%
\begin{pgfscope}%
\pgfpathrectangle{\pgfqpoint{0.539299in}{0.078740in}}{\pgfqpoint{7.842520in}{7.842520in}}%
\pgfusepath{clip}%
\pgfsetbuttcap%
\pgfsetroundjoin%
\definecolor{currentfill}{rgb}{0.226397,0.728888,0.462789}%
\pgfsetfillcolor{currentfill}%
\pgfsetlinewidth{0.000000pt}%
\definecolor{currentstroke}{rgb}{0.195860,0.395433,0.555276}%
\pgfsetstrokecolor{currentstroke}%
\pgfsetdash{}{0pt}%
\pgfpathmoveto{\pgfqpoint{4.260950in}{4.734983in}}%
\pgfpathlineto{\pgfqpoint{4.485130in}{4.425835in}}%
\pgfpathlineto{\pgfqpoint{4.343978in}{4.638176in}}%
\pgfpathclose%
\pgfusepath{fill}%
\end{pgfscope}%
\begin{pgfscope}%
\pgfpathrectangle{\pgfqpoint{0.539299in}{0.078740in}}{\pgfqpoint{7.842520in}{7.842520in}}%
\pgfusepath{clip}%
\pgfsetbuttcap%
\pgfsetroundjoin%
\definecolor{currentfill}{rgb}{0.120638,0.625828,0.533488}%
\pgfsetfillcolor{currentfill}%
\pgfsetlinewidth{0.000000pt}%
\definecolor{currentstroke}{rgb}{0.194100,0.399323,0.555565}%
\pgfsetstrokecolor{currentstroke}%
\pgfsetdash{}{0pt}%
\pgfpathmoveto{\pgfqpoint{4.686160in}{4.082636in}}%
\pgfpathlineto{\pgfqpoint{4.766857in}{3.997620in}}%
\pgfpathlineto{\pgfqpoint{4.544630in}{4.302516in}}%
\pgfpathclose%
\pgfusepath{fill}%
\end{pgfscope}%
\begin{pgfscope}%
\pgfpathrectangle{\pgfqpoint{0.539299in}{0.078740in}}{\pgfqpoint{7.842520in}{7.842520in}}%
\pgfusepath{clip}%
\pgfsetbuttcap%
\pgfsetroundjoin%
\definecolor{currentfill}{rgb}{0.192357,0.403199,0.555836}%
\pgfsetfillcolor{currentfill}%
\pgfsetlinewidth{0.000000pt}%
\definecolor{currentstroke}{rgb}{0.192357,0.403199,0.555836}%
\pgfsetstrokecolor{currentstroke}%
\pgfsetdash{}{0pt}%
\pgfpathmoveto{\pgfqpoint{5.313135in}{3.090200in}}%
\pgfpathlineto{\pgfqpoint{5.391316in}{3.053947in}}%
\pgfpathlineto{\pgfqpoint{5.250481in}{3.245064in}}%
\pgfpathclose%
\pgfusepath{fill}%
\end{pgfscope}%
\begin{pgfscope}%
\pgfpathrectangle{\pgfqpoint{0.539299in}{0.078740in}}{\pgfqpoint{7.842520in}{7.842520in}}%
\pgfusepath{clip}%
\pgfsetbuttcap%
\pgfsetroundjoin%
\definecolor{currentfill}{rgb}{0.192357,0.403199,0.555836}%
\pgfsetfillcolor{currentfill}%
\pgfsetlinewidth{0.000000pt}%
\definecolor{currentstroke}{rgb}{0.190631,0.407061,0.556089}%
\pgfsetstrokecolor{currentstroke}%
\pgfsetdash{}{0pt}%
\pgfpathmoveto{\pgfqpoint{1.804002in}{3.315816in}}%
\pgfpathlineto{\pgfqpoint{1.675575in}{3.148455in}}%
\pgfpathlineto{\pgfqpoint{1.726778in}{2.916957in}}%
\pgfpathclose%
\pgfusepath{fill}%
\end{pgfscope}%
\begin{pgfscope}%
\pgfpathrectangle{\pgfqpoint{0.539299in}{0.078740in}}{\pgfqpoint{7.842520in}{7.842520in}}%
\pgfusepath{clip}%
\pgfsetbuttcap%
\pgfsetroundjoin%
\definecolor{currentfill}{rgb}{0.288921,0.758394,0.428426}%
\pgfsetfillcolor{currentfill}%
\pgfsetlinewidth{0.000000pt}%
\definecolor{currentstroke}{rgb}{0.188923,0.410910,0.556326}%
\pgfsetstrokecolor{currentstroke}%
\pgfsetdash{}{0pt}%
\pgfpathmoveto{\pgfqpoint{4.260950in}{4.734983in}}%
\pgfpathlineto{\pgfqpoint{4.343978in}{4.638176in}}%
\pgfpathlineto{\pgfqpoint{4.202671in}{4.843466in}}%
\pgfpathclose%
\pgfusepath{fill}%
\end{pgfscope}%
\begin{pgfscope}%
\pgfpathrectangle{\pgfqpoint{0.539299in}{0.078740in}}{\pgfqpoint{7.842520in}{7.842520in}}%
\pgfusepath{clip}%
\pgfsetbuttcap%
\pgfsetroundjoin%
\definecolor{currentfill}{rgb}{0.276022,0.044167,0.370164}%
\pgfsetfillcolor{currentfill}%
\pgfsetlinewidth{0.000000pt}%
\definecolor{currentstroke}{rgb}{0.187231,0.414746,0.556547}%
\pgfsetstrokecolor{currentstroke}%
\pgfsetdash{}{0pt}%
\pgfpathmoveto{\pgfqpoint{6.446395in}{1.760841in}}%
\pgfpathlineto{\pgfqpoint{6.522373in}{1.832361in}}%
\pgfpathlineto{\pgfqpoint{6.380073in}{1.954750in}}%
\pgfpathclose%
\pgfusepath{fill}%
\end{pgfscope}%
\begin{pgfscope}%
\pgfpathrectangle{\pgfqpoint{0.539299in}{0.078740in}}{\pgfqpoint{7.842520in}{7.842520in}}%
\pgfusepath{clip}%
\pgfsetbuttcap%
\pgfsetroundjoin%
\definecolor{currentfill}{rgb}{0.688944,0.865448,0.182725}%
\pgfsetfillcolor{currentfill}%
\pgfsetlinewidth{0.000000pt}%
\definecolor{currentstroke}{rgb}{0.185556,0.418570,0.556753}%
\pgfsetstrokecolor{currentstroke}%
\pgfsetdash{}{0pt}%
\pgfpathmoveto{\pgfqpoint{3.778860in}{5.358543in}}%
\pgfpathlineto{\pgfqpoint{3.638229in}{5.473510in}}%
\pgfpathlineto{\pgfqpoint{3.552031in}{5.538964in}}%
\pgfpathclose%
\pgfusepath{fill}%
\end{pgfscope}%
\begin{pgfscope}%
\pgfpathrectangle{\pgfqpoint{0.539299in}{0.078740in}}{\pgfqpoint{7.842520in}{7.842520in}}%
\pgfusepath{clip}%
\pgfsetbuttcap%
\pgfsetroundjoin%
\definecolor{currentfill}{rgb}{0.267968,0.223549,0.512008}%
\pgfsetfillcolor{currentfill}%
\pgfsetlinewidth{0.000000pt}%
\definecolor{currentstroke}{rgb}{0.183898,0.422383,0.556944}%
\pgfsetstrokecolor{currentstroke}%
\pgfsetdash{}{0pt}%
\pgfpathmoveto{\pgfqpoint{5.878547in}{2.354530in}}%
\pgfpathlineto{\pgfqpoint{5.955205in}{2.375680in}}%
\pgfpathlineto{\pgfqpoint{5.737117in}{2.526203in}}%
\pgfpathclose%
\pgfusepath{fill}%
\end{pgfscope}%
\begin{pgfscope}%
\pgfpathrectangle{\pgfqpoint{0.539299in}{0.078740in}}{\pgfqpoint{7.842520in}{7.842520in}}%
\pgfusepath{clip}%
\pgfsetbuttcap%
\pgfsetroundjoin%
\definecolor{currentfill}{rgb}{0.278826,0.175490,0.483397}%
\pgfsetfillcolor{currentfill}%
\pgfsetlinewidth{0.000000pt}%
\definecolor{currentstroke}{rgb}{0.182256,0.426184,0.557120}%
\pgfsetstrokecolor{currentstroke}%
\pgfsetdash{}{0pt}%
\pgfpathmoveto{\pgfqpoint{6.096558in}{2.227023in}}%
\pgfpathlineto{\pgfqpoint{5.878547in}{2.354530in}}%
\pgfpathlineto{\pgfqpoint{6.020118in}{2.191364in}}%
\pgfpathclose%
\pgfusepath{fill}%
\end{pgfscope}%
\begin{pgfscope}%
\pgfpathrectangle{\pgfqpoint{0.539299in}{0.078740in}}{\pgfqpoint{7.842520in}{7.842520in}}%
\pgfusepath{clip}%
\pgfsetbuttcap%
\pgfsetroundjoin%
\definecolor{currentfill}{rgb}{0.282290,0.145912,0.461510}%
\pgfsetfillcolor{currentfill}%
\pgfsetlinewidth{0.000000pt}%
\definecolor{currentstroke}{rgb}{0.180629,0.429975,0.557282}%
\pgfsetstrokecolor{currentstroke}%
\pgfsetdash{}{0pt}%
\pgfpathmoveto{\pgfqpoint{6.020118in}{2.191364in}}%
\pgfpathlineto{\pgfqpoint{6.161891in}{2.037228in}}%
\pgfpathlineto{\pgfqpoint{6.096558in}{2.227023in}}%
\pgfpathclose%
\pgfusepath{fill}%
\end{pgfscope}%
\begin{pgfscope}%
\pgfpathrectangle{\pgfqpoint{0.539299in}{0.078740in}}{\pgfqpoint{7.842520in}{7.842520in}}%
\pgfusepath{clip}%
\pgfsetbuttcap%
\pgfsetroundjoin%
\definecolor{currentfill}{rgb}{0.244972,0.287675,0.537260}%
\pgfsetfillcolor{currentfill}%
\pgfsetlinewidth{0.000000pt}%
\definecolor{currentstroke}{rgb}{0.179019,0.433756,0.557430}%
\pgfsetstrokecolor{currentstroke}%
\pgfsetdash{}{0pt}%
\pgfpathmoveto{\pgfqpoint{5.673051in}{2.697726in}}%
\pgfpathlineto{\pgfqpoint{5.595772in}{2.706138in}}%
\pgfpathlineto{\pgfqpoint{5.737117in}{2.526203in}}%
\pgfpathclose%
\pgfusepath{fill}%
\end{pgfscope}%
\begin{pgfscope}%
\pgfpathrectangle{\pgfqpoint{0.539299in}{0.078740in}}{\pgfqpoint{7.842520in}{7.842520in}}%
\pgfusepath{clip}%
\pgfsetbuttcap%
\pgfsetroundjoin%
\definecolor{currentfill}{rgb}{0.122606,0.585371,0.546557}%
\pgfsetfillcolor{currentfill}%
\pgfsetlinewidth{0.000000pt}%
\definecolor{currentstroke}{rgb}{0.177423,0.437527,0.557565}%
\pgfsetstrokecolor{currentstroke}%
\pgfsetdash{}{0pt}%
\pgfpathmoveto{\pgfqpoint{1.965641in}{3.935827in}}%
\pgfpathlineto{\pgfqpoint{2.049048in}{4.173228in}}%
\pgfpathlineto{\pgfqpoint{1.839990in}{3.669004in}}%
\pgfpathclose%
\pgfusepath{fill}%
\end{pgfscope}%
\begin{pgfscope}%
\pgfpathrectangle{\pgfqpoint{0.539299in}{0.078740in}}{\pgfqpoint{7.842520in}{7.842520in}}%
\pgfusepath{clip}%
\pgfsetbuttcap%
\pgfsetroundjoin%
\definecolor{currentfill}{rgb}{0.165117,0.467423,0.558141}%
\pgfsetfillcolor{currentfill}%
\pgfsetlinewidth{0.000000pt}%
\definecolor{currentstroke}{rgb}{0.175841,0.441290,0.557685}%
\pgfsetstrokecolor{currentstroke}%
\pgfsetdash{}{0pt}%
\pgfpathmoveto{\pgfqpoint{5.250481in}{3.245064in}}%
\pgfpathlineto{\pgfqpoint{5.109597in}{3.444492in}}%
\pgfpathlineto{\pgfqpoint{5.030239in}{3.504331in}}%
\pgfpathclose%
\pgfusepath{fill}%
\end{pgfscope}%
\begin{pgfscope}%
\pgfpathrectangle{\pgfqpoint{0.539299in}{0.078740in}}{\pgfqpoint{7.842520in}{7.842520in}}%
\pgfusepath{clip}%
\pgfsetbuttcap%
\pgfsetroundjoin%
\definecolor{currentfill}{rgb}{0.369214,0.788888,0.382914}%
\pgfsetfillcolor{currentfill}%
\pgfsetlinewidth{0.000000pt}%
\definecolor{currentstroke}{rgb}{0.174274,0.445044,0.557792}%
\pgfsetstrokecolor{currentstroke}%
\pgfsetdash{}{0pt}%
\pgfpathmoveto{\pgfqpoint{2.305286in}{4.666266in}}%
\pgfpathlineto{\pgfqpoint{2.433328in}{4.953363in}}%
\pgfpathlineto{\pgfqpoint{2.519975in}{5.063797in}}%
\pgfpathclose%
\pgfusepath{fill}%
\end{pgfscope}%
\begin{pgfscope}%
\pgfpathrectangle{\pgfqpoint{0.539299in}{0.078740in}}{\pgfqpoint{7.842520in}{7.842520in}}%
\pgfusepath{clip}%
\pgfsetbuttcap%
\pgfsetroundjoin%
\definecolor{currentfill}{rgb}{0.259857,0.745492,0.444467}%
\pgfsetfillcolor{currentfill}%
\pgfsetlinewidth{0.000000pt}%
\definecolor{currentstroke}{rgb}{0.172719,0.448791,0.557885}%
\pgfsetstrokecolor{currentstroke}%
\pgfsetdash{}{0pt}%
\pgfpathmoveto{\pgfqpoint{2.305286in}{4.666266in}}%
\pgfpathlineto{\pgfqpoint{2.219156in}{4.533644in}}%
\pgfpathlineto{\pgfqpoint{2.347219in}{4.810962in}}%
\pgfpathclose%
\pgfusepath{fill}%
\end{pgfscope}%
\begin{pgfscope}%
\pgfpathrectangle{\pgfqpoint{0.539299in}{0.078740in}}{\pgfqpoint{7.842520in}{7.842520in}}%
\pgfusepath{clip}%
\pgfsetbuttcap%
\pgfsetroundjoin%
\definecolor{currentfill}{rgb}{0.386433,0.794644,0.372886}%
\pgfsetfillcolor{currentfill}%
\pgfsetlinewidth{0.000000pt}%
\definecolor{currentstroke}{rgb}{0.171176,0.452530,0.557965}%
\pgfsetstrokecolor{currentstroke}%
\pgfsetdash{}{0pt}%
\pgfpathmoveto{\pgfqpoint{4.061290in}{5.036247in}}%
\pgfpathlineto{\pgfqpoint{4.118884in}{4.938971in}}%
\pgfpathlineto{\pgfqpoint{4.202671in}{4.843466in}}%
\pgfpathclose%
\pgfusepath{fill}%
\end{pgfscope}%
\begin{pgfscope}%
\pgfpathrectangle{\pgfqpoint{0.539299in}{0.078740in}}{\pgfqpoint{7.842520in}{7.842520in}}%
\pgfusepath{clip}%
\pgfsetbuttcap%
\pgfsetroundjoin%
\definecolor{currentfill}{rgb}{0.282910,0.105393,0.426902}%
\pgfsetfillcolor{currentfill}%
\pgfsetlinewidth{0.000000pt}%
\definecolor{currentstroke}{rgb}{0.169646,0.456262,0.558030}%
\pgfsetstrokecolor{currentstroke}%
\pgfsetdash{}{0pt}%
\pgfpathmoveto{\pgfqpoint{6.238160in}{2.086590in}}%
\pgfpathlineto{\pgfqpoint{6.161891in}{2.037228in}}%
\pgfpathlineto{\pgfqpoint{6.303947in}{1.893136in}}%
\pgfpathclose%
\pgfusepath{fill}%
\end{pgfscope}%
\begin{pgfscope}%
\pgfpathrectangle{\pgfqpoint{0.539299in}{0.078740in}}{\pgfqpoint{7.842520in}{7.842520in}}%
\pgfusepath{clip}%
\pgfsetbuttcap%
\pgfsetroundjoin%
\definecolor{currentfill}{rgb}{0.223925,0.334994,0.548053}%
\pgfsetfillcolor{currentfill}%
\pgfsetlinewidth{0.000000pt}%
\definecolor{currentstroke}{rgb}{0.168126,0.459988,0.558082}%
\pgfsetstrokecolor{currentstroke}%
\pgfsetdash{}{0pt}%
\pgfpathmoveto{\pgfqpoint{5.532155in}{2.871486in}}%
\pgfpathlineto{\pgfqpoint{5.454461in}{2.894198in}}%
\pgfpathlineto{\pgfqpoint{5.595772in}{2.706138in}}%
\pgfpathclose%
\pgfusepath{fill}%
\end{pgfscope}%
\begin{pgfscope}%
\pgfpathrectangle{\pgfqpoint{0.539299in}{0.078740in}}{\pgfqpoint{7.842520in}{7.842520in}}%
\pgfusepath{clip}%
\pgfsetbuttcap%
\pgfsetroundjoin%
\definecolor{currentfill}{rgb}{0.157729,0.485932,0.558013}%
\pgfsetfillcolor{currentfill}%
\pgfsetlinewidth{0.000000pt}%
\definecolor{currentstroke}{rgb}{0.166617,0.463708,0.558119}%
\pgfsetstrokecolor{currentstroke}%
\pgfsetdash{}{0pt}%
\pgfpathmoveto{\pgfqpoint{1.804002in}{3.315816in}}%
\pgfpathlineto{\pgfqpoint{1.883808in}{3.652469in}}%
\pgfpathlineto{\pgfqpoint{1.756965in}{3.430597in}}%
\pgfpathclose%
\pgfusepath{fill}%
\end{pgfscope}%
\begin{pgfscope}%
\pgfpathrectangle{\pgfqpoint{0.539299in}{0.078740in}}{\pgfqpoint{7.842520in}{7.842520in}}%
\pgfusepath{clip}%
\pgfsetbuttcap%
\pgfsetroundjoin%
\definecolor{currentfill}{rgb}{0.170948,0.694384,0.493803}%
\pgfsetfillcolor{currentfill}%
\pgfsetlinewidth{0.000000pt}%
\definecolor{currentstroke}{rgb}{0.165117,0.467423,0.558141}%
\pgfsetstrokecolor{currentstroke}%
\pgfsetdash{}{0pt}%
\pgfpathmoveto{\pgfqpoint{4.544630in}{4.302516in}}%
\pgfpathlineto{\pgfqpoint{4.485130in}{4.425835in}}%
\pgfpathlineto{\pgfqpoint{4.402885in}{4.521271in}}%
\pgfpathclose%
\pgfusepath{fill}%
\end{pgfscope}%
\begin{pgfscope}%
\pgfpathrectangle{\pgfqpoint{0.539299in}{0.078740in}}{\pgfqpoint{7.842520in}{7.842520in}}%
\pgfusepath{clip}%
\pgfsetbuttcap%
\pgfsetroundjoin%
\definecolor{currentfill}{rgb}{0.151918,0.500685,0.557587}%
\pgfsetfillcolor{currentfill}%
\pgfsetlinewidth{0.000000pt}%
\definecolor{currentstroke}{rgb}{0.163625,0.471133,0.558148}%
\pgfsetstrokecolor{currentstroke}%
\pgfsetdash{}{0pt}%
\pgfpathmoveto{\pgfqpoint{5.030239in}{3.504331in}}%
\pgfpathlineto{\pgfqpoint{5.109597in}{3.444492in}}%
\pgfpathlineto{\pgfqpoint{4.968612in}{3.651484in}}%
\pgfpathclose%
\pgfusepath{fill}%
\end{pgfscope}%
\begin{pgfscope}%
\pgfpathrectangle{\pgfqpoint{0.539299in}{0.078740in}}{\pgfqpoint{7.842520in}{7.842520in}}%
\pgfusepath{clip}%
\pgfsetbuttcap%
\pgfsetroundjoin%
\definecolor{currentfill}{rgb}{0.535621,0.835785,0.281908}%
\pgfsetfillcolor{currentfill}%
\pgfsetlinewidth{0.000000pt}%
\definecolor{currentstroke}{rgb}{0.162142,0.474838,0.558140}%
\pgfsetstrokecolor{currentstroke}%
\pgfsetdash{}{0pt}%
\pgfpathmoveto{\pgfqpoint{2.738785in}{5.362101in}}%
\pgfpathlineto{\pgfqpoint{2.606940in}{5.146364in}}%
\pgfpathlineto{\pgfqpoint{2.519975in}{5.063797in}}%
\pgfpathclose%
\pgfusepath{fill}%
\end{pgfscope}%
\begin{pgfscope}%
\pgfpathrectangle{\pgfqpoint{0.539299in}{0.078740in}}{\pgfqpoint{7.842520in}{7.842520in}}%
\pgfusepath{clip}%
\pgfsetbuttcap%
\pgfsetroundjoin%
\definecolor{currentfill}{rgb}{0.595839,0.848717,0.243329}%
\pgfsetfillcolor{currentfill}%
\pgfsetlinewidth{0.000000pt}%
\definecolor{currentstroke}{rgb}{0.160665,0.478540,0.558115}%
\pgfsetstrokecolor{currentstroke}%
\pgfsetdash{}{0pt}%
\pgfpathmoveto{\pgfqpoint{3.834801in}{5.295343in}}%
\pgfpathlineto{\pgfqpoint{3.919960in}{5.210264in}}%
\pgfpathlineto{\pgfqpoint{3.778860in}{5.358543in}}%
\pgfpathclose%
\pgfusepath{fill}%
\end{pgfscope}%
\begin{pgfscope}%
\pgfpathrectangle{\pgfqpoint{0.539299in}{0.078740in}}{\pgfqpoint{7.842520in}{7.842520in}}%
\pgfusepath{clip}%
\pgfsetbuttcap%
\pgfsetroundjoin%
\definecolor{currentfill}{rgb}{0.525776,0.833491,0.288127}%
\pgfsetfillcolor{currentfill}%
\pgfsetlinewidth{0.000000pt}%
\definecolor{currentstroke}{rgb}{0.159194,0.482237,0.558073}%
\pgfsetstrokecolor{currentstroke}%
\pgfsetdash{}{0pt}%
\pgfpathmoveto{\pgfqpoint{4.061290in}{5.036247in}}%
\pgfpathlineto{\pgfqpoint{3.919960in}{5.210264in}}%
\pgfpathlineto{\pgfqpoint{3.834801in}{5.295343in}}%
\pgfpathclose%
\pgfusepath{fill}%
\end{pgfscope}%
\begin{pgfscope}%
\pgfpathrectangle{\pgfqpoint{0.539299in}{0.078740in}}{\pgfqpoint{7.842520in}{7.842520in}}%
\pgfusepath{clip}%
\pgfsetbuttcap%
\pgfsetroundjoin%
\definecolor{currentfill}{rgb}{0.208623,0.367752,0.552675}%
\pgfsetfillcolor{currentfill}%
\pgfsetlinewidth{0.000000pt}%
\definecolor{currentstroke}{rgb}{0.157729,0.485932,0.558013}%
\pgfsetstrokecolor{currentstroke}%
\pgfsetdash{}{0pt}%
\pgfpathmoveto{\pgfqpoint{5.313135in}{3.090200in}}%
\pgfpathlineto{\pgfqpoint{5.454461in}{2.894198in}}%
\pgfpathlineto{\pgfqpoint{5.532155in}{2.871486in}}%
\pgfpathclose%
\pgfusepath{fill}%
\end{pgfscope}%
\begin{pgfscope}%
\pgfpathrectangle{\pgfqpoint{0.539299in}{0.078740in}}{\pgfqpoint{7.842520in}{7.842520in}}%
\pgfusepath{clip}%
\pgfsetbuttcap%
\pgfsetroundjoin%
\definecolor{currentfill}{rgb}{0.139147,0.533812,0.555298}%
\pgfsetfillcolor{currentfill}%
\pgfsetlinewidth{0.000000pt}%
\definecolor{currentstroke}{rgb}{0.156270,0.489624,0.557936}%
\pgfsetstrokecolor{currentstroke}%
\pgfsetdash{}{0pt}%
\pgfpathmoveto{\pgfqpoint{1.756965in}{3.430597in}}%
\pgfpathlineto{\pgfqpoint{1.883808in}{3.652469in}}%
\pgfpathlineto{\pgfqpoint{1.965641in}{3.935827in}}%
\pgfpathclose%
\pgfusepath{fill}%
\end{pgfscope}%
\begin{pgfscope}%
\pgfpathrectangle{\pgfqpoint{0.539299in}{0.078740in}}{\pgfqpoint{7.842520in}{7.842520in}}%
\pgfusepath{clip}%
\pgfsetbuttcap%
\pgfsetroundjoin%
\definecolor{currentfill}{rgb}{0.278791,0.062145,0.386592}%
\pgfsetfillcolor{currentfill}%
\pgfsetlinewidth{0.000000pt}%
\definecolor{currentstroke}{rgb}{0.154815,0.493313,0.557840}%
\pgfsetstrokecolor{currentstroke}%
\pgfsetdash{}{0pt}%
\pgfpathmoveto{\pgfqpoint{6.380073in}{1.954750in}}%
\pgfpathlineto{\pgfqpoint{6.303947in}{1.893136in}}%
\pgfpathlineto{\pgfqpoint{6.446395in}{1.760841in}}%
\pgfpathclose%
\pgfusepath{fill}%
\end{pgfscope}%
\begin{pgfscope}%
\pgfpathrectangle{\pgfqpoint{0.539299in}{0.078740in}}{\pgfqpoint{7.842520in}{7.842520in}}%
\pgfusepath{clip}%
\pgfsetbuttcap%
\pgfsetroundjoin%
\definecolor{currentfill}{rgb}{0.751884,0.874951,0.143228}%
\pgfsetfillcolor{currentfill}%
\pgfsetlinewidth{0.000000pt}%
\definecolor{currentstroke}{rgb}{0.153364,0.497000,0.557724}%
\pgfsetstrokecolor{currentstroke}%
\pgfsetdash{}{0pt}%
\pgfpathmoveto{\pgfqpoint{3.411816in}{5.600141in}}%
\pgfpathlineto{\pgfqpoint{3.552031in}{5.538964in}}%
\pgfpathlineto{\pgfqpoint{3.638229in}{5.473510in}}%
\pgfpathclose%
\pgfusepath{fill}%
\end{pgfscope}%
\begin{pgfscope}%
\pgfpathrectangle{\pgfqpoint{0.539299in}{0.078740in}}{\pgfqpoint{7.842520in}{7.842520in}}%
\pgfusepath{clip}%
\pgfsetbuttcap%
\pgfsetroundjoin%
\definecolor{currentfill}{rgb}{0.220124,0.725509,0.466226}%
\pgfsetfillcolor{currentfill}%
\pgfsetlinewidth{0.000000pt}%
\definecolor{currentstroke}{rgb}{0.151918,0.500685,0.557587}%
\pgfsetstrokecolor{currentstroke}%
\pgfsetdash{}{0pt}%
\pgfpathmoveto{\pgfqpoint{4.402885in}{4.521271in}}%
\pgfpathlineto{\pgfqpoint{4.485130in}{4.425835in}}%
\pgfpathlineto{\pgfqpoint{4.260950in}{4.734983in}}%
\pgfpathclose%
\pgfusepath{fill}%
\end{pgfscope}%
\begin{pgfscope}%
\pgfpathrectangle{\pgfqpoint{0.539299in}{0.078740in}}{\pgfqpoint{7.842520in}{7.842520in}}%
\pgfusepath{clip}%
\pgfsetbuttcap%
\pgfsetroundjoin%
\definecolor{currentfill}{rgb}{0.225863,0.330805,0.547314}%
\pgfsetfillcolor{currentfill}%
\pgfsetlinewidth{0.000000pt}%
\definecolor{currentstroke}{rgb}{0.150476,0.504369,0.557430}%
\pgfsetstrokecolor{currentstroke}%
\pgfsetdash{}{0pt}%
\pgfpathmoveto{\pgfqpoint{1.786943in}{2.460503in}}%
\pgfpathlineto{\pgfqpoint{1.859345in}{2.989662in}}%
\pgfpathlineto{\pgfqpoint{1.726778in}{2.916957in}}%
\pgfpathclose%
\pgfusepath{fill}%
\end{pgfscope}%
\begin{pgfscope}%
\pgfpathrectangle{\pgfqpoint{0.539299in}{0.078740in}}{\pgfqpoint{7.842520in}{7.842520in}}%
\pgfusepath{clip}%
\pgfsetbuttcap%
\pgfsetroundjoin%
\definecolor{currentfill}{rgb}{0.804182,0.882046,0.114965}%
\pgfsetfillcolor{currentfill}%
\pgfsetlinewidth{0.000000pt}%
\definecolor{currentstroke}{rgb}{0.149039,0.508051,0.557250}%
\pgfsetstrokecolor{currentstroke}%
\pgfsetdash{}{0pt}%
\pgfpathmoveto{\pgfqpoint{3.185707in}{5.629892in}}%
\pgfpathlineto{\pgfqpoint{3.272857in}{5.610573in}}%
\pgfpathlineto{\pgfqpoint{3.048388in}{5.568482in}}%
\pgfpathclose%
\pgfusepath{fill}%
\end{pgfscope}%
\begin{pgfscope}%
\pgfpathrectangle{\pgfqpoint{0.539299in}{0.078740in}}{\pgfqpoint{7.842520in}{7.842520in}}%
\pgfusepath{clip}%
\pgfsetbuttcap%
\pgfsetroundjoin%
\definecolor{currentfill}{rgb}{0.180629,0.429975,0.557282}%
\pgfsetfillcolor{currentfill}%
\pgfsetlinewidth{0.000000pt}%
\definecolor{currentstroke}{rgb}{0.147607,0.511733,0.557049}%
\pgfsetstrokecolor{currentstroke}%
\pgfsetdash{}{0pt}%
\pgfpathmoveto{\pgfqpoint{5.171743in}{3.293790in}}%
\pgfpathlineto{\pgfqpoint{5.313135in}{3.090200in}}%
\pgfpathlineto{\pgfqpoint{5.250481in}{3.245064in}}%
\pgfpathclose%
\pgfusepath{fill}%
\end{pgfscope}%
\begin{pgfscope}%
\pgfpathrectangle{\pgfqpoint{0.539299in}{0.078740in}}{\pgfqpoint{7.842520in}{7.842520in}}%
\pgfusepath{clip}%
\pgfsetbuttcap%
\pgfsetroundjoin%
\definecolor{currentfill}{rgb}{0.751884,0.874951,0.143228}%
\pgfsetfillcolor{currentfill}%
\pgfsetlinewidth{0.000000pt}%
\definecolor{currentstroke}{rgb}{0.146180,0.515413,0.556823}%
\pgfsetstrokecolor{currentstroke}%
\pgfsetdash{}{0pt}%
\pgfpathmoveto{\pgfqpoint{2.826074in}{5.413789in}}%
\pgfpathlineto{\pgfqpoint{2.961006in}{5.551827in}}%
\pgfpathlineto{\pgfqpoint{3.048388in}{5.568482in}}%
\pgfpathclose%
\pgfusepath{fill}%
\end{pgfscope}%
\begin{pgfscope}%
\pgfpathrectangle{\pgfqpoint{0.539299in}{0.078740in}}{\pgfqpoint{7.842520in}{7.842520in}}%
\pgfusepath{clip}%
\pgfsetbuttcap%
\pgfsetroundjoin%
\definecolor{currentfill}{rgb}{0.232815,0.732247,0.459277}%
\pgfsetfillcolor{currentfill}%
\pgfsetlinewidth{0.000000pt}%
\definecolor{currentstroke}{rgb}{0.144759,0.519093,0.556572}%
\pgfsetstrokecolor{currentstroke}%
\pgfsetdash{}{0pt}%
\pgfpathmoveto{\pgfqpoint{2.347219in}{4.810962in}}%
\pgfpathlineto{\pgfqpoint{2.219156in}{4.533644in}}%
\pgfpathlineto{\pgfqpoint{2.133656in}{4.370779in}}%
\pgfpathclose%
\pgfusepath{fill}%
\end{pgfscope}%
\begin{pgfscope}%
\pgfpathrectangle{\pgfqpoint{0.539299in}{0.078740in}}{\pgfqpoint{7.842520in}{7.842520in}}%
\pgfusepath{clip}%
\pgfsetbuttcap%
\pgfsetroundjoin%
\definecolor{currentfill}{rgb}{0.126453,0.570633,0.549841}%
\pgfsetfillcolor{currentfill}%
\pgfsetlinewidth{0.000000pt}%
\definecolor{currentstroke}{rgb}{0.143343,0.522773,0.556295}%
\pgfsetstrokecolor{currentstroke}%
\pgfsetdash{}{0pt}%
\pgfpathmoveto{\pgfqpoint{4.968612in}{3.651484in}}%
\pgfpathlineto{\pgfqpoint{4.827480in}{3.864804in}}%
\pgfpathlineto{\pgfqpoint{4.746722in}{3.941721in}}%
\pgfpathclose%
\pgfusepath{fill}%
\end{pgfscope}%
\begin{pgfscope}%
\pgfpathrectangle{\pgfqpoint{0.539299in}{0.078740in}}{\pgfqpoint{7.842520in}{7.842520in}}%
\pgfusepath{clip}%
\pgfsetbuttcap%
\pgfsetroundjoin%
\definecolor{currentfill}{rgb}{0.166617,0.463708,0.558119}%
\pgfsetfillcolor{currentfill}%
\pgfsetlinewidth{0.000000pt}%
\definecolor{currentstroke}{rgb}{0.141935,0.526453,0.555991}%
\pgfsetstrokecolor{currentstroke}%
\pgfsetdash{}{0pt}%
\pgfpathmoveto{\pgfqpoint{5.250481in}{3.245064in}}%
\pgfpathlineto{\pgfqpoint{5.030239in}{3.504331in}}%
\pgfpathlineto{\pgfqpoint{5.171743in}{3.293790in}}%
\pgfpathclose%
\pgfusepath{fill}%
\end{pgfscope}%
\begin{pgfscope}%
\pgfpathrectangle{\pgfqpoint{0.539299in}{0.078740in}}{\pgfqpoint{7.842520in}{7.842520in}}%
\pgfusepath{clip}%
\pgfsetbuttcap%
\pgfsetroundjoin%
\definecolor{currentfill}{rgb}{0.344074,0.780029,0.397381}%
\pgfsetfillcolor{currentfill}%
\pgfsetlinewidth{0.000000pt}%
\definecolor{currentstroke}{rgb}{0.140536,0.530132,0.555659}%
\pgfsetstrokecolor{currentstroke}%
\pgfsetdash{}{0pt}%
\pgfpathmoveto{\pgfqpoint{2.347219in}{4.810962in}}%
\pgfpathlineto{\pgfqpoint{2.433328in}{4.953363in}}%
\pgfpathlineto{\pgfqpoint{2.305286in}{4.666266in}}%
\pgfpathclose%
\pgfusepath{fill}%
\end{pgfscope}%
\begin{pgfscope}%
\pgfpathrectangle{\pgfqpoint{0.539299in}{0.078740in}}{\pgfqpoint{7.842520in}{7.842520in}}%
\pgfusepath{clip}%
\pgfsetbuttcap%
\pgfsetroundjoin%
\definecolor{currentfill}{rgb}{0.352360,0.783011,0.392636}%
\pgfsetfillcolor{currentfill}%
\pgfsetlinewidth{0.000000pt}%
\definecolor{currentstroke}{rgb}{0.139147,0.533812,0.555298}%
\pgfsetstrokecolor{currentstroke}%
\pgfsetdash{}{0pt}%
\pgfpathmoveto{\pgfqpoint{4.202671in}{4.843466in}}%
\pgfpathlineto{\pgfqpoint{4.118884in}{4.938971in}}%
\pgfpathlineto{\pgfqpoint{4.260950in}{4.734983in}}%
\pgfpathclose%
\pgfusepath{fill}%
\end{pgfscope}%
\begin{pgfscope}%
\pgfpathrectangle{\pgfqpoint{0.539299in}{0.078740in}}{\pgfqpoint{7.842520in}{7.842520in}}%
\pgfusepath{clip}%
\pgfsetbuttcap%
\pgfsetroundjoin%
\definecolor{currentfill}{rgb}{0.709898,0.868751,0.169257}%
\pgfsetfillcolor{currentfill}%
\pgfsetlinewidth{0.000000pt}%
\definecolor{currentstroke}{rgb}{0.137770,0.537492,0.554906}%
\pgfsetstrokecolor{currentstroke}%
\pgfsetdash{}{0pt}%
\pgfpathmoveto{\pgfqpoint{3.552031in}{5.538964in}}%
\pgfpathlineto{\pgfqpoint{3.693131in}{5.434822in}}%
\pgfpathlineto{\pgfqpoint{3.778860in}{5.358543in}}%
\pgfpathclose%
\pgfusepath{fill}%
\end{pgfscope}%
\begin{pgfscope}%
\pgfpathrectangle{\pgfqpoint{0.539299in}{0.078740in}}{\pgfqpoint{7.842520in}{7.842520in}}%
\pgfusepath{clip}%
\pgfsetbuttcap%
\pgfsetroundjoin%
\definecolor{currentfill}{rgb}{0.241237,0.296485,0.539709}%
\pgfsetfillcolor{currentfill}%
\pgfsetlinewidth{0.000000pt}%
\definecolor{currentstroke}{rgb}{0.136408,0.541173,0.554483}%
\pgfsetstrokecolor{currentstroke}%
\pgfsetdash{}{0pt}%
\pgfpathmoveto{\pgfqpoint{1.786943in}{2.460503in}}%
\pgfpathlineto{\pgfqpoint{1.922231in}{2.462670in}}%
\pgfpathlineto{\pgfqpoint{1.859345in}{2.989662in}}%
\pgfpathclose%
\pgfusepath{fill}%
\end{pgfscope}%
\begin{pgfscope}%
\pgfpathrectangle{\pgfqpoint{0.539299in}{0.078740in}}{\pgfqpoint{7.842520in}{7.842520in}}%
\pgfusepath{clip}%
\pgfsetbuttcap%
\pgfsetroundjoin%
\definecolor{currentfill}{rgb}{0.468053,0.818921,0.323998}%
\pgfsetfillcolor{currentfill}%
\pgfsetlinewidth{0.000000pt}%
\definecolor{currentstroke}{rgb}{0.135066,0.544853,0.554029}%
\pgfsetstrokecolor{currentstroke}%
\pgfsetdash{}{0pt}%
\pgfpathmoveto{\pgfqpoint{4.061290in}{5.036247in}}%
\pgfpathlineto{\pgfqpoint{3.976785in}{5.127804in}}%
\pgfpathlineto{\pgfqpoint{4.118884in}{4.938971in}}%
\pgfpathclose%
\pgfusepath{fill}%
\end{pgfscope}%
\begin{pgfscope}%
\pgfpathrectangle{\pgfqpoint{0.539299in}{0.078740in}}{\pgfqpoint{7.842520in}{7.842520in}}%
\pgfusepath{clip}%
\pgfsetbuttcap%
\pgfsetroundjoin%
\definecolor{currentfill}{rgb}{0.668054,0.861999,0.196293}%
\pgfsetfillcolor{currentfill}%
\pgfsetlinewidth{0.000000pt}%
\definecolor{currentstroke}{rgb}{0.133743,0.548535,0.553541}%
\pgfsetstrokecolor{currentstroke}%
\pgfsetdash{}{0pt}%
\pgfpathmoveto{\pgfqpoint{3.778860in}{5.358543in}}%
\pgfpathlineto{\pgfqpoint{3.693131in}{5.434822in}}%
\pgfpathlineto{\pgfqpoint{3.834801in}{5.295343in}}%
\pgfpathclose%
\pgfusepath{fill}%
\end{pgfscope}%
\begin{pgfscope}%
\pgfpathrectangle{\pgfqpoint{0.539299in}{0.078740in}}{\pgfqpoint{7.842520in}{7.842520in}}%
\pgfusepath{clip}%
\pgfsetbuttcap%
\pgfsetroundjoin%
\definecolor{currentfill}{rgb}{0.824940,0.884720,0.106217}%
\pgfsetfillcolor{currentfill}%
\pgfsetlinewidth{0.000000pt}%
\definecolor{currentstroke}{rgb}{0.132444,0.552216,0.553018}%
\pgfsetstrokecolor{currentstroke}%
\pgfsetdash{}{0pt}%
\pgfpathmoveto{\pgfqpoint{3.411816in}{5.600141in}}%
\pgfpathlineto{\pgfqpoint{3.272857in}{5.610573in}}%
\pgfpathlineto{\pgfqpoint{3.324832in}{5.632651in}}%
\pgfpathclose%
\pgfusepath{fill}%
\end{pgfscope}%
\begin{pgfscope}%
\pgfpathrectangle{\pgfqpoint{0.539299in}{0.078740in}}{\pgfqpoint{7.842520in}{7.842520in}}%
\pgfusepath{clip}%
\pgfsetbuttcap%
\pgfsetroundjoin%
\definecolor{currentfill}{rgb}{0.535621,0.835785,0.281908}%
\pgfsetfillcolor{currentfill}%
\pgfsetlinewidth{0.000000pt}%
\definecolor{currentstroke}{rgb}{0.131172,0.555899,0.552459}%
\pgfsetstrokecolor{currentstroke}%
\pgfsetdash{}{0pt}%
\pgfpathmoveto{\pgfqpoint{3.834801in}{5.295343in}}%
\pgfpathlineto{\pgfqpoint{3.976785in}{5.127804in}}%
\pgfpathlineto{\pgfqpoint{4.061290in}{5.036247in}}%
\pgfpathclose%
\pgfusepath{fill}%
\end{pgfscope}%
\begin{pgfscope}%
\pgfpathrectangle{\pgfqpoint{0.539299in}{0.078740in}}{\pgfqpoint{7.842520in}{7.842520in}}%
\pgfusepath{clip}%
\pgfsetbuttcap%
\pgfsetroundjoin%
\definecolor{currentfill}{rgb}{0.120081,0.622161,0.534946}%
\pgfsetfillcolor{currentfill}%
\pgfsetlinewidth{0.000000pt}%
\definecolor{currentstroke}{rgb}{0.129933,0.559582,0.551864}%
\pgfsetstrokecolor{currentstroke}%
\pgfsetdash{}{0pt}%
\pgfpathmoveto{\pgfqpoint{4.686160in}{4.082636in}}%
\pgfpathlineto{\pgfqpoint{4.604647in}{4.165055in}}%
\pgfpathlineto{\pgfqpoint{4.827480in}{3.864804in}}%
\pgfpathclose%
\pgfusepath{fill}%
\end{pgfscope}%
\begin{pgfscope}%
\pgfpathrectangle{\pgfqpoint{0.539299in}{0.078740in}}{\pgfqpoint{7.842520in}{7.842520in}}%
\pgfusepath{clip}%
\pgfsetbuttcap%
\pgfsetroundjoin%
\definecolor{currentfill}{rgb}{0.140536,0.530132,0.555659}%
\pgfsetfillcolor{currentfill}%
\pgfsetlinewidth{0.000000pt}%
\definecolor{currentstroke}{rgb}{0.128729,0.563265,0.551229}%
\pgfsetstrokecolor{currentstroke}%
\pgfsetdash{}{0pt}%
\pgfpathmoveto{\pgfqpoint{4.888578in}{3.720803in}}%
\pgfpathlineto{\pgfqpoint{5.030239in}{3.504331in}}%
\pgfpathlineto{\pgfqpoint{4.968612in}{3.651484in}}%
\pgfpathclose%
\pgfusepath{fill}%
\end{pgfscope}%
\begin{pgfscope}%
\pgfpathrectangle{\pgfqpoint{0.539299in}{0.078740in}}{\pgfqpoint{7.842520in}{7.842520in}}%
\pgfusepath{clip}%
\pgfsetbuttcap%
\pgfsetroundjoin%
\definecolor{currentfill}{rgb}{0.175841,0.441290,0.557685}%
\pgfsetfillcolor{currentfill}%
\pgfsetlinewidth{0.000000pt}%
\definecolor{currentstroke}{rgb}{0.127568,0.566949,0.550556}%
\pgfsetstrokecolor{currentstroke}%
\pgfsetdash{}{0pt}%
\pgfpathmoveto{\pgfqpoint{1.804002in}{3.315816in}}%
\pgfpathlineto{\pgfqpoint{1.726778in}{2.916957in}}%
\pgfpathlineto{\pgfqpoint{1.935199in}{3.440713in}}%
\pgfpathclose%
\pgfusepath{fill}%
\end{pgfscope}%
\begin{pgfscope}%
\pgfpathrectangle{\pgfqpoint{0.539299in}{0.078740in}}{\pgfqpoint{7.842520in}{7.842520in}}%
\pgfusepath{clip}%
\pgfsetbuttcap%
\pgfsetroundjoin%
\definecolor{currentfill}{rgb}{0.280868,0.160771,0.472899}%
\pgfsetfillcolor{currentfill}%
\pgfsetlinewidth{0.000000pt}%
\definecolor{currentstroke}{rgb}{0.126453,0.570633,0.549841}%
\pgfsetstrokecolor{currentstroke}%
\pgfsetdash{}{0pt}%
\pgfpathmoveto{\pgfqpoint{6.020118in}{2.191364in}}%
\pgfpathlineto{\pgfqpoint{5.943414in}{2.169358in}}%
\pgfpathlineto{\pgfqpoint{6.161891in}{2.037228in}}%
\pgfpathclose%
\pgfusepath{fill}%
\end{pgfscope}%
\begin{pgfscope}%
\pgfpathrectangle{\pgfqpoint{0.539299in}{0.078740in}}{\pgfqpoint{7.842520in}{7.842520in}}%
\pgfusepath{clip}%
\pgfsetbuttcap%
\pgfsetroundjoin%
\definecolor{currentfill}{rgb}{0.271828,0.209303,0.504434}%
\pgfsetfillcolor{currentfill}%
\pgfsetlinewidth{0.000000pt}%
\definecolor{currentstroke}{rgb}{0.125394,0.574318,0.549086}%
\pgfsetstrokecolor{currentstroke}%
\pgfsetdash{}{0pt}%
\pgfpathmoveto{\pgfqpoint{6.020118in}{2.191364in}}%
\pgfpathlineto{\pgfqpoint{5.878547in}{2.354530in}}%
\pgfpathlineto{\pgfqpoint{5.801571in}{2.347348in}}%
\pgfpathclose%
\pgfusepath{fill}%
\end{pgfscope}%
\begin{pgfscope}%
\pgfpathrectangle{\pgfqpoint{0.539299in}{0.078740in}}{\pgfqpoint{7.842520in}{7.842520in}}%
\pgfusepath{clip}%
\pgfsetbuttcap%
\pgfsetroundjoin%
\definecolor{currentfill}{rgb}{0.263663,0.237631,0.518762}%
\pgfsetfillcolor{currentfill}%
\pgfsetlinewidth{0.000000pt}%
\definecolor{currentstroke}{rgb}{0.124395,0.578002,0.548287}%
\pgfsetstrokecolor{currentstroke}%
\pgfsetdash{}{0pt}%
\pgfpathmoveto{\pgfqpoint{5.737117in}{2.526203in}}%
\pgfpathlineto{\pgfqpoint{5.801571in}{2.347348in}}%
\pgfpathlineto{\pgfqpoint{5.878547in}{2.354530in}}%
\pgfpathclose%
\pgfusepath{fill}%
\end{pgfscope}%
\begin{pgfscope}%
\pgfpathrectangle{\pgfqpoint{0.539299in}{0.078740in}}{\pgfqpoint{7.842520in}{7.842520in}}%
\pgfusepath{clip}%
\pgfsetbuttcap%
\pgfsetroundjoin%
\definecolor{currentfill}{rgb}{0.282327,0.094955,0.417331}%
\pgfsetfillcolor{currentfill}%
\pgfsetlinewidth{0.000000pt}%
\definecolor{currentstroke}{rgb}{0.123463,0.581687,0.547445}%
\pgfsetstrokecolor{currentstroke}%
\pgfsetdash{}{0pt}%
\pgfpathmoveto{\pgfqpoint{6.303947in}{1.893136in}}%
\pgfpathlineto{\pgfqpoint{6.161891in}{2.037228in}}%
\pgfpathlineto{\pgfqpoint{6.227580in}{1.841168in}}%
\pgfpathclose%
\pgfusepath{fill}%
\end{pgfscope}%
\begin{pgfscope}%
\pgfpathrectangle{\pgfqpoint{0.539299in}{0.078740in}}{\pgfqpoint{7.842520in}{7.842520in}}%
\pgfusepath{clip}%
\pgfsetbuttcap%
\pgfsetroundjoin%
\definecolor{currentfill}{rgb}{0.814576,0.883393,0.110347}%
\pgfsetfillcolor{currentfill}%
\pgfsetlinewidth{0.000000pt}%
\definecolor{currentstroke}{rgb}{0.122606,0.585371,0.546557}%
\pgfsetstrokecolor{currentstroke}%
\pgfsetdash{}{0pt}%
\pgfpathmoveto{\pgfqpoint{3.185707in}{5.629892in}}%
\pgfpathlineto{\pgfqpoint{3.048388in}{5.568482in}}%
\pgfpathlineto{\pgfqpoint{2.961006in}{5.551827in}}%
\pgfpathclose%
\pgfusepath{fill}%
\end{pgfscope}%
\begin{pgfscope}%
\pgfpathrectangle{\pgfqpoint{0.539299in}{0.078740in}}{\pgfqpoint{7.842520in}{7.842520in}}%
\pgfusepath{clip}%
\pgfsetbuttcap%
\pgfsetroundjoin%
\definecolor{currentfill}{rgb}{0.132268,0.655014,0.519661}%
\pgfsetfillcolor{currentfill}%
\pgfsetlinewidth{0.000000pt}%
\definecolor{currentstroke}{rgb}{0.121831,0.589055,0.545623}%
\pgfsetstrokecolor{currentstroke}%
\pgfsetdash{}{0pt}%
\pgfpathmoveto{\pgfqpoint{4.544630in}{4.302516in}}%
\pgfpathlineto{\pgfqpoint{4.604647in}{4.165055in}}%
\pgfpathlineto{\pgfqpoint{4.686160in}{4.082636in}}%
\pgfpathclose%
\pgfusepath{fill}%
\end{pgfscope}%
\begin{pgfscope}%
\pgfpathrectangle{\pgfqpoint{0.539299in}{0.078740in}}{\pgfqpoint{7.842520in}{7.842520in}}%
\pgfusepath{clip}%
\pgfsetbuttcap%
\pgfsetroundjoin%
\definecolor{currentfill}{rgb}{0.279566,0.067836,0.391917}%
\pgfsetfillcolor{currentfill}%
\pgfsetlinewidth{0.000000pt}%
\definecolor{currentstroke}{rgb}{0.121148,0.592739,0.544641}%
\pgfsetstrokecolor{currentstroke}%
\pgfsetdash{}{0pt}%
\pgfpathmoveto{\pgfqpoint{6.227580in}{1.841168in}}%
\pgfpathlineto{\pgfqpoint{6.446395in}{1.760841in}}%
\pgfpathlineto{\pgfqpoint{6.303947in}{1.893136in}}%
\pgfpathclose%
\pgfusepath{fill}%
\end{pgfscope}%
\begin{pgfscope}%
\pgfpathrectangle{\pgfqpoint{0.539299in}{0.078740in}}{\pgfqpoint{7.842520in}{7.842520in}}%
\pgfusepath{clip}%
\pgfsetbuttcap%
\pgfsetroundjoin%
\definecolor{currentfill}{rgb}{0.835270,0.886029,0.102646}%
\pgfsetfillcolor{currentfill}%
\pgfsetlinewidth{0.000000pt}%
\definecolor{currentstroke}{rgb}{0.120565,0.596422,0.543611}%
\pgfsetstrokecolor{currentstroke}%
\pgfsetdash{}{0pt}%
\pgfpathmoveto{\pgfqpoint{3.324832in}{5.632651in}}%
\pgfpathlineto{\pgfqpoint{3.272857in}{5.610573in}}%
\pgfpathlineto{\pgfqpoint{3.185707in}{5.629892in}}%
\pgfpathclose%
\pgfusepath{fill}%
\end{pgfscope}%
\begin{pgfscope}%
\pgfpathrectangle{\pgfqpoint{0.539299in}{0.078740in}}{\pgfqpoint{7.842520in}{7.842520in}}%
\pgfusepath{clip}%
\pgfsetbuttcap%
\pgfsetroundjoin%
\definecolor{currentfill}{rgb}{0.235526,0.309527,0.542944}%
\pgfsetfillcolor{currentfill}%
\pgfsetlinewidth{0.000000pt}%
\definecolor{currentstroke}{rgb}{0.120092,0.600104,0.542530}%
\pgfsetstrokecolor{currentstroke}%
\pgfsetdash{}{0pt}%
\pgfpathmoveto{\pgfqpoint{5.595772in}{2.706138in}}%
\pgfpathlineto{\pgfqpoint{5.518023in}{2.726615in}}%
\pgfpathlineto{\pgfqpoint{5.737117in}{2.526203in}}%
\pgfpathclose%
\pgfusepath{fill}%
\end{pgfscope}%
\begin{pgfscope}%
\pgfpathrectangle{\pgfqpoint{0.539299in}{0.078740in}}{\pgfqpoint{7.842520in}{7.842520in}}%
\pgfusepath{clip}%
\pgfsetbuttcap%
\pgfsetroundjoin%
\definecolor{currentfill}{rgb}{0.127568,0.566949,0.550556}%
\pgfsetfillcolor{currentfill}%
\pgfsetlinewidth{0.000000pt}%
\definecolor{currentstroke}{rgb}{0.119738,0.603785,0.541400}%
\pgfsetstrokecolor{currentstroke}%
\pgfsetdash{}{0pt}%
\pgfpathmoveto{\pgfqpoint{4.746722in}{3.941721in}}%
\pgfpathlineto{\pgfqpoint{4.888578in}{3.720803in}}%
\pgfpathlineto{\pgfqpoint{4.968612in}{3.651484in}}%
\pgfpathclose%
\pgfusepath{fill}%
\end{pgfscope}%
\begin{pgfscope}%
\pgfpathrectangle{\pgfqpoint{0.539299in}{0.078740in}}{\pgfqpoint{7.842520in}{7.842520in}}%
\pgfusepath{clip}%
\pgfsetbuttcap%
\pgfsetroundjoin%
\definecolor{currentfill}{rgb}{0.180653,0.701402,0.488189}%
\pgfsetfillcolor{currentfill}%
\pgfsetlinewidth{0.000000pt}%
\definecolor{currentstroke}{rgb}{0.119512,0.607464,0.540218}%
\pgfsetstrokecolor{currentstroke}%
\pgfsetdash{}{0pt}%
\pgfpathmoveto{\pgfqpoint{2.133656in}{4.370779in}}%
\pgfpathlineto{\pgfqpoint{2.049048in}{4.173228in}}%
\pgfpathlineto{\pgfqpoint{2.261900in}{4.632062in}}%
\pgfpathclose%
\pgfusepath{fill}%
\end{pgfscope}%
\begin{pgfscope}%
\pgfpathrectangle{\pgfqpoint{0.539299in}{0.078740in}}{\pgfqpoint{7.842520in}{7.842520in}}%
\pgfusepath{clip}%
\pgfsetbuttcap%
\pgfsetroundjoin%
\definecolor{currentfill}{rgb}{0.183898,0.422383,0.556944}%
\pgfsetfillcolor{currentfill}%
\pgfsetlinewidth{0.000000pt}%
\definecolor{currentstroke}{rgb}{0.119423,0.611141,0.538982}%
\pgfsetstrokecolor{currentstroke}%
\pgfsetdash{}{0pt}%
\pgfpathmoveto{\pgfqpoint{1.935199in}{3.440713in}}%
\pgfpathlineto{\pgfqpoint{1.726778in}{2.916957in}}%
\pgfpathlineto{\pgfqpoint{1.859345in}{2.989662in}}%
\pgfpathclose%
\pgfusepath{fill}%
\end{pgfscope}%
\begin{pgfscope}%
\pgfpathrectangle{\pgfqpoint{0.539299in}{0.078740in}}{\pgfqpoint{7.842520in}{7.842520in}}%
\pgfusepath{clip}%
\pgfsetbuttcap%
\pgfsetroundjoin%
\definecolor{currentfill}{rgb}{0.595839,0.848717,0.243329}%
\pgfsetfillcolor{currentfill}%
\pgfsetlinewidth{0.000000pt}%
\definecolor{currentstroke}{rgb}{0.119483,0.614817,0.537692}%
\pgfsetstrokecolor{currentstroke}%
\pgfsetdash{}{0pt}%
\pgfpathmoveto{\pgfqpoint{2.519975in}{5.063797in}}%
\pgfpathlineto{\pgfqpoint{2.651662in}{5.282437in}}%
\pgfpathlineto{\pgfqpoint{2.738785in}{5.362101in}}%
\pgfpathclose%
\pgfusepath{fill}%
\end{pgfscope}%
\begin{pgfscope}%
\pgfpathrectangle{\pgfqpoint{0.539299in}{0.078740in}}{\pgfqpoint{7.842520in}{7.842520in}}%
\pgfusepath{clip}%
\pgfsetbuttcap%
\pgfsetroundjoin%
\definecolor{currentfill}{rgb}{0.151918,0.500685,0.557587}%
\pgfsetfillcolor{currentfill}%
\pgfsetlinewidth{0.000000pt}%
\definecolor{currentstroke}{rgb}{0.119699,0.618490,0.536347}%
\pgfsetstrokecolor{currentstroke}%
\pgfsetdash{}{0pt}%
\pgfpathmoveto{\pgfqpoint{1.804002in}{3.315816in}}%
\pgfpathlineto{\pgfqpoint{1.935199in}{3.440713in}}%
\pgfpathlineto{\pgfqpoint{1.883808in}{3.652469in}}%
\pgfpathclose%
\pgfusepath{fill}%
\end{pgfscope}%
\begin{pgfscope}%
\pgfpathrectangle{\pgfqpoint{0.539299in}{0.078740in}}{\pgfqpoint{7.842520in}{7.842520in}}%
\pgfusepath{clip}%
\pgfsetbuttcap%
\pgfsetroundjoin%
\definecolor{currentfill}{rgb}{0.214298,0.355619,0.551184}%
\pgfsetfillcolor{currentfill}%
\pgfsetlinewidth{0.000000pt}%
\definecolor{currentstroke}{rgb}{0.120081,0.622161,0.534946}%
\pgfsetstrokecolor{currentstroke}%
\pgfsetdash{}{0pt}%
\pgfpathmoveto{\pgfqpoint{5.595772in}{2.706138in}}%
\pgfpathlineto{\pgfqpoint{5.454461in}{2.894198in}}%
\pgfpathlineto{\pgfqpoint{5.376216in}{2.927023in}}%
\pgfpathclose%
\pgfusepath{fill}%
\end{pgfscope}%
\begin{pgfscope}%
\pgfpathrectangle{\pgfqpoint{0.539299in}{0.078740in}}{\pgfqpoint{7.842520in}{7.842520in}}%
\pgfusepath{clip}%
\pgfsetbuttcap%
\pgfsetroundjoin%
\definecolor{currentfill}{rgb}{0.824940,0.884720,0.106217}%
\pgfsetfillcolor{currentfill}%
\pgfsetlinewidth{0.000000pt}%
\definecolor{currentstroke}{rgb}{0.120638,0.625828,0.533488}%
\pgfsetstrokecolor{currentstroke}%
\pgfsetdash{}{0pt}%
\pgfpathmoveto{\pgfqpoint{3.324832in}{5.632651in}}%
\pgfpathlineto{\pgfqpoint{3.552031in}{5.538964in}}%
\pgfpathlineto{\pgfqpoint{3.411816in}{5.600141in}}%
\pgfpathclose%
\pgfusepath{fill}%
\end{pgfscope}%
\begin{pgfscope}%
\pgfpathrectangle{\pgfqpoint{0.539299in}{0.078740in}}{\pgfqpoint{7.842520in}{7.842520in}}%
\pgfusepath{clip}%
\pgfsetbuttcap%
\pgfsetroundjoin%
\definecolor{currentfill}{rgb}{0.720391,0.870350,0.162603}%
\pgfsetfillcolor{currentfill}%
\pgfsetlinewidth{0.000000pt}%
\definecolor{currentstroke}{rgb}{0.121380,0.629492,0.531973}%
\pgfsetstrokecolor{currentstroke}%
\pgfsetdash{}{0pt}%
\pgfpathmoveto{\pgfqpoint{2.826074in}{5.413789in}}%
\pgfpathlineto{\pgfqpoint{2.738785in}{5.362101in}}%
\pgfpathlineto{\pgfqpoint{2.873601in}{5.508440in}}%
\pgfpathclose%
\pgfusepath{fill}%
\end{pgfscope}%
\begin{pgfscope}%
\pgfpathrectangle{\pgfqpoint{0.539299in}{0.078740in}}{\pgfqpoint{7.842520in}{7.842520in}}%
\pgfusepath{clip}%
\pgfsetbuttcap%
\pgfsetroundjoin%
\definecolor{currentfill}{rgb}{0.197636,0.391528,0.554969}%
\pgfsetfillcolor{currentfill}%
\pgfsetlinewidth{0.000000pt}%
\definecolor{currentstroke}{rgb}{0.122312,0.633153,0.530398}%
\pgfsetstrokecolor{currentstroke}%
\pgfsetdash{}{0pt}%
\pgfpathmoveto{\pgfqpoint{5.376216in}{2.927023in}}%
\pgfpathlineto{\pgfqpoint{5.454461in}{2.894198in}}%
\pgfpathlineto{\pgfqpoint{5.313135in}{3.090200in}}%
\pgfpathclose%
\pgfusepath{fill}%
\end{pgfscope}%
\begin{pgfscope}%
\pgfpathrectangle{\pgfqpoint{0.539299in}{0.078740in}}{\pgfqpoint{7.842520in}{7.842520in}}%
\pgfusepath{clip}%
\pgfsetbuttcap%
\pgfsetroundjoin%
\definecolor{currentfill}{rgb}{0.119699,0.618490,0.536347}%
\pgfsetfillcolor{currentfill}%
\pgfsetlinewidth{0.000000pt}%
\definecolor{currentstroke}{rgb}{0.123444,0.636809,0.528763}%
\pgfsetstrokecolor{currentstroke}%
\pgfsetdash{}{0pt}%
\pgfpathmoveto{\pgfqpoint{4.604647in}{4.165055in}}%
\pgfpathlineto{\pgfqpoint{4.746722in}{3.941721in}}%
\pgfpathlineto{\pgfqpoint{4.827480in}{3.864804in}}%
\pgfpathclose%
\pgfusepath{fill}%
\end{pgfscope}%
\begin{pgfscope}%
\pgfpathrectangle{\pgfqpoint{0.539299in}{0.078740in}}{\pgfqpoint{7.842520in}{7.842520in}}%
\pgfusepath{clip}%
\pgfsetbuttcap%
\pgfsetroundjoin%
\definecolor{currentfill}{rgb}{0.283197,0.115680,0.436115}%
\pgfsetfillcolor{currentfill}%
\pgfsetlinewidth{0.000000pt}%
\definecolor{currentstroke}{rgb}{0.124780,0.640461,0.527068}%
\pgfsetstrokecolor{currentstroke}%
\pgfsetdash{}{0pt}%
\pgfpathmoveto{\pgfqpoint{6.227580in}{1.841168in}}%
\pgfpathlineto{\pgfqpoint{6.161891in}{2.037228in}}%
\pgfpathlineto{\pgfqpoint{6.085388in}{2.000146in}}%
\pgfpathclose%
\pgfusepath{fill}%
\end{pgfscope}%
\begin{pgfscope}%
\pgfpathrectangle{\pgfqpoint{0.539299in}{0.078740in}}{\pgfqpoint{7.842520in}{7.842520in}}%
\pgfusepath{clip}%
\pgfsetbuttcap%
\pgfsetroundjoin%
\definecolor{currentfill}{rgb}{0.281887,0.150881,0.465405}%
\pgfsetfillcolor{currentfill}%
\pgfsetlinewidth{0.000000pt}%
\definecolor{currentstroke}{rgb}{0.126326,0.644107,0.525311}%
\pgfsetstrokecolor{currentstroke}%
\pgfsetdash{}{0pt}%
\pgfpathmoveto{\pgfqpoint{6.161891in}{2.037228in}}%
\pgfpathlineto{\pgfqpoint{5.943414in}{2.169358in}}%
\pgfpathlineto{\pgfqpoint{6.085388in}{2.000146in}}%
\pgfpathclose%
\pgfusepath{fill}%
\end{pgfscope}%
\begin{pgfscope}%
\pgfpathrectangle{\pgfqpoint{0.539299in}{0.078740in}}{\pgfqpoint{7.842520in}{7.842520in}}%
\pgfusepath{clip}%
\pgfsetbuttcap%
\pgfsetroundjoin%
\definecolor{currentfill}{rgb}{0.274128,0.199721,0.498911}%
\pgfsetfillcolor{currentfill}%
\pgfsetlinewidth{0.000000pt}%
\definecolor{currentstroke}{rgb}{0.128087,0.647749,0.523491}%
\pgfsetstrokecolor{currentstroke}%
\pgfsetdash{}{0pt}%
\pgfpathmoveto{\pgfqpoint{5.801571in}{2.347348in}}%
\pgfpathlineto{\pgfqpoint{5.943414in}{2.169358in}}%
\pgfpathlineto{\pgfqpoint{6.020118in}{2.191364in}}%
\pgfpathclose%
\pgfusepath{fill}%
\end{pgfscope}%
\begin{pgfscope}%
\pgfpathrectangle{\pgfqpoint{0.539299in}{0.078740in}}{\pgfqpoint{7.842520in}{7.842520in}}%
\pgfusepath{clip}%
\pgfsetbuttcap%
\pgfsetroundjoin%
\definecolor{currentfill}{rgb}{0.772852,0.877868,0.131109}%
\pgfsetfillcolor{currentfill}%
\pgfsetlinewidth{0.000000pt}%
\definecolor{currentstroke}{rgb}{0.130067,0.651384,0.521608}%
\pgfsetstrokecolor{currentstroke}%
\pgfsetdash{}{0pt}%
\pgfpathmoveto{\pgfqpoint{2.873601in}{5.508440in}}%
\pgfpathlineto{\pgfqpoint{2.961006in}{5.551827in}}%
\pgfpathlineto{\pgfqpoint{2.826074in}{5.413789in}}%
\pgfpathclose%
\pgfusepath{fill}%
\end{pgfscope}%
\begin{pgfscope}%
\pgfpathrectangle{\pgfqpoint{0.539299in}{0.078740in}}{\pgfqpoint{7.842520in}{7.842520in}}%
\pgfusepath{clip}%
\pgfsetbuttcap%
\pgfsetroundjoin%
\definecolor{currentfill}{rgb}{0.220124,0.725509,0.466226}%
\pgfsetfillcolor{currentfill}%
\pgfsetlinewidth{0.000000pt}%
\definecolor{currentstroke}{rgb}{0.132268,0.655014,0.519661}%
\pgfsetstrokecolor{currentstroke}%
\pgfsetdash{}{0pt}%
\pgfpathmoveto{\pgfqpoint{4.319824in}{4.607800in}}%
\pgfpathlineto{\pgfqpoint{4.544630in}{4.302516in}}%
\pgfpathlineto{\pgfqpoint{4.402885in}{4.521271in}}%
\pgfpathclose%
\pgfusepath{fill}%
\end{pgfscope}%
\begin{pgfscope}%
\pgfpathrectangle{\pgfqpoint{0.539299in}{0.078740in}}{\pgfqpoint{7.842520in}{7.842520in}}%
\pgfusepath{clip}%
\pgfsetbuttcap%
\pgfsetroundjoin%
\definecolor{currentfill}{rgb}{0.124780,0.640461,0.527068}%
\pgfsetfillcolor{currentfill}%
\pgfsetlinewidth{0.000000pt}%
\definecolor{currentstroke}{rgb}{0.134692,0.658636,0.517649}%
\pgfsetstrokecolor{currentstroke}%
\pgfsetdash{}{0pt}%
\pgfpathmoveto{\pgfqpoint{2.094863in}{4.143974in}}%
\pgfpathlineto{\pgfqpoint{2.049048in}{4.173228in}}%
\pgfpathlineto{\pgfqpoint{1.965641in}{3.935827in}}%
\pgfpathclose%
\pgfusepath{fill}%
\end{pgfscope}%
\begin{pgfscope}%
\pgfpathrectangle{\pgfqpoint{0.539299in}{0.078740in}}{\pgfqpoint{7.842520in}{7.842520in}}%
\pgfusepath{clip}%
\pgfsetbuttcap%
\pgfsetroundjoin%
\definecolor{currentfill}{rgb}{0.253935,0.265254,0.529983}%
\pgfsetfillcolor{currentfill}%
\pgfsetlinewidth{0.000000pt}%
\definecolor{currentstroke}{rgb}{0.137339,0.662252,0.515571}%
\pgfsetstrokecolor{currentstroke}%
\pgfsetdash{}{0pt}%
\pgfpathmoveto{\pgfqpoint{5.659792in}{2.533268in}}%
\pgfpathlineto{\pgfqpoint{5.801571in}{2.347348in}}%
\pgfpathlineto{\pgfqpoint{5.737117in}{2.526203in}}%
\pgfpathclose%
\pgfusepath{fill}%
\end{pgfscope}%
\begin{pgfscope}%
\pgfpathrectangle{\pgfqpoint{0.539299in}{0.078740in}}{\pgfqpoint{7.842520in}{7.842520in}}%
\pgfusepath{clip}%
\pgfsetbuttcap%
\pgfsetroundjoin%
\definecolor{currentfill}{rgb}{0.212395,0.359683,0.551710}%
\pgfsetfillcolor{currentfill}%
\pgfsetlinewidth{0.000000pt}%
\definecolor{currentstroke}{rgb}{0.140210,0.665859,0.513427}%
\pgfsetstrokecolor{currentstroke}%
\pgfsetdash{}{0pt}%
\pgfpathmoveto{\pgfqpoint{1.859345in}{2.989662in}}%
\pgfpathlineto{\pgfqpoint{1.922231in}{2.462670in}}%
\pgfpathlineto{\pgfqpoint{1.993699in}{3.037069in}}%
\pgfpathclose%
\pgfusepath{fill}%
\end{pgfscope}%
\begin{pgfscope}%
\pgfpathrectangle{\pgfqpoint{0.539299in}{0.078740in}}{\pgfqpoint{7.842520in}{7.842520in}}%
\pgfusepath{clip}%
\pgfsetbuttcap%
\pgfsetroundjoin%
\definecolor{currentfill}{rgb}{0.506271,0.828786,0.300362}%
\pgfsetfillcolor{currentfill}%
\pgfsetlinewidth{0.000000pt}%
\definecolor{currentstroke}{rgb}{0.143303,0.669459,0.511215}%
\pgfsetstrokecolor{currentstroke}%
\pgfsetdash{}{0pt}%
\pgfpathmoveto{\pgfqpoint{2.519975in}{5.063797in}}%
\pgfpathlineto{\pgfqpoint{2.433328in}{4.953363in}}%
\pgfpathlineto{\pgfqpoint{2.564916in}{5.170485in}}%
\pgfpathclose%
\pgfusepath{fill}%
\end{pgfscope}%
\begin{pgfscope}%
\pgfpathrectangle{\pgfqpoint{0.539299in}{0.078740in}}{\pgfqpoint{7.842520in}{7.842520in}}%
\pgfusepath{clip}%
\pgfsetbuttcap%
\pgfsetroundjoin%
\definecolor{currentfill}{rgb}{0.274149,0.751988,0.436601}%
\pgfsetfillcolor{currentfill}%
\pgfsetlinewidth{0.000000pt}%
\definecolor{currentstroke}{rgb}{0.146616,0.673050,0.508936}%
\pgfsetstrokecolor{currentstroke}%
\pgfsetdash{}{0pt}%
\pgfpathmoveto{\pgfqpoint{2.133656in}{4.370779in}}%
\pgfpathlineto{\pgfqpoint{2.261900in}{4.632062in}}%
\pgfpathlineto{\pgfqpoint{2.347219in}{4.810962in}}%
\pgfpathclose%
\pgfusepath{fill}%
\end{pgfscope}%
\begin{pgfscope}%
\pgfpathrectangle{\pgfqpoint{0.539299in}{0.078740in}}{\pgfqpoint{7.842520in}{7.842520in}}%
\pgfusepath{clip}%
\pgfsetbuttcap%
\pgfsetroundjoin%
\definecolor{currentfill}{rgb}{0.277941,0.056324,0.381191}%
\pgfsetfillcolor{currentfill}%
\pgfsetlinewidth{0.000000pt}%
\definecolor{currentstroke}{rgb}{0.150148,0.676631,0.506589}%
\pgfsetstrokecolor{currentstroke}%
\pgfsetdash{}{0pt}%
\pgfpathmoveto{\pgfqpoint{6.370114in}{1.694823in}}%
\pgfpathlineto{\pgfqpoint{6.446395in}{1.760841in}}%
\pgfpathlineto{\pgfqpoint{6.227580in}{1.841168in}}%
\pgfpathclose%
\pgfusepath{fill}%
\end{pgfscope}%
\begin{pgfscope}%
\pgfpathrectangle{\pgfqpoint{0.539299in}{0.078740in}}{\pgfqpoint{7.842520in}{7.842520in}}%
\pgfusepath{clip}%
\pgfsetbuttcap%
\pgfsetroundjoin%
\definecolor{currentfill}{rgb}{0.119738,0.603785,0.541400}%
\pgfsetfillcolor{currentfill}%
\pgfsetlinewidth{0.000000pt}%
\definecolor{currentstroke}{rgb}{0.153894,0.680203,0.504172}%
\pgfsetstrokecolor{currentstroke}%
\pgfsetdash{}{0pt}%
\pgfpathmoveto{\pgfqpoint{1.883808in}{3.652469in}}%
\pgfpathlineto{\pgfqpoint{2.094863in}{4.143974in}}%
\pgfpathlineto{\pgfqpoint{1.965641in}{3.935827in}}%
\pgfpathclose%
\pgfusepath{fill}%
\end{pgfscope}%
\begin{pgfscope}%
\pgfpathrectangle{\pgfqpoint{0.539299in}{0.078740in}}{\pgfqpoint{7.842520in}{7.842520in}}%
\pgfusepath{clip}%
\pgfsetbuttcap%
\pgfsetroundjoin%
\definecolor{currentfill}{rgb}{0.239346,0.300855,0.540844}%
\pgfsetfillcolor{currentfill}%
\pgfsetlinewidth{0.000000pt}%
\definecolor{currentstroke}{rgb}{0.157851,0.683765,0.501686}%
\pgfsetstrokecolor{currentstroke}%
\pgfsetdash{}{0pt}%
\pgfpathmoveto{\pgfqpoint{5.737117in}{2.526203in}}%
\pgfpathlineto{\pgfqpoint{5.518023in}{2.726615in}}%
\pgfpathlineto{\pgfqpoint{5.659792in}{2.533268in}}%
\pgfpathclose%
\pgfusepath{fill}%
\end{pgfscope}%
\begin{pgfscope}%
\pgfpathrectangle{\pgfqpoint{0.539299in}{0.078740in}}{\pgfqpoint{7.842520in}{7.842520in}}%
\pgfusepath{clip}%
\pgfsetbuttcap%
\pgfsetroundjoin%
\definecolor{currentfill}{rgb}{0.169646,0.456262,0.558030}%
\pgfsetfillcolor{currentfill}%
\pgfsetlinewidth{0.000000pt}%
\definecolor{currentstroke}{rgb}{0.162016,0.687316,0.499129}%
\pgfsetstrokecolor{currentstroke}%
\pgfsetdash{}{0pt}%
\pgfpathmoveto{\pgfqpoint{5.313135in}{3.090200in}}%
\pgfpathlineto{\pgfqpoint{5.171743in}{3.293790in}}%
\pgfpathlineto{\pgfqpoint{5.092313in}{3.347287in}}%
\pgfpathclose%
\pgfusepath{fill}%
\end{pgfscope}%
\begin{pgfscope}%
\pgfpathrectangle{\pgfqpoint{0.539299in}{0.078740in}}{\pgfqpoint{7.842520in}{7.842520in}}%
\pgfusepath{clip}%
\pgfsetbuttcap%
\pgfsetroundjoin%
\definecolor{currentfill}{rgb}{0.281477,0.755203,0.432552}%
\pgfsetfillcolor{currentfill}%
\pgfsetlinewidth{0.000000pt}%
\definecolor{currentstroke}{rgb}{0.166383,0.690856,0.496502}%
\pgfsetstrokecolor{currentstroke}%
\pgfsetdash{}{0pt}%
\pgfpathmoveto{\pgfqpoint{4.402885in}{4.521271in}}%
\pgfpathlineto{\pgfqpoint{4.260950in}{4.734983in}}%
\pgfpathlineto{\pgfqpoint{4.319824in}{4.607800in}}%
\pgfpathclose%
\pgfusepath{fill}%
\end{pgfscope}%
\begin{pgfscope}%
\pgfpathrectangle{\pgfqpoint{0.539299in}{0.078740in}}{\pgfqpoint{7.842520in}{7.842520in}}%
\pgfusepath{clip}%
\pgfsetbuttcap%
\pgfsetroundjoin%
\definecolor{currentfill}{rgb}{0.231674,0.318106,0.544834}%
\pgfsetfillcolor{currentfill}%
\pgfsetlinewidth{0.000000pt}%
\definecolor{currentstroke}{rgb}{0.170948,0.694384,0.493803}%
\pgfsetstrokecolor{currentstroke}%
\pgfsetdash{}{0pt}%
\pgfpathmoveto{\pgfqpoint{1.993699in}{3.037069in}}%
\pgfpathlineto{\pgfqpoint{1.922231in}{2.462670in}}%
\pgfpathlineto{\pgfqpoint{2.058527in}{2.453906in}}%
\pgfpathclose%
\pgfusepath{fill}%
\end{pgfscope}%
\begin{pgfscope}%
\pgfpathrectangle{\pgfqpoint{0.539299in}{0.078740in}}{\pgfqpoint{7.842520in}{7.842520in}}%
\pgfusepath{clip}%
\pgfsetbuttcap%
\pgfsetroundjoin%
\definecolor{currentfill}{rgb}{0.216210,0.351535,0.550627}%
\pgfsetfillcolor{currentfill}%
\pgfsetlinewidth{0.000000pt}%
\definecolor{currentstroke}{rgb}{0.175707,0.697900,0.491033}%
\pgfsetstrokecolor{currentstroke}%
\pgfsetdash{}{0pt}%
\pgfpathmoveto{\pgfqpoint{5.376216in}{2.927023in}}%
\pgfpathlineto{\pgfqpoint{5.518023in}{2.726615in}}%
\pgfpathlineto{\pgfqpoint{5.595772in}{2.706138in}}%
\pgfpathclose%
\pgfusepath{fill}%
\end{pgfscope}%
\begin{pgfscope}%
\pgfpathrectangle{\pgfqpoint{0.539299in}{0.078740in}}{\pgfqpoint{7.842520in}{7.842520in}}%
\pgfusepath{clip}%
\pgfsetbuttcap%
\pgfsetroundjoin%
\definecolor{currentfill}{rgb}{0.793760,0.880678,0.120005}%
\pgfsetfillcolor{currentfill}%
\pgfsetlinewidth{0.000000pt}%
\definecolor{currentstroke}{rgb}{0.180653,0.701402,0.488189}%
\pgfsetstrokecolor{currentstroke}%
\pgfsetdash{}{0pt}%
\pgfpathmoveto{\pgfqpoint{3.693131in}{5.434822in}}%
\pgfpathlineto{\pgfqpoint{3.552031in}{5.538964in}}%
\pgfpathlineto{\pgfqpoint{3.465333in}{5.584033in}}%
\pgfpathclose%
\pgfusepath{fill}%
\end{pgfscope}%
\begin{pgfscope}%
\pgfpathrectangle{\pgfqpoint{0.539299in}{0.078740in}}{\pgfqpoint{7.842520in}{7.842520in}}%
\pgfusepath{clip}%
\pgfsetbuttcap%
\pgfsetroundjoin%
\definecolor{currentfill}{rgb}{0.156270,0.489624,0.557936}%
\pgfsetfillcolor{currentfill}%
\pgfsetlinewidth{0.000000pt}%
\definecolor{currentstroke}{rgb}{0.185783,0.704891,0.485273}%
\pgfsetstrokecolor{currentstroke}%
\pgfsetdash{}{0pt}%
\pgfpathmoveto{\pgfqpoint{5.092313in}{3.347287in}}%
\pgfpathlineto{\pgfqpoint{5.171743in}{3.293790in}}%
\pgfpathlineto{\pgfqpoint{5.030239in}{3.504331in}}%
\pgfpathclose%
\pgfusepath{fill}%
\end{pgfscope}%
\begin{pgfscope}%
\pgfpathrectangle{\pgfqpoint{0.539299in}{0.078740in}}{\pgfqpoint{7.842520in}{7.842520in}}%
\pgfusepath{clip}%
\pgfsetbuttcap%
\pgfsetroundjoin%
\definecolor{currentfill}{rgb}{0.162016,0.687316,0.499129}%
\pgfsetfillcolor{currentfill}%
\pgfsetlinewidth{0.000000pt}%
\definecolor{currentstroke}{rgb}{0.191090,0.708366,0.482284}%
\pgfsetstrokecolor{currentstroke}%
\pgfsetdash{}{0pt}%
\pgfpathmoveto{\pgfqpoint{4.544630in}{4.302516in}}%
\pgfpathlineto{\pgfqpoint{4.462343in}{4.388175in}}%
\pgfpathlineto{\pgfqpoint{4.604647in}{4.165055in}}%
\pgfpathclose%
\pgfusepath{fill}%
\end{pgfscope}%
\begin{pgfscope}%
\pgfpathrectangle{\pgfqpoint{0.539299in}{0.078740in}}{\pgfqpoint{7.842520in}{7.842520in}}%
\pgfusepath{clip}%
\pgfsetbuttcap%
\pgfsetroundjoin%
\definecolor{currentfill}{rgb}{0.855810,0.888601,0.097452}%
\pgfsetfillcolor{currentfill}%
\pgfsetlinewidth{0.000000pt}%
\definecolor{currentstroke}{rgb}{0.196571,0.711827,0.479221}%
\pgfsetstrokecolor{currentstroke}%
\pgfsetdash{}{0pt}%
\pgfpathmoveto{\pgfqpoint{2.961006in}{5.551827in}}%
\pgfpathlineto{\pgfqpoint{3.098325in}{5.625023in}}%
\pgfpathlineto{\pgfqpoint{3.185707in}{5.629892in}}%
\pgfpathclose%
\pgfusepath{fill}%
\end{pgfscope}%
\begin{pgfscope}%
\pgfpathrectangle{\pgfqpoint{0.539299in}{0.078740in}}{\pgfqpoint{7.842520in}{7.842520in}}%
\pgfusepath{clip}%
\pgfsetbuttcap%
\pgfsetroundjoin%
\definecolor{currentfill}{rgb}{0.585678,0.846661,0.249897}%
\pgfsetfillcolor{currentfill}%
\pgfsetlinewidth{0.000000pt}%
\definecolor{currentstroke}{rgb}{0.202219,0.715272,0.476084}%
\pgfsetstrokecolor{currentstroke}%
\pgfsetdash{}{0pt}%
\pgfpathmoveto{\pgfqpoint{2.564916in}{5.170485in}}%
\pgfpathlineto{\pgfqpoint{2.651662in}{5.282437in}}%
\pgfpathlineto{\pgfqpoint{2.519975in}{5.063797in}}%
\pgfpathclose%
\pgfusepath{fill}%
\end{pgfscope}%
\begin{pgfscope}%
\pgfpathrectangle{\pgfqpoint{0.539299in}{0.078740in}}{\pgfqpoint{7.842520in}{7.842520in}}%
\pgfusepath{clip}%
\pgfsetbuttcap%
\pgfsetroundjoin%
\definecolor{currentfill}{rgb}{0.133743,0.548535,0.553541}%
\pgfsetfillcolor{currentfill}%
\pgfsetlinewidth{0.000000pt}%
\definecolor{currentstroke}{rgb}{0.208030,0.718701,0.472873}%
\pgfsetstrokecolor{currentstroke}%
\pgfsetdash{}{0pt}%
\pgfpathmoveto{\pgfqpoint{1.883808in}{3.652469in}}%
\pgfpathlineto{\pgfqpoint{1.935199in}{3.440713in}}%
\pgfpathlineto{\pgfqpoint{2.013886in}{3.822753in}}%
\pgfpathclose%
\pgfusepath{fill}%
\end{pgfscope}%
\begin{pgfscope}%
\pgfpathrectangle{\pgfqpoint{0.539299in}{0.078740in}}{\pgfqpoint{7.842520in}{7.842520in}}%
\pgfusepath{clip}%
\pgfsetbuttcap%
\pgfsetroundjoin%
\definecolor{currentfill}{rgb}{0.430983,0.808473,0.346476}%
\pgfsetfillcolor{currentfill}%
\pgfsetlinewidth{0.000000pt}%
\definecolor{currentstroke}{rgb}{0.214000,0.722114,0.469588}%
\pgfsetstrokecolor{currentstroke}%
\pgfsetdash{}{0pt}%
\pgfpathmoveto{\pgfqpoint{4.034345in}{5.020034in}}%
\pgfpathlineto{\pgfqpoint{4.260950in}{4.734983in}}%
\pgfpathlineto{\pgfqpoint{4.118884in}{4.938971in}}%
\pgfpathclose%
\pgfusepath{fill}%
\end{pgfscope}%
\begin{pgfscope}%
\pgfpathrectangle{\pgfqpoint{0.539299in}{0.078740in}}{\pgfqpoint{7.842520in}{7.842520in}}%
\pgfusepath{clip}%
\pgfsetbuttcap%
\pgfsetroundjoin%
\definecolor{currentfill}{rgb}{0.175841,0.441290,0.557685}%
\pgfsetfillcolor{currentfill}%
\pgfsetlinewidth{0.000000pt}%
\definecolor{currentstroke}{rgb}{0.220124,0.725509,0.466226}%
\pgfsetstrokecolor{currentstroke}%
\pgfsetdash{}{0pt}%
\pgfpathmoveto{\pgfqpoint{1.935199in}{3.440713in}}%
\pgfpathlineto{\pgfqpoint{1.859345in}{2.989662in}}%
\pgfpathlineto{\pgfqpoint{1.993699in}{3.037069in}}%
\pgfpathclose%
\pgfusepath{fill}%
\end{pgfscope}%
\begin{pgfscope}%
\pgfpathrectangle{\pgfqpoint{0.539299in}{0.078740in}}{\pgfqpoint{7.842520in}{7.842520in}}%
\pgfusepath{clip}%
\pgfsetbuttcap%
\pgfsetroundjoin%
\definecolor{currentfill}{rgb}{0.202219,0.715272,0.476084}%
\pgfsetfillcolor{currentfill}%
\pgfsetlinewidth{0.000000pt}%
\definecolor{currentstroke}{rgb}{0.226397,0.728888,0.462789}%
\pgfsetstrokecolor{currentstroke}%
\pgfsetdash{}{0pt}%
\pgfpathmoveto{\pgfqpoint{2.261900in}{4.632062in}}%
\pgfpathlineto{\pgfqpoint{2.049048in}{4.173228in}}%
\pgfpathlineto{\pgfqpoint{2.177668in}{4.411616in}}%
\pgfpathclose%
\pgfusepath{fill}%
\end{pgfscope}%
\begin{pgfscope}%
\pgfpathrectangle{\pgfqpoint{0.539299in}{0.078740in}}{\pgfqpoint{7.842520in}{7.842520in}}%
\pgfusepath{clip}%
\pgfsetbuttcap%
\pgfsetroundjoin%
\definecolor{currentfill}{rgb}{0.185556,0.418570,0.556753}%
\pgfsetfillcolor{currentfill}%
\pgfsetlinewidth{0.000000pt}%
\definecolor{currentstroke}{rgb}{0.232815,0.732247,0.459277}%
\pgfsetstrokecolor{currentstroke}%
\pgfsetdash{}{0pt}%
\pgfpathmoveto{\pgfqpoint{5.376216in}{2.927023in}}%
\pgfpathlineto{\pgfqpoint{5.313135in}{3.090200in}}%
\pgfpathlineto{\pgfqpoint{5.234327in}{3.134099in}}%
\pgfpathclose%
\pgfusepath{fill}%
\end{pgfscope}%
\begin{pgfscope}%
\pgfpathrectangle{\pgfqpoint{0.539299in}{0.078740in}}{\pgfqpoint{7.842520in}{7.842520in}}%
\pgfusepath{clip}%
\pgfsetbuttcap%
\pgfsetroundjoin%
\definecolor{currentfill}{rgb}{0.214000,0.722114,0.469588}%
\pgfsetfillcolor{currentfill}%
\pgfsetlinewidth{0.000000pt}%
\definecolor{currentstroke}{rgb}{0.239374,0.735588,0.455688}%
\pgfsetstrokecolor{currentstroke}%
\pgfsetdash{}{0pt}%
\pgfpathmoveto{\pgfqpoint{4.462343in}{4.388175in}}%
\pgfpathlineto{\pgfqpoint{4.544630in}{4.302516in}}%
\pgfpathlineto{\pgfqpoint{4.319824in}{4.607800in}}%
\pgfpathclose%
\pgfusepath{fill}%
\end{pgfscope}%
\begin{pgfscope}%
\pgfpathrectangle{\pgfqpoint{0.539299in}{0.078740in}}{\pgfqpoint{7.842520in}{7.842520in}}%
\pgfusepath{clip}%
\pgfsetbuttcap%
\pgfsetroundjoin%
\definecolor{currentfill}{rgb}{0.845561,0.887322,0.099702}%
\pgfsetfillcolor{currentfill}%
\pgfsetlinewidth{0.000000pt}%
\definecolor{currentstroke}{rgb}{0.246070,0.738910,0.452024}%
\pgfsetstrokecolor{currentstroke}%
\pgfsetdash{}{0pt}%
\pgfpathmoveto{\pgfqpoint{3.465333in}{5.584033in}}%
\pgfpathlineto{\pgfqpoint{3.552031in}{5.538964in}}%
\pgfpathlineto{\pgfqpoint{3.324832in}{5.632651in}}%
\pgfpathclose%
\pgfusepath{fill}%
\end{pgfscope}%
\begin{pgfscope}%
\pgfpathrectangle{\pgfqpoint{0.539299in}{0.078740in}}{\pgfqpoint{7.842520in}{7.842520in}}%
\pgfusepath{clip}%
\pgfsetbuttcap%
\pgfsetroundjoin%
\definecolor{currentfill}{rgb}{0.506271,0.828786,0.300362}%
\pgfsetfillcolor{currentfill}%
\pgfsetlinewidth{0.000000pt}%
\definecolor{currentstroke}{rgb}{0.252899,0.742211,0.448284}%
\pgfsetstrokecolor{currentstroke}%
\pgfsetdash{}{0pt}%
\pgfpathmoveto{\pgfqpoint{4.118884in}{4.938971in}}%
\pgfpathlineto{\pgfqpoint{3.976785in}{5.127804in}}%
\pgfpathlineto{\pgfqpoint{4.034345in}{5.020034in}}%
\pgfpathclose%
\pgfusepath{fill}%
\end{pgfscope}%
\begin{pgfscope}%
\pgfpathrectangle{\pgfqpoint{0.539299in}{0.078740in}}{\pgfqpoint{7.842520in}{7.842520in}}%
\pgfusepath{clip}%
\pgfsetbuttcap%
\pgfsetroundjoin%
\definecolor{currentfill}{rgb}{0.477504,0.821444,0.318195}%
\pgfsetfillcolor{currentfill}%
\pgfsetlinewidth{0.000000pt}%
\definecolor{currentstroke}{rgb}{0.259857,0.745492,0.444467}%
\pgfsetstrokecolor{currentstroke}%
\pgfsetdash{}{0pt}%
\pgfpathmoveto{\pgfqpoint{2.564916in}{5.170485in}}%
\pgfpathlineto{\pgfqpoint{2.433328in}{4.953363in}}%
\pgfpathlineto{\pgfqpoint{2.347219in}{4.810962in}}%
\pgfpathclose%
\pgfusepath{fill}%
\end{pgfscope}%
\begin{pgfscope}%
\pgfpathrectangle{\pgfqpoint{0.539299in}{0.078740in}}{\pgfqpoint{7.842520in}{7.842520in}}%
\pgfusepath{clip}%
\pgfsetbuttcap%
\pgfsetroundjoin%
\definecolor{currentfill}{rgb}{0.720391,0.870350,0.162603}%
\pgfsetfillcolor{currentfill}%
\pgfsetlinewidth{0.000000pt}%
\definecolor{currentstroke}{rgb}{0.266941,0.748751,0.440573}%
\pgfsetstrokecolor{currentstroke}%
\pgfsetdash{}{0pt}%
\pgfpathmoveto{\pgfqpoint{2.873601in}{5.508440in}}%
\pgfpathlineto{\pgfqpoint{2.738785in}{5.362101in}}%
\pgfpathlineto{\pgfqpoint{2.651662in}{5.282437in}}%
\pgfpathclose%
\pgfusepath{fill}%
\end{pgfscope}%
\begin{pgfscope}%
\pgfpathrectangle{\pgfqpoint{0.539299in}{0.078740in}}{\pgfqpoint{7.842520in}{7.842520in}}%
\pgfusepath{clip}%
\pgfsetbuttcap%
\pgfsetroundjoin%
\definecolor{currentfill}{rgb}{0.709898,0.868751,0.169257}%
\pgfsetfillcolor{currentfill}%
\pgfsetlinewidth{0.000000pt}%
\definecolor{currentstroke}{rgb}{0.274149,0.751988,0.436601}%
\pgfsetstrokecolor{currentstroke}%
\pgfsetdash{}{0pt}%
\pgfpathmoveto{\pgfqpoint{3.749006in}{5.361964in}}%
\pgfpathlineto{\pgfqpoint{3.834801in}{5.295343in}}%
\pgfpathlineto{\pgfqpoint{3.693131in}{5.434822in}}%
\pgfpathclose%
\pgfusepath{fill}%
\end{pgfscope}%
\begin{pgfscope}%
\pgfpathrectangle{\pgfqpoint{0.539299in}{0.078740in}}{\pgfqpoint{7.842520in}{7.842520in}}%
\pgfusepath{clip}%
\pgfsetbuttcap%
\pgfsetroundjoin%
\definecolor{currentfill}{rgb}{0.876168,0.891125,0.095250}%
\pgfsetfillcolor{currentfill}%
\pgfsetlinewidth{0.000000pt}%
\definecolor{currentstroke}{rgb}{0.281477,0.755203,0.432552}%
\pgfsetstrokecolor{currentstroke}%
\pgfsetdash{}{0pt}%
\pgfpathmoveto{\pgfqpoint{3.185707in}{5.629892in}}%
\pgfpathlineto{\pgfqpoint{3.098325in}{5.625023in}}%
\pgfpathlineto{\pgfqpoint{3.324832in}{5.632651in}}%
\pgfpathclose%
\pgfusepath{fill}%
\end{pgfscope}%
\begin{pgfscope}%
\pgfpathrectangle{\pgfqpoint{0.539299in}{0.078740in}}{\pgfqpoint{7.842520in}{7.842520in}}%
\pgfusepath{clip}%
\pgfsetbuttcap%
\pgfsetroundjoin%
\definecolor{currentfill}{rgb}{0.157851,0.683765,0.501686}%
\pgfsetfillcolor{currentfill}%
\pgfsetlinewidth{0.000000pt}%
\definecolor{currentstroke}{rgb}{0.288921,0.758394,0.428426}%
\pgfsetstrokecolor{currentstroke}%
\pgfsetdash{}{0pt}%
\pgfpathmoveto{\pgfqpoint{2.177668in}{4.411616in}}%
\pgfpathlineto{\pgfqpoint{2.049048in}{4.173228in}}%
\pgfpathlineto{\pgfqpoint{2.094863in}{4.143974in}}%
\pgfpathclose%
\pgfusepath{fill}%
\end{pgfscope}%
\begin{pgfscope}%
\pgfpathrectangle{\pgfqpoint{0.539299in}{0.078740in}}{\pgfqpoint{7.842520in}{7.842520in}}%
\pgfusepath{clip}%
\pgfsetbuttcap%
\pgfsetroundjoin%
\definecolor{currentfill}{rgb}{0.616293,0.852709,0.230052}%
\pgfsetfillcolor{currentfill}%
\pgfsetlinewidth{0.000000pt}%
\definecolor{currentstroke}{rgb}{0.296479,0.761561,0.424223}%
\pgfsetstrokecolor{currentstroke}%
\pgfsetdash{}{0pt}%
\pgfpathmoveto{\pgfqpoint{3.891582in}{5.202680in}}%
\pgfpathlineto{\pgfqpoint{3.976785in}{5.127804in}}%
\pgfpathlineto{\pgfqpoint{3.834801in}{5.295343in}}%
\pgfpathclose%
\pgfusepath{fill}%
\end{pgfscope}%
\begin{pgfscope}%
\pgfpathrectangle{\pgfqpoint{0.539299in}{0.078740in}}{\pgfqpoint{7.842520in}{7.842520in}}%
\pgfusepath{clip}%
\pgfsetbuttcap%
\pgfsetroundjoin%
\definecolor{currentfill}{rgb}{0.171176,0.452530,0.557965}%
\pgfsetfillcolor{currentfill}%
\pgfsetlinewidth{0.000000pt}%
\definecolor{currentstroke}{rgb}{0.304148,0.764704,0.419943}%
\pgfsetstrokecolor{currentstroke}%
\pgfsetdash{}{0pt}%
\pgfpathmoveto{\pgfqpoint{5.092313in}{3.347287in}}%
\pgfpathlineto{\pgfqpoint{5.234327in}{3.134099in}}%
\pgfpathlineto{\pgfqpoint{5.313135in}{3.090200in}}%
\pgfpathclose%
\pgfusepath{fill}%
\end{pgfscope}%
\begin{pgfscope}%
\pgfpathrectangle{\pgfqpoint{0.539299in}{0.078740in}}{\pgfqpoint{7.842520in}{7.842520in}}%
\pgfusepath{clip}%
\pgfsetbuttcap%
\pgfsetroundjoin%
\definecolor{currentfill}{rgb}{0.119738,0.603785,0.541400}%
\pgfsetfillcolor{currentfill}%
\pgfsetlinewidth{0.000000pt}%
\definecolor{currentstroke}{rgb}{0.311925,0.767822,0.415586}%
\pgfsetstrokecolor{currentstroke}%
\pgfsetdash{}{0pt}%
\pgfpathmoveto{\pgfqpoint{2.013886in}{3.822753in}}%
\pgfpathlineto{\pgfqpoint{2.094863in}{4.143974in}}%
\pgfpathlineto{\pgfqpoint{1.883808in}{3.652469in}}%
\pgfpathclose%
\pgfusepath{fill}%
\end{pgfscope}%
\begin{pgfscope}%
\pgfpathrectangle{\pgfqpoint{0.539299in}{0.078740in}}{\pgfqpoint{7.842520in}{7.842520in}}%
\pgfusepath{clip}%
\pgfsetbuttcap%
\pgfsetroundjoin%
\definecolor{currentfill}{rgb}{0.129933,0.559582,0.551864}%
\pgfsetfillcolor{currentfill}%
\pgfsetlinewidth{0.000000pt}%
\definecolor{currentstroke}{rgb}{0.319809,0.770914,0.411152}%
\pgfsetstrokecolor{currentstroke}%
\pgfsetdash{}{0pt}%
\pgfpathmoveto{\pgfqpoint{5.030239in}{3.504331in}}%
\pgfpathlineto{\pgfqpoint{4.888578in}{3.720803in}}%
\pgfpathlineto{\pgfqpoint{4.807756in}{3.788327in}}%
\pgfpathclose%
\pgfusepath{fill}%
\end{pgfscope}%
\begin{pgfscope}%
\pgfpathrectangle{\pgfqpoint{0.539299in}{0.078740in}}{\pgfqpoint{7.842520in}{7.842520in}}%
\pgfusepath{clip}%
\pgfsetbuttcap%
\pgfsetroundjoin%
\definecolor{currentfill}{rgb}{0.283072,0.130895,0.449241}%
\pgfsetfillcolor{currentfill}%
\pgfsetlinewidth{0.000000pt}%
\definecolor{currentstroke}{rgb}{0.327796,0.773980,0.406640}%
\pgfsetstrokecolor{currentstroke}%
\pgfsetdash{}{0pt}%
\pgfpathmoveto{\pgfqpoint{6.085388in}{2.000146in}}%
\pgfpathlineto{\pgfqpoint{6.008600in}{1.974963in}}%
\pgfpathlineto{\pgfqpoint{6.227580in}{1.841168in}}%
\pgfpathclose%
\pgfusepath{fill}%
\end{pgfscope}%
\begin{pgfscope}%
\pgfpathrectangle{\pgfqpoint{0.539299in}{0.078740in}}{\pgfqpoint{7.842520in}{7.842520in}}%
\pgfusepath{clip}%
\pgfsetbuttcap%
\pgfsetroundjoin%
\definecolor{currentfill}{rgb}{0.280868,0.160771,0.472899}%
\pgfsetfillcolor{currentfill}%
\pgfsetlinewidth{0.000000pt}%
\definecolor{currentstroke}{rgb}{0.335885,0.777018,0.402049}%
\pgfsetstrokecolor{currentstroke}%
\pgfsetdash{}{0pt}%
\pgfpathmoveto{\pgfqpoint{5.943414in}{2.169358in}}%
\pgfpathlineto{\pgfqpoint{6.008600in}{1.974963in}}%
\pgfpathlineto{\pgfqpoint{6.085388in}{2.000146in}}%
\pgfpathclose%
\pgfusepath{fill}%
\end{pgfscope}%
\begin{pgfscope}%
\pgfpathrectangle{\pgfqpoint{0.539299in}{0.078740in}}{\pgfqpoint{7.842520in}{7.842520in}}%
\pgfusepath{clip}%
\pgfsetbuttcap%
\pgfsetroundjoin%
\definecolor{currentfill}{rgb}{0.278791,0.062145,0.386592}%
\pgfsetfillcolor{currentfill}%
\pgfsetlinewidth{0.000000pt}%
\definecolor{currentstroke}{rgb}{0.344074,0.780029,0.397381}%
\pgfsetstrokecolor{currentstroke}%
\pgfsetdash{}{0pt}%
\pgfpathmoveto{\pgfqpoint{6.227580in}{1.841168in}}%
\pgfpathlineto{\pgfqpoint{6.293516in}{1.634687in}}%
\pgfpathlineto{\pgfqpoint{6.370114in}{1.694823in}}%
\pgfpathclose%
\pgfusepath{fill}%
\end{pgfscope}%
\begin{pgfscope}%
\pgfpathrectangle{\pgfqpoint{0.539299in}{0.078740in}}{\pgfqpoint{7.842520in}{7.842520in}}%
\pgfusepath{clip}%
\pgfsetbuttcap%
\pgfsetroundjoin%
\definecolor{currentfill}{rgb}{0.352360,0.783011,0.392636}%
\pgfsetfillcolor{currentfill}%
\pgfsetlinewidth{0.000000pt}%
\definecolor{currentstroke}{rgb}{0.352360,0.783011,0.392636}%
\pgfsetstrokecolor{currentstroke}%
\pgfsetdash{}{0pt}%
\pgfpathmoveto{\pgfqpoint{4.319824in}{4.607800in}}%
\pgfpathlineto{\pgfqpoint{4.260950in}{4.734983in}}%
\pgfpathlineto{\pgfqpoint{4.177132in}{4.819968in}}%
\pgfpathclose%
\pgfusepath{fill}%
\end{pgfscope}%
\begin{pgfscope}%
\pgfpathrectangle{\pgfqpoint{0.539299in}{0.078740in}}{\pgfqpoint{7.842520in}{7.842520in}}%
\pgfusepath{clip}%
\pgfsetbuttcap%
\pgfsetroundjoin%
\definecolor{currentfill}{rgb}{0.265145,0.232956,0.516599}%
\pgfsetfillcolor{currentfill}%
\pgfsetlinewidth{0.000000pt}%
\definecolor{currentstroke}{rgb}{0.360741,0.785964,0.387814}%
\pgfsetstrokecolor{currentstroke}%
\pgfsetdash{}{0pt}%
\pgfpathmoveto{\pgfqpoint{5.943414in}{2.169358in}}%
\pgfpathlineto{\pgfqpoint{5.801571in}{2.347348in}}%
\pgfpathlineto{\pgfqpoint{5.724190in}{2.352097in}}%
\pgfpathclose%
\pgfusepath{fill}%
\end{pgfscope}%
\begin{pgfscope}%
\pgfpathrectangle{\pgfqpoint{0.539299in}{0.078740in}}{\pgfqpoint{7.842520in}{7.842520in}}%
\pgfusepath{clip}%
\pgfsetbuttcap%
\pgfsetroundjoin%
\definecolor{currentfill}{rgb}{0.814576,0.883393,0.110347}%
\pgfsetfillcolor{currentfill}%
\pgfsetlinewidth{0.000000pt}%
\definecolor{currentstroke}{rgb}{0.369214,0.788888,0.382914}%
\pgfsetstrokecolor{currentstroke}%
\pgfsetdash{}{0pt}%
\pgfpathmoveto{\pgfqpoint{3.465333in}{5.584033in}}%
\pgfpathlineto{\pgfqpoint{3.606833in}{5.491393in}}%
\pgfpathlineto{\pgfqpoint{3.693131in}{5.434822in}}%
\pgfpathclose%
\pgfusepath{fill}%
\end{pgfscope}%
\begin{pgfscope}%
\pgfpathrectangle{\pgfqpoint{0.539299in}{0.078740in}}{\pgfqpoint{7.842520in}{7.842520in}}%
\pgfusepath{clip}%
\pgfsetbuttcap%
\pgfsetroundjoin%
\definecolor{currentfill}{rgb}{0.119423,0.611141,0.538982}%
\pgfsetfillcolor{currentfill}%
\pgfsetlinewidth{0.000000pt}%
\definecolor{currentstroke}{rgb}{0.377779,0.791781,0.377939}%
\pgfsetstrokecolor{currentstroke}%
\pgfsetdash{}{0pt}%
\pgfpathmoveto{\pgfqpoint{4.888578in}{3.720803in}}%
\pgfpathlineto{\pgfqpoint{4.746722in}{3.941721in}}%
\pgfpathlineto{\pgfqpoint{4.665156in}{4.013355in}}%
\pgfpathclose%
\pgfusepath{fill}%
\end{pgfscope}%
\begin{pgfscope}%
\pgfpathrectangle{\pgfqpoint{0.539299in}{0.078740in}}{\pgfqpoint{7.842520in}{7.842520in}}%
\pgfusepath{clip}%
\pgfsetbuttcap%
\pgfsetroundjoin%
\definecolor{currentfill}{rgb}{0.144759,0.519093,0.556572}%
\pgfsetfillcolor{currentfill}%
\pgfsetlinewidth{0.000000pt}%
\definecolor{currentstroke}{rgb}{0.386433,0.794644,0.372886}%
\pgfsetstrokecolor{currentstroke}%
\pgfsetdash{}{0pt}%
\pgfpathmoveto{\pgfqpoint{5.030239in}{3.504331in}}%
\pgfpathlineto{\pgfqpoint{4.950134in}{3.565760in}}%
\pgfpathlineto{\pgfqpoint{5.092313in}{3.347287in}}%
\pgfpathclose%
\pgfusepath{fill}%
\end{pgfscope}%
\begin{pgfscope}%
\pgfpathrectangle{\pgfqpoint{0.539299in}{0.078740in}}{\pgfqpoint{7.842520in}{7.842520in}}%
\pgfusepath{clip}%
\pgfsetbuttcap%
\pgfsetroundjoin%
\definecolor{currentfill}{rgb}{0.203063,0.379716,0.553925}%
\pgfsetfillcolor{currentfill}%
\pgfsetlinewidth{0.000000pt}%
\definecolor{currentstroke}{rgb}{0.395174,0.797475,0.367757}%
\pgfsetstrokecolor{currentstroke}%
\pgfsetdash{}{0pt}%
\pgfpathmoveto{\pgfqpoint{2.058527in}{2.453906in}}%
\pgfpathlineto{\pgfqpoint{2.129568in}{3.063212in}}%
\pgfpathlineto{\pgfqpoint{1.993699in}{3.037069in}}%
\pgfpathclose%
\pgfusepath{fill}%
\end{pgfscope}%
\begin{pgfscope}%
\pgfpathrectangle{\pgfqpoint{0.539299in}{0.078740in}}{\pgfqpoint{7.842520in}{7.842520in}}%
\pgfusepath{clip}%
\pgfsetbuttcap%
\pgfsetroundjoin%
\definecolor{currentfill}{rgb}{0.421908,0.805774,0.351910}%
\pgfsetfillcolor{currentfill}%
\pgfsetlinewidth{0.000000pt}%
\definecolor{currentstroke}{rgb}{0.404001,0.800275,0.362552}%
\pgfsetstrokecolor{currentstroke}%
\pgfsetdash{}{0pt}%
\pgfpathmoveto{\pgfqpoint{4.177132in}{4.819968in}}%
\pgfpathlineto{\pgfqpoint{4.260950in}{4.734983in}}%
\pgfpathlineto{\pgfqpoint{4.034345in}{5.020034in}}%
\pgfpathclose%
\pgfusepath{fill}%
\end{pgfscope}%
\begin{pgfscope}%
\pgfpathrectangle{\pgfqpoint{0.539299in}{0.078740in}}{\pgfqpoint{7.842520in}{7.842520in}}%
\pgfusepath{clip}%
\pgfsetbuttcap%
\pgfsetroundjoin%
\definecolor{currentfill}{rgb}{0.244972,0.287675,0.537260}%
\pgfsetfillcolor{currentfill}%
\pgfsetlinewidth{0.000000pt}%
\definecolor{currentstroke}{rgb}{0.412913,0.803041,0.357269}%
\pgfsetstrokecolor{currentstroke}%
\pgfsetdash{}{0pt}%
\pgfpathmoveto{\pgfqpoint{5.581983in}{2.550780in}}%
\pgfpathlineto{\pgfqpoint{5.801571in}{2.347348in}}%
\pgfpathlineto{\pgfqpoint{5.659792in}{2.533268in}}%
\pgfpathclose%
\pgfusepath{fill}%
\end{pgfscope}%
\begin{pgfscope}%
\pgfpathrectangle{\pgfqpoint{0.539299in}{0.078740in}}{\pgfqpoint{7.842520in}{7.842520in}}%
\pgfusepath{clip}%
\pgfsetbuttcap%
\pgfsetroundjoin%
\definecolor{currentfill}{rgb}{0.772852,0.877868,0.131109}%
\pgfsetfillcolor{currentfill}%
\pgfsetlinewidth{0.000000pt}%
\definecolor{currentstroke}{rgb}{0.421908,0.805774,0.351910}%
\pgfsetstrokecolor{currentstroke}%
\pgfsetdash{}{0pt}%
\pgfpathmoveto{\pgfqpoint{3.693131in}{5.434822in}}%
\pgfpathlineto{\pgfqpoint{3.606833in}{5.491393in}}%
\pgfpathlineto{\pgfqpoint{3.749006in}{5.361964in}}%
\pgfpathclose%
\pgfusepath{fill}%
\end{pgfscope}%
\begin{pgfscope}%
\pgfpathrectangle{\pgfqpoint{0.539299in}{0.078740in}}{\pgfqpoint{7.842520in}{7.842520in}}%
\pgfusepath{clip}%
\pgfsetbuttcap%
\pgfsetroundjoin%
\definecolor{currentfill}{rgb}{0.229739,0.322361,0.545706}%
\pgfsetfillcolor{currentfill}%
\pgfsetlinewidth{0.000000pt}%
\definecolor{currentstroke}{rgb}{0.430983,0.808473,0.346476}%
\pgfsetstrokecolor{currentstroke}%
\pgfsetdash{}{0pt}%
\pgfpathmoveto{\pgfqpoint{5.659792in}{2.533268in}}%
\pgfpathlineto{\pgfqpoint{5.518023in}{2.726615in}}%
\pgfpathlineto{\pgfqpoint{5.581983in}{2.550780in}}%
\pgfpathclose%
\pgfusepath{fill}%
\end{pgfscope}%
\begin{pgfscope}%
\pgfpathrectangle{\pgfqpoint{0.539299in}{0.078740in}}{\pgfqpoint{7.842520in}{7.842520in}}%
\pgfusepath{clip}%
\pgfsetbuttcap%
\pgfsetroundjoin%
\definecolor{currentfill}{rgb}{0.131172,0.555899,0.552459}%
\pgfsetfillcolor{currentfill}%
\pgfsetlinewidth{0.000000pt}%
\definecolor{currentstroke}{rgb}{0.440137,0.811138,0.340967}%
\pgfsetstrokecolor{currentstroke}%
\pgfsetdash{}{0pt}%
\pgfpathmoveto{\pgfqpoint{5.030239in}{3.504331in}}%
\pgfpathlineto{\pgfqpoint{4.807756in}{3.788327in}}%
\pgfpathlineto{\pgfqpoint{4.950134in}{3.565760in}}%
\pgfpathclose%
\pgfusepath{fill}%
\end{pgfscope}%
\begin{pgfscope}%
\pgfpathrectangle{\pgfqpoint{0.539299in}{0.078740in}}{\pgfqpoint{7.842520in}{7.842520in}}%
\pgfusepath{clip}%
\pgfsetbuttcap%
\pgfsetroundjoin%
\definecolor{currentfill}{rgb}{0.154815,0.493313,0.557840}%
\pgfsetfillcolor{currentfill}%
\pgfsetlinewidth{0.000000pt}%
\definecolor{currentstroke}{rgb}{0.449368,0.813768,0.335384}%
\pgfsetstrokecolor{currentstroke}%
\pgfsetdash{}{0pt}%
\pgfpathmoveto{\pgfqpoint{1.993699in}{3.037069in}}%
\pgfpathlineto{\pgfqpoint{2.068708in}{3.529775in}}%
\pgfpathlineto{\pgfqpoint{1.935199in}{3.440713in}}%
\pgfpathclose%
\pgfusepath{fill}%
\end{pgfscope}%
\begin{pgfscope}%
\pgfpathrectangle{\pgfqpoint{0.539299in}{0.078740in}}{\pgfqpoint{7.842520in}{7.842520in}}%
\pgfusepath{clip}%
\pgfsetbuttcap%
\pgfsetroundjoin%
\definecolor{currentfill}{rgb}{0.585678,0.846661,0.249897}%
\pgfsetfillcolor{currentfill}%
\pgfsetlinewidth{0.000000pt}%
\definecolor{currentstroke}{rgb}{0.458674,0.816363,0.329727}%
\pgfsetstrokecolor{currentstroke}%
\pgfsetdash{}{0pt}%
\pgfpathmoveto{\pgfqpoint{3.976785in}{5.127804in}}%
\pgfpathlineto{\pgfqpoint{3.891582in}{5.202680in}}%
\pgfpathlineto{\pgfqpoint{4.034345in}{5.020034in}}%
\pgfpathclose%
\pgfusepath{fill}%
\end{pgfscope}%
\begin{pgfscope}%
\pgfpathrectangle{\pgfqpoint{0.539299in}{0.078740in}}{\pgfqpoint{7.842520in}{7.842520in}}%
\pgfusepath{clip}%
\pgfsetbuttcap%
\pgfsetroundjoin%
\definecolor{currentfill}{rgb}{0.688944,0.865448,0.182725}%
\pgfsetfillcolor{currentfill}%
\pgfsetlinewidth{0.000000pt}%
\definecolor{currentstroke}{rgb}{0.468053,0.818921,0.323998}%
\pgfsetstrokecolor{currentstroke}%
\pgfsetdash{}{0pt}%
\pgfpathmoveto{\pgfqpoint{3.891582in}{5.202680in}}%
\pgfpathlineto{\pgfqpoint{3.834801in}{5.295343in}}%
\pgfpathlineto{\pgfqpoint{3.749006in}{5.361964in}}%
\pgfpathclose%
\pgfusepath{fill}%
\end{pgfscope}%
\begin{pgfscope}%
\pgfpathrectangle{\pgfqpoint{0.539299in}{0.078740in}}{\pgfqpoint{7.842520in}{7.842520in}}%
\pgfusepath{clip}%
\pgfsetbuttcap%
\pgfsetroundjoin%
\definecolor{currentfill}{rgb}{0.281446,0.084320,0.407414}%
\pgfsetfillcolor{currentfill}%
\pgfsetlinewidth{0.000000pt}%
\definecolor{currentstroke}{rgb}{0.477504,0.821444,0.318195}%
\pgfsetstrokecolor{currentstroke}%
\pgfsetdash{}{0pt}%
\pgfpathmoveto{\pgfqpoint{6.150940in}{1.799026in}}%
\pgfpathlineto{\pgfqpoint{6.293516in}{1.634687in}}%
\pgfpathlineto{\pgfqpoint{6.227580in}{1.841168in}}%
\pgfpathclose%
\pgfusepath{fill}%
\end{pgfscope}%
\begin{pgfscope}%
\pgfpathrectangle{\pgfqpoint{0.539299in}{0.078740in}}{\pgfqpoint{7.842520in}{7.842520in}}%
\pgfusepath{clip}%
\pgfsetbuttcap%
\pgfsetroundjoin%
\definecolor{currentfill}{rgb}{0.283197,0.115680,0.436115}%
\pgfsetfillcolor{currentfill}%
\pgfsetlinewidth{0.000000pt}%
\definecolor{currentstroke}{rgb}{0.487026,0.823929,0.312321}%
\pgfsetstrokecolor{currentstroke}%
\pgfsetdash{}{0pt}%
\pgfpathmoveto{\pgfqpoint{6.227580in}{1.841168in}}%
\pgfpathlineto{\pgfqpoint{6.008600in}{1.974963in}}%
\pgfpathlineto{\pgfqpoint{6.150940in}{1.799026in}}%
\pgfpathclose%
\pgfusepath{fill}%
\end{pgfscope}%
\begin{pgfscope}%
\pgfpathrectangle{\pgfqpoint{0.539299in}{0.078740in}}{\pgfqpoint{7.842520in}{7.842520in}}%
\pgfusepath{clip}%
\pgfsetbuttcap%
\pgfsetroundjoin%
\definecolor{currentfill}{rgb}{0.140210,0.665859,0.513427}%
\pgfsetfillcolor{currentfill}%
\pgfsetlinewidth{0.000000pt}%
\definecolor{currentstroke}{rgb}{0.496615,0.826376,0.306377}%
\pgfsetstrokecolor{currentstroke}%
\pgfsetdash{}{0pt}%
\pgfpathmoveto{\pgfqpoint{4.604647in}{4.165055in}}%
\pgfpathlineto{\pgfqpoint{4.522320in}{4.238703in}}%
\pgfpathlineto{\pgfqpoint{4.746722in}{3.941721in}}%
\pgfpathclose%
\pgfusepath{fill}%
\end{pgfscope}%
\begin{pgfscope}%
\pgfpathrectangle{\pgfqpoint{0.539299in}{0.078740in}}{\pgfqpoint{7.842520in}{7.842520in}}%
\pgfusepath{clip}%
\pgfsetbuttcap%
\pgfsetroundjoin%
\definecolor{currentfill}{rgb}{0.876168,0.891125,0.095250}%
\pgfsetfillcolor{currentfill}%
\pgfsetlinewidth{0.000000pt}%
\definecolor{currentstroke}{rgb}{0.506271,0.828786,0.300362}%
\pgfsetstrokecolor{currentstroke}%
\pgfsetdash{}{0pt}%
\pgfpathmoveto{\pgfqpoint{3.010887in}{5.591305in}}%
\pgfpathlineto{\pgfqpoint{3.098325in}{5.625023in}}%
\pgfpathlineto{\pgfqpoint{2.961006in}{5.551827in}}%
\pgfpathclose%
\pgfusepath{fill}%
\end{pgfscope}%
\begin{pgfscope}%
\pgfpathrectangle{\pgfqpoint{0.539299in}{0.078740in}}{\pgfqpoint{7.842520in}{7.842520in}}%
\pgfusepath{clip}%
\pgfsetbuttcap%
\pgfsetroundjoin%
\definecolor{currentfill}{rgb}{0.855810,0.888601,0.097452}%
\pgfsetfillcolor{currentfill}%
\pgfsetlinewidth{0.000000pt}%
\definecolor{currentstroke}{rgb}{0.515992,0.831158,0.294279}%
\pgfsetstrokecolor{currentstroke}%
\pgfsetdash{}{0pt}%
\pgfpathmoveto{\pgfqpoint{2.961006in}{5.551827in}}%
\pgfpathlineto{\pgfqpoint{2.873601in}{5.508440in}}%
\pgfpathlineto{\pgfqpoint{3.010887in}{5.591305in}}%
\pgfpathclose%
\pgfusepath{fill}%
\end{pgfscope}%
\begin{pgfscope}%
\pgfpathrectangle{\pgfqpoint{0.539299in}{0.078740in}}{\pgfqpoint{7.842520in}{7.842520in}}%
\pgfusepath{clip}%
\pgfsetbuttcap%
\pgfsetroundjoin%
\definecolor{currentfill}{rgb}{0.277134,0.185228,0.489898}%
\pgfsetfillcolor{currentfill}%
\pgfsetlinewidth{0.000000pt}%
\definecolor{currentstroke}{rgb}{0.525776,0.833491,0.288127}%
\pgfsetstrokecolor{currentstroke}%
\pgfsetdash{}{0pt}%
\pgfpathmoveto{\pgfqpoint{5.866377in}{2.159848in}}%
\pgfpathlineto{\pgfqpoint{6.008600in}{1.974963in}}%
\pgfpathlineto{\pgfqpoint{5.943414in}{2.169358in}}%
\pgfpathclose%
\pgfusepath{fill}%
\end{pgfscope}%
\begin{pgfscope}%
\pgfpathrectangle{\pgfqpoint{0.539299in}{0.078740in}}{\pgfqpoint{7.842520in}{7.842520in}}%
\pgfusepath{clip}%
\pgfsetbuttcap%
\pgfsetroundjoin%
\definecolor{currentfill}{rgb}{0.223925,0.334994,0.548053}%
\pgfsetfillcolor{currentfill}%
\pgfsetlinewidth{0.000000pt}%
\definecolor{currentstroke}{rgb}{0.535621,0.835785,0.281908}%
\pgfsetstrokecolor{currentstroke}%
\pgfsetdash{}{0pt}%
\pgfpathmoveto{\pgfqpoint{2.058527in}{2.453906in}}%
\pgfpathlineto{\pgfqpoint{2.195719in}{2.435884in}}%
\pgfpathlineto{\pgfqpoint{2.266737in}{3.071192in}}%
\pgfpathclose%
\pgfusepath{fill}%
\end{pgfscope}%
\begin{pgfscope}%
\pgfpathrectangle{\pgfqpoint{0.539299in}{0.078740in}}{\pgfqpoint{7.842520in}{7.842520in}}%
\pgfusepath{clip}%
\pgfsetbuttcap%
\pgfsetroundjoin%
\definecolor{currentfill}{rgb}{0.906311,0.894855,0.098125}%
\pgfsetfillcolor{currentfill}%
\pgfsetlinewidth{0.000000pt}%
\definecolor{currentstroke}{rgb}{0.545524,0.838039,0.275626}%
\pgfsetstrokecolor{currentstroke}%
\pgfsetdash{}{0pt}%
\pgfpathmoveto{\pgfqpoint{3.324832in}{5.632651in}}%
\pgfpathlineto{\pgfqpoint{3.098325in}{5.625023in}}%
\pgfpathlineto{\pgfqpoint{3.237559in}{5.639984in}}%
\pgfpathclose%
\pgfusepath{fill}%
\end{pgfscope}%
\begin{pgfscope}%
\pgfpathrectangle{\pgfqpoint{0.539299in}{0.078740in}}{\pgfqpoint{7.842520in}{7.842520in}}%
\pgfusepath{clip}%
\pgfsetbuttcap%
\pgfsetroundjoin%
\definecolor{currentfill}{rgb}{0.386433,0.794644,0.372886}%
\pgfsetfillcolor{currentfill}%
\pgfsetlinewidth{0.000000pt}%
\definecolor{currentstroke}{rgb}{0.555484,0.840254,0.269281}%
\pgfsetstrokecolor{currentstroke}%
\pgfsetdash{}{0pt}%
\pgfpathmoveto{\pgfqpoint{2.347219in}{4.810962in}}%
\pgfpathlineto{\pgfqpoint{2.261900in}{4.632062in}}%
\pgfpathlineto{\pgfqpoint{2.393562in}{4.830606in}}%
\pgfpathclose%
\pgfusepath{fill}%
\end{pgfscope}%
\begin{pgfscope}%
\pgfpathrectangle{\pgfqpoint{0.539299in}{0.078740in}}{\pgfqpoint{7.842520in}{7.842520in}}%
\pgfusepath{clip}%
\pgfsetbuttcap%
\pgfsetroundjoin%
\definecolor{currentfill}{rgb}{0.267968,0.223549,0.512008}%
\pgfsetfillcolor{currentfill}%
\pgfsetlinewidth{0.000000pt}%
\definecolor{currentstroke}{rgb}{0.565498,0.842430,0.262877}%
\pgfsetstrokecolor{currentstroke}%
\pgfsetdash{}{0pt}%
\pgfpathmoveto{\pgfqpoint{5.866377in}{2.159848in}}%
\pgfpathlineto{\pgfqpoint{5.943414in}{2.169358in}}%
\pgfpathlineto{\pgfqpoint{5.724190in}{2.352097in}}%
\pgfpathclose%
\pgfusepath{fill}%
\end{pgfscope}%
\begin{pgfscope}%
\pgfpathrectangle{\pgfqpoint{0.539299in}{0.078740in}}{\pgfqpoint{7.842520in}{7.842520in}}%
\pgfusepath{clip}%
\pgfsetbuttcap%
\pgfsetroundjoin%
\definecolor{currentfill}{rgb}{0.197636,0.391528,0.554969}%
\pgfsetfillcolor{currentfill}%
\pgfsetlinewidth{0.000000pt}%
\definecolor{currentstroke}{rgb}{0.575563,0.844566,0.256415}%
\pgfsetstrokecolor{currentstroke}%
\pgfsetdash{}{0pt}%
\pgfpathmoveto{\pgfqpoint{5.518023in}{2.726615in}}%
\pgfpathlineto{\pgfqpoint{5.376216in}{2.927023in}}%
\pgfpathlineto{\pgfqpoint{5.297333in}{2.965317in}}%
\pgfpathclose%
\pgfusepath{fill}%
\end{pgfscope}%
\begin{pgfscope}%
\pgfpathrectangle{\pgfqpoint{0.539299in}{0.078740in}}{\pgfqpoint{7.842520in}{7.842520in}}%
\pgfusepath{clip}%
\pgfsetbuttcap%
\pgfsetroundjoin%
\definecolor{currentfill}{rgb}{0.525776,0.833491,0.288127}%
\pgfsetfillcolor{currentfill}%
\pgfsetlinewidth{0.000000pt}%
\definecolor{currentstroke}{rgb}{0.585678,0.846661,0.249897}%
\pgfsetstrokecolor{currentstroke}%
\pgfsetdash{}{0pt}%
\pgfpathmoveto{\pgfqpoint{2.347219in}{4.810962in}}%
\pgfpathlineto{\pgfqpoint{2.478789in}{5.021565in}}%
\pgfpathlineto{\pgfqpoint{2.564916in}{5.170485in}}%
\pgfpathclose%
\pgfusepath{fill}%
\end{pgfscope}%
\begin{pgfscope}%
\pgfpathrectangle{\pgfqpoint{0.539299in}{0.078740in}}{\pgfqpoint{7.842520in}{7.842520in}}%
\pgfusepath{clip}%
\pgfsetbuttcap%
\pgfsetroundjoin%
\definecolor{currentfill}{rgb}{0.772852,0.877868,0.131109}%
\pgfsetfillcolor{currentfill}%
\pgfsetlinewidth{0.000000pt}%
\definecolor{currentstroke}{rgb}{0.595839,0.848717,0.243329}%
\pgfsetstrokecolor{currentstroke}%
\pgfsetdash{}{0pt}%
\pgfpathmoveto{\pgfqpoint{2.651662in}{5.282437in}}%
\pgfpathlineto{\pgfqpoint{2.786367in}{5.433811in}}%
\pgfpathlineto{\pgfqpoint{2.873601in}{5.508440in}}%
\pgfpathclose%
\pgfusepath{fill}%
\end{pgfscope}%
\begin{pgfscope}%
\pgfpathrectangle{\pgfqpoint{0.539299in}{0.078740in}}{\pgfqpoint{7.842520in}{7.842520in}}%
\pgfusepath{clip}%
\pgfsetbuttcap%
\pgfsetroundjoin%
\definecolor{currentfill}{rgb}{0.119512,0.607464,0.540218}%
\pgfsetfillcolor{currentfill}%
\pgfsetlinewidth{0.000000pt}%
\definecolor{currentstroke}{rgb}{0.606045,0.850733,0.236712}%
\pgfsetstrokecolor{currentstroke}%
\pgfsetdash{}{0pt}%
\pgfpathmoveto{\pgfqpoint{4.665156in}{4.013355in}}%
\pgfpathlineto{\pgfqpoint{4.807756in}{3.788327in}}%
\pgfpathlineto{\pgfqpoint{4.888578in}{3.720803in}}%
\pgfpathclose%
\pgfusepath{fill}%
\end{pgfscope}%
\begin{pgfscope}%
\pgfpathrectangle{\pgfqpoint{0.539299in}{0.078740in}}{\pgfqpoint{7.842520in}{7.842520in}}%
\pgfusepath{clip}%
\pgfsetbuttcap%
\pgfsetroundjoin%
\definecolor{currentfill}{rgb}{0.122606,0.585371,0.546557}%
\pgfsetfillcolor{currentfill}%
\pgfsetlinewidth{0.000000pt}%
\definecolor{currentstroke}{rgb}{0.616293,0.852709,0.230052}%
\pgfsetstrokecolor{currentstroke}%
\pgfsetdash{}{0pt}%
\pgfpathmoveto{\pgfqpoint{2.013886in}{3.822753in}}%
\pgfpathlineto{\pgfqpoint{1.935199in}{3.440713in}}%
\pgfpathlineto{\pgfqpoint{2.146689in}{3.948369in}}%
\pgfpathclose%
\pgfusepath{fill}%
\end{pgfscope}%
\begin{pgfscope}%
\pgfpathrectangle{\pgfqpoint{0.539299in}{0.078740in}}{\pgfqpoint{7.842520in}{7.842520in}}%
\pgfusepath{clip}%
\pgfsetbuttcap%
\pgfsetroundjoin%
\definecolor{currentfill}{rgb}{0.248629,0.278775,0.534556}%
\pgfsetfillcolor{currentfill}%
\pgfsetlinewidth{0.000000pt}%
\definecolor{currentstroke}{rgb}{0.626579,0.854645,0.223353}%
\pgfsetstrokecolor{currentstroke}%
\pgfsetdash{}{0pt}%
\pgfpathmoveto{\pgfqpoint{5.581983in}{2.550780in}}%
\pgfpathlineto{\pgfqpoint{5.724190in}{2.352097in}}%
\pgfpathlineto{\pgfqpoint{5.801571in}{2.347348in}}%
\pgfpathclose%
\pgfusepath{fill}%
\end{pgfscope}%
\begin{pgfscope}%
\pgfpathrectangle{\pgfqpoint{0.539299in}{0.078740in}}{\pgfqpoint{7.842520in}{7.842520in}}%
\pgfusepath{clip}%
\pgfsetbuttcap%
\pgfsetroundjoin%
\definecolor{currentfill}{rgb}{0.182256,0.426184,0.557120}%
\pgfsetfillcolor{currentfill}%
\pgfsetlinewidth{0.000000pt}%
\definecolor{currentstroke}{rgb}{0.636902,0.856542,0.216620}%
\pgfsetstrokecolor{currentstroke}%
\pgfsetdash{}{0pt}%
\pgfpathmoveto{\pgfqpoint{5.297333in}{2.965317in}}%
\pgfpathlineto{\pgfqpoint{5.376216in}{2.927023in}}%
\pgfpathlineto{\pgfqpoint{5.234327in}{3.134099in}}%
\pgfpathclose%
\pgfusepath{fill}%
\end{pgfscope}%
\begin{pgfscope}%
\pgfpathrectangle{\pgfqpoint{0.539299in}{0.078740in}}{\pgfqpoint{7.842520in}{7.842520in}}%
\pgfusepath{clip}%
\pgfsetbuttcap%
\pgfsetroundjoin%
\definecolor{currentfill}{rgb}{0.202219,0.715272,0.476084}%
\pgfsetfillcolor{currentfill}%
\pgfsetlinewidth{0.000000pt}%
\definecolor{currentstroke}{rgb}{0.647257,0.858400,0.209861}%
\pgfsetstrokecolor{currentstroke}%
\pgfsetdash{}{0pt}%
\pgfpathmoveto{\pgfqpoint{4.604647in}{4.165055in}}%
\pgfpathlineto{\pgfqpoint{4.462343in}{4.388175in}}%
\pgfpathlineto{\pgfqpoint{4.379257in}{4.461675in}}%
\pgfpathclose%
\pgfusepath{fill}%
\end{pgfscope}%
\begin{pgfscope}%
\pgfpathrectangle{\pgfqpoint{0.539299in}{0.078740in}}{\pgfqpoint{7.842520in}{7.842520in}}%
\pgfusepath{clip}%
\pgfsetbuttcap%
\pgfsetroundjoin%
\definecolor{currentfill}{rgb}{0.163625,0.471133,0.558148}%
\pgfsetfillcolor{currentfill}%
\pgfsetlinewidth{0.000000pt}%
\definecolor{currentstroke}{rgb}{0.657642,0.860219,0.203082}%
\pgfsetstrokecolor{currentstroke}%
\pgfsetdash{}{0pt}%
\pgfpathmoveto{\pgfqpoint{1.993699in}{3.037069in}}%
\pgfpathlineto{\pgfqpoint{2.129568in}{3.063212in}}%
\pgfpathlineto{\pgfqpoint{2.068708in}{3.529775in}}%
\pgfpathclose%
\pgfusepath{fill}%
\end{pgfscope}%
\begin{pgfscope}%
\pgfpathrectangle{\pgfqpoint{0.539299in}{0.078740in}}{\pgfqpoint{7.842520in}{7.842520in}}%
\pgfusepath{clip}%
\pgfsetbuttcap%
\pgfsetroundjoin%
\definecolor{currentfill}{rgb}{0.896320,0.893616,0.096335}%
\pgfsetfillcolor{currentfill}%
\pgfsetlinewidth{0.000000pt}%
\definecolor{currentstroke}{rgb}{0.668054,0.861999,0.196293}%
\pgfsetstrokecolor{currentstroke}%
\pgfsetdash{}{0pt}%
\pgfpathmoveto{\pgfqpoint{3.465333in}{5.584033in}}%
\pgfpathlineto{\pgfqpoint{3.324832in}{5.632651in}}%
\pgfpathlineto{\pgfqpoint{3.378279in}{5.603577in}}%
\pgfpathclose%
\pgfusepath{fill}%
\end{pgfscope}%
\begin{pgfscope}%
\pgfpathrectangle{\pgfqpoint{0.539299in}{0.078740in}}{\pgfqpoint{7.842520in}{7.842520in}}%
\pgfusepath{clip}%
\pgfsetbuttcap%
\pgfsetroundjoin%
\definecolor{currentfill}{rgb}{0.197636,0.391528,0.554969}%
\pgfsetfillcolor{currentfill}%
\pgfsetlinewidth{0.000000pt}%
\definecolor{currentstroke}{rgb}{0.678489,0.863742,0.189503}%
\pgfsetstrokecolor{currentstroke}%
\pgfsetdash{}{0pt}%
\pgfpathmoveto{\pgfqpoint{2.266737in}{3.071192in}}%
\pgfpathlineto{\pgfqpoint{2.129568in}{3.063212in}}%
\pgfpathlineto{\pgfqpoint{2.058527in}{2.453906in}}%
\pgfpathclose%
\pgfusepath{fill}%
\end{pgfscope}%
\begin{pgfscope}%
\pgfpathrectangle{\pgfqpoint{0.539299in}{0.078740in}}{\pgfqpoint{7.842520in}{7.842520in}}%
\pgfusepath{clip}%
\pgfsetbuttcap%
\pgfsetroundjoin%
\definecolor{currentfill}{rgb}{0.216210,0.351535,0.550627}%
\pgfsetfillcolor{currentfill}%
\pgfsetlinewidth{0.000000pt}%
\definecolor{currentstroke}{rgb}{0.688944,0.865448,0.182725}%
\pgfsetstrokecolor{currentstroke}%
\pgfsetdash{}{0pt}%
\pgfpathmoveto{\pgfqpoint{5.518023in}{2.726615in}}%
\pgfpathlineto{\pgfqpoint{5.439710in}{2.755331in}}%
\pgfpathlineto{\pgfqpoint{5.581983in}{2.550780in}}%
\pgfpathclose%
\pgfusepath{fill}%
\end{pgfscope}%
\begin{pgfscope}%
\pgfpathrectangle{\pgfqpoint{0.539299in}{0.078740in}}{\pgfqpoint{7.842520in}{7.842520in}}%
\pgfusepath{clip}%
\pgfsetbuttcap%
\pgfsetroundjoin%
\definecolor{currentfill}{rgb}{0.296479,0.761561,0.424223}%
\pgfsetfillcolor{currentfill}%
\pgfsetlinewidth{0.000000pt}%
\definecolor{currentstroke}{rgb}{0.699415,0.867117,0.175971}%
\pgfsetstrokecolor{currentstroke}%
\pgfsetdash{}{0pt}%
\pgfpathmoveto{\pgfqpoint{2.177668in}{4.411616in}}%
\pgfpathlineto{\pgfqpoint{2.309553in}{4.592125in}}%
\pgfpathlineto{\pgfqpoint{2.261900in}{4.632062in}}%
\pgfpathclose%
\pgfusepath{fill}%
\end{pgfscope}%
\begin{pgfscope}%
\pgfpathrectangle{\pgfqpoint{0.539299in}{0.078740in}}{\pgfqpoint{7.842520in}{7.842520in}}%
\pgfusepath{clip}%
\pgfsetbuttcap%
\pgfsetroundjoin%
\definecolor{currentfill}{rgb}{0.266941,0.748751,0.440573}%
\pgfsetfillcolor{currentfill}%
\pgfsetlinewidth{0.000000pt}%
\definecolor{currentstroke}{rgb}{0.709898,0.868751,0.169257}%
\pgfsetstrokecolor{currentstroke}%
\pgfsetdash{}{0pt}%
\pgfpathmoveto{\pgfqpoint{4.319824in}{4.607800in}}%
\pgfpathlineto{\pgfqpoint{4.379257in}{4.461675in}}%
\pgfpathlineto{\pgfqpoint{4.462343in}{4.388175in}}%
\pgfpathclose%
\pgfusepath{fill}%
\end{pgfscope}%
\begin{pgfscope}%
\pgfpathrectangle{\pgfqpoint{0.539299in}{0.078740in}}{\pgfqpoint{7.842520in}{7.842520in}}%
\pgfusepath{clip}%
\pgfsetbuttcap%
\pgfsetroundjoin%
\definecolor{currentfill}{rgb}{0.137339,0.662252,0.515571}%
\pgfsetfillcolor{currentfill}%
\pgfsetlinewidth{0.000000pt}%
\definecolor{currentstroke}{rgb}{0.720391,0.870350,0.162603}%
\pgfsetstrokecolor{currentstroke}%
\pgfsetdash{}{0pt}%
\pgfpathmoveto{\pgfqpoint{4.746722in}{3.941721in}}%
\pgfpathlineto{\pgfqpoint{4.522320in}{4.238703in}}%
\pgfpathlineto{\pgfqpoint{4.665156in}{4.013355in}}%
\pgfpathclose%
\pgfusepath{fill}%
\end{pgfscope}%
\begin{pgfscope}%
\pgfpathrectangle{\pgfqpoint{0.539299in}{0.078740in}}{\pgfqpoint{7.842520in}{7.842520in}}%
\pgfusepath{clip}%
\pgfsetbuttcap%
\pgfsetroundjoin%
\definecolor{currentfill}{rgb}{0.127568,0.566949,0.550556}%
\pgfsetfillcolor{currentfill}%
\pgfsetlinewidth{0.000000pt}%
\definecolor{currentstroke}{rgb}{0.730889,0.871916,0.156029}%
\pgfsetstrokecolor{currentstroke}%
\pgfsetdash{}{0pt}%
\pgfpathmoveto{\pgfqpoint{2.146689in}{3.948369in}}%
\pgfpathlineto{\pgfqpoint{1.935199in}{3.440713in}}%
\pgfpathlineto{\pgfqpoint{2.068708in}{3.529775in}}%
\pgfpathclose%
\pgfusepath{fill}%
\end{pgfscope}%
\begin{pgfscope}%
\pgfpathrectangle{\pgfqpoint{0.539299in}{0.078740in}}{\pgfqpoint{7.842520in}{7.842520in}}%
\pgfusepath{clip}%
\pgfsetbuttcap%
\pgfsetroundjoin%
\definecolor{currentfill}{rgb}{0.162142,0.474838,0.558140}%
\pgfsetfillcolor{currentfill}%
\pgfsetlinewidth{0.000000pt}%
\definecolor{currentstroke}{rgb}{0.741388,0.873449,0.149561}%
\pgfsetstrokecolor{currentstroke}%
\pgfsetdash{}{0pt}%
\pgfpathmoveto{\pgfqpoint{5.154816in}{3.180272in}}%
\pgfpathlineto{\pgfqpoint{5.234327in}{3.134099in}}%
\pgfpathlineto{\pgfqpoint{5.092313in}{3.347287in}}%
\pgfpathclose%
\pgfusepath{fill}%
\end{pgfscope}%
\begin{pgfscope}%
\pgfpathrectangle{\pgfqpoint{0.539299in}{0.078740in}}{\pgfqpoint{7.842520in}{7.842520in}}%
\pgfusepath{clip}%
\pgfsetbuttcap%
\pgfsetroundjoin%
\definecolor{currentfill}{rgb}{0.468053,0.818921,0.323998}%
\pgfsetfillcolor{currentfill}%
\pgfsetlinewidth{0.000000pt}%
\definecolor{currentstroke}{rgb}{0.751884,0.874951,0.143228}%
\pgfsetstrokecolor{currentstroke}%
\pgfsetdash{}{0pt}%
\pgfpathmoveto{\pgfqpoint{2.393562in}{4.830606in}}%
\pgfpathlineto{\pgfqpoint{2.478789in}{5.021565in}}%
\pgfpathlineto{\pgfqpoint{2.347219in}{4.810962in}}%
\pgfpathclose%
\pgfusepath{fill}%
\end{pgfscope}%
\begin{pgfscope}%
\pgfpathrectangle{\pgfqpoint{0.539299in}{0.078740in}}{\pgfqpoint{7.842520in}{7.842520in}}%
\pgfusepath{clip}%
\pgfsetbuttcap%
\pgfsetroundjoin%
\definecolor{currentfill}{rgb}{0.143303,0.669459,0.511215}%
\pgfsetfillcolor{currentfill}%
\pgfsetlinewidth{0.000000pt}%
\definecolor{currentstroke}{rgb}{0.762373,0.876424,0.137064}%
\pgfsetstrokecolor{currentstroke}%
\pgfsetdash{}{0pt}%
\pgfpathmoveto{\pgfqpoint{2.227125in}{4.300185in}}%
\pgfpathlineto{\pgfqpoint{2.094863in}{4.143974in}}%
\pgfpathlineto{\pgfqpoint{2.013886in}{3.822753in}}%
\pgfpathclose%
\pgfusepath{fill}%
\end{pgfscope}%
\begin{pgfscope}%
\pgfpathrectangle{\pgfqpoint{0.539299in}{0.078740in}}{\pgfqpoint{7.842520in}{7.842520in}}%
\pgfusepath{clip}%
\pgfsetbuttcap%
\pgfsetroundjoin%
\definecolor{currentfill}{rgb}{0.699415,0.867117,0.175971}%
\pgfsetfillcolor{currentfill}%
\pgfsetlinewidth{0.000000pt}%
\definecolor{currentstroke}{rgb}{0.772852,0.877868,0.131109}%
\pgfsetstrokecolor{currentstroke}%
\pgfsetdash{}{0pt}%
\pgfpathmoveto{\pgfqpoint{2.699533in}{5.323077in}}%
\pgfpathlineto{\pgfqpoint{2.651662in}{5.282437in}}%
\pgfpathlineto{\pgfqpoint{2.564916in}{5.170485in}}%
\pgfpathclose%
\pgfusepath{fill}%
\end{pgfscope}%
\begin{pgfscope}%
\pgfpathrectangle{\pgfqpoint{0.539299in}{0.078740in}}{\pgfqpoint{7.842520in}{7.842520in}}%
\pgfusepath{clip}%
\pgfsetbuttcap%
\pgfsetroundjoin%
\definecolor{currentfill}{rgb}{0.191090,0.708366,0.482284}%
\pgfsetfillcolor{currentfill}%
\pgfsetlinewidth{0.000000pt}%
\definecolor{currentstroke}{rgb}{0.783315,0.879285,0.125405}%
\pgfsetstrokecolor{currentstroke}%
\pgfsetdash{}{0pt}%
\pgfpathmoveto{\pgfqpoint{2.177668in}{4.411616in}}%
\pgfpathlineto{\pgfqpoint{2.094863in}{4.143974in}}%
\pgfpathlineto{\pgfqpoint{2.227125in}{4.300185in}}%
\pgfpathclose%
\pgfusepath{fill}%
\end{pgfscope}%
\begin{pgfscope}%
\pgfpathrectangle{\pgfqpoint{0.539299in}{0.078740in}}{\pgfqpoint{7.842520in}{7.842520in}}%
\pgfusepath{clip}%
\pgfsetbuttcap%
\pgfsetroundjoin%
\definecolor{currentfill}{rgb}{0.926106,0.897330,0.104071}%
\pgfsetfillcolor{currentfill}%
\pgfsetlinewidth{0.000000pt}%
\definecolor{currentstroke}{rgb}{0.793760,0.880678,0.120005}%
\pgfsetstrokecolor{currentstroke}%
\pgfsetdash{}{0pt}%
\pgfpathmoveto{\pgfqpoint{3.378279in}{5.603577in}}%
\pgfpathlineto{\pgfqpoint{3.324832in}{5.632651in}}%
\pgfpathlineto{\pgfqpoint{3.237559in}{5.639984in}}%
\pgfpathclose%
\pgfusepath{fill}%
\end{pgfscope}%
\begin{pgfscope}%
\pgfpathrectangle{\pgfqpoint{0.539299in}{0.078740in}}{\pgfqpoint{7.842520in}{7.842520in}}%
\pgfusepath{clip}%
\pgfsetbuttcap%
\pgfsetroundjoin%
\definecolor{currentfill}{rgb}{0.199430,0.387607,0.554642}%
\pgfsetfillcolor{currentfill}%
\pgfsetlinewidth{0.000000pt}%
\definecolor{currentstroke}{rgb}{0.804182,0.882046,0.114965}%
\pgfsetstrokecolor{currentstroke}%
\pgfsetdash{}{0pt}%
\pgfpathmoveto{\pgfqpoint{5.297333in}{2.965317in}}%
\pgfpathlineto{\pgfqpoint{5.439710in}{2.755331in}}%
\pgfpathlineto{\pgfqpoint{5.518023in}{2.726615in}}%
\pgfpathclose%
\pgfusepath{fill}%
\end{pgfscope}%
\begin{pgfscope}%
\pgfpathrectangle{\pgfqpoint{0.539299in}{0.078740in}}{\pgfqpoint{7.842520in}{7.842520in}}%
\pgfusepath{clip}%
\pgfsetbuttcap%
\pgfsetroundjoin%
\definecolor{currentfill}{rgb}{0.886271,0.892374,0.095374}%
\pgfsetfillcolor{currentfill}%
\pgfsetlinewidth{0.000000pt}%
\definecolor{currentstroke}{rgb}{0.814576,0.883393,0.110347}%
\pgfsetstrokecolor{currentstroke}%
\pgfsetdash{}{0pt}%
\pgfpathmoveto{\pgfqpoint{3.378279in}{5.603577in}}%
\pgfpathlineto{\pgfqpoint{3.606833in}{5.491393in}}%
\pgfpathlineto{\pgfqpoint{3.465333in}{5.584033in}}%
\pgfpathclose%
\pgfusepath{fill}%
\end{pgfscope}%
\begin{pgfscope}%
\pgfpathrectangle{\pgfqpoint{0.539299in}{0.078740in}}{\pgfqpoint{7.842520in}{7.842520in}}%
\pgfusepath{clip}%
\pgfsetbuttcap%
\pgfsetroundjoin%
\definecolor{currentfill}{rgb}{0.855810,0.888601,0.097452}%
\pgfsetfillcolor{currentfill}%
\pgfsetlinewidth{0.000000pt}%
\definecolor{currentstroke}{rgb}{0.824940,0.884720,0.106217}%
\pgfsetstrokecolor{currentstroke}%
\pgfsetdash{}{0pt}%
\pgfpathmoveto{\pgfqpoint{3.010887in}{5.591305in}}%
\pgfpathlineto{\pgfqpoint{2.873601in}{5.508440in}}%
\pgfpathlineto{\pgfqpoint{2.786367in}{5.433811in}}%
\pgfpathclose%
\pgfusepath{fill}%
\end{pgfscope}%
\begin{pgfscope}%
\pgfpathrectangle{\pgfqpoint{0.539299in}{0.078740in}}{\pgfqpoint{7.842520in}{7.842520in}}%
\pgfusepath{clip}%
\pgfsetbuttcap%
\pgfsetroundjoin%
\definecolor{currentfill}{rgb}{0.196571,0.711827,0.479221}%
\pgfsetfillcolor{currentfill}%
\pgfsetlinewidth{0.000000pt}%
\definecolor{currentstroke}{rgb}{0.835270,0.886029,0.102646}%
\pgfsetstrokecolor{currentstroke}%
\pgfsetdash{}{0pt}%
\pgfpathmoveto{\pgfqpoint{4.379257in}{4.461675in}}%
\pgfpathlineto{\pgfqpoint{4.522320in}{4.238703in}}%
\pgfpathlineto{\pgfqpoint{4.604647in}{4.165055in}}%
\pgfpathclose%
\pgfusepath{fill}%
\end{pgfscope}%
\begin{pgfscope}%
\pgfpathrectangle{\pgfqpoint{0.539299in}{0.078740in}}{\pgfqpoint{7.842520in}{7.842520in}}%
\pgfusepath{clip}%
\pgfsetbuttcap%
\pgfsetroundjoin%
\definecolor{currentfill}{rgb}{0.412913,0.803041,0.357269}%
\pgfsetfillcolor{currentfill}%
\pgfsetlinewidth{0.000000pt}%
\definecolor{currentstroke}{rgb}{0.845561,0.887322,0.099702}%
\pgfsetstrokecolor{currentstroke}%
\pgfsetdash{}{0pt}%
\pgfpathmoveto{\pgfqpoint{4.177132in}{4.819968in}}%
\pgfpathlineto{\pgfqpoint{4.092594in}{4.886710in}}%
\pgfpathlineto{\pgfqpoint{4.319824in}{4.607800in}}%
\pgfpathclose%
\pgfusepath{fill}%
\end{pgfscope}%
\begin{pgfscope}%
\pgfpathrectangle{\pgfqpoint{0.539299in}{0.078740in}}{\pgfqpoint{7.842520in}{7.842520in}}%
\pgfusepath{clip}%
\pgfsetbuttcap%
\pgfsetroundjoin%
\definecolor{currentfill}{rgb}{0.218130,0.347432,0.550038}%
\pgfsetfillcolor{currentfill}%
\pgfsetlinewidth{0.000000pt}%
\definecolor{currentstroke}{rgb}{0.855810,0.888601,0.097452}%
\pgfsetstrokecolor{currentstroke}%
\pgfsetdash{}{0pt}%
\pgfpathmoveto{\pgfqpoint{2.195719in}{2.435884in}}%
\pgfpathlineto{\pgfqpoint{2.333716in}{2.409916in}}%
\pgfpathlineto{\pgfqpoint{2.405032in}{3.063501in}}%
\pgfpathclose%
\pgfusepath{fill}%
\end{pgfscope}%
\begin{pgfscope}%
\pgfpathrectangle{\pgfqpoint{0.539299in}{0.078740in}}{\pgfqpoint{7.842520in}{7.842520in}}%
\pgfusepath{clip}%
\pgfsetbuttcap%
\pgfsetroundjoin%
\definecolor{currentfill}{rgb}{0.762373,0.876424,0.137064}%
\pgfsetfillcolor{currentfill}%
\pgfsetlinewidth{0.000000pt}%
\definecolor{currentstroke}{rgb}{0.866013,0.889868,0.095953}%
\pgfsetstrokecolor{currentstroke}%
\pgfsetdash{}{0pt}%
\pgfpathmoveto{\pgfqpoint{2.786367in}{5.433811in}}%
\pgfpathlineto{\pgfqpoint{2.651662in}{5.282437in}}%
\pgfpathlineto{\pgfqpoint{2.699533in}{5.323077in}}%
\pgfpathclose%
\pgfusepath{fill}%
\end{pgfscope}%
\begin{pgfscope}%
\pgfpathrectangle{\pgfqpoint{0.539299in}{0.078740in}}{\pgfqpoint{7.842520in}{7.842520in}}%
\pgfusepath{clip}%
\pgfsetbuttcap%
\pgfsetroundjoin%
\definecolor{currentfill}{rgb}{0.282327,0.094955,0.417331}%
\pgfsetfillcolor{currentfill}%
\pgfsetlinewidth{0.000000pt}%
\definecolor{currentstroke}{rgb}{0.876168,0.891125,0.095250}%
\pgfsetstrokecolor{currentstroke}%
\pgfsetdash{}{0pt}%
\pgfpathmoveto{\pgfqpoint{6.073966in}{1.765621in}}%
\pgfpathlineto{\pgfqpoint{6.293516in}{1.634687in}}%
\pgfpathlineto{\pgfqpoint{6.150940in}{1.799026in}}%
\pgfpathclose%
\pgfusepath{fill}%
\end{pgfscope}%
\begin{pgfscope}%
\pgfpathrectangle{\pgfqpoint{0.539299in}{0.078740in}}{\pgfqpoint{7.842520in}{7.842520in}}%
\pgfusepath{clip}%
\pgfsetbuttcap%
\pgfsetroundjoin%
\definecolor{currentfill}{rgb}{0.283072,0.130895,0.449241}%
\pgfsetfillcolor{currentfill}%
\pgfsetlinewidth{0.000000pt}%
\definecolor{currentstroke}{rgb}{0.886271,0.892374,0.095374}%
\pgfsetstrokecolor{currentstroke}%
\pgfsetdash{}{0pt}%
\pgfpathmoveto{\pgfqpoint{6.150940in}{1.799026in}}%
\pgfpathlineto{\pgfqpoint{6.008600in}{1.974963in}}%
\pgfpathlineto{\pgfqpoint{6.073966in}{1.765621in}}%
\pgfpathclose%
\pgfusepath{fill}%
\end{pgfscope}%
\begin{pgfscope}%
\pgfpathrectangle{\pgfqpoint{0.539299in}{0.078740in}}{\pgfqpoint{7.842520in}{7.842520in}}%
\pgfusepath{clip}%
\pgfsetbuttcap%
\pgfsetroundjoin%
\definecolor{currentfill}{rgb}{0.135066,0.544853,0.554029}%
\pgfsetfillcolor{currentfill}%
\pgfsetlinewidth{0.000000pt}%
\definecolor{currentstroke}{rgb}{0.896320,0.893616,0.096335}%
\pgfsetstrokecolor{currentstroke}%
\pgfsetdash{}{0pt}%
\pgfpathmoveto{\pgfqpoint{5.092313in}{3.347287in}}%
\pgfpathlineto{\pgfqpoint{4.950134in}{3.565760in}}%
\pgfpathlineto{\pgfqpoint{4.869240in}{3.622273in}}%
\pgfpathclose%
\pgfusepath{fill}%
\end{pgfscope}%
\begin{pgfscope}%
\pgfpathrectangle{\pgfqpoint{0.539299in}{0.078740in}}{\pgfqpoint{7.842520in}{7.842520in}}%
\pgfusepath{clip}%
\pgfsetbuttcap%
\pgfsetroundjoin%
\definecolor{currentfill}{rgb}{0.171176,0.452530,0.557965}%
\pgfsetfillcolor{currentfill}%
\pgfsetlinewidth{0.000000pt}%
\definecolor{currentstroke}{rgb}{0.906311,0.894855,0.098125}%
\pgfsetstrokecolor{currentstroke}%
\pgfsetdash{}{0pt}%
\pgfpathmoveto{\pgfqpoint{5.234327in}{3.134099in}}%
\pgfpathlineto{\pgfqpoint{5.154816in}{3.180272in}}%
\pgfpathlineto{\pgfqpoint{5.297333in}{2.965317in}}%
\pgfpathclose%
\pgfusepath{fill}%
\end{pgfscope}%
\begin{pgfscope}%
\pgfpathrectangle{\pgfqpoint{0.539299in}{0.078740in}}{\pgfqpoint{7.842520in}{7.842520in}}%
\pgfusepath{clip}%
\pgfsetbuttcap%
\pgfsetroundjoin%
\definecolor{currentfill}{rgb}{0.496615,0.826376,0.306377}%
\pgfsetfillcolor{currentfill}%
\pgfsetlinewidth{0.000000pt}%
\definecolor{currentstroke}{rgb}{0.916242,0.896091,0.100717}%
\pgfsetstrokecolor{currentstroke}%
\pgfsetdash{}{0pt}%
\pgfpathmoveto{\pgfqpoint{4.034345in}{5.020034in}}%
\pgfpathlineto{\pgfqpoint{4.092594in}{4.886710in}}%
\pgfpathlineto{\pgfqpoint{4.177132in}{4.819968in}}%
\pgfpathclose%
\pgfusepath{fill}%
\end{pgfscope}%
\begin{pgfscope}%
\pgfpathrectangle{\pgfqpoint{0.539299in}{0.078740in}}{\pgfqpoint{7.842520in}{7.842520in}}%
\pgfusepath{clip}%
\pgfsetbuttcap%
\pgfsetroundjoin%
\definecolor{currentfill}{rgb}{0.277134,0.185228,0.489898}%
\pgfsetfillcolor{currentfill}%
\pgfsetlinewidth{0.000000pt}%
\definecolor{currentstroke}{rgb}{0.926106,0.897330,0.104071}%
\pgfsetstrokecolor{currentstroke}%
\pgfsetdash{}{0pt}%
\pgfpathmoveto{\pgfqpoint{5.931443in}{1.959659in}}%
\pgfpathlineto{\pgfqpoint{6.008600in}{1.974963in}}%
\pgfpathlineto{\pgfqpoint{5.866377in}{2.159848in}}%
\pgfpathclose%
\pgfusepath{fill}%
\end{pgfscope}%
\begin{pgfscope}%
\pgfpathrectangle{\pgfqpoint{0.539299in}{0.078740in}}{\pgfqpoint{7.842520in}{7.842520in}}%
\pgfusepath{clip}%
\pgfsetbuttcap%
\pgfsetroundjoin%
\definecolor{currentfill}{rgb}{0.377779,0.791781,0.377939}%
\pgfsetfillcolor{currentfill}%
\pgfsetlinewidth{0.000000pt}%
\definecolor{currentstroke}{rgb}{0.935904,0.898570,0.108131}%
\pgfsetstrokecolor{currentstroke}%
\pgfsetdash{}{0pt}%
\pgfpathmoveto{\pgfqpoint{2.261900in}{4.632062in}}%
\pgfpathlineto{\pgfqpoint{2.309553in}{4.592125in}}%
\pgfpathlineto{\pgfqpoint{2.393562in}{4.830606in}}%
\pgfpathclose%
\pgfusepath{fill}%
\end{pgfscope}%
\begin{pgfscope}%
\pgfpathrectangle{\pgfqpoint{0.539299in}{0.078740in}}{\pgfqpoint{7.842520in}{7.842520in}}%
\pgfusepath{clip}%
\pgfsetbuttcap%
\pgfsetroundjoin%
\definecolor{currentfill}{rgb}{0.134692,0.658636,0.517649}%
\pgfsetfillcolor{currentfill}%
\pgfsetlinewidth{0.000000pt}%
\definecolor{currentstroke}{rgb}{0.945636,0.899815,0.112838}%
\pgfsetstrokecolor{currentstroke}%
\pgfsetdash{}{0pt}%
\pgfpathmoveto{\pgfqpoint{2.013886in}{3.822753in}}%
\pgfpathlineto{\pgfqpoint{2.146689in}{3.948369in}}%
\pgfpathlineto{\pgfqpoint{2.227125in}{4.300185in}}%
\pgfpathclose%
\pgfusepath{fill}%
\end{pgfscope}%
\begin{pgfscope}%
\pgfpathrectangle{\pgfqpoint{0.539299in}{0.078740in}}{\pgfqpoint{7.842520in}{7.842520in}}%
\pgfusepath{clip}%
\pgfsetbuttcap%
\pgfsetroundjoin%
\definecolor{currentfill}{rgb}{0.935904,0.898570,0.108131}%
\pgfsetfillcolor{currentfill}%
\pgfsetlinewidth{0.000000pt}%
\definecolor{currentstroke}{rgb}{0.955300,0.901065,0.118128}%
\pgfsetstrokecolor{currentstroke}%
\pgfsetdash{}{0pt}%
\pgfpathmoveto{\pgfqpoint{3.237559in}{5.639984in}}%
\pgfpathlineto{\pgfqpoint{3.098325in}{5.625023in}}%
\pgfpathlineto{\pgfqpoint{3.150177in}{5.616991in}}%
\pgfpathclose%
\pgfusepath{fill}%
\end{pgfscope}%
\begin{pgfscope}%
\pgfpathrectangle{\pgfqpoint{0.539299in}{0.078740in}}{\pgfqpoint{7.842520in}{7.842520in}}%
\pgfusepath{clip}%
\pgfsetbuttcap%
\pgfsetroundjoin%
\definecolor{currentfill}{rgb}{0.814576,0.883393,0.110347}%
\pgfsetfillcolor{currentfill}%
\pgfsetlinewidth{0.000000pt}%
\definecolor{currentstroke}{rgb}{0.964894,0.902323,0.123941}%
\pgfsetstrokecolor{currentstroke}%
\pgfsetdash{}{0pt}%
\pgfpathmoveto{\pgfqpoint{3.749006in}{5.361964in}}%
\pgfpathlineto{\pgfqpoint{3.606833in}{5.491393in}}%
\pgfpathlineto{\pgfqpoint{3.662699in}{5.404278in}}%
\pgfpathclose%
\pgfusepath{fill}%
\end{pgfscope}%
\begin{pgfscope}%
\pgfpathrectangle{\pgfqpoint{0.539299in}{0.078740in}}{\pgfqpoint{7.842520in}{7.842520in}}%
\pgfusepath{clip}%
\pgfsetbuttcap%
\pgfsetroundjoin%
\definecolor{currentfill}{rgb}{0.143343,0.522773,0.556295}%
\pgfsetfillcolor{currentfill}%
\pgfsetlinewidth{0.000000pt}%
\definecolor{currentstroke}{rgb}{0.974417,0.903590,0.130215}%
\pgfsetstrokecolor{currentstroke}%
\pgfsetdash{}{0pt}%
\pgfpathmoveto{\pgfqpoint{2.068708in}{3.529775in}}%
\pgfpathlineto{\pgfqpoint{2.129568in}{3.063212in}}%
\pgfpathlineto{\pgfqpoint{2.204165in}{3.588171in}}%
\pgfpathclose%
\pgfusepath{fill}%
\end{pgfscope}%
\begin{pgfscope}%
\pgfpathrectangle{\pgfqpoint{0.539299in}{0.078740in}}{\pgfqpoint{7.842520in}{7.842520in}}%
\pgfusepath{clip}%
\pgfsetbuttcap%
\pgfsetroundjoin%
\definecolor{currentfill}{rgb}{0.751884,0.874951,0.143228}%
\pgfsetfillcolor{currentfill}%
\pgfsetlinewidth{0.000000pt}%
\definecolor{currentstroke}{rgb}{0.983868,0.904867,0.136897}%
\pgfsetstrokecolor{currentstroke}%
\pgfsetdash{}{0pt}%
\pgfpathmoveto{\pgfqpoint{3.662699in}{5.404278in}}%
\pgfpathlineto{\pgfqpoint{3.891582in}{5.202680in}}%
\pgfpathlineto{\pgfqpoint{3.749006in}{5.361964in}}%
\pgfpathclose%
\pgfusepath{fill}%
\end{pgfscope}%
\begin{pgfscope}%
\pgfpathrectangle{\pgfqpoint{0.539299in}{0.078740in}}{\pgfqpoint{7.842520in}{7.842520in}}%
\pgfusepath{clip}%
\pgfsetbuttcap%
\pgfsetroundjoin%
\definecolor{currentfill}{rgb}{0.657642,0.860219,0.203082}%
\pgfsetfillcolor{currentfill}%
\pgfsetlinewidth{0.000000pt}%
\definecolor{currentstroke}{rgb}{0.993248,0.906157,0.143936}%
\pgfsetstrokecolor{currentstroke}%
\pgfsetdash{}{0pt}%
\pgfpathmoveto{\pgfqpoint{4.034345in}{5.020034in}}%
\pgfpathlineto{\pgfqpoint{3.891582in}{5.202680in}}%
\pgfpathlineto{\pgfqpoint{3.805790in}{5.254759in}}%
\pgfpathclose%
\pgfusepath{fill}%
\end{pgfscope}%
\begin{pgfscope}%
\pgfpathrectangle{\pgfqpoint{0.539299in}{0.078740in}}{\pgfqpoint{7.842520in}{7.842520in}}%
\pgfusepath{clip}%
\pgfsetbuttcap%
\pgfsetroundjoin%
\definecolor{currentfill}{rgb}{0.262138,0.242286,0.520837}%
\pgfsetfillcolor{currentfill}%
\pgfsetlinewidth{0.000000pt}%
\definecolor{currentstroke}{rgb}{0.267004,0.004874,0.329415}%
\pgfsetstrokecolor{currentstroke}%
\pgfsetdash{}{0pt}%
\pgfpathmoveto{\pgfqpoint{5.788906in}{2.159810in}}%
\pgfpathlineto{\pgfqpoint{5.866377in}{2.159848in}}%
\pgfpathlineto{\pgfqpoint{5.724190in}{2.352097in}}%
\pgfpathclose%
\pgfusepath{fill}%
\end{pgfscope}%
\begin{pgfscope}%
\pgfpathrectangle{\pgfqpoint{0.539299in}{0.078740in}}{\pgfqpoint{7.842520in}{7.842520in}}%
\pgfusepath{clip}%
\pgfsetbuttcap%
\pgfsetroundjoin%
\definecolor{currentfill}{rgb}{0.657642,0.860219,0.203082}%
\pgfsetfillcolor{currentfill}%
\pgfsetlinewidth{0.000000pt}%
\definecolor{currentstroke}{rgb}{0.268510,0.009605,0.335427}%
\pgfsetstrokecolor{currentstroke}%
\pgfsetdash{}{0pt}%
\pgfpathmoveto{\pgfqpoint{2.564916in}{5.170485in}}%
\pgfpathlineto{\pgfqpoint{2.478789in}{5.021565in}}%
\pgfpathlineto{\pgfqpoint{2.699533in}{5.323077in}}%
\pgfpathclose%
\pgfusepath{fill}%
\end{pgfscope}%
\begin{pgfscope}%
\pgfpathrectangle{\pgfqpoint{0.539299in}{0.078740in}}{\pgfqpoint{7.842520in}{7.842520in}}%
\pgfusepath{clip}%
\pgfsetbuttcap%
\pgfsetroundjoin%
\definecolor{currentfill}{rgb}{0.935904,0.898570,0.108131}%
\pgfsetfillcolor{currentfill}%
\pgfsetlinewidth{0.000000pt}%
\definecolor{currentstroke}{rgb}{0.269944,0.014625,0.341379}%
\pgfsetstrokecolor{currentstroke}%
\pgfsetdash{}{0pt}%
\pgfpathmoveto{\pgfqpoint{3.150177in}{5.616991in}}%
\pgfpathlineto{\pgfqpoint{3.098325in}{5.625023in}}%
\pgfpathlineto{\pgfqpoint{3.010887in}{5.591305in}}%
\pgfpathclose%
\pgfusepath{fill}%
\end{pgfscope}%
\begin{pgfscope}%
\pgfpathrectangle{\pgfqpoint{0.539299in}{0.078740in}}{\pgfqpoint{7.842520in}{7.842520in}}%
\pgfusepath{clip}%
\pgfsetbuttcap%
\pgfsetroundjoin%
\definecolor{currentfill}{rgb}{0.266941,0.748751,0.440573}%
\pgfsetfillcolor{currentfill}%
\pgfsetlinewidth{0.000000pt}%
\definecolor{currentstroke}{rgb}{0.271305,0.019942,0.347269}%
\pgfsetstrokecolor{currentstroke}%
\pgfsetdash{}{0pt}%
\pgfpathmoveto{\pgfqpoint{2.227125in}{4.300185in}}%
\pgfpathlineto{\pgfqpoint{2.309553in}{4.592125in}}%
\pgfpathlineto{\pgfqpoint{2.177668in}{4.411616in}}%
\pgfpathclose%
\pgfusepath{fill}%
\end{pgfscope}%
\begin{pgfscope}%
\pgfpathrectangle{\pgfqpoint{0.539299in}{0.078740in}}{\pgfqpoint{7.842520in}{7.842520in}}%
\pgfusepath{clip}%
\pgfsetbuttcap%
\pgfsetroundjoin%
\definecolor{currentfill}{rgb}{0.327796,0.773980,0.406640}%
\pgfsetfillcolor{currentfill}%
\pgfsetlinewidth{0.000000pt}%
\definecolor{currentstroke}{rgb}{0.272594,0.025563,0.353093}%
\pgfsetstrokecolor{currentstroke}%
\pgfsetdash{}{0pt}%
\pgfpathmoveto{\pgfqpoint{4.235995in}{4.678979in}}%
\pgfpathlineto{\pgfqpoint{4.379257in}{4.461675in}}%
\pgfpathlineto{\pgfqpoint{4.319824in}{4.607800in}}%
\pgfpathclose%
\pgfusepath{fill}%
\end{pgfscope}%
\begin{pgfscope}%
\pgfpathrectangle{\pgfqpoint{0.539299in}{0.078740in}}{\pgfqpoint{7.842520in}{7.842520in}}%
\pgfusepath{clip}%
\pgfsetbuttcap%
\pgfsetroundjoin%
\definecolor{currentfill}{rgb}{0.120092,0.600104,0.542530}%
\pgfsetfillcolor{currentfill}%
\pgfsetlinewidth{0.000000pt}%
\definecolor{currentstroke}{rgb}{0.273809,0.031497,0.358853}%
\pgfsetstrokecolor{currentstroke}%
\pgfsetdash{}{0pt}%
\pgfpathmoveto{\pgfqpoint{4.950134in}{3.565760in}}%
\pgfpathlineto{\pgfqpoint{4.807756in}{3.788327in}}%
\pgfpathlineto{\pgfqpoint{4.726130in}{3.847156in}}%
\pgfpathclose%
\pgfusepath{fill}%
\end{pgfscope}%
\begin{pgfscope}%
\pgfpathrectangle{\pgfqpoint{0.539299in}{0.078740in}}{\pgfqpoint{7.842520in}{7.842520in}}%
\pgfusepath{clip}%
\pgfsetbuttcap%
\pgfsetroundjoin%
\definecolor{currentfill}{rgb}{0.192357,0.403199,0.555836}%
\pgfsetfillcolor{currentfill}%
\pgfsetlinewidth{0.000000pt}%
\definecolor{currentstroke}{rgb}{0.274952,0.037752,0.364543}%
\pgfsetstrokecolor{currentstroke}%
\pgfsetdash{}{0pt}%
\pgfpathmoveto{\pgfqpoint{2.266737in}{3.071192in}}%
\pgfpathlineto{\pgfqpoint{2.195719in}{2.435884in}}%
\pgfpathlineto{\pgfqpoint{2.405032in}{3.063501in}}%
\pgfpathclose%
\pgfusepath{fill}%
\end{pgfscope}%
\begin{pgfscope}%
\pgfpathrectangle{\pgfqpoint{0.539299in}{0.078740in}}{\pgfqpoint{7.842520in}{7.842520in}}%
\pgfusepath{clip}%
\pgfsetbuttcap%
\pgfsetroundjoin%
\definecolor{currentfill}{rgb}{0.150476,0.504369,0.557430}%
\pgfsetfillcolor{currentfill}%
\pgfsetlinewidth{0.000000pt}%
\definecolor{currentstroke}{rgb}{0.276022,0.044167,0.370164}%
\pgfsetstrokecolor{currentstroke}%
\pgfsetdash{}{0pt}%
\pgfpathmoveto{\pgfqpoint{5.092313in}{3.347287in}}%
\pgfpathlineto{\pgfqpoint{5.012128in}{3.399559in}}%
\pgfpathlineto{\pgfqpoint{5.154816in}{3.180272in}}%
\pgfpathclose%
\pgfusepath{fill}%
\end{pgfscope}%
\begin{pgfscope}%
\pgfpathrectangle{\pgfqpoint{0.539299in}{0.078740in}}{\pgfqpoint{7.842520in}{7.842520in}}%
\pgfusepath{clip}%
\pgfsetbuttcap%
\pgfsetroundjoin%
\definecolor{currentfill}{rgb}{0.280894,0.078907,0.402329}%
\pgfsetfillcolor{currentfill}%
\pgfsetlinewidth{0.000000pt}%
\definecolor{currentstroke}{rgb}{0.277018,0.050344,0.375715}%
\pgfsetstrokecolor{currentstroke}%
\pgfsetdash{}{0pt}%
\pgfpathmoveto{\pgfqpoint{6.216568in}{1.580071in}}%
\pgfpathlineto{\pgfqpoint{6.293516in}{1.634687in}}%
\pgfpathlineto{\pgfqpoint{6.073966in}{1.765621in}}%
\pgfpathclose%
\pgfusepath{fill}%
\end{pgfscope}%
\begin{pgfscope}%
\pgfpathrectangle{\pgfqpoint{0.539299in}{0.078740in}}{\pgfqpoint{7.842520in}{7.842520in}}%
\pgfusepath{clip}%
\pgfsetbuttcap%
\pgfsetroundjoin%
\definecolor{currentfill}{rgb}{0.896320,0.893616,0.096335}%
\pgfsetfillcolor{currentfill}%
\pgfsetlinewidth{0.000000pt}%
\definecolor{currentstroke}{rgb}{0.277941,0.056324,0.381191}%
\pgfsetstrokecolor{currentstroke}%
\pgfsetdash{}{0pt}%
\pgfpathmoveto{\pgfqpoint{3.520103in}{5.522735in}}%
\pgfpathlineto{\pgfqpoint{3.606833in}{5.491393in}}%
\pgfpathlineto{\pgfqpoint{3.378279in}{5.603577in}}%
\pgfpathclose%
\pgfusepath{fill}%
\end{pgfscope}%
\begin{pgfscope}%
\pgfpathrectangle{\pgfqpoint{0.539299in}{0.078740in}}{\pgfqpoint{7.842520in}{7.842520in}}%
\pgfusepath{clip}%
\pgfsetbuttcap%
\pgfsetroundjoin%
\definecolor{currentfill}{rgb}{0.239346,0.300855,0.540844}%
\pgfsetfillcolor{currentfill}%
\pgfsetlinewidth{0.000000pt}%
\definecolor{currentstroke}{rgb}{0.278791,0.062145,0.386592}%
\pgfsetstrokecolor{currentstroke}%
\pgfsetdash{}{0pt}%
\pgfpathmoveto{\pgfqpoint{5.646299in}{2.364748in}}%
\pgfpathlineto{\pgfqpoint{5.724190in}{2.352097in}}%
\pgfpathlineto{\pgfqpoint{5.581983in}{2.550780in}}%
\pgfpathclose%
\pgfusepath{fill}%
\end{pgfscope}%
\begin{pgfscope}%
\pgfpathrectangle{\pgfqpoint{0.539299in}{0.078740in}}{\pgfqpoint{7.842520in}{7.842520in}}%
\pgfusepath{clip}%
\pgfsetbuttcap%
\pgfsetroundjoin%
\definecolor{currentfill}{rgb}{0.218130,0.347432,0.550038}%
\pgfsetfillcolor{currentfill}%
\pgfsetlinewidth{0.000000pt}%
\definecolor{currentstroke}{rgb}{0.279566,0.067836,0.391917}%
\pgfsetstrokecolor{currentstroke}%
\pgfsetdash{}{0pt}%
\pgfpathmoveto{\pgfqpoint{2.405032in}{3.063501in}}%
\pgfpathlineto{\pgfqpoint{2.333716in}{2.409916in}}%
\pgfpathlineto{\pgfqpoint{2.472446in}{2.377069in}}%
\pgfpathclose%
\pgfusepath{fill}%
\end{pgfscope}%
\begin{pgfscope}%
\pgfpathrectangle{\pgfqpoint{0.539299in}{0.078740in}}{\pgfqpoint{7.842520in}{7.842520in}}%
\pgfusepath{clip}%
\pgfsetbuttcap%
\pgfsetroundjoin%
\definecolor{currentfill}{rgb}{0.121148,0.592739,0.544641}%
\pgfsetfillcolor{currentfill}%
\pgfsetlinewidth{0.000000pt}%
\definecolor{currentstroke}{rgb}{0.280267,0.073417,0.397163}%
\pgfsetstrokecolor{currentstroke}%
\pgfsetdash{}{0pt}%
\pgfpathmoveto{\pgfqpoint{2.146689in}{3.948369in}}%
\pgfpathlineto{\pgfqpoint{2.068708in}{3.529775in}}%
\pgfpathlineto{\pgfqpoint{2.204165in}{3.588171in}}%
\pgfpathclose%
\pgfusepath{fill}%
\end{pgfscope}%
\begin{pgfscope}%
\pgfpathrectangle{\pgfqpoint{0.539299in}{0.078740in}}{\pgfqpoint{7.842520in}{7.842520in}}%
\pgfusepath{clip}%
\pgfsetbuttcap%
\pgfsetroundjoin%
\definecolor{currentfill}{rgb}{0.404001,0.800275,0.362552}%
\pgfsetfillcolor{currentfill}%
\pgfsetlinewidth{0.000000pt}%
\definecolor{currentstroke}{rgb}{0.280894,0.078907,0.402329}%
\pgfsetstrokecolor{currentstroke}%
\pgfsetdash{}{0pt}%
\pgfpathmoveto{\pgfqpoint{4.319824in}{4.607800in}}%
\pgfpathlineto{\pgfqpoint{4.092594in}{4.886710in}}%
\pgfpathlineto{\pgfqpoint{4.235995in}{4.678979in}}%
\pgfpathclose%
\pgfusepath{fill}%
\end{pgfscope}%
\begin{pgfscope}%
\pgfpathrectangle{\pgfqpoint{0.539299in}{0.078740in}}{\pgfqpoint{7.842520in}{7.842520in}}%
\pgfusepath{clip}%
\pgfsetbuttcap%
\pgfsetroundjoin%
\definecolor{currentfill}{rgb}{0.281412,0.155834,0.469201}%
\pgfsetfillcolor{currentfill}%
\pgfsetlinewidth{0.000000pt}%
\definecolor{currentstroke}{rgb}{0.281446,0.084320,0.407414}%
\pgfsetstrokecolor{currentstroke}%
\pgfsetdash{}{0pt}%
\pgfpathmoveto{\pgfqpoint{6.008600in}{1.974963in}}%
\pgfpathlineto{\pgfqpoint{5.931443in}{1.959659in}}%
\pgfpathlineto{\pgfqpoint{6.073966in}{1.765621in}}%
\pgfpathclose%
\pgfusepath{fill}%
\end{pgfscope}%
\begin{pgfscope}%
\pgfpathrectangle{\pgfqpoint{0.539299in}{0.078740in}}{\pgfqpoint{7.842520in}{7.842520in}}%
\pgfusepath{clip}%
\pgfsetbuttcap%
\pgfsetroundjoin%
\definecolor{currentfill}{rgb}{0.136408,0.541173,0.554483}%
\pgfsetfillcolor{currentfill}%
\pgfsetlinewidth{0.000000pt}%
\definecolor{currentstroke}{rgb}{0.281924,0.089666,0.412415}%
\pgfsetstrokecolor{currentstroke}%
\pgfsetdash{}{0pt}%
\pgfpathmoveto{\pgfqpoint{4.869240in}{3.622273in}}%
\pgfpathlineto{\pgfqpoint{5.012128in}{3.399559in}}%
\pgfpathlineto{\pgfqpoint{5.092313in}{3.347287in}}%
\pgfpathclose%
\pgfusepath{fill}%
\end{pgfscope}%
\begin{pgfscope}%
\pgfpathrectangle{\pgfqpoint{0.539299in}{0.078740in}}{\pgfqpoint{7.842520in}{7.842520in}}%
\pgfusepath{clip}%
\pgfsetbuttcap%
\pgfsetroundjoin%
\definecolor{currentfill}{rgb}{0.122312,0.633153,0.530398}%
\pgfsetfillcolor{currentfill}%
\pgfsetlinewidth{0.000000pt}%
\definecolor{currentstroke}{rgb}{0.282327,0.094955,0.417331}%
\pgfsetstrokecolor{currentstroke}%
\pgfsetdash{}{0pt}%
\pgfpathmoveto{\pgfqpoint{4.726130in}{3.847156in}}%
\pgfpathlineto{\pgfqpoint{4.807756in}{3.788327in}}%
\pgfpathlineto{\pgfqpoint{4.665156in}{4.013355in}}%
\pgfpathclose%
\pgfusepath{fill}%
\end{pgfscope}%
\begin{pgfscope}%
\pgfpathrectangle{\pgfqpoint{0.539299in}{0.078740in}}{\pgfqpoint{7.842520in}{7.842520in}}%
\pgfusepath{clip}%
\pgfsetbuttcap%
\pgfsetroundjoin%
\definecolor{currentfill}{rgb}{0.270595,0.214069,0.507052}%
\pgfsetfillcolor{currentfill}%
\pgfsetlinewidth{0.000000pt}%
\definecolor{currentstroke}{rgb}{0.282656,0.100196,0.422160}%
\pgfsetstrokecolor{currentstroke}%
\pgfsetdash{}{0pt}%
\pgfpathmoveto{\pgfqpoint{5.931443in}{1.959659in}}%
\pgfpathlineto{\pgfqpoint{5.866377in}{2.159848in}}%
\pgfpathlineto{\pgfqpoint{5.788906in}{2.159810in}}%
\pgfpathclose%
\pgfusepath{fill}%
\end{pgfscope}%
\begin{pgfscope}%
\pgfpathrectangle{\pgfqpoint{0.539299in}{0.078740in}}{\pgfqpoint{7.842520in}{7.842520in}}%
\pgfusepath{clip}%
\pgfsetbuttcap%
\pgfsetroundjoin%
\definecolor{currentfill}{rgb}{0.866013,0.889868,0.095953}%
\pgfsetfillcolor{currentfill}%
\pgfsetlinewidth{0.000000pt}%
\definecolor{currentstroke}{rgb}{0.282910,0.105393,0.426902}%
\pgfsetstrokecolor{currentstroke}%
\pgfsetdash{}{0pt}%
\pgfpathmoveto{\pgfqpoint{3.662699in}{5.404278in}}%
\pgfpathlineto{\pgfqpoint{3.606833in}{5.491393in}}%
\pgfpathlineto{\pgfqpoint{3.520103in}{5.522735in}}%
\pgfpathclose%
\pgfusepath{fill}%
\end{pgfscope}%
\begin{pgfscope}%
\pgfpathrectangle{\pgfqpoint{0.539299in}{0.078740in}}{\pgfqpoint{7.842520in}{7.842520in}}%
\pgfusepath{clip}%
\pgfsetbuttcap%
\pgfsetroundjoin%
\definecolor{currentfill}{rgb}{0.955300,0.901065,0.118128}%
\pgfsetfillcolor{currentfill}%
\pgfsetlinewidth{0.000000pt}%
\definecolor{currentstroke}{rgb}{0.283091,0.110553,0.431554}%
\pgfsetstrokecolor{currentstroke}%
\pgfsetdash{}{0pt}%
\pgfpathmoveto{\pgfqpoint{3.237559in}{5.639984in}}%
\pgfpathlineto{\pgfqpoint{3.150177in}{5.616991in}}%
\pgfpathlineto{\pgfqpoint{3.378279in}{5.603577in}}%
\pgfpathclose%
\pgfusepath{fill}%
\end{pgfscope}%
\begin{pgfscope}%
\pgfpathrectangle{\pgfqpoint{0.539299in}{0.078740in}}{\pgfqpoint{7.842520in}{7.842520in}}%
\pgfusepath{clip}%
\pgfsetbuttcap%
\pgfsetroundjoin%
\definecolor{currentfill}{rgb}{0.896320,0.893616,0.096335}%
\pgfsetfillcolor{currentfill}%
\pgfsetlinewidth{0.000000pt}%
\definecolor{currentstroke}{rgb}{0.283197,0.115680,0.436115}%
\pgfsetstrokecolor{currentstroke}%
\pgfsetdash{}{0pt}%
\pgfpathmoveto{\pgfqpoint{2.786367in}{5.433811in}}%
\pgfpathlineto{\pgfqpoint{2.923598in}{5.523700in}}%
\pgfpathlineto{\pgfqpoint{3.010887in}{5.591305in}}%
\pgfpathclose%
\pgfusepath{fill}%
\end{pgfscope}%
\begin{pgfscope}%
\pgfpathrectangle{\pgfqpoint{0.539299in}{0.078740in}}{\pgfqpoint{7.842520in}{7.842520in}}%
\pgfusepath{clip}%
\pgfsetbuttcap%
\pgfsetroundjoin%
\definecolor{currentfill}{rgb}{0.206756,0.371758,0.553117}%
\pgfsetfillcolor{currentfill}%
\pgfsetlinewidth{0.000000pt}%
\definecolor{currentstroke}{rgb}{0.283229,0.120777,0.440584}%
\pgfsetstrokecolor{currentstroke}%
\pgfsetdash{}{0pt}%
\pgfpathmoveto{\pgfqpoint{5.439710in}{2.755331in}}%
\pgfpathlineto{\pgfqpoint{5.360735in}{2.786461in}}%
\pgfpathlineto{\pgfqpoint{5.581983in}{2.550780in}}%
\pgfpathclose%
\pgfusepath{fill}%
\end{pgfscope}%
\begin{pgfscope}%
\pgfpathrectangle{\pgfqpoint{0.539299in}{0.078740in}}{\pgfqpoint{7.842520in}{7.842520in}}%
\pgfusepath{clip}%
\pgfsetbuttcap%
\pgfsetroundjoin%
\definecolor{currentfill}{rgb}{0.575563,0.844566,0.256415}%
\pgfsetfillcolor{currentfill}%
\pgfsetlinewidth{0.000000pt}%
\definecolor{currentstroke}{rgb}{0.283187,0.125848,0.444960}%
\pgfsetstrokecolor{currentstroke}%
\pgfsetdash{}{0pt}%
\pgfpathmoveto{\pgfqpoint{3.949148in}{5.080344in}}%
\pgfpathlineto{\pgfqpoint{4.092594in}{4.886710in}}%
\pgfpathlineto{\pgfqpoint{4.034345in}{5.020034in}}%
\pgfpathclose%
\pgfusepath{fill}%
\end{pgfscope}%
\begin{pgfscope}%
\pgfpathrectangle{\pgfqpoint{0.539299in}{0.078740in}}{\pgfqpoint{7.842520in}{7.842520in}}%
\pgfusepath{clip}%
\pgfsetbuttcap%
\pgfsetroundjoin%
\definecolor{currentfill}{rgb}{0.647257,0.858400,0.209861}%
\pgfsetfillcolor{currentfill}%
\pgfsetlinewidth{0.000000pt}%
\definecolor{currentstroke}{rgb}{0.283072,0.130895,0.449241}%
\pgfsetstrokecolor{currentstroke}%
\pgfsetdash{}{0pt}%
\pgfpathmoveto{\pgfqpoint{4.034345in}{5.020034in}}%
\pgfpathlineto{\pgfqpoint{3.805790in}{5.254759in}}%
\pgfpathlineto{\pgfqpoint{3.949148in}{5.080344in}}%
\pgfpathclose%
\pgfusepath{fill}%
\end{pgfscope}%
\begin{pgfscope}%
\pgfpathrectangle{\pgfqpoint{0.539299in}{0.078740in}}{\pgfqpoint{7.842520in}{7.842520in}}%
\pgfusepath{clip}%
\pgfsetbuttcap%
\pgfsetroundjoin%
\definecolor{currentfill}{rgb}{0.121148,0.592739,0.544641}%
\pgfsetfillcolor{currentfill}%
\pgfsetlinewidth{0.000000pt}%
\definecolor{currentstroke}{rgb}{0.282884,0.135920,0.453427}%
\pgfsetstrokecolor{currentstroke}%
\pgfsetdash{}{0pt}%
\pgfpathmoveto{\pgfqpoint{4.726130in}{3.847156in}}%
\pgfpathlineto{\pgfqpoint{4.869240in}{3.622273in}}%
\pgfpathlineto{\pgfqpoint{4.950134in}{3.565760in}}%
\pgfpathclose%
\pgfusepath{fill}%
\end{pgfscope}%
\begin{pgfscope}%
\pgfpathrectangle{\pgfqpoint{0.539299in}{0.078740in}}{\pgfqpoint{7.842520in}{7.842520in}}%
\pgfusepath{clip}%
\pgfsetbuttcap%
\pgfsetroundjoin%
\definecolor{currentfill}{rgb}{0.151918,0.500685,0.557587}%
\pgfsetfillcolor{currentfill}%
\pgfsetlinewidth{0.000000pt}%
\definecolor{currentstroke}{rgb}{0.282623,0.140926,0.457517}%
\pgfsetstrokecolor{currentstroke}%
\pgfsetdash{}{0pt}%
\pgfpathmoveto{\pgfqpoint{2.341276in}{3.620065in}}%
\pgfpathlineto{\pgfqpoint{2.129568in}{3.063212in}}%
\pgfpathlineto{\pgfqpoint{2.266737in}{3.071192in}}%
\pgfpathclose%
\pgfusepath{fill}%
\end{pgfscope}%
\begin{pgfscope}%
\pgfpathrectangle{\pgfqpoint{0.539299in}{0.078740in}}{\pgfqpoint{7.842520in}{7.842520in}}%
\pgfusepath{clip}%
\pgfsetbuttcap%
\pgfsetroundjoin%
\definecolor{currentfill}{rgb}{0.751884,0.874951,0.143228}%
\pgfsetfillcolor{currentfill}%
\pgfsetlinewidth{0.000000pt}%
\definecolor{currentstroke}{rgb}{0.282290,0.145912,0.461510}%
\pgfsetstrokecolor{currentstroke}%
\pgfsetdash{}{0pt}%
\pgfpathmoveto{\pgfqpoint{3.805790in}{5.254759in}}%
\pgfpathlineto{\pgfqpoint{3.891582in}{5.202680in}}%
\pgfpathlineto{\pgfqpoint{3.662699in}{5.404278in}}%
\pgfpathclose%
\pgfusepath{fill}%
\end{pgfscope}%
\begin{pgfscope}%
\pgfpathrectangle{\pgfqpoint{0.539299in}{0.078740in}}{\pgfqpoint{7.842520in}{7.842520in}}%
\pgfusepath{clip}%
\pgfsetbuttcap%
\pgfsetroundjoin%
\definecolor{currentfill}{rgb}{0.250425,0.274290,0.533103}%
\pgfsetfillcolor{currentfill}%
\pgfsetlinewidth{0.000000pt}%
\definecolor{currentstroke}{rgb}{0.281887,0.150881,0.465405}%
\pgfsetstrokecolor{currentstroke}%
\pgfsetdash{}{0pt}%
\pgfpathmoveto{\pgfqpoint{5.724190in}{2.352097in}}%
\pgfpathlineto{\pgfqpoint{5.646299in}{2.364748in}}%
\pgfpathlineto{\pgfqpoint{5.788906in}{2.159810in}}%
\pgfpathclose%
\pgfusepath{fill}%
\end{pgfscope}%
\begin{pgfscope}%
\pgfpathrectangle{\pgfqpoint{0.539299in}{0.078740in}}{\pgfqpoint{7.842520in}{7.842520in}}%
\pgfusepath{clip}%
\pgfsetbuttcap%
\pgfsetroundjoin%
\definecolor{currentfill}{rgb}{0.190631,0.407061,0.556089}%
\pgfsetfillcolor{currentfill}%
\pgfsetlinewidth{0.000000pt}%
\definecolor{currentstroke}{rgb}{0.281412,0.155834,0.469201}%
\pgfsetstrokecolor{currentstroke}%
\pgfsetdash{}{0pt}%
\pgfpathmoveto{\pgfqpoint{5.297333in}{2.965317in}}%
\pgfpathlineto{\pgfqpoint{5.360735in}{2.786461in}}%
\pgfpathlineto{\pgfqpoint{5.439710in}{2.755331in}}%
\pgfpathclose%
\pgfusepath{fill}%
\end{pgfscope}%
\begin{pgfscope}%
\pgfpathrectangle{\pgfqpoint{0.539299in}{0.078740in}}{\pgfqpoint{7.842520in}{7.842520in}}%
\pgfusepath{clip}%
\pgfsetbuttcap%
\pgfsetroundjoin%
\definecolor{currentfill}{rgb}{0.845561,0.887322,0.099702}%
\pgfsetfillcolor{currentfill}%
\pgfsetlinewidth{0.000000pt}%
\definecolor{currentstroke}{rgb}{0.280868,0.160771,0.472899}%
\pgfsetstrokecolor{currentstroke}%
\pgfsetdash{}{0pt}%
\pgfpathmoveto{\pgfqpoint{2.923598in}{5.523700in}}%
\pgfpathlineto{\pgfqpoint{2.786367in}{5.433811in}}%
\pgfpathlineto{\pgfqpoint{2.699533in}{5.323077in}}%
\pgfpathclose%
\pgfusepath{fill}%
\end{pgfscope}%
\begin{pgfscope}%
\pgfpathrectangle{\pgfqpoint{0.539299in}{0.078740in}}{\pgfqpoint{7.842520in}{7.842520in}}%
\pgfusepath{clip}%
\pgfsetbuttcap%
\pgfsetroundjoin%
\definecolor{currentfill}{rgb}{0.136408,0.541173,0.554483}%
\pgfsetfillcolor{currentfill}%
\pgfsetlinewidth{0.000000pt}%
\definecolor{currentstroke}{rgb}{0.280255,0.165693,0.476498}%
\pgfsetstrokecolor{currentstroke}%
\pgfsetdash{}{0pt}%
\pgfpathmoveto{\pgfqpoint{2.204165in}{3.588171in}}%
\pgfpathlineto{\pgfqpoint{2.129568in}{3.063212in}}%
\pgfpathlineto{\pgfqpoint{2.341276in}{3.620065in}}%
\pgfpathclose%
\pgfusepath{fill}%
\end{pgfscope}%
\begin{pgfscope}%
\pgfpathrectangle{\pgfqpoint{0.539299in}{0.078740in}}{\pgfqpoint{7.842520in}{7.842520in}}%
\pgfusepath{clip}%
\pgfsetbuttcap%
\pgfsetroundjoin%
\definecolor{currentfill}{rgb}{0.185783,0.704891,0.485273}%
\pgfsetfillcolor{currentfill}%
\pgfsetlinewidth{0.000000pt}%
\definecolor{currentstroke}{rgb}{0.279574,0.170599,0.479997}%
\pgfsetstrokecolor{currentstroke}%
\pgfsetdash{}{0pt}%
\pgfpathmoveto{\pgfqpoint{4.665156in}{4.013355in}}%
\pgfpathlineto{\pgfqpoint{4.522320in}{4.238703in}}%
\pgfpathlineto{\pgfqpoint{4.439213in}{4.296234in}}%
\pgfpathclose%
\pgfusepath{fill}%
\end{pgfscope}%
\begin{pgfscope}%
\pgfpathrectangle{\pgfqpoint{0.539299in}{0.078740in}}{\pgfqpoint{7.842520in}{7.842520in}}%
\pgfusepath{clip}%
\pgfsetbuttcap%
\pgfsetroundjoin%
\definecolor{currentfill}{rgb}{0.688944,0.865448,0.182725}%
\pgfsetfillcolor{currentfill}%
\pgfsetlinewidth{0.000000pt}%
\definecolor{currentstroke}{rgb}{0.278826,0.175490,0.483397}%
\pgfsetstrokecolor{currentstroke}%
\pgfsetdash{}{0pt}%
\pgfpathmoveto{\pgfqpoint{2.699533in}{5.323077in}}%
\pgfpathlineto{\pgfqpoint{2.478789in}{5.021565in}}%
\pgfpathlineto{\pgfqpoint{2.613359in}{5.171039in}}%
\pgfpathclose%
\pgfusepath{fill}%
\end{pgfscope}%
\begin{pgfscope}%
\pgfpathrectangle{\pgfqpoint{0.539299in}{0.078740in}}{\pgfqpoint{7.842520in}{7.842520in}}%
\pgfusepath{clip}%
\pgfsetbuttcap%
\pgfsetroundjoin%
\definecolor{currentfill}{rgb}{0.545524,0.838039,0.275626}%
\pgfsetfillcolor{currentfill}%
\pgfsetlinewidth{0.000000pt}%
\definecolor{currentstroke}{rgb}{0.278012,0.180367,0.486697}%
\pgfsetstrokecolor{currentstroke}%
\pgfsetdash{}{0pt}%
\pgfpathmoveto{\pgfqpoint{2.393562in}{4.830606in}}%
\pgfpathlineto{\pgfqpoint{2.528143in}{4.972174in}}%
\pgfpathlineto{\pgfqpoint{2.478789in}{5.021565in}}%
\pgfpathclose%
\pgfusepath{fill}%
\end{pgfscope}%
\begin{pgfscope}%
\pgfpathrectangle{\pgfqpoint{0.539299in}{0.078740in}}{\pgfqpoint{7.842520in}{7.842520in}}%
\pgfusepath{clip}%
\pgfsetbuttcap%
\pgfsetroundjoin%
\definecolor{currentfill}{rgb}{0.225863,0.330805,0.547314}%
\pgfsetfillcolor{currentfill}%
\pgfsetlinewidth{0.000000pt}%
\definecolor{currentstroke}{rgb}{0.277134,0.185228,0.489898}%
\pgfsetstrokecolor{currentstroke}%
\pgfsetdash{}{0pt}%
\pgfpathmoveto{\pgfqpoint{5.503585in}{2.573766in}}%
\pgfpathlineto{\pgfqpoint{5.646299in}{2.364748in}}%
\pgfpathlineto{\pgfqpoint{5.581983in}{2.550780in}}%
\pgfpathclose%
\pgfusepath{fill}%
\end{pgfscope}%
\begin{pgfscope}%
\pgfpathrectangle{\pgfqpoint{0.539299in}{0.078740in}}{\pgfqpoint{7.842520in}{7.842520in}}%
\pgfusepath{clip}%
\pgfsetbuttcap%
\pgfsetroundjoin%
\definecolor{currentfill}{rgb}{0.162016,0.687316,0.499129}%
\pgfsetfillcolor{currentfill}%
\pgfsetlinewidth{0.000000pt}%
\definecolor{currentstroke}{rgb}{0.276194,0.190074,0.493001}%
\pgfsetstrokecolor{currentstroke}%
\pgfsetdash{}{0pt}%
\pgfpathmoveto{\pgfqpoint{2.227125in}{4.300185in}}%
\pgfpathlineto{\pgfqpoint{2.146689in}{3.948369in}}%
\pgfpathlineto{\pgfqpoint{2.281795in}{4.035068in}}%
\pgfpathclose%
\pgfusepath{fill}%
\end{pgfscope}%
\begin{pgfscope}%
\pgfpathrectangle{\pgfqpoint{0.539299in}{0.078740in}}{\pgfqpoint{7.842520in}{7.842520in}}%
\pgfusepath{clip}%
\pgfsetbuttcap%
\pgfsetroundjoin%
\definecolor{currentfill}{rgb}{0.945636,0.899815,0.112838}%
\pgfsetfillcolor{currentfill}%
\pgfsetlinewidth{0.000000pt}%
\definecolor{currentstroke}{rgb}{0.275191,0.194905,0.496005}%
\pgfsetstrokecolor{currentstroke}%
\pgfsetdash{}{0pt}%
\pgfpathmoveto{\pgfqpoint{3.010887in}{5.591305in}}%
\pgfpathlineto{\pgfqpoint{2.923598in}{5.523700in}}%
\pgfpathlineto{\pgfqpoint{3.150177in}{5.616991in}}%
\pgfpathclose%
\pgfusepath{fill}%
\end{pgfscope}%
\begin{pgfscope}%
\pgfpathrectangle{\pgfqpoint{0.539299in}{0.078740in}}{\pgfqpoint{7.842520in}{7.842520in}}%
\pgfusepath{clip}%
\pgfsetbuttcap%
\pgfsetroundjoin%
\definecolor{currentfill}{rgb}{0.123444,0.636809,0.528763}%
\pgfsetfillcolor{currentfill}%
\pgfsetlinewidth{0.000000pt}%
\definecolor{currentstroke}{rgb}{0.274128,0.199721,0.498911}%
\pgfsetstrokecolor{currentstroke}%
\pgfsetdash{}{0pt}%
\pgfpathmoveto{\pgfqpoint{2.204165in}{3.588171in}}%
\pgfpathlineto{\pgfqpoint{2.281795in}{4.035068in}}%
\pgfpathlineto{\pgfqpoint{2.146689in}{3.948369in}}%
\pgfpathclose%
\pgfusepath{fill}%
\end{pgfscope}%
\begin{pgfscope}%
\pgfpathrectangle{\pgfqpoint{0.539299in}{0.078740in}}{\pgfqpoint{7.842520in}{7.842520in}}%
\pgfusepath{clip}%
\pgfsetbuttcap%
\pgfsetroundjoin%
\definecolor{currentfill}{rgb}{0.246070,0.738910,0.452024}%
\pgfsetfillcolor{currentfill}%
\pgfsetlinewidth{0.000000pt}%
\definecolor{currentstroke}{rgb}{0.273006,0.204520,0.501721}%
\pgfsetstrokecolor{currentstroke}%
\pgfsetdash{}{0pt}%
\pgfpathmoveto{\pgfqpoint{4.522320in}{4.238703in}}%
\pgfpathlineto{\pgfqpoint{4.379257in}{4.461675in}}%
\pgfpathlineto{\pgfqpoint{4.439213in}{4.296234in}}%
\pgfpathclose%
\pgfusepath{fill}%
\end{pgfscope}%
\begin{pgfscope}%
\pgfpathrectangle{\pgfqpoint{0.539299in}{0.078740in}}{\pgfqpoint{7.842520in}{7.842520in}}%
\pgfusepath{clip}%
\pgfsetbuttcap%
\pgfsetroundjoin%
\definecolor{currentfill}{rgb}{0.208623,0.367752,0.552675}%
\pgfsetfillcolor{currentfill}%
\pgfsetlinewidth{0.000000pt}%
\definecolor{currentstroke}{rgb}{0.271828,0.209303,0.504434}%
\pgfsetstrokecolor{currentstroke}%
\pgfsetdash{}{0pt}%
\pgfpathmoveto{\pgfqpoint{5.581983in}{2.550780in}}%
\pgfpathlineto{\pgfqpoint{5.360735in}{2.786461in}}%
\pgfpathlineto{\pgfqpoint{5.503585in}{2.573766in}}%
\pgfpathclose%
\pgfusepath{fill}%
\end{pgfscope}%
\begin{pgfscope}%
\pgfpathrectangle{\pgfqpoint{0.539299in}{0.078740in}}{\pgfqpoint{7.842520in}{7.842520in}}%
\pgfusepath{clip}%
\pgfsetbuttcap%
\pgfsetroundjoin%
\definecolor{currentfill}{rgb}{0.160665,0.478540,0.558115}%
\pgfsetfillcolor{currentfill}%
\pgfsetlinewidth{0.000000pt}%
\definecolor{currentstroke}{rgb}{0.270595,0.214069,0.507052}%
\pgfsetstrokecolor{currentstroke}%
\pgfsetdash{}{0pt}%
\pgfpathmoveto{\pgfqpoint{5.154816in}{3.180272in}}%
\pgfpathlineto{\pgfqpoint{5.074536in}{3.221540in}}%
\pgfpathlineto{\pgfqpoint{5.297333in}{2.965317in}}%
\pgfpathclose%
\pgfusepath{fill}%
\end{pgfscope}%
\begin{pgfscope}%
\pgfpathrectangle{\pgfqpoint{0.539299in}{0.078740in}}{\pgfqpoint{7.842520in}{7.842520in}}%
\pgfusepath{clip}%
\pgfsetbuttcap%
\pgfsetroundjoin%
\definecolor{currentfill}{rgb}{0.281924,0.089666,0.412415}%
\pgfsetfillcolor{currentfill}%
\pgfsetlinewidth{0.000000pt}%
\definecolor{currentstroke}{rgb}{0.269308,0.218818,0.509577}%
\pgfsetstrokecolor{currentstroke}%
\pgfsetdash{}{0pt}%
\pgfpathmoveto{\pgfqpoint{6.073966in}{1.765621in}}%
\pgfpathlineto{\pgfqpoint{6.139221in}{1.529769in}}%
\pgfpathlineto{\pgfqpoint{6.216568in}{1.580071in}}%
\pgfpathclose%
\pgfusepath{fill}%
\end{pgfscope}%
\begin{pgfscope}%
\pgfpathrectangle{\pgfqpoint{0.539299in}{0.078740in}}{\pgfqpoint{7.842520in}{7.842520in}}%
\pgfusepath{clip}%
\pgfsetbuttcap%
\pgfsetroundjoin%
\definecolor{currentfill}{rgb}{0.140210,0.665859,0.513427}%
\pgfsetfillcolor{currentfill}%
\pgfsetlinewidth{0.000000pt}%
\definecolor{currentstroke}{rgb}{0.267968,0.223549,0.512008}%
\pgfsetstrokecolor{currentstroke}%
\pgfsetdash{}{0pt}%
\pgfpathmoveto{\pgfqpoint{4.665156in}{4.013355in}}%
\pgfpathlineto{\pgfqpoint{4.582787in}{4.072527in}}%
\pgfpathlineto{\pgfqpoint{4.726130in}{3.847156in}}%
\pgfpathclose%
\pgfusepath{fill}%
\end{pgfscope}%
\begin{pgfscope}%
\pgfpathrectangle{\pgfqpoint{0.539299in}{0.078740in}}{\pgfqpoint{7.842520in}{7.842520in}}%
\pgfusepath{clip}%
\pgfsetbuttcap%
\pgfsetroundjoin%
\definecolor{currentfill}{rgb}{0.430983,0.808473,0.346476}%
\pgfsetfillcolor{currentfill}%
\pgfsetlinewidth{0.000000pt}%
\definecolor{currentstroke}{rgb}{0.266580,0.228262,0.514349}%
\pgfsetstrokecolor{currentstroke}%
\pgfsetdash{}{0pt}%
\pgfpathmoveto{\pgfqpoint{2.393562in}{4.830606in}}%
\pgfpathlineto{\pgfqpoint{2.309553in}{4.592125in}}%
\pgfpathlineto{\pgfqpoint{2.444220in}{4.720662in}}%
\pgfpathclose%
\pgfusepath{fill}%
\end{pgfscope}%
\begin{pgfscope}%
\pgfpathrectangle{\pgfqpoint{0.539299in}{0.078740in}}{\pgfqpoint{7.842520in}{7.842520in}}%
\pgfusepath{clip}%
\pgfsetbuttcap%
\pgfsetroundjoin%
\definecolor{currentfill}{rgb}{0.187231,0.414746,0.556547}%
\pgfsetfillcolor{currentfill}%
\pgfsetlinewidth{0.000000pt}%
\definecolor{currentstroke}{rgb}{0.265145,0.232956,0.516599}%
\pgfsetstrokecolor{currentstroke}%
\pgfsetdash{}{0pt}%
\pgfpathmoveto{\pgfqpoint{2.405032in}{3.063501in}}%
\pgfpathlineto{\pgfqpoint{2.472446in}{2.377069in}}%
\pgfpathlineto{\pgfqpoint{2.544311in}{3.042210in}}%
\pgfpathclose%
\pgfusepath{fill}%
\end{pgfscope}%
\begin{pgfscope}%
\pgfpathrectangle{\pgfqpoint{0.539299in}{0.078740in}}{\pgfqpoint{7.842520in}{7.842520in}}%
\pgfusepath{clip}%
\pgfsetbuttcap%
\pgfsetroundjoin%
\definecolor{currentfill}{rgb}{0.964894,0.902323,0.123941}%
\pgfsetfillcolor{currentfill}%
\pgfsetlinewidth{0.000000pt}%
\definecolor{currentstroke}{rgb}{0.263663,0.237631,0.518762}%
\pgfsetstrokecolor{currentstroke}%
\pgfsetdash{}{0pt}%
\pgfpathmoveto{\pgfqpoint{3.378279in}{5.603577in}}%
\pgfpathlineto{\pgfqpoint{3.150177in}{5.616991in}}%
\pgfpathlineto{\pgfqpoint{3.291045in}{5.591989in}}%
\pgfpathclose%
\pgfusepath{fill}%
\end{pgfscope}%
\begin{pgfscope}%
\pgfpathrectangle{\pgfqpoint{0.539299in}{0.078740in}}{\pgfqpoint{7.842520in}{7.842520in}}%
\pgfusepath{clip}%
\pgfsetbuttcap%
\pgfsetroundjoin%
\definecolor{currentfill}{rgb}{0.214298,0.355619,0.551184}%
\pgfsetfillcolor{currentfill}%
\pgfsetlinewidth{0.000000pt}%
\definecolor{currentstroke}{rgb}{0.262138,0.242286,0.520837}%
\pgfsetstrokecolor{currentstroke}%
\pgfsetdash{}{0pt}%
\pgfpathmoveto{\pgfqpoint{2.472446in}{2.377069in}}%
\pgfpathlineto{\pgfqpoint{2.611847in}{2.338243in}}%
\pgfpathlineto{\pgfqpoint{2.684453in}{3.009085in}}%
\pgfpathclose%
\pgfusepath{fill}%
\end{pgfscope}%
\begin{pgfscope}%
\pgfpathrectangle{\pgfqpoint{0.539299in}{0.078740in}}{\pgfqpoint{7.842520in}{7.842520in}}%
\pgfusepath{clip}%
\pgfsetbuttcap%
\pgfsetroundjoin%
\definecolor{currentfill}{rgb}{0.636902,0.856542,0.216620}%
\pgfsetfillcolor{currentfill}%
\pgfsetlinewidth{0.000000pt}%
\definecolor{currentstroke}{rgb}{0.260571,0.246922,0.522828}%
\pgfsetstrokecolor{currentstroke}%
\pgfsetdash{}{0pt}%
\pgfpathmoveto{\pgfqpoint{2.478789in}{5.021565in}}%
\pgfpathlineto{\pgfqpoint{2.528143in}{4.972174in}}%
\pgfpathlineto{\pgfqpoint{2.613359in}{5.171039in}}%
\pgfpathclose%
\pgfusepath{fill}%
\end{pgfscope}%
\begin{pgfscope}%
\pgfpathrectangle{\pgfqpoint{0.539299in}{0.078740in}}{\pgfqpoint{7.842520in}{7.842520in}}%
\pgfusepath{clip}%
\pgfsetbuttcap%
\pgfsetroundjoin%
\definecolor{currentfill}{rgb}{0.149039,0.508051,0.557250}%
\pgfsetfillcolor{currentfill}%
\pgfsetlinewidth{0.000000pt}%
\definecolor{currentstroke}{rgb}{0.258965,0.251537,0.524736}%
\pgfsetstrokecolor{currentstroke}%
\pgfsetdash{}{0pt}%
\pgfpathmoveto{\pgfqpoint{2.405032in}{3.063501in}}%
\pgfpathlineto{\pgfqpoint{2.341276in}{3.620065in}}%
\pgfpathlineto{\pgfqpoint{2.266737in}{3.071192in}}%
\pgfpathclose%
\pgfusepath{fill}%
\end{pgfscope}%
\begin{pgfscope}%
\pgfpathrectangle{\pgfqpoint{0.539299in}{0.078740in}}{\pgfqpoint{7.842520in}{7.842520in}}%
\pgfusepath{clip}%
\pgfsetbuttcap%
\pgfsetroundjoin%
\definecolor{currentfill}{rgb}{0.344074,0.780029,0.397381}%
\pgfsetfillcolor{currentfill}%
\pgfsetlinewidth{0.000000pt}%
\definecolor{currentstroke}{rgb}{0.257322,0.256130,0.526563}%
\pgfsetstrokecolor{currentstroke}%
\pgfsetdash{}{0pt}%
\pgfpathmoveto{\pgfqpoint{2.227125in}{4.300185in}}%
\pgfpathlineto{\pgfqpoint{2.444220in}{4.720662in}}%
\pgfpathlineto{\pgfqpoint{2.309553in}{4.592125in}}%
\pgfpathclose%
\pgfusepath{fill}%
\end{pgfscope}%
\begin{pgfscope}%
\pgfpathrectangle{\pgfqpoint{0.539299in}{0.078740in}}{\pgfqpoint{7.842520in}{7.842520in}}%
\pgfusepath{clip}%
\pgfsetbuttcap%
\pgfsetroundjoin%
\definecolor{currentfill}{rgb}{0.175707,0.697900,0.491033}%
\pgfsetfillcolor{currentfill}%
\pgfsetlinewidth{0.000000pt}%
\definecolor{currentstroke}{rgb}{0.255645,0.260703,0.528312}%
\pgfsetstrokecolor{currentstroke}%
\pgfsetdash{}{0pt}%
\pgfpathmoveto{\pgfqpoint{4.439213in}{4.296234in}}%
\pgfpathlineto{\pgfqpoint{4.582787in}{4.072527in}}%
\pgfpathlineto{\pgfqpoint{4.665156in}{4.013355in}}%
\pgfpathclose%
\pgfusepath{fill}%
\end{pgfscope}%
\begin{pgfscope}%
\pgfpathrectangle{\pgfqpoint{0.539299in}{0.078740in}}{\pgfqpoint{7.842520in}{7.842520in}}%
\pgfusepath{clip}%
\pgfsetbuttcap%
\pgfsetroundjoin%
\definecolor{currentfill}{rgb}{0.935904,0.898570,0.108131}%
\pgfsetfillcolor{currentfill}%
\pgfsetlinewidth{0.000000pt}%
\definecolor{currentstroke}{rgb}{0.253935,0.265254,0.529983}%
\pgfsetstrokecolor{currentstroke}%
\pgfsetdash{}{0pt}%
\pgfpathmoveto{\pgfqpoint{3.433110in}{5.522817in}}%
\pgfpathlineto{\pgfqpoint{3.520103in}{5.522735in}}%
\pgfpathlineto{\pgfqpoint{3.378279in}{5.603577in}}%
\pgfpathclose%
\pgfusepath{fill}%
\end{pgfscope}%
\begin{pgfscope}%
\pgfpathrectangle{\pgfqpoint{0.539299in}{0.078740in}}{\pgfqpoint{7.842520in}{7.842520in}}%
\pgfusepath{clip}%
\pgfsetbuttcap%
\pgfsetroundjoin%
\definecolor{currentfill}{rgb}{0.278826,0.175490,0.483397}%
\pgfsetfillcolor{currentfill}%
\pgfsetlinewidth{0.000000pt}%
\definecolor{currentstroke}{rgb}{0.252194,0.269783,0.531579}%
\pgfsetstrokecolor{currentstroke}%
\pgfsetdash{}{0pt}%
\pgfpathmoveto{\pgfqpoint{6.073966in}{1.765621in}}%
\pgfpathlineto{\pgfqpoint{5.931443in}{1.959659in}}%
\pgfpathlineto{\pgfqpoint{5.853809in}{1.950429in}}%
\pgfpathclose%
\pgfusepath{fill}%
\end{pgfscope}%
\begin{pgfscope}%
\pgfpathrectangle{\pgfqpoint{0.539299in}{0.078740in}}{\pgfqpoint{7.842520in}{7.842520in}}%
\pgfusepath{clip}%
\pgfsetbuttcap%
\pgfsetroundjoin%
\definecolor{currentfill}{rgb}{0.140536,0.530132,0.555659}%
\pgfsetfillcolor{currentfill}%
\pgfsetlinewidth{0.000000pt}%
\definecolor{currentstroke}{rgb}{0.250425,0.274290,0.533103}%
\pgfsetstrokecolor{currentstroke}%
\pgfsetdash{}{0pt}%
\pgfpathmoveto{\pgfqpoint{4.931146in}{3.442939in}}%
\pgfpathlineto{\pgfqpoint{5.154816in}{3.180272in}}%
\pgfpathlineto{\pgfqpoint{5.012128in}{3.399559in}}%
\pgfpathclose%
\pgfusepath{fill}%
\end{pgfscope}%
\begin{pgfscope}%
\pgfpathrectangle{\pgfqpoint{0.539299in}{0.078740in}}{\pgfqpoint{7.842520in}{7.842520in}}%
\pgfusepath{clip}%
\pgfsetbuttcap%
\pgfsetroundjoin%
\definecolor{currentfill}{rgb}{0.386433,0.794644,0.372886}%
\pgfsetfillcolor{currentfill}%
\pgfsetlinewidth{0.000000pt}%
\definecolor{currentstroke}{rgb}{0.248629,0.278775,0.534556}%
\pgfsetstrokecolor{currentstroke}%
\pgfsetdash{}{0pt}%
\pgfpathmoveto{\pgfqpoint{4.235995in}{4.678979in}}%
\pgfpathlineto{\pgfqpoint{4.151482in}{4.727445in}}%
\pgfpathlineto{\pgfqpoint{4.379257in}{4.461675in}}%
\pgfpathclose%
\pgfusepath{fill}%
\end{pgfscope}%
\begin{pgfscope}%
\pgfpathrectangle{\pgfqpoint{0.539299in}{0.078740in}}{\pgfqpoint{7.842520in}{7.842520in}}%
\pgfusepath{clip}%
\pgfsetbuttcap%
\pgfsetroundjoin%
\definecolor{currentfill}{rgb}{0.177423,0.437527,0.557565}%
\pgfsetfillcolor{currentfill}%
\pgfsetlinewidth{0.000000pt}%
\definecolor{currentstroke}{rgb}{0.246811,0.283237,0.535941}%
\pgfsetstrokecolor{currentstroke}%
\pgfsetdash{}{0pt}%
\pgfpathmoveto{\pgfqpoint{5.217726in}{3.002518in}}%
\pgfpathlineto{\pgfqpoint{5.360735in}{2.786461in}}%
\pgfpathlineto{\pgfqpoint{5.297333in}{2.965317in}}%
\pgfpathclose%
\pgfusepath{fill}%
\end{pgfscope}%
\begin{pgfscope}%
\pgfpathrectangle{\pgfqpoint{0.539299in}{0.078740in}}{\pgfqpoint{7.842520in}{7.842520in}}%
\pgfusepath{clip}%
\pgfsetbuttcap%
\pgfsetroundjoin%
\definecolor{currentfill}{rgb}{0.120081,0.622161,0.534946}%
\pgfsetfillcolor{currentfill}%
\pgfsetlinewidth{0.000000pt}%
\definecolor{currentstroke}{rgb}{0.244972,0.287675,0.537260}%
\pgfsetstrokecolor{currentstroke}%
\pgfsetdash{}{0pt}%
\pgfpathmoveto{\pgfqpoint{2.204165in}{3.588171in}}%
\pgfpathlineto{\pgfqpoint{2.341276in}{3.620065in}}%
\pgfpathlineto{\pgfqpoint{2.281795in}{4.035068in}}%
\pgfpathclose%
\pgfusepath{fill}%
\end{pgfscope}%
\begin{pgfscope}%
\pgfpathrectangle{\pgfqpoint{0.539299in}{0.078740in}}{\pgfqpoint{7.842520in}{7.842520in}}%
\pgfusepath{clip}%
\pgfsetbuttcap%
\pgfsetroundjoin%
\definecolor{currentfill}{rgb}{0.265145,0.232956,0.516599}%
\pgfsetfillcolor{currentfill}%
\pgfsetlinewidth{0.000000pt}%
\definecolor{currentstroke}{rgb}{0.243113,0.292092,0.538516}%
\pgfsetstrokecolor{currentstroke}%
\pgfsetdash{}{0pt}%
\pgfpathmoveto{\pgfqpoint{5.788906in}{2.159810in}}%
\pgfpathlineto{\pgfqpoint{5.710883in}{2.164223in}}%
\pgfpathlineto{\pgfqpoint{5.931443in}{1.959659in}}%
\pgfpathclose%
\pgfusepath{fill}%
\end{pgfscope}%
\begin{pgfscope}%
\pgfpathrectangle{\pgfqpoint{0.539299in}{0.078740in}}{\pgfqpoint{7.842520in}{7.842520in}}%
\pgfusepath{clip}%
\pgfsetbuttcap%
\pgfsetroundjoin%
\definecolor{currentfill}{rgb}{0.866013,0.889868,0.095953}%
\pgfsetfillcolor{currentfill}%
\pgfsetlinewidth{0.000000pt}%
\definecolor{currentstroke}{rgb}{0.241237,0.296485,0.539709}%
\pgfsetstrokecolor{currentstroke}%
\pgfsetdash{}{0pt}%
\pgfpathmoveto{\pgfqpoint{2.836696in}{5.416830in}}%
\pgfpathlineto{\pgfqpoint{2.923598in}{5.523700in}}%
\pgfpathlineto{\pgfqpoint{2.699533in}{5.323077in}}%
\pgfpathclose%
\pgfusepath{fill}%
\end{pgfscope}%
\begin{pgfscope}%
\pgfpathrectangle{\pgfqpoint{0.539299in}{0.078740in}}{\pgfqpoint{7.842520in}{7.842520in}}%
\pgfusepath{clip}%
\pgfsetbuttcap%
\pgfsetroundjoin%
\definecolor{currentfill}{rgb}{0.226397,0.728888,0.462789}%
\pgfsetfillcolor{currentfill}%
\pgfsetlinewidth{0.000000pt}%
\definecolor{currentstroke}{rgb}{0.239346,0.300855,0.540844}%
\pgfsetstrokecolor{currentstroke}%
\pgfsetdash{}{0pt}%
\pgfpathmoveto{\pgfqpoint{2.281795in}{4.035068in}}%
\pgfpathlineto{\pgfqpoint{2.361965in}{4.410404in}}%
\pgfpathlineto{\pgfqpoint{2.227125in}{4.300185in}}%
\pgfpathclose%
\pgfusepath{fill}%
\end{pgfscope}%
\begin{pgfscope}%
\pgfpathrectangle{\pgfqpoint{0.539299in}{0.078740in}}{\pgfqpoint{7.842520in}{7.842520in}}%
\pgfusepath{clip}%
\pgfsetbuttcap%
\pgfsetroundjoin%
\definecolor{currentfill}{rgb}{0.906311,0.894855,0.098125}%
\pgfsetfillcolor{currentfill}%
\pgfsetlinewidth{0.000000pt}%
\definecolor{currentstroke}{rgb}{0.237441,0.305202,0.541921}%
\pgfsetstrokecolor{currentstroke}%
\pgfsetdash{}{0pt}%
\pgfpathmoveto{\pgfqpoint{3.662699in}{5.404278in}}%
\pgfpathlineto{\pgfqpoint{3.520103in}{5.522735in}}%
\pgfpathlineto{\pgfqpoint{3.433110in}{5.522817in}}%
\pgfpathclose%
\pgfusepath{fill}%
\end{pgfscope}%
\begin{pgfscope}%
\pgfpathrectangle{\pgfqpoint{0.539299in}{0.078740in}}{\pgfqpoint{7.842520in}{7.842520in}}%
\pgfusepath{clip}%
\pgfsetbuttcap%
\pgfsetroundjoin%
\definecolor{currentfill}{rgb}{0.128729,0.563265,0.551229}%
\pgfsetfillcolor{currentfill}%
\pgfsetlinewidth{0.000000pt}%
\definecolor{currentstroke}{rgb}{0.235526,0.309527,0.542944}%
\pgfsetstrokecolor{currentstroke}%
\pgfsetdash{}{0pt}%
\pgfpathmoveto{\pgfqpoint{5.012128in}{3.399559in}}%
\pgfpathlineto{\pgfqpoint{4.869240in}{3.622273in}}%
\pgfpathlineto{\pgfqpoint{4.931146in}{3.442939in}}%
\pgfpathclose%
\pgfusepath{fill}%
\end{pgfscope}%
\begin{pgfscope}%
\pgfpathrectangle{\pgfqpoint{0.539299in}{0.078740in}}{\pgfqpoint{7.842520in}{7.842520in}}%
\pgfusepath{clip}%
\pgfsetbuttcap%
\pgfsetroundjoin%
\definecolor{currentfill}{rgb}{0.163625,0.471133,0.558148}%
\pgfsetfillcolor{currentfill}%
\pgfsetlinewidth{0.000000pt}%
\definecolor{currentstroke}{rgb}{0.233603,0.313828,0.543914}%
\pgfsetstrokecolor{currentstroke}%
\pgfsetdash{}{0pt}%
\pgfpathmoveto{\pgfqpoint{5.297333in}{2.965317in}}%
\pgfpathlineto{\pgfqpoint{5.074536in}{3.221540in}}%
\pgfpathlineto{\pgfqpoint{5.217726in}{3.002518in}}%
\pgfpathclose%
\pgfusepath{fill}%
\end{pgfscope}%
\begin{pgfscope}%
\pgfpathrectangle{\pgfqpoint{0.539299in}{0.078740in}}{\pgfqpoint{7.842520in}{7.842520in}}%
\pgfusepath{clip}%
\pgfsetbuttcap%
\pgfsetroundjoin%
\definecolor{currentfill}{rgb}{0.283197,0.115680,0.436115}%
\pgfsetfillcolor{currentfill}%
\pgfsetlinewidth{0.000000pt}%
\definecolor{currentstroke}{rgb}{0.231674,0.318106,0.544834}%
\pgfsetstrokecolor{currentstroke}%
\pgfsetdash{}{0pt}%
\pgfpathmoveto{\pgfqpoint{6.073966in}{1.765621in}}%
\pgfpathlineto{\pgfqpoint{5.996574in}{1.738442in}}%
\pgfpathlineto{\pgfqpoint{6.139221in}{1.529769in}}%
\pgfpathclose%
\pgfusepath{fill}%
\end{pgfscope}%
\begin{pgfscope}%
\pgfpathrectangle{\pgfqpoint{0.539299in}{0.078740in}}{\pgfqpoint{7.842520in}{7.842520in}}%
\pgfusepath{clip}%
\pgfsetbuttcap%
\pgfsetroundjoin%
\definecolor{currentfill}{rgb}{0.506271,0.828786,0.300362}%
\pgfsetfillcolor{currentfill}%
\pgfsetlinewidth{0.000000pt}%
\definecolor{currentstroke}{rgb}{0.229739,0.322361,0.545706}%
\pgfsetstrokecolor{currentstroke}%
\pgfsetdash{}{0pt}%
\pgfpathmoveto{\pgfqpoint{4.235995in}{4.678979in}}%
\pgfpathlineto{\pgfqpoint{4.092594in}{4.886710in}}%
\pgfpathlineto{\pgfqpoint{4.007443in}{4.927974in}}%
\pgfpathclose%
\pgfusepath{fill}%
\end{pgfscope}%
\begin{pgfscope}%
\pgfpathrectangle{\pgfqpoint{0.539299in}{0.078740in}}{\pgfqpoint{7.842520in}{7.842520in}}%
\pgfusepath{clip}%
\pgfsetbuttcap%
\pgfsetroundjoin%
\definecolor{currentfill}{rgb}{0.515992,0.831158,0.294279}%
\pgfsetfillcolor{currentfill}%
\pgfsetlinewidth{0.000000pt}%
\definecolor{currentstroke}{rgb}{0.227802,0.326594,0.546532}%
\pgfsetstrokecolor{currentstroke}%
\pgfsetdash{}{0pt}%
\pgfpathmoveto{\pgfqpoint{2.444220in}{4.720662in}}%
\pgfpathlineto{\pgfqpoint{2.528143in}{4.972174in}}%
\pgfpathlineto{\pgfqpoint{2.393562in}{4.830606in}}%
\pgfpathclose%
\pgfusepath{fill}%
\end{pgfscope}%
\begin{pgfscope}%
\pgfpathrectangle{\pgfqpoint{0.539299in}{0.078740in}}{\pgfqpoint{7.842520in}{7.842520in}}%
\pgfusepath{clip}%
\pgfsetbuttcap%
\pgfsetroundjoin%
\definecolor{currentfill}{rgb}{0.964894,0.902323,0.123941}%
\pgfsetfillcolor{currentfill}%
\pgfsetlinewidth{0.000000pt}%
\definecolor{currentstroke}{rgb}{0.225863,0.330805,0.547314}%
\pgfsetstrokecolor{currentstroke}%
\pgfsetdash{}{0pt}%
\pgfpathmoveto{\pgfqpoint{3.291045in}{5.591989in}}%
\pgfpathlineto{\pgfqpoint{3.433110in}{5.522817in}}%
\pgfpathlineto{\pgfqpoint{3.378279in}{5.603577in}}%
\pgfpathclose%
\pgfusepath{fill}%
\end{pgfscope}%
\begin{pgfscope}%
\pgfpathrectangle{\pgfqpoint{0.539299in}{0.078740in}}{\pgfqpoint{7.842520in}{7.842520in}}%
\pgfusepath{clip}%
\pgfsetbuttcap%
\pgfsetroundjoin%
\definecolor{currentfill}{rgb}{0.955300,0.901065,0.118128}%
\pgfsetfillcolor{currentfill}%
\pgfsetlinewidth{0.000000pt}%
\definecolor{currentstroke}{rgb}{0.223925,0.334994,0.548053}%
\pgfsetstrokecolor{currentstroke}%
\pgfsetdash{}{0pt}%
\pgfpathmoveto{\pgfqpoint{2.923598in}{5.523700in}}%
\pgfpathlineto{\pgfqpoint{3.062895in}{5.558135in}}%
\pgfpathlineto{\pgfqpoint{3.150177in}{5.616991in}}%
\pgfpathclose%
\pgfusepath{fill}%
\end{pgfscope}%
\begin{pgfscope}%
\pgfpathrectangle{\pgfqpoint{0.539299in}{0.078740in}}{\pgfqpoint{7.842520in}{7.842520in}}%
\pgfusepath{clip}%
\pgfsetbuttcap%
\pgfsetroundjoin%
\definecolor{currentfill}{rgb}{0.214298,0.355619,0.551184}%
\pgfsetfillcolor{currentfill}%
\pgfsetlinewidth{0.000000pt}%
\definecolor{currentstroke}{rgb}{0.221989,0.339161,0.548752}%
\pgfsetstrokecolor{currentstroke}%
\pgfsetdash{}{0pt}%
\pgfpathmoveto{\pgfqpoint{2.611847in}{2.338243in}}%
\pgfpathlineto{\pgfqpoint{2.751870in}{2.294211in}}%
\pgfpathlineto{\pgfqpoint{2.684453in}{3.009085in}}%
\pgfpathclose%
\pgfusepath{fill}%
\end{pgfscope}%
\begin{pgfscope}%
\pgfpathrectangle{\pgfqpoint{0.539299in}{0.078740in}}{\pgfqpoint{7.842520in}{7.842520in}}%
\pgfusepath{clip}%
\pgfsetbuttcap%
\pgfsetroundjoin%
\definecolor{currentfill}{rgb}{0.185556,0.418570,0.556753}%
\pgfsetfillcolor{currentfill}%
\pgfsetlinewidth{0.000000pt}%
\definecolor{currentstroke}{rgb}{0.220057,0.343307,0.549413}%
\pgfsetstrokecolor{currentstroke}%
\pgfsetdash{}{0pt}%
\pgfpathmoveto{\pgfqpoint{2.544311in}{3.042210in}}%
\pgfpathlineto{\pgfqpoint{2.472446in}{2.377069in}}%
\pgfpathlineto{\pgfqpoint{2.684453in}{3.009085in}}%
\pgfpathclose%
\pgfusepath{fill}%
\end{pgfscope}%
\begin{pgfscope}%
\pgfpathrectangle{\pgfqpoint{0.539299in}{0.078740in}}{\pgfqpoint{7.842520in}{7.842520in}}%
\pgfusepath{clip}%
\pgfsetbuttcap%
\pgfsetroundjoin%
\definecolor{currentfill}{rgb}{0.304148,0.764704,0.419943}%
\pgfsetfillcolor{currentfill}%
\pgfsetlinewidth{0.000000pt}%
\definecolor{currentstroke}{rgb}{0.218130,0.347432,0.550038}%
\pgfsetstrokecolor{currentstroke}%
\pgfsetdash{}{0pt}%
\pgfpathmoveto{\pgfqpoint{4.439213in}{4.296234in}}%
\pgfpathlineto{\pgfqpoint{4.379257in}{4.461675in}}%
\pgfpathlineto{\pgfqpoint{4.295430in}{4.515609in}}%
\pgfpathclose%
\pgfusepath{fill}%
\end{pgfscope}%
\begin{pgfscope}%
\pgfpathrectangle{\pgfqpoint{0.539299in}{0.078740in}}{\pgfqpoint{7.842520in}{7.842520in}}%
\pgfusepath{clip}%
\pgfsetbuttcap%
\pgfsetroundjoin%
\definecolor{currentfill}{rgb}{0.595839,0.848717,0.243329}%
\pgfsetfillcolor{currentfill}%
\pgfsetlinewidth{0.000000pt}%
\definecolor{currentstroke}{rgb}{0.216210,0.351535,0.550627}%
\pgfsetstrokecolor{currentstroke}%
\pgfsetdash{}{0pt}%
\pgfpathmoveto{\pgfqpoint{4.007443in}{4.927974in}}%
\pgfpathlineto{\pgfqpoint{4.092594in}{4.886710in}}%
\pgfpathlineto{\pgfqpoint{3.949148in}{5.080344in}}%
\pgfpathclose%
\pgfusepath{fill}%
\end{pgfscope}%
\begin{pgfscope}%
\pgfpathrectangle{\pgfqpoint{0.539299in}{0.078740in}}{\pgfqpoint{7.842520in}{7.842520in}}%
\pgfusepath{clip}%
\pgfsetbuttcap%
\pgfsetroundjoin%
\definecolor{currentfill}{rgb}{0.327796,0.773980,0.406640}%
\pgfsetfillcolor{currentfill}%
\pgfsetlinewidth{0.000000pt}%
\definecolor{currentstroke}{rgb}{0.214298,0.355619,0.551184}%
\pgfsetstrokecolor{currentstroke}%
\pgfsetdash{}{0pt}%
\pgfpathmoveto{\pgfqpoint{2.227125in}{4.300185in}}%
\pgfpathlineto{\pgfqpoint{2.361965in}{4.410404in}}%
\pgfpathlineto{\pgfqpoint{2.444220in}{4.720662in}}%
\pgfpathclose%
\pgfusepath{fill}%
\end{pgfscope}%
\begin{pgfscope}%
\pgfpathrectangle{\pgfqpoint{0.539299in}{0.078740in}}{\pgfqpoint{7.842520in}{7.842520in}}%
\pgfusepath{clip}%
\pgfsetbuttcap%
\pgfsetroundjoin%
\definecolor{currentfill}{rgb}{0.280868,0.160771,0.472899}%
\pgfsetfillcolor{currentfill}%
\pgfsetlinewidth{0.000000pt}%
\definecolor{currentstroke}{rgb}{0.212395,0.359683,0.551710}%
\pgfsetstrokecolor{currentstroke}%
\pgfsetdash{}{0pt}%
\pgfpathmoveto{\pgfqpoint{5.853809in}{1.950429in}}%
\pgfpathlineto{\pgfqpoint{5.996574in}{1.738442in}}%
\pgfpathlineto{\pgfqpoint{6.073966in}{1.765621in}}%
\pgfpathclose%
\pgfusepath{fill}%
\end{pgfscope}%
\begin{pgfscope}%
\pgfpathrectangle{\pgfqpoint{0.539299in}{0.078740in}}{\pgfqpoint{7.842520in}{7.842520in}}%
\pgfusepath{clip}%
\pgfsetbuttcap%
\pgfsetroundjoin%
\definecolor{currentfill}{rgb}{0.241237,0.296485,0.539709}%
\pgfsetfillcolor{currentfill}%
\pgfsetlinewidth{0.000000pt}%
\definecolor{currentstroke}{rgb}{0.210503,0.363727,0.552206}%
\pgfsetstrokecolor{currentstroke}%
\pgfsetdash{}{0pt}%
\pgfpathmoveto{\pgfqpoint{5.788906in}{2.159810in}}%
\pgfpathlineto{\pgfqpoint{5.646299in}{2.364748in}}%
\pgfpathlineto{\pgfqpoint{5.567779in}{2.379196in}}%
\pgfpathclose%
\pgfusepath{fill}%
\end{pgfscope}%
\begin{pgfscope}%
\pgfpathrectangle{\pgfqpoint{0.539299in}{0.078740in}}{\pgfqpoint{7.842520in}{7.842520in}}%
\pgfusepath{clip}%
\pgfsetbuttcap%
\pgfsetroundjoin%
\definecolor{currentfill}{rgb}{0.814576,0.883393,0.110347}%
\pgfsetfillcolor{currentfill}%
\pgfsetlinewidth{0.000000pt}%
\definecolor{currentstroke}{rgb}{0.208623,0.367752,0.552675}%
\pgfsetstrokecolor{currentstroke}%
\pgfsetdash{}{0pt}%
\pgfpathmoveto{\pgfqpoint{3.719555in}{5.277291in}}%
\pgfpathlineto{\pgfqpoint{3.805790in}{5.254759in}}%
\pgfpathlineto{\pgfqpoint{3.662699in}{5.404278in}}%
\pgfpathclose%
\pgfusepath{fill}%
\end{pgfscope}%
\begin{pgfscope}%
\pgfpathrectangle{\pgfqpoint{0.539299in}{0.078740in}}{\pgfqpoint{7.842520in}{7.842520in}}%
\pgfusepath{clip}%
\pgfsetbuttcap%
\pgfsetroundjoin%
\definecolor{currentfill}{rgb}{0.709898,0.868751,0.169257}%
\pgfsetfillcolor{currentfill}%
\pgfsetlinewidth{0.000000pt}%
\definecolor{currentstroke}{rgb}{0.206756,0.371758,0.553117}%
\pgfsetstrokecolor{currentstroke}%
\pgfsetdash{}{0pt}%
\pgfpathmoveto{\pgfqpoint{3.949148in}{5.080344in}}%
\pgfpathlineto{\pgfqpoint{3.805790in}{5.254759in}}%
\pgfpathlineto{\pgfqpoint{3.863419in}{5.112875in}}%
\pgfpathclose%
\pgfusepath{fill}%
\end{pgfscope}%
\begin{pgfscope}%
\pgfpathrectangle{\pgfqpoint{0.539299in}{0.078740in}}{\pgfqpoint{7.842520in}{7.842520in}}%
\pgfusepath{clip}%
\pgfsetbuttcap%
\pgfsetroundjoin%
\definecolor{currentfill}{rgb}{0.143343,0.522773,0.556295}%
\pgfsetfillcolor{currentfill}%
\pgfsetlinewidth{0.000000pt}%
\definecolor{currentstroke}{rgb}{0.204903,0.375746,0.553533}%
\pgfsetstrokecolor{currentstroke}%
\pgfsetdash{}{0pt}%
\pgfpathmoveto{\pgfqpoint{4.931146in}{3.442939in}}%
\pgfpathlineto{\pgfqpoint{5.074536in}{3.221540in}}%
\pgfpathlineto{\pgfqpoint{5.154816in}{3.180272in}}%
\pgfpathclose%
\pgfusepath{fill}%
\end{pgfscope}%
\begin{pgfscope}%
\pgfpathrectangle{\pgfqpoint{0.539299in}{0.078740in}}{\pgfqpoint{7.842520in}{7.842520in}}%
\pgfusepath{clip}%
\pgfsetbuttcap%
\pgfsetroundjoin%
\definecolor{currentfill}{rgb}{0.119699,0.618490,0.536347}%
\pgfsetfillcolor{currentfill}%
\pgfsetlinewidth{0.000000pt}%
\definecolor{currentstroke}{rgb}{0.203063,0.379716,0.553925}%
\pgfsetstrokecolor{currentstroke}%
\pgfsetdash{}{0pt}%
\pgfpathmoveto{\pgfqpoint{4.869240in}{3.622273in}}%
\pgfpathlineto{\pgfqpoint{4.726130in}{3.847156in}}%
\pgfpathlineto{\pgfqpoint{4.787540in}{3.665841in}}%
\pgfpathclose%
\pgfusepath{fill}%
\end{pgfscope}%
\begin{pgfscope}%
\pgfpathrectangle{\pgfqpoint{0.539299in}{0.078740in}}{\pgfqpoint{7.842520in}{7.842520in}}%
\pgfusepath{clip}%
\pgfsetbuttcap%
\pgfsetroundjoin%
\definecolor{currentfill}{rgb}{0.783315,0.879285,0.125405}%
\pgfsetfillcolor{currentfill}%
\pgfsetlinewidth{0.000000pt}%
\definecolor{currentstroke}{rgb}{0.201239,0.383670,0.554294}%
\pgfsetstrokecolor{currentstroke}%
\pgfsetdash{}{0pt}%
\pgfpathmoveto{\pgfqpoint{2.699533in}{5.323077in}}%
\pgfpathlineto{\pgfqpoint{2.613359in}{5.171039in}}%
\pgfpathlineto{\pgfqpoint{2.750456in}{5.265020in}}%
\pgfpathclose%
\pgfusepath{fill}%
\end{pgfscope}%
\begin{pgfscope}%
\pgfpathrectangle{\pgfqpoint{0.539299in}{0.078740in}}{\pgfqpoint{7.842520in}{7.842520in}}%
\pgfusepath{clip}%
\pgfsetbuttcap%
\pgfsetroundjoin%
\definecolor{currentfill}{rgb}{0.377779,0.791781,0.377939}%
\pgfsetfillcolor{currentfill}%
\pgfsetlinewidth{0.000000pt}%
\definecolor{currentstroke}{rgb}{0.199430,0.387607,0.554642}%
\pgfsetstrokecolor{currentstroke}%
\pgfsetdash{}{0pt}%
\pgfpathmoveto{\pgfqpoint{4.379257in}{4.461675in}}%
\pgfpathlineto{\pgfqpoint{4.151482in}{4.727445in}}%
\pgfpathlineto{\pgfqpoint{4.295430in}{4.515609in}}%
\pgfpathclose%
\pgfusepath{fill}%
\end{pgfscope}%
\begin{pgfscope}%
\pgfpathrectangle{\pgfqpoint{0.539299in}{0.078740in}}{\pgfqpoint{7.842520in}{7.842520in}}%
\pgfusepath{clip}%
\pgfsetbuttcap%
\pgfsetroundjoin%
\definecolor{currentfill}{rgb}{0.267968,0.223549,0.512008}%
\pgfsetfillcolor{currentfill}%
\pgfsetlinewidth{0.000000pt}%
\definecolor{currentstroke}{rgb}{0.197636,0.391528,0.554969}%
\pgfsetstrokecolor{currentstroke}%
\pgfsetdash{}{0pt}%
\pgfpathmoveto{\pgfqpoint{5.710883in}{2.164223in}}%
\pgfpathlineto{\pgfqpoint{5.853809in}{1.950429in}}%
\pgfpathlineto{\pgfqpoint{5.931443in}{1.959659in}}%
\pgfpathclose%
\pgfusepath{fill}%
\end{pgfscope}%
\begin{pgfscope}%
\pgfpathrectangle{\pgfqpoint{0.539299in}{0.078740in}}{\pgfqpoint{7.842520in}{7.842520in}}%
\pgfusepath{clip}%
\pgfsetbuttcap%
\pgfsetroundjoin%
\definecolor{currentfill}{rgb}{0.216210,0.351535,0.550627}%
\pgfsetfillcolor{currentfill}%
\pgfsetlinewidth{0.000000pt}%
\definecolor{currentstroke}{rgb}{0.195860,0.395433,0.555276}%
\pgfsetstrokecolor{currentstroke}%
\pgfsetdash{}{0pt}%
\pgfpathmoveto{\pgfqpoint{5.424487in}{2.595182in}}%
\pgfpathlineto{\pgfqpoint{5.646299in}{2.364748in}}%
\pgfpathlineto{\pgfqpoint{5.503585in}{2.573766in}}%
\pgfpathclose%
\pgfusepath{fill}%
\end{pgfscope}%
\begin{pgfscope}%
\pgfpathrectangle{\pgfqpoint{0.539299in}{0.078740in}}{\pgfqpoint{7.842520in}{7.842520in}}%
\pgfusepath{clip}%
\pgfsetbuttcap%
\pgfsetroundjoin%
\definecolor{currentfill}{rgb}{0.935904,0.898570,0.108131}%
\pgfsetfillcolor{currentfill}%
\pgfsetlinewidth{0.000000pt}%
\definecolor{currentstroke}{rgb}{0.194100,0.399323,0.555565}%
\pgfsetstrokecolor{currentstroke}%
\pgfsetdash{}{0pt}%
\pgfpathmoveto{\pgfqpoint{3.062895in}{5.558135in}}%
\pgfpathlineto{\pgfqpoint{2.923598in}{5.523700in}}%
\pgfpathlineto{\pgfqpoint{2.836696in}{5.416830in}}%
\pgfpathclose%
\pgfusepath{fill}%
\end{pgfscope}%
\begin{pgfscope}%
\pgfpathrectangle{\pgfqpoint{0.539299in}{0.078740in}}{\pgfqpoint{7.842520in}{7.842520in}}%
\pgfusepath{clip}%
\pgfsetbuttcap%
\pgfsetroundjoin%
\definecolor{currentfill}{rgb}{0.129933,0.559582,0.551864}%
\pgfsetfillcolor{currentfill}%
\pgfsetlinewidth{0.000000pt}%
\definecolor{currentstroke}{rgb}{0.192357,0.403199,0.555836}%
\pgfsetstrokecolor{currentstroke}%
\pgfsetdash{}{0pt}%
\pgfpathmoveto{\pgfqpoint{2.479796in}{3.628915in}}%
\pgfpathlineto{\pgfqpoint{2.341276in}{3.620065in}}%
\pgfpathlineto{\pgfqpoint{2.405032in}{3.063501in}}%
\pgfpathclose%
\pgfusepath{fill}%
\end{pgfscope}%
\begin{pgfscope}%
\pgfpathrectangle{\pgfqpoint{0.539299in}{0.078740in}}{\pgfqpoint{7.842520in}{7.842520in}}%
\pgfusepath{clip}%
\pgfsetbuttcap%
\pgfsetroundjoin%
\definecolor{currentfill}{rgb}{0.835270,0.886029,0.102646}%
\pgfsetfillcolor{currentfill}%
\pgfsetlinewidth{0.000000pt}%
\definecolor{currentstroke}{rgb}{0.190631,0.407061,0.556089}%
\pgfsetstrokecolor{currentstroke}%
\pgfsetdash{}{0pt}%
\pgfpathmoveto{\pgfqpoint{2.699533in}{5.323077in}}%
\pgfpathlineto{\pgfqpoint{2.750456in}{5.265020in}}%
\pgfpathlineto{\pgfqpoint{2.836696in}{5.416830in}}%
\pgfpathclose%
\pgfusepath{fill}%
\end{pgfscope}%
\begin{pgfscope}%
\pgfpathrectangle{\pgfqpoint{0.539299in}{0.078740in}}{\pgfqpoint{7.842520in}{7.842520in}}%
\pgfusepath{clip}%
\pgfsetbuttcap%
\pgfsetroundjoin%
\definecolor{currentfill}{rgb}{0.496615,0.826376,0.306377}%
\pgfsetfillcolor{currentfill}%
\pgfsetlinewidth{0.000000pt}%
\definecolor{currentstroke}{rgb}{0.188923,0.410910,0.556326}%
\pgfsetstrokecolor{currentstroke}%
\pgfsetdash{}{0pt}%
\pgfpathmoveto{\pgfqpoint{4.007443in}{4.927974in}}%
\pgfpathlineto{\pgfqpoint{4.151482in}{4.727445in}}%
\pgfpathlineto{\pgfqpoint{4.235995in}{4.678979in}}%
\pgfpathclose%
\pgfusepath{fill}%
\end{pgfscope}%
\begin{pgfscope}%
\pgfpathrectangle{\pgfqpoint{0.539299in}{0.078740in}}{\pgfqpoint{7.842520in}{7.842520in}}%
\pgfusepath{clip}%
\pgfsetbuttcap%
\pgfsetroundjoin%
\definecolor{currentfill}{rgb}{0.983868,0.904867,0.136897}%
\pgfsetfillcolor{currentfill}%
\pgfsetlinewidth{0.000000pt}%
\definecolor{currentstroke}{rgb}{0.187231,0.414746,0.556547}%
\pgfsetstrokecolor{currentstroke}%
\pgfsetdash{}{0pt}%
\pgfpathmoveto{\pgfqpoint{3.150177in}{5.616991in}}%
\pgfpathlineto{\pgfqpoint{3.203841in}{5.543255in}}%
\pgfpathlineto{\pgfqpoint{3.291045in}{5.591989in}}%
\pgfpathclose%
\pgfusepath{fill}%
\end{pgfscope}%
\begin{pgfscope}%
\pgfpathrectangle{\pgfqpoint{0.539299in}{0.078740in}}{\pgfqpoint{7.842520in}{7.842520in}}%
\pgfusepath{clip}%
\pgfsetbuttcap%
\pgfsetroundjoin%
\definecolor{currentfill}{rgb}{0.906311,0.894855,0.098125}%
\pgfsetfillcolor{currentfill}%
\pgfsetlinewidth{0.000000pt}%
\definecolor{currentstroke}{rgb}{0.185556,0.418570,0.556753}%
\pgfsetstrokecolor{currentstroke}%
\pgfsetdash{}{0pt}%
\pgfpathmoveto{\pgfqpoint{3.433110in}{5.522817in}}%
\pgfpathlineto{\pgfqpoint{3.576041in}{5.415869in}}%
\pgfpathlineto{\pgfqpoint{3.662699in}{5.404278in}}%
\pgfpathclose%
\pgfusepath{fill}%
\end{pgfscope}%
\begin{pgfscope}%
\pgfpathrectangle{\pgfqpoint{0.539299in}{0.078740in}}{\pgfqpoint{7.842520in}{7.842520in}}%
\pgfusepath{clip}%
\pgfsetbuttcap%
\pgfsetroundjoin%
\definecolor{currentfill}{rgb}{0.199430,0.387607,0.554642}%
\pgfsetfillcolor{currentfill}%
\pgfsetlinewidth{0.000000pt}%
\definecolor{currentstroke}{rgb}{0.183898,0.422383,0.556944}%
\pgfsetstrokecolor{currentstroke}%
\pgfsetdash{}{0pt}%
\pgfpathmoveto{\pgfqpoint{5.503585in}{2.573766in}}%
\pgfpathlineto{\pgfqpoint{5.360735in}{2.786461in}}%
\pgfpathlineto{\pgfqpoint{5.424487in}{2.595182in}}%
\pgfpathclose%
\pgfusepath{fill}%
\end{pgfscope}%
\begin{pgfscope}%
\pgfpathrectangle{\pgfqpoint{0.539299in}{0.078740in}}{\pgfqpoint{7.842520in}{7.842520in}}%
\pgfusepath{clip}%
\pgfsetbuttcap%
\pgfsetroundjoin%
\definecolor{currentfill}{rgb}{0.140210,0.665859,0.513427}%
\pgfsetfillcolor{currentfill}%
\pgfsetlinewidth{0.000000pt}%
\definecolor{currentstroke}{rgb}{0.182256,0.426184,0.557120}%
\pgfsetstrokecolor{currentstroke}%
\pgfsetdash{}{0pt}%
\pgfpathmoveto{\pgfqpoint{2.281795in}{4.035068in}}%
\pgfpathlineto{\pgfqpoint{2.341276in}{3.620065in}}%
\pgfpathlineto{\pgfqpoint{2.418849in}{4.087714in}}%
\pgfpathclose%
\pgfusepath{fill}%
\end{pgfscope}%
\begin{pgfscope}%
\pgfpathrectangle{\pgfqpoint{0.539299in}{0.078740in}}{\pgfqpoint{7.842520in}{7.842520in}}%
\pgfusepath{clip}%
\pgfsetbuttcap%
\pgfsetroundjoin%
\definecolor{currentfill}{rgb}{0.246811,0.283237,0.535941}%
\pgfsetfillcolor{currentfill}%
\pgfsetlinewidth{0.000000pt}%
\definecolor{currentstroke}{rgb}{0.180629,0.429975,0.557282}%
\pgfsetstrokecolor{currentstroke}%
\pgfsetdash{}{0pt}%
\pgfpathmoveto{\pgfqpoint{5.567779in}{2.379196in}}%
\pgfpathlineto{\pgfqpoint{5.710883in}{2.164223in}}%
\pgfpathlineto{\pgfqpoint{5.788906in}{2.159810in}}%
\pgfpathclose%
\pgfusepath{fill}%
\end{pgfscope}%
\begin{pgfscope}%
\pgfpathrectangle{\pgfqpoint{0.539299in}{0.078740in}}{\pgfqpoint{7.842520in}{7.842520in}}%
\pgfusepath{clip}%
\pgfsetbuttcap%
\pgfsetroundjoin%
\definecolor{currentfill}{rgb}{0.121148,0.592739,0.544641}%
\pgfsetfillcolor{currentfill}%
\pgfsetlinewidth{0.000000pt}%
\definecolor{currentstroke}{rgb}{0.179019,0.433756,0.557430}%
\pgfsetstrokecolor{currentstroke}%
\pgfsetdash{}{0pt}%
\pgfpathmoveto{\pgfqpoint{4.869240in}{3.622273in}}%
\pgfpathlineto{\pgfqpoint{4.787540in}{3.665841in}}%
\pgfpathlineto{\pgfqpoint{4.931146in}{3.442939in}}%
\pgfpathclose%
\pgfusepath{fill}%
\end{pgfscope}%
\begin{pgfscope}%
\pgfpathrectangle{\pgfqpoint{0.539299in}{0.078740in}}{\pgfqpoint{7.842520in}{7.842520in}}%
\pgfusepath{clip}%
\pgfsetbuttcap%
\pgfsetroundjoin%
\definecolor{currentfill}{rgb}{0.214000,0.722114,0.469588}%
\pgfsetfillcolor{currentfill}%
\pgfsetlinewidth{0.000000pt}%
\definecolor{currentstroke}{rgb}{0.177423,0.437527,0.557565}%
\pgfsetstrokecolor{currentstroke}%
\pgfsetdash{}{0pt}%
\pgfpathmoveto{\pgfqpoint{2.361965in}{4.410404in}}%
\pgfpathlineto{\pgfqpoint{2.281795in}{4.035068in}}%
\pgfpathlineto{\pgfqpoint{2.418849in}{4.087714in}}%
\pgfpathclose%
\pgfusepath{fill}%
\end{pgfscope}%
\begin{pgfscope}%
\pgfpathrectangle{\pgfqpoint{0.539299in}{0.078740in}}{\pgfqpoint{7.842520in}{7.842520in}}%
\pgfusepath{clip}%
\pgfsetbuttcap%
\pgfsetroundjoin%
\definecolor{currentfill}{rgb}{0.720391,0.870350,0.162603}%
\pgfsetfillcolor{currentfill}%
\pgfsetlinewidth{0.000000pt}%
\definecolor{currentstroke}{rgb}{0.175841,0.441290,0.557685}%
\pgfsetstrokecolor{currentstroke}%
\pgfsetdash{}{0pt}%
\pgfpathmoveto{\pgfqpoint{2.750456in}{5.265020in}}%
\pgfpathlineto{\pgfqpoint{2.613359in}{5.171039in}}%
\pgfpathlineto{\pgfqpoint{2.528143in}{4.972174in}}%
\pgfpathclose%
\pgfusepath{fill}%
\end{pgfscope}%
\begin{pgfscope}%
\pgfpathrectangle{\pgfqpoint{0.539299in}{0.078740in}}{\pgfqpoint{7.842520in}{7.842520in}}%
\pgfusepath{clip}%
\pgfsetbuttcap%
\pgfsetroundjoin%
\definecolor{currentfill}{rgb}{0.983868,0.904867,0.136897}%
\pgfsetfillcolor{currentfill}%
\pgfsetlinewidth{0.000000pt}%
\definecolor{currentstroke}{rgb}{0.174274,0.445044,0.557792}%
\pgfsetstrokecolor{currentstroke}%
\pgfsetdash{}{0pt}%
\pgfpathmoveto{\pgfqpoint{3.062895in}{5.558135in}}%
\pgfpathlineto{\pgfqpoint{3.203841in}{5.543255in}}%
\pgfpathlineto{\pgfqpoint{3.150177in}{5.616991in}}%
\pgfpathclose%
\pgfusepath{fill}%
\end{pgfscope}%
\begin{pgfscope}%
\pgfpathrectangle{\pgfqpoint{0.539299in}{0.078740in}}{\pgfqpoint{7.842520in}{7.842520in}}%
\pgfusepath{clip}%
\pgfsetbuttcap%
\pgfsetroundjoin%
\definecolor{currentfill}{rgb}{0.162016,0.687316,0.499129}%
\pgfsetfillcolor{currentfill}%
\pgfsetlinewidth{0.000000pt}%
\definecolor{currentstroke}{rgb}{0.172719,0.448791,0.557885}%
\pgfsetstrokecolor{currentstroke}%
\pgfsetdash{}{0pt}%
\pgfpathmoveto{\pgfqpoint{4.499656in}{4.110842in}}%
\pgfpathlineto{\pgfqpoint{4.726130in}{3.847156in}}%
\pgfpathlineto{\pgfqpoint{4.582787in}{4.072527in}}%
\pgfpathclose%
\pgfusepath{fill}%
\end{pgfscope}%
\begin{pgfscope}%
\pgfpathrectangle{\pgfqpoint{0.539299in}{0.078740in}}{\pgfqpoint{7.842520in}{7.842520in}}%
\pgfusepath{clip}%
\pgfsetbuttcap%
\pgfsetroundjoin%
\definecolor{currentfill}{rgb}{0.866013,0.889868,0.095953}%
\pgfsetfillcolor{currentfill}%
\pgfsetlinewidth{0.000000pt}%
\definecolor{currentstroke}{rgb}{0.171176,0.452530,0.557965}%
\pgfsetstrokecolor{currentstroke}%
\pgfsetdash{}{0pt}%
\pgfpathmoveto{\pgfqpoint{3.662699in}{5.404278in}}%
\pgfpathlineto{\pgfqpoint{3.576041in}{5.415869in}}%
\pgfpathlineto{\pgfqpoint{3.719555in}{5.277291in}}%
\pgfpathclose%
\pgfusepath{fill}%
\end{pgfscope}%
\begin{pgfscope}%
\pgfpathrectangle{\pgfqpoint{0.539299in}{0.078740in}}{\pgfqpoint{7.842520in}{7.842520in}}%
\pgfusepath{clip}%
\pgfsetbuttcap%
\pgfsetroundjoin%
\definecolor{currentfill}{rgb}{0.668054,0.861999,0.196293}%
\pgfsetfillcolor{currentfill}%
\pgfsetlinewidth{0.000000pt}%
\definecolor{currentstroke}{rgb}{0.169646,0.456262,0.558030}%
\pgfsetstrokecolor{currentstroke}%
\pgfsetdash{}{0pt}%
\pgfpathmoveto{\pgfqpoint{3.949148in}{5.080344in}}%
\pgfpathlineto{\pgfqpoint{3.863419in}{5.112875in}}%
\pgfpathlineto{\pgfqpoint{4.007443in}{4.927974in}}%
\pgfpathclose%
\pgfusepath{fill}%
\end{pgfscope}%
\begin{pgfscope}%
\pgfpathrectangle{\pgfqpoint{0.539299in}{0.078740in}}{\pgfqpoint{7.842520in}{7.842520in}}%
\pgfusepath{clip}%
\pgfsetbuttcap%
\pgfsetroundjoin%
\definecolor{currentfill}{rgb}{0.772852,0.877868,0.131109}%
\pgfsetfillcolor{currentfill}%
\pgfsetlinewidth{0.000000pt}%
\definecolor{currentstroke}{rgb}{0.168126,0.459988,0.558082}%
\pgfsetstrokecolor{currentstroke}%
\pgfsetdash{}{0pt}%
\pgfpathmoveto{\pgfqpoint{3.863419in}{5.112875in}}%
\pgfpathlineto{\pgfqpoint{3.805790in}{5.254759in}}%
\pgfpathlineto{\pgfqpoint{3.719555in}{5.277291in}}%
\pgfpathclose%
\pgfusepath{fill}%
\end{pgfscope}%
\begin{pgfscope}%
\pgfpathrectangle{\pgfqpoint{0.539299in}{0.078740in}}{\pgfqpoint{7.842520in}{7.842520in}}%
\pgfusepath{clip}%
\pgfsetbuttcap%
\pgfsetroundjoin%
\definecolor{currentfill}{rgb}{0.143343,0.522773,0.556295}%
\pgfsetfillcolor{currentfill}%
\pgfsetlinewidth{0.000000pt}%
\definecolor{currentstroke}{rgb}{0.166617,0.463708,0.558119}%
\pgfsetstrokecolor{currentstroke}%
\pgfsetdash{}{0pt}%
\pgfpathmoveto{\pgfqpoint{2.619520in}{3.617668in}}%
\pgfpathlineto{\pgfqpoint{2.405032in}{3.063501in}}%
\pgfpathlineto{\pgfqpoint{2.544311in}{3.042210in}}%
\pgfpathclose%
\pgfusepath{fill}%
\end{pgfscope}%
\begin{pgfscope}%
\pgfpathrectangle{\pgfqpoint{0.539299in}{0.078740in}}{\pgfqpoint{7.842520in}{7.842520in}}%
\pgfusepath{clip}%
\pgfsetbuttcap%
\pgfsetroundjoin%
\definecolor{currentfill}{rgb}{0.282910,0.105393,0.426902}%
\pgfsetfillcolor{currentfill}%
\pgfsetlinewidth{0.000000pt}%
\definecolor{currentstroke}{rgb}{0.165117,0.467423,0.558141}%
\pgfsetstrokecolor{currentstroke}%
\pgfsetdash{}{0pt}%
\pgfpathmoveto{\pgfqpoint{6.061409in}{1.481685in}}%
\pgfpathlineto{\pgfqpoint{6.139221in}{1.529769in}}%
\pgfpathlineto{\pgfqpoint{5.996574in}{1.738442in}}%
\pgfpathclose%
\pgfusepath{fill}%
\end{pgfscope}%
\begin{pgfscope}%
\pgfpathrectangle{\pgfqpoint{0.539299in}{0.078740in}}{\pgfqpoint{7.842520in}{7.842520in}}%
\pgfusepath{clip}%
\pgfsetbuttcap%
\pgfsetroundjoin%
\definecolor{currentfill}{rgb}{0.220057,0.343307,0.549413}%
\pgfsetfillcolor{currentfill}%
\pgfsetlinewidth{0.000000pt}%
\definecolor{currentstroke}{rgb}{0.163625,0.471133,0.558148}%
\pgfsetstrokecolor{currentstroke}%
\pgfsetdash{}{0pt}%
\pgfpathmoveto{\pgfqpoint{5.424487in}{2.595182in}}%
\pgfpathlineto{\pgfqpoint{5.567779in}{2.379196in}}%
\pgfpathlineto{\pgfqpoint{5.646299in}{2.364748in}}%
\pgfpathclose%
\pgfusepath{fill}%
\end{pgfscope}%
\begin{pgfscope}%
\pgfpathrectangle{\pgfqpoint{0.539299in}{0.078740in}}{\pgfqpoint{7.842520in}{7.842520in}}%
\pgfusepath{clip}%
\pgfsetbuttcap%
\pgfsetroundjoin%
\definecolor{currentfill}{rgb}{0.214000,0.722114,0.469588}%
\pgfsetfillcolor{currentfill}%
\pgfsetlinewidth{0.000000pt}%
\definecolor{currentstroke}{rgb}{0.162142,0.474838,0.558140}%
\pgfsetstrokecolor{currentstroke}%
\pgfsetdash{}{0pt}%
\pgfpathmoveto{\pgfqpoint{4.582787in}{4.072527in}}%
\pgfpathlineto{\pgfqpoint{4.439213in}{4.296234in}}%
\pgfpathlineto{\pgfqpoint{4.499656in}{4.110842in}}%
\pgfpathclose%
\pgfusepath{fill}%
\end{pgfscope}%
\begin{pgfscope}%
\pgfpathrectangle{\pgfqpoint{0.539299in}{0.078740in}}{\pgfqpoint{7.842520in}{7.842520in}}%
\pgfusepath{clip}%
\pgfsetbuttcap%
\pgfsetroundjoin%
\definecolor{currentfill}{rgb}{0.168126,0.459988,0.558082}%
\pgfsetfillcolor{currentfill}%
\pgfsetlinewidth{0.000000pt}%
\definecolor{currentstroke}{rgb}{0.160665,0.478540,0.558115}%
\pgfsetstrokecolor{currentstroke}%
\pgfsetdash{}{0pt}%
\pgfpathmoveto{\pgfqpoint{5.217726in}{3.002518in}}%
\pgfpathlineto{\pgfqpoint{5.137323in}{3.030184in}}%
\pgfpathlineto{\pgfqpoint{5.360735in}{2.786461in}}%
\pgfpathclose%
\pgfusepath{fill}%
\end{pgfscope}%
\begin{pgfscope}%
\pgfpathrectangle{\pgfqpoint{0.539299in}{0.078740in}}{\pgfqpoint{7.842520in}{7.842520in}}%
\pgfusepath{clip}%
\pgfsetbuttcap%
\pgfsetroundjoin%
\definecolor{currentfill}{rgb}{0.983868,0.904867,0.136897}%
\pgfsetfillcolor{currentfill}%
\pgfsetlinewidth{0.000000pt}%
\definecolor{currentstroke}{rgb}{0.159194,0.482237,0.558073}%
\pgfsetstrokecolor{currentstroke}%
\pgfsetdash{}{0pt}%
\pgfpathmoveto{\pgfqpoint{3.203841in}{5.543255in}}%
\pgfpathlineto{\pgfqpoint{3.433110in}{5.522817in}}%
\pgfpathlineto{\pgfqpoint{3.291045in}{5.591989in}}%
\pgfpathclose%
\pgfusepath{fill}%
\end{pgfscope}%
\begin{pgfscope}%
\pgfpathrectangle{\pgfqpoint{0.539299in}{0.078740in}}{\pgfqpoint{7.842520in}{7.842520in}}%
\pgfusepath{clip}%
\pgfsetbuttcap%
\pgfsetroundjoin%
\definecolor{currentfill}{rgb}{0.214298,0.355619,0.551184}%
\pgfsetfillcolor{currentfill}%
\pgfsetlinewidth{0.000000pt}%
\definecolor{currentstroke}{rgb}{0.157729,0.485932,0.558013}%
\pgfsetstrokecolor{currentstroke}%
\pgfsetdash{}{0pt}%
\pgfpathmoveto{\pgfqpoint{2.751870in}{2.294211in}}%
\pgfpathlineto{\pgfqpoint{2.892470in}{2.245647in}}%
\pgfpathlineto{\pgfqpoint{2.966934in}{2.913287in}}%
\pgfpathclose%
\pgfusepath{fill}%
\end{pgfscope}%
\begin{pgfscope}%
\pgfpathrectangle{\pgfqpoint{0.539299in}{0.078740in}}{\pgfqpoint{7.842520in}{7.842520in}}%
\pgfusepath{clip}%
\pgfsetbuttcap%
\pgfsetroundjoin%
\definecolor{currentfill}{rgb}{0.128087,0.647749,0.523491}%
\pgfsetfillcolor{currentfill}%
\pgfsetlinewidth{0.000000pt}%
\definecolor{currentstroke}{rgb}{0.156270,0.489624,0.557936}%
\pgfsetstrokecolor{currentstroke}%
\pgfsetdash{}{0pt}%
\pgfpathmoveto{\pgfqpoint{4.787540in}{3.665841in}}%
\pgfpathlineto{\pgfqpoint{4.726130in}{3.847156in}}%
\pgfpathlineto{\pgfqpoint{4.643709in}{3.889018in}}%
\pgfpathclose%
\pgfusepath{fill}%
\end{pgfscope}%
\begin{pgfscope}%
\pgfpathrectangle{\pgfqpoint{0.539299in}{0.078740in}}{\pgfqpoint{7.842520in}{7.842520in}}%
\pgfusepath{clip}%
\pgfsetbuttcap%
\pgfsetroundjoin%
\definecolor{currentfill}{rgb}{0.185556,0.418570,0.556753}%
\pgfsetfillcolor{currentfill}%
\pgfsetlinewidth{0.000000pt}%
\definecolor{currentstroke}{rgb}{0.154815,0.493313,0.557840}%
\pgfsetstrokecolor{currentstroke}%
\pgfsetdash{}{0pt}%
\pgfpathmoveto{\pgfqpoint{2.684453in}{3.009085in}}%
\pgfpathlineto{\pgfqpoint{2.751870in}{2.294211in}}%
\pgfpathlineto{\pgfqpoint{2.825356in}{2.965660in}}%
\pgfpathclose%
\pgfusepath{fill}%
\end{pgfscope}%
\begin{pgfscope}%
\pgfpathrectangle{\pgfqpoint{0.539299in}{0.078740in}}{\pgfqpoint{7.842520in}{7.842520in}}%
\pgfusepath{clip}%
\pgfsetbuttcap%
\pgfsetroundjoin%
\definecolor{currentfill}{rgb}{0.595839,0.848717,0.243329}%
\pgfsetfillcolor{currentfill}%
\pgfsetlinewidth{0.000000pt}%
\definecolor{currentstroke}{rgb}{0.153364,0.497000,0.557724}%
\pgfsetstrokecolor{currentstroke}%
\pgfsetdash{}{0pt}%
\pgfpathmoveto{\pgfqpoint{2.665187in}{5.062348in}}%
\pgfpathlineto{\pgfqpoint{2.528143in}{4.972174in}}%
\pgfpathlineto{\pgfqpoint{2.444220in}{4.720662in}}%
\pgfpathclose%
\pgfusepath{fill}%
\end{pgfscope}%
\begin{pgfscope}%
\pgfpathrectangle{\pgfqpoint{0.539299in}{0.078740in}}{\pgfqpoint{7.842520in}{7.842520in}}%
\pgfusepath{clip}%
\pgfsetbuttcap%
\pgfsetroundjoin%
\definecolor{currentfill}{rgb}{0.127568,0.566949,0.550556}%
\pgfsetfillcolor{currentfill}%
\pgfsetlinewidth{0.000000pt}%
\definecolor{currentstroke}{rgb}{0.151918,0.500685,0.557587}%
\pgfsetstrokecolor{currentstroke}%
\pgfsetdash{}{0pt}%
\pgfpathmoveto{\pgfqpoint{2.479796in}{3.628915in}}%
\pgfpathlineto{\pgfqpoint{2.405032in}{3.063501in}}%
\pgfpathlineto{\pgfqpoint{2.619520in}{3.617668in}}%
\pgfpathclose%
\pgfusepath{fill}%
\end{pgfscope}%
\begin{pgfscope}%
\pgfpathrectangle{\pgfqpoint{0.539299in}{0.078740in}}{\pgfqpoint{7.842520in}{7.842520in}}%
\pgfusepath{clip}%
\pgfsetbuttcap%
\pgfsetroundjoin%
\definecolor{currentfill}{rgb}{0.945636,0.899815,0.112838}%
\pgfsetfillcolor{currentfill}%
\pgfsetlinewidth{0.000000pt}%
\definecolor{currentstroke}{rgb}{0.150476,0.504369,0.557430}%
\pgfsetstrokecolor{currentstroke}%
\pgfsetdash{}{0pt}%
\pgfpathmoveto{\pgfqpoint{3.062895in}{5.558135in}}%
\pgfpathlineto{\pgfqpoint{2.836696in}{5.416830in}}%
\pgfpathlineto{\pgfqpoint{2.975959in}{5.457547in}}%
\pgfpathclose%
\pgfusepath{fill}%
\end{pgfscope}%
\begin{pgfscope}%
\pgfpathrectangle{\pgfqpoint{0.539299in}{0.078740in}}{\pgfqpoint{7.842520in}{7.842520in}}%
\pgfusepath{clip}%
\pgfsetbuttcap%
\pgfsetroundjoin%
\definecolor{currentfill}{rgb}{0.440137,0.811138,0.340967}%
\pgfsetfillcolor{currentfill}%
\pgfsetlinewidth{0.000000pt}%
\definecolor{currentstroke}{rgb}{0.149039,0.508051,0.557250}%
\pgfsetstrokecolor{currentstroke}%
\pgfsetdash{}{0pt}%
\pgfpathmoveto{\pgfqpoint{2.581236in}{4.802712in}}%
\pgfpathlineto{\pgfqpoint{2.444220in}{4.720662in}}%
\pgfpathlineto{\pgfqpoint{2.361965in}{4.410404in}}%
\pgfpathclose%
\pgfusepath{fill}%
\end{pgfscope}%
\begin{pgfscope}%
\pgfpathrectangle{\pgfqpoint{0.539299in}{0.078740in}}{\pgfqpoint{7.842520in}{7.842520in}}%
\pgfusepath{clip}%
\pgfsetbuttcap%
\pgfsetroundjoin%
\definecolor{currentfill}{rgb}{0.154815,0.493313,0.557840}%
\pgfsetfillcolor{currentfill}%
\pgfsetlinewidth{0.000000pt}%
\definecolor{currentstroke}{rgb}{0.147607,0.511733,0.557049}%
\pgfsetstrokecolor{currentstroke}%
\pgfsetdash{}{0pt}%
\pgfpathmoveto{\pgfqpoint{5.217726in}{3.002518in}}%
\pgfpathlineto{\pgfqpoint{5.074536in}{3.221540in}}%
\pgfpathlineto{\pgfqpoint{5.137323in}{3.030184in}}%
\pgfpathclose%
\pgfusepath{fill}%
\end{pgfscope}%
\begin{pgfscope}%
\pgfpathrectangle{\pgfqpoint{0.539299in}{0.078740in}}{\pgfqpoint{7.842520in}{7.842520in}}%
\pgfusepath{clip}%
\pgfsetbuttcap%
\pgfsetroundjoin%
\definecolor{currentfill}{rgb}{0.153894,0.680203,0.504172}%
\pgfsetfillcolor{currentfill}%
\pgfsetlinewidth{0.000000pt}%
\definecolor{currentstroke}{rgb}{0.146180,0.515413,0.556823}%
\pgfsetstrokecolor{currentstroke}%
\pgfsetdash{}{0pt}%
\pgfpathmoveto{\pgfqpoint{4.643709in}{3.889018in}}%
\pgfpathlineto{\pgfqpoint{4.726130in}{3.847156in}}%
\pgfpathlineto{\pgfqpoint{4.499656in}{4.110842in}}%
\pgfpathclose%
\pgfusepath{fill}%
\end{pgfscope}%
\begin{pgfscope}%
\pgfpathrectangle{\pgfqpoint{0.539299in}{0.078740in}}{\pgfqpoint{7.842520in}{7.842520in}}%
\pgfusepath{clip}%
\pgfsetbuttcap%
\pgfsetroundjoin%
\definecolor{currentfill}{rgb}{0.187231,0.414746,0.556547}%
\pgfsetfillcolor{currentfill}%
\pgfsetlinewidth{0.000000pt}%
\definecolor{currentstroke}{rgb}{0.144759,0.519093,0.556572}%
\pgfsetstrokecolor{currentstroke}%
\pgfsetdash{}{0pt}%
\pgfpathmoveto{\pgfqpoint{5.424487in}{2.595182in}}%
\pgfpathlineto{\pgfqpoint{5.360735in}{2.786461in}}%
\pgfpathlineto{\pgfqpoint{5.281003in}{2.812185in}}%
\pgfpathclose%
\pgfusepath{fill}%
\end{pgfscope}%
\begin{pgfscope}%
\pgfpathrectangle{\pgfqpoint{0.539299in}{0.078740in}}{\pgfqpoint{7.842520in}{7.842520in}}%
\pgfusepath{clip}%
\pgfsetbuttcap%
\pgfsetroundjoin%
\definecolor{currentfill}{rgb}{0.274128,0.199721,0.498911}%
\pgfsetfillcolor{currentfill}%
\pgfsetlinewidth{0.000000pt}%
\definecolor{currentstroke}{rgb}{0.143343,0.522773,0.556295}%
\pgfsetstrokecolor{currentstroke}%
\pgfsetdash{}{0pt}%
\pgfpathmoveto{\pgfqpoint{5.996574in}{1.738442in}}%
\pgfpathlineto{\pgfqpoint{5.853809in}{1.950429in}}%
\pgfpathlineto{\pgfqpoint{5.775575in}{1.941647in}}%
\pgfpathclose%
\pgfusepath{fill}%
\end{pgfscope}%
\begin{pgfscope}%
\pgfpathrectangle{\pgfqpoint{0.539299in}{0.078740in}}{\pgfqpoint{7.842520in}{7.842520in}}%
\pgfusepath{clip}%
\pgfsetbuttcap%
\pgfsetroundjoin%
\definecolor{currentfill}{rgb}{0.288921,0.758394,0.428426}%
\pgfsetfillcolor{currentfill}%
\pgfsetlinewidth{0.000000pt}%
\definecolor{currentstroke}{rgb}{0.141935,0.526453,0.555991}%
\pgfsetstrokecolor{currentstroke}%
\pgfsetdash{}{0pt}%
\pgfpathmoveto{\pgfqpoint{2.418849in}{4.087714in}}%
\pgfpathlineto{\pgfqpoint{2.498985in}{4.479903in}}%
\pgfpathlineto{\pgfqpoint{2.361965in}{4.410404in}}%
\pgfpathclose%
\pgfusepath{fill}%
\end{pgfscope}%
\begin{pgfscope}%
\pgfpathrectangle{\pgfqpoint{0.539299in}{0.078740in}}{\pgfqpoint{7.842520in}{7.842520in}}%
\pgfusepath{clip}%
\pgfsetbuttcap%
\pgfsetroundjoin%
\definecolor{currentfill}{rgb}{0.352360,0.783011,0.392636}%
\pgfsetfillcolor{currentfill}%
\pgfsetlinewidth{0.000000pt}%
\definecolor{currentstroke}{rgb}{0.140536,0.530132,0.555659}%
\pgfsetstrokecolor{currentstroke}%
\pgfsetdash{}{0pt}%
\pgfpathmoveto{\pgfqpoint{4.295430in}{4.515609in}}%
\pgfpathlineto{\pgfqpoint{4.210962in}{4.541645in}}%
\pgfpathlineto{\pgfqpoint{4.439213in}{4.296234in}}%
\pgfpathclose%
\pgfusepath{fill}%
\end{pgfscope}%
\begin{pgfscope}%
\pgfpathrectangle{\pgfqpoint{0.539299in}{0.078740in}}{\pgfqpoint{7.842520in}{7.842520in}}%
\pgfusepath{clip}%
\pgfsetbuttcap%
\pgfsetroundjoin%
\definecolor{currentfill}{rgb}{0.720391,0.870350,0.162603}%
\pgfsetfillcolor{currentfill}%
\pgfsetlinewidth{0.000000pt}%
\definecolor{currentstroke}{rgb}{0.139147,0.533812,0.555298}%
\pgfsetstrokecolor{currentstroke}%
\pgfsetdash{}{0pt}%
\pgfpathmoveto{\pgfqpoint{2.528143in}{4.972174in}}%
\pgfpathlineto{\pgfqpoint{2.665187in}{5.062348in}}%
\pgfpathlineto{\pgfqpoint{2.750456in}{5.265020in}}%
\pgfpathclose%
\pgfusepath{fill}%
\end{pgfscope}%
\begin{pgfscope}%
\pgfpathrectangle{\pgfqpoint{0.539299in}{0.078740in}}{\pgfqpoint{7.842520in}{7.842520in}}%
\pgfusepath{clip}%
\pgfsetbuttcap%
\pgfsetroundjoin%
\definecolor{currentfill}{rgb}{0.128087,0.647749,0.523491}%
\pgfsetfillcolor{currentfill}%
\pgfsetlinewidth{0.000000pt}%
\definecolor{currentstroke}{rgb}{0.137770,0.537492,0.554906}%
\pgfsetstrokecolor{currentstroke}%
\pgfsetdash{}{0pt}%
\pgfpathmoveto{\pgfqpoint{2.479796in}{3.628915in}}%
\pgfpathlineto{\pgfqpoint{2.557549in}{4.110498in}}%
\pgfpathlineto{\pgfqpoint{2.341276in}{3.620065in}}%
\pgfpathclose%
\pgfusepath{fill}%
\end{pgfscope}%
\begin{pgfscope}%
\pgfpathrectangle{\pgfqpoint{0.539299in}{0.078740in}}{\pgfqpoint{7.842520in}{7.842520in}}%
\pgfusepath{clip}%
\pgfsetbuttcap%
\pgfsetroundjoin%
\definecolor{currentfill}{rgb}{0.141935,0.526453,0.555991}%
\pgfsetfillcolor{currentfill}%
\pgfsetlinewidth{0.000000pt}%
\definecolor{currentstroke}{rgb}{0.136408,0.541173,0.554483}%
\pgfsetstrokecolor{currentstroke}%
\pgfsetdash{}{0pt}%
\pgfpathmoveto{\pgfqpoint{2.684453in}{3.009085in}}%
\pgfpathlineto{\pgfqpoint{2.619520in}{3.617668in}}%
\pgfpathlineto{\pgfqpoint{2.544311in}{3.042210in}}%
\pgfpathclose%
\pgfusepath{fill}%
\end{pgfscope}%
\begin{pgfscope}%
\pgfpathrectangle{\pgfqpoint{0.539299in}{0.078740in}}{\pgfqpoint{7.842520in}{7.842520in}}%
\pgfusepath{clip}%
\pgfsetbuttcap%
\pgfsetroundjoin%
\definecolor{currentfill}{rgb}{0.153894,0.680203,0.504172}%
\pgfsetfillcolor{currentfill}%
\pgfsetlinewidth{0.000000pt}%
\definecolor{currentstroke}{rgb}{0.135066,0.544853,0.554029}%
\pgfsetstrokecolor{currentstroke}%
\pgfsetdash{}{0pt}%
\pgfpathmoveto{\pgfqpoint{2.341276in}{3.620065in}}%
\pgfpathlineto{\pgfqpoint{2.557549in}{4.110498in}}%
\pgfpathlineto{\pgfqpoint{2.418849in}{4.087714in}}%
\pgfpathclose%
\pgfusepath{fill}%
\end{pgfscope}%
\begin{pgfscope}%
\pgfpathrectangle{\pgfqpoint{0.539299in}{0.078740in}}{\pgfqpoint{7.842520in}{7.842520in}}%
\pgfusepath{clip}%
\pgfsetbuttcap%
\pgfsetroundjoin%
\definecolor{currentfill}{rgb}{0.216210,0.351535,0.550627}%
\pgfsetfillcolor{currentfill}%
\pgfsetlinewidth{0.000000pt}%
\definecolor{currentstroke}{rgb}{0.133743,0.548535,0.553541}%
\pgfsetstrokecolor{currentstroke}%
\pgfsetdash{}{0pt}%
\pgfpathmoveto{\pgfqpoint{2.966934in}{2.913287in}}%
\pgfpathlineto{\pgfqpoint{2.892470in}{2.245647in}}%
\pgfpathlineto{\pgfqpoint{3.033612in}{2.193144in}}%
\pgfpathclose%
\pgfusepath{fill}%
\end{pgfscope}%
\begin{pgfscope}%
\pgfpathrectangle{\pgfqpoint{0.539299in}{0.078740in}}{\pgfqpoint{7.842520in}{7.842520in}}%
\pgfusepath{clip}%
\pgfsetbuttcap%
\pgfsetroundjoin%
\definecolor{currentfill}{rgb}{0.282884,0.135920,0.453427}%
\pgfsetfillcolor{currentfill}%
\pgfsetlinewidth{0.000000pt}%
\definecolor{currentstroke}{rgb}{0.132444,0.552216,0.553018}%
\pgfsetstrokecolor{currentstroke}%
\pgfsetdash{}{0pt}%
\pgfpathmoveto{\pgfqpoint{5.918659in}{1.713508in}}%
\pgfpathlineto{\pgfqpoint{6.061409in}{1.481685in}}%
\pgfpathlineto{\pgfqpoint{5.996574in}{1.738442in}}%
\pgfpathclose%
\pgfusepath{fill}%
\end{pgfscope}%
\begin{pgfscope}%
\pgfpathrectangle{\pgfqpoint{0.539299in}{0.078740in}}{\pgfqpoint{7.842520in}{7.842520in}}%
\pgfusepath{clip}%
\pgfsetbuttcap%
\pgfsetroundjoin%
\definecolor{currentfill}{rgb}{0.171176,0.452530,0.557965}%
\pgfsetfillcolor{currentfill}%
\pgfsetlinewidth{0.000000pt}%
\definecolor{currentstroke}{rgb}{0.131172,0.555899,0.552459}%
\pgfsetstrokecolor{currentstroke}%
\pgfsetdash{}{0pt}%
\pgfpathmoveto{\pgfqpoint{5.360735in}{2.786461in}}%
\pgfpathlineto{\pgfqpoint{5.137323in}{3.030184in}}%
\pgfpathlineto{\pgfqpoint{5.281003in}{2.812185in}}%
\pgfpathclose%
\pgfusepath{fill}%
\end{pgfscope}%
\begin{pgfscope}%
\pgfpathrectangle{\pgfqpoint{0.539299in}{0.078740in}}{\pgfqpoint{7.842520in}{7.842520in}}%
\pgfusepath{clip}%
\pgfsetbuttcap%
\pgfsetroundjoin%
\definecolor{currentfill}{rgb}{0.185556,0.418570,0.556753}%
\pgfsetfillcolor{currentfill}%
\pgfsetlinewidth{0.000000pt}%
\definecolor{currentstroke}{rgb}{0.129933,0.559582,0.551864}%
\pgfsetstrokecolor{currentstroke}%
\pgfsetdash{}{0pt}%
\pgfpathmoveto{\pgfqpoint{2.966934in}{2.913287in}}%
\pgfpathlineto{\pgfqpoint{2.825356in}{2.965660in}}%
\pgfpathlineto{\pgfqpoint{2.751870in}{2.294211in}}%
\pgfpathclose%
\pgfusepath{fill}%
\end{pgfscope}%
\begin{pgfscope}%
\pgfpathrectangle{\pgfqpoint{0.539299in}{0.078740in}}{\pgfqpoint{7.842520in}{7.842520in}}%
\pgfusepath{clip}%
\pgfsetbuttcap%
\pgfsetroundjoin%
\definecolor{currentfill}{rgb}{0.430983,0.808473,0.346476}%
\pgfsetfillcolor{currentfill}%
\pgfsetlinewidth{0.000000pt}%
\definecolor{currentstroke}{rgb}{0.128729,0.563265,0.551229}%
\pgfsetstrokecolor{currentstroke}%
\pgfsetdash{}{0pt}%
\pgfpathmoveto{\pgfqpoint{4.295430in}{4.515609in}}%
\pgfpathlineto{\pgfqpoint{4.151482in}{4.727445in}}%
\pgfpathlineto{\pgfqpoint{4.210962in}{4.541645in}}%
\pgfpathclose%
\pgfusepath{fill}%
\end{pgfscope}%
\begin{pgfscope}%
\pgfpathrectangle{\pgfqpoint{0.539299in}{0.078740in}}{\pgfqpoint{7.842520in}{7.842520in}}%
\pgfusepath{clip}%
\pgfsetbuttcap%
\pgfsetroundjoin%
\definecolor{currentfill}{rgb}{0.412913,0.803041,0.357269}%
\pgfsetfillcolor{currentfill}%
\pgfsetlinewidth{0.000000pt}%
\definecolor{currentstroke}{rgb}{0.127568,0.566949,0.550556}%
\pgfsetstrokecolor{currentstroke}%
\pgfsetdash{}{0pt}%
\pgfpathmoveto{\pgfqpoint{2.361965in}{4.410404in}}%
\pgfpathlineto{\pgfqpoint{2.498985in}{4.479903in}}%
\pgfpathlineto{\pgfqpoint{2.581236in}{4.802712in}}%
\pgfpathclose%
\pgfusepath{fill}%
\end{pgfscope}%
\begin{pgfscope}%
\pgfpathrectangle{\pgfqpoint{0.539299in}{0.078740in}}{\pgfqpoint{7.842520in}{7.842520in}}%
\pgfusepath{clip}%
\pgfsetbuttcap%
\pgfsetroundjoin%
\definecolor{currentfill}{rgb}{0.128729,0.563265,0.551229}%
\pgfsetfillcolor{currentfill}%
\pgfsetlinewidth{0.000000pt}%
\definecolor{currentstroke}{rgb}{0.126453,0.570633,0.549841}%
\pgfsetstrokecolor{currentstroke}%
\pgfsetdash{}{0pt}%
\pgfpathmoveto{\pgfqpoint{5.074536in}{3.221540in}}%
\pgfpathlineto{\pgfqpoint{4.931146in}{3.442939in}}%
\pgfpathlineto{\pgfqpoint{4.849351in}{3.468201in}}%
\pgfpathclose%
\pgfusepath{fill}%
\end{pgfscope}%
\begin{pgfscope}%
\pgfpathrectangle{\pgfqpoint{0.539299in}{0.078740in}}{\pgfqpoint{7.842520in}{7.842520in}}%
\pgfusepath{clip}%
\pgfsetbuttcap%
\pgfsetroundjoin%
\definecolor{currentfill}{rgb}{0.253935,0.265254,0.529983}%
\pgfsetfillcolor{currentfill}%
\pgfsetlinewidth{0.000000pt}%
\definecolor{currentstroke}{rgb}{0.125394,0.574318,0.549086}%
\pgfsetstrokecolor{currentstroke}%
\pgfsetdash{}{0pt}%
\pgfpathmoveto{\pgfqpoint{5.710883in}{2.164223in}}%
\pgfpathlineto{\pgfqpoint{5.632182in}{2.166073in}}%
\pgfpathlineto{\pgfqpoint{5.853809in}{1.950429in}}%
\pgfpathclose%
\pgfusepath{fill}%
\end{pgfscope}%
\begin{pgfscope}%
\pgfpathrectangle{\pgfqpoint{0.539299in}{0.078740in}}{\pgfqpoint{7.842520in}{7.842520in}}%
\pgfusepath{clip}%
\pgfsetbuttcap%
\pgfsetroundjoin%
\definecolor{currentfill}{rgb}{0.935904,0.898570,0.108131}%
\pgfsetfillcolor{currentfill}%
\pgfsetlinewidth{0.000000pt}%
\definecolor{currentstroke}{rgb}{0.124395,0.578002,0.548287}%
\pgfsetstrokecolor{currentstroke}%
\pgfsetdash{}{0pt}%
\pgfpathmoveto{\pgfqpoint{3.489230in}{5.389845in}}%
\pgfpathlineto{\pgfqpoint{3.576041in}{5.415869in}}%
\pgfpathlineto{\pgfqpoint{3.433110in}{5.522817in}}%
\pgfpathclose%
\pgfusepath{fill}%
\end{pgfscope}%
\begin{pgfscope}%
\pgfpathrectangle{\pgfqpoint{0.539299in}{0.078740in}}{\pgfqpoint{7.842520in}{7.842520in}}%
\pgfusepath{clip}%
\pgfsetbuttcap%
\pgfsetroundjoin%
\definecolor{currentfill}{rgb}{0.585678,0.846661,0.249897}%
\pgfsetfillcolor{currentfill}%
\pgfsetlinewidth{0.000000pt}%
\definecolor{currentstroke}{rgb}{0.123463,0.581687,0.547445}%
\pgfsetstrokecolor{currentstroke}%
\pgfsetdash{}{0pt}%
\pgfpathmoveto{\pgfqpoint{2.581236in}{4.802712in}}%
\pgfpathlineto{\pgfqpoint{2.665187in}{5.062348in}}%
\pgfpathlineto{\pgfqpoint{2.444220in}{4.720662in}}%
\pgfpathclose%
\pgfusepath{fill}%
\end{pgfscope}%
\begin{pgfscope}%
\pgfpathrectangle{\pgfqpoint{0.539299in}{0.078740in}}{\pgfqpoint{7.842520in}{7.842520in}}%
\pgfusepath{clip}%
\pgfsetbuttcap%
\pgfsetroundjoin%
\definecolor{currentfill}{rgb}{0.983868,0.904867,0.136897}%
\pgfsetfillcolor{currentfill}%
\pgfsetlinewidth{0.000000pt}%
\definecolor{currentstroke}{rgb}{0.122606,0.585371,0.546557}%
\pgfsetstrokecolor{currentstroke}%
\pgfsetdash{}{0pt}%
\pgfpathmoveto{\pgfqpoint{3.346062in}{5.485171in}}%
\pgfpathlineto{\pgfqpoint{3.433110in}{5.522817in}}%
\pgfpathlineto{\pgfqpoint{3.203841in}{5.543255in}}%
\pgfpathclose%
\pgfusepath{fill}%
\end{pgfscope}%
\begin{pgfscope}%
\pgfpathrectangle{\pgfqpoint{0.539299in}{0.078740in}}{\pgfqpoint{7.842520in}{7.842520in}}%
\pgfusepath{clip}%
\pgfsetbuttcap%
\pgfsetroundjoin%
\definecolor{currentfill}{rgb}{0.876168,0.891125,0.095250}%
\pgfsetfillcolor{currentfill}%
\pgfsetlinewidth{0.000000pt}%
\definecolor{currentstroke}{rgb}{0.121831,0.589055,0.545623}%
\pgfsetstrokecolor{currentstroke}%
\pgfsetdash{}{0pt}%
\pgfpathmoveto{\pgfqpoint{2.836696in}{5.416830in}}%
\pgfpathlineto{\pgfqpoint{2.750456in}{5.265020in}}%
\pgfpathlineto{\pgfqpoint{2.889650in}{5.309100in}}%
\pgfpathclose%
\pgfusepath{fill}%
\end{pgfscope}%
\begin{pgfscope}%
\pgfpathrectangle{\pgfqpoint{0.539299in}{0.078740in}}{\pgfqpoint{7.842520in}{7.842520in}}%
\pgfusepath{clip}%
\pgfsetbuttcap%
\pgfsetroundjoin%
\definecolor{currentfill}{rgb}{0.259857,0.745492,0.444467}%
\pgfsetfillcolor{currentfill}%
\pgfsetlinewidth{0.000000pt}%
\definecolor{currentstroke}{rgb}{0.121148,0.592739,0.544641}%
\pgfsetstrokecolor{currentstroke}%
\pgfsetdash{}{0pt}%
\pgfpathmoveto{\pgfqpoint{4.499656in}{4.110842in}}%
\pgfpathlineto{\pgfqpoint{4.439213in}{4.296234in}}%
\pgfpathlineto{\pgfqpoint{4.355396in}{4.329235in}}%
\pgfpathclose%
\pgfusepath{fill}%
\end{pgfscope}%
\begin{pgfscope}%
\pgfpathrectangle{\pgfqpoint{0.539299in}{0.078740in}}{\pgfqpoint{7.842520in}{7.842520in}}%
\pgfusepath{clip}%
\pgfsetbuttcap%
\pgfsetroundjoin%
\definecolor{currentfill}{rgb}{0.277134,0.185228,0.489898}%
\pgfsetfillcolor{currentfill}%
\pgfsetlinewidth{0.000000pt}%
\definecolor{currentstroke}{rgb}{0.120565,0.596422,0.543611}%
\pgfsetstrokecolor{currentstroke}%
\pgfsetdash{}{0pt}%
\pgfpathmoveto{\pgfqpoint{5.996574in}{1.738442in}}%
\pgfpathlineto{\pgfqpoint{5.775575in}{1.941647in}}%
\pgfpathlineto{\pgfqpoint{5.918659in}{1.713508in}}%
\pgfpathclose%
\pgfusepath{fill}%
\end{pgfscope}%
\begin{pgfscope}%
\pgfpathrectangle{\pgfqpoint{0.539299in}{0.078740in}}{\pgfqpoint{7.842520in}{7.842520in}}%
\pgfusepath{clip}%
\pgfsetbuttcap%
\pgfsetroundjoin%
\definecolor{currentfill}{rgb}{0.555484,0.840254,0.269281}%
\pgfsetfillcolor{currentfill}%
\pgfsetlinewidth{0.000000pt}%
\definecolor{currentstroke}{rgb}{0.120092,0.600104,0.542530}%
\pgfsetstrokecolor{currentstroke}%
\pgfsetdash{}{0pt}%
\pgfpathmoveto{\pgfqpoint{4.066411in}{4.745025in}}%
\pgfpathlineto{\pgfqpoint{4.151482in}{4.727445in}}%
\pgfpathlineto{\pgfqpoint{4.007443in}{4.927974in}}%
\pgfpathclose%
\pgfusepath{fill}%
\end{pgfscope}%
\begin{pgfscope}%
\pgfpathrectangle{\pgfqpoint{0.539299in}{0.078740in}}{\pgfqpoint{7.842520in}{7.842520in}}%
\pgfusepath{clip}%
\pgfsetbuttcap%
\pgfsetroundjoin%
\definecolor{currentfill}{rgb}{0.916242,0.896091,0.100717}%
\pgfsetfillcolor{currentfill}%
\pgfsetlinewidth{0.000000pt}%
\definecolor{currentstroke}{rgb}{0.119738,0.603785,0.541400}%
\pgfsetstrokecolor{currentstroke}%
\pgfsetdash{}{0pt}%
\pgfpathmoveto{\pgfqpoint{2.975959in}{5.457547in}}%
\pgfpathlineto{\pgfqpoint{2.836696in}{5.416830in}}%
\pgfpathlineto{\pgfqpoint{2.889650in}{5.309100in}}%
\pgfpathclose%
\pgfusepath{fill}%
\end{pgfscope}%
\begin{pgfscope}%
\pgfpathrectangle{\pgfqpoint{0.539299in}{0.078740in}}{\pgfqpoint{7.842520in}{7.842520in}}%
\pgfusepath{clip}%
\pgfsetbuttcap%
\pgfsetroundjoin%
\definecolor{currentfill}{rgb}{0.896320,0.893616,0.096335}%
\pgfsetfillcolor{currentfill}%
\pgfsetlinewidth{0.000000pt}%
\definecolor{currentstroke}{rgb}{0.119512,0.607464,0.540218}%
\pgfsetstrokecolor{currentstroke}%
\pgfsetdash{}{0pt}%
\pgfpathmoveto{\pgfqpoint{3.719555in}{5.277291in}}%
\pgfpathlineto{\pgfqpoint{3.576041in}{5.415869in}}%
\pgfpathlineto{\pgfqpoint{3.489230in}{5.389845in}}%
\pgfpathclose%
\pgfusepath{fill}%
\end{pgfscope}%
\begin{pgfscope}%
\pgfpathrectangle{\pgfqpoint{0.539299in}{0.078740in}}{\pgfqpoint{7.842520in}{7.842520in}}%
\pgfusepath{clip}%
\pgfsetbuttcap%
\pgfsetroundjoin%
\definecolor{currentfill}{rgb}{0.983868,0.904867,0.136897}%
\pgfsetfillcolor{currentfill}%
\pgfsetlinewidth{0.000000pt}%
\definecolor{currentstroke}{rgb}{0.119423,0.611141,0.538982}%
\pgfsetstrokecolor{currentstroke}%
\pgfsetdash{}{0pt}%
\pgfpathmoveto{\pgfqpoint{3.116915in}{5.451034in}}%
\pgfpathlineto{\pgfqpoint{3.203841in}{5.543255in}}%
\pgfpathlineto{\pgfqpoint{3.062895in}{5.558135in}}%
\pgfpathclose%
\pgfusepath{fill}%
\end{pgfscope}%
\begin{pgfscope}%
\pgfpathrectangle{\pgfqpoint{0.539299in}{0.078740in}}{\pgfqpoint{7.842520in}{7.842520in}}%
\pgfusepath{clip}%
\pgfsetbuttcap%
\pgfsetroundjoin%
\definecolor{currentfill}{rgb}{0.237441,0.305202,0.541921}%
\pgfsetfillcolor{currentfill}%
\pgfsetlinewidth{0.000000pt}%
\definecolor{currentstroke}{rgb}{0.119483,0.614817,0.537692}%
\pgfsetstrokecolor{currentstroke}%
\pgfsetdash{}{0pt}%
\pgfpathmoveto{\pgfqpoint{5.567779in}{2.379196in}}%
\pgfpathlineto{\pgfqpoint{5.632182in}{2.166073in}}%
\pgfpathlineto{\pgfqpoint{5.710883in}{2.164223in}}%
\pgfpathclose%
\pgfusepath{fill}%
\end{pgfscope}%
\begin{pgfscope}%
\pgfpathrectangle{\pgfqpoint{0.539299in}{0.078740in}}{\pgfqpoint{7.842520in}{7.842520in}}%
\pgfusepath{clip}%
\pgfsetbuttcap%
\pgfsetroundjoin%
\definecolor{currentfill}{rgb}{0.120565,0.596422,0.543611}%
\pgfsetfillcolor{currentfill}%
\pgfsetlinewidth{0.000000pt}%
\definecolor{currentstroke}{rgb}{0.119699,0.618490,0.536347}%
\pgfsetstrokecolor{currentstroke}%
\pgfsetdash{}{0pt}%
\pgfpathmoveto{\pgfqpoint{4.849351in}{3.468201in}}%
\pgfpathlineto{\pgfqpoint{4.931146in}{3.442939in}}%
\pgfpathlineto{\pgfqpoint{4.787540in}{3.665841in}}%
\pgfpathclose%
\pgfusepath{fill}%
\end{pgfscope}%
\begin{pgfscope}%
\pgfpathrectangle{\pgfqpoint{0.539299in}{0.078740in}}{\pgfqpoint{7.842520in}{7.842520in}}%
\pgfusepath{clip}%
\pgfsetbuttcap%
\pgfsetroundjoin%
\definecolor{currentfill}{rgb}{0.335885,0.777018,0.402049}%
\pgfsetfillcolor{currentfill}%
\pgfsetlinewidth{0.000000pt}%
\definecolor{currentstroke}{rgb}{0.120081,0.622161,0.534946}%
\pgfsetstrokecolor{currentstroke}%
\pgfsetdash{}{0pt}%
\pgfpathmoveto{\pgfqpoint{4.439213in}{4.296234in}}%
\pgfpathlineto{\pgfqpoint{4.210962in}{4.541645in}}%
\pgfpathlineto{\pgfqpoint{4.355396in}{4.329235in}}%
\pgfpathclose%
\pgfusepath{fill}%
\end{pgfscope}%
\begin{pgfscope}%
\pgfpathrectangle{\pgfqpoint{0.539299in}{0.078740in}}{\pgfqpoint{7.842520in}{7.842520in}}%
\pgfusepath{clip}%
\pgfsetbuttcap%
\pgfsetroundjoin%
\definecolor{currentfill}{rgb}{0.974417,0.903590,0.130215}%
\pgfsetfillcolor{currentfill}%
\pgfsetlinewidth{0.000000pt}%
\definecolor{currentstroke}{rgb}{0.120638,0.625828,0.533488}%
\pgfsetstrokecolor{currentstroke}%
\pgfsetdash{}{0pt}%
\pgfpathmoveto{\pgfqpoint{3.116915in}{5.451034in}}%
\pgfpathlineto{\pgfqpoint{3.062895in}{5.558135in}}%
\pgfpathlineto{\pgfqpoint{2.975959in}{5.457547in}}%
\pgfpathclose%
\pgfusepath{fill}%
\end{pgfscope}%
\begin{pgfscope}%
\pgfpathrectangle{\pgfqpoint{0.539299in}{0.078740in}}{\pgfqpoint{7.842520in}{7.842520in}}%
\pgfusepath{clip}%
\pgfsetbuttcap%
\pgfsetroundjoin%
\definecolor{currentfill}{rgb}{0.144759,0.519093,0.556572}%
\pgfsetfillcolor{currentfill}%
\pgfsetlinewidth{0.000000pt}%
\definecolor{currentstroke}{rgb}{0.121380,0.629492,0.531973}%
\pgfsetstrokecolor{currentstroke}%
\pgfsetdash{}{0pt}%
\pgfpathmoveto{\pgfqpoint{5.137323in}{3.030184in}}%
\pgfpathlineto{\pgfqpoint{5.074536in}{3.221540in}}%
\pgfpathlineto{\pgfqpoint{4.993441in}{3.248999in}}%
\pgfpathclose%
\pgfusepath{fill}%
\end{pgfscope}%
\begin{pgfscope}%
\pgfpathrectangle{\pgfqpoint{0.539299in}{0.078740in}}{\pgfqpoint{7.842520in}{7.842520in}}%
\pgfusepath{clip}%
\pgfsetbuttcap%
\pgfsetroundjoin%
\definecolor{currentfill}{rgb}{0.709898,0.868751,0.169257}%
\pgfsetfillcolor{currentfill}%
\pgfsetlinewidth{0.000000pt}%
\definecolor{currentstroke}{rgb}{0.122312,0.633153,0.530398}%
\pgfsetstrokecolor{currentstroke}%
\pgfsetdash{}{0pt}%
\pgfpathmoveto{\pgfqpoint{3.863419in}{5.112875in}}%
\pgfpathlineto{\pgfqpoint{3.777327in}{5.109990in}}%
\pgfpathlineto{\pgfqpoint{4.007443in}{4.927974in}}%
\pgfpathclose%
\pgfusepath{fill}%
\end{pgfscope}%
\begin{pgfscope}%
\pgfpathrectangle{\pgfqpoint{0.539299in}{0.078740in}}{\pgfqpoint{7.842520in}{7.842520in}}%
\pgfusepath{clip}%
\pgfsetbuttcap%
\pgfsetroundjoin%
\definecolor{currentfill}{rgb}{0.783315,0.879285,0.125405}%
\pgfsetfillcolor{currentfill}%
\pgfsetlinewidth{0.000000pt}%
\definecolor{currentstroke}{rgb}{0.123444,0.636809,0.528763}%
\pgfsetstrokecolor{currentstroke}%
\pgfsetdash{}{0pt}%
\pgfpathmoveto{\pgfqpoint{3.777327in}{5.109990in}}%
\pgfpathlineto{\pgfqpoint{3.863419in}{5.112875in}}%
\pgfpathlineto{\pgfqpoint{3.719555in}{5.277291in}}%
\pgfpathclose%
\pgfusepath{fill}%
\end{pgfscope}%
\begin{pgfscope}%
\pgfpathrectangle{\pgfqpoint{0.539299in}{0.078740in}}{\pgfqpoint{7.842520in}{7.842520in}}%
\pgfusepath{clip}%
\pgfsetbuttcap%
\pgfsetroundjoin%
\definecolor{currentfill}{rgb}{0.964894,0.902323,0.123941}%
\pgfsetfillcolor{currentfill}%
\pgfsetlinewidth{0.000000pt}%
\definecolor{currentstroke}{rgb}{0.124780,0.640461,0.527068}%
\pgfsetstrokecolor{currentstroke}%
\pgfsetdash{}{0pt}%
\pgfpathmoveto{\pgfqpoint{3.489230in}{5.389845in}}%
\pgfpathlineto{\pgfqpoint{3.433110in}{5.522817in}}%
\pgfpathlineto{\pgfqpoint{3.346062in}{5.485171in}}%
\pgfpathclose%
\pgfusepath{fill}%
\end{pgfscope}%
\begin{pgfscope}%
\pgfpathrectangle{\pgfqpoint{0.539299in}{0.078740in}}{\pgfqpoint{7.842520in}{7.842520in}}%
\pgfusepath{clip}%
\pgfsetbuttcap%
\pgfsetroundjoin%
\definecolor{currentfill}{rgb}{0.132268,0.655014,0.519661}%
\pgfsetfillcolor{currentfill}%
\pgfsetlinewidth{0.000000pt}%
\definecolor{currentstroke}{rgb}{0.126326,0.644107,0.525311}%
\pgfsetstrokecolor{currentstroke}%
\pgfsetdash{}{0pt}%
\pgfpathmoveto{\pgfqpoint{2.619520in}{3.617668in}}%
\pgfpathlineto{\pgfqpoint{2.557549in}{4.110498in}}%
\pgfpathlineto{\pgfqpoint{2.479796in}{3.628915in}}%
\pgfpathclose%
\pgfusepath{fill}%
\end{pgfscope}%
\begin{pgfscope}%
\pgfpathrectangle{\pgfqpoint{0.539299in}{0.078740in}}{\pgfqpoint{7.842520in}{7.842520in}}%
\pgfusepath{clip}%
\pgfsetbuttcap%
\pgfsetroundjoin%
\definecolor{currentfill}{rgb}{0.258965,0.251537,0.524736}%
\pgfsetfillcolor{currentfill}%
\pgfsetlinewidth{0.000000pt}%
\definecolor{currentstroke}{rgb}{0.128087,0.647749,0.523491}%
\pgfsetstrokecolor{currentstroke}%
\pgfsetdash{}{0pt}%
\pgfpathmoveto{\pgfqpoint{5.853809in}{1.950429in}}%
\pgfpathlineto{\pgfqpoint{5.632182in}{2.166073in}}%
\pgfpathlineto{\pgfqpoint{5.775575in}{1.941647in}}%
\pgfpathclose%
\pgfusepath{fill}%
\end{pgfscope}%
\begin{pgfscope}%
\pgfpathrectangle{\pgfqpoint{0.539299in}{0.078740in}}{\pgfqpoint{7.842520in}{7.842520in}}%
\pgfusepath{clip}%
\pgfsetbuttcap%
\pgfsetroundjoin%
\definecolor{currentfill}{rgb}{0.814576,0.883393,0.110347}%
\pgfsetfillcolor{currentfill}%
\pgfsetlinewidth{0.000000pt}%
\definecolor{currentstroke}{rgb}{0.130067,0.651384,0.521608}%
\pgfsetstrokecolor{currentstroke}%
\pgfsetdash{}{0pt}%
\pgfpathmoveto{\pgfqpoint{2.750456in}{5.265020in}}%
\pgfpathlineto{\pgfqpoint{2.665187in}{5.062348in}}%
\pgfpathlineto{\pgfqpoint{2.889650in}{5.309100in}}%
\pgfpathclose%
\pgfusepath{fill}%
\end{pgfscope}%
\begin{pgfscope}%
\pgfpathrectangle{\pgfqpoint{0.539299in}{0.078740in}}{\pgfqpoint{7.842520in}{7.842520in}}%
\pgfusepath{clip}%
\pgfsetbuttcap%
\pgfsetroundjoin%
\definecolor{currentfill}{rgb}{0.210503,0.363727,0.552206}%
\pgfsetfillcolor{currentfill}%
\pgfsetlinewidth{0.000000pt}%
\definecolor{currentstroke}{rgb}{0.132268,0.655014,0.519661}%
\pgfsetstrokecolor{currentstroke}%
\pgfsetdash{}{0pt}%
\pgfpathmoveto{\pgfqpoint{5.488510in}{2.387296in}}%
\pgfpathlineto{\pgfqpoint{5.567779in}{2.379196in}}%
\pgfpathlineto{\pgfqpoint{5.424487in}{2.595182in}}%
\pgfpathclose%
\pgfusepath{fill}%
\end{pgfscope}%
\begin{pgfscope}%
\pgfpathrectangle{\pgfqpoint{0.539299in}{0.078740in}}{\pgfqpoint{7.842520in}{7.842520in}}%
\pgfusepath{clip}%
\pgfsetbuttcap%
\pgfsetroundjoin%
\definecolor{currentfill}{rgb}{0.131172,0.555899,0.552459}%
\pgfsetfillcolor{currentfill}%
\pgfsetlinewidth{0.000000pt}%
\definecolor{currentstroke}{rgb}{0.134692,0.658636,0.517649}%
\pgfsetstrokecolor{currentstroke}%
\pgfsetdash{}{0pt}%
\pgfpathmoveto{\pgfqpoint{4.993441in}{3.248999in}}%
\pgfpathlineto{\pgfqpoint{5.074536in}{3.221540in}}%
\pgfpathlineto{\pgfqpoint{4.849351in}{3.468201in}}%
\pgfpathclose%
\pgfusepath{fill}%
\end{pgfscope}%
\begin{pgfscope}%
\pgfpathrectangle{\pgfqpoint{0.539299in}{0.078740in}}{\pgfqpoint{7.842520in}{7.842520in}}%
\pgfusepath{clip}%
\pgfsetbuttcap%
\pgfsetroundjoin%
\definecolor{currentfill}{rgb}{0.130067,0.651384,0.521608}%
\pgfsetfillcolor{currentfill}%
\pgfsetlinewidth{0.000000pt}%
\definecolor{currentstroke}{rgb}{0.137339,0.662252,0.515571}%
\pgfsetstrokecolor{currentstroke}%
\pgfsetdash{}{0pt}%
\pgfpathmoveto{\pgfqpoint{4.705051in}{3.687048in}}%
\pgfpathlineto{\pgfqpoint{4.787540in}{3.665841in}}%
\pgfpathlineto{\pgfqpoint{4.643709in}{3.889018in}}%
\pgfpathclose%
\pgfusepath{fill}%
\end{pgfscope}%
\begin{pgfscope}%
\pgfpathrectangle{\pgfqpoint{0.539299in}{0.078740in}}{\pgfqpoint{7.842520in}{7.842520in}}%
\pgfusepath{clip}%
\pgfsetbuttcap%
\pgfsetroundjoin%
\definecolor{currentfill}{rgb}{0.327796,0.773980,0.406640}%
\pgfsetfillcolor{currentfill}%
\pgfsetlinewidth{0.000000pt}%
\definecolor{currentstroke}{rgb}{0.140210,0.665859,0.513427}%
\pgfsetstrokecolor{currentstroke}%
\pgfsetdash{}{0pt}%
\pgfpathmoveto{\pgfqpoint{2.637836in}{4.513404in}}%
\pgfpathlineto{\pgfqpoint{2.498985in}{4.479903in}}%
\pgfpathlineto{\pgfqpoint{2.418849in}{4.087714in}}%
\pgfpathclose%
\pgfusepath{fill}%
\end{pgfscope}%
\begin{pgfscope}%
\pgfpathrectangle{\pgfqpoint{0.539299in}{0.078740in}}{\pgfqpoint{7.842520in}{7.842520in}}%
\pgfusepath{clip}%
\pgfsetbuttcap%
\pgfsetroundjoin%
\definecolor{currentfill}{rgb}{0.487026,0.823929,0.312321}%
\pgfsetfillcolor{currentfill}%
\pgfsetlinewidth{0.000000pt}%
\definecolor{currentstroke}{rgb}{0.143303,0.669459,0.511215}%
\pgfsetstrokecolor{currentstroke}%
\pgfsetdash{}{0pt}%
\pgfpathmoveto{\pgfqpoint{4.210962in}{4.541645in}}%
\pgfpathlineto{\pgfqpoint{4.151482in}{4.727445in}}%
\pgfpathlineto{\pgfqpoint{4.066411in}{4.745025in}}%
\pgfpathclose%
\pgfusepath{fill}%
\end{pgfscope}%
\begin{pgfscope}%
\pgfpathrectangle{\pgfqpoint{0.539299in}{0.078740in}}{\pgfqpoint{7.842520in}{7.842520in}}%
\pgfusepath{clip}%
\pgfsetbuttcap%
\pgfsetroundjoin%
\definecolor{currentfill}{rgb}{0.125394,0.574318,0.549086}%
\pgfsetfillcolor{currentfill}%
\pgfsetlinewidth{0.000000pt}%
\definecolor{currentstroke}{rgb}{0.146616,0.673050,0.508936}%
\pgfsetstrokecolor{currentstroke}%
\pgfsetdash{}{0pt}%
\pgfpathmoveto{\pgfqpoint{2.760273in}{3.588885in}}%
\pgfpathlineto{\pgfqpoint{2.619520in}{3.617668in}}%
\pgfpathlineto{\pgfqpoint{2.684453in}{3.009085in}}%
\pgfpathclose%
\pgfusepath{fill}%
\end{pgfscope}%
\begin{pgfscope}%
\pgfpathrectangle{\pgfqpoint{0.539299in}{0.078740in}}{\pgfqpoint{7.842520in}{7.842520in}}%
\pgfusepath{clip}%
\pgfsetbuttcap%
\pgfsetroundjoin%
\definecolor{currentfill}{rgb}{0.218130,0.347432,0.550038}%
\pgfsetfillcolor{currentfill}%
\pgfsetlinewidth{0.000000pt}%
\definecolor{currentstroke}{rgb}{0.150148,0.676631,0.506589}%
\pgfsetstrokecolor{currentstroke}%
\pgfsetdash{}{0pt}%
\pgfpathmoveto{\pgfqpoint{3.175266in}{2.137225in}}%
\pgfpathlineto{\pgfqpoint{3.251833in}{2.786336in}}%
\pgfpathlineto{\pgfqpoint{3.033612in}{2.193144in}}%
\pgfpathclose%
\pgfusepath{fill}%
\end{pgfscope}%
\begin{pgfscope}%
\pgfpathrectangle{\pgfqpoint{0.539299in}{0.078740in}}{\pgfqpoint{7.842520in}{7.842520in}}%
\pgfusepath{clip}%
\pgfsetbuttcap%
\pgfsetroundjoin%
\definecolor{currentfill}{rgb}{0.274149,0.751988,0.436601}%
\pgfsetfillcolor{currentfill}%
\pgfsetlinewidth{0.000000pt}%
\definecolor{currentstroke}{rgb}{0.153894,0.680203,0.504172}%
\pgfsetstrokecolor{currentstroke}%
\pgfsetdash{}{0pt}%
\pgfpathmoveto{\pgfqpoint{2.418849in}{4.087714in}}%
\pgfpathlineto{\pgfqpoint{2.557549in}{4.110498in}}%
\pgfpathlineto{\pgfqpoint{2.637836in}{4.513404in}}%
\pgfpathclose%
\pgfusepath{fill}%
\end{pgfscope}%
\begin{pgfscope}%
\pgfpathrectangle{\pgfqpoint{0.539299in}{0.078740in}}{\pgfqpoint{7.842520in}{7.842520in}}%
\pgfusepath{clip}%
\pgfsetbuttcap%
\pgfsetroundjoin%
\definecolor{currentfill}{rgb}{0.283229,0.120777,0.440584}%
\pgfsetfillcolor{currentfill}%
\pgfsetlinewidth{0.000000pt}%
\definecolor{currentstroke}{rgb}{0.157851,0.683765,0.501686}%
\pgfsetstrokecolor{currentstroke}%
\pgfsetdash{}{0pt}%
\pgfpathmoveto{\pgfqpoint{5.983060in}{1.432854in}}%
\pgfpathlineto{\pgfqpoint{6.061409in}{1.481685in}}%
\pgfpathlineto{\pgfqpoint{5.918659in}{1.713508in}}%
\pgfpathclose%
\pgfusepath{fill}%
\end{pgfscope}%
\begin{pgfscope}%
\pgfpathrectangle{\pgfqpoint{0.539299in}{0.078740in}}{\pgfqpoint{7.842520in}{7.842520in}}%
\pgfusepath{clip}%
\pgfsetbuttcap%
\pgfsetroundjoin%
\definecolor{currentfill}{rgb}{0.886271,0.892374,0.095374}%
\pgfsetfillcolor{currentfill}%
\pgfsetlinewidth{0.000000pt}%
\definecolor{currentstroke}{rgb}{0.162016,0.687316,0.499129}%
\pgfsetstrokecolor{currentstroke}%
\pgfsetdash{}{0pt}%
\pgfpathmoveto{\pgfqpoint{3.489230in}{5.389845in}}%
\pgfpathlineto{\pgfqpoint{3.633064in}{5.262991in}}%
\pgfpathlineto{\pgfqpoint{3.719555in}{5.277291in}}%
\pgfpathclose%
\pgfusepath{fill}%
\end{pgfscope}%
\begin{pgfscope}%
\pgfpathrectangle{\pgfqpoint{0.539299in}{0.078740in}}{\pgfqpoint{7.842520in}{7.842520in}}%
\pgfusepath{clip}%
\pgfsetbuttcap%
\pgfsetroundjoin%
\definecolor{currentfill}{rgb}{0.141935,0.526453,0.555991}%
\pgfsetfillcolor{currentfill}%
\pgfsetlinewidth{0.000000pt}%
\definecolor{currentstroke}{rgb}{0.166383,0.690856,0.496502}%
\pgfsetstrokecolor{currentstroke}%
\pgfsetdash{}{0pt}%
\pgfpathmoveto{\pgfqpoint{2.684453in}{3.009085in}}%
\pgfpathlineto{\pgfqpoint{2.825356in}{2.965660in}}%
\pgfpathlineto{\pgfqpoint{2.901907in}{3.544823in}}%
\pgfpathclose%
\pgfusepath{fill}%
\end{pgfscope}%
\begin{pgfscope}%
\pgfpathrectangle{\pgfqpoint{0.539299in}{0.078740in}}{\pgfqpoint{7.842520in}{7.842520in}}%
\pgfusepath{clip}%
\pgfsetbuttcap%
\pgfsetroundjoin%
\definecolor{currentfill}{rgb}{0.616293,0.852709,0.230052}%
\pgfsetfillcolor{currentfill}%
\pgfsetlinewidth{0.000000pt}%
\definecolor{currentstroke}{rgb}{0.170948,0.694384,0.493803}%
\pgfsetstrokecolor{currentstroke}%
\pgfsetdash{}{0pt}%
\pgfpathmoveto{\pgfqpoint{4.007443in}{4.927974in}}%
\pgfpathlineto{\pgfqpoint{3.921827in}{4.935824in}}%
\pgfpathlineto{\pgfqpoint{4.066411in}{4.745025in}}%
\pgfpathclose%
\pgfusepath{fill}%
\end{pgfscope}%
\begin{pgfscope}%
\pgfpathrectangle{\pgfqpoint{0.539299in}{0.078740in}}{\pgfqpoint{7.842520in}{7.842520in}}%
\pgfusepath{clip}%
\pgfsetbuttcap%
\pgfsetroundjoin%
\definecolor{currentfill}{rgb}{0.187231,0.414746,0.556547}%
\pgfsetfillcolor{currentfill}%
\pgfsetlinewidth{0.000000pt}%
\definecolor{currentstroke}{rgb}{0.175707,0.697900,0.491033}%
\pgfsetstrokecolor{currentstroke}%
\pgfsetdash{}{0pt}%
\pgfpathmoveto{\pgfqpoint{3.033612in}{2.193144in}}%
\pgfpathlineto{\pgfqpoint{3.109114in}{2.853159in}}%
\pgfpathlineto{\pgfqpoint{2.966934in}{2.913287in}}%
\pgfpathclose%
\pgfusepath{fill}%
\end{pgfscope}%
\begin{pgfscope}%
\pgfpathrectangle{\pgfqpoint{0.539299in}{0.078740in}}{\pgfqpoint{7.842520in}{7.842520in}}%
\pgfusepath{clip}%
\pgfsetbuttcap%
\pgfsetroundjoin%
\definecolor{currentfill}{rgb}{0.458674,0.816363,0.329727}%
\pgfsetfillcolor{currentfill}%
\pgfsetlinewidth{0.000000pt}%
\definecolor{currentstroke}{rgb}{0.180653,0.701402,0.488189}%
\pgfsetstrokecolor{currentstroke}%
\pgfsetdash{}{0pt}%
\pgfpathmoveto{\pgfqpoint{2.581236in}{4.802712in}}%
\pgfpathlineto{\pgfqpoint{2.498985in}{4.479903in}}%
\pgfpathlineto{\pgfqpoint{2.637836in}{4.513404in}}%
\pgfpathclose%
\pgfusepath{fill}%
\end{pgfscope}%
\begin{pgfscope}%
\pgfpathrectangle{\pgfqpoint{0.539299in}{0.078740in}}{\pgfqpoint{7.842520in}{7.842520in}}%
\pgfusepath{clip}%
\pgfsetbuttcap%
\pgfsetroundjoin%
\definecolor{currentfill}{rgb}{0.688944,0.865448,0.182725}%
\pgfsetfillcolor{currentfill}%
\pgfsetlinewidth{0.000000pt}%
\definecolor{currentstroke}{rgb}{0.185783,0.704891,0.485273}%
\pgfsetstrokecolor{currentstroke}%
\pgfsetdash{}{0pt}%
\pgfpathmoveto{\pgfqpoint{3.777327in}{5.109990in}}%
\pgfpathlineto{\pgfqpoint{3.921827in}{4.935824in}}%
\pgfpathlineto{\pgfqpoint{4.007443in}{4.927974in}}%
\pgfpathclose%
\pgfusepath{fill}%
\end{pgfscope}%
\begin{pgfscope}%
\pgfpathrectangle{\pgfqpoint{0.539299in}{0.078740in}}{\pgfqpoint{7.842520in}{7.842520in}}%
\pgfusepath{clip}%
\pgfsetbuttcap%
\pgfsetroundjoin%
\definecolor{currentfill}{rgb}{0.180653,0.701402,0.488189}%
\pgfsetfillcolor{currentfill}%
\pgfsetlinewidth{0.000000pt}%
\definecolor{currentstroke}{rgb}{0.191090,0.708366,0.482284}%
\pgfsetstrokecolor{currentstroke}%
\pgfsetdash{}{0pt}%
\pgfpathmoveto{\pgfqpoint{4.499656in}{4.110842in}}%
\pgfpathlineto{\pgfqpoint{4.560545in}{3.904435in}}%
\pgfpathlineto{\pgfqpoint{4.643709in}{3.889018in}}%
\pgfpathclose%
\pgfusepath{fill}%
\end{pgfscope}%
\begin{pgfscope}%
\pgfpathrectangle{\pgfqpoint{0.539299in}{0.078740in}}{\pgfqpoint{7.842520in}{7.842520in}}%
\pgfusepath{clip}%
\pgfsetbuttcap%
\pgfsetroundjoin%
\definecolor{currentfill}{rgb}{0.835270,0.886029,0.102646}%
\pgfsetfillcolor{currentfill}%
\pgfsetlinewidth{0.000000pt}%
\definecolor{currentstroke}{rgb}{0.196571,0.711827,0.479221}%
\pgfsetstrokecolor{currentstroke}%
\pgfsetdash{}{0pt}%
\pgfpathmoveto{\pgfqpoint{3.719555in}{5.277291in}}%
\pgfpathlineto{\pgfqpoint{3.633064in}{5.262991in}}%
\pgfpathlineto{\pgfqpoint{3.777327in}{5.109990in}}%
\pgfpathclose%
\pgfusepath{fill}%
\end{pgfscope}%
\begin{pgfscope}%
\pgfpathrectangle{\pgfqpoint{0.539299in}{0.078740in}}{\pgfqpoint{7.842520in}{7.842520in}}%
\pgfusepath{clip}%
\pgfsetbuttcap%
\pgfsetroundjoin%
\definecolor{currentfill}{rgb}{0.223925,0.334994,0.548053}%
\pgfsetfillcolor{currentfill}%
\pgfsetlinewidth{0.000000pt}%
\definecolor{currentstroke}{rgb}{0.202219,0.715272,0.476084}%
\pgfsetstrokecolor{currentstroke}%
\pgfsetdash{}{0pt}%
\pgfpathmoveto{\pgfqpoint{5.567779in}{2.379196in}}%
\pgfpathlineto{\pgfqpoint{5.488510in}{2.387296in}}%
\pgfpathlineto{\pgfqpoint{5.632182in}{2.166073in}}%
\pgfpathclose%
\pgfusepath{fill}%
\end{pgfscope}%
\begin{pgfscope}%
\pgfpathrectangle{\pgfqpoint{0.539299in}{0.078740in}}{\pgfqpoint{7.842520in}{7.842520in}}%
\pgfusepath{clip}%
\pgfsetbuttcap%
\pgfsetroundjoin%
\definecolor{currentfill}{rgb}{0.945636,0.899815,0.112838}%
\pgfsetfillcolor{currentfill}%
\pgfsetlinewidth{0.000000pt}%
\definecolor{currentstroke}{rgb}{0.208030,0.718701,0.472873}%
\pgfsetstrokecolor{currentstroke}%
\pgfsetdash{}{0pt}%
\pgfpathmoveto{\pgfqpoint{2.975959in}{5.457547in}}%
\pgfpathlineto{\pgfqpoint{2.889650in}{5.309100in}}%
\pgfpathlineto{\pgfqpoint{3.116915in}{5.451034in}}%
\pgfpathclose%
\pgfusepath{fill}%
\end{pgfscope}%
\begin{pgfscope}%
\pgfpathrectangle{\pgfqpoint{0.539299in}{0.078740in}}{\pgfqpoint{7.842520in}{7.842520in}}%
\pgfusepath{clip}%
\pgfsetbuttcap%
\pgfsetroundjoin%
\definecolor{currentfill}{rgb}{0.179019,0.433756,0.557430}%
\pgfsetfillcolor{currentfill}%
\pgfsetlinewidth{0.000000pt}%
\definecolor{currentstroke}{rgb}{0.214000,0.722114,0.469588}%
\pgfsetstrokecolor{currentstroke}%
\pgfsetdash{}{0pt}%
\pgfpathmoveto{\pgfqpoint{5.281003in}{2.812185in}}%
\pgfpathlineto{\pgfqpoint{5.200440in}{2.822813in}}%
\pgfpathlineto{\pgfqpoint{5.424487in}{2.595182in}}%
\pgfpathclose%
\pgfusepath{fill}%
\end{pgfscope}%
\begin{pgfscope}%
\pgfpathrectangle{\pgfqpoint{0.539299in}{0.078740in}}{\pgfqpoint{7.842520in}{7.842520in}}%
\pgfusepath{clip}%
\pgfsetbuttcap%
\pgfsetroundjoin%
\definecolor{currentfill}{rgb}{0.636902,0.856542,0.216620}%
\pgfsetfillcolor{currentfill}%
\pgfsetlinewidth{0.000000pt}%
\definecolor{currentstroke}{rgb}{0.220124,0.725509,0.466226}%
\pgfsetstrokecolor{currentstroke}%
\pgfsetdash{}{0pt}%
\pgfpathmoveto{\pgfqpoint{2.720220in}{4.843367in}}%
\pgfpathlineto{\pgfqpoint{2.665187in}{5.062348in}}%
\pgfpathlineto{\pgfqpoint{2.581236in}{4.802712in}}%
\pgfpathclose%
\pgfusepath{fill}%
\end{pgfscope}%
\begin{pgfscope}%
\pgfpathrectangle{\pgfqpoint{0.539299in}{0.078740in}}{\pgfqpoint{7.842520in}{7.842520in}}%
\pgfusepath{clip}%
\pgfsetbuttcap%
\pgfsetroundjoin%
\definecolor{currentfill}{rgb}{0.983868,0.904867,0.136897}%
\pgfsetfillcolor{currentfill}%
\pgfsetlinewidth{0.000000pt}%
\definecolor{currentstroke}{rgb}{0.226397,0.728888,0.462789}%
\pgfsetstrokecolor{currentstroke}%
\pgfsetdash{}{0pt}%
\pgfpathmoveto{\pgfqpoint{3.346062in}{5.485171in}}%
\pgfpathlineto{\pgfqpoint{3.203841in}{5.543255in}}%
\pgfpathlineto{\pgfqpoint{3.259203in}{5.402997in}}%
\pgfpathclose%
\pgfusepath{fill}%
\end{pgfscope}%
\begin{pgfscope}%
\pgfpathrectangle{\pgfqpoint{0.539299in}{0.078740in}}{\pgfqpoint{7.842520in}{7.842520in}}%
\pgfusepath{clip}%
\pgfsetbuttcap%
\pgfsetroundjoin%
\definecolor{currentfill}{rgb}{0.120638,0.625828,0.533488}%
\pgfsetfillcolor{currentfill}%
\pgfsetlinewidth{0.000000pt}%
\definecolor{currentstroke}{rgb}{0.232815,0.732247,0.459277}%
\pgfsetstrokecolor{currentstroke}%
\pgfsetdash{}{0pt}%
\pgfpathmoveto{\pgfqpoint{4.787540in}{3.665841in}}%
\pgfpathlineto{\pgfqpoint{4.705051in}{3.687048in}}%
\pgfpathlineto{\pgfqpoint{4.849351in}{3.468201in}}%
\pgfpathclose%
\pgfusepath{fill}%
\end{pgfscope}%
\begin{pgfscope}%
\pgfpathrectangle{\pgfqpoint{0.539299in}{0.078740in}}{\pgfqpoint{7.842520in}{7.842520in}}%
\pgfusepath{clip}%
\pgfsetbuttcap%
\pgfsetroundjoin%
\definecolor{currentfill}{rgb}{0.804182,0.882046,0.114965}%
\pgfsetfillcolor{currentfill}%
\pgfsetlinewidth{0.000000pt}%
\definecolor{currentstroke}{rgb}{0.239374,0.735588,0.455688}%
\pgfsetstrokecolor{currentstroke}%
\pgfsetdash{}{0pt}%
\pgfpathmoveto{\pgfqpoint{2.889650in}{5.309100in}}%
\pgfpathlineto{\pgfqpoint{2.665187in}{5.062348in}}%
\pgfpathlineto{\pgfqpoint{2.804286in}{5.106496in}}%
\pgfpathclose%
\pgfusepath{fill}%
\end{pgfscope}%
\begin{pgfscope}%
\pgfpathrectangle{\pgfqpoint{0.539299in}{0.078740in}}{\pgfqpoint{7.842520in}{7.842520in}}%
\pgfusepath{clip}%
\pgfsetbuttcap%
\pgfsetroundjoin%
\definecolor{currentfill}{rgb}{0.983868,0.904867,0.136897}%
\pgfsetfillcolor{currentfill}%
\pgfsetlinewidth{0.000000pt}%
\definecolor{currentstroke}{rgb}{0.246070,0.738910,0.452024}%
\pgfsetstrokecolor{currentstroke}%
\pgfsetdash{}{0pt}%
\pgfpathmoveto{\pgfqpoint{3.203841in}{5.543255in}}%
\pgfpathlineto{\pgfqpoint{3.116915in}{5.451034in}}%
\pgfpathlineto{\pgfqpoint{3.259203in}{5.402997in}}%
\pgfpathclose%
\pgfusepath{fill}%
\end{pgfscope}%
\begin{pgfscope}%
\pgfpathrectangle{\pgfqpoint{0.539299in}{0.078740in}}{\pgfqpoint{7.842520in}{7.842520in}}%
\pgfusepath{clip}%
\pgfsetbuttcap%
\pgfsetroundjoin%
\definecolor{currentfill}{rgb}{0.175707,0.697900,0.491033}%
\pgfsetfillcolor{currentfill}%
\pgfsetlinewidth{0.000000pt}%
\definecolor{currentstroke}{rgb}{0.252899,0.742211,0.448284}%
\pgfsetstrokecolor{currentstroke}%
\pgfsetdash{}{0pt}%
\pgfpathmoveto{\pgfqpoint{2.619520in}{3.617668in}}%
\pgfpathlineto{\pgfqpoint{2.697635in}{4.107094in}}%
\pgfpathlineto{\pgfqpoint{2.557549in}{4.110498in}}%
\pgfpathclose%
\pgfusepath{fill}%
\end{pgfscope}%
\begin{pgfscope}%
\pgfpathrectangle{\pgfqpoint{0.539299in}{0.078740in}}{\pgfqpoint{7.842520in}{7.842520in}}%
\pgfusepath{clip}%
\pgfsetbuttcap%
\pgfsetroundjoin%
\definecolor{currentfill}{rgb}{0.163625,0.471133,0.558148}%
\pgfsetfillcolor{currentfill}%
\pgfsetlinewidth{0.000000pt}%
\definecolor{currentstroke}{rgb}{0.259857,0.745492,0.444467}%
\pgfsetstrokecolor{currentstroke}%
\pgfsetdash{}{0pt}%
\pgfpathmoveto{\pgfqpoint{5.137323in}{3.030184in}}%
\pgfpathlineto{\pgfqpoint{5.200440in}{2.822813in}}%
\pgfpathlineto{\pgfqpoint{5.281003in}{2.812185in}}%
\pgfpathclose%
\pgfusepath{fill}%
\end{pgfscope}%
\begin{pgfscope}%
\pgfpathrectangle{\pgfqpoint{0.539299in}{0.078740in}}{\pgfqpoint{7.842520in}{7.842520in}}%
\pgfusepath{clip}%
\pgfsetbuttcap%
\pgfsetroundjoin%
\definecolor{currentfill}{rgb}{0.124395,0.578002,0.548287}%
\pgfsetfillcolor{currentfill}%
\pgfsetlinewidth{0.000000pt}%
\definecolor{currentstroke}{rgb}{0.266941,0.748751,0.440573}%
\pgfsetstrokecolor{currentstroke}%
\pgfsetdash{}{0pt}%
\pgfpathmoveto{\pgfqpoint{2.684453in}{3.009085in}}%
\pgfpathlineto{\pgfqpoint{2.901907in}{3.544823in}}%
\pgfpathlineto{\pgfqpoint{2.760273in}{3.588885in}}%
\pgfpathclose%
\pgfusepath{fill}%
\end{pgfscope}%
\begin{pgfscope}%
\pgfpathrectangle{\pgfqpoint{0.539299in}{0.078740in}}{\pgfqpoint{7.842520in}{7.842520in}}%
\pgfusepath{clip}%
\pgfsetbuttcap%
\pgfsetroundjoin%
\definecolor{currentfill}{rgb}{0.197636,0.391528,0.554969}%
\pgfsetfillcolor{currentfill}%
\pgfsetlinewidth{0.000000pt}%
\definecolor{currentstroke}{rgb}{0.274149,0.751988,0.436601}%
\pgfsetstrokecolor{currentstroke}%
\pgfsetdash{}{0pt}%
\pgfpathmoveto{\pgfqpoint{5.424487in}{2.595182in}}%
\pgfpathlineto{\pgfqpoint{5.344589in}{2.606000in}}%
\pgfpathlineto{\pgfqpoint{5.488510in}{2.387296in}}%
\pgfpathclose%
\pgfusepath{fill}%
\end{pgfscope}%
\begin{pgfscope}%
\pgfpathrectangle{\pgfqpoint{0.539299in}{0.078740in}}{\pgfqpoint{7.842520in}{7.842520in}}%
\pgfusepath{clip}%
\pgfsetbuttcap%
\pgfsetroundjoin%
\definecolor{currentfill}{rgb}{0.188923,0.410910,0.556326}%
\pgfsetfillcolor{currentfill}%
\pgfsetlinewidth{0.000000pt}%
\definecolor{currentstroke}{rgb}{0.281477,0.755203,0.432552}%
\pgfsetstrokecolor{currentstroke}%
\pgfsetdash{}{0pt}%
\pgfpathmoveto{\pgfqpoint{3.033612in}{2.193144in}}%
\pgfpathlineto{\pgfqpoint{3.251833in}{2.786336in}}%
\pgfpathlineto{\pgfqpoint{3.109114in}{2.853159in}}%
\pgfpathclose%
\pgfusepath{fill}%
\end{pgfscope}%
\begin{pgfscope}%
\pgfpathrectangle{\pgfqpoint{0.539299in}{0.078740in}}{\pgfqpoint{7.842520in}{7.842520in}}%
\pgfusepath{clip}%
\pgfsetbuttcap%
\pgfsetroundjoin%
\definecolor{currentfill}{rgb}{0.296479,0.761561,0.424223}%
\pgfsetfillcolor{currentfill}%
\pgfsetlinewidth{0.000000pt}%
\definecolor{currentstroke}{rgb}{0.288921,0.758394,0.428426}%
\pgfsetstrokecolor{currentstroke}%
\pgfsetdash{}{0pt}%
\pgfpathmoveto{\pgfqpoint{4.355396in}{4.329235in}}%
\pgfpathlineto{\pgfqpoint{4.270980in}{4.328400in}}%
\pgfpathlineto{\pgfqpoint{4.499656in}{4.110842in}}%
\pgfpathclose%
\pgfusepath{fill}%
\end{pgfscope}%
\begin{pgfscope}%
\pgfpathrectangle{\pgfqpoint{0.539299in}{0.078740in}}{\pgfqpoint{7.842520in}{7.842520in}}%
\pgfusepath{clip}%
\pgfsetbuttcap%
\pgfsetroundjoin%
\definecolor{currentfill}{rgb}{0.150148,0.676631,0.506589}%
\pgfsetfillcolor{currentfill}%
\pgfsetlinewidth{0.000000pt}%
\definecolor{currentstroke}{rgb}{0.296479,0.761561,0.424223}%
\pgfsetstrokecolor{currentstroke}%
\pgfsetdash{}{0pt}%
\pgfpathmoveto{\pgfqpoint{4.643709in}{3.889018in}}%
\pgfpathlineto{\pgfqpoint{4.560545in}{3.904435in}}%
\pgfpathlineto{\pgfqpoint{4.705051in}{3.687048in}}%
\pgfpathclose%
\pgfusepath{fill}%
\end{pgfscope}%
\begin{pgfscope}%
\pgfpathrectangle{\pgfqpoint{0.539299in}{0.078740in}}{\pgfqpoint{7.842520in}{7.842520in}}%
\pgfusepath{clip}%
\pgfsetbuttcap%
\pgfsetroundjoin%
\definecolor{currentfill}{rgb}{0.709898,0.868751,0.169257}%
\pgfsetfillcolor{currentfill}%
\pgfsetlinewidth{0.000000pt}%
\definecolor{currentstroke}{rgb}{0.304148,0.764704,0.419943}%
\pgfsetstrokecolor{currentstroke}%
\pgfsetdash{}{0pt}%
\pgfpathmoveto{\pgfqpoint{2.804286in}{5.106496in}}%
\pgfpathlineto{\pgfqpoint{2.665187in}{5.062348in}}%
\pgfpathlineto{\pgfqpoint{2.720220in}{4.843367in}}%
\pgfpathclose%
\pgfusepath{fill}%
\end{pgfscope}%
\begin{pgfscope}%
\pgfpathrectangle{\pgfqpoint{0.539299in}{0.078740in}}{\pgfqpoint{7.842520in}{7.842520in}}%
\pgfusepath{clip}%
\pgfsetbuttcap%
\pgfsetroundjoin%
\definecolor{currentfill}{rgb}{0.964894,0.902323,0.123941}%
\pgfsetfillcolor{currentfill}%
\pgfsetlinewidth{0.000000pt}%
\definecolor{currentstroke}{rgb}{0.311925,0.767822,0.415586}%
\pgfsetstrokecolor{currentstroke}%
\pgfsetdash{}{0pt}%
\pgfpathmoveto{\pgfqpoint{3.259203in}{5.402997in}}%
\pgfpathlineto{\pgfqpoint{3.489230in}{5.389845in}}%
\pgfpathlineto{\pgfqpoint{3.346062in}{5.485171in}}%
\pgfpathclose%
\pgfusepath{fill}%
\end{pgfscope}%
\begin{pgfscope}%
\pgfpathrectangle{\pgfqpoint{0.539299in}{0.078740in}}{\pgfqpoint{7.842520in}{7.842520in}}%
\pgfusepath{clip}%
\pgfsetbuttcap%
\pgfsetroundjoin%
\definecolor{currentfill}{rgb}{0.545524,0.838039,0.275626}%
\pgfsetfillcolor{currentfill}%
\pgfsetlinewidth{0.000000pt}%
\definecolor{currentstroke}{rgb}{0.319809,0.770914,0.411152}%
\pgfsetstrokecolor{currentstroke}%
\pgfsetdash{}{0pt}%
\pgfpathmoveto{\pgfqpoint{2.637836in}{4.513404in}}%
\pgfpathlineto{\pgfqpoint{2.720220in}{4.843367in}}%
\pgfpathlineto{\pgfqpoint{2.581236in}{4.802712in}}%
\pgfpathclose%
\pgfusepath{fill}%
\end{pgfscope}%
\begin{pgfscope}%
\pgfpathrectangle{\pgfqpoint{0.539299in}{0.078740in}}{\pgfqpoint{7.842520in}{7.842520in}}%
\pgfusepath{clip}%
\pgfsetbuttcap%
\pgfsetroundjoin%
\definecolor{currentfill}{rgb}{0.221989,0.339161,0.548752}%
\pgfsetfillcolor{currentfill}%
\pgfsetlinewidth{0.000000pt}%
\definecolor{currentstroke}{rgb}{0.327796,0.773980,0.406640}%
\pgfsetstrokecolor{currentstroke}%
\pgfsetdash{}{0pt}%
\pgfpathmoveto{\pgfqpoint{3.395040in}{2.713747in}}%
\pgfpathlineto{\pgfqpoint{3.175266in}{2.137225in}}%
\pgfpathlineto{\pgfqpoint{3.317407in}{2.078347in}}%
\pgfpathclose%
\pgfusepath{fill}%
\end{pgfscope}%
\begin{pgfscope}%
\pgfpathrectangle{\pgfqpoint{0.539299in}{0.078740in}}{\pgfqpoint{7.842520in}{7.842520in}}%
\pgfusepath{clip}%
\pgfsetbuttcap%
\pgfsetroundjoin%
\definecolor{currentfill}{rgb}{0.267968,0.223549,0.512008}%
\pgfsetfillcolor{currentfill}%
\pgfsetlinewidth{0.000000pt}%
\definecolor{currentstroke}{rgb}{0.335885,0.777018,0.402049}%
\pgfsetstrokecolor{currentstroke}%
\pgfsetdash{}{0pt}%
\pgfpathmoveto{\pgfqpoint{5.918659in}{1.713508in}}%
\pgfpathlineto{\pgfqpoint{5.775575in}{1.941647in}}%
\pgfpathlineto{\pgfqpoint{5.696616in}{1.925940in}}%
\pgfpathclose%
\pgfusepath{fill}%
\end{pgfscope}%
\begin{pgfscope}%
\pgfpathrectangle{\pgfqpoint{0.539299in}{0.078740in}}{\pgfqpoint{7.842520in}{7.842520in}}%
\pgfusepath{clip}%
\pgfsetbuttcap%
\pgfsetroundjoin%
\definecolor{currentfill}{rgb}{0.182256,0.426184,0.557120}%
\pgfsetfillcolor{currentfill}%
\pgfsetlinewidth{0.000000pt}%
\definecolor{currentstroke}{rgb}{0.344074,0.780029,0.397381}%
\pgfsetstrokecolor{currentstroke}%
\pgfsetdash{}{0pt}%
\pgfpathmoveto{\pgfqpoint{5.200440in}{2.822813in}}%
\pgfpathlineto{\pgfqpoint{5.344589in}{2.606000in}}%
\pgfpathlineto{\pgfqpoint{5.424487in}{2.595182in}}%
\pgfpathclose%
\pgfusepath{fill}%
\end{pgfscope}%
\begin{pgfscope}%
\pgfpathrectangle{\pgfqpoint{0.539299in}{0.078740in}}{\pgfqpoint{7.842520in}{7.842520in}}%
\pgfusepath{clip}%
\pgfsetbuttcap%
\pgfsetroundjoin%
\definecolor{currentfill}{rgb}{0.296479,0.761561,0.424223}%
\pgfsetfillcolor{currentfill}%
\pgfsetlinewidth{0.000000pt}%
\definecolor{currentstroke}{rgb}{0.352360,0.783011,0.392636}%
\pgfsetstrokecolor{currentstroke}%
\pgfsetdash{}{0pt}%
\pgfpathmoveto{\pgfqpoint{2.557549in}{4.110498in}}%
\pgfpathlineto{\pgfqpoint{2.697635in}{4.107094in}}%
\pgfpathlineto{\pgfqpoint{2.637836in}{4.513404in}}%
\pgfpathclose%
\pgfusepath{fill}%
\end{pgfscope}%
\begin{pgfscope}%
\pgfpathrectangle{\pgfqpoint{0.539299in}{0.078740in}}{\pgfqpoint{7.842520in}{7.842520in}}%
\pgfusepath{clip}%
\pgfsetbuttcap%
\pgfsetroundjoin%
\definecolor{currentfill}{rgb}{0.281887,0.150881,0.465405}%
\pgfsetfillcolor{currentfill}%
\pgfsetlinewidth{0.000000pt}%
\definecolor{currentstroke}{rgb}{0.360741,0.785964,0.387814}%
\pgfsetstrokecolor{currentstroke}%
\pgfsetdash{}{0pt}%
\pgfpathmoveto{\pgfqpoint{5.918659in}{1.713508in}}%
\pgfpathlineto{\pgfqpoint{5.840111in}{1.685406in}}%
\pgfpathlineto{\pgfqpoint{5.983060in}{1.432854in}}%
\pgfpathclose%
\pgfusepath{fill}%
\end{pgfscope}%
\begin{pgfscope}%
\pgfpathrectangle{\pgfqpoint{0.539299in}{0.078740in}}{\pgfqpoint{7.842520in}{7.842520in}}%
\pgfusepath{clip}%
\pgfsetbuttcap%
\pgfsetroundjoin%
\definecolor{currentfill}{rgb}{0.369214,0.788888,0.382914}%
\pgfsetfillcolor{currentfill}%
\pgfsetlinewidth{0.000000pt}%
\definecolor{currentstroke}{rgb}{0.369214,0.788888,0.382914}%
\pgfsetstrokecolor{currentstroke}%
\pgfsetdash{}{0pt}%
\pgfpathmoveto{\pgfqpoint{4.210962in}{4.541645in}}%
\pgfpathlineto{\pgfqpoint{4.270980in}{4.328400in}}%
\pgfpathlineto{\pgfqpoint{4.355396in}{4.329235in}}%
\pgfpathclose%
\pgfusepath{fill}%
\end{pgfscope}%
\begin{pgfscope}%
\pgfpathrectangle{\pgfqpoint{0.539299in}{0.078740in}}{\pgfqpoint{7.842520in}{7.842520in}}%
\pgfusepath{clip}%
\pgfsetbuttcap%
\pgfsetroundjoin%
\definecolor{currentfill}{rgb}{0.141935,0.526453,0.555991}%
\pgfsetfillcolor{currentfill}%
\pgfsetlinewidth{0.000000pt}%
\definecolor{currentstroke}{rgb}{0.377779,0.791781,0.377939}%
\pgfsetstrokecolor{currentstroke}%
\pgfsetdash{}{0pt}%
\pgfpathmoveto{\pgfqpoint{3.044293in}{3.487491in}}%
\pgfpathlineto{\pgfqpoint{2.825356in}{2.965660in}}%
\pgfpathlineto{\pgfqpoint{2.966934in}{2.913287in}}%
\pgfpathclose%
\pgfusepath{fill}%
\end{pgfscope}%
\begin{pgfscope}%
\pgfpathrectangle{\pgfqpoint{0.539299in}{0.078740in}}{\pgfqpoint{7.842520in}{7.842520in}}%
\pgfusepath{clip}%
\pgfsetbuttcap%
\pgfsetroundjoin%
\definecolor{currentfill}{rgb}{0.935904,0.898570,0.108131}%
\pgfsetfillcolor{currentfill}%
\pgfsetlinewidth{0.000000pt}%
\definecolor{currentstroke}{rgb}{0.386433,0.794644,0.372886}%
\pgfsetstrokecolor{currentstroke}%
\pgfsetdash{}{0pt}%
\pgfpathmoveto{\pgfqpoint{3.116915in}{5.451034in}}%
\pgfpathlineto{\pgfqpoint{2.889650in}{5.309100in}}%
\pgfpathlineto{\pgfqpoint{3.030553in}{5.308756in}}%
\pgfpathclose%
\pgfusepath{fill}%
\end{pgfscope}%
\begin{pgfscope}%
\pgfpathrectangle{\pgfqpoint{0.539299in}{0.078740in}}{\pgfqpoint{7.842520in}{7.842520in}}%
\pgfusepath{clip}%
\pgfsetbuttcap%
\pgfsetroundjoin%
\definecolor{currentfill}{rgb}{0.137770,0.537492,0.554906}%
\pgfsetfillcolor{currentfill}%
\pgfsetlinewidth{0.000000pt}%
\definecolor{currentstroke}{rgb}{0.395174,0.797475,0.367757}%
\pgfsetstrokecolor{currentstroke}%
\pgfsetdash{}{0pt}%
\pgfpathmoveto{\pgfqpoint{4.993441in}{3.248999in}}%
\pgfpathlineto{\pgfqpoint{4.911519in}{3.252217in}}%
\pgfpathlineto{\pgfqpoint{5.137323in}{3.030184in}}%
\pgfpathclose%
\pgfusepath{fill}%
\end{pgfscope}%
\begin{pgfscope}%
\pgfpathrectangle{\pgfqpoint{0.539299in}{0.078740in}}{\pgfqpoint{7.842520in}{7.842520in}}%
\pgfusepath{clip}%
\pgfsetbuttcap%
\pgfsetroundjoin%
\definecolor{currentfill}{rgb}{0.220124,0.725509,0.466226}%
\pgfsetfillcolor{currentfill}%
\pgfsetlinewidth{0.000000pt}%
\definecolor{currentstroke}{rgb}{0.404001,0.800275,0.362552}%
\pgfsetstrokecolor{currentstroke}%
\pgfsetdash{}{0pt}%
\pgfpathmoveto{\pgfqpoint{4.499656in}{4.110842in}}%
\pgfpathlineto{\pgfqpoint{4.415846in}{4.118860in}}%
\pgfpathlineto{\pgfqpoint{4.560545in}{3.904435in}}%
\pgfpathclose%
\pgfusepath{fill}%
\end{pgfscope}%
\begin{pgfscope}%
\pgfpathrectangle{\pgfqpoint{0.539299in}{0.078740in}}{\pgfqpoint{7.842520in}{7.842520in}}%
\pgfusepath{clip}%
\pgfsetbuttcap%
\pgfsetroundjoin%
\definecolor{currentfill}{rgb}{0.252194,0.269783,0.531579}%
\pgfsetfillcolor{currentfill}%
\pgfsetlinewidth{0.000000pt}%
\definecolor{currentstroke}{rgb}{0.412913,0.803041,0.357269}%
\pgfsetstrokecolor{currentstroke}%
\pgfsetdash{}{0pt}%
\pgfpathmoveto{\pgfqpoint{5.775575in}{1.941647in}}%
\pgfpathlineto{\pgfqpoint{5.632182in}{2.166073in}}%
\pgfpathlineto{\pgfqpoint{5.696616in}{1.925940in}}%
\pgfpathclose%
\pgfusepath{fill}%
\end{pgfscope}%
\begin{pgfscope}%
\pgfpathrectangle{\pgfqpoint{0.539299in}{0.078740in}}{\pgfqpoint{7.842520in}{7.842520in}}%
\pgfusepath{clip}%
\pgfsetbuttcap%
\pgfsetroundjoin%
\definecolor{currentfill}{rgb}{0.140210,0.665859,0.513427}%
\pgfsetfillcolor{currentfill}%
\pgfsetlinewidth{0.000000pt}%
\definecolor{currentstroke}{rgb}{0.421908,0.805774,0.351910}%
\pgfsetstrokecolor{currentstroke}%
\pgfsetdash{}{0pt}%
\pgfpathmoveto{\pgfqpoint{2.838886in}{4.080766in}}%
\pgfpathlineto{\pgfqpoint{2.619520in}{3.617668in}}%
\pgfpathlineto{\pgfqpoint{2.760273in}{3.588885in}}%
\pgfpathclose%
\pgfusepath{fill}%
\end{pgfscope}%
\begin{pgfscope}%
\pgfpathrectangle{\pgfqpoint{0.539299in}{0.078740in}}{\pgfqpoint{7.842520in}{7.842520in}}%
\pgfusepath{clip}%
\pgfsetbuttcap%
\pgfsetroundjoin%
\definecolor{currentfill}{rgb}{0.185783,0.704891,0.485273}%
\pgfsetfillcolor{currentfill}%
\pgfsetlinewidth{0.000000pt}%
\definecolor{currentstroke}{rgb}{0.430983,0.808473,0.346476}%
\pgfsetstrokecolor{currentstroke}%
\pgfsetdash{}{0pt}%
\pgfpathmoveto{\pgfqpoint{2.697635in}{4.107094in}}%
\pgfpathlineto{\pgfqpoint{2.619520in}{3.617668in}}%
\pgfpathlineto{\pgfqpoint{2.838886in}{4.080766in}}%
\pgfpathclose%
\pgfusepath{fill}%
\end{pgfscope}%
\begin{pgfscope}%
\pgfpathrectangle{\pgfqpoint{0.539299in}{0.078740in}}{\pgfqpoint{7.842520in}{7.842520in}}%
\pgfusepath{clip}%
\pgfsetbuttcap%
\pgfsetroundjoin%
\definecolor{currentfill}{rgb}{0.126453,0.570633,0.549841}%
\pgfsetfillcolor{currentfill}%
\pgfsetlinewidth{0.000000pt}%
\definecolor{currentstroke}{rgb}{0.440137,0.811138,0.340967}%
\pgfsetstrokecolor{currentstroke}%
\pgfsetdash{}{0pt}%
\pgfpathmoveto{\pgfqpoint{4.849351in}{3.468201in}}%
\pgfpathlineto{\pgfqpoint{4.911519in}{3.252217in}}%
\pgfpathlineto{\pgfqpoint{4.993441in}{3.248999in}}%
\pgfpathclose%
\pgfusepath{fill}%
\end{pgfscope}%
\begin{pgfscope}%
\pgfpathrectangle{\pgfqpoint{0.539299in}{0.078740in}}{\pgfqpoint{7.842520in}{7.842520in}}%
\pgfusepath{clip}%
\pgfsetbuttcap%
\pgfsetroundjoin%
\definecolor{currentfill}{rgb}{0.525776,0.833491,0.288127}%
\pgfsetfillcolor{currentfill}%
\pgfsetlinewidth{0.000000pt}%
\definecolor{currentstroke}{rgb}{0.449368,0.813768,0.335384}%
\pgfsetstrokecolor{currentstroke}%
\pgfsetdash{}{0pt}%
\pgfpathmoveto{\pgfqpoint{4.066411in}{4.745025in}}%
\pgfpathlineto{\pgfqpoint{3.980947in}{4.722912in}}%
\pgfpathlineto{\pgfqpoint{4.210962in}{4.541645in}}%
\pgfpathclose%
\pgfusepath{fill}%
\end{pgfscope}%
\begin{pgfscope}%
\pgfpathrectangle{\pgfqpoint{0.539299in}{0.078740in}}{\pgfqpoint{7.842520in}{7.842520in}}%
\pgfusepath{clip}%
\pgfsetbuttcap%
\pgfsetroundjoin%
\definecolor{currentfill}{rgb}{0.274149,0.751988,0.436601}%
\pgfsetfillcolor{currentfill}%
\pgfsetlinewidth{0.000000pt}%
\definecolor{currentstroke}{rgb}{0.458674,0.816363,0.329727}%
\pgfsetstrokecolor{currentstroke}%
\pgfsetdash{}{0pt}%
\pgfpathmoveto{\pgfqpoint{4.270980in}{4.328400in}}%
\pgfpathlineto{\pgfqpoint{4.415846in}{4.118860in}}%
\pgfpathlineto{\pgfqpoint{4.499656in}{4.110842in}}%
\pgfpathclose%
\pgfusepath{fill}%
\end{pgfscope}%
\begin{pgfscope}%
\pgfpathrectangle{\pgfqpoint{0.539299in}{0.078740in}}{\pgfqpoint{7.842520in}{7.842520in}}%
\pgfusepath{clip}%
\pgfsetbuttcap%
\pgfsetroundjoin%
\definecolor{currentfill}{rgb}{0.896320,0.893616,0.096335}%
\pgfsetfillcolor{currentfill}%
\pgfsetlinewidth{0.000000pt}%
\definecolor{currentstroke}{rgb}{0.468053,0.818921,0.323998}%
\pgfsetstrokecolor{currentstroke}%
\pgfsetdash{}{0pt}%
\pgfpathmoveto{\pgfqpoint{3.489230in}{5.389845in}}%
\pgfpathlineto{\pgfqpoint{3.546541in}{5.204178in}}%
\pgfpathlineto{\pgfqpoint{3.633064in}{5.262991in}}%
\pgfpathclose%
\pgfusepath{fill}%
\end{pgfscope}%
\begin{pgfscope}%
\pgfpathrectangle{\pgfqpoint{0.539299in}{0.078740in}}{\pgfqpoint{7.842520in}{7.842520in}}%
\pgfusepath{clip}%
\pgfsetbuttcap%
\pgfsetroundjoin%
\definecolor{currentfill}{rgb}{0.273006,0.204520,0.501721}%
\pgfsetfillcolor{currentfill}%
\pgfsetlinewidth{0.000000pt}%
\definecolor{currentstroke}{rgb}{0.477504,0.821444,0.318195}%
\pgfsetstrokecolor{currentstroke}%
\pgfsetdash{}{0pt}%
\pgfpathmoveto{\pgfqpoint{5.696616in}{1.925940in}}%
\pgfpathlineto{\pgfqpoint{5.840111in}{1.685406in}}%
\pgfpathlineto{\pgfqpoint{5.918659in}{1.713508in}}%
\pgfpathclose%
\pgfusepath{fill}%
\end{pgfscope}%
\begin{pgfscope}%
\pgfpathrectangle{\pgfqpoint{0.539299in}{0.078740in}}{\pgfqpoint{7.842520in}{7.842520in}}%
\pgfusepath{clip}%
\pgfsetbuttcap%
\pgfsetroundjoin%
\definecolor{currentfill}{rgb}{0.955300,0.901065,0.118128}%
\pgfsetfillcolor{currentfill}%
\pgfsetlinewidth{0.000000pt}%
\definecolor{currentstroke}{rgb}{0.487026,0.823929,0.312321}%
\pgfsetstrokecolor{currentstroke}%
\pgfsetdash{}{0pt}%
\pgfpathmoveto{\pgfqpoint{3.259203in}{5.402997in}}%
\pgfpathlineto{\pgfqpoint{3.116915in}{5.451034in}}%
\pgfpathlineto{\pgfqpoint{3.030553in}{5.308756in}}%
\pgfpathclose%
\pgfusepath{fill}%
\end{pgfscope}%
\begin{pgfscope}%
\pgfpathrectangle{\pgfqpoint{0.539299in}{0.078740in}}{\pgfqpoint{7.842520in}{7.842520in}}%
\pgfusepath{clip}%
\pgfsetbuttcap%
\pgfsetroundjoin%
\definecolor{currentfill}{rgb}{0.223925,0.334994,0.548053}%
\pgfsetfillcolor{currentfill}%
\pgfsetlinewidth{0.000000pt}%
\definecolor{currentstroke}{rgb}{0.496615,0.826376,0.306377}%
\pgfsetstrokecolor{currentstroke}%
\pgfsetdash{}{0pt}%
\pgfpathmoveto{\pgfqpoint{3.460014in}{2.016906in}}%
\pgfpathlineto{\pgfqpoint{3.395040in}{2.713747in}}%
\pgfpathlineto{\pgfqpoint{3.317407in}{2.078347in}}%
\pgfpathclose%
\pgfusepath{fill}%
\end{pgfscope}%
\begin{pgfscope}%
\pgfpathrectangle{\pgfqpoint{0.539299in}{0.078740in}}{\pgfqpoint{7.842520in}{7.842520in}}%
\pgfusepath{clip}%
\pgfsetbuttcap%
\pgfsetroundjoin%
\definecolor{currentfill}{rgb}{0.835270,0.886029,0.102646}%
\pgfsetfillcolor{currentfill}%
\pgfsetlinewidth{0.000000pt}%
\definecolor{currentstroke}{rgb}{0.506271,0.828786,0.300362}%
\pgfsetstrokecolor{currentstroke}%
\pgfsetdash{}{0pt}%
\pgfpathmoveto{\pgfqpoint{2.945073in}{5.109745in}}%
\pgfpathlineto{\pgfqpoint{2.889650in}{5.309100in}}%
\pgfpathlineto{\pgfqpoint{2.804286in}{5.106496in}}%
\pgfpathclose%
\pgfusepath{fill}%
\end{pgfscope}%
\begin{pgfscope}%
\pgfpathrectangle{\pgfqpoint{0.539299in}{0.078740in}}{\pgfqpoint{7.842520in}{7.842520in}}%
\pgfusepath{clip}%
\pgfsetbuttcap%
\pgfsetroundjoin%
\definecolor{currentfill}{rgb}{0.153364,0.497000,0.557724}%
\pgfsetfillcolor{currentfill}%
\pgfsetlinewidth{0.000000pt}%
\definecolor{currentstroke}{rgb}{0.515992,0.831158,0.294279}%
\pgfsetstrokecolor{currentstroke}%
\pgfsetdash{}{0pt}%
\pgfpathmoveto{\pgfqpoint{5.056080in}{3.038167in}}%
\pgfpathlineto{\pgfqpoint{5.200440in}{2.822813in}}%
\pgfpathlineto{\pgfqpoint{5.137323in}{3.030184in}}%
\pgfpathclose%
\pgfusepath{fill}%
\end{pgfscope}%
\begin{pgfscope}%
\pgfpathrectangle{\pgfqpoint{0.539299in}{0.078740in}}{\pgfqpoint{7.842520in}{7.842520in}}%
\pgfusepath{clip}%
\pgfsetbuttcap%
\pgfsetroundjoin%
\definecolor{currentfill}{rgb}{0.955300,0.901065,0.118128}%
\pgfsetfillcolor{currentfill}%
\pgfsetlinewidth{0.000000pt}%
\definecolor{currentstroke}{rgb}{0.525776,0.833491,0.288127}%
\pgfsetstrokecolor{currentstroke}%
\pgfsetdash{}{0pt}%
\pgfpathmoveto{\pgfqpoint{3.402504in}{5.318957in}}%
\pgfpathlineto{\pgfqpoint{3.489230in}{5.389845in}}%
\pgfpathlineto{\pgfqpoint{3.259203in}{5.402997in}}%
\pgfpathclose%
\pgfusepath{fill}%
\end{pgfscope}%
\begin{pgfscope}%
\pgfpathrectangle{\pgfqpoint{0.539299in}{0.078740in}}{\pgfqpoint{7.842520in}{7.842520in}}%
\pgfusepath{clip}%
\pgfsetbuttcap%
\pgfsetroundjoin%
\definecolor{currentfill}{rgb}{0.125394,0.574318,0.549086}%
\pgfsetfillcolor{currentfill}%
\pgfsetlinewidth{0.000000pt}%
\definecolor{currentstroke}{rgb}{0.535621,0.835785,0.281908}%
\pgfsetstrokecolor{currentstroke}%
\pgfsetdash{}{0pt}%
\pgfpathmoveto{\pgfqpoint{2.901907in}{3.544823in}}%
\pgfpathlineto{\pgfqpoint{2.825356in}{2.965660in}}%
\pgfpathlineto{\pgfqpoint{3.044293in}{3.487491in}}%
\pgfpathclose%
\pgfusepath{fill}%
\end{pgfscope}%
\begin{pgfscope}%
\pgfpathrectangle{\pgfqpoint{0.539299in}{0.078740in}}{\pgfqpoint{7.842520in}{7.842520in}}%
\pgfusepath{clip}%
\pgfsetbuttcap%
\pgfsetroundjoin%
\definecolor{currentfill}{rgb}{0.606045,0.850733,0.236712}%
\pgfsetfillcolor{currentfill}%
\pgfsetlinewidth{0.000000pt}%
\definecolor{currentstroke}{rgb}{0.545524,0.838039,0.275626}%
\pgfsetstrokecolor{currentstroke}%
\pgfsetdash{}{0pt}%
\pgfpathmoveto{\pgfqpoint{3.921827in}{4.935824in}}%
\pgfpathlineto{\pgfqpoint{3.980947in}{4.722912in}}%
\pgfpathlineto{\pgfqpoint{4.066411in}{4.745025in}}%
\pgfpathclose%
\pgfusepath{fill}%
\end{pgfscope}%
\begin{pgfscope}%
\pgfpathrectangle{\pgfqpoint{0.539299in}{0.078740in}}{\pgfqpoint{7.842520in}{7.842520in}}%
\pgfusepath{clip}%
\pgfsetbuttcap%
\pgfsetroundjoin%
\definecolor{currentfill}{rgb}{0.192357,0.403199,0.555836}%
\pgfsetfillcolor{currentfill}%
\pgfsetlinewidth{0.000000pt}%
\definecolor{currentstroke}{rgb}{0.555484,0.840254,0.269281}%
\pgfsetstrokecolor{currentstroke}%
\pgfsetdash{}{0pt}%
\pgfpathmoveto{\pgfqpoint{3.395040in}{2.713747in}}%
\pgfpathlineto{\pgfqpoint{3.251833in}{2.786336in}}%
\pgfpathlineto{\pgfqpoint{3.175266in}{2.137225in}}%
\pgfpathclose%
\pgfusepath{fill}%
\end{pgfscope}%
\begin{pgfscope}%
\pgfpathrectangle{\pgfqpoint{0.539299in}{0.078740in}}{\pgfqpoint{7.842520in}{7.842520in}}%
\pgfusepath{clip}%
\pgfsetbuttcap%
\pgfsetroundjoin%
\definecolor{currentfill}{rgb}{0.845561,0.887322,0.099702}%
\pgfsetfillcolor{currentfill}%
\pgfsetlinewidth{0.000000pt}%
\definecolor{currentstroke}{rgb}{0.565498,0.842430,0.262877}%
\pgfsetstrokecolor{currentstroke}%
\pgfsetdash{}{0pt}%
\pgfpathmoveto{\pgfqpoint{3.633064in}{5.262991in}}%
\pgfpathlineto{\pgfqpoint{3.546541in}{5.204178in}}%
\pgfpathlineto{\pgfqpoint{3.777327in}{5.109990in}}%
\pgfpathclose%
\pgfusepath{fill}%
\end{pgfscope}%
\begin{pgfscope}%
\pgfpathrectangle{\pgfqpoint{0.539299in}{0.078740in}}{\pgfqpoint{7.842520in}{7.842520in}}%
\pgfusepath{clip}%
\pgfsetbuttcap%
\pgfsetroundjoin%
\definecolor{currentfill}{rgb}{0.751884,0.874951,0.143228}%
\pgfsetfillcolor{currentfill}%
\pgfsetlinewidth{0.000000pt}%
\definecolor{currentstroke}{rgb}{0.575563,0.844566,0.256415}%
\pgfsetstrokecolor{currentstroke}%
\pgfsetdash{}{0pt}%
\pgfpathmoveto{\pgfqpoint{3.921827in}{4.935824in}}%
\pgfpathlineto{\pgfqpoint{3.777327in}{5.109990in}}%
\pgfpathlineto{\pgfqpoint{3.691083in}{5.063599in}}%
\pgfpathclose%
\pgfusepath{fill}%
\end{pgfscope}%
\begin{pgfscope}%
\pgfpathrectangle{\pgfqpoint{0.539299in}{0.078740in}}{\pgfqpoint{7.842520in}{7.842520in}}%
\pgfusepath{clip}%
\pgfsetbuttcap%
\pgfsetroundjoin%
\definecolor{currentfill}{rgb}{0.143343,0.522773,0.556295}%
\pgfsetfillcolor{currentfill}%
\pgfsetlinewidth{0.000000pt}%
\definecolor{currentstroke}{rgb}{0.585678,0.846661,0.249897}%
\pgfsetstrokecolor{currentstroke}%
\pgfsetdash{}{0pt}%
\pgfpathmoveto{\pgfqpoint{2.966934in}{2.913287in}}%
\pgfpathlineto{\pgfqpoint{3.109114in}{2.853159in}}%
\pgfpathlineto{\pgfqpoint{3.044293in}{3.487491in}}%
\pgfpathclose%
\pgfusepath{fill}%
\end{pgfscope}%
\begin{pgfscope}%
\pgfpathrectangle{\pgfqpoint{0.539299in}{0.078740in}}{\pgfqpoint{7.842520in}{7.842520in}}%
\pgfusepath{clip}%
\pgfsetbuttcap%
\pgfsetroundjoin%
\definecolor{currentfill}{rgb}{0.225863,0.330805,0.547314}%
\pgfsetfillcolor{currentfill}%
\pgfsetlinewidth{0.000000pt}%
\definecolor{currentstroke}{rgb}{0.595839,0.848717,0.243329}%
\pgfsetstrokecolor{currentstroke}%
\pgfsetdash{}{0pt}%
\pgfpathmoveto{\pgfqpoint{5.552680in}{2.156474in}}%
\pgfpathlineto{\pgfqpoint{5.632182in}{2.166073in}}%
\pgfpathlineto{\pgfqpoint{5.488510in}{2.387296in}}%
\pgfpathclose%
\pgfusepath{fill}%
\end{pgfscope}%
\begin{pgfscope}%
\pgfpathrectangle{\pgfqpoint{0.539299in}{0.078740in}}{\pgfqpoint{7.842520in}{7.842520in}}%
\pgfusepath{clip}%
\pgfsetbuttcap%
\pgfsetroundjoin%
\definecolor{currentfill}{rgb}{0.886271,0.892374,0.095374}%
\pgfsetfillcolor{currentfill}%
\pgfsetlinewidth{0.000000pt}%
\definecolor{currentstroke}{rgb}{0.606045,0.850733,0.236712}%
\pgfsetstrokecolor{currentstroke}%
\pgfsetdash{}{0pt}%
\pgfpathmoveto{\pgfqpoint{3.030553in}{5.308756in}}%
\pgfpathlineto{\pgfqpoint{2.889650in}{5.309100in}}%
\pgfpathlineto{\pgfqpoint{2.945073in}{5.109745in}}%
\pgfpathclose%
\pgfusepath{fill}%
\end{pgfscope}%
\begin{pgfscope}%
\pgfpathrectangle{\pgfqpoint{0.539299in}{0.078740in}}{\pgfqpoint{7.842520in}{7.842520in}}%
\pgfusepath{clip}%
\pgfsetbuttcap%
\pgfsetroundjoin%
\definecolor{currentfill}{rgb}{0.377779,0.791781,0.377939}%
\pgfsetfillcolor{currentfill}%
\pgfsetlinewidth{0.000000pt}%
\definecolor{currentstroke}{rgb}{0.616293,0.852709,0.230052}%
\pgfsetstrokecolor{currentstroke}%
\pgfsetdash{}{0pt}%
\pgfpathmoveto{\pgfqpoint{2.637836in}{4.513404in}}%
\pgfpathlineto{\pgfqpoint{2.697635in}{4.107094in}}%
\pgfpathlineto{\pgfqpoint{2.778216in}{4.515166in}}%
\pgfpathclose%
\pgfusepath{fill}%
\end{pgfscope}%
\begin{pgfscope}%
\pgfpathrectangle{\pgfqpoint{0.539299in}{0.078740in}}{\pgfqpoint{7.842520in}{7.842520in}}%
\pgfusepath{clip}%
\pgfsetbuttcap%
\pgfsetroundjoin%
\definecolor{currentfill}{rgb}{0.140536,0.530132,0.555659}%
\pgfsetfillcolor{currentfill}%
\pgfsetlinewidth{0.000000pt}%
\definecolor{currentstroke}{rgb}{0.626579,0.854645,0.223353}%
\pgfsetstrokecolor{currentstroke}%
\pgfsetdash{}{0pt}%
\pgfpathmoveto{\pgfqpoint{4.911519in}{3.252217in}}%
\pgfpathlineto{\pgfqpoint{5.056080in}{3.038167in}}%
\pgfpathlineto{\pgfqpoint{5.137323in}{3.030184in}}%
\pgfpathclose%
\pgfusepath{fill}%
\end{pgfscope}%
\begin{pgfscope}%
\pgfpathrectangle{\pgfqpoint{0.539299in}{0.078740in}}{\pgfqpoint{7.842520in}{7.842520in}}%
\pgfusepath{clip}%
\pgfsetbuttcap%
\pgfsetroundjoin%
\definecolor{currentfill}{rgb}{0.575563,0.844566,0.256415}%
\pgfsetfillcolor{currentfill}%
\pgfsetlinewidth{0.000000pt}%
\definecolor{currentstroke}{rgb}{0.636902,0.856542,0.216620}%
\pgfsetstrokecolor{currentstroke}%
\pgfsetdash{}{0pt}%
\pgfpathmoveto{\pgfqpoint{2.720220in}{4.843367in}}%
\pgfpathlineto{\pgfqpoint{2.637836in}{4.513404in}}%
\pgfpathlineto{\pgfqpoint{2.860832in}{4.847363in}}%
\pgfpathclose%
\pgfusepath{fill}%
\end{pgfscope}%
\begin{pgfscope}%
\pgfpathrectangle{\pgfqpoint{0.539299in}{0.078740in}}{\pgfqpoint{7.842520in}{7.842520in}}%
\pgfusepath{clip}%
\pgfsetbuttcap%
\pgfsetroundjoin%
\definecolor{currentfill}{rgb}{0.699415,0.867117,0.175971}%
\pgfsetfillcolor{currentfill}%
\pgfsetlinewidth{0.000000pt}%
\definecolor{currentstroke}{rgb}{0.647257,0.858400,0.209861}%
\pgfsetstrokecolor{currentstroke}%
\pgfsetdash{}{0pt}%
\pgfpathmoveto{\pgfqpoint{2.804286in}{5.106496in}}%
\pgfpathlineto{\pgfqpoint{2.720220in}{4.843367in}}%
\pgfpathlineto{\pgfqpoint{2.860832in}{4.847363in}}%
\pgfpathclose%
\pgfusepath{fill}%
\end{pgfscope}%
\begin{pgfscope}%
\pgfpathrectangle{\pgfqpoint{0.539299in}{0.078740in}}{\pgfqpoint{7.842520in}{7.842520in}}%
\pgfusepath{clip}%
\pgfsetbuttcap%
\pgfsetroundjoin%
\definecolor{currentfill}{rgb}{0.421908,0.805774,0.351910}%
\pgfsetfillcolor{currentfill}%
\pgfsetlinewidth{0.000000pt}%
\definecolor{currentstroke}{rgb}{0.657642,0.860219,0.203082}%
\pgfsetstrokecolor{currentstroke}%
\pgfsetdash{}{0pt}%
\pgfpathmoveto{\pgfqpoint{4.125992in}{4.530691in}}%
\pgfpathlineto{\pgfqpoint{4.270980in}{4.328400in}}%
\pgfpathlineto{\pgfqpoint{4.210962in}{4.541645in}}%
\pgfpathclose%
\pgfusepath{fill}%
\end{pgfscope}%
\begin{pgfscope}%
\pgfpathrectangle{\pgfqpoint{0.539299in}{0.078740in}}{\pgfqpoint{7.842520in}{7.842520in}}%
\pgfusepath{clip}%
\pgfsetbuttcap%
\pgfsetroundjoin%
\definecolor{currentfill}{rgb}{0.926106,0.897330,0.104071}%
\pgfsetfillcolor{currentfill}%
\pgfsetlinewidth{0.000000pt}%
\definecolor{currentstroke}{rgb}{0.668054,0.861999,0.196293}%
\pgfsetstrokecolor{currentstroke}%
\pgfsetdash{}{0pt}%
\pgfpathmoveto{\pgfqpoint{3.402504in}{5.318957in}}%
\pgfpathlineto{\pgfqpoint{3.546541in}{5.204178in}}%
\pgfpathlineto{\pgfqpoint{3.489230in}{5.389845in}}%
\pgfpathclose%
\pgfusepath{fill}%
\end{pgfscope}%
\begin{pgfscope}%
\pgfpathrectangle{\pgfqpoint{0.539299in}{0.078740in}}{\pgfqpoint{7.842520in}{7.842520in}}%
\pgfusepath{clip}%
\pgfsetbuttcap%
\pgfsetroundjoin%
\definecolor{currentfill}{rgb}{0.124780,0.640461,0.527068}%
\pgfsetfillcolor{currentfill}%
\pgfsetlinewidth{0.000000pt}%
\definecolor{currentstroke}{rgb}{0.678489,0.863742,0.189503}%
\pgfsetstrokecolor{currentstroke}%
\pgfsetdash{}{0pt}%
\pgfpathmoveto{\pgfqpoint{4.849351in}{3.468201in}}%
\pgfpathlineto{\pgfqpoint{4.705051in}{3.687048in}}%
\pgfpathlineto{\pgfqpoint{4.621833in}{3.675341in}}%
\pgfpathclose%
\pgfusepath{fill}%
\end{pgfscope}%
\begin{pgfscope}%
\pgfpathrectangle{\pgfqpoint{0.539299in}{0.078740in}}{\pgfqpoint{7.842520in}{7.842520in}}%
\pgfusepath{clip}%
\pgfsetbuttcap%
\pgfsetroundjoin%
\definecolor{currentfill}{rgb}{0.140210,0.665859,0.513427}%
\pgfsetfillcolor{currentfill}%
\pgfsetlinewidth{0.000000pt}%
\definecolor{currentstroke}{rgb}{0.688944,0.865448,0.182725}%
\pgfsetstrokecolor{currentstroke}%
\pgfsetdash{}{0pt}%
\pgfpathmoveto{\pgfqpoint{2.760273in}{3.588885in}}%
\pgfpathlineto{\pgfqpoint{2.901907in}{3.544823in}}%
\pgfpathlineto{\pgfqpoint{2.838886in}{4.080766in}}%
\pgfpathclose%
\pgfusepath{fill}%
\end{pgfscope}%
\begin{pgfscope}%
\pgfpathrectangle{\pgfqpoint{0.539299in}{0.078740in}}{\pgfqpoint{7.842520in}{7.842520in}}%
\pgfusepath{clip}%
\pgfsetbuttcap%
\pgfsetroundjoin%
\definecolor{currentfill}{rgb}{0.496615,0.826376,0.306377}%
\pgfsetfillcolor{currentfill}%
\pgfsetlinewidth{0.000000pt}%
\definecolor{currentstroke}{rgb}{0.699415,0.867117,0.175971}%
\pgfsetstrokecolor{currentstroke}%
\pgfsetdash{}{0pt}%
\pgfpathmoveto{\pgfqpoint{4.210962in}{4.541645in}}%
\pgfpathlineto{\pgfqpoint{3.980947in}{4.722912in}}%
\pgfpathlineto{\pgfqpoint{4.125992in}{4.530691in}}%
\pgfpathclose%
\pgfusepath{fill}%
\end{pgfscope}%
\begin{pgfscope}%
\pgfpathrectangle{\pgfqpoint{0.539299in}{0.078740in}}{\pgfqpoint{7.842520in}{7.842520in}}%
\pgfusepath{clip}%
\pgfsetbuttcap%
\pgfsetroundjoin%
\definecolor{currentfill}{rgb}{0.199430,0.387607,0.554642}%
\pgfsetfillcolor{currentfill}%
\pgfsetlinewidth{0.000000pt}%
\definecolor{currentstroke}{rgb}{0.709898,0.868751,0.169257}%
\pgfsetstrokecolor{currentstroke}%
\pgfsetdash{}{0pt}%
\pgfpathmoveto{\pgfqpoint{5.408389in}{2.379042in}}%
\pgfpathlineto{\pgfqpoint{5.488510in}{2.387296in}}%
\pgfpathlineto{\pgfqpoint{5.344589in}{2.606000in}}%
\pgfpathclose%
\pgfusepath{fill}%
\end{pgfscope}%
\begin{pgfscope}%
\pgfpathrectangle{\pgfqpoint{0.539299in}{0.078740in}}{\pgfqpoint{7.842520in}{7.842520in}}%
\pgfusepath{clip}%
\pgfsetbuttcap%
\pgfsetroundjoin%
\definecolor{currentfill}{rgb}{0.241237,0.296485,0.539709}%
\pgfsetfillcolor{currentfill}%
\pgfsetlinewidth{0.000000pt}%
\definecolor{currentstroke}{rgb}{0.720391,0.870350,0.162603}%
\pgfsetstrokecolor{currentstroke}%
\pgfsetdash{}{0pt}%
\pgfpathmoveto{\pgfqpoint{5.696616in}{1.925940in}}%
\pgfpathlineto{\pgfqpoint{5.632182in}{2.166073in}}%
\pgfpathlineto{\pgfqpoint{5.552680in}{2.156474in}}%
\pgfpathclose%
\pgfusepath{fill}%
\end{pgfscope}%
\begin{pgfscope}%
\pgfpathrectangle{\pgfqpoint{0.539299in}{0.078740in}}{\pgfqpoint{7.842520in}{7.842520in}}%
\pgfusepath{clip}%
\pgfsetbuttcap%
\pgfsetroundjoin%
\definecolor{currentfill}{rgb}{0.515992,0.831158,0.294279}%
\pgfsetfillcolor{currentfill}%
\pgfsetlinewidth{0.000000pt}%
\definecolor{currentstroke}{rgb}{0.730889,0.871916,0.156029}%
\pgfsetstrokecolor{currentstroke}%
\pgfsetdash{}{0pt}%
\pgfpathmoveto{\pgfqpoint{2.860832in}{4.847363in}}%
\pgfpathlineto{\pgfqpoint{2.637836in}{4.513404in}}%
\pgfpathlineto{\pgfqpoint{2.778216in}{4.515166in}}%
\pgfpathclose%
\pgfusepath{fill}%
\end{pgfscope}%
\begin{pgfscope}%
\pgfpathrectangle{\pgfqpoint{0.539299in}{0.078740in}}{\pgfqpoint{7.842520in}{7.842520in}}%
\pgfusepath{clip}%
\pgfsetbuttcap%
\pgfsetroundjoin%
\definecolor{currentfill}{rgb}{0.945636,0.899815,0.112838}%
\pgfsetfillcolor{currentfill}%
\pgfsetlinewidth{0.000000pt}%
\definecolor{currentstroke}{rgb}{0.741388,0.873449,0.149561}%
\pgfsetstrokecolor{currentstroke}%
\pgfsetdash{}{0pt}%
\pgfpathmoveto{\pgfqpoint{3.030553in}{5.308756in}}%
\pgfpathlineto{\pgfqpoint{3.172817in}{5.269283in}}%
\pgfpathlineto{\pgfqpoint{3.259203in}{5.402997in}}%
\pgfpathclose%
\pgfusepath{fill}%
\end{pgfscope}%
\begin{pgfscope}%
\pgfpathrectangle{\pgfqpoint{0.539299in}{0.078740in}}{\pgfqpoint{7.842520in}{7.842520in}}%
\pgfusepath{clip}%
\pgfsetbuttcap%
\pgfsetroundjoin%
\definecolor{currentfill}{rgb}{0.311925,0.767822,0.415586}%
\pgfsetfillcolor{currentfill}%
\pgfsetlinewidth{0.000000pt}%
\definecolor{currentstroke}{rgb}{0.751884,0.874951,0.143228}%
\pgfsetstrokecolor{currentstroke}%
\pgfsetdash{}{0pt}%
\pgfpathmoveto{\pgfqpoint{2.778216in}{4.515166in}}%
\pgfpathlineto{\pgfqpoint{2.697635in}{4.107094in}}%
\pgfpathlineto{\pgfqpoint{2.838886in}{4.080766in}}%
\pgfpathclose%
\pgfusepath{fill}%
\end{pgfscope}%
\begin{pgfscope}%
\pgfpathrectangle{\pgfqpoint{0.539299in}{0.078740in}}{\pgfqpoint{7.842520in}{7.842520in}}%
\pgfusepath{clip}%
\pgfsetbuttcap%
\pgfsetroundjoin%
\definecolor{currentfill}{rgb}{0.762373,0.876424,0.137064}%
\pgfsetfillcolor{currentfill}%
\pgfsetlinewidth{0.000000pt}%
\definecolor{currentstroke}{rgb}{0.762373,0.876424,0.137064}%
\pgfsetstrokecolor{currentstroke}%
\pgfsetdash{}{0pt}%
\pgfpathmoveto{\pgfqpoint{2.860832in}{4.847363in}}%
\pgfpathlineto{\pgfqpoint{2.945073in}{5.109745in}}%
\pgfpathlineto{\pgfqpoint{2.804286in}{5.106496in}}%
\pgfpathclose%
\pgfusepath{fill}%
\end{pgfscope}%
\begin{pgfscope}%
\pgfpathrectangle{\pgfqpoint{0.539299in}{0.078740in}}{\pgfqpoint{7.842520in}{7.842520in}}%
\pgfusepath{clip}%
\pgfsetbuttcap%
\pgfsetroundjoin%
\definecolor{currentfill}{rgb}{0.824940,0.884720,0.106217}%
\pgfsetfillcolor{currentfill}%
\pgfsetlinewidth{0.000000pt}%
\definecolor{currentstroke}{rgb}{0.772852,0.877868,0.131109}%
\pgfsetstrokecolor{currentstroke}%
\pgfsetdash{}{0pt}%
\pgfpathmoveto{\pgfqpoint{3.777327in}{5.109990in}}%
\pgfpathlineto{\pgfqpoint{3.546541in}{5.204178in}}%
\pgfpathlineto{\pgfqpoint{3.691083in}{5.063599in}}%
\pgfpathclose%
\pgfusepath{fill}%
\end{pgfscope}%
\begin{pgfscope}%
\pgfpathrectangle{\pgfqpoint{0.539299in}{0.078740in}}{\pgfqpoint{7.842520in}{7.842520in}}%
\pgfusepath{clip}%
\pgfsetbuttcap%
\pgfsetroundjoin%
\definecolor{currentfill}{rgb}{0.162016,0.687316,0.499129}%
\pgfsetfillcolor{currentfill}%
\pgfsetlinewidth{0.000000pt}%
\definecolor{currentstroke}{rgb}{0.783315,0.879285,0.125405}%
\pgfsetstrokecolor{currentstroke}%
\pgfsetdash{}{0pt}%
\pgfpathmoveto{\pgfqpoint{4.705051in}{3.687048in}}%
\pgfpathlineto{\pgfqpoint{4.560545in}{3.904435in}}%
\pgfpathlineto{\pgfqpoint{4.476731in}{3.882966in}}%
\pgfpathclose%
\pgfusepath{fill}%
\end{pgfscope}%
\begin{pgfscope}%
\pgfpathrectangle{\pgfqpoint{0.539299in}{0.078740in}}{\pgfqpoint{7.842520in}{7.842520in}}%
\pgfusepath{clip}%
\pgfsetbuttcap%
\pgfsetroundjoin%
\definecolor{currentfill}{rgb}{0.657642,0.860219,0.203082}%
\pgfsetfillcolor{currentfill}%
\pgfsetlinewidth{0.000000pt}%
\definecolor{currentstroke}{rgb}{0.793760,0.880678,0.120005}%
\pgfsetstrokecolor{currentstroke}%
\pgfsetdash{}{0pt}%
\pgfpathmoveto{\pgfqpoint{3.835937in}{4.901790in}}%
\pgfpathlineto{\pgfqpoint{3.980947in}{4.722912in}}%
\pgfpathlineto{\pgfqpoint{3.921827in}{4.935824in}}%
\pgfpathclose%
\pgfusepath{fill}%
\end{pgfscope}%
\begin{pgfscope}%
\pgfpathrectangle{\pgfqpoint{0.539299in}{0.078740in}}{\pgfqpoint{7.842520in}{7.842520in}}%
\pgfusepath{clip}%
\pgfsetbuttcap%
\pgfsetroundjoin%
\definecolor{currentfill}{rgb}{0.120565,0.596422,0.543611}%
\pgfsetfillcolor{currentfill}%
\pgfsetlinewidth{0.000000pt}%
\definecolor{currentstroke}{rgb}{0.804182,0.882046,0.114965}%
\pgfsetstrokecolor{currentstroke}%
\pgfsetdash{}{0pt}%
\pgfpathmoveto{\pgfqpoint{4.766767in}{3.464787in}}%
\pgfpathlineto{\pgfqpoint{4.911519in}{3.252217in}}%
\pgfpathlineto{\pgfqpoint{4.849351in}{3.468201in}}%
\pgfpathclose%
\pgfusepath{fill}%
\end{pgfscope}%
\begin{pgfscope}%
\pgfpathrectangle{\pgfqpoint{0.539299in}{0.078740in}}{\pgfqpoint{7.842520in}{7.842520in}}%
\pgfusepath{clip}%
\pgfsetbuttcap%
\pgfsetroundjoin%
\definecolor{currentfill}{rgb}{0.730889,0.871916,0.156029}%
\pgfsetfillcolor{currentfill}%
\pgfsetlinewidth{0.000000pt}%
\definecolor{currentstroke}{rgb}{0.814576,0.883393,0.110347}%
\pgfsetstrokecolor{currentstroke}%
\pgfsetdash{}{0pt}%
\pgfpathmoveto{\pgfqpoint{3.691083in}{5.063599in}}%
\pgfpathlineto{\pgfqpoint{3.835937in}{4.901790in}}%
\pgfpathlineto{\pgfqpoint{3.921827in}{4.935824in}}%
\pgfpathclose%
\pgfusepath{fill}%
\end{pgfscope}%
\begin{pgfscope}%
\pgfpathrectangle{\pgfqpoint{0.539299in}{0.078740in}}{\pgfqpoint{7.842520in}{7.842520in}}%
\pgfusepath{clip}%
\pgfsetbuttcap%
\pgfsetroundjoin%
\definecolor{currentfill}{rgb}{0.280255,0.165693,0.476498}%
\pgfsetfillcolor{currentfill}%
\pgfsetlinewidth{0.000000pt}%
\definecolor{currentstroke}{rgb}{0.824940,0.884720,0.106217}%
\pgfsetstrokecolor{currentstroke}%
\pgfsetdash{}{0pt}%
\pgfpathmoveto{\pgfqpoint{5.983060in}{1.432854in}}%
\pgfpathlineto{\pgfqpoint{5.840111in}{1.685406in}}%
\pgfpathlineto{\pgfqpoint{5.760821in}{1.647418in}}%
\pgfpathclose%
\pgfusepath{fill}%
\end{pgfscope}%
\begin{pgfscope}%
\pgfpathrectangle{\pgfqpoint{0.539299in}{0.078740in}}{\pgfqpoint{7.842520in}{7.842520in}}%
\pgfusepath{clip}%
\pgfsetbuttcap%
\pgfsetroundjoin%
\definecolor{currentfill}{rgb}{0.227802,0.326594,0.546532}%
\pgfsetfillcolor{currentfill}%
\pgfsetlinewidth{0.000000pt}%
\definecolor{currentstroke}{rgb}{0.835270,0.886029,0.102646}%
\pgfsetstrokecolor{currentstroke}%
\pgfsetdash{}{0pt}%
\pgfpathmoveto{\pgfqpoint{3.460014in}{2.016906in}}%
\pgfpathlineto{\pgfqpoint{3.603070in}{1.953238in}}%
\pgfpathlineto{\pgfqpoint{3.682755in}{2.554388in}}%
\pgfpathclose%
\pgfusepath{fill}%
\end{pgfscope}%
\begin{pgfscope}%
\pgfpathrectangle{\pgfqpoint{0.539299in}{0.078740in}}{\pgfqpoint{7.842520in}{7.842520in}}%
\pgfusepath{clip}%
\pgfsetbuttcap%
\pgfsetroundjoin%
\definecolor{currentfill}{rgb}{0.121380,0.629492,0.531973}%
\pgfsetfillcolor{currentfill}%
\pgfsetlinewidth{0.000000pt}%
\definecolor{currentstroke}{rgb}{0.845561,0.887322,0.099702}%
\pgfsetstrokecolor{currentstroke}%
\pgfsetdash{}{0pt}%
\pgfpathmoveto{\pgfqpoint{4.849351in}{3.468201in}}%
\pgfpathlineto{\pgfqpoint{4.621833in}{3.675341in}}%
\pgfpathlineto{\pgfqpoint{4.766767in}{3.464787in}}%
\pgfpathclose%
\pgfusepath{fill}%
\end{pgfscope}%
\begin{pgfscope}%
\pgfpathrectangle{\pgfqpoint{0.539299in}{0.078740in}}{\pgfqpoint{7.842520in}{7.842520in}}%
\pgfusepath{clip}%
\pgfsetbuttcap%
\pgfsetroundjoin%
\definecolor{currentfill}{rgb}{0.935904,0.898570,0.108131}%
\pgfsetfillcolor{currentfill}%
\pgfsetlinewidth{0.000000pt}%
\definecolor{currentstroke}{rgb}{0.855810,0.888601,0.097452}%
\pgfsetstrokecolor{currentstroke}%
\pgfsetdash{}{0pt}%
\pgfpathmoveto{\pgfqpoint{3.259203in}{5.402997in}}%
\pgfpathlineto{\pgfqpoint{3.316139in}{5.195751in}}%
\pgfpathlineto{\pgfqpoint{3.402504in}{5.318957in}}%
\pgfpathclose%
\pgfusepath{fill}%
\end{pgfscope}%
\begin{pgfscope}%
\pgfpathrectangle{\pgfqpoint{0.539299in}{0.078740in}}{\pgfqpoint{7.842520in}{7.842520in}}%
\pgfusepath{clip}%
\pgfsetbuttcap%
\pgfsetroundjoin%
\definecolor{currentfill}{rgb}{0.208030,0.718701,0.472873}%
\pgfsetfillcolor{currentfill}%
\pgfsetlinewidth{0.000000pt}%
\definecolor{currentstroke}{rgb}{0.866013,0.889868,0.095953}%
\pgfsetstrokecolor{currentstroke}%
\pgfsetdash{}{0pt}%
\pgfpathmoveto{\pgfqpoint{4.476731in}{3.882966in}}%
\pgfpathlineto{\pgfqpoint{4.560545in}{3.904435in}}%
\pgfpathlineto{\pgfqpoint{4.415846in}{4.118860in}}%
\pgfpathclose%
\pgfusepath{fill}%
\end{pgfscope}%
\begin{pgfscope}%
\pgfpathrectangle{\pgfqpoint{0.539299in}{0.078740in}}{\pgfqpoint{7.842520in}{7.842520in}}%
\pgfusepath{clip}%
\pgfsetbuttcap%
\pgfsetroundjoin%
\definecolor{currentfill}{rgb}{0.212395,0.359683,0.551710}%
\pgfsetfillcolor{currentfill}%
\pgfsetlinewidth{0.000000pt}%
\definecolor{currentstroke}{rgb}{0.876168,0.891125,0.095250}%
\pgfsetstrokecolor{currentstroke}%
\pgfsetdash{}{0pt}%
\pgfpathmoveto{\pgfqpoint{5.488510in}{2.387296in}}%
\pgfpathlineto{\pgfqpoint{5.408389in}{2.379042in}}%
\pgfpathlineto{\pgfqpoint{5.552680in}{2.156474in}}%
\pgfpathclose%
\pgfusepath{fill}%
\end{pgfscope}%
\begin{pgfscope}%
\pgfpathrectangle{\pgfqpoint{0.539299in}{0.078740in}}{\pgfqpoint{7.842520in}{7.842520in}}%
\pgfusepath{clip}%
\pgfsetbuttcap%
\pgfsetroundjoin%
\definecolor{currentfill}{rgb}{0.195860,0.395433,0.555276}%
\pgfsetfillcolor{currentfill}%
\pgfsetlinewidth{0.000000pt}%
\definecolor{currentstroke}{rgb}{0.886271,0.892374,0.095374}%
\pgfsetstrokecolor{currentstroke}%
\pgfsetdash{}{0pt}%
\pgfpathmoveto{\pgfqpoint{3.538693in}{2.636201in}}%
\pgfpathlineto{\pgfqpoint{3.395040in}{2.713747in}}%
\pgfpathlineto{\pgfqpoint{3.460014in}{2.016906in}}%
\pgfpathclose%
\pgfusepath{fill}%
\end{pgfscope}%
\begin{pgfscope}%
\pgfpathrectangle{\pgfqpoint{0.539299in}{0.078740in}}{\pgfqpoint{7.842520in}{7.842520in}}%
\pgfusepath{clip}%
\pgfsetbuttcap%
\pgfsetroundjoin%
\definecolor{currentfill}{rgb}{0.168126,0.459988,0.558082}%
\pgfsetfillcolor{currentfill}%
\pgfsetlinewidth{0.000000pt}%
\definecolor{currentstroke}{rgb}{0.896320,0.893616,0.096335}%
\pgfsetstrokecolor{currentstroke}%
\pgfsetdash{}{0pt}%
\pgfpathmoveto{\pgfqpoint{5.344589in}{2.606000in}}%
\pgfpathlineto{\pgfqpoint{5.200440in}{2.822813in}}%
\pgfpathlineto{\pgfqpoint{5.119001in}{2.807062in}}%
\pgfpathclose%
\pgfusepath{fill}%
\end{pgfscope}%
\begin{pgfscope}%
\pgfpathrectangle{\pgfqpoint{0.539299in}{0.078740in}}{\pgfqpoint{7.842520in}{7.842520in}}%
\pgfusepath{clip}%
\pgfsetbuttcap%
\pgfsetroundjoin%
\definecolor{currentfill}{rgb}{0.126453,0.570633,0.549841}%
\pgfsetfillcolor{currentfill}%
\pgfsetlinewidth{0.000000pt}%
\definecolor{currentstroke}{rgb}{0.906311,0.894855,0.098125}%
\pgfsetstrokecolor{currentstroke}%
\pgfsetdash{}{0pt}%
\pgfpathmoveto{\pgfqpoint{3.109114in}{2.853159in}}%
\pgfpathlineto{\pgfqpoint{3.187325in}{3.418679in}}%
\pgfpathlineto{\pgfqpoint{3.044293in}{3.487491in}}%
\pgfpathclose%
\pgfusepath{fill}%
\end{pgfscope}%
\begin{pgfscope}%
\pgfpathrectangle{\pgfqpoint{0.539299in}{0.078740in}}{\pgfqpoint{7.842520in}{7.842520in}}%
\pgfusepath{clip}%
\pgfsetbuttcap%
\pgfsetroundjoin%
\definecolor{currentfill}{rgb}{0.855810,0.888601,0.097452}%
\pgfsetfillcolor{currentfill}%
\pgfsetlinewidth{0.000000pt}%
\definecolor{currentstroke}{rgb}{0.916242,0.896091,0.100717}%
\pgfsetstrokecolor{currentstroke}%
\pgfsetdash{}{0pt}%
\pgfpathmoveto{\pgfqpoint{3.030553in}{5.308756in}}%
\pgfpathlineto{\pgfqpoint{2.945073in}{5.109745in}}%
\pgfpathlineto{\pgfqpoint{3.087223in}{5.076965in}}%
\pgfpathclose%
\pgfusepath{fill}%
\end{pgfscope}%
\begin{pgfscope}%
\pgfpathrectangle{\pgfqpoint{0.539299in}{0.078740in}}{\pgfqpoint{7.842520in}{7.842520in}}%
\pgfusepath{clip}%
\pgfsetbuttcap%
\pgfsetroundjoin%
\definecolor{currentfill}{rgb}{0.147607,0.511733,0.557049}%
\pgfsetfillcolor{currentfill}%
\pgfsetlinewidth{0.000000pt}%
\definecolor{currentstroke}{rgb}{0.926106,0.897330,0.104071}%
\pgfsetstrokecolor{currentstroke}%
\pgfsetdash{}{0pt}%
\pgfpathmoveto{\pgfqpoint{3.109114in}{2.853159in}}%
\pgfpathlineto{\pgfqpoint{3.251833in}{2.786336in}}%
\pgfpathlineto{\pgfqpoint{3.330909in}{3.339983in}}%
\pgfpathclose%
\pgfusepath{fill}%
\end{pgfscope}%
\begin{pgfscope}%
\pgfpathrectangle{\pgfqpoint{0.539299in}{0.078740in}}{\pgfqpoint{7.842520in}{7.842520in}}%
\pgfusepath{clip}%
\pgfsetbuttcap%
\pgfsetroundjoin%
\definecolor{currentfill}{rgb}{0.935904,0.898570,0.108131}%
\pgfsetfillcolor{currentfill}%
\pgfsetlinewidth{0.000000pt}%
\definecolor{currentstroke}{rgb}{0.935904,0.898570,0.108131}%
\pgfsetstrokecolor{currentstroke}%
\pgfsetdash{}{0pt}%
\pgfpathmoveto{\pgfqpoint{3.172817in}{5.269283in}}%
\pgfpathlineto{\pgfqpoint{3.316139in}{5.195751in}}%
\pgfpathlineto{\pgfqpoint{3.259203in}{5.402997in}}%
\pgfpathclose%
\pgfusepath{fill}%
\end{pgfscope}%
\begin{pgfscope}%
\pgfpathrectangle{\pgfqpoint{0.539299in}{0.078740in}}{\pgfqpoint{7.842520in}{7.842520in}}%
\pgfusepath{clip}%
\pgfsetbuttcap%
\pgfsetroundjoin%
\definecolor{currentfill}{rgb}{0.185783,0.704891,0.485273}%
\pgfsetfillcolor{currentfill}%
\pgfsetlinewidth{0.000000pt}%
\definecolor{currentstroke}{rgb}{0.945636,0.899815,0.112838}%
\pgfsetstrokecolor{currentstroke}%
\pgfsetdash{}{0pt}%
\pgfpathmoveto{\pgfqpoint{2.901907in}{3.544823in}}%
\pgfpathlineto{\pgfqpoint{2.981110in}{4.034439in}}%
\pgfpathlineto{\pgfqpoint{2.838886in}{4.080766in}}%
\pgfpathclose%
\pgfusepath{fill}%
\end{pgfscope}%
\begin{pgfscope}%
\pgfpathrectangle{\pgfqpoint{0.539299in}{0.078740in}}{\pgfqpoint{7.842520in}{7.842520in}}%
\pgfusepath{clip}%
\pgfsetbuttcap%
\pgfsetroundjoin%
\definecolor{currentfill}{rgb}{0.896320,0.893616,0.096335}%
\pgfsetfillcolor{currentfill}%
\pgfsetlinewidth{0.000000pt}%
\definecolor{currentstroke}{rgb}{0.955300,0.901065,0.118128}%
\pgfsetstrokecolor{currentstroke}%
\pgfsetdash{}{0pt}%
\pgfpathmoveto{\pgfqpoint{3.087223in}{5.076965in}}%
\pgfpathlineto{\pgfqpoint{3.172817in}{5.269283in}}%
\pgfpathlineto{\pgfqpoint{3.030553in}{5.308756in}}%
\pgfpathclose%
\pgfusepath{fill}%
\end{pgfscope}%
\begin{pgfscope}%
\pgfpathrectangle{\pgfqpoint{0.539299in}{0.078740in}}{\pgfqpoint{7.842520in}{7.842520in}}%
\pgfusepath{clip}%
\pgfsetbuttcap%
\pgfsetroundjoin%
\definecolor{currentfill}{rgb}{0.282623,0.140926,0.457517}%
\pgfsetfillcolor{currentfill}%
\pgfsetlinewidth{0.000000pt}%
\definecolor{currentstroke}{rgb}{0.964894,0.902323,0.123941}%
\pgfsetstrokecolor{currentstroke}%
\pgfsetdash{}{0pt}%
\pgfpathmoveto{\pgfqpoint{5.760821in}{1.647418in}}%
\pgfpathlineto{\pgfqpoint{5.904099in}{1.379502in}}%
\pgfpathlineto{\pgfqpoint{5.983060in}{1.432854in}}%
\pgfpathclose%
\pgfusepath{fill}%
\end{pgfscope}%
\begin{pgfscope}%
\pgfpathrectangle{\pgfqpoint{0.539299in}{0.078740in}}{\pgfqpoint{7.842520in}{7.842520in}}%
\pgfusepath{clip}%
\pgfsetbuttcap%
\pgfsetroundjoin%
\definecolor{currentfill}{rgb}{0.269308,0.218818,0.509577}%
\pgfsetfillcolor{currentfill}%
\pgfsetlinewidth{0.000000pt}%
\definecolor{currentstroke}{rgb}{0.974417,0.903590,0.130215}%
\pgfsetstrokecolor{currentstroke}%
\pgfsetdash{}{0pt}%
\pgfpathmoveto{\pgfqpoint{5.760821in}{1.647418in}}%
\pgfpathlineto{\pgfqpoint{5.840111in}{1.685406in}}%
\pgfpathlineto{\pgfqpoint{5.696616in}{1.925940in}}%
\pgfpathclose%
\pgfusepath{fill}%
\end{pgfscope}%
\begin{pgfscope}%
\pgfpathrectangle{\pgfqpoint{0.539299in}{0.078740in}}{\pgfqpoint{7.842520in}{7.842520in}}%
\pgfusepath{clip}%
\pgfsetbuttcap%
\pgfsetroundjoin%
\definecolor{currentfill}{rgb}{0.140210,0.665859,0.513427}%
\pgfsetfillcolor{currentfill}%
\pgfsetlinewidth{0.000000pt}%
\definecolor{currentstroke}{rgb}{0.983868,0.904867,0.136897}%
\pgfsetstrokecolor{currentstroke}%
\pgfsetdash{}{0pt}%
\pgfpathmoveto{\pgfqpoint{3.044293in}{3.487491in}}%
\pgfpathlineto{\pgfqpoint{2.981110in}{4.034439in}}%
\pgfpathlineto{\pgfqpoint{2.901907in}{3.544823in}}%
\pgfpathclose%
\pgfusepath{fill}%
\end{pgfscope}%
\begin{pgfscope}%
\pgfpathrectangle{\pgfqpoint{0.539299in}{0.078740in}}{\pgfqpoint{7.842520in}{7.842520in}}%
\pgfusepath{clip}%
\pgfsetbuttcap%
\pgfsetroundjoin%
\definecolor{currentfill}{rgb}{0.150148,0.676631,0.506589}%
\pgfsetfillcolor{currentfill}%
\pgfsetlinewidth{0.000000pt}%
\definecolor{currentstroke}{rgb}{0.993248,0.906157,0.143936}%
\pgfsetstrokecolor{currentstroke}%
\pgfsetdash{}{0pt}%
\pgfpathmoveto{\pgfqpoint{4.476731in}{3.882966in}}%
\pgfpathlineto{\pgfqpoint{4.621833in}{3.675341in}}%
\pgfpathlineto{\pgfqpoint{4.705051in}{3.687048in}}%
\pgfpathclose%
\pgfusepath{fill}%
\end{pgfscope}%
\begin{pgfscope}%
\pgfpathrectangle{\pgfqpoint{0.539299in}{0.078740in}}{\pgfqpoint{7.842520in}{7.842520in}}%
\pgfusepath{clip}%
\pgfsetbuttcap%
\pgfsetroundjoin%
\definecolor{currentfill}{rgb}{0.906311,0.894855,0.098125}%
\pgfsetfillcolor{currentfill}%
\pgfsetlinewidth{0.000000pt}%
\definecolor{currentstroke}{rgb}{0.267004,0.004874,0.329415}%
\pgfsetstrokecolor{currentstroke}%
\pgfsetdash{}{0pt}%
\pgfpathmoveto{\pgfqpoint{3.316139in}{5.195751in}}%
\pgfpathlineto{\pgfqpoint{3.546541in}{5.204178in}}%
\pgfpathlineto{\pgfqpoint{3.402504in}{5.318957in}}%
\pgfpathclose%
\pgfusepath{fill}%
\end{pgfscope}%
\begin{pgfscope}%
\pgfpathrectangle{\pgfqpoint{0.539299in}{0.078740in}}{\pgfqpoint{7.842520in}{7.842520in}}%
\pgfusepath{clip}%
\pgfsetbuttcap%
\pgfsetroundjoin%
\definecolor{currentfill}{rgb}{0.395174,0.797475,0.367757}%
\pgfsetfillcolor{currentfill}%
\pgfsetlinewidth{0.000000pt}%
\definecolor{currentstroke}{rgb}{0.268510,0.009605,0.335427}%
\pgfsetstrokecolor{currentstroke}%
\pgfsetdash{}{0pt}%
\pgfpathmoveto{\pgfqpoint{2.838886in}{4.080766in}}%
\pgfpathlineto{\pgfqpoint{2.919862in}{4.489063in}}%
\pgfpathlineto{\pgfqpoint{2.778216in}{4.515166in}}%
\pgfpathclose%
\pgfusepath{fill}%
\end{pgfscope}%
\begin{pgfscope}%
\pgfpathrectangle{\pgfqpoint{0.539299in}{0.078740in}}{\pgfqpoint{7.842520in}{7.842520in}}%
\pgfusepath{clip}%
\pgfsetbuttcap%
\pgfsetroundjoin%
\definecolor{currentfill}{rgb}{0.335885,0.777018,0.402049}%
\pgfsetfillcolor{currentfill}%
\pgfsetlinewidth{0.000000pt}%
\definecolor{currentstroke}{rgb}{0.269944,0.014625,0.341379}%
\pgfsetstrokecolor{currentstroke}%
\pgfsetdash{}{0pt}%
\pgfpathmoveto{\pgfqpoint{4.415846in}{4.118860in}}%
\pgfpathlineto{\pgfqpoint{4.270980in}{4.328400in}}%
\pgfpathlineto{\pgfqpoint{4.186126in}{4.283787in}}%
\pgfpathclose%
\pgfusepath{fill}%
\end{pgfscope}%
\begin{pgfscope}%
\pgfpathrectangle{\pgfqpoint{0.539299in}{0.078740in}}{\pgfqpoint{7.842520in}{7.842520in}}%
\pgfusepath{clip}%
\pgfsetbuttcap%
\pgfsetroundjoin%
\definecolor{currentfill}{rgb}{0.535621,0.835785,0.281908}%
\pgfsetfillcolor{currentfill}%
\pgfsetlinewidth{0.000000pt}%
\definecolor{currentstroke}{rgb}{0.271305,0.019942,0.347269}%
\pgfsetstrokecolor{currentstroke}%
\pgfsetdash{}{0pt}%
\pgfpathmoveto{\pgfqpoint{2.778216in}{4.515166in}}%
\pgfpathlineto{\pgfqpoint{2.919862in}{4.489063in}}%
\pgfpathlineto{\pgfqpoint{2.860832in}{4.847363in}}%
\pgfpathclose%
\pgfusepath{fill}%
\end{pgfscope}%
\begin{pgfscope}%
\pgfpathrectangle{\pgfqpoint{0.539299in}{0.078740in}}{\pgfqpoint{7.842520in}{7.842520in}}%
\pgfusepath{clip}%
\pgfsetbuttcap%
\pgfsetroundjoin%
\definecolor{currentfill}{rgb}{0.772852,0.877868,0.131109}%
\pgfsetfillcolor{currentfill}%
\pgfsetlinewidth{0.000000pt}%
\definecolor{currentstroke}{rgb}{0.272594,0.025563,0.353093}%
\pgfsetstrokecolor{currentstroke}%
\pgfsetdash{}{0pt}%
\pgfpathmoveto{\pgfqpoint{2.945073in}{5.109745in}}%
\pgfpathlineto{\pgfqpoint{2.860832in}{4.847363in}}%
\pgfpathlineto{\pgfqpoint{3.087223in}{5.076965in}}%
\pgfpathclose%
\pgfusepath{fill}%
\end{pgfscope}%
\begin{pgfscope}%
\pgfpathrectangle{\pgfqpoint{0.539299in}{0.078740in}}{\pgfqpoint{7.842520in}{7.842520in}}%
\pgfusepath{clip}%
\pgfsetbuttcap%
\pgfsetroundjoin%
\definecolor{currentfill}{rgb}{0.187231,0.414746,0.556547}%
\pgfsetfillcolor{currentfill}%
\pgfsetlinewidth{0.000000pt}%
\definecolor{currentstroke}{rgb}{0.273809,0.031497,0.358853}%
\pgfsetstrokecolor{currentstroke}%
\pgfsetdash{}{0pt}%
\pgfpathmoveto{\pgfqpoint{5.263813in}{2.595433in}}%
\pgfpathlineto{\pgfqpoint{5.408389in}{2.379042in}}%
\pgfpathlineto{\pgfqpoint{5.344589in}{2.606000in}}%
\pgfpathclose%
\pgfusepath{fill}%
\end{pgfscope}%
\begin{pgfscope}%
\pgfpathrectangle{\pgfqpoint{0.539299in}{0.078740in}}{\pgfqpoint{7.842520in}{7.842520in}}%
\pgfusepath{clip}%
\pgfsetbuttcap%
\pgfsetroundjoin%
\definecolor{currentfill}{rgb}{0.149039,0.508051,0.557250}%
\pgfsetfillcolor{currentfill}%
\pgfsetlinewidth{0.000000pt}%
\definecolor{currentstroke}{rgb}{0.274952,0.037752,0.364543}%
\pgfsetstrokecolor{currentstroke}%
\pgfsetdash{}{0pt}%
\pgfpathmoveto{\pgfqpoint{5.200440in}{2.822813in}}%
\pgfpathlineto{\pgfqpoint{5.056080in}{3.038167in}}%
\pgfpathlineto{\pgfqpoint{4.973989in}{3.014923in}}%
\pgfpathclose%
\pgfusepath{fill}%
\end{pgfscope}%
\begin{pgfscope}%
\pgfpathrectangle{\pgfqpoint{0.539299in}{0.078740in}}{\pgfqpoint{7.842520in}{7.842520in}}%
\pgfusepath{clip}%
\pgfsetbuttcap%
\pgfsetroundjoin%
\definecolor{currentfill}{rgb}{0.229739,0.322361,0.545706}%
\pgfsetfillcolor{currentfill}%
\pgfsetlinewidth{0.000000pt}%
\definecolor{currentstroke}{rgb}{0.276022,0.044167,0.370164}%
\pgfsetstrokecolor{currentstroke}%
\pgfsetdash{}{0pt}%
\pgfpathmoveto{\pgfqpoint{3.682755in}{2.554388in}}%
\pgfpathlineto{\pgfqpoint{3.603070in}{1.953238in}}%
\pgfpathlineto{\pgfqpoint{3.746562in}{1.887616in}}%
\pgfpathclose%
\pgfusepath{fill}%
\end{pgfscope}%
\begin{pgfscope}%
\pgfpathrectangle{\pgfqpoint{0.539299in}{0.078740in}}{\pgfqpoint{7.842520in}{7.842520in}}%
\pgfusepath{clip}%
\pgfsetbuttcap%
\pgfsetroundjoin%
\definecolor{currentfill}{rgb}{0.199430,0.387607,0.554642}%
\pgfsetfillcolor{currentfill}%
\pgfsetlinewidth{0.000000pt}%
\definecolor{currentstroke}{rgb}{0.277018,0.050344,0.375715}%
\pgfsetstrokecolor{currentstroke}%
\pgfsetdash{}{0pt}%
\pgfpathmoveto{\pgfqpoint{3.682755in}{2.554388in}}%
\pgfpathlineto{\pgfqpoint{3.538693in}{2.636201in}}%
\pgfpathlineto{\pgfqpoint{3.460014in}{2.016906in}}%
\pgfpathclose%
\pgfusepath{fill}%
\end{pgfscope}%
\begin{pgfscope}%
\pgfpathrectangle{\pgfqpoint{0.539299in}{0.078740in}}{\pgfqpoint{7.842520in}{7.842520in}}%
\pgfusepath{clip}%
\pgfsetbuttcap%
\pgfsetroundjoin%
\definecolor{currentfill}{rgb}{0.128729,0.563265,0.551229}%
\pgfsetfillcolor{currentfill}%
\pgfsetlinewidth{0.000000pt}%
\definecolor{currentstroke}{rgb}{0.277941,0.056324,0.381191}%
\pgfsetstrokecolor{currentstroke}%
\pgfsetdash{}{0pt}%
\pgfpathmoveto{\pgfqpoint{3.330909in}{3.339983in}}%
\pgfpathlineto{\pgfqpoint{3.187325in}{3.418679in}}%
\pgfpathlineto{\pgfqpoint{3.109114in}{2.853159in}}%
\pgfpathclose%
\pgfusepath{fill}%
\end{pgfscope}%
\begin{pgfscope}%
\pgfpathrectangle{\pgfqpoint{0.539299in}{0.078740in}}{\pgfqpoint{7.842520in}{7.842520in}}%
\pgfusepath{clip}%
\pgfsetbuttcap%
\pgfsetroundjoin%
\definecolor{currentfill}{rgb}{0.440137,0.811138,0.340967}%
\pgfsetfillcolor{currentfill}%
\pgfsetlinewidth{0.000000pt}%
\definecolor{currentstroke}{rgb}{0.278791,0.062145,0.386592}%
\pgfsetstrokecolor{currentstroke}%
\pgfsetdash{}{0pt}%
\pgfpathmoveto{\pgfqpoint{4.270980in}{4.328400in}}%
\pgfpathlineto{\pgfqpoint{4.125992in}{4.530691in}}%
\pgfpathlineto{\pgfqpoint{4.040710in}{4.473147in}}%
\pgfpathclose%
\pgfusepath{fill}%
\end{pgfscope}%
\begin{pgfscope}%
\pgfpathrectangle{\pgfqpoint{0.539299in}{0.078740in}}{\pgfqpoint{7.842520in}{7.842520in}}%
\pgfusepath{clip}%
\pgfsetbuttcap%
\pgfsetroundjoin%
\definecolor{currentfill}{rgb}{0.149039,0.508051,0.557250}%
\pgfsetfillcolor{currentfill}%
\pgfsetlinewidth{0.000000pt}%
\definecolor{currentstroke}{rgb}{0.279566,0.067836,0.391917}%
\pgfsetstrokecolor{currentstroke}%
\pgfsetdash{}{0pt}%
\pgfpathmoveto{\pgfqpoint{3.251833in}{2.786336in}}%
\pgfpathlineto{\pgfqpoint{3.395040in}{2.713747in}}%
\pgfpathlineto{\pgfqpoint{3.330909in}{3.339983in}}%
\pgfpathclose%
\pgfusepath{fill}%
\end{pgfscope}%
\begin{pgfscope}%
\pgfpathrectangle{\pgfqpoint{0.539299in}{0.078740in}}{\pgfqpoint{7.842520in}{7.842520in}}%
\pgfusepath{clip}%
\pgfsetbuttcap%
\pgfsetroundjoin%
\definecolor{currentfill}{rgb}{0.172719,0.448791,0.557885}%
\pgfsetfillcolor{currentfill}%
\pgfsetlinewidth{0.000000pt}%
\definecolor{currentstroke}{rgb}{0.280267,0.073417,0.397163}%
\pgfsetstrokecolor{currentstroke}%
\pgfsetdash{}{0pt}%
\pgfpathmoveto{\pgfqpoint{5.344589in}{2.606000in}}%
\pgfpathlineto{\pgfqpoint{5.119001in}{2.807062in}}%
\pgfpathlineto{\pgfqpoint{5.263813in}{2.595433in}}%
\pgfpathclose%
\pgfusepath{fill}%
\end{pgfscope}%
\begin{pgfscope}%
\pgfpathrectangle{\pgfqpoint{0.539299in}{0.078740in}}{\pgfqpoint{7.842520in}{7.842520in}}%
\pgfusepath{clip}%
\pgfsetbuttcap%
\pgfsetroundjoin%
\definecolor{currentfill}{rgb}{0.246070,0.738910,0.452024}%
\pgfsetfillcolor{currentfill}%
\pgfsetlinewidth{0.000000pt}%
\definecolor{currentstroke}{rgb}{0.280894,0.078907,0.402329}%
\pgfsetstrokecolor{currentstroke}%
\pgfsetdash{}{0pt}%
\pgfpathmoveto{\pgfqpoint{4.415846in}{4.118860in}}%
\pgfpathlineto{\pgfqpoint{4.331483in}{4.086352in}}%
\pgfpathlineto{\pgfqpoint{4.476731in}{3.882966in}}%
\pgfpathclose%
\pgfusepath{fill}%
\end{pgfscope}%
\begin{pgfscope}%
\pgfpathrectangle{\pgfqpoint{0.539299in}{0.078740in}}{\pgfqpoint{7.842520in}{7.842520in}}%
\pgfusepath{clip}%
\pgfsetbuttcap%
\pgfsetroundjoin%
\definecolor{currentfill}{rgb}{0.824940,0.884720,0.106217}%
\pgfsetfillcolor{currentfill}%
\pgfsetlinewidth{0.000000pt}%
\definecolor{currentstroke}{rgb}{0.281446,0.084320,0.407414}%
\pgfsetstrokecolor{currentstroke}%
\pgfsetdash{}{0pt}%
\pgfpathmoveto{\pgfqpoint{3.546541in}{5.204178in}}%
\pgfpathlineto{\pgfqpoint{3.604941in}{4.965350in}}%
\pgfpathlineto{\pgfqpoint{3.691083in}{5.063599in}}%
\pgfpathclose%
\pgfusepath{fill}%
\end{pgfscope}%
\begin{pgfscope}%
\pgfpathrectangle{\pgfqpoint{0.539299in}{0.078740in}}{\pgfqpoint{7.842520in}{7.842520in}}%
\pgfusepath{clip}%
\pgfsetbuttcap%
\pgfsetroundjoin%
\definecolor{currentfill}{rgb}{0.515992,0.831158,0.294279}%
\pgfsetfillcolor{currentfill}%
\pgfsetlinewidth{0.000000pt}%
\definecolor{currentstroke}{rgb}{0.281924,0.089666,0.412415}%
\pgfsetstrokecolor{currentstroke}%
\pgfsetdash{}{0pt}%
\pgfpathmoveto{\pgfqpoint{4.040710in}{4.473147in}}%
\pgfpathlineto{\pgfqpoint{4.125992in}{4.530691in}}%
\pgfpathlineto{\pgfqpoint{3.980947in}{4.722912in}}%
\pgfpathclose%
\pgfusepath{fill}%
\end{pgfscope}%
\begin{pgfscope}%
\pgfpathrectangle{\pgfqpoint{0.539299in}{0.078740in}}{\pgfqpoint{7.842520in}{7.842520in}}%
\pgfusepath{clip}%
\pgfsetbuttcap%
\pgfsetroundjoin%
\definecolor{currentfill}{rgb}{0.129933,0.559582,0.551864}%
\pgfsetfillcolor{currentfill}%
\pgfsetlinewidth{0.000000pt}%
\definecolor{currentstroke}{rgb}{0.282327,0.094955,0.417331}%
\pgfsetstrokecolor{currentstroke}%
\pgfsetdash{}{0pt}%
\pgfpathmoveto{\pgfqpoint{4.828804in}{3.219577in}}%
\pgfpathlineto{\pgfqpoint{5.056080in}{3.038167in}}%
\pgfpathlineto{\pgfqpoint{4.911519in}{3.252217in}}%
\pgfpathclose%
\pgfusepath{fill}%
\end{pgfscope}%
\begin{pgfscope}%
\pgfpathrectangle{\pgfqpoint{0.539299in}{0.078740in}}{\pgfqpoint{7.842520in}{7.842520in}}%
\pgfusepath{clip}%
\pgfsetbuttcap%
\pgfsetroundjoin%
\definecolor{currentfill}{rgb}{0.886271,0.892374,0.095374}%
\pgfsetfillcolor{currentfill}%
\pgfsetlinewidth{0.000000pt}%
\definecolor{currentstroke}{rgb}{0.282656,0.100196,0.422160}%
\pgfsetstrokecolor{currentstroke}%
\pgfsetdash{}{0pt}%
\pgfpathmoveto{\pgfqpoint{3.087223in}{5.076965in}}%
\pgfpathlineto{\pgfqpoint{3.316139in}{5.195751in}}%
\pgfpathlineto{\pgfqpoint{3.172817in}{5.269283in}}%
\pgfpathclose%
\pgfusepath{fill}%
\end{pgfscope}%
\begin{pgfscope}%
\pgfpathrectangle{\pgfqpoint{0.539299in}{0.078740in}}{\pgfqpoint{7.842520in}{7.842520in}}%
\pgfusepath{clip}%
\pgfsetbuttcap%
\pgfsetroundjoin%
\definecolor{currentfill}{rgb}{0.606045,0.850733,0.236712}%
\pgfsetfillcolor{currentfill}%
\pgfsetlinewidth{0.000000pt}%
\definecolor{currentstroke}{rgb}{0.282910,0.105393,0.426902}%
\pgfsetstrokecolor{currentstroke}%
\pgfsetdash{}{0pt}%
\pgfpathmoveto{\pgfqpoint{2.919862in}{4.489063in}}%
\pgfpathlineto{\pgfqpoint{3.002777in}{4.819100in}}%
\pgfpathlineto{\pgfqpoint{2.860832in}{4.847363in}}%
\pgfpathclose%
\pgfusepath{fill}%
\end{pgfscope}%
\begin{pgfscope}%
\pgfpathrectangle{\pgfqpoint{0.539299in}{0.078740in}}{\pgfqpoint{7.842520in}{7.842520in}}%
\pgfusepath{clip}%
\pgfsetbuttcap%
\pgfsetroundjoin%
\definecolor{currentfill}{rgb}{0.762373,0.876424,0.137064}%
\pgfsetfillcolor{currentfill}%
\pgfsetlinewidth{0.000000pt}%
\definecolor{currentstroke}{rgb}{0.283091,0.110553,0.431554}%
\pgfsetstrokecolor{currentstroke}%
\pgfsetdash{}{0pt}%
\pgfpathmoveto{\pgfqpoint{3.604941in}{4.965350in}}%
\pgfpathlineto{\pgfqpoint{3.835937in}{4.901790in}}%
\pgfpathlineto{\pgfqpoint{3.691083in}{5.063599in}}%
\pgfpathclose%
\pgfusepath{fill}%
\end{pgfscope}%
\begin{pgfscope}%
\pgfpathrectangle{\pgfqpoint{0.539299in}{0.078740in}}{\pgfqpoint{7.842520in}{7.842520in}}%
\pgfusepath{clip}%
\pgfsetbuttcap%
\pgfsetroundjoin%
\definecolor{currentfill}{rgb}{0.720391,0.870350,0.162603}%
\pgfsetfillcolor{currentfill}%
\pgfsetlinewidth{0.000000pt}%
\definecolor{currentstroke}{rgb}{0.283197,0.115680,0.436115}%
\pgfsetstrokecolor{currentstroke}%
\pgfsetdash{}{0pt}%
\pgfpathmoveto{\pgfqpoint{3.087223in}{5.076965in}}%
\pgfpathlineto{\pgfqpoint{2.860832in}{4.847363in}}%
\pgfpathlineto{\pgfqpoint{3.002777in}{4.819100in}}%
\pgfpathclose%
\pgfusepath{fill}%
\end{pgfscope}%
\begin{pgfscope}%
\pgfpathrectangle{\pgfqpoint{0.539299in}{0.078740in}}{\pgfqpoint{7.842520in}{7.842520in}}%
\pgfusepath{clip}%
\pgfsetbuttcap%
\pgfsetroundjoin%
\definecolor{currentfill}{rgb}{0.668054,0.861999,0.196293}%
\pgfsetfillcolor{currentfill}%
\pgfsetlinewidth{0.000000pt}%
\definecolor{currentstroke}{rgb}{0.283229,0.120777,0.440584}%
\pgfsetstrokecolor{currentstroke}%
\pgfsetdash{}{0pt}%
\pgfpathmoveto{\pgfqpoint{3.980947in}{4.722912in}}%
\pgfpathlineto{\pgfqpoint{3.835937in}{4.901790in}}%
\pgfpathlineto{\pgfqpoint{3.750009in}{4.817083in}}%
\pgfpathclose%
\pgfusepath{fill}%
\end{pgfscope}%
\begin{pgfscope}%
\pgfpathrectangle{\pgfqpoint{0.539299in}{0.078740in}}{\pgfqpoint{7.842520in}{7.842520in}}%
\pgfusepath{clip}%
\pgfsetbuttcap%
\pgfsetroundjoin%
\definecolor{currentfill}{rgb}{0.876168,0.891125,0.095250}%
\pgfsetfillcolor{currentfill}%
\pgfsetlinewidth{0.000000pt}%
\definecolor{currentstroke}{rgb}{0.283187,0.125848,0.444960}%
\pgfsetstrokecolor{currentstroke}%
\pgfsetdash{}{0pt}%
\pgfpathmoveto{\pgfqpoint{3.460255in}{5.092948in}}%
\pgfpathlineto{\pgfqpoint{3.546541in}{5.204178in}}%
\pgfpathlineto{\pgfqpoint{3.316139in}{5.195751in}}%
\pgfpathclose%
\pgfusepath{fill}%
\end{pgfscope}%
\begin{pgfscope}%
\pgfpathrectangle{\pgfqpoint{0.539299in}{0.078740in}}{\pgfqpoint{7.842520in}{7.842520in}}%
\pgfusepath{clip}%
\pgfsetbuttcap%
\pgfsetroundjoin%
\definecolor{currentfill}{rgb}{0.304148,0.764704,0.419943}%
\pgfsetfillcolor{currentfill}%
\pgfsetlinewidth{0.000000pt}%
\definecolor{currentstroke}{rgb}{0.283072,0.130895,0.449241}%
\pgfsetstrokecolor{currentstroke}%
\pgfsetdash{}{0pt}%
\pgfpathmoveto{\pgfqpoint{4.186126in}{4.283787in}}%
\pgfpathlineto{\pgfqpoint{4.331483in}{4.086352in}}%
\pgfpathlineto{\pgfqpoint{4.415846in}{4.118860in}}%
\pgfpathclose%
\pgfusepath{fill}%
\end{pgfscope}%
\begin{pgfscope}%
\pgfpathrectangle{\pgfqpoint{0.539299in}{0.078740in}}{\pgfqpoint{7.842520in}{7.842520in}}%
\pgfusepath{clip}%
\pgfsetbuttcap%
\pgfsetroundjoin%
\definecolor{currentfill}{rgb}{0.395174,0.797475,0.367757}%
\pgfsetfillcolor{currentfill}%
\pgfsetlinewidth{0.000000pt}%
\definecolor{currentstroke}{rgb}{0.282884,0.135920,0.453427}%
\pgfsetstrokecolor{currentstroke}%
\pgfsetdash{}{0pt}%
\pgfpathmoveto{\pgfqpoint{2.838886in}{4.080766in}}%
\pgfpathlineto{\pgfqpoint{3.062545in}{4.438625in}}%
\pgfpathlineto{\pgfqpoint{2.919862in}{4.489063in}}%
\pgfpathclose%
\pgfusepath{fill}%
\end{pgfscope}%
\begin{pgfscope}%
\pgfpathrectangle{\pgfqpoint{0.539299in}{0.078740in}}{\pgfqpoint{7.842520in}{7.842520in}}%
\pgfusepath{clip}%
\pgfsetbuttcap%
\pgfsetroundjoin%
\definecolor{currentfill}{rgb}{0.235526,0.309527,0.542944}%
\pgfsetfillcolor{currentfill}%
\pgfsetlinewidth{0.000000pt}%
\definecolor{currentstroke}{rgb}{0.282623,0.140926,0.457517}%
\pgfsetstrokecolor{currentstroke}%
\pgfsetdash{}{0pt}%
\pgfpathmoveto{\pgfqpoint{5.472274in}{2.124924in}}%
\pgfpathlineto{\pgfqpoint{5.696616in}{1.925940in}}%
\pgfpathlineto{\pgfqpoint{5.552680in}{2.156474in}}%
\pgfpathclose%
\pgfusepath{fill}%
\end{pgfscope}%
\begin{pgfscope}%
\pgfpathrectangle{\pgfqpoint{0.539299in}{0.078740in}}{\pgfqpoint{7.842520in}{7.842520in}}%
\pgfusepath{clip}%
\pgfsetbuttcap%
\pgfsetroundjoin%
\definecolor{currentfill}{rgb}{0.319809,0.770914,0.411152}%
\pgfsetfillcolor{currentfill}%
\pgfsetlinewidth{0.000000pt}%
\definecolor{currentstroke}{rgb}{0.282290,0.145912,0.461510}%
\pgfsetstrokecolor{currentstroke}%
\pgfsetdash{}{0pt}%
\pgfpathmoveto{\pgfqpoint{2.981110in}{4.034439in}}%
\pgfpathlineto{\pgfqpoint{3.062545in}{4.438625in}}%
\pgfpathlineto{\pgfqpoint{2.838886in}{4.080766in}}%
\pgfpathclose%
\pgfusepath{fill}%
\end{pgfscope}%
\begin{pgfscope}%
\pgfpathrectangle{\pgfqpoint{0.539299in}{0.078740in}}{\pgfqpoint{7.842520in}{7.842520in}}%
\pgfusepath{clip}%
\pgfsetbuttcap%
\pgfsetroundjoin%
\definecolor{currentfill}{rgb}{0.153364,0.497000,0.557724}%
\pgfsetfillcolor{currentfill}%
\pgfsetlinewidth{0.000000pt}%
\definecolor{currentstroke}{rgb}{0.281887,0.150881,0.465405}%
\pgfsetstrokecolor{currentstroke}%
\pgfsetdash{}{0pt}%
\pgfpathmoveto{\pgfqpoint{4.973989in}{3.014923in}}%
\pgfpathlineto{\pgfqpoint{5.119001in}{2.807062in}}%
\pgfpathlineto{\pgfqpoint{5.200440in}{2.822813in}}%
\pgfpathclose%
\pgfusepath{fill}%
\end{pgfscope}%
\begin{pgfscope}%
\pgfpathrectangle{\pgfqpoint{0.539299in}{0.078740in}}{\pgfqpoint{7.842520in}{7.842520in}}%
\pgfusepath{clip}%
\pgfsetbuttcap%
\pgfsetroundjoin%
\definecolor{currentfill}{rgb}{0.180653,0.701402,0.488189}%
\pgfsetfillcolor{currentfill}%
\pgfsetlinewidth{0.000000pt}%
\definecolor{currentstroke}{rgb}{0.281412,0.155834,0.469201}%
\pgfsetstrokecolor{currentstroke}%
\pgfsetdash{}{0pt}%
\pgfpathmoveto{\pgfqpoint{3.044293in}{3.487491in}}%
\pgfpathlineto{\pgfqpoint{3.124144in}{3.970746in}}%
\pgfpathlineto{\pgfqpoint{2.981110in}{4.034439in}}%
\pgfpathclose%
\pgfusepath{fill}%
\end{pgfscope}%
\begin{pgfscope}%
\pgfpathrectangle{\pgfqpoint{0.539299in}{0.078740in}}{\pgfqpoint{7.842520in}{7.842520in}}%
\pgfusepath{clip}%
\pgfsetbuttcap%
\pgfsetroundjoin%
\definecolor{currentfill}{rgb}{0.119512,0.607464,0.540218}%
\pgfsetfillcolor{currentfill}%
\pgfsetlinewidth{0.000000pt}%
\definecolor{currentstroke}{rgb}{0.280868,0.160771,0.472899}%
\pgfsetstrokecolor{currentstroke}%
\pgfsetdash{}{0pt}%
\pgfpathmoveto{\pgfqpoint{4.683466in}{3.421156in}}%
\pgfpathlineto{\pgfqpoint{4.911519in}{3.252217in}}%
\pgfpathlineto{\pgfqpoint{4.766767in}{3.464787in}}%
\pgfpathclose%
\pgfusepath{fill}%
\end{pgfscope}%
\begin{pgfscope}%
\pgfpathrectangle{\pgfqpoint{0.539299in}{0.078740in}}{\pgfqpoint{7.842520in}{7.842520in}}%
\pgfusepath{clip}%
\pgfsetbuttcap%
\pgfsetroundjoin%
\definecolor{currentfill}{rgb}{0.260571,0.246922,0.522828}%
\pgfsetfillcolor{currentfill}%
\pgfsetlinewidth{0.000000pt}%
\definecolor{currentstroke}{rgb}{0.280255,0.165693,0.476498}%
\pgfsetstrokecolor{currentstroke}%
\pgfsetdash{}{0pt}%
\pgfpathmoveto{\pgfqpoint{5.696616in}{1.925940in}}%
\pgfpathlineto{\pgfqpoint{5.616816in}{1.894378in}}%
\pgfpathlineto{\pgfqpoint{5.760821in}{1.647418in}}%
\pgfpathclose%
\pgfusepath{fill}%
\end{pgfscope}%
\begin{pgfscope}%
\pgfpathrectangle{\pgfqpoint{0.539299in}{0.078740in}}{\pgfqpoint{7.842520in}{7.842520in}}%
\pgfusepath{clip}%
\pgfsetbuttcap%
\pgfsetroundjoin%
\definecolor{currentfill}{rgb}{0.835270,0.886029,0.102646}%
\pgfsetfillcolor{currentfill}%
\pgfsetlinewidth{0.000000pt}%
\definecolor{currentstroke}{rgb}{0.279574,0.170599,0.479997}%
\pgfsetstrokecolor{currentstroke}%
\pgfsetdash{}{0pt}%
\pgfpathmoveto{\pgfqpoint{3.460255in}{5.092948in}}%
\pgfpathlineto{\pgfqpoint{3.604941in}{4.965350in}}%
\pgfpathlineto{\pgfqpoint{3.546541in}{5.204178in}}%
\pgfpathclose%
\pgfusepath{fill}%
\end{pgfscope}%
\begin{pgfscope}%
\pgfpathrectangle{\pgfqpoint{0.539299in}{0.078740in}}{\pgfqpoint{7.842520in}{7.842520in}}%
\pgfusepath{clip}%
\pgfsetbuttcap%
\pgfsetroundjoin%
\definecolor{currentfill}{rgb}{0.412913,0.803041,0.357269}%
\pgfsetfillcolor{currentfill}%
\pgfsetlinewidth{0.000000pt}%
\definecolor{currentstroke}{rgb}{0.278826,0.175490,0.483397}%
\pgfsetstrokecolor{currentstroke}%
\pgfsetdash{}{0pt}%
\pgfpathmoveto{\pgfqpoint{4.040710in}{4.473147in}}%
\pgfpathlineto{\pgfqpoint{4.186126in}{4.283787in}}%
\pgfpathlineto{\pgfqpoint{4.270980in}{4.328400in}}%
\pgfpathclose%
\pgfusepath{fill}%
\end{pgfscope}%
\begin{pgfscope}%
\pgfpathrectangle{\pgfqpoint{0.539299in}{0.078740in}}{\pgfqpoint{7.842520in}{7.842520in}}%
\pgfusepath{clip}%
\pgfsetbuttcap%
\pgfsetroundjoin%
\definecolor{currentfill}{rgb}{0.845561,0.887322,0.099702}%
\pgfsetfillcolor{currentfill}%
\pgfsetlinewidth{0.000000pt}%
\definecolor{currentstroke}{rgb}{0.278012,0.180367,0.486697}%
\pgfsetstrokecolor{currentstroke}%
\pgfsetdash{}{0pt}%
\pgfpathmoveto{\pgfqpoint{3.230452in}{5.012748in}}%
\pgfpathlineto{\pgfqpoint{3.316139in}{5.195751in}}%
\pgfpathlineto{\pgfqpoint{3.087223in}{5.076965in}}%
\pgfpathclose%
\pgfusepath{fill}%
\end{pgfscope}%
\begin{pgfscope}%
\pgfpathrectangle{\pgfqpoint{0.539299in}{0.078740in}}{\pgfqpoint{7.842520in}{7.842520in}}%
\pgfusepath{clip}%
\pgfsetbuttcap%
\pgfsetroundjoin%
\definecolor{currentfill}{rgb}{0.216210,0.351535,0.550627}%
\pgfsetfillcolor{currentfill}%
\pgfsetlinewidth{0.000000pt}%
\definecolor{currentstroke}{rgb}{0.277134,0.185228,0.489898}%
\pgfsetstrokecolor{currentstroke}%
\pgfsetdash{}{0pt}%
\pgfpathmoveto{\pgfqpoint{5.552680in}{2.156474in}}%
\pgfpathlineto{\pgfqpoint{5.408389in}{2.379042in}}%
\pgfpathlineto{\pgfqpoint{5.472274in}{2.124924in}}%
\pgfpathclose%
\pgfusepath{fill}%
\end{pgfscope}%
\begin{pgfscope}%
\pgfpathrectangle{\pgfqpoint{0.539299in}{0.078740in}}{\pgfqpoint{7.842520in}{7.842520in}}%
\pgfusepath{clip}%
\pgfsetbuttcap%
\pgfsetroundjoin%
\definecolor{currentfill}{rgb}{0.132268,0.655014,0.519661}%
\pgfsetfillcolor{currentfill}%
\pgfsetlinewidth{0.000000pt}%
\definecolor{currentstroke}{rgb}{0.276194,0.190074,0.493001}%
\pgfsetstrokecolor{currentstroke}%
\pgfsetdash{}{0pt}%
\pgfpathmoveto{\pgfqpoint{3.267846in}{3.892059in}}%
\pgfpathlineto{\pgfqpoint{3.044293in}{3.487491in}}%
\pgfpathlineto{\pgfqpoint{3.187325in}{3.418679in}}%
\pgfpathclose%
\pgfusepath{fill}%
\end{pgfscope}%
\begin{pgfscope}%
\pgfpathrectangle{\pgfqpoint{0.539299in}{0.078740in}}{\pgfqpoint{7.842520in}{7.842520in}}%
\pgfusepath{clip}%
\pgfsetbuttcap%
\pgfsetroundjoin%
\definecolor{currentfill}{rgb}{0.135066,0.544853,0.554029}%
\pgfsetfillcolor{currentfill}%
\pgfsetlinewidth{0.000000pt}%
\definecolor{currentstroke}{rgb}{0.275191,0.194905,0.496005}%
\pgfsetstrokecolor{currentstroke}%
\pgfsetdash{}{0pt}%
\pgfpathmoveto{\pgfqpoint{4.973989in}{3.014923in}}%
\pgfpathlineto{\pgfqpoint{5.056080in}{3.038167in}}%
\pgfpathlineto{\pgfqpoint{4.828804in}{3.219577in}}%
\pgfpathclose%
\pgfusepath{fill}%
\end{pgfscope}%
\begin{pgfscope}%
\pgfpathrectangle{\pgfqpoint{0.539299in}{0.078740in}}{\pgfqpoint{7.842520in}{7.842520in}}%
\pgfusepath{clip}%
\pgfsetbuttcap%
\pgfsetroundjoin%
\definecolor{currentfill}{rgb}{0.535621,0.835785,0.281908}%
\pgfsetfillcolor{currentfill}%
\pgfsetlinewidth{0.000000pt}%
\definecolor{currentstroke}{rgb}{0.274128,0.199721,0.498911}%
\pgfsetstrokecolor{currentstroke}%
\pgfsetdash{}{0pt}%
\pgfpathmoveto{\pgfqpoint{2.919862in}{4.489063in}}%
\pgfpathlineto{\pgfqpoint{3.062545in}{4.438625in}}%
\pgfpathlineto{\pgfqpoint{3.002777in}{4.819100in}}%
\pgfpathclose%
\pgfusepath{fill}%
\end{pgfscope}%
\begin{pgfscope}%
\pgfpathrectangle{\pgfqpoint{0.539299in}{0.078740in}}{\pgfqpoint{7.842520in}{7.842520in}}%
\pgfusepath{clip}%
\pgfsetbuttcap%
\pgfsetroundjoin%
\definecolor{currentfill}{rgb}{0.730889,0.871916,0.156029}%
\pgfsetfillcolor{currentfill}%
\pgfsetlinewidth{0.000000pt}%
\definecolor{currentstroke}{rgb}{0.273006,0.204520,0.501721}%
\pgfsetstrokecolor{currentstroke}%
\pgfsetdash{}{0pt}%
\pgfpathmoveto{\pgfqpoint{3.750009in}{4.817083in}}%
\pgfpathlineto{\pgfqpoint{3.835937in}{4.901790in}}%
\pgfpathlineto{\pgfqpoint{3.604941in}{4.965350in}}%
\pgfpathclose%
\pgfusepath{fill}%
\end{pgfscope}%
\begin{pgfscope}%
\pgfpathrectangle{\pgfqpoint{0.539299in}{0.078740in}}{\pgfqpoint{7.842520in}{7.842520in}}%
\pgfusepath{clip}%
\pgfsetbuttcap%
\pgfsetroundjoin%
\definecolor{currentfill}{rgb}{0.555484,0.840254,0.269281}%
\pgfsetfillcolor{currentfill}%
\pgfsetlinewidth{0.000000pt}%
\definecolor{currentstroke}{rgb}{0.271828,0.209303,0.504434}%
\pgfsetstrokecolor{currentstroke}%
\pgfsetdash{}{0pt}%
\pgfpathmoveto{\pgfqpoint{3.895306in}{4.651896in}}%
\pgfpathlineto{\pgfqpoint{4.040710in}{4.473147in}}%
\pgfpathlineto{\pgfqpoint{3.980947in}{4.722912in}}%
\pgfpathclose%
\pgfusepath{fill}%
\end{pgfscope}%
\begin{pgfscope}%
\pgfpathrectangle{\pgfqpoint{0.539299in}{0.078740in}}{\pgfqpoint{7.842520in}{7.842520in}}%
\pgfusepath{clip}%
\pgfsetbuttcap%
\pgfsetroundjoin%
\definecolor{currentfill}{rgb}{0.132268,0.655014,0.519661}%
\pgfsetfillcolor{currentfill}%
\pgfsetlinewidth{0.000000pt}%
\definecolor{currentstroke}{rgb}{0.270595,0.214069,0.507052}%
\pgfsetstrokecolor{currentstroke}%
\pgfsetdash{}{0pt}%
\pgfpathmoveto{\pgfqpoint{4.766767in}{3.464787in}}%
\pgfpathlineto{\pgfqpoint{4.621833in}{3.675341in}}%
\pgfpathlineto{\pgfqpoint{4.537995in}{3.619381in}}%
\pgfpathclose%
\pgfusepath{fill}%
\end{pgfscope}%
\begin{pgfscope}%
\pgfpathrectangle{\pgfqpoint{0.539299in}{0.078740in}}{\pgfqpoint{7.842520in}{7.842520in}}%
\pgfusepath{clip}%
\pgfsetbuttcap%
\pgfsetroundjoin%
\definecolor{currentfill}{rgb}{0.626579,0.854645,0.223353}%
\pgfsetfillcolor{currentfill}%
\pgfsetlinewidth{0.000000pt}%
\definecolor{currentstroke}{rgb}{0.269308,0.218818,0.509577}%
\pgfsetstrokecolor{currentstroke}%
\pgfsetdash{}{0pt}%
\pgfpathmoveto{\pgfqpoint{3.980947in}{4.722912in}}%
\pgfpathlineto{\pgfqpoint{3.750009in}{4.817083in}}%
\pgfpathlineto{\pgfqpoint{3.895306in}{4.651896in}}%
\pgfpathclose%
\pgfusepath{fill}%
\end{pgfscope}%
\begin{pgfscope}%
\pgfpathrectangle{\pgfqpoint{0.539299in}{0.078740in}}{\pgfqpoint{7.842520in}{7.842520in}}%
\pgfusepath{clip}%
\pgfsetbuttcap%
\pgfsetroundjoin%
\definecolor{currentfill}{rgb}{0.237441,0.305202,0.541921}%
\pgfsetfillcolor{currentfill}%
\pgfsetlinewidth{0.000000pt}%
\definecolor{currentstroke}{rgb}{0.267968,0.223549,0.512008}%
\pgfsetstrokecolor{currentstroke}%
\pgfsetdash{}{0pt}%
\pgfpathmoveto{\pgfqpoint{3.746562in}{1.887616in}}%
\pgfpathlineto{\pgfqpoint{3.890478in}{1.820244in}}%
\pgfpathlineto{\pgfqpoint{3.971998in}{2.380078in}}%
\pgfpathclose%
\pgfusepath{fill}%
\end{pgfscope}%
\begin{pgfscope}%
\pgfpathrectangle{\pgfqpoint{0.539299in}{0.078740in}}{\pgfqpoint{7.842520in}{7.842520in}}%
\pgfusepath{clip}%
\pgfsetbuttcap%
\pgfsetroundjoin%
\definecolor{currentfill}{rgb}{0.282290,0.145912,0.461510}%
\pgfsetfillcolor{currentfill}%
\pgfsetlinewidth{0.000000pt}%
\definecolor{currentstroke}{rgb}{0.266580,0.228262,0.514349}%
\pgfsetstrokecolor{currentstroke}%
\pgfsetdash{}{0pt}%
\pgfpathmoveto{\pgfqpoint{5.824454in}{1.317143in}}%
\pgfpathlineto{\pgfqpoint{5.904099in}{1.379502in}}%
\pgfpathlineto{\pgfqpoint{5.760821in}{1.647418in}}%
\pgfpathclose%
\pgfusepath{fill}%
\end{pgfscope}%
\begin{pgfscope}%
\pgfpathrectangle{\pgfqpoint{0.539299in}{0.078740in}}{\pgfqpoint{7.842520in}{7.842520in}}%
\pgfusepath{clip}%
\pgfsetbuttcap%
\pgfsetroundjoin%
\definecolor{currentfill}{rgb}{0.720391,0.870350,0.162603}%
\pgfsetfillcolor{currentfill}%
\pgfsetlinewidth{0.000000pt}%
\definecolor{currentstroke}{rgb}{0.265145,0.232956,0.516599}%
\pgfsetstrokecolor{currentstroke}%
\pgfsetdash{}{0pt}%
\pgfpathmoveto{\pgfqpoint{3.087223in}{5.076965in}}%
\pgfpathlineto{\pgfqpoint{3.002777in}{4.819100in}}%
\pgfpathlineto{\pgfqpoint{3.145794in}{4.762662in}}%
\pgfpathclose%
\pgfusepath{fill}%
\end{pgfscope}%
\begin{pgfscope}%
\pgfpathrectangle{\pgfqpoint{0.539299in}{0.078740in}}{\pgfqpoint{7.842520in}{7.842520in}}%
\pgfusepath{clip}%
\pgfsetbuttcap%
\pgfsetroundjoin%
\definecolor{currentfill}{rgb}{0.241237,0.296485,0.539709}%
\pgfsetfillcolor{currentfill}%
\pgfsetlinewidth{0.000000pt}%
\definecolor{currentstroke}{rgb}{0.263663,0.237631,0.518762}%
\pgfsetstrokecolor{currentstroke}%
\pgfsetdash{}{0pt}%
\pgfpathmoveto{\pgfqpoint{5.616816in}{1.894378in}}%
\pgfpathlineto{\pgfqpoint{5.696616in}{1.925940in}}%
\pgfpathlineto{\pgfqpoint{5.472274in}{2.124924in}}%
\pgfpathclose%
\pgfusepath{fill}%
\end{pgfscope}%
\begin{pgfscope}%
\pgfpathrectangle{\pgfqpoint{0.539299in}{0.078740in}}{\pgfqpoint{7.842520in}{7.842520in}}%
\pgfusepath{clip}%
\pgfsetbuttcap%
\pgfsetroundjoin%
\definecolor{currentfill}{rgb}{0.204903,0.375746,0.553533}%
\pgfsetfillcolor{currentfill}%
\pgfsetlinewidth{0.000000pt}%
\definecolor{currentstroke}{rgb}{0.262138,0.242286,0.520837}%
\pgfsetstrokecolor{currentstroke}%
\pgfsetdash{}{0pt}%
\pgfpathmoveto{\pgfqpoint{3.746562in}{1.887616in}}%
\pgfpathlineto{\pgfqpoint{3.827198in}{2.468870in}}%
\pgfpathlineto{\pgfqpoint{3.682755in}{2.554388in}}%
\pgfpathclose%
\pgfusepath{fill}%
\end{pgfscope}%
\begin{pgfscope}%
\pgfpathrectangle{\pgfqpoint{0.539299in}{0.078740in}}{\pgfqpoint{7.842520in}{7.842520in}}%
\pgfusepath{clip}%
\pgfsetbuttcap%
\pgfsetroundjoin%
\definecolor{currentfill}{rgb}{0.154815,0.493313,0.557840}%
\pgfsetfillcolor{currentfill}%
\pgfsetlinewidth{0.000000pt}%
\definecolor{currentstroke}{rgb}{0.260571,0.246922,0.522828}%
\pgfsetstrokecolor{currentstroke}%
\pgfsetdash{}{0pt}%
\pgfpathmoveto{\pgfqpoint{3.619444in}{3.158415in}}%
\pgfpathlineto{\pgfqpoint{3.395040in}{2.713747in}}%
\pgfpathlineto{\pgfqpoint{3.538693in}{2.636201in}}%
\pgfpathclose%
\pgfusepath{fill}%
\end{pgfscope}%
\begin{pgfscope}%
\pgfpathrectangle{\pgfqpoint{0.539299in}{0.078740in}}{\pgfqpoint{7.842520in}{7.842520in}}%
\pgfusepath{clip}%
\pgfsetbuttcap%
\pgfsetroundjoin%
\definecolor{currentfill}{rgb}{0.175707,0.697900,0.491033}%
\pgfsetfillcolor{currentfill}%
\pgfsetlinewidth{0.000000pt}%
\definecolor{currentstroke}{rgb}{0.258965,0.251537,0.524736}%
\pgfsetstrokecolor{currentstroke}%
\pgfsetdash{}{0pt}%
\pgfpathmoveto{\pgfqpoint{3.124144in}{3.970746in}}%
\pgfpathlineto{\pgfqpoint{3.044293in}{3.487491in}}%
\pgfpathlineto{\pgfqpoint{3.267846in}{3.892059in}}%
\pgfpathclose%
\pgfusepath{fill}%
\end{pgfscope}%
\begin{pgfscope}%
\pgfpathrectangle{\pgfqpoint{0.539299in}{0.078740in}}{\pgfqpoint{7.842520in}{7.842520in}}%
\pgfusepath{clip}%
\pgfsetbuttcap%
\pgfsetroundjoin%
\definecolor{currentfill}{rgb}{0.132444,0.552216,0.553018}%
\pgfsetfillcolor{currentfill}%
\pgfsetlinewidth{0.000000pt}%
\definecolor{currentstroke}{rgb}{0.257322,0.256130,0.526563}%
\pgfsetstrokecolor{currentstroke}%
\pgfsetdash{}{0pt}%
\pgfpathmoveto{\pgfqpoint{3.330909in}{3.339983in}}%
\pgfpathlineto{\pgfqpoint{3.395040in}{2.713747in}}%
\pgfpathlineto{\pgfqpoint{3.474971in}{3.252817in}}%
\pgfpathclose%
\pgfusepath{fill}%
\end{pgfscope}%
\begin{pgfscope}%
\pgfpathrectangle{\pgfqpoint{0.539299in}{0.078740in}}{\pgfqpoint{7.842520in}{7.842520in}}%
\pgfusepath{clip}%
\pgfsetbuttcap%
\pgfsetroundjoin%
\definecolor{currentfill}{rgb}{0.311925,0.767822,0.415586}%
\pgfsetfillcolor{currentfill}%
\pgfsetlinewidth{0.000000pt}%
\definecolor{currentstroke}{rgb}{0.255645,0.260703,0.528312}%
\pgfsetstrokecolor{currentstroke}%
\pgfsetdash{}{0pt}%
\pgfpathmoveto{\pgfqpoint{3.124144in}{3.970746in}}%
\pgfpathlineto{\pgfqpoint{3.062545in}{4.438625in}}%
\pgfpathlineto{\pgfqpoint{2.981110in}{4.034439in}}%
\pgfpathclose%
\pgfusepath{fill}%
\end{pgfscope}%
\begin{pgfscope}%
\pgfpathrectangle{\pgfqpoint{0.539299in}{0.078740in}}{\pgfqpoint{7.842520in}{7.842520in}}%
\pgfusepath{clip}%
\pgfsetbuttcap%
\pgfsetroundjoin%
\definecolor{currentfill}{rgb}{0.157851,0.683765,0.501686}%
\pgfsetfillcolor{currentfill}%
\pgfsetlinewidth{0.000000pt}%
\definecolor{currentstroke}{rgb}{0.253935,0.265254,0.529983}%
\pgfsetstrokecolor{currentstroke}%
\pgfsetdash{}{0pt}%
\pgfpathmoveto{\pgfqpoint{4.537995in}{3.619381in}}%
\pgfpathlineto{\pgfqpoint{4.621833in}{3.675341in}}%
\pgfpathlineto{\pgfqpoint{4.476731in}{3.882966in}}%
\pgfpathclose%
\pgfusepath{fill}%
\end{pgfscope}%
\begin{pgfscope}%
\pgfpathrectangle{\pgfqpoint{0.539299in}{0.078740in}}{\pgfqpoint{7.842520in}{7.842520in}}%
\pgfusepath{clip}%
\pgfsetbuttcap%
\pgfsetroundjoin%
\definecolor{currentfill}{rgb}{0.835270,0.886029,0.102646}%
\pgfsetfillcolor{currentfill}%
\pgfsetlinewidth{0.000000pt}%
\definecolor{currentstroke}{rgb}{0.252194,0.269783,0.531579}%
\pgfsetstrokecolor{currentstroke}%
\pgfsetdash{}{0pt}%
\pgfpathmoveto{\pgfqpoint{3.316139in}{5.195751in}}%
\pgfpathlineto{\pgfqpoint{3.374513in}{4.921390in}}%
\pgfpathlineto{\pgfqpoint{3.460255in}{5.092948in}}%
\pgfpathclose%
\pgfusepath{fill}%
\end{pgfscope}%
\begin{pgfscope}%
\pgfpathrectangle{\pgfqpoint{0.539299in}{0.078740in}}{\pgfqpoint{7.842520in}{7.842520in}}%
\pgfusepath{clip}%
\pgfsetbuttcap%
\pgfsetroundjoin%
\definecolor{currentfill}{rgb}{0.121148,0.592739,0.544641}%
\pgfsetfillcolor{currentfill}%
\pgfsetlinewidth{0.000000pt}%
\definecolor{currentstroke}{rgb}{0.250425,0.274290,0.533103}%
\pgfsetstrokecolor{currentstroke}%
\pgfsetdash{}{0pt}%
\pgfpathmoveto{\pgfqpoint{4.683466in}{3.421156in}}%
\pgfpathlineto{\pgfqpoint{4.828804in}{3.219577in}}%
\pgfpathlineto{\pgfqpoint{4.911519in}{3.252217in}}%
\pgfpathclose%
\pgfusepath{fill}%
\end{pgfscope}%
\begin{pgfscope}%
\pgfpathrectangle{\pgfqpoint{0.539299in}{0.078740in}}{\pgfqpoint{7.842520in}{7.842520in}}%
\pgfusepath{clip}%
\pgfsetbuttcap%
\pgfsetroundjoin%
\definecolor{currentfill}{rgb}{0.772852,0.877868,0.131109}%
\pgfsetfillcolor{currentfill}%
\pgfsetlinewidth{0.000000pt}%
\definecolor{currentstroke}{rgb}{0.248629,0.278775,0.534556}%
\pgfsetstrokecolor{currentstroke}%
\pgfsetdash{}{0pt}%
\pgfpathmoveto{\pgfqpoint{3.087223in}{5.076965in}}%
\pgfpathlineto{\pgfqpoint{3.145794in}{4.762662in}}%
\pgfpathlineto{\pgfqpoint{3.230452in}{5.012748in}}%
\pgfpathclose%
\pgfusepath{fill}%
\end{pgfscope}%
\begin{pgfscope}%
\pgfpathrectangle{\pgfqpoint{0.539299in}{0.078740in}}{\pgfqpoint{7.842520in}{7.842520in}}%
\pgfusepath{clip}%
\pgfsetbuttcap%
\pgfsetroundjoin%
\definecolor{currentfill}{rgb}{0.190631,0.407061,0.556089}%
\pgfsetfillcolor{currentfill}%
\pgfsetlinewidth{0.000000pt}%
\definecolor{currentstroke}{rgb}{0.246811,0.283237,0.535941}%
\pgfsetstrokecolor{currentstroke}%
\pgfsetdash{}{0pt}%
\pgfpathmoveto{\pgfqpoint{5.327340in}{2.342890in}}%
\pgfpathlineto{\pgfqpoint{5.408389in}{2.379042in}}%
\pgfpathlineto{\pgfqpoint{5.263813in}{2.595433in}}%
\pgfpathclose%
\pgfusepath{fill}%
\end{pgfscope}%
\begin{pgfscope}%
\pgfpathrectangle{\pgfqpoint{0.539299in}{0.078740in}}{\pgfqpoint{7.842520in}{7.842520in}}%
\pgfusepath{clip}%
\pgfsetbuttcap%
\pgfsetroundjoin%
\definecolor{currentfill}{rgb}{0.824940,0.884720,0.106217}%
\pgfsetfillcolor{currentfill}%
\pgfsetlinewidth{0.000000pt}%
\definecolor{currentstroke}{rgb}{0.244972,0.287675,0.537260}%
\pgfsetstrokecolor{currentstroke}%
\pgfsetdash{}{0pt}%
\pgfpathmoveto{\pgfqpoint{3.230452in}{5.012748in}}%
\pgfpathlineto{\pgfqpoint{3.374513in}{4.921390in}}%
\pgfpathlineto{\pgfqpoint{3.316139in}{5.195751in}}%
\pgfpathclose%
\pgfusepath{fill}%
\end{pgfscope}%
\begin{pgfscope}%
\pgfpathrectangle{\pgfqpoint{0.539299in}{0.078740in}}{\pgfqpoint{7.842520in}{7.842520in}}%
\pgfusepath{clip}%
\pgfsetbuttcap%
\pgfsetroundjoin%
\definecolor{currentfill}{rgb}{0.606045,0.850733,0.236712}%
\pgfsetfillcolor{currentfill}%
\pgfsetlinewidth{0.000000pt}%
\definecolor{currentstroke}{rgb}{0.243113,0.292092,0.538516}%
\pgfsetstrokecolor{currentstroke}%
\pgfsetdash{}{0pt}%
\pgfpathmoveto{\pgfqpoint{3.002777in}{4.819100in}}%
\pgfpathlineto{\pgfqpoint{3.062545in}{4.438625in}}%
\pgfpathlineto{\pgfqpoint{3.145794in}{4.762662in}}%
\pgfpathclose%
\pgfusepath{fill}%
\end{pgfscope}%
\begin{pgfscope}%
\pgfpathrectangle{\pgfqpoint{0.539299in}{0.078740in}}{\pgfqpoint{7.842520in}{7.842520in}}%
\pgfusepath{clip}%
\pgfsetbuttcap%
\pgfsetroundjoin%
\definecolor{currentfill}{rgb}{0.226397,0.728888,0.462789}%
\pgfsetfillcolor{currentfill}%
\pgfsetlinewidth{0.000000pt}%
\definecolor{currentstroke}{rgb}{0.241237,0.296485,0.539709}%
\pgfsetstrokecolor{currentstroke}%
\pgfsetdash{}{0pt}%
\pgfpathmoveto{\pgfqpoint{4.331483in}{4.086352in}}%
\pgfpathlineto{\pgfqpoint{4.392412in}{3.813568in}}%
\pgfpathlineto{\pgfqpoint{4.476731in}{3.882966in}}%
\pgfpathclose%
\pgfusepath{fill}%
\end{pgfscope}%
\begin{pgfscope}%
\pgfpathrectangle{\pgfqpoint{0.539299in}{0.078740in}}{\pgfqpoint{7.842520in}{7.842520in}}%
\pgfusepath{clip}%
\pgfsetbuttcap%
\pgfsetroundjoin%
\definecolor{currentfill}{rgb}{0.804182,0.882046,0.114965}%
\pgfsetfillcolor{currentfill}%
\pgfsetlinewidth{0.000000pt}%
\definecolor{currentstroke}{rgb}{0.239346,0.300855,0.540844}%
\pgfsetstrokecolor{currentstroke}%
\pgfsetdash{}{0pt}%
\pgfpathmoveto{\pgfqpoint{3.460255in}{5.092948in}}%
\pgfpathlineto{\pgfqpoint{3.374513in}{4.921390in}}%
\pgfpathlineto{\pgfqpoint{3.604941in}{4.965350in}}%
\pgfpathclose%
\pgfusepath{fill}%
\end{pgfscope}%
\begin{pgfscope}%
\pgfpathrectangle{\pgfqpoint{0.539299in}{0.078740in}}{\pgfqpoint{7.842520in}{7.842520in}}%
\pgfusepath{clip}%
\pgfsetbuttcap%
\pgfsetroundjoin%
\definecolor{currentfill}{rgb}{0.210503,0.363727,0.552206}%
\pgfsetfillcolor{currentfill}%
\pgfsetlinewidth{0.000000pt}%
\definecolor{currentstroke}{rgb}{0.237441,0.305202,0.541921}%
\pgfsetstrokecolor{currentstroke}%
\pgfsetdash{}{0pt}%
\pgfpathmoveto{\pgfqpoint{3.971998in}{2.380078in}}%
\pgfpathlineto{\pgfqpoint{3.827198in}{2.468870in}}%
\pgfpathlineto{\pgfqpoint{3.746562in}{1.887616in}}%
\pgfpathclose%
\pgfusepath{fill}%
\end{pgfscope}%
\begin{pgfscope}%
\pgfpathrectangle{\pgfqpoint{0.539299in}{0.078740in}}{\pgfqpoint{7.842520in}{7.842520in}}%
\pgfusepath{clip}%
\pgfsetbuttcap%
\pgfsetroundjoin%
\definecolor{currentfill}{rgb}{0.124780,0.640461,0.527068}%
\pgfsetfillcolor{currentfill}%
\pgfsetlinewidth{0.000000pt}%
\definecolor{currentstroke}{rgb}{0.235526,0.309527,0.542944}%
\pgfsetstrokecolor{currentstroke}%
\pgfsetdash{}{0pt}%
\pgfpathmoveto{\pgfqpoint{4.537995in}{3.619381in}}%
\pgfpathlineto{\pgfqpoint{4.683466in}{3.421156in}}%
\pgfpathlineto{\pgfqpoint{4.766767in}{3.464787in}}%
\pgfpathclose%
\pgfusepath{fill}%
\end{pgfscope}%
\begin{pgfscope}%
\pgfpathrectangle{\pgfqpoint{0.539299in}{0.078740in}}{\pgfqpoint{7.842520in}{7.842520in}}%
\pgfusepath{clip}%
\pgfsetbuttcap%
\pgfsetroundjoin%
\definecolor{currentfill}{rgb}{0.136408,0.541173,0.554483}%
\pgfsetfillcolor{currentfill}%
\pgfsetlinewidth{0.000000pt}%
\definecolor{currentstroke}{rgb}{0.233603,0.313828,0.543914}%
\pgfsetstrokecolor{currentstroke}%
\pgfsetdash{}{0pt}%
\pgfpathmoveto{\pgfqpoint{3.474971in}{3.252817in}}%
\pgfpathlineto{\pgfqpoint{3.395040in}{2.713747in}}%
\pgfpathlineto{\pgfqpoint{3.619444in}{3.158415in}}%
\pgfpathclose%
\pgfusepath{fill}%
\end{pgfscope}%
\begin{pgfscope}%
\pgfpathrectangle{\pgfqpoint{0.539299in}{0.078740in}}{\pgfqpoint{7.842520in}{7.842520in}}%
\pgfusepath{clip}%
\pgfsetbuttcap%
\pgfsetroundjoin%
\definecolor{currentfill}{rgb}{0.239346,0.300855,0.540844}%
\pgfsetfillcolor{currentfill}%
\pgfsetlinewidth{0.000000pt}%
\definecolor{currentstroke}{rgb}{0.231674,0.318106,0.544834}%
\pgfsetstrokecolor{currentstroke}%
\pgfsetdash{}{0pt}%
\pgfpathmoveto{\pgfqpoint{3.971998in}{2.380078in}}%
\pgfpathlineto{\pgfqpoint{3.890478in}{1.820244in}}%
\pgfpathlineto{\pgfqpoint{4.034810in}{1.751253in}}%
\pgfpathclose%
\pgfusepath{fill}%
\end{pgfscope}%
\begin{pgfscope}%
\pgfpathrectangle{\pgfqpoint{0.539299in}{0.078740in}}{\pgfqpoint{7.842520in}{7.842520in}}%
\pgfusepath{clip}%
\pgfsetbuttcap%
\pgfsetroundjoin%
\definecolor{currentfill}{rgb}{0.126326,0.644107,0.525311}%
\pgfsetfillcolor{currentfill}%
\pgfsetlinewidth{0.000000pt}%
\definecolor{currentstroke}{rgb}{0.229739,0.322361,0.545706}%
\pgfsetstrokecolor{currentstroke}%
\pgfsetdash{}{0pt}%
\pgfpathmoveto{\pgfqpoint{3.187325in}{3.418679in}}%
\pgfpathlineto{\pgfqpoint{3.330909in}{3.339983in}}%
\pgfpathlineto{\pgfqpoint{3.412099in}{3.800510in}}%
\pgfpathclose%
\pgfusepath{fill}%
\end{pgfscope}%
\begin{pgfscope}%
\pgfpathrectangle{\pgfqpoint{0.539299in}{0.078740in}}{\pgfqpoint{7.842520in}{7.842520in}}%
\pgfusepath{clip}%
\pgfsetbuttcap%
\pgfsetroundjoin%
\definecolor{currentfill}{rgb}{0.377779,0.791781,0.377939}%
\pgfsetfillcolor{currentfill}%
\pgfsetlinewidth{0.000000pt}%
\definecolor{currentstroke}{rgb}{0.227802,0.326594,0.546532}%
\pgfsetstrokecolor{currentstroke}%
\pgfsetdash{}{0pt}%
\pgfpathmoveto{\pgfqpoint{3.206070in}{4.367077in}}%
\pgfpathlineto{\pgfqpoint{3.062545in}{4.438625in}}%
\pgfpathlineto{\pgfqpoint{3.124144in}{3.970746in}}%
\pgfpathclose%
\pgfusepath{fill}%
\end{pgfscope}%
\begin{pgfscope}%
\pgfpathrectangle{\pgfqpoint{0.539299in}{0.078740in}}{\pgfqpoint{7.842520in}{7.842520in}}%
\pgfusepath{clip}%
\pgfsetbuttcap%
\pgfsetroundjoin%
\definecolor{currentfill}{rgb}{0.204903,0.375746,0.553533}%
\pgfsetfillcolor{currentfill}%
\pgfsetlinewidth{0.000000pt}%
\definecolor{currentstroke}{rgb}{0.225863,0.330805,0.547314}%
\pgfsetstrokecolor{currentstroke}%
\pgfsetdash{}{0pt}%
\pgfpathmoveto{\pgfqpoint{5.408389in}{2.379042in}}%
\pgfpathlineto{\pgfqpoint{5.327340in}{2.342890in}}%
\pgfpathlineto{\pgfqpoint{5.472274in}{2.124924in}}%
\pgfpathclose%
\pgfusepath{fill}%
\end{pgfscope}%
\begin{pgfscope}%
\pgfpathrectangle{\pgfqpoint{0.539299in}{0.078740in}}{\pgfqpoint{7.842520in}{7.842520in}}%
\pgfusepath{clip}%
\pgfsetbuttcap%
\pgfsetroundjoin%
\definecolor{currentfill}{rgb}{0.162142,0.474838,0.558140}%
\pgfsetfillcolor{currentfill}%
\pgfsetlinewidth{0.000000pt}%
\definecolor{currentstroke}{rgb}{0.223925,0.334994,0.548053}%
\pgfsetstrokecolor{currentstroke}%
\pgfsetdash{}{0pt}%
\pgfpathmoveto{\pgfqpoint{5.263813in}{2.595433in}}%
\pgfpathlineto{\pgfqpoint{5.119001in}{2.807062in}}%
\pgfpathlineto{\pgfqpoint{5.036687in}{2.752478in}}%
\pgfpathclose%
\pgfusepath{fill}%
\end{pgfscope}%
\begin{pgfscope}%
\pgfpathrectangle{\pgfqpoint{0.539299in}{0.078740in}}{\pgfqpoint{7.842520in}{7.842520in}}%
\pgfusepath{clip}%
\pgfsetbuttcap%
\pgfsetroundjoin%
\definecolor{currentfill}{rgb}{0.344074,0.780029,0.397381}%
\pgfsetfillcolor{currentfill}%
\pgfsetlinewidth{0.000000pt}%
\definecolor{currentstroke}{rgb}{0.221989,0.339161,0.548752}%
\pgfsetstrokecolor{currentstroke}%
\pgfsetdash{}{0pt}%
\pgfpathmoveto{\pgfqpoint{4.186126in}{4.283787in}}%
\pgfpathlineto{\pgfqpoint{4.101044in}{4.185140in}}%
\pgfpathlineto{\pgfqpoint{4.331483in}{4.086352in}}%
\pgfpathclose%
\pgfusepath{fill}%
\end{pgfscope}%
\begin{pgfscope}%
\pgfpathrectangle{\pgfqpoint{0.539299in}{0.078740in}}{\pgfqpoint{7.842520in}{7.842520in}}%
\pgfusepath{clip}%
\pgfsetbuttcap%
\pgfsetroundjoin%
\definecolor{currentfill}{rgb}{0.159194,0.482237,0.558073}%
\pgfsetfillcolor{currentfill}%
\pgfsetlinewidth{0.000000pt}%
\definecolor{currentstroke}{rgb}{0.220057,0.343307,0.549413}%
\pgfsetstrokecolor{currentstroke}%
\pgfsetdash{}{0pt}%
\pgfpathmoveto{\pgfqpoint{3.764278in}{3.057833in}}%
\pgfpathlineto{\pgfqpoint{3.538693in}{2.636201in}}%
\pgfpathlineto{\pgfqpoint{3.682755in}{2.554388in}}%
\pgfpathclose%
\pgfusepath{fill}%
\end{pgfscope}%
\begin{pgfscope}%
\pgfpathrectangle{\pgfqpoint{0.539299in}{0.078740in}}{\pgfqpoint{7.842520in}{7.842520in}}%
\pgfusepath{clip}%
\pgfsetbuttcap%
\pgfsetroundjoin%
\definecolor{currentfill}{rgb}{0.162016,0.687316,0.499129}%
\pgfsetfillcolor{currentfill}%
\pgfsetlinewidth{0.000000pt}%
\definecolor{currentstroke}{rgb}{0.218130,0.347432,0.550038}%
\pgfsetstrokecolor{currentstroke}%
\pgfsetdash{}{0pt}%
\pgfpathmoveto{\pgfqpoint{3.412099in}{3.800510in}}%
\pgfpathlineto{\pgfqpoint{3.267846in}{3.892059in}}%
\pgfpathlineto{\pgfqpoint{3.187325in}{3.418679in}}%
\pgfpathclose%
\pgfusepath{fill}%
\end{pgfscope}%
\begin{pgfscope}%
\pgfpathrectangle{\pgfqpoint{0.539299in}{0.078740in}}{\pgfqpoint{7.842520in}{7.842520in}}%
\pgfusepath{clip}%
\pgfsetbuttcap%
\pgfsetroundjoin%
\definecolor{currentfill}{rgb}{0.699415,0.867117,0.175971}%
\pgfsetfillcolor{currentfill}%
\pgfsetlinewidth{0.000000pt}%
\definecolor{currentstroke}{rgb}{0.216210,0.351535,0.550627}%
\pgfsetstrokecolor{currentstroke}%
\pgfsetdash{}{0pt}%
\pgfpathmoveto{\pgfqpoint{3.230452in}{5.012748in}}%
\pgfpathlineto{\pgfqpoint{3.145794in}{4.762662in}}%
\pgfpathlineto{\pgfqpoint{3.289659in}{4.681825in}}%
\pgfpathclose%
\pgfusepath{fill}%
\end{pgfscope}%
\begin{pgfscope}%
\pgfpathrectangle{\pgfqpoint{0.539299in}{0.078740in}}{\pgfqpoint{7.842520in}{7.842520in}}%
\pgfusepath{clip}%
\pgfsetbuttcap%
\pgfsetroundjoin%
\definecolor{currentfill}{rgb}{0.585678,0.846661,0.249897}%
\pgfsetfillcolor{currentfill}%
\pgfsetlinewidth{0.000000pt}%
\definecolor{currentstroke}{rgb}{0.214298,0.355619,0.551184}%
\pgfsetstrokecolor{currentstroke}%
\pgfsetdash{}{0pt}%
\pgfpathmoveto{\pgfqpoint{3.062545in}{4.438625in}}%
\pgfpathlineto{\pgfqpoint{3.289659in}{4.681825in}}%
\pgfpathlineto{\pgfqpoint{3.145794in}{4.762662in}}%
\pgfpathclose%
\pgfusepath{fill}%
\end{pgfscope}%
\begin{pgfscope}%
\pgfpathrectangle{\pgfqpoint{0.539299in}{0.078740in}}{\pgfqpoint{7.842520in}{7.842520in}}%
\pgfusepath{clip}%
\pgfsetbuttcap%
\pgfsetroundjoin%
\definecolor{currentfill}{rgb}{0.412913,0.803041,0.357269}%
\pgfsetfillcolor{currentfill}%
\pgfsetlinewidth{0.000000pt}%
\definecolor{currentstroke}{rgb}{0.212395,0.359683,0.551710}%
\pgfsetstrokecolor{currentstroke}%
\pgfsetdash{}{0pt}%
\pgfpathmoveto{\pgfqpoint{4.101044in}{4.185140in}}%
\pgfpathlineto{\pgfqpoint{4.186126in}{4.283787in}}%
\pgfpathlineto{\pgfqpoint{4.040710in}{4.473147in}}%
\pgfpathclose%
\pgfusepath{fill}%
\end{pgfscope}%
\begin{pgfscope}%
\pgfpathrectangle{\pgfqpoint{0.539299in}{0.078740in}}{\pgfqpoint{7.842520in}{7.842520in}}%
\pgfusepath{clip}%
\pgfsetbuttcap%
\pgfsetroundjoin%
\definecolor{currentfill}{rgb}{0.699415,0.867117,0.175971}%
\pgfsetfillcolor{currentfill}%
\pgfsetlinewidth{0.000000pt}%
\definecolor{currentstroke}{rgb}{0.210503,0.363727,0.552206}%
\pgfsetstrokecolor{currentstroke}%
\pgfsetdash{}{0pt}%
\pgfpathmoveto{\pgfqpoint{3.664322in}{4.672860in}}%
\pgfpathlineto{\pgfqpoint{3.750009in}{4.817083in}}%
\pgfpathlineto{\pgfqpoint{3.604941in}{4.965350in}}%
\pgfpathclose%
\pgfusepath{fill}%
\end{pgfscope}%
\begin{pgfscope}%
\pgfpathrectangle{\pgfqpoint{0.539299in}{0.078740in}}{\pgfqpoint{7.842520in}{7.842520in}}%
\pgfusepath{clip}%
\pgfsetbuttcap%
\pgfsetroundjoin%
\definecolor{currentfill}{rgb}{0.180653,0.701402,0.488189}%
\pgfsetfillcolor{currentfill}%
\pgfsetlinewidth{0.000000pt}%
\definecolor{currentstroke}{rgb}{0.208623,0.367752,0.552675}%
\pgfsetstrokecolor{currentstroke}%
\pgfsetdash{}{0pt}%
\pgfpathmoveto{\pgfqpoint{4.476731in}{3.882966in}}%
\pgfpathlineto{\pgfqpoint{4.392412in}{3.813568in}}%
\pgfpathlineto{\pgfqpoint{4.537995in}{3.619381in}}%
\pgfpathclose%
\pgfusepath{fill}%
\end{pgfscope}%
\begin{pgfscope}%
\pgfpathrectangle{\pgfqpoint{0.539299in}{0.078740in}}{\pgfqpoint{7.842520in}{7.842520in}}%
\pgfusepath{clip}%
\pgfsetbuttcap%
\pgfsetroundjoin%
\definecolor{currentfill}{rgb}{0.751884,0.874951,0.143228}%
\pgfsetfillcolor{currentfill}%
\pgfsetlinewidth{0.000000pt}%
\definecolor{currentstroke}{rgb}{0.206756,0.371758,0.553117}%
\pgfsetstrokecolor{currentstroke}%
\pgfsetdash{}{0pt}%
\pgfpathmoveto{\pgfqpoint{3.604941in}{4.965350in}}%
\pgfpathlineto{\pgfqpoint{3.374513in}{4.921390in}}%
\pgfpathlineto{\pgfqpoint{3.519195in}{4.806876in}}%
\pgfpathclose%
\pgfusepath{fill}%
\end{pgfscope}%
\begin{pgfscope}%
\pgfpathrectangle{\pgfqpoint{0.539299in}{0.078740in}}{\pgfqpoint{7.842520in}{7.842520in}}%
\pgfusepath{clip}%
\pgfsetbuttcap%
\pgfsetroundjoin%
\definecolor{currentfill}{rgb}{0.741388,0.873449,0.149561}%
\pgfsetfillcolor{currentfill}%
\pgfsetlinewidth{0.000000pt}%
\definecolor{currentstroke}{rgb}{0.204903,0.375746,0.553533}%
\pgfsetstrokecolor{currentstroke}%
\pgfsetdash{}{0pt}%
\pgfpathmoveto{\pgfqpoint{3.289659in}{4.681825in}}%
\pgfpathlineto{\pgfqpoint{3.374513in}{4.921390in}}%
\pgfpathlineto{\pgfqpoint{3.230452in}{5.012748in}}%
\pgfpathclose%
\pgfusepath{fill}%
\end{pgfscope}%
\begin{pgfscope}%
\pgfpathrectangle{\pgfqpoint{0.539299in}{0.078740in}}{\pgfqpoint{7.842520in}{7.842520in}}%
\pgfusepath{clip}%
\pgfsetbuttcap%
\pgfsetroundjoin%
\definecolor{currentfill}{rgb}{0.278012,0.180367,0.486697}%
\pgfsetfillcolor{currentfill}%
\pgfsetlinewidth{0.000000pt}%
\definecolor{currentstroke}{rgb}{0.203063,0.379716,0.553925}%
\pgfsetstrokecolor{currentstroke}%
\pgfsetdash{}{0pt}%
\pgfpathmoveto{\pgfqpoint{5.760821in}{1.647418in}}%
\pgfpathlineto{\pgfqpoint{5.680690in}{1.591684in}}%
\pgfpathlineto{\pgfqpoint{5.824454in}{1.317143in}}%
\pgfpathclose%
\pgfusepath{fill}%
\end{pgfscope}%
\begin{pgfscope}%
\pgfpathrectangle{\pgfqpoint{0.539299in}{0.078740in}}{\pgfqpoint{7.842520in}{7.842520in}}%
\pgfusepath{clip}%
\pgfsetbuttcap%
\pgfsetroundjoin%
\definecolor{currentfill}{rgb}{0.150476,0.504369,0.557430}%
\pgfsetfillcolor{currentfill}%
\pgfsetlinewidth{0.000000pt}%
\definecolor{currentstroke}{rgb}{0.201239,0.383670,0.554294}%
\pgfsetstrokecolor{currentstroke}%
\pgfsetdash{}{0pt}%
\pgfpathmoveto{\pgfqpoint{5.036687in}{2.752478in}}%
\pgfpathlineto{\pgfqpoint{5.119001in}{2.807062in}}%
\pgfpathlineto{\pgfqpoint{4.973989in}{3.014923in}}%
\pgfpathclose%
\pgfusepath{fill}%
\end{pgfscope}%
\begin{pgfscope}%
\pgfpathrectangle{\pgfqpoint{0.539299in}{0.078740in}}{\pgfqpoint{7.842520in}{7.842520in}}%
\pgfusepath{clip}%
\pgfsetbuttcap%
\pgfsetroundjoin%
\definecolor{currentfill}{rgb}{0.545524,0.838039,0.275626}%
\pgfsetfillcolor{currentfill}%
\pgfsetlinewidth{0.000000pt}%
\definecolor{currentstroke}{rgb}{0.199430,0.387607,0.554642}%
\pgfsetstrokecolor{currentstroke}%
\pgfsetdash{}{0pt}%
\pgfpathmoveto{\pgfqpoint{3.895306in}{4.651896in}}%
\pgfpathlineto{\pgfqpoint{3.809748in}{4.522655in}}%
\pgfpathlineto{\pgfqpoint{4.040710in}{4.473147in}}%
\pgfpathclose%
\pgfusepath{fill}%
\end{pgfscope}%
\begin{pgfscope}%
\pgfpathrectangle{\pgfqpoint{0.539299in}{0.078740in}}{\pgfqpoint{7.842520in}{7.842520in}}%
\pgfusepath{clip}%
\pgfsetbuttcap%
\pgfsetroundjoin%
\definecolor{currentfill}{rgb}{0.606045,0.850733,0.236712}%
\pgfsetfillcolor{currentfill}%
\pgfsetlinewidth{0.000000pt}%
\definecolor{currentstroke}{rgb}{0.197636,0.391528,0.554969}%
\pgfsetstrokecolor{currentstroke}%
\pgfsetdash{}{0pt}%
\pgfpathmoveto{\pgfqpoint{3.895306in}{4.651896in}}%
\pgfpathlineto{\pgfqpoint{3.750009in}{4.817083in}}%
\pgfpathlineto{\pgfqpoint{3.809748in}{4.522655in}}%
\pgfpathclose%
\pgfusepath{fill}%
\end{pgfscope}%
\begin{pgfscope}%
\pgfpathrectangle{\pgfqpoint{0.539299in}{0.078740in}}{\pgfqpoint{7.842520in}{7.842520in}}%
\pgfusepath{clip}%
\pgfsetbuttcap%
\pgfsetroundjoin%
\definecolor{currentfill}{rgb}{0.123444,0.636809,0.528763}%
\pgfsetfillcolor{currentfill}%
\pgfsetlinewidth{0.000000pt}%
\definecolor{currentstroke}{rgb}{0.195860,0.395433,0.555276}%
\pgfsetstrokecolor{currentstroke}%
\pgfsetdash{}{0pt}%
\pgfpathmoveto{\pgfqpoint{3.412099in}{3.800510in}}%
\pgfpathlineto{\pgfqpoint{3.330909in}{3.339983in}}%
\pgfpathlineto{\pgfqpoint{3.474971in}{3.252817in}}%
\pgfpathclose%
\pgfusepath{fill}%
\end{pgfscope}%
\begin{pgfscope}%
\pgfpathrectangle{\pgfqpoint{0.539299in}{0.078740in}}{\pgfqpoint{7.842520in}{7.842520in}}%
\pgfusepath{clip}%
\pgfsetbuttcap%
\pgfsetroundjoin%
\definecolor{currentfill}{rgb}{0.257322,0.256130,0.526563}%
\pgfsetfillcolor{currentfill}%
\pgfsetlinewidth{0.000000pt}%
\definecolor{currentstroke}{rgb}{0.194100,0.399323,0.555565}%
\pgfsetstrokecolor{currentstroke}%
\pgfsetdash{}{0pt}%
\pgfpathmoveto{\pgfqpoint{5.760821in}{1.647418in}}%
\pgfpathlineto{\pgfqpoint{5.616816in}{1.894378in}}%
\pgfpathlineto{\pgfqpoint{5.536081in}{1.836733in}}%
\pgfpathclose%
\pgfusepath{fill}%
\end{pgfscope}%
\begin{pgfscope}%
\pgfpathrectangle{\pgfqpoint{0.539299in}{0.078740in}}{\pgfqpoint{7.842520in}{7.842520in}}%
\pgfusepath{clip}%
\pgfsetbuttcap%
\pgfsetroundjoin%
\definecolor{currentfill}{rgb}{0.515992,0.831158,0.294279}%
\pgfsetfillcolor{currentfill}%
\pgfsetlinewidth{0.000000pt}%
\definecolor{currentstroke}{rgb}{0.192357,0.403199,0.555836}%
\pgfsetstrokecolor{currentstroke}%
\pgfsetdash{}{0pt}%
\pgfpathmoveto{\pgfqpoint{3.206070in}{4.367077in}}%
\pgfpathlineto{\pgfqpoint{3.289659in}{4.681825in}}%
\pgfpathlineto{\pgfqpoint{3.062545in}{4.438625in}}%
\pgfpathclose%
\pgfusepath{fill}%
\end{pgfscope}%
\begin{pgfscope}%
\pgfpathrectangle{\pgfqpoint{0.539299in}{0.078740in}}{\pgfqpoint{7.842520in}{7.842520in}}%
\pgfusepath{clip}%
\pgfsetbuttcap%
\pgfsetroundjoin%
\definecolor{currentfill}{rgb}{0.180629,0.429975,0.557282}%
\pgfsetfillcolor{currentfill}%
\pgfsetlinewidth{0.000000pt}%
\definecolor{currentstroke}{rgb}{0.190631,0.407061,0.556089}%
\pgfsetstrokecolor{currentstroke}%
\pgfsetdash{}{0pt}%
\pgfpathmoveto{\pgfqpoint{5.263813in}{2.595433in}}%
\pgfpathlineto{\pgfqpoint{5.182120in}{2.551317in}}%
\pgfpathlineto{\pgfqpoint{5.327340in}{2.342890in}}%
\pgfpathclose%
\pgfusepath{fill}%
\end{pgfscope}%
\begin{pgfscope}%
\pgfpathrectangle{\pgfqpoint{0.539299in}{0.078740in}}{\pgfqpoint{7.842520in}{7.842520in}}%
\pgfusepath{clip}%
\pgfsetbuttcap%
\pgfsetroundjoin%
\definecolor{currentfill}{rgb}{0.288921,0.758394,0.428426}%
\pgfsetfillcolor{currentfill}%
\pgfsetlinewidth{0.000000pt}%
\definecolor{currentstroke}{rgb}{0.188923,0.410910,0.556326}%
\pgfsetstrokecolor{currentstroke}%
\pgfsetdash{}{0pt}%
\pgfpathmoveto{\pgfqpoint{3.124144in}{3.970746in}}%
\pgfpathlineto{\pgfqpoint{3.267846in}{3.892059in}}%
\pgfpathlineto{\pgfqpoint{3.350268in}{4.277351in}}%
\pgfpathclose%
\pgfusepath{fill}%
\end{pgfscope}%
\begin{pgfscope}%
\pgfpathrectangle{\pgfqpoint{0.539299in}{0.078740in}}{\pgfqpoint{7.842520in}{7.842520in}}%
\pgfusepath{clip}%
\pgfsetbuttcap%
\pgfsetroundjoin%
\definecolor{currentfill}{rgb}{0.252899,0.742211,0.448284}%
\pgfsetfillcolor{currentfill}%
\pgfsetlinewidth{0.000000pt}%
\definecolor{currentstroke}{rgb}{0.187231,0.414746,0.556547}%
\pgfsetstrokecolor{currentstroke}%
\pgfsetdash{}{0pt}%
\pgfpathmoveto{\pgfqpoint{4.246749in}{4.002642in}}%
\pgfpathlineto{\pgfqpoint{4.392412in}{3.813568in}}%
\pgfpathlineto{\pgfqpoint{4.331483in}{4.086352in}}%
\pgfpathclose%
\pgfusepath{fill}%
\end{pgfscope}%
\begin{pgfscope}%
\pgfpathrectangle{\pgfqpoint{0.539299in}{0.078740in}}{\pgfqpoint{7.842520in}{7.842520in}}%
\pgfusepath{clip}%
\pgfsetbuttcap%
\pgfsetroundjoin%
\definecolor{currentfill}{rgb}{0.141935,0.526453,0.555991}%
\pgfsetfillcolor{currentfill}%
\pgfsetlinewidth{0.000000pt}%
\definecolor{currentstroke}{rgb}{0.185556,0.418570,0.556753}%
\pgfsetstrokecolor{currentstroke}%
\pgfsetdash{}{0pt}%
\pgfpathmoveto{\pgfqpoint{3.764278in}{3.057833in}}%
\pgfpathlineto{\pgfqpoint{3.619444in}{3.158415in}}%
\pgfpathlineto{\pgfqpoint{3.538693in}{2.636201in}}%
\pgfpathclose%
\pgfusepath{fill}%
\end{pgfscope}%
\begin{pgfscope}%
\pgfpathrectangle{\pgfqpoint{0.539299in}{0.078740in}}{\pgfqpoint{7.842520in}{7.842520in}}%
\pgfusepath{clip}%
\pgfsetbuttcap%
\pgfsetroundjoin%
\definecolor{currentfill}{rgb}{0.360741,0.785964,0.387814}%
\pgfsetfillcolor{currentfill}%
\pgfsetlinewidth{0.000000pt}%
\definecolor{currentstroke}{rgb}{0.183898,0.422383,0.556944}%
\pgfsetstrokecolor{currentstroke}%
\pgfsetdash{}{0pt}%
\pgfpathmoveto{\pgfqpoint{3.206070in}{4.367077in}}%
\pgfpathlineto{\pgfqpoint{3.124144in}{3.970746in}}%
\pgfpathlineto{\pgfqpoint{3.350268in}{4.277351in}}%
\pgfpathclose%
\pgfusepath{fill}%
\end{pgfscope}%
\begin{pgfscope}%
\pgfpathrectangle{\pgfqpoint{0.539299in}{0.078740in}}{\pgfqpoint{7.842520in}{7.842520in}}%
\pgfusepath{clip}%
\pgfsetbuttcap%
\pgfsetroundjoin%
\definecolor{currentfill}{rgb}{0.709898,0.868751,0.169257}%
\pgfsetfillcolor{currentfill}%
\pgfsetlinewidth{0.000000pt}%
\definecolor{currentstroke}{rgb}{0.182256,0.426184,0.557120}%
\pgfsetstrokecolor{currentstroke}%
\pgfsetdash{}{0pt}%
\pgfpathmoveto{\pgfqpoint{3.519195in}{4.806876in}}%
\pgfpathlineto{\pgfqpoint{3.664322in}{4.672860in}}%
\pgfpathlineto{\pgfqpoint{3.604941in}{4.965350in}}%
\pgfpathclose%
\pgfusepath{fill}%
\end{pgfscope}%
\begin{pgfscope}%
\pgfpathrectangle{\pgfqpoint{0.539299in}{0.078740in}}{\pgfqpoint{7.842520in}{7.842520in}}%
\pgfusepath{clip}%
\pgfsetbuttcap%
\pgfsetroundjoin%
\definecolor{currentfill}{rgb}{0.304148,0.764704,0.419943}%
\pgfsetfillcolor{currentfill}%
\pgfsetlinewidth{0.000000pt}%
\definecolor{currentstroke}{rgb}{0.180629,0.429975,0.557282}%
\pgfsetstrokecolor{currentstroke}%
\pgfsetdash{}{0pt}%
\pgfpathmoveto{\pgfqpoint{4.331483in}{4.086352in}}%
\pgfpathlineto{\pgfqpoint{4.101044in}{4.185140in}}%
\pgfpathlineto{\pgfqpoint{4.246749in}{4.002642in}}%
\pgfpathclose%
\pgfusepath{fill}%
\end{pgfscope}%
\begin{pgfscope}%
\pgfpathrectangle{\pgfqpoint{0.539299in}{0.078740in}}{\pgfqpoint{7.842520in}{7.842520in}}%
\pgfusepath{clip}%
\pgfsetbuttcap%
\pgfsetroundjoin%
\definecolor{currentfill}{rgb}{0.132444,0.552216,0.553018}%
\pgfsetfillcolor{currentfill}%
\pgfsetlinewidth{0.000000pt}%
\definecolor{currentstroke}{rgb}{0.179019,0.433756,0.557430}%
\pgfsetstrokecolor{currentstroke}%
\pgfsetdash{}{0pt}%
\pgfpathmoveto{\pgfqpoint{4.828804in}{3.219577in}}%
\pgfpathlineto{\pgfqpoint{4.891094in}{2.947957in}}%
\pgfpathlineto{\pgfqpoint{4.973989in}{3.014923in}}%
\pgfpathclose%
\pgfusepath{fill}%
\end{pgfscope}%
\begin{pgfscope}%
\pgfpathrectangle{\pgfqpoint{0.539299in}{0.078740in}}{\pgfqpoint{7.842520in}{7.842520in}}%
\pgfusepath{clip}%
\pgfsetbuttcap%
\pgfsetroundjoin%
\definecolor{currentfill}{rgb}{0.168126,0.459988,0.558082}%
\pgfsetfillcolor{currentfill}%
\pgfsetlinewidth{0.000000pt}%
\definecolor{currentstroke}{rgb}{0.177423,0.437527,0.557565}%
\pgfsetstrokecolor{currentstroke}%
\pgfsetdash{}{0pt}%
\pgfpathmoveto{\pgfqpoint{5.036687in}{2.752478in}}%
\pgfpathlineto{\pgfqpoint{5.182120in}{2.551317in}}%
\pgfpathlineto{\pgfqpoint{5.263813in}{2.595433in}}%
\pgfpathclose%
\pgfusepath{fill}%
\end{pgfscope}%
\begin{pgfscope}%
\pgfpathrectangle{\pgfqpoint{0.539299in}{0.078740in}}{\pgfqpoint{7.842520in}{7.842520in}}%
\pgfusepath{clip}%
\pgfsetbuttcap%
\pgfsetroundjoin%
\definecolor{currentfill}{rgb}{0.699415,0.867117,0.175971}%
\pgfsetfillcolor{currentfill}%
\pgfsetlinewidth{0.000000pt}%
\definecolor{currentstroke}{rgb}{0.175841,0.441290,0.557685}%
\pgfsetstrokecolor{currentstroke}%
\pgfsetdash{}{0pt}%
\pgfpathmoveto{\pgfqpoint{3.519195in}{4.806876in}}%
\pgfpathlineto{\pgfqpoint{3.374513in}{4.921390in}}%
\pgfpathlineto{\pgfqpoint{3.289659in}{4.681825in}}%
\pgfpathclose%
\pgfusepath{fill}%
\end{pgfscope}%
\begin{pgfscope}%
\pgfpathrectangle{\pgfqpoint{0.539299in}{0.078740in}}{\pgfqpoint{7.842520in}{7.842520in}}%
\pgfusepath{clip}%
\pgfsetbuttcap%
\pgfsetroundjoin%
\definecolor{currentfill}{rgb}{0.626579,0.854645,0.223353}%
\pgfsetfillcolor{currentfill}%
\pgfsetlinewidth{0.000000pt}%
\definecolor{currentstroke}{rgb}{0.174274,0.445044,0.557792}%
\pgfsetstrokecolor{currentstroke}%
\pgfsetdash{}{0pt}%
\pgfpathmoveto{\pgfqpoint{3.809748in}{4.522655in}}%
\pgfpathlineto{\pgfqpoint{3.750009in}{4.817083in}}%
\pgfpathlineto{\pgfqpoint{3.664322in}{4.672860in}}%
\pgfpathclose%
\pgfusepath{fill}%
\end{pgfscope}%
\begin{pgfscope}%
\pgfpathrectangle{\pgfqpoint{0.539299in}{0.078740in}}{\pgfqpoint{7.842520in}{7.842520in}}%
\pgfusepath{clip}%
\pgfsetbuttcap%
\pgfsetroundjoin%
\definecolor{currentfill}{rgb}{0.165117,0.467423,0.558141}%
\pgfsetfillcolor{currentfill}%
\pgfsetlinewidth{0.000000pt}%
\definecolor{currentstroke}{rgb}{0.172719,0.448791,0.557885}%
\pgfsetstrokecolor{currentstroke}%
\pgfsetdash{}{0pt}%
\pgfpathmoveto{\pgfqpoint{3.682755in}{2.554388in}}%
\pgfpathlineto{\pgfqpoint{3.827198in}{2.468870in}}%
\pgfpathlineto{\pgfqpoint{3.909430in}{2.951944in}}%
\pgfpathclose%
\pgfusepath{fill}%
\end{pgfscope}%
\begin{pgfscope}%
\pgfpathrectangle{\pgfqpoint{0.539299in}{0.078740in}}{\pgfqpoint{7.842520in}{7.842520in}}%
\pgfusepath{clip}%
\pgfsetbuttcap%
\pgfsetroundjoin%
\definecolor{currentfill}{rgb}{0.430983,0.808473,0.346476}%
\pgfsetfillcolor{currentfill}%
\pgfsetlinewidth{0.000000pt}%
\definecolor{currentstroke}{rgb}{0.171176,0.452530,0.557965}%
\pgfsetstrokecolor{currentstroke}%
\pgfsetdash{}{0pt}%
\pgfpathmoveto{\pgfqpoint{4.101044in}{4.185140in}}%
\pgfpathlineto{\pgfqpoint{4.040710in}{4.473147in}}%
\pgfpathlineto{\pgfqpoint{3.955353in}{4.359218in}}%
\pgfpathclose%
\pgfusepath{fill}%
\end{pgfscope}%
\begin{pgfscope}%
\pgfpathrectangle{\pgfqpoint{0.539299in}{0.078740in}}{\pgfqpoint{7.842520in}{7.842520in}}%
\pgfusepath{clip}%
\pgfsetbuttcap%
\pgfsetroundjoin%
\definecolor{currentfill}{rgb}{0.496615,0.826376,0.306377}%
\pgfsetfillcolor{currentfill}%
\pgfsetlinewidth{0.000000pt}%
\definecolor{currentstroke}{rgb}{0.169646,0.456262,0.558030}%
\pgfsetstrokecolor{currentstroke}%
\pgfsetdash{}{0pt}%
\pgfpathmoveto{\pgfqpoint{4.040710in}{4.473147in}}%
\pgfpathlineto{\pgfqpoint{3.809748in}{4.522655in}}%
\pgfpathlineto{\pgfqpoint{3.955353in}{4.359218in}}%
\pgfpathclose%
\pgfusepath{fill}%
\end{pgfscope}%
\begin{pgfscope}%
\pgfpathrectangle{\pgfqpoint{0.539299in}{0.078740in}}{\pgfqpoint{7.842520in}{7.842520in}}%
\pgfusepath{clip}%
\pgfsetbuttcap%
\pgfsetroundjoin%
\definecolor{currentfill}{rgb}{0.237441,0.305202,0.541921}%
\pgfsetfillcolor{currentfill}%
\pgfsetlinewidth{0.000000pt}%
\definecolor{currentstroke}{rgb}{0.168126,0.459988,0.558082}%
\pgfsetstrokecolor{currentstroke}%
\pgfsetdash{}{0pt}%
\pgfpathmoveto{\pgfqpoint{5.472274in}{2.124924in}}%
\pgfpathlineto{\pgfqpoint{5.536081in}{1.836733in}}%
\pgfpathlineto{\pgfqpoint{5.616816in}{1.894378in}}%
\pgfpathclose%
\pgfusepath{fill}%
\end{pgfscope}%
\begin{pgfscope}%
\pgfpathrectangle{\pgfqpoint{0.539299in}{0.078740in}}{\pgfqpoint{7.842520in}{7.842520in}}%
\pgfusepath{clip}%
\pgfsetbuttcap%
\pgfsetroundjoin%
\definecolor{currentfill}{rgb}{0.265145,0.232956,0.516599}%
\pgfsetfillcolor{currentfill}%
\pgfsetlinewidth{0.000000pt}%
\definecolor{currentstroke}{rgb}{0.166617,0.463708,0.558119}%
\pgfsetstrokecolor{currentstroke}%
\pgfsetdash{}{0pt}%
\pgfpathmoveto{\pgfqpoint{5.536081in}{1.836733in}}%
\pgfpathlineto{\pgfqpoint{5.680690in}{1.591684in}}%
\pgfpathlineto{\pgfqpoint{5.760821in}{1.647418in}}%
\pgfpathclose%
\pgfusepath{fill}%
\end{pgfscope}%
\begin{pgfscope}%
\pgfpathrectangle{\pgfqpoint{0.539299in}{0.078740in}}{\pgfqpoint{7.842520in}{7.842520in}}%
\pgfusepath{clip}%
\pgfsetbuttcap%
\pgfsetroundjoin%
\definecolor{currentfill}{rgb}{0.216210,0.351535,0.550627}%
\pgfsetfillcolor{currentfill}%
\pgfsetlinewidth{0.000000pt}%
\definecolor{currentstroke}{rgb}{0.165117,0.467423,0.558141}%
\pgfsetstrokecolor{currentstroke}%
\pgfsetdash{}{0pt}%
\pgfpathmoveto{\pgfqpoint{4.034810in}{1.751253in}}%
\pgfpathlineto{\pgfqpoint{4.117136in}{2.288294in}}%
\pgfpathlineto{\pgfqpoint{3.971998in}{2.380078in}}%
\pgfpathclose%
\pgfusepath{fill}%
\end{pgfscope}%
\begin{pgfscope}%
\pgfpathrectangle{\pgfqpoint{0.539299in}{0.078740in}}{\pgfqpoint{7.842520in}{7.842520in}}%
\pgfusepath{clip}%
\pgfsetbuttcap%
\pgfsetroundjoin%
\definecolor{currentfill}{rgb}{0.246811,0.283237,0.535941}%
\pgfsetfillcolor{currentfill}%
\pgfsetlinewidth{0.000000pt}%
\definecolor{currentstroke}{rgb}{0.163625,0.471133,0.558148}%
\pgfsetstrokecolor{currentstroke}%
\pgfsetdash{}{0pt}%
\pgfpathmoveto{\pgfqpoint{4.262595in}{2.193629in}}%
\pgfpathlineto{\pgfqpoint{4.034810in}{1.751253in}}%
\pgfpathlineto{\pgfqpoint{4.179552in}{1.680684in}}%
\pgfpathclose%
\pgfusepath{fill}%
\end{pgfscope}%
\begin{pgfscope}%
\pgfpathrectangle{\pgfqpoint{0.539299in}{0.078740in}}{\pgfqpoint{7.842520in}{7.842520in}}%
\pgfusepath{clip}%
\pgfsetbuttcap%
\pgfsetroundjoin%
\definecolor{currentfill}{rgb}{0.120565,0.596422,0.543611}%
\pgfsetfillcolor{currentfill}%
\pgfsetlinewidth{0.000000pt}%
\definecolor{currentstroke}{rgb}{0.162142,0.474838,0.558140}%
\pgfsetstrokecolor{currentstroke}%
\pgfsetdash{}{0pt}%
\pgfpathmoveto{\pgfqpoint{4.745380in}{3.138731in}}%
\pgfpathlineto{\pgfqpoint{4.828804in}{3.219577in}}%
\pgfpathlineto{\pgfqpoint{4.683466in}{3.421156in}}%
\pgfpathclose%
\pgfusepath{fill}%
\end{pgfscope}%
\begin{pgfscope}%
\pgfpathrectangle{\pgfqpoint{0.539299in}{0.078740in}}{\pgfqpoint{7.842520in}{7.842520in}}%
\pgfusepath{clip}%
\pgfsetbuttcap%
\pgfsetroundjoin%
\definecolor{currentfill}{rgb}{0.496615,0.826376,0.306377}%
\pgfsetfillcolor{currentfill}%
\pgfsetlinewidth{0.000000pt}%
\definecolor{currentstroke}{rgb}{0.160665,0.478540,0.558115}%
\pgfsetstrokecolor{currentstroke}%
\pgfsetdash{}{0pt}%
\pgfpathmoveto{\pgfqpoint{3.350268in}{4.277351in}}%
\pgfpathlineto{\pgfqpoint{3.289659in}{4.681825in}}%
\pgfpathlineto{\pgfqpoint{3.206070in}{4.367077in}}%
\pgfpathclose%
\pgfusepath{fill}%
\end{pgfscope}%
\begin{pgfscope}%
\pgfpathrectangle{\pgfqpoint{0.539299in}{0.078740in}}{\pgfqpoint{7.842520in}{7.842520in}}%
\pgfusepath{clip}%
\pgfsetbuttcap%
\pgfsetroundjoin%
\definecolor{currentfill}{rgb}{0.636902,0.856542,0.216620}%
\pgfsetfillcolor{currentfill}%
\pgfsetlinewidth{0.000000pt}%
\definecolor{currentstroke}{rgb}{0.159194,0.482237,0.558073}%
\pgfsetstrokecolor{currentstroke}%
\pgfsetdash{}{0pt}%
\pgfpathmoveto{\pgfqpoint{3.289659in}{4.681825in}}%
\pgfpathlineto{\pgfqpoint{3.434182in}{4.580061in}}%
\pgfpathlineto{\pgfqpoint{3.519195in}{4.806876in}}%
\pgfpathclose%
\pgfusepath{fill}%
\end{pgfscope}%
\begin{pgfscope}%
\pgfpathrectangle{\pgfqpoint{0.539299in}{0.078740in}}{\pgfqpoint{7.842520in}{7.842520in}}%
\pgfusepath{clip}%
\pgfsetbuttcap%
\pgfsetroundjoin%
\definecolor{currentfill}{rgb}{0.143303,0.669459,0.511215}%
\pgfsetfillcolor{currentfill}%
\pgfsetlinewidth{0.000000pt}%
\definecolor{currentstroke}{rgb}{0.157729,0.485932,0.558013}%
\pgfsetstrokecolor{currentstroke}%
\pgfsetdash{}{0pt}%
\pgfpathmoveto{\pgfqpoint{3.474971in}{3.252817in}}%
\pgfpathlineto{\pgfqpoint{3.556804in}{3.697996in}}%
\pgfpathlineto{\pgfqpoint{3.412099in}{3.800510in}}%
\pgfpathclose%
\pgfusepath{fill}%
\end{pgfscope}%
\begin{pgfscope}%
\pgfpathrectangle{\pgfqpoint{0.539299in}{0.078740in}}{\pgfqpoint{7.842520in}{7.842520in}}%
\pgfusepath{clip}%
\pgfsetbuttcap%
\pgfsetroundjoin%
\definecolor{currentfill}{rgb}{0.143343,0.522773,0.556295}%
\pgfsetfillcolor{currentfill}%
\pgfsetlinewidth{0.000000pt}%
\definecolor{currentstroke}{rgb}{0.156270,0.489624,0.557936}%
\pgfsetstrokecolor{currentstroke}%
\pgfsetdash{}{0pt}%
\pgfpathmoveto{\pgfqpoint{4.973989in}{3.014923in}}%
\pgfpathlineto{\pgfqpoint{4.891094in}{2.947957in}}%
\pgfpathlineto{\pgfqpoint{5.036687in}{2.752478in}}%
\pgfpathclose%
\pgfusepath{fill}%
\end{pgfscope}%
\begin{pgfscope}%
\pgfpathrectangle{\pgfqpoint{0.539299in}{0.078740in}}{\pgfqpoint{7.842520in}{7.842520in}}%
\pgfusepath{clip}%
\pgfsetbuttcap%
\pgfsetroundjoin%
\definecolor{currentfill}{rgb}{0.124780,0.640461,0.527068}%
\pgfsetfillcolor{currentfill}%
\pgfsetlinewidth{0.000000pt}%
\definecolor{currentstroke}{rgb}{0.154815,0.493313,0.557840}%
\pgfsetstrokecolor{currentstroke}%
\pgfsetdash{}{0pt}%
\pgfpathmoveto{\pgfqpoint{4.599574in}{3.325252in}}%
\pgfpathlineto{\pgfqpoint{4.683466in}{3.421156in}}%
\pgfpathlineto{\pgfqpoint{4.537995in}{3.619381in}}%
\pgfpathclose%
\pgfusepath{fill}%
\end{pgfscope}%
\begin{pgfscope}%
\pgfpathrectangle{\pgfqpoint{0.539299in}{0.078740in}}{\pgfqpoint{7.842520in}{7.842520in}}%
\pgfusepath{clip}%
\pgfsetbuttcap%
\pgfsetroundjoin%
\definecolor{currentfill}{rgb}{0.259857,0.745492,0.444467}%
\pgfsetfillcolor{currentfill}%
\pgfsetlinewidth{0.000000pt}%
\definecolor{currentstroke}{rgb}{0.153364,0.497000,0.557724}%
\pgfsetstrokecolor{currentstroke}%
\pgfsetdash{}{0pt}%
\pgfpathmoveto{\pgfqpoint{3.267846in}{3.892059in}}%
\pgfpathlineto{\pgfqpoint{3.412099in}{3.800510in}}%
\pgfpathlineto{\pgfqpoint{3.494996in}{4.172102in}}%
\pgfpathclose%
\pgfusepath{fill}%
\end{pgfscope}%
\begin{pgfscope}%
\pgfpathrectangle{\pgfqpoint{0.539299in}{0.078740in}}{\pgfqpoint{7.842520in}{7.842520in}}%
\pgfusepath{clip}%
\pgfsetbuttcap%
\pgfsetroundjoin%
\definecolor{currentfill}{rgb}{0.147607,0.511733,0.557049}%
\pgfsetfillcolor{currentfill}%
\pgfsetlinewidth{0.000000pt}%
\definecolor{currentstroke}{rgb}{0.151918,0.500685,0.557587}%
\pgfsetstrokecolor{currentstroke}%
\pgfsetdash{}{0pt}%
\pgfpathmoveto{\pgfqpoint{3.909430in}{2.951944in}}%
\pgfpathlineto{\pgfqpoint{3.764278in}{3.057833in}}%
\pgfpathlineto{\pgfqpoint{3.682755in}{2.554388in}}%
\pgfpathclose%
\pgfusepath{fill}%
\end{pgfscope}%
\begin{pgfscope}%
\pgfpathrectangle{\pgfqpoint{0.539299in}{0.078740in}}{\pgfqpoint{7.842520in}{7.842520in}}%
\pgfusepath{clip}%
\pgfsetbuttcap%
\pgfsetroundjoin%
\definecolor{currentfill}{rgb}{0.535621,0.835785,0.281908}%
\pgfsetfillcolor{currentfill}%
\pgfsetlinewidth{0.000000pt}%
\definecolor{currentstroke}{rgb}{0.150476,0.504369,0.557430}%
\pgfsetstrokecolor{currentstroke}%
\pgfsetdash{}{0pt}%
\pgfpathmoveto{\pgfqpoint{3.434182in}{4.580061in}}%
\pgfpathlineto{\pgfqpoint{3.289659in}{4.681825in}}%
\pgfpathlineto{\pgfqpoint{3.350268in}{4.277351in}}%
\pgfpathclose%
\pgfusepath{fill}%
\end{pgfscope}%
\begin{pgfscope}%
\pgfpathrectangle{\pgfqpoint{0.539299in}{0.078740in}}{\pgfqpoint{7.842520in}{7.842520in}}%
\pgfusepath{clip}%
\pgfsetbuttcap%
\pgfsetroundjoin%
\definecolor{currentfill}{rgb}{0.119483,0.614817,0.537692}%
\pgfsetfillcolor{currentfill}%
\pgfsetlinewidth{0.000000pt}%
\definecolor{currentstroke}{rgb}{0.149039,0.508051,0.557250}%
\pgfsetstrokecolor{currentstroke}%
\pgfsetdash{}{0pt}%
\pgfpathmoveto{\pgfqpoint{3.474971in}{3.252817in}}%
\pgfpathlineto{\pgfqpoint{3.619444in}{3.158415in}}%
\pgfpathlineto{\pgfqpoint{3.701876in}{3.586183in}}%
\pgfpathclose%
\pgfusepath{fill}%
\end{pgfscope}%
\begin{pgfscope}%
\pgfpathrectangle{\pgfqpoint{0.539299in}{0.078740in}}{\pgfqpoint{7.842520in}{7.842520in}}%
\pgfusepath{clip}%
\pgfsetbuttcap%
\pgfsetroundjoin%
\definecolor{currentfill}{rgb}{0.327796,0.773980,0.406640}%
\pgfsetfillcolor{currentfill}%
\pgfsetlinewidth{0.000000pt}%
\definecolor{currentstroke}{rgb}{0.147607,0.511733,0.557049}%
\pgfsetstrokecolor{currentstroke}%
\pgfsetdash{}{0pt}%
\pgfpathmoveto{\pgfqpoint{3.494996in}{4.172102in}}%
\pgfpathlineto{\pgfqpoint{3.350268in}{4.277351in}}%
\pgfpathlineto{\pgfqpoint{3.267846in}{3.892059in}}%
\pgfpathclose%
\pgfusepath{fill}%
\end{pgfscope}%
\begin{pgfscope}%
\pgfpathrectangle{\pgfqpoint{0.539299in}{0.078740in}}{\pgfqpoint{7.842520in}{7.842520in}}%
\pgfusepath{clip}%
\pgfsetbuttcap%
\pgfsetroundjoin%
\definecolor{currentfill}{rgb}{0.616293,0.852709,0.230052}%
\pgfsetfillcolor{currentfill}%
\pgfsetlinewidth{0.000000pt}%
\definecolor{currentstroke}{rgb}{0.146180,0.515413,0.556823}%
\pgfsetstrokecolor{currentstroke}%
\pgfsetdash{}{0pt}%
\pgfpathmoveto{\pgfqpoint{3.579199in}{4.460532in}}%
\pgfpathlineto{\pgfqpoint{3.664322in}{4.672860in}}%
\pgfpathlineto{\pgfqpoint{3.519195in}{4.806876in}}%
\pgfpathclose%
\pgfusepath{fill}%
\end{pgfscope}%
\begin{pgfscope}%
\pgfpathrectangle{\pgfqpoint{0.539299in}{0.078740in}}{\pgfqpoint{7.842520in}{7.842520in}}%
\pgfusepath{clip}%
\pgfsetbuttcap%
\pgfsetroundjoin%
\definecolor{currentfill}{rgb}{0.168126,0.459988,0.558082}%
\pgfsetfillcolor{currentfill}%
\pgfsetlinewidth{0.000000pt}%
\definecolor{currentstroke}{rgb}{0.144759,0.519093,0.556572}%
\pgfsetstrokecolor{currentstroke}%
\pgfsetdash{}{0pt}%
\pgfpathmoveto{\pgfqpoint{3.909430in}{2.951944in}}%
\pgfpathlineto{\pgfqpoint{3.827198in}{2.468870in}}%
\pgfpathlineto{\pgfqpoint{3.971998in}{2.380078in}}%
\pgfpathclose%
\pgfusepath{fill}%
\end{pgfscope}%
\begin{pgfscope}%
\pgfpathrectangle{\pgfqpoint{0.539299in}{0.078740in}}{\pgfqpoint{7.842520in}{7.842520in}}%
\pgfusepath{clip}%
\pgfsetbuttcap%
\pgfsetroundjoin%
\definecolor{currentfill}{rgb}{0.221989,0.339161,0.548752}%
\pgfsetfillcolor{currentfill}%
\pgfsetlinewidth{0.000000pt}%
\definecolor{currentstroke}{rgb}{0.143343,0.522773,0.556295}%
\pgfsetstrokecolor{currentstroke}%
\pgfsetdash{}{0pt}%
\pgfpathmoveto{\pgfqpoint{4.117136in}{2.288294in}}%
\pgfpathlineto{\pgfqpoint{4.034810in}{1.751253in}}%
\pgfpathlineto{\pgfqpoint{4.262595in}{2.193629in}}%
\pgfpathclose%
\pgfusepath{fill}%
\end{pgfscope}%
\begin{pgfscope}%
\pgfpathrectangle{\pgfqpoint{0.539299in}{0.078740in}}{\pgfqpoint{7.842520in}{7.842520in}}%
\pgfusepath{clip}%
\pgfsetbuttcap%
\pgfsetroundjoin%
\definecolor{currentfill}{rgb}{0.606045,0.850733,0.236712}%
\pgfsetfillcolor{currentfill}%
\pgfsetlinewidth{0.000000pt}%
\definecolor{currentstroke}{rgb}{0.141935,0.526453,0.555991}%
\pgfsetstrokecolor{currentstroke}%
\pgfsetdash{}{0pt}%
\pgfpathmoveto{\pgfqpoint{3.519195in}{4.806876in}}%
\pgfpathlineto{\pgfqpoint{3.434182in}{4.580061in}}%
\pgfpathlineto{\pgfqpoint{3.579199in}{4.460532in}}%
\pgfpathclose%
\pgfusepath{fill}%
\end{pgfscope}%
\begin{pgfscope}%
\pgfpathrectangle{\pgfqpoint{0.539299in}{0.078740in}}{\pgfqpoint{7.842520in}{7.842520in}}%
\pgfusepath{clip}%
\pgfsetbuttcap%
\pgfsetroundjoin%
\definecolor{currentfill}{rgb}{0.127568,0.566949,0.550556}%
\pgfsetfillcolor{currentfill}%
\pgfsetlinewidth{0.000000pt}%
\definecolor{currentstroke}{rgb}{0.140536,0.530132,0.555659}%
\pgfsetstrokecolor{currentstroke}%
\pgfsetdash{}{0pt}%
\pgfpathmoveto{\pgfqpoint{4.745380in}{3.138731in}}%
\pgfpathlineto{\pgfqpoint{4.891094in}{2.947957in}}%
\pgfpathlineto{\pgfqpoint{4.828804in}{3.219577in}}%
\pgfpathclose%
\pgfusepath{fill}%
\end{pgfscope}%
\begin{pgfscope}%
\pgfpathrectangle{\pgfqpoint{0.539299in}{0.078740in}}{\pgfqpoint{7.842520in}{7.842520in}}%
\pgfusepath{clip}%
\pgfsetbuttcap%
\pgfsetroundjoin%
\definecolor{currentfill}{rgb}{0.175707,0.697900,0.491033}%
\pgfsetfillcolor{currentfill}%
\pgfsetlinewidth{0.000000pt}%
\definecolor{currentstroke}{rgb}{0.139147,0.533812,0.555298}%
\pgfsetstrokecolor{currentstroke}%
\pgfsetdash{}{0pt}%
\pgfpathmoveto{\pgfqpoint{4.537995in}{3.619381in}}%
\pgfpathlineto{\pgfqpoint{4.392412in}{3.813568in}}%
\pgfpathlineto{\pgfqpoint{4.307790in}{3.685037in}}%
\pgfpathclose%
\pgfusepath{fill}%
\end{pgfscope}%
\begin{pgfscope}%
\pgfpathrectangle{\pgfqpoint{0.539299in}{0.078740in}}{\pgfqpoint{7.842520in}{7.842520in}}%
\pgfusepath{clip}%
\pgfsetbuttcap%
\pgfsetroundjoin%
\definecolor{currentfill}{rgb}{0.565498,0.842430,0.262877}%
\pgfsetfillcolor{currentfill}%
\pgfsetlinewidth{0.000000pt}%
\definecolor{currentstroke}{rgb}{0.137770,0.537492,0.554906}%
\pgfsetstrokecolor{currentstroke}%
\pgfsetdash{}{0pt}%
\pgfpathmoveto{\pgfqpoint{3.579199in}{4.460532in}}%
\pgfpathlineto{\pgfqpoint{3.809748in}{4.522655in}}%
\pgfpathlineto{\pgfqpoint{3.664322in}{4.672860in}}%
\pgfpathclose%
\pgfusepath{fill}%
\end{pgfscope}%
\begin{pgfscope}%
\pgfpathrectangle{\pgfqpoint{0.539299in}{0.078740in}}{\pgfqpoint{7.842520in}{7.842520in}}%
\pgfusepath{clip}%
\pgfsetbuttcap%
\pgfsetroundjoin%
\definecolor{currentfill}{rgb}{0.132268,0.655014,0.519661}%
\pgfsetfillcolor{currentfill}%
\pgfsetlinewidth{0.000000pt}%
\definecolor{currentstroke}{rgb}{0.136408,0.541173,0.554483}%
\pgfsetstrokecolor{currentstroke}%
\pgfsetdash{}{0pt}%
\pgfpathmoveto{\pgfqpoint{3.701876in}{3.586183in}}%
\pgfpathlineto{\pgfqpoint{3.556804in}{3.697996in}}%
\pgfpathlineto{\pgfqpoint{3.474971in}{3.252817in}}%
\pgfpathclose%
\pgfusepath{fill}%
\end{pgfscope}%
\begin{pgfscope}%
\pgfpathrectangle{\pgfqpoint{0.539299in}{0.078740in}}{\pgfqpoint{7.842520in}{7.842520in}}%
\pgfusepath{clip}%
\pgfsetbuttcap%
\pgfsetroundjoin%
\definecolor{currentfill}{rgb}{0.203063,0.379716,0.553925}%
\pgfsetfillcolor{currentfill}%
\pgfsetlinewidth{0.000000pt}%
\definecolor{currentstroke}{rgb}{0.135066,0.544853,0.554029}%
\pgfsetstrokecolor{currentstroke}%
\pgfsetdash{}{0pt}%
\pgfpathmoveto{\pgfqpoint{5.472274in}{2.124924in}}%
\pgfpathlineto{\pgfqpoint{5.327340in}{2.342890in}}%
\pgfpathlineto{\pgfqpoint{5.245331in}{2.266200in}}%
\pgfpathclose%
\pgfusepath{fill}%
\end{pgfscope}%
\begin{pgfscope}%
\pgfpathrectangle{\pgfqpoint{0.539299in}{0.078740in}}{\pgfqpoint{7.842520in}{7.842520in}}%
\pgfusepath{clip}%
\pgfsetbuttcap%
\pgfsetroundjoin%
\definecolor{currentfill}{rgb}{0.248629,0.278775,0.534556}%
\pgfsetfillcolor{currentfill}%
\pgfsetlinewidth{0.000000pt}%
\definecolor{currentstroke}{rgb}{0.133743,0.548535,0.553541}%
\pgfsetstrokecolor{currentstroke}%
\pgfsetdash{}{0pt}%
\pgfpathmoveto{\pgfqpoint{4.179552in}{1.680684in}}%
\pgfpathlineto{\pgfqpoint{4.324695in}{1.608470in}}%
\pgfpathlineto{\pgfqpoint{4.262595in}{2.193629in}}%
\pgfpathclose%
\pgfusepath{fill}%
\end{pgfscope}%
\begin{pgfscope}%
\pgfpathrectangle{\pgfqpoint{0.539299in}{0.078740in}}{\pgfqpoint{7.842520in}{7.842520in}}%
\pgfusepath{clip}%
\pgfsetbuttcap%
\pgfsetroundjoin%
\definecolor{currentfill}{rgb}{0.449368,0.813768,0.335384}%
\pgfsetfillcolor{currentfill}%
\pgfsetlinewidth{0.000000pt}%
\definecolor{currentstroke}{rgb}{0.132444,0.552216,0.553018}%
\pgfsetstrokecolor{currentstroke}%
\pgfsetdash{}{0pt}%
\pgfpathmoveto{\pgfqpoint{3.434182in}{4.580061in}}%
\pgfpathlineto{\pgfqpoint{3.350268in}{4.277351in}}%
\pgfpathlineto{\pgfqpoint{3.494996in}{4.172102in}}%
\pgfpathclose%
\pgfusepath{fill}%
\end{pgfscope}%
\begin{pgfscope}%
\pgfpathrectangle{\pgfqpoint{0.539299in}{0.078740in}}{\pgfqpoint{7.842520in}{7.842520in}}%
\pgfusepath{clip}%
\pgfsetbuttcap%
\pgfsetroundjoin%
\definecolor{currentfill}{rgb}{0.239374,0.735588,0.455688}%
\pgfsetfillcolor{currentfill}%
\pgfsetlinewidth{0.000000pt}%
\definecolor{currentstroke}{rgb}{0.131172,0.555899,0.552459}%
\pgfsetstrokecolor{currentstroke}%
\pgfsetdash{}{0pt}%
\pgfpathmoveto{\pgfqpoint{4.161873in}{3.857033in}}%
\pgfpathlineto{\pgfqpoint{4.392412in}{3.813568in}}%
\pgfpathlineto{\pgfqpoint{4.246749in}{4.002642in}}%
\pgfpathclose%
\pgfusepath{fill}%
\end{pgfscope}%
\begin{pgfscope}%
\pgfpathrectangle{\pgfqpoint{0.539299in}{0.078740in}}{\pgfqpoint{7.842520in}{7.842520in}}%
\pgfusepath{clip}%
\pgfsetbuttcap%
\pgfsetroundjoin%
\definecolor{currentfill}{rgb}{0.119423,0.611141,0.538982}%
\pgfsetfillcolor{currentfill}%
\pgfsetlinewidth{0.000000pt}%
\definecolor{currentstroke}{rgb}{0.129933,0.559582,0.551864}%
\pgfsetstrokecolor{currentstroke}%
\pgfsetdash{}{0pt}%
\pgfpathmoveto{\pgfqpoint{4.683466in}{3.421156in}}%
\pgfpathlineto{\pgfqpoint{4.599574in}{3.325252in}}%
\pgfpathlineto{\pgfqpoint{4.745380in}{3.138731in}}%
\pgfpathclose%
\pgfusepath{fill}%
\end{pgfscope}%
\begin{pgfscope}%
\pgfpathrectangle{\pgfqpoint{0.539299in}{0.078740in}}{\pgfqpoint{7.842520in}{7.842520in}}%
\pgfusepath{clip}%
\pgfsetbuttcap%
\pgfsetroundjoin%
\definecolor{currentfill}{rgb}{0.239374,0.735588,0.455688}%
\pgfsetfillcolor{currentfill}%
\pgfsetlinewidth{0.000000pt}%
\definecolor{currentstroke}{rgb}{0.128729,0.563265,0.551229}%
\pgfsetstrokecolor{currentstroke}%
\pgfsetdash{}{0pt}%
\pgfpathmoveto{\pgfqpoint{3.494996in}{4.172102in}}%
\pgfpathlineto{\pgfqpoint{3.412099in}{3.800510in}}%
\pgfpathlineto{\pgfqpoint{3.556804in}{3.697996in}}%
\pgfpathclose%
\pgfusepath{fill}%
\end{pgfscope}%
\begin{pgfscope}%
\pgfpathrectangle{\pgfqpoint{0.539299in}{0.078740in}}{\pgfqpoint{7.842520in}{7.842520in}}%
\pgfusepath{clip}%
\pgfsetbuttcap%
\pgfsetroundjoin%
\definecolor{currentfill}{rgb}{0.311925,0.767822,0.415586}%
\pgfsetfillcolor{currentfill}%
\pgfsetlinewidth{0.000000pt}%
\definecolor{currentstroke}{rgb}{0.127568,0.566949,0.550556}%
\pgfsetstrokecolor{currentstroke}%
\pgfsetdash{}{0pt}%
\pgfpathmoveto{\pgfqpoint{4.246749in}{4.002642in}}%
\pgfpathlineto{\pgfqpoint{4.101044in}{4.185140in}}%
\pgfpathlineto{\pgfqpoint{4.015993in}{4.022293in}}%
\pgfpathclose%
\pgfusepath{fill}%
\end{pgfscope}%
\begin{pgfscope}%
\pgfpathrectangle{\pgfqpoint{0.539299in}{0.078740in}}{\pgfqpoint{7.842520in}{7.842520in}}%
\pgfusepath{clip}%
\pgfsetbuttcap%
\pgfsetroundjoin%
\definecolor{currentfill}{rgb}{0.119738,0.603785,0.541400}%
\pgfsetfillcolor{currentfill}%
\pgfsetlinewidth{0.000000pt}%
\definecolor{currentstroke}{rgb}{0.126453,0.570633,0.549841}%
\pgfsetstrokecolor{currentstroke}%
\pgfsetdash{}{0pt}%
\pgfpathmoveto{\pgfqpoint{3.619444in}{3.158415in}}%
\pgfpathlineto{\pgfqpoint{3.764278in}{3.057833in}}%
\pgfpathlineto{\pgfqpoint{3.701876in}{3.586183in}}%
\pgfpathclose%
\pgfusepath{fill}%
\end{pgfscope}%
\begin{pgfscope}%
\pgfpathrectangle{\pgfqpoint{0.539299in}{0.078740in}}{\pgfqpoint{7.842520in}{7.842520in}}%
\pgfusepath{clip}%
\pgfsetbuttcap%
\pgfsetroundjoin%
\definecolor{currentfill}{rgb}{0.458674,0.816363,0.329727}%
\pgfsetfillcolor{currentfill}%
\pgfsetlinewidth{0.000000pt}%
\definecolor{currentstroke}{rgb}{0.125394,0.574318,0.549086}%
\pgfsetstrokecolor{currentstroke}%
\pgfsetdash{}{0pt}%
\pgfpathmoveto{\pgfqpoint{3.809748in}{4.522655in}}%
\pgfpathlineto{\pgfqpoint{3.870205in}{4.179274in}}%
\pgfpathlineto{\pgfqpoint{3.955353in}{4.359218in}}%
\pgfpathclose%
\pgfusepath{fill}%
\end{pgfscope}%
\begin{pgfscope}%
\pgfpathrectangle{\pgfqpoint{0.539299in}{0.078740in}}{\pgfqpoint{7.842520in}{7.842520in}}%
\pgfusepath{clip}%
\pgfsetbuttcap%
\pgfsetroundjoin%
\definecolor{currentfill}{rgb}{0.278012,0.180367,0.486697}%
\pgfsetfillcolor{currentfill}%
\pgfsetlinewidth{0.000000pt}%
\definecolor{currentstroke}{rgb}{0.124395,0.578002,0.548287}%
\pgfsetstrokecolor{currentstroke}%
\pgfsetdash{}{0pt}%
\pgfpathmoveto{\pgfqpoint{5.680690in}{1.591684in}}%
\pgfpathlineto{\pgfqpoint{5.599643in}{1.509341in}}%
\pgfpathlineto{\pgfqpoint{5.824454in}{1.317143in}}%
\pgfpathclose%
\pgfusepath{fill}%
\end{pgfscope}%
\begin{pgfscope}%
\pgfpathrectangle{\pgfqpoint{0.539299in}{0.078740in}}{\pgfqpoint{7.842520in}{7.842520in}}%
\pgfusepath{clip}%
\pgfsetbuttcap%
\pgfsetroundjoin%
\definecolor{currentfill}{rgb}{0.377779,0.791781,0.377939}%
\pgfsetfillcolor{currentfill}%
\pgfsetlinewidth{0.000000pt}%
\definecolor{currentstroke}{rgb}{0.123463,0.581687,0.547445}%
\pgfsetstrokecolor{currentstroke}%
\pgfsetdash{}{0pt}%
\pgfpathmoveto{\pgfqpoint{4.015993in}{4.022293in}}%
\pgfpathlineto{\pgfqpoint{4.101044in}{4.185140in}}%
\pgfpathlineto{\pgfqpoint{3.955353in}{4.359218in}}%
\pgfpathclose%
\pgfusepath{fill}%
\end{pgfscope}%
\begin{pgfscope}%
\pgfpathrectangle{\pgfqpoint{0.539299in}{0.078740in}}{\pgfqpoint{7.842520in}{7.842520in}}%
\pgfusepath{clip}%
\pgfsetbuttcap%
\pgfsetroundjoin%
\definecolor{currentfill}{rgb}{0.487026,0.823929,0.312321}%
\pgfsetfillcolor{currentfill}%
\pgfsetlinewidth{0.000000pt}%
\definecolor{currentstroke}{rgb}{0.122606,0.585371,0.546557}%
\pgfsetstrokecolor{currentstroke}%
\pgfsetdash{}{0pt}%
\pgfpathmoveto{\pgfqpoint{3.494996in}{4.172102in}}%
\pgfpathlineto{\pgfqpoint{3.579199in}{4.460532in}}%
\pgfpathlineto{\pgfqpoint{3.434182in}{4.580061in}}%
\pgfpathclose%
\pgfusepath{fill}%
\end{pgfscope}%
\begin{pgfscope}%
\pgfpathrectangle{\pgfqpoint{0.539299in}{0.078740in}}{\pgfqpoint{7.842520in}{7.842520in}}%
\pgfusepath{clip}%
\pgfsetbuttcap%
\pgfsetroundjoin%
\definecolor{currentfill}{rgb}{0.227802,0.326594,0.546532}%
\pgfsetfillcolor{currentfill}%
\pgfsetlinewidth{0.000000pt}%
\definecolor{currentstroke}{rgb}{0.121831,0.589055,0.545623}%
\pgfsetstrokecolor{currentstroke}%
\pgfsetdash{}{0pt}%
\pgfpathmoveto{\pgfqpoint{5.390896in}{2.059660in}}%
\pgfpathlineto{\pgfqpoint{5.536081in}{1.836733in}}%
\pgfpathlineto{\pgfqpoint{5.472274in}{2.124924in}}%
\pgfpathclose%
\pgfusepath{fill}%
\end{pgfscope}%
\begin{pgfscope}%
\pgfpathrectangle{\pgfqpoint{0.539299in}{0.078740in}}{\pgfqpoint{7.842520in}{7.842520in}}%
\pgfusepath{clip}%
\pgfsetbuttcap%
\pgfsetroundjoin%
\definecolor{currentfill}{rgb}{0.506271,0.828786,0.300362}%
\pgfsetfillcolor{currentfill}%
\pgfsetlinewidth{0.000000pt}%
\definecolor{currentstroke}{rgb}{0.121148,0.592739,0.544641}%
\pgfsetstrokecolor{currentstroke}%
\pgfsetdash{}{0pt}%
\pgfpathmoveto{\pgfqpoint{3.724577in}{4.326091in}}%
\pgfpathlineto{\pgfqpoint{3.809748in}{4.522655in}}%
\pgfpathlineto{\pgfqpoint{3.579199in}{4.460532in}}%
\pgfpathclose%
\pgfusepath{fill}%
\end{pgfscope}%
\begin{pgfscope}%
\pgfpathrectangle{\pgfqpoint{0.539299in}{0.078740in}}{\pgfqpoint{7.842520in}{7.842520in}}%
\pgfusepath{clip}%
\pgfsetbuttcap%
\pgfsetroundjoin%
\definecolor{currentfill}{rgb}{0.130067,0.651384,0.521608}%
\pgfsetfillcolor{currentfill}%
\pgfsetlinewidth{0.000000pt}%
\definecolor{currentstroke}{rgb}{0.120565,0.596422,0.543611}%
\pgfsetstrokecolor{currentstroke}%
\pgfsetdash{}{0pt}%
\pgfpathmoveto{\pgfqpoint{4.537995in}{3.619381in}}%
\pgfpathlineto{\pgfqpoint{4.453701in}{3.507500in}}%
\pgfpathlineto{\pgfqpoint{4.599574in}{3.325252in}}%
\pgfpathclose%
\pgfusepath{fill}%
\end{pgfscope}%
\begin{pgfscope}%
\pgfpathrectangle{\pgfqpoint{0.539299in}{0.078740in}}{\pgfqpoint{7.842520in}{7.842520in}}%
\pgfusepath{clip}%
\pgfsetbuttcap%
\pgfsetroundjoin%
\definecolor{currentfill}{rgb}{0.180629,0.429975,0.557282}%
\pgfsetfillcolor{currentfill}%
\pgfsetlinewidth{0.000000pt}%
\definecolor{currentstroke}{rgb}{0.120092,0.600104,0.542530}%
\pgfsetstrokecolor{currentstroke}%
\pgfsetdash{}{0pt}%
\pgfpathmoveto{\pgfqpoint{5.327340in}{2.342890in}}%
\pgfpathlineto{\pgfqpoint{5.182120in}{2.551317in}}%
\pgfpathlineto{\pgfqpoint{5.099519in}{2.460623in}}%
\pgfpathclose%
\pgfusepath{fill}%
\end{pgfscope}%
\begin{pgfscope}%
\pgfpathrectangle{\pgfqpoint{0.539299in}{0.078740in}}{\pgfqpoint{7.842520in}{7.842520in}}%
\pgfusepath{clip}%
\pgfsetbuttcap%
\pgfsetroundjoin%
\definecolor{currentfill}{rgb}{0.153894,0.680203,0.504172}%
\pgfsetfillcolor{currentfill}%
\pgfsetlinewidth{0.000000pt}%
\definecolor{currentstroke}{rgb}{0.119738,0.603785,0.541400}%
\pgfsetstrokecolor{currentstroke}%
\pgfsetdash{}{0pt}%
\pgfpathmoveto{\pgfqpoint{4.307790in}{3.685037in}}%
\pgfpathlineto{\pgfqpoint{4.453701in}{3.507500in}}%
\pgfpathlineto{\pgfqpoint{4.537995in}{3.619381in}}%
\pgfpathclose%
\pgfusepath{fill}%
\end{pgfscope}%
\begin{pgfscope}%
\pgfpathrectangle{\pgfqpoint{0.539299in}{0.078740in}}{\pgfqpoint{7.842520in}{7.842520in}}%
\pgfusepath{clip}%
\pgfsetbuttcap%
\pgfsetroundjoin%
\definecolor{currentfill}{rgb}{0.458674,0.816363,0.329727}%
\pgfsetfillcolor{currentfill}%
\pgfsetlinewidth{0.000000pt}%
\definecolor{currentstroke}{rgb}{0.119512,0.607464,0.540218}%
\pgfsetstrokecolor{currentstroke}%
\pgfsetdash{}{0pt}%
\pgfpathmoveto{\pgfqpoint{3.724577in}{4.326091in}}%
\pgfpathlineto{\pgfqpoint{3.870205in}{4.179274in}}%
\pgfpathlineto{\pgfqpoint{3.809748in}{4.522655in}}%
\pgfpathclose%
\pgfusepath{fill}%
\end{pgfscope}%
\begin{pgfscope}%
\pgfpathrectangle{\pgfqpoint{0.539299in}{0.078740in}}{\pgfqpoint{7.842520in}{7.842520in}}%
\pgfusepath{clip}%
\pgfsetbuttcap%
\pgfsetroundjoin%
\definecolor{currentfill}{rgb}{0.449368,0.813768,0.335384}%
\pgfsetfillcolor{currentfill}%
\pgfsetlinewidth{0.000000pt}%
\definecolor{currentstroke}{rgb}{0.119423,0.611141,0.538982}%
\pgfsetstrokecolor{currentstroke}%
\pgfsetdash{}{0pt}%
\pgfpathmoveto{\pgfqpoint{3.724577in}{4.326091in}}%
\pgfpathlineto{\pgfqpoint{3.579199in}{4.460532in}}%
\pgfpathlineto{\pgfqpoint{3.494996in}{4.172102in}}%
\pgfpathclose%
\pgfusepath{fill}%
\end{pgfscope}%
\begin{pgfscope}%
\pgfpathrectangle{\pgfqpoint{0.539299in}{0.078740in}}{\pgfqpoint{7.842520in}{7.842520in}}%
\pgfusepath{clip}%
\pgfsetbuttcap%
\pgfsetroundjoin%
\definecolor{currentfill}{rgb}{0.208030,0.718701,0.472873}%
\pgfsetfillcolor{currentfill}%
\pgfsetlinewidth{0.000000pt}%
\definecolor{currentstroke}{rgb}{0.119483,0.614817,0.537692}%
\pgfsetstrokecolor{currentstroke}%
\pgfsetdash{}{0pt}%
\pgfpathmoveto{\pgfqpoint{4.307790in}{3.685037in}}%
\pgfpathlineto{\pgfqpoint{4.392412in}{3.813568in}}%
\pgfpathlineto{\pgfqpoint{4.161873in}{3.857033in}}%
\pgfpathclose%
\pgfusepath{fill}%
\end{pgfscope}%
\begin{pgfscope}%
\pgfpathrectangle{\pgfqpoint{0.539299in}{0.078740in}}{\pgfqpoint{7.842520in}{7.842520in}}%
\pgfusepath{clip}%
\pgfsetbuttcap%
\pgfsetroundjoin%
\definecolor{currentfill}{rgb}{0.281477,0.755203,0.432552}%
\pgfsetfillcolor{currentfill}%
\pgfsetlinewidth{0.000000pt}%
\definecolor{currentstroke}{rgb}{0.119699,0.618490,0.536347}%
\pgfsetstrokecolor{currentstroke}%
\pgfsetdash{}{0pt}%
\pgfpathmoveto{\pgfqpoint{3.556804in}{3.697996in}}%
\pgfpathlineto{\pgfqpoint{3.640137in}{4.053707in}}%
\pgfpathlineto{\pgfqpoint{3.494996in}{4.172102in}}%
\pgfpathclose%
\pgfusepath{fill}%
\end{pgfscope}%
\begin{pgfscope}%
\pgfpathrectangle{\pgfqpoint{0.539299in}{0.078740in}}{\pgfqpoint{7.842520in}{7.842520in}}%
\pgfusepath{clip}%
\pgfsetbuttcap%
\pgfsetroundjoin%
\definecolor{currentfill}{rgb}{0.210503,0.363727,0.552206}%
\pgfsetfillcolor{currentfill}%
\pgfsetlinewidth{0.000000pt}%
\definecolor{currentstroke}{rgb}{0.120081,0.622161,0.534946}%
\pgfsetstrokecolor{currentstroke}%
\pgfsetdash{}{0pt}%
\pgfpathmoveto{\pgfqpoint{5.245331in}{2.266200in}}%
\pgfpathlineto{\pgfqpoint{5.390896in}{2.059660in}}%
\pgfpathlineto{\pgfqpoint{5.472274in}{2.124924in}}%
\pgfpathclose%
\pgfusepath{fill}%
\end{pgfscope}%
\begin{pgfscope}%
\pgfpathrectangle{\pgfqpoint{0.539299in}{0.078740in}}{\pgfqpoint{7.842520in}{7.842520in}}%
\pgfusepath{clip}%
\pgfsetbuttcap%
\pgfsetroundjoin%
\definecolor{currentfill}{rgb}{0.281887,0.150881,0.465405}%
\pgfsetfillcolor{currentfill}%
\pgfsetlinewidth{0.000000pt}%
\definecolor{currentstroke}{rgb}{0.120638,0.625828,0.533488}%
\pgfsetstrokecolor{currentstroke}%
\pgfsetdash{}{0pt}%
\pgfpathmoveto{\pgfqpoint{5.824454in}{1.317143in}}%
\pgfpathlineto{\pgfqpoint{5.599643in}{1.509341in}}%
\pgfpathlineto{\pgfqpoint{5.744062in}{1.240624in}}%
\pgfpathclose%
\pgfusepath{fill}%
\end{pgfscope}%
\begin{pgfscope}%
\pgfpathrectangle{\pgfqpoint{0.539299in}{0.078740in}}{\pgfqpoint{7.842520in}{7.842520in}}%
\pgfusepath{clip}%
\pgfsetbuttcap%
\pgfsetroundjoin%
\definecolor{currentfill}{rgb}{0.274149,0.751988,0.436601}%
\pgfsetfillcolor{currentfill}%
\pgfsetlinewidth{0.000000pt}%
\definecolor{currentstroke}{rgb}{0.121380,0.629492,0.531973}%
\pgfsetstrokecolor{currentstroke}%
\pgfsetdash{}{0pt}%
\pgfpathmoveto{\pgfqpoint{4.015993in}{4.022293in}}%
\pgfpathlineto{\pgfqpoint{4.161873in}{3.857033in}}%
\pgfpathlineto{\pgfqpoint{4.246749in}{4.002642in}}%
\pgfpathclose%
\pgfusepath{fill}%
\end{pgfscope}%
\begin{pgfscope}%
\pgfpathrectangle{\pgfqpoint{0.539299in}{0.078740in}}{\pgfqpoint{7.842520in}{7.842520in}}%
\pgfusepath{clip}%
\pgfsetbuttcap%
\pgfsetroundjoin%
\definecolor{currentfill}{rgb}{0.154815,0.493313,0.557840}%
\pgfsetfillcolor{currentfill}%
\pgfsetlinewidth{0.000000pt}%
\definecolor{currentstroke}{rgb}{0.122312,0.633153,0.530398}%
\pgfsetstrokecolor{currentstroke}%
\pgfsetdash{}{0pt}%
\pgfpathmoveto{\pgfqpoint{3.971998in}{2.380078in}}%
\pgfpathlineto{\pgfqpoint{4.054864in}{2.841426in}}%
\pgfpathlineto{\pgfqpoint{3.909430in}{2.951944in}}%
\pgfpathclose%
\pgfusepath{fill}%
\end{pgfscope}%
\begin{pgfscope}%
\pgfpathrectangle{\pgfqpoint{0.539299in}{0.078740in}}{\pgfqpoint{7.842520in}{7.842520in}}%
\pgfusepath{clip}%
\pgfsetbuttcap%
\pgfsetroundjoin%
\definecolor{currentfill}{rgb}{0.377779,0.791781,0.377939}%
\pgfsetfillcolor{currentfill}%
\pgfsetlinewidth{0.000000pt}%
\definecolor{currentstroke}{rgb}{0.123444,0.636809,0.528763}%
\pgfsetstrokecolor{currentstroke}%
\pgfsetdash{}{0pt}%
\pgfpathmoveto{\pgfqpoint{3.955353in}{4.359218in}}%
\pgfpathlineto{\pgfqpoint{3.870205in}{4.179274in}}%
\pgfpathlineto{\pgfqpoint{4.015993in}{4.022293in}}%
\pgfpathclose%
\pgfusepath{fill}%
\end{pgfscope}%
\begin{pgfscope}%
\pgfpathrectangle{\pgfqpoint{0.539299in}{0.078740in}}{\pgfqpoint{7.842520in}{7.842520in}}%
\pgfusepath{clip}%
\pgfsetbuttcap%
\pgfsetroundjoin%
\definecolor{currentfill}{rgb}{0.162142,0.474838,0.558140}%
\pgfsetfillcolor{currentfill}%
\pgfsetlinewidth{0.000000pt}%
\definecolor{currentstroke}{rgb}{0.124780,0.640461,0.527068}%
\pgfsetstrokecolor{currentstroke}%
\pgfsetdash{}{0pt}%
\pgfpathmoveto{\pgfqpoint{5.182120in}{2.551317in}}%
\pgfpathlineto{\pgfqpoint{5.036687in}{2.752478in}}%
\pgfpathlineto{\pgfqpoint{4.953553in}{2.645992in}}%
\pgfpathclose%
\pgfusepath{fill}%
\end{pgfscope}%
\begin{pgfscope}%
\pgfpathrectangle{\pgfqpoint{0.539299in}{0.078740in}}{\pgfqpoint{7.842520in}{7.842520in}}%
\pgfusepath{clip}%
\pgfsetbuttcap%
\pgfsetroundjoin%
\definecolor{currentfill}{rgb}{0.177423,0.437527,0.557565}%
\pgfsetfillcolor{currentfill}%
\pgfsetlinewidth{0.000000pt}%
\definecolor{currentstroke}{rgb}{0.126326,0.644107,0.525311}%
\pgfsetstrokecolor{currentstroke}%
\pgfsetdash{}{0pt}%
\pgfpathmoveto{\pgfqpoint{3.971998in}{2.380078in}}%
\pgfpathlineto{\pgfqpoint{4.117136in}{2.288294in}}%
\pgfpathlineto{\pgfqpoint{4.200551in}{2.726746in}}%
\pgfpathclose%
\pgfusepath{fill}%
\end{pgfscope}%
\begin{pgfscope}%
\pgfpathrectangle{\pgfqpoint{0.539299in}{0.078740in}}{\pgfqpoint{7.842520in}{7.842520in}}%
\pgfusepath{clip}%
\pgfsetbuttcap%
\pgfsetroundjoin%
\definecolor{currentfill}{rgb}{0.202219,0.715272,0.476084}%
\pgfsetfillcolor{currentfill}%
\pgfsetlinewidth{0.000000pt}%
\definecolor{currentstroke}{rgb}{0.128087,0.647749,0.523491}%
\pgfsetstrokecolor{currentstroke}%
\pgfsetdash{}{0pt}%
\pgfpathmoveto{\pgfqpoint{3.701876in}{3.586183in}}%
\pgfpathlineto{\pgfqpoint{3.640137in}{4.053707in}}%
\pgfpathlineto{\pgfqpoint{3.556804in}{3.697996in}}%
\pgfpathclose%
\pgfusepath{fill}%
\end{pgfscope}%
\begin{pgfscope}%
\pgfpathrectangle{\pgfqpoint{0.539299in}{0.078740in}}{\pgfqpoint{7.842520in}{7.842520in}}%
\pgfusepath{clip}%
\pgfsetbuttcap%
\pgfsetroundjoin%
\definecolor{currentfill}{rgb}{0.377779,0.791781,0.377939}%
\pgfsetfillcolor{currentfill}%
\pgfsetlinewidth{0.000000pt}%
\definecolor{currentstroke}{rgb}{0.130067,0.651384,0.521608}%
\pgfsetstrokecolor{currentstroke}%
\pgfsetdash{}{0pt}%
\pgfpathmoveto{\pgfqpoint{3.494996in}{4.172102in}}%
\pgfpathlineto{\pgfqpoint{3.640137in}{4.053707in}}%
\pgfpathlineto{\pgfqpoint{3.724577in}{4.326091in}}%
\pgfpathclose%
\pgfusepath{fill}%
\end{pgfscope}%
\begin{pgfscope}%
\pgfpathrectangle{\pgfqpoint{0.539299in}{0.078740in}}{\pgfqpoint{7.842520in}{7.842520in}}%
\pgfusepath{clip}%
\pgfsetbuttcap%
\pgfsetroundjoin%
\definecolor{currentfill}{rgb}{0.122606,0.585371,0.546557}%
\pgfsetfillcolor{currentfill}%
\pgfsetlinewidth{0.000000pt}%
\definecolor{currentstroke}{rgb}{0.132268,0.655014,0.519661}%
\pgfsetstrokecolor{currentstroke}%
\pgfsetdash{}{0pt}%
\pgfpathmoveto{\pgfqpoint{3.847250in}{3.466509in}}%
\pgfpathlineto{\pgfqpoint{3.764278in}{3.057833in}}%
\pgfpathlineto{\pgfqpoint{3.909430in}{2.951944in}}%
\pgfpathclose%
\pgfusepath{fill}%
\end{pgfscope}%
\begin{pgfscope}%
\pgfpathrectangle{\pgfqpoint{0.539299in}{0.078740in}}{\pgfqpoint{7.842520in}{7.842520in}}%
\pgfusepath{clip}%
\pgfsetbuttcap%
\pgfsetroundjoin%
\definecolor{currentfill}{rgb}{0.122312,0.633153,0.530398}%
\pgfsetfillcolor{currentfill}%
\pgfsetlinewidth{0.000000pt}%
\definecolor{currentstroke}{rgb}{0.134692,0.658636,0.517649}%
\pgfsetstrokecolor{currentstroke}%
\pgfsetdash{}{0pt}%
\pgfpathmoveto{\pgfqpoint{3.701876in}{3.586183in}}%
\pgfpathlineto{\pgfqpoint{3.764278in}{3.057833in}}%
\pgfpathlineto{\pgfqpoint{3.847250in}{3.466509in}}%
\pgfpathclose%
\pgfusepath{fill}%
\end{pgfscope}%
\begin{pgfscope}%
\pgfpathrectangle{\pgfqpoint{0.539299in}{0.078740in}}{\pgfqpoint{7.842520in}{7.842520in}}%
\pgfusepath{clip}%
\pgfsetbuttcap%
\pgfsetroundjoin%
\definecolor{currentfill}{rgb}{0.263663,0.237631,0.518762}%
\pgfsetfillcolor{currentfill}%
\pgfsetlinewidth{0.000000pt}%
\definecolor{currentstroke}{rgb}{0.137339,0.662252,0.515571}%
\pgfsetstrokecolor{currentstroke}%
\pgfsetdash{}{0pt}%
\pgfpathmoveto{\pgfqpoint{5.536081in}{1.836733in}}%
\pgfpathlineto{\pgfqpoint{5.599643in}{1.509341in}}%
\pgfpathlineto{\pgfqpoint{5.680690in}{1.591684in}}%
\pgfpathclose%
\pgfusepath{fill}%
\end{pgfscope}%
\begin{pgfscope}%
\pgfpathrectangle{\pgfqpoint{0.539299in}{0.078740in}}{\pgfqpoint{7.842520in}{7.842520in}}%
\pgfusepath{clip}%
\pgfsetbuttcap%
\pgfsetroundjoin%
\definecolor{currentfill}{rgb}{0.150476,0.504369,0.557430}%
\pgfsetfillcolor{currentfill}%
\pgfsetlinewidth{0.000000pt}%
\definecolor{currentstroke}{rgb}{0.140210,0.665859,0.513427}%
\pgfsetstrokecolor{currentstroke}%
\pgfsetdash{}{0pt}%
\pgfpathmoveto{\pgfqpoint{4.953553in}{2.645992in}}%
\pgfpathlineto{\pgfqpoint{5.036687in}{2.752478in}}%
\pgfpathlineto{\pgfqpoint{4.891094in}{2.947957in}}%
\pgfpathclose%
\pgfusepath{fill}%
\end{pgfscope}%
\begin{pgfscope}%
\pgfpathrectangle{\pgfqpoint{0.539299in}{0.078740in}}{\pgfqpoint{7.842520in}{7.842520in}}%
\pgfusepath{clip}%
\pgfsetbuttcap%
\pgfsetroundjoin%
\definecolor{currentfill}{rgb}{0.386433,0.794644,0.372886}%
\pgfsetfillcolor{currentfill}%
\pgfsetlinewidth{0.000000pt}%
\definecolor{currentstroke}{rgb}{0.143303,0.669459,0.511215}%
\pgfsetstrokecolor{currentstroke}%
\pgfsetdash{}{0pt}%
\pgfpathmoveto{\pgfqpoint{3.640137in}{4.053707in}}%
\pgfpathlineto{\pgfqpoint{3.870205in}{4.179274in}}%
\pgfpathlineto{\pgfqpoint{3.724577in}{4.326091in}}%
\pgfpathclose%
\pgfusepath{fill}%
\end{pgfscope}%
\begin{pgfscope}%
\pgfpathrectangle{\pgfqpoint{0.539299in}{0.078740in}}{\pgfqpoint{7.842520in}{7.842520in}}%
\pgfusepath{clip}%
\pgfsetbuttcap%
\pgfsetroundjoin%
\definecolor{currentfill}{rgb}{0.253935,0.265254,0.529983}%
\pgfsetfillcolor{currentfill}%
\pgfsetlinewidth{0.000000pt}%
\definecolor{currentstroke}{rgb}{0.146616,0.673050,0.508936}%
\pgfsetstrokecolor{currentstroke}%
\pgfsetdash{}{0pt}%
\pgfpathmoveto{\pgfqpoint{4.408361in}{2.095987in}}%
\pgfpathlineto{\pgfqpoint{4.324695in}{1.608470in}}%
\pgfpathlineto{\pgfqpoint{4.470234in}{1.534400in}}%
\pgfpathclose%
\pgfusepath{fill}%
\end{pgfscope}%
\begin{pgfscope}%
\pgfpathrectangle{\pgfqpoint{0.539299in}{0.078740in}}{\pgfqpoint{7.842520in}{7.842520in}}%
\pgfusepath{clip}%
\pgfsetbuttcap%
\pgfsetroundjoin%
\definecolor{currentfill}{rgb}{0.187231,0.414746,0.556547}%
\pgfsetfillcolor{currentfill}%
\pgfsetlinewidth{0.000000pt}%
\definecolor{currentstroke}{rgb}{0.150148,0.676631,0.506589}%
\pgfsetstrokecolor{currentstroke}%
\pgfsetdash{}{0pt}%
\pgfpathmoveto{\pgfqpoint{5.327340in}{2.342890in}}%
\pgfpathlineto{\pgfqpoint{5.099519in}{2.460623in}}%
\pgfpathlineto{\pgfqpoint{5.245331in}{2.266200in}}%
\pgfpathclose%
\pgfusepath{fill}%
\end{pgfscope}%
\begin{pgfscope}%
\pgfpathrectangle{\pgfqpoint{0.539299in}{0.078740in}}{\pgfqpoint{7.842520in}{7.842520in}}%
\pgfusepath{clip}%
\pgfsetbuttcap%
\pgfsetroundjoin%
\definecolor{currentfill}{rgb}{0.160665,0.478540,0.558115}%
\pgfsetfillcolor{currentfill}%
\pgfsetlinewidth{0.000000pt}%
\definecolor{currentstroke}{rgb}{0.153894,0.680203,0.504172}%
\pgfsetstrokecolor{currentstroke}%
\pgfsetdash{}{0pt}%
\pgfpathmoveto{\pgfqpoint{4.200551in}{2.726746in}}%
\pgfpathlineto{\pgfqpoint{4.054864in}{2.841426in}}%
\pgfpathlineto{\pgfqpoint{3.971998in}{2.380078in}}%
\pgfpathclose%
\pgfusepath{fill}%
\end{pgfscope}%
\begin{pgfscope}%
\pgfpathrectangle{\pgfqpoint{0.539299in}{0.078740in}}{\pgfqpoint{7.842520in}{7.842520in}}%
\pgfusepath{clip}%
\pgfsetbuttcap%
\pgfsetroundjoin%
\definecolor{currentfill}{rgb}{0.229739,0.322361,0.545706}%
\pgfsetfillcolor{currentfill}%
\pgfsetlinewidth{0.000000pt}%
\definecolor{currentstroke}{rgb}{0.157851,0.683765,0.501686}%
\pgfsetstrokecolor{currentstroke}%
\pgfsetdash{}{0pt}%
\pgfpathmoveto{\pgfqpoint{4.324695in}{1.608470in}}%
\pgfpathlineto{\pgfqpoint{4.408361in}{2.095987in}}%
\pgfpathlineto{\pgfqpoint{4.262595in}{2.193629in}}%
\pgfpathclose%
\pgfusepath{fill}%
\end{pgfscope}%
\begin{pgfscope}%
\pgfpathrectangle{\pgfqpoint{0.539299in}{0.078740in}}{\pgfqpoint{7.842520in}{7.842520in}}%
\pgfusepath{clip}%
\pgfsetbuttcap%
\pgfsetroundjoin%
\definecolor{currentfill}{rgb}{0.239374,0.735588,0.455688}%
\pgfsetfillcolor{currentfill}%
\pgfsetlinewidth{0.000000pt}%
\definecolor{currentstroke}{rgb}{0.162016,0.687316,0.499129}%
\pgfsetstrokecolor{currentstroke}%
\pgfsetdash{}{0pt}%
\pgfpathmoveto{\pgfqpoint{3.785592in}{3.924272in}}%
\pgfpathlineto{\pgfqpoint{3.640137in}{4.053707in}}%
\pgfpathlineto{\pgfqpoint{3.701876in}{3.586183in}}%
\pgfpathclose%
\pgfusepath{fill}%
\end{pgfscope}%
\begin{pgfscope}%
\pgfpathrectangle{\pgfqpoint{0.539299in}{0.078740in}}{\pgfqpoint{7.842520in}{7.842520in}}%
\pgfusepath{clip}%
\pgfsetbuttcap%
\pgfsetroundjoin%
\definecolor{currentfill}{rgb}{0.319809,0.770914,0.411152}%
\pgfsetfillcolor{currentfill}%
\pgfsetlinewidth{0.000000pt}%
\definecolor{currentstroke}{rgb}{0.166383,0.690856,0.496502}%
\pgfsetstrokecolor{currentstroke}%
\pgfsetdash{}{0pt}%
\pgfpathmoveto{\pgfqpoint{3.640137in}{4.053707in}}%
\pgfpathlineto{\pgfqpoint{3.785592in}{3.924272in}}%
\pgfpathlineto{\pgfqpoint{3.870205in}{4.179274in}}%
\pgfpathclose%
\pgfusepath{fill}%
\end{pgfscope}%
\begin{pgfscope}%
\pgfpathrectangle{\pgfqpoint{0.539299in}{0.078740in}}{\pgfqpoint{7.842520in}{7.842520in}}%
\pgfusepath{clip}%
\pgfsetbuttcap%
\pgfsetroundjoin%
\definecolor{currentfill}{rgb}{0.133743,0.548535,0.553541}%
\pgfsetfillcolor{currentfill}%
\pgfsetlinewidth{0.000000pt}%
\definecolor{currentstroke}{rgb}{0.170948,0.694384,0.493803}%
\pgfsetstrokecolor{currentstroke}%
\pgfsetdash{}{0pt}%
\pgfpathmoveto{\pgfqpoint{4.807495in}{2.824405in}}%
\pgfpathlineto{\pgfqpoint{4.891094in}{2.947957in}}%
\pgfpathlineto{\pgfqpoint{4.745380in}{3.138731in}}%
\pgfpathclose%
\pgfusepath{fill}%
\end{pgfscope}%
\begin{pgfscope}%
\pgfpathrectangle{\pgfqpoint{0.539299in}{0.078740in}}{\pgfqpoint{7.842520in}{7.842520in}}%
\pgfusepath{clip}%
\pgfsetbuttcap%
\pgfsetroundjoin%
\definecolor{currentfill}{rgb}{0.319809,0.770914,0.411152}%
\pgfsetfillcolor{currentfill}%
\pgfsetlinewidth{0.000000pt}%
\definecolor{currentstroke}{rgb}{0.175707,0.697900,0.491033}%
\pgfsetstrokecolor{currentstroke}%
\pgfsetdash{}{0pt}%
\pgfpathmoveto{\pgfqpoint{3.870205in}{4.179274in}}%
\pgfpathlineto{\pgfqpoint{3.785592in}{3.924272in}}%
\pgfpathlineto{\pgfqpoint{4.015993in}{4.022293in}}%
\pgfpathclose%
\pgfusepath{fill}%
\end{pgfscope}%
\begin{pgfscope}%
\pgfpathrectangle{\pgfqpoint{0.539299in}{0.078740in}}{\pgfqpoint{7.842520in}{7.842520in}}%
\pgfusepath{clip}%
\pgfsetbuttcap%
\pgfsetroundjoin%
\definecolor{currentfill}{rgb}{0.182256,0.426184,0.557120}%
\pgfsetfillcolor{currentfill}%
\pgfsetlinewidth{0.000000pt}%
\definecolor{currentstroke}{rgb}{0.180653,0.701402,0.488189}%
\pgfsetstrokecolor{currentstroke}%
\pgfsetdash{}{0pt}%
\pgfpathmoveto{\pgfqpoint{4.117136in}{2.288294in}}%
\pgfpathlineto{\pgfqpoint{4.262595in}{2.193629in}}%
\pgfpathlineto{\pgfqpoint{4.200551in}{2.726746in}}%
\pgfpathclose%
\pgfusepath{fill}%
\end{pgfscope}%
\begin{pgfscope}%
\pgfpathrectangle{\pgfqpoint{0.539299in}{0.078740in}}{\pgfqpoint{7.842520in}{7.842520in}}%
\pgfusepath{clip}%
\pgfsetbuttcap%
\pgfsetroundjoin%
\definecolor{currentfill}{rgb}{0.121831,0.589055,0.545623}%
\pgfsetfillcolor{currentfill}%
\pgfsetlinewidth{0.000000pt}%
\definecolor{currentstroke}{rgb}{0.185783,0.704891,0.485273}%
\pgfsetstrokecolor{currentstroke}%
\pgfsetdash{}{0pt}%
\pgfpathmoveto{\pgfqpoint{4.599574in}{3.325252in}}%
\pgfpathlineto{\pgfqpoint{4.661391in}{2.997194in}}%
\pgfpathlineto{\pgfqpoint{4.745380in}{3.138731in}}%
\pgfpathclose%
\pgfusepath{fill}%
\end{pgfscope}%
\begin{pgfscope}%
\pgfpathrectangle{\pgfqpoint{0.539299in}{0.078740in}}{\pgfqpoint{7.842520in}{7.842520in}}%
\pgfusepath{clip}%
\pgfsetbuttcap%
\pgfsetroundjoin%
\definecolor{currentfill}{rgb}{0.168126,0.459988,0.558082}%
\pgfsetfillcolor{currentfill}%
\pgfsetlinewidth{0.000000pt}%
\definecolor{currentstroke}{rgb}{0.191090,0.708366,0.482284}%
\pgfsetstrokecolor{currentstroke}%
\pgfsetdash{}{0pt}%
\pgfpathmoveto{\pgfqpoint{5.099519in}{2.460623in}}%
\pgfpathlineto{\pgfqpoint{5.182120in}{2.551317in}}%
\pgfpathlineto{\pgfqpoint{4.953553in}{2.645992in}}%
\pgfpathclose%
\pgfusepath{fill}%
\end{pgfscope}%
\begin{pgfscope}%
\pgfpathrectangle{\pgfqpoint{0.539299in}{0.078740in}}{\pgfqpoint{7.842520in}{7.842520in}}%
\pgfusepath{clip}%
\pgfsetbuttcap%
\pgfsetroundjoin%
\definecolor{currentfill}{rgb}{0.208030,0.718701,0.472873}%
\pgfsetfillcolor{currentfill}%
\pgfsetlinewidth{0.000000pt}%
\definecolor{currentstroke}{rgb}{0.196571,0.711827,0.479221}%
\pgfsetstrokecolor{currentstroke}%
\pgfsetdash{}{0pt}%
\pgfpathmoveto{\pgfqpoint{3.931280in}{3.785621in}}%
\pgfpathlineto{\pgfqpoint{3.785592in}{3.924272in}}%
\pgfpathlineto{\pgfqpoint{3.701876in}{3.586183in}}%
\pgfpathclose%
\pgfusepath{fill}%
\end{pgfscope}%
\begin{pgfscope}%
\pgfpathrectangle{\pgfqpoint{0.539299in}{0.078740in}}{\pgfqpoint{7.842520in}{7.842520in}}%
\pgfusepath{clip}%
\pgfsetbuttcap%
\pgfsetroundjoin%
\definecolor{currentfill}{rgb}{0.162016,0.687316,0.499129}%
\pgfsetfillcolor{currentfill}%
\pgfsetlinewidth{0.000000pt}%
\definecolor{currentstroke}{rgb}{0.202219,0.715272,0.476084}%
\pgfsetstrokecolor{currentstroke}%
\pgfsetdash{}{0pt}%
\pgfpathmoveto{\pgfqpoint{3.701876in}{3.586183in}}%
\pgfpathlineto{\pgfqpoint{3.847250in}{3.466509in}}%
\pgfpathlineto{\pgfqpoint{3.931280in}{3.785621in}}%
\pgfpathclose%
\pgfusepath{fill}%
\end{pgfscope}%
\begin{pgfscope}%
\pgfpathrectangle{\pgfqpoint{0.539299in}{0.078740in}}{\pgfqpoint{7.842520in}{7.842520in}}%
\pgfusepath{clip}%
\pgfsetbuttcap%
\pgfsetroundjoin%
\definecolor{currentfill}{rgb}{0.119423,0.611141,0.538982}%
\pgfsetfillcolor{currentfill}%
\pgfsetlinewidth{0.000000pt}%
\definecolor{currentstroke}{rgb}{0.208030,0.718701,0.472873}%
\pgfsetstrokecolor{currentstroke}%
\pgfsetdash{}{0pt}%
\pgfpathmoveto{\pgfqpoint{3.909430in}{2.951944in}}%
\pgfpathlineto{\pgfqpoint{3.992871in}{3.340173in}}%
\pgfpathlineto{\pgfqpoint{3.847250in}{3.466509in}}%
\pgfpathclose%
\pgfusepath{fill}%
\end{pgfscope}%
\begin{pgfscope}%
\pgfpathrectangle{\pgfqpoint{0.539299in}{0.078740in}}{\pgfqpoint{7.842520in}{7.842520in}}%
\pgfusepath{clip}%
\pgfsetbuttcap%
\pgfsetroundjoin%
\definecolor{currentfill}{rgb}{0.121380,0.629492,0.531973}%
\pgfsetfillcolor{currentfill}%
\pgfsetlinewidth{0.000000pt}%
\definecolor{currentstroke}{rgb}{0.214000,0.722114,0.469588}%
\pgfsetstrokecolor{currentstroke}%
\pgfsetdash{}{0pt}%
\pgfpathmoveto{\pgfqpoint{4.515273in}{3.165074in}}%
\pgfpathlineto{\pgfqpoint{4.599574in}{3.325252in}}%
\pgfpathlineto{\pgfqpoint{4.453701in}{3.507500in}}%
\pgfpathclose%
\pgfusepath{fill}%
\end{pgfscope}%
\begin{pgfscope}%
\pgfpathrectangle{\pgfqpoint{0.539299in}{0.078740in}}{\pgfqpoint{7.842520in}{7.842520in}}%
\pgfusepath{clip}%
\pgfsetbuttcap%
\pgfsetroundjoin%
\definecolor{currentfill}{rgb}{0.259857,0.745492,0.444467}%
\pgfsetfillcolor{currentfill}%
\pgfsetlinewidth{0.000000pt}%
\definecolor{currentstroke}{rgb}{0.220124,0.725509,0.466226}%
\pgfsetstrokecolor{currentstroke}%
\pgfsetdash{}{0pt}%
\pgfpathmoveto{\pgfqpoint{4.015993in}{4.022293in}}%
\pgfpathlineto{\pgfqpoint{3.785592in}{3.924272in}}%
\pgfpathlineto{\pgfqpoint{3.931280in}{3.785621in}}%
\pgfpathclose%
\pgfusepath{fill}%
\end{pgfscope}%
\begin{pgfscope}%
\pgfpathrectangle{\pgfqpoint{0.539299in}{0.078740in}}{\pgfqpoint{7.842520in}{7.842520in}}%
\pgfusepath{clip}%
\pgfsetbuttcap%
\pgfsetroundjoin%
\definecolor{currentfill}{rgb}{0.232815,0.732247,0.459277}%
\pgfsetfillcolor{currentfill}%
\pgfsetlinewidth{0.000000pt}%
\definecolor{currentstroke}{rgb}{0.226397,0.728888,0.462789}%
\pgfsetstrokecolor{currentstroke}%
\pgfsetdash{}{0pt}%
\pgfpathmoveto{\pgfqpoint{4.077139in}{3.639292in}}%
\pgfpathlineto{\pgfqpoint{4.161873in}{3.857033in}}%
\pgfpathlineto{\pgfqpoint{4.015993in}{4.022293in}}%
\pgfpathclose%
\pgfusepath{fill}%
\end{pgfscope}%
\begin{pgfscope}%
\pgfpathrectangle{\pgfqpoint{0.539299in}{0.078740in}}{\pgfqpoint{7.842520in}{7.842520in}}%
\pgfusepath{clip}%
\pgfsetbuttcap%
\pgfsetroundjoin%
\definecolor{currentfill}{rgb}{0.150148,0.676631,0.506589}%
\pgfsetfillcolor{currentfill}%
\pgfsetlinewidth{0.000000pt}%
\definecolor{currentstroke}{rgb}{0.232815,0.732247,0.459277}%
\pgfsetstrokecolor{currentstroke}%
\pgfsetdash{}{0pt}%
\pgfpathmoveto{\pgfqpoint{4.307790in}{3.685037in}}%
\pgfpathlineto{\pgfqpoint{4.223116in}{3.486528in}}%
\pgfpathlineto{\pgfqpoint{4.453701in}{3.507500in}}%
\pgfpathclose%
\pgfusepath{fill}%
\end{pgfscope}%
\begin{pgfscope}%
\pgfpathrectangle{\pgfqpoint{0.539299in}{0.078740in}}{\pgfqpoint{7.842520in}{7.842520in}}%
\pgfusepath{clip}%
\pgfsetbuttcap%
\pgfsetroundjoin%
\definecolor{currentfill}{rgb}{0.180653,0.701402,0.488189}%
\pgfsetfillcolor{currentfill}%
\pgfsetlinewidth{0.000000pt}%
\definecolor{currentstroke}{rgb}{0.239374,0.735588,0.455688}%
\pgfsetstrokecolor{currentstroke}%
\pgfsetdash{}{0pt}%
\pgfpathmoveto{\pgfqpoint{4.161873in}{3.857033in}}%
\pgfpathlineto{\pgfqpoint{4.223116in}{3.486528in}}%
\pgfpathlineto{\pgfqpoint{4.307790in}{3.685037in}}%
\pgfpathclose%
\pgfusepath{fill}%
\end{pgfscope}%
\begin{pgfscope}%
\pgfpathrectangle{\pgfqpoint{0.539299in}{0.078740in}}{\pgfqpoint{7.842520in}{7.842520in}}%
\pgfusepath{clip}%
\pgfsetbuttcap%
\pgfsetroundjoin%
\definecolor{currentfill}{rgb}{0.131172,0.555899,0.552459}%
\pgfsetfillcolor{currentfill}%
\pgfsetlinewidth{0.000000pt}%
\definecolor{currentstroke}{rgb}{0.246070,0.738910,0.452024}%
\pgfsetstrokecolor{currentstroke}%
\pgfsetdash{}{0pt}%
\pgfpathmoveto{\pgfqpoint{3.909430in}{2.951944in}}%
\pgfpathlineto{\pgfqpoint{4.054864in}{2.841426in}}%
\pgfpathlineto{\pgfqpoint{4.138693in}{3.208126in}}%
\pgfpathclose%
\pgfusepath{fill}%
\end{pgfscope}%
\begin{pgfscope}%
\pgfpathrectangle{\pgfqpoint{0.539299in}{0.078740in}}{\pgfqpoint{7.842520in}{7.842520in}}%
\pgfusepath{clip}%
\pgfsetbuttcap%
\pgfsetroundjoin%
\definecolor{currentfill}{rgb}{0.226397,0.728888,0.462789}%
\pgfsetfillcolor{currentfill}%
\pgfsetlinewidth{0.000000pt}%
\definecolor{currentstroke}{rgb}{0.252899,0.742211,0.448284}%
\pgfsetstrokecolor{currentstroke}%
\pgfsetdash{}{0pt}%
\pgfpathmoveto{\pgfqpoint{3.931280in}{3.785621in}}%
\pgfpathlineto{\pgfqpoint{4.077139in}{3.639292in}}%
\pgfpathlineto{\pgfqpoint{4.015993in}{4.022293in}}%
\pgfpathclose%
\pgfusepath{fill}%
\end{pgfscope}%
\begin{pgfscope}%
\pgfpathrectangle{\pgfqpoint{0.539299in}{0.078740in}}{\pgfqpoint{7.842520in}{7.842520in}}%
\pgfusepath{clip}%
\pgfsetbuttcap%
\pgfsetroundjoin%
\definecolor{currentfill}{rgb}{0.146180,0.515413,0.556823}%
\pgfsetfillcolor{currentfill}%
\pgfsetlinewidth{0.000000pt}%
\definecolor{currentstroke}{rgb}{0.259857,0.745492,0.444467}%
\pgfsetstrokecolor{currentstroke}%
\pgfsetdash{}{0pt}%
\pgfpathmoveto{\pgfqpoint{4.891094in}{2.947957in}}%
\pgfpathlineto{\pgfqpoint{4.807495in}{2.824405in}}%
\pgfpathlineto{\pgfqpoint{4.953553in}{2.645992in}}%
\pgfpathclose%
\pgfusepath{fill}%
\end{pgfscope}%
\begin{pgfscope}%
\pgfpathrectangle{\pgfqpoint{0.539299in}{0.078740in}}{\pgfqpoint{7.842520in}{7.842520in}}%
\pgfusepath{clip}%
\pgfsetbuttcap%
\pgfsetroundjoin%
\definecolor{currentfill}{rgb}{0.121148,0.592739,0.544641}%
\pgfsetfillcolor{currentfill}%
\pgfsetlinewidth{0.000000pt}%
\definecolor{currentstroke}{rgb}{0.266941,0.748751,0.440573}%
\pgfsetstrokecolor{currentstroke}%
\pgfsetdash{}{0pt}%
\pgfpathmoveto{\pgfqpoint{4.138693in}{3.208126in}}%
\pgfpathlineto{\pgfqpoint{3.992871in}{3.340173in}}%
\pgfpathlineto{\pgfqpoint{3.909430in}{2.951944in}}%
\pgfpathclose%
\pgfusepath{fill}%
\end{pgfscope}%
\begin{pgfscope}%
\pgfpathrectangle{\pgfqpoint{0.539299in}{0.078740in}}{\pgfqpoint{7.842520in}{7.842520in}}%
\pgfusepath{clip}%
\pgfsetbuttcap%
\pgfsetroundjoin%
\definecolor{currentfill}{rgb}{0.146616,0.673050,0.508936}%
\pgfsetfillcolor{currentfill}%
\pgfsetlinewidth{0.000000pt}%
\definecolor{currentstroke}{rgb}{0.274149,0.751988,0.436601}%
\pgfsetstrokecolor{currentstroke}%
\pgfsetdash{}{0pt}%
\pgfpathmoveto{\pgfqpoint{3.931280in}{3.785621in}}%
\pgfpathlineto{\pgfqpoint{3.847250in}{3.466509in}}%
\pgfpathlineto{\pgfqpoint{3.992871in}{3.340173in}}%
\pgfpathclose%
\pgfusepath{fill}%
\end{pgfscope}%
\begin{pgfscope}%
\pgfpathrectangle{\pgfqpoint{0.539299in}{0.078740in}}{\pgfqpoint{7.842520in}{7.842520in}}%
\pgfusepath{clip}%
\pgfsetbuttcap%
\pgfsetroundjoin%
\definecolor{currentfill}{rgb}{0.229739,0.322361,0.545706}%
\pgfsetfillcolor{currentfill}%
\pgfsetlinewidth{0.000000pt}%
\definecolor{currentstroke}{rgb}{0.281477,0.755203,0.432552}%
\pgfsetstrokecolor{currentstroke}%
\pgfsetdash{}{0pt}%
\pgfpathmoveto{\pgfqpoint{5.390896in}{2.059660in}}%
\pgfpathlineto{\pgfqpoint{5.308523in}{1.948082in}}%
\pgfpathlineto{\pgfqpoint{5.536081in}{1.836733in}}%
\pgfpathclose%
\pgfusepath{fill}%
\end{pgfscope}%
\begin{pgfscope}%
\pgfpathrectangle{\pgfqpoint{0.539299in}{0.078740in}}{\pgfqpoint{7.842520in}{7.842520in}}%
\pgfusepath{clip}%
\pgfsetbuttcap%
\pgfsetroundjoin%
\definecolor{currentfill}{rgb}{0.175707,0.697900,0.491033}%
\pgfsetfillcolor{currentfill}%
\pgfsetlinewidth{0.000000pt}%
\definecolor{currentstroke}{rgb}{0.288921,0.758394,0.428426}%
\pgfsetstrokecolor{currentstroke}%
\pgfsetdash{}{0pt}%
\pgfpathmoveto{\pgfqpoint{4.161873in}{3.857033in}}%
\pgfpathlineto{\pgfqpoint{4.077139in}{3.639292in}}%
\pgfpathlineto{\pgfqpoint{4.223116in}{3.486528in}}%
\pgfpathclose%
\pgfusepath{fill}%
\end{pgfscope}%
\begin{pgfscope}%
\pgfpathrectangle{\pgfqpoint{0.539299in}{0.078740in}}{\pgfqpoint{7.842520in}{7.842520in}}%
\pgfusepath{clip}%
\pgfsetbuttcap%
\pgfsetroundjoin%
\definecolor{currentfill}{rgb}{0.131172,0.555899,0.552459}%
\pgfsetfillcolor{currentfill}%
\pgfsetlinewidth{0.000000pt}%
\definecolor{currentstroke}{rgb}{0.296479,0.761561,0.424223}%
\pgfsetstrokecolor{currentstroke}%
\pgfsetdash{}{0pt}%
\pgfpathmoveto{\pgfqpoint{4.745380in}{3.138731in}}%
\pgfpathlineto{\pgfqpoint{4.661391in}{2.997194in}}%
\pgfpathlineto{\pgfqpoint{4.807495in}{2.824405in}}%
\pgfpathclose%
\pgfusepath{fill}%
\end{pgfscope}%
\begin{pgfscope}%
\pgfpathrectangle{\pgfqpoint{0.539299in}{0.078740in}}{\pgfqpoint{7.842520in}{7.842520in}}%
\pgfusepath{clip}%
\pgfsetbuttcap%
\pgfsetroundjoin%
\definecolor{currentfill}{rgb}{0.132268,0.655014,0.519661}%
\pgfsetfillcolor{currentfill}%
\pgfsetlinewidth{0.000000pt}%
\definecolor{currentstroke}{rgb}{0.304148,0.764704,0.419943}%
\pgfsetstrokecolor{currentstroke}%
\pgfsetdash{}{0pt}%
\pgfpathmoveto{\pgfqpoint{4.223116in}{3.486528in}}%
\pgfpathlineto{\pgfqpoint{4.369172in}{3.328258in}}%
\pgfpathlineto{\pgfqpoint{4.453701in}{3.507500in}}%
\pgfpathclose%
\pgfusepath{fill}%
\end{pgfscope}%
\begin{pgfscope}%
\pgfpathrectangle{\pgfqpoint{0.539299in}{0.078740in}}{\pgfqpoint{7.842520in}{7.842520in}}%
\pgfusepath{clip}%
\pgfsetbuttcap%
\pgfsetroundjoin%
\definecolor{currentfill}{rgb}{0.121148,0.592739,0.544641}%
\pgfsetfillcolor{currentfill}%
\pgfsetlinewidth{0.000000pt}%
\definecolor{currentstroke}{rgb}{0.311925,0.767822,0.415586}%
\pgfsetstrokecolor{currentstroke}%
\pgfsetdash{}{0pt}%
\pgfpathmoveto{\pgfqpoint{4.515273in}{3.165074in}}%
\pgfpathlineto{\pgfqpoint{4.661391in}{2.997194in}}%
\pgfpathlineto{\pgfqpoint{4.599574in}{3.325252in}}%
\pgfpathclose%
\pgfusepath{fill}%
\end{pgfscope}%
\begin{pgfscope}%
\pgfpathrectangle{\pgfqpoint{0.539299in}{0.078740in}}{\pgfqpoint{7.842520in}{7.842520in}}%
\pgfusepath{clip}%
\pgfsetbuttcap%
\pgfsetroundjoin%
\definecolor{currentfill}{rgb}{0.157851,0.683765,0.501686}%
\pgfsetfillcolor{currentfill}%
\pgfsetlinewidth{0.000000pt}%
\definecolor{currentstroke}{rgb}{0.319809,0.770914,0.411152}%
\pgfsetstrokecolor{currentstroke}%
\pgfsetdash{}{0pt}%
\pgfpathmoveto{\pgfqpoint{3.992871in}{3.340173in}}%
\pgfpathlineto{\pgfqpoint{4.077139in}{3.639292in}}%
\pgfpathlineto{\pgfqpoint{3.931280in}{3.785621in}}%
\pgfpathclose%
\pgfusepath{fill}%
\end{pgfscope}%
\begin{pgfscope}%
\pgfpathrectangle{\pgfqpoint{0.539299in}{0.078740in}}{\pgfqpoint{7.842520in}{7.842520in}}%
\pgfusepath{clip}%
\pgfsetbuttcap%
\pgfsetroundjoin%
\definecolor{currentfill}{rgb}{0.122312,0.633153,0.530398}%
\pgfsetfillcolor{currentfill}%
\pgfsetlinewidth{0.000000pt}%
\definecolor{currentstroke}{rgb}{0.327796,0.773980,0.406640}%
\pgfsetstrokecolor{currentstroke}%
\pgfsetdash{}{0pt}%
\pgfpathmoveto{\pgfqpoint{4.453701in}{3.507500in}}%
\pgfpathlineto{\pgfqpoint{4.369172in}{3.328258in}}%
\pgfpathlineto{\pgfqpoint{4.515273in}{3.165074in}}%
\pgfpathclose%
\pgfusepath{fill}%
\end{pgfscope}%
\begin{pgfscope}%
\pgfpathrectangle{\pgfqpoint{0.539299in}{0.078740in}}{\pgfqpoint{7.842520in}{7.842520in}}%
\pgfusepath{clip}%
\pgfsetbuttcap%
\pgfsetroundjoin%
\definecolor{currentfill}{rgb}{0.258965,0.251537,0.524736}%
\pgfsetfillcolor{currentfill}%
\pgfsetlinewidth{0.000000pt}%
\definecolor{currentstroke}{rgb}{0.335885,0.777018,0.402049}%
\pgfsetstrokecolor{currentstroke}%
\pgfsetdash{}{0pt}%
\pgfpathmoveto{\pgfqpoint{4.554418in}{1.995003in}}%
\pgfpathlineto{\pgfqpoint{4.470234in}{1.534400in}}%
\pgfpathlineto{\pgfqpoint{4.616159in}{1.458069in}}%
\pgfpathclose%
\pgfusepath{fill}%
\end{pgfscope}%
\begin{pgfscope}%
\pgfpathrectangle{\pgfqpoint{0.539299in}{0.078740in}}{\pgfqpoint{7.842520in}{7.842520in}}%
\pgfusepath{clip}%
\pgfsetbuttcap%
\pgfsetroundjoin%
\definecolor{currentfill}{rgb}{0.255645,0.260703,0.528312}%
\pgfsetfillcolor{currentfill}%
\pgfsetlinewidth{0.000000pt}%
\definecolor{currentstroke}{rgb}{0.344074,0.780029,0.397381}%
\pgfsetstrokecolor{currentstroke}%
\pgfsetdash{}{0pt}%
\pgfpathmoveto{\pgfqpoint{5.536081in}{1.836733in}}%
\pgfpathlineto{\pgfqpoint{5.454351in}{1.741746in}}%
\pgfpathlineto{\pgfqpoint{5.599643in}{1.509341in}}%
\pgfpathclose%
\pgfusepath{fill}%
\end{pgfscope}%
\begin{pgfscope}%
\pgfpathrectangle{\pgfqpoint{0.539299in}{0.078740in}}{\pgfqpoint{7.842520in}{7.842520in}}%
\pgfusepath{clip}%
\pgfsetbuttcap%
\pgfsetroundjoin%
\definecolor{currentfill}{rgb}{0.237441,0.305202,0.541921}%
\pgfsetfillcolor{currentfill}%
\pgfsetlinewidth{0.000000pt}%
\definecolor{currentstroke}{rgb}{0.352360,0.783011,0.392636}%
\pgfsetstrokecolor{currentstroke}%
\pgfsetdash{}{0pt}%
\pgfpathmoveto{\pgfqpoint{4.408361in}{2.095987in}}%
\pgfpathlineto{\pgfqpoint{4.470234in}{1.534400in}}%
\pgfpathlineto{\pgfqpoint{4.554418in}{1.995003in}}%
\pgfpathclose%
\pgfusepath{fill}%
\end{pgfscope}%
\begin{pgfscope}%
\pgfpathrectangle{\pgfqpoint{0.539299in}{0.078740in}}{\pgfqpoint{7.842520in}{7.842520in}}%
\pgfusepath{clip}%
\pgfsetbuttcap%
\pgfsetroundjoin%
\definecolor{currentfill}{rgb}{0.169646,0.456262,0.558030}%
\pgfsetfillcolor{currentfill}%
\pgfsetlinewidth{0.000000pt}%
\definecolor{currentstroke}{rgb}{0.360741,0.785964,0.387814}%
\pgfsetstrokecolor{currentstroke}%
\pgfsetdash{}{0pt}%
\pgfpathmoveto{\pgfqpoint{4.200551in}{2.726746in}}%
\pgfpathlineto{\pgfqpoint{4.262595in}{2.193629in}}%
\pgfpathlineto{\pgfqpoint{4.346469in}{2.608126in}}%
\pgfpathclose%
\pgfusepath{fill}%
\end{pgfscope}%
\begin{pgfscope}%
\pgfpathrectangle{\pgfqpoint{0.539299in}{0.078740in}}{\pgfqpoint{7.842520in}{7.842520in}}%
\pgfusepath{clip}%
\pgfsetbuttcap%
\pgfsetroundjoin%
\definecolor{currentfill}{rgb}{0.140210,0.665859,0.513427}%
\pgfsetfillcolor{currentfill}%
\pgfsetlinewidth{0.000000pt}%
\definecolor{currentstroke}{rgb}{0.369214,0.788888,0.382914}%
\pgfsetstrokecolor{currentstroke}%
\pgfsetdash{}{0pt}%
\pgfpathmoveto{\pgfqpoint{4.223116in}{3.486528in}}%
\pgfpathlineto{\pgfqpoint{4.077139in}{3.639292in}}%
\pgfpathlineto{\pgfqpoint{3.992871in}{3.340173in}}%
\pgfpathclose%
\pgfusepath{fill}%
\end{pgfscope}%
\begin{pgfscope}%
\pgfpathrectangle{\pgfqpoint{0.539299in}{0.078740in}}{\pgfqpoint{7.842520in}{7.842520in}}%
\pgfusepath{clip}%
\pgfsetbuttcap%
\pgfsetroundjoin%
\definecolor{currentfill}{rgb}{0.139147,0.533812,0.555298}%
\pgfsetfillcolor{currentfill}%
\pgfsetlinewidth{0.000000pt}%
\definecolor{currentstroke}{rgb}{0.377779,0.791781,0.377939}%
\pgfsetstrokecolor{currentstroke}%
\pgfsetdash{}{0pt}%
\pgfpathmoveto{\pgfqpoint{4.054864in}{2.841426in}}%
\pgfpathlineto{\pgfqpoint{4.200551in}{2.726746in}}%
\pgfpathlineto{\pgfqpoint{4.284682in}{3.071051in}}%
\pgfpathclose%
\pgfusepath{fill}%
\end{pgfscope}%
\begin{pgfscope}%
\pgfpathrectangle{\pgfqpoint{0.539299in}{0.078740in}}{\pgfqpoint{7.842520in}{7.842520in}}%
\pgfusepath{clip}%
\pgfsetbuttcap%
\pgfsetroundjoin%
\definecolor{currentfill}{rgb}{0.212395,0.359683,0.551710}%
\pgfsetfillcolor{currentfill}%
\pgfsetlinewidth{0.000000pt}%
\definecolor{currentstroke}{rgb}{0.386433,0.794644,0.372886}%
\pgfsetstrokecolor{currentstroke}%
\pgfsetdash{}{0pt}%
\pgfpathmoveto{\pgfqpoint{5.308523in}{1.948082in}}%
\pgfpathlineto{\pgfqpoint{5.390896in}{2.059660in}}%
\pgfpathlineto{\pgfqpoint{5.245331in}{2.266200in}}%
\pgfpathclose%
\pgfusepath{fill}%
\end{pgfscope}%
\begin{pgfscope}%
\pgfpathrectangle{\pgfqpoint{0.539299in}{0.078740in}}{\pgfqpoint{7.842520in}{7.842520in}}%
\pgfusepath{clip}%
\pgfsetbuttcap%
\pgfsetroundjoin%
\definecolor{currentfill}{rgb}{0.123444,0.636809,0.528763}%
\pgfsetfillcolor{currentfill}%
\pgfsetlinewidth{0.000000pt}%
\definecolor{currentstroke}{rgb}{0.395174,0.797475,0.367757}%
\pgfsetstrokecolor{currentstroke}%
\pgfsetdash{}{0pt}%
\pgfpathmoveto{\pgfqpoint{3.992871in}{3.340173in}}%
\pgfpathlineto{\pgfqpoint{4.138693in}{3.208126in}}%
\pgfpathlineto{\pgfqpoint{4.223116in}{3.486528in}}%
\pgfpathclose%
\pgfusepath{fill}%
\end{pgfscope}%
\begin{pgfscope}%
\pgfpathrectangle{\pgfqpoint{0.539299in}{0.078740in}}{\pgfqpoint{7.842520in}{7.842520in}}%
\pgfusepath{clip}%
\pgfsetbuttcap%
\pgfsetroundjoin%
\definecolor{currentfill}{rgb}{0.126453,0.570633,0.549841}%
\pgfsetfillcolor{currentfill}%
\pgfsetlinewidth{0.000000pt}%
\definecolor{currentstroke}{rgb}{0.404001,0.800275,0.362552}%
\pgfsetstrokecolor{currentstroke}%
\pgfsetdash{}{0pt}%
\pgfpathmoveto{\pgfqpoint{4.138693in}{3.208126in}}%
\pgfpathlineto{\pgfqpoint{4.054864in}{2.841426in}}%
\pgfpathlineto{\pgfqpoint{4.284682in}{3.071051in}}%
\pgfpathclose%
\pgfusepath{fill}%
\end{pgfscope}%
\begin{pgfscope}%
\pgfpathrectangle{\pgfqpoint{0.539299in}{0.078740in}}{\pgfqpoint{7.842520in}{7.842520in}}%
\pgfusepath{clip}%
\pgfsetbuttcap%
\pgfsetroundjoin%
\definecolor{currentfill}{rgb}{0.281887,0.150881,0.465405}%
\pgfsetfillcolor{currentfill}%
\pgfsetlinewidth{0.000000pt}%
\definecolor{currentstroke}{rgb}{0.412913,0.803041,0.357269}%
\pgfsetstrokecolor{currentstroke}%
\pgfsetdash{}{0pt}%
\pgfpathmoveto{\pgfqpoint{5.599643in}{1.509341in}}%
\pgfpathlineto{\pgfqpoint{5.662874in}{1.144010in}}%
\pgfpathlineto{\pgfqpoint{5.744062in}{1.240624in}}%
\pgfpathclose%
\pgfusepath{fill}%
\end{pgfscope}%
\begin{pgfscope}%
\pgfpathrectangle{\pgfqpoint{0.539299in}{0.078740in}}{\pgfqpoint{7.842520in}{7.842520in}}%
\pgfusepath{clip}%
\pgfsetbuttcap%
\pgfsetroundjoin%
\definecolor{currentfill}{rgb}{0.194100,0.399323,0.555565}%
\pgfsetfillcolor{currentfill}%
\pgfsetlinewidth{0.000000pt}%
\definecolor{currentstroke}{rgb}{0.421908,0.805774,0.351910}%
\pgfsetstrokecolor{currentstroke}%
\pgfsetdash{}{0pt}%
\pgfpathmoveto{\pgfqpoint{4.492593in}{2.485503in}}%
\pgfpathlineto{\pgfqpoint{4.262595in}{2.193629in}}%
\pgfpathlineto{\pgfqpoint{4.408361in}{2.095987in}}%
\pgfpathclose%
\pgfusepath{fill}%
\end{pgfscope}%
\begin{pgfscope}%
\pgfpathrectangle{\pgfqpoint{0.539299in}{0.078740in}}{\pgfqpoint{7.842520in}{7.842520in}}%
\pgfusepath{clip}%
\pgfsetbuttcap%
\pgfsetroundjoin%
\definecolor{currentfill}{rgb}{0.239346,0.300855,0.540844}%
\pgfsetfillcolor{currentfill}%
\pgfsetlinewidth{0.000000pt}%
\definecolor{currentstroke}{rgb}{0.430983,0.808473,0.346476}%
\pgfsetstrokecolor{currentstroke}%
\pgfsetdash{}{0pt}%
\pgfpathmoveto{\pgfqpoint{5.536081in}{1.836733in}}%
\pgfpathlineto{\pgfqpoint{5.308523in}{1.948082in}}%
\pgfpathlineto{\pgfqpoint{5.454351in}{1.741746in}}%
\pgfpathclose%
\pgfusepath{fill}%
\end{pgfscope}%
\begin{pgfscope}%
\pgfpathrectangle{\pgfqpoint{0.539299in}{0.078740in}}{\pgfqpoint{7.842520in}{7.842520in}}%
\pgfusepath{clip}%
\pgfsetbuttcap%
\pgfsetroundjoin%
\definecolor{currentfill}{rgb}{0.175841,0.441290,0.557685}%
\pgfsetfillcolor{currentfill}%
\pgfsetlinewidth{0.000000pt}%
\definecolor{currentstroke}{rgb}{0.440137,0.811138,0.340967}%
\pgfsetstrokecolor{currentstroke}%
\pgfsetdash{}{0pt}%
\pgfpathmoveto{\pgfqpoint{4.346469in}{2.608126in}}%
\pgfpathlineto{\pgfqpoint{4.262595in}{2.193629in}}%
\pgfpathlineto{\pgfqpoint{4.492593in}{2.485503in}}%
\pgfpathclose%
\pgfusepath{fill}%
\end{pgfscope}%
\begin{pgfscope}%
\pgfpathrectangle{\pgfqpoint{0.539299in}{0.078740in}}{\pgfqpoint{7.842520in}{7.842520in}}%
\pgfusepath{clip}%
\pgfsetbuttcap%
\pgfsetroundjoin%
\definecolor{currentfill}{rgb}{0.120638,0.625828,0.533488}%
\pgfsetfillcolor{currentfill}%
\pgfsetlinewidth{0.000000pt}%
\definecolor{currentstroke}{rgb}{0.449368,0.813768,0.335384}%
\pgfsetstrokecolor{currentstroke}%
\pgfsetdash{}{0pt}%
\pgfpathmoveto{\pgfqpoint{4.284682in}{3.071051in}}%
\pgfpathlineto{\pgfqpoint{4.369172in}{3.328258in}}%
\pgfpathlineto{\pgfqpoint{4.223116in}{3.486528in}}%
\pgfpathclose%
\pgfusepath{fill}%
\end{pgfscope}%
\begin{pgfscope}%
\pgfpathrectangle{\pgfqpoint{0.539299in}{0.078740in}}{\pgfqpoint{7.842520in}{7.842520in}}%
\pgfusepath{clip}%
\pgfsetbuttcap%
\pgfsetroundjoin%
\definecolor{currentfill}{rgb}{0.119699,0.618490,0.536347}%
\pgfsetfillcolor{currentfill}%
\pgfsetlinewidth{0.000000pt}%
\definecolor{currentstroke}{rgb}{0.458674,0.816363,0.329727}%
\pgfsetstrokecolor{currentstroke}%
\pgfsetdash{}{0pt}%
\pgfpathmoveto{\pgfqpoint{4.223116in}{3.486528in}}%
\pgfpathlineto{\pgfqpoint{4.138693in}{3.208126in}}%
\pgfpathlineto{\pgfqpoint{4.284682in}{3.071051in}}%
\pgfpathclose%
\pgfusepath{fill}%
\end{pgfscope}%
\begin{pgfscope}%
\pgfpathrectangle{\pgfqpoint{0.539299in}{0.078740in}}{\pgfqpoint{7.842520in}{7.842520in}}%
\pgfusepath{clip}%
\pgfsetbuttcap%
\pgfsetroundjoin%
\definecolor{currentfill}{rgb}{0.119738,0.603785,0.541400}%
\pgfsetfillcolor{currentfill}%
\pgfsetlinewidth{0.000000pt}%
\definecolor{currentstroke}{rgb}{0.468053,0.818921,0.323998}%
\pgfsetstrokecolor{currentstroke}%
\pgfsetdash{}{0pt}%
\pgfpathmoveto{\pgfqpoint{4.515273in}{3.165074in}}%
\pgfpathlineto{\pgfqpoint{4.369172in}{3.328258in}}%
\pgfpathlineto{\pgfqpoint{4.284682in}{3.071051in}}%
\pgfpathclose%
\pgfusepath{fill}%
\end{pgfscope}%
\begin{pgfscope}%
\pgfpathrectangle{\pgfqpoint{0.539299in}{0.078740in}}{\pgfqpoint{7.842520in}{7.842520in}}%
\pgfusepath{clip}%
\pgfsetbuttcap%
\pgfsetroundjoin%
\definecolor{currentfill}{rgb}{0.183898,0.422383,0.556944}%
\pgfsetfillcolor{currentfill}%
\pgfsetlinewidth{0.000000pt}%
\definecolor{currentstroke}{rgb}{0.477504,0.821444,0.318195}%
\pgfsetstrokecolor{currentstroke}%
\pgfsetdash{}{0pt}%
\pgfpathmoveto{\pgfqpoint{5.245331in}{2.266200in}}%
\pgfpathlineto{\pgfqpoint{5.099519in}{2.460623in}}%
\pgfpathlineto{\pgfqpoint{5.016082in}{2.310084in}}%
\pgfpathclose%
\pgfusepath{fill}%
\end{pgfscope}%
\begin{pgfscope}%
\pgfpathrectangle{\pgfqpoint{0.539299in}{0.078740in}}{\pgfqpoint{7.842520in}{7.842520in}}%
\pgfusepath{clip}%
\pgfsetbuttcap%
\pgfsetroundjoin%
\definecolor{currentfill}{rgb}{0.143343,0.522773,0.556295}%
\pgfsetfillcolor{currentfill}%
\pgfsetlinewidth{0.000000pt}%
\definecolor{currentstroke}{rgb}{0.487026,0.823929,0.312321}%
\pgfsetstrokecolor{currentstroke}%
\pgfsetdash{}{0pt}%
\pgfpathmoveto{\pgfqpoint{4.284682in}{3.071051in}}%
\pgfpathlineto{\pgfqpoint{4.200551in}{2.726746in}}%
\pgfpathlineto{\pgfqpoint{4.346469in}{2.608126in}}%
\pgfpathclose%
\pgfusepath{fill}%
\end{pgfscope}%
\begin{pgfscope}%
\pgfpathrectangle{\pgfqpoint{0.539299in}{0.078740in}}{\pgfqpoint{7.842520in}{7.842520in}}%
\pgfusepath{clip}%
\pgfsetbuttcap%
\pgfsetroundjoin%
\definecolor{currentfill}{rgb}{0.171176,0.452530,0.557965}%
\pgfsetfillcolor{currentfill}%
\pgfsetlinewidth{0.000000pt}%
\definecolor{currentstroke}{rgb}{0.496615,0.826376,0.306377}%
\pgfsetstrokecolor{currentstroke}%
\pgfsetdash{}{0pt}%
\pgfpathmoveto{\pgfqpoint{5.099519in}{2.460623in}}%
\pgfpathlineto{\pgfqpoint{4.953553in}{2.645992in}}%
\pgfpathlineto{\pgfqpoint{5.016082in}{2.310084in}}%
\pgfpathclose%
\pgfusepath{fill}%
\end{pgfscope}%
\begin{pgfscope}%
\pgfpathrectangle{\pgfqpoint{0.539299in}{0.078740in}}{\pgfqpoint{7.842520in}{7.842520in}}%
\pgfusepath{clip}%
\pgfsetbuttcap%
\pgfsetroundjoin%
\definecolor{currentfill}{rgb}{0.137770,0.537492,0.554906}%
\pgfsetfillcolor{currentfill}%
\pgfsetlinewidth{0.000000pt}%
\definecolor{currentstroke}{rgb}{0.506271,0.828786,0.300362}%
\pgfsetstrokecolor{currentstroke}%
\pgfsetdash{}{0pt}%
\pgfpathmoveto{\pgfqpoint{4.661391in}{2.997194in}}%
\pgfpathlineto{\pgfqpoint{4.577041in}{2.783012in}}%
\pgfpathlineto{\pgfqpoint{4.807495in}{2.824405in}}%
\pgfpathclose%
\pgfusepath{fill}%
\end{pgfscope}%
\begin{pgfscope}%
\pgfpathrectangle{\pgfqpoint{0.539299in}{0.078740in}}{\pgfqpoint{7.842520in}{7.842520in}}%
\pgfusepath{clip}%
\pgfsetbuttcap%
\pgfsetroundjoin%
\definecolor{currentfill}{rgb}{0.129933,0.559582,0.551864}%
\pgfsetfillcolor{currentfill}%
\pgfsetlinewidth{0.000000pt}%
\definecolor{currentstroke}{rgb}{0.515992,0.831158,0.294279}%
\pgfsetstrokecolor{currentstroke}%
\pgfsetdash{}{0pt}%
\pgfpathmoveto{\pgfqpoint{4.577041in}{2.783012in}}%
\pgfpathlineto{\pgfqpoint{4.661391in}{2.997194in}}%
\pgfpathlineto{\pgfqpoint{4.515273in}{3.165074in}}%
\pgfpathclose%
\pgfusepath{fill}%
\end{pgfscope}%
\begin{pgfscope}%
\pgfpathrectangle{\pgfqpoint{0.539299in}{0.078740in}}{\pgfqpoint{7.842520in}{7.842520in}}%
\pgfusepath{clip}%
\pgfsetbuttcap%
\pgfsetroundjoin%
\definecolor{currentfill}{rgb}{0.151918,0.500685,0.557587}%
\pgfsetfillcolor{currentfill}%
\pgfsetlinewidth{0.000000pt}%
\definecolor{currentstroke}{rgb}{0.525776,0.833491,0.288127}%
\pgfsetstrokecolor{currentstroke}%
\pgfsetdash{}{0pt}%
\pgfpathmoveto{\pgfqpoint{4.953553in}{2.645992in}}%
\pgfpathlineto{\pgfqpoint{4.807495in}{2.824405in}}%
\pgfpathlineto{\pgfqpoint{4.723356in}{2.631725in}}%
\pgfpathclose%
\pgfusepath{fill}%
\end{pgfscope}%
\begin{pgfscope}%
\pgfpathrectangle{\pgfqpoint{0.539299in}{0.078740in}}{\pgfqpoint{7.842520in}{7.842520in}}%
\pgfusepath{clip}%
\pgfsetbuttcap%
\pgfsetroundjoin%
\definecolor{currentfill}{rgb}{0.124395,0.578002,0.548287}%
\pgfsetfillcolor{currentfill}%
\pgfsetlinewidth{0.000000pt}%
\definecolor{currentstroke}{rgb}{0.535621,0.835785,0.281908}%
\pgfsetstrokecolor{currentstroke}%
\pgfsetdash{}{0pt}%
\pgfpathmoveto{\pgfqpoint{4.284682in}{3.071051in}}%
\pgfpathlineto{\pgfqpoint{4.430807in}{2.929332in}}%
\pgfpathlineto{\pgfqpoint{4.515273in}{3.165074in}}%
\pgfpathclose%
\pgfusepath{fill}%
\end{pgfscope}%
\begin{pgfscope}%
\pgfpathrectangle{\pgfqpoint{0.539299in}{0.078740in}}{\pgfqpoint{7.842520in}{7.842520in}}%
\pgfusepath{clip}%
\pgfsetbuttcap%
\pgfsetroundjoin%
\definecolor{currentfill}{rgb}{0.199430,0.387607,0.554642}%
\pgfsetfillcolor{currentfill}%
\pgfsetlinewidth{0.000000pt}%
\definecolor{currentstroke}{rgb}{0.545524,0.838039,0.275626}%
\pgfsetstrokecolor{currentstroke}%
\pgfsetdash{}{0pt}%
\pgfpathmoveto{\pgfqpoint{4.492593in}{2.485503in}}%
\pgfpathlineto{\pgfqpoint{4.408361in}{2.095987in}}%
\pgfpathlineto{\pgfqpoint{4.554418in}{1.995003in}}%
\pgfpathclose%
\pgfusepath{fill}%
\end{pgfscope}%
\begin{pgfscope}%
\pgfpathrectangle{\pgfqpoint{0.539299in}{0.078740in}}{\pgfqpoint{7.842520in}{7.842520in}}%
\pgfusepath{clip}%
\pgfsetbuttcap%
\pgfsetroundjoin%
\definecolor{currentfill}{rgb}{0.137770,0.537492,0.554906}%
\pgfsetfillcolor{currentfill}%
\pgfsetlinewidth{0.000000pt}%
\definecolor{currentstroke}{rgb}{0.555484,0.840254,0.269281}%
\pgfsetstrokecolor{currentstroke}%
\pgfsetdash{}{0pt}%
\pgfpathmoveto{\pgfqpoint{4.346469in}{2.608126in}}%
\pgfpathlineto{\pgfqpoint{4.430807in}{2.929332in}}%
\pgfpathlineto{\pgfqpoint{4.284682in}{3.071051in}}%
\pgfpathclose%
\pgfusepath{fill}%
\end{pgfscope}%
\begin{pgfscope}%
\pgfpathrectangle{\pgfqpoint{0.539299in}{0.078740in}}{\pgfqpoint{7.842520in}{7.842520in}}%
\pgfusepath{clip}%
\pgfsetbuttcap%
\pgfsetroundjoin%
\definecolor{currentfill}{rgb}{0.243113,0.292092,0.538516}%
\pgfsetfillcolor{currentfill}%
\pgfsetlinewidth{0.000000pt}%
\definecolor{currentstroke}{rgb}{0.565498,0.842430,0.262877}%
\pgfsetstrokecolor{currentstroke}%
\pgfsetdash{}{0pt}%
\pgfpathmoveto{\pgfqpoint{4.616159in}{1.458069in}}%
\pgfpathlineto{\pgfqpoint{4.700745in}{1.889957in}}%
\pgfpathlineto{\pgfqpoint{4.554418in}{1.995003in}}%
\pgfpathclose%
\pgfusepath{fill}%
\end{pgfscope}%
\begin{pgfscope}%
\pgfpathrectangle{\pgfqpoint{0.539299in}{0.078740in}}{\pgfqpoint{7.842520in}{7.842520in}}%
\pgfusepath{clip}%
\pgfsetbuttcap%
\pgfsetroundjoin%
\definecolor{currentfill}{rgb}{0.131172,0.555899,0.552459}%
\pgfsetfillcolor{currentfill}%
\pgfsetlinewidth{0.000000pt}%
\definecolor{currentstroke}{rgb}{0.575563,0.844566,0.256415}%
\pgfsetstrokecolor{currentstroke}%
\pgfsetdash{}{0pt}%
\pgfpathmoveto{\pgfqpoint{4.515273in}{3.165074in}}%
\pgfpathlineto{\pgfqpoint{4.430807in}{2.929332in}}%
\pgfpathlineto{\pgfqpoint{4.577041in}{2.783012in}}%
\pgfpathclose%
\pgfusepath{fill}%
\end{pgfscope}%
\begin{pgfscope}%
\pgfpathrectangle{\pgfqpoint{0.539299in}{0.078740in}}{\pgfqpoint{7.842520in}{7.842520in}}%
\pgfusepath{clip}%
\pgfsetbuttcap%
\pgfsetroundjoin%
\definecolor{currentfill}{rgb}{0.206756,0.371758,0.553117}%
\pgfsetfillcolor{currentfill}%
\pgfsetlinewidth{0.000000pt}%
\definecolor{currentstroke}{rgb}{0.585678,0.846661,0.249897}%
\pgfsetstrokecolor{currentstroke}%
\pgfsetdash{}{0pt}%
\pgfpathmoveto{\pgfqpoint{5.245331in}{2.266200in}}%
\pgfpathlineto{\pgfqpoint{5.162383in}{2.135782in}}%
\pgfpathlineto{\pgfqpoint{5.308523in}{1.948082in}}%
\pgfpathclose%
\pgfusepath{fill}%
\end{pgfscope}%
\begin{pgfscope}%
\pgfpathrectangle{\pgfqpoint{0.539299in}{0.078740in}}{\pgfqpoint{7.842520in}{7.842520in}}%
\pgfusepath{clip}%
\pgfsetbuttcap%
\pgfsetroundjoin%
\definecolor{currentfill}{rgb}{0.153364,0.497000,0.557724}%
\pgfsetfillcolor{currentfill}%
\pgfsetlinewidth{0.000000pt}%
\definecolor{currentstroke}{rgb}{0.595839,0.848717,0.243329}%
\pgfsetstrokecolor{currentstroke}%
\pgfsetdash{}{0pt}%
\pgfpathmoveto{\pgfqpoint{4.492593in}{2.485503in}}%
\pgfpathlineto{\pgfqpoint{4.430807in}{2.929332in}}%
\pgfpathlineto{\pgfqpoint{4.346469in}{2.608126in}}%
\pgfpathclose%
\pgfusepath{fill}%
\end{pgfscope}%
\begin{pgfscope}%
\pgfpathrectangle{\pgfqpoint{0.539299in}{0.078740in}}{\pgfqpoint{7.842520in}{7.842520in}}%
\pgfusepath{clip}%
\pgfsetbuttcap%
\pgfsetroundjoin%
\definecolor{currentfill}{rgb}{0.192357,0.403199,0.555836}%
\pgfsetfillcolor{currentfill}%
\pgfsetlinewidth{0.000000pt}%
\definecolor{currentstroke}{rgb}{0.606045,0.850733,0.236712}%
\pgfsetstrokecolor{currentstroke}%
\pgfsetdash{}{0pt}%
\pgfpathmoveto{\pgfqpoint{5.016082in}{2.310084in}}%
\pgfpathlineto{\pgfqpoint{5.162383in}{2.135782in}}%
\pgfpathlineto{\pgfqpoint{5.245331in}{2.266200in}}%
\pgfpathclose%
\pgfusepath{fill}%
\end{pgfscope}%
\begin{pgfscope}%
\pgfpathrectangle{\pgfqpoint{0.539299in}{0.078740in}}{\pgfqpoint{7.842520in}{7.842520in}}%
\pgfusepath{clip}%
\pgfsetbuttcap%
\pgfsetroundjoin%
\definecolor{currentfill}{rgb}{0.147607,0.511733,0.557049}%
\pgfsetfillcolor{currentfill}%
\pgfsetlinewidth{0.000000pt}%
\definecolor{currentstroke}{rgb}{0.616293,0.852709,0.230052}%
\pgfsetstrokecolor{currentstroke}%
\pgfsetdash{}{0pt}%
\pgfpathmoveto{\pgfqpoint{4.807495in}{2.824405in}}%
\pgfpathlineto{\pgfqpoint{4.577041in}{2.783012in}}%
\pgfpathlineto{\pgfqpoint{4.723356in}{2.631725in}}%
\pgfpathclose%
\pgfusepath{fill}%
\end{pgfscope}%
\begin{pgfscope}%
\pgfpathrectangle{\pgfqpoint{0.539299in}{0.078740in}}{\pgfqpoint{7.842520in}{7.842520in}}%
\pgfusepath{clip}%
\pgfsetbuttcap%
\pgfsetroundjoin%
\definecolor{currentfill}{rgb}{0.160665,0.478540,0.558115}%
\pgfsetfillcolor{currentfill}%
\pgfsetlinewidth{0.000000pt}%
\definecolor{currentstroke}{rgb}{0.626579,0.854645,0.223353}%
\pgfsetstrokecolor{currentstroke}%
\pgfsetdash{}{0pt}%
\pgfpathmoveto{\pgfqpoint{4.723356in}{2.631725in}}%
\pgfpathlineto{\pgfqpoint{4.869718in}{2.474594in}}%
\pgfpathlineto{\pgfqpoint{4.953553in}{2.645992in}}%
\pgfpathclose%
\pgfusepath{fill}%
\end{pgfscope}%
\begin{pgfscope}%
\pgfpathrectangle{\pgfqpoint{0.539299in}{0.078740in}}{\pgfqpoint{7.842520in}{7.842520in}}%
\pgfusepath{clip}%
\pgfsetbuttcap%
\pgfsetroundjoin%
\definecolor{currentfill}{rgb}{0.266580,0.228262,0.514349}%
\pgfsetfillcolor{currentfill}%
\pgfsetlinewidth{0.000000pt}%
\definecolor{currentstroke}{rgb}{0.636902,0.856542,0.216620}%
\pgfsetstrokecolor{currentstroke}%
\pgfsetdash{}{0pt}%
\pgfpathmoveto{\pgfqpoint{4.616159in}{1.458069in}}%
\pgfpathlineto{\pgfqpoint{4.762457in}{1.378786in}}%
\pgfpathlineto{\pgfqpoint{4.847314in}{1.779621in}}%
\pgfpathclose%
\pgfusepath{fill}%
\end{pgfscope}%
\begin{pgfscope}%
\pgfpathrectangle{\pgfqpoint{0.539299in}{0.078740in}}{\pgfqpoint{7.842520in}{7.842520in}}%
\pgfusepath{clip}%
\pgfsetbuttcap%
\pgfsetroundjoin%
\definecolor{currentfill}{rgb}{0.266580,0.228262,0.514349}%
\pgfsetfillcolor{currentfill}%
\pgfsetlinewidth{0.000000pt}%
\definecolor{currentstroke}{rgb}{0.647257,0.858400,0.209861}%
\pgfsetstrokecolor{currentstroke}%
\pgfsetdash{}{0pt}%
\pgfpathmoveto{\pgfqpoint{5.517631in}{1.390496in}}%
\pgfpathlineto{\pgfqpoint{5.599643in}{1.509341in}}%
\pgfpathlineto{\pgfqpoint{5.454351in}{1.741746in}}%
\pgfpathclose%
\pgfusepath{fill}%
\end{pgfscope}%
\begin{pgfscope}%
\pgfpathrectangle{\pgfqpoint{0.539299in}{0.078740in}}{\pgfqpoint{7.842520in}{7.842520in}}%
\pgfusepath{clip}%
\pgfsetbuttcap%
\pgfsetroundjoin%
\definecolor{currentfill}{rgb}{0.169646,0.456262,0.558030}%
\pgfsetfillcolor{currentfill}%
\pgfsetlinewidth{0.000000pt}%
\definecolor{currentstroke}{rgb}{0.657642,0.860219,0.203082}%
\pgfsetstrokecolor{currentstroke}%
\pgfsetdash{}{0pt}%
\pgfpathmoveto{\pgfqpoint{5.016082in}{2.310084in}}%
\pgfpathlineto{\pgfqpoint{4.953553in}{2.645992in}}%
\pgfpathlineto{\pgfqpoint{4.869718in}{2.474594in}}%
\pgfpathclose%
\pgfusepath{fill}%
\end{pgfscope}%
\begin{pgfscope}%
\pgfpathrectangle{\pgfqpoint{0.539299in}{0.078740in}}{\pgfqpoint{7.842520in}{7.842520in}}%
\pgfusepath{clip}%
\pgfsetbuttcap%
\pgfsetroundjoin%
\definecolor{currentfill}{rgb}{0.147607,0.511733,0.557049}%
\pgfsetfillcolor{currentfill}%
\pgfsetlinewidth{0.000000pt}%
\definecolor{currentstroke}{rgb}{0.668054,0.861999,0.196293}%
\pgfsetstrokecolor{currentstroke}%
\pgfsetdash{}{0pt}%
\pgfpathmoveto{\pgfqpoint{4.492593in}{2.485503in}}%
\pgfpathlineto{\pgfqpoint{4.577041in}{2.783012in}}%
\pgfpathlineto{\pgfqpoint{4.430807in}{2.929332in}}%
\pgfpathclose%
\pgfusepath{fill}%
\end{pgfscope}%
\begin{pgfscope}%
\pgfpathrectangle{\pgfqpoint{0.539299in}{0.078740in}}{\pgfqpoint{7.842520in}{7.842520in}}%
\pgfusepath{clip}%
\pgfsetbuttcap%
\pgfsetroundjoin%
\definecolor{currentfill}{rgb}{0.278826,0.175490,0.483397}%
\pgfsetfillcolor{currentfill}%
\pgfsetlinewidth{0.000000pt}%
\definecolor{currentstroke}{rgb}{0.678489,0.863742,0.189503}%
\pgfsetstrokecolor{currentstroke}%
\pgfsetdash{}{0pt}%
\pgfpathmoveto{\pgfqpoint{5.517631in}{1.390496in}}%
\pgfpathlineto{\pgfqpoint{5.662874in}{1.144010in}}%
\pgfpathlineto{\pgfqpoint{5.599643in}{1.509341in}}%
\pgfpathclose%
\pgfusepath{fill}%
\end{pgfscope}%
\begin{pgfscope}%
\pgfpathrectangle{\pgfqpoint{0.539299in}{0.078740in}}{\pgfqpoint{7.842520in}{7.842520in}}%
\pgfusepath{clip}%
\pgfsetbuttcap%
\pgfsetroundjoin%
\definecolor{currentfill}{rgb}{0.188923,0.410910,0.556326}%
\pgfsetfillcolor{currentfill}%
\pgfsetlinewidth{0.000000pt}%
\definecolor{currentstroke}{rgb}{0.688944,0.865448,0.182725}%
\pgfsetstrokecolor{currentstroke}%
\pgfsetdash{}{0pt}%
\pgfpathmoveto{\pgfqpoint{4.554418in}{1.995003in}}%
\pgfpathlineto{\pgfqpoint{4.638902in}{2.358455in}}%
\pgfpathlineto{\pgfqpoint{4.492593in}{2.485503in}}%
\pgfpathclose%
\pgfusepath{fill}%
\end{pgfscope}%
\begin{pgfscope}%
\pgfpathrectangle{\pgfqpoint{0.539299in}{0.078740in}}{\pgfqpoint{7.842520in}{7.842520in}}%
\pgfusepath{clip}%
\pgfsetbuttcap%
\pgfsetroundjoin%
\definecolor{currentfill}{rgb}{0.250425,0.274290,0.533103}%
\pgfsetfillcolor{currentfill}%
\pgfsetlinewidth{0.000000pt}%
\definecolor{currentstroke}{rgb}{0.699415,0.867117,0.175971}%
\pgfsetstrokecolor{currentstroke}%
\pgfsetdash{}{0pt}%
\pgfpathmoveto{\pgfqpoint{4.847314in}{1.779621in}}%
\pgfpathlineto{\pgfqpoint{4.700745in}{1.889957in}}%
\pgfpathlineto{\pgfqpoint{4.616159in}{1.458069in}}%
\pgfpathclose%
\pgfusepath{fill}%
\end{pgfscope}%
\begin{pgfscope}%
\pgfpathrectangle{\pgfqpoint{0.539299in}{0.078740in}}{\pgfqpoint{7.842520in}{7.842520in}}%
\pgfusepath{clip}%
\pgfsetbuttcap%
\pgfsetroundjoin%
\definecolor{currentfill}{rgb}{0.156270,0.489624,0.557936}%
\pgfsetfillcolor{currentfill}%
\pgfsetlinewidth{0.000000pt}%
\definecolor{currentstroke}{rgb}{0.709898,0.868751,0.169257}%
\pgfsetstrokecolor{currentstroke}%
\pgfsetdash{}{0pt}%
\pgfpathmoveto{\pgfqpoint{4.723356in}{2.631725in}}%
\pgfpathlineto{\pgfqpoint{4.577041in}{2.783012in}}%
\pgfpathlineto{\pgfqpoint{4.492593in}{2.485503in}}%
\pgfpathclose%
\pgfusepath{fill}%
\end{pgfscope}%
\begin{pgfscope}%
\pgfpathrectangle{\pgfqpoint{0.539299in}{0.078740in}}{\pgfqpoint{7.842520in}{7.842520in}}%
\pgfusepath{clip}%
\pgfsetbuttcap%
\pgfsetroundjoin%
\definecolor{currentfill}{rgb}{0.168126,0.459988,0.558082}%
\pgfsetfillcolor{currentfill}%
\pgfsetlinewidth{0.000000pt}%
\definecolor{currentstroke}{rgb}{0.720391,0.870350,0.162603}%
\pgfsetstrokecolor{currentstroke}%
\pgfsetdash{}{0pt}%
\pgfpathmoveto{\pgfqpoint{4.492593in}{2.485503in}}%
\pgfpathlineto{\pgfqpoint{4.638902in}{2.358455in}}%
\pgfpathlineto{\pgfqpoint{4.723356in}{2.631725in}}%
\pgfpathclose%
\pgfusepath{fill}%
\end{pgfscope}%
\begin{pgfscope}%
\pgfpathrectangle{\pgfqpoint{0.539299in}{0.078740in}}{\pgfqpoint{7.842520in}{7.842520in}}%
\pgfusepath{clip}%
\pgfsetbuttcap%
\pgfsetroundjoin%
\definecolor{currentfill}{rgb}{0.212395,0.359683,0.551710}%
\pgfsetfillcolor{currentfill}%
\pgfsetlinewidth{0.000000pt}%
\definecolor{currentstroke}{rgb}{0.730889,0.871916,0.156029}%
\pgfsetstrokecolor{currentstroke}%
\pgfsetdash{}{0pt}%
\pgfpathmoveto{\pgfqpoint{4.554418in}{1.995003in}}%
\pgfpathlineto{\pgfqpoint{4.700745in}{1.889957in}}%
\pgfpathlineto{\pgfqpoint{4.785366in}{2.226099in}}%
\pgfpathclose%
\pgfusepath{fill}%
\end{pgfscope}%
\begin{pgfscope}%
\pgfpathrectangle{\pgfqpoint{0.539299in}{0.078740in}}{\pgfqpoint{7.842520in}{7.842520in}}%
\pgfusepath{clip}%
\pgfsetbuttcap%
\pgfsetroundjoin%
\definecolor{currentfill}{rgb}{0.195860,0.395433,0.555276}%
\pgfsetfillcolor{currentfill}%
\pgfsetlinewidth{0.000000pt}%
\definecolor{currentstroke}{rgb}{0.741388,0.873449,0.149561}%
\pgfsetstrokecolor{currentstroke}%
\pgfsetdash{}{0pt}%
\pgfpathmoveto{\pgfqpoint{4.785366in}{2.226099in}}%
\pgfpathlineto{\pgfqpoint{4.638902in}{2.358455in}}%
\pgfpathlineto{\pgfqpoint{4.554418in}{1.995003in}}%
\pgfpathclose%
\pgfusepath{fill}%
\end{pgfscope}%
\begin{pgfscope}%
\pgfpathrectangle{\pgfqpoint{0.539299in}{0.078740in}}{\pgfqpoint{7.842520in}{7.842520in}}%
\pgfusepath{clip}%
\pgfsetbuttcap%
\pgfsetroundjoin%
\definecolor{currentfill}{rgb}{0.168126,0.459988,0.558082}%
\pgfsetfillcolor{currentfill}%
\pgfsetlinewidth{0.000000pt}%
\definecolor{currentstroke}{rgb}{0.751884,0.874951,0.143228}%
\pgfsetstrokecolor{currentstroke}%
\pgfsetdash{}{0pt}%
\pgfpathmoveto{\pgfqpoint{4.723356in}{2.631725in}}%
\pgfpathlineto{\pgfqpoint{4.638902in}{2.358455in}}%
\pgfpathlineto{\pgfqpoint{4.869718in}{2.474594in}}%
\pgfpathclose%
\pgfusepath{fill}%
\end{pgfscope}%
\begin{pgfscope}%
\pgfpathrectangle{\pgfqpoint{0.539299in}{0.078740in}}{\pgfqpoint{7.842520in}{7.842520in}}%
\pgfusepath{clip}%
\pgfsetbuttcap%
\pgfsetroundjoin%
\definecolor{currentfill}{rgb}{0.243113,0.292092,0.538516}%
\pgfsetfillcolor{currentfill}%
\pgfsetlinewidth{0.000000pt}%
\definecolor{currentstroke}{rgb}{0.762373,0.876424,0.137064}%
\pgfsetstrokecolor{currentstroke}%
\pgfsetdash{}{0pt}%
\pgfpathmoveto{\pgfqpoint{5.371612in}{1.597343in}}%
\pgfpathlineto{\pgfqpoint{5.454351in}{1.741746in}}%
\pgfpathlineto{\pgfqpoint{5.308523in}{1.948082in}}%
\pgfpathclose%
\pgfusepath{fill}%
\end{pgfscope}%
\begin{pgfscope}%
\pgfpathrectangle{\pgfqpoint{0.539299in}{0.078740in}}{\pgfqpoint{7.842520in}{7.842520in}}%
\pgfusepath{clip}%
\pgfsetbuttcap%
\pgfsetroundjoin%
\definecolor{currentfill}{rgb}{0.269308,0.218818,0.509577}%
\pgfsetfillcolor{currentfill}%
\pgfsetlinewidth{0.000000pt}%
\definecolor{currentstroke}{rgb}{0.772852,0.877868,0.131109}%
\pgfsetstrokecolor{currentstroke}%
\pgfsetdash{}{0pt}%
\pgfpathmoveto{\pgfqpoint{4.847314in}{1.779621in}}%
\pgfpathlineto{\pgfqpoint{4.762457in}{1.378786in}}%
\pgfpathlineto{\pgfqpoint{4.909107in}{1.295428in}}%
\pgfpathclose%
\pgfusepath{fill}%
\end{pgfscope}%
\begin{pgfscope}%
\pgfpathrectangle{\pgfqpoint{0.539299in}{0.078740in}}{\pgfqpoint{7.842520in}{7.842520in}}%
\pgfusepath{clip}%
\pgfsetbuttcap%
\pgfsetroundjoin%
\definecolor{currentfill}{rgb}{0.180629,0.429975,0.557282}%
\pgfsetfillcolor{currentfill}%
\pgfsetlinewidth{0.000000pt}%
\definecolor{currentstroke}{rgb}{0.783315,0.879285,0.125405}%
\pgfsetstrokecolor{currentstroke}%
\pgfsetdash{}{0pt}%
\pgfpathmoveto{\pgfqpoint{5.016082in}{2.310084in}}%
\pgfpathlineto{\pgfqpoint{4.869718in}{2.474594in}}%
\pgfpathlineto{\pgfqpoint{4.785366in}{2.226099in}}%
\pgfpathclose%
\pgfusepath{fill}%
\end{pgfscope}%
\begin{pgfscope}%
\pgfpathrectangle{\pgfqpoint{0.539299in}{0.078740in}}{\pgfqpoint{7.842520in}{7.842520in}}%
\pgfusepath{clip}%
\pgfsetbuttcap%
\pgfsetroundjoin%
\definecolor{currentfill}{rgb}{0.179019,0.433756,0.557430}%
\pgfsetfillcolor{currentfill}%
\pgfsetlinewidth{0.000000pt}%
\definecolor{currentstroke}{rgb}{0.793760,0.880678,0.120005}%
\pgfsetstrokecolor{currentstroke}%
\pgfsetdash{}{0pt}%
\pgfpathmoveto{\pgfqpoint{4.638902in}{2.358455in}}%
\pgfpathlineto{\pgfqpoint{4.785366in}{2.226099in}}%
\pgfpathlineto{\pgfqpoint{4.869718in}{2.474594in}}%
\pgfpathclose%
\pgfusepath{fill}%
\end{pgfscope}%
\begin{pgfscope}%
\pgfpathrectangle{\pgfqpoint{0.539299in}{0.078740in}}{\pgfqpoint{7.842520in}{7.842520in}}%
\pgfusepath{clip}%
\pgfsetbuttcap%
\pgfsetroundjoin%
\definecolor{currentfill}{rgb}{0.214298,0.355619,0.551184}%
\pgfsetfillcolor{currentfill}%
\pgfsetlinewidth{0.000000pt}%
\definecolor{currentstroke}{rgb}{0.804182,0.882046,0.114965}%
\pgfsetstrokecolor{currentstroke}%
\pgfsetdash{}{0pt}%
\pgfpathmoveto{\pgfqpoint{5.308523in}{1.948082in}}%
\pgfpathlineto{\pgfqpoint{5.162383in}{2.135782in}}%
\pgfpathlineto{\pgfqpoint{5.078583in}{1.938515in}}%
\pgfpathclose%
\pgfusepath{fill}%
\end{pgfscope}%
\begin{pgfscope}%
\pgfpathrectangle{\pgfqpoint{0.539299in}{0.078740in}}{\pgfqpoint{7.842520in}{7.842520in}}%
\pgfusepath{clip}%
\pgfsetbuttcap%
\pgfsetroundjoin%
\definecolor{currentfill}{rgb}{0.260571,0.246922,0.522828}%
\pgfsetfillcolor{currentfill}%
\pgfsetlinewidth{0.000000pt}%
\definecolor{currentstroke}{rgb}{0.814576,0.883393,0.110347}%
\pgfsetstrokecolor{currentstroke}%
\pgfsetdash{}{0pt}%
\pgfpathmoveto{\pgfqpoint{5.454351in}{1.741746in}}%
\pgfpathlineto{\pgfqpoint{5.371612in}{1.597343in}}%
\pgfpathlineto{\pgfqpoint{5.517631in}{1.390496in}}%
\pgfpathclose%
\pgfusepath{fill}%
\end{pgfscope}%
\begin{pgfscope}%
\pgfpathrectangle{\pgfqpoint{0.539299in}{0.078740in}}{\pgfqpoint{7.842520in}{7.842520in}}%
\pgfusepath{clip}%
\pgfsetbuttcap%
\pgfsetroundjoin%
\definecolor{currentfill}{rgb}{0.201239,0.383670,0.554294}%
\pgfsetfillcolor{currentfill}%
\pgfsetlinewidth{0.000000pt}%
\definecolor{currentstroke}{rgb}{0.824940,0.884720,0.106217}%
\pgfsetstrokecolor{currentstroke}%
\pgfsetdash{}{0pt}%
\pgfpathmoveto{\pgfqpoint{5.078583in}{1.938515in}}%
\pgfpathlineto{\pgfqpoint{5.162383in}{2.135782in}}%
\pgfpathlineto{\pgfqpoint{5.016082in}{2.310084in}}%
\pgfpathclose%
\pgfusepath{fill}%
\end{pgfscope}%
\begin{pgfscope}%
\pgfpathrectangle{\pgfqpoint{0.539299in}{0.078740in}}{\pgfqpoint{7.842520in}{7.842520in}}%
\pgfusepath{clip}%
\pgfsetbuttcap%
\pgfsetroundjoin%
\definecolor{currentfill}{rgb}{0.206756,0.371758,0.553117}%
\pgfsetfillcolor{currentfill}%
\pgfsetlinewidth{0.000000pt}%
\definecolor{currentstroke}{rgb}{0.835270,0.886029,0.102646}%
\pgfsetstrokecolor{currentstroke}%
\pgfsetdash{}{0pt}%
\pgfpathmoveto{\pgfqpoint{4.931945in}{2.086920in}}%
\pgfpathlineto{\pgfqpoint{4.785366in}{2.226099in}}%
\pgfpathlineto{\pgfqpoint{4.700745in}{1.889957in}}%
\pgfpathclose%
\pgfusepath{fill}%
\end{pgfscope}%
\begin{pgfscope}%
\pgfpathrectangle{\pgfqpoint{0.539299in}{0.078740in}}{\pgfqpoint{7.842520in}{7.842520in}}%
\pgfusepath{clip}%
\pgfsetbuttcap%
\pgfsetroundjoin%
\definecolor{currentfill}{rgb}{0.192357,0.403199,0.555836}%
\pgfsetfillcolor{currentfill}%
\pgfsetlinewidth{0.000000pt}%
\definecolor{currentstroke}{rgb}{0.845561,0.887322,0.099702}%
\pgfsetstrokecolor{currentstroke}%
\pgfsetdash{}{0pt}%
\pgfpathmoveto{\pgfqpoint{4.785366in}{2.226099in}}%
\pgfpathlineto{\pgfqpoint{4.931945in}{2.086920in}}%
\pgfpathlineto{\pgfqpoint{5.016082in}{2.310084in}}%
\pgfpathclose%
\pgfusepath{fill}%
\end{pgfscope}%
\begin{pgfscope}%
\pgfpathrectangle{\pgfqpoint{0.539299in}{0.078740in}}{\pgfqpoint{7.842520in}{7.842520in}}%
\pgfusepath{clip}%
\pgfsetbuttcap%
\pgfsetroundjoin%
\definecolor{currentfill}{rgb}{0.223925,0.334994,0.548053}%
\pgfsetfillcolor{currentfill}%
\pgfsetlinewidth{0.000000pt}%
\definecolor{currentstroke}{rgb}{0.855810,0.888601,0.097452}%
\pgfsetstrokecolor{currentstroke}%
\pgfsetdash{}{0pt}%
\pgfpathmoveto{\pgfqpoint{4.931945in}{2.086920in}}%
\pgfpathlineto{\pgfqpoint{4.700745in}{1.889957in}}%
\pgfpathlineto{\pgfqpoint{4.847314in}{1.779621in}}%
\pgfpathclose%
\pgfusepath{fill}%
\end{pgfscope}%
\begin{pgfscope}%
\pgfpathrectangle{\pgfqpoint{0.539299in}{0.078740in}}{\pgfqpoint{7.842520in}{7.842520in}}%
\pgfusepath{clip}%
\pgfsetbuttcap%
\pgfsetroundjoin%
\definecolor{currentfill}{rgb}{0.201239,0.383670,0.554294}%
\pgfsetfillcolor{currentfill}%
\pgfsetlinewidth{0.000000pt}%
\definecolor{currentstroke}{rgb}{0.866013,0.889868,0.095953}%
\pgfsetstrokecolor{currentstroke}%
\pgfsetdash{}{0pt}%
\pgfpathmoveto{\pgfqpoint{5.016082in}{2.310084in}}%
\pgfpathlineto{\pgfqpoint{4.931945in}{2.086920in}}%
\pgfpathlineto{\pgfqpoint{5.078583in}{1.938515in}}%
\pgfpathclose%
\pgfusepath{fill}%
\end{pgfscope}%
\begin{pgfscope}%
\pgfpathrectangle{\pgfqpoint{0.539299in}{0.078740in}}{\pgfqpoint{7.842520in}{7.842520in}}%
\pgfusepath{clip}%
\pgfsetbuttcap%
\pgfsetroundjoin%
\definecolor{currentfill}{rgb}{0.239346,0.300855,0.540844}%
\pgfsetfillcolor{currentfill}%
\pgfsetlinewidth{0.000000pt}%
\definecolor{currentstroke}{rgb}{0.876168,0.891125,0.095250}%
\pgfsetstrokecolor{currentstroke}%
\pgfsetdash{}{0pt}%
\pgfpathmoveto{\pgfqpoint{5.225186in}{1.777193in}}%
\pgfpathlineto{\pgfqpoint{5.371612in}{1.597343in}}%
\pgfpathlineto{\pgfqpoint{5.308523in}{1.948082in}}%
\pgfpathclose%
\pgfusepath{fill}%
\end{pgfscope}%
\begin{pgfscope}%
\pgfpathrectangle{\pgfqpoint{0.539299in}{0.078740in}}{\pgfqpoint{7.842520in}{7.842520in}}%
\pgfusepath{clip}%
\pgfsetbuttcap%
\pgfsetroundjoin%
\definecolor{currentfill}{rgb}{0.225863,0.330805,0.547314}%
\pgfsetfillcolor{currentfill}%
\pgfsetlinewidth{0.000000pt}%
\definecolor{currentstroke}{rgb}{0.886271,0.892374,0.095374}%
\pgfsetstrokecolor{currentstroke}%
\pgfsetdash{}{0pt}%
\pgfpathmoveto{\pgfqpoint{5.308523in}{1.948082in}}%
\pgfpathlineto{\pgfqpoint{5.078583in}{1.938515in}}%
\pgfpathlineto{\pgfqpoint{5.225186in}{1.777193in}}%
\pgfpathclose%
\pgfusepath{fill}%
\end{pgfscope}%
\begin{pgfscope}%
\pgfpathrectangle{\pgfqpoint{0.539299in}{0.078740in}}{\pgfqpoint{7.842520in}{7.842520in}}%
\pgfusepath{clip}%
\pgfsetbuttcap%
\pgfsetroundjoin%
\definecolor{currentfill}{rgb}{0.258965,0.251537,0.524736}%
\pgfsetfillcolor{currentfill}%
\pgfsetlinewidth{0.000000pt}%
\definecolor{currentstroke}{rgb}{0.896320,0.893616,0.096335}%
\pgfsetstrokecolor{currentstroke}%
\pgfsetdash{}{0pt}%
\pgfpathmoveto{\pgfqpoint{4.909107in}{1.295428in}}%
\pgfpathlineto{\pgfqpoint{4.994083in}{1.662022in}}%
\pgfpathlineto{\pgfqpoint{4.847314in}{1.779621in}}%
\pgfpathclose%
\pgfusepath{fill}%
\end{pgfscope}%
\begin{pgfscope}%
\pgfpathrectangle{\pgfqpoint{0.539299in}{0.078740in}}{\pgfqpoint{7.842520in}{7.842520in}}%
\pgfusepath{clip}%
\pgfsetbuttcap%
\pgfsetroundjoin%
\definecolor{currentfill}{rgb}{0.280868,0.160771,0.472899}%
\pgfsetfillcolor{currentfill}%
\pgfsetlinewidth{0.000000pt}%
\definecolor{currentstroke}{rgb}{0.906311,0.894855,0.098125}%
\pgfsetstrokecolor{currentstroke}%
\pgfsetdash{}{0pt}%
\pgfpathmoveto{\pgfqpoint{5.434633in}{1.223811in}}%
\pgfpathlineto{\pgfqpoint{5.662874in}{1.144010in}}%
\pgfpathlineto{\pgfqpoint{5.517631in}{1.390496in}}%
\pgfpathclose%
\pgfusepath{fill}%
\end{pgfscope}%
\begin{pgfscope}%
\pgfpathrectangle{\pgfqpoint{0.539299in}{0.078740in}}{\pgfqpoint{7.842520in}{7.842520in}}%
\pgfusepath{clip}%
\pgfsetbuttcap%
\pgfsetroundjoin%
\definecolor{currentfill}{rgb}{0.231674,0.318106,0.544834}%
\pgfsetfillcolor{currentfill}%
\pgfsetlinewidth{0.000000pt}%
\definecolor{currentstroke}{rgb}{0.916242,0.896091,0.100717}%
\pgfsetstrokecolor{currentstroke}%
\pgfsetdash{}{0pt}%
\pgfpathmoveto{\pgfqpoint{4.847314in}{1.779621in}}%
\pgfpathlineto{\pgfqpoint{4.994083in}{1.662022in}}%
\pgfpathlineto{\pgfqpoint{4.931945in}{2.086920in}}%
\pgfpathclose%
\pgfusepath{fill}%
\end{pgfscope}%
\begin{pgfscope}%
\pgfpathrectangle{\pgfqpoint{0.539299in}{0.078740in}}{\pgfqpoint{7.842520in}{7.842520in}}%
\pgfusepath{clip}%
\pgfsetbuttcap%
\pgfsetroundjoin%
\definecolor{currentfill}{rgb}{0.223925,0.334994,0.548053}%
\pgfsetfillcolor{currentfill}%
\pgfsetlinewidth{0.000000pt}%
\definecolor{currentstroke}{rgb}{0.926106,0.897330,0.104071}%
\pgfsetstrokecolor{currentstroke}%
\pgfsetdash{}{0pt}%
\pgfpathmoveto{\pgfqpoint{4.994083in}{1.662022in}}%
\pgfpathlineto{\pgfqpoint{5.078583in}{1.938515in}}%
\pgfpathlineto{\pgfqpoint{4.931945in}{2.086920in}}%
\pgfpathclose%
\pgfusepath{fill}%
\end{pgfscope}%
\begin{pgfscope}%
\pgfpathrectangle{\pgfqpoint{0.539299in}{0.078740in}}{\pgfqpoint{7.842520in}{7.842520in}}%
\pgfusepath{clip}%
\pgfsetbuttcap%
\pgfsetroundjoin%
\definecolor{currentfill}{rgb}{0.265145,0.232956,0.516599}%
\pgfsetfillcolor{currentfill}%
\pgfsetlinewidth{0.000000pt}%
\definecolor{currentstroke}{rgb}{0.935904,0.898570,0.108131}%
\pgfsetstrokecolor{currentstroke}%
\pgfsetdash{}{0pt}%
\pgfpathmoveto{\pgfqpoint{4.909107in}{1.295428in}}%
\pgfpathlineto{\pgfqpoint{5.140982in}{1.534035in}}%
\pgfpathlineto{\pgfqpoint{4.994083in}{1.662022in}}%
\pgfpathclose%
\pgfusepath{fill}%
\end{pgfscope}%
\begin{pgfscope}%
\pgfpathrectangle{\pgfqpoint{0.539299in}{0.078740in}}{\pgfqpoint{7.842520in}{7.842520in}}%
\pgfusepath{clip}%
\pgfsetbuttcap%
\pgfsetroundjoin%
\definecolor{currentfill}{rgb}{0.283187,0.125848,0.444960}%
\pgfsetfillcolor{currentfill}%
\pgfsetlinewidth{0.000000pt}%
\definecolor{currentstroke}{rgb}{0.945636,0.899815,0.112838}%
\pgfsetstrokecolor{currentstroke}%
\pgfsetdash{}{0pt}%
\pgfpathmoveto{\pgfqpoint{5.434633in}{1.223811in}}%
\pgfpathlineto{\pgfqpoint{5.580846in}{1.019988in}}%
\pgfpathlineto{\pgfqpoint{5.662874in}{1.144010in}}%
\pgfpathclose%
\pgfusepath{fill}%
\end{pgfscope}%
\begin{pgfscope}%
\pgfpathrectangle{\pgfqpoint{0.539299in}{0.078740in}}{\pgfqpoint{7.842520in}{7.842520in}}%
\pgfusepath{clip}%
\pgfsetbuttcap%
\pgfsetroundjoin%
\definecolor{currentfill}{rgb}{0.276194,0.190074,0.493001}%
\pgfsetfillcolor{currentfill}%
\pgfsetlinewidth{0.000000pt}%
\definecolor{currentstroke}{rgb}{0.955300,0.901065,0.118128}%
\pgfsetstrokecolor{currentstroke}%
\pgfsetdash{}{0pt}%
\pgfpathmoveto{\pgfqpoint{5.056074in}{1.206164in}}%
\pgfpathlineto{\pgfqpoint{5.140982in}{1.534035in}}%
\pgfpathlineto{\pgfqpoint{4.909107in}{1.295428in}}%
\pgfpathclose%
\pgfusepath{fill}%
\end{pgfscope}%
\begin{pgfscope}%
\pgfpathrectangle{\pgfqpoint{0.539299in}{0.078740in}}{\pgfqpoint{7.842520in}{7.842520in}}%
\pgfusepath{clip}%
\pgfsetbuttcap%
\pgfsetroundjoin%
\definecolor{currentfill}{rgb}{0.267968,0.223549,0.512008}%
\pgfsetfillcolor{currentfill}%
\pgfsetlinewidth{0.000000pt}%
\definecolor{currentstroke}{rgb}{0.964894,0.902323,0.123941}%
\pgfsetstrokecolor{currentstroke}%
\pgfsetdash{}{0pt}%
\pgfpathmoveto{\pgfqpoint{5.517631in}{1.390496in}}%
\pgfpathlineto{\pgfqpoint{5.371612in}{1.597343in}}%
\pgfpathlineto{\pgfqpoint{5.287897in}{1.390668in}}%
\pgfpathclose%
\pgfusepath{fill}%
\end{pgfscope}%
\begin{pgfscope}%
\pgfpathrectangle{\pgfqpoint{0.539299in}{0.078740in}}{\pgfqpoint{7.842520in}{7.842520in}}%
\pgfusepath{clip}%
\pgfsetbuttcap%
\pgfsetroundjoin%
\definecolor{currentfill}{rgb}{0.239346,0.300855,0.540844}%
\pgfsetfillcolor{currentfill}%
\pgfsetlinewidth{0.000000pt}%
\definecolor{currentstroke}{rgb}{0.974417,0.903590,0.130215}%
\pgfsetstrokecolor{currentstroke}%
\pgfsetdash{}{0pt}%
\pgfpathmoveto{\pgfqpoint{5.078583in}{1.938515in}}%
\pgfpathlineto{\pgfqpoint{5.140982in}{1.534035in}}%
\pgfpathlineto{\pgfqpoint{5.225186in}{1.777193in}}%
\pgfpathclose%
\pgfusepath{fill}%
\end{pgfscope}%
\begin{pgfscope}%
\pgfpathrectangle{\pgfqpoint{0.539299in}{0.078740in}}{\pgfqpoint{7.842520in}{7.842520in}}%
\pgfusepath{clip}%
\pgfsetbuttcap%
\pgfsetroundjoin%
\definecolor{currentfill}{rgb}{0.243113,0.292092,0.538516}%
\pgfsetfillcolor{currentfill}%
\pgfsetlinewidth{0.000000pt}%
\definecolor{currentstroke}{rgb}{0.983868,0.904867,0.136897}%
\pgfsetstrokecolor{currentstroke}%
\pgfsetdash{}{0pt}%
\pgfpathmoveto{\pgfqpoint{4.994083in}{1.662022in}}%
\pgfpathlineto{\pgfqpoint{5.140982in}{1.534035in}}%
\pgfpathlineto{\pgfqpoint{5.078583in}{1.938515in}}%
\pgfpathclose%
\pgfusepath{fill}%
\end{pgfscope}%
\begin{pgfscope}%
\pgfpathrectangle{\pgfqpoint{0.539299in}{0.078740in}}{\pgfqpoint{7.842520in}{7.842520in}}%
\pgfusepath{clip}%
\pgfsetbuttcap%
\pgfsetroundjoin%
\definecolor{currentfill}{rgb}{0.255645,0.260703,0.528312}%
\pgfsetfillcolor{currentfill}%
\pgfsetlinewidth{0.000000pt}%
\definecolor{currentstroke}{rgb}{0.993248,0.906157,0.143936}%
\pgfsetstrokecolor{currentstroke}%
\pgfsetdash{}{0pt}%
\pgfpathmoveto{\pgfqpoint{5.287897in}{1.390668in}}%
\pgfpathlineto{\pgfqpoint{5.371612in}{1.597343in}}%
\pgfpathlineto{\pgfqpoint{5.225186in}{1.777193in}}%
\pgfpathclose%
\pgfusepath{fill}%
\end{pgfscope}%
\begin{pgfscope}%
\pgfpathrectangle{\pgfqpoint{0.539299in}{0.078740in}}{\pgfqpoint{7.842520in}{7.842520in}}%
\pgfusepath{clip}%
\pgfsetbuttcap%
\pgfsetroundjoin%
\definecolor{currentfill}{rgb}{0.276194,0.190074,0.493001}%
\pgfsetfillcolor{currentfill}%
\pgfsetlinewidth{0.000000pt}%
\definecolor{currentstroke}{rgb}{0.267004,0.004874,0.329415}%
\pgfsetstrokecolor{currentstroke}%
\pgfsetdash{}{0pt}%
\pgfpathmoveto{\pgfqpoint{5.287897in}{1.390668in}}%
\pgfpathlineto{\pgfqpoint{5.434633in}{1.223811in}}%
\pgfpathlineto{\pgfqpoint{5.517631in}{1.390496in}}%
\pgfpathclose%
\pgfusepath{fill}%
\end{pgfscope}%
\begin{pgfscope}%
\pgfpathrectangle{\pgfqpoint{0.539299in}{0.078740in}}{\pgfqpoint{7.842520in}{7.842520in}}%
\pgfusepath{clip}%
\pgfsetbuttcap%
\pgfsetroundjoin%
\definecolor{currentfill}{rgb}{0.257322,0.256130,0.526563}%
\pgfsetfillcolor{currentfill}%
\pgfsetlinewidth{0.000000pt}%
\definecolor{currentstroke}{rgb}{0.268510,0.009605,0.335427}%
\pgfsetstrokecolor{currentstroke}%
\pgfsetdash{}{0pt}%
\pgfpathmoveto{\pgfqpoint{5.225186in}{1.777193in}}%
\pgfpathlineto{\pgfqpoint{5.140982in}{1.534035in}}%
\pgfpathlineto{\pgfqpoint{5.287897in}{1.390668in}}%
\pgfpathclose%
\pgfusepath{fill}%
\end{pgfscope}%
\begin{pgfscope}%
\pgfpathrectangle{\pgfqpoint{0.539299in}{0.078740in}}{\pgfqpoint{7.842520in}{7.842520in}}%
\pgfusepath{clip}%
\pgfsetbuttcap%
\pgfsetroundjoin%
\definecolor{currentfill}{rgb}{0.273006,0.204520,0.501721}%
\pgfsetfillcolor{currentfill}%
\pgfsetlinewidth{0.000000pt}%
\definecolor{currentstroke}{rgb}{0.269944,0.014625,0.341379}%
\pgfsetstrokecolor{currentstroke}%
\pgfsetdash{}{0pt}%
\pgfpathmoveto{\pgfqpoint{5.287897in}{1.390668in}}%
\pgfpathlineto{\pgfqpoint{5.140982in}{1.534035in}}%
\pgfpathlineto{\pgfqpoint{5.056074in}{1.206164in}}%
\pgfpathclose%
\pgfusepath{fill}%
\end{pgfscope}%
\begin{pgfscope}%
\pgfpathrectangle{\pgfqpoint{0.539299in}{0.078740in}}{\pgfqpoint{7.842520in}{7.842520in}}%
\pgfusepath{clip}%
\pgfsetbuttcap%
\pgfsetroundjoin%
\definecolor{currentfill}{rgb}{0.280255,0.165693,0.476498}%
\pgfsetfillcolor{currentfill}%
\pgfsetlinewidth{0.000000pt}%
\definecolor{currentstroke}{rgb}{0.271305,0.019942,0.347269}%
\pgfsetstrokecolor{currentstroke}%
\pgfsetdash{}{0pt}%
\pgfpathmoveto{\pgfqpoint{5.056074in}{1.206164in}}%
\pgfpathlineto{\pgfqpoint{5.203294in}{1.107929in}}%
\pgfpathlineto{\pgfqpoint{5.287897in}{1.390668in}}%
\pgfpathclose%
\pgfusepath{fill}%
\end{pgfscope}%
\begin{pgfscope}%
\pgfpathrectangle{\pgfqpoint{0.539299in}{0.078740in}}{\pgfqpoint{7.842520in}{7.842520in}}%
\pgfusepath{clip}%
\pgfsetbuttcap%
\pgfsetroundjoin%
\definecolor{currentfill}{rgb}{0.280868,0.160771,0.472899}%
\pgfsetfillcolor{currentfill}%
\pgfsetlinewidth{0.000000pt}%
\definecolor{currentstroke}{rgb}{0.272594,0.025563,0.353093}%
\pgfsetstrokecolor{currentstroke}%
\pgfsetdash{}{0pt}%
\pgfpathmoveto{\pgfqpoint{5.350654in}{0.995313in}}%
\pgfpathlineto{\pgfqpoint{5.434633in}{1.223811in}}%
\pgfpathlineto{\pgfqpoint{5.287897in}{1.390668in}}%
\pgfpathclose%
\pgfusepath{fill}%
\end{pgfscope}%
\begin{pgfscope}%
\pgfpathrectangle{\pgfqpoint{0.539299in}{0.078740in}}{\pgfqpoint{7.842520in}{7.842520in}}%
\pgfusepath{clip}%
\pgfsetbuttcap%
\pgfsetroundjoin%
\definecolor{currentfill}{rgb}{0.283091,0.110553,0.431554}%
\pgfsetfillcolor{currentfill}%
\pgfsetlinewidth{0.000000pt}%
\definecolor{currentstroke}{rgb}{0.273809,0.031497,0.358853}%
\pgfsetstrokecolor{currentstroke}%
\pgfsetdash{}{0pt}%
\pgfpathmoveto{\pgfqpoint{5.497926in}{0.857909in}}%
\pgfpathlineto{\pgfqpoint{5.580846in}{1.019988in}}%
\pgfpathlineto{\pgfqpoint{5.434633in}{1.223811in}}%
\pgfpathclose%
\pgfusepath{fill}%
\end{pgfscope}%
\begin{pgfscope}%
\pgfpathrectangle{\pgfqpoint{0.539299in}{0.078740in}}{\pgfqpoint{7.842520in}{7.842520in}}%
\pgfusepath{clip}%
\pgfsetbuttcap%
\pgfsetroundjoin%
\definecolor{currentfill}{rgb}{0.281887,0.150881,0.465405}%
\pgfsetfillcolor{currentfill}%
\pgfsetlinewidth{0.000000pt}%
\definecolor{currentstroke}{rgb}{0.274952,0.037752,0.364543}%
\pgfsetstrokecolor{currentstroke}%
\pgfsetdash{}{0pt}%
\pgfpathmoveto{\pgfqpoint{5.203294in}{1.107929in}}%
\pgfpathlineto{\pgfqpoint{5.350654in}{0.995313in}}%
\pgfpathlineto{\pgfqpoint{5.287897in}{1.390668in}}%
\pgfpathclose%
\pgfusepath{fill}%
\end{pgfscope}%
\begin{pgfscope}%
\pgfpathrectangle{\pgfqpoint{0.539299in}{0.078740in}}{\pgfqpoint{7.842520in}{7.842520in}}%
\pgfusepath{clip}%
\pgfsetbuttcap%
\pgfsetroundjoin%
\definecolor{currentfill}{rgb}{0.283091,0.110553,0.431554}%
\pgfsetfillcolor{currentfill}%
\pgfsetlinewidth{0.000000pt}%
\definecolor{currentstroke}{rgb}{0.276022,0.044167,0.370164}%
\pgfsetstrokecolor{currentstroke}%
\pgfsetdash{}{0pt}%
\pgfpathmoveto{\pgfqpoint{5.434633in}{1.223811in}}%
\pgfpathlineto{\pgfqpoint{5.350654in}{0.995313in}}%
\pgfpathlineto{\pgfqpoint{5.497926in}{0.857909in}}%
\pgfpathclose%
\pgfusepath{fill}%
\end{pgfscope}%
\begin{pgfscope}%
\pgfsetbuttcap%
\pgfsetmiterjoin%
\definecolor{currentfill}{rgb}{1.000000,1.000000,1.000000}%
\pgfsetfillcolor{currentfill}%
\pgfsetlinewidth{0.000000pt}%
\definecolor{currentstroke}{rgb}{0.000000,0.000000,0.000000}%
\pgfsetstrokecolor{currentstroke}%
\pgfsetstrokeopacity{0.000000}%
\pgfsetdash{}{0pt}%
\pgfpathmoveto{\pgfqpoint{11.180860in}{0.157480in}}%
\pgfpathlineto{\pgfqpoint{11.495820in}{0.157480in}}%
\pgfpathlineto{\pgfqpoint{11.495820in}{7.842520in}}%
\pgfpathlineto{\pgfqpoint{11.180860in}{7.842520in}}%
\pgfpathclose%
\pgfusepath{fill}%
\end{pgfscope}%
\begin{pgfscope}%
\pgfpathrectangle{\pgfqpoint{11.180860in}{0.157480in}}{\pgfqpoint{0.314961in}{7.685039in}}%
\pgfusepath{clip}%
\pgfsetbuttcap%
\pgfsetmiterjoin%
\definecolor{currentfill}{rgb}{1.000000,1.000000,1.000000}%
\pgfsetfillcolor{currentfill}%
\pgfsetlinewidth{0.010037pt}%
\definecolor{currentstroke}{rgb}{1.000000,1.000000,1.000000}%
\pgfsetstrokecolor{currentstroke}%
\pgfsetdash{}{0pt}%
\pgfpathmoveto{\pgfqpoint{11.180860in}{0.157480in}}%
\pgfpathlineto{\pgfqpoint{11.180860in}{0.187500in}}%
\pgfpathlineto{\pgfqpoint{11.180860in}{7.812500in}}%
\pgfpathlineto{\pgfqpoint{11.180860in}{7.842520in}}%
\pgfpathlineto{\pgfqpoint{11.495820in}{7.842520in}}%
\pgfpathlineto{\pgfqpoint{11.495820in}{7.812500in}}%
\pgfpathlineto{\pgfqpoint{11.495820in}{0.187500in}}%
\pgfpathlineto{\pgfqpoint{11.495820in}{0.157480in}}%
\pgfpathlineto{\pgfqpoint{11.495820in}{0.157480in}}%
\pgfpathclose%
\pgfusepath{stroke,fill}%
\end{pgfscope}%
\begin{pgfscope}%
\pgfsys@transformshift{11.180000in}{0.160000in}%
\pgftext[left,bottom]{\includegraphics[interpolate=true,width=0.320000in,height=7.680000in]{unknown1-img0.png}}%
\end{pgfscope}%
\begin{pgfscope}%
\pgfsetbuttcap%
\pgfsetroundjoin%
\definecolor{currentfill}{rgb}{0.000000,0.000000,0.000000}%
\pgfsetfillcolor{currentfill}%
\pgfsetlinewidth{0.501875pt}%
\definecolor{currentstroke}{rgb}{0.000000,0.000000,0.000000}%
\pgfsetstrokecolor{currentstroke}%
\pgfsetdash{}{0pt}%
\pgfsys@defobject{currentmarker}{\pgfqpoint{-0.034722in}{0.000000in}}{\pgfqpoint{-0.000000in}{0.000000in}}{%
\pgfpathmoveto{\pgfqpoint{-0.000000in}{0.000000in}}%
\pgfpathlineto{\pgfqpoint{-0.034722in}{0.000000in}}%
\pgfusepath{stroke,fill}%
}%
\begin{pgfscope}%
\pgfsys@transformshift{11.495820in}{1.494658in}%
\pgfsys@useobject{currentmarker}{}%
\end{pgfscope}%
\end{pgfscope}%
\begin{pgfscope}%
\definecolor{textcolor}{rgb}{0.980392,0.811765,0.352941}%
\pgfsetstrokecolor{textcolor}%
\pgfsetfillcolor{textcolor}%
\pgftext[x=11.044431in, y=1.452449in, right, base]{\color{textcolor}\sffamily\fontsize{18.000000}{9.600000}\selectfont $\displaystyle 0.005$}%
\end{pgfscope}%
\begin{pgfscope}%
\pgfsetbuttcap%
\pgfsetroundjoin%
\definecolor{currentfill}{rgb}{0.000000,0.000000,0.000000}%
\pgfsetfillcolor{currentfill}%
\pgfsetlinewidth{0.501875pt}%
\definecolor{currentstroke}{rgb}{0.000000,0.000000,0.000000}%
\pgfsetstrokecolor{currentstroke}%
\pgfsetdash{}{0pt}%
\pgfsys@defobject{currentmarker}{\pgfqpoint{-0.034722in}{0.000000in}}{\pgfqpoint{-0.000000in}{0.000000in}}{%
\pgfpathmoveto{\pgfqpoint{-0.000000in}{0.000000in}}%
\pgfpathlineto{\pgfqpoint{-0.034722in}{0.000000in}}%
\pgfusepath{stroke,fill}%
}%
\begin{pgfscope}%
\pgfsys@transformshift{11.495820in}{2.832718in}%
\pgfsys@useobject{currentmarker}{}%
\end{pgfscope}%
\end{pgfscope}%
\begin{pgfscope}%
\definecolor{textcolor}{rgb}{0.980392,0.811765,0.352941}%
\pgfsetstrokecolor{textcolor}%
\pgfsetfillcolor{textcolor}%
\pgftext[x=11.044431in, y=2.790509in, right, base]{\color{textcolor}\sffamily\fontsize{18.000000}{9.600000}\selectfont $\displaystyle 0.010$}%
\end{pgfscope}%
\begin{pgfscope}%
\pgfsetbuttcap%
\pgfsetroundjoin%
\definecolor{currentfill}{rgb}{0.000000,0.000000,0.000000}%
\pgfsetfillcolor{currentfill}%
\pgfsetlinewidth{0.501875pt}%
\definecolor{currentstroke}{rgb}{0.000000,0.000000,0.000000}%
\pgfsetstrokecolor{currentstroke}%
\pgfsetdash{}{0pt}%
\pgfsys@defobject{currentmarker}{\pgfqpoint{-0.034722in}{0.000000in}}{\pgfqpoint{-0.000000in}{0.000000in}}{%
\pgfpathmoveto{\pgfqpoint{-0.000000in}{0.000000in}}%
\pgfpathlineto{\pgfqpoint{-0.034722in}{0.000000in}}%
\pgfusepath{stroke,fill}%
}%
\begin{pgfscope}%
\pgfsys@transformshift{11.495820in}{4.170779in}%
\pgfsys@useobject{currentmarker}{}%
\end{pgfscope}%
\end{pgfscope}%
\begin{pgfscope}%
\definecolor{textcolor}{rgb}{0.980392,0.811765,0.352941}%
\pgfsetstrokecolor{textcolor}%
\pgfsetfillcolor{textcolor}%
\pgftext[x=11.044431in, y=4.128569in, right, base]{\color{textcolor}\sffamily\fontsize{18.000000}{9.600000}\selectfont $\displaystyle 0.015$}%
\end{pgfscope}%
\begin{pgfscope}%
\pgfsetbuttcap%
\pgfsetroundjoin%
\definecolor{currentfill}{rgb}{0.000000,0.000000,0.000000}%
\pgfsetfillcolor{currentfill}%
\pgfsetlinewidth{0.501875pt}%
\definecolor{currentstroke}{rgb}{0.000000,0.000000,0.000000}%
\pgfsetstrokecolor{currentstroke}%
\pgfsetdash{}{0pt}%
\pgfsys@defobject{currentmarker}{\pgfqpoint{-0.034722in}{0.000000in}}{\pgfqpoint{-0.000000in}{0.000000in}}{%
\pgfpathmoveto{\pgfqpoint{-0.000000in}{0.000000in}}%
\pgfpathlineto{\pgfqpoint{-0.034722in}{0.000000in}}%
\pgfusepath{stroke,fill}%
}%
\begin{pgfscope}%
\pgfsys@transformshift{11.495820in}{5.508839in}%
\pgfsys@useobject{currentmarker}{}%
\end{pgfscope}%
\end{pgfscope}%
\begin{pgfscope}%
\definecolor{textcolor}{rgb}{0.980392,0.811765,0.352941}%
\pgfsetstrokecolor{textcolor}%
\pgfsetfillcolor{textcolor}%
\pgftext[x=11.044431in, y=5.466629in, right, base]{\color{textcolor}\sffamily\fontsize{18.000000}{9.600000}\selectfont $\displaystyle 0.020$}%
\end{pgfscope}%
\begin{pgfscope}%
\pgfsetbuttcap%
\pgfsetroundjoin%
\definecolor{currentfill}{rgb}{0.000000,0.000000,0.000000}%
\pgfsetfillcolor{currentfill}%
\pgfsetlinewidth{0.501875pt}%
\definecolor{currentstroke}{rgb}{0.000000,0.000000,0.000000}%
\pgfsetstrokecolor{currentstroke}%
\pgfsetdash{}{0pt}%
\pgfsys@defobject{currentmarker}{\pgfqpoint{-0.034722in}{0.000000in}}{\pgfqpoint{-0.000000in}{0.000000in}}{%
\pgfpathmoveto{\pgfqpoint{-0.000000in}{0.000000in}}%
\pgfpathlineto{\pgfqpoint{-0.034722in}{0.000000in}}%
\pgfusepath{stroke,fill}%
}%
\begin{pgfscope}%
\pgfsys@transformshift{11.495820in}{6.846899in}%
\pgfsys@useobject{currentmarker}{}%
\end{pgfscope}%
\end{pgfscope}%
\begin{pgfscope}%
\definecolor{textcolor}{rgb}{0.980392,0.811765,0.352941}%
\pgfsetstrokecolor{textcolor}%
\pgfsetfillcolor{textcolor}%
\pgftext[x=11.044431in, y=6.804689in, right, base]{\color{textcolor}\sffamily\fontsize{18.000000}{9.600000}\selectfont $\displaystyle 0.025$}%
\end{pgfscope}%
\begin{pgfscope}%
\pgfsetrectcap%
\pgfsetmiterjoin%
\pgfsetlinewidth{1.003750pt}%
\definecolor{currentstroke}{rgb}{0.000000,0.000000,0.000000}%
\pgfsetstrokecolor{currentstroke}%
\pgfsetdash{}{0pt}%
\pgfpathmoveto{\pgfqpoint{11.180860in}{0.157480in}}%
\pgfpathlineto{\pgfqpoint{11.180860in}{0.187500in}}%
\pgfpathlineto{\pgfqpoint{11.180860in}{7.812500in}}%
\pgfpathlineto{\pgfqpoint{11.180860in}{7.842520in}}%
\pgfpathlineto{\pgfqpoint{11.495820in}{7.842520in}}%
\pgfpathlineto{\pgfqpoint{11.495820in}{7.812500in}}%
\pgfpathlineto{\pgfqpoint{11.495820in}{0.187500in}}%
\pgfpathlineto{\pgfqpoint{11.495820in}{0.157480in}}%
\pgfpathclose%
\pgfusepath{stroke}%
\end{pgfscope}%
\end{pgfpicture}%
\makeatother%
\endgroup%
}
% 	\caption{$h = 2^{-5}$ 时差分逼近解}\label{fig:explicit1}
% \end{figure}

\subsubsection{稳定性条件不成立}

在数值实验中, 取定$t_{\max} =1$. 令$\mu = 0.51>0.5$, 不满足稳定性条件. 
计算$\mathbb{L}^{\infty}$与$\mathbb{L}^{2}$误差可知, 在不满足稳定性条件时, 显式格式所求得的数值解不收敛. 
\begin{table}[H]\centering\heiti\zihao{-5}
	\caption{显式格式不同步长时的 $\mathbb{L}^2$, $\mathbb{L}^\infty$ 误差及收敛阶}\label{tab:expliciterr2}
	\begin{tabular}{|c|c|c|c|c|}\hline
		收敛阶	&	$\mathbb{L}^2$ 误差	&	$h$	&	$\mathbb{L}^\infty$ 误差		&	收敛阶\\\hline	&	$9.9655\times10^{-3}$	&	$2^3$	&	$7.0467\times10^{-3}$	&	\\\hline
		$2.0229$	&	$2.4522\times10^{-3}$	&	$2^4$	&	$1.7340\times10^{-3}$	&	$2.02286$\\\hline
		$-3.3210$	&	$2.4507\times10^{-2}$	&	$2^5$	&	$1.7288\times10^{-2}$	&	$-3.31758$\\\hline
		$-171.0$	&	$7.3858\times10^{49}$	&	$2^6$	&	$5.2225\times10^{49}$	&	$-171.013$\\\hline
		$-680.7$	&	$5.8712\times10^{254}$	&	$2^7$	&	$4.1516\times10^{254}$	&	$-680.664$\\\hline
		$-\infty$	&	$\infty$	&	$2^8$	&	$\infty$	&	$-\infty$\\\hline
	\end{tabular}
\end{table}
% 数值解$U_{i,j}$与$u_{i,j}$在$t = t_{\max}$时刻图像如下所示, 
% \begin{figure}[H]\centering
% 	% \resizebox{0.9\linewidth}{!}{%% Creator: Matplotlib, PGF backend
%%
%% To include the figure in your LaTeX document, write
%%   \input{<filename>.pgf}
%%
%% Make sure the required packages are loaded in your preamble
%%   \usepackage{pgf}
%%
%% Figures using additional raster images can only be included by \input if
%% they are in the same directory as the main LaTeX file. For loading figures
%% from other directories you can use the `import` package
%%   \usepackage{import}
%%
%% and then include the figures with
%%   \import{<path to file>}{<filename>.pgf}
%%
%% Matplotlib used the following preamble
%%   \usepackage{fontspec}
%%   \setmainfont{DejaVuSerif.ttf}[Path=\detokenize{/Users/quejiahao/.julia/conda/3/lib/python3.9/site-packages/matplotlib/mpl-data/fonts/ttf/}]
%%   \setsansfont{DejaVuSans.ttf}[Path=\detokenize{/Users/quejiahao/.julia/conda/3/lib/python3.9/site-packages/matplotlib/mpl-data/fonts/ttf/}]
%%   \setmonofont{DejaVuSansMono.ttf}[Path=\detokenize{/Users/quejiahao/.julia/conda/3/lib/python3.9/site-packages/matplotlib/mpl-data/fonts/ttf/}]
%%
\begingroup%
\makeatletter%
\begin{pgfpicture}%
\pgfpathrectangle{\pgfpointorigin}{\pgfqpoint{12.000000in}{8.000000in}}%
\pgfusepath{use as bounding box, clip}%
\begin{pgfscope}%
\pgfsetbuttcap%
\pgfsetmiterjoin%
\definecolor{currentfill}{rgb}{0.152941,0.098039,0.141176}%
\pgfsetfillcolor{currentfill}%
\pgfsetlinewidth{0.000000pt}%
\definecolor{currentstroke}{rgb}{1.000000,1.000000,1.000000}%
\pgfsetstrokecolor{currentstroke}%
\pgfsetdash{}{0pt}%
\pgfpathmoveto{\pgfqpoint{0.000000in}{0.000000in}}%
\pgfpathlineto{\pgfqpoint{12.000000in}{0.000000in}}%
\pgfpathlineto{\pgfqpoint{12.000000in}{8.000000in}}%
\pgfpathlineto{\pgfqpoint{0.000000in}{8.000000in}}%
\pgfpathclose%
\pgfusepath{fill}%
\end{pgfscope}%
\begin{pgfscope}%
\pgfsetbuttcap%
\pgfsetmiterjoin%
\definecolor{currentfill}{rgb}{0.152941,0.098039,0.141176}%
\pgfsetfillcolor{currentfill}%
\pgfsetlinewidth{0.000000pt}%
\definecolor{currentstroke}{rgb}{0.000000,0.000000,0.000000}%
\pgfsetstrokecolor{currentstroke}%
\pgfsetstrokeopacity{0.000000}%
\pgfsetdash{}{0pt}%
\pgfpathmoveto{\pgfqpoint{0.539299in}{0.078740in}}%
\pgfpathlineto{\pgfqpoint{8.381819in}{0.078740in}}%
\pgfpathlineto{\pgfqpoint{8.381819in}{7.921260in}}%
\pgfpathlineto{\pgfqpoint{0.539299in}{7.921260in}}%
\pgfpathclose%
\pgfusepath{fill}%
\end{pgfscope}%
\begin{pgfscope}%
\pgfsetbuttcap%
\pgfsetmiterjoin%
\definecolor{currentfill}{rgb}{0.950000,0.950000,0.950000}%
\pgfsetfillcolor{currentfill}%
\pgfsetfillopacity{0.500000}%
\pgfsetlinewidth{1.003750pt}%
\definecolor{currentstroke}{rgb}{0.950000,0.950000,0.950000}%
\pgfsetstrokecolor{currentstroke}%
\pgfsetstrokeopacity{0.500000}%
\pgfsetdash{}{0pt}%
\pgfpathmoveto{\pgfqpoint{1.131463in}{2.012454in}}%
\pgfpathlineto{\pgfqpoint{3.721319in}{4.183323in}}%
\pgfpathlineto{\pgfqpoint{3.685318in}{7.314104in}}%
\pgfpathlineto{\pgfqpoint{0.971524in}{5.333700in}}%
\pgfusepath{stroke,fill}%
\end{pgfscope}%
\begin{pgfscope}%
\pgfsetbuttcap%
\pgfsetmiterjoin%
\definecolor{currentfill}{rgb}{0.900000,0.900000,0.900000}%
\pgfsetfillcolor{currentfill}%
\pgfsetfillopacity{0.500000}%
\pgfsetlinewidth{1.003750pt}%
\definecolor{currentstroke}{rgb}{0.900000,0.900000,0.900000}%
\pgfsetstrokecolor{currentstroke}%
\pgfsetstrokeopacity{0.500000}%
\pgfsetdash{}{0pt}%
\pgfpathmoveto{\pgfqpoint{3.721319in}{4.183323in}}%
\pgfpathlineto{\pgfqpoint{7.877114in}{2.975397in}}%
\pgfpathlineto{\pgfqpoint{8.025420in}{6.214014in}}%
\pgfpathlineto{\pgfqpoint{3.685318in}{7.314104in}}%
\pgfusepath{stroke,fill}%
\end{pgfscope}%
\begin{pgfscope}%
\pgfsetbuttcap%
\pgfsetmiterjoin%
\definecolor{currentfill}{rgb}{0.925000,0.925000,0.925000}%
\pgfsetfillcolor{currentfill}%
\pgfsetfillopacity{0.500000}%
\pgfsetlinewidth{1.003750pt}%
\definecolor{currentstroke}{rgb}{0.925000,0.925000,0.925000}%
\pgfsetstrokecolor{currentstroke}%
\pgfsetstrokeopacity{0.500000}%
\pgfsetdash{}{0pt}%
\pgfpathmoveto{\pgfqpoint{1.131463in}{2.012454in}}%
\pgfpathlineto{\pgfqpoint{5.536809in}{0.573668in}}%
\pgfpathlineto{\pgfqpoint{7.877114in}{2.975397in}}%
\pgfpathlineto{\pgfqpoint{3.721319in}{4.183323in}}%
\pgfusepath{stroke,fill}%
\end{pgfscope}%
\begin{pgfscope}%
\pgfsetrectcap%
\pgfsetroundjoin%
\pgfsetlinewidth{0.803000pt}%
\definecolor{currentstroke}{rgb}{0.000000,0.000000,0.000000}%
\pgfsetstrokecolor{currentstroke}%
\pgfsetdash{}{0pt}%
\pgfpathmoveto{\pgfqpoint{1.131463in}{2.012454in}}%
\pgfpathlineto{\pgfqpoint{5.536809in}{0.573668in}}%
\pgfusepath{stroke}%
\end{pgfscope}%
\begin{pgfscope}%
\definecolor{textcolor}{rgb}{0.525490,0.694118,0.356863}%
\pgfsetstrokecolor{textcolor}%
\pgfsetfillcolor{textcolor}%
\pgftext[x=3.050786in,y=0.824279in,,]{\color{textcolor}\sffamily\fontsize{24.000000}{13.200000}\bfseries\selectfont $x$}%
\end{pgfscope}%
\begin{pgfscope}%
\pgfsetbuttcap%
\pgfsetroundjoin%
\pgfsetlinewidth{0.803000pt}%
\definecolor{currentstroke}{rgb}{0.690196,0.690196,0.690196}%
\pgfsetstrokecolor{currentstroke}%
\pgfsetdash{}{0pt}%
\pgfpathmoveto{\pgfqpoint{1.945241in}{1.746674in}}%
\pgfpathlineto{\pgfqpoint{4.491766in}{3.959384in}}%
\pgfpathlineto{\pgfqpoint{4.488551in}{7.110508in}}%
\pgfusepath{stroke}%
\end{pgfscope}%
\begin{pgfscope}%
\pgfsetbuttcap%
\pgfsetroundjoin%
\pgfsetlinewidth{0.803000pt}%
\definecolor{currentstroke}{rgb}{0.690196,0.690196,0.690196}%
\pgfsetstrokecolor{currentstroke}%
\pgfsetdash{}{0pt}%
\pgfpathmoveto{\pgfqpoint{2.827694in}{1.458465in}}%
\pgfpathlineto{\pgfqpoint{5.325810in}{3.716961in}}%
\pgfpathlineto{\pgfqpoint{5.358796in}{6.889926in}}%
\pgfusepath{stroke}%
\end{pgfscope}%
\begin{pgfscope}%
\pgfsetbuttcap%
\pgfsetroundjoin%
\pgfsetlinewidth{0.803000pt}%
\definecolor{currentstroke}{rgb}{0.690196,0.690196,0.690196}%
\pgfsetstrokecolor{currentstroke}%
\pgfsetdash{}{0pt}%
\pgfpathmoveto{\pgfqpoint{3.729308in}{1.163998in}}%
\pgfpathlineto{\pgfqpoint{6.176441in}{3.469716in}}%
\pgfpathlineto{\pgfqpoint{6.247107in}{6.664765in}}%
\pgfusepath{stroke}%
\end{pgfscope}%
\begin{pgfscope}%
\pgfsetbuttcap%
\pgfsetroundjoin%
\pgfsetlinewidth{0.803000pt}%
\definecolor{currentstroke}{rgb}{0.690196,0.690196,0.690196}%
\pgfsetstrokecolor{currentstroke}%
\pgfsetdash{}{0pt}%
\pgfpathmoveto{\pgfqpoint{4.650714in}{0.863067in}}%
\pgfpathlineto{\pgfqpoint{7.044159in}{3.217505in}}%
\pgfpathlineto{\pgfqpoint{7.154054in}{6.434880in}}%
\pgfusepath{stroke}%
\end{pgfscope}%
\begin{pgfscope}%
\pgfpathrectangle{\pgfqpoint{0.539299in}{0.078740in}}{\pgfqpoint{7.842520in}{7.842520in}}%
\pgfusepath{clip}%
\pgfsetrectcap%
\pgfsetroundjoin%
\pgfsetlinewidth{0.501875pt}%
\definecolor{currentstroke}{rgb}{0.980392,0.811765,0.352941}%
\pgfsetstrokecolor{currentstroke}%
\pgfsetstrokeopacity{0.100000}%
\pgfsetdash{}{0pt}%
\pgfpathmoveto{\pgfqpoint{4.566539in}{4.105980in}}%
\pgfusepath{stroke}%
\end{pgfscope}%
\begin{pgfscope}%
\pgfsetrectcap%
\pgfsetroundjoin%
\pgfsetlinewidth{0.803000pt}%
\definecolor{currentstroke}{rgb}{0.980392,0.811765,0.352941}%
\pgfsetstrokecolor{currentstroke}%
\pgfsetdash{}{0pt}%
\pgfpathmoveto{\pgfqpoint{1.967428in}{1.765953in}}%
\pgfpathlineto{\pgfqpoint{1.900771in}{1.708034in}}%
\pgfusepath{stroke}%
\end{pgfscope}%
\begin{pgfscope}%
\definecolor{textcolor}{rgb}{0.525490,0.694118,0.356863}%
\pgfsetstrokecolor{textcolor}%
\pgfsetfillcolor{textcolor}%
\pgftext[x=1.841138in,y=1.527121in,,top]{\color{textcolor}\sffamily\fontsize{18.000000}{9.600000}\selectfont $\displaystyle 0.2$}%
\end{pgfscope}%
\begin{pgfscope}%
\pgfpathrectangle{\pgfqpoint{0.539299in}{0.078740in}}{\pgfqpoint{7.842520in}{7.842520in}}%
\pgfusepath{clip}%
\pgfsetrectcap%
\pgfsetroundjoin%
\pgfsetlinewidth{0.501875pt}%
\definecolor{currentstroke}{rgb}{0.980392,0.811765,0.352941}%
\pgfsetstrokecolor{currentstroke}%
\pgfsetstrokeopacity{0.100000}%
\pgfsetdash{}{0pt}%
\pgfpathmoveto{\pgfqpoint{4.566539in}{4.105980in}}%
\pgfusepath{stroke}%
\end{pgfscope}%
\begin{pgfscope}%
\pgfsetrectcap%
\pgfsetroundjoin%
\pgfsetlinewidth{0.803000pt}%
\definecolor{currentstroke}{rgb}{0.980392,0.811765,0.352941}%
\pgfsetstrokecolor{currentstroke}%
\pgfsetdash{}{0pt}%
\pgfpathmoveto{\pgfqpoint{2.849478in}{1.478160in}}%
\pgfpathlineto{\pgfqpoint{2.784030in}{1.418990in}}%
\pgfusepath{stroke}%
\end{pgfscope}%
\begin{pgfscope}%
\definecolor{textcolor}{rgb}{0.525490,0.694118,0.356863}%
\pgfsetstrokecolor{textcolor}%
\pgfsetfillcolor{textcolor}%
\pgftext[x=2.724041in,y=1.236037in,,top]{\color{textcolor}\sffamily\fontsize{18.000000}{9.600000}\selectfont $\displaystyle 0.4$}%
\end{pgfscope}%
\begin{pgfscope}%
\pgfpathrectangle{\pgfqpoint{0.539299in}{0.078740in}}{\pgfqpoint{7.842520in}{7.842520in}}%
\pgfusepath{clip}%
\pgfsetrectcap%
\pgfsetroundjoin%
\pgfsetlinewidth{0.501875pt}%
\definecolor{currentstroke}{rgb}{0.980392,0.811765,0.352941}%
\pgfsetstrokecolor{currentstroke}%
\pgfsetstrokeopacity{0.100000}%
\pgfsetdash{}{0pt}%
\pgfpathmoveto{\pgfqpoint{4.566539in}{4.105980in}}%
\pgfusepath{stroke}%
\end{pgfscope}%
\begin{pgfscope}%
\pgfsetrectcap%
\pgfsetroundjoin%
\pgfsetlinewidth{0.803000pt}%
\definecolor{currentstroke}{rgb}{0.980392,0.811765,0.352941}%
\pgfsetstrokecolor{currentstroke}%
\pgfsetdash{}{0pt}%
\pgfpathmoveto{\pgfqpoint{3.750667in}{1.184123in}}%
\pgfpathlineto{\pgfqpoint{3.686496in}{1.123660in}}%
\pgfusepath{stroke}%
\end{pgfscope}%
\begin{pgfscope}%
\definecolor{textcolor}{rgb}{0.525490,0.694118,0.356863}%
\pgfsetstrokecolor{textcolor}%
\pgfsetfillcolor{textcolor}%
\pgftext[x=3.626149in,y=0.938621in,,top]{\color{textcolor}\sffamily\fontsize{18.000000}{9.600000}\selectfont $\displaystyle 0.6$}%
\end{pgfscope}%
\begin{pgfscope}%
\pgfpathrectangle{\pgfqpoint{0.539299in}{0.078740in}}{\pgfqpoint{7.842520in}{7.842520in}}%
\pgfusepath{clip}%
\pgfsetrectcap%
\pgfsetroundjoin%
\pgfsetlinewidth{0.501875pt}%
\definecolor{currentstroke}{rgb}{0.980392,0.811765,0.352941}%
\pgfsetstrokecolor{currentstroke}%
\pgfsetstrokeopacity{0.100000}%
\pgfsetdash{}{0pt}%
\pgfpathmoveto{\pgfqpoint{4.566539in}{4.105980in}}%
\pgfusepath{stroke}%
\end{pgfscope}%
\begin{pgfscope}%
\pgfsetrectcap%
\pgfsetroundjoin%
\pgfsetlinewidth{0.803000pt}%
\definecolor{currentstroke}{rgb}{0.980392,0.811765,0.352941}%
\pgfsetstrokecolor{currentstroke}%
\pgfsetdash{}{0pt}%
\pgfpathmoveto{\pgfqpoint{4.671624in}{0.883635in}}%
\pgfpathlineto{\pgfqpoint{4.608802in}{0.821838in}}%
\pgfusepath{stroke}%
\end{pgfscope}%
\begin{pgfscope}%
\definecolor{textcolor}{rgb}{0.525490,0.694118,0.356863}%
\pgfsetstrokecolor{textcolor}%
\pgfsetfillcolor{textcolor}%
\pgftext[x=4.548096in,y=0.634665in,,top]{\color{textcolor}\sffamily\fontsize{18.000000}{9.600000}\selectfont $\displaystyle 0.8$}%
\end{pgfscope}%
\begin{pgfscope}%
\pgfsetrectcap%
\pgfsetroundjoin%
\pgfsetlinewidth{0.803000pt}%
\definecolor{currentstroke}{rgb}{0.000000,0.000000,0.000000}%
\pgfsetstrokecolor{currentstroke}%
\pgfsetdash{}{0pt}%
\pgfpathmoveto{\pgfqpoint{7.877114in}{2.975397in}}%
\pgfpathlineto{\pgfqpoint{5.536809in}{0.573668in}}%
\pgfusepath{stroke}%
\end{pgfscope}%
\begin{pgfscope}%
\definecolor{textcolor}{rgb}{0.525490,0.694118,0.356863}%
\pgfsetstrokecolor{textcolor}%
\pgfsetfillcolor{textcolor}%
\pgftext[x=7.133707in,y=1.439527in,,]{\color{textcolor}\sffamily\fontsize{24.000000}{13.200000}\bfseries\selectfont $y$}%
\end{pgfscope}%
\begin{pgfscope}%
\pgfsetbuttcap%
\pgfsetroundjoin%
\pgfsetlinewidth{0.803000pt}%
\definecolor{currentstroke}{rgb}{0.690196,0.690196,0.690196}%
\pgfsetstrokecolor{currentstroke}%
\pgfsetdash{}{0pt}%
\pgfpathmoveto{\pgfqpoint{1.533379in}{5.743717in}}%
\pgfpathlineto{\pgfqpoint{1.666046in}{2.460552in}}%
\pgfpathlineto{\pgfqpoint{6.021576in}{1.071158in}}%
\pgfusepath{stroke}%
\end{pgfscope}%
\begin{pgfscope}%
\pgfsetbuttcap%
\pgfsetroundjoin%
\pgfsetlinewidth{0.803000pt}%
\definecolor{currentstroke}{rgb}{0.690196,0.690196,0.690196}%
\pgfsetstrokecolor{currentstroke}%
\pgfsetdash{}{0pt}%
\pgfpathmoveto{\pgfqpoint{2.108895in}{6.163702in}}%
\pgfpathlineto{\pgfqpoint{2.214498in}{2.920275in}}%
\pgfpathlineto{\pgfqpoint{6.518001in}{1.580612in}}%
\pgfusepath{stroke}%
\end{pgfscope}%
\begin{pgfscope}%
\pgfsetbuttcap%
\pgfsetroundjoin%
\pgfsetlinewidth{0.803000pt}%
\definecolor{currentstroke}{rgb}{0.690196,0.690196,0.690196}%
\pgfsetstrokecolor{currentstroke}%
\pgfsetdash{}{0pt}%
\pgfpathmoveto{\pgfqpoint{2.663627in}{6.568520in}}%
\pgfpathlineto{\pgfqpoint{2.743979in}{3.364097in}}%
\pgfpathlineto{\pgfqpoint{6.996374in}{2.071541in}}%
\pgfusepath{stroke}%
\end{pgfscope}%
\begin{pgfscope}%
\pgfsetbuttcap%
\pgfsetroundjoin%
\pgfsetlinewidth{0.803000pt}%
\definecolor{currentstroke}{rgb}{0.690196,0.690196,0.690196}%
\pgfsetstrokecolor{currentstroke}%
\pgfsetdash{}{0pt}%
\pgfpathmoveto{\pgfqpoint{3.198678in}{6.958976in}}%
\pgfpathlineto{\pgfqpoint{3.255455in}{3.792826in}}%
\pgfpathlineto{\pgfqpoint{7.457662in}{2.544936in}}%
\pgfusepath{stroke}%
\end{pgfscope}%
\begin{pgfscope}%
\pgfpathrectangle{\pgfqpoint{0.539299in}{0.078740in}}{\pgfqpoint{7.842520in}{7.842520in}}%
\pgfusepath{clip}%
\pgfsetrectcap%
\pgfsetroundjoin%
\pgfsetlinewidth{0.501875pt}%
\definecolor{currentstroke}{rgb}{0.980392,0.811765,0.352941}%
\pgfsetstrokecolor{currentstroke}%
\pgfsetstrokeopacity{0.100000}%
\pgfsetdash{}{0pt}%
\pgfpathmoveto{\pgfqpoint{4.566539in}{4.105980in}}%
\pgfusepath{stroke}%
\end{pgfscope}%
\begin{pgfscope}%
\pgfsetrectcap%
\pgfsetroundjoin%
\pgfsetlinewidth{0.803000pt}%
\definecolor{currentstroke}{rgb}{0.980392,0.811765,0.352941}%
\pgfsetstrokecolor{currentstroke}%
\pgfsetdash{}{0pt}%
\pgfpathmoveto{\pgfqpoint{5.984893in}{1.082859in}}%
\pgfpathlineto{\pgfqpoint{6.095033in}{1.047725in}}%
\pgfusepath{stroke}%
\end{pgfscope}%
\begin{pgfscope}%
\definecolor{textcolor}{rgb}{0.525490,0.694118,0.356863}%
\pgfsetstrokecolor{textcolor}%
\pgfsetfillcolor{textcolor}%
\pgftext[x=6.198378in,y=0.887093in,,top]{\color{textcolor}\sffamily\fontsize{18.000000}{9.600000}\selectfont $\displaystyle 0.2$}%
\end{pgfscope}%
\begin{pgfscope}%
\pgfpathrectangle{\pgfqpoint{0.539299in}{0.078740in}}{\pgfqpoint{7.842520in}{7.842520in}}%
\pgfusepath{clip}%
\pgfsetrectcap%
\pgfsetroundjoin%
\pgfsetlinewidth{0.501875pt}%
\definecolor{currentstroke}{rgb}{0.980392,0.811765,0.352941}%
\pgfsetstrokecolor{currentstroke}%
\pgfsetstrokeopacity{0.100000}%
\pgfsetdash{}{0pt}%
\pgfpathmoveto{\pgfqpoint{4.566539in}{4.105980in}}%
\pgfusepath{stroke}%
\end{pgfscope}%
\begin{pgfscope}%
\pgfsetrectcap%
\pgfsetroundjoin%
\pgfsetlinewidth{0.803000pt}%
\definecolor{currentstroke}{rgb}{0.980392,0.811765,0.352941}%
\pgfsetstrokecolor{currentstroke}%
\pgfsetdash{}{0pt}%
\pgfpathmoveto{\pgfqpoint{6.481791in}{1.591885in}}%
\pgfpathlineto{\pgfqpoint{6.590512in}{1.558040in}}%
\pgfusepath{stroke}%
\end{pgfscope}%
\begin{pgfscope}%
\definecolor{textcolor}{rgb}{0.525490,0.694118,0.356863}%
\pgfsetstrokecolor{textcolor}%
\pgfsetfillcolor{textcolor}%
\pgftext[x=6.691664in,y=1.400097in,,top]{\color{textcolor}\sffamily\fontsize{18.000000}{9.600000}\selectfont $\displaystyle 0.4$}%
\end{pgfscope}%
\begin{pgfscope}%
\pgfpathrectangle{\pgfqpoint{0.539299in}{0.078740in}}{\pgfqpoint{7.842520in}{7.842520in}}%
\pgfusepath{clip}%
\pgfsetrectcap%
\pgfsetroundjoin%
\pgfsetlinewidth{0.501875pt}%
\definecolor{currentstroke}{rgb}{0.980392,0.811765,0.352941}%
\pgfsetstrokecolor{currentstroke}%
\pgfsetstrokeopacity{0.100000}%
\pgfsetdash{}{0pt}%
\pgfpathmoveto{\pgfqpoint{4.566539in}{4.105980in}}%
\pgfusepath{stroke}%
\end{pgfscope}%
\begin{pgfscope}%
\pgfsetrectcap%
\pgfsetroundjoin%
\pgfsetlinewidth{0.803000pt}%
\definecolor{currentstroke}{rgb}{0.980392,0.811765,0.352941}%
\pgfsetstrokecolor{currentstroke}%
\pgfsetdash{}{0pt}%
\pgfpathmoveto{\pgfqpoint{6.960625in}{2.082407in}}%
\pgfpathlineto{\pgfqpoint{7.067958in}{2.049782in}}%
\pgfusepath{stroke}%
\end{pgfscope}%
\begin{pgfscope}%
\definecolor{textcolor}{rgb}{0.525490,0.694118,0.356863}%
\pgfsetstrokecolor{textcolor}%
\pgfsetfillcolor{textcolor}%
\pgftext[x=7.167007in,y=1.894440in,,top]{\color{textcolor}\sffamily\fontsize{18.000000}{9.600000}\selectfont $\displaystyle 0.6$}%
\end{pgfscope}%
\begin{pgfscope}%
\pgfpathrectangle{\pgfqpoint{0.539299in}{0.078740in}}{\pgfqpoint{7.842520in}{7.842520in}}%
\pgfusepath{clip}%
\pgfsetrectcap%
\pgfsetroundjoin%
\pgfsetlinewidth{0.501875pt}%
\definecolor{currentstroke}{rgb}{0.980392,0.811765,0.352941}%
\pgfsetstrokecolor{currentstroke}%
\pgfsetstrokeopacity{0.100000}%
\pgfsetdash{}{0pt}%
\pgfpathmoveto{\pgfqpoint{4.566539in}{4.105980in}}%
\pgfusepath{stroke}%
\end{pgfscope}%
\begin{pgfscope}%
\pgfsetrectcap%
\pgfsetroundjoin%
\pgfsetlinewidth{0.803000pt}%
\definecolor{currentstroke}{rgb}{0.980392,0.811765,0.352941}%
\pgfsetstrokecolor{currentstroke}%
\pgfsetdash{}{0pt}%
\pgfpathmoveto{\pgfqpoint{7.422366in}{2.555417in}}%
\pgfpathlineto{\pgfqpoint{7.528339in}{2.523947in}}%
\pgfusepath{stroke}%
\end{pgfscope}%
\begin{pgfscope}%
\definecolor{textcolor}{rgb}{0.525490,0.694118,0.356863}%
\pgfsetstrokecolor{textcolor}%
\pgfsetfillcolor{textcolor}%
\pgftext[x=7.625368in,y=2.371123in,,top]{\color{textcolor}\sffamily\fontsize{18.000000}{9.600000}\selectfont $\displaystyle 0.8$}%
\end{pgfscope}%
\begin{pgfscope}%
\pgfsetrectcap%
\pgfsetroundjoin%
\pgfsetlinewidth{0.803000pt}%
\definecolor{currentstroke}{rgb}{0.000000,0.000000,0.000000}%
\pgfsetstrokecolor{currentstroke}%
\pgfsetdash{}{0pt}%
\pgfpathmoveto{\pgfqpoint{7.877114in}{2.975397in}}%
\pgfpathlineto{\pgfqpoint{8.025420in}{6.214014in}}%
\pgfusepath{stroke}%
\end{pgfscope}%
\begin{pgfscope}%
\definecolor{textcolor}{rgb}{0.525490,0.694118,0.356863}%
\pgfsetstrokecolor{textcolor}%
\pgfsetfillcolor{textcolor}%
\pgftext[x=8.812392in,y=4.634624in,,]{\color{textcolor}\sffamily\fontsize{24.000000}{13.200000}\bfseries\selectfont $U$}%
\end{pgfscope}%
\begin{pgfscope}%
\pgfsetbuttcap%
\pgfsetroundjoin%
\pgfsetlinewidth{0.803000pt}%
\definecolor{currentstroke}{rgb}{0.690196,0.690196,0.690196}%
\pgfsetstrokecolor{currentstroke}%
\pgfsetdash{}{0pt}%
\pgfpathmoveto{\pgfqpoint{7.903878in}{3.559855in}}%
\pgfpathlineto{\pgfqpoint{3.714811in}{4.749298in}}%
\pgfpathlineto{\pgfqpoint{1.102640in}{2.610996in}}%
\pgfusepath{stroke}%
\end{pgfscope}%
\begin{pgfscope}%
\pgfsetbuttcap%
\pgfsetroundjoin%
\pgfsetlinewidth{0.803000pt}%
\definecolor{currentstroke}{rgb}{0.690196,0.690196,0.690196}%
\pgfsetstrokecolor{currentstroke}%
\pgfsetdash{}{0pt}%
\pgfpathmoveto{\pgfqpoint{7.928165in}{4.090216in}}%
\pgfpathlineto{\pgfqpoint{3.708910in}{5.262514in}}%
\pgfpathlineto{\pgfqpoint{1.076469in}{3.154452in}}%
\pgfusepath{stroke}%
\end{pgfscope}%
\begin{pgfscope}%
\pgfsetbuttcap%
\pgfsetroundjoin%
\pgfsetlinewidth{0.803000pt}%
\definecolor{currentstroke}{rgb}{0.690196,0.690196,0.690196}%
\pgfsetstrokecolor{currentstroke}%
\pgfsetdash{}{0pt}%
\pgfpathmoveto{\pgfqpoint{7.952808in}{4.628352in}}%
\pgfpathlineto{\pgfqpoint{3.702926in}{5.782891in}}%
\pgfpathlineto{\pgfqpoint{1.049899in}{3.706182in}}%
\pgfusepath{stroke}%
\end{pgfscope}%
\begin{pgfscope}%
\pgfsetbuttcap%
\pgfsetroundjoin%
\pgfsetlinewidth{0.803000pt}%
\definecolor{currentstroke}{rgb}{0.690196,0.690196,0.690196}%
\pgfsetstrokecolor{currentstroke}%
\pgfsetdash{}{0pt}%
\pgfpathmoveto{\pgfqpoint{7.977815in}{5.174436in}}%
\pgfpathlineto{\pgfqpoint{3.696858in}{6.310580in}}%
\pgfpathlineto{\pgfqpoint{1.022922in}{4.266378in}}%
\pgfusepath{stroke}%
\end{pgfscope}%
\begin{pgfscope}%
\pgfsetbuttcap%
\pgfsetroundjoin%
\pgfsetlinewidth{0.803000pt}%
\definecolor{currentstroke}{rgb}{0.690196,0.690196,0.690196}%
\pgfsetstrokecolor{currentstroke}%
\pgfsetdash{}{0pt}%
\pgfpathmoveto{\pgfqpoint{8.003194in}{5.728645in}}%
\pgfpathlineto{\pgfqpoint{3.690704in}{6.845738in}}%
\pgfpathlineto{\pgfqpoint{0.995528in}{4.835234in}}%
\pgfusepath{stroke}%
\end{pgfscope}%
\begin{pgfscope}%
\pgfpathrectangle{\pgfqpoint{0.539299in}{0.078740in}}{\pgfqpoint{7.842520in}{7.842520in}}%
\pgfusepath{clip}%
\pgfsetrectcap%
\pgfsetroundjoin%
\pgfsetlinewidth{0.501875pt}%
\definecolor{currentstroke}{rgb}{0.980392,0.811765,0.352941}%
\pgfsetstrokecolor{currentstroke}%
\pgfsetstrokeopacity{0.100000}%
\pgfsetdash{}{0pt}%
\pgfpathmoveto{\pgfqpoint{4.566539in}{4.105980in}}%
\pgfusepath{stroke}%
\end{pgfscope}%
\begin{pgfscope}%
\pgfsetrectcap%
\pgfsetroundjoin%
\pgfsetlinewidth{0.803000pt}%
\definecolor{currentstroke}{rgb}{0.980392,0.811765,0.352941}%
\pgfsetstrokecolor{currentstroke}%
\pgfsetdash{}{0pt}%
\pgfpathmoveto{\pgfqpoint{7.868707in}{3.569842in}}%
\pgfpathlineto{\pgfqpoint{7.974305in}{3.539859in}}%
\pgfusepath{stroke}%
\end{pgfscope}%
\begin{pgfscope}%
\definecolor{textcolor}{rgb}{0.525490,0.694118,0.356863}%
\pgfsetstrokecolor{textcolor}%
\pgfsetfillcolor{textcolor}%
\pgftext[x=8.359745in,y=3.595552in,,top]{\color{textcolor}\sffamily\fontsize{18.000000}{9.600000}\selectfont $\displaystyle 0.005$}%
\end{pgfscope}%
\begin{pgfscope}%
\pgfpathrectangle{\pgfqpoint{0.539299in}{0.078740in}}{\pgfqpoint{7.842520in}{7.842520in}}%
\pgfusepath{clip}%
\pgfsetrectcap%
\pgfsetroundjoin%
\pgfsetlinewidth{0.501875pt}%
\definecolor{currentstroke}{rgb}{0.980392,0.811765,0.352941}%
\pgfsetstrokecolor{currentstroke}%
\pgfsetstrokeopacity{0.100000}%
\pgfsetdash{}{0pt}%
\pgfpathmoveto{\pgfqpoint{4.566539in}{4.105980in}}%
\pgfusepath{stroke}%
\end{pgfscope}%
\begin{pgfscope}%
\pgfsetrectcap%
\pgfsetroundjoin%
\pgfsetlinewidth{0.803000pt}%
\definecolor{currentstroke}{rgb}{0.980392,0.811765,0.352941}%
\pgfsetstrokecolor{currentstroke}%
\pgfsetdash{}{0pt}%
\pgfpathmoveto{\pgfqpoint{7.892728in}{4.100062in}}%
\pgfpathlineto{\pgfqpoint{7.999124in}{4.070500in}}%
\pgfusepath{stroke}%
\end{pgfscope}%
\begin{pgfscope}%
\definecolor{textcolor}{rgb}{0.525490,0.694118,0.356863}%
\pgfsetstrokecolor{textcolor}%
\pgfsetfillcolor{textcolor}%
\pgftext[x=8.385857in,y=4.125410in,,top]{\color{textcolor}\sffamily\fontsize{18.000000}{9.600000}\selectfont $\displaystyle 0.010$}%
\end{pgfscope}%
\begin{pgfscope}%
\pgfpathrectangle{\pgfqpoint{0.539299in}{0.078740in}}{\pgfqpoint{7.842520in}{7.842520in}}%
\pgfusepath{clip}%
\pgfsetrectcap%
\pgfsetroundjoin%
\pgfsetlinewidth{0.501875pt}%
\definecolor{currentstroke}{rgb}{0.980392,0.811765,0.352941}%
\pgfsetstrokecolor{currentstroke}%
\pgfsetstrokeopacity{0.100000}%
\pgfsetdash{}{0pt}%
\pgfpathmoveto{\pgfqpoint{4.566539in}{4.105980in}}%
\pgfusepath{stroke}%
\end{pgfscope}%
\begin{pgfscope}%
\pgfsetrectcap%
\pgfsetroundjoin%
\pgfsetlinewidth{0.803000pt}%
\definecolor{currentstroke}{rgb}{0.980392,0.811765,0.352941}%
\pgfsetstrokecolor{currentstroke}%
\pgfsetdash{}{0pt}%
\pgfpathmoveto{\pgfqpoint{7.917102in}{4.638052in}}%
\pgfpathlineto{\pgfqpoint{8.024308in}{4.608928in}}%
\pgfusepath{stroke}%
\end{pgfscope}%
\begin{pgfscope}%
\definecolor{textcolor}{rgb}{0.525490,0.694118,0.356863}%
\pgfsetstrokecolor{textcolor}%
\pgfsetfillcolor{textcolor}%
\pgftext[x=8.412350in,y=4.663024in,,top]{\color{textcolor}\sffamily\fontsize{18.000000}{9.600000}\selectfont $\displaystyle 0.015$}%
\end{pgfscope}%
\begin{pgfscope}%
\pgfpathrectangle{\pgfqpoint{0.539299in}{0.078740in}}{\pgfqpoint{7.842520in}{7.842520in}}%
\pgfusepath{clip}%
\pgfsetrectcap%
\pgfsetroundjoin%
\pgfsetlinewidth{0.501875pt}%
\definecolor{currentstroke}{rgb}{0.980392,0.811765,0.352941}%
\pgfsetstrokecolor{currentstroke}%
\pgfsetstrokeopacity{0.100000}%
\pgfsetdash{}{0pt}%
\pgfpathmoveto{\pgfqpoint{4.566539in}{4.105980in}}%
\pgfusepath{stroke}%
\end{pgfscope}%
\begin{pgfscope}%
\pgfsetrectcap%
\pgfsetroundjoin%
\pgfsetlinewidth{0.803000pt}%
\definecolor{currentstroke}{rgb}{0.980392,0.811765,0.352941}%
\pgfsetstrokecolor{currentstroke}%
\pgfsetdash{}{0pt}%
\pgfpathmoveto{\pgfqpoint{7.941835in}{5.183985in}}%
\pgfpathlineto{\pgfqpoint{8.049863in}{5.155315in}}%
\pgfusepath{stroke}%
\end{pgfscope}%
\begin{pgfscope}%
\definecolor{textcolor}{rgb}{0.525490,0.694118,0.356863}%
\pgfsetstrokecolor{textcolor}%
\pgfsetfillcolor{textcolor}%
\pgftext[x=8.439234in,y=5.208568in,,top]{\color{textcolor}\sffamily\fontsize{18.000000}{9.600000}\selectfont $\displaystyle 0.020$}%
\end{pgfscope}%
\begin{pgfscope}%
\pgfpathrectangle{\pgfqpoint{0.539299in}{0.078740in}}{\pgfqpoint{7.842520in}{7.842520in}}%
\pgfusepath{clip}%
\pgfsetrectcap%
\pgfsetroundjoin%
\pgfsetlinewidth{0.501875pt}%
\definecolor{currentstroke}{rgb}{0.980392,0.811765,0.352941}%
\pgfsetstrokecolor{currentstroke}%
\pgfsetstrokeopacity{0.100000}%
\pgfsetdash{}{0pt}%
\pgfpathmoveto{\pgfqpoint{4.566539in}{4.105980in}}%
\pgfusepath{stroke}%
\end{pgfscope}%
\begin{pgfscope}%
\pgfsetrectcap%
\pgfsetroundjoin%
\pgfsetlinewidth{0.803000pt}%
\definecolor{currentstroke}{rgb}{0.980392,0.811765,0.352941}%
\pgfsetstrokecolor{currentstroke}%
\pgfsetdash{}{0pt}%
\pgfpathmoveto{\pgfqpoint{7.966936in}{5.738038in}}%
\pgfpathlineto{\pgfqpoint{8.075799in}{5.709838in}}%
\pgfusepath{stroke}%
\end{pgfscope}%
\begin{pgfscope}%
\definecolor{textcolor}{rgb}{0.525490,0.694118,0.356863}%
\pgfsetstrokecolor{textcolor}%
\pgfsetfillcolor{textcolor}%
\pgftext[x=8.466518in,y=5.762217in,,top]{\color{textcolor}\sffamily\fontsize{18.000000}{9.600000}\selectfont $\displaystyle 0.025$}%
\end{pgfscope}%
\begin{pgfscope}%
\pgfpathrectangle{\pgfqpoint{0.539299in}{0.078740in}}{\pgfqpoint{7.842520in}{7.842520in}}%
\pgfusepath{clip}%
\pgfsetbuttcap%
\pgfsetroundjoin%
\definecolor{currentfill}{rgb}{0.283091,0.110553,0.431554}%
\pgfsetfillcolor{currentfill}%
\pgfsetlinewidth{0.000000pt}%
\definecolor{currentstroke}{rgb}{0.267004,0.004874,0.329415}%
\pgfsetstrokecolor{currentstroke}%
\pgfsetdash{}{0pt}%
\pgfpathmoveto{\pgfqpoint{3.799922in}{4.421780in}}%
\pgfpathlineto{\pgfqpoint{3.877896in}{4.341545in}}%
\pgfpathlineto{\pgfqpoint{3.750979in}{4.305549in}}%
\pgfpathclose%
\pgfusepath{fill}%
\end{pgfscope}%
\begin{pgfscope}%
\pgfpathrectangle{\pgfqpoint{0.539299in}{0.078740in}}{\pgfqpoint{7.842520in}{7.842520in}}%
\pgfusepath{clip}%
\pgfsetbuttcap%
\pgfsetroundjoin%
\definecolor{currentfill}{rgb}{0.283091,0.110553,0.431554}%
\pgfsetfillcolor{currentfill}%
\pgfsetlinewidth{0.000000pt}%
\definecolor{currentstroke}{rgb}{0.268510,0.009605,0.335427}%
\pgfsetstrokecolor{currentstroke}%
\pgfsetdash{}{0pt}%
\pgfpathmoveto{\pgfqpoint{3.750979in}{4.305549in}}%
\pgfpathlineto{\pgfqpoint{3.673363in}{4.323977in}}%
\pgfpathlineto{\pgfqpoint{3.799922in}{4.421780in}}%
\pgfpathclose%
\pgfusepath{fill}%
\end{pgfscope}%
\begin{pgfscope}%
\pgfpathrectangle{\pgfqpoint{0.539299in}{0.078740in}}{\pgfqpoint{7.842520in}{7.842520in}}%
\pgfusepath{clip}%
\pgfsetbuttcap%
\pgfsetroundjoin%
\definecolor{currentfill}{rgb}{0.283187,0.125848,0.444960}%
\pgfsetfillcolor{currentfill}%
\pgfsetlinewidth{0.000000pt}%
\definecolor{currentstroke}{rgb}{0.269944,0.014625,0.341379}%
\pgfsetstrokecolor{currentstroke}%
\pgfsetdash{}{0pt}%
\pgfpathmoveto{\pgfqpoint{4.005716in}{4.343136in}}%
\pgfpathlineto{\pgfqpoint{3.877896in}{4.341545in}}%
\pgfpathlineto{\pgfqpoint{3.799922in}{4.421780in}}%
\pgfpathclose%
\pgfusepath{fill}%
\end{pgfscope}%
\begin{pgfscope}%
\pgfpathrectangle{\pgfqpoint{0.539299in}{0.078740in}}{\pgfqpoint{7.842520in}{7.842520in}}%
\pgfusepath{clip}%
\pgfsetbuttcap%
\pgfsetroundjoin%
\definecolor{currentfill}{rgb}{0.280868,0.160771,0.472899}%
\pgfsetfillcolor{currentfill}%
\pgfsetlinewidth{0.000000pt}%
\definecolor{currentstroke}{rgb}{0.271305,0.019942,0.347269}%
\pgfsetstrokecolor{currentstroke}%
\pgfsetdash{}{0pt}%
\pgfpathmoveto{\pgfqpoint{3.927826in}{4.463428in}}%
\pgfpathlineto{\pgfqpoint{4.005716in}{4.343136in}}%
\pgfpathlineto{\pgfqpoint{3.799922in}{4.421780in}}%
\pgfpathclose%
\pgfusepath{fill}%
\end{pgfscope}%
\begin{pgfscope}%
\pgfpathrectangle{\pgfqpoint{0.539299in}{0.078740in}}{\pgfqpoint{7.842520in}{7.842520in}}%
\pgfusepath{clip}%
\pgfsetbuttcap%
\pgfsetroundjoin%
\definecolor{currentfill}{rgb}{0.280868,0.160771,0.472899}%
\pgfsetfillcolor{currentfill}%
\pgfsetlinewidth{0.000000pt}%
\definecolor{currentstroke}{rgb}{0.272594,0.025563,0.353093}%
\pgfsetstrokecolor{currentstroke}%
\pgfsetdash{}{0pt}%
\pgfpathmoveto{\pgfqpoint{3.721620in}{4.468366in}}%
\pgfpathlineto{\pgfqpoint{3.799922in}{4.421780in}}%
\pgfpathlineto{\pgfqpoint{3.673363in}{4.323977in}}%
\pgfpathclose%
\pgfusepath{fill}%
\end{pgfscope}%
\begin{pgfscope}%
\pgfpathrectangle{\pgfqpoint{0.539299in}{0.078740in}}{\pgfqpoint{7.842520in}{7.842520in}}%
\pgfusepath{clip}%
\pgfsetbuttcap%
\pgfsetroundjoin%
\definecolor{currentfill}{rgb}{0.281887,0.150881,0.465405}%
\pgfsetfillcolor{currentfill}%
\pgfsetlinewidth{0.000000pt}%
\definecolor{currentstroke}{rgb}{0.273809,0.031497,0.358853}%
\pgfsetstrokecolor{currentstroke}%
\pgfsetdash{}{0pt}%
\pgfpathmoveto{\pgfqpoint{3.721620in}{4.468366in}}%
\pgfpathlineto{\pgfqpoint{3.673363in}{4.323977in}}%
\pgfpathlineto{\pgfqpoint{3.595462in}{4.319772in}}%
\pgfpathclose%
\pgfusepath{fill}%
\end{pgfscope}%
\begin{pgfscope}%
\pgfpathrectangle{\pgfqpoint{0.539299in}{0.078740in}}{\pgfqpoint{7.842520in}{7.842520in}}%
\pgfusepath{clip}%
\pgfsetbuttcap%
\pgfsetroundjoin%
\definecolor{currentfill}{rgb}{0.281887,0.150881,0.465405}%
\pgfsetfillcolor{currentfill}%
\pgfsetlinewidth{0.000000pt}%
\definecolor{currentstroke}{rgb}{0.274952,0.037752,0.364543}%
\pgfsetstrokecolor{currentstroke}%
\pgfsetdash{}{0pt}%
\pgfpathmoveto{\pgfqpoint{4.056700in}{4.461449in}}%
\pgfpathlineto{\pgfqpoint{4.134205in}{4.319900in}}%
\pgfpathlineto{\pgfqpoint{4.005716in}{4.343136in}}%
\pgfpathclose%
\pgfusepath{fill}%
\end{pgfscope}%
\begin{pgfscope}%
\pgfpathrectangle{\pgfqpoint{0.539299in}{0.078740in}}{\pgfqpoint{7.842520in}{7.842520in}}%
\pgfusepath{clip}%
\pgfsetbuttcap%
\pgfsetroundjoin%
\definecolor{currentfill}{rgb}{0.281887,0.150881,0.465405}%
\pgfsetfillcolor{currentfill}%
\pgfsetlinewidth{0.000000pt}%
\definecolor{currentstroke}{rgb}{0.276022,0.044167,0.370164}%
\pgfsetstrokecolor{currentstroke}%
\pgfsetdash{}{0pt}%
\pgfpathmoveto{\pgfqpoint{4.263211in}{4.278418in}}%
\pgfpathlineto{\pgfqpoint{4.134205in}{4.319900in}}%
\pgfpathlineto{\pgfqpoint{4.056700in}{4.461449in}}%
\pgfpathclose%
\pgfusepath{fill}%
\end{pgfscope}%
\begin{pgfscope}%
\pgfpathrectangle{\pgfqpoint{0.539299in}{0.078740in}}{\pgfqpoint{7.842520in}{7.842520in}}%
\pgfusepath{clip}%
\pgfsetbuttcap%
\pgfsetroundjoin%
\definecolor{currentfill}{rgb}{0.278826,0.175490,0.483397}%
\pgfsetfillcolor{currentfill}%
\pgfsetlinewidth{0.000000pt}%
\definecolor{currentstroke}{rgb}{0.277018,0.050344,0.375715}%
\pgfsetstrokecolor{currentstroke}%
\pgfsetdash{}{0pt}%
\pgfpathmoveto{\pgfqpoint{4.056700in}{4.461449in}}%
\pgfpathlineto{\pgfqpoint{4.005716in}{4.343136in}}%
\pgfpathlineto{\pgfqpoint{3.927826in}{4.463428in}}%
\pgfpathclose%
\pgfusepath{fill}%
\end{pgfscope}%
\begin{pgfscope}%
\pgfpathrectangle{\pgfqpoint{0.539299in}{0.078740in}}{\pgfqpoint{7.842520in}{7.842520in}}%
\pgfusepath{clip}%
\pgfsetbuttcap%
\pgfsetroundjoin%
\definecolor{currentfill}{rgb}{0.276194,0.190074,0.493001}%
\pgfsetfillcolor{currentfill}%
\pgfsetlinewidth{0.000000pt}%
\definecolor{currentstroke}{rgb}{0.277941,0.056324,0.381191}%
\pgfsetstrokecolor{currentstroke}%
\pgfsetdash{}{0pt}%
\pgfpathmoveto{\pgfqpoint{3.927826in}{4.463428in}}%
\pgfpathlineto{\pgfqpoint{3.799922in}{4.421780in}}%
\pgfpathlineto{\pgfqpoint{3.721620in}{4.468366in}}%
\pgfpathclose%
\pgfusepath{fill}%
\end{pgfscope}%
\begin{pgfscope}%
\pgfpathrectangle{\pgfqpoint{0.539299in}{0.078740in}}{\pgfqpoint{7.842520in}{7.842520in}}%
\pgfusepath{clip}%
\pgfsetbuttcap%
\pgfsetroundjoin%
\definecolor{currentfill}{rgb}{0.280255,0.165693,0.476498}%
\pgfsetfillcolor{currentfill}%
\pgfsetlinewidth{0.000000pt}%
\definecolor{currentstroke}{rgb}{0.278791,0.062145,0.386592}%
\pgfsetstrokecolor{currentstroke}%
\pgfsetdash{}{0pt}%
\pgfpathmoveto{\pgfqpoint{3.595462in}{4.319772in}}%
\pgfpathlineto{\pgfqpoint{3.517208in}{4.302391in}}%
\pgfpathlineto{\pgfqpoint{3.721620in}{4.468366in}}%
\pgfpathclose%
\pgfusepath{fill}%
\end{pgfscope}%
\begin{pgfscope}%
\pgfpathrectangle{\pgfqpoint{0.539299in}{0.078740in}}{\pgfqpoint{7.842520in}{7.842520in}}%
\pgfusepath{clip}%
\pgfsetbuttcap%
\pgfsetroundjoin%
\definecolor{currentfill}{rgb}{0.281887,0.150881,0.465405}%
\pgfsetfillcolor{currentfill}%
\pgfsetlinewidth{0.000000pt}%
\definecolor{currentstroke}{rgb}{0.279566,0.067836,0.391917}%
\pgfsetstrokecolor{currentstroke}%
\pgfsetdash{}{0pt}%
\pgfpathmoveto{\pgfqpoint{4.186265in}{4.426031in}}%
\pgfpathlineto{\pgfqpoint{4.392627in}{4.224057in}}%
\pgfpathlineto{\pgfqpoint{4.263211in}{4.278418in}}%
\pgfpathclose%
\pgfusepath{fill}%
\end{pgfscope}%
\begin{pgfscope}%
\pgfpathrectangle{\pgfqpoint{0.539299in}{0.078740in}}{\pgfqpoint{7.842520in}{7.842520in}}%
\pgfusepath{clip}%
\pgfsetbuttcap%
\pgfsetroundjoin%
\definecolor{currentfill}{rgb}{0.278012,0.180367,0.486697}%
\pgfsetfillcolor{currentfill}%
\pgfsetlinewidth{0.000000pt}%
\definecolor{currentstroke}{rgb}{0.280267,0.073417,0.397163}%
\pgfsetstrokecolor{currentstroke}%
\pgfsetdash{}{0pt}%
\pgfpathmoveto{\pgfqpoint{4.056700in}{4.461449in}}%
\pgfpathlineto{\pgfqpoint{4.186265in}{4.426031in}}%
\pgfpathlineto{\pgfqpoint{4.263211in}{4.278418in}}%
\pgfpathclose%
\pgfusepath{fill}%
\end{pgfscope}%
\begin{pgfscope}%
\pgfpathrectangle{\pgfqpoint{0.539299in}{0.078740in}}{\pgfqpoint{7.842520in}{7.842520in}}%
\pgfusepath{clip}%
\pgfsetbuttcap%
\pgfsetroundjoin%
\definecolor{currentfill}{rgb}{0.282884,0.135920,0.453427}%
\pgfsetfillcolor{currentfill}%
\pgfsetlinewidth{0.000000pt}%
\definecolor{currentstroke}{rgb}{0.280894,0.078907,0.402329}%
\pgfsetstrokecolor{currentstroke}%
\pgfsetdash{}{0pt}%
\pgfpathmoveto{\pgfqpoint{4.522384in}{4.161514in}}%
\pgfpathlineto{\pgfqpoint{4.392627in}{4.224057in}}%
\pgfpathlineto{\pgfqpoint{4.446717in}{4.289781in}}%
\pgfpathclose%
\pgfusepath{fill}%
\end{pgfscope}%
\begin{pgfscope}%
\pgfpathrectangle{\pgfqpoint{0.539299in}{0.078740in}}{\pgfqpoint{7.842520in}{7.842520in}}%
\pgfusepath{clip}%
\pgfsetbuttcap%
\pgfsetroundjoin%
\definecolor{currentfill}{rgb}{0.267968,0.223549,0.512008}%
\pgfsetfillcolor{currentfill}%
\pgfsetlinewidth{0.000000pt}%
\definecolor{currentstroke}{rgb}{0.281446,0.084320,0.407414}%
\pgfsetstrokecolor{currentstroke}%
\pgfsetdash{}{0pt}%
\pgfpathmoveto{\pgfqpoint{3.721620in}{4.468366in}}%
\pgfpathlineto{\pgfqpoint{3.849491in}{4.547694in}}%
\pgfpathlineto{\pgfqpoint{3.927826in}{4.463428in}}%
\pgfpathclose%
\pgfusepath{fill}%
\end{pgfscope}%
\begin{pgfscope}%
\pgfpathrectangle{\pgfqpoint{0.539299in}{0.078740in}}{\pgfqpoint{7.842520in}{7.842520in}}%
\pgfusepath{clip}%
\pgfsetbuttcap%
\pgfsetroundjoin%
\definecolor{currentfill}{rgb}{0.267968,0.223549,0.512008}%
\pgfsetfillcolor{currentfill}%
\pgfsetlinewidth{0.000000pt}%
\definecolor{currentstroke}{rgb}{0.281924,0.089666,0.412415}%
\pgfsetstrokecolor{currentstroke}%
\pgfsetdash{}{0pt}%
\pgfpathmoveto{\pgfqpoint{3.927826in}{4.463428in}}%
\pgfpathlineto{\pgfqpoint{3.849491in}{4.547694in}}%
\pgfpathlineto{\pgfqpoint{4.056700in}{4.461449in}}%
\pgfpathclose%
\pgfusepath{fill}%
\end{pgfscope}%
\begin{pgfscope}%
\pgfpathrectangle{\pgfqpoint{0.539299in}{0.078740in}}{\pgfqpoint{7.842520in}{7.842520in}}%
\pgfusepath{clip}%
\pgfsetbuttcap%
\pgfsetroundjoin%
\definecolor{currentfill}{rgb}{0.278826,0.175490,0.483397}%
\pgfsetfillcolor{currentfill}%
\pgfsetlinewidth{0.000000pt}%
\definecolor{currentstroke}{rgb}{0.282327,0.094955,0.417331}%
\pgfsetstrokecolor{currentstroke}%
\pgfsetdash{}{0pt}%
\pgfpathmoveto{\pgfqpoint{4.316318in}{4.366115in}}%
\pgfpathlineto{\pgfqpoint{4.392627in}{4.224057in}}%
\pgfpathlineto{\pgfqpoint{4.186265in}{4.426031in}}%
\pgfpathclose%
\pgfusepath{fill}%
\end{pgfscope}%
\begin{pgfscope}%
\pgfpathrectangle{\pgfqpoint{0.539299in}{0.078740in}}{\pgfqpoint{7.842520in}{7.842520in}}%
\pgfusepath{clip}%
\pgfsetbuttcap%
\pgfsetroundjoin%
\definecolor{currentfill}{rgb}{0.273006,0.204520,0.501721}%
\pgfsetfillcolor{currentfill}%
\pgfsetlinewidth{0.000000pt}%
\definecolor{currentstroke}{rgb}{0.282656,0.100196,0.422160}%
\pgfsetstrokecolor{currentstroke}%
\pgfsetdash{}{0pt}%
\pgfpathmoveto{\pgfqpoint{3.517208in}{4.302391in}}%
\pgfpathlineto{\pgfqpoint{3.642937in}{4.493560in}}%
\pgfpathlineto{\pgfqpoint{3.721620in}{4.468366in}}%
\pgfpathclose%
\pgfusepath{fill}%
\end{pgfscope}%
\begin{pgfscope}%
\pgfpathrectangle{\pgfqpoint{0.539299in}{0.078740in}}{\pgfqpoint{7.842520in}{7.842520in}}%
\pgfusepath{clip}%
\pgfsetbuttcap%
\pgfsetroundjoin%
\definecolor{currentfill}{rgb}{0.283187,0.125848,0.444960}%
\pgfsetfillcolor{currentfill}%
\pgfsetlinewidth{0.000000pt}%
\definecolor{currentstroke}{rgb}{0.282910,0.105393,0.426902}%
\pgfsetstrokecolor{currentstroke}%
\pgfsetdash{}{0pt}%
\pgfpathmoveto{\pgfqpoint{4.652446in}{4.094919in}}%
\pgfpathlineto{\pgfqpoint{4.522384in}{4.161514in}}%
\pgfpathlineto{\pgfqpoint{4.446717in}{4.289781in}}%
\pgfpathclose%
\pgfusepath{fill}%
\end{pgfscope}%
\begin{pgfscope}%
\pgfpathrectangle{\pgfqpoint{0.539299in}{0.078740in}}{\pgfqpoint{7.842520in}{7.842520in}}%
\pgfusepath{clip}%
\pgfsetbuttcap%
\pgfsetroundjoin%
\definecolor{currentfill}{rgb}{0.279574,0.170599,0.479997}%
\pgfsetfillcolor{currentfill}%
\pgfsetlinewidth{0.000000pt}%
\definecolor{currentstroke}{rgb}{0.283091,0.110553,0.431554}%
\pgfsetstrokecolor{currentstroke}%
\pgfsetdash{}{0pt}%
\pgfpathmoveto{\pgfqpoint{4.446717in}{4.289781in}}%
\pgfpathlineto{\pgfqpoint{4.392627in}{4.224057in}}%
\pgfpathlineto{\pgfqpoint{4.316318in}{4.366115in}}%
\pgfpathclose%
\pgfusepath{fill}%
\end{pgfscope}%
\begin{pgfscope}%
\pgfpathrectangle{\pgfqpoint{0.539299in}{0.078740in}}{\pgfqpoint{7.842520in}{7.842520in}}%
\pgfusepath{clip}%
\pgfsetbuttcap%
\pgfsetroundjoin%
\definecolor{currentfill}{rgb}{0.276194,0.190074,0.493001}%
\pgfsetfillcolor{currentfill}%
\pgfsetlinewidth{0.000000pt}%
\definecolor{currentstroke}{rgb}{0.283197,0.115680,0.436115}%
\pgfsetstrokecolor{currentstroke}%
\pgfsetdash{}{0pt}%
\pgfpathmoveto{\pgfqpoint{3.642937in}{4.493560in}}%
\pgfpathlineto{\pgfqpoint{3.517208in}{4.302391in}}%
\pgfpathlineto{\pgfqpoint{3.438560in}{4.276751in}}%
\pgfpathclose%
\pgfusepath{fill}%
\end{pgfscope}%
\begin{pgfscope}%
\pgfpathrectangle{\pgfqpoint{0.539299in}{0.078740in}}{\pgfqpoint{7.842520in}{7.842520in}}%
\pgfusepath{clip}%
\pgfsetbuttcap%
\pgfsetroundjoin%
\definecolor{currentfill}{rgb}{0.282910,0.105393,0.426902}%
\pgfsetfillcolor{currentfill}%
\pgfsetlinewidth{0.000000pt}%
\definecolor{currentstroke}{rgb}{0.283229,0.120777,0.440584}%
\pgfsetstrokecolor{currentstroke}%
\pgfsetdash{}{0pt}%
\pgfpathmoveto{\pgfqpoint{4.708270in}{4.115796in}}%
\pgfpathlineto{\pgfqpoint{4.782803in}{4.027779in}}%
\pgfpathlineto{\pgfqpoint{4.652446in}{4.094919in}}%
\pgfpathclose%
\pgfusepath{fill}%
\end{pgfscope}%
\begin{pgfscope}%
\pgfpathrectangle{\pgfqpoint{0.539299in}{0.078740in}}{\pgfqpoint{7.842520in}{7.842520in}}%
\pgfusepath{clip}%
\pgfsetbuttcap%
\pgfsetroundjoin%
\definecolor{currentfill}{rgb}{0.258965,0.251537,0.524736}%
\pgfsetfillcolor{currentfill}%
\pgfsetlinewidth{0.000000pt}%
\definecolor{currentstroke}{rgb}{0.283187,0.125848,0.444960}%
\pgfsetstrokecolor{currentstroke}%
\pgfsetdash{}{0pt}%
\pgfpathmoveto{\pgfqpoint{4.056700in}{4.461449in}}%
\pgfpathlineto{\pgfqpoint{3.849491in}{4.547694in}}%
\pgfpathlineto{\pgfqpoint{3.978618in}{4.570065in}}%
\pgfpathclose%
\pgfusepath{fill}%
\end{pgfscope}%
\begin{pgfscope}%
\pgfpathrectangle{\pgfqpoint{0.539299in}{0.078740in}}{\pgfqpoint{7.842520in}{7.842520in}}%
\pgfusepath{clip}%
\pgfsetbuttcap%
\pgfsetroundjoin%
\definecolor{currentfill}{rgb}{0.282290,0.145912,0.461510}%
\pgfsetfillcolor{currentfill}%
\pgfsetlinewidth{0.000000pt}%
\definecolor{currentstroke}{rgb}{0.283072,0.130895,0.449241}%
\pgfsetstrokecolor{currentstroke}%
\pgfsetdash{}{0pt}%
\pgfpathmoveto{\pgfqpoint{4.577380in}{4.204259in}}%
\pgfpathlineto{\pgfqpoint{4.652446in}{4.094919in}}%
\pgfpathlineto{\pgfqpoint{4.446717in}{4.289781in}}%
\pgfpathclose%
\pgfusepath{fill}%
\end{pgfscope}%
\begin{pgfscope}%
\pgfpathrectangle{\pgfqpoint{0.539299in}{0.078740in}}{\pgfqpoint{7.842520in}{7.842520in}}%
\pgfusepath{clip}%
\pgfsetbuttcap%
\pgfsetroundjoin%
\definecolor{currentfill}{rgb}{0.282327,0.094955,0.417331}%
\pgfsetfillcolor{currentfill}%
\pgfsetlinewidth{0.000000pt}%
\definecolor{currentstroke}{rgb}{0.282884,0.135920,0.453427}%
\pgfsetstrokecolor{currentstroke}%
\pgfsetdash{}{0pt}%
\pgfpathmoveto{\pgfqpoint{4.913467in}{3.962882in}}%
\pgfpathlineto{\pgfqpoint{4.782803in}{4.027779in}}%
\pgfpathlineto{\pgfqpoint{4.708270in}{4.115796in}}%
\pgfpathclose%
\pgfusepath{fill}%
\end{pgfscope}%
\begin{pgfscope}%
\pgfpathrectangle{\pgfqpoint{0.539299in}{0.078740in}}{\pgfqpoint{7.842520in}{7.842520in}}%
\pgfusepath{clip}%
\pgfsetbuttcap%
\pgfsetroundjoin%
\definecolor{currentfill}{rgb}{0.263663,0.237631,0.518762}%
\pgfsetfillcolor{currentfill}%
\pgfsetlinewidth{0.000000pt}%
\definecolor{currentstroke}{rgb}{0.282623,0.140926,0.457517}%
\pgfsetstrokecolor{currentstroke}%
\pgfsetdash{}{0pt}%
\pgfpathmoveto{\pgfqpoint{4.108635in}{4.547005in}}%
\pgfpathlineto{\pgfqpoint{4.186265in}{4.426031in}}%
\pgfpathlineto{\pgfqpoint{4.056700in}{4.461449in}}%
\pgfpathclose%
\pgfusepath{fill}%
\end{pgfscope}%
\begin{pgfscope}%
\pgfpathrectangle{\pgfqpoint{0.539299in}{0.078740in}}{\pgfqpoint{7.842520in}{7.842520in}}%
\pgfusepath{clip}%
\pgfsetbuttcap%
\pgfsetroundjoin%
\definecolor{currentfill}{rgb}{0.255645,0.260703,0.528312}%
\pgfsetfillcolor{currentfill}%
\pgfsetlinewidth{0.000000pt}%
\definecolor{currentstroke}{rgb}{0.282290,0.145912,0.461510}%
\pgfsetstrokecolor{currentstroke}%
\pgfsetdash{}{0pt}%
\pgfpathmoveto{\pgfqpoint{3.770691in}{4.607440in}}%
\pgfpathlineto{\pgfqpoint{3.849491in}{4.547694in}}%
\pgfpathlineto{\pgfqpoint{3.721620in}{4.468366in}}%
\pgfpathclose%
\pgfusepath{fill}%
\end{pgfscope}%
\begin{pgfscope}%
\pgfpathrectangle{\pgfqpoint{0.539299in}{0.078740in}}{\pgfqpoint{7.842520in}{7.842520in}}%
\pgfusepath{clip}%
\pgfsetbuttcap%
\pgfsetroundjoin%
\definecolor{currentfill}{rgb}{0.282884,0.135920,0.453427}%
\pgfsetfillcolor{currentfill}%
\pgfsetlinewidth{0.000000pt}%
\definecolor{currentstroke}{rgb}{0.281887,0.150881,0.465405}%
\pgfsetstrokecolor{currentstroke}%
\pgfsetdash{}{0pt}%
\pgfpathmoveto{\pgfqpoint{4.708270in}{4.115796in}}%
\pgfpathlineto{\pgfqpoint{4.652446in}{4.094919in}}%
\pgfpathlineto{\pgfqpoint{4.577380in}{4.204259in}}%
\pgfpathclose%
\pgfusepath{fill}%
\end{pgfscope}%
\begin{pgfscope}%
\pgfpathrectangle{\pgfqpoint{0.539299in}{0.078740in}}{\pgfqpoint{7.842520in}{7.842520in}}%
\pgfusepath{clip}%
\pgfsetbuttcap%
\pgfsetroundjoin%
\definecolor{currentfill}{rgb}{0.257322,0.256130,0.526563}%
\pgfsetfillcolor{currentfill}%
\pgfsetlinewidth{0.000000pt}%
\definecolor{currentstroke}{rgb}{0.281412,0.155834,0.469201}%
\pgfsetstrokecolor{currentstroke}%
\pgfsetdash{}{0pt}%
\pgfpathmoveto{\pgfqpoint{3.721620in}{4.468366in}}%
\pgfpathlineto{\pgfqpoint{3.642937in}{4.493560in}}%
\pgfpathlineto{\pgfqpoint{3.770691in}{4.607440in}}%
\pgfpathclose%
\pgfusepath{fill}%
\end{pgfscope}%
\begin{pgfscope}%
\pgfpathrectangle{\pgfqpoint{0.539299in}{0.078740in}}{\pgfqpoint{7.842520in}{7.842520in}}%
\pgfusepath{clip}%
\pgfsetbuttcap%
\pgfsetroundjoin%
\definecolor{currentfill}{rgb}{0.280894,0.078907,0.402329}%
\pgfsetfillcolor{currentfill}%
\pgfsetlinewidth{0.000000pt}%
\definecolor{currentstroke}{rgb}{0.280868,0.160771,0.472899}%
\pgfsetstrokecolor{currentstroke}%
\pgfsetdash{}{0pt}%
\pgfpathmoveto{\pgfqpoint{4.839392in}{4.029475in}}%
\pgfpathlineto{\pgfqpoint{5.044466in}{3.902225in}}%
\pgfpathlineto{\pgfqpoint{4.913467in}{3.962882in}}%
\pgfpathclose%
\pgfusepath{fill}%
\end{pgfscope}%
\begin{pgfscope}%
\pgfpathrectangle{\pgfqpoint{0.539299in}{0.078740in}}{\pgfqpoint{7.842520in}{7.842520in}}%
\pgfusepath{clip}%
\pgfsetbuttcap%
\pgfsetroundjoin%
\definecolor{currentfill}{rgb}{0.255645,0.260703,0.528312}%
\pgfsetfillcolor{currentfill}%
\pgfsetlinewidth{0.000000pt}%
\definecolor{currentstroke}{rgb}{0.280255,0.165693,0.476498}%
\pgfsetstrokecolor{currentstroke}%
\pgfsetdash{}{0pt}%
\pgfpathmoveto{\pgfqpoint{3.978618in}{4.570065in}}%
\pgfpathlineto{\pgfqpoint{4.108635in}{4.547005in}}%
\pgfpathlineto{\pgfqpoint{4.056700in}{4.461449in}}%
\pgfpathclose%
\pgfusepath{fill}%
\end{pgfscope}%
\begin{pgfscope}%
\pgfpathrectangle{\pgfqpoint{0.539299in}{0.078740in}}{\pgfqpoint{7.842520in}{7.842520in}}%
\pgfusepath{clip}%
\pgfsetbuttcap%
\pgfsetroundjoin%
\definecolor{currentfill}{rgb}{0.265145,0.232956,0.516599}%
\pgfsetfillcolor{currentfill}%
\pgfsetlinewidth{0.000000pt}%
\definecolor{currentstroke}{rgb}{0.279574,0.170599,0.479997}%
\pgfsetstrokecolor{currentstroke}%
\pgfsetdash{}{0pt}%
\pgfpathmoveto{\pgfqpoint{4.186265in}{4.426031in}}%
\pgfpathlineto{\pgfqpoint{4.239257in}{4.489420in}}%
\pgfpathlineto{\pgfqpoint{4.316318in}{4.366115in}}%
\pgfpathclose%
\pgfusepath{fill}%
\end{pgfscope}%
\begin{pgfscope}%
\pgfpathrectangle{\pgfqpoint{0.539299in}{0.078740in}}{\pgfqpoint{7.842520in}{7.842520in}}%
\pgfusepath{clip}%
\pgfsetbuttcap%
\pgfsetroundjoin%
\definecolor{currentfill}{rgb}{0.265145,0.232956,0.516599}%
\pgfsetfillcolor{currentfill}%
\pgfsetlinewidth{0.000000pt}%
\definecolor{currentstroke}{rgb}{0.278826,0.175490,0.483397}%
\pgfsetstrokecolor{currentstroke}%
\pgfsetdash{}{0pt}%
\pgfpathmoveto{\pgfqpoint{3.438560in}{4.276751in}}%
\pgfpathlineto{\pgfqpoint{3.563833in}{4.504742in}}%
\pgfpathlineto{\pgfqpoint{3.642937in}{4.493560in}}%
\pgfpathclose%
\pgfusepath{fill}%
\end{pgfscope}%
\begin{pgfscope}%
\pgfpathrectangle{\pgfqpoint{0.539299in}{0.078740in}}{\pgfqpoint{7.842520in}{7.842520in}}%
\pgfusepath{clip}%
\pgfsetbuttcap%
\pgfsetroundjoin%
\definecolor{currentfill}{rgb}{0.283091,0.110553,0.431554}%
\pgfsetfillcolor{currentfill}%
\pgfsetlinewidth{0.000000pt}%
\definecolor{currentstroke}{rgb}{0.278012,0.180367,0.486697}%
\pgfsetstrokecolor{currentstroke}%
\pgfsetdash{}{0pt}%
\pgfpathmoveto{\pgfqpoint{4.708270in}{4.115796in}}%
\pgfpathlineto{\pgfqpoint{4.839392in}{4.029475in}}%
\pgfpathlineto{\pgfqpoint{4.913467in}{3.962882in}}%
\pgfpathclose%
\pgfusepath{fill}%
\end{pgfscope}%
\begin{pgfscope}%
\pgfpathrectangle{\pgfqpoint{0.539299in}{0.078740in}}{\pgfqpoint{7.842520in}{7.842520in}}%
\pgfusepath{clip}%
\pgfsetbuttcap%
\pgfsetroundjoin%
\definecolor{currentfill}{rgb}{0.271828,0.209303,0.504434}%
\pgfsetfillcolor{currentfill}%
\pgfsetlinewidth{0.000000pt}%
\definecolor{currentstroke}{rgb}{0.277134,0.185228,0.489898}%
\pgfsetstrokecolor{currentstroke}%
\pgfsetdash{}{0pt}%
\pgfpathmoveto{\pgfqpoint{3.438560in}{4.276751in}}%
\pgfpathlineto{\pgfqpoint{3.359496in}{4.245638in}}%
\pgfpathlineto{\pgfqpoint{3.563833in}{4.504742in}}%
\pgfpathclose%
\pgfusepath{fill}%
\end{pgfscope}%
\begin{pgfscope}%
\pgfpathrectangle{\pgfqpoint{0.539299in}{0.078740in}}{\pgfqpoint{7.842520in}{7.842520in}}%
\pgfusepath{clip}%
\pgfsetbuttcap%
\pgfsetroundjoin%
\definecolor{currentfill}{rgb}{0.267968,0.223549,0.512008}%
\pgfsetfillcolor{currentfill}%
\pgfsetlinewidth{0.000000pt}%
\definecolor{currentstroke}{rgb}{0.276194,0.190074,0.493001}%
\pgfsetstrokecolor{currentstroke}%
\pgfsetdash{}{0pt}%
\pgfpathmoveto{\pgfqpoint{4.239257in}{4.489420in}}%
\pgfpathlineto{\pgfqpoint{4.446717in}{4.289781in}}%
\pgfpathlineto{\pgfqpoint{4.316318in}{4.366115in}}%
\pgfpathclose%
\pgfusepath{fill}%
\end{pgfscope}%
\begin{pgfscope}%
\pgfpathrectangle{\pgfqpoint{0.539299in}{0.078740in}}{\pgfqpoint{7.842520in}{7.842520in}}%
\pgfusepath{clip}%
\pgfsetbuttcap%
\pgfsetroundjoin%
\definecolor{currentfill}{rgb}{0.278791,0.062145,0.386592}%
\pgfsetfillcolor{currentfill}%
\pgfsetlinewidth{0.000000pt}%
\definecolor{currentstroke}{rgb}{0.275191,0.194905,0.496005}%
\pgfsetstrokecolor{currentstroke}%
\pgfsetdash{}{0pt}%
\pgfpathmoveto{\pgfqpoint{5.175833in}{3.846987in}}%
\pgfpathlineto{\pgfqpoint{5.044466in}{3.902225in}}%
\pgfpathlineto{\pgfqpoint{5.102484in}{3.877072in}}%
\pgfpathclose%
\pgfusepath{fill}%
\end{pgfscope}%
\begin{pgfscope}%
\pgfpathrectangle{\pgfqpoint{0.539299in}{0.078740in}}{\pgfqpoint{7.842520in}{7.842520in}}%
\pgfusepath{clip}%
\pgfsetbuttcap%
\pgfsetroundjoin%
\definecolor{currentfill}{rgb}{0.255645,0.260703,0.528312}%
\pgfsetfillcolor{currentfill}%
\pgfsetlinewidth{0.000000pt}%
\definecolor{currentstroke}{rgb}{0.274128,0.199721,0.498911}%
\pgfsetstrokecolor{currentstroke}%
\pgfsetdash{}{0pt}%
\pgfpathmoveto{\pgfqpoint{4.108635in}{4.547005in}}%
\pgfpathlineto{\pgfqpoint{4.239257in}{4.489420in}}%
\pgfpathlineto{\pgfqpoint{4.186265in}{4.426031in}}%
\pgfpathclose%
\pgfusepath{fill}%
\end{pgfscope}%
\begin{pgfscope}%
\pgfpathrectangle{\pgfqpoint{0.539299in}{0.078740in}}{\pgfqpoint{7.842520in}{7.842520in}}%
\pgfusepath{clip}%
\pgfsetbuttcap%
\pgfsetroundjoin%
\definecolor{currentfill}{rgb}{0.282327,0.094955,0.417331}%
\pgfsetfillcolor{currentfill}%
\pgfsetlinewidth{0.000000pt}%
\definecolor{currentstroke}{rgb}{0.273006,0.204520,0.501721}%
\pgfsetstrokecolor{currentstroke}%
\pgfsetdash{}{0pt}%
\pgfpathmoveto{\pgfqpoint{4.970779in}{3.949088in}}%
\pgfpathlineto{\pgfqpoint{5.044466in}{3.902225in}}%
\pgfpathlineto{\pgfqpoint{4.839392in}{4.029475in}}%
\pgfpathclose%
\pgfusepath{fill}%
\end{pgfscope}%
\begin{pgfscope}%
\pgfpathrectangle{\pgfqpoint{0.539299in}{0.078740in}}{\pgfqpoint{7.842520in}{7.842520in}}%
\pgfusepath{clip}%
\pgfsetbuttcap%
\pgfsetroundjoin%
\definecolor{currentfill}{rgb}{0.243113,0.292092,0.538516}%
\pgfsetfillcolor{currentfill}%
\pgfsetlinewidth{0.000000pt}%
\definecolor{currentstroke}{rgb}{0.271828,0.209303,0.504434}%
\pgfsetstrokecolor{currentstroke}%
\pgfsetdash{}{0pt}%
\pgfpathmoveto{\pgfqpoint{3.978618in}{4.570065in}}%
\pgfpathlineto{\pgfqpoint{3.849491in}{4.547694in}}%
\pgfpathlineto{\pgfqpoint{3.899971in}{4.655078in}}%
\pgfpathclose%
\pgfusepath{fill}%
\end{pgfscope}%
\begin{pgfscope}%
\pgfpathrectangle{\pgfqpoint{0.539299in}{0.078740in}}{\pgfqpoint{7.842520in}{7.842520in}}%
\pgfusepath{clip}%
\pgfsetbuttcap%
\pgfsetroundjoin%
\definecolor{currentfill}{rgb}{0.275191,0.194905,0.496005}%
\pgfsetfillcolor{currentfill}%
\pgfsetlinewidth{0.000000pt}%
\definecolor{currentstroke}{rgb}{0.270595,0.214069,0.507052}%
\pgfsetstrokecolor{currentstroke}%
\pgfsetdash{}{0pt}%
\pgfpathmoveto{\pgfqpoint{4.577380in}{4.204259in}}%
\pgfpathlineto{\pgfqpoint{4.446717in}{4.289781in}}%
\pgfpathlineto{\pgfqpoint{4.501556in}{4.310839in}}%
\pgfpathclose%
\pgfusepath{fill}%
\end{pgfscope}%
\begin{pgfscope}%
\pgfpathrectangle{\pgfqpoint{0.539299in}{0.078740in}}{\pgfqpoint{7.842520in}{7.842520in}}%
\pgfusepath{clip}%
\pgfsetbuttcap%
\pgfsetroundjoin%
\definecolor{currentfill}{rgb}{0.277941,0.056324,0.381191}%
\pgfsetfillcolor{currentfill}%
\pgfsetlinewidth{0.000000pt}%
\definecolor{currentstroke}{rgb}{0.269308,0.218818,0.509577}%
\pgfsetstrokecolor{currentstroke}%
\pgfsetdash{}{0pt}%
\pgfpathmoveto{\pgfqpoint{5.102484in}{3.877072in}}%
\pgfpathlineto{\pgfqpoint{5.307605in}{3.797555in}}%
\pgfpathlineto{\pgfqpoint{5.175833in}{3.846987in}}%
\pgfpathclose%
\pgfusepath{fill}%
\end{pgfscope}%
\begin{pgfscope}%
\pgfpathrectangle{\pgfqpoint{0.539299in}{0.078740in}}{\pgfqpoint{7.842520in}{7.842520in}}%
\pgfusepath{clip}%
\pgfsetbuttcap%
\pgfsetroundjoin%
\definecolor{currentfill}{rgb}{0.281446,0.084320,0.407414}%
\pgfsetfillcolor{currentfill}%
\pgfsetlinewidth{0.000000pt}%
\definecolor{currentstroke}{rgb}{0.267968,0.223549,0.512008}%
\pgfsetstrokecolor{currentstroke}%
\pgfsetdash{}{0pt}%
\pgfpathmoveto{\pgfqpoint{5.102484in}{3.877072in}}%
\pgfpathlineto{\pgfqpoint{5.044466in}{3.902225in}}%
\pgfpathlineto{\pgfqpoint{4.970779in}{3.949088in}}%
\pgfpathclose%
\pgfusepath{fill}%
\end{pgfscope}%
\begin{pgfscope}%
\pgfpathrectangle{\pgfqpoint{0.539299in}{0.078740in}}{\pgfqpoint{7.842520in}{7.842520in}}%
\pgfusepath{clip}%
\pgfsetbuttcap%
\pgfsetroundjoin%
\definecolor{currentfill}{rgb}{0.239346,0.300855,0.540844}%
\pgfsetfillcolor{currentfill}%
\pgfsetlinewidth{0.000000pt}%
\definecolor{currentstroke}{rgb}{0.266580,0.228262,0.514349}%
\pgfsetstrokecolor{currentstroke}%
\pgfsetdash{}{0pt}%
\pgfpathmoveto{\pgfqpoint{3.899971in}{4.655078in}}%
\pgfpathlineto{\pgfqpoint{3.849491in}{4.547694in}}%
\pgfpathlineto{\pgfqpoint{3.770691in}{4.607440in}}%
\pgfpathclose%
\pgfusepath{fill}%
\end{pgfscope}%
\begin{pgfscope}%
\pgfpathrectangle{\pgfqpoint{0.539299in}{0.078740in}}{\pgfqpoint{7.842520in}{7.842520in}}%
\pgfusepath{clip}%
\pgfsetbuttcap%
\pgfsetroundjoin%
\definecolor{currentfill}{rgb}{0.277134,0.185228,0.489898}%
\pgfsetfillcolor{currentfill}%
\pgfsetlinewidth{0.000000pt}%
\definecolor{currentstroke}{rgb}{0.265145,0.232956,0.516599}%
\pgfsetstrokecolor{currentstroke}%
\pgfsetdash{}{0pt}%
\pgfpathmoveto{\pgfqpoint{4.577380in}{4.204259in}}%
\pgfpathlineto{\pgfqpoint{4.501556in}{4.310839in}}%
\pgfpathlineto{\pgfqpoint{4.708270in}{4.115796in}}%
\pgfpathclose%
\pgfusepath{fill}%
\end{pgfscope}%
\begin{pgfscope}%
\pgfpathrectangle{\pgfqpoint{0.539299in}{0.078740in}}{\pgfqpoint{7.842520in}{7.842520in}}%
\pgfusepath{clip}%
\pgfsetbuttcap%
\pgfsetroundjoin%
\definecolor{currentfill}{rgb}{0.262138,0.242286,0.520837}%
\pgfsetfillcolor{currentfill}%
\pgfsetlinewidth{0.000000pt}%
\definecolor{currentstroke}{rgb}{0.263663,0.237631,0.518762}%
\pgfsetstrokecolor{currentstroke}%
\pgfsetdash{}{0pt}%
\pgfpathmoveto{\pgfqpoint{4.370276in}{4.407562in}}%
\pgfpathlineto{\pgfqpoint{4.446717in}{4.289781in}}%
\pgfpathlineto{\pgfqpoint{4.239257in}{4.489420in}}%
\pgfpathclose%
\pgfusepath{fill}%
\end{pgfscope}%
\begin{pgfscope}%
\pgfpathrectangle{\pgfqpoint{0.539299in}{0.078740in}}{\pgfqpoint{7.842520in}{7.842520in}}%
\pgfusepath{clip}%
\pgfsetbuttcap%
\pgfsetroundjoin%
\definecolor{currentfill}{rgb}{0.239346,0.300855,0.540844}%
\pgfsetfillcolor{currentfill}%
\pgfsetlinewidth{0.000000pt}%
\definecolor{currentstroke}{rgb}{0.262138,0.242286,0.520837}%
\pgfsetstrokecolor{currentstroke}%
\pgfsetdash{}{0pt}%
\pgfpathmoveto{\pgfqpoint{3.899971in}{4.655078in}}%
\pgfpathlineto{\pgfqpoint{4.108635in}{4.547005in}}%
\pgfpathlineto{\pgfqpoint{3.978618in}{4.570065in}}%
\pgfpathclose%
\pgfusepath{fill}%
\end{pgfscope}%
\begin{pgfscope}%
\pgfpathrectangle{\pgfqpoint{0.539299in}{0.078740in}}{\pgfqpoint{7.842520in}{7.842520in}}%
\pgfusepath{clip}%
\pgfsetbuttcap%
\pgfsetroundjoin%
\definecolor{currentfill}{rgb}{0.239346,0.300855,0.540844}%
\pgfsetfillcolor{currentfill}%
\pgfsetlinewidth{0.000000pt}%
\definecolor{currentstroke}{rgb}{0.260571,0.246922,0.522828}%
\pgfsetstrokecolor{currentstroke}%
\pgfsetdash{}{0pt}%
\pgfpathmoveto{\pgfqpoint{3.642937in}{4.493560in}}%
\pgfpathlineto{\pgfqpoint{3.691406in}{4.650385in}}%
\pgfpathlineto{\pgfqpoint{3.770691in}{4.607440in}}%
\pgfpathclose%
\pgfusepath{fill}%
\end{pgfscope}%
\begin{pgfscope}%
\pgfpathrectangle{\pgfqpoint{0.539299in}{0.078740in}}{\pgfqpoint{7.842520in}{7.842520in}}%
\pgfusepath{clip}%
\pgfsetbuttcap%
\pgfsetroundjoin%
\definecolor{currentfill}{rgb}{0.243113,0.292092,0.538516}%
\pgfsetfillcolor{currentfill}%
\pgfsetlinewidth{0.000000pt}%
\definecolor{currentstroke}{rgb}{0.258965,0.251537,0.524736}%
\pgfsetstrokecolor{currentstroke}%
\pgfsetdash{}{0pt}%
\pgfpathmoveto{\pgfqpoint{3.563833in}{4.504742in}}%
\pgfpathlineto{\pgfqpoint{3.691406in}{4.650385in}}%
\pgfpathlineto{\pgfqpoint{3.642937in}{4.493560in}}%
\pgfpathclose%
\pgfusepath{fill}%
\end{pgfscope}%
\begin{pgfscope}%
\pgfpathrectangle{\pgfqpoint{0.539299in}{0.078740in}}{\pgfqpoint{7.842520in}{7.842520in}}%
\pgfusepath{clip}%
\pgfsetbuttcap%
\pgfsetroundjoin%
\definecolor{currentfill}{rgb}{0.266580,0.228262,0.514349}%
\pgfsetfillcolor{currentfill}%
\pgfsetlinewidth{0.000000pt}%
\definecolor{currentstroke}{rgb}{0.257322,0.256130,0.526563}%
\pgfsetstrokecolor{currentstroke}%
\pgfsetdash{}{0pt}%
\pgfpathmoveto{\pgfqpoint{4.501556in}{4.310839in}}%
\pgfpathlineto{\pgfqpoint{4.446717in}{4.289781in}}%
\pgfpathlineto{\pgfqpoint{4.370276in}{4.407562in}}%
\pgfpathclose%
\pgfusepath{fill}%
\end{pgfscope}%
\begin{pgfscope}%
\pgfpathrectangle{\pgfqpoint{0.539299in}{0.078740in}}{\pgfqpoint{7.842520in}{7.842520in}}%
\pgfusepath{clip}%
\pgfsetbuttcap%
\pgfsetroundjoin%
\definecolor{currentfill}{rgb}{0.257322,0.256130,0.526563}%
\pgfsetfillcolor{currentfill}%
\pgfsetlinewidth{0.000000pt}%
\definecolor{currentstroke}{rgb}{0.255645,0.260703,0.528312}%
\pgfsetstrokecolor{currentstroke}%
\pgfsetdash{}{0pt}%
\pgfpathmoveto{\pgfqpoint{3.563833in}{4.504742in}}%
\pgfpathlineto{\pgfqpoint{3.359496in}{4.245638in}}%
\pgfpathlineto{\pgfqpoint{3.484281in}{4.506457in}}%
\pgfpathclose%
\pgfusepath{fill}%
\end{pgfscope}%
\begin{pgfscope}%
\pgfpathrectangle{\pgfqpoint{0.539299in}{0.078740in}}{\pgfqpoint{7.842520in}{7.842520in}}%
\pgfusepath{clip}%
\pgfsetbuttcap%
\pgfsetroundjoin%
\definecolor{currentfill}{rgb}{0.281887,0.150881,0.465405}%
\pgfsetfillcolor{currentfill}%
\pgfsetlinewidth{0.000000pt}%
\definecolor{currentstroke}{rgb}{0.253935,0.265254,0.529983}%
\pgfsetstrokecolor{currentstroke}%
\pgfsetdash{}{0pt}%
\pgfpathmoveto{\pgfqpoint{4.839392in}{4.029475in}}%
\pgfpathlineto{\pgfqpoint{4.708270in}{4.115796in}}%
\pgfpathlineto{\pgfqpoint{4.764671in}{4.104606in}}%
\pgfpathclose%
\pgfusepath{fill}%
\end{pgfscope}%
\begin{pgfscope}%
\pgfpathrectangle{\pgfqpoint{0.539299in}{0.078740in}}{\pgfqpoint{7.842520in}{7.842520in}}%
\pgfusepath{clip}%
\pgfsetbuttcap%
\pgfsetroundjoin%
\definecolor{currentfill}{rgb}{0.266580,0.228262,0.514349}%
\pgfsetfillcolor{currentfill}%
\pgfsetlinewidth{0.000000pt}%
\definecolor{currentstroke}{rgb}{0.252194,0.269783,0.531579}%
\pgfsetstrokecolor{currentstroke}%
\pgfsetdash{}{0pt}%
\pgfpathmoveto{\pgfqpoint{3.484281in}{4.506457in}}%
\pgfpathlineto{\pgfqpoint{3.359496in}{4.245638in}}%
\pgfpathlineto{\pgfqpoint{3.279998in}{4.210712in}}%
\pgfpathclose%
\pgfusepath{fill}%
\end{pgfscope}%
\begin{pgfscope}%
\pgfpathrectangle{\pgfqpoint{0.539299in}{0.078740in}}{\pgfqpoint{7.842520in}{7.842520in}}%
\pgfusepath{clip}%
\pgfsetbuttcap%
\pgfsetroundjoin%
\definecolor{currentfill}{rgb}{0.279566,0.067836,0.391917}%
\pgfsetfillcolor{currentfill}%
\pgfsetlinewidth{0.000000pt}%
\definecolor{currentstroke}{rgb}{0.250425,0.274290,0.533103}%
\pgfsetstrokecolor{currentstroke}%
\pgfsetdash{}{0pt}%
\pgfpathmoveto{\pgfqpoint{5.234563in}{3.814537in}}%
\pgfpathlineto{\pgfqpoint{5.307605in}{3.797555in}}%
\pgfpathlineto{\pgfqpoint{5.102484in}{3.877072in}}%
\pgfpathclose%
\pgfusepath{fill}%
\end{pgfscope}%
\begin{pgfscope}%
\pgfpathrectangle{\pgfqpoint{0.539299in}{0.078740in}}{\pgfqpoint{7.842520in}{7.842520in}}%
\pgfusepath{clip}%
\pgfsetbuttcap%
\pgfsetroundjoin%
\definecolor{currentfill}{rgb}{0.276022,0.044167,0.370164}%
\pgfsetfillcolor{currentfill}%
\pgfsetlinewidth{0.000000pt}%
\definecolor{currentstroke}{rgb}{0.248629,0.278775,0.534556}%
\pgfsetstrokecolor{currentstroke}%
\pgfsetdash{}{0pt}%
\pgfpathmoveto{\pgfqpoint{5.439811in}{3.753615in}}%
\pgfpathlineto{\pgfqpoint{5.307605in}{3.797555in}}%
\pgfpathlineto{\pgfqpoint{5.367069in}{3.761390in}}%
\pgfpathclose%
\pgfusepath{fill}%
\end{pgfscope}%
\begin{pgfscope}%
\pgfpathrectangle{\pgfqpoint{0.539299in}{0.078740in}}{\pgfqpoint{7.842520in}{7.842520in}}%
\pgfusepath{clip}%
\pgfsetbuttcap%
\pgfsetroundjoin%
\definecolor{currentfill}{rgb}{0.282884,0.135920,0.453427}%
\pgfsetfillcolor{currentfill}%
\pgfsetlinewidth{0.000000pt}%
\definecolor{currentstroke}{rgb}{0.246811,0.283237,0.535941}%
\pgfsetstrokecolor{currentstroke}%
\pgfsetdash{}{0pt}%
\pgfpathmoveto{\pgfqpoint{4.970779in}{3.949088in}}%
\pgfpathlineto{\pgfqpoint{4.839392in}{4.029475in}}%
\pgfpathlineto{\pgfqpoint{4.764671in}{4.104606in}}%
\pgfpathclose%
\pgfusepath{fill}%
\end{pgfscope}%
\begin{pgfscope}%
\pgfpathrectangle{\pgfqpoint{0.539299in}{0.078740in}}{\pgfqpoint{7.842520in}{7.842520in}}%
\pgfusepath{clip}%
\pgfsetbuttcap%
\pgfsetroundjoin%
\definecolor{currentfill}{rgb}{0.274128,0.199721,0.498911}%
\pgfsetfillcolor{currentfill}%
\pgfsetlinewidth{0.000000pt}%
\definecolor{currentstroke}{rgb}{0.244972,0.287675,0.537260}%
\pgfsetstrokecolor{currentstroke}%
\pgfsetdash{}{0pt}%
\pgfpathmoveto{\pgfqpoint{4.501556in}{4.310839in}}%
\pgfpathlineto{\pgfqpoint{4.633026in}{4.207547in}}%
\pgfpathlineto{\pgfqpoint{4.708270in}{4.115796in}}%
\pgfpathclose%
\pgfusepath{fill}%
\end{pgfscope}%
\begin{pgfscope}%
\pgfpathrectangle{\pgfqpoint{0.539299in}{0.078740in}}{\pgfqpoint{7.842520in}{7.842520in}}%
\pgfusepath{clip}%
\pgfsetbuttcap%
\pgfsetroundjoin%
\definecolor{currentfill}{rgb}{0.278791,0.062145,0.386592}%
\pgfsetfillcolor{currentfill}%
\pgfsetlinewidth{0.000000pt}%
\definecolor{currentstroke}{rgb}{0.243113,0.292092,0.538516}%
\pgfsetstrokecolor{currentstroke}%
\pgfsetdash{}{0pt}%
\pgfpathmoveto{\pgfqpoint{5.367069in}{3.761390in}}%
\pgfpathlineto{\pgfqpoint{5.307605in}{3.797555in}}%
\pgfpathlineto{\pgfqpoint{5.234563in}{3.814537in}}%
\pgfpathclose%
\pgfusepath{fill}%
\end{pgfscope}%
\begin{pgfscope}%
\pgfpathrectangle{\pgfqpoint{0.539299in}{0.078740in}}{\pgfqpoint{7.842520in}{7.842520in}}%
\pgfusepath{clip}%
\pgfsetbuttcap%
\pgfsetroundjoin%
\definecolor{currentfill}{rgb}{0.278012,0.180367,0.486697}%
\pgfsetfillcolor{currentfill}%
\pgfsetlinewidth{0.000000pt}%
\definecolor{currentstroke}{rgb}{0.241237,0.296485,0.539709}%
\pgfsetstrokecolor{currentstroke}%
\pgfsetdash{}{0pt}%
\pgfpathmoveto{\pgfqpoint{4.764671in}{4.104606in}}%
\pgfpathlineto{\pgfqpoint{4.708270in}{4.115796in}}%
\pgfpathlineto{\pgfqpoint{4.633026in}{4.207547in}}%
\pgfpathclose%
\pgfusepath{fill}%
\end{pgfscope}%
\begin{pgfscope}%
\pgfpathrectangle{\pgfqpoint{0.539299in}{0.078740in}}{\pgfqpoint{7.842520in}{7.842520in}}%
\pgfusepath{clip}%
\pgfsetbuttcap%
\pgfsetroundjoin%
\definecolor{currentfill}{rgb}{0.276022,0.044167,0.370164}%
\pgfsetfillcolor{currentfill}%
\pgfsetlinewidth{0.000000pt}%
\definecolor{currentstroke}{rgb}{0.239346,0.300855,0.540844}%
\pgfsetstrokecolor{currentstroke}%
\pgfsetdash{}{0pt}%
\pgfpathmoveto{\pgfqpoint{5.367069in}{3.761390in}}%
\pgfpathlineto{\pgfqpoint{5.572472in}{3.714282in}}%
\pgfpathlineto{\pgfqpoint{5.439811in}{3.753615in}}%
\pgfpathclose%
\pgfusepath{fill}%
\end{pgfscope}%
\begin{pgfscope}%
\pgfpathrectangle{\pgfqpoint{0.539299in}{0.078740in}}{\pgfqpoint{7.842520in}{7.842520in}}%
\pgfusepath{clip}%
\pgfsetbuttcap%
\pgfsetroundjoin%
\definecolor{currentfill}{rgb}{0.237441,0.305202,0.541921}%
\pgfsetfillcolor{currentfill}%
\pgfsetlinewidth{0.000000pt}%
\definecolor{currentstroke}{rgb}{0.237441,0.305202,0.541921}%
\pgfsetstrokecolor{currentstroke}%
\pgfsetdash{}{0pt}%
\pgfpathmoveto{\pgfqpoint{4.161473in}{4.598110in}}%
\pgfpathlineto{\pgfqpoint{4.239257in}{4.489420in}}%
\pgfpathlineto{\pgfqpoint{4.108635in}{4.547005in}}%
\pgfpathclose%
\pgfusepath{fill}%
\end{pgfscope}%
\begin{pgfscope}%
\pgfpathrectangle{\pgfqpoint{0.539299in}{0.078740in}}{\pgfqpoint{7.842520in}{7.842520in}}%
\pgfusepath{clip}%
\pgfsetbuttcap%
\pgfsetroundjoin%
\definecolor{currentfill}{rgb}{0.225863,0.330805,0.547314}%
\pgfsetfillcolor{currentfill}%
\pgfsetlinewidth{0.000000pt}%
\definecolor{currentstroke}{rgb}{0.235526,0.309527,0.542944}%
\pgfsetstrokecolor{currentstroke}%
\pgfsetdash{}{0pt}%
\pgfpathmoveto{\pgfqpoint{3.770691in}{4.607440in}}%
\pgfpathlineto{\pgfqpoint{3.691406in}{4.650385in}}%
\pgfpathlineto{\pgfqpoint{3.899971in}{4.655078in}}%
\pgfpathclose%
\pgfusepath{fill}%
\end{pgfscope}%
\begin{pgfscope}%
\pgfpathrectangle{\pgfqpoint{0.539299in}{0.078740in}}{\pgfqpoint{7.842520in}{7.842520in}}%
\pgfusepath{clip}%
\pgfsetbuttcap%
\pgfsetroundjoin%
\definecolor{currentfill}{rgb}{0.229739,0.322361,0.545706}%
\pgfsetfillcolor{currentfill}%
\pgfsetlinewidth{0.000000pt}%
\definecolor{currentstroke}{rgb}{0.233603,0.313828,0.543914}%
\pgfsetstrokecolor{currentstroke}%
\pgfsetdash{}{0pt}%
\pgfpathmoveto{\pgfqpoint{3.899971in}{4.655078in}}%
\pgfpathlineto{\pgfqpoint{4.030348in}{4.648062in}}%
\pgfpathlineto{\pgfqpoint{4.108635in}{4.547005in}}%
\pgfpathclose%
\pgfusepath{fill}%
\end{pgfscope}%
\begin{pgfscope}%
\pgfpathrectangle{\pgfqpoint{0.539299in}{0.078740in}}{\pgfqpoint{7.842520in}{7.842520in}}%
\pgfusepath{clip}%
\pgfsetbuttcap%
\pgfsetroundjoin%
\definecolor{currentfill}{rgb}{0.283091,0.110553,0.431554}%
\pgfsetfillcolor{currentfill}%
\pgfsetlinewidth{0.000000pt}%
\definecolor{currentstroke}{rgb}{0.231674,0.318106,0.544834}%
\pgfsetstrokecolor{currentstroke}%
\pgfsetdash{}{0pt}%
\pgfpathmoveto{\pgfqpoint{5.028622in}{3.919505in}}%
\pgfpathlineto{\pgfqpoint{5.102484in}{3.877072in}}%
\pgfpathlineto{\pgfqpoint{4.970779in}{3.949088in}}%
\pgfpathclose%
\pgfusepath{fill}%
\end{pgfscope}%
\begin{pgfscope}%
\pgfpathrectangle{\pgfqpoint{0.539299in}{0.078740in}}{\pgfqpoint{7.842520in}{7.842520in}}%
\pgfusepath{clip}%
\pgfsetbuttcap%
\pgfsetroundjoin%
\definecolor{currentfill}{rgb}{0.281887,0.150881,0.465405}%
\pgfsetfillcolor{currentfill}%
\pgfsetlinewidth{0.000000pt}%
\definecolor{currentstroke}{rgb}{0.229739,0.322361,0.545706}%
\pgfsetstrokecolor{currentstroke}%
\pgfsetdash{}{0pt}%
\pgfpathmoveto{\pgfqpoint{4.896518in}{4.007367in}}%
\pgfpathlineto{\pgfqpoint{4.970779in}{3.949088in}}%
\pgfpathlineto{\pgfqpoint{4.764671in}{4.104606in}}%
\pgfpathclose%
\pgfusepath{fill}%
\end{pgfscope}%
\begin{pgfscope}%
\pgfpathrectangle{\pgfqpoint{0.539299in}{0.078740in}}{\pgfqpoint{7.842520in}{7.842520in}}%
\pgfusepath{clip}%
\pgfsetbuttcap%
\pgfsetroundjoin%
\definecolor{currentfill}{rgb}{0.282656,0.100196,0.422160}%
\pgfsetfillcolor{currentfill}%
\pgfsetlinewidth{0.000000pt}%
\definecolor{currentstroke}{rgb}{0.227802,0.326594,0.546532}%
\pgfsetstrokecolor{currentstroke}%
\pgfsetdash{}{0pt}%
\pgfpathmoveto{\pgfqpoint{5.234563in}{3.814537in}}%
\pgfpathlineto{\pgfqpoint{5.102484in}{3.877072in}}%
\pgfpathlineto{\pgfqpoint{5.028622in}{3.919505in}}%
\pgfpathclose%
\pgfusepath{fill}%
\end{pgfscope}%
\begin{pgfscope}%
\pgfpathrectangle{\pgfqpoint{0.539299in}{0.078740in}}{\pgfqpoint{7.842520in}{7.842520in}}%
\pgfusepath{clip}%
\pgfsetbuttcap%
\pgfsetroundjoin%
\definecolor{currentfill}{rgb}{0.244972,0.287675,0.537260}%
\pgfsetfillcolor{currentfill}%
\pgfsetlinewidth{0.000000pt}%
\definecolor{currentstroke}{rgb}{0.225863,0.330805,0.547314}%
\pgfsetstrokecolor{currentstroke}%
\pgfsetdash{}{0pt}%
\pgfpathmoveto{\pgfqpoint{4.370276in}{4.407562in}}%
\pgfpathlineto{\pgfqpoint{4.239257in}{4.489420in}}%
\pgfpathlineto{\pgfqpoint{4.293076in}{4.516692in}}%
\pgfpathclose%
\pgfusepath{fill}%
\end{pgfscope}%
\begin{pgfscope}%
\pgfpathrectangle{\pgfqpoint{0.539299in}{0.078740in}}{\pgfqpoint{7.842520in}{7.842520in}}%
\pgfusepath{clip}%
\pgfsetbuttcap%
\pgfsetroundjoin%
\definecolor{currentfill}{rgb}{0.227802,0.326594,0.546532}%
\pgfsetfillcolor{currentfill}%
\pgfsetlinewidth{0.000000pt}%
\definecolor{currentstroke}{rgb}{0.223925,0.334994,0.548053}%
\pgfsetstrokecolor{currentstroke}%
\pgfsetdash{}{0pt}%
\pgfpathmoveto{\pgfqpoint{4.108635in}{4.547005in}}%
\pgfpathlineto{\pgfqpoint{4.030348in}{4.648062in}}%
\pgfpathlineto{\pgfqpoint{4.161473in}{4.598110in}}%
\pgfpathclose%
\pgfusepath{fill}%
\end{pgfscope}%
\begin{pgfscope}%
\pgfpathrectangle{\pgfqpoint{0.539299in}{0.078740in}}{\pgfqpoint{7.842520in}{7.842520in}}%
\pgfusepath{clip}%
\pgfsetbuttcap%
\pgfsetroundjoin%
\definecolor{currentfill}{rgb}{0.282884,0.135920,0.453427}%
\pgfsetfillcolor{currentfill}%
\pgfsetlinewidth{0.000000pt}%
\definecolor{currentstroke}{rgb}{0.221989,0.339161,0.548752}%
\pgfsetstrokecolor{currentstroke}%
\pgfsetdash{}{0pt}%
\pgfpathmoveto{\pgfqpoint{5.028622in}{3.919505in}}%
\pgfpathlineto{\pgfqpoint{4.970779in}{3.949088in}}%
\pgfpathlineto{\pgfqpoint{4.896518in}{4.007367in}}%
\pgfpathclose%
\pgfusepath{fill}%
\end{pgfscope}%
\begin{pgfscope}%
\pgfpathrectangle{\pgfqpoint{0.539299in}{0.078740in}}{\pgfqpoint{7.842520in}{7.842520in}}%
\pgfusepath{clip}%
\pgfsetbuttcap%
\pgfsetroundjoin%
\definecolor{currentfill}{rgb}{0.277941,0.056324,0.381191}%
\pgfsetfillcolor{currentfill}%
\pgfsetlinewidth{0.000000pt}%
\definecolor{currentstroke}{rgb}{0.220057,0.343307,0.549413}%
\pgfsetstrokecolor{currentstroke}%
\pgfsetdash{}{0pt}%
\pgfpathmoveto{\pgfqpoint{5.500041in}{3.716527in}}%
\pgfpathlineto{\pgfqpoint{5.572472in}{3.714282in}}%
\pgfpathlineto{\pgfqpoint{5.367069in}{3.761390in}}%
\pgfpathclose%
\pgfusepath{fill}%
\end{pgfscope}%
\begin{pgfscope}%
\pgfpathrectangle{\pgfqpoint{0.539299in}{0.078740in}}{\pgfqpoint{7.842520in}{7.842520in}}%
\pgfusepath{clip}%
\pgfsetbuttcap%
\pgfsetroundjoin%
\definecolor{currentfill}{rgb}{0.250425,0.274290,0.533103}%
\pgfsetfillcolor{currentfill}%
\pgfsetlinewidth{0.000000pt}%
\definecolor{currentstroke}{rgb}{0.218130,0.347432,0.550038}%
\pgfsetstrokecolor{currentstroke}%
\pgfsetdash{}{0pt}%
\pgfpathmoveto{\pgfqpoint{4.293076in}{4.516692in}}%
\pgfpathlineto{\pgfqpoint{4.501556in}{4.310839in}}%
\pgfpathlineto{\pgfqpoint{4.370276in}{4.407562in}}%
\pgfpathclose%
\pgfusepath{fill}%
\end{pgfscope}%
\begin{pgfscope}%
\pgfpathrectangle{\pgfqpoint{0.539299in}{0.078740in}}{\pgfqpoint{7.842520in}{7.842520in}}%
\pgfusepath{clip}%
\pgfsetbuttcap%
\pgfsetroundjoin%
\definecolor{currentfill}{rgb}{0.231674,0.318106,0.544834}%
\pgfsetfillcolor{currentfill}%
\pgfsetlinewidth{0.000000pt}%
\definecolor{currentstroke}{rgb}{0.216210,0.351535,0.550627}%
\pgfsetstrokecolor{currentstroke}%
\pgfsetdash{}{0pt}%
\pgfpathmoveto{\pgfqpoint{3.611618in}{4.681644in}}%
\pgfpathlineto{\pgfqpoint{3.563833in}{4.504742in}}%
\pgfpathlineto{\pgfqpoint{3.484281in}{4.506457in}}%
\pgfpathclose%
\pgfusepath{fill}%
\end{pgfscope}%
\begin{pgfscope}%
\pgfpathrectangle{\pgfqpoint{0.539299in}{0.078740in}}{\pgfqpoint{7.842520in}{7.842520in}}%
\pgfusepath{clip}%
\pgfsetbuttcap%
\pgfsetroundjoin%
\definecolor{currentfill}{rgb}{0.250425,0.274290,0.533103}%
\pgfsetfillcolor{currentfill}%
\pgfsetlinewidth{0.000000pt}%
\definecolor{currentstroke}{rgb}{0.214298,0.355619,0.551184}%
\pgfsetstrokecolor{currentstroke}%
\pgfsetdash{}{0pt}%
\pgfpathmoveto{\pgfqpoint{3.279998in}{4.210712in}}%
\pgfpathlineto{\pgfqpoint{3.404261in}{4.501552in}}%
\pgfpathlineto{\pgfqpoint{3.484281in}{4.506457in}}%
\pgfpathclose%
\pgfusepath{fill}%
\end{pgfscope}%
\begin{pgfscope}%
\pgfpathrectangle{\pgfqpoint{0.539299in}{0.078740in}}{\pgfqpoint{7.842520in}{7.842520in}}%
\pgfusepath{clip}%
\pgfsetbuttcap%
\pgfsetroundjoin%
\definecolor{currentfill}{rgb}{0.223925,0.334994,0.548053}%
\pgfsetfillcolor{currentfill}%
\pgfsetlinewidth{0.000000pt}%
\definecolor{currentstroke}{rgb}{0.212395,0.359683,0.551710}%
\pgfsetstrokecolor{currentstroke}%
\pgfsetdash{}{0pt}%
\pgfpathmoveto{\pgfqpoint{3.611618in}{4.681644in}}%
\pgfpathlineto{\pgfqpoint{3.691406in}{4.650385in}}%
\pgfpathlineto{\pgfqpoint{3.563833in}{4.504742in}}%
\pgfpathclose%
\pgfusepath{fill}%
\end{pgfscope}%
\begin{pgfscope}%
\pgfpathrectangle{\pgfqpoint{0.539299in}{0.078740in}}{\pgfqpoint{7.842520in}{7.842520in}}%
\pgfusepath{clip}%
\pgfsetbuttcap%
\pgfsetroundjoin%
\definecolor{currentfill}{rgb}{0.276022,0.044167,0.370164}%
\pgfsetfillcolor{currentfill}%
\pgfsetlinewidth{0.000000pt}%
\definecolor{currentstroke}{rgb}{0.210503,0.363727,0.552206}%
\pgfsetstrokecolor{currentstroke}%
\pgfsetdash{}{0pt}%
\pgfpathmoveto{\pgfqpoint{5.705595in}{3.678274in}}%
\pgfpathlineto{\pgfqpoint{5.572472in}{3.714282in}}%
\pgfpathlineto{\pgfqpoint{5.633500in}{3.678097in}}%
\pgfpathclose%
\pgfusepath{fill}%
\end{pgfscope}%
\begin{pgfscope}%
\pgfpathrectangle{\pgfqpoint{0.539299in}{0.078740in}}{\pgfqpoint{7.842520in}{7.842520in}}%
\pgfusepath{clip}%
\pgfsetbuttcap%
\pgfsetroundjoin%
\definecolor{currentfill}{rgb}{0.258965,0.251537,0.524736}%
\pgfsetfillcolor{currentfill}%
\pgfsetlinewidth{0.000000pt}%
\definecolor{currentstroke}{rgb}{0.208623,0.367752,0.552675}%
\pgfsetstrokecolor{currentstroke}%
\pgfsetdash{}{0pt}%
\pgfpathmoveto{\pgfqpoint{4.501556in}{4.310839in}}%
\pgfpathlineto{\pgfqpoint{4.424973in}{4.414618in}}%
\pgfpathlineto{\pgfqpoint{4.633026in}{4.207547in}}%
\pgfpathclose%
\pgfusepath{fill}%
\end{pgfscope}%
\begin{pgfscope}%
\pgfpathrectangle{\pgfqpoint{0.539299in}{0.078740in}}{\pgfqpoint{7.842520in}{7.842520in}}%
\pgfusepath{clip}%
\pgfsetbuttcap%
\pgfsetroundjoin%
\definecolor{currentfill}{rgb}{0.214298,0.355619,0.551184}%
\pgfsetfillcolor{currentfill}%
\pgfsetlinewidth{0.000000pt}%
\definecolor{currentstroke}{rgb}{0.206756,0.371758,0.553117}%
\pgfsetstrokecolor{currentstroke}%
\pgfsetdash{}{0pt}%
\pgfpathmoveto{\pgfqpoint{3.691406in}{4.650385in}}%
\pgfpathlineto{\pgfqpoint{3.820757in}{4.723273in}}%
\pgfpathlineto{\pgfqpoint{3.899971in}{4.655078in}}%
\pgfpathclose%
\pgfusepath{fill}%
\end{pgfscope}%
\begin{pgfscope}%
\pgfpathrectangle{\pgfqpoint{0.539299in}{0.078740in}}{\pgfqpoint{7.842520in}{7.842520in}}%
\pgfusepath{clip}%
\pgfsetbuttcap%
\pgfsetroundjoin%
\definecolor{currentfill}{rgb}{0.231674,0.318106,0.544834}%
\pgfsetfillcolor{currentfill}%
\pgfsetlinewidth{0.000000pt}%
\definecolor{currentstroke}{rgb}{0.204903,0.375746,0.553533}%
\pgfsetstrokecolor{currentstroke}%
\pgfsetdash{}{0pt}%
\pgfpathmoveto{\pgfqpoint{4.239257in}{4.489420in}}%
\pgfpathlineto{\pgfqpoint{4.161473in}{4.598110in}}%
\pgfpathlineto{\pgfqpoint{4.293076in}{4.516692in}}%
\pgfpathclose%
\pgfusepath{fill}%
\end{pgfscope}%
\begin{pgfscope}%
\pgfpathrectangle{\pgfqpoint{0.539299in}{0.078740in}}{\pgfqpoint{7.842520in}{7.842520in}}%
\pgfusepath{clip}%
\pgfsetbuttcap%
\pgfsetroundjoin%
\definecolor{currentfill}{rgb}{0.277941,0.056324,0.381191}%
\pgfsetfillcolor{currentfill}%
\pgfsetlinewidth{0.000000pt}%
\definecolor{currentstroke}{rgb}{0.203063,0.379716,0.553925}%
\pgfsetstrokecolor{currentstroke}%
\pgfsetdash{}{0pt}%
\pgfpathmoveto{\pgfqpoint{5.633500in}{3.678097in}}%
\pgfpathlineto{\pgfqpoint{5.572472in}{3.714282in}}%
\pgfpathlineto{\pgfqpoint{5.500041in}{3.716527in}}%
\pgfpathclose%
\pgfusepath{fill}%
\end{pgfscope}%
\begin{pgfscope}%
\pgfpathrectangle{\pgfqpoint{0.539299in}{0.078740in}}{\pgfqpoint{7.842520in}{7.842520in}}%
\pgfusepath{clip}%
\pgfsetbuttcap%
\pgfsetroundjoin%
\definecolor{currentfill}{rgb}{0.281446,0.084320,0.407414}%
\pgfsetfillcolor{currentfill}%
\pgfsetlinewidth{0.000000pt}%
\definecolor{currentstroke}{rgb}{0.201239,0.383670,0.554294}%
\pgfsetstrokecolor{currentstroke}%
\pgfsetdash{}{0pt}%
\pgfpathmoveto{\pgfqpoint{5.293876in}{3.778414in}}%
\pgfpathlineto{\pgfqpoint{5.367069in}{3.761390in}}%
\pgfpathlineto{\pgfqpoint{5.234563in}{3.814537in}}%
\pgfpathclose%
\pgfusepath{fill}%
\end{pgfscope}%
\begin{pgfscope}%
\pgfpathrectangle{\pgfqpoint{0.539299in}{0.078740in}}{\pgfqpoint{7.842520in}{7.842520in}}%
\pgfusepath{clip}%
\pgfsetbuttcap%
\pgfsetroundjoin%
\definecolor{currentfill}{rgb}{0.283091,0.110553,0.431554}%
\pgfsetfillcolor{currentfill}%
\pgfsetlinewidth{0.000000pt}%
\definecolor{currentstroke}{rgb}{0.199430,0.387607,0.554642}%
\pgfsetstrokecolor{currentstroke}%
\pgfsetdash{}{0pt}%
\pgfpathmoveto{\pgfqpoint{5.028622in}{3.919505in}}%
\pgfpathlineto{\pgfqpoint{5.161052in}{3.843033in}}%
\pgfpathlineto{\pgfqpoint{5.234563in}{3.814537in}}%
\pgfpathclose%
\pgfusepath{fill}%
\end{pgfscope}%
\begin{pgfscope}%
\pgfpathrectangle{\pgfqpoint{0.539299in}{0.078740in}}{\pgfqpoint{7.842520in}{7.842520in}}%
\pgfusepath{clip}%
\pgfsetbuttcap%
\pgfsetroundjoin%
\definecolor{currentfill}{rgb}{0.267968,0.223549,0.512008}%
\pgfsetfillcolor{currentfill}%
\pgfsetlinewidth{0.000000pt}%
\definecolor{currentstroke}{rgb}{0.197636,0.391528,0.554969}%
\pgfsetstrokecolor{currentstroke}%
\pgfsetdash{}{0pt}%
\pgfpathmoveto{\pgfqpoint{4.764671in}{4.104606in}}%
\pgfpathlineto{\pgfqpoint{4.633026in}{4.207547in}}%
\pgfpathlineto{\pgfqpoint{4.557051in}{4.301638in}}%
\pgfpathclose%
\pgfusepath{fill}%
\end{pgfscope}%
\begin{pgfscope}%
\pgfpathrectangle{\pgfqpoint{0.539299in}{0.078740in}}{\pgfqpoint{7.842520in}{7.842520in}}%
\pgfusepath{clip}%
\pgfsetbuttcap%
\pgfsetroundjoin%
\definecolor{currentfill}{rgb}{0.258965,0.251537,0.524736}%
\pgfsetfillcolor{currentfill}%
\pgfsetlinewidth{0.000000pt}%
\definecolor{currentstroke}{rgb}{0.195860,0.395433,0.555276}%
\pgfsetstrokecolor{currentstroke}%
\pgfsetdash{}{0pt}%
\pgfpathmoveto{\pgfqpoint{3.279998in}{4.210712in}}%
\pgfpathlineto{\pgfqpoint{3.200055in}{4.172995in}}%
\pgfpathlineto{\pgfqpoint{3.323758in}{4.491822in}}%
\pgfpathclose%
\pgfusepath{fill}%
\end{pgfscope}%
\begin{pgfscope}%
\pgfpathrectangle{\pgfqpoint{0.539299in}{0.078740in}}{\pgfqpoint{7.842520in}{7.842520in}}%
\pgfusepath{clip}%
\pgfsetbuttcap%
\pgfsetroundjoin%
\definecolor{currentfill}{rgb}{0.212395,0.359683,0.551710}%
\pgfsetfillcolor{currentfill}%
\pgfsetlinewidth{0.000000pt}%
\definecolor{currentstroke}{rgb}{0.194100,0.399323,0.555565}%
\pgfsetstrokecolor{currentstroke}%
\pgfsetdash{}{0pt}%
\pgfpathmoveto{\pgfqpoint{3.951415in}{4.734438in}}%
\pgfpathlineto{\pgfqpoint{4.030348in}{4.648062in}}%
\pgfpathlineto{\pgfqpoint{3.899971in}{4.655078in}}%
\pgfpathclose%
\pgfusepath{fill}%
\end{pgfscope}%
\begin{pgfscope}%
\pgfpathrectangle{\pgfqpoint{0.539299in}{0.078740in}}{\pgfqpoint{7.842520in}{7.842520in}}%
\pgfusepath{clip}%
\pgfsetbuttcap%
\pgfsetroundjoin%
\definecolor{currentfill}{rgb}{0.276022,0.044167,0.370164}%
\pgfsetfillcolor{currentfill}%
\pgfsetlinewidth{0.000000pt}%
\definecolor{currentstroke}{rgb}{0.192357,0.403199,0.555836}%
\pgfsetstrokecolor{currentstroke}%
\pgfsetdash{}{0pt}%
\pgfpathmoveto{\pgfqpoint{5.767446in}{3.643788in}}%
\pgfpathlineto{\pgfqpoint{5.839174in}{3.644098in}}%
\pgfpathlineto{\pgfqpoint{5.705595in}{3.678274in}}%
\pgfpathclose%
\pgfusepath{fill}%
\end{pgfscope}%
\begin{pgfscope}%
\pgfpathrectangle{\pgfqpoint{0.539299in}{0.078740in}}{\pgfqpoint{7.842520in}{7.842520in}}%
\pgfusepath{clip}%
\pgfsetbuttcap%
\pgfsetroundjoin%
\definecolor{currentfill}{rgb}{0.282910,0.105393,0.426902}%
\pgfsetfillcolor{currentfill}%
\pgfsetlinewidth{0.000000pt}%
\definecolor{currentstroke}{rgb}{0.190631,0.407061,0.556089}%
\pgfsetstrokecolor{currentstroke}%
\pgfsetdash{}{0pt}%
\pgfpathmoveto{\pgfqpoint{5.234563in}{3.814537in}}%
\pgfpathlineto{\pgfqpoint{5.161052in}{3.843033in}}%
\pgfpathlineto{\pgfqpoint{5.293876in}{3.778414in}}%
\pgfpathclose%
\pgfusepath{fill}%
\end{pgfscope}%
\begin{pgfscope}%
\pgfpathrectangle{\pgfqpoint{0.539299in}{0.078740in}}{\pgfqpoint{7.842520in}{7.842520in}}%
\pgfusepath{clip}%
\pgfsetbuttcap%
\pgfsetroundjoin%
\definecolor{currentfill}{rgb}{0.243113,0.292092,0.538516}%
\pgfsetfillcolor{currentfill}%
\pgfsetlinewidth{0.000000pt}%
\definecolor{currentstroke}{rgb}{0.188923,0.410910,0.556326}%
\pgfsetstrokecolor{currentstroke}%
\pgfsetdash{}{0pt}%
\pgfpathmoveto{\pgfqpoint{4.293076in}{4.516692in}}%
\pgfpathlineto{\pgfqpoint{4.424973in}{4.414618in}}%
\pgfpathlineto{\pgfqpoint{4.501556in}{4.310839in}}%
\pgfpathclose%
\pgfusepath{fill}%
\end{pgfscope}%
\begin{pgfscope}%
\pgfpathrectangle{\pgfqpoint{0.539299in}{0.078740in}}{\pgfqpoint{7.842520in}{7.842520in}}%
\pgfusepath{clip}%
\pgfsetbuttcap%
\pgfsetroundjoin%
\definecolor{currentfill}{rgb}{0.277134,0.185228,0.489898}%
\pgfsetfillcolor{currentfill}%
\pgfsetlinewidth{0.000000pt}%
\definecolor{currentstroke}{rgb}{0.187231,0.414746,0.556547}%
\pgfsetstrokecolor{currentstroke}%
\pgfsetdash{}{0pt}%
\pgfpathmoveto{\pgfqpoint{4.764671in}{4.104606in}}%
\pgfpathlineto{\pgfqpoint{4.821630in}{4.074629in}}%
\pgfpathlineto{\pgfqpoint{4.896518in}{4.007367in}}%
\pgfpathclose%
\pgfusepath{fill}%
\end{pgfscope}%
\begin{pgfscope}%
\pgfpathrectangle{\pgfqpoint{0.539299in}{0.078740in}}{\pgfqpoint{7.842520in}{7.842520in}}%
\pgfusepath{clip}%
\pgfsetbuttcap%
\pgfsetroundjoin%
\definecolor{currentfill}{rgb}{0.280894,0.078907,0.402329}%
\pgfsetfillcolor{currentfill}%
\pgfsetlinewidth{0.000000pt}%
\definecolor{currentstroke}{rgb}{0.185556,0.418570,0.556753}%
\pgfsetstrokecolor{currentstroke}%
\pgfsetdash{}{0pt}%
\pgfpathmoveto{\pgfqpoint{5.500041in}{3.716527in}}%
\pgfpathlineto{\pgfqpoint{5.367069in}{3.761390in}}%
\pgfpathlineto{\pgfqpoint{5.427151in}{3.724780in}}%
\pgfpathclose%
\pgfusepath{fill}%
\end{pgfscope}%
\begin{pgfscope}%
\pgfpathrectangle{\pgfqpoint{0.539299in}{0.078740in}}{\pgfqpoint{7.842520in}{7.842520in}}%
\pgfusepath{clip}%
\pgfsetbuttcap%
\pgfsetroundjoin%
\definecolor{currentfill}{rgb}{0.223925,0.334994,0.548053}%
\pgfsetfillcolor{currentfill}%
\pgfsetlinewidth{0.000000pt}%
\definecolor{currentstroke}{rgb}{0.183898,0.422383,0.556944}%
\pgfsetstrokecolor{currentstroke}%
\pgfsetdash{}{0pt}%
\pgfpathmoveto{\pgfqpoint{3.484281in}{4.506457in}}%
\pgfpathlineto{\pgfqpoint{3.404261in}{4.501552in}}%
\pgfpathlineto{\pgfqpoint{3.611618in}{4.681644in}}%
\pgfpathclose%
\pgfusepath{fill}%
\end{pgfscope}%
\begin{pgfscope}%
\pgfpathrectangle{\pgfqpoint{0.539299in}{0.078740in}}{\pgfqpoint{7.842520in}{7.842520in}}%
\pgfusepath{clip}%
\pgfsetbuttcap%
\pgfsetroundjoin%
\definecolor{currentfill}{rgb}{0.243113,0.292092,0.538516}%
\pgfsetfillcolor{currentfill}%
\pgfsetlinewidth{0.000000pt}%
\definecolor{currentstroke}{rgb}{0.182256,0.426184,0.557120}%
\pgfsetstrokecolor{currentstroke}%
\pgfsetdash{}{0pt}%
\pgfpathmoveto{\pgfqpoint{3.323758in}{4.491822in}}%
\pgfpathlineto{\pgfqpoint{3.404261in}{4.501552in}}%
\pgfpathlineto{\pgfqpoint{3.279998in}{4.210712in}}%
\pgfpathclose%
\pgfusepath{fill}%
\end{pgfscope}%
\begin{pgfscope}%
\pgfpathrectangle{\pgfqpoint{0.539299in}{0.078740in}}{\pgfqpoint{7.842520in}{7.842520in}}%
\pgfusepath{clip}%
\pgfsetbuttcap%
\pgfsetroundjoin%
\definecolor{currentfill}{rgb}{0.277941,0.056324,0.381191}%
\pgfsetfillcolor{currentfill}%
\pgfsetlinewidth{0.000000pt}%
\definecolor{currentstroke}{rgb}{0.180629,0.429975,0.557282}%
\pgfsetstrokecolor{currentstroke}%
\pgfsetdash{}{0pt}%
\pgfpathmoveto{\pgfqpoint{5.705595in}{3.678274in}}%
\pgfpathlineto{\pgfqpoint{5.633500in}{3.678097in}}%
\pgfpathlineto{\pgfqpoint{5.767446in}{3.643788in}}%
\pgfpathclose%
\pgfusepath{fill}%
\end{pgfscope}%
\begin{pgfscope}%
\pgfpathrectangle{\pgfqpoint{0.539299in}{0.078740in}}{\pgfqpoint{7.842520in}{7.842520in}}%
\pgfusepath{clip}%
\pgfsetbuttcap%
\pgfsetroundjoin%
\definecolor{currentfill}{rgb}{0.253935,0.265254,0.529983}%
\pgfsetfillcolor{currentfill}%
\pgfsetlinewidth{0.000000pt}%
\definecolor{currentstroke}{rgb}{0.179019,0.433756,0.557430}%
\pgfsetstrokecolor{currentstroke}%
\pgfsetdash{}{0pt}%
\pgfpathmoveto{\pgfqpoint{4.633026in}{4.207547in}}%
\pgfpathlineto{\pgfqpoint{4.424973in}{4.414618in}}%
\pgfpathlineto{\pgfqpoint{4.557051in}{4.301638in}}%
\pgfpathclose%
\pgfusepath{fill}%
\end{pgfscope}%
\begin{pgfscope}%
\pgfpathrectangle{\pgfqpoint{0.539299in}{0.078740in}}{\pgfqpoint{7.842520in}{7.842520in}}%
\pgfusepath{clip}%
\pgfsetbuttcap%
\pgfsetroundjoin%
\definecolor{currentfill}{rgb}{0.206756,0.371758,0.553117}%
\pgfsetfillcolor{currentfill}%
\pgfsetlinewidth{0.000000pt}%
\definecolor{currentstroke}{rgb}{0.177423,0.437527,0.557565}%
\pgfsetstrokecolor{currentstroke}%
\pgfsetdash{}{0pt}%
\pgfpathmoveto{\pgfqpoint{3.899971in}{4.655078in}}%
\pgfpathlineto{\pgfqpoint{3.820757in}{4.723273in}}%
\pgfpathlineto{\pgfqpoint{3.951415in}{4.734438in}}%
\pgfpathclose%
\pgfusepath{fill}%
\end{pgfscope}%
\begin{pgfscope}%
\pgfpathrectangle{\pgfqpoint{0.539299in}{0.078740in}}{\pgfqpoint{7.842520in}{7.842520in}}%
\pgfusepath{clip}%
\pgfsetbuttcap%
\pgfsetroundjoin%
\definecolor{currentfill}{rgb}{0.208623,0.367752,0.552675}%
\pgfsetfillcolor{currentfill}%
\pgfsetlinewidth{0.000000pt}%
\definecolor{currentstroke}{rgb}{0.175841,0.441290,0.557685}%
\pgfsetstrokecolor{currentstroke}%
\pgfsetdash{}{0pt}%
\pgfpathmoveto{\pgfqpoint{3.611618in}{4.681644in}}%
\pgfpathlineto{\pgfqpoint{3.820757in}{4.723273in}}%
\pgfpathlineto{\pgfqpoint{3.691406in}{4.650385in}}%
\pgfpathclose%
\pgfusepath{fill}%
\end{pgfscope}%
\begin{pgfscope}%
\pgfpathrectangle{\pgfqpoint{0.539299in}{0.078740in}}{\pgfqpoint{7.842520in}{7.842520in}}%
\pgfusepath{clip}%
\pgfsetbuttcap%
\pgfsetroundjoin%
\definecolor{currentfill}{rgb}{0.279574,0.170599,0.479997}%
\pgfsetfillcolor{currentfill}%
\pgfsetlinewidth{0.000000pt}%
\definecolor{currentstroke}{rgb}{0.174274,0.445044,0.557792}%
\pgfsetstrokecolor{currentstroke}%
\pgfsetdash{}{0pt}%
\pgfpathmoveto{\pgfqpoint{4.821630in}{4.074629in}}%
\pgfpathlineto{\pgfqpoint{5.028622in}{3.919505in}}%
\pgfpathlineto{\pgfqpoint{4.896518in}{4.007367in}}%
\pgfpathclose%
\pgfusepath{fill}%
\end{pgfscope}%
\begin{pgfscope}%
\pgfpathrectangle{\pgfqpoint{0.539299in}{0.078740in}}{\pgfqpoint{7.842520in}{7.842520in}}%
\pgfusepath{clip}%
\pgfsetbuttcap%
\pgfsetroundjoin%
\definecolor{currentfill}{rgb}{0.277018,0.050344,0.375715}%
\pgfsetfillcolor{currentfill}%
\pgfsetlinewidth{0.000000pt}%
\definecolor{currentstroke}{rgb}{0.172719,0.448791,0.557885}%
\pgfsetstrokecolor{currentstroke}%
\pgfsetdash{}{0pt}%
\pgfpathmoveto{\pgfqpoint{5.973195in}{3.610242in}}%
\pgfpathlineto{\pgfqpoint{5.839174in}{3.644098in}}%
\pgfpathlineto{\pgfqpoint{5.767446in}{3.643788in}}%
\pgfpathclose%
\pgfusepath{fill}%
\end{pgfscope}%
\begin{pgfscope}%
\pgfpathrectangle{\pgfqpoint{0.539299in}{0.078740in}}{\pgfqpoint{7.842520in}{7.842520in}}%
\pgfusepath{clip}%
\pgfsetbuttcap%
\pgfsetroundjoin%
\definecolor{currentfill}{rgb}{0.265145,0.232956,0.516599}%
\pgfsetfillcolor{currentfill}%
\pgfsetlinewidth{0.000000pt}%
\definecolor{currentstroke}{rgb}{0.171176,0.452530,0.557965}%
\pgfsetstrokecolor{currentstroke}%
\pgfsetdash{}{0pt}%
\pgfpathmoveto{\pgfqpoint{4.557051in}{4.301638in}}%
\pgfpathlineto{\pgfqpoint{4.689266in}{4.186089in}}%
\pgfpathlineto{\pgfqpoint{4.764671in}{4.104606in}}%
\pgfpathclose%
\pgfusepath{fill}%
\end{pgfscope}%
\begin{pgfscope}%
\pgfpathrectangle{\pgfqpoint{0.539299in}{0.078740in}}{\pgfqpoint{7.842520in}{7.842520in}}%
\pgfusepath{clip}%
\pgfsetbuttcap%
\pgfsetroundjoin%
\definecolor{currentfill}{rgb}{0.282327,0.094955,0.417331}%
\pgfsetfillcolor{currentfill}%
\pgfsetlinewidth{0.000000pt}%
\definecolor{currentstroke}{rgb}{0.169646,0.456262,0.558030}%
\pgfsetstrokecolor{currentstroke}%
\pgfsetdash{}{0pt}%
\pgfpathmoveto{\pgfqpoint{5.367069in}{3.761390in}}%
\pgfpathlineto{\pgfqpoint{5.293876in}{3.778414in}}%
\pgfpathlineto{\pgfqpoint{5.427151in}{3.724780in}}%
\pgfpathclose%
\pgfusepath{fill}%
\end{pgfscope}%
\begin{pgfscope}%
\pgfpathrectangle{\pgfqpoint{0.539299in}{0.078740in}}{\pgfqpoint{7.842520in}{7.842520in}}%
\pgfusepath{clip}%
\pgfsetbuttcap%
\pgfsetroundjoin%
\definecolor{currentfill}{rgb}{0.210503,0.363727,0.552206}%
\pgfsetfillcolor{currentfill}%
\pgfsetlinewidth{0.000000pt}%
\definecolor{currentstroke}{rgb}{0.168126,0.459988,0.558082}%
\pgfsetstrokecolor{currentstroke}%
\pgfsetdash{}{0pt}%
\pgfpathmoveto{\pgfqpoint{4.082979in}{4.695694in}}%
\pgfpathlineto{\pgfqpoint{4.161473in}{4.598110in}}%
\pgfpathlineto{\pgfqpoint{4.030348in}{4.648062in}}%
\pgfpathclose%
\pgfusepath{fill}%
\end{pgfscope}%
\begin{pgfscope}%
\pgfpathrectangle{\pgfqpoint{0.539299in}{0.078740in}}{\pgfqpoint{7.842520in}{7.842520in}}%
\pgfusepath{clip}%
\pgfsetbuttcap%
\pgfsetroundjoin%
\definecolor{currentfill}{rgb}{0.280894,0.078907,0.402329}%
\pgfsetfillcolor{currentfill}%
\pgfsetlinewidth{0.000000pt}%
\definecolor{currentstroke}{rgb}{0.166617,0.463708,0.558119}%
\pgfsetstrokecolor{currentstroke}%
\pgfsetdash{}{0pt}%
\pgfpathmoveto{\pgfqpoint{5.427151in}{3.724780in}}%
\pgfpathlineto{\pgfqpoint{5.633500in}{3.678097in}}%
\pgfpathlineto{\pgfqpoint{5.500041in}{3.716527in}}%
\pgfpathclose%
\pgfusepath{fill}%
\end{pgfscope}%
\begin{pgfscope}%
\pgfpathrectangle{\pgfqpoint{0.539299in}{0.078740in}}{\pgfqpoint{7.842520in}{7.842520in}}%
\pgfusepath{clip}%
\pgfsetbuttcap%
\pgfsetroundjoin%
\definecolor{currentfill}{rgb}{0.270595,0.214069,0.507052}%
\pgfsetfillcolor{currentfill}%
\pgfsetlinewidth{0.000000pt}%
\definecolor{currentstroke}{rgb}{0.165117,0.467423,0.558141}%
\pgfsetstrokecolor{currentstroke}%
\pgfsetdash{}{0pt}%
\pgfpathmoveto{\pgfqpoint{4.689266in}{4.186089in}}%
\pgfpathlineto{\pgfqpoint{4.821630in}{4.074629in}}%
\pgfpathlineto{\pgfqpoint{4.764671in}{4.104606in}}%
\pgfpathclose%
\pgfusepath{fill}%
\end{pgfscope}%
\begin{pgfscope}%
\pgfpathrectangle{\pgfqpoint{0.539299in}{0.078740in}}{\pgfqpoint{7.842520in}{7.842520in}}%
\pgfusepath{clip}%
\pgfsetbuttcap%
\pgfsetroundjoin%
\definecolor{currentfill}{rgb}{0.282623,0.140926,0.457517}%
\pgfsetfillcolor{currentfill}%
\pgfsetlinewidth{0.000000pt}%
\definecolor{currentstroke}{rgb}{0.163625,0.471133,0.558148}%
\pgfsetstrokecolor{currentstroke}%
\pgfsetdash{}{0pt}%
\pgfpathmoveto{\pgfqpoint{5.087019in}{3.881455in}}%
\pgfpathlineto{\pgfqpoint{5.161052in}{3.843033in}}%
\pgfpathlineto{\pgfqpoint{5.028622in}{3.919505in}}%
\pgfpathclose%
\pgfusepath{fill}%
\end{pgfscope}%
\begin{pgfscope}%
\pgfpathrectangle{\pgfqpoint{0.539299in}{0.078740in}}{\pgfqpoint{7.842520in}{7.842520in}}%
\pgfusepath{clip}%
\pgfsetbuttcap%
\pgfsetroundjoin%
\definecolor{currentfill}{rgb}{0.214298,0.355619,0.551184}%
\pgfsetfillcolor{currentfill}%
\pgfsetlinewidth{0.000000pt}%
\definecolor{currentstroke}{rgb}{0.162142,0.474838,0.558140}%
\pgfsetstrokecolor{currentstroke}%
\pgfsetdash{}{0pt}%
\pgfpathmoveto{\pgfqpoint{4.082979in}{4.695694in}}%
\pgfpathlineto{\pgfqpoint{4.293076in}{4.516692in}}%
\pgfpathlineto{\pgfqpoint{4.161473in}{4.598110in}}%
\pgfpathclose%
\pgfusepath{fill}%
\end{pgfscope}%
\begin{pgfscope}%
\pgfpathrectangle{\pgfqpoint{0.539299in}{0.078740in}}{\pgfqpoint{7.842520in}{7.842520in}}%
\pgfusepath{clip}%
\pgfsetbuttcap%
\pgfsetroundjoin%
\definecolor{currentfill}{rgb}{0.278012,0.180367,0.486697}%
\pgfsetfillcolor{currentfill}%
\pgfsetlinewidth{0.000000pt}%
\definecolor{currentstroke}{rgb}{0.160665,0.478540,0.558115}%
\pgfsetstrokecolor{currentstroke}%
\pgfsetdash{}{0pt}%
\pgfpathmoveto{\pgfqpoint{4.954191in}{3.972092in}}%
\pgfpathlineto{\pgfqpoint{5.028622in}{3.919505in}}%
\pgfpathlineto{\pgfqpoint{4.821630in}{4.074629in}}%
\pgfpathclose%
\pgfusepath{fill}%
\end{pgfscope}%
\begin{pgfscope}%
\pgfpathrectangle{\pgfqpoint{0.539299in}{0.078740in}}{\pgfqpoint{7.842520in}{7.842520in}}%
\pgfusepath{clip}%
\pgfsetbuttcap%
\pgfsetroundjoin%
\definecolor{currentfill}{rgb}{0.203063,0.379716,0.553925}%
\pgfsetfillcolor{currentfill}%
\pgfsetlinewidth{0.000000pt}%
\definecolor{currentstroke}{rgb}{0.159194,0.482237,0.558073}%
\pgfsetstrokecolor{currentstroke}%
\pgfsetdash{}{0pt}%
\pgfpathmoveto{\pgfqpoint{4.082979in}{4.695694in}}%
\pgfpathlineto{\pgfqpoint{4.030348in}{4.648062in}}%
\pgfpathlineto{\pgfqpoint{3.951415in}{4.734438in}}%
\pgfpathclose%
\pgfusepath{fill}%
\end{pgfscope}%
\begin{pgfscope}%
\pgfpathrectangle{\pgfqpoint{0.539299in}{0.078740in}}{\pgfqpoint{7.842520in}{7.842520in}}%
\pgfusepath{clip}%
\pgfsetbuttcap%
\pgfsetroundjoin%
\definecolor{currentfill}{rgb}{0.277018,0.050344,0.375715}%
\pgfsetfillcolor{currentfill}%
\pgfsetlinewidth{0.000000pt}%
\definecolor{currentstroke}{rgb}{0.157729,0.485932,0.558013}%
\pgfsetstrokecolor{currentstroke}%
\pgfsetdash{}{0pt}%
\pgfpathmoveto{\pgfqpoint{6.107631in}{3.575340in}}%
\pgfpathlineto{\pgfqpoint{5.973195in}{3.610242in}}%
\pgfpathlineto{\pgfqpoint{5.901860in}{3.611134in}}%
\pgfpathclose%
\pgfusepath{fill}%
\end{pgfscope}%
\begin{pgfscope}%
\pgfpathrectangle{\pgfqpoint{0.539299in}{0.078740in}}{\pgfqpoint{7.842520in}{7.842520in}}%
\pgfusepath{clip}%
\pgfsetbuttcap%
\pgfsetroundjoin%
\definecolor{currentfill}{rgb}{0.253935,0.265254,0.529983}%
\pgfsetfillcolor{currentfill}%
\pgfsetlinewidth{0.000000pt}%
\definecolor{currentstroke}{rgb}{0.156270,0.489624,0.557936}%
\pgfsetstrokecolor{currentstroke}%
\pgfsetdash{}{0pt}%
\pgfpathmoveto{\pgfqpoint{3.200055in}{4.172995in}}%
\pgfpathlineto{\pgfqpoint{3.119658in}{4.133109in}}%
\pgfpathlineto{\pgfqpoint{3.242761in}{4.478380in}}%
\pgfpathclose%
\pgfusepath{fill}%
\end{pgfscope}%
\begin{pgfscope}%
\pgfpathrectangle{\pgfqpoint{0.539299in}{0.078740in}}{\pgfqpoint{7.842520in}{7.842520in}}%
\pgfusepath{clip}%
\pgfsetbuttcap%
\pgfsetroundjoin%
\definecolor{currentfill}{rgb}{0.278791,0.062145,0.386592}%
\pgfsetfillcolor{currentfill}%
\pgfsetlinewidth{0.000000pt}%
\definecolor{currentstroke}{rgb}{0.154815,0.493313,0.557840}%
\pgfsetstrokecolor{currentstroke}%
\pgfsetdash{}{0pt}%
\pgfpathmoveto{\pgfqpoint{5.901860in}{3.611134in}}%
\pgfpathlineto{\pgfqpoint{5.973195in}{3.610242in}}%
\pgfpathlineto{\pgfqpoint{5.767446in}{3.643788in}}%
\pgfpathclose%
\pgfusepath{fill}%
\end{pgfscope}%
\begin{pgfscope}%
\pgfpathrectangle{\pgfqpoint{0.539299in}{0.078740in}}{\pgfqpoint{7.842520in}{7.842520in}}%
\pgfusepath{clip}%
\pgfsetbuttcap%
\pgfsetroundjoin%
\definecolor{currentfill}{rgb}{0.283187,0.125848,0.444960}%
\pgfsetfillcolor{currentfill}%
\pgfsetlinewidth{0.000000pt}%
\definecolor{currentstroke}{rgb}{0.153364,0.497000,0.557724}%
\pgfsetstrokecolor{currentstroke}%
\pgfsetdash{}{0pt}%
\pgfpathmoveto{\pgfqpoint{5.220192in}{3.803938in}}%
\pgfpathlineto{\pgfqpoint{5.293876in}{3.778414in}}%
\pgfpathlineto{\pgfqpoint{5.161052in}{3.843033in}}%
\pgfpathclose%
\pgfusepath{fill}%
\end{pgfscope}%
\begin{pgfscope}%
\pgfpathrectangle{\pgfqpoint{0.539299in}{0.078740in}}{\pgfqpoint{7.842520in}{7.842520in}}%
\pgfusepath{clip}%
\pgfsetbuttcap%
\pgfsetroundjoin%
\definecolor{currentfill}{rgb}{0.280255,0.165693,0.476498}%
\pgfsetfillcolor{currentfill}%
\pgfsetlinewidth{0.000000pt}%
\definecolor{currentstroke}{rgb}{0.151918,0.500685,0.557587}%
\pgfsetstrokecolor{currentstroke}%
\pgfsetdash{}{0pt}%
\pgfpathmoveto{\pgfqpoint{5.028622in}{3.919505in}}%
\pgfpathlineto{\pgfqpoint{4.954191in}{3.972092in}}%
\pgfpathlineto{\pgfqpoint{5.087019in}{3.881455in}}%
\pgfpathclose%
\pgfusepath{fill}%
\end{pgfscope}%
\begin{pgfscope}%
\pgfpathrectangle{\pgfqpoint{0.539299in}{0.078740in}}{\pgfqpoint{7.842520in}{7.842520in}}%
\pgfusepath{clip}%
\pgfsetbuttcap%
\pgfsetroundjoin%
\definecolor{currentfill}{rgb}{0.280894,0.078907,0.402329}%
\pgfsetfillcolor{currentfill}%
\pgfsetlinewidth{0.000000pt}%
\definecolor{currentstroke}{rgb}{0.150476,0.504369,0.557430}%
\pgfsetstrokecolor{currentstroke}%
\pgfsetdash{}{0pt}%
\pgfpathmoveto{\pgfqpoint{5.560913in}{3.680214in}}%
\pgfpathlineto{\pgfqpoint{5.767446in}{3.643788in}}%
\pgfpathlineto{\pgfqpoint{5.633500in}{3.678097in}}%
\pgfpathclose%
\pgfusepath{fill}%
\end{pgfscope}%
\begin{pgfscope}%
\pgfpathrectangle{\pgfqpoint{0.539299in}{0.078740in}}{\pgfqpoint{7.842520in}{7.842520in}}%
\pgfusepath{clip}%
\pgfsetbuttcap%
\pgfsetroundjoin%
\definecolor{currentfill}{rgb}{0.195860,0.395433,0.555276}%
\pgfsetfillcolor{currentfill}%
\pgfsetlinewidth{0.000000pt}%
\definecolor{currentstroke}{rgb}{0.149039,0.508051,0.557250}%
\pgfsetstrokecolor{currentstroke}%
\pgfsetdash{}{0pt}%
\pgfpathmoveto{\pgfqpoint{3.740970in}{4.779433in}}%
\pgfpathlineto{\pgfqpoint{3.820757in}{4.723273in}}%
\pgfpathlineto{\pgfqpoint{3.611618in}{4.681644in}}%
\pgfpathclose%
\pgfusepath{fill}%
\end{pgfscope}%
\begin{pgfscope}%
\pgfpathrectangle{\pgfqpoint{0.539299in}{0.078740in}}{\pgfqpoint{7.842520in}{7.842520in}}%
\pgfusepath{clip}%
\pgfsetbuttcap%
\pgfsetroundjoin%
\definecolor{currentfill}{rgb}{0.281924,0.089666,0.412415}%
\pgfsetfillcolor{currentfill}%
\pgfsetlinewidth{0.000000pt}%
\definecolor{currentstroke}{rgb}{0.147607,0.511733,0.557049}%
\pgfsetstrokecolor{currentstroke}%
\pgfsetdash{}{0pt}%
\pgfpathmoveto{\pgfqpoint{5.427151in}{3.724780in}}%
\pgfpathlineto{\pgfqpoint{5.560913in}{3.680214in}}%
\pgfpathlineto{\pgfqpoint{5.633500in}{3.678097in}}%
\pgfpathclose%
\pgfusepath{fill}%
\end{pgfscope}%
\begin{pgfscope}%
\pgfpathrectangle{\pgfqpoint{0.539299in}{0.078740in}}{\pgfqpoint{7.842520in}{7.842520in}}%
\pgfusepath{clip}%
\pgfsetbuttcap%
\pgfsetroundjoin%
\definecolor{currentfill}{rgb}{0.225863,0.330805,0.547314}%
\pgfsetfillcolor{currentfill}%
\pgfsetlinewidth{0.000000pt}%
\definecolor{currentstroke}{rgb}{0.146180,0.515413,0.556823}%
\pgfsetstrokecolor{currentstroke}%
\pgfsetdash{}{0pt}%
\pgfpathmoveto{\pgfqpoint{4.347630in}{4.516065in}}%
\pgfpathlineto{\pgfqpoint{4.424973in}{4.414618in}}%
\pgfpathlineto{\pgfqpoint{4.293076in}{4.516692in}}%
\pgfpathclose%
\pgfusepath{fill}%
\end{pgfscope}%
\begin{pgfscope}%
\pgfpathrectangle{\pgfqpoint{0.539299in}{0.078740in}}{\pgfqpoint{7.842520in}{7.842520in}}%
\pgfusepath{clip}%
\pgfsetbuttcap%
\pgfsetroundjoin%
\definecolor{currentfill}{rgb}{0.206756,0.371758,0.553117}%
\pgfsetfillcolor{currentfill}%
\pgfsetlinewidth{0.000000pt}%
\definecolor{currentstroke}{rgb}{0.144759,0.519093,0.556572}%
\pgfsetstrokecolor{currentstroke}%
\pgfsetdash{}{0pt}%
\pgfpathmoveto{\pgfqpoint{3.404261in}{4.501552in}}%
\pgfpathlineto{\pgfqpoint{3.531309in}{4.704589in}}%
\pgfpathlineto{\pgfqpoint{3.611618in}{4.681644in}}%
\pgfpathclose%
\pgfusepath{fill}%
\end{pgfscope}%
\begin{pgfscope}%
\pgfpathrectangle{\pgfqpoint{0.539299in}{0.078740in}}{\pgfqpoint{7.842520in}{7.842520in}}%
\pgfusepath{clip}%
\pgfsetbuttcap%
\pgfsetroundjoin%
\definecolor{currentfill}{rgb}{0.283229,0.120777,0.440584}%
\pgfsetfillcolor{currentfill}%
\pgfsetlinewidth{0.000000pt}%
\definecolor{currentstroke}{rgb}{0.143343,0.522773,0.556295}%
\pgfsetstrokecolor{currentstroke}%
\pgfsetdash{}{0pt}%
\pgfpathmoveto{\pgfqpoint{5.220192in}{3.803938in}}%
\pgfpathlineto{\pgfqpoint{5.427151in}{3.724780in}}%
\pgfpathlineto{\pgfqpoint{5.293876in}{3.778414in}}%
\pgfpathclose%
\pgfusepath{fill}%
\end{pgfscope}%
\begin{pgfscope}%
\pgfpathrectangle{\pgfqpoint{0.539299in}{0.078740in}}{\pgfqpoint{7.842520in}{7.842520in}}%
\pgfusepath{clip}%
\pgfsetbuttcap%
\pgfsetroundjoin%
\definecolor{currentfill}{rgb}{0.237441,0.305202,0.541921}%
\pgfsetfillcolor{currentfill}%
\pgfsetlinewidth{0.000000pt}%
\definecolor{currentstroke}{rgb}{0.141935,0.526453,0.555991}%
\pgfsetstrokecolor{currentstroke}%
\pgfsetdash{}{0pt}%
\pgfpathmoveto{\pgfqpoint{3.323758in}{4.491822in}}%
\pgfpathlineto{\pgfqpoint{3.200055in}{4.172995in}}%
\pgfpathlineto{\pgfqpoint{3.242761in}{4.478380in}}%
\pgfpathclose%
\pgfusepath{fill}%
\end{pgfscope}%
\begin{pgfscope}%
\pgfpathrectangle{\pgfqpoint{0.539299in}{0.078740in}}{\pgfqpoint{7.842520in}{7.842520in}}%
\pgfusepath{clip}%
\pgfsetbuttcap%
\pgfsetroundjoin%
\definecolor{currentfill}{rgb}{0.237441,0.305202,0.541921}%
\pgfsetfillcolor{currentfill}%
\pgfsetlinewidth{0.000000pt}%
\definecolor{currentstroke}{rgb}{0.140536,0.530132,0.555659}%
\pgfsetstrokecolor{currentstroke}%
\pgfsetdash{}{0pt}%
\pgfpathmoveto{\pgfqpoint{4.557051in}{4.301638in}}%
\pgfpathlineto{\pgfqpoint{4.424973in}{4.414618in}}%
\pgfpathlineto{\pgfqpoint{4.480330in}{4.397499in}}%
\pgfpathclose%
\pgfusepath{fill}%
\end{pgfscope}%
\begin{pgfscope}%
\pgfpathrectangle{\pgfqpoint{0.539299in}{0.078740in}}{\pgfqpoint{7.842520in}{7.842520in}}%
\pgfusepath{clip}%
\pgfsetbuttcap%
\pgfsetroundjoin%
\definecolor{currentfill}{rgb}{0.282290,0.145912,0.461510}%
\pgfsetfillcolor{currentfill}%
\pgfsetlinewidth{0.000000pt}%
\definecolor{currentstroke}{rgb}{0.139147,0.533812,0.555298}%
\pgfsetstrokecolor{currentstroke}%
\pgfsetdash{}{0pt}%
\pgfpathmoveto{\pgfqpoint{5.220192in}{3.803938in}}%
\pgfpathlineto{\pgfqpoint{5.161052in}{3.843033in}}%
\pgfpathlineto{\pgfqpoint{5.087019in}{3.881455in}}%
\pgfpathclose%
\pgfusepath{fill}%
\end{pgfscope}%
\begin{pgfscope}%
\pgfpathrectangle{\pgfqpoint{0.539299in}{0.078740in}}{\pgfqpoint{7.842520in}{7.842520in}}%
\pgfusepath{clip}%
\pgfsetbuttcap%
\pgfsetroundjoin%
\definecolor{currentfill}{rgb}{0.212395,0.359683,0.551710}%
\pgfsetfillcolor{currentfill}%
\pgfsetlinewidth{0.000000pt}%
\definecolor{currentstroke}{rgb}{0.137770,0.537492,0.554906}%
\pgfsetstrokecolor{currentstroke}%
\pgfsetdash{}{0pt}%
\pgfpathmoveto{\pgfqpoint{3.531309in}{4.704589in}}%
\pgfpathlineto{\pgfqpoint{3.404261in}{4.501552in}}%
\pgfpathlineto{\pgfqpoint{3.323758in}{4.491822in}}%
\pgfpathclose%
\pgfusepath{fill}%
\end{pgfscope}%
\begin{pgfscope}%
\pgfpathrectangle{\pgfqpoint{0.539299in}{0.078740in}}{\pgfqpoint{7.842520in}{7.842520in}}%
\pgfusepath{clip}%
\pgfsetbuttcap%
\pgfsetroundjoin%
\definecolor{currentfill}{rgb}{0.206756,0.371758,0.553117}%
\pgfsetfillcolor{currentfill}%
\pgfsetlinewidth{0.000000pt}%
\definecolor{currentstroke}{rgb}{0.136408,0.541173,0.554483}%
\pgfsetstrokecolor{currentstroke}%
\pgfsetdash{}{0pt}%
\pgfpathmoveto{\pgfqpoint{4.215129in}{4.619034in}}%
\pgfpathlineto{\pgfqpoint{4.293076in}{4.516692in}}%
\pgfpathlineto{\pgfqpoint{4.082979in}{4.695694in}}%
\pgfpathclose%
\pgfusepath{fill}%
\end{pgfscope}%
\begin{pgfscope}%
\pgfpathrectangle{\pgfqpoint{0.539299in}{0.078740in}}{\pgfqpoint{7.842520in}{7.842520in}}%
\pgfusepath{clip}%
\pgfsetbuttcap%
\pgfsetroundjoin%
\definecolor{currentfill}{rgb}{0.246811,0.283237,0.535941}%
\pgfsetfillcolor{currentfill}%
\pgfsetlinewidth{0.000000pt}%
\definecolor{currentstroke}{rgb}{0.135066,0.544853,0.554029}%
\pgfsetstrokecolor{currentstroke}%
\pgfsetdash{}{0pt}%
\pgfpathmoveto{\pgfqpoint{4.480330in}{4.397499in}}%
\pgfpathlineto{\pgfqpoint{4.689266in}{4.186089in}}%
\pgfpathlineto{\pgfqpoint{4.557051in}{4.301638in}}%
\pgfpathclose%
\pgfusepath{fill}%
\end{pgfscope}%
\begin{pgfscope}%
\pgfpathrectangle{\pgfqpoint{0.539299in}{0.078740in}}{\pgfqpoint{7.842520in}{7.842520in}}%
\pgfusepath{clip}%
\pgfsetbuttcap%
\pgfsetroundjoin%
\definecolor{currentfill}{rgb}{0.190631,0.407061,0.556089}%
\pgfsetfillcolor{currentfill}%
\pgfsetlinewidth{0.000000pt}%
\definecolor{currentstroke}{rgb}{0.133743,0.548535,0.553541}%
\pgfsetstrokecolor{currentstroke}%
\pgfsetdash{}{0pt}%
\pgfpathmoveto{\pgfqpoint{3.951415in}{4.734438in}}%
\pgfpathlineto{\pgfqpoint{3.820757in}{4.723273in}}%
\pgfpathlineto{\pgfqpoint{3.871838in}{4.810044in}}%
\pgfpathclose%
\pgfusepath{fill}%
\end{pgfscope}%
\begin{pgfscope}%
\pgfpathrectangle{\pgfqpoint{0.539299in}{0.078740in}}{\pgfqpoint{7.842520in}{7.842520in}}%
\pgfusepath{clip}%
\pgfsetbuttcap%
\pgfsetroundjoin%
\definecolor{currentfill}{rgb}{0.278791,0.062145,0.386592}%
\pgfsetfillcolor{currentfill}%
\pgfsetlinewidth{0.000000pt}%
\definecolor{currentstroke}{rgb}{0.132444,0.552216,0.553018}%
\pgfsetstrokecolor{currentstroke}%
\pgfsetdash{}{0pt}%
\pgfpathmoveto{\pgfqpoint{5.901860in}{3.611134in}}%
\pgfpathlineto{\pgfqpoint{6.036706in}{3.577796in}}%
\pgfpathlineto{\pgfqpoint{6.107631in}{3.575340in}}%
\pgfpathclose%
\pgfusepath{fill}%
\end{pgfscope}%
\begin{pgfscope}%
\pgfpathrectangle{\pgfqpoint{0.539299in}{0.078740in}}{\pgfqpoint{7.842520in}{7.842520in}}%
\pgfusepath{clip}%
\pgfsetbuttcap%
\pgfsetroundjoin%
\definecolor{currentfill}{rgb}{0.257322,0.256130,0.526563}%
\pgfsetfillcolor{currentfill}%
\pgfsetlinewidth{0.000000pt}%
\definecolor{currentstroke}{rgb}{0.131172,0.555899,0.552459}%
\pgfsetstrokecolor{currentstroke}%
\pgfsetdash{}{0pt}%
\pgfpathmoveto{\pgfqpoint{4.821630in}{4.074629in}}%
\pgfpathlineto{\pgfqpoint{4.689266in}{4.186089in}}%
\pgfpathlineto{\pgfqpoint{4.613151in}{4.272715in}}%
\pgfpathclose%
\pgfusepath{fill}%
\end{pgfscope}%
\begin{pgfscope}%
\pgfpathrectangle{\pgfqpoint{0.539299in}{0.078740in}}{\pgfqpoint{7.842520in}{7.842520in}}%
\pgfusepath{clip}%
\pgfsetbuttcap%
\pgfsetroundjoin%
\definecolor{currentfill}{rgb}{0.212395,0.359683,0.551710}%
\pgfsetfillcolor{currentfill}%
\pgfsetlinewidth{0.000000pt}%
\definecolor{currentstroke}{rgb}{0.129933,0.559582,0.551864}%
\pgfsetstrokecolor{currentstroke}%
\pgfsetdash{}{0pt}%
\pgfpathmoveto{\pgfqpoint{4.293076in}{4.516692in}}%
\pgfpathlineto{\pgfqpoint{4.215129in}{4.619034in}}%
\pgfpathlineto{\pgfqpoint{4.347630in}{4.516065in}}%
\pgfpathclose%
\pgfusepath{fill}%
\end{pgfscope}%
\begin{pgfscope}%
\pgfpathrectangle{\pgfqpoint{0.539299in}{0.078740in}}{\pgfqpoint{7.842520in}{7.842520in}}%
\pgfusepath{clip}%
\pgfsetbuttcap%
\pgfsetroundjoin%
\definecolor{currentfill}{rgb}{0.252194,0.269783,0.531579}%
\pgfsetfillcolor{currentfill}%
\pgfsetlinewidth{0.000000pt}%
\definecolor{currentstroke}{rgb}{0.128729,0.563265,0.551229}%
\pgfsetstrokecolor{currentstroke}%
\pgfsetdash{}{0pt}%
\pgfpathmoveto{\pgfqpoint{3.242761in}{4.478380in}}%
\pgfpathlineto{\pgfqpoint{3.119658in}{4.133109in}}%
\pgfpathlineto{\pgfqpoint{3.038801in}{4.091418in}}%
\pgfpathclose%
\pgfusepath{fill}%
\end{pgfscope}%
\begin{pgfscope}%
\pgfpathrectangle{\pgfqpoint{0.539299in}{0.078740in}}{\pgfqpoint{7.842520in}{7.842520in}}%
\pgfusepath{clip}%
\pgfsetbuttcap%
\pgfsetroundjoin%
\definecolor{currentfill}{rgb}{0.281924,0.089666,0.412415}%
\pgfsetfillcolor{currentfill}%
\pgfsetlinewidth{0.000000pt}%
\definecolor{currentstroke}{rgb}{0.127568,0.566949,0.550556}%
\pgfsetstrokecolor{currentstroke}%
\pgfsetdash{}{0pt}%
\pgfpathmoveto{\pgfqpoint{5.695175in}{3.642100in}}%
\pgfpathlineto{\pgfqpoint{5.767446in}{3.643788in}}%
\pgfpathlineto{\pgfqpoint{5.560913in}{3.680214in}}%
\pgfpathclose%
\pgfusepath{fill}%
\end{pgfscope}%
\begin{pgfscope}%
\pgfpathrectangle{\pgfqpoint{0.539299in}{0.078740in}}{\pgfqpoint{7.842520in}{7.842520in}}%
\pgfusepath{clip}%
\pgfsetbuttcap%
\pgfsetroundjoin%
\definecolor{currentfill}{rgb}{0.277941,0.056324,0.381191}%
\pgfsetfillcolor{currentfill}%
\pgfsetlinewidth{0.000000pt}%
\definecolor{currentstroke}{rgb}{0.126453,0.570633,0.549841}%
\pgfsetstrokecolor{currentstroke}%
\pgfsetdash{}{0pt}%
\pgfpathmoveto{\pgfqpoint{6.107631in}{3.575340in}}%
\pgfpathlineto{\pgfqpoint{6.171939in}{3.541802in}}%
\pgfpathlineto{\pgfqpoint{6.242456in}{3.538310in}}%
\pgfpathclose%
\pgfusepath{fill}%
\end{pgfscope}%
\begin{pgfscope}%
\pgfpathrectangle{\pgfqpoint{0.539299in}{0.078740in}}{\pgfqpoint{7.842520in}{7.842520in}}%
\pgfusepath{clip}%
\pgfsetbuttcap%
\pgfsetroundjoin%
\definecolor{currentfill}{rgb}{0.187231,0.414746,0.556547}%
\pgfsetfillcolor{currentfill}%
\pgfsetlinewidth{0.000000pt}%
\definecolor{currentstroke}{rgb}{0.125394,0.574318,0.549086}%
\pgfsetstrokecolor{currentstroke}%
\pgfsetdash{}{0pt}%
\pgfpathmoveto{\pgfqpoint{3.871838in}{4.810044in}}%
\pgfpathlineto{\pgfqpoint{3.820757in}{4.723273in}}%
\pgfpathlineto{\pgfqpoint{3.740970in}{4.779433in}}%
\pgfpathclose%
\pgfusepath{fill}%
\end{pgfscope}%
\begin{pgfscope}%
\pgfpathrectangle{\pgfqpoint{0.539299in}{0.078740in}}{\pgfqpoint{7.842520in}{7.842520in}}%
\pgfusepath{clip}%
\pgfsetbuttcap%
\pgfsetroundjoin%
\definecolor{currentfill}{rgb}{0.281446,0.084320,0.407414}%
\pgfsetfillcolor{currentfill}%
\pgfsetlinewidth{0.000000pt}%
\definecolor{currentstroke}{rgb}{0.124395,0.578002,0.548287}%
\pgfsetstrokecolor{currentstroke}%
\pgfsetdash{}{0pt}%
\pgfpathmoveto{\pgfqpoint{5.767446in}{3.643788in}}%
\pgfpathlineto{\pgfqpoint{5.829924in}{3.607475in}}%
\pgfpathlineto{\pgfqpoint{5.901860in}{3.611134in}}%
\pgfpathclose%
\pgfusepath{fill}%
\end{pgfscope}%
\begin{pgfscope}%
\pgfpathrectangle{\pgfqpoint{0.539299in}{0.078740in}}{\pgfqpoint{7.842520in}{7.842520in}}%
\pgfusepath{clip}%
\pgfsetbuttcap%
\pgfsetroundjoin%
\definecolor{currentfill}{rgb}{0.223925,0.334994,0.548053}%
\pgfsetfillcolor{currentfill}%
\pgfsetlinewidth{0.000000pt}%
\definecolor{currentstroke}{rgb}{0.123463,0.581687,0.547445}%
\pgfsetstrokecolor{currentstroke}%
\pgfsetdash{}{0pt}%
\pgfpathmoveto{\pgfqpoint{4.480330in}{4.397499in}}%
\pgfpathlineto{\pgfqpoint{4.424973in}{4.414618in}}%
\pgfpathlineto{\pgfqpoint{4.347630in}{4.516065in}}%
\pgfpathclose%
\pgfusepath{fill}%
\end{pgfscope}%
\begin{pgfscope}%
\pgfpathrectangle{\pgfqpoint{0.539299in}{0.078740in}}{\pgfqpoint{7.842520in}{7.842520in}}%
\pgfusepath{clip}%
\pgfsetbuttcap%
\pgfsetroundjoin%
\definecolor{currentfill}{rgb}{0.270595,0.214069,0.507052}%
\pgfsetfillcolor{currentfill}%
\pgfsetlinewidth{0.000000pt}%
\definecolor{currentstroke}{rgb}{0.122606,0.585371,0.546557}%
\pgfsetstrokecolor{currentstroke}%
\pgfsetdash{}{0pt}%
\pgfpathmoveto{\pgfqpoint{4.879148in}{4.033520in}}%
\pgfpathlineto{\pgfqpoint{4.954191in}{3.972092in}}%
\pgfpathlineto{\pgfqpoint{4.821630in}{4.074629in}}%
\pgfpathclose%
\pgfusepath{fill}%
\end{pgfscope}%
\begin{pgfscope}%
\pgfpathrectangle{\pgfqpoint{0.539299in}{0.078740in}}{\pgfqpoint{7.842520in}{7.842520in}}%
\pgfusepath{clip}%
\pgfsetbuttcap%
\pgfsetroundjoin%
\definecolor{currentfill}{rgb}{0.278791,0.062145,0.386592}%
\pgfsetfillcolor{currentfill}%
\pgfsetlinewidth{0.000000pt}%
\definecolor{currentstroke}{rgb}{0.121831,0.589055,0.545623}%
\pgfsetstrokecolor{currentstroke}%
\pgfsetdash{}{0pt}%
\pgfpathmoveto{\pgfqpoint{6.036706in}{3.577796in}}%
\pgfpathlineto{\pgfqpoint{6.171939in}{3.541802in}}%
\pgfpathlineto{\pgfqpoint{6.107631in}{3.575340in}}%
\pgfpathclose%
\pgfusepath{fill}%
\end{pgfscope}%
\begin{pgfscope}%
\pgfpathrectangle{\pgfqpoint{0.539299in}{0.078740in}}{\pgfqpoint{7.842520in}{7.842520in}}%
\pgfusepath{clip}%
\pgfsetbuttcap%
\pgfsetroundjoin%
\definecolor{currentfill}{rgb}{0.283072,0.130895,0.449241}%
\pgfsetfillcolor{currentfill}%
\pgfsetlinewidth{0.000000pt}%
\definecolor{currentstroke}{rgb}{0.121148,0.592739,0.544641}%
\pgfsetstrokecolor{currentstroke}%
\pgfsetdash{}{0pt}%
\pgfpathmoveto{\pgfqpoint{5.353780in}{3.739197in}}%
\pgfpathlineto{\pgfqpoint{5.427151in}{3.724780in}}%
\pgfpathlineto{\pgfqpoint{5.220192in}{3.803938in}}%
\pgfpathclose%
\pgfusepath{fill}%
\end{pgfscope}%
\begin{pgfscope}%
\pgfpathrectangle{\pgfqpoint{0.539299in}{0.078740in}}{\pgfqpoint{7.842520in}{7.842520in}}%
\pgfusepath{clip}%
\pgfsetbuttcap%
\pgfsetroundjoin%
\definecolor{currentfill}{rgb}{0.187231,0.414746,0.556547}%
\pgfsetfillcolor{currentfill}%
\pgfsetlinewidth{0.000000pt}%
\definecolor{currentstroke}{rgb}{0.120565,0.596422,0.543611}%
\pgfsetstrokecolor{currentstroke}%
\pgfsetdash{}{0pt}%
\pgfpathmoveto{\pgfqpoint{3.951415in}{4.734438in}}%
\pgfpathlineto{\pgfqpoint{3.871838in}{4.810044in}}%
\pgfpathlineto{\pgfqpoint{4.082979in}{4.695694in}}%
\pgfpathclose%
\pgfusepath{fill}%
\end{pgfscope}%
\begin{pgfscope}%
\pgfpathrectangle{\pgfqpoint{0.539299in}{0.078740in}}{\pgfqpoint{7.842520in}{7.842520in}}%
\pgfusepath{clip}%
\pgfsetbuttcap%
\pgfsetroundjoin%
\definecolor{currentfill}{rgb}{0.283091,0.110553,0.431554}%
\pgfsetfillcolor{currentfill}%
\pgfsetlinewidth{0.000000pt}%
\definecolor{currentstroke}{rgb}{0.120092,0.600104,0.542530}%
\pgfsetstrokecolor{currentstroke}%
\pgfsetdash{}{0pt}%
\pgfpathmoveto{\pgfqpoint{5.487835in}{3.685615in}}%
\pgfpathlineto{\pgfqpoint{5.560913in}{3.680214in}}%
\pgfpathlineto{\pgfqpoint{5.427151in}{3.724780in}}%
\pgfpathclose%
\pgfusepath{fill}%
\end{pgfscope}%
\begin{pgfscope}%
\pgfpathrectangle{\pgfqpoint{0.539299in}{0.078740in}}{\pgfqpoint{7.842520in}{7.842520in}}%
\pgfusepath{clip}%
\pgfsetbuttcap%
\pgfsetroundjoin%
\definecolor{currentfill}{rgb}{0.281924,0.089666,0.412415}%
\pgfsetfillcolor{currentfill}%
\pgfsetlinewidth{0.000000pt}%
\definecolor{currentstroke}{rgb}{0.119738,0.603785,0.541400}%
\pgfsetstrokecolor{currentstroke}%
\pgfsetdash{}{0pt}%
\pgfpathmoveto{\pgfqpoint{5.695175in}{3.642100in}}%
\pgfpathlineto{\pgfqpoint{5.829924in}{3.607475in}}%
\pgfpathlineto{\pgfqpoint{5.767446in}{3.643788in}}%
\pgfpathclose%
\pgfusepath{fill}%
\end{pgfscope}%
\begin{pgfscope}%
\pgfpathrectangle{\pgfqpoint{0.539299in}{0.078740in}}{\pgfqpoint{7.842520in}{7.842520in}}%
\pgfusepath{clip}%
\pgfsetbuttcap%
\pgfsetroundjoin%
\definecolor{currentfill}{rgb}{0.277941,0.056324,0.381191}%
\pgfsetfillcolor{currentfill}%
\pgfsetlinewidth{0.000000pt}%
\definecolor{currentstroke}{rgb}{0.119512,0.607464,0.540218}%
\pgfsetstrokecolor{currentstroke}%
\pgfsetdash{}{0pt}%
\pgfpathmoveto{\pgfqpoint{6.242456in}{3.538310in}}%
\pgfpathlineto{\pgfqpoint{6.171939in}{3.541802in}}%
\pgfpathlineto{\pgfqpoint{6.377643in}{3.498447in}}%
\pgfpathclose%
\pgfusepath{fill}%
\end{pgfscope}%
\begin{pgfscope}%
\pgfpathrectangle{\pgfqpoint{0.539299in}{0.078740in}}{\pgfqpoint{7.842520in}{7.842520in}}%
\pgfusepath{clip}%
\pgfsetbuttcap%
\pgfsetroundjoin%
\definecolor{currentfill}{rgb}{0.241237,0.296485,0.539709}%
\pgfsetfillcolor{currentfill}%
\pgfsetlinewidth{0.000000pt}%
\definecolor{currentstroke}{rgb}{0.119423,0.611141,0.538982}%
\pgfsetstrokecolor{currentstroke}%
\pgfsetdash{}{0pt}%
\pgfpathmoveto{\pgfqpoint{4.613151in}{4.272715in}}%
\pgfpathlineto{\pgfqpoint{4.689266in}{4.186089in}}%
\pgfpathlineto{\pgfqpoint{4.480330in}{4.397499in}}%
\pgfpathclose%
\pgfusepath{fill}%
\end{pgfscope}%
\begin{pgfscope}%
\pgfpathrectangle{\pgfqpoint{0.539299in}{0.078740in}}{\pgfqpoint{7.842520in}{7.842520in}}%
\pgfusepath{clip}%
\pgfsetbuttcap%
\pgfsetroundjoin%
\definecolor{currentfill}{rgb}{0.185556,0.418570,0.556753}%
\pgfsetfillcolor{currentfill}%
\pgfsetlinewidth{0.000000pt}%
\definecolor{currentstroke}{rgb}{0.119483,0.614817,0.537692}%
\pgfsetstrokecolor{currentstroke}%
\pgfsetdash{}{0pt}%
\pgfpathmoveto{\pgfqpoint{3.611618in}{4.681644in}}%
\pgfpathlineto{\pgfqpoint{3.660600in}{4.826856in}}%
\pgfpathlineto{\pgfqpoint{3.740970in}{4.779433in}}%
\pgfpathclose%
\pgfusepath{fill}%
\end{pgfscope}%
\begin{pgfscope}%
\pgfpathrectangle{\pgfqpoint{0.539299in}{0.078740in}}{\pgfqpoint{7.842520in}{7.842520in}}%
\pgfusepath{clip}%
\pgfsetbuttcap%
\pgfsetroundjoin%
\definecolor{currentfill}{rgb}{0.187231,0.414746,0.556547}%
\pgfsetfillcolor{currentfill}%
\pgfsetlinewidth{0.000000pt}%
\definecolor{currentstroke}{rgb}{0.119699,0.618490,0.536347}%
\pgfsetstrokecolor{currentstroke}%
\pgfsetdash{}{0pt}%
\pgfpathmoveto{\pgfqpoint{3.611618in}{4.681644in}}%
\pgfpathlineto{\pgfqpoint{3.531309in}{4.704589in}}%
\pgfpathlineto{\pgfqpoint{3.660600in}{4.826856in}}%
\pgfpathclose%
\pgfusepath{fill}%
\end{pgfscope}%
\begin{pgfscope}%
\pgfpathrectangle{\pgfqpoint{0.539299in}{0.078740in}}{\pgfqpoint{7.842520in}{7.842520in}}%
\pgfusepath{clip}%
\pgfsetbuttcap%
\pgfsetroundjoin%
\definecolor{currentfill}{rgb}{0.276194,0.190074,0.493001}%
\pgfsetfillcolor{currentfill}%
\pgfsetlinewidth{0.000000pt}%
\definecolor{currentstroke}{rgb}{0.120081,0.622161,0.534946}%
\pgfsetstrokecolor{currentstroke}%
\pgfsetdash{}{0pt}%
\pgfpathmoveto{\pgfqpoint{4.954191in}{3.972092in}}%
\pgfpathlineto{\pgfqpoint{5.012423in}{3.928911in}}%
\pgfpathlineto{\pgfqpoint{5.087019in}{3.881455in}}%
\pgfpathclose%
\pgfusepath{fill}%
\end{pgfscope}%
\begin{pgfscope}%
\pgfpathrectangle{\pgfqpoint{0.539299in}{0.078740in}}{\pgfqpoint{7.842520in}{7.842520in}}%
\pgfusepath{clip}%
\pgfsetbuttcap%
\pgfsetroundjoin%
\definecolor{currentfill}{rgb}{0.253935,0.265254,0.529983}%
\pgfsetfillcolor{currentfill}%
\pgfsetlinewidth{0.000000pt}%
\definecolor{currentstroke}{rgb}{0.120638,0.625828,0.533488}%
\pgfsetstrokecolor{currentstroke}%
\pgfsetdash{}{0pt}%
\pgfpathmoveto{\pgfqpoint{4.613151in}{4.272715in}}%
\pgfpathlineto{\pgfqpoint{4.746079in}{4.149433in}}%
\pgfpathlineto{\pgfqpoint{4.821630in}{4.074629in}}%
\pgfpathclose%
\pgfusepath{fill}%
\end{pgfscope}%
\begin{pgfscope}%
\pgfpathrectangle{\pgfqpoint{0.539299in}{0.078740in}}{\pgfqpoint{7.842520in}{7.842520in}}%
\pgfusepath{clip}%
\pgfsetbuttcap%
\pgfsetroundjoin%
\definecolor{currentfill}{rgb}{0.283187,0.125848,0.444960}%
\pgfsetfillcolor{currentfill}%
\pgfsetlinewidth{0.000000pt}%
\definecolor{currentstroke}{rgb}{0.121380,0.629492,0.531973}%
\pgfsetstrokecolor{currentstroke}%
\pgfsetdash{}{0pt}%
\pgfpathmoveto{\pgfqpoint{5.353780in}{3.739197in}}%
\pgfpathlineto{\pgfqpoint{5.487835in}{3.685615in}}%
\pgfpathlineto{\pgfqpoint{5.427151in}{3.724780in}}%
\pgfpathclose%
\pgfusepath{fill}%
\end{pgfscope}%
\begin{pgfscope}%
\pgfpathrectangle{\pgfqpoint{0.539299in}{0.078740in}}{\pgfqpoint{7.842520in}{7.842520in}}%
\pgfusepath{clip}%
\pgfsetbuttcap%
\pgfsetroundjoin%
\definecolor{currentfill}{rgb}{0.262138,0.242286,0.520837}%
\pgfsetfillcolor{currentfill}%
\pgfsetlinewidth{0.000000pt}%
\definecolor{currentstroke}{rgb}{0.122312,0.633153,0.530398}%
\pgfsetstrokecolor{currentstroke}%
\pgfsetdash{}{0pt}%
\pgfpathmoveto{\pgfqpoint{4.821630in}{4.074629in}}%
\pgfpathlineto{\pgfqpoint{4.746079in}{4.149433in}}%
\pgfpathlineto{\pgfqpoint{4.879148in}{4.033520in}}%
\pgfpathclose%
\pgfusepath{fill}%
\end{pgfscope}%
\begin{pgfscope}%
\pgfpathrectangle{\pgfqpoint{0.539299in}{0.078740in}}{\pgfqpoint{7.842520in}{7.842520in}}%
\pgfusepath{clip}%
\pgfsetbuttcap%
\pgfsetroundjoin%
\definecolor{currentfill}{rgb}{0.278826,0.175490,0.483397}%
\pgfsetfillcolor{currentfill}%
\pgfsetlinewidth{0.000000pt}%
\definecolor{currentstroke}{rgb}{0.123444,0.636809,0.528763}%
\pgfsetstrokecolor{currentstroke}%
\pgfsetdash{}{0pt}%
\pgfpathmoveto{\pgfqpoint{5.087019in}{3.881455in}}%
\pgfpathlineto{\pgfqpoint{5.012423in}{3.928911in}}%
\pgfpathlineto{\pgfqpoint{5.220192in}{3.803938in}}%
\pgfpathclose%
\pgfusepath{fill}%
\end{pgfscope}%
\begin{pgfscope}%
\pgfpathrectangle{\pgfqpoint{0.539299in}{0.078740in}}{\pgfqpoint{7.842520in}{7.842520in}}%
\pgfusepath{clip}%
\pgfsetbuttcap%
\pgfsetroundjoin%
\definecolor{currentfill}{rgb}{0.281446,0.084320,0.407414}%
\pgfsetfillcolor{currentfill}%
\pgfsetlinewidth{0.000000pt}%
\definecolor{currentstroke}{rgb}{0.124780,0.640461,0.527068}%
\pgfsetstrokecolor{currentstroke}%
\pgfsetdash{}{0pt}%
\pgfpathmoveto{\pgfqpoint{5.965125in}{3.573380in}}%
\pgfpathlineto{\pgfqpoint{6.036706in}{3.577796in}}%
\pgfpathlineto{\pgfqpoint{5.901860in}{3.611134in}}%
\pgfpathclose%
\pgfusepath{fill}%
\end{pgfscope}%
\begin{pgfscope}%
\pgfpathrectangle{\pgfqpoint{0.539299in}{0.078740in}}{\pgfqpoint{7.842520in}{7.842520in}}%
\pgfusepath{clip}%
\pgfsetbuttcap%
\pgfsetroundjoin%
\definecolor{currentfill}{rgb}{0.195860,0.395433,0.555276}%
\pgfsetfillcolor{currentfill}%
\pgfsetlinewidth{0.000000pt}%
\definecolor{currentstroke}{rgb}{0.126326,0.644107,0.525311}%
\pgfsetstrokecolor{currentstroke}%
\pgfsetdash{}{0pt}%
\pgfpathmoveto{\pgfqpoint{3.323758in}{4.491822in}}%
\pgfpathlineto{\pgfqpoint{3.450467in}{4.721409in}}%
\pgfpathlineto{\pgfqpoint{3.531309in}{4.704589in}}%
\pgfpathclose%
\pgfusepath{fill}%
\end{pgfscope}%
\begin{pgfscope}%
\pgfpathrectangle{\pgfqpoint{0.539299in}{0.078740in}}{\pgfqpoint{7.842520in}{7.842520in}}%
\pgfusepath{clip}%
\pgfsetbuttcap%
\pgfsetroundjoin%
\definecolor{currentfill}{rgb}{0.283091,0.110553,0.431554}%
\pgfsetfillcolor{currentfill}%
\pgfsetlinewidth{0.000000pt}%
\definecolor{currentstroke}{rgb}{0.128087,0.647749,0.523491}%
\pgfsetstrokecolor{currentstroke}%
\pgfsetdash{}{0pt}%
\pgfpathmoveto{\pgfqpoint{5.560913in}{3.680214in}}%
\pgfpathlineto{\pgfqpoint{5.622384in}{3.640649in}}%
\pgfpathlineto{\pgfqpoint{5.695175in}{3.642100in}}%
\pgfpathclose%
\pgfusepath{fill}%
\end{pgfscope}%
\begin{pgfscope}%
\pgfpathrectangle{\pgfqpoint{0.539299in}{0.078740in}}{\pgfqpoint{7.842520in}{7.842520in}}%
\pgfusepath{clip}%
\pgfsetbuttcap%
\pgfsetroundjoin%
\definecolor{currentfill}{rgb}{0.270595,0.214069,0.507052}%
\pgfsetfillcolor{currentfill}%
\pgfsetlinewidth{0.000000pt}%
\definecolor{currentstroke}{rgb}{0.130067,0.651384,0.521608}%
\pgfsetstrokecolor{currentstroke}%
\pgfsetdash{}{0pt}%
\pgfpathmoveto{\pgfqpoint{4.879148in}{4.033520in}}%
\pgfpathlineto{\pgfqpoint{5.012423in}{3.928911in}}%
\pgfpathlineto{\pgfqpoint{4.954191in}{3.972092in}}%
\pgfpathclose%
\pgfusepath{fill}%
\end{pgfscope}%
\begin{pgfscope}%
\pgfpathrectangle{\pgfqpoint{0.539299in}{0.078740in}}{\pgfqpoint{7.842520in}{7.842520in}}%
\pgfusepath{clip}%
\pgfsetbuttcap%
\pgfsetroundjoin%
\definecolor{currentfill}{rgb}{0.190631,0.407061,0.556089}%
\pgfsetfillcolor{currentfill}%
\pgfsetlinewidth{0.000000pt}%
\definecolor{currentstroke}{rgb}{0.132268,0.655014,0.519661}%
\pgfsetstrokecolor{currentstroke}%
\pgfsetdash{}{0pt}%
\pgfpathmoveto{\pgfqpoint{4.136438in}{4.716235in}}%
\pgfpathlineto{\pgfqpoint{4.215129in}{4.619034in}}%
\pgfpathlineto{\pgfqpoint{4.082979in}{4.695694in}}%
\pgfpathclose%
\pgfusepath{fill}%
\end{pgfscope}%
\begin{pgfscope}%
\pgfpathrectangle{\pgfqpoint{0.539299in}{0.078740in}}{\pgfqpoint{7.842520in}{7.842520in}}%
\pgfusepath{clip}%
\pgfsetbuttcap%
\pgfsetroundjoin%
\definecolor{currentfill}{rgb}{0.282327,0.094955,0.417331}%
\pgfsetfillcolor{currentfill}%
\pgfsetlinewidth{0.000000pt}%
\definecolor{currentstroke}{rgb}{0.134692,0.658636,0.517649}%
\pgfsetstrokecolor{currentstroke}%
\pgfsetdash{}{0pt}%
\pgfpathmoveto{\pgfqpoint{5.901860in}{3.611134in}}%
\pgfpathlineto{\pgfqpoint{5.829924in}{3.607475in}}%
\pgfpathlineto{\pgfqpoint{5.965125in}{3.573380in}}%
\pgfpathclose%
\pgfusepath{fill}%
\end{pgfscope}%
\begin{pgfscope}%
\pgfpathrectangle{\pgfqpoint{0.539299in}{0.078740in}}{\pgfqpoint{7.842520in}{7.842520in}}%
\pgfusepath{clip}%
\pgfsetbuttcap%
\pgfsetroundjoin%
\definecolor{currentfill}{rgb}{0.279566,0.067836,0.391917}%
\pgfsetfillcolor{currentfill}%
\pgfsetlinewidth{0.000000pt}%
\definecolor{currentstroke}{rgb}{0.137339,0.662252,0.515571}%
\pgfsetstrokecolor{currentstroke}%
\pgfsetdash{}{0pt}%
\pgfpathmoveto{\pgfqpoint{6.171939in}{3.541802in}}%
\pgfpathlineto{\pgfqpoint{6.307514in}{3.501723in}}%
\pgfpathlineto{\pgfqpoint{6.377643in}{3.498447in}}%
\pgfpathclose%
\pgfusepath{fill}%
\end{pgfscope}%
\begin{pgfscope}%
\pgfpathrectangle{\pgfqpoint{0.539299in}{0.078740in}}{\pgfqpoint{7.842520in}{7.842520in}}%
\pgfusepath{clip}%
\pgfsetbuttcap%
\pgfsetroundjoin%
\definecolor{currentfill}{rgb}{0.203063,0.379716,0.553925}%
\pgfsetfillcolor{currentfill}%
\pgfsetlinewidth{0.000000pt}%
\definecolor{currentstroke}{rgb}{0.140210,0.665859,0.513427}%
\pgfsetstrokecolor{currentstroke}%
\pgfsetdash{}{0pt}%
\pgfpathmoveto{\pgfqpoint{3.242761in}{4.478380in}}%
\pgfpathlineto{\pgfqpoint{3.450467in}{4.721409in}}%
\pgfpathlineto{\pgfqpoint{3.323758in}{4.491822in}}%
\pgfpathclose%
\pgfusepath{fill}%
\end{pgfscope}%
\begin{pgfscope}%
\pgfpathrectangle{\pgfqpoint{0.539299in}{0.078740in}}{\pgfqpoint{7.842520in}{7.842520in}}%
\pgfusepath{clip}%
\pgfsetbuttcap%
\pgfsetroundjoin%
\definecolor{currentfill}{rgb}{0.180629,0.429975,0.557282}%
\pgfsetfillcolor{currentfill}%
\pgfsetlinewidth{0.000000pt}%
\definecolor{currentstroke}{rgb}{0.143303,0.669459,0.511215}%
\pgfsetstrokecolor{currentstroke}%
\pgfsetdash{}{0pt}%
\pgfpathmoveto{\pgfqpoint{4.082979in}{4.695694in}}%
\pgfpathlineto{\pgfqpoint{3.871838in}{4.810044in}}%
\pgfpathlineto{\pgfqpoint{4.003782in}{4.784962in}}%
\pgfpathclose%
\pgfusepath{fill}%
\end{pgfscope}%
\begin{pgfscope}%
\pgfpathrectangle{\pgfqpoint{0.539299in}{0.078740in}}{\pgfqpoint{7.842520in}{7.842520in}}%
\pgfusepath{clip}%
\pgfsetbuttcap%
\pgfsetroundjoin%
\definecolor{currentfill}{rgb}{0.277018,0.050344,0.375715}%
\pgfsetfillcolor{currentfill}%
\pgfsetlinewidth{0.000000pt}%
\definecolor{currentstroke}{rgb}{0.146616,0.673050,0.508936}%
\pgfsetstrokecolor{currentstroke}%
\pgfsetdash{}{0pt}%
\pgfpathmoveto{\pgfqpoint{6.377643in}{3.498447in}}%
\pgfpathlineto{\pgfqpoint{6.443391in}{3.456772in}}%
\pgfpathlineto{\pgfqpoint{6.513170in}{3.455462in}}%
\pgfpathclose%
\pgfusepath{fill}%
\end{pgfscope}%
\begin{pgfscope}%
\pgfpathrectangle{\pgfqpoint{0.539299in}{0.078740in}}{\pgfqpoint{7.842520in}{7.842520in}}%
\pgfusepath{clip}%
\pgfsetbuttcap%
\pgfsetroundjoin%
\definecolor{currentfill}{rgb}{0.283229,0.120777,0.440584}%
\pgfsetfillcolor{currentfill}%
\pgfsetlinewidth{0.000000pt}%
\definecolor{currentstroke}{rgb}{0.150148,0.676631,0.506589}%
\pgfsetstrokecolor{currentstroke}%
\pgfsetdash{}{0pt}%
\pgfpathmoveto{\pgfqpoint{5.560913in}{3.680214in}}%
\pgfpathlineto{\pgfqpoint{5.487835in}{3.685615in}}%
\pgfpathlineto{\pgfqpoint{5.622384in}{3.640649in}}%
\pgfpathclose%
\pgfusepath{fill}%
\end{pgfscope}%
\begin{pgfscope}%
\pgfpathrectangle{\pgfqpoint{0.539299in}{0.078740in}}{\pgfqpoint{7.842520in}{7.842520in}}%
\pgfusepath{clip}%
\pgfsetbuttcap%
\pgfsetroundjoin%
\definecolor{currentfill}{rgb}{0.281446,0.084320,0.407414}%
\pgfsetfillcolor{currentfill}%
\pgfsetlinewidth{0.000000pt}%
\definecolor{currentstroke}{rgb}{0.153894,0.680203,0.504172}%
\pgfsetstrokecolor{currentstroke}%
\pgfsetdash{}{0pt}%
\pgfpathmoveto{\pgfqpoint{5.965125in}{3.573380in}}%
\pgfpathlineto{\pgfqpoint{6.171939in}{3.541802in}}%
\pgfpathlineto{\pgfqpoint{6.036706in}{3.577796in}}%
\pgfpathclose%
\pgfusepath{fill}%
\end{pgfscope}%
\begin{pgfscope}%
\pgfpathrectangle{\pgfqpoint{0.539299in}{0.078740in}}{\pgfqpoint{7.842520in}{7.842520in}}%
\pgfusepath{clip}%
\pgfsetbuttcap%
\pgfsetroundjoin%
\definecolor{currentfill}{rgb}{0.246811,0.283237,0.535941}%
\pgfsetfillcolor{currentfill}%
\pgfsetlinewidth{0.000000pt}%
\definecolor{currentstroke}{rgb}{0.157851,0.683765,0.501686}%
\pgfsetstrokecolor{currentstroke}%
\pgfsetdash{}{0pt}%
\pgfpathmoveto{\pgfqpoint{2.957481in}{4.048102in}}%
\pgfpathlineto{\pgfqpoint{3.161266in}{4.461876in}}%
\pgfpathlineto{\pgfqpoint{3.038801in}{4.091418in}}%
\pgfpathclose%
\pgfusepath{fill}%
\end{pgfscope}%
\begin{pgfscope}%
\pgfpathrectangle{\pgfqpoint{0.539299in}{0.078740in}}{\pgfqpoint{7.842520in}{7.842520in}}%
\pgfusepath{clip}%
\pgfsetbuttcap%
\pgfsetroundjoin%
\definecolor{currentfill}{rgb}{0.227802,0.326594,0.546532}%
\pgfsetfillcolor{currentfill}%
\pgfsetlinewidth{0.000000pt}%
\definecolor{currentstroke}{rgb}{0.162016,0.687316,0.499129}%
\pgfsetstrokecolor{currentstroke}%
\pgfsetdash{}{0pt}%
\pgfpathmoveto{\pgfqpoint{3.038801in}{4.091418in}}%
\pgfpathlineto{\pgfqpoint{3.161266in}{4.461876in}}%
\pgfpathlineto{\pgfqpoint{3.242761in}{4.478380in}}%
\pgfpathclose%
\pgfusepath{fill}%
\end{pgfscope}%
\begin{pgfscope}%
\pgfpathrectangle{\pgfqpoint{0.539299in}{0.078740in}}{\pgfqpoint{7.842520in}{7.842520in}}%
\pgfusepath{clip}%
\pgfsetbuttcap%
\pgfsetroundjoin%
\definecolor{currentfill}{rgb}{0.281412,0.155834,0.469201}%
\pgfsetfillcolor{currentfill}%
\pgfsetlinewidth{0.000000pt}%
\definecolor{currentstroke}{rgb}{0.166383,0.690856,0.496502}%
\pgfsetstrokecolor{currentstroke}%
\pgfsetdash{}{0pt}%
\pgfpathmoveto{\pgfqpoint{5.220192in}{3.803938in}}%
\pgfpathlineto{\pgfqpoint{5.279908in}{3.760172in}}%
\pgfpathlineto{\pgfqpoint{5.353780in}{3.739197in}}%
\pgfpathclose%
\pgfusepath{fill}%
\end{pgfscope}%
\begin{pgfscope}%
\pgfpathrectangle{\pgfqpoint{0.539299in}{0.078740in}}{\pgfqpoint{7.842520in}{7.842520in}}%
\pgfusepath{clip}%
\pgfsetbuttcap%
\pgfsetroundjoin%
\definecolor{currentfill}{rgb}{0.283091,0.110553,0.431554}%
\pgfsetfillcolor{currentfill}%
\pgfsetlinewidth{0.000000pt}%
\definecolor{currentstroke}{rgb}{0.170948,0.694384,0.493803}%
\pgfsetstrokecolor{currentstroke}%
\pgfsetdash{}{0pt}%
\pgfpathmoveto{\pgfqpoint{5.622384in}{3.640649in}}%
\pgfpathlineto{\pgfqpoint{5.829924in}{3.607475in}}%
\pgfpathlineto{\pgfqpoint{5.695175in}{3.642100in}}%
\pgfpathclose%
\pgfusepath{fill}%
\end{pgfscope}%
\begin{pgfscope}%
\pgfpathrectangle{\pgfqpoint{0.539299in}{0.078740in}}{\pgfqpoint{7.842520in}{7.842520in}}%
\pgfusepath{clip}%
\pgfsetbuttcap%
\pgfsetroundjoin%
\definecolor{currentfill}{rgb}{0.199430,0.387607,0.554642}%
\pgfsetfillcolor{currentfill}%
\pgfsetlinewidth{0.000000pt}%
\definecolor{currentstroke}{rgb}{0.175707,0.697900,0.491033}%
\pgfsetstrokecolor{currentstroke}%
\pgfsetdash{}{0pt}%
\pgfpathmoveto{\pgfqpoint{4.215129in}{4.619034in}}%
\pgfpathlineto{\pgfqpoint{4.269526in}{4.615818in}}%
\pgfpathlineto{\pgfqpoint{4.347630in}{4.516065in}}%
\pgfpathclose%
\pgfusepath{fill}%
\end{pgfscope}%
\begin{pgfscope}%
\pgfpathrectangle{\pgfqpoint{0.539299in}{0.078740in}}{\pgfqpoint{7.842520in}{7.842520in}}%
\pgfusepath{clip}%
\pgfsetbuttcap%
\pgfsetroundjoin%
\definecolor{currentfill}{rgb}{0.277134,0.185228,0.489898}%
\pgfsetfillcolor{currentfill}%
\pgfsetlinewidth{0.000000pt}%
\definecolor{currentstroke}{rgb}{0.180653,0.701402,0.488189}%
\pgfsetstrokecolor{currentstroke}%
\pgfsetdash{}{0pt}%
\pgfpathmoveto{\pgfqpoint{5.012423in}{3.928911in}}%
\pgfpathlineto{\pgfqpoint{5.145984in}{3.837674in}}%
\pgfpathlineto{\pgfqpoint{5.220192in}{3.803938in}}%
\pgfpathclose%
\pgfusepath{fill}%
\end{pgfscope}%
\begin{pgfscope}%
\pgfpathrectangle{\pgfqpoint{0.539299in}{0.078740in}}{\pgfqpoint{7.842520in}{7.842520in}}%
\pgfusepath{clip}%
\pgfsetbuttcap%
\pgfsetroundjoin%
\definecolor{currentfill}{rgb}{0.278791,0.062145,0.386592}%
\pgfsetfillcolor{currentfill}%
\pgfsetlinewidth{0.000000pt}%
\definecolor{currentstroke}{rgb}{0.185783,0.704891,0.485273}%
\pgfsetstrokecolor{currentstroke}%
\pgfsetdash{}{0pt}%
\pgfpathmoveto{\pgfqpoint{6.377643in}{3.498447in}}%
\pgfpathlineto{\pgfqpoint{6.307514in}{3.501723in}}%
\pgfpathlineto{\pgfqpoint{6.443391in}{3.456772in}}%
\pgfpathclose%
\pgfusepath{fill}%
\end{pgfscope}%
\begin{pgfscope}%
\pgfpathrectangle{\pgfqpoint{0.539299in}{0.078740in}}{\pgfqpoint{7.842520in}{7.842520in}}%
\pgfusepath{clip}%
\pgfsetbuttcap%
\pgfsetroundjoin%
\definecolor{currentfill}{rgb}{0.206756,0.371758,0.553117}%
\pgfsetfillcolor{currentfill}%
\pgfsetlinewidth{0.000000pt}%
\definecolor{currentstroke}{rgb}{0.191090,0.708366,0.482284}%
\pgfsetstrokecolor{currentstroke}%
\pgfsetdash{}{0pt}%
\pgfpathmoveto{\pgfqpoint{4.347630in}{4.516065in}}%
\pgfpathlineto{\pgfqpoint{4.269526in}{4.615818in}}%
\pgfpathlineto{\pgfqpoint{4.480330in}{4.397499in}}%
\pgfpathclose%
\pgfusepath{fill}%
\end{pgfscope}%
\begin{pgfscope}%
\pgfpathrectangle{\pgfqpoint{0.539299in}{0.078740in}}{\pgfqpoint{7.842520in}{7.842520in}}%
\pgfusepath{clip}%
\pgfsetbuttcap%
\pgfsetroundjoin%
\definecolor{currentfill}{rgb}{0.180629,0.429975,0.557282}%
\pgfsetfillcolor{currentfill}%
\pgfsetlinewidth{0.000000pt}%
\definecolor{currentstroke}{rgb}{0.196571,0.711827,0.479221}%
\pgfsetstrokecolor{currentstroke}%
\pgfsetdash{}{0pt}%
\pgfpathmoveto{\pgfqpoint{4.082979in}{4.695694in}}%
\pgfpathlineto{\pgfqpoint{4.003782in}{4.784962in}}%
\pgfpathlineto{\pgfqpoint{4.136438in}{4.716235in}}%
\pgfpathclose%
\pgfusepath{fill}%
\end{pgfscope}%
\begin{pgfscope}%
\pgfpathrectangle{\pgfqpoint{0.539299in}{0.078740in}}{\pgfqpoint{7.842520in}{7.842520in}}%
\pgfusepath{clip}%
\pgfsetbuttcap%
\pgfsetroundjoin%
\definecolor{currentfill}{rgb}{0.171176,0.452530,0.557965}%
\pgfsetfillcolor{currentfill}%
\pgfsetlinewidth{0.000000pt}%
\definecolor{currentstroke}{rgb}{0.202219,0.715272,0.476084}%
\pgfsetstrokecolor{currentstroke}%
\pgfsetdash{}{0pt}%
\pgfpathmoveto{\pgfqpoint{3.791614in}{4.877687in}}%
\pgfpathlineto{\pgfqpoint{3.871838in}{4.810044in}}%
\pgfpathlineto{\pgfqpoint{3.740970in}{4.779433in}}%
\pgfpathclose%
\pgfusepath{fill}%
\end{pgfscope}%
\begin{pgfscope}%
\pgfpathrectangle{\pgfqpoint{0.539299in}{0.078740in}}{\pgfqpoint{7.842520in}{7.842520in}}%
\pgfusepath{clip}%
\pgfsetbuttcap%
\pgfsetroundjoin%
\definecolor{currentfill}{rgb}{0.281887,0.150881,0.465405}%
\pgfsetfillcolor{currentfill}%
\pgfsetlinewidth{0.000000pt}%
\definecolor{currentstroke}{rgb}{0.208030,0.718701,0.472873}%
\pgfsetstrokecolor{currentstroke}%
\pgfsetdash{}{0pt}%
\pgfpathmoveto{\pgfqpoint{5.279908in}{3.760172in}}%
\pgfpathlineto{\pgfqpoint{5.487835in}{3.685615in}}%
\pgfpathlineto{\pgfqpoint{5.353780in}{3.739197in}}%
\pgfpathclose%
\pgfusepath{fill}%
\end{pgfscope}%
\begin{pgfscope}%
\pgfpathrectangle{\pgfqpoint{0.539299in}{0.078740in}}{\pgfqpoint{7.842520in}{7.842520in}}%
\pgfusepath{clip}%
\pgfsetbuttcap%
\pgfsetroundjoin%
\definecolor{currentfill}{rgb}{0.279574,0.170599,0.479997}%
\pgfsetfillcolor{currentfill}%
\pgfsetlinewidth{0.000000pt}%
\definecolor{currentstroke}{rgb}{0.214000,0.722114,0.469588}%
\pgfsetstrokecolor{currentstroke}%
\pgfsetdash{}{0pt}%
\pgfpathmoveto{\pgfqpoint{5.220192in}{3.803938in}}%
\pgfpathlineto{\pgfqpoint{5.145984in}{3.837674in}}%
\pgfpathlineto{\pgfqpoint{5.279908in}{3.760172in}}%
\pgfpathclose%
\pgfusepath{fill}%
\end{pgfscope}%
\begin{pgfscope}%
\pgfpathrectangle{\pgfqpoint{0.539299in}{0.078740in}}{\pgfqpoint{7.842520in}{7.842520in}}%
\pgfusepath{clip}%
\pgfsetbuttcap%
\pgfsetroundjoin%
\definecolor{currentfill}{rgb}{0.225863,0.330805,0.547314}%
\pgfsetfillcolor{currentfill}%
\pgfsetlinewidth{0.000000pt}%
\definecolor{currentstroke}{rgb}{0.220124,0.725509,0.466226}%
\pgfsetstrokecolor{currentstroke}%
\pgfsetdash{}{0pt}%
\pgfpathmoveto{\pgfqpoint{4.536301in}{4.363851in}}%
\pgfpathlineto{\pgfqpoint{4.613151in}{4.272715in}}%
\pgfpathlineto{\pgfqpoint{4.480330in}{4.397499in}}%
\pgfpathclose%
\pgfusepath{fill}%
\end{pgfscope}%
\begin{pgfscope}%
\pgfpathrectangle{\pgfqpoint{0.539299in}{0.078740in}}{\pgfqpoint{7.842520in}{7.842520in}}%
\pgfusepath{clip}%
\pgfsetbuttcap%
\pgfsetroundjoin%
\definecolor{currentfill}{rgb}{0.277018,0.050344,0.375715}%
\pgfsetfillcolor{currentfill}%
\pgfsetlinewidth{0.000000pt}%
\definecolor{currentstroke}{rgb}{0.226397,0.728888,0.462789}%
\pgfsetstrokecolor{currentstroke}%
\pgfsetdash{}{0pt}%
\pgfpathmoveto{\pgfqpoint{6.579539in}{3.406827in}}%
\pgfpathlineto{\pgfqpoint{6.649027in}{3.409472in}}%
\pgfpathlineto{\pgfqpoint{6.513170in}{3.455462in}}%
\pgfpathclose%
\pgfusepath{fill}%
\end{pgfscope}%
\begin{pgfscope}%
\pgfpathrectangle{\pgfqpoint{0.539299in}{0.078740in}}{\pgfqpoint{7.842520in}{7.842520in}}%
\pgfusepath{clip}%
\pgfsetbuttcap%
\pgfsetroundjoin%
\definecolor{currentfill}{rgb}{0.169646,0.456262,0.558030}%
\pgfsetfillcolor{currentfill}%
\pgfsetlinewidth{0.000000pt}%
\definecolor{currentstroke}{rgb}{0.232815,0.732247,0.459277}%
\pgfsetstrokecolor{currentstroke}%
\pgfsetdash{}{0pt}%
\pgfpathmoveto{\pgfqpoint{3.740970in}{4.779433in}}%
\pgfpathlineto{\pgfqpoint{3.660600in}{4.826856in}}%
\pgfpathlineto{\pgfqpoint{3.791614in}{4.877687in}}%
\pgfpathclose%
\pgfusepath{fill}%
\end{pgfscope}%
\begin{pgfscope}%
\pgfpathrectangle{\pgfqpoint{0.539299in}{0.078740in}}{\pgfqpoint{7.842520in}{7.842520in}}%
\pgfusepath{clip}%
\pgfsetbuttcap%
\pgfsetroundjoin%
\definecolor{currentfill}{rgb}{0.281446,0.084320,0.407414}%
\pgfsetfillcolor{currentfill}%
\pgfsetlinewidth{0.000000pt}%
\definecolor{currentstroke}{rgb}{0.239374,0.735588,0.455688}%
\pgfsetstrokecolor{currentstroke}%
\pgfsetdash{}{0pt}%
\pgfpathmoveto{\pgfqpoint{6.307514in}{3.501723in}}%
\pgfpathlineto{\pgfqpoint{6.171939in}{3.541802in}}%
\pgfpathlineto{\pgfqpoint{6.100728in}{3.537174in}}%
\pgfpathclose%
\pgfusepath{fill}%
\end{pgfscope}%
\begin{pgfscope}%
\pgfpathrectangle{\pgfqpoint{0.539299in}{0.078740in}}{\pgfqpoint{7.842520in}{7.842520in}}%
\pgfusepath{clip}%
\pgfsetbuttcap%
\pgfsetroundjoin%
\definecolor{currentfill}{rgb}{0.239346,0.300855,0.540844}%
\pgfsetfillcolor{currentfill}%
\pgfsetlinewidth{0.000000pt}%
\definecolor{currentstroke}{rgb}{0.246070,0.738910,0.452024}%
\pgfsetstrokecolor{currentstroke}%
\pgfsetdash{}{0pt}%
\pgfpathmoveto{\pgfqpoint{4.669832in}{4.230954in}}%
\pgfpathlineto{\pgfqpoint{4.746079in}{4.149433in}}%
\pgfpathlineto{\pgfqpoint{4.613151in}{4.272715in}}%
\pgfpathclose%
\pgfusepath{fill}%
\end{pgfscope}%
\begin{pgfscope}%
\pgfpathrectangle{\pgfqpoint{0.539299in}{0.078740in}}{\pgfqpoint{7.842520in}{7.842520in}}%
\pgfusepath{clip}%
\pgfsetbuttcap%
\pgfsetroundjoin%
\definecolor{currentfill}{rgb}{0.282327,0.094955,0.417331}%
\pgfsetfillcolor{currentfill}%
\pgfsetlinewidth{0.000000pt}%
\definecolor{currentstroke}{rgb}{0.252899,0.742211,0.448284}%
\pgfsetstrokecolor{currentstroke}%
\pgfsetdash{}{0pt}%
\pgfpathmoveto{\pgfqpoint{6.100728in}{3.537174in}}%
\pgfpathlineto{\pgfqpoint{6.171939in}{3.541802in}}%
\pgfpathlineto{\pgfqpoint{5.965125in}{3.573380in}}%
\pgfpathclose%
\pgfusepath{fill}%
\end{pgfscope}%
\begin{pgfscope}%
\pgfpathrectangle{\pgfqpoint{0.539299in}{0.078740in}}{\pgfqpoint{7.842520in}{7.842520in}}%
\pgfusepath{clip}%
\pgfsetbuttcap%
\pgfsetroundjoin%
\definecolor{currentfill}{rgb}{0.187231,0.414746,0.556547}%
\pgfsetfillcolor{currentfill}%
\pgfsetlinewidth{0.000000pt}%
\definecolor{currentstroke}{rgb}{0.259857,0.745492,0.444467}%
\pgfsetstrokecolor{currentstroke}%
\pgfsetdash{}{0pt}%
\pgfpathmoveto{\pgfqpoint{4.136438in}{4.716235in}}%
\pgfpathlineto{\pgfqpoint{4.269526in}{4.615818in}}%
\pgfpathlineto{\pgfqpoint{4.215129in}{4.619034in}}%
\pgfpathclose%
\pgfusepath{fill}%
\end{pgfscope}%
\begin{pgfscope}%
\pgfpathrectangle{\pgfqpoint{0.539299in}{0.078740in}}{\pgfqpoint{7.842520in}{7.842520in}}%
\pgfusepath{clip}%
\pgfsetbuttcap%
\pgfsetroundjoin%
\definecolor{currentfill}{rgb}{0.283091,0.110553,0.431554}%
\pgfsetfillcolor{currentfill}%
\pgfsetlinewidth{0.000000pt}%
\definecolor{currentstroke}{rgb}{0.266941,0.748751,0.440573}%
\pgfsetstrokecolor{currentstroke}%
\pgfsetdash{}{0pt}%
\pgfpathmoveto{\pgfqpoint{5.965125in}{3.573380in}}%
\pgfpathlineto{\pgfqpoint{5.829924in}{3.607475in}}%
\pgfpathlineto{\pgfqpoint{5.757427in}{3.601206in}}%
\pgfpathclose%
\pgfusepath{fill}%
\end{pgfscope}%
\begin{pgfscope}%
\pgfpathrectangle{\pgfqpoint{0.539299in}{0.078740in}}{\pgfqpoint{7.842520in}{7.842520in}}%
\pgfusepath{clip}%
\pgfsetbuttcap%
\pgfsetroundjoin%
\definecolor{currentfill}{rgb}{0.248629,0.278775,0.534556}%
\pgfsetfillcolor{currentfill}%
\pgfsetlinewidth{0.000000pt}%
\definecolor{currentstroke}{rgb}{0.274149,0.751988,0.436601}%
\pgfsetstrokecolor{currentstroke}%
\pgfsetdash{}{0pt}%
\pgfpathmoveto{\pgfqpoint{4.879148in}{4.033520in}}%
\pgfpathlineto{\pgfqpoint{4.746079in}{4.149433in}}%
\pgfpathlineto{\pgfqpoint{4.669832in}{4.230954in}}%
\pgfpathclose%
\pgfusepath{fill}%
\end{pgfscope}%
\begin{pgfscope}%
\pgfpathrectangle{\pgfqpoint{0.539299in}{0.078740in}}{\pgfqpoint{7.842520in}{7.842520in}}%
\pgfusepath{clip}%
\pgfsetbuttcap%
\pgfsetroundjoin%
\definecolor{currentfill}{rgb}{0.283229,0.120777,0.440584}%
\pgfsetfillcolor{currentfill}%
\pgfsetlinewidth{0.000000pt}%
\definecolor{currentstroke}{rgb}{0.281477,0.755203,0.432552}%
\pgfsetstrokecolor{currentstroke}%
\pgfsetdash{}{0pt}%
\pgfpathmoveto{\pgfqpoint{5.757427in}{3.601206in}}%
\pgfpathlineto{\pgfqpoint{5.829924in}{3.607475in}}%
\pgfpathlineto{\pgfqpoint{5.622384in}{3.640649in}}%
\pgfpathclose%
\pgfusepath{fill}%
\end{pgfscope}%
\begin{pgfscope}%
\pgfpathrectangle{\pgfqpoint{0.539299in}{0.078740in}}{\pgfqpoint{7.842520in}{7.842520in}}%
\pgfusepath{clip}%
\pgfsetbuttcap%
\pgfsetroundjoin%
\definecolor{currentfill}{rgb}{0.171176,0.452530,0.557965}%
\pgfsetfillcolor{currentfill}%
\pgfsetlinewidth{0.000000pt}%
\definecolor{currentstroke}{rgb}{0.288921,0.758394,0.428426}%
\pgfsetstrokecolor{currentstroke}%
\pgfsetdash{}{0pt}%
\pgfpathmoveto{\pgfqpoint{3.579639in}{4.867750in}}%
\pgfpathlineto{\pgfqpoint{3.660600in}{4.826856in}}%
\pgfpathlineto{\pgfqpoint{3.531309in}{4.704589in}}%
\pgfpathclose%
\pgfusepath{fill}%
\end{pgfscope}%
\begin{pgfscope}%
\pgfpathrectangle{\pgfqpoint{0.539299in}{0.078740in}}{\pgfqpoint{7.842520in}{7.842520in}}%
\pgfusepath{clip}%
\pgfsetbuttcap%
\pgfsetroundjoin%
\definecolor{currentfill}{rgb}{0.278791,0.062145,0.386592}%
\pgfsetfillcolor{currentfill}%
\pgfsetlinewidth{0.000000pt}%
\definecolor{currentstroke}{rgb}{0.296479,0.761561,0.424223}%
\pgfsetstrokecolor{currentstroke}%
\pgfsetdash{}{0pt}%
\pgfpathmoveto{\pgfqpoint{6.513170in}{3.455462in}}%
\pgfpathlineto{\pgfqpoint{6.443391in}{3.456772in}}%
\pgfpathlineto{\pgfqpoint{6.579539in}{3.406827in}}%
\pgfpathclose%
\pgfusepath{fill}%
\end{pgfscope}%
\begin{pgfscope}%
\pgfpathrectangle{\pgfqpoint{0.539299in}{0.078740in}}{\pgfqpoint{7.842520in}{7.842520in}}%
\pgfusepath{clip}%
\pgfsetbuttcap%
\pgfsetroundjoin%
\definecolor{currentfill}{rgb}{0.174274,0.445044,0.557792}%
\pgfsetfillcolor{currentfill}%
\pgfsetlinewidth{0.000000pt}%
\definecolor{currentstroke}{rgb}{0.304148,0.764704,0.419943}%
\pgfsetstrokecolor{currentstroke}%
\pgfsetdash{}{0pt}%
\pgfpathmoveto{\pgfqpoint{3.531309in}{4.704589in}}%
\pgfpathlineto{\pgfqpoint{3.450467in}{4.721409in}}%
\pgfpathlineto{\pgfqpoint{3.579639in}{4.867750in}}%
\pgfpathclose%
\pgfusepath{fill}%
\end{pgfscope}%
\begin{pgfscope}%
\pgfpathrectangle{\pgfqpoint{0.539299in}{0.078740in}}{\pgfqpoint{7.842520in}{7.842520in}}%
\pgfusepath{clip}%
\pgfsetbuttcap%
\pgfsetroundjoin%
\definecolor{currentfill}{rgb}{0.168126,0.459988,0.558082}%
\pgfsetfillcolor{currentfill}%
\pgfsetlinewidth{0.000000pt}%
\definecolor{currentstroke}{rgb}{0.311925,0.767822,0.415586}%
\pgfsetstrokecolor{currentstroke}%
\pgfsetdash{}{0pt}%
\pgfpathmoveto{\pgfqpoint{4.003782in}{4.784962in}}%
\pgfpathlineto{\pgfqpoint{3.871838in}{4.810044in}}%
\pgfpathlineto{\pgfqpoint{3.791614in}{4.877687in}}%
\pgfpathclose%
\pgfusepath{fill}%
\end{pgfscope}%
\begin{pgfscope}%
\pgfpathrectangle{\pgfqpoint{0.539299in}{0.078740in}}{\pgfqpoint{7.842520in}{7.842520in}}%
\pgfusepath{clip}%
\pgfsetbuttcap%
\pgfsetroundjoin%
\definecolor{currentfill}{rgb}{0.276022,0.044167,0.370164}%
\pgfsetfillcolor{currentfill}%
\pgfsetlinewidth{0.000000pt}%
\definecolor{currentstroke}{rgb}{0.319809,0.770914,0.411152}%
\pgfsetstrokecolor{currentstroke}%
\pgfsetdash{}{0pt}%
\pgfpathmoveto{\pgfqpoint{6.785212in}{3.360948in}}%
\pgfpathlineto{\pgfqpoint{6.649027in}{3.409472in}}%
\pgfpathlineto{\pgfqpoint{6.579539in}{3.406827in}}%
\pgfpathclose%
\pgfusepath{fill}%
\end{pgfscope}%
\begin{pgfscope}%
\pgfpathrectangle{\pgfqpoint{0.539299in}{0.078740in}}{\pgfqpoint{7.842520in}{7.842520in}}%
\pgfusepath{clip}%
\pgfsetbuttcap%
\pgfsetroundjoin%
\definecolor{currentfill}{rgb}{0.258965,0.251537,0.524736}%
\pgfsetfillcolor{currentfill}%
\pgfsetlinewidth{0.000000pt}%
\definecolor{currentstroke}{rgb}{0.327796,0.773980,0.406640}%
\pgfsetstrokecolor{currentstroke}%
\pgfsetdash{}{0pt}%
\pgfpathmoveto{\pgfqpoint{4.803456in}{4.103048in}}%
\pgfpathlineto{\pgfqpoint{5.012423in}{3.928911in}}%
\pgfpathlineto{\pgfqpoint{4.879148in}{4.033520in}}%
\pgfpathclose%
\pgfusepath{fill}%
\end{pgfscope}%
\begin{pgfscope}%
\pgfpathrectangle{\pgfqpoint{0.539299in}{0.078740in}}{\pgfqpoint{7.842520in}{7.842520in}}%
\pgfusepath{clip}%
\pgfsetbuttcap%
\pgfsetroundjoin%
\definecolor{currentfill}{rgb}{0.201239,0.383670,0.554294}%
\pgfsetfillcolor{currentfill}%
\pgfsetlinewidth{0.000000pt}%
\definecolor{currentstroke}{rgb}{0.335885,0.777018,0.402049}%
\pgfsetstrokecolor{currentstroke}%
\pgfsetdash{}{0pt}%
\pgfpathmoveto{\pgfqpoint{4.480330in}{4.397499in}}%
\pgfpathlineto{\pgfqpoint{4.269526in}{4.615818in}}%
\pgfpathlineto{\pgfqpoint{4.402851in}{4.494989in}}%
\pgfpathclose%
\pgfusepath{fill}%
\end{pgfscope}%
\begin{pgfscope}%
\pgfpathrectangle{\pgfqpoint{0.539299in}{0.078740in}}{\pgfqpoint{7.842520in}{7.842520in}}%
\pgfusepath{clip}%
\pgfsetbuttcap%
\pgfsetroundjoin%
\definecolor{currentfill}{rgb}{0.185556,0.418570,0.556753}%
\pgfsetfillcolor{currentfill}%
\pgfsetlinewidth{0.000000pt}%
\definecolor{currentstroke}{rgb}{0.344074,0.780029,0.397381}%
\pgfsetstrokecolor{currentstroke}%
\pgfsetdash{}{0pt}%
\pgfpathmoveto{\pgfqpoint{3.369083in}{4.733483in}}%
\pgfpathlineto{\pgfqpoint{3.450467in}{4.721409in}}%
\pgfpathlineto{\pgfqpoint{3.242761in}{4.478380in}}%
\pgfpathclose%
\pgfusepath{fill}%
\end{pgfscope}%
\begin{pgfscope}%
\pgfpathrectangle{\pgfqpoint{0.539299in}{0.078740in}}{\pgfqpoint{7.842520in}{7.842520in}}%
\pgfusepath{clip}%
\pgfsetbuttcap%
\pgfsetroundjoin%
\definecolor{currentfill}{rgb}{0.210503,0.363727,0.552206}%
\pgfsetfillcolor{currentfill}%
\pgfsetlinewidth{0.000000pt}%
\definecolor{currentstroke}{rgb}{0.352360,0.783011,0.392636}%
\pgfsetstrokecolor{currentstroke}%
\pgfsetdash{}{0pt}%
\pgfpathmoveto{\pgfqpoint{4.480330in}{4.397499in}}%
\pgfpathlineto{\pgfqpoint{4.402851in}{4.494989in}}%
\pgfpathlineto{\pgfqpoint{4.536301in}{4.363851in}}%
\pgfpathclose%
\pgfusepath{fill}%
\end{pgfscope}%
\begin{pgfscope}%
\pgfpathrectangle{\pgfqpoint{0.539299in}{0.078740in}}{\pgfqpoint{7.842520in}{7.842520in}}%
\pgfusepath{clip}%
\pgfsetbuttcap%
\pgfsetroundjoin%
\definecolor{currentfill}{rgb}{0.281412,0.155834,0.469201}%
\pgfsetfillcolor{currentfill}%
\pgfsetlinewidth{0.000000pt}%
\definecolor{currentstroke}{rgb}{0.360741,0.785964,0.387814}%
\pgfsetstrokecolor{currentstroke}%
\pgfsetdash{}{0pt}%
\pgfpathmoveto{\pgfqpoint{5.414261in}{3.695329in}}%
\pgfpathlineto{\pgfqpoint{5.487835in}{3.685615in}}%
\pgfpathlineto{\pgfqpoint{5.279908in}{3.760172in}}%
\pgfpathclose%
\pgfusepath{fill}%
\end{pgfscope}%
\begin{pgfscope}%
\pgfpathrectangle{\pgfqpoint{0.539299in}{0.078740in}}{\pgfqpoint{7.842520in}{7.842520in}}%
\pgfusepath{clip}%
\pgfsetbuttcap%
\pgfsetroundjoin%
\definecolor{currentfill}{rgb}{0.282623,0.140926,0.457517}%
\pgfsetfillcolor{currentfill}%
\pgfsetlinewidth{0.000000pt}%
\definecolor{currentstroke}{rgb}{0.369214,0.788888,0.382914}%
\pgfsetstrokecolor{currentstroke}%
\pgfsetdash{}{0pt}%
\pgfpathmoveto{\pgfqpoint{5.487835in}{3.685615in}}%
\pgfpathlineto{\pgfqpoint{5.549086in}{3.640953in}}%
\pgfpathlineto{\pgfqpoint{5.622384in}{3.640649in}}%
\pgfpathclose%
\pgfusepath{fill}%
\end{pgfscope}%
\begin{pgfscope}%
\pgfpathrectangle{\pgfqpoint{0.539299in}{0.078740in}}{\pgfqpoint{7.842520in}{7.842520in}}%
\pgfusepath{clip}%
\pgfsetbuttcap%
\pgfsetroundjoin%
\definecolor{currentfill}{rgb}{0.225863,0.330805,0.547314}%
\pgfsetfillcolor{currentfill}%
\pgfsetlinewidth{0.000000pt}%
\definecolor{currentstroke}{rgb}{0.377779,0.791781,0.377939}%
\pgfsetstrokecolor{currentstroke}%
\pgfsetdash{}{0pt}%
\pgfpathmoveto{\pgfqpoint{4.669832in}{4.230954in}}%
\pgfpathlineto{\pgfqpoint{4.613151in}{4.272715in}}%
\pgfpathlineto{\pgfqpoint{4.536301in}{4.363851in}}%
\pgfpathclose%
\pgfusepath{fill}%
\end{pgfscope}%
\begin{pgfscope}%
\pgfpathrectangle{\pgfqpoint{0.539299in}{0.078740in}}{\pgfqpoint{7.842520in}{7.842520in}}%
\pgfusepath{clip}%
\pgfsetbuttcap%
\pgfsetroundjoin%
\definecolor{currentfill}{rgb}{0.267968,0.223549,0.512008}%
\pgfsetfillcolor{currentfill}%
\pgfsetlinewidth{0.000000pt}%
\definecolor{currentstroke}{rgb}{0.386433,0.794644,0.372886}%
\pgfsetstrokecolor{currentstroke}%
\pgfsetdash{}{0pt}%
\pgfpathmoveto{\pgfqpoint{5.012423in}{3.928911in}}%
\pgfpathlineto{\pgfqpoint{4.937227in}{3.984970in}}%
\pgfpathlineto{\pgfqpoint{5.145984in}{3.837674in}}%
\pgfpathclose%
\pgfusepath{fill}%
\end{pgfscope}%
\begin{pgfscope}%
\pgfpathrectangle{\pgfqpoint{0.539299in}{0.078740in}}{\pgfqpoint{7.842520in}{7.842520in}}%
\pgfusepath{clip}%
\pgfsetbuttcap%
\pgfsetroundjoin%
\definecolor{currentfill}{rgb}{0.282327,0.094955,0.417331}%
\pgfsetfillcolor{currentfill}%
\pgfsetlinewidth{0.000000pt}%
\definecolor{currentstroke}{rgb}{0.395174,0.797475,0.367757}%
\pgfsetstrokecolor{currentstroke}%
\pgfsetdash{}{0pt}%
\pgfpathmoveto{\pgfqpoint{6.100728in}{3.537174in}}%
\pgfpathlineto{\pgfqpoint{6.236676in}{3.496772in}}%
\pgfpathlineto{\pgfqpoint{6.307514in}{3.501723in}}%
\pgfpathclose%
\pgfusepath{fill}%
\end{pgfscope}%
\begin{pgfscope}%
\pgfpathrectangle{\pgfqpoint{0.539299in}{0.078740in}}{\pgfqpoint{7.842520in}{7.842520in}}%
\pgfusepath{clip}%
\pgfsetbuttcap%
\pgfsetroundjoin%
\definecolor{currentfill}{rgb}{0.221989,0.339161,0.548752}%
\pgfsetfillcolor{currentfill}%
\pgfsetlinewidth{0.000000pt}%
\definecolor{currentstroke}{rgb}{0.404001,0.800275,0.362552}%
\pgfsetstrokecolor{currentstroke}%
\pgfsetdash{}{0pt}%
\pgfpathmoveto{\pgfqpoint{2.957481in}{4.048102in}}%
\pgfpathlineto{\pgfqpoint{3.079268in}{4.442627in}}%
\pgfpathlineto{\pgfqpoint{3.161266in}{4.461876in}}%
\pgfpathclose%
\pgfusepath{fill}%
\end{pgfscope}%
\begin{pgfscope}%
\pgfpathrectangle{\pgfqpoint{0.539299in}{0.078740in}}{\pgfqpoint{7.842520in}{7.842520in}}%
\pgfusepath{clip}%
\pgfsetbuttcap%
\pgfsetroundjoin%
\definecolor{currentfill}{rgb}{0.281446,0.084320,0.407414}%
\pgfsetfillcolor{currentfill}%
\pgfsetlinewidth{0.000000pt}%
\definecolor{currentstroke}{rgb}{0.412913,0.803041,0.357269}%
\pgfsetstrokecolor{currentstroke}%
\pgfsetdash{}{0pt}%
\pgfpathmoveto{\pgfqpoint{6.372915in}{3.450815in}}%
\pgfpathlineto{\pgfqpoint{6.443391in}{3.456772in}}%
\pgfpathlineto{\pgfqpoint{6.307514in}{3.501723in}}%
\pgfpathclose%
\pgfusepath{fill}%
\end{pgfscope}%
\begin{pgfscope}%
\pgfpathrectangle{\pgfqpoint{0.539299in}{0.078740in}}{\pgfqpoint{7.842520in}{7.842520in}}%
\pgfusepath{clip}%
\pgfsetbuttcap%
\pgfsetroundjoin%
\definecolor{currentfill}{rgb}{0.244972,0.287675,0.537260}%
\pgfsetfillcolor{currentfill}%
\pgfsetlinewidth{0.000000pt}%
\definecolor{currentstroke}{rgb}{0.421908,0.805774,0.351910}%
\pgfsetstrokecolor{currentstroke}%
\pgfsetdash{}{0pt}%
\pgfpathmoveto{\pgfqpoint{4.669832in}{4.230954in}}%
\pgfpathlineto{\pgfqpoint{4.803456in}{4.103048in}}%
\pgfpathlineto{\pgfqpoint{4.879148in}{4.033520in}}%
\pgfpathclose%
\pgfusepath{fill}%
\end{pgfscope}%
\begin{pgfscope}%
\pgfpathrectangle{\pgfqpoint{0.539299in}{0.078740in}}{\pgfqpoint{7.842520in}{7.842520in}}%
\pgfusepath{clip}%
\pgfsetbuttcap%
\pgfsetroundjoin%
\definecolor{currentfill}{rgb}{0.283229,0.120777,0.440584}%
\pgfsetfillcolor{currentfill}%
\pgfsetlinewidth{0.000000pt}%
\definecolor{currentstroke}{rgb}{0.430983,0.808473,0.346476}%
\pgfsetstrokecolor{currentstroke}%
\pgfsetdash{}{0pt}%
\pgfpathmoveto{\pgfqpoint{5.757427in}{3.601206in}}%
\pgfpathlineto{\pgfqpoint{5.892938in}{3.564025in}}%
\pgfpathlineto{\pgfqpoint{5.965125in}{3.573380in}}%
\pgfpathclose%
\pgfusepath{fill}%
\end{pgfscope}%
\begin{pgfscope}%
\pgfpathrectangle{\pgfqpoint{0.539299in}{0.078740in}}{\pgfqpoint{7.842520in}{7.842520in}}%
\pgfusepath{clip}%
\pgfsetbuttcap%
\pgfsetroundjoin%
\definecolor{currentfill}{rgb}{0.281887,0.150881,0.465405}%
\pgfsetfillcolor{currentfill}%
\pgfsetlinewidth{0.000000pt}%
\definecolor{currentstroke}{rgb}{0.440137,0.811138,0.340967}%
\pgfsetstrokecolor{currentstroke}%
\pgfsetdash{}{0pt}%
\pgfpathmoveto{\pgfqpoint{5.414261in}{3.695329in}}%
\pgfpathlineto{\pgfqpoint{5.549086in}{3.640953in}}%
\pgfpathlineto{\pgfqpoint{5.487835in}{3.685615in}}%
\pgfpathclose%
\pgfusepath{fill}%
\end{pgfscope}%
\begin{pgfscope}%
\pgfpathrectangle{\pgfqpoint{0.539299in}{0.078740in}}{\pgfqpoint{7.842520in}{7.842520in}}%
\pgfusepath{clip}%
\pgfsetbuttcap%
\pgfsetroundjoin%
\definecolor{currentfill}{rgb}{0.274952,0.037752,0.364543}%
\pgfsetfillcolor{currentfill}%
\pgfsetlinewidth{0.000000pt}%
\definecolor{currentstroke}{rgb}{0.449368,0.813768,0.335384}%
\pgfsetstrokecolor{currentstroke}%
\pgfsetdash{}{0pt}%
\pgfpathmoveto{\pgfqpoint{6.921738in}{3.310623in}}%
\pgfpathlineto{\pgfqpoint{6.785212in}{3.360948in}}%
\pgfpathlineto{\pgfqpoint{6.715949in}{3.352378in}}%
\pgfpathclose%
\pgfusepath{fill}%
\end{pgfscope}%
\begin{pgfscope}%
\pgfpathrectangle{\pgfqpoint{0.539299in}{0.078740in}}{\pgfqpoint{7.842520in}{7.842520in}}%
\pgfusepath{clip}%
\pgfsetbuttcap%
\pgfsetroundjoin%
\definecolor{currentfill}{rgb}{0.277941,0.056324,0.381191}%
\pgfsetfillcolor{currentfill}%
\pgfsetlinewidth{0.000000pt}%
\definecolor{currentstroke}{rgb}{0.458674,0.816363,0.329727}%
\pgfsetstrokecolor{currentstroke}%
\pgfsetdash{}{0pt}%
\pgfpathmoveto{\pgfqpoint{6.715949in}{3.352378in}}%
\pgfpathlineto{\pgfqpoint{6.785212in}{3.360948in}}%
\pgfpathlineto{\pgfqpoint{6.579539in}{3.406827in}}%
\pgfpathclose%
\pgfusepath{fill}%
\end{pgfscope}%
\begin{pgfscope}%
\pgfpathrectangle{\pgfqpoint{0.539299in}{0.078740in}}{\pgfqpoint{7.842520in}{7.842520in}}%
\pgfusepath{clip}%
\pgfsetbuttcap%
\pgfsetroundjoin%
\definecolor{currentfill}{rgb}{0.283197,0.115680,0.436115}%
\pgfsetfillcolor{currentfill}%
\pgfsetlinewidth{0.000000pt}%
\definecolor{currentstroke}{rgb}{0.468053,0.818921,0.323998}%
\pgfsetstrokecolor{currentstroke}%
\pgfsetdash{}{0pt}%
\pgfpathmoveto{\pgfqpoint{6.100728in}{3.537174in}}%
\pgfpathlineto{\pgfqpoint{5.965125in}{3.573380in}}%
\pgfpathlineto{\pgfqpoint{6.028872in}{3.526024in}}%
\pgfpathclose%
\pgfusepath{fill}%
\end{pgfscope}%
\begin{pgfscope}%
\pgfpathrectangle{\pgfqpoint{0.539299in}{0.078740in}}{\pgfqpoint{7.842520in}{7.842520in}}%
\pgfusepath{clip}%
\pgfsetbuttcap%
\pgfsetroundjoin%
\definecolor{currentfill}{rgb}{0.257322,0.256130,0.526563}%
\pgfsetfillcolor{currentfill}%
\pgfsetlinewidth{0.000000pt}%
\definecolor{currentstroke}{rgb}{0.477504,0.821444,0.318195}%
\pgfsetstrokecolor{currentstroke}%
\pgfsetdash{}{0pt}%
\pgfpathmoveto{\pgfqpoint{4.803456in}{4.103048in}}%
\pgfpathlineto{\pgfqpoint{4.937227in}{3.984970in}}%
\pgfpathlineto{\pgfqpoint{5.012423in}{3.928911in}}%
\pgfpathclose%
\pgfusepath{fill}%
\end{pgfscope}%
\begin{pgfscope}%
\pgfpathrectangle{\pgfqpoint{0.539299in}{0.078740in}}{\pgfqpoint{7.842520in}{7.842520in}}%
\pgfusepath{clip}%
\pgfsetbuttcap%
\pgfsetroundjoin%
\definecolor{currentfill}{rgb}{0.239346,0.300855,0.540844}%
\pgfsetfillcolor{currentfill}%
\pgfsetlinewidth{0.000000pt}%
\definecolor{currentstroke}{rgb}{0.487026,0.823929,0.312321}%
\pgfsetstrokecolor{currentstroke}%
\pgfsetdash{}{0pt}%
\pgfpathmoveto{\pgfqpoint{2.875698in}{4.003213in}}%
\pgfpathlineto{\pgfqpoint{2.996771in}{4.420700in}}%
\pgfpathlineto{\pgfqpoint{2.957481in}{4.048102in}}%
\pgfpathclose%
\pgfusepath{fill}%
\end{pgfscope}%
\begin{pgfscope}%
\pgfpathrectangle{\pgfqpoint{0.539299in}{0.078740in}}{\pgfqpoint{7.842520in}{7.842520in}}%
\pgfusepath{clip}%
\pgfsetbuttcap%
\pgfsetroundjoin%
\definecolor{currentfill}{rgb}{0.275191,0.194905,0.496005}%
\pgfsetfillcolor{currentfill}%
\pgfsetlinewidth{0.000000pt}%
\definecolor{currentstroke}{rgb}{0.496615,0.826376,0.306377}%
\pgfsetstrokecolor{currentstroke}%
\pgfsetdash{}{0pt}%
\pgfpathmoveto{\pgfqpoint{5.205512in}{3.788278in}}%
\pgfpathlineto{\pgfqpoint{5.279908in}{3.760172in}}%
\pgfpathlineto{\pgfqpoint{5.145984in}{3.837674in}}%
\pgfpathclose%
\pgfusepath{fill}%
\end{pgfscope}%
\begin{pgfscope}%
\pgfpathrectangle{\pgfqpoint{0.539299in}{0.078740in}}{\pgfqpoint{7.842520in}{7.842520in}}%
\pgfusepath{clip}%
\pgfsetbuttcap%
\pgfsetroundjoin%
\definecolor{currentfill}{rgb}{0.160665,0.478540,0.558115}%
\pgfsetfillcolor{currentfill}%
\pgfsetlinewidth{0.000000pt}%
\definecolor{currentstroke}{rgb}{0.506271,0.828786,0.300362}%
\pgfsetstrokecolor{currentstroke}%
\pgfsetdash{}{0pt}%
\pgfpathmoveto{\pgfqpoint{3.791614in}{4.877687in}}%
\pgfpathlineto{\pgfqpoint{3.660600in}{4.826856in}}%
\pgfpathlineto{\pgfqpoint{3.579639in}{4.867750in}}%
\pgfpathclose%
\pgfusepath{fill}%
\end{pgfscope}%
\begin{pgfscope}%
\pgfpathrectangle{\pgfqpoint{0.539299in}{0.078740in}}{\pgfqpoint{7.842520in}{7.842520in}}%
\pgfusepath{clip}%
\pgfsetbuttcap%
\pgfsetroundjoin%
\definecolor{currentfill}{rgb}{0.282327,0.094955,0.417331}%
\pgfsetfillcolor{currentfill}%
\pgfsetlinewidth{0.000000pt}%
\definecolor{currentstroke}{rgb}{0.515992,0.831158,0.294279}%
\pgfsetstrokecolor{currentstroke}%
\pgfsetdash{}{0pt}%
\pgfpathmoveto{\pgfqpoint{6.307514in}{3.501723in}}%
\pgfpathlineto{\pgfqpoint{6.236676in}{3.496772in}}%
\pgfpathlineto{\pgfqpoint{6.372915in}{3.450815in}}%
\pgfpathclose%
\pgfusepath{fill}%
\end{pgfscope}%
\begin{pgfscope}%
\pgfpathrectangle{\pgfqpoint{0.539299in}{0.078740in}}{\pgfqpoint{7.842520in}{7.842520in}}%
\pgfusepath{clip}%
\pgfsetbuttcap%
\pgfsetroundjoin%
\definecolor{currentfill}{rgb}{0.169646,0.456262,0.558030}%
\pgfsetfillcolor{currentfill}%
\pgfsetlinewidth{0.000000pt}%
\definecolor{currentstroke}{rgb}{0.525776,0.833491,0.288127}%
\pgfsetstrokecolor{currentstroke}%
\pgfsetdash{}{0pt}%
\pgfpathmoveto{\pgfqpoint{4.003782in}{4.784962in}}%
\pgfpathlineto{\pgfqpoint{4.057001in}{4.809605in}}%
\pgfpathlineto{\pgfqpoint{4.136438in}{4.716235in}}%
\pgfpathclose%
\pgfusepath{fill}%
\end{pgfscope}%
\begin{pgfscope}%
\pgfpathrectangle{\pgfqpoint{0.539299in}{0.078740in}}{\pgfqpoint{7.842520in}{7.842520in}}%
\pgfusepath{clip}%
\pgfsetbuttcap%
\pgfsetroundjoin%
\definecolor{currentfill}{rgb}{0.282884,0.135920,0.453427}%
\pgfsetfillcolor{currentfill}%
\pgfsetlinewidth{0.000000pt}%
\definecolor{currentstroke}{rgb}{0.535621,0.835785,0.281908}%
\pgfsetstrokecolor{currentstroke}%
\pgfsetdash{}{0pt}%
\pgfpathmoveto{\pgfqpoint{5.757427in}{3.601206in}}%
\pgfpathlineto{\pgfqpoint{5.622384in}{3.640649in}}%
\pgfpathlineto{\pgfqpoint{5.684398in}{3.594113in}}%
\pgfpathclose%
\pgfusepath{fill}%
\end{pgfscope}%
\begin{pgfscope}%
\pgfpathrectangle{\pgfqpoint{0.539299in}{0.078740in}}{\pgfqpoint{7.842520in}{7.842520in}}%
\pgfusepath{clip}%
\pgfsetbuttcap%
\pgfsetroundjoin%
\definecolor{currentfill}{rgb}{0.162142,0.474838,0.558140}%
\pgfsetfillcolor{currentfill}%
\pgfsetlinewidth{0.000000pt}%
\definecolor{currentstroke}{rgb}{0.545524,0.838039,0.275626}%
\pgfsetstrokecolor{currentstroke}%
\pgfsetdash{}{0pt}%
\pgfpathmoveto{\pgfqpoint{3.923882in}{4.868004in}}%
\pgfpathlineto{\pgfqpoint{4.003782in}{4.784962in}}%
\pgfpathlineto{\pgfqpoint{3.791614in}{4.877687in}}%
\pgfpathclose%
\pgfusepath{fill}%
\end{pgfscope}%
\begin{pgfscope}%
\pgfpathrectangle{\pgfqpoint{0.539299in}{0.078740in}}{\pgfqpoint{7.842520in}{7.842520in}}%
\pgfusepath{clip}%
\pgfsetbuttcap%
\pgfsetroundjoin%
\definecolor{currentfill}{rgb}{0.190631,0.407061,0.556089}%
\pgfsetfillcolor{currentfill}%
\pgfsetlinewidth{0.000000pt}%
\definecolor{currentstroke}{rgb}{0.555484,0.840254,0.269281}%
\pgfsetstrokecolor{currentstroke}%
\pgfsetdash{}{0pt}%
\pgfpathmoveto{\pgfqpoint{3.242761in}{4.478380in}}%
\pgfpathlineto{\pgfqpoint{3.161266in}{4.461876in}}%
\pgfpathlineto{\pgfqpoint{3.287152in}{4.741607in}}%
\pgfpathclose%
\pgfusepath{fill}%
\end{pgfscope}%
\begin{pgfscope}%
\pgfpathrectangle{\pgfqpoint{0.539299in}{0.078740in}}{\pgfqpoint{7.842520in}{7.842520in}}%
\pgfusepath{clip}%
\pgfsetbuttcap%
\pgfsetroundjoin%
\definecolor{currentfill}{rgb}{0.281446,0.084320,0.407414}%
\pgfsetfillcolor{currentfill}%
\pgfsetlinewidth{0.000000pt}%
\definecolor{currentstroke}{rgb}{0.565498,0.842430,0.262877}%
\pgfsetstrokecolor{currentstroke}%
\pgfsetdash{}{0pt}%
\pgfpathmoveto{\pgfqpoint{6.579539in}{3.406827in}}%
\pgfpathlineto{\pgfqpoint{6.443391in}{3.456772in}}%
\pgfpathlineto{\pgfqpoint{6.372915in}{3.450815in}}%
\pgfpathclose%
\pgfusepath{fill}%
\end{pgfscope}%
\begin{pgfscope}%
\pgfpathrectangle{\pgfqpoint{0.539299in}{0.078740in}}{\pgfqpoint{7.842520in}{7.842520in}}%
\pgfusepath{clip}%
\pgfsetbuttcap%
\pgfsetroundjoin%
\definecolor{currentfill}{rgb}{0.283229,0.120777,0.440584}%
\pgfsetfillcolor{currentfill}%
\pgfsetlinewidth{0.000000pt}%
\definecolor{currentstroke}{rgb}{0.575563,0.844566,0.256415}%
\pgfsetstrokecolor{currentstroke}%
\pgfsetdash{}{0pt}%
\pgfpathmoveto{\pgfqpoint{6.028872in}{3.526024in}}%
\pgfpathlineto{\pgfqpoint{5.965125in}{3.573380in}}%
\pgfpathlineto{\pgfqpoint{5.892938in}{3.564025in}}%
\pgfpathclose%
\pgfusepath{fill}%
\end{pgfscope}%
\begin{pgfscope}%
\pgfpathrectangle{\pgfqpoint{0.539299in}{0.078740in}}{\pgfqpoint{7.842520in}{7.842520in}}%
\pgfusepath{clip}%
\pgfsetbuttcap%
\pgfsetroundjoin%
\definecolor{currentfill}{rgb}{0.265145,0.232956,0.516599}%
\pgfsetfillcolor{currentfill}%
\pgfsetlinewidth{0.000000pt}%
\definecolor{currentstroke}{rgb}{0.585678,0.846661,0.249897}%
\pgfsetstrokecolor{currentstroke}%
\pgfsetdash{}{0pt}%
\pgfpathmoveto{\pgfqpoint{5.145984in}{3.837674in}}%
\pgfpathlineto{\pgfqpoint{4.937227in}{3.984970in}}%
\pgfpathlineto{\pgfqpoint{5.071218in}{3.879659in}}%
\pgfpathclose%
\pgfusepath{fill}%
\end{pgfscope}%
\begin{pgfscope}%
\pgfpathrectangle{\pgfqpoint{0.539299in}{0.078740in}}{\pgfqpoint{7.842520in}{7.842520in}}%
\pgfusepath{clip}%
\pgfsetbuttcap%
\pgfsetroundjoin%
\definecolor{currentfill}{rgb}{0.172719,0.448791,0.557885}%
\pgfsetfillcolor{currentfill}%
\pgfsetlinewidth{0.000000pt}%
\definecolor{currentstroke}{rgb}{0.595839,0.848717,0.243329}%
\pgfsetstrokecolor{currentstroke}%
\pgfsetdash{}{0pt}%
\pgfpathmoveto{\pgfqpoint{4.136438in}{4.716235in}}%
\pgfpathlineto{\pgfqpoint{4.057001in}{4.809605in}}%
\pgfpathlineto{\pgfqpoint{4.269526in}{4.615818in}}%
\pgfpathclose%
\pgfusepath{fill}%
\end{pgfscope}%
\begin{pgfscope}%
\pgfpathrectangle{\pgfqpoint{0.539299in}{0.078740in}}{\pgfqpoint{7.842520in}{7.842520in}}%
\pgfusepath{clip}%
\pgfsetbuttcap%
\pgfsetroundjoin%
\definecolor{currentfill}{rgb}{0.282290,0.145912,0.461510}%
\pgfsetfillcolor{currentfill}%
\pgfsetlinewidth{0.000000pt}%
\definecolor{currentstroke}{rgb}{0.606045,0.850733,0.236712}%
\pgfsetstrokecolor{currentstroke}%
\pgfsetdash{}{0pt}%
\pgfpathmoveto{\pgfqpoint{5.622384in}{3.640649in}}%
\pgfpathlineto{\pgfqpoint{5.549086in}{3.640953in}}%
\pgfpathlineto{\pgfqpoint{5.684398in}{3.594113in}}%
\pgfpathclose%
\pgfusepath{fill}%
\end{pgfscope}%
\begin{pgfscope}%
\pgfpathrectangle{\pgfqpoint{0.539299in}{0.078740in}}{\pgfqpoint{7.842520in}{7.842520in}}%
\pgfusepath{clip}%
\pgfsetbuttcap%
\pgfsetroundjoin%
\definecolor{currentfill}{rgb}{0.179019,0.433756,0.557430}%
\pgfsetfillcolor{currentfill}%
\pgfsetlinewidth{0.000000pt}%
\definecolor{currentstroke}{rgb}{0.616293,0.852709,0.230052}%
\pgfsetstrokecolor{currentstroke}%
\pgfsetdash{}{0pt}%
\pgfpathmoveto{\pgfqpoint{3.287152in}{4.741607in}}%
\pgfpathlineto{\pgfqpoint{3.369083in}{4.733483in}}%
\pgfpathlineto{\pgfqpoint{3.242761in}{4.478380in}}%
\pgfpathclose%
\pgfusepath{fill}%
\end{pgfscope}%
\begin{pgfscope}%
\pgfpathrectangle{\pgfqpoint{0.539299in}{0.078740in}}{\pgfqpoint{7.842520in}{7.842520in}}%
\pgfusepath{clip}%
\pgfsetbuttcap%
\pgfsetroundjoin%
\definecolor{currentfill}{rgb}{0.270595,0.214069,0.507052}%
\pgfsetfillcolor{currentfill}%
\pgfsetlinewidth{0.000000pt}%
\definecolor{currentstroke}{rgb}{0.626579,0.854645,0.223353}%
\pgfsetstrokecolor{currentstroke}%
\pgfsetdash{}{0pt}%
\pgfpathmoveto{\pgfqpoint{5.071218in}{3.879659in}}%
\pgfpathlineto{\pgfqpoint{5.205512in}{3.788278in}}%
\pgfpathlineto{\pgfqpoint{5.145984in}{3.837674in}}%
\pgfpathclose%
\pgfusepath{fill}%
\end{pgfscope}%
\begin{pgfscope}%
\pgfpathrectangle{\pgfqpoint{0.539299in}{0.078740in}}{\pgfqpoint{7.842520in}{7.842520in}}%
\pgfusepath{clip}%
\pgfsetbuttcap%
\pgfsetroundjoin%
\definecolor{currentfill}{rgb}{0.278012,0.180367,0.486697}%
\pgfsetfillcolor{currentfill}%
\pgfsetlinewidth{0.000000pt}%
\definecolor{currentstroke}{rgb}{0.636902,0.856542,0.216620}%
\pgfsetstrokecolor{currentstroke}%
\pgfsetdash{}{0pt}%
\pgfpathmoveto{\pgfqpoint{5.414261in}{3.695329in}}%
\pgfpathlineto{\pgfqpoint{5.279908in}{3.760172in}}%
\pgfpathlineto{\pgfqpoint{5.340183in}{3.710444in}}%
\pgfpathclose%
\pgfusepath{fill}%
\end{pgfscope}%
\begin{pgfscope}%
\pgfpathrectangle{\pgfqpoint{0.539299in}{0.078740in}}{\pgfqpoint{7.842520in}{7.842520in}}%
\pgfusepath{clip}%
\pgfsetbuttcap%
\pgfsetroundjoin%
\definecolor{currentfill}{rgb}{0.276022,0.044167,0.370164}%
\pgfsetfillcolor{currentfill}%
\pgfsetlinewidth{0.000000pt}%
\definecolor{currentstroke}{rgb}{0.647257,0.858400,0.209861}%
\pgfsetstrokecolor{currentstroke}%
\pgfsetdash{}{0pt}%
\pgfpathmoveto{\pgfqpoint{6.715949in}{3.352378in}}%
\pgfpathlineto{\pgfqpoint{6.852629in}{3.294416in}}%
\pgfpathlineto{\pgfqpoint{6.921738in}{3.310623in}}%
\pgfpathclose%
\pgfusepath{fill}%
\end{pgfscope}%
\begin{pgfscope}%
\pgfpathrectangle{\pgfqpoint{0.539299in}{0.078740in}}{\pgfqpoint{7.842520in}{7.842520in}}%
\pgfusepath{clip}%
\pgfsetbuttcap%
\pgfsetroundjoin%
\definecolor{currentfill}{rgb}{0.160665,0.478540,0.558115}%
\pgfsetfillcolor{currentfill}%
\pgfsetlinewidth{0.000000pt}%
\definecolor{currentstroke}{rgb}{0.657642,0.860219,0.203082}%
\pgfsetstrokecolor{currentstroke}%
\pgfsetdash{}{0pt}%
\pgfpathmoveto{\pgfqpoint{3.923882in}{4.868004in}}%
\pgfpathlineto{\pgfqpoint{4.057001in}{4.809605in}}%
\pgfpathlineto{\pgfqpoint{4.003782in}{4.784962in}}%
\pgfpathclose%
\pgfusepath{fill}%
\end{pgfscope}%
\begin{pgfscope}%
\pgfpathrectangle{\pgfqpoint{0.539299in}{0.078740in}}{\pgfqpoint{7.842520in}{7.842520in}}%
\pgfusepath{clip}%
\pgfsetbuttcap%
\pgfsetroundjoin%
\definecolor{currentfill}{rgb}{0.272594,0.025563,0.353093}%
\pgfsetfillcolor{currentfill}%
\pgfsetlinewidth{0.000000pt}%
\definecolor{currentstroke}{rgb}{0.668054,0.861999,0.196293}%
\pgfsetstrokecolor{currentstroke}%
\pgfsetdash{}{0pt}%
\pgfpathmoveto{\pgfqpoint{6.989604in}{3.234274in}}%
\pgfpathlineto{\pgfqpoint{7.058628in}{3.259387in}}%
\pgfpathlineto{\pgfqpoint{6.921738in}{3.310623in}}%
\pgfpathclose%
\pgfusepath{fill}%
\end{pgfscope}%
\begin{pgfscope}%
\pgfpathrectangle{\pgfqpoint{0.539299in}{0.078740in}}{\pgfqpoint{7.842520in}{7.842520in}}%
\pgfusepath{clip}%
\pgfsetbuttcap%
\pgfsetroundjoin%
\definecolor{currentfill}{rgb}{0.283197,0.115680,0.436115}%
\pgfsetfillcolor{currentfill}%
\pgfsetlinewidth{0.000000pt}%
\definecolor{currentstroke}{rgb}{0.678489,0.863742,0.189503}%
\pgfsetstrokecolor{currentstroke}%
\pgfsetdash{}{0pt}%
\pgfpathmoveto{\pgfqpoint{6.165167in}{3.484596in}}%
\pgfpathlineto{\pgfqpoint{6.236676in}{3.496772in}}%
\pgfpathlineto{\pgfqpoint{6.100728in}{3.537174in}}%
\pgfpathclose%
\pgfusepath{fill}%
\end{pgfscope}%
\begin{pgfscope}%
\pgfpathrectangle{\pgfqpoint{0.539299in}{0.078740in}}{\pgfqpoint{7.842520in}{7.842520in}}%
\pgfusepath{clip}%
\pgfsetbuttcap%
\pgfsetroundjoin%
\definecolor{currentfill}{rgb}{0.216210,0.351535,0.550627}%
\pgfsetfillcolor{currentfill}%
\pgfsetlinewidth{0.000000pt}%
\definecolor{currentstroke}{rgb}{0.688944,0.865448,0.182725}%
\pgfsetstrokecolor{currentstroke}%
\pgfsetdash{}{0pt}%
\pgfpathmoveto{\pgfqpoint{2.996771in}{4.420700in}}%
\pgfpathlineto{\pgfqpoint{3.079268in}{4.442627in}}%
\pgfpathlineto{\pgfqpoint{2.957481in}{4.048102in}}%
\pgfpathclose%
\pgfusepath{fill}%
\end{pgfscope}%
\begin{pgfscope}%
\pgfpathrectangle{\pgfqpoint{0.539299in}{0.078740in}}{\pgfqpoint{7.842520in}{7.842520in}}%
\pgfusepath{clip}%
\pgfsetbuttcap%
\pgfsetroundjoin%
\definecolor{currentfill}{rgb}{0.185556,0.418570,0.556753}%
\pgfsetfillcolor{currentfill}%
\pgfsetlinewidth{0.000000pt}%
\definecolor{currentstroke}{rgb}{0.699415,0.867117,0.175971}%
\pgfsetstrokecolor{currentstroke}%
\pgfsetdash{}{0pt}%
\pgfpathmoveto{\pgfqpoint{4.402851in}{4.494989in}}%
\pgfpathlineto{\pgfqpoint{4.269526in}{4.615818in}}%
\pgfpathlineto{\pgfqpoint{4.324600in}{4.594132in}}%
\pgfpathclose%
\pgfusepath{fill}%
\end{pgfscope}%
\begin{pgfscope}%
\pgfpathrectangle{\pgfqpoint{0.539299in}{0.078740in}}{\pgfqpoint{7.842520in}{7.842520in}}%
\pgfusepath{clip}%
\pgfsetbuttcap%
\pgfsetroundjoin%
\definecolor{currentfill}{rgb}{0.282884,0.135920,0.453427}%
\pgfsetfillcolor{currentfill}%
\pgfsetlinewidth{0.000000pt}%
\definecolor{currentstroke}{rgb}{0.709898,0.868751,0.169257}%
\pgfsetstrokecolor{currentstroke}%
\pgfsetdash{}{0pt}%
\pgfpathmoveto{\pgfqpoint{5.820187in}{3.551515in}}%
\pgfpathlineto{\pgfqpoint{5.892938in}{3.564025in}}%
\pgfpathlineto{\pgfqpoint{5.757427in}{3.601206in}}%
\pgfpathclose%
\pgfusepath{fill}%
\end{pgfscope}%
\begin{pgfscope}%
\pgfpathrectangle{\pgfqpoint{0.539299in}{0.078740in}}{\pgfqpoint{7.842520in}{7.842520in}}%
\pgfusepath{clip}%
\pgfsetbuttcap%
\pgfsetroundjoin%
\definecolor{currentfill}{rgb}{0.163625,0.471133,0.558148}%
\pgfsetfillcolor{currentfill}%
\pgfsetlinewidth{0.000000pt}%
\definecolor{currentstroke}{rgb}{0.720391,0.870350,0.162603}%
\pgfsetstrokecolor{currentstroke}%
\pgfsetdash{}{0pt}%
\pgfpathmoveto{\pgfqpoint{3.498080in}{4.903513in}}%
\pgfpathlineto{\pgfqpoint{3.450467in}{4.721409in}}%
\pgfpathlineto{\pgfqpoint{3.369083in}{4.733483in}}%
\pgfpathclose%
\pgfusepath{fill}%
\end{pgfscope}%
\begin{pgfscope}%
\pgfpathrectangle{\pgfqpoint{0.539299in}{0.078740in}}{\pgfqpoint{7.842520in}{7.842520in}}%
\pgfusepath{clip}%
\pgfsetbuttcap%
\pgfsetroundjoin%
\definecolor{currentfill}{rgb}{0.194100,0.399323,0.555565}%
\pgfsetfillcolor{currentfill}%
\pgfsetlinewidth{0.000000pt}%
\definecolor{currentstroke}{rgb}{0.730889,0.871916,0.156029}%
\pgfsetstrokecolor{currentstroke}%
\pgfsetdash{}{0pt}%
\pgfpathmoveto{\pgfqpoint{4.536301in}{4.363851in}}%
\pgfpathlineto{\pgfqpoint{4.402851in}{4.494989in}}%
\pgfpathlineto{\pgfqpoint{4.324600in}{4.594132in}}%
\pgfpathclose%
\pgfusepath{fill}%
\end{pgfscope}%
\begin{pgfscope}%
\pgfpathrectangle{\pgfqpoint{0.539299in}{0.078740in}}{\pgfqpoint{7.842520in}{7.842520in}}%
\pgfusepath{clip}%
\pgfsetbuttcap%
\pgfsetroundjoin%
\definecolor{currentfill}{rgb}{0.159194,0.482237,0.558073}%
\pgfsetfillcolor{currentfill}%
\pgfsetlinewidth{0.000000pt}%
\definecolor{currentstroke}{rgb}{0.741388,0.873449,0.149561}%
\pgfsetstrokecolor{currentstroke}%
\pgfsetdash{}{0pt}%
\pgfpathmoveto{\pgfqpoint{3.579639in}{4.867750in}}%
\pgfpathlineto{\pgfqpoint{3.450467in}{4.721409in}}%
\pgfpathlineto{\pgfqpoint{3.498080in}{4.903513in}}%
\pgfpathclose%
\pgfusepath{fill}%
\end{pgfscope}%
\begin{pgfscope}%
\pgfpathrectangle{\pgfqpoint{0.539299in}{0.078740in}}{\pgfqpoint{7.842520in}{7.842520in}}%
\pgfusepath{clip}%
\pgfsetbuttcap%
\pgfsetroundjoin%
\definecolor{currentfill}{rgb}{0.278826,0.175490,0.483397}%
\pgfsetfillcolor{currentfill}%
\pgfsetlinewidth{0.000000pt}%
\definecolor{currentstroke}{rgb}{0.751884,0.874951,0.143228}%
\pgfsetstrokecolor{currentstroke}%
\pgfsetdash{}{0pt}%
\pgfpathmoveto{\pgfqpoint{5.340183in}{3.710444in}}%
\pgfpathlineto{\pgfqpoint{5.549086in}{3.640953in}}%
\pgfpathlineto{\pgfqpoint{5.414261in}{3.695329in}}%
\pgfpathclose%
\pgfusepath{fill}%
\end{pgfscope}%
\begin{pgfscope}%
\pgfpathrectangle{\pgfqpoint{0.539299in}{0.078740in}}{\pgfqpoint{7.842520in}{7.842520in}}%
\pgfusepath{clip}%
\pgfsetbuttcap%
\pgfsetroundjoin%
\definecolor{currentfill}{rgb}{0.237441,0.305202,0.541921}%
\pgfsetfillcolor{currentfill}%
\pgfsetlinewidth{0.000000pt}%
\definecolor{currentstroke}{rgb}{0.762373,0.876424,0.137064}%
\pgfsetstrokecolor{currentstroke}%
\pgfsetdash{}{0pt}%
\pgfpathmoveto{\pgfqpoint{2.793452in}{3.956696in}}%
\pgfpathlineto{\pgfqpoint{2.996771in}{4.420700in}}%
\pgfpathlineto{\pgfqpoint{2.875698in}{4.003213in}}%
\pgfpathclose%
\pgfusepath{fill}%
\end{pgfscope}%
\begin{pgfscope}%
\pgfpathrectangle{\pgfqpoint{0.539299in}{0.078740in}}{\pgfqpoint{7.842520in}{7.842520in}}%
\pgfusepath{clip}%
\pgfsetbuttcap%
\pgfsetroundjoin%
\definecolor{currentfill}{rgb}{0.281924,0.089666,0.412415}%
\pgfsetfillcolor{currentfill}%
\pgfsetlinewidth{0.000000pt}%
\definecolor{currentstroke}{rgb}{0.772852,0.877868,0.131109}%
\pgfsetstrokecolor{currentstroke}%
\pgfsetdash{}{0pt}%
\pgfpathmoveto{\pgfqpoint{6.372915in}{3.450815in}}%
\pgfpathlineto{\pgfqpoint{6.509400in}{3.398734in}}%
\pgfpathlineto{\pgfqpoint{6.579539in}{3.406827in}}%
\pgfpathclose%
\pgfusepath{fill}%
\end{pgfscope}%
\begin{pgfscope}%
\pgfpathrectangle{\pgfqpoint{0.539299in}{0.078740in}}{\pgfqpoint{7.842520in}{7.842520in}}%
\pgfusepath{clip}%
\pgfsetbuttcap%
\pgfsetroundjoin%
\definecolor{currentfill}{rgb}{0.275191,0.194905,0.496005}%
\pgfsetfillcolor{currentfill}%
\pgfsetlinewidth{0.000000pt}%
\definecolor{currentstroke}{rgb}{0.783315,0.879285,0.125405}%
\pgfsetstrokecolor{currentstroke}%
\pgfsetdash{}{0pt}%
\pgfpathmoveto{\pgfqpoint{5.340183in}{3.710444in}}%
\pgfpathlineto{\pgfqpoint{5.279908in}{3.760172in}}%
\pgfpathlineto{\pgfqpoint{5.205512in}{3.788278in}}%
\pgfpathclose%
\pgfusepath{fill}%
\end{pgfscope}%
\begin{pgfscope}%
\pgfpathrectangle{\pgfqpoint{0.539299in}{0.078740in}}{\pgfqpoint{7.842520in}{7.842520in}}%
\pgfusepath{clip}%
\pgfsetbuttcap%
\pgfsetroundjoin%
\definecolor{currentfill}{rgb}{0.283229,0.120777,0.440584}%
\pgfsetfillcolor{currentfill}%
\pgfsetlinewidth{0.000000pt}%
\definecolor{currentstroke}{rgb}{0.793760,0.880678,0.120005}%
\pgfsetstrokecolor{currentstroke}%
\pgfsetdash{}{0pt}%
\pgfpathmoveto{\pgfqpoint{6.100728in}{3.537174in}}%
\pgfpathlineto{\pgfqpoint{6.028872in}{3.526024in}}%
\pgfpathlineto{\pgfqpoint{6.165167in}{3.484596in}}%
\pgfpathclose%
\pgfusepath{fill}%
\end{pgfscope}%
\begin{pgfscope}%
\pgfpathrectangle{\pgfqpoint{0.539299in}{0.078740in}}{\pgfqpoint{7.842520in}{7.842520in}}%
\pgfusepath{clip}%
\pgfsetbuttcap%
\pgfsetroundjoin%
\definecolor{currentfill}{rgb}{0.274952,0.037752,0.364543}%
\pgfsetfillcolor{currentfill}%
\pgfsetlinewidth{0.000000pt}%
\definecolor{currentstroke}{rgb}{0.804182,0.882046,0.114965}%
\pgfsetstrokecolor{currentstroke}%
\pgfsetdash{}{0pt}%
\pgfpathmoveto{\pgfqpoint{6.921738in}{3.310623in}}%
\pgfpathlineto{\pgfqpoint{6.852629in}{3.294416in}}%
\pgfpathlineto{\pgfqpoint{6.989604in}{3.234274in}}%
\pgfpathclose%
\pgfusepath{fill}%
\end{pgfscope}%
\begin{pgfscope}%
\pgfpathrectangle{\pgfqpoint{0.539299in}{0.078740in}}{\pgfqpoint{7.842520in}{7.842520in}}%
\pgfusepath{clip}%
\pgfsetbuttcap%
\pgfsetroundjoin%
\definecolor{currentfill}{rgb}{0.208623,0.367752,0.552675}%
\pgfsetfillcolor{currentfill}%
\pgfsetlinewidth{0.000000pt}%
\definecolor{currentstroke}{rgb}{0.814576,0.883393,0.110347}%
\pgfsetstrokecolor{currentstroke}%
\pgfsetdash{}{0pt}%
\pgfpathmoveto{\pgfqpoint{4.536301in}{4.363851in}}%
\pgfpathlineto{\pgfqpoint{4.458695in}{4.459156in}}%
\pgfpathlineto{\pgfqpoint{4.669832in}{4.230954in}}%
\pgfpathclose%
\pgfusepath{fill}%
\end{pgfscope}%
\begin{pgfscope}%
\pgfpathrectangle{\pgfqpoint{0.539299in}{0.078740in}}{\pgfqpoint{7.842520in}{7.842520in}}%
\pgfusepath{clip}%
\pgfsetbuttcap%
\pgfsetroundjoin%
\definecolor{currentfill}{rgb}{0.280267,0.073417,0.397163}%
\pgfsetfillcolor{currentfill}%
\pgfsetlinewidth{0.000000pt}%
\definecolor{currentstroke}{rgb}{0.824940,0.884720,0.106217}%
\pgfsetstrokecolor{currentstroke}%
\pgfsetdash{}{0pt}%
\pgfpathmoveto{\pgfqpoint{6.579539in}{3.406827in}}%
\pgfpathlineto{\pgfqpoint{6.646106in}{3.340736in}}%
\pgfpathlineto{\pgfqpoint{6.715949in}{3.352378in}}%
\pgfpathclose%
\pgfusepath{fill}%
\end{pgfscope}%
\begin{pgfscope}%
\pgfpathrectangle{\pgfqpoint{0.539299in}{0.078740in}}{\pgfqpoint{7.842520in}{7.842520in}}%
\pgfusepath{clip}%
\pgfsetbuttcap%
\pgfsetroundjoin%
\definecolor{currentfill}{rgb}{0.282290,0.145912,0.461510}%
\pgfsetfillcolor{currentfill}%
\pgfsetlinewidth{0.000000pt}%
\definecolor{currentstroke}{rgb}{0.835270,0.886029,0.102646}%
\pgfsetstrokecolor{currentstroke}%
\pgfsetdash{}{0pt}%
\pgfpathmoveto{\pgfqpoint{5.757427in}{3.601206in}}%
\pgfpathlineto{\pgfqpoint{5.684398in}{3.594113in}}%
\pgfpathlineto{\pgfqpoint{5.820187in}{3.551515in}}%
\pgfpathclose%
\pgfusepath{fill}%
\end{pgfscope}%
\begin{pgfscope}%
\pgfpathrectangle{\pgfqpoint{0.539299in}{0.078740in}}{\pgfqpoint{7.842520in}{7.842520in}}%
\pgfusepath{clip}%
\pgfsetbuttcap%
\pgfsetroundjoin%
\definecolor{currentfill}{rgb}{0.283091,0.110553,0.431554}%
\pgfsetfillcolor{currentfill}%
\pgfsetlinewidth{0.000000pt}%
\definecolor{currentstroke}{rgb}{0.845561,0.887322,0.099702}%
\pgfsetstrokecolor{currentstroke}%
\pgfsetdash{}{0pt}%
\pgfpathmoveto{\pgfqpoint{6.372915in}{3.450815in}}%
\pgfpathlineto{\pgfqpoint{6.236676in}{3.496772in}}%
\pgfpathlineto{\pgfqpoint{6.165167in}{3.484596in}}%
\pgfpathclose%
\pgfusepath{fill}%
\end{pgfscope}%
\begin{pgfscope}%
\pgfpathrectangle{\pgfqpoint{0.539299in}{0.078740in}}{\pgfqpoint{7.842520in}{7.842520in}}%
\pgfusepath{clip}%
\pgfsetbuttcap%
\pgfsetroundjoin%
\definecolor{currentfill}{rgb}{0.151918,0.500685,0.557587}%
\pgfsetfillcolor{currentfill}%
\pgfsetlinewidth{0.000000pt}%
\definecolor{currentstroke}{rgb}{0.855810,0.888601,0.097452}%
\pgfsetstrokecolor{currentstroke}%
\pgfsetdash{}{0pt}%
\pgfpathmoveto{\pgfqpoint{3.579639in}{4.867750in}}%
\pgfpathlineto{\pgfqpoint{3.710739in}{4.939290in}}%
\pgfpathlineto{\pgfqpoint{3.791614in}{4.877687in}}%
\pgfpathclose%
\pgfusepath{fill}%
\end{pgfscope}%
\begin{pgfscope}%
\pgfpathrectangle{\pgfqpoint{0.539299in}{0.078740in}}{\pgfqpoint{7.842520in}{7.842520in}}%
\pgfusepath{clip}%
\pgfsetbuttcap%
\pgfsetroundjoin%
\definecolor{currentfill}{rgb}{0.229739,0.322361,0.545706}%
\pgfsetfillcolor{currentfill}%
\pgfsetlinewidth{0.000000pt}%
\definecolor{currentstroke}{rgb}{0.866013,0.889868,0.095953}%
\pgfsetstrokecolor{currentstroke}%
\pgfsetdash{}{0pt}%
\pgfpathmoveto{\pgfqpoint{4.727082in}{4.180259in}}%
\pgfpathlineto{\pgfqpoint{4.803456in}{4.103048in}}%
\pgfpathlineto{\pgfqpoint{4.669832in}{4.230954in}}%
\pgfpathclose%
\pgfusepath{fill}%
\end{pgfscope}%
\begin{pgfscope}%
\pgfpathrectangle{\pgfqpoint{0.539299in}{0.078740in}}{\pgfqpoint{7.842520in}{7.842520in}}%
\pgfusepath{clip}%
\pgfsetbuttcap%
\pgfsetroundjoin%
\definecolor{currentfill}{rgb}{0.185556,0.418570,0.556753}%
\pgfsetfillcolor{currentfill}%
\pgfsetlinewidth{0.000000pt}%
\definecolor{currentstroke}{rgb}{0.876168,0.891125,0.095250}%
\pgfsetstrokecolor{currentstroke}%
\pgfsetdash{}{0pt}%
\pgfpathmoveto{\pgfqpoint{3.287152in}{4.741607in}}%
\pgfpathlineto{\pgfqpoint{3.161266in}{4.461876in}}%
\pgfpathlineto{\pgfqpoint{3.079268in}{4.442627in}}%
\pgfpathclose%
\pgfusepath{fill}%
\end{pgfscope}%
\begin{pgfscope}%
\pgfpathrectangle{\pgfqpoint{0.539299in}{0.078740in}}{\pgfqpoint{7.842520in}{7.842520in}}%
\pgfusepath{clip}%
\pgfsetbuttcap%
\pgfsetroundjoin%
\definecolor{currentfill}{rgb}{0.168126,0.459988,0.558082}%
\pgfsetfillcolor{currentfill}%
\pgfsetlinewidth{0.000000pt}%
\definecolor{currentstroke}{rgb}{0.886271,0.892374,0.095374}%
\pgfsetstrokecolor{currentstroke}%
\pgfsetdash{}{0pt}%
\pgfpathmoveto{\pgfqpoint{4.269526in}{4.615818in}}%
\pgfpathlineto{\pgfqpoint{4.057001in}{4.809605in}}%
\pgfpathlineto{\pgfqpoint{4.190652in}{4.714468in}}%
\pgfpathclose%
\pgfusepath{fill}%
\end{pgfscope}%
\begin{pgfscope}%
\pgfpathrectangle{\pgfqpoint{0.539299in}{0.078740in}}{\pgfqpoint{7.842520in}{7.842520in}}%
\pgfusepath{clip}%
\pgfsetbuttcap%
\pgfsetroundjoin%
\definecolor{currentfill}{rgb}{0.271305,0.019942,0.347269}%
\pgfsetfillcolor{currentfill}%
\pgfsetlinewidth{0.000000pt}%
\definecolor{currentstroke}{rgb}{0.896320,0.893616,0.096335}%
\pgfsetstrokecolor{currentstroke}%
\pgfsetdash{}{0pt}%
\pgfpathmoveto{\pgfqpoint{7.126919in}{3.173464in}}%
\pgfpathlineto{\pgfqpoint{7.195913in}{3.208174in}}%
\pgfpathlineto{\pgfqpoint{7.058628in}{3.259387in}}%
\pgfpathclose%
\pgfusepath{fill}%
\end{pgfscope}%
\begin{pgfscope}%
\pgfpathrectangle{\pgfqpoint{0.539299in}{0.078740in}}{\pgfqpoint{7.842520in}{7.842520in}}%
\pgfusepath{clip}%
\pgfsetbuttcap%
\pgfsetroundjoin%
\definecolor{currentfill}{rgb}{0.282623,0.140926,0.457517}%
\pgfsetfillcolor{currentfill}%
\pgfsetlinewidth{0.000000pt}%
\definecolor{currentstroke}{rgb}{0.906311,0.894855,0.098125}%
\pgfsetstrokecolor{currentstroke}%
\pgfsetdash{}{0pt}%
\pgfpathmoveto{\pgfqpoint{5.820187in}{3.551515in}}%
\pgfpathlineto{\pgfqpoint{6.028872in}{3.526024in}}%
\pgfpathlineto{\pgfqpoint{5.892938in}{3.564025in}}%
\pgfpathclose%
\pgfusepath{fill}%
\end{pgfscope}%
\begin{pgfscope}%
\pgfpathrectangle{\pgfqpoint{0.539299in}{0.078740in}}{\pgfqpoint{7.842520in}{7.842520in}}%
\pgfusepath{clip}%
\pgfsetbuttcap%
\pgfsetroundjoin%
\definecolor{currentfill}{rgb}{0.244972,0.287675,0.537260}%
\pgfsetfillcolor{currentfill}%
\pgfsetlinewidth{0.000000pt}%
\definecolor{currentstroke}{rgb}{0.916242,0.896091,0.100717}%
\pgfsetstrokecolor{currentstroke}%
\pgfsetdash{}{0pt}%
\pgfpathmoveto{\pgfqpoint{4.803456in}{4.103048in}}%
\pgfpathlineto{\pgfqpoint{4.861395in}{4.049469in}}%
\pgfpathlineto{\pgfqpoint{4.937227in}{3.984970in}}%
\pgfpathclose%
\pgfusepath{fill}%
\end{pgfscope}%
\begin{pgfscope}%
\pgfpathrectangle{\pgfqpoint{0.539299in}{0.078740in}}{\pgfqpoint{7.842520in}{7.842520in}}%
\pgfusepath{clip}%
\pgfsetbuttcap%
\pgfsetroundjoin%
\definecolor{currentfill}{rgb}{0.281446,0.084320,0.407414}%
\pgfsetfillcolor{currentfill}%
\pgfsetlinewidth{0.000000pt}%
\definecolor{currentstroke}{rgb}{0.926106,0.897330,0.104071}%
\pgfsetstrokecolor{currentstroke}%
\pgfsetdash{}{0pt}%
\pgfpathmoveto{\pgfqpoint{6.509400in}{3.398734in}}%
\pgfpathlineto{\pgfqpoint{6.646106in}{3.340736in}}%
\pgfpathlineto{\pgfqpoint{6.579539in}{3.406827in}}%
\pgfpathclose%
\pgfusepath{fill}%
\end{pgfscope}%
\begin{pgfscope}%
\pgfpathrectangle{\pgfqpoint{0.539299in}{0.078740in}}{\pgfqpoint{7.842520in}{7.842520in}}%
\pgfusepath{clip}%
\pgfsetbuttcap%
\pgfsetroundjoin%
\definecolor{currentfill}{rgb}{0.174274,0.445044,0.557792}%
\pgfsetfillcolor{currentfill}%
\pgfsetlinewidth{0.000000pt}%
\definecolor{currentstroke}{rgb}{0.935904,0.898570,0.108131}%
\pgfsetstrokecolor{currentstroke}%
\pgfsetdash{}{0pt}%
\pgfpathmoveto{\pgfqpoint{4.324600in}{4.594132in}}%
\pgfpathlineto{\pgfqpoint{4.269526in}{4.615818in}}%
\pgfpathlineto{\pgfqpoint{4.190652in}{4.714468in}}%
\pgfpathclose%
\pgfusepath{fill}%
\end{pgfscope}%
\begin{pgfscope}%
\pgfpathrectangle{\pgfqpoint{0.539299in}{0.078740in}}{\pgfqpoint{7.842520in}{7.842520in}}%
\pgfusepath{clip}%
\pgfsetbuttcap%
\pgfsetroundjoin%
\definecolor{currentfill}{rgb}{0.252194,0.269783,0.531579}%
\pgfsetfillcolor{currentfill}%
\pgfsetlinewidth{0.000000pt}%
\definecolor{currentstroke}{rgb}{0.945636,0.899815,0.112838}%
\pgfsetstrokecolor{currentstroke}%
\pgfsetdash{}{0pt}%
\pgfpathmoveto{\pgfqpoint{5.071218in}{3.879659in}}%
\pgfpathlineto{\pgfqpoint{4.937227in}{3.984970in}}%
\pgfpathlineto{\pgfqpoint{4.861395in}{4.049469in}}%
\pgfpathclose%
\pgfusepath{fill}%
\end{pgfscope}%
\begin{pgfscope}%
\pgfpathrectangle{\pgfqpoint{0.539299in}{0.078740in}}{\pgfqpoint{7.842520in}{7.842520in}}%
\pgfusepath{clip}%
\pgfsetbuttcap%
\pgfsetroundjoin%
\definecolor{currentfill}{rgb}{0.280255,0.165693,0.476498}%
\pgfsetfillcolor{currentfill}%
\pgfsetlinewidth{0.000000pt}%
\definecolor{currentstroke}{rgb}{0.955300,0.901065,0.118128}%
\pgfsetstrokecolor{currentstroke}%
\pgfsetdash{}{0pt}%
\pgfpathmoveto{\pgfqpoint{5.684398in}{3.594113in}}%
\pgfpathlineto{\pgfqpoint{5.549086in}{3.640953in}}%
\pgfpathlineto{\pgfqpoint{5.475289in}{3.644517in}}%
\pgfpathclose%
\pgfusepath{fill}%
\end{pgfscope}%
\begin{pgfscope}%
\pgfpathrectangle{\pgfqpoint{0.539299in}{0.078740in}}{\pgfqpoint{7.842520in}{7.842520in}}%
\pgfusepath{clip}%
\pgfsetbuttcap%
\pgfsetroundjoin%
\definecolor{currentfill}{rgb}{0.150476,0.504369,0.557430}%
\pgfsetfillcolor{currentfill}%
\pgfsetlinewidth{0.000000pt}%
\definecolor{currentstroke}{rgb}{0.964894,0.902323,0.123941}%
\pgfsetstrokecolor{currentstroke}%
\pgfsetdash{}{0pt}%
\pgfpathmoveto{\pgfqpoint{3.791614in}{4.877687in}}%
\pgfpathlineto{\pgfqpoint{3.843276in}{4.946277in}}%
\pgfpathlineto{\pgfqpoint{3.923882in}{4.868004in}}%
\pgfpathclose%
\pgfusepath{fill}%
\end{pgfscope}%
\begin{pgfscope}%
\pgfpathrectangle{\pgfqpoint{0.539299in}{0.078740in}}{\pgfqpoint{7.842520in}{7.842520in}}%
\pgfusepath{clip}%
\pgfsetbuttcap%
\pgfsetroundjoin%
\definecolor{currentfill}{rgb}{0.273809,0.031497,0.358853}%
\pgfsetfillcolor{currentfill}%
\pgfsetlinewidth{0.000000pt}%
\definecolor{currentstroke}{rgb}{0.974417,0.903590,0.130215}%
\pgfsetstrokecolor{currentstroke}%
\pgfsetdash{}{0pt}%
\pgfpathmoveto{\pgfqpoint{7.126919in}{3.173464in}}%
\pgfpathlineto{\pgfqpoint{7.058628in}{3.259387in}}%
\pgfpathlineto{\pgfqpoint{6.989604in}{3.234274in}}%
\pgfpathclose%
\pgfusepath{fill}%
\end{pgfscope}%
\begin{pgfscope}%
\pgfpathrectangle{\pgfqpoint{0.539299in}{0.078740in}}{\pgfqpoint{7.842520in}{7.842520in}}%
\pgfusepath{clip}%
\pgfsetbuttcap%
\pgfsetroundjoin%
\definecolor{currentfill}{rgb}{0.278791,0.062145,0.386592}%
\pgfsetfillcolor{currentfill}%
\pgfsetlinewidth{0.000000pt}%
\definecolor{currentstroke}{rgb}{0.983868,0.904867,0.136897}%
\pgfsetstrokecolor{currentstroke}%
\pgfsetdash{}{0pt}%
\pgfpathmoveto{\pgfqpoint{6.783027in}{3.277708in}}%
\pgfpathlineto{\pgfqpoint{6.852629in}{3.294416in}}%
\pgfpathlineto{\pgfqpoint{6.715949in}{3.352378in}}%
\pgfpathclose%
\pgfusepath{fill}%
\end{pgfscope}%
\begin{pgfscope}%
\pgfpathrectangle{\pgfqpoint{0.539299in}{0.078740in}}{\pgfqpoint{7.842520in}{7.842520in}}%
\pgfusepath{clip}%
\pgfsetbuttcap%
\pgfsetroundjoin%
\definecolor{currentfill}{rgb}{0.190631,0.407061,0.556089}%
\pgfsetfillcolor{currentfill}%
\pgfsetlinewidth{0.000000pt}%
\definecolor{currentstroke}{rgb}{0.993248,0.906157,0.143936}%
\pgfsetstrokecolor{currentstroke}%
\pgfsetdash{}{0pt}%
\pgfpathmoveto{\pgfqpoint{4.324600in}{4.594132in}}%
\pgfpathlineto{\pgfqpoint{4.458695in}{4.459156in}}%
\pgfpathlineto{\pgfqpoint{4.536301in}{4.363851in}}%
\pgfpathclose%
\pgfusepath{fill}%
\end{pgfscope}%
\begin{pgfscope}%
\pgfpathrectangle{\pgfqpoint{0.539299in}{0.078740in}}{\pgfqpoint{7.842520in}{7.842520in}}%
\pgfusepath{clip}%
\pgfsetbuttcap%
\pgfsetroundjoin%
\definecolor{currentfill}{rgb}{0.278012,0.180367,0.486697}%
\pgfsetfillcolor{currentfill}%
\pgfsetlinewidth{0.000000pt}%
\definecolor{currentstroke}{rgb}{0.267004,0.004874,0.329415}%
\pgfsetstrokecolor{currentstroke}%
\pgfsetdash{}{0pt}%
\pgfpathmoveto{\pgfqpoint{5.475289in}{3.644517in}}%
\pgfpathlineto{\pgfqpoint{5.549086in}{3.640953in}}%
\pgfpathlineto{\pgfqpoint{5.340183in}{3.710444in}}%
\pgfpathclose%
\pgfusepath{fill}%
\end{pgfscope}%
\begin{pgfscope}%
\pgfpathrectangle{\pgfqpoint{0.539299in}{0.078740in}}{\pgfqpoint{7.842520in}{7.842520in}}%
\pgfusepath{clip}%
\pgfsetbuttcap%
\pgfsetroundjoin%
\definecolor{currentfill}{rgb}{0.271305,0.019942,0.347269}%
\pgfsetfillcolor{currentfill}%
\pgfsetlinewidth{0.000000pt}%
\definecolor{currentstroke}{rgb}{0.268510,0.009605,0.335427}%
\pgfsetstrokecolor{currentstroke}%
\pgfsetdash{}{0pt}%
\pgfpathmoveto{\pgfqpoint{7.333628in}{3.157861in}}%
\pgfpathlineto{\pgfqpoint{7.195913in}{3.208174in}}%
\pgfpathlineto{\pgfqpoint{7.126919in}{3.173464in}}%
\pgfpathclose%
\pgfusepath{fill}%
\end{pgfscope}%
\begin{pgfscope}%
\pgfpathrectangle{\pgfqpoint{0.539299in}{0.078740in}}{\pgfqpoint{7.842520in}{7.842520in}}%
\pgfusepath{clip}%
\pgfsetbuttcap%
\pgfsetroundjoin%
\definecolor{currentfill}{rgb}{0.206756,0.371758,0.553117}%
\pgfsetfillcolor{currentfill}%
\pgfsetlinewidth{0.000000pt}%
\definecolor{currentstroke}{rgb}{0.269944,0.014625,0.341379}%
\pgfsetstrokecolor{currentstroke}%
\pgfsetdash{}{0pt}%
\pgfpathmoveto{\pgfqpoint{4.458695in}{4.459156in}}%
\pgfpathlineto{\pgfqpoint{4.592860in}{4.318703in}}%
\pgfpathlineto{\pgfqpoint{4.669832in}{4.230954in}}%
\pgfpathclose%
\pgfusepath{fill}%
\end{pgfscope}%
\begin{pgfscope}%
\pgfpathrectangle{\pgfqpoint{0.539299in}{0.078740in}}{\pgfqpoint{7.842520in}{7.842520in}}%
\pgfusepath{clip}%
\pgfsetbuttcap%
\pgfsetroundjoin%
\definecolor{currentfill}{rgb}{0.283091,0.110553,0.431554}%
\pgfsetfillcolor{currentfill}%
\pgfsetlinewidth{0.000000pt}%
\definecolor{currentstroke}{rgb}{0.271305,0.019942,0.347269}%
\pgfsetstrokecolor{currentstroke}%
\pgfsetdash{}{0pt}%
\pgfpathmoveto{\pgfqpoint{6.301764in}{3.437824in}}%
\pgfpathlineto{\pgfqpoint{6.509400in}{3.398734in}}%
\pgfpathlineto{\pgfqpoint{6.372915in}{3.450815in}}%
\pgfpathclose%
\pgfusepath{fill}%
\end{pgfscope}%
\begin{pgfscope}%
\pgfpathrectangle{\pgfqpoint{0.539299in}{0.078740in}}{\pgfqpoint{7.842520in}{7.842520in}}%
\pgfusepath{clip}%
\pgfsetbuttcap%
\pgfsetroundjoin%
\definecolor{currentfill}{rgb}{0.216210,0.351535,0.550627}%
\pgfsetfillcolor{currentfill}%
\pgfsetlinewidth{0.000000pt}%
\definecolor{currentstroke}{rgb}{0.272594,0.025563,0.353093}%
\pgfsetstrokecolor{currentstroke}%
\pgfsetdash{}{0pt}%
\pgfpathmoveto{\pgfqpoint{4.669832in}{4.230954in}}%
\pgfpathlineto{\pgfqpoint{4.592860in}{4.318703in}}%
\pgfpathlineto{\pgfqpoint{4.727082in}{4.180259in}}%
\pgfpathclose%
\pgfusepath{fill}%
\end{pgfscope}%
\begin{pgfscope}%
\pgfpathrectangle{\pgfqpoint{0.539299in}{0.078740in}}{\pgfqpoint{7.842520in}{7.842520in}}%
\pgfusepath{clip}%
\pgfsetbuttcap%
\pgfsetroundjoin%
\definecolor{currentfill}{rgb}{0.283229,0.120777,0.440584}%
\pgfsetfillcolor{currentfill}%
\pgfsetlinewidth{0.000000pt}%
\definecolor{currentstroke}{rgb}{0.273809,0.031497,0.358853}%
\pgfsetstrokecolor{currentstroke}%
\pgfsetdash{}{0pt}%
\pgfpathmoveto{\pgfqpoint{6.165167in}{3.484596in}}%
\pgfpathlineto{\pgfqpoint{6.301764in}{3.437824in}}%
\pgfpathlineto{\pgfqpoint{6.372915in}{3.450815in}}%
\pgfpathclose%
\pgfusepath{fill}%
\end{pgfscope}%
\begin{pgfscope}%
\pgfpathrectangle{\pgfqpoint{0.539299in}{0.078740in}}{\pgfqpoint{7.842520in}{7.842520in}}%
\pgfusepath{clip}%
\pgfsetbuttcap%
\pgfsetroundjoin%
\definecolor{currentfill}{rgb}{0.263663,0.237631,0.518762}%
\pgfsetfillcolor{currentfill}%
\pgfsetlinewidth{0.000000pt}%
\definecolor{currentstroke}{rgb}{0.274952,0.037752,0.364543}%
\pgfsetstrokecolor{currentstroke}%
\pgfsetdash{}{0pt}%
\pgfpathmoveto{\pgfqpoint{5.130570in}{3.824189in}}%
\pgfpathlineto{\pgfqpoint{5.205512in}{3.788278in}}%
\pgfpathlineto{\pgfqpoint{5.071218in}{3.879659in}}%
\pgfpathclose%
\pgfusepath{fill}%
\end{pgfscope}%
\begin{pgfscope}%
\pgfpathrectangle{\pgfqpoint{0.539299in}{0.078740in}}{\pgfqpoint{7.842520in}{7.842520in}}%
\pgfusepath{clip}%
\pgfsetbuttcap%
\pgfsetroundjoin%
\definecolor{currentfill}{rgb}{0.146180,0.515413,0.556823}%
\pgfsetfillcolor{currentfill}%
\pgfsetlinewidth{0.000000pt}%
\definecolor{currentstroke}{rgb}{0.276022,0.044167,0.370164}%
\pgfsetstrokecolor{currentstroke}%
\pgfsetdash{}{0pt}%
\pgfpathmoveto{\pgfqpoint{3.710739in}{4.939290in}}%
\pgfpathlineto{\pgfqpoint{3.843276in}{4.946277in}}%
\pgfpathlineto{\pgfqpoint{3.791614in}{4.877687in}}%
\pgfpathclose%
\pgfusepath{fill}%
\end{pgfscope}%
\begin{pgfscope}%
\pgfpathrectangle{\pgfqpoint{0.539299in}{0.078740in}}{\pgfqpoint{7.842520in}{7.842520in}}%
\pgfusepath{clip}%
\pgfsetbuttcap%
\pgfsetroundjoin%
\definecolor{currentfill}{rgb}{0.280894,0.078907,0.402329}%
\pgfsetfillcolor{currentfill}%
\pgfsetlinewidth{0.000000pt}%
\definecolor{currentstroke}{rgb}{0.277018,0.050344,0.375715}%
\pgfsetstrokecolor{currentstroke}%
\pgfsetdash{}{0pt}%
\pgfpathmoveto{\pgfqpoint{6.715949in}{3.352378in}}%
\pgfpathlineto{\pgfqpoint{6.646106in}{3.340736in}}%
\pgfpathlineto{\pgfqpoint{6.783027in}{3.277708in}}%
\pgfpathclose%
\pgfusepath{fill}%
\end{pgfscope}%
\begin{pgfscope}%
\pgfpathrectangle{\pgfqpoint{0.539299in}{0.078740in}}{\pgfqpoint{7.842520in}{7.842520in}}%
\pgfusepath{clip}%
\pgfsetbuttcap%
\pgfsetroundjoin%
\definecolor{currentfill}{rgb}{0.231674,0.318106,0.544834}%
\pgfsetfillcolor{currentfill}%
\pgfsetlinewidth{0.000000pt}%
\definecolor{currentstroke}{rgb}{0.277941,0.056324,0.381191}%
\pgfsetstrokecolor{currentstroke}%
\pgfsetdash{}{0pt}%
\pgfpathmoveto{\pgfqpoint{4.727082in}{4.180259in}}%
\pgfpathlineto{\pgfqpoint{4.861395in}{4.049469in}}%
\pgfpathlineto{\pgfqpoint{4.803456in}{4.103048in}}%
\pgfpathclose%
\pgfusepath{fill}%
\end{pgfscope}%
\begin{pgfscope}%
\pgfpathrectangle{\pgfqpoint{0.539299in}{0.078740in}}{\pgfqpoint{7.842520in}{7.842520in}}%
\pgfusepath{clip}%
\pgfsetbuttcap%
\pgfsetroundjoin%
\definecolor{currentfill}{rgb}{0.147607,0.511733,0.557049}%
\pgfsetfillcolor{currentfill}%
\pgfsetlinewidth{0.000000pt}%
\definecolor{currentstroke}{rgb}{0.278791,0.062145,0.386592}%
\pgfsetstrokecolor{currentstroke}%
\pgfsetdash{}{0pt}%
\pgfpathmoveto{\pgfqpoint{3.498080in}{4.903513in}}%
\pgfpathlineto{\pgfqpoint{3.710739in}{4.939290in}}%
\pgfpathlineto{\pgfqpoint{3.579639in}{4.867750in}}%
\pgfpathclose%
\pgfusepath{fill}%
\end{pgfscope}%
\begin{pgfscope}%
\pgfpathrectangle{\pgfqpoint{0.539299in}{0.078740in}}{\pgfqpoint{7.842520in}{7.842520in}}%
\pgfusepath{clip}%
\pgfsetbuttcap%
\pgfsetroundjoin%
\definecolor{currentfill}{rgb}{0.277941,0.056324,0.381191}%
\pgfsetfillcolor{currentfill}%
\pgfsetlinewidth{0.000000pt}%
\definecolor{currentstroke}{rgb}{0.279566,0.067836,0.391917}%
\pgfsetstrokecolor{currentstroke}%
\pgfsetdash{}{0pt}%
\pgfpathmoveto{\pgfqpoint{6.989604in}{3.234274in}}%
\pgfpathlineto{\pgfqpoint{6.852629in}{3.294416in}}%
\pgfpathlineto{\pgfqpoint{6.783027in}{3.277708in}}%
\pgfpathclose%
\pgfusepath{fill}%
\end{pgfscope}%
\begin{pgfscope}%
\pgfpathrectangle{\pgfqpoint{0.539299in}{0.078740in}}{\pgfqpoint{7.842520in}{7.842520in}}%
\pgfusepath{clip}%
\pgfsetbuttcap%
\pgfsetroundjoin%
\definecolor{currentfill}{rgb}{0.282290,0.145912,0.461510}%
\pgfsetfillcolor{currentfill}%
\pgfsetlinewidth{0.000000pt}%
\definecolor{currentstroke}{rgb}{0.280267,0.073417,0.397163}%
\pgfsetstrokecolor{currentstroke}%
\pgfsetdash{}{0pt}%
\pgfpathmoveto{\pgfqpoint{5.956414in}{3.509868in}}%
\pgfpathlineto{\pgfqpoint{6.028872in}{3.526024in}}%
\pgfpathlineto{\pgfqpoint{5.820187in}{3.551515in}}%
\pgfpathclose%
\pgfusepath{fill}%
\end{pgfscope}%
\begin{pgfscope}%
\pgfpathrectangle{\pgfqpoint{0.539299in}{0.078740in}}{\pgfqpoint{7.842520in}{7.842520in}}%
\pgfusepath{clip}%
\pgfsetbuttcap%
\pgfsetroundjoin%
\definecolor{currentfill}{rgb}{0.280255,0.165693,0.476498}%
\pgfsetfillcolor{currentfill}%
\pgfsetlinewidth{0.000000pt}%
\definecolor{currentstroke}{rgb}{0.280894,0.078907,0.402329}%
\pgfsetstrokecolor{currentstroke}%
\pgfsetdash{}{0pt}%
\pgfpathmoveto{\pgfqpoint{5.820187in}{3.551515in}}%
\pgfpathlineto{\pgfqpoint{5.684398in}{3.594113in}}%
\pgfpathlineto{\pgfqpoint{5.610860in}{3.587946in}}%
\pgfpathclose%
\pgfusepath{fill}%
\end{pgfscope}%
\begin{pgfscope}%
\pgfpathrectangle{\pgfqpoint{0.539299in}{0.078740in}}{\pgfqpoint{7.842520in}{7.842520in}}%
\pgfusepath{clip}%
\pgfsetbuttcap%
\pgfsetroundjoin%
\definecolor{currentfill}{rgb}{0.267968,0.223549,0.512008}%
\pgfsetfillcolor{currentfill}%
\pgfsetlinewidth{0.000000pt}%
\definecolor{currentstroke}{rgb}{0.281446,0.084320,0.407414}%
\pgfsetstrokecolor{currentstroke}%
\pgfsetdash{}{0pt}%
\pgfpathmoveto{\pgfqpoint{5.130570in}{3.824189in}}%
\pgfpathlineto{\pgfqpoint{5.340183in}{3.710444in}}%
\pgfpathlineto{\pgfqpoint{5.205512in}{3.788278in}}%
\pgfpathclose%
\pgfusepath{fill}%
\end{pgfscope}%
\begin{pgfscope}%
\pgfpathrectangle{\pgfqpoint{0.539299in}{0.078740in}}{\pgfqpoint{7.842520in}{7.842520in}}%
\pgfusepath{clip}%
\pgfsetbuttcap%
\pgfsetroundjoin%
\definecolor{currentfill}{rgb}{0.282884,0.135920,0.453427}%
\pgfsetfillcolor{currentfill}%
\pgfsetlinewidth{0.000000pt}%
\definecolor{currentstroke}{rgb}{0.281924,0.089666,0.412415}%
\pgfsetstrokecolor{currentstroke}%
\pgfsetdash{}{0pt}%
\pgfpathmoveto{\pgfqpoint{6.093026in}{3.466206in}}%
\pgfpathlineto{\pgfqpoint{6.165167in}{3.484596in}}%
\pgfpathlineto{\pgfqpoint{6.028872in}{3.526024in}}%
\pgfpathclose%
\pgfusepath{fill}%
\end{pgfscope}%
\begin{pgfscope}%
\pgfpathrectangle{\pgfqpoint{0.539299in}{0.078740in}}{\pgfqpoint{7.842520in}{7.842520in}}%
\pgfusepath{clip}%
\pgfsetbuttcap%
\pgfsetroundjoin%
\definecolor{currentfill}{rgb}{0.248629,0.278775,0.534556}%
\pgfsetfillcolor{currentfill}%
\pgfsetlinewidth{0.000000pt}%
\definecolor{currentstroke}{rgb}{0.282327,0.094955,0.417331}%
\pgfsetstrokecolor{currentstroke}%
\pgfsetdash{}{0pt}%
\pgfpathmoveto{\pgfqpoint{4.995865in}{3.930113in}}%
\pgfpathlineto{\pgfqpoint{5.071218in}{3.879659in}}%
\pgfpathlineto{\pgfqpoint{4.861395in}{4.049469in}}%
\pgfpathclose%
\pgfusepath{fill}%
\end{pgfscope}%
\begin{pgfscope}%
\pgfpathrectangle{\pgfqpoint{0.539299in}{0.078740in}}{\pgfqpoint{7.842520in}{7.842520in}}%
\pgfusepath{clip}%
\pgfsetbuttcap%
\pgfsetroundjoin%
\definecolor{currentfill}{rgb}{0.150476,0.504369,0.557430}%
\pgfsetfillcolor{currentfill}%
\pgfsetlinewidth{0.000000pt}%
\definecolor{currentstroke}{rgb}{0.282656,0.100196,0.422160}%
\pgfsetstrokecolor{currentstroke}%
\pgfsetdash{}{0pt}%
\pgfpathmoveto{\pgfqpoint{3.976815in}{4.900063in}}%
\pgfpathlineto{\pgfqpoint{4.057001in}{4.809605in}}%
\pgfpathlineto{\pgfqpoint{3.923882in}{4.868004in}}%
\pgfpathclose%
\pgfusepath{fill}%
\end{pgfscope}%
\begin{pgfscope}%
\pgfpathrectangle{\pgfqpoint{0.539299in}{0.078740in}}{\pgfqpoint{7.842520in}{7.842520in}}%
\pgfusepath{clip}%
\pgfsetbuttcap%
\pgfsetroundjoin%
\definecolor{currentfill}{rgb}{0.278826,0.175490,0.483397}%
\pgfsetfillcolor{currentfill}%
\pgfsetlinewidth{0.000000pt}%
\definecolor{currentstroke}{rgb}{0.282910,0.105393,0.426902}%
\pgfsetstrokecolor{currentstroke}%
\pgfsetdash{}{0pt}%
\pgfpathmoveto{\pgfqpoint{5.475289in}{3.644517in}}%
\pgfpathlineto{\pgfqpoint{5.610860in}{3.587946in}}%
\pgfpathlineto{\pgfqpoint{5.684398in}{3.594113in}}%
\pgfpathclose%
\pgfusepath{fill}%
\end{pgfscope}%
\begin{pgfscope}%
\pgfpathrectangle{\pgfqpoint{0.539299in}{0.078740in}}{\pgfqpoint{7.842520in}{7.842520in}}%
\pgfusepath{clip}%
\pgfsetbuttcap%
\pgfsetroundjoin%
\definecolor{currentfill}{rgb}{0.269944,0.014625,0.341379}%
\pgfsetfillcolor{currentfill}%
\pgfsetlinewidth{0.000000pt}%
\definecolor{currentstroke}{rgb}{0.283091,0.110553,0.431554}%
\pgfsetstrokecolor{currentstroke}%
\pgfsetdash{}{0pt}%
\pgfpathmoveto{\pgfqpoint{7.264628in}{3.113510in}}%
\pgfpathlineto{\pgfqpoint{7.471813in}{3.109193in}}%
\pgfpathlineto{\pgfqpoint{7.333628in}{3.157861in}}%
\pgfpathclose%
\pgfusepath{fill}%
\end{pgfscope}%
\begin{pgfscope}%
\pgfpathrectangle{\pgfqpoint{0.539299in}{0.078740in}}{\pgfqpoint{7.842520in}{7.842520in}}%
\pgfusepath{clip}%
\pgfsetbuttcap%
\pgfsetroundjoin%
\definecolor{currentfill}{rgb}{0.271305,0.019942,0.347269}%
\pgfsetfillcolor{currentfill}%
\pgfsetlinewidth{0.000000pt}%
\definecolor{currentstroke}{rgb}{0.283197,0.115680,0.436115}%
\pgfsetstrokecolor{currentstroke}%
\pgfsetdash{}{0pt}%
\pgfpathmoveto{\pgfqpoint{7.126919in}{3.173464in}}%
\pgfpathlineto{\pgfqpoint{7.264628in}{3.113510in}}%
\pgfpathlineto{\pgfqpoint{7.333628in}{3.157861in}}%
\pgfpathclose%
\pgfusepath{fill}%
\end{pgfscope}%
\begin{pgfscope}%
\pgfpathrectangle{\pgfqpoint{0.539299in}{0.078740in}}{\pgfqpoint{7.842520in}{7.842520in}}%
\pgfusepath{clip}%
\pgfsetbuttcap%
\pgfsetroundjoin%
\definecolor{currentfill}{rgb}{0.255645,0.260703,0.528312}%
\pgfsetfillcolor{currentfill}%
\pgfsetlinewidth{0.000000pt}%
\definecolor{currentstroke}{rgb}{0.283229,0.120777,0.440584}%
\pgfsetstrokecolor{currentstroke}%
\pgfsetdash{}{0pt}%
\pgfpathmoveto{\pgfqpoint{5.130570in}{3.824189in}}%
\pgfpathlineto{\pgfqpoint{5.071218in}{3.879659in}}%
\pgfpathlineto{\pgfqpoint{4.995865in}{3.930113in}}%
\pgfpathclose%
\pgfusepath{fill}%
\end{pgfscope}%
\begin{pgfscope}%
\pgfpathrectangle{\pgfqpoint{0.539299in}{0.078740in}}{\pgfqpoint{7.842520in}{7.842520in}}%
\pgfusepath{clip}%
\pgfsetbuttcap%
\pgfsetroundjoin%
\definecolor{currentfill}{rgb}{0.283197,0.115680,0.436115}%
\pgfsetfillcolor{currentfill}%
\pgfsetlinewidth{0.000000pt}%
\definecolor{currentstroke}{rgb}{0.283187,0.125848,0.444960}%
\pgfsetstrokecolor{currentstroke}%
\pgfsetdash{}{0pt}%
\pgfpathmoveto{\pgfqpoint{6.301764in}{3.437824in}}%
\pgfpathlineto{\pgfqpoint{6.438604in}{3.384604in}}%
\pgfpathlineto{\pgfqpoint{6.509400in}{3.398734in}}%
\pgfpathclose%
\pgfusepath{fill}%
\end{pgfscope}%
\begin{pgfscope}%
\pgfpathrectangle{\pgfqpoint{0.539299in}{0.078740in}}{\pgfqpoint{7.842520in}{7.842520in}}%
\pgfusepath{clip}%
\pgfsetbuttcap%
\pgfsetroundjoin%
\definecolor{currentfill}{rgb}{0.282290,0.145912,0.461510}%
\pgfsetfillcolor{currentfill}%
\pgfsetlinewidth{0.000000pt}%
\definecolor{currentstroke}{rgb}{0.283072,0.130895,0.449241}%
\pgfsetstrokecolor{currentstroke}%
\pgfsetdash{}{0pt}%
\pgfpathmoveto{\pgfqpoint{6.093026in}{3.466206in}}%
\pgfpathlineto{\pgfqpoint{6.028872in}{3.526024in}}%
\pgfpathlineto{\pgfqpoint{5.956414in}{3.509868in}}%
\pgfpathclose%
\pgfusepath{fill}%
\end{pgfscope}%
\begin{pgfscope}%
\pgfpathrectangle{\pgfqpoint{0.539299in}{0.078740in}}{\pgfqpoint{7.842520in}{7.842520in}}%
\pgfusepath{clip}%
\pgfsetbuttcap%
\pgfsetroundjoin%
\definecolor{currentfill}{rgb}{0.153364,0.497000,0.557724}%
\pgfsetfillcolor{currentfill}%
\pgfsetlinewidth{0.000000pt}%
\definecolor{currentstroke}{rgb}{0.282884,0.135920,0.453427}%
\pgfsetstrokecolor{currentstroke}%
\pgfsetdash{}{0pt}%
\pgfpathmoveto{\pgfqpoint{4.190652in}{4.714468in}}%
\pgfpathlineto{\pgfqpoint{4.057001in}{4.809605in}}%
\pgfpathlineto{\pgfqpoint{3.976815in}{4.900063in}}%
\pgfpathclose%
\pgfusepath{fill}%
\end{pgfscope}%
\begin{pgfscope}%
\pgfpathrectangle{\pgfqpoint{0.539299in}{0.078740in}}{\pgfqpoint{7.842520in}{7.842520in}}%
\pgfusepath{clip}%
\pgfsetbuttcap%
\pgfsetroundjoin%
\definecolor{currentfill}{rgb}{0.153364,0.497000,0.557724}%
\pgfsetfillcolor{currentfill}%
\pgfsetlinewidth{0.000000pt}%
\definecolor{currentstroke}{rgb}{0.282623,0.140926,0.457517}%
\pgfsetstrokecolor{currentstroke}%
\pgfsetdash{}{0pt}%
\pgfpathmoveto{\pgfqpoint{3.415921in}{4.934934in}}%
\pgfpathlineto{\pgfqpoint{3.369083in}{4.733483in}}%
\pgfpathlineto{\pgfqpoint{3.287152in}{4.741607in}}%
\pgfpathclose%
\pgfusepath{fill}%
\end{pgfscope}%
\begin{pgfscope}%
\pgfpathrectangle{\pgfqpoint{0.539299in}{0.078740in}}{\pgfqpoint{7.842520in}{7.842520in}}%
\pgfusepath{clip}%
\pgfsetbuttcap%
\pgfsetroundjoin%
\definecolor{currentfill}{rgb}{0.210503,0.363727,0.552206}%
\pgfsetfillcolor{currentfill}%
\pgfsetlinewidth{0.000000pt}%
\definecolor{currentstroke}{rgb}{0.282290,0.145912,0.461510}%
\pgfsetstrokecolor{currentstroke}%
\pgfsetdash{}{0pt}%
\pgfpathmoveto{\pgfqpoint{2.913780in}{4.395967in}}%
\pgfpathlineto{\pgfqpoint{2.996771in}{4.420700in}}%
\pgfpathlineto{\pgfqpoint{2.793452in}{3.956696in}}%
\pgfpathclose%
\pgfusepath{fill}%
\end{pgfscope}%
\begin{pgfscope}%
\pgfpathrectangle{\pgfqpoint{0.539299in}{0.078740in}}{\pgfqpoint{7.842520in}{7.842520in}}%
\pgfusepath{clip}%
\pgfsetbuttcap%
\pgfsetroundjoin%
\definecolor{currentfill}{rgb}{0.282656,0.100196,0.422160}%
\pgfsetfillcolor{currentfill}%
\pgfsetlinewidth{0.000000pt}%
\definecolor{currentstroke}{rgb}{0.281887,0.150881,0.465405}%
\pgfsetstrokecolor{currentstroke}%
\pgfsetdash{}{0pt}%
\pgfpathmoveto{\pgfqpoint{6.575647in}{3.324679in}}%
\pgfpathlineto{\pgfqpoint{6.646106in}{3.340736in}}%
\pgfpathlineto{\pgfqpoint{6.509400in}{3.398734in}}%
\pgfpathclose%
\pgfusepath{fill}%
\end{pgfscope}%
\begin{pgfscope}%
\pgfpathrectangle{\pgfqpoint{0.539299in}{0.078740in}}{\pgfqpoint{7.842520in}{7.842520in}}%
\pgfusepath{clip}%
\pgfsetbuttcap%
\pgfsetroundjoin%
\definecolor{currentfill}{rgb}{0.168126,0.459988,0.558082}%
\pgfsetfillcolor{currentfill}%
\pgfsetlinewidth{0.000000pt}%
\definecolor{currentstroke}{rgb}{0.281412,0.155834,0.469201}%
\pgfsetstrokecolor{currentstroke}%
\pgfsetdash{}{0pt}%
\pgfpathmoveto{\pgfqpoint{3.079268in}{4.442627in}}%
\pgfpathlineto{\pgfqpoint{3.204675in}{4.746147in}}%
\pgfpathlineto{\pgfqpoint{3.287152in}{4.741607in}}%
\pgfpathclose%
\pgfusepath{fill}%
\end{pgfscope}%
\begin{pgfscope}%
\pgfpathrectangle{\pgfqpoint{0.539299in}{0.078740in}}{\pgfqpoint{7.842520in}{7.842520in}}%
\pgfusepath{clip}%
\pgfsetbuttcap%
\pgfsetroundjoin%
\definecolor{currentfill}{rgb}{0.147607,0.511733,0.557049}%
\pgfsetfillcolor{currentfill}%
\pgfsetlinewidth{0.000000pt}%
\definecolor{currentstroke}{rgb}{0.280868,0.160771,0.472899}%
\pgfsetstrokecolor{currentstroke}%
\pgfsetdash{}{0pt}%
\pgfpathmoveto{\pgfqpoint{3.415921in}{4.934934in}}%
\pgfpathlineto{\pgfqpoint{3.498080in}{4.903513in}}%
\pgfpathlineto{\pgfqpoint{3.369083in}{4.733483in}}%
\pgfpathclose%
\pgfusepath{fill}%
\end{pgfscope}%
\begin{pgfscope}%
\pgfpathrectangle{\pgfqpoint{0.539299in}{0.078740in}}{\pgfqpoint{7.842520in}{7.842520in}}%
\pgfusepath{clip}%
\pgfsetbuttcap%
\pgfsetroundjoin%
\definecolor{currentfill}{rgb}{0.282884,0.135920,0.453427}%
\pgfsetfillcolor{currentfill}%
\pgfsetlinewidth{0.000000pt}%
\definecolor{currentstroke}{rgb}{0.280255,0.165693,0.476498}%
\pgfsetstrokecolor{currentstroke}%
\pgfsetdash{}{0pt}%
\pgfpathmoveto{\pgfqpoint{6.301764in}{3.437824in}}%
\pgfpathlineto{\pgfqpoint{6.165167in}{3.484596in}}%
\pgfpathlineto{\pgfqpoint{6.093026in}{3.466206in}}%
\pgfpathclose%
\pgfusepath{fill}%
\end{pgfscope}%
\begin{pgfscope}%
\pgfpathrectangle{\pgfqpoint{0.539299in}{0.078740in}}{\pgfqpoint{7.842520in}{7.842520in}}%
\pgfusepath{clip}%
\pgfsetbuttcap%
\pgfsetroundjoin%
\definecolor{currentfill}{rgb}{0.278791,0.062145,0.386592}%
\pgfsetfillcolor{currentfill}%
\pgfsetlinewidth{0.000000pt}%
\definecolor{currentstroke}{rgb}{0.279574,0.170599,0.479997}%
\pgfsetstrokecolor{currentstroke}%
\pgfsetdash{}{0pt}%
\pgfpathmoveto{\pgfqpoint{6.783027in}{3.277708in}}%
\pgfpathlineto{\pgfqpoint{6.920184in}{3.211063in}}%
\pgfpathlineto{\pgfqpoint{6.989604in}{3.234274in}}%
\pgfpathclose%
\pgfusepath{fill}%
\end{pgfscope}%
\begin{pgfscope}%
\pgfpathrectangle{\pgfqpoint{0.539299in}{0.078740in}}{\pgfqpoint{7.842520in}{7.842520in}}%
\pgfusepath{clip}%
\pgfsetbuttcap%
\pgfsetroundjoin%
\definecolor{currentfill}{rgb}{0.276022,0.044167,0.370164}%
\pgfsetfillcolor{currentfill}%
\pgfsetlinewidth{0.000000pt}%
\definecolor{currentstroke}{rgb}{0.278826,0.175490,0.483397}%
\pgfsetstrokecolor{currentstroke}%
\pgfsetdash{}{0pt}%
\pgfpathmoveto{\pgfqpoint{7.057617in}{3.142554in}}%
\pgfpathlineto{\pgfqpoint{7.126919in}{3.173464in}}%
\pgfpathlineto{\pgfqpoint{6.989604in}{3.234274in}}%
\pgfpathclose%
\pgfusepath{fill}%
\end{pgfscope}%
\begin{pgfscope}%
\pgfpathrectangle{\pgfqpoint{0.539299in}{0.078740in}}{\pgfqpoint{7.842520in}{7.842520in}}%
\pgfusepath{clip}%
\pgfsetbuttcap%
\pgfsetroundjoin%
\definecolor{currentfill}{rgb}{0.143343,0.522773,0.556295}%
\pgfsetfillcolor{currentfill}%
\pgfsetlinewidth{0.000000pt}%
\definecolor{currentstroke}{rgb}{0.278012,0.180367,0.486697}%
\pgfsetstrokecolor{currentstroke}%
\pgfsetdash{}{0pt}%
\pgfpathmoveto{\pgfqpoint{3.923882in}{4.868004in}}%
\pgfpathlineto{\pgfqpoint{3.843276in}{4.946277in}}%
\pgfpathlineto{\pgfqpoint{3.976815in}{4.900063in}}%
\pgfpathclose%
\pgfusepath{fill}%
\end{pgfscope}%
\begin{pgfscope}%
\pgfpathrectangle{\pgfqpoint{0.539299in}{0.078740in}}{\pgfqpoint{7.842520in}{7.842520in}}%
\pgfusepath{clip}%
\pgfsetbuttcap%
\pgfsetroundjoin%
\definecolor{currentfill}{rgb}{0.273006,0.204520,0.501721}%
\pgfsetfillcolor{currentfill}%
\pgfsetlinewidth{0.000000pt}%
\definecolor{currentstroke}{rgb}{0.277134,0.185228,0.489898}%
\pgfsetstrokecolor{currentstroke}%
\pgfsetdash{}{0pt}%
\pgfpathmoveto{\pgfqpoint{5.340183in}{3.710444in}}%
\pgfpathlineto{\pgfqpoint{5.400994in}{3.652853in}}%
\pgfpathlineto{\pgfqpoint{5.475289in}{3.644517in}}%
\pgfpathclose%
\pgfusepath{fill}%
\end{pgfscope}%
\begin{pgfscope}%
\pgfpathrectangle{\pgfqpoint{0.539299in}{0.078740in}}{\pgfqpoint{7.842520in}{7.842520in}}%
\pgfusepath{clip}%
\pgfsetbuttcap%
\pgfsetroundjoin%
\definecolor{currentfill}{rgb}{0.229739,0.322361,0.545706}%
\pgfsetfillcolor{currentfill}%
\pgfsetlinewidth{0.000000pt}%
\definecolor{currentstroke}{rgb}{0.276194,0.190074,0.493001}%
\pgfsetstrokecolor{currentstroke}%
\pgfsetdash{}{0pt}%
\pgfpathmoveto{\pgfqpoint{2.793452in}{3.956696in}}%
\pgfpathlineto{\pgfqpoint{2.710747in}{3.908413in}}%
\pgfpathlineto{\pgfqpoint{2.830306in}{4.368134in}}%
\pgfpathclose%
\pgfusepath{fill}%
\end{pgfscope}%
\begin{pgfscope}%
\pgfpathrectangle{\pgfqpoint{0.539299in}{0.078740in}}{\pgfqpoint{7.842520in}{7.842520in}}%
\pgfusepath{clip}%
\pgfsetbuttcap%
\pgfsetroundjoin%
\definecolor{currentfill}{rgb}{0.160665,0.478540,0.558115}%
\pgfsetfillcolor{currentfill}%
\pgfsetlinewidth{0.000000pt}%
\definecolor{currentstroke}{rgb}{0.275191,0.194905,0.496005}%
\pgfsetstrokecolor{currentstroke}%
\pgfsetdash{}{0pt}%
\pgfpathmoveto{\pgfqpoint{4.190652in}{4.714468in}}%
\pgfpathlineto{\pgfqpoint{4.111002in}{4.812426in}}%
\pgfpathlineto{\pgfqpoint{4.324600in}{4.594132in}}%
\pgfpathclose%
\pgfusepath{fill}%
\end{pgfscope}%
\begin{pgfscope}%
\pgfpathrectangle{\pgfqpoint{0.539299in}{0.078740in}}{\pgfqpoint{7.842520in}{7.842520in}}%
\pgfusepath{clip}%
\pgfsetbuttcap%
\pgfsetroundjoin%
\definecolor{currentfill}{rgb}{0.283091,0.110553,0.431554}%
\pgfsetfillcolor{currentfill}%
\pgfsetlinewidth{0.000000pt}%
\definecolor{currentstroke}{rgb}{0.274128,0.199721,0.498911}%
\pgfsetstrokecolor{currentstroke}%
\pgfsetdash{}{0pt}%
\pgfpathmoveto{\pgfqpoint{6.509400in}{3.398734in}}%
\pgfpathlineto{\pgfqpoint{6.438604in}{3.384604in}}%
\pgfpathlineto{\pgfqpoint{6.575647in}{3.324679in}}%
\pgfpathclose%
\pgfusepath{fill}%
\end{pgfscope}%
\begin{pgfscope}%
\pgfpathrectangle{\pgfqpoint{0.539299in}{0.078740in}}{\pgfqpoint{7.842520in}{7.842520in}}%
\pgfusepath{clip}%
\pgfsetbuttcap%
\pgfsetroundjoin%
\definecolor{currentfill}{rgb}{0.265145,0.232956,0.516599}%
\pgfsetfillcolor{currentfill}%
\pgfsetlinewidth{0.000000pt}%
\definecolor{currentstroke}{rgb}{0.273006,0.204520,0.501721}%
\pgfsetstrokecolor{currentstroke}%
\pgfsetdash{}{0pt}%
\pgfpathmoveto{\pgfqpoint{5.265591in}{3.732089in}}%
\pgfpathlineto{\pgfqpoint{5.340183in}{3.710444in}}%
\pgfpathlineto{\pgfqpoint{5.130570in}{3.824189in}}%
\pgfpathclose%
\pgfusepath{fill}%
\end{pgfscope}%
\begin{pgfscope}%
\pgfpathrectangle{\pgfqpoint{0.539299in}{0.078740in}}{\pgfqpoint{7.842520in}{7.842520in}}%
\pgfusepath{clip}%
\pgfsetbuttcap%
\pgfsetroundjoin%
\definecolor{currentfill}{rgb}{0.279574,0.170599,0.479997}%
\pgfsetfillcolor{currentfill}%
\pgfsetlinewidth{0.000000pt}%
\definecolor{currentstroke}{rgb}{0.271828,0.209303,0.504434}%
\pgfsetstrokecolor{currentstroke}%
\pgfsetdash{}{0pt}%
\pgfpathmoveto{\pgfqpoint{5.610860in}{3.587946in}}%
\pgfpathlineto{\pgfqpoint{5.746902in}{3.537634in}}%
\pgfpathlineto{\pgfqpoint{5.820187in}{3.551515in}}%
\pgfpathclose%
\pgfusepath{fill}%
\end{pgfscope}%
\begin{pgfscope}%
\pgfpathrectangle{\pgfqpoint{0.539299in}{0.078740in}}{\pgfqpoint{7.842520in}{7.842520in}}%
\pgfusepath{clip}%
\pgfsetbuttcap%
\pgfsetroundjoin%
\definecolor{currentfill}{rgb}{0.282327,0.094955,0.417331}%
\pgfsetfillcolor{currentfill}%
\pgfsetlinewidth{0.000000pt}%
\definecolor{currentstroke}{rgb}{0.270595,0.214069,0.507052}%
\pgfsetstrokecolor{currentstroke}%
\pgfsetdash{}{0pt}%
\pgfpathmoveto{\pgfqpoint{6.575647in}{3.324679in}}%
\pgfpathlineto{\pgfqpoint{6.783027in}{3.277708in}}%
\pgfpathlineto{\pgfqpoint{6.646106in}{3.340736in}}%
\pgfpathclose%
\pgfusepath{fill}%
\end{pgfscope}%
\begin{pgfscope}%
\pgfpathrectangle{\pgfqpoint{0.539299in}{0.078740in}}{\pgfqpoint{7.842520in}{7.842520in}}%
\pgfusepath{clip}%
\pgfsetbuttcap%
\pgfsetroundjoin%
\definecolor{currentfill}{rgb}{0.269944,0.014625,0.341379}%
\pgfsetfillcolor{currentfill}%
\pgfsetlinewidth{0.000000pt}%
\definecolor{currentstroke}{rgb}{0.269308,0.218818,0.509577}%
\pgfsetstrokecolor{currentstroke}%
\pgfsetdash{}{0pt}%
\pgfpathmoveto{\pgfqpoint{7.402794in}{3.055818in}}%
\pgfpathlineto{\pgfqpoint{7.471813in}{3.109193in}}%
\pgfpathlineto{\pgfqpoint{7.264628in}{3.113510in}}%
\pgfpathclose%
\pgfusepath{fill}%
\end{pgfscope}%
\begin{pgfscope}%
\pgfpathrectangle{\pgfqpoint{0.539299in}{0.078740in}}{\pgfqpoint{7.842520in}{7.842520in}}%
\pgfusepath{clip}%
\pgfsetbuttcap%
\pgfsetroundjoin%
\definecolor{currentfill}{rgb}{0.277941,0.056324,0.381191}%
\pgfsetfillcolor{currentfill}%
\pgfsetlinewidth{0.000000pt}%
\definecolor{currentstroke}{rgb}{0.267968,0.223549,0.512008}%
\pgfsetstrokecolor{currentstroke}%
\pgfsetdash{}{0pt}%
\pgfpathmoveto{\pgfqpoint{6.989604in}{3.234274in}}%
\pgfpathlineto{\pgfqpoint{6.920184in}{3.211063in}}%
\pgfpathlineto{\pgfqpoint{7.057617in}{3.142554in}}%
\pgfpathclose%
\pgfusepath{fill}%
\end{pgfscope}%
\begin{pgfscope}%
\pgfpathrectangle{\pgfqpoint{0.539299in}{0.078740in}}{\pgfqpoint{7.842520in}{7.842520in}}%
\pgfusepath{clip}%
\pgfsetbuttcap%
\pgfsetroundjoin%
\definecolor{currentfill}{rgb}{0.137770,0.537492,0.554906}%
\pgfsetfillcolor{currentfill}%
\pgfsetlinewidth{0.000000pt}%
\definecolor{currentstroke}{rgb}{0.266580,0.228262,0.514349}%
\pgfsetstrokecolor{currentstroke}%
\pgfsetdash{}{0pt}%
\pgfpathmoveto{\pgfqpoint{3.629210in}{4.996070in}}%
\pgfpathlineto{\pgfqpoint{3.710739in}{4.939290in}}%
\pgfpathlineto{\pgfqpoint{3.498080in}{4.903513in}}%
\pgfpathclose%
\pgfusepath{fill}%
\end{pgfscope}%
\begin{pgfscope}%
\pgfpathrectangle{\pgfqpoint{0.539299in}{0.078740in}}{\pgfqpoint{7.842520in}{7.842520in}}%
\pgfusepath{clip}%
\pgfsetbuttcap%
\pgfsetroundjoin%
\definecolor{currentfill}{rgb}{0.280868,0.160771,0.472899}%
\pgfsetfillcolor{currentfill}%
\pgfsetlinewidth{0.000000pt}%
\definecolor{currentstroke}{rgb}{0.265145,0.232956,0.516599}%
\pgfsetstrokecolor{currentstroke}%
\pgfsetdash{}{0pt}%
\pgfpathmoveto{\pgfqpoint{5.820187in}{3.551515in}}%
\pgfpathlineto{\pgfqpoint{5.883392in}{3.490293in}}%
\pgfpathlineto{\pgfqpoint{5.956414in}{3.509868in}}%
\pgfpathclose%
\pgfusepath{fill}%
\end{pgfscope}%
\begin{pgfscope}%
\pgfpathrectangle{\pgfqpoint{0.539299in}{0.078740in}}{\pgfqpoint{7.842520in}{7.842520in}}%
\pgfusepath{clip}%
\pgfsetbuttcap%
\pgfsetroundjoin%
\definecolor{currentfill}{rgb}{0.175841,0.441290,0.557685}%
\pgfsetfillcolor{currentfill}%
\pgfsetlinewidth{0.000000pt}%
\definecolor{currentstroke}{rgb}{0.263663,0.237631,0.518762}%
\pgfsetstrokecolor{currentstroke}%
\pgfsetdash{}{0pt}%
\pgfpathmoveto{\pgfqpoint{4.324600in}{4.594132in}}%
\pgfpathlineto{\pgfqpoint{4.380311in}{4.558363in}}%
\pgfpathlineto{\pgfqpoint{4.458695in}{4.459156in}}%
\pgfpathclose%
\pgfusepath{fill}%
\end{pgfscope}%
\begin{pgfscope}%
\pgfpathrectangle{\pgfqpoint{0.539299in}{0.078740in}}{\pgfqpoint{7.842520in}{7.842520in}}%
\pgfusepath{clip}%
\pgfsetbuttcap%
\pgfsetroundjoin%
\definecolor{currentfill}{rgb}{0.273809,0.031497,0.358853}%
\pgfsetfillcolor{currentfill}%
\pgfsetlinewidth{0.000000pt}%
\definecolor{currentstroke}{rgb}{0.262138,0.242286,0.520837}%
\pgfsetstrokecolor{currentstroke}%
\pgfsetdash{}{0pt}%
\pgfpathmoveto{\pgfqpoint{7.264628in}{3.113510in}}%
\pgfpathlineto{\pgfqpoint{7.126919in}{3.173464in}}%
\pgfpathlineto{\pgfqpoint{7.057617in}{3.142554in}}%
\pgfpathclose%
\pgfusepath{fill}%
\end{pgfscope}%
\begin{pgfscope}%
\pgfpathrectangle{\pgfqpoint{0.539299in}{0.078740in}}{\pgfqpoint{7.842520in}{7.842520in}}%
\pgfusepath{clip}%
\pgfsetbuttcap%
\pgfsetroundjoin%
\definecolor{currentfill}{rgb}{0.185556,0.418570,0.556753}%
\pgfsetfillcolor{currentfill}%
\pgfsetlinewidth{0.000000pt}%
\definecolor{currentstroke}{rgb}{0.260571,0.246922,0.522828}%
\pgfsetstrokecolor{currentstroke}%
\pgfsetdash{}{0pt}%
\pgfpathmoveto{\pgfqpoint{4.380311in}{4.558363in}}%
\pgfpathlineto{\pgfqpoint{4.592860in}{4.318703in}}%
\pgfpathlineto{\pgfqpoint{4.458695in}{4.459156in}}%
\pgfpathclose%
\pgfusepath{fill}%
\end{pgfscope}%
\begin{pgfscope}%
\pgfpathrectangle{\pgfqpoint{0.539299in}{0.078740in}}{\pgfqpoint{7.842520in}{7.842520in}}%
\pgfusepath{clip}%
\pgfsetbuttcap%
\pgfsetroundjoin%
\definecolor{currentfill}{rgb}{0.267004,0.004874,0.329415}%
\pgfsetfillcolor{currentfill}%
\pgfsetlinewidth{0.000000pt}%
\definecolor{currentstroke}{rgb}{0.258965,0.251537,0.524736}%
\pgfsetstrokecolor{currentstroke}%
\pgfsetdash{}{0pt}%
\pgfpathmoveto{\pgfqpoint{7.541482in}{3.001589in}}%
\pgfpathlineto{\pgfqpoint{7.610502in}{3.062741in}}%
\pgfpathlineto{\pgfqpoint{7.471813in}{3.109193in}}%
\pgfpathclose%
\pgfusepath{fill}%
\end{pgfscope}%
\begin{pgfscope}%
\pgfpathrectangle{\pgfqpoint{0.539299in}{0.078740in}}{\pgfqpoint{7.842520in}{7.842520in}}%
\pgfusepath{clip}%
\pgfsetbuttcap%
\pgfsetroundjoin%
\definecolor{currentfill}{rgb}{0.269308,0.218818,0.509577}%
\pgfsetfillcolor{currentfill}%
\pgfsetlinewidth{0.000000pt}%
\definecolor{currentstroke}{rgb}{0.257322,0.256130,0.526563}%
\pgfsetstrokecolor{currentstroke}%
\pgfsetdash{}{0pt}%
\pgfpathmoveto{\pgfqpoint{5.265591in}{3.732089in}}%
\pgfpathlineto{\pgfqpoint{5.400994in}{3.652853in}}%
\pgfpathlineto{\pgfqpoint{5.340183in}{3.710444in}}%
\pgfpathclose%
\pgfusepath{fill}%
\end{pgfscope}%
\begin{pgfscope}%
\pgfpathrectangle{\pgfqpoint{0.539299in}{0.078740in}}{\pgfqpoint{7.842520in}{7.842520in}}%
\pgfusepath{clip}%
\pgfsetbuttcap%
\pgfsetroundjoin%
\definecolor{currentfill}{rgb}{0.136408,0.541173,0.554483}%
\pgfsetfillcolor{currentfill}%
\pgfsetlinewidth{0.000000pt}%
\definecolor{currentstroke}{rgb}{0.255645,0.260703,0.528312}%
\pgfsetstrokecolor{currentstroke}%
\pgfsetdash{}{0pt}%
\pgfpathmoveto{\pgfqpoint{3.843276in}{4.946277in}}%
\pgfpathlineto{\pgfqpoint{3.710739in}{4.939290in}}%
\pgfpathlineto{\pgfqpoint{3.629210in}{4.996070in}}%
\pgfpathclose%
\pgfusepath{fill}%
\end{pgfscope}%
\begin{pgfscope}%
\pgfpathrectangle{\pgfqpoint{0.539299in}{0.078740in}}{\pgfqpoint{7.842520in}{7.842520in}}%
\pgfusepath{clip}%
\pgfsetbuttcap%
\pgfsetroundjoin%
\definecolor{currentfill}{rgb}{0.175841,0.441290,0.557685}%
\pgfsetfillcolor{currentfill}%
\pgfsetlinewidth{0.000000pt}%
\definecolor{currentstroke}{rgb}{0.253935,0.265254,0.529983}%
\pgfsetstrokecolor{currentstroke}%
\pgfsetdash{}{0pt}%
\pgfpathmoveto{\pgfqpoint{3.079268in}{4.442627in}}%
\pgfpathlineto{\pgfqpoint{2.996771in}{4.420700in}}%
\pgfpathlineto{\pgfqpoint{3.121655in}{4.747134in}}%
\pgfpathclose%
\pgfusepath{fill}%
\end{pgfscope}%
\begin{pgfscope}%
\pgfpathrectangle{\pgfqpoint{0.539299in}{0.078740in}}{\pgfqpoint{7.842520in}{7.842520in}}%
\pgfusepath{clip}%
\pgfsetbuttcap%
\pgfsetroundjoin%
\definecolor{currentfill}{rgb}{0.204903,0.375746,0.553533}%
\pgfsetfillcolor{currentfill}%
\pgfsetlinewidth{0.000000pt}%
\definecolor{currentstroke}{rgb}{0.252194,0.269783,0.531579}%
\pgfsetstrokecolor{currentstroke}%
\pgfsetdash{}{0pt}%
\pgfpathmoveto{\pgfqpoint{4.592860in}{4.318703in}}%
\pgfpathlineto{\pgfqpoint{4.649993in}{4.264866in}}%
\pgfpathlineto{\pgfqpoint{4.727082in}{4.180259in}}%
\pgfpathclose%
\pgfusepath{fill}%
\end{pgfscope}%
\begin{pgfscope}%
\pgfpathrectangle{\pgfqpoint{0.539299in}{0.078740in}}{\pgfqpoint{7.842520in}{7.842520in}}%
\pgfusepath{clip}%
\pgfsetbuttcap%
\pgfsetroundjoin%
\definecolor{currentfill}{rgb}{0.216210,0.351535,0.550627}%
\pgfsetfillcolor{currentfill}%
\pgfsetlinewidth{0.000000pt}%
\definecolor{currentstroke}{rgb}{0.250425,0.274290,0.533103}%
\pgfsetstrokecolor{currentstroke}%
\pgfsetdash{}{0pt}%
\pgfpathmoveto{\pgfqpoint{4.727082in}{4.180259in}}%
\pgfpathlineto{\pgfqpoint{4.649993in}{4.264866in}}%
\pgfpathlineto{\pgfqpoint{4.861395in}{4.049469in}}%
\pgfpathclose%
\pgfusepath{fill}%
\end{pgfscope}%
\begin{pgfscope}%
\pgfpathrectangle{\pgfqpoint{0.539299in}{0.078740in}}{\pgfqpoint{7.842520in}{7.842520in}}%
\pgfusepath{clip}%
\pgfsetbuttcap%
\pgfsetroundjoin%
\definecolor{currentfill}{rgb}{0.275191,0.194905,0.496005}%
\pgfsetfillcolor{currentfill}%
\pgfsetlinewidth{0.000000pt}%
\definecolor{currentstroke}{rgb}{0.248629,0.278775,0.534556}%
\pgfsetstrokecolor{currentstroke}%
\pgfsetdash{}{0pt}%
\pgfpathmoveto{\pgfqpoint{5.536826in}{3.584474in}}%
\pgfpathlineto{\pgfqpoint{5.610860in}{3.587946in}}%
\pgfpathlineto{\pgfqpoint{5.475289in}{3.644517in}}%
\pgfpathclose%
\pgfusepath{fill}%
\end{pgfscope}%
\begin{pgfscope}%
\pgfpathrectangle{\pgfqpoint{0.539299in}{0.078740in}}{\pgfqpoint{7.842520in}{7.842520in}}%
\pgfusepath{clip}%
\pgfsetbuttcap%
\pgfsetroundjoin%
\definecolor{currentfill}{rgb}{0.279574,0.170599,0.479997}%
\pgfsetfillcolor{currentfill}%
\pgfsetlinewidth{0.000000pt}%
\definecolor{currentstroke}{rgb}{0.246811,0.283237,0.535941}%
\pgfsetstrokecolor{currentstroke}%
\pgfsetdash{}{0pt}%
\pgfpathmoveto{\pgfqpoint{5.820187in}{3.551515in}}%
\pgfpathlineto{\pgfqpoint{5.746902in}{3.537634in}}%
\pgfpathlineto{\pgfqpoint{5.883392in}{3.490293in}}%
\pgfpathclose%
\pgfusepath{fill}%
\end{pgfscope}%
\begin{pgfscope}%
\pgfpathrectangle{\pgfqpoint{0.539299in}{0.078740in}}{\pgfqpoint{7.842520in}{7.842520in}}%
\pgfusepath{clip}%
\pgfsetbuttcap%
\pgfsetroundjoin%
\definecolor{currentfill}{rgb}{0.268510,0.009605,0.335427}%
\pgfsetfillcolor{currentfill}%
\pgfsetlinewidth{0.000000pt}%
\definecolor{currentstroke}{rgb}{0.244972,0.287675,0.537260}%
\pgfsetstrokecolor{currentstroke}%
\pgfsetdash{}{0pt}%
\pgfpathmoveto{\pgfqpoint{7.541482in}{3.001589in}}%
\pgfpathlineto{\pgfqpoint{7.471813in}{3.109193in}}%
\pgfpathlineto{\pgfqpoint{7.402794in}{3.055818in}}%
\pgfpathclose%
\pgfusepath{fill}%
\end{pgfscope}%
\begin{pgfscope}%
\pgfpathrectangle{\pgfqpoint{0.539299in}{0.078740in}}{\pgfqpoint{7.842520in}{7.842520in}}%
\pgfusepath{clip}%
\pgfsetbuttcap%
\pgfsetroundjoin%
\definecolor{currentfill}{rgb}{0.282623,0.140926,0.457517}%
\pgfsetfillcolor{currentfill}%
\pgfsetlinewidth{0.000000pt}%
\definecolor{currentstroke}{rgb}{0.243113,0.292092,0.538516}%
\pgfsetstrokecolor{currentstroke}%
\pgfsetdash{}{0pt}%
\pgfpathmoveto{\pgfqpoint{6.229961in}{3.418141in}}%
\pgfpathlineto{\pgfqpoint{6.301764in}{3.437824in}}%
\pgfpathlineto{\pgfqpoint{6.093026in}{3.466206in}}%
\pgfpathclose%
\pgfusepath{fill}%
\end{pgfscope}%
\begin{pgfscope}%
\pgfpathrectangle{\pgfqpoint{0.539299in}{0.078740in}}{\pgfqpoint{7.842520in}{7.842520in}}%
\pgfusepath{clip}%
\pgfsetbuttcap%
\pgfsetroundjoin%
\definecolor{currentfill}{rgb}{0.149039,0.508051,0.557250}%
\pgfsetfillcolor{currentfill}%
\pgfsetlinewidth{0.000000pt}%
\definecolor{currentstroke}{rgb}{0.241237,0.296485,0.539709}%
\pgfsetstrokecolor{currentstroke}%
\pgfsetdash{}{0pt}%
\pgfpathmoveto{\pgfqpoint{3.976815in}{4.900063in}}%
\pgfpathlineto{\pgfqpoint{4.111002in}{4.812426in}}%
\pgfpathlineto{\pgfqpoint{4.190652in}{4.714468in}}%
\pgfpathclose%
\pgfusepath{fill}%
\end{pgfscope}%
\begin{pgfscope}%
\pgfpathrectangle{\pgfqpoint{0.539299in}{0.078740in}}{\pgfqpoint{7.842520in}{7.842520in}}%
\pgfusepath{clip}%
\pgfsetbuttcap%
\pgfsetroundjoin%
\definecolor{currentfill}{rgb}{0.283072,0.130895,0.449241}%
\pgfsetfillcolor{currentfill}%
\pgfsetlinewidth{0.000000pt}%
\definecolor{currentstroke}{rgb}{0.239346,0.300855,0.540844}%
\pgfsetstrokecolor{currentstroke}%
\pgfsetdash{}{0pt}%
\pgfpathmoveto{\pgfqpoint{6.438604in}{3.384604in}}%
\pgfpathlineto{\pgfqpoint{6.301764in}{3.437824in}}%
\pgfpathlineto{\pgfqpoint{6.367154in}{3.364036in}}%
\pgfpathclose%
\pgfusepath{fill}%
\end{pgfscope}%
\begin{pgfscope}%
\pgfpathrectangle{\pgfqpoint{0.539299in}{0.078740in}}{\pgfqpoint{7.842520in}{7.842520in}}%
\pgfusepath{clip}%
\pgfsetbuttcap%
\pgfsetroundjoin%
\definecolor{currentfill}{rgb}{0.235526,0.309527,0.542944}%
\pgfsetfillcolor{currentfill}%
\pgfsetlinewidth{0.000000pt}%
\definecolor{currentstroke}{rgb}{0.237441,0.305202,0.541921}%
\pgfsetstrokecolor{currentstroke}%
\pgfsetdash{}{0pt}%
\pgfpathmoveto{\pgfqpoint{4.861395in}{4.049469in}}%
\pgfpathlineto{\pgfqpoint{4.919892in}{3.989315in}}%
\pgfpathlineto{\pgfqpoint{4.995865in}{3.930113in}}%
\pgfpathclose%
\pgfusepath{fill}%
\end{pgfscope}%
\begin{pgfscope}%
\pgfpathrectangle{\pgfqpoint{0.539299in}{0.078740in}}{\pgfqpoint{7.842520in}{7.842520in}}%
\pgfusepath{clip}%
\pgfsetbuttcap%
\pgfsetroundjoin%
\definecolor{currentfill}{rgb}{0.204903,0.375746,0.553533}%
\pgfsetfillcolor{currentfill}%
\pgfsetlinewidth{0.000000pt}%
\definecolor{currentstroke}{rgb}{0.235526,0.309527,0.542944}%
\pgfsetstrokecolor{currentstroke}%
\pgfsetdash{}{0pt}%
\pgfpathmoveto{\pgfqpoint{2.830306in}{4.368134in}}%
\pgfpathlineto{\pgfqpoint{2.913780in}{4.395967in}}%
\pgfpathlineto{\pgfqpoint{2.793452in}{3.956696in}}%
\pgfpathclose%
\pgfusepath{fill}%
\end{pgfscope}%
\begin{pgfscope}%
\pgfpathrectangle{\pgfqpoint{0.539299in}{0.078740in}}{\pgfqpoint{7.842520in}{7.842520in}}%
\pgfusepath{clip}%
\pgfsetbuttcap%
\pgfsetroundjoin%
\definecolor{currentfill}{rgb}{0.267004,0.004874,0.329415}%
\pgfsetfillcolor{currentfill}%
\pgfsetlinewidth{0.000000pt}%
\definecolor{currentstroke}{rgb}{0.233603,0.313828,0.543914}%
\pgfsetstrokecolor{currentstroke}%
\pgfsetdash{}{0pt}%
\pgfpathmoveto{\pgfqpoint{7.749729in}{3.018900in}}%
\pgfpathlineto{\pgfqpoint{7.610502in}{3.062741in}}%
\pgfpathlineto{\pgfqpoint{7.541482in}{3.001589in}}%
\pgfpathclose%
\pgfusepath{fill}%
\end{pgfscope}%
\begin{pgfscope}%
\pgfpathrectangle{\pgfqpoint{0.539299in}{0.078740in}}{\pgfqpoint{7.842520in}{7.842520in}}%
\pgfusepath{clip}%
\pgfsetbuttcap%
\pgfsetroundjoin%
\definecolor{currentfill}{rgb}{0.282656,0.100196,0.422160}%
\pgfsetfillcolor{currentfill}%
\pgfsetlinewidth{0.000000pt}%
\definecolor{currentstroke}{rgb}{0.231674,0.318106,0.544834}%
\pgfsetstrokecolor{currentstroke}%
\pgfsetdash{}{0pt}%
\pgfpathmoveto{\pgfqpoint{6.712873in}{3.258583in}}%
\pgfpathlineto{\pgfqpoint{6.783027in}{3.277708in}}%
\pgfpathlineto{\pgfqpoint{6.575647in}{3.324679in}}%
\pgfpathclose%
\pgfusepath{fill}%
\end{pgfscope}%
\begin{pgfscope}%
\pgfpathrectangle{\pgfqpoint{0.539299in}{0.078740in}}{\pgfqpoint{7.842520in}{7.842520in}}%
\pgfusepath{clip}%
\pgfsetbuttcap%
\pgfsetroundjoin%
\definecolor{currentfill}{rgb}{0.162142,0.474838,0.558140}%
\pgfsetfillcolor{currentfill}%
\pgfsetlinewidth{0.000000pt}%
\definecolor{currentstroke}{rgb}{0.229739,0.322361,0.545706}%
\pgfsetstrokecolor{currentstroke}%
\pgfsetdash{}{0pt}%
\pgfpathmoveto{\pgfqpoint{3.121655in}{4.747134in}}%
\pgfpathlineto{\pgfqpoint{3.204675in}{4.746147in}}%
\pgfpathlineto{\pgfqpoint{3.079268in}{4.442627in}}%
\pgfpathclose%
\pgfusepath{fill}%
\end{pgfscope}%
\begin{pgfscope}%
\pgfpathrectangle{\pgfqpoint{0.539299in}{0.078740in}}{\pgfqpoint{7.842520in}{7.842520in}}%
\pgfusepath{clip}%
\pgfsetbuttcap%
\pgfsetroundjoin%
\definecolor{currentfill}{rgb}{0.157729,0.485932,0.558013}%
\pgfsetfillcolor{currentfill}%
\pgfsetlinewidth{0.000000pt}%
\definecolor{currentstroke}{rgb}{0.227802,0.326594,0.546532}%
\pgfsetstrokecolor{currentstroke}%
\pgfsetdash{}{0pt}%
\pgfpathmoveto{\pgfqpoint{4.324600in}{4.594132in}}%
\pgfpathlineto{\pgfqpoint{4.111002in}{4.812426in}}%
\pgfpathlineto{\pgfqpoint{4.245563in}{4.694927in}}%
\pgfpathclose%
\pgfusepath{fill}%
\end{pgfscope}%
\begin{pgfscope}%
\pgfpathrectangle{\pgfqpoint{0.539299in}{0.078740in}}{\pgfqpoint{7.842520in}{7.842520in}}%
\pgfusepath{clip}%
\pgfsetbuttcap%
\pgfsetroundjoin%
\definecolor{currentfill}{rgb}{0.280894,0.078907,0.402329}%
\pgfsetfillcolor{currentfill}%
\pgfsetlinewidth{0.000000pt}%
\definecolor{currentstroke}{rgb}{0.225863,0.330805,0.547314}%
\pgfsetstrokecolor{currentstroke}%
\pgfsetdash{}{0pt}%
\pgfpathmoveto{\pgfqpoint{6.783027in}{3.277708in}}%
\pgfpathlineto{\pgfqpoint{6.850286in}{3.187517in}}%
\pgfpathlineto{\pgfqpoint{6.920184in}{3.211063in}}%
\pgfpathclose%
\pgfusepath{fill}%
\end{pgfscope}%
\begin{pgfscope}%
\pgfpathrectangle{\pgfqpoint{0.539299in}{0.078740in}}{\pgfqpoint{7.842520in}{7.842520in}}%
\pgfusepath{clip}%
\pgfsetbuttcap%
\pgfsetroundjoin%
\definecolor{currentfill}{rgb}{0.273006,0.204520,0.501721}%
\pgfsetfillcolor{currentfill}%
\pgfsetlinewidth{0.000000pt}%
\definecolor{currentstroke}{rgb}{0.223925,0.334994,0.548053}%
\pgfsetstrokecolor{currentstroke}%
\pgfsetdash{}{0pt}%
\pgfpathmoveto{\pgfqpoint{5.475289in}{3.644517in}}%
\pgfpathlineto{\pgfqpoint{5.400994in}{3.652853in}}%
\pgfpathlineto{\pgfqpoint{5.536826in}{3.584474in}}%
\pgfpathclose%
\pgfusepath{fill}%
\end{pgfscope}%
\begin{pgfscope}%
\pgfpathrectangle{\pgfqpoint{0.539299in}{0.078740in}}{\pgfqpoint{7.842520in}{7.842520in}}%
\pgfusepath{clip}%
\pgfsetbuttcap%
\pgfsetroundjoin%
\definecolor{currentfill}{rgb}{0.243113,0.292092,0.538516}%
\pgfsetfillcolor{currentfill}%
\pgfsetlinewidth{0.000000pt}%
\definecolor{currentstroke}{rgb}{0.221989,0.339161,0.548752}%
\pgfsetstrokecolor{currentstroke}%
\pgfsetdash{}{0pt}%
\pgfpathmoveto{\pgfqpoint{4.995865in}{3.930113in}}%
\pgfpathlineto{\pgfqpoint{4.919892in}{3.989315in}}%
\pgfpathlineto{\pgfqpoint{5.130570in}{3.824189in}}%
\pgfpathclose%
\pgfusepath{fill}%
\end{pgfscope}%
\begin{pgfscope}%
\pgfpathrectangle{\pgfqpoint{0.539299in}{0.078740in}}{\pgfqpoint{7.842520in}{7.842520in}}%
\pgfusepath{clip}%
\pgfsetbuttcap%
\pgfsetroundjoin%
\definecolor{currentfill}{rgb}{0.280868,0.160771,0.472899}%
\pgfsetfillcolor{currentfill}%
\pgfsetlinewidth{0.000000pt}%
\definecolor{currentstroke}{rgb}{0.220057,0.343307,0.549413}%
\pgfsetstrokecolor{currentstroke}%
\pgfsetdash{}{0pt}%
\pgfpathmoveto{\pgfqpoint{6.020289in}{3.442779in}}%
\pgfpathlineto{\pgfqpoint{6.093026in}{3.466206in}}%
\pgfpathlineto{\pgfqpoint{5.956414in}{3.509868in}}%
\pgfpathclose%
\pgfusepath{fill}%
\end{pgfscope}%
\begin{pgfscope}%
\pgfpathrectangle{\pgfqpoint{0.539299in}{0.078740in}}{\pgfqpoint{7.842520in}{7.842520in}}%
\pgfusepath{clip}%
\pgfsetbuttcap%
\pgfsetroundjoin%
\definecolor{currentfill}{rgb}{0.276194,0.190074,0.493001}%
\pgfsetfillcolor{currentfill}%
\pgfsetlinewidth{0.000000pt}%
\definecolor{currentstroke}{rgb}{0.218130,0.347432,0.550038}%
\pgfsetstrokecolor{currentstroke}%
\pgfsetdash{}{0pt}%
\pgfpathmoveto{\pgfqpoint{5.536826in}{3.584474in}}%
\pgfpathlineto{\pgfqpoint{5.746902in}{3.537634in}}%
\pgfpathlineto{\pgfqpoint{5.610860in}{3.587946in}}%
\pgfpathclose%
\pgfusepath{fill}%
\end{pgfscope}%
\begin{pgfscope}%
\pgfpathrectangle{\pgfqpoint{0.539299in}{0.078740in}}{\pgfqpoint{7.842520in}{7.842520in}}%
\pgfusepath{clip}%
\pgfsetbuttcap%
\pgfsetroundjoin%
\definecolor{currentfill}{rgb}{0.274952,0.037752,0.364543}%
\pgfsetfillcolor{currentfill}%
\pgfsetlinewidth{0.000000pt}%
\definecolor{currentstroke}{rgb}{0.216210,0.351535,0.550627}%
\pgfsetstrokecolor{currentstroke}%
\pgfsetdash{}{0pt}%
\pgfpathmoveto{\pgfqpoint{7.195388in}{3.074084in}}%
\pgfpathlineto{\pgfqpoint{7.264628in}{3.113510in}}%
\pgfpathlineto{\pgfqpoint{7.057617in}{3.142554in}}%
\pgfpathclose%
\pgfusepath{fill}%
\end{pgfscope}%
\begin{pgfscope}%
\pgfpathrectangle{\pgfqpoint{0.539299in}{0.078740in}}{\pgfqpoint{7.842520in}{7.842520in}}%
\pgfusepath{clip}%
\pgfsetbuttcap%
\pgfsetroundjoin%
\definecolor{currentfill}{rgb}{0.165117,0.467423,0.558141}%
\pgfsetfillcolor{currentfill}%
\pgfsetlinewidth{0.000000pt}%
\definecolor{currentstroke}{rgb}{0.214298,0.355619,0.551184}%
\pgfsetstrokecolor{currentstroke}%
\pgfsetdash{}{0pt}%
\pgfpathmoveto{\pgfqpoint{4.245563in}{4.694927in}}%
\pgfpathlineto{\pgfqpoint{4.380311in}{4.558363in}}%
\pgfpathlineto{\pgfqpoint{4.324600in}{4.594132in}}%
\pgfpathclose%
\pgfusepath{fill}%
\end{pgfscope}%
\begin{pgfscope}%
\pgfpathrectangle{\pgfqpoint{0.539299in}{0.078740in}}{\pgfqpoint{7.842520in}{7.842520in}}%
\pgfusepath{clip}%
\pgfsetbuttcap%
\pgfsetroundjoin%
\definecolor{currentfill}{rgb}{0.135066,0.544853,0.554029}%
\pgfsetfillcolor{currentfill}%
\pgfsetlinewidth{0.000000pt}%
\definecolor{currentstroke}{rgb}{0.212395,0.359683,0.551710}%
\pgfsetstrokecolor{currentstroke}%
\pgfsetdash{}{0pt}%
\pgfpathmoveto{\pgfqpoint{3.629210in}{4.996070in}}%
\pgfpathlineto{\pgfqpoint{3.498080in}{4.903513in}}%
\pgfpathlineto{\pgfqpoint{3.415921in}{4.934934in}}%
\pgfpathclose%
\pgfusepath{fill}%
\end{pgfscope}%
\begin{pgfscope}%
\pgfpathrectangle{\pgfqpoint{0.539299in}{0.078740in}}{\pgfqpoint{7.842520in}{7.842520in}}%
\pgfusepath{clip}%
\pgfsetbuttcap%
\pgfsetroundjoin%
\definecolor{currentfill}{rgb}{0.282623,0.140926,0.457517}%
\pgfsetfillcolor{currentfill}%
\pgfsetlinewidth{0.000000pt}%
\definecolor{currentstroke}{rgb}{0.210503,0.363727,0.552206}%
\pgfsetstrokecolor{currentstroke}%
\pgfsetdash{}{0pt}%
\pgfpathmoveto{\pgfqpoint{6.367154in}{3.364036in}}%
\pgfpathlineto{\pgfqpoint{6.301764in}{3.437824in}}%
\pgfpathlineto{\pgfqpoint{6.229961in}{3.418141in}}%
\pgfpathclose%
\pgfusepath{fill}%
\end{pgfscope}%
\begin{pgfscope}%
\pgfpathrectangle{\pgfqpoint{0.539299in}{0.078740in}}{\pgfqpoint{7.842520in}{7.842520in}}%
\pgfusepath{clip}%
\pgfsetbuttcap%
\pgfsetroundjoin%
\definecolor{currentfill}{rgb}{0.271305,0.019942,0.347269}%
\pgfsetfillcolor{currentfill}%
\pgfsetlinewidth{0.000000pt}%
\definecolor{currentstroke}{rgb}{0.208623,0.367752,0.552675}%
\pgfsetstrokecolor{currentstroke}%
\pgfsetdash{}{0pt}%
\pgfpathmoveto{\pgfqpoint{7.264628in}{3.113510in}}%
\pgfpathlineto{\pgfqpoint{7.333570in}{3.007542in}}%
\pgfpathlineto{\pgfqpoint{7.402794in}{3.055818in}}%
\pgfpathclose%
\pgfusepath{fill}%
\end{pgfscope}%
\begin{pgfscope}%
\pgfpathrectangle{\pgfqpoint{0.539299in}{0.078740in}}{\pgfqpoint{7.842520in}{7.842520in}}%
\pgfusepath{clip}%
\pgfsetbuttcap%
\pgfsetroundjoin%
\definecolor{currentfill}{rgb}{0.182256,0.426184,0.557120}%
\pgfsetfillcolor{currentfill}%
\pgfsetlinewidth{0.000000pt}%
\definecolor{currentstroke}{rgb}{0.206756,0.371758,0.553117}%
\pgfsetstrokecolor{currentstroke}%
\pgfsetdash{}{0pt}%
\pgfpathmoveto{\pgfqpoint{4.515135in}{4.412321in}}%
\pgfpathlineto{\pgfqpoint{4.592860in}{4.318703in}}%
\pgfpathlineto{\pgfqpoint{4.380311in}{4.558363in}}%
\pgfpathclose%
\pgfusepath{fill}%
\end{pgfscope}%
\begin{pgfscope}%
\pgfpathrectangle{\pgfqpoint{0.539299in}{0.078740in}}{\pgfqpoint{7.842520in}{7.842520in}}%
\pgfusepath{clip}%
\pgfsetbuttcap%
\pgfsetroundjoin%
\definecolor{currentfill}{rgb}{0.192357,0.403199,0.555836}%
\pgfsetfillcolor{currentfill}%
\pgfsetlinewidth{0.000000pt}%
\definecolor{currentstroke}{rgb}{0.204903,0.375746,0.553533}%
\pgfsetstrokecolor{currentstroke}%
\pgfsetdash{}{0pt}%
\pgfpathmoveto{\pgfqpoint{4.515135in}{4.412321in}}%
\pgfpathlineto{\pgfqpoint{4.649993in}{4.264866in}}%
\pgfpathlineto{\pgfqpoint{4.592860in}{4.318703in}}%
\pgfpathclose%
\pgfusepath{fill}%
\end{pgfscope}%
\begin{pgfscope}%
\pgfpathrectangle{\pgfqpoint{0.539299in}{0.078740in}}{\pgfqpoint{7.842520in}{7.842520in}}%
\pgfusepath{clip}%
\pgfsetbuttcap%
\pgfsetroundjoin%
\definecolor{currentfill}{rgb}{0.281924,0.089666,0.412415}%
\pgfsetfillcolor{currentfill}%
\pgfsetlinewidth{0.000000pt}%
\definecolor{currentstroke}{rgb}{0.203063,0.379716,0.553925}%
\pgfsetstrokecolor{currentstroke}%
\pgfsetdash{}{0pt}%
\pgfpathmoveto{\pgfqpoint{6.712873in}{3.258583in}}%
\pgfpathlineto{\pgfqpoint{6.850286in}{3.187517in}}%
\pgfpathlineto{\pgfqpoint{6.783027in}{3.277708in}}%
\pgfpathclose%
\pgfusepath{fill}%
\end{pgfscope}%
\begin{pgfscope}%
\pgfpathrectangle{\pgfqpoint{0.539299in}{0.078740in}}{\pgfqpoint{7.842520in}{7.842520in}}%
\pgfusepath{clip}%
\pgfsetbuttcap%
\pgfsetroundjoin%
\definecolor{currentfill}{rgb}{0.280255,0.165693,0.476498}%
\pgfsetfillcolor{currentfill}%
\pgfsetlinewidth{0.000000pt}%
\definecolor{currentstroke}{rgb}{0.201239,0.383670,0.554294}%
\pgfsetstrokecolor{currentstroke}%
\pgfsetdash{}{0pt}%
\pgfpathmoveto{\pgfqpoint{5.956414in}{3.509868in}}%
\pgfpathlineto{\pgfqpoint{5.883392in}{3.490293in}}%
\pgfpathlineto{\pgfqpoint{6.020289in}{3.442779in}}%
\pgfpathclose%
\pgfusepath{fill}%
\end{pgfscope}%
\begin{pgfscope}%
\pgfpathrectangle{\pgfqpoint{0.539299in}{0.078740in}}{\pgfqpoint{7.842520in}{7.842520in}}%
\pgfusepath{clip}%
\pgfsetbuttcap%
\pgfsetroundjoin%
\definecolor{currentfill}{rgb}{0.279566,0.067836,0.391917}%
\pgfsetfillcolor{currentfill}%
\pgfsetlinewidth{0.000000pt}%
\definecolor{currentstroke}{rgb}{0.199430,0.387607,0.554642}%
\pgfsetstrokecolor{currentstroke}%
\pgfsetdash{}{0pt}%
\pgfpathmoveto{\pgfqpoint{6.920184in}{3.211063in}}%
\pgfpathlineto{\pgfqpoint{6.850286in}{3.187517in}}%
\pgfpathlineto{\pgfqpoint{7.057617in}{3.142554in}}%
\pgfpathclose%
\pgfusepath{fill}%
\end{pgfscope}%
\begin{pgfscope}%
\pgfpathrectangle{\pgfqpoint{0.539299in}{0.078740in}}{\pgfqpoint{7.842520in}{7.842520in}}%
\pgfusepath{clip}%
\pgfsetbuttcap%
\pgfsetroundjoin%
\definecolor{currentfill}{rgb}{0.214298,0.355619,0.551184}%
\pgfsetfillcolor{currentfill}%
\pgfsetlinewidth{0.000000pt}%
\definecolor{currentstroke}{rgb}{0.197636,0.391528,0.554969}%
\pgfsetstrokecolor{currentstroke}%
\pgfsetdash{}{0pt}%
\pgfpathmoveto{\pgfqpoint{4.861395in}{4.049469in}}%
\pgfpathlineto{\pgfqpoint{4.649993in}{4.264866in}}%
\pgfpathlineto{\pgfqpoint{4.784895in}{4.122344in}}%
\pgfpathclose%
\pgfusepath{fill}%
\end{pgfscope}%
\begin{pgfscope}%
\pgfpathrectangle{\pgfqpoint{0.539299in}{0.078740in}}{\pgfqpoint{7.842520in}{7.842520in}}%
\pgfusepath{clip}%
\pgfsetbuttcap%
\pgfsetroundjoin%
\definecolor{currentfill}{rgb}{0.257322,0.256130,0.526563}%
\pgfsetfillcolor{currentfill}%
\pgfsetlinewidth{0.000000pt}%
\definecolor{currentstroke}{rgb}{0.195860,0.395433,0.555276}%
\pgfsetstrokecolor{currentstroke}%
\pgfsetdash{}{0pt}%
\pgfpathmoveto{\pgfqpoint{5.190465in}{3.761377in}}%
\pgfpathlineto{\pgfqpoint{5.265591in}{3.732089in}}%
\pgfpathlineto{\pgfqpoint{5.130570in}{3.824189in}}%
\pgfpathclose%
\pgfusepath{fill}%
\end{pgfscope}%
\begin{pgfscope}%
\pgfpathrectangle{\pgfqpoint{0.539299in}{0.078740in}}{\pgfqpoint{7.842520in}{7.842520in}}%
\pgfusepath{clip}%
\pgfsetbuttcap%
\pgfsetroundjoin%
\definecolor{currentfill}{rgb}{0.132444,0.552216,0.553018}%
\pgfsetfillcolor{currentfill}%
\pgfsetlinewidth{0.000000pt}%
\definecolor{currentstroke}{rgb}{0.194100,0.399323,0.555565}%
\pgfsetstrokecolor{currentstroke}%
\pgfsetdash{}{0pt}%
\pgfpathmoveto{\pgfqpoint{3.976815in}{4.900063in}}%
\pgfpathlineto{\pgfqpoint{3.843276in}{4.946277in}}%
\pgfpathlineto{\pgfqpoint{3.761962in}{5.020676in}}%
\pgfpathclose%
\pgfusepath{fill}%
\end{pgfscope}%
\begin{pgfscope}%
\pgfpathrectangle{\pgfqpoint{0.539299in}{0.078740in}}{\pgfqpoint{7.842520in}{7.842520in}}%
\pgfusepath{clip}%
\pgfsetbuttcap%
\pgfsetroundjoin%
\definecolor{currentfill}{rgb}{0.283229,0.120777,0.440584}%
\pgfsetfillcolor{currentfill}%
\pgfsetlinewidth{0.000000pt}%
\definecolor{currentstroke}{rgb}{0.192357,0.403199,0.555836}%
\pgfsetstrokecolor{currentstroke}%
\pgfsetdash{}{0pt}%
\pgfpathmoveto{\pgfqpoint{6.504552in}{3.303089in}}%
\pgfpathlineto{\pgfqpoint{6.575647in}{3.324679in}}%
\pgfpathlineto{\pgfqpoint{6.438604in}{3.384604in}}%
\pgfpathclose%
\pgfusepath{fill}%
\end{pgfscope}%
\begin{pgfscope}%
\pgfpathrectangle{\pgfqpoint{0.539299in}{0.078740in}}{\pgfqpoint{7.842520in}{7.842520in}}%
\pgfusepath{clip}%
\pgfsetbuttcap%
\pgfsetroundjoin%
\definecolor{currentfill}{rgb}{0.272594,0.025563,0.353093}%
\pgfsetfillcolor{currentfill}%
\pgfsetlinewidth{0.000000pt}%
\definecolor{currentstroke}{rgb}{0.190631,0.407061,0.556089}%
\pgfsetstrokecolor{currentstroke}%
\pgfsetdash{}{0pt}%
\pgfpathmoveto{\pgfqpoint{7.195388in}{3.074084in}}%
\pgfpathlineto{\pgfqpoint{7.333570in}{3.007542in}}%
\pgfpathlineto{\pgfqpoint{7.264628in}{3.113510in}}%
\pgfpathclose%
\pgfusepath{fill}%
\end{pgfscope}%
\begin{pgfscope}%
\pgfpathrectangle{\pgfqpoint{0.539299in}{0.078740in}}{\pgfqpoint{7.842520in}{7.842520in}}%
\pgfusepath{clip}%
\pgfsetbuttcap%
\pgfsetroundjoin%
\definecolor{currentfill}{rgb}{0.223925,0.334994,0.548053}%
\pgfsetfillcolor{currentfill}%
\pgfsetlinewidth{0.000000pt}%
\definecolor{currentstroke}{rgb}{0.188923,0.410910,0.556326}%
\pgfsetstrokecolor{currentstroke}%
\pgfsetdash{}{0pt}%
\pgfpathmoveto{\pgfqpoint{4.784895in}{4.122344in}}%
\pgfpathlineto{\pgfqpoint{4.919892in}{3.989315in}}%
\pgfpathlineto{\pgfqpoint{4.861395in}{4.049469in}}%
\pgfpathclose%
\pgfusepath{fill}%
\end{pgfscope}%
\begin{pgfscope}%
\pgfpathrectangle{\pgfqpoint{0.539299in}{0.078740in}}{\pgfqpoint{7.842520in}{7.842520in}}%
\pgfusepath{clip}%
\pgfsetbuttcap%
\pgfsetroundjoin%
\definecolor{currentfill}{rgb}{0.225863,0.330805,0.547314}%
\pgfsetfillcolor{currentfill}%
\pgfsetlinewidth{0.000000pt}%
\definecolor{currentstroke}{rgb}{0.187231,0.414746,0.556547}%
\pgfsetstrokecolor{currentstroke}%
\pgfsetdash{}{0pt}%
\pgfpathmoveto{\pgfqpoint{2.710747in}{3.908413in}}%
\pgfpathlineto{\pgfqpoint{2.627589in}{3.858151in}}%
\pgfpathlineto{\pgfqpoint{2.746363in}{4.336760in}}%
\pgfpathclose%
\pgfusepath{fill}%
\end{pgfscope}%
\begin{pgfscope}%
\pgfpathrectangle{\pgfqpoint{0.539299in}{0.078740in}}{\pgfqpoint{7.842520in}{7.842520in}}%
\pgfusepath{clip}%
\pgfsetbuttcap%
\pgfsetroundjoin%
\definecolor{currentfill}{rgb}{0.129933,0.559582,0.551864}%
\pgfsetfillcolor{currentfill}%
\pgfsetlinewidth{0.000000pt}%
\definecolor{currentstroke}{rgb}{0.185556,0.418570,0.556753}%
\pgfsetstrokecolor{currentstroke}%
\pgfsetdash{}{0pt}%
\pgfpathmoveto{\pgfqpoint{3.629210in}{4.996070in}}%
\pgfpathlineto{\pgfqpoint{3.761962in}{5.020676in}}%
\pgfpathlineto{\pgfqpoint{3.843276in}{4.946277in}}%
\pgfpathclose%
\pgfusepath{fill}%
\end{pgfscope}%
\begin{pgfscope}%
\pgfpathrectangle{\pgfqpoint{0.539299in}{0.078740in}}{\pgfqpoint{7.842520in}{7.842520in}}%
\pgfusepath{clip}%
\pgfsetbuttcap%
\pgfsetroundjoin%
\definecolor{currentfill}{rgb}{0.143343,0.522773,0.556295}%
\pgfsetfillcolor{currentfill}%
\pgfsetlinewidth{0.000000pt}%
\definecolor{currentstroke}{rgb}{0.183898,0.422383,0.556944}%
\pgfsetstrokecolor{currentstroke}%
\pgfsetdash{}{0pt}%
\pgfpathmoveto{\pgfqpoint{3.287152in}{4.741607in}}%
\pgfpathlineto{\pgfqpoint{3.204675in}{4.746147in}}%
\pgfpathlineto{\pgfqpoint{3.333165in}{4.962328in}}%
\pgfpathclose%
\pgfusepath{fill}%
\end{pgfscope}%
\begin{pgfscope}%
\pgfpathrectangle{\pgfqpoint{0.539299in}{0.078740in}}{\pgfqpoint{7.842520in}{7.842520in}}%
\pgfusepath{clip}%
\pgfsetbuttcap%
\pgfsetroundjoin%
\definecolor{currentfill}{rgb}{0.267004,0.004874,0.329415}%
\pgfsetfillcolor{currentfill}%
\pgfsetlinewidth{0.000000pt}%
\definecolor{currentstroke}{rgb}{0.182256,0.426184,0.557120}%
\pgfsetstrokecolor{currentstroke}%
\pgfsetdash{}{0pt}%
\pgfpathmoveto{\pgfqpoint{7.541482in}{3.001589in}}%
\pgfpathlineto{\pgfqpoint{7.680757in}{2.951802in}}%
\pgfpathlineto{\pgfqpoint{7.749729in}{3.018900in}}%
\pgfpathclose%
\pgfusepath{fill}%
\end{pgfscope}%
\begin{pgfscope}%
\pgfpathrectangle{\pgfqpoint{0.539299in}{0.078740in}}{\pgfqpoint{7.842520in}{7.842520in}}%
\pgfusepath{clip}%
\pgfsetbuttcap%
\pgfsetroundjoin%
\definecolor{currentfill}{rgb}{0.262138,0.242286,0.520837}%
\pgfsetfillcolor{currentfill}%
\pgfsetlinewidth{0.000000pt}%
\definecolor{currentstroke}{rgb}{0.180629,0.429975,0.557282}%
\pgfsetstrokecolor{currentstroke}%
\pgfsetdash{}{0pt}%
\pgfpathmoveto{\pgfqpoint{5.400994in}{3.652853in}}%
\pgfpathlineto{\pgfqpoint{5.265591in}{3.732089in}}%
\pgfpathlineto{\pgfqpoint{5.190465in}{3.761377in}}%
\pgfpathclose%
\pgfusepath{fill}%
\end{pgfscope}%
\begin{pgfscope}%
\pgfpathrectangle{\pgfqpoint{0.539299in}{0.078740in}}{\pgfqpoint{7.842520in}{7.842520in}}%
\pgfusepath{clip}%
\pgfsetbuttcap%
\pgfsetroundjoin%
\definecolor{currentfill}{rgb}{0.281412,0.155834,0.469201}%
\pgfsetfillcolor{currentfill}%
\pgfsetlinewidth{0.000000pt}%
\definecolor{currentstroke}{rgb}{0.179019,0.433756,0.557430}%
\pgfsetstrokecolor{currentstroke}%
\pgfsetdash{}{0pt}%
\pgfpathmoveto{\pgfqpoint{6.229961in}{3.418141in}}%
\pgfpathlineto{\pgfqpoint{6.093026in}{3.466206in}}%
\pgfpathlineto{\pgfqpoint{6.157532in}{3.392366in}}%
\pgfpathclose%
\pgfusepath{fill}%
\end{pgfscope}%
\begin{pgfscope}%
\pgfpathrectangle{\pgfqpoint{0.539299in}{0.078740in}}{\pgfqpoint{7.842520in}{7.842520in}}%
\pgfusepath{clip}%
\pgfsetbuttcap%
\pgfsetroundjoin%
\definecolor{currentfill}{rgb}{0.137770,0.537492,0.554906}%
\pgfsetfillcolor{currentfill}%
\pgfsetlinewidth{0.000000pt}%
\definecolor{currentstroke}{rgb}{0.177423,0.437527,0.557565}%
\pgfsetstrokecolor{currentstroke}%
\pgfsetdash{}{0pt}%
\pgfpathmoveto{\pgfqpoint{3.333165in}{4.962328in}}%
\pgfpathlineto{\pgfqpoint{3.415921in}{4.934934in}}%
\pgfpathlineto{\pgfqpoint{3.287152in}{4.741607in}}%
\pgfpathclose%
\pgfusepath{fill}%
\end{pgfscope}%
\begin{pgfscope}%
\pgfpathrectangle{\pgfqpoint{0.539299in}{0.078740in}}{\pgfqpoint{7.842520in}{7.842520in}}%
\pgfusepath{clip}%
\pgfsetbuttcap%
\pgfsetroundjoin%
\definecolor{currentfill}{rgb}{0.283072,0.130895,0.449241}%
\pgfsetfillcolor{currentfill}%
\pgfsetlinewidth{0.000000pt}%
\definecolor{currentstroke}{rgb}{0.175841,0.441290,0.557685}%
\pgfsetstrokecolor{currentstroke}%
\pgfsetdash{}{0pt}%
\pgfpathmoveto{\pgfqpoint{6.438604in}{3.384604in}}%
\pgfpathlineto{\pgfqpoint{6.367154in}{3.364036in}}%
\pgfpathlineto{\pgfqpoint{6.504552in}{3.303089in}}%
\pgfpathclose%
\pgfusepath{fill}%
\end{pgfscope}%
\begin{pgfscope}%
\pgfpathrectangle{\pgfqpoint{0.539299in}{0.078740in}}{\pgfqpoint{7.842520in}{7.842520in}}%
\pgfusepath{clip}%
\pgfsetbuttcap%
\pgfsetroundjoin%
\definecolor{currentfill}{rgb}{0.241237,0.296485,0.539709}%
\pgfsetfillcolor{currentfill}%
\pgfsetlinewidth{0.000000pt}%
\definecolor{currentstroke}{rgb}{0.174274,0.445044,0.557792}%
\pgfsetstrokecolor{currentstroke}%
\pgfsetdash{}{0pt}%
\pgfpathmoveto{\pgfqpoint{5.130570in}{3.824189in}}%
\pgfpathlineto{\pgfqpoint{4.919892in}{3.989315in}}%
\pgfpathlineto{\pgfqpoint{5.055056in}{3.868593in}}%
\pgfpathclose%
\pgfusepath{fill}%
\end{pgfscope}%
\begin{pgfscope}%
\pgfpathrectangle{\pgfqpoint{0.539299in}{0.078740in}}{\pgfqpoint{7.842520in}{7.842520in}}%
\pgfusepath{clip}%
\pgfsetbuttcap%
\pgfsetroundjoin%
\definecolor{currentfill}{rgb}{0.268510,0.009605,0.335427}%
\pgfsetfillcolor{currentfill}%
\pgfsetlinewidth{0.000000pt}%
\definecolor{currentstroke}{rgb}{0.172719,0.448791,0.557885}%
\pgfsetstrokecolor{currentstroke}%
\pgfsetdash{}{0pt}%
\pgfpathmoveto{\pgfqpoint{7.472249in}{2.944684in}}%
\pgfpathlineto{\pgfqpoint{7.541482in}{3.001589in}}%
\pgfpathlineto{\pgfqpoint{7.402794in}{3.055818in}}%
\pgfpathclose%
\pgfusepath{fill}%
\end{pgfscope}%
\begin{pgfscope}%
\pgfpathrectangle{\pgfqpoint{0.539299in}{0.078740in}}{\pgfqpoint{7.842520in}{7.842520in}}%
\pgfusepath{clip}%
\pgfsetbuttcap%
\pgfsetroundjoin%
\definecolor{currentfill}{rgb}{0.283197,0.115680,0.436115}%
\pgfsetfillcolor{currentfill}%
\pgfsetlinewidth{0.000000pt}%
\definecolor{currentstroke}{rgb}{0.171176,0.452530,0.557965}%
\pgfsetstrokecolor{currentstroke}%
\pgfsetdash{}{0pt}%
\pgfpathmoveto{\pgfqpoint{6.712873in}{3.258583in}}%
\pgfpathlineto{\pgfqpoint{6.575647in}{3.324679in}}%
\pgfpathlineto{\pgfqpoint{6.504552in}{3.303089in}}%
\pgfpathclose%
\pgfusepath{fill}%
\end{pgfscope}%
\begin{pgfscope}%
\pgfpathrectangle{\pgfqpoint{0.539299in}{0.078740in}}{\pgfqpoint{7.842520in}{7.842520in}}%
\pgfusepath{clip}%
\pgfsetbuttcap%
\pgfsetroundjoin%
\definecolor{currentfill}{rgb}{0.275191,0.194905,0.496005}%
\pgfsetfillcolor{currentfill}%
\pgfsetlinewidth{0.000000pt}%
\definecolor{currentstroke}{rgb}{0.169646,0.456262,0.558030}%
\pgfsetstrokecolor{currentstroke}%
\pgfsetdash{}{0pt}%
\pgfpathmoveto{\pgfqpoint{5.673111in}{3.524234in}}%
\pgfpathlineto{\pgfqpoint{5.746902in}{3.537634in}}%
\pgfpathlineto{\pgfqpoint{5.536826in}{3.584474in}}%
\pgfpathclose%
\pgfusepath{fill}%
\end{pgfscope}%
\begin{pgfscope}%
\pgfpathrectangle{\pgfqpoint{0.539299in}{0.078740in}}{\pgfqpoint{7.842520in}{7.842520in}}%
\pgfusepath{clip}%
\pgfsetbuttcap%
\pgfsetroundjoin%
\definecolor{currentfill}{rgb}{0.278012,0.180367,0.486697}%
\pgfsetfillcolor{currentfill}%
\pgfsetlinewidth{0.000000pt}%
\definecolor{currentstroke}{rgb}{0.168126,0.459988,0.558082}%
\pgfsetstrokecolor{currentstroke}%
\pgfsetdash{}{0pt}%
\pgfpathmoveto{\pgfqpoint{5.809842in}{3.469040in}}%
\pgfpathlineto{\pgfqpoint{5.883392in}{3.490293in}}%
\pgfpathlineto{\pgfqpoint{5.746902in}{3.537634in}}%
\pgfpathclose%
\pgfusepath{fill}%
\end{pgfscope}%
\begin{pgfscope}%
\pgfpathrectangle{\pgfqpoint{0.539299in}{0.078740in}}{\pgfqpoint{7.842520in}{7.842520in}}%
\pgfusepath{clip}%
\pgfsetbuttcap%
\pgfsetroundjoin%
\definecolor{currentfill}{rgb}{0.248629,0.278775,0.534556}%
\pgfsetfillcolor{currentfill}%
\pgfsetlinewidth{0.000000pt}%
\definecolor{currentstroke}{rgb}{0.166617,0.463708,0.558119}%
\pgfsetstrokecolor{currentstroke}%
\pgfsetdash{}{0pt}%
\pgfpathmoveto{\pgfqpoint{5.130570in}{3.824189in}}%
\pgfpathlineto{\pgfqpoint{5.055056in}{3.868593in}}%
\pgfpathlineto{\pgfqpoint{5.190465in}{3.761377in}}%
\pgfpathclose%
\pgfusepath{fill}%
\end{pgfscope}%
\begin{pgfscope}%
\pgfpathrectangle{\pgfqpoint{0.539299in}{0.078740in}}{\pgfqpoint{7.842520in}{7.842520in}}%
\pgfusepath{clip}%
\pgfsetbuttcap%
\pgfsetroundjoin%
\definecolor{currentfill}{rgb}{0.280255,0.165693,0.476498}%
\pgfsetfillcolor{currentfill}%
\pgfsetlinewidth{0.000000pt}%
\definecolor{currentstroke}{rgb}{0.165117,0.467423,0.558141}%
\pgfsetstrokecolor{currentstroke}%
\pgfsetdash{}{0pt}%
\pgfpathmoveto{\pgfqpoint{6.093026in}{3.466206in}}%
\pgfpathlineto{\pgfqpoint{6.020289in}{3.442779in}}%
\pgfpathlineto{\pgfqpoint{6.157532in}{3.392366in}}%
\pgfpathclose%
\pgfusepath{fill}%
\end{pgfscope}%
\begin{pgfscope}%
\pgfpathrectangle{\pgfqpoint{0.539299in}{0.078740in}}{\pgfqpoint{7.842520in}{7.842520in}}%
\pgfusepath{clip}%
\pgfsetbuttcap%
\pgfsetroundjoin%
\definecolor{currentfill}{rgb}{0.280267,0.073417,0.397163}%
\pgfsetfillcolor{currentfill}%
\pgfsetlinewidth{0.000000pt}%
\definecolor{currentstroke}{rgb}{0.163625,0.471133,0.558148}%
\pgfsetstrokecolor{currentstroke}%
\pgfsetdash{}{0pt}%
\pgfpathmoveto{\pgfqpoint{6.850286in}{3.187517in}}%
\pgfpathlineto{\pgfqpoint{6.987917in}{3.113178in}}%
\pgfpathlineto{\pgfqpoint{7.057617in}{3.142554in}}%
\pgfpathclose%
\pgfusepath{fill}%
\end{pgfscope}%
\begin{pgfscope}%
\pgfpathrectangle{\pgfqpoint{0.539299in}{0.078740in}}{\pgfqpoint{7.842520in}{7.842520in}}%
\pgfusepath{clip}%
\pgfsetbuttcap%
\pgfsetroundjoin%
\definecolor{currentfill}{rgb}{0.269944,0.014625,0.341379}%
\pgfsetfillcolor{currentfill}%
\pgfsetlinewidth{0.000000pt}%
\definecolor{currentstroke}{rgb}{0.162142,0.474838,0.558140}%
\pgfsetstrokecolor{currentstroke}%
\pgfsetdash{}{0pt}%
\pgfpathmoveto{\pgfqpoint{7.402794in}{3.055818in}}%
\pgfpathlineto{\pgfqpoint{7.333570in}{3.007542in}}%
\pgfpathlineto{\pgfqpoint{7.472249in}{2.944684in}}%
\pgfpathclose%
\pgfusepath{fill}%
\end{pgfscope}%
\begin{pgfscope}%
\pgfpathrectangle{\pgfqpoint{0.539299in}{0.078740in}}{\pgfqpoint{7.842520in}{7.842520in}}%
\pgfusepath{clip}%
\pgfsetbuttcap%
\pgfsetroundjoin%
\definecolor{currentfill}{rgb}{0.277018,0.050344,0.375715}%
\pgfsetfillcolor{currentfill}%
\pgfsetlinewidth{0.000000pt}%
\definecolor{currentstroke}{rgb}{0.160665,0.478540,0.558115}%
\pgfsetstrokecolor{currentstroke}%
\pgfsetdash{}{0pt}%
\pgfpathmoveto{\pgfqpoint{7.195388in}{3.074084in}}%
\pgfpathlineto{\pgfqpoint{7.057617in}{3.142554in}}%
\pgfpathlineto{\pgfqpoint{7.125818in}{3.037569in}}%
\pgfpathclose%
\pgfusepath{fill}%
\end{pgfscope}%
\begin{pgfscope}%
\pgfpathrectangle{\pgfqpoint{0.539299in}{0.078740in}}{\pgfqpoint{7.842520in}{7.842520in}}%
\pgfusepath{clip}%
\pgfsetbuttcap%
\pgfsetroundjoin%
\definecolor{currentfill}{rgb}{0.168126,0.459988,0.558082}%
\pgfsetfillcolor{currentfill}%
\pgfsetlinewidth{0.000000pt}%
\definecolor{currentstroke}{rgb}{0.159194,0.482237,0.558073}%
\pgfsetstrokecolor{currentstroke}%
\pgfsetdash{}{0pt}%
\pgfpathmoveto{\pgfqpoint{3.038103in}{4.744324in}}%
\pgfpathlineto{\pgfqpoint{2.996771in}{4.420700in}}%
\pgfpathlineto{\pgfqpoint{2.913780in}{4.395967in}}%
\pgfpathclose%
\pgfusepath{fill}%
\end{pgfscope}%
\begin{pgfscope}%
\pgfpathrectangle{\pgfqpoint{0.539299in}{0.078740in}}{\pgfqpoint{7.842520in}{7.842520in}}%
\pgfusepath{clip}%
\pgfsetbuttcap%
\pgfsetroundjoin%
\definecolor{currentfill}{rgb}{0.281887,0.150881,0.465405}%
\pgfsetfillcolor{currentfill}%
\pgfsetlinewidth{0.000000pt}%
\definecolor{currentstroke}{rgb}{0.157729,0.485932,0.558013}%
\pgfsetstrokecolor{currentstroke}%
\pgfsetdash{}{0pt}%
\pgfpathmoveto{\pgfqpoint{6.157532in}{3.392366in}}%
\pgfpathlineto{\pgfqpoint{6.367154in}{3.364036in}}%
\pgfpathlineto{\pgfqpoint{6.229961in}{3.418141in}}%
\pgfpathclose%
\pgfusepath{fill}%
\end{pgfscope}%
\begin{pgfscope}%
\pgfpathrectangle{\pgfqpoint{0.539299in}{0.078740in}}{\pgfqpoint{7.842520in}{7.842520in}}%
\pgfusepath{clip}%
\pgfsetbuttcap%
\pgfsetroundjoin%
\definecolor{currentfill}{rgb}{0.266580,0.228262,0.514349}%
\pgfsetfillcolor{currentfill}%
\pgfsetlinewidth{0.000000pt}%
\definecolor{currentstroke}{rgb}{0.156270,0.489624,0.557936}%
\pgfsetstrokecolor{currentstroke}%
\pgfsetdash{}{0pt}%
\pgfpathmoveto{\pgfqpoint{5.536826in}{3.584474in}}%
\pgfpathlineto{\pgfqpoint{5.400994in}{3.652853in}}%
\pgfpathlineto{\pgfqpoint{5.326196in}{3.667457in}}%
\pgfpathclose%
\pgfusepath{fill}%
\end{pgfscope}%
\begin{pgfscope}%
\pgfpathrectangle{\pgfqpoint{0.539299in}{0.078740in}}{\pgfqpoint{7.842520in}{7.842520in}}%
\pgfusepath{clip}%
\pgfsetbuttcap%
\pgfsetroundjoin%
\definecolor{currentfill}{rgb}{0.136408,0.541173,0.554483}%
\pgfsetfillcolor{currentfill}%
\pgfsetlinewidth{0.000000pt}%
\definecolor{currentstroke}{rgb}{0.154815,0.493313,0.557840}%
\pgfsetstrokecolor{currentstroke}%
\pgfsetdash{}{0pt}%
\pgfpathmoveto{\pgfqpoint{3.976815in}{4.900063in}}%
\pgfpathlineto{\pgfqpoint{4.030569in}{4.909838in}}%
\pgfpathlineto{\pgfqpoint{4.111002in}{4.812426in}}%
\pgfpathclose%
\pgfusepath{fill}%
\end{pgfscope}%
\begin{pgfscope}%
\pgfpathrectangle{\pgfqpoint{0.539299in}{0.078740in}}{\pgfqpoint{7.842520in}{7.842520in}}%
\pgfusepath{clip}%
\pgfsetbuttcap%
\pgfsetroundjoin%
\definecolor{currentfill}{rgb}{0.268510,0.009605,0.335427}%
\pgfsetfillcolor{currentfill}%
\pgfsetlinewidth{0.000000pt}%
\definecolor{currentstroke}{rgb}{0.153364,0.497000,0.557724}%
\pgfsetstrokecolor{currentstroke}%
\pgfsetdash{}{0pt}%
\pgfpathmoveto{\pgfqpoint{7.680757in}{2.951802in}}%
\pgfpathlineto{\pgfqpoint{7.541482in}{3.001589in}}%
\pgfpathlineto{\pgfqpoint{7.472249in}{2.944684in}}%
\pgfpathclose%
\pgfusepath{fill}%
\end{pgfscope}%
\begin{pgfscope}%
\pgfpathrectangle{\pgfqpoint{0.539299in}{0.078740in}}{\pgfqpoint{7.842520in}{7.842520in}}%
\pgfusepath{clip}%
\pgfsetbuttcap%
\pgfsetroundjoin%
\definecolor{currentfill}{rgb}{0.276194,0.190074,0.493001}%
\pgfsetfillcolor{currentfill}%
\pgfsetlinewidth{0.000000pt}%
\definecolor{currentstroke}{rgb}{0.151918,0.500685,0.557587}%
\pgfsetstrokecolor{currentstroke}%
\pgfsetdash{}{0pt}%
\pgfpathmoveto{\pgfqpoint{5.673111in}{3.524234in}}%
\pgfpathlineto{\pgfqpoint{5.809842in}{3.469040in}}%
\pgfpathlineto{\pgfqpoint{5.746902in}{3.537634in}}%
\pgfpathclose%
\pgfusepath{fill}%
\end{pgfscope}%
\begin{pgfscope}%
\pgfpathrectangle{\pgfqpoint{0.539299in}{0.078740in}}{\pgfqpoint{7.842520in}{7.842520in}}%
\pgfusepath{clip}%
\pgfsetbuttcap%
\pgfsetroundjoin%
\definecolor{currentfill}{rgb}{0.126453,0.570633,0.549841}%
\pgfsetfillcolor{currentfill}%
\pgfsetlinewidth{0.000000pt}%
\definecolor{currentstroke}{rgb}{0.150476,0.504369,0.557430}%
\pgfsetstrokecolor{currentstroke}%
\pgfsetdash{}{0pt}%
\pgfpathmoveto{\pgfqpoint{3.547026in}{5.048670in}}%
\pgfpathlineto{\pgfqpoint{3.629210in}{4.996070in}}%
\pgfpathlineto{\pgfqpoint{3.415921in}{4.934934in}}%
\pgfpathclose%
\pgfusepath{fill}%
\end{pgfscope}%
\begin{pgfscope}%
\pgfpathrectangle{\pgfqpoint{0.539299in}{0.078740in}}{\pgfqpoint{7.842520in}{7.842520in}}%
\pgfusepath{clip}%
\pgfsetbuttcap%
\pgfsetroundjoin%
\definecolor{currentfill}{rgb}{0.201239,0.383670,0.554294}%
\pgfsetfillcolor{currentfill}%
\pgfsetlinewidth{0.000000pt}%
\definecolor{currentstroke}{rgb}{0.149039,0.508051,0.557250}%
\pgfsetstrokecolor{currentstroke}%
\pgfsetdash{}{0pt}%
\pgfpathmoveto{\pgfqpoint{2.830306in}{4.368134in}}%
\pgfpathlineto{\pgfqpoint{2.710747in}{3.908413in}}%
\pgfpathlineto{\pgfqpoint{2.746363in}{4.336760in}}%
\pgfpathclose%
\pgfusepath{fill}%
\end{pgfscope}%
\begin{pgfscope}%
\pgfpathrectangle{\pgfqpoint{0.539299in}{0.078740in}}{\pgfqpoint{7.842520in}{7.842520in}}%
\pgfusepath{clip}%
\pgfsetbuttcap%
\pgfsetroundjoin%
\definecolor{currentfill}{rgb}{0.282910,0.105393,0.426902}%
\pgfsetfillcolor{currentfill}%
\pgfsetlinewidth{0.000000pt}%
\definecolor{currentstroke}{rgb}{0.147607,0.511733,0.557049}%
\pgfsetstrokecolor{currentstroke}%
\pgfsetdash{}{0pt}%
\pgfpathmoveto{\pgfqpoint{6.712873in}{3.258583in}}%
\pgfpathlineto{\pgfqpoint{6.642120in}{3.235328in}}%
\pgfpathlineto{\pgfqpoint{6.850286in}{3.187517in}}%
\pgfpathclose%
\pgfusepath{fill}%
\end{pgfscope}%
\begin{pgfscope}%
\pgfpathrectangle{\pgfqpoint{0.539299in}{0.078740in}}{\pgfqpoint{7.842520in}{7.842520in}}%
\pgfusepath{clip}%
\pgfsetbuttcap%
\pgfsetroundjoin%
\definecolor{currentfill}{rgb}{0.278791,0.062145,0.386592}%
\pgfsetfillcolor{currentfill}%
\pgfsetlinewidth{0.000000pt}%
\definecolor{currentstroke}{rgb}{0.146180,0.515413,0.556823}%
\pgfsetstrokecolor{currentstroke}%
\pgfsetdash{}{0pt}%
\pgfpathmoveto{\pgfqpoint{7.125818in}{3.037569in}}%
\pgfpathlineto{\pgfqpoint{7.057617in}{3.142554in}}%
\pgfpathlineto{\pgfqpoint{6.987917in}{3.113178in}}%
\pgfpathclose%
\pgfusepath{fill}%
\end{pgfscope}%
\begin{pgfscope}%
\pgfpathrectangle{\pgfqpoint{0.539299in}{0.078740in}}{\pgfqpoint{7.842520in}{7.842520in}}%
\pgfusepath{clip}%
\pgfsetbuttcap%
\pgfsetroundjoin%
\definecolor{currentfill}{rgb}{0.127568,0.566949,0.550556}%
\pgfsetfillcolor{currentfill}%
\pgfsetlinewidth{0.000000pt}%
\definecolor{currentstroke}{rgb}{0.144759,0.519093,0.556572}%
\pgfsetstrokecolor{currentstroke}%
\pgfsetdash{}{0pt}%
\pgfpathmoveto{\pgfqpoint{3.761962in}{5.020676in}}%
\pgfpathlineto{\pgfqpoint{3.895878in}{4.988126in}}%
\pgfpathlineto{\pgfqpoint{3.976815in}{4.900063in}}%
\pgfpathclose%
\pgfusepath{fill}%
\end{pgfscope}%
\begin{pgfscope}%
\pgfpathrectangle{\pgfqpoint{0.539299in}{0.078740in}}{\pgfqpoint{7.842520in}{7.842520in}}%
\pgfusepath{clip}%
\pgfsetbuttcap%
\pgfsetroundjoin%
\definecolor{currentfill}{rgb}{0.140536,0.530132,0.555659}%
\pgfsetfillcolor{currentfill}%
\pgfsetlinewidth{0.000000pt}%
\definecolor{currentstroke}{rgb}{0.143343,0.522773,0.556295}%
\pgfsetstrokecolor{currentstroke}%
\pgfsetdash{}{0pt}%
\pgfpathmoveto{\pgfqpoint{4.245563in}{4.694927in}}%
\pgfpathlineto{\pgfqpoint{4.111002in}{4.812426in}}%
\pgfpathlineto{\pgfqpoint{4.030569in}{4.909838in}}%
\pgfpathclose%
\pgfusepath{fill}%
\end{pgfscope}%
\begin{pgfscope}%
\pgfpathrectangle{\pgfqpoint{0.539299in}{0.078740in}}{\pgfqpoint{7.842520in}{7.842520in}}%
\pgfusepath{clip}%
\pgfsetbuttcap%
\pgfsetroundjoin%
\definecolor{currentfill}{rgb}{0.274952,0.037752,0.364543}%
\pgfsetfillcolor{currentfill}%
\pgfsetlinewidth{0.000000pt}%
\definecolor{currentstroke}{rgb}{0.141935,0.526453,0.555991}%
\pgfsetstrokecolor{currentstroke}%
\pgfsetdash{}{0pt}%
\pgfpathmoveto{\pgfqpoint{7.125818in}{3.037569in}}%
\pgfpathlineto{\pgfqpoint{7.333570in}{3.007542in}}%
\pgfpathlineto{\pgfqpoint{7.195388in}{3.074084in}}%
\pgfpathclose%
\pgfusepath{fill}%
\end{pgfscope}%
\begin{pgfscope}%
\pgfpathrectangle{\pgfqpoint{0.539299in}{0.078740in}}{\pgfqpoint{7.842520in}{7.842520in}}%
\pgfusepath{clip}%
\pgfsetbuttcap%
\pgfsetroundjoin%
\definecolor{currentfill}{rgb}{0.258965,0.251537,0.524736}%
\pgfsetfillcolor{currentfill}%
\pgfsetlinewidth{0.000000pt}%
\definecolor{currentstroke}{rgb}{0.140536,0.530132,0.555659}%
\pgfsetstrokecolor{currentstroke}%
\pgfsetdash{}{0pt}%
\pgfpathmoveto{\pgfqpoint{5.190465in}{3.761377in}}%
\pgfpathlineto{\pgfqpoint{5.326196in}{3.667457in}}%
\pgfpathlineto{\pgfqpoint{5.400994in}{3.652853in}}%
\pgfpathclose%
\pgfusepath{fill}%
\end{pgfscope}%
\begin{pgfscope}%
\pgfpathrectangle{\pgfqpoint{0.539299in}{0.078740in}}{\pgfqpoint{7.842520in}{7.842520in}}%
\pgfusepath{clip}%
\pgfsetbuttcap%
\pgfsetroundjoin%
\definecolor{currentfill}{rgb}{0.283229,0.120777,0.440584}%
\pgfsetfillcolor{currentfill}%
\pgfsetlinewidth{0.000000pt}%
\definecolor{currentstroke}{rgb}{0.139147,0.533812,0.555298}%
\pgfsetstrokecolor{currentstroke}%
\pgfsetdash{}{0pt}%
\pgfpathmoveto{\pgfqpoint{6.504552in}{3.303089in}}%
\pgfpathlineto{\pgfqpoint{6.642120in}{3.235328in}}%
\pgfpathlineto{\pgfqpoint{6.712873in}{3.258583in}}%
\pgfpathclose%
\pgfusepath{fill}%
\end{pgfscope}%
\begin{pgfscope}%
\pgfpathrectangle{\pgfqpoint{0.539299in}{0.078740in}}{\pgfqpoint{7.842520in}{7.842520in}}%
\pgfusepath{clip}%
\pgfsetbuttcap%
\pgfsetroundjoin%
\definecolor{currentfill}{rgb}{0.278012,0.180367,0.486697}%
\pgfsetfillcolor{currentfill}%
\pgfsetlinewidth{0.000000pt}%
\definecolor{currentstroke}{rgb}{0.137770,0.537492,0.554906}%
\pgfsetstrokecolor{currentstroke}%
\pgfsetdash{}{0pt}%
\pgfpathmoveto{\pgfqpoint{5.946992in}{3.415751in}}%
\pgfpathlineto{\pgfqpoint{6.020289in}{3.442779in}}%
\pgfpathlineto{\pgfqpoint{5.883392in}{3.490293in}}%
\pgfpathclose%
\pgfusepath{fill}%
\end{pgfscope}%
\begin{pgfscope}%
\pgfpathrectangle{\pgfqpoint{0.539299in}{0.078740in}}{\pgfqpoint{7.842520in}{7.842520in}}%
\pgfusepath{clip}%
\pgfsetbuttcap%
\pgfsetroundjoin%
\definecolor{currentfill}{rgb}{0.154815,0.493313,0.557840}%
\pgfsetfillcolor{currentfill}%
\pgfsetlinewidth{0.000000pt}%
\definecolor{currentstroke}{rgb}{0.136408,0.541173,0.554483}%
\pgfsetstrokecolor{currentstroke}%
\pgfsetdash{}{0pt}%
\pgfpathmoveto{\pgfqpoint{3.121655in}{4.747134in}}%
\pgfpathlineto{\pgfqpoint{2.996771in}{4.420700in}}%
\pgfpathlineto{\pgfqpoint{3.038103in}{4.744324in}}%
\pgfpathclose%
\pgfusepath{fill}%
\end{pgfscope}%
\begin{pgfscope}%
\pgfpathrectangle{\pgfqpoint{0.539299in}{0.078740in}}{\pgfqpoint{7.842520in}{7.842520in}}%
\pgfusepath{clip}%
\pgfsetbuttcap%
\pgfsetroundjoin%
\definecolor{currentfill}{rgb}{0.150476,0.504369,0.557430}%
\pgfsetfillcolor{currentfill}%
\pgfsetlinewidth{0.000000pt}%
\definecolor{currentstroke}{rgb}{0.135066,0.544853,0.554029}%
\pgfsetstrokecolor{currentstroke}%
\pgfsetdash{}{0pt}%
\pgfpathmoveto{\pgfqpoint{4.380311in}{4.558363in}}%
\pgfpathlineto{\pgfqpoint{4.245563in}{4.694927in}}%
\pgfpathlineto{\pgfqpoint{4.165728in}{4.797221in}}%
\pgfpathclose%
\pgfusepath{fill}%
\end{pgfscope}%
\begin{pgfscope}%
\pgfpathrectangle{\pgfqpoint{0.539299in}{0.078740in}}{\pgfqpoint{7.842520in}{7.842520in}}%
\pgfusepath{clip}%
\pgfsetbuttcap%
\pgfsetroundjoin%
\definecolor{currentfill}{rgb}{0.139147,0.533812,0.555298}%
\pgfsetfillcolor{currentfill}%
\pgfsetlinewidth{0.000000pt}%
\definecolor{currentstroke}{rgb}{0.133743,0.548535,0.553541}%
\pgfsetstrokecolor{currentstroke}%
\pgfsetdash{}{0pt}%
\pgfpathmoveto{\pgfqpoint{3.333165in}{4.962328in}}%
\pgfpathlineto{\pgfqpoint{3.204675in}{4.746147in}}%
\pgfpathlineto{\pgfqpoint{3.121655in}{4.747134in}}%
\pgfpathclose%
\pgfusepath{fill}%
\end{pgfscope}%
\begin{pgfscope}%
\pgfpathrectangle{\pgfqpoint{0.539299in}{0.078740in}}{\pgfqpoint{7.842520in}{7.842520in}}%
\pgfusepath{clip}%
\pgfsetbuttcap%
\pgfsetroundjoin%
\definecolor{currentfill}{rgb}{0.168126,0.459988,0.558082}%
\pgfsetfillcolor{currentfill}%
\pgfsetlinewidth{0.000000pt}%
\definecolor{currentstroke}{rgb}{0.132444,0.552216,0.553018}%
\pgfsetstrokecolor{currentstroke}%
\pgfsetdash{}{0pt}%
\pgfpathmoveto{\pgfqpoint{4.380311in}{4.558363in}}%
\pgfpathlineto{\pgfqpoint{4.436631in}{4.511408in}}%
\pgfpathlineto{\pgfqpoint{4.515135in}{4.412321in}}%
\pgfpathclose%
\pgfusepath{fill}%
\end{pgfscope}%
\begin{pgfscope}%
\pgfpathrectangle{\pgfqpoint{0.539299in}{0.078740in}}{\pgfqpoint{7.842520in}{7.842520in}}%
\pgfusepath{clip}%
\pgfsetbuttcap%
\pgfsetroundjoin%
\definecolor{currentfill}{rgb}{0.282290,0.145912,0.461510}%
\pgfsetfillcolor{currentfill}%
\pgfsetlinewidth{0.000000pt}%
\definecolor{currentstroke}{rgb}{0.131172,0.555899,0.552459}%
\pgfsetstrokecolor{currentstroke}%
\pgfsetdash{}{0pt}%
\pgfpathmoveto{\pgfqpoint{6.504552in}{3.303089in}}%
\pgfpathlineto{\pgfqpoint{6.367154in}{3.364036in}}%
\pgfpathlineto{\pgfqpoint{6.295060in}{3.336953in}}%
\pgfpathclose%
\pgfusepath{fill}%
\end{pgfscope}%
\begin{pgfscope}%
\pgfpathrectangle{\pgfqpoint{0.539299in}{0.078740in}}{\pgfqpoint{7.842520in}{7.842520in}}%
\pgfusepath{clip}%
\pgfsetbuttcap%
\pgfsetroundjoin%
\definecolor{currentfill}{rgb}{0.177423,0.437527,0.557565}%
\pgfsetfillcolor{currentfill}%
\pgfsetlinewidth{0.000000pt}%
\definecolor{currentstroke}{rgb}{0.129933,0.559582,0.551864}%
\pgfsetstrokecolor{currentstroke}%
\pgfsetdash{}{0pt}%
\pgfpathmoveto{\pgfqpoint{4.515135in}{4.412321in}}%
\pgfpathlineto{\pgfqpoint{4.436631in}{4.511408in}}%
\pgfpathlineto{\pgfqpoint{4.649993in}{4.264866in}}%
\pgfpathclose%
\pgfusepath{fill}%
\end{pgfscope}%
\begin{pgfscope}%
\pgfpathrectangle{\pgfqpoint{0.539299in}{0.078740in}}{\pgfqpoint{7.842520in}{7.842520in}}%
\pgfusepath{clip}%
\pgfsetbuttcap%
\pgfsetroundjoin%
\definecolor{currentfill}{rgb}{0.128729,0.563265,0.551229}%
\pgfsetfillcolor{currentfill}%
\pgfsetlinewidth{0.000000pt}%
\definecolor{currentstroke}{rgb}{0.128729,0.563265,0.551229}%
\pgfsetstrokecolor{currentstroke}%
\pgfsetdash{}{0pt}%
\pgfpathmoveto{\pgfqpoint{3.976815in}{4.900063in}}%
\pgfpathlineto{\pgfqpoint{3.895878in}{4.988126in}}%
\pgfpathlineto{\pgfqpoint{4.030569in}{4.909838in}}%
\pgfpathclose%
\pgfusepath{fill}%
\end{pgfscope}%
\begin{pgfscope}%
\pgfpathrectangle{\pgfqpoint{0.539299in}{0.078740in}}{\pgfqpoint{7.842520in}{7.842520in}}%
\pgfusepath{clip}%
\pgfsetbuttcap%
\pgfsetroundjoin%
\definecolor{currentfill}{rgb}{0.197636,0.391528,0.554969}%
\pgfsetfillcolor{currentfill}%
\pgfsetlinewidth{0.000000pt}%
\definecolor{currentstroke}{rgb}{0.127568,0.566949,0.550556}%
\pgfsetstrokecolor{currentstroke}%
\pgfsetdash{}{0pt}%
\pgfpathmoveto{\pgfqpoint{4.784895in}{4.122344in}}%
\pgfpathlineto{\pgfqpoint{4.649993in}{4.264866in}}%
\pgfpathlineto{\pgfqpoint{4.707691in}{4.203491in}}%
\pgfpathclose%
\pgfusepath{fill}%
\end{pgfscope}%
\begin{pgfscope}%
\pgfpathrectangle{\pgfqpoint{0.539299in}{0.078740in}}{\pgfqpoint{7.842520in}{7.842520in}}%
\pgfusepath{clip}%
\pgfsetbuttcap%
\pgfsetroundjoin%
\definecolor{currentfill}{rgb}{0.223925,0.334994,0.548053}%
\pgfsetfillcolor{currentfill}%
\pgfsetlinewidth{0.000000pt}%
\definecolor{currentstroke}{rgb}{0.126453,0.570633,0.549841}%
\pgfsetstrokecolor{currentstroke}%
\pgfsetdash{}{0pt}%
\pgfpathmoveto{\pgfqpoint{2.661974in}{4.301274in}}%
\pgfpathlineto{\pgfqpoint{2.627589in}{3.858151in}}%
\pgfpathlineto{\pgfqpoint{2.543987in}{3.805632in}}%
\pgfpathclose%
\pgfusepath{fill}%
\end{pgfscope}%
\begin{pgfscope}%
\pgfpathrectangle{\pgfqpoint{0.539299in}{0.078740in}}{\pgfqpoint{7.842520in}{7.842520in}}%
\pgfusepath{clip}%
\pgfsetbuttcap%
\pgfsetroundjoin%
\definecolor{currentfill}{rgb}{0.281412,0.155834,0.469201}%
\pgfsetfillcolor{currentfill}%
\pgfsetlinewidth{0.000000pt}%
\definecolor{currentstroke}{rgb}{0.125394,0.574318,0.549086}%
\pgfsetstrokecolor{currentstroke}%
\pgfsetdash{}{0pt}%
\pgfpathmoveto{\pgfqpoint{6.295060in}{3.336953in}}%
\pgfpathlineto{\pgfqpoint{6.367154in}{3.364036in}}%
\pgfpathlineto{\pgfqpoint{6.157532in}{3.392366in}}%
\pgfpathclose%
\pgfusepath{fill}%
\end{pgfscope}%
\begin{pgfscope}%
\pgfpathrectangle{\pgfqpoint{0.539299in}{0.078740in}}{\pgfqpoint{7.842520in}{7.842520in}}%
\pgfusepath{clip}%
\pgfsetbuttcap%
\pgfsetroundjoin%
\definecolor{currentfill}{rgb}{0.277134,0.185228,0.489898}%
\pgfsetfillcolor{currentfill}%
\pgfsetlinewidth{0.000000pt}%
\definecolor{currentstroke}{rgb}{0.124395,0.578002,0.548287}%
\pgfsetstrokecolor{currentstroke}%
\pgfsetdash{}{0pt}%
\pgfpathmoveto{\pgfqpoint{5.946992in}{3.415751in}}%
\pgfpathlineto{\pgfqpoint{5.883392in}{3.490293in}}%
\pgfpathlineto{\pgfqpoint{5.809842in}{3.469040in}}%
\pgfpathclose%
\pgfusepath{fill}%
\end{pgfscope}%
\begin{pgfscope}%
\pgfpathrectangle{\pgfqpoint{0.539299in}{0.078740in}}{\pgfqpoint{7.842520in}{7.842520in}}%
\pgfusepath{clip}%
\pgfsetbuttcap%
\pgfsetroundjoin%
\definecolor{currentfill}{rgb}{0.268510,0.009605,0.335427}%
\pgfsetfillcolor{currentfill}%
\pgfsetlinewidth{0.000000pt}%
\definecolor{currentstroke}{rgb}{0.123463,0.581687,0.547445}%
\pgfsetstrokecolor{currentstroke}%
\pgfsetdash{}{0pt}%
\pgfpathmoveto{\pgfqpoint{7.611514in}{2.887087in}}%
\pgfpathlineto{\pgfqpoint{7.680757in}{2.951802in}}%
\pgfpathlineto{\pgfqpoint{7.472249in}{2.944684in}}%
\pgfpathclose%
\pgfusepath{fill}%
\end{pgfscope}%
\begin{pgfscope}%
\pgfpathrectangle{\pgfqpoint{0.539299in}{0.078740in}}{\pgfqpoint{7.842520in}{7.842520in}}%
\pgfusepath{clip}%
\pgfsetbuttcap%
\pgfsetroundjoin%
\definecolor{currentfill}{rgb}{0.281924,0.089666,0.412415}%
\pgfsetfillcolor{currentfill}%
\pgfsetlinewidth{0.000000pt}%
\definecolor{currentstroke}{rgb}{0.122606,0.585371,0.546557}%
\pgfsetstrokecolor{currentstroke}%
\pgfsetdash{}{0pt}%
\pgfpathmoveto{\pgfqpoint{6.987917in}{3.113178in}}%
\pgfpathlineto{\pgfqpoint{6.850286in}{3.187517in}}%
\pgfpathlineto{\pgfqpoint{6.779843in}{3.161531in}}%
\pgfpathclose%
\pgfusepath{fill}%
\end{pgfscope}%
\begin{pgfscope}%
\pgfpathrectangle{\pgfqpoint{0.539299in}{0.078740in}}{\pgfqpoint{7.842520in}{7.842520in}}%
\pgfusepath{clip}%
\pgfsetbuttcap%
\pgfsetroundjoin%
\definecolor{currentfill}{rgb}{0.271828,0.209303,0.504434}%
\pgfsetfillcolor{currentfill}%
\pgfsetlinewidth{0.000000pt}%
\definecolor{currentstroke}{rgb}{0.121831,0.589055,0.545623}%
\pgfsetstrokecolor{currentstroke}%
\pgfsetdash{}{0pt}%
\pgfpathmoveto{\pgfqpoint{5.536826in}{3.584474in}}%
\pgfpathlineto{\pgfqpoint{5.598833in}{3.513257in}}%
\pgfpathlineto{\pgfqpoint{5.673111in}{3.524234in}}%
\pgfpathclose%
\pgfusepath{fill}%
\end{pgfscope}%
\begin{pgfscope}%
\pgfpathrectangle{\pgfqpoint{0.539299in}{0.078740in}}{\pgfqpoint{7.842520in}{7.842520in}}%
\pgfusepath{clip}%
\pgfsetbuttcap%
\pgfsetroundjoin%
\definecolor{currentfill}{rgb}{0.208623,0.367752,0.552675}%
\pgfsetfillcolor{currentfill}%
\pgfsetlinewidth{0.000000pt}%
\definecolor{currentstroke}{rgb}{0.121148,0.592739,0.544641}%
\pgfsetstrokecolor{currentstroke}%
\pgfsetdash{}{0pt}%
\pgfpathmoveto{\pgfqpoint{4.707691in}{4.203491in}}%
\pgfpathlineto{\pgfqpoint{4.919892in}{3.989315in}}%
\pgfpathlineto{\pgfqpoint{4.784895in}{4.122344in}}%
\pgfpathclose%
\pgfusepath{fill}%
\end{pgfscope}%
\begin{pgfscope}%
\pgfpathrectangle{\pgfqpoint{0.539299in}{0.078740in}}{\pgfqpoint{7.842520in}{7.842520in}}%
\pgfusepath{clip}%
\pgfsetbuttcap%
\pgfsetroundjoin%
\definecolor{currentfill}{rgb}{0.265145,0.232956,0.516599}%
\pgfsetfillcolor{currentfill}%
\pgfsetlinewidth{0.000000pt}%
\definecolor{currentstroke}{rgb}{0.120565,0.596422,0.543611}%
\pgfsetstrokecolor{currentstroke}%
\pgfsetdash{}{0pt}%
\pgfpathmoveto{\pgfqpoint{5.326196in}{3.667457in}}%
\pgfpathlineto{\pgfqpoint{5.462306in}{3.585484in}}%
\pgfpathlineto{\pgfqpoint{5.536826in}{3.584474in}}%
\pgfpathclose%
\pgfusepath{fill}%
\end{pgfscope}%
\begin{pgfscope}%
\pgfpathrectangle{\pgfqpoint{0.539299in}{0.078740in}}{\pgfqpoint{7.842520in}{7.842520in}}%
\pgfusepath{clip}%
\pgfsetbuttcap%
\pgfsetroundjoin%
\definecolor{currentfill}{rgb}{0.121831,0.589055,0.545623}%
\pgfsetfillcolor{currentfill}%
\pgfsetlinewidth{0.000000pt}%
\definecolor{currentstroke}{rgb}{0.120092,0.600104,0.542530}%
\pgfsetstrokecolor{currentstroke}%
\pgfsetdash{}{0pt}%
\pgfpathmoveto{\pgfqpoint{3.679940in}{5.091602in}}%
\pgfpathlineto{\pgfqpoint{3.761962in}{5.020676in}}%
\pgfpathlineto{\pgfqpoint{3.629210in}{4.996070in}}%
\pgfpathclose%
\pgfusepath{fill}%
\end{pgfscope}%
\begin{pgfscope}%
\pgfpathrectangle{\pgfqpoint{0.539299in}{0.078740in}}{\pgfqpoint{7.842520in}{7.842520in}}%
\pgfusepath{clip}%
\pgfsetbuttcap%
\pgfsetroundjoin%
\definecolor{currentfill}{rgb}{0.272594,0.025563,0.353093}%
\pgfsetfillcolor{currentfill}%
\pgfsetlinewidth{0.000000pt}%
\definecolor{currentstroke}{rgb}{0.119738,0.603785,0.541400}%
\pgfsetstrokecolor{currentstroke}%
\pgfsetdash{}{0pt}%
\pgfpathmoveto{\pgfqpoint{7.472249in}{2.944684in}}%
\pgfpathlineto{\pgfqpoint{7.333570in}{3.007542in}}%
\pgfpathlineto{\pgfqpoint{7.264065in}{2.962827in}}%
\pgfpathclose%
\pgfusepath{fill}%
\end{pgfscope}%
\begin{pgfscope}%
\pgfpathrectangle{\pgfqpoint{0.539299in}{0.078740in}}{\pgfqpoint{7.842520in}{7.842520in}}%
\pgfusepath{clip}%
\pgfsetbuttcap%
\pgfsetroundjoin%
\definecolor{currentfill}{rgb}{0.124395,0.578002,0.548287}%
\pgfsetfillcolor{currentfill}%
\pgfsetlinewidth{0.000000pt}%
\definecolor{currentstroke}{rgb}{0.119512,0.607464,0.540218}%
\pgfsetstrokecolor{currentstroke}%
\pgfsetdash{}{0pt}%
\pgfpathmoveto{\pgfqpoint{3.547026in}{5.048670in}}%
\pgfpathlineto{\pgfqpoint{3.415921in}{4.934934in}}%
\pgfpathlineto{\pgfqpoint{3.333165in}{4.962328in}}%
\pgfpathclose%
\pgfusepath{fill}%
\end{pgfscope}%
\begin{pgfscope}%
\pgfpathrectangle{\pgfqpoint{0.539299in}{0.078740in}}{\pgfqpoint{7.842520in}{7.842520in}}%
\pgfusepath{clip}%
\pgfsetbuttcap%
\pgfsetroundjoin%
\definecolor{currentfill}{rgb}{0.278826,0.175490,0.483397}%
\pgfsetfillcolor{currentfill}%
\pgfsetlinewidth{0.000000pt}%
\definecolor{currentstroke}{rgb}{0.119423,0.611141,0.538982}%
\pgfsetstrokecolor{currentstroke}%
\pgfsetdash{}{0pt}%
\pgfpathmoveto{\pgfqpoint{5.946992in}{3.415751in}}%
\pgfpathlineto{\pgfqpoint{6.157532in}{3.392366in}}%
\pgfpathlineto{\pgfqpoint{6.020289in}{3.442779in}}%
\pgfpathclose%
\pgfusepath{fill}%
\end{pgfscope}%
\begin{pgfscope}%
\pgfpathrectangle{\pgfqpoint{0.539299in}{0.078740in}}{\pgfqpoint{7.842520in}{7.842520in}}%
\pgfusepath{clip}%
\pgfsetbuttcap%
\pgfsetroundjoin%
\definecolor{currentfill}{rgb}{0.282910,0.105393,0.426902}%
\pgfsetfillcolor{currentfill}%
\pgfsetlinewidth{0.000000pt}%
\definecolor{currentstroke}{rgb}{0.119483,0.614817,0.537692}%
\pgfsetstrokecolor{currentstroke}%
\pgfsetdash{}{0pt}%
\pgfpathmoveto{\pgfqpoint{6.850286in}{3.187517in}}%
\pgfpathlineto{\pgfqpoint{6.642120in}{3.235328in}}%
\pgfpathlineto{\pgfqpoint{6.779843in}{3.161531in}}%
\pgfpathclose%
\pgfusepath{fill}%
\end{pgfscope}%
\begin{pgfscope}%
\pgfpathrectangle{\pgfqpoint{0.539299in}{0.078740in}}{\pgfqpoint{7.842520in}{7.842520in}}%
\pgfusepath{clip}%
\pgfsetbuttcap%
\pgfsetroundjoin%
\definecolor{currentfill}{rgb}{0.274952,0.037752,0.364543}%
\pgfsetfillcolor{currentfill}%
\pgfsetlinewidth{0.000000pt}%
\definecolor{currentstroke}{rgb}{0.119699,0.618490,0.536347}%
\pgfsetstrokecolor{currentstroke}%
\pgfsetdash{}{0pt}%
\pgfpathmoveto{\pgfqpoint{7.264065in}{2.962827in}}%
\pgfpathlineto{\pgfqpoint{7.333570in}{3.007542in}}%
\pgfpathlineto{\pgfqpoint{7.125818in}{3.037569in}}%
\pgfpathclose%
\pgfusepath{fill}%
\end{pgfscope}%
\begin{pgfscope}%
\pgfpathrectangle{\pgfqpoint{0.539299in}{0.078740in}}{\pgfqpoint{7.842520in}{7.842520in}}%
\pgfusepath{clip}%
\pgfsetbuttcap%
\pgfsetroundjoin%
\definecolor{currentfill}{rgb}{0.229739,0.322361,0.545706}%
\pgfsetfillcolor{currentfill}%
\pgfsetlinewidth{0.000000pt}%
\definecolor{currentstroke}{rgb}{0.120081,0.622161,0.534946}%
\pgfsetstrokecolor{currentstroke}%
\pgfsetdash{}{0pt}%
\pgfpathmoveto{\pgfqpoint{5.055056in}{3.868593in}}%
\pgfpathlineto{\pgfqpoint{4.919892in}{3.989315in}}%
\pgfpathlineto{\pgfqpoint{4.978938in}{3.922091in}}%
\pgfpathclose%
\pgfusepath{fill}%
\end{pgfscope}%
\begin{pgfscope}%
\pgfpathrectangle{\pgfqpoint{0.539299in}{0.078740in}}{\pgfqpoint{7.842520in}{7.842520in}}%
\pgfusepath{clip}%
\pgfsetbuttcap%
\pgfsetroundjoin%
\definecolor{currentfill}{rgb}{0.121148,0.592739,0.544641}%
\pgfsetfillcolor{currentfill}%
\pgfsetlinewidth{0.000000pt}%
\definecolor{currentstroke}{rgb}{0.120638,0.625828,0.533488}%
\pgfsetstrokecolor{currentstroke}%
\pgfsetdash{}{0pt}%
\pgfpathmoveto{\pgfqpoint{3.679940in}{5.091602in}}%
\pgfpathlineto{\pgfqpoint{3.629210in}{4.996070in}}%
\pgfpathlineto{\pgfqpoint{3.547026in}{5.048670in}}%
\pgfpathclose%
\pgfusepath{fill}%
\end{pgfscope}%
\begin{pgfscope}%
\pgfpathrectangle{\pgfqpoint{0.539299in}{0.078740in}}{\pgfqpoint{7.842520in}{7.842520in}}%
\pgfusepath{clip}%
\pgfsetbuttcap%
\pgfsetroundjoin%
\definecolor{currentfill}{rgb}{0.267968,0.223549,0.512008}%
\pgfsetfillcolor{currentfill}%
\pgfsetlinewidth{0.000000pt}%
\definecolor{currentstroke}{rgb}{0.121380,0.629492,0.531973}%
\pgfsetstrokecolor{currentstroke}%
\pgfsetdash{}{0pt}%
\pgfpathmoveto{\pgfqpoint{5.462306in}{3.585484in}}%
\pgfpathlineto{\pgfqpoint{5.598833in}{3.513257in}}%
\pgfpathlineto{\pgfqpoint{5.536826in}{3.584474in}}%
\pgfpathclose%
\pgfusepath{fill}%
\end{pgfscope}%
\begin{pgfscope}%
\pgfpathrectangle{\pgfqpoint{0.539299in}{0.078740in}}{\pgfqpoint{7.842520in}{7.842520in}}%
\pgfusepath{clip}%
\pgfsetbuttcap%
\pgfsetroundjoin%
\definecolor{currentfill}{rgb}{0.137770,0.537492,0.554906}%
\pgfsetfillcolor{currentfill}%
\pgfsetlinewidth{0.000000pt}%
\definecolor{currentstroke}{rgb}{0.122312,0.633153,0.530398}%
\pgfsetstrokecolor{currentstroke}%
\pgfsetdash{}{0pt}%
\pgfpathmoveto{\pgfqpoint{4.165728in}{4.797221in}}%
\pgfpathlineto{\pgfqpoint{4.245563in}{4.694927in}}%
\pgfpathlineto{\pgfqpoint{4.030569in}{4.909838in}}%
\pgfpathclose%
\pgfusepath{fill}%
\end{pgfscope}%
\begin{pgfscope}%
\pgfpathrectangle{\pgfqpoint{0.539299in}{0.078740in}}{\pgfqpoint{7.842520in}{7.842520in}}%
\pgfusepath{clip}%
\pgfsetbuttcap%
\pgfsetroundjoin%
\definecolor{currentfill}{rgb}{0.241237,0.296485,0.539709}%
\pgfsetfillcolor{currentfill}%
\pgfsetlinewidth{0.000000pt}%
\definecolor{currentstroke}{rgb}{0.123444,0.636809,0.528763}%
\pgfsetstrokecolor{currentstroke}%
\pgfsetdash{}{0pt}%
\pgfpathmoveto{\pgfqpoint{5.190465in}{3.761377in}}%
\pgfpathlineto{\pgfqpoint{5.055056in}{3.868593in}}%
\pgfpathlineto{\pgfqpoint{5.114785in}{3.799326in}}%
\pgfpathclose%
\pgfusepath{fill}%
\end{pgfscope}%
\begin{pgfscope}%
\pgfpathrectangle{\pgfqpoint{0.539299in}{0.078740in}}{\pgfqpoint{7.842520in}{7.842520in}}%
\pgfusepath{clip}%
\pgfsetbuttcap%
\pgfsetroundjoin%
\definecolor{currentfill}{rgb}{0.280267,0.073417,0.397163}%
\pgfsetfillcolor{currentfill}%
\pgfsetlinewidth{0.000000pt}%
\definecolor{currentstroke}{rgb}{0.124780,0.640461,0.527068}%
\pgfsetstrokecolor{currentstroke}%
\pgfsetdash{}{0pt}%
\pgfpathmoveto{\pgfqpoint{7.125818in}{3.037569in}}%
\pgfpathlineto{\pgfqpoint{6.987917in}{3.113178in}}%
\pgfpathlineto{\pgfqpoint{6.917735in}{3.083093in}}%
\pgfpathclose%
\pgfusepath{fill}%
\end{pgfscope}%
\begin{pgfscope}%
\pgfpathrectangle{\pgfqpoint{0.539299in}{0.078740in}}{\pgfqpoint{7.842520in}{7.842520in}}%
\pgfusepath{clip}%
\pgfsetbuttcap%
\pgfsetroundjoin%
\definecolor{currentfill}{rgb}{0.147607,0.511733,0.557049}%
\pgfsetfillcolor{currentfill}%
\pgfsetlinewidth{0.000000pt}%
\definecolor{currentstroke}{rgb}{0.126326,0.644107,0.525311}%
\pgfsetstrokecolor{currentstroke}%
\pgfsetdash{}{0pt}%
\pgfpathmoveto{\pgfqpoint{4.165728in}{4.797221in}}%
\pgfpathlineto{\pgfqpoint{4.301131in}{4.661130in}}%
\pgfpathlineto{\pgfqpoint{4.380311in}{4.558363in}}%
\pgfpathclose%
\pgfusepath{fill}%
\end{pgfscope}%
\begin{pgfscope}%
\pgfpathrectangle{\pgfqpoint{0.539299in}{0.078740in}}{\pgfqpoint{7.842520in}{7.842520in}}%
\pgfusepath{clip}%
\pgfsetbuttcap%
\pgfsetroundjoin%
\definecolor{currentfill}{rgb}{0.281887,0.150881,0.465405}%
\pgfsetfillcolor{currentfill}%
\pgfsetlinewidth{0.000000pt}%
\definecolor{currentstroke}{rgb}{0.128087,0.647749,0.523491}%
\pgfsetstrokecolor{currentstroke}%
\pgfsetdash{}{0pt}%
\pgfpathmoveto{\pgfqpoint{6.504552in}{3.303089in}}%
\pgfpathlineto{\pgfqpoint{6.295060in}{3.336953in}}%
\pgfpathlineto{\pgfqpoint{6.432810in}{3.275195in}}%
\pgfpathclose%
\pgfusepath{fill}%
\end{pgfscope}%
\begin{pgfscope}%
\pgfpathrectangle{\pgfqpoint{0.539299in}{0.078740in}}{\pgfqpoint{7.842520in}{7.842520in}}%
\pgfusepath{clip}%
\pgfsetbuttcap%
\pgfsetroundjoin%
\definecolor{currentfill}{rgb}{0.156270,0.489624,0.557936}%
\pgfsetfillcolor{currentfill}%
\pgfsetlinewidth{0.000000pt}%
\definecolor{currentstroke}{rgb}{0.130067,0.651384,0.521608}%
\pgfsetstrokecolor{currentstroke}%
\pgfsetdash{}{0pt}%
\pgfpathmoveto{\pgfqpoint{4.380311in}{4.558363in}}%
\pgfpathlineto{\pgfqpoint{4.301131in}{4.661130in}}%
\pgfpathlineto{\pgfqpoint{4.436631in}{4.511408in}}%
\pgfpathclose%
\pgfusepath{fill}%
\end{pgfscope}%
\begin{pgfscope}%
\pgfpathrectangle{\pgfqpoint{0.539299in}{0.078740in}}{\pgfqpoint{7.842520in}{7.842520in}}%
\pgfusepath{clip}%
\pgfsetbuttcap%
\pgfsetroundjoin%
\definecolor{currentfill}{rgb}{0.273006,0.204520,0.501721}%
\pgfsetfillcolor{currentfill}%
\pgfsetlinewidth{0.000000pt}%
\definecolor{currentstroke}{rgb}{0.132268,0.655014,0.519661}%
\pgfsetstrokecolor{currentstroke}%
\pgfsetdash{}{0pt}%
\pgfpathmoveto{\pgfqpoint{5.735793in}{3.448039in}}%
\pgfpathlineto{\pgfqpoint{5.809842in}{3.469040in}}%
\pgfpathlineto{\pgfqpoint{5.673111in}{3.524234in}}%
\pgfpathclose%
\pgfusepath{fill}%
\end{pgfscope}%
\begin{pgfscope}%
\pgfpathrectangle{\pgfqpoint{0.539299in}{0.078740in}}{\pgfqpoint{7.842520in}{7.842520in}}%
\pgfusepath{clip}%
\pgfsetbuttcap%
\pgfsetroundjoin%
\definecolor{currentfill}{rgb}{0.283072,0.130895,0.449241}%
\pgfsetfillcolor{currentfill}%
\pgfsetlinewidth{0.000000pt}%
\definecolor{currentstroke}{rgb}{0.134692,0.658636,0.517649}%
\pgfsetstrokecolor{currentstroke}%
\pgfsetdash{}{0pt}%
\pgfpathmoveto{\pgfqpoint{6.570734in}{3.206547in}}%
\pgfpathlineto{\pgfqpoint{6.642120in}{3.235328in}}%
\pgfpathlineto{\pgfqpoint{6.504552in}{3.303089in}}%
\pgfpathclose%
\pgfusepath{fill}%
\end{pgfscope}%
\begin{pgfscope}%
\pgfpathrectangle{\pgfqpoint{0.539299in}{0.078740in}}{\pgfqpoint{7.842520in}{7.842520in}}%
\pgfusepath{clip}%
\pgfsetbuttcap%
\pgfsetroundjoin%
\definecolor{currentfill}{rgb}{0.175841,0.441290,0.557685}%
\pgfsetfillcolor{currentfill}%
\pgfsetlinewidth{0.000000pt}%
\definecolor{currentstroke}{rgb}{0.137339,0.662252,0.515571}%
\pgfsetstrokecolor{currentstroke}%
\pgfsetdash{}{0pt}%
\pgfpathmoveto{\pgfqpoint{4.436631in}{4.511408in}}%
\pgfpathlineto{\pgfqpoint{4.572157in}{4.356552in}}%
\pgfpathlineto{\pgfqpoint{4.649993in}{4.264866in}}%
\pgfpathclose%
\pgfusepath{fill}%
\end{pgfscope}%
\begin{pgfscope}%
\pgfpathrectangle{\pgfqpoint{0.539299in}{0.078740in}}{\pgfqpoint{7.842520in}{7.842520in}}%
\pgfusepath{clip}%
\pgfsetbuttcap%
\pgfsetroundjoin%
\definecolor{currentfill}{rgb}{0.185556,0.418570,0.556753}%
\pgfsetfillcolor{currentfill}%
\pgfsetlinewidth{0.000000pt}%
\definecolor{currentstroke}{rgb}{0.140210,0.665859,0.513427}%
\pgfsetstrokecolor{currentstroke}%
\pgfsetdash{}{0pt}%
\pgfpathmoveto{\pgfqpoint{4.707691in}{4.203491in}}%
\pgfpathlineto{\pgfqpoint{4.649993in}{4.264866in}}%
\pgfpathlineto{\pgfqpoint{4.572157in}{4.356552in}}%
\pgfpathclose%
\pgfusepath{fill}%
\end{pgfscope}%
\begin{pgfscope}%
\pgfpathrectangle{\pgfqpoint{0.539299in}{0.078740in}}{\pgfqpoint{7.842520in}{7.842520in}}%
\pgfusepath{clip}%
\pgfsetbuttcap%
\pgfsetroundjoin%
\definecolor{currentfill}{rgb}{0.281924,0.089666,0.412415}%
\pgfsetfillcolor{currentfill}%
\pgfsetlinewidth{0.000000pt}%
\definecolor{currentstroke}{rgb}{0.143303,0.669459,0.511215}%
\pgfsetstrokecolor{currentstroke}%
\pgfsetdash{}{0pt}%
\pgfpathmoveto{\pgfqpoint{6.779843in}{3.161531in}}%
\pgfpathlineto{\pgfqpoint{6.917735in}{3.083093in}}%
\pgfpathlineto{\pgfqpoint{6.987917in}{3.113178in}}%
\pgfpathclose%
\pgfusepath{fill}%
\end{pgfscope}%
\begin{pgfscope}%
\pgfpathrectangle{\pgfqpoint{0.539299in}{0.078740in}}{\pgfqpoint{7.842520in}{7.842520in}}%
\pgfusepath{clip}%
\pgfsetbuttcap%
\pgfsetroundjoin%
\definecolor{currentfill}{rgb}{0.120565,0.596422,0.543611}%
\pgfsetfillcolor{currentfill}%
\pgfsetlinewidth{0.000000pt}%
\definecolor{currentstroke}{rgb}{0.146616,0.673050,0.508936}%
\pgfsetstrokecolor{currentstroke}%
\pgfsetdash{}{0pt}%
\pgfpathmoveto{\pgfqpoint{3.761962in}{5.020676in}}%
\pgfpathlineto{\pgfqpoint{3.679940in}{5.091602in}}%
\pgfpathlineto{\pgfqpoint{3.895878in}{4.988126in}}%
\pgfpathclose%
\pgfusepath{fill}%
\end{pgfscope}%
\begin{pgfscope}%
\pgfpathrectangle{\pgfqpoint{0.539299in}{0.078740in}}{\pgfqpoint{7.842520in}{7.842520in}}%
\pgfusepath{clip}%
\pgfsetbuttcap%
\pgfsetroundjoin%
\definecolor{currentfill}{rgb}{0.272594,0.025563,0.353093}%
\pgfsetfillcolor{currentfill}%
\pgfsetlinewidth{0.000000pt}%
\definecolor{currentstroke}{rgb}{0.150148,0.676631,0.506589}%
\pgfsetstrokecolor{currentstroke}%
\pgfsetdash{}{0pt}%
\pgfpathmoveto{\pgfqpoint{7.264065in}{2.962827in}}%
\pgfpathlineto{\pgfqpoint{7.402749in}{2.891078in}}%
\pgfpathlineto{\pgfqpoint{7.472249in}{2.944684in}}%
\pgfpathclose%
\pgfusepath{fill}%
\end{pgfscope}%
\begin{pgfscope}%
\pgfpathrectangle{\pgfqpoint{0.539299in}{0.078740in}}{\pgfqpoint{7.842520in}{7.842520in}}%
\pgfusepath{clip}%
\pgfsetbuttcap%
\pgfsetroundjoin%
\definecolor{currentfill}{rgb}{0.163625,0.471133,0.558148}%
\pgfsetfillcolor{currentfill}%
\pgfsetlinewidth{0.000000pt}%
\definecolor{currentstroke}{rgb}{0.153894,0.680203,0.504172}%
\pgfsetstrokecolor{currentstroke}%
\pgfsetdash{}{0pt}%
\pgfpathmoveto{\pgfqpoint{2.954036in}{4.737238in}}%
\pgfpathlineto{\pgfqpoint{2.913780in}{4.395967in}}%
\pgfpathlineto{\pgfqpoint{2.830306in}{4.368134in}}%
\pgfpathclose%
\pgfusepath{fill}%
\end{pgfscope}%
\begin{pgfscope}%
\pgfpathrectangle{\pgfqpoint{0.539299in}{0.078740in}}{\pgfqpoint{7.842520in}{7.842520in}}%
\pgfusepath{clip}%
\pgfsetbuttcap%
\pgfsetroundjoin%
\definecolor{currentfill}{rgb}{0.206756,0.371758,0.553117}%
\pgfsetfillcolor{currentfill}%
\pgfsetlinewidth{0.000000pt}%
\definecolor{currentstroke}{rgb}{0.157851,0.683765,0.501686}%
\pgfsetstrokecolor{currentstroke}%
\pgfsetdash{}{0pt}%
\pgfpathmoveto{\pgfqpoint{4.843266in}{4.057477in}}%
\pgfpathlineto{\pgfqpoint{4.919892in}{3.989315in}}%
\pgfpathlineto{\pgfqpoint{4.707691in}{4.203491in}}%
\pgfpathclose%
\pgfusepath{fill}%
\end{pgfscope}%
\begin{pgfscope}%
\pgfpathrectangle{\pgfqpoint{0.539299in}{0.078740in}}{\pgfqpoint{7.842520in}{7.842520in}}%
\pgfusepath{clip}%
\pgfsetbuttcap%
\pgfsetroundjoin%
\definecolor{currentfill}{rgb}{0.252194,0.269783,0.531579}%
\pgfsetfillcolor{currentfill}%
\pgfsetlinewidth{0.000000pt}%
\definecolor{currentstroke}{rgb}{0.162016,0.687316,0.499129}%
\pgfsetstrokecolor{currentstroke}%
\pgfsetdash{}{0pt}%
\pgfpathmoveto{\pgfqpoint{5.250882in}{3.689752in}}%
\pgfpathlineto{\pgfqpoint{5.326196in}{3.667457in}}%
\pgfpathlineto{\pgfqpoint{5.190465in}{3.761377in}}%
\pgfpathclose%
\pgfusepath{fill}%
\end{pgfscope}%
\begin{pgfscope}%
\pgfpathrectangle{\pgfqpoint{0.539299in}{0.078740in}}{\pgfqpoint{7.842520in}{7.842520in}}%
\pgfusepath{clip}%
\pgfsetbuttcap%
\pgfsetroundjoin%
\definecolor{currentfill}{rgb}{0.216210,0.351535,0.550627}%
\pgfsetfillcolor{currentfill}%
\pgfsetlinewidth{0.000000pt}%
\definecolor{currentstroke}{rgb}{0.166383,0.690856,0.496502}%
\pgfsetstrokecolor{currentstroke}%
\pgfsetdash{}{0pt}%
\pgfpathmoveto{\pgfqpoint{4.978938in}{3.922091in}}%
\pgfpathlineto{\pgfqpoint{4.919892in}{3.989315in}}%
\pgfpathlineto{\pgfqpoint{4.843266in}{4.057477in}}%
\pgfpathclose%
\pgfusepath{fill}%
\end{pgfscope}%
\begin{pgfscope}%
\pgfpathrectangle{\pgfqpoint{0.539299in}{0.078740in}}{\pgfqpoint{7.842520in}{7.842520in}}%
\pgfusepath{clip}%
\pgfsetbuttcap%
\pgfsetroundjoin%
\definecolor{currentfill}{rgb}{0.195860,0.395433,0.555276}%
\pgfsetfillcolor{currentfill}%
\pgfsetlinewidth{0.000000pt}%
\definecolor{currentstroke}{rgb}{0.170948,0.694384,0.493803}%
\pgfsetstrokecolor{currentstroke}%
\pgfsetdash{}{0pt}%
\pgfpathmoveto{\pgfqpoint{2.661974in}{4.301274in}}%
\pgfpathlineto{\pgfqpoint{2.746363in}{4.336760in}}%
\pgfpathlineto{\pgfqpoint{2.627589in}{3.858151in}}%
\pgfpathclose%
\pgfusepath{fill}%
\end{pgfscope}%
\begin{pgfscope}%
\pgfpathrectangle{\pgfqpoint{0.539299in}{0.078740in}}{\pgfqpoint{7.842520in}{7.842520in}}%
\pgfusepath{clip}%
\pgfsetbuttcap%
\pgfsetroundjoin%
\definecolor{currentfill}{rgb}{0.278012,0.180367,0.486697}%
\pgfsetfillcolor{currentfill}%
\pgfsetlinewidth{0.000000pt}%
\definecolor{currentstroke}{rgb}{0.175707,0.697900,0.491033}%
\pgfsetstrokecolor{currentstroke}%
\pgfsetdash{}{0pt}%
\pgfpathmoveto{\pgfqpoint{6.084512in}{3.361457in}}%
\pgfpathlineto{\pgfqpoint{6.157532in}{3.392366in}}%
\pgfpathlineto{\pgfqpoint{5.946992in}{3.415751in}}%
\pgfpathclose%
\pgfusepath{fill}%
\end{pgfscope}%
\begin{pgfscope}%
\pgfpathrectangle{\pgfqpoint{0.539299in}{0.078740in}}{\pgfqpoint{7.842520in}{7.842520in}}%
\pgfusepath{clip}%
\pgfsetbuttcap%
\pgfsetroundjoin%
\definecolor{currentfill}{rgb}{0.268510,0.009605,0.335427}%
\pgfsetfillcolor{currentfill}%
\pgfsetlinewidth{0.000000pt}%
\definecolor{currentstroke}{rgb}{0.180653,0.701402,0.488189}%
\pgfsetstrokecolor{currentstroke}%
\pgfsetdash{}{0pt}%
\pgfpathmoveto{\pgfqpoint{7.472249in}{2.944684in}}%
\pgfpathlineto{\pgfqpoint{7.541973in}{2.824378in}}%
\pgfpathlineto{\pgfqpoint{7.611514in}{2.887087in}}%
\pgfpathclose%
\pgfusepath{fill}%
\end{pgfscope}%
\begin{pgfscope}%
\pgfpathrectangle{\pgfqpoint{0.539299in}{0.078740in}}{\pgfqpoint{7.842520in}{7.842520in}}%
\pgfusepath{clip}%
\pgfsetbuttcap%
\pgfsetroundjoin%
\definecolor{currentfill}{rgb}{0.221989,0.339161,0.548752}%
\pgfsetfillcolor{currentfill}%
\pgfsetlinewidth{0.000000pt}%
\definecolor{currentstroke}{rgb}{0.185783,0.704891,0.485273}%
\pgfsetstrokecolor{currentstroke}%
\pgfsetdash{}{0pt}%
\pgfpathmoveto{\pgfqpoint{2.543987in}{3.805632in}}%
\pgfpathlineto{\pgfqpoint{2.459954in}{3.750514in}}%
\pgfpathlineto{\pgfqpoint{2.661974in}{4.301274in}}%
\pgfpathclose%
\pgfusepath{fill}%
\end{pgfscope}%
\begin{pgfscope}%
\pgfpathrectangle{\pgfqpoint{0.539299in}{0.078740in}}{\pgfqpoint{7.842520in}{7.842520in}}%
\pgfusepath{clip}%
\pgfsetbuttcap%
\pgfsetroundjoin%
\definecolor{currentfill}{rgb}{0.270595,0.214069,0.507052}%
\pgfsetfillcolor{currentfill}%
\pgfsetlinewidth{0.000000pt}%
\definecolor{currentstroke}{rgb}{0.191090,0.708366,0.482284}%
\pgfsetstrokecolor{currentstroke}%
\pgfsetdash{}{0pt}%
\pgfpathmoveto{\pgfqpoint{5.673111in}{3.524234in}}%
\pgfpathlineto{\pgfqpoint{5.598833in}{3.513257in}}%
\pgfpathlineto{\pgfqpoint{5.735793in}{3.448039in}}%
\pgfpathclose%
\pgfusepath{fill}%
\end{pgfscope}%
\begin{pgfscope}%
\pgfpathrectangle{\pgfqpoint{0.539299in}{0.078740in}}{\pgfqpoint{7.842520in}{7.842520in}}%
\pgfusepath{clip}%
\pgfsetbuttcap%
\pgfsetroundjoin%
\definecolor{currentfill}{rgb}{0.282623,0.140926,0.457517}%
\pgfsetfillcolor{currentfill}%
\pgfsetlinewidth{0.000000pt}%
\definecolor{currentstroke}{rgb}{0.196571,0.711827,0.479221}%
\pgfsetstrokecolor{currentstroke}%
\pgfsetdash{}{0pt}%
\pgfpathmoveto{\pgfqpoint{6.504552in}{3.303089in}}%
\pgfpathlineto{\pgfqpoint{6.432810in}{3.275195in}}%
\pgfpathlineto{\pgfqpoint{6.570734in}{3.206547in}}%
\pgfpathclose%
\pgfusepath{fill}%
\end{pgfscope}%
\begin{pgfscope}%
\pgfpathrectangle{\pgfqpoint{0.539299in}{0.078740in}}{\pgfqpoint{7.842520in}{7.842520in}}%
\pgfusepath{clip}%
\pgfsetbuttcap%
\pgfsetroundjoin%
\definecolor{currentfill}{rgb}{0.280255,0.165693,0.476498}%
\pgfsetfillcolor{currentfill}%
\pgfsetlinewidth{0.000000pt}%
\definecolor{currentstroke}{rgb}{0.202219,0.715272,0.476084}%
\pgfsetstrokecolor{currentstroke}%
\pgfsetdash{}{0pt}%
\pgfpathmoveto{\pgfqpoint{6.222345in}{3.303711in}}%
\pgfpathlineto{\pgfqpoint{6.295060in}{3.336953in}}%
\pgfpathlineto{\pgfqpoint{6.157532in}{3.392366in}}%
\pgfpathclose%
\pgfusepath{fill}%
\end{pgfscope}%
\begin{pgfscope}%
\pgfpathrectangle{\pgfqpoint{0.539299in}{0.078740in}}{\pgfqpoint{7.842520in}{7.842520in}}%
\pgfusepath{clip}%
\pgfsetbuttcap%
\pgfsetroundjoin%
\definecolor{currentfill}{rgb}{0.231674,0.318106,0.544834}%
\pgfsetfillcolor{currentfill}%
\pgfsetlinewidth{0.000000pt}%
\definecolor{currentstroke}{rgb}{0.208030,0.718701,0.472873}%
\pgfsetstrokecolor{currentstroke}%
\pgfsetdash{}{0pt}%
\pgfpathmoveto{\pgfqpoint{5.055056in}{3.868593in}}%
\pgfpathlineto{\pgfqpoint{4.978938in}{3.922091in}}%
\pgfpathlineto{\pgfqpoint{5.114785in}{3.799326in}}%
\pgfpathclose%
\pgfusepath{fill}%
\end{pgfscope}%
\begin{pgfscope}%
\pgfpathrectangle{\pgfqpoint{0.539299in}{0.078740in}}{\pgfqpoint{7.842520in}{7.842520in}}%
\pgfusepath{clip}%
\pgfsetbuttcap%
\pgfsetroundjoin%
\definecolor{currentfill}{rgb}{0.274128,0.199721,0.498911}%
\pgfsetfillcolor{currentfill}%
\pgfsetlinewidth{0.000000pt}%
\definecolor{currentstroke}{rgb}{0.214000,0.722114,0.469588}%
\pgfsetstrokecolor{currentstroke}%
\pgfsetdash{}{0pt}%
\pgfpathmoveto{\pgfqpoint{5.809842in}{3.469040in}}%
\pgfpathlineto{\pgfqpoint{5.735793in}{3.448039in}}%
\pgfpathlineto{\pgfqpoint{5.946992in}{3.415751in}}%
\pgfpathclose%
\pgfusepath{fill}%
\end{pgfscope}%
\begin{pgfscope}%
\pgfpathrectangle{\pgfqpoint{0.539299in}{0.078740in}}{\pgfqpoint{7.842520in}{7.842520in}}%
\pgfusepath{clip}%
\pgfsetbuttcap%
\pgfsetroundjoin%
\definecolor{currentfill}{rgb}{0.269944,0.014625,0.341379}%
\pgfsetfillcolor{currentfill}%
\pgfsetlinewidth{0.000000pt}%
\definecolor{currentstroke}{rgb}{0.220124,0.725509,0.466226}%
\pgfsetstrokecolor{currentstroke}%
\pgfsetdash{}{0pt}%
\pgfpathmoveto{\pgfqpoint{7.472249in}{2.944684in}}%
\pgfpathlineto{\pgfqpoint{7.402749in}{2.891078in}}%
\pgfpathlineto{\pgfqpoint{7.541973in}{2.824378in}}%
\pgfpathclose%
\pgfusepath{fill}%
\end{pgfscope}%
\begin{pgfscope}%
\pgfpathrectangle{\pgfqpoint{0.539299in}{0.078740in}}{\pgfqpoint{7.842520in}{7.842520in}}%
\pgfusepath{clip}%
\pgfsetbuttcap%
\pgfsetroundjoin%
\definecolor{currentfill}{rgb}{0.277018,0.050344,0.375715}%
\pgfsetfillcolor{currentfill}%
\pgfsetlinewidth{0.000000pt}%
\definecolor{currentstroke}{rgb}{0.226397,0.728888,0.462789}%
\pgfsetstrokecolor{currentstroke}%
\pgfsetdash{}{0pt}%
\pgfpathmoveto{\pgfqpoint{7.125818in}{3.037569in}}%
\pgfpathlineto{\pgfqpoint{7.194201in}{2.919934in}}%
\pgfpathlineto{\pgfqpoint{7.264065in}{2.962827in}}%
\pgfpathclose%
\pgfusepath{fill}%
\end{pgfscope}%
\begin{pgfscope}%
\pgfpathrectangle{\pgfqpoint{0.539299in}{0.078740in}}{\pgfqpoint{7.842520in}{7.842520in}}%
\pgfusepath{clip}%
\pgfsetbuttcap%
\pgfsetroundjoin%
\definecolor{currentfill}{rgb}{0.280267,0.073417,0.397163}%
\pgfsetfillcolor{currentfill}%
\pgfsetlinewidth{0.000000pt}%
\definecolor{currentstroke}{rgb}{0.232815,0.732247,0.459277}%
\pgfsetstrokecolor{currentstroke}%
\pgfsetdash{}{0pt}%
\pgfpathmoveto{\pgfqpoint{6.917735in}{3.083093in}}%
\pgfpathlineto{\pgfqpoint{7.055833in}{3.001854in}}%
\pgfpathlineto{\pgfqpoint{7.125818in}{3.037569in}}%
\pgfpathclose%
\pgfusepath{fill}%
\end{pgfscope}%
\begin{pgfscope}%
\pgfpathrectangle{\pgfqpoint{0.539299in}{0.078740in}}{\pgfqpoint{7.842520in}{7.842520in}}%
\pgfusepath{clip}%
\pgfsetbuttcap%
\pgfsetroundjoin%
\definecolor{currentfill}{rgb}{0.126453,0.570633,0.549841}%
\pgfsetfillcolor{currentfill}%
\pgfsetlinewidth{0.000000pt}%
\definecolor{currentstroke}{rgb}{0.239374,0.735588,0.455688}%
\pgfsetstrokecolor{currentstroke}%
\pgfsetdash{}{0pt}%
\pgfpathmoveto{\pgfqpoint{3.121655in}{4.747134in}}%
\pgfpathlineto{\pgfqpoint{3.249818in}{4.985619in}}%
\pgfpathlineto{\pgfqpoint{3.333165in}{4.962328in}}%
\pgfpathclose%
\pgfusepath{fill}%
\end{pgfscope}%
\begin{pgfscope}%
\pgfpathrectangle{\pgfqpoint{0.539299in}{0.078740in}}{\pgfqpoint{7.842520in}{7.842520in}}%
\pgfusepath{clip}%
\pgfsetbuttcap%
\pgfsetroundjoin%
\definecolor{currentfill}{rgb}{0.283197,0.115680,0.436115}%
\pgfsetfillcolor{currentfill}%
\pgfsetlinewidth{0.000000pt}%
\definecolor{currentstroke}{rgb}{0.246070,0.738910,0.452024}%
\pgfsetstrokecolor{currentstroke}%
\pgfsetdash{}{0pt}%
\pgfpathmoveto{\pgfqpoint{6.779843in}{3.161531in}}%
\pgfpathlineto{\pgfqpoint{6.642120in}{3.235328in}}%
\pgfpathlineto{\pgfqpoint{6.708800in}{3.131236in}}%
\pgfpathclose%
\pgfusepath{fill}%
\end{pgfscope}%
\begin{pgfscope}%
\pgfpathrectangle{\pgfqpoint{0.539299in}{0.078740in}}{\pgfqpoint{7.842520in}{7.842520in}}%
\pgfusepath{clip}%
\pgfsetbuttcap%
\pgfsetroundjoin%
\definecolor{currentfill}{rgb}{0.258965,0.251537,0.524736}%
\pgfsetfillcolor{currentfill}%
\pgfsetlinewidth{0.000000pt}%
\definecolor{currentstroke}{rgb}{0.252899,0.742211,0.448284}%
\pgfsetstrokecolor{currentstroke}%
\pgfsetdash{}{0pt}%
\pgfpathmoveto{\pgfqpoint{5.387297in}{3.592738in}}%
\pgfpathlineto{\pgfqpoint{5.462306in}{3.585484in}}%
\pgfpathlineto{\pgfqpoint{5.326196in}{3.667457in}}%
\pgfpathclose%
\pgfusepath{fill}%
\end{pgfscope}%
\begin{pgfscope}%
\pgfpathrectangle{\pgfqpoint{0.539299in}{0.078740in}}{\pgfqpoint{7.842520in}{7.842520in}}%
\pgfusepath{clip}%
\pgfsetbuttcap%
\pgfsetroundjoin%
\definecolor{currentfill}{rgb}{0.278826,0.175490,0.483397}%
\pgfsetfillcolor{currentfill}%
\pgfsetlinewidth{0.000000pt}%
\definecolor{currentstroke}{rgb}{0.259857,0.745492,0.444467}%
\pgfsetstrokecolor{currentstroke}%
\pgfsetdash{}{0pt}%
\pgfpathmoveto{\pgfqpoint{6.157532in}{3.392366in}}%
\pgfpathlineto{\pgfqpoint{6.084512in}{3.361457in}}%
\pgfpathlineto{\pgfqpoint{6.222345in}{3.303711in}}%
\pgfpathclose%
\pgfusepath{fill}%
\end{pgfscope}%
\begin{pgfscope}%
\pgfpathrectangle{\pgfqpoint{0.539299in}{0.078740in}}{\pgfqpoint{7.842520in}{7.842520in}}%
\pgfusepath{clip}%
\pgfsetbuttcap%
\pgfsetroundjoin%
\definecolor{currentfill}{rgb}{0.243113,0.292092,0.538516}%
\pgfsetfillcolor{currentfill}%
\pgfsetlinewidth{0.000000pt}%
\definecolor{currentstroke}{rgb}{0.266941,0.748751,0.440573}%
\pgfsetstrokecolor{currentstroke}%
\pgfsetdash{}{0pt}%
\pgfpathmoveto{\pgfqpoint{5.190465in}{3.761377in}}%
\pgfpathlineto{\pgfqpoint{5.114785in}{3.799326in}}%
\pgfpathlineto{\pgfqpoint{5.250882in}{3.689752in}}%
\pgfpathclose%
\pgfusepath{fill}%
\end{pgfscope}%
\begin{pgfscope}%
\pgfpathrectangle{\pgfqpoint{0.539299in}{0.078740in}}{\pgfqpoint{7.842520in}{7.842520in}}%
\pgfusepath{clip}%
\pgfsetbuttcap%
\pgfsetroundjoin%
\definecolor{currentfill}{rgb}{0.119738,0.603785,0.541400}%
\pgfsetfillcolor{currentfill}%
\pgfsetlinewidth{0.000000pt}%
\definecolor{currentstroke}{rgb}{0.274149,0.751988,0.436601}%
\pgfsetstrokecolor{currentstroke}%
\pgfsetdash{}{0pt}%
\pgfpathmoveto{\pgfqpoint{3.333165in}{4.962328in}}%
\pgfpathlineto{\pgfqpoint{3.464190in}{5.097259in}}%
\pgfpathlineto{\pgfqpoint{3.547026in}{5.048670in}}%
\pgfpathclose%
\pgfusepath{fill}%
\end{pgfscope}%
\begin{pgfscope}%
\pgfpathrectangle{\pgfqpoint{0.539299in}{0.078740in}}{\pgfqpoint{7.842520in}{7.842520in}}%
\pgfusepath{clip}%
\pgfsetbuttcap%
\pgfsetroundjoin%
\definecolor{currentfill}{rgb}{0.149039,0.508051,0.557250}%
\pgfsetfillcolor{currentfill}%
\pgfsetlinewidth{0.000000pt}%
\definecolor{currentstroke}{rgb}{0.281477,0.755203,0.432552}%
\pgfsetstrokecolor{currentstroke}%
\pgfsetdash{}{0pt}%
\pgfpathmoveto{\pgfqpoint{2.954036in}{4.737238in}}%
\pgfpathlineto{\pgfqpoint{3.038103in}{4.744324in}}%
\pgfpathlineto{\pgfqpoint{2.913780in}{4.395967in}}%
\pgfpathclose%
\pgfusepath{fill}%
\end{pgfscope}%
\begin{pgfscope}%
\pgfpathrectangle{\pgfqpoint{0.539299in}{0.078740in}}{\pgfqpoint{7.842520in}{7.842520in}}%
\pgfusepath{clip}%
\pgfsetbuttcap%
\pgfsetroundjoin%
\definecolor{currentfill}{rgb}{0.278791,0.062145,0.386592}%
\pgfsetfillcolor{currentfill}%
\pgfsetlinewidth{0.000000pt}%
\definecolor{currentstroke}{rgb}{0.288921,0.758394,0.428426}%
\pgfsetstrokecolor{currentstroke}%
\pgfsetdash{}{0pt}%
\pgfpathmoveto{\pgfqpoint{7.125818in}{3.037569in}}%
\pgfpathlineto{\pgfqpoint{7.055833in}{3.001854in}}%
\pgfpathlineto{\pgfqpoint{7.194201in}{2.919934in}}%
\pgfpathclose%
\pgfusepath{fill}%
\end{pgfscope}%
\begin{pgfscope}%
\pgfpathrectangle{\pgfqpoint{0.539299in}{0.078740in}}{\pgfqpoint{7.842520in}{7.842520in}}%
\pgfusepath{clip}%
\pgfsetbuttcap%
\pgfsetroundjoin%
\definecolor{currentfill}{rgb}{0.274952,0.037752,0.364543}%
\pgfsetfillcolor{currentfill}%
\pgfsetlinewidth{0.000000pt}%
\definecolor{currentstroke}{rgb}{0.296479,0.761561,0.424223}%
\pgfsetstrokecolor{currentstroke}%
\pgfsetdash{}{0pt}%
\pgfpathmoveto{\pgfqpoint{7.264065in}{2.962827in}}%
\pgfpathlineto{\pgfqpoint{7.194201in}{2.919934in}}%
\pgfpathlineto{\pgfqpoint{7.402749in}{2.891078in}}%
\pgfpathclose%
\pgfusepath{fill}%
\end{pgfscope}%
\begin{pgfscope}%
\pgfpathrectangle{\pgfqpoint{0.539299in}{0.078740in}}{\pgfqpoint{7.842520in}{7.842520in}}%
\pgfusepath{clip}%
\pgfsetbuttcap%
\pgfsetroundjoin%
\definecolor{currentfill}{rgb}{0.131172,0.555899,0.552459}%
\pgfsetfillcolor{currentfill}%
\pgfsetlinewidth{0.000000pt}%
\definecolor{currentstroke}{rgb}{0.304148,0.764704,0.419943}%
\pgfsetstrokecolor{currentstroke}%
\pgfsetdash{}{0pt}%
\pgfpathmoveto{\pgfqpoint{3.121655in}{4.747134in}}%
\pgfpathlineto{\pgfqpoint{3.038103in}{4.744324in}}%
\pgfpathlineto{\pgfqpoint{3.249818in}{4.985619in}}%
\pgfpathclose%
\pgfusepath{fill}%
\end{pgfscope}%
\begin{pgfscope}%
\pgfpathrectangle{\pgfqpoint{0.539299in}{0.078740in}}{\pgfqpoint{7.842520in}{7.842520in}}%
\pgfusepath{clip}%
\pgfsetbuttcap%
\pgfsetroundjoin%
\definecolor{currentfill}{rgb}{0.262138,0.242286,0.520837}%
\pgfsetfillcolor{currentfill}%
\pgfsetlinewidth{0.000000pt}%
\definecolor{currentstroke}{rgb}{0.311925,0.767822,0.415586}%
\pgfsetstrokecolor{currentstroke}%
\pgfsetdash{}{0pt}%
\pgfpathmoveto{\pgfqpoint{5.387297in}{3.592738in}}%
\pgfpathlineto{\pgfqpoint{5.598833in}{3.513257in}}%
\pgfpathlineto{\pgfqpoint{5.462306in}{3.585484in}}%
\pgfpathclose%
\pgfusepath{fill}%
\end{pgfscope}%
\begin{pgfscope}%
\pgfpathrectangle{\pgfqpoint{0.539299in}{0.078740in}}{\pgfqpoint{7.842520in}{7.842520in}}%
\pgfusepath{clip}%
\pgfsetbuttcap%
\pgfsetroundjoin%
\definecolor{currentfill}{rgb}{0.283072,0.130895,0.449241}%
\pgfsetfillcolor{currentfill}%
\pgfsetlinewidth{0.000000pt}%
\definecolor{currentstroke}{rgb}{0.319809,0.770914,0.411152}%
\pgfsetstrokecolor{currentstroke}%
\pgfsetdash{}{0pt}%
\pgfpathmoveto{\pgfqpoint{6.708800in}{3.131236in}}%
\pgfpathlineto{\pgfqpoint{6.642120in}{3.235328in}}%
\pgfpathlineto{\pgfqpoint{6.570734in}{3.206547in}}%
\pgfpathclose%
\pgfusepath{fill}%
\end{pgfscope}%
\begin{pgfscope}%
\pgfpathrectangle{\pgfqpoint{0.539299in}{0.078740in}}{\pgfqpoint{7.842520in}{7.842520in}}%
\pgfusepath{clip}%
\pgfsetbuttcap%
\pgfsetroundjoin%
\definecolor{currentfill}{rgb}{0.121148,0.592739,0.544641}%
\pgfsetfillcolor{currentfill}%
\pgfsetlinewidth{0.000000pt}%
\definecolor{currentstroke}{rgb}{0.327796,0.773980,0.406640}%
\pgfsetstrokecolor{currentstroke}%
\pgfsetdash{}{0pt}%
\pgfpathmoveto{\pgfqpoint{3.949347in}{5.006539in}}%
\pgfpathlineto{\pgfqpoint{4.030569in}{4.909838in}}%
\pgfpathlineto{\pgfqpoint{3.895878in}{4.988126in}}%
\pgfpathclose%
\pgfusepath{fill}%
\end{pgfscope}%
\begin{pgfscope}%
\pgfpathrectangle{\pgfqpoint{0.539299in}{0.078740in}}{\pgfqpoint{7.842520in}{7.842520in}}%
\pgfusepath{clip}%
\pgfsetbuttcap%
\pgfsetroundjoin%
\definecolor{currentfill}{rgb}{0.280868,0.160771,0.472899}%
\pgfsetfillcolor{currentfill}%
\pgfsetlinewidth{0.000000pt}%
\definecolor{currentstroke}{rgb}{0.335885,0.777018,0.402049}%
\pgfsetstrokecolor{currentstroke}%
\pgfsetdash{}{0pt}%
\pgfpathmoveto{\pgfqpoint{6.432810in}{3.275195in}}%
\pgfpathlineto{\pgfqpoint{6.295060in}{3.336953in}}%
\pgfpathlineto{\pgfqpoint{6.360427in}{3.240693in}}%
\pgfpathclose%
\pgfusepath{fill}%
\end{pgfscope}%
\begin{pgfscope}%
\pgfpathrectangle{\pgfqpoint{0.539299in}{0.078740in}}{\pgfqpoint{7.842520in}{7.842520in}}%
\pgfusepath{clip}%
\pgfsetbuttcap%
\pgfsetroundjoin%
\definecolor{currentfill}{rgb}{0.252194,0.269783,0.531579}%
\pgfsetfillcolor{currentfill}%
\pgfsetlinewidth{0.000000pt}%
\definecolor{currentstroke}{rgb}{0.344074,0.780029,0.397381}%
\pgfsetstrokecolor{currentstroke}%
\pgfsetdash{}{0pt}%
\pgfpathmoveto{\pgfqpoint{5.326196in}{3.667457in}}%
\pgfpathlineto{\pgfqpoint{5.250882in}{3.689752in}}%
\pgfpathlineto{\pgfqpoint{5.387297in}{3.592738in}}%
\pgfpathclose%
\pgfusepath{fill}%
\end{pgfscope}%
\begin{pgfscope}%
\pgfpathrectangle{\pgfqpoint{0.539299in}{0.078740in}}{\pgfqpoint{7.842520in}{7.842520in}}%
\pgfusepath{clip}%
\pgfsetbuttcap%
\pgfsetroundjoin%
\definecolor{currentfill}{rgb}{0.119699,0.618490,0.536347}%
\pgfsetfillcolor{currentfill}%
\pgfsetlinewidth{0.000000pt}%
\definecolor{currentstroke}{rgb}{0.352360,0.783011,0.392636}%
\pgfsetstrokecolor{currentstroke}%
\pgfsetdash{}{0pt}%
\pgfpathmoveto{\pgfqpoint{3.547026in}{5.048670in}}%
\pgfpathlineto{\pgfqpoint{3.464190in}{5.097259in}}%
\pgfpathlineto{\pgfqpoint{3.679940in}{5.091602in}}%
\pgfpathclose%
\pgfusepath{fill}%
\end{pgfscope}%
\begin{pgfscope}%
\pgfpathrectangle{\pgfqpoint{0.539299in}{0.078740in}}{\pgfqpoint{7.842520in}{7.842520in}}%
\pgfusepath{clip}%
\pgfsetbuttcap%
\pgfsetroundjoin%
\definecolor{currentfill}{rgb}{0.119423,0.611141,0.538982}%
\pgfsetfillcolor{currentfill}%
\pgfsetlinewidth{0.000000pt}%
\definecolor{currentstroke}{rgb}{0.360741,0.785964,0.387814}%
\pgfsetstrokecolor{currentstroke}%
\pgfsetdash{}{0pt}%
\pgfpathmoveto{\pgfqpoint{3.895878in}{4.988126in}}%
\pgfpathlineto{\pgfqpoint{3.679940in}{5.091602in}}%
\pgfpathlineto{\pgfqpoint{3.814186in}{5.073908in}}%
\pgfpathclose%
\pgfusepath{fill}%
\end{pgfscope}%
\begin{pgfscope}%
\pgfpathrectangle{\pgfqpoint{0.539299in}{0.078740in}}{\pgfqpoint{7.842520in}{7.842520in}}%
\pgfusepath{clip}%
\pgfsetbuttcap%
\pgfsetroundjoin%
\definecolor{currentfill}{rgb}{0.282656,0.100196,0.422160}%
\pgfsetfillcolor{currentfill}%
\pgfsetlinewidth{0.000000pt}%
\definecolor{currentstroke}{rgb}{0.369214,0.788888,0.382914}%
\pgfsetstrokecolor{currentstroke}%
\pgfsetdash{}{0pt}%
\pgfpathmoveto{\pgfqpoint{6.847001in}{3.050168in}}%
\pgfpathlineto{\pgfqpoint{6.917735in}{3.083093in}}%
\pgfpathlineto{\pgfqpoint{6.779843in}{3.161531in}}%
\pgfpathclose%
\pgfusepath{fill}%
\end{pgfscope}%
\begin{pgfscope}%
\pgfpathrectangle{\pgfqpoint{0.539299in}{0.078740in}}{\pgfqpoint{7.842520in}{7.842520in}}%
\pgfusepath{clip}%
\pgfsetbuttcap%
\pgfsetroundjoin%
\definecolor{currentfill}{rgb}{0.273006,0.204520,0.501721}%
\pgfsetfillcolor{currentfill}%
\pgfsetlinewidth{0.000000pt}%
\definecolor{currentstroke}{rgb}{0.377779,0.791781,0.377939}%
\pgfsetstrokecolor{currentstroke}%
\pgfsetdash{}{0pt}%
\pgfpathmoveto{\pgfqpoint{5.946992in}{3.415751in}}%
\pgfpathlineto{\pgfqpoint{5.735793in}{3.448039in}}%
\pgfpathlineto{\pgfqpoint{5.873171in}{3.386859in}}%
\pgfpathclose%
\pgfusepath{fill}%
\end{pgfscope}%
\begin{pgfscope}%
\pgfpathrectangle{\pgfqpoint{0.539299in}{0.078740in}}{\pgfqpoint{7.842520in}{7.842520in}}%
\pgfusepath{clip}%
\pgfsetbuttcap%
\pgfsetroundjoin%
\definecolor{currentfill}{rgb}{0.159194,0.482237,0.558073}%
\pgfsetfillcolor{currentfill}%
\pgfsetlinewidth{0.000000pt}%
\definecolor{currentstroke}{rgb}{0.386433,0.794644,0.372886}%
\pgfsetstrokecolor{currentstroke}%
\pgfsetdash{}{0pt}%
\pgfpathmoveto{\pgfqpoint{2.830306in}{4.368134in}}%
\pgfpathlineto{\pgfqpoint{2.746363in}{4.336760in}}%
\pgfpathlineto{\pgfqpoint{2.954036in}{4.737238in}}%
\pgfpathclose%
\pgfusepath{fill}%
\end{pgfscope}%
\begin{pgfscope}%
\pgfpathrectangle{\pgfqpoint{0.539299in}{0.078740in}}{\pgfqpoint{7.842520in}{7.842520in}}%
\pgfusepath{clip}%
\pgfsetbuttcap%
\pgfsetroundjoin%
\definecolor{currentfill}{rgb}{0.276194,0.190074,0.493001}%
\pgfsetfillcolor{currentfill}%
\pgfsetlinewidth{0.000000pt}%
\definecolor{currentstroke}{rgb}{0.395174,0.797475,0.367757}%
\pgfsetstrokecolor{currentstroke}%
\pgfsetdash{}{0pt}%
\pgfpathmoveto{\pgfqpoint{6.010937in}{3.326788in}}%
\pgfpathlineto{\pgfqpoint{6.084512in}{3.361457in}}%
\pgfpathlineto{\pgfqpoint{5.946992in}{3.415751in}}%
\pgfpathclose%
\pgfusepath{fill}%
\end{pgfscope}%
\begin{pgfscope}%
\pgfpathrectangle{\pgfqpoint{0.539299in}{0.078740in}}{\pgfqpoint{7.842520in}{7.842520in}}%
\pgfusepath{clip}%
\pgfsetbuttcap%
\pgfsetroundjoin%
\definecolor{currentfill}{rgb}{0.280255,0.165693,0.476498}%
\pgfsetfillcolor{currentfill}%
\pgfsetlinewidth{0.000000pt}%
\definecolor{currentstroke}{rgb}{0.404001,0.800275,0.362552}%
\pgfsetstrokecolor{currentstroke}%
\pgfsetdash{}{0pt}%
\pgfpathmoveto{\pgfqpoint{6.360427in}{3.240693in}}%
\pgfpathlineto{\pgfqpoint{6.295060in}{3.336953in}}%
\pgfpathlineto{\pgfqpoint{6.222345in}{3.303711in}}%
\pgfpathclose%
\pgfusepath{fill}%
\end{pgfscope}%
\begin{pgfscope}%
\pgfpathrectangle{\pgfqpoint{0.539299in}{0.078740in}}{\pgfqpoint{7.842520in}{7.842520in}}%
\pgfusepath{clip}%
\pgfsetbuttcap%
\pgfsetroundjoin%
\definecolor{currentfill}{rgb}{0.126453,0.570633,0.549841}%
\pgfsetfillcolor{currentfill}%
\pgfsetlinewidth{0.000000pt}%
\definecolor{currentstroke}{rgb}{0.412913,0.803041,0.357269}%
\pgfsetstrokecolor{currentstroke}%
\pgfsetdash{}{0pt}%
\pgfpathmoveto{\pgfqpoint{4.165728in}{4.797221in}}%
\pgfpathlineto{\pgfqpoint{4.030569in}{4.909838in}}%
\pgfpathlineto{\pgfqpoint{4.085083in}{4.900621in}}%
\pgfpathclose%
\pgfusepath{fill}%
\end{pgfscope}%
\begin{pgfscope}%
\pgfpathrectangle{\pgfqpoint{0.539299in}{0.078740in}}{\pgfqpoint{7.842520in}{7.842520in}}%
\pgfusepath{clip}%
\pgfsetbuttcap%
\pgfsetroundjoin%
\definecolor{currentfill}{rgb}{0.281887,0.150881,0.465405}%
\pgfsetfillcolor{currentfill}%
\pgfsetlinewidth{0.000000pt}%
\definecolor{currentstroke}{rgb}{0.421908,0.805774,0.351910}%
\pgfsetstrokecolor{currentstroke}%
\pgfsetdash{}{0pt}%
\pgfpathmoveto{\pgfqpoint{6.570734in}{3.206547in}}%
\pgfpathlineto{\pgfqpoint{6.432810in}{3.275195in}}%
\pgfpathlineto{\pgfqpoint{6.360427in}{3.240693in}}%
\pgfpathclose%
\pgfusepath{fill}%
\end{pgfscope}%
\begin{pgfscope}%
\pgfpathrectangle{\pgfqpoint{0.539299in}{0.078740in}}{\pgfqpoint{7.842520in}{7.842520in}}%
\pgfusepath{clip}%
\pgfsetbuttcap%
\pgfsetroundjoin%
\definecolor{currentfill}{rgb}{0.281924,0.089666,0.412415}%
\pgfsetfillcolor{currentfill}%
\pgfsetlinewidth{0.000000pt}%
\definecolor{currentstroke}{rgb}{0.430983,0.808473,0.346476}%
\pgfsetstrokecolor{currentstroke}%
\pgfsetdash{}{0pt}%
\pgfpathmoveto{\pgfqpoint{6.847001in}{3.050168in}}%
\pgfpathlineto{\pgfqpoint{7.055833in}{3.001854in}}%
\pgfpathlineto{\pgfqpoint{6.917735in}{3.083093in}}%
\pgfpathclose%
\pgfusepath{fill}%
\end{pgfscope}%
\begin{pgfscope}%
\pgfpathrectangle{\pgfqpoint{0.539299in}{0.078740in}}{\pgfqpoint{7.842520in}{7.842520in}}%
\pgfusepath{clip}%
\pgfsetbuttcap%
\pgfsetroundjoin%
\definecolor{currentfill}{rgb}{0.283197,0.115680,0.436115}%
\pgfsetfillcolor{currentfill}%
\pgfsetlinewidth{0.000000pt}%
\definecolor{currentstroke}{rgb}{0.440137,0.811138,0.340967}%
\pgfsetstrokecolor{currentstroke}%
\pgfsetdash{}{0pt}%
\pgfpathmoveto{\pgfqpoint{6.779843in}{3.161531in}}%
\pgfpathlineto{\pgfqpoint{6.708800in}{3.131236in}}%
\pgfpathlineto{\pgfqpoint{6.847001in}{3.050168in}}%
\pgfpathclose%
\pgfusepath{fill}%
\end{pgfscope}%
\begin{pgfscope}%
\pgfpathrectangle{\pgfqpoint{0.539299in}{0.078740in}}{\pgfqpoint{7.842520in}{7.842520in}}%
\pgfusepath{clip}%
\pgfsetbuttcap%
\pgfsetroundjoin%
\definecolor{currentfill}{rgb}{0.266580,0.228262,0.514349}%
\pgfsetfillcolor{currentfill}%
\pgfsetlinewidth{0.000000pt}%
\definecolor{currentstroke}{rgb}{0.449368,0.813768,0.335384}%
\pgfsetstrokecolor{currentstroke}%
\pgfsetdash{}{0pt}%
\pgfpathmoveto{\pgfqpoint{5.735793in}{3.448039in}}%
\pgfpathlineto{\pgfqpoint{5.598833in}{3.513257in}}%
\pgfpathlineto{\pgfqpoint{5.524083in}{3.506700in}}%
\pgfpathclose%
\pgfusepath{fill}%
\end{pgfscope}%
\begin{pgfscope}%
\pgfpathrectangle{\pgfqpoint{0.539299in}{0.078740in}}{\pgfqpoint{7.842520in}{7.842520in}}%
\pgfusepath{clip}%
\pgfsetbuttcap%
\pgfsetroundjoin%
\definecolor{currentfill}{rgb}{0.131172,0.555899,0.552459}%
\pgfsetfillcolor{currentfill}%
\pgfsetlinewidth{0.000000pt}%
\definecolor{currentstroke}{rgb}{0.458674,0.816363,0.329727}%
\pgfsetstrokecolor{currentstroke}%
\pgfsetdash{}{0pt}%
\pgfpathmoveto{\pgfqpoint{4.085083in}{4.900621in}}%
\pgfpathlineto{\pgfqpoint{4.301131in}{4.661130in}}%
\pgfpathlineto{\pgfqpoint{4.165728in}{4.797221in}}%
\pgfpathclose%
\pgfusepath{fill}%
\end{pgfscope}%
\begin{pgfscope}%
\pgfpathrectangle{\pgfqpoint{0.539299in}{0.078740in}}{\pgfqpoint{7.842520in}{7.842520in}}%
\pgfusepath{clip}%
\pgfsetbuttcap%
\pgfsetroundjoin%
\definecolor{currentfill}{rgb}{0.119423,0.611141,0.538982}%
\pgfsetfillcolor{currentfill}%
\pgfsetlinewidth{0.000000pt}%
\definecolor{currentstroke}{rgb}{0.468053,0.818921,0.323998}%
\pgfsetstrokecolor{currentstroke}%
\pgfsetdash{}{0pt}%
\pgfpathmoveto{\pgfqpoint{3.814186in}{5.073908in}}%
\pgfpathlineto{\pgfqpoint{3.949347in}{5.006539in}}%
\pgfpathlineto{\pgfqpoint{3.895878in}{4.988126in}}%
\pgfpathclose%
\pgfusepath{fill}%
\end{pgfscope}%
\begin{pgfscope}%
\pgfpathrectangle{\pgfqpoint{0.539299in}{0.078740in}}{\pgfqpoint{7.842520in}{7.842520in}}%
\pgfusepath{clip}%
\pgfsetbuttcap%
\pgfsetroundjoin%
\definecolor{currentfill}{rgb}{0.273809,0.031497,0.358853}%
\pgfsetfillcolor{currentfill}%
\pgfsetlinewidth{0.000000pt}%
\definecolor{currentstroke}{rgb}{0.477504,0.821444,0.318195}%
\pgfsetstrokecolor{currentstroke}%
\pgfsetdash{}{0pt}%
\pgfpathmoveto{\pgfqpoint{7.402749in}{2.891078in}}%
\pgfpathlineto{\pgfqpoint{7.194201in}{2.919934in}}%
\pgfpathlineto{\pgfqpoint{7.332923in}{2.839586in}}%
\pgfpathclose%
\pgfusepath{fill}%
\end{pgfscope}%
\begin{pgfscope}%
\pgfpathrectangle{\pgfqpoint{0.539299in}{0.078740in}}{\pgfqpoint{7.842520in}{7.842520in}}%
\pgfusepath{clip}%
\pgfsetbuttcap%
\pgfsetroundjoin%
\definecolor{currentfill}{rgb}{0.275191,0.194905,0.496005}%
\pgfsetfillcolor{currentfill}%
\pgfsetlinewidth{0.000000pt}%
\definecolor{currentstroke}{rgb}{0.487026,0.823929,0.312321}%
\pgfsetstrokecolor{currentstroke}%
\pgfsetdash{}{0pt}%
\pgfpathmoveto{\pgfqpoint{5.946992in}{3.415751in}}%
\pgfpathlineto{\pgfqpoint{5.873171in}{3.386859in}}%
\pgfpathlineto{\pgfqpoint{6.010937in}{3.326788in}}%
\pgfpathclose%
\pgfusepath{fill}%
\end{pgfscope}%
\begin{pgfscope}%
\pgfpathrectangle{\pgfqpoint{0.539299in}{0.078740in}}{\pgfqpoint{7.842520in}{7.842520in}}%
\pgfusepath{clip}%
\pgfsetbuttcap%
\pgfsetroundjoin%
\definecolor{currentfill}{rgb}{0.147607,0.511733,0.557049}%
\pgfsetfillcolor{currentfill}%
\pgfsetlinewidth{0.000000pt}%
\definecolor{currentstroke}{rgb}{0.496615,0.826376,0.306377}%
\pgfsetstrokecolor{currentstroke}%
\pgfsetdash{}{0pt}%
\pgfpathmoveto{\pgfqpoint{4.301131in}{4.661130in}}%
\pgfpathlineto{\pgfqpoint{4.357325in}{4.615401in}}%
\pgfpathlineto{\pgfqpoint{4.436631in}{4.511408in}}%
\pgfpathclose%
\pgfusepath{fill}%
\end{pgfscope}%
\begin{pgfscope}%
\pgfpathrectangle{\pgfqpoint{0.539299in}{0.078740in}}{\pgfqpoint{7.842520in}{7.842520in}}%
\pgfusepath{clip}%
\pgfsetbuttcap%
\pgfsetroundjoin%
\definecolor{currentfill}{rgb}{0.162142,0.474838,0.558140}%
\pgfsetfillcolor{currentfill}%
\pgfsetlinewidth{0.000000pt}%
\definecolor{currentstroke}{rgb}{0.506271,0.828786,0.300362}%
\pgfsetstrokecolor{currentstroke}%
\pgfsetdash{}{0pt}%
\pgfpathmoveto{\pgfqpoint{4.493544in}{4.454843in}}%
\pgfpathlineto{\pgfqpoint{4.572157in}{4.356552in}}%
\pgfpathlineto{\pgfqpoint{4.436631in}{4.511408in}}%
\pgfpathclose%
\pgfusepath{fill}%
\end{pgfscope}%
\begin{pgfscope}%
\pgfpathrectangle{\pgfqpoint{0.539299in}{0.078740in}}{\pgfqpoint{7.842520in}{7.842520in}}%
\pgfusepath{clip}%
\pgfsetbuttcap%
\pgfsetroundjoin%
\definecolor{currentfill}{rgb}{0.171176,0.452530,0.557965}%
\pgfsetfillcolor{currentfill}%
\pgfsetlinewidth{0.000000pt}%
\definecolor{currentstroke}{rgb}{0.515992,0.831158,0.294279}%
\pgfsetstrokecolor{currentstroke}%
\pgfsetdash{}{0pt}%
\pgfpathmoveto{\pgfqpoint{4.572157in}{4.356552in}}%
\pgfpathlineto{\pgfqpoint{4.493544in}{4.454843in}}%
\pgfpathlineto{\pgfqpoint{4.707691in}{4.203491in}}%
\pgfpathclose%
\pgfusepath{fill}%
\end{pgfscope}%
\begin{pgfscope}%
\pgfpathrectangle{\pgfqpoint{0.539299in}{0.078740in}}{\pgfqpoint{7.842520in}{7.842520in}}%
\pgfusepath{clip}%
\pgfsetbuttcap%
\pgfsetroundjoin%
\definecolor{currentfill}{rgb}{0.192357,0.403199,0.555836}%
\pgfsetfillcolor{currentfill}%
\pgfsetlinewidth{0.000000pt}%
\definecolor{currentstroke}{rgb}{0.525776,0.833491,0.288127}%
\pgfsetstrokecolor{currentstroke}%
\pgfsetdash{}{0pt}%
\pgfpathmoveto{\pgfqpoint{4.707691in}{4.203491in}}%
\pgfpathlineto{\pgfqpoint{4.765949in}{4.134630in}}%
\pgfpathlineto{\pgfqpoint{4.843266in}{4.057477in}}%
\pgfpathclose%
\pgfusepath{fill}%
\end{pgfscope}%
\begin{pgfscope}%
\pgfpathrectangle{\pgfqpoint{0.539299in}{0.078740in}}{\pgfqpoint{7.842520in}{7.842520in}}%
\pgfusepath{clip}%
\pgfsetbuttcap%
\pgfsetroundjoin%
\definecolor{currentfill}{rgb}{0.269944,0.014625,0.341379}%
\pgfsetfillcolor{currentfill}%
\pgfsetlinewidth{0.000000pt}%
\definecolor{currentstroke}{rgb}{0.535621,0.835785,0.281908}%
\pgfsetstrokecolor{currentstroke}%
\pgfsetdash{}{0pt}%
\pgfpathmoveto{\pgfqpoint{7.472102in}{2.763114in}}%
\pgfpathlineto{\pgfqpoint{7.541973in}{2.824378in}}%
\pgfpathlineto{\pgfqpoint{7.402749in}{2.891078in}}%
\pgfpathclose%
\pgfusepath{fill}%
\end{pgfscope}%
\begin{pgfscope}%
\pgfpathrectangle{\pgfqpoint{0.539299in}{0.078740in}}{\pgfqpoint{7.842520in}{7.842520in}}%
\pgfusepath{clip}%
\pgfsetbuttcap%
\pgfsetroundjoin%
\definecolor{currentfill}{rgb}{0.218130,0.347432,0.550038}%
\pgfsetfillcolor{currentfill}%
\pgfsetlinewidth{0.000000pt}%
\definecolor{currentstroke}{rgb}{0.545524,0.838039,0.275626}%
\pgfsetstrokecolor{currentstroke}%
\pgfsetdash{}{0pt}%
\pgfpathmoveto{\pgfqpoint{2.577164in}{4.260977in}}%
\pgfpathlineto{\pgfqpoint{2.459954in}{3.750514in}}%
\pgfpathlineto{\pgfqpoint{2.375504in}{3.692390in}}%
\pgfpathclose%
\pgfusepath{fill}%
\end{pgfscope}%
\begin{pgfscope}%
\pgfpathrectangle{\pgfqpoint{0.539299in}{0.078740in}}{\pgfqpoint{7.842520in}{7.842520in}}%
\pgfusepath{clip}%
\pgfsetbuttcap%
\pgfsetroundjoin%
\definecolor{currentfill}{rgb}{0.260571,0.246922,0.522828}%
\pgfsetfillcolor{currentfill}%
\pgfsetlinewidth{0.000000pt}%
\definecolor{currentstroke}{rgb}{0.555484,0.840254,0.269281}%
\pgfsetstrokecolor{currentstroke}%
\pgfsetdash{}{0pt}%
\pgfpathmoveto{\pgfqpoint{5.524083in}{3.506700in}}%
\pgfpathlineto{\pgfqpoint{5.598833in}{3.513257in}}%
\pgfpathlineto{\pgfqpoint{5.387297in}{3.592738in}}%
\pgfpathclose%
\pgfusepath{fill}%
\end{pgfscope}%
\begin{pgfscope}%
\pgfpathrectangle{\pgfqpoint{0.539299in}{0.078740in}}{\pgfqpoint{7.842520in}{7.842520in}}%
\pgfusepath{clip}%
\pgfsetbuttcap%
\pgfsetroundjoin%
\definecolor{currentfill}{rgb}{0.203063,0.379716,0.553925}%
\pgfsetfillcolor{currentfill}%
\pgfsetlinewidth{0.000000pt}%
\definecolor{currentstroke}{rgb}{0.565498,0.842430,0.262877}%
\pgfsetstrokecolor{currentstroke}%
\pgfsetdash{}{0pt}%
\pgfpathmoveto{\pgfqpoint{4.765949in}{4.134630in}}%
\pgfpathlineto{\pgfqpoint{4.978938in}{3.922091in}}%
\pgfpathlineto{\pgfqpoint{4.843266in}{4.057477in}}%
\pgfpathclose%
\pgfusepath{fill}%
\end{pgfscope}%
\begin{pgfscope}%
\pgfpathrectangle{\pgfqpoint{0.539299in}{0.078740in}}{\pgfqpoint{7.842520in}{7.842520in}}%
\pgfusepath{clip}%
\pgfsetbuttcap%
\pgfsetroundjoin%
\definecolor{currentfill}{rgb}{0.218130,0.347432,0.550038}%
\pgfsetfillcolor{currentfill}%
\pgfsetlinewidth{0.000000pt}%
\definecolor{currentstroke}{rgb}{0.575563,0.844566,0.256415}%
\pgfsetstrokecolor{currentstroke}%
\pgfsetdash{}{0pt}%
\pgfpathmoveto{\pgfqpoint{4.978938in}{3.922091in}}%
\pgfpathlineto{\pgfqpoint{4.902184in}{3.985093in}}%
\pgfpathlineto{\pgfqpoint{5.114785in}{3.799326in}}%
\pgfpathclose%
\pgfusepath{fill}%
\end{pgfscope}%
\begin{pgfscope}%
\pgfpathrectangle{\pgfqpoint{0.539299in}{0.078740in}}{\pgfqpoint{7.842520in}{7.842520in}}%
\pgfusepath{clip}%
\pgfsetbuttcap%
\pgfsetroundjoin%
\definecolor{currentfill}{rgb}{0.271305,0.019942,0.347269}%
\pgfsetfillcolor{currentfill}%
\pgfsetlinewidth{0.000000pt}%
\definecolor{currentstroke}{rgb}{0.585678,0.846661,0.249897}%
\pgfsetstrokecolor{currentstroke}%
\pgfsetdash{}{0pt}%
\pgfpathmoveto{\pgfqpoint{7.402749in}{2.891078in}}%
\pgfpathlineto{\pgfqpoint{7.332923in}{2.839586in}}%
\pgfpathlineto{\pgfqpoint{7.472102in}{2.763114in}}%
\pgfpathclose%
\pgfusepath{fill}%
\end{pgfscope}%
\begin{pgfscope}%
\pgfpathrectangle{\pgfqpoint{0.539299in}{0.078740in}}{\pgfqpoint{7.842520in}{7.842520in}}%
\pgfusepath{clip}%
\pgfsetbuttcap%
\pgfsetroundjoin%
\definecolor{currentfill}{rgb}{0.278012,0.180367,0.486697}%
\pgfsetfillcolor{currentfill}%
\pgfsetlinewidth{0.000000pt}%
\definecolor{currentstroke}{rgb}{0.595839,0.848717,0.243329}%
\pgfsetstrokecolor{currentstroke}%
\pgfsetdash{}{0pt}%
\pgfpathmoveto{\pgfqpoint{6.084512in}{3.361457in}}%
\pgfpathlineto{\pgfqpoint{6.149041in}{3.265181in}}%
\pgfpathlineto{\pgfqpoint{6.222345in}{3.303711in}}%
\pgfpathclose%
\pgfusepath{fill}%
\end{pgfscope}%
\begin{pgfscope}%
\pgfpathrectangle{\pgfqpoint{0.539299in}{0.078740in}}{\pgfqpoint{7.842520in}{7.842520in}}%
\pgfusepath{clip}%
\pgfsetbuttcap%
\pgfsetroundjoin%
\definecolor{currentfill}{rgb}{0.282623,0.140926,0.457517}%
\pgfsetfillcolor{currentfill}%
\pgfsetlinewidth{0.000000pt}%
\definecolor{currentstroke}{rgb}{0.606045,0.850733,0.236712}%
\pgfsetstrokecolor{currentstroke}%
\pgfsetdash{}{0pt}%
\pgfpathmoveto{\pgfqpoint{6.570734in}{3.206547in}}%
\pgfpathlineto{\pgfqpoint{6.498702in}{3.171293in}}%
\pgfpathlineto{\pgfqpoint{6.708800in}{3.131236in}}%
\pgfpathclose%
\pgfusepath{fill}%
\end{pgfscope}%
\begin{pgfscope}%
\pgfpathrectangle{\pgfqpoint{0.539299in}{0.078740in}}{\pgfqpoint{7.842520in}{7.842520in}}%
\pgfusepath{clip}%
\pgfsetbuttcap%
\pgfsetroundjoin%
\definecolor{currentfill}{rgb}{0.120565,0.596422,0.543611}%
\pgfsetfillcolor{currentfill}%
\pgfsetlinewidth{0.000000pt}%
\definecolor{currentstroke}{rgb}{0.616293,0.852709,0.230052}%
\pgfsetstrokecolor{currentstroke}%
\pgfsetdash{}{0pt}%
\pgfpathmoveto{\pgfqpoint{4.085083in}{4.900621in}}%
\pgfpathlineto{\pgfqpoint{4.030569in}{4.909838in}}%
\pgfpathlineto{\pgfqpoint{3.949347in}{5.006539in}}%
\pgfpathclose%
\pgfusepath{fill}%
\end{pgfscope}%
\begin{pgfscope}%
\pgfpathrectangle{\pgfqpoint{0.539299in}{0.078740in}}{\pgfqpoint{7.842520in}{7.842520in}}%
\pgfusepath{clip}%
\pgfsetbuttcap%
\pgfsetroundjoin%
\definecolor{currentfill}{rgb}{0.120638,0.625828,0.533488}%
\pgfsetfillcolor{currentfill}%
\pgfsetlinewidth{0.000000pt}%
\definecolor{currentstroke}{rgb}{0.626579,0.854645,0.223353}%
\pgfsetstrokecolor{currentstroke}%
\pgfsetdash{}{0pt}%
\pgfpathmoveto{\pgfqpoint{3.464190in}{5.097259in}}%
\pgfpathlineto{\pgfqpoint{3.333165in}{4.962328in}}%
\pgfpathlineto{\pgfqpoint{3.380711in}{5.141597in}}%
\pgfpathclose%
\pgfusepath{fill}%
\end{pgfscope}%
\begin{pgfscope}%
\pgfpathrectangle{\pgfqpoint{0.539299in}{0.078740in}}{\pgfqpoint{7.842520in}{7.842520in}}%
\pgfusepath{clip}%
\pgfsetbuttcap%
\pgfsetroundjoin%
\definecolor{currentfill}{rgb}{0.278791,0.062145,0.386592}%
\pgfsetfillcolor{currentfill}%
\pgfsetlinewidth{0.000000pt}%
\definecolor{currentstroke}{rgb}{0.636902,0.856542,0.216620}%
\pgfsetstrokecolor{currentstroke}%
\pgfsetdash{}{0pt}%
\pgfpathmoveto{\pgfqpoint{7.055833in}{3.001854in}}%
\pgfpathlineto{\pgfqpoint{7.123901in}{2.876974in}}%
\pgfpathlineto{\pgfqpoint{7.194201in}{2.919934in}}%
\pgfpathclose%
\pgfusepath{fill}%
\end{pgfscope}%
\begin{pgfscope}%
\pgfpathrectangle{\pgfqpoint{0.539299in}{0.078740in}}{\pgfqpoint{7.842520in}{7.842520in}}%
\pgfusepath{clip}%
\pgfsetbuttcap%
\pgfsetroundjoin%
\definecolor{currentfill}{rgb}{0.119699,0.618490,0.536347}%
\pgfsetfillcolor{currentfill}%
\pgfsetlinewidth{0.000000pt}%
\definecolor{currentstroke}{rgb}{0.647257,0.858400,0.209861}%
\pgfsetstrokecolor{currentstroke}%
\pgfsetdash{}{0pt}%
\pgfpathmoveto{\pgfqpoint{3.380711in}{5.141597in}}%
\pgfpathlineto{\pgfqpoint{3.333165in}{4.962328in}}%
\pgfpathlineto{\pgfqpoint{3.249818in}{4.985619in}}%
\pgfpathclose%
\pgfusepath{fill}%
\end{pgfscope}%
\begin{pgfscope}%
\pgfpathrectangle{\pgfqpoint{0.539299in}{0.078740in}}{\pgfqpoint{7.842520in}{7.842520in}}%
\pgfusepath{clip}%
\pgfsetbuttcap%
\pgfsetroundjoin%
\definecolor{currentfill}{rgb}{0.281924,0.089666,0.412415}%
\pgfsetfillcolor{currentfill}%
\pgfsetlinewidth{0.000000pt}%
\definecolor{currentstroke}{rgb}{0.657642,0.860219,0.203082}%
\pgfsetstrokecolor{currentstroke}%
\pgfsetdash{}{0pt}%
\pgfpathmoveto{\pgfqpoint{6.985354in}{2.964798in}}%
\pgfpathlineto{\pgfqpoint{7.055833in}{3.001854in}}%
\pgfpathlineto{\pgfqpoint{6.847001in}{3.050168in}}%
\pgfpathclose%
\pgfusepath{fill}%
\end{pgfscope}%
\begin{pgfscope}%
\pgfpathrectangle{\pgfqpoint{0.539299in}{0.078740in}}{\pgfqpoint{7.842520in}{7.842520in}}%
\pgfusepath{clip}%
\pgfsetbuttcap%
\pgfsetroundjoin%
\definecolor{currentfill}{rgb}{0.281412,0.155834,0.469201}%
\pgfsetfillcolor{currentfill}%
\pgfsetlinewidth{0.000000pt}%
\definecolor{currentstroke}{rgb}{0.668054,0.861999,0.196293}%
\pgfsetstrokecolor{currentstroke}%
\pgfsetdash{}{0pt}%
\pgfpathmoveto{\pgfqpoint{6.360427in}{3.240693in}}%
\pgfpathlineto{\pgfqpoint{6.498702in}{3.171293in}}%
\pgfpathlineto{\pgfqpoint{6.570734in}{3.206547in}}%
\pgfpathclose%
\pgfusepath{fill}%
\end{pgfscope}%
\begin{pgfscope}%
\pgfpathrectangle{\pgfqpoint{0.539299in}{0.078740in}}{\pgfqpoint{7.842520in}{7.842520in}}%
\pgfusepath{clip}%
\pgfsetbuttcap%
\pgfsetroundjoin%
\definecolor{currentfill}{rgb}{0.235526,0.309527,0.542944}%
\pgfsetfillcolor{currentfill}%
\pgfsetlinewidth{0.000000pt}%
\definecolor{currentstroke}{rgb}{0.678489,0.863742,0.189503}%
\pgfsetstrokecolor{currentstroke}%
\pgfsetdash{}{0pt}%
\pgfpathmoveto{\pgfqpoint{5.175033in}{3.721001in}}%
\pgfpathlineto{\pgfqpoint{5.250882in}{3.689752in}}%
\pgfpathlineto{\pgfqpoint{5.114785in}{3.799326in}}%
\pgfpathclose%
\pgfusepath{fill}%
\end{pgfscope}%
\begin{pgfscope}%
\pgfpathrectangle{\pgfqpoint{0.539299in}{0.078740in}}{\pgfqpoint{7.842520in}{7.842520in}}%
\pgfusepath{clip}%
\pgfsetbuttcap%
\pgfsetroundjoin%
\definecolor{currentfill}{rgb}{0.192357,0.403199,0.555836}%
\pgfsetfillcolor{currentfill}%
\pgfsetlinewidth{0.000000pt}%
\definecolor{currentstroke}{rgb}{0.688944,0.865448,0.182725}%
\pgfsetstrokecolor{currentstroke}%
\pgfsetdash{}{0pt}%
\pgfpathmoveto{\pgfqpoint{2.459954in}{3.750514in}}%
\pgfpathlineto{\pgfqpoint{2.577164in}{4.260977in}}%
\pgfpathlineto{\pgfqpoint{2.661974in}{4.301274in}}%
\pgfpathclose%
\pgfusepath{fill}%
\end{pgfscope}%
\begin{pgfscope}%
\pgfpathrectangle{\pgfqpoint{0.539299in}{0.078740in}}{\pgfqpoint{7.842520in}{7.842520in}}%
\pgfusepath{clip}%
\pgfsetbuttcap%
\pgfsetroundjoin%
\definecolor{currentfill}{rgb}{0.265145,0.232956,0.516599}%
\pgfsetfillcolor{currentfill}%
\pgfsetlinewidth{0.000000pt}%
\definecolor{currentstroke}{rgb}{0.699415,0.867117,0.175971}%
\pgfsetstrokecolor{currentstroke}%
\pgfsetdash{}{0pt}%
\pgfpathmoveto{\pgfqpoint{5.524083in}{3.506700in}}%
\pgfpathlineto{\pgfqpoint{5.661270in}{3.429383in}}%
\pgfpathlineto{\pgfqpoint{5.735793in}{3.448039in}}%
\pgfpathclose%
\pgfusepath{fill}%
\end{pgfscope}%
\begin{pgfscope}%
\pgfpathrectangle{\pgfqpoint{0.539299in}{0.078740in}}{\pgfqpoint{7.842520in}{7.842520in}}%
\pgfusepath{clip}%
\pgfsetbuttcap%
\pgfsetroundjoin%
\definecolor{currentfill}{rgb}{0.124780,0.640461,0.527068}%
\pgfsetfillcolor{currentfill}%
\pgfsetlinewidth{0.000000pt}%
\definecolor{currentstroke}{rgb}{0.709898,0.868751,0.169257}%
\pgfsetstrokecolor{currentstroke}%
\pgfsetdash{}{0pt}%
\pgfpathmoveto{\pgfqpoint{3.679940in}{5.091602in}}%
\pgfpathlineto{\pgfqpoint{3.464190in}{5.097259in}}%
\pgfpathlineto{\pgfqpoint{3.597214in}{5.159015in}}%
\pgfpathclose%
\pgfusepath{fill}%
\end{pgfscope}%
\begin{pgfscope}%
\pgfpathrectangle{\pgfqpoint{0.539299in}{0.078740in}}{\pgfqpoint{7.842520in}{7.842520in}}%
\pgfusepath{clip}%
\pgfsetbuttcap%
\pgfsetroundjoin%
\definecolor{currentfill}{rgb}{0.270595,0.214069,0.507052}%
\pgfsetfillcolor{currentfill}%
\pgfsetlinewidth{0.000000pt}%
\definecolor{currentstroke}{rgb}{0.720391,0.870350,0.162603}%
\pgfsetstrokecolor{currentstroke}%
\pgfsetdash{}{0pt}%
\pgfpathmoveto{\pgfqpoint{5.798863in}{3.358134in}}%
\pgfpathlineto{\pgfqpoint{5.873171in}{3.386859in}}%
\pgfpathlineto{\pgfqpoint{5.735793in}{3.448039in}}%
\pgfpathclose%
\pgfusepath{fill}%
\end{pgfscope}%
\begin{pgfscope}%
\pgfpathrectangle{\pgfqpoint{0.539299in}{0.078740in}}{\pgfqpoint{7.842520in}{7.842520in}}%
\pgfusepath{clip}%
\pgfsetbuttcap%
\pgfsetroundjoin%
\definecolor{currentfill}{rgb}{0.276194,0.190074,0.493001}%
\pgfsetfillcolor{currentfill}%
\pgfsetlinewidth{0.000000pt}%
\definecolor{currentstroke}{rgb}{0.730889,0.871916,0.156029}%
\pgfsetstrokecolor{currentstroke}%
\pgfsetdash{}{0pt}%
\pgfpathmoveto{\pgfqpoint{6.010937in}{3.326788in}}%
\pgfpathlineto{\pgfqpoint{6.149041in}{3.265181in}}%
\pgfpathlineto{\pgfqpoint{6.084512in}{3.361457in}}%
\pgfpathclose%
\pgfusepath{fill}%
\end{pgfscope}%
\begin{pgfscope}%
\pgfpathrectangle{\pgfqpoint{0.539299in}{0.078740in}}{\pgfqpoint{7.842520in}{7.842520in}}%
\pgfusepath{clip}%
\pgfsetbuttcap%
\pgfsetroundjoin%
\definecolor{currentfill}{rgb}{0.128729,0.563265,0.551229}%
\pgfsetfillcolor{currentfill}%
\pgfsetlinewidth{0.000000pt}%
\definecolor{currentstroke}{rgb}{0.741388,0.873449,0.149561}%
\pgfsetstrokecolor{currentstroke}%
\pgfsetdash{}{0pt}%
\pgfpathmoveto{\pgfqpoint{4.221135in}{4.766924in}}%
\pgfpathlineto{\pgfqpoint{4.301131in}{4.661130in}}%
\pgfpathlineto{\pgfqpoint{4.085083in}{4.900621in}}%
\pgfpathclose%
\pgfusepath{fill}%
\end{pgfscope}%
\begin{pgfscope}%
\pgfpathrectangle{\pgfqpoint{0.539299in}{0.078740in}}{\pgfqpoint{7.842520in}{7.842520in}}%
\pgfusepath{clip}%
\pgfsetbuttcap%
\pgfsetroundjoin%
\definecolor{currentfill}{rgb}{0.243113,0.292092,0.538516}%
\pgfsetfillcolor{currentfill}%
\pgfsetlinewidth{0.000000pt}%
\definecolor{currentstroke}{rgb}{0.751884,0.874951,0.143228}%
\pgfsetstrokecolor{currentstroke}%
\pgfsetdash{}{0pt}%
\pgfpathmoveto{\pgfqpoint{5.250882in}{3.689752in}}%
\pgfpathlineto{\pgfqpoint{5.175033in}{3.721001in}}%
\pgfpathlineto{\pgfqpoint{5.387297in}{3.592738in}}%
\pgfpathclose%
\pgfusepath{fill}%
\end{pgfscope}%
\begin{pgfscope}%
\pgfpathrectangle{\pgfqpoint{0.539299in}{0.078740in}}{\pgfqpoint{7.842520in}{7.842520in}}%
\pgfusepath{clip}%
\pgfsetbuttcap%
\pgfsetroundjoin%
\definecolor{currentfill}{rgb}{0.278826,0.175490,0.483397}%
\pgfsetfillcolor{currentfill}%
\pgfsetlinewidth{0.000000pt}%
\definecolor{currentstroke}{rgb}{0.762373,0.876424,0.137064}%
\pgfsetstrokecolor{currentstroke}%
\pgfsetdash{}{0pt}%
\pgfpathmoveto{\pgfqpoint{6.222345in}{3.303711in}}%
\pgfpathlineto{\pgfqpoint{6.149041in}{3.265181in}}%
\pgfpathlineto{\pgfqpoint{6.360427in}{3.240693in}}%
\pgfpathclose%
\pgfusepath{fill}%
\end{pgfscope}%
\begin{pgfscope}%
\pgfpathrectangle{\pgfqpoint{0.539299in}{0.078740in}}{\pgfqpoint{7.842520in}{7.842520in}}%
\pgfusepath{clip}%
\pgfsetbuttcap%
\pgfsetroundjoin%
\definecolor{currentfill}{rgb}{0.121148,0.592739,0.544641}%
\pgfsetfillcolor{currentfill}%
\pgfsetlinewidth{0.000000pt}%
\definecolor{currentstroke}{rgb}{0.772852,0.877868,0.131109}%
\pgfsetstrokecolor{currentstroke}%
\pgfsetdash{}{0pt}%
\pgfpathmoveto{\pgfqpoint{3.249818in}{4.985619in}}%
\pgfpathlineto{\pgfqpoint{3.038103in}{4.744324in}}%
\pgfpathlineto{\pgfqpoint{3.165894in}{5.004406in}}%
\pgfpathclose%
\pgfusepath{fill}%
\end{pgfscope}%
\begin{pgfscope}%
\pgfpathrectangle{\pgfqpoint{0.539299in}{0.078740in}}{\pgfqpoint{7.842520in}{7.842520in}}%
\pgfusepath{clip}%
\pgfsetbuttcap%
\pgfsetroundjoin%
\definecolor{currentfill}{rgb}{0.136408,0.541173,0.554483}%
\pgfsetfillcolor{currentfill}%
\pgfsetlinewidth{0.000000pt}%
\definecolor{currentstroke}{rgb}{0.783315,0.879285,0.125405}%
\pgfsetstrokecolor{currentstroke}%
\pgfsetdash{}{0pt}%
\pgfpathmoveto{\pgfqpoint{4.221135in}{4.766924in}}%
\pgfpathlineto{\pgfqpoint{4.357325in}{4.615401in}}%
\pgfpathlineto{\pgfqpoint{4.301131in}{4.661130in}}%
\pgfpathclose%
\pgfusepath{fill}%
\end{pgfscope}%
\begin{pgfscope}%
\pgfpathrectangle{\pgfqpoint{0.539299in}{0.078740in}}{\pgfqpoint{7.842520in}{7.842520in}}%
\pgfusepath{clip}%
\pgfsetbuttcap%
\pgfsetroundjoin%
\definecolor{currentfill}{rgb}{0.180629,0.429975,0.557282}%
\pgfsetfillcolor{currentfill}%
\pgfsetlinewidth{0.000000pt}%
\definecolor{currentstroke}{rgb}{0.793760,0.880678,0.120005}%
\pgfsetstrokecolor{currentstroke}%
\pgfsetdash{}{0pt}%
\pgfpathmoveto{\pgfqpoint{4.629750in}{4.292637in}}%
\pgfpathlineto{\pgfqpoint{4.765949in}{4.134630in}}%
\pgfpathlineto{\pgfqpoint{4.707691in}{4.203491in}}%
\pgfpathclose%
\pgfusepath{fill}%
\end{pgfscope}%
\begin{pgfscope}%
\pgfpathrectangle{\pgfqpoint{0.539299in}{0.078740in}}{\pgfqpoint{7.842520in}{7.842520in}}%
\pgfusepath{clip}%
\pgfsetbuttcap%
\pgfsetroundjoin%
\definecolor{currentfill}{rgb}{0.169646,0.456262,0.558030}%
\pgfsetfillcolor{currentfill}%
\pgfsetlinewidth{0.000000pt}%
\definecolor{currentstroke}{rgb}{0.804182,0.882046,0.114965}%
\pgfsetstrokecolor{currentstroke}%
\pgfsetdash{}{0pt}%
\pgfpathmoveto{\pgfqpoint{4.707691in}{4.203491in}}%
\pgfpathlineto{\pgfqpoint{4.493544in}{4.454843in}}%
\pgfpathlineto{\pgfqpoint{4.629750in}{4.292637in}}%
\pgfpathclose%
\pgfusepath{fill}%
\end{pgfscope}%
\begin{pgfscope}%
\pgfpathrectangle{\pgfqpoint{0.539299in}{0.078740in}}{\pgfqpoint{7.842520in}{7.842520in}}%
\pgfusepath{clip}%
\pgfsetbuttcap%
\pgfsetroundjoin%
\definecolor{currentfill}{rgb}{0.277018,0.050344,0.375715}%
\pgfsetfillcolor{currentfill}%
\pgfsetlinewidth{0.000000pt}%
\definecolor{currentstroke}{rgb}{0.814576,0.883393,0.110347}%
\pgfsetstrokecolor{currentstroke}%
\pgfsetdash{}{0pt}%
\pgfpathmoveto{\pgfqpoint{7.332923in}{2.839586in}}%
\pgfpathlineto{\pgfqpoint{7.194201in}{2.919934in}}%
\pgfpathlineto{\pgfqpoint{7.123901in}{2.876974in}}%
\pgfpathclose%
\pgfusepath{fill}%
\end{pgfscope}%
\begin{pgfscope}%
\pgfpathrectangle{\pgfqpoint{0.539299in}{0.078740in}}{\pgfqpoint{7.842520in}{7.842520in}}%
\pgfusepath{clip}%
\pgfsetbuttcap%
\pgfsetroundjoin%
\definecolor{currentfill}{rgb}{0.150476,0.504369,0.557430}%
\pgfsetfillcolor{currentfill}%
\pgfsetlinewidth{0.000000pt}%
\definecolor{currentstroke}{rgb}{0.824940,0.884720,0.106217}%
\pgfsetstrokecolor{currentstroke}%
\pgfsetdash{}{0pt}%
\pgfpathmoveto{\pgfqpoint{4.436631in}{4.511408in}}%
\pgfpathlineto{\pgfqpoint{4.357325in}{4.615401in}}%
\pgfpathlineto{\pgfqpoint{4.493544in}{4.454843in}}%
\pgfpathclose%
\pgfusepath{fill}%
\end{pgfscope}%
\begin{pgfscope}%
\pgfpathrectangle{\pgfqpoint{0.539299in}{0.078740in}}{\pgfqpoint{7.842520in}{7.842520in}}%
\pgfusepath{clip}%
\pgfsetbuttcap%
\pgfsetroundjoin%
\definecolor{currentfill}{rgb}{0.280267,0.073417,0.397163}%
\pgfsetfillcolor{currentfill}%
\pgfsetlinewidth{0.000000pt}%
\definecolor{currentstroke}{rgb}{0.835270,0.886029,0.102646}%
\pgfsetstrokecolor{currentstroke}%
\pgfsetdash{}{0pt}%
\pgfpathmoveto{\pgfqpoint{6.985354in}{2.964798in}}%
\pgfpathlineto{\pgfqpoint{7.123901in}{2.876974in}}%
\pgfpathlineto{\pgfqpoint{7.055833in}{3.001854in}}%
\pgfpathclose%
\pgfusepath{fill}%
\end{pgfscope}%
\begin{pgfscope}%
\pgfpathrectangle{\pgfqpoint{0.539299in}{0.078740in}}{\pgfqpoint{7.842520in}{7.842520in}}%
\pgfusepath{clip}%
\pgfsetbuttcap%
\pgfsetroundjoin%
\definecolor{currentfill}{rgb}{0.201239,0.383670,0.554294}%
\pgfsetfillcolor{currentfill}%
\pgfsetlinewidth{0.000000pt}%
\definecolor{currentstroke}{rgb}{0.845561,0.887322,0.099702}%
\pgfsetstrokecolor{currentstroke}%
\pgfsetdash{}{0pt}%
\pgfpathmoveto{\pgfqpoint{4.765949in}{4.134630in}}%
\pgfpathlineto{\pgfqpoint{4.902184in}{3.985093in}}%
\pgfpathlineto{\pgfqpoint{4.978938in}{3.922091in}}%
\pgfpathclose%
\pgfusepath{fill}%
\end{pgfscope}%
\begin{pgfscope}%
\pgfpathrectangle{\pgfqpoint{0.539299in}{0.078740in}}{\pgfqpoint{7.842520in}{7.842520in}}%
\pgfusepath{clip}%
\pgfsetbuttcap%
\pgfsetroundjoin%
\definecolor{currentfill}{rgb}{0.283187,0.125848,0.444960}%
\pgfsetfillcolor{currentfill}%
\pgfsetlinewidth{0.000000pt}%
\definecolor{currentstroke}{rgb}{0.855810,0.888601,0.097452}%
\pgfsetstrokecolor{currentstroke}%
\pgfsetdash{}{0pt}%
\pgfpathmoveto{\pgfqpoint{6.847001in}{3.050168in}}%
\pgfpathlineto{\pgfqpoint{6.708800in}{3.131236in}}%
\pgfpathlineto{\pgfqpoint{6.637123in}{3.095135in}}%
\pgfpathclose%
\pgfusepath{fill}%
\end{pgfscope}%
\begin{pgfscope}%
\pgfpathrectangle{\pgfqpoint{0.539299in}{0.078740in}}{\pgfqpoint{7.842520in}{7.842520in}}%
\pgfusepath{clip}%
\pgfsetbuttcap%
\pgfsetroundjoin%
\definecolor{currentfill}{rgb}{0.267968,0.223549,0.512008}%
\pgfsetfillcolor{currentfill}%
\pgfsetlinewidth{0.000000pt}%
\definecolor{currentstroke}{rgb}{0.866013,0.889868,0.095953}%
\pgfsetstrokecolor{currentstroke}%
\pgfsetdash{}{0pt}%
\pgfpathmoveto{\pgfqpoint{5.735793in}{3.448039in}}%
\pgfpathlineto{\pgfqpoint{5.661270in}{3.429383in}}%
\pgfpathlineto{\pgfqpoint{5.798863in}{3.358134in}}%
\pgfpathclose%
\pgfusepath{fill}%
\end{pgfscope}%
\begin{pgfscope}%
\pgfpathrectangle{\pgfqpoint{0.539299in}{0.078740in}}{\pgfqpoint{7.842520in}{7.842520in}}%
\pgfusepath{clip}%
\pgfsetbuttcap%
\pgfsetroundjoin%
\definecolor{currentfill}{rgb}{0.124780,0.640461,0.527068}%
\pgfsetfillcolor{currentfill}%
\pgfsetlinewidth{0.000000pt}%
\definecolor{currentstroke}{rgb}{0.876168,0.891125,0.095250}%
\pgfsetstrokecolor{currentstroke}%
\pgfsetdash{}{0pt}%
\pgfpathmoveto{\pgfqpoint{3.679940in}{5.091602in}}%
\pgfpathlineto{\pgfqpoint{3.731742in}{5.157127in}}%
\pgfpathlineto{\pgfqpoint{3.814186in}{5.073908in}}%
\pgfpathclose%
\pgfusepath{fill}%
\end{pgfscope}%
\begin{pgfscope}%
\pgfpathrectangle{\pgfqpoint{0.539299in}{0.078740in}}{\pgfqpoint{7.842520in}{7.842520in}}%
\pgfusepath{clip}%
\pgfsetbuttcap%
\pgfsetroundjoin%
\definecolor{currentfill}{rgb}{0.216210,0.351535,0.550627}%
\pgfsetfillcolor{currentfill}%
\pgfsetlinewidth{0.000000pt}%
\definecolor{currentstroke}{rgb}{0.886271,0.892374,0.095374}%
\pgfsetstrokecolor{currentstroke}%
\pgfsetdash{}{0pt}%
\pgfpathmoveto{\pgfqpoint{5.114785in}{3.799326in}}%
\pgfpathlineto{\pgfqpoint{4.902184in}{3.985093in}}%
\pgfpathlineto{\pgfqpoint{5.038520in}{3.846770in}}%
\pgfpathclose%
\pgfusepath{fill}%
\end{pgfscope}%
\begin{pgfscope}%
\pgfpathrectangle{\pgfqpoint{0.539299in}{0.078740in}}{\pgfqpoint{7.842520in}{7.842520in}}%
\pgfusepath{clip}%
\pgfsetbuttcap%
\pgfsetroundjoin%
\definecolor{currentfill}{rgb}{0.282623,0.140926,0.457517}%
\pgfsetfillcolor{currentfill}%
\pgfsetlinewidth{0.000000pt}%
\definecolor{currentstroke}{rgb}{0.896320,0.893616,0.096335}%
\pgfsetstrokecolor{currentstroke}%
\pgfsetdash{}{0pt}%
\pgfpathmoveto{\pgfqpoint{6.708800in}{3.131236in}}%
\pgfpathlineto{\pgfqpoint{6.498702in}{3.171293in}}%
\pgfpathlineto{\pgfqpoint{6.637123in}{3.095135in}}%
\pgfpathclose%
\pgfusepath{fill}%
\end{pgfscope}%
\begin{pgfscope}%
\pgfpathrectangle{\pgfqpoint{0.539299in}{0.078740in}}{\pgfqpoint{7.842520in}{7.842520in}}%
\pgfusepath{clip}%
\pgfsetbuttcap%
\pgfsetroundjoin%
\definecolor{currentfill}{rgb}{0.124395,0.578002,0.548287}%
\pgfsetfillcolor{currentfill}%
\pgfsetlinewidth{0.000000pt}%
\definecolor{currentstroke}{rgb}{0.906311,0.894855,0.098125}%
\pgfsetstrokecolor{currentstroke}%
\pgfsetdash{}{0pt}%
\pgfpathmoveto{\pgfqpoint{3.165894in}{5.004406in}}%
\pgfpathlineto{\pgfqpoint{3.038103in}{4.744324in}}%
\pgfpathlineto{\pgfqpoint{2.954036in}{4.737238in}}%
\pgfpathclose%
\pgfusepath{fill}%
\end{pgfscope}%
\begin{pgfscope}%
\pgfpathrectangle{\pgfqpoint{0.539299in}{0.078740in}}{\pgfqpoint{7.842520in}{7.842520in}}%
\pgfusepath{clip}%
\pgfsetbuttcap%
\pgfsetroundjoin%
\definecolor{currentfill}{rgb}{0.225863,0.330805,0.547314}%
\pgfsetfillcolor{currentfill}%
\pgfsetlinewidth{0.000000pt}%
\definecolor{currentstroke}{rgb}{0.916242,0.896091,0.100717}%
\pgfsetstrokecolor{currentstroke}%
\pgfsetdash{}{0pt}%
\pgfpathmoveto{\pgfqpoint{5.175033in}{3.721001in}}%
\pgfpathlineto{\pgfqpoint{5.114785in}{3.799326in}}%
\pgfpathlineto{\pgfqpoint{5.038520in}{3.846770in}}%
\pgfpathclose%
\pgfusepath{fill}%
\end{pgfscope}%
\begin{pgfscope}%
\pgfpathrectangle{\pgfqpoint{0.539299in}{0.078740in}}{\pgfqpoint{7.842520in}{7.842520in}}%
\pgfusepath{clip}%
\pgfsetbuttcap%
\pgfsetroundjoin%
\definecolor{currentfill}{rgb}{0.273006,0.204520,0.501721}%
\pgfsetfillcolor{currentfill}%
\pgfsetlinewidth{0.000000pt}%
\definecolor{currentstroke}{rgb}{0.926106,0.897330,0.104071}%
\pgfsetstrokecolor{currentstroke}%
\pgfsetdash{}{0pt}%
\pgfpathmoveto{\pgfqpoint{5.873171in}{3.386859in}}%
\pgfpathlineto{\pgfqpoint{5.936847in}{3.290164in}}%
\pgfpathlineto{\pgfqpoint{6.010937in}{3.326788in}}%
\pgfpathclose%
\pgfusepath{fill}%
\end{pgfscope}%
\begin{pgfscope}%
\pgfpathrectangle{\pgfqpoint{0.539299in}{0.078740in}}{\pgfqpoint{7.842520in}{7.842520in}}%
\pgfusepath{clip}%
\pgfsetbuttcap%
\pgfsetroundjoin%
\definecolor{currentfill}{rgb}{0.253935,0.265254,0.529983}%
\pgfsetfillcolor{currentfill}%
\pgfsetlinewidth{0.000000pt}%
\definecolor{currentstroke}{rgb}{0.935904,0.898570,0.108131}%
\pgfsetstrokecolor{currentstroke}%
\pgfsetdash{}{0pt}%
\pgfpathmoveto{\pgfqpoint{5.387297in}{3.592738in}}%
\pgfpathlineto{\pgfqpoint{5.448865in}{3.506545in}}%
\pgfpathlineto{\pgfqpoint{5.524083in}{3.506700in}}%
\pgfpathclose%
\pgfusepath{fill}%
\end{pgfscope}%
\begin{pgfscope}%
\pgfpathrectangle{\pgfqpoint{0.539299in}{0.078740in}}{\pgfqpoint{7.842520in}{7.842520in}}%
\pgfusepath{clip}%
\pgfsetbuttcap%
\pgfsetroundjoin%
\definecolor{currentfill}{rgb}{0.141935,0.526453,0.555991}%
\pgfsetfillcolor{currentfill}%
\pgfsetlinewidth{0.000000pt}%
\definecolor{currentstroke}{rgb}{0.945636,0.899815,0.112838}%
\pgfsetstrokecolor{currentstroke}%
\pgfsetdash{}{0pt}%
\pgfpathmoveto{\pgfqpoint{2.954036in}{4.737238in}}%
\pgfpathlineto{\pgfqpoint{2.746363in}{4.336760in}}%
\pgfpathlineto{\pgfqpoint{2.869477in}{4.725189in}}%
\pgfpathclose%
\pgfusepath{fill}%
\end{pgfscope}%
\begin{pgfscope}%
\pgfpathrectangle{\pgfqpoint{0.539299in}{0.078740in}}{\pgfqpoint{7.842520in}{7.842520in}}%
\pgfusepath{clip}%
\pgfsetbuttcap%
\pgfsetroundjoin%
\definecolor{currentfill}{rgb}{0.130067,0.651384,0.521608}%
\pgfsetfillcolor{currentfill}%
\pgfsetlinewidth{0.000000pt}%
\definecolor{currentstroke}{rgb}{0.955300,0.901065,0.118128}%
\pgfsetstrokecolor{currentstroke}%
\pgfsetdash{}{0pt}%
\pgfpathmoveto{\pgfqpoint{3.597214in}{5.159015in}}%
\pgfpathlineto{\pgfqpoint{3.731742in}{5.157127in}}%
\pgfpathlineto{\pgfqpoint{3.679940in}{5.091602in}}%
\pgfpathclose%
\pgfusepath{fill}%
\end{pgfscope}%
\begin{pgfscope}%
\pgfpathrectangle{\pgfqpoint{0.539299in}{0.078740in}}{\pgfqpoint{7.842520in}{7.842520in}}%
\pgfusepath{clip}%
\pgfsetbuttcap%
\pgfsetroundjoin%
\definecolor{currentfill}{rgb}{0.271305,0.019942,0.347269}%
\pgfsetfillcolor{currentfill}%
\pgfsetlinewidth{0.000000pt}%
\definecolor{currentstroke}{rgb}{0.964894,0.902323,0.123941}%
\pgfsetstrokecolor{currentstroke}%
\pgfsetdash{}{0pt}%
\pgfpathmoveto{\pgfqpoint{7.472102in}{2.763114in}}%
\pgfpathlineto{\pgfqpoint{7.332923in}{2.839586in}}%
\pgfpathlineto{\pgfqpoint{7.401863in}{2.702549in}}%
\pgfpathclose%
\pgfusepath{fill}%
\end{pgfscope}%
\begin{pgfscope}%
\pgfpathrectangle{\pgfqpoint{0.539299in}{0.078740in}}{\pgfqpoint{7.842520in}{7.842520in}}%
\pgfusepath{clip}%
\pgfsetbuttcap%
\pgfsetroundjoin%
\definecolor{currentfill}{rgb}{0.276022,0.044167,0.370164}%
\pgfsetfillcolor{currentfill}%
\pgfsetlinewidth{0.000000pt}%
\definecolor{currentstroke}{rgb}{0.974417,0.903590,0.130215}%
\pgfsetstrokecolor{currentstroke}%
\pgfsetdash{}{0pt}%
\pgfpathmoveto{\pgfqpoint{7.262707in}{2.788802in}}%
\pgfpathlineto{\pgfqpoint{7.332923in}{2.839586in}}%
\pgfpathlineto{\pgfqpoint{7.123901in}{2.876974in}}%
\pgfpathclose%
\pgfusepath{fill}%
\end{pgfscope}%
\begin{pgfscope}%
\pgfpathrectangle{\pgfqpoint{0.539299in}{0.078740in}}{\pgfqpoint{7.842520in}{7.842520in}}%
\pgfusepath{clip}%
\pgfsetbuttcap%
\pgfsetroundjoin%
\definecolor{currentfill}{rgb}{0.278012,0.180367,0.486697}%
\pgfsetfillcolor{currentfill}%
\pgfsetlinewidth{0.000000pt}%
\definecolor{currentstroke}{rgb}{0.983868,0.904867,0.136897}%
\pgfsetstrokecolor{currentstroke}%
\pgfsetdash{}{0pt}%
\pgfpathmoveto{\pgfqpoint{6.360427in}{3.240693in}}%
\pgfpathlineto{\pgfqpoint{6.149041in}{3.265181in}}%
\pgfpathlineto{\pgfqpoint{6.287425in}{3.199853in}}%
\pgfpathclose%
\pgfusepath{fill}%
\end{pgfscope}%
\begin{pgfscope}%
\pgfpathrectangle{\pgfqpoint{0.539299in}{0.078740in}}{\pgfqpoint{7.842520in}{7.842520in}}%
\pgfusepath{clip}%
\pgfsetbuttcap%
\pgfsetroundjoin%
\definecolor{currentfill}{rgb}{0.216210,0.351535,0.550627}%
\pgfsetfillcolor{currentfill}%
\pgfsetlinewidth{0.000000pt}%
\definecolor{currentstroke}{rgb}{0.993248,0.906157,0.143936}%
\pgfsetstrokecolor{currentstroke}%
\pgfsetdash{}{0pt}%
\pgfpathmoveto{\pgfqpoint{2.375504in}{3.692390in}}%
\pgfpathlineto{\pgfqpoint{2.290658in}{3.630794in}}%
\pgfpathlineto{\pgfqpoint{2.491969in}{4.215044in}}%
\pgfpathclose%
\pgfusepath{fill}%
\end{pgfscope}%
\begin{pgfscope}%
\pgfpathrectangle{\pgfqpoint{0.539299in}{0.078740in}}{\pgfqpoint{7.842520in}{7.842520in}}%
\pgfusepath{clip}%
\pgfsetbuttcap%
\pgfsetroundjoin%
\definecolor{currentfill}{rgb}{0.241237,0.296485,0.539709}%
\pgfsetfillcolor{currentfill}%
\pgfsetlinewidth{0.000000pt}%
\definecolor{currentstroke}{rgb}{0.267004,0.004874,0.329415}%
\pgfsetstrokecolor{currentstroke}%
\pgfsetdash{}{0pt}%
\pgfpathmoveto{\pgfqpoint{5.387297in}{3.592738in}}%
\pgfpathlineto{\pgfqpoint{5.175033in}{3.721001in}}%
\pgfpathlineto{\pgfqpoint{5.311793in}{3.607895in}}%
\pgfpathclose%
\pgfusepath{fill}%
\end{pgfscope}%
\begin{pgfscope}%
\pgfpathrectangle{\pgfqpoint{0.539299in}{0.078740in}}{\pgfqpoint{7.842520in}{7.842520in}}%
\pgfusepath{clip}%
\pgfsetbuttcap%
\pgfsetroundjoin%
\definecolor{currentfill}{rgb}{0.282656,0.100196,0.422160}%
\pgfsetfillcolor{currentfill}%
\pgfsetlinewidth{0.000000pt}%
\definecolor{currentstroke}{rgb}{0.268510,0.009605,0.335427}%
\pgfsetstrokecolor{currentstroke}%
\pgfsetdash{}{0pt}%
\pgfpathmoveto{\pgfqpoint{6.847001in}{3.050168in}}%
\pgfpathlineto{\pgfqpoint{6.914313in}{2.924367in}}%
\pgfpathlineto{\pgfqpoint{6.985354in}{2.964798in}}%
\pgfpathclose%
\pgfusepath{fill}%
\end{pgfscope}%
\begin{pgfscope}%
\pgfpathrectangle{\pgfqpoint{0.539299in}{0.078740in}}{\pgfqpoint{7.842520in}{7.842520in}}%
\pgfusepath{clip}%
\pgfsetbuttcap%
\pgfsetroundjoin%
\definecolor{currentfill}{rgb}{0.280868,0.160771,0.472899}%
\pgfsetfillcolor{currentfill}%
\pgfsetlinewidth{0.000000pt}%
\definecolor{currentstroke}{rgb}{0.269944,0.014625,0.341379}%
\pgfsetstrokecolor{currentstroke}%
\pgfsetdash{}{0pt}%
\pgfpathmoveto{\pgfqpoint{6.426028in}{3.129202in}}%
\pgfpathlineto{\pgfqpoint{6.498702in}{3.171293in}}%
\pgfpathlineto{\pgfqpoint{6.360427in}{3.240693in}}%
\pgfpathclose%
\pgfusepath{fill}%
\end{pgfscope}%
\begin{pgfscope}%
\pgfpathrectangle{\pgfqpoint{0.539299in}{0.078740in}}{\pgfqpoint{7.842520in}{7.842520in}}%
\pgfusepath{clip}%
\pgfsetbuttcap%
\pgfsetroundjoin%
\definecolor{currentfill}{rgb}{0.132268,0.655014,0.519661}%
\pgfsetfillcolor{currentfill}%
\pgfsetlinewidth{0.000000pt}%
\definecolor{currentstroke}{rgb}{0.271305,0.019942,0.347269}%
\pgfsetstrokecolor{currentstroke}%
\pgfsetdash{}{0pt}%
\pgfpathmoveto{\pgfqpoint{3.597214in}{5.159015in}}%
\pgfpathlineto{\pgfqpoint{3.464190in}{5.097259in}}%
\pgfpathlineto{\pgfqpoint{3.380711in}{5.141597in}}%
\pgfpathclose%
\pgfusepath{fill}%
\end{pgfscope}%
\begin{pgfscope}%
\pgfpathrectangle{\pgfqpoint{0.539299in}{0.078740in}}{\pgfqpoint{7.842520in}{7.842520in}}%
\pgfusepath{clip}%
\pgfsetbuttcap%
\pgfsetroundjoin%
\definecolor{currentfill}{rgb}{0.270595,0.214069,0.507052}%
\pgfsetfillcolor{currentfill}%
\pgfsetlinewidth{0.000000pt}%
\definecolor{currentstroke}{rgb}{0.272594,0.025563,0.353093}%
\pgfsetstrokecolor{currentstroke}%
\pgfsetdash{}{0pt}%
\pgfpathmoveto{\pgfqpoint{5.798863in}{3.358134in}}%
\pgfpathlineto{\pgfqpoint{5.936847in}{3.290164in}}%
\pgfpathlineto{\pgfqpoint{5.873171in}{3.386859in}}%
\pgfpathclose%
\pgfusepath{fill}%
\end{pgfscope}%
\begin{pgfscope}%
\pgfpathrectangle{\pgfqpoint{0.539299in}{0.078740in}}{\pgfqpoint{7.842520in}{7.842520in}}%
\pgfusepath{clip}%
\pgfsetbuttcap%
\pgfsetroundjoin%
\definecolor{currentfill}{rgb}{0.283229,0.120777,0.440584}%
\pgfsetfillcolor{currentfill}%
\pgfsetlinewidth{0.000000pt}%
\definecolor{currentstroke}{rgb}{0.273809,0.031497,0.358853}%
\pgfsetstrokecolor{currentstroke}%
\pgfsetdash{}{0pt}%
\pgfpathmoveto{\pgfqpoint{6.637123in}{3.095135in}}%
\pgfpathlineto{\pgfqpoint{6.775662in}{3.012527in}}%
\pgfpathlineto{\pgfqpoint{6.847001in}{3.050168in}}%
\pgfpathclose%
\pgfusepath{fill}%
\end{pgfscope}%
\begin{pgfscope}%
\pgfpathrectangle{\pgfqpoint{0.539299in}{0.078740in}}{\pgfqpoint{7.842520in}{7.842520in}}%
\pgfusepath{clip}%
\pgfsetbuttcap%
\pgfsetroundjoin%
\definecolor{currentfill}{rgb}{0.258965,0.251537,0.524736}%
\pgfsetfillcolor{currentfill}%
\pgfsetlinewidth{0.000000pt}%
\definecolor{currentstroke}{rgb}{0.274952,0.037752,0.364543}%
\pgfsetstrokecolor{currentstroke}%
\pgfsetdash{}{0pt}%
\pgfpathmoveto{\pgfqpoint{5.448865in}{3.506545in}}%
\pgfpathlineto{\pgfqpoint{5.661270in}{3.429383in}}%
\pgfpathlineto{\pgfqpoint{5.524083in}{3.506700in}}%
\pgfpathclose%
\pgfusepath{fill}%
\end{pgfscope}%
\begin{pgfscope}%
\pgfpathrectangle{\pgfqpoint{0.539299in}{0.078740in}}{\pgfqpoint{7.842520in}{7.842520in}}%
\pgfusepath{clip}%
\pgfsetbuttcap%
\pgfsetroundjoin%
\definecolor{currentfill}{rgb}{0.274128,0.199721,0.498911}%
\pgfsetfillcolor{currentfill}%
\pgfsetlinewidth{0.000000pt}%
\definecolor{currentstroke}{rgb}{0.276022,0.044167,0.370164}%
\pgfsetstrokecolor{currentstroke}%
\pgfsetdash{}{0pt}%
\pgfpathmoveto{\pgfqpoint{5.936847in}{3.290164in}}%
\pgfpathlineto{\pgfqpoint{6.149041in}{3.265181in}}%
\pgfpathlineto{\pgfqpoint{6.010937in}{3.326788in}}%
\pgfpathclose%
\pgfusepath{fill}%
\end{pgfscope}%
\begin{pgfscope}%
\pgfpathrectangle{\pgfqpoint{0.539299in}{0.078740in}}{\pgfqpoint{7.842520in}{7.842520in}}%
\pgfusepath{clip}%
\pgfsetbuttcap%
\pgfsetroundjoin%
\definecolor{currentfill}{rgb}{0.124780,0.640461,0.527068}%
\pgfsetfillcolor{currentfill}%
\pgfsetlinewidth{0.000000pt}%
\definecolor{currentstroke}{rgb}{0.277018,0.050344,0.375715}%
\pgfsetstrokecolor{currentstroke}%
\pgfsetdash{}{0pt}%
\pgfpathmoveto{\pgfqpoint{3.867334in}{5.102013in}}%
\pgfpathlineto{\pgfqpoint{3.949347in}{5.006539in}}%
\pgfpathlineto{\pgfqpoint{3.814186in}{5.073908in}}%
\pgfpathclose%
\pgfusepath{fill}%
\end{pgfscope}%
\begin{pgfscope}%
\pgfpathrectangle{\pgfqpoint{0.539299in}{0.078740in}}{\pgfqpoint{7.842520in}{7.842520in}}%
\pgfusepath{clip}%
\pgfsetbuttcap%
\pgfsetroundjoin%
\definecolor{currentfill}{rgb}{0.153364,0.497000,0.557724}%
\pgfsetfillcolor{currentfill}%
\pgfsetlinewidth{0.000000pt}%
\definecolor{currentstroke}{rgb}{0.277941,0.056324,0.381191}%
\pgfsetstrokecolor{currentstroke}%
\pgfsetdash{}{0pt}%
\pgfpathmoveto{\pgfqpoint{2.746363in}{4.336760in}}%
\pgfpathlineto{\pgfqpoint{2.661974in}{4.301274in}}%
\pgfpathlineto{\pgfqpoint{2.784458in}{4.707293in}}%
\pgfpathclose%
\pgfusepath{fill}%
\end{pgfscope}%
\begin{pgfscope}%
\pgfpathrectangle{\pgfqpoint{0.539299in}{0.078740in}}{\pgfqpoint{7.842520in}{7.842520in}}%
\pgfusepath{clip}%
\pgfsetbuttcap%
\pgfsetroundjoin%
\definecolor{currentfill}{rgb}{0.123444,0.636809,0.528763}%
\pgfsetfillcolor{currentfill}%
\pgfsetlinewidth{0.000000pt}%
\definecolor{currentstroke}{rgb}{0.278791,0.062145,0.386592}%
\pgfsetstrokecolor{currentstroke}%
\pgfsetdash{}{0pt}%
\pgfpathmoveto{\pgfqpoint{3.380711in}{5.141597in}}%
\pgfpathlineto{\pgfqpoint{3.249818in}{4.985619in}}%
\pgfpathlineto{\pgfqpoint{3.165894in}{5.004406in}}%
\pgfpathclose%
\pgfusepath{fill}%
\end{pgfscope}%
\begin{pgfscope}%
\pgfpathrectangle{\pgfqpoint{0.539299in}{0.078740in}}{\pgfqpoint{7.842520in}{7.842520in}}%
\pgfusepath{clip}%
\pgfsetbuttcap%
\pgfsetroundjoin%
\definecolor{currentfill}{rgb}{0.248629,0.278775,0.534556}%
\pgfsetfillcolor{currentfill}%
\pgfsetlinewidth{0.000000pt}%
\definecolor{currentstroke}{rgb}{0.279566,0.067836,0.391917}%
\pgfsetstrokecolor{currentstroke}%
\pgfsetdash{}{0pt}%
\pgfpathmoveto{\pgfqpoint{5.387297in}{3.592738in}}%
\pgfpathlineto{\pgfqpoint{5.311793in}{3.607895in}}%
\pgfpathlineto{\pgfqpoint{5.448865in}{3.506545in}}%
\pgfpathclose%
\pgfusepath{fill}%
\end{pgfscope}%
\begin{pgfscope}%
\pgfpathrectangle{\pgfqpoint{0.539299in}{0.078740in}}{\pgfqpoint{7.842520in}{7.842520in}}%
\pgfusepath{clip}%
\pgfsetbuttcap%
\pgfsetroundjoin%
\definecolor{currentfill}{rgb}{0.273809,0.031497,0.358853}%
\pgfsetfillcolor{currentfill}%
\pgfsetlinewidth{0.000000pt}%
\definecolor{currentstroke}{rgb}{0.280267,0.073417,0.397163}%
\pgfsetstrokecolor{currentstroke}%
\pgfsetdash{}{0pt}%
\pgfpathmoveto{\pgfqpoint{7.401863in}{2.702549in}}%
\pgfpathlineto{\pgfqpoint{7.332923in}{2.839586in}}%
\pgfpathlineto{\pgfqpoint{7.262707in}{2.788802in}}%
\pgfpathclose%
\pgfusepath{fill}%
\end{pgfscope}%
\begin{pgfscope}%
\pgfpathrectangle{\pgfqpoint{0.539299in}{0.078740in}}{\pgfqpoint{7.842520in}{7.842520in}}%
\pgfusepath{clip}%
\pgfsetbuttcap%
\pgfsetroundjoin%
\definecolor{currentfill}{rgb}{0.281446,0.084320,0.407414}%
\pgfsetfillcolor{currentfill}%
\pgfsetlinewidth{0.000000pt}%
\definecolor{currentstroke}{rgb}{0.280894,0.078907,0.402329}%
\pgfsetstrokecolor{currentstroke}%
\pgfsetdash{}{0pt}%
\pgfpathmoveto{\pgfqpoint{6.914313in}{2.924367in}}%
\pgfpathlineto{\pgfqpoint{7.123901in}{2.876974in}}%
\pgfpathlineto{\pgfqpoint{6.985354in}{2.964798in}}%
\pgfpathclose%
\pgfusepath{fill}%
\end{pgfscope}%
\begin{pgfscope}%
\pgfpathrectangle{\pgfqpoint{0.539299in}{0.078740in}}{\pgfqpoint{7.842520in}{7.842520in}}%
\pgfusepath{clip}%
\pgfsetbuttcap%
\pgfsetroundjoin%
\definecolor{currentfill}{rgb}{0.279574,0.170599,0.479997}%
\pgfsetfillcolor{currentfill}%
\pgfsetlinewidth{0.000000pt}%
\definecolor{currentstroke}{rgb}{0.281446,0.084320,0.407414}%
\pgfsetstrokecolor{currentstroke}%
\pgfsetdash{}{0pt}%
\pgfpathmoveto{\pgfqpoint{6.360427in}{3.240693in}}%
\pgfpathlineto{\pgfqpoint{6.287425in}{3.199853in}}%
\pgfpathlineto{\pgfqpoint{6.426028in}{3.129202in}}%
\pgfpathclose%
\pgfusepath{fill}%
\end{pgfscope}%
\begin{pgfscope}%
\pgfpathrectangle{\pgfqpoint{0.539299in}{0.078740in}}{\pgfqpoint{7.842520in}{7.842520in}}%
\pgfusepath{clip}%
\pgfsetbuttcap%
\pgfsetroundjoin%
\definecolor{currentfill}{rgb}{0.283091,0.110553,0.431554}%
\pgfsetfillcolor{currentfill}%
\pgfsetlinewidth{0.000000pt}%
\definecolor{currentstroke}{rgb}{0.281924,0.089666,0.412415}%
\pgfsetstrokecolor{currentstroke}%
\pgfsetdash{}{0pt}%
\pgfpathmoveto{\pgfqpoint{6.775662in}{3.012527in}}%
\pgfpathlineto{\pgfqpoint{6.914313in}{2.924367in}}%
\pgfpathlineto{\pgfqpoint{6.847001in}{3.050168in}}%
\pgfpathclose%
\pgfusepath{fill}%
\end{pgfscope}%
\begin{pgfscope}%
\pgfpathrectangle{\pgfqpoint{0.539299in}{0.078740in}}{\pgfqpoint{7.842520in}{7.842520in}}%
\pgfusepath{clip}%
\pgfsetbuttcap%
\pgfsetroundjoin%
\definecolor{currentfill}{rgb}{0.121380,0.629492,0.531973}%
\pgfsetfillcolor{currentfill}%
\pgfsetlinewidth{0.000000pt}%
\definecolor{currentstroke}{rgb}{0.282327,0.094955,0.417331}%
\pgfsetstrokecolor{currentstroke}%
\pgfsetdash{}{0pt}%
\pgfpathmoveto{\pgfqpoint{3.867334in}{5.102013in}}%
\pgfpathlineto{\pgfqpoint{4.085083in}{4.900621in}}%
\pgfpathlineto{\pgfqpoint{3.949347in}{5.006539in}}%
\pgfpathclose%
\pgfusepath{fill}%
\end{pgfscope}%
\begin{pgfscope}%
\pgfpathrectangle{\pgfqpoint{0.539299in}{0.078740in}}{\pgfqpoint{7.842520in}{7.842520in}}%
\pgfusepath{clip}%
\pgfsetbuttcap%
\pgfsetroundjoin%
\definecolor{currentfill}{rgb}{0.188923,0.410910,0.556326}%
\pgfsetfillcolor{currentfill}%
\pgfsetlinewidth{0.000000pt}%
\definecolor{currentstroke}{rgb}{0.282656,0.100196,0.422160}%
\pgfsetstrokecolor{currentstroke}%
\pgfsetdash{}{0pt}%
\pgfpathmoveto{\pgfqpoint{2.375504in}{3.692390in}}%
\pgfpathlineto{\pgfqpoint{2.491969in}{4.215044in}}%
\pgfpathlineto{\pgfqpoint{2.577164in}{4.260977in}}%
\pgfpathclose%
\pgfusepath{fill}%
\end{pgfscope}%
\begin{pgfscope}%
\pgfpathrectangle{\pgfqpoint{0.539299in}{0.078740in}}{\pgfqpoint{7.842520in}{7.842520in}}%
\pgfusepath{clip}%
\pgfsetbuttcap%
\pgfsetroundjoin%
\definecolor{currentfill}{rgb}{0.281887,0.150881,0.465405}%
\pgfsetfillcolor{currentfill}%
\pgfsetlinewidth{0.000000pt}%
\definecolor{currentstroke}{rgb}{0.282910,0.105393,0.426902}%
\pgfsetstrokecolor{currentstroke}%
\pgfsetdash{}{0pt}%
\pgfpathmoveto{\pgfqpoint{6.637123in}{3.095135in}}%
\pgfpathlineto{\pgfqpoint{6.498702in}{3.171293in}}%
\pgfpathlineto{\pgfqpoint{6.564795in}{3.052267in}}%
\pgfpathclose%
\pgfusepath{fill}%
\end{pgfscope}%
\begin{pgfscope}%
\pgfpathrectangle{\pgfqpoint{0.539299in}{0.078740in}}{\pgfqpoint{7.842520in}{7.842520in}}%
\pgfusepath{clip}%
\pgfsetbuttcap%
\pgfsetroundjoin%
\definecolor{currentfill}{rgb}{0.265145,0.232956,0.516599}%
\pgfsetfillcolor{currentfill}%
\pgfsetlinewidth{0.000000pt}%
\definecolor{currentstroke}{rgb}{0.283091,0.110553,0.431554}%
\pgfsetstrokecolor{currentstroke}%
\pgfsetdash{}{0pt}%
\pgfpathmoveto{\pgfqpoint{5.661270in}{3.429383in}}%
\pgfpathlineto{\pgfqpoint{5.724096in}{3.331845in}}%
\pgfpathlineto{\pgfqpoint{5.798863in}{3.358134in}}%
\pgfpathclose%
\pgfusepath{fill}%
\end{pgfscope}%
\begin{pgfscope}%
\pgfpathrectangle{\pgfqpoint{0.539299in}{0.078740in}}{\pgfqpoint{7.842520in}{7.842520in}}%
\pgfusepath{clip}%
\pgfsetbuttcap%
\pgfsetroundjoin%
\definecolor{currentfill}{rgb}{0.134692,0.658636,0.517649}%
\pgfsetfillcolor{currentfill}%
\pgfsetlinewidth{0.000000pt}%
\definecolor{currentstroke}{rgb}{0.283197,0.115680,0.436115}%
\pgfsetstrokecolor{currentstroke}%
\pgfsetdash{}{0pt}%
\pgfpathmoveto{\pgfqpoint{3.814186in}{5.073908in}}%
\pgfpathlineto{\pgfqpoint{3.731742in}{5.157127in}}%
\pgfpathlineto{\pgfqpoint{3.867334in}{5.102013in}}%
\pgfpathclose%
\pgfusepath{fill}%
\end{pgfscope}%
\begin{pgfscope}%
\pgfpathrectangle{\pgfqpoint{0.539299in}{0.078740in}}{\pgfqpoint{7.842520in}{7.842520in}}%
\pgfusepath{clip}%
\pgfsetbuttcap%
\pgfsetroundjoin%
\definecolor{currentfill}{rgb}{0.257322,0.256130,0.526563}%
\pgfsetfillcolor{currentfill}%
\pgfsetlinewidth{0.000000pt}%
\definecolor{currentstroke}{rgb}{0.283229,0.120777,0.440584}%
\pgfsetstrokecolor{currentstroke}%
\pgfsetdash{}{0pt}%
\pgfpathmoveto{\pgfqpoint{5.586291in}{3.415267in}}%
\pgfpathlineto{\pgfqpoint{5.661270in}{3.429383in}}%
\pgfpathlineto{\pgfqpoint{5.448865in}{3.506545in}}%
\pgfpathclose%
\pgfusepath{fill}%
\end{pgfscope}%
\begin{pgfscope}%
\pgfpathrectangle{\pgfqpoint{0.539299in}{0.078740in}}{\pgfqpoint{7.842520in}{7.842520in}}%
\pgfusepath{clip}%
\pgfsetbuttcap%
\pgfsetroundjoin%
\definecolor{currentfill}{rgb}{0.187231,0.414746,0.556547}%
\pgfsetfillcolor{currentfill}%
\pgfsetlinewidth{0.000000pt}%
\definecolor{currentstroke}{rgb}{0.283187,0.125848,0.444960}%
\pgfsetstrokecolor{currentstroke}%
\pgfsetdash{}{0pt}%
\pgfpathmoveto{\pgfqpoint{4.765949in}{4.134630in}}%
\pgfpathlineto{\pgfqpoint{4.824755in}{4.057714in}}%
\pgfpathlineto{\pgfqpoint{4.902184in}{3.985093in}}%
\pgfpathclose%
\pgfusepath{fill}%
\end{pgfscope}%
\begin{pgfscope}%
\pgfpathrectangle{\pgfqpoint{0.539299in}{0.078740in}}{\pgfqpoint{7.842520in}{7.842520in}}%
\pgfusepath{clip}%
\pgfsetbuttcap%
\pgfsetroundjoin%
\definecolor{currentfill}{rgb}{0.120565,0.596422,0.543611}%
\pgfsetfillcolor{currentfill}%
\pgfsetlinewidth{0.000000pt}%
\definecolor{currentstroke}{rgb}{0.283072,0.130895,0.449241}%
\pgfsetstrokecolor{currentstroke}%
\pgfsetdash{}{0pt}%
\pgfpathmoveto{\pgfqpoint{4.085083in}{4.900621in}}%
\pgfpathlineto{\pgfqpoint{4.140312in}{4.874931in}}%
\pgfpathlineto{\pgfqpoint{4.221135in}{4.766924in}}%
\pgfpathclose%
\pgfusepath{fill}%
\end{pgfscope}%
\begin{pgfscope}%
\pgfpathrectangle{\pgfqpoint{0.539299in}{0.078740in}}{\pgfqpoint{7.842520in}{7.842520in}}%
\pgfusepath{clip}%
\pgfsetbuttcap%
\pgfsetroundjoin%
\definecolor{currentfill}{rgb}{0.166617,0.463708,0.558119}%
\pgfsetfillcolor{currentfill}%
\pgfsetlinewidth{0.000000pt}%
\definecolor{currentstroke}{rgb}{0.282884,0.135920,0.453427}%
\pgfsetstrokecolor{currentstroke}%
\pgfsetdash{}{0pt}%
\pgfpathmoveto{\pgfqpoint{4.551037in}{4.389232in}}%
\pgfpathlineto{\pgfqpoint{4.765949in}{4.134630in}}%
\pgfpathlineto{\pgfqpoint{4.629750in}{4.292637in}}%
\pgfpathclose%
\pgfusepath{fill}%
\end{pgfscope}%
\begin{pgfscope}%
\pgfpathrectangle{\pgfqpoint{0.539299in}{0.078740in}}{\pgfqpoint{7.842520in}{7.842520in}}%
\pgfusepath{clip}%
\pgfsetbuttcap%
\pgfsetroundjoin%
\definecolor{currentfill}{rgb}{0.156270,0.489624,0.557936}%
\pgfsetfillcolor{currentfill}%
\pgfsetlinewidth{0.000000pt}%
\definecolor{currentstroke}{rgb}{0.282623,0.140926,0.457517}%
\pgfsetstrokecolor{currentstroke}%
\pgfsetdash{}{0pt}%
\pgfpathmoveto{\pgfqpoint{4.629750in}{4.292637in}}%
\pgfpathlineto{\pgfqpoint{4.493544in}{4.454843in}}%
\pgfpathlineto{\pgfqpoint{4.551037in}{4.389232in}}%
\pgfpathclose%
\pgfusepath{fill}%
\end{pgfscope}%
\begin{pgfscope}%
\pgfpathrectangle{\pgfqpoint{0.539299in}{0.078740in}}{\pgfqpoint{7.842520in}{7.842520in}}%
\pgfusepath{clip}%
\pgfsetbuttcap%
\pgfsetroundjoin%
\definecolor{currentfill}{rgb}{0.137770,0.537492,0.554906}%
\pgfsetfillcolor{currentfill}%
\pgfsetlinewidth{0.000000pt}%
\definecolor{currentstroke}{rgb}{0.282290,0.145912,0.461510}%
\pgfsetstrokecolor{currentstroke}%
\pgfsetdash{}{0pt}%
\pgfpathmoveto{\pgfqpoint{2.869477in}{4.725189in}}%
\pgfpathlineto{\pgfqpoint{2.746363in}{4.336760in}}%
\pgfpathlineto{\pgfqpoint{2.784458in}{4.707293in}}%
\pgfpathclose%
\pgfusepath{fill}%
\end{pgfscope}%
\begin{pgfscope}%
\pgfpathrectangle{\pgfqpoint{0.539299in}{0.078740in}}{\pgfqpoint{7.842520in}{7.842520in}}%
\pgfusepath{clip}%
\pgfsetbuttcap%
\pgfsetroundjoin%
\definecolor{currentfill}{rgb}{0.274128,0.199721,0.498911}%
\pgfsetfillcolor{currentfill}%
\pgfsetlinewidth{0.000000pt}%
\definecolor{currentstroke}{rgb}{0.281887,0.150881,0.465405}%
\pgfsetstrokecolor{currentstroke}%
\pgfsetdash{}{0pt}%
\pgfpathmoveto{\pgfqpoint{6.075188in}{3.222781in}}%
\pgfpathlineto{\pgfqpoint{6.149041in}{3.265181in}}%
\pgfpathlineto{\pgfqpoint{5.936847in}{3.290164in}}%
\pgfpathclose%
\pgfusepath{fill}%
\end{pgfscope}%
\begin{pgfscope}%
\pgfpathrectangle{\pgfqpoint{0.539299in}{0.078740in}}{\pgfqpoint{7.842520in}{7.842520in}}%
\pgfusepath{clip}%
\pgfsetbuttcap%
\pgfsetroundjoin%
\definecolor{currentfill}{rgb}{0.197636,0.391528,0.554969}%
\pgfsetfillcolor{currentfill}%
\pgfsetlinewidth{0.000000pt}%
\definecolor{currentstroke}{rgb}{0.281412,0.155834,0.469201}%
\pgfsetstrokecolor{currentstroke}%
\pgfsetdash{}{0pt}%
\pgfpathmoveto{\pgfqpoint{5.038520in}{3.846770in}}%
\pgfpathlineto{\pgfqpoint{4.902184in}{3.985093in}}%
\pgfpathlineto{\pgfqpoint{4.824755in}{4.057714in}}%
\pgfpathclose%
\pgfusepath{fill}%
\end{pgfscope}%
\begin{pgfscope}%
\pgfpathrectangle{\pgfqpoint{0.539299in}{0.078740in}}{\pgfqpoint{7.842520in}{7.842520in}}%
\pgfusepath{clip}%
\pgfsetbuttcap%
\pgfsetroundjoin%
\definecolor{currentfill}{rgb}{0.124395,0.578002,0.548287}%
\pgfsetfillcolor{currentfill}%
\pgfsetlinewidth{0.000000pt}%
\definecolor{currentstroke}{rgb}{0.280868,0.160771,0.472899}%
\pgfsetstrokecolor{currentstroke}%
\pgfsetdash{}{0pt}%
\pgfpathmoveto{\pgfqpoint{4.140312in}{4.874931in}}%
\pgfpathlineto{\pgfqpoint{4.357325in}{4.615401in}}%
\pgfpathlineto{\pgfqpoint{4.221135in}{4.766924in}}%
\pgfpathclose%
\pgfusepath{fill}%
\end{pgfscope}%
\begin{pgfscope}%
\pgfpathrectangle{\pgfqpoint{0.539299in}{0.078740in}}{\pgfqpoint{7.842520in}{7.842520in}}%
\pgfusepath{clip}%
\pgfsetbuttcap%
\pgfsetroundjoin%
\definecolor{currentfill}{rgb}{0.281412,0.155834,0.469201}%
\pgfsetfillcolor{currentfill}%
\pgfsetlinewidth{0.000000pt}%
\definecolor{currentstroke}{rgb}{0.280255,0.165693,0.476498}%
\pgfsetstrokecolor{currentstroke}%
\pgfsetdash{}{0pt}%
\pgfpathmoveto{\pgfqpoint{6.564795in}{3.052267in}}%
\pgfpathlineto{\pgfqpoint{6.498702in}{3.171293in}}%
\pgfpathlineto{\pgfqpoint{6.426028in}{3.129202in}}%
\pgfpathclose%
\pgfusepath{fill}%
\end{pgfscope}%
\begin{pgfscope}%
\pgfpathrectangle{\pgfqpoint{0.539299in}{0.078740in}}{\pgfqpoint{7.842520in}{7.842520in}}%
\pgfusepath{clip}%
\pgfsetbuttcap%
\pgfsetroundjoin%
\definecolor{currentfill}{rgb}{0.136408,0.541173,0.554483}%
\pgfsetfillcolor{currentfill}%
\pgfsetlinewidth{0.000000pt}%
\definecolor{currentstroke}{rgb}{0.279574,0.170599,0.479997}%
\pgfsetstrokecolor{currentstroke}%
\pgfsetdash{}{0pt}%
\pgfpathmoveto{\pgfqpoint{4.493544in}{4.454843in}}%
\pgfpathlineto{\pgfqpoint{4.357325in}{4.615401in}}%
\pgfpathlineto{\pgfqpoint{4.277193in}{4.723467in}}%
\pgfpathclose%
\pgfusepath{fill}%
\end{pgfscope}%
\begin{pgfscope}%
\pgfpathrectangle{\pgfqpoint{0.539299in}{0.078740in}}{\pgfqpoint{7.842520in}{7.842520in}}%
\pgfusepath{clip}%
\pgfsetbuttcap%
\pgfsetroundjoin%
\definecolor{currentfill}{rgb}{0.277941,0.056324,0.381191}%
\pgfsetfillcolor{currentfill}%
\pgfsetlinewidth{0.000000pt}%
\definecolor{currentstroke}{rgb}{0.278826,0.175490,0.483397}%
\pgfsetstrokecolor{currentstroke}%
\pgfsetdash{}{0pt}%
\pgfpathmoveto{\pgfqpoint{7.123901in}{2.876974in}}%
\pgfpathlineto{\pgfqpoint{7.192040in}{2.737188in}}%
\pgfpathlineto{\pgfqpoint{7.262707in}{2.788802in}}%
\pgfpathclose%
\pgfusepath{fill}%
\end{pgfscope}%
\begin{pgfscope}%
\pgfpathrectangle{\pgfqpoint{0.539299in}{0.078740in}}{\pgfqpoint{7.842520in}{7.842520in}}%
\pgfusepath{clip}%
\pgfsetbuttcap%
\pgfsetroundjoin%
\definecolor{currentfill}{rgb}{0.277134,0.185228,0.489898}%
\pgfsetfillcolor{currentfill}%
\pgfsetlinewidth{0.000000pt}%
\definecolor{currentstroke}{rgb}{0.278012,0.180367,0.486697}%
\pgfsetstrokecolor{currentstroke}%
\pgfsetdash{}{0pt}%
\pgfpathmoveto{\pgfqpoint{6.287425in}{3.199853in}}%
\pgfpathlineto{\pgfqpoint{6.149041in}{3.265181in}}%
\pgfpathlineto{\pgfqpoint{6.213837in}{3.153582in}}%
\pgfpathclose%
\pgfusepath{fill}%
\end{pgfscope}%
\begin{pgfscope}%
\pgfpathrectangle{\pgfqpoint{0.539299in}{0.078740in}}{\pgfqpoint{7.842520in}{7.842520in}}%
\pgfusepath{clip}%
\pgfsetbuttcap%
\pgfsetroundjoin%
\definecolor{currentfill}{rgb}{0.281446,0.084320,0.407414}%
\pgfsetfillcolor{currentfill}%
\pgfsetlinewidth{0.000000pt}%
\definecolor{currentstroke}{rgb}{0.277134,0.185228,0.489898}%
\pgfsetstrokecolor{currentstroke}%
\pgfsetdash{}{0pt}%
\pgfpathmoveto{\pgfqpoint{7.053093in}{2.832020in}}%
\pgfpathlineto{\pgfqpoint{7.123901in}{2.876974in}}%
\pgfpathlineto{\pgfqpoint{6.914313in}{2.924367in}}%
\pgfpathclose%
\pgfusepath{fill}%
\end{pgfscope}%
\begin{pgfscope}%
\pgfpathrectangle{\pgfqpoint{0.539299in}{0.078740in}}{\pgfqpoint{7.842520in}{7.842520in}}%
\pgfusepath{clip}%
\pgfsetbuttcap%
\pgfsetroundjoin%
\definecolor{currentfill}{rgb}{0.212395,0.359683,0.551710}%
\pgfsetfillcolor{currentfill}%
\pgfsetlinewidth{0.000000pt}%
\definecolor{currentstroke}{rgb}{0.276194,0.190074,0.493001}%
\pgfsetstrokecolor{currentstroke}%
\pgfsetdash{}{0pt}%
\pgfpathmoveto{\pgfqpoint{5.175033in}{3.721001in}}%
\pgfpathlineto{\pgfqpoint{5.038520in}{3.846770in}}%
\pgfpathlineto{\pgfqpoint{4.961637in}{3.904256in}}%
\pgfpathclose%
\pgfusepath{fill}%
\end{pgfscope}%
\begin{pgfscope}%
\pgfpathrectangle{\pgfqpoint{0.539299in}{0.078740in}}{\pgfqpoint{7.842520in}{7.842520in}}%
\pgfusepath{clip}%
\pgfsetbuttcap%
\pgfsetroundjoin%
\definecolor{currentfill}{rgb}{0.119699,0.618490,0.536347}%
\pgfsetfillcolor{currentfill}%
\pgfsetlinewidth{0.000000pt}%
\definecolor{currentstroke}{rgb}{0.275191,0.194905,0.496005}%
\pgfsetstrokecolor{currentstroke}%
\pgfsetdash{}{0pt}%
\pgfpathmoveto{\pgfqpoint{3.081416in}{5.017996in}}%
\pgfpathlineto{\pgfqpoint{3.165894in}{5.004406in}}%
\pgfpathlineto{\pgfqpoint{2.954036in}{4.737238in}}%
\pgfpathclose%
\pgfusepath{fill}%
\end{pgfscope}%
\begin{pgfscope}%
\pgfpathrectangle{\pgfqpoint{0.539299in}{0.078740in}}{\pgfqpoint{7.842520in}{7.842520in}}%
\pgfusepath{clip}%
\pgfsetbuttcap%
\pgfsetroundjoin%
\definecolor{currentfill}{rgb}{0.283072,0.130895,0.449241}%
\pgfsetfillcolor{currentfill}%
\pgfsetlinewidth{0.000000pt}%
\definecolor{currentstroke}{rgb}{0.274128,0.199721,0.498911}%
\pgfsetstrokecolor{currentstroke}%
\pgfsetdash{}{0pt}%
\pgfpathmoveto{\pgfqpoint{6.703683in}{2.968722in}}%
\pgfpathlineto{\pgfqpoint{6.775662in}{3.012527in}}%
\pgfpathlineto{\pgfqpoint{6.637123in}{3.095135in}}%
\pgfpathclose%
\pgfusepath{fill}%
\end{pgfscope}%
\begin{pgfscope}%
\pgfpathrectangle{\pgfqpoint{0.539299in}{0.078740in}}{\pgfqpoint{7.842520in}{7.842520in}}%
\pgfusepath{clip}%
\pgfsetbuttcap%
\pgfsetroundjoin%
\definecolor{currentfill}{rgb}{0.262138,0.242286,0.520837}%
\pgfsetfillcolor{currentfill}%
\pgfsetlinewidth{0.000000pt}%
\definecolor{currentstroke}{rgb}{0.273006,0.204520,0.501721}%
\pgfsetstrokecolor{currentstroke}%
\pgfsetdash{}{0pt}%
\pgfpathmoveto{\pgfqpoint{5.586291in}{3.415267in}}%
\pgfpathlineto{\pgfqpoint{5.724096in}{3.331845in}}%
\pgfpathlineto{\pgfqpoint{5.661270in}{3.429383in}}%
\pgfpathclose%
\pgfusepath{fill}%
\end{pgfscope}%
\begin{pgfscope}%
\pgfpathrectangle{\pgfqpoint{0.539299in}{0.078740in}}{\pgfqpoint{7.842520in}{7.842520in}}%
\pgfusepath{clip}%
\pgfsetbuttcap%
\pgfsetroundjoin%
\definecolor{currentfill}{rgb}{0.267968,0.223549,0.512008}%
\pgfsetfillcolor{currentfill}%
\pgfsetlinewidth{0.000000pt}%
\definecolor{currentstroke}{rgb}{0.271828,0.209303,0.504434}%
\pgfsetstrokecolor{currentstroke}%
\pgfsetdash{}{0pt}%
\pgfpathmoveto{\pgfqpoint{5.798863in}{3.358134in}}%
\pgfpathlineto{\pgfqpoint{5.724096in}{3.331845in}}%
\pgfpathlineto{\pgfqpoint{5.936847in}{3.290164in}}%
\pgfpathclose%
\pgfusepath{fill}%
\end{pgfscope}%
\begin{pgfscope}%
\pgfpathrectangle{\pgfqpoint{0.539299in}{0.078740in}}{\pgfqpoint{7.842520in}{7.842520in}}%
\pgfusepath{clip}%
\pgfsetbuttcap%
\pgfsetroundjoin%
\definecolor{currentfill}{rgb}{0.150476,0.504369,0.557430}%
\pgfsetfillcolor{currentfill}%
\pgfsetlinewidth{0.000000pt}%
\definecolor{currentstroke}{rgb}{0.270595,0.214069,0.507052}%
\pgfsetstrokecolor{currentstroke}%
\pgfsetdash{}{0pt}%
\pgfpathmoveto{\pgfqpoint{2.784458in}{4.707293in}}%
\pgfpathlineto{\pgfqpoint{2.661974in}{4.301274in}}%
\pgfpathlineto{\pgfqpoint{2.577164in}{4.260977in}}%
\pgfpathclose%
\pgfusepath{fill}%
\end{pgfscope}%
\begin{pgfscope}%
\pgfpathrectangle{\pgfqpoint{0.539299in}{0.078740in}}{\pgfqpoint{7.842520in}{7.842520in}}%
\pgfusepath{clip}%
\pgfsetbuttcap%
\pgfsetroundjoin%
\definecolor{currentfill}{rgb}{0.150148,0.676631,0.506589}%
\pgfsetfillcolor{currentfill}%
\pgfsetlinewidth{0.000000pt}%
\definecolor{currentstroke}{rgb}{0.269308,0.218818,0.509577}%
\pgfsetstrokecolor{currentstroke}%
\pgfsetdash{}{0pt}%
\pgfpathmoveto{\pgfqpoint{3.380711in}{5.141597in}}%
\pgfpathlineto{\pgfqpoint{3.513793in}{5.222466in}}%
\pgfpathlineto{\pgfqpoint{3.597214in}{5.159015in}}%
\pgfpathclose%
\pgfusepath{fill}%
\end{pgfscope}%
\begin{pgfscope}%
\pgfpathrectangle{\pgfqpoint{0.539299in}{0.078740in}}{\pgfqpoint{7.842520in}{7.842520in}}%
\pgfusepath{clip}%
\pgfsetbuttcap%
\pgfsetroundjoin%
\definecolor{currentfill}{rgb}{0.275191,0.194905,0.496005}%
\pgfsetfillcolor{currentfill}%
\pgfsetlinewidth{0.000000pt}%
\definecolor{currentstroke}{rgb}{0.267968,0.223549,0.512008}%
\pgfsetstrokecolor{currentstroke}%
\pgfsetdash{}{0pt}%
\pgfpathmoveto{\pgfqpoint{6.213837in}{3.153582in}}%
\pgfpathlineto{\pgfqpoint{6.149041in}{3.265181in}}%
\pgfpathlineto{\pgfqpoint{6.075188in}{3.222781in}}%
\pgfpathclose%
\pgfusepath{fill}%
\end{pgfscope}%
\begin{pgfscope}%
\pgfpathrectangle{\pgfqpoint{0.539299in}{0.078740in}}{\pgfqpoint{7.842520in}{7.842520in}}%
\pgfusepath{clip}%
\pgfsetbuttcap%
\pgfsetroundjoin%
\definecolor{currentfill}{rgb}{0.279566,0.067836,0.391917}%
\pgfsetfillcolor{currentfill}%
\pgfsetlinewidth{0.000000pt}%
\definecolor{currentstroke}{rgb}{0.266580,0.228262,0.514349}%
\pgfsetstrokecolor{currentstroke}%
\pgfsetdash{}{0pt}%
\pgfpathmoveto{\pgfqpoint{7.123901in}{2.876974in}}%
\pgfpathlineto{\pgfqpoint{7.053093in}{2.832020in}}%
\pgfpathlineto{\pgfqpoint{7.192040in}{2.737188in}}%
\pgfpathclose%
\pgfusepath{fill}%
\end{pgfscope}%
\begin{pgfscope}%
\pgfpathrectangle{\pgfqpoint{0.539299in}{0.078740in}}{\pgfqpoint{7.842520in}{7.842520in}}%
\pgfusepath{clip}%
\pgfsetbuttcap%
\pgfsetroundjoin%
\definecolor{currentfill}{rgb}{0.231674,0.318106,0.544834}%
\pgfsetfillcolor{currentfill}%
\pgfsetlinewidth{0.000000pt}%
\definecolor{currentstroke}{rgb}{0.265145,0.232956,0.516599}%
\pgfsetstrokecolor{currentstroke}%
\pgfsetdash{}{0pt}%
\pgfpathmoveto{\pgfqpoint{5.235775in}{3.632407in}}%
\pgfpathlineto{\pgfqpoint{5.311793in}{3.607895in}}%
\pgfpathlineto{\pgfqpoint{5.175033in}{3.721001in}}%
\pgfpathclose%
\pgfusepath{fill}%
\end{pgfscope}%
\begin{pgfscope}%
\pgfpathrectangle{\pgfqpoint{0.539299in}{0.078740in}}{\pgfqpoint{7.842520in}{7.842520in}}%
\pgfusepath{clip}%
\pgfsetbuttcap%
\pgfsetroundjoin%
\definecolor{currentfill}{rgb}{0.153894,0.680203,0.504172}%
\pgfsetfillcolor{currentfill}%
\pgfsetlinewidth{0.000000pt}%
\definecolor{currentstroke}{rgb}{0.263663,0.237631,0.518762}%
\pgfsetstrokecolor{currentstroke}%
\pgfsetdash{}{0pt}%
\pgfpathmoveto{\pgfqpoint{3.513793in}{5.222466in}}%
\pgfpathlineto{\pgfqpoint{3.731742in}{5.157127in}}%
\pgfpathlineto{\pgfqpoint{3.597214in}{5.159015in}}%
\pgfpathclose%
\pgfusepath{fill}%
\end{pgfscope}%
\begin{pgfscope}%
\pgfpathrectangle{\pgfqpoint{0.539299in}{0.078740in}}{\pgfqpoint{7.842520in}{7.842520in}}%
\pgfusepath{clip}%
\pgfsetbuttcap%
\pgfsetroundjoin%
\definecolor{currentfill}{rgb}{0.272594,0.025563,0.353093}%
\pgfsetfillcolor{currentfill}%
\pgfsetlinewidth{0.000000pt}%
\definecolor{currentstroke}{rgb}{0.262138,0.242286,0.520837}%
\pgfsetstrokecolor{currentstroke}%
\pgfsetdash{}{0pt}%
\pgfpathmoveto{\pgfqpoint{7.331215in}{2.641797in}}%
\pgfpathlineto{\pgfqpoint{7.401863in}{2.702549in}}%
\pgfpathlineto{\pgfqpoint{7.262707in}{2.788802in}}%
\pgfpathclose%
\pgfusepath{fill}%
\end{pgfscope}%
\begin{pgfscope}%
\pgfpathrectangle{\pgfqpoint{0.539299in}{0.078740in}}{\pgfqpoint{7.842520in}{7.842520in}}%
\pgfusepath{clip}%
\pgfsetbuttcap%
\pgfsetroundjoin%
\definecolor{currentfill}{rgb}{0.283229,0.120777,0.440584}%
\pgfsetfillcolor{currentfill}%
\pgfsetlinewidth{0.000000pt}%
\definecolor{currentstroke}{rgb}{0.260571,0.246922,0.522828}%
\pgfsetstrokecolor{currentstroke}%
\pgfsetdash{}{0pt}%
\pgfpathmoveto{\pgfqpoint{6.703683in}{2.968722in}}%
\pgfpathlineto{\pgfqpoint{6.914313in}{2.924367in}}%
\pgfpathlineto{\pgfqpoint{6.775662in}{3.012527in}}%
\pgfpathclose%
\pgfusepath{fill}%
\end{pgfscope}%
\begin{pgfscope}%
\pgfpathrectangle{\pgfqpoint{0.539299in}{0.078740in}}{\pgfqpoint{7.842520in}{7.842520in}}%
\pgfusepath{clip}%
\pgfsetbuttcap%
\pgfsetroundjoin%
\definecolor{currentfill}{rgb}{0.124780,0.640461,0.527068}%
\pgfsetfillcolor{currentfill}%
\pgfsetlinewidth{0.000000pt}%
\definecolor{currentstroke}{rgb}{0.258965,0.251537,0.524736}%
\pgfsetstrokecolor{currentstroke}%
\pgfsetdash{}{0pt}%
\pgfpathmoveto{\pgfqpoint{4.003621in}{5.004426in}}%
\pgfpathlineto{\pgfqpoint{4.085083in}{4.900621in}}%
\pgfpathlineto{\pgfqpoint{3.867334in}{5.102013in}}%
\pgfpathclose%
\pgfusepath{fill}%
\end{pgfscope}%
\begin{pgfscope}%
\pgfpathrectangle{\pgfqpoint{0.539299in}{0.078740in}}{\pgfqpoint{7.842520in}{7.842520in}}%
\pgfusepath{clip}%
\pgfsetbuttcap%
\pgfsetroundjoin%
\definecolor{currentfill}{rgb}{0.282623,0.140926,0.457517}%
\pgfsetfillcolor{currentfill}%
\pgfsetlinewidth{0.000000pt}%
\definecolor{currentstroke}{rgb}{0.257322,0.256130,0.526563}%
\pgfsetstrokecolor{currentstroke}%
\pgfsetdash{}{0pt}%
\pgfpathmoveto{\pgfqpoint{6.637123in}{3.095135in}}%
\pgfpathlineto{\pgfqpoint{6.564795in}{3.052267in}}%
\pgfpathlineto{\pgfqpoint{6.703683in}{2.968722in}}%
\pgfpathclose%
\pgfusepath{fill}%
\end{pgfscope}%
\begin{pgfscope}%
\pgfpathrectangle{\pgfqpoint{0.539299in}{0.078740in}}{\pgfqpoint{7.842520in}{7.842520in}}%
\pgfusepath{clip}%
\pgfsetbuttcap%
\pgfsetroundjoin%
\definecolor{currentfill}{rgb}{0.278826,0.175490,0.483397}%
\pgfsetfillcolor{currentfill}%
\pgfsetlinewidth{0.000000pt}%
\definecolor{currentstroke}{rgb}{0.255645,0.260703,0.528312}%
\pgfsetstrokecolor{currentstroke}%
\pgfsetdash{}{0pt}%
\pgfpathmoveto{\pgfqpoint{6.426028in}{3.129202in}}%
\pgfpathlineto{\pgfqpoint{6.287425in}{3.199853in}}%
\pgfpathlineto{\pgfqpoint{6.352735in}{3.080591in}}%
\pgfpathclose%
\pgfusepath{fill}%
\end{pgfscope}%
\begin{pgfscope}%
\pgfpathrectangle{\pgfqpoint{0.539299in}{0.078740in}}{\pgfqpoint{7.842520in}{7.842520in}}%
\pgfusepath{clip}%
\pgfsetbuttcap%
\pgfsetroundjoin%
\definecolor{currentfill}{rgb}{0.140210,0.665859,0.513427}%
\pgfsetfillcolor{currentfill}%
\pgfsetlinewidth{0.000000pt}%
\definecolor{currentstroke}{rgb}{0.253935,0.265254,0.529983}%
\pgfsetstrokecolor{currentstroke}%
\pgfsetdash{}{0pt}%
\pgfpathmoveto{\pgfqpoint{3.165894in}{5.004406in}}%
\pgfpathlineto{\pgfqpoint{3.296607in}{5.181085in}}%
\pgfpathlineto{\pgfqpoint{3.380711in}{5.141597in}}%
\pgfpathclose%
\pgfusepath{fill}%
\end{pgfscope}%
\begin{pgfscope}%
\pgfpathrectangle{\pgfqpoint{0.539299in}{0.078740in}}{\pgfqpoint{7.842520in}{7.842520in}}%
\pgfusepath{clip}%
\pgfsetbuttcap%
\pgfsetroundjoin%
\definecolor{currentfill}{rgb}{0.243113,0.292092,0.538516}%
\pgfsetfillcolor{currentfill}%
\pgfsetlinewidth{0.000000pt}%
\definecolor{currentstroke}{rgb}{0.252194,0.269783,0.531579}%
\pgfsetstrokecolor{currentstroke}%
\pgfsetdash{}{0pt}%
\pgfpathmoveto{\pgfqpoint{5.311793in}{3.607895in}}%
\pgfpathlineto{\pgfqpoint{5.373172in}{3.514648in}}%
\pgfpathlineto{\pgfqpoint{5.448865in}{3.506545in}}%
\pgfpathclose%
\pgfusepath{fill}%
\end{pgfscope}%
\begin{pgfscope}%
\pgfpathrectangle{\pgfqpoint{0.539299in}{0.078740in}}{\pgfqpoint{7.842520in}{7.842520in}}%
\pgfusepath{clip}%
\pgfsetbuttcap%
\pgfsetroundjoin%
\definecolor{currentfill}{rgb}{0.175841,0.441290,0.557685}%
\pgfsetfillcolor{currentfill}%
\pgfsetlinewidth{0.000000pt}%
\definecolor{currentstroke}{rgb}{0.250425,0.274290,0.533103}%
\pgfsetstrokecolor{currentstroke}%
\pgfsetdash{}{0pt}%
\pgfpathmoveto{\pgfqpoint{4.687905in}{4.220515in}}%
\pgfpathlineto{\pgfqpoint{4.824755in}{4.057714in}}%
\pgfpathlineto{\pgfqpoint{4.765949in}{4.134630in}}%
\pgfpathclose%
\pgfusepath{fill}%
\end{pgfscope}%
\begin{pgfscope}%
\pgfpathrectangle{\pgfqpoint{0.539299in}{0.078740in}}{\pgfqpoint{7.842520in}{7.842520in}}%
\pgfusepath{clip}%
\pgfsetbuttcap%
\pgfsetroundjoin%
\definecolor{currentfill}{rgb}{0.120638,0.625828,0.533488}%
\pgfsetfillcolor{currentfill}%
\pgfsetlinewidth{0.000000pt}%
\definecolor{currentstroke}{rgb}{0.248629,0.278775,0.534556}%
\pgfsetstrokecolor{currentstroke}%
\pgfsetdash{}{0pt}%
\pgfpathmoveto{\pgfqpoint{4.003621in}{5.004426in}}%
\pgfpathlineto{\pgfqpoint{4.140312in}{4.874931in}}%
\pgfpathlineto{\pgfqpoint{4.085083in}{4.900621in}}%
\pgfpathclose%
\pgfusepath{fill}%
\end{pgfscope}%
\begin{pgfscope}%
\pgfpathrectangle{\pgfqpoint{0.539299in}{0.078740in}}{\pgfqpoint{7.842520in}{7.842520in}}%
\pgfusepath{clip}%
\pgfsetbuttcap%
\pgfsetroundjoin%
\definecolor{currentfill}{rgb}{0.165117,0.467423,0.558141}%
\pgfsetfillcolor{currentfill}%
\pgfsetlinewidth{0.000000pt}%
\definecolor{currentstroke}{rgb}{0.246811,0.283237,0.535941}%
\pgfsetstrokecolor{currentstroke}%
\pgfsetdash{}{0pt}%
\pgfpathmoveto{\pgfqpoint{4.687905in}{4.220515in}}%
\pgfpathlineto{\pgfqpoint{4.765949in}{4.134630in}}%
\pgfpathlineto{\pgfqpoint{4.551037in}{4.389232in}}%
\pgfpathclose%
\pgfusepath{fill}%
\end{pgfscope}%
\begin{pgfscope}%
\pgfpathrectangle{\pgfqpoint{0.539299in}{0.078740in}}{\pgfqpoint{7.842520in}{7.842520in}}%
\pgfusepath{clip}%
\pgfsetbuttcap%
\pgfsetroundjoin%
\definecolor{currentfill}{rgb}{0.195860,0.395433,0.555276}%
\pgfsetfillcolor{currentfill}%
\pgfsetlinewidth{0.000000pt}%
\definecolor{currentstroke}{rgb}{0.244972,0.287675,0.537260}%
\pgfsetstrokecolor{currentstroke}%
\pgfsetdash{}{0pt}%
\pgfpathmoveto{\pgfqpoint{4.961637in}{3.904256in}}%
\pgfpathlineto{\pgfqpoint{5.038520in}{3.846770in}}%
\pgfpathlineto{\pgfqpoint{4.824755in}{4.057714in}}%
\pgfpathclose%
\pgfusepath{fill}%
\end{pgfscope}%
\begin{pgfscope}%
\pgfpathrectangle{\pgfqpoint{0.539299in}{0.078740in}}{\pgfqpoint{7.842520in}{7.842520in}}%
\pgfusepath{clip}%
\pgfsetbuttcap%
\pgfsetroundjoin%
\definecolor{currentfill}{rgb}{0.144759,0.519093,0.556572}%
\pgfsetfillcolor{currentfill}%
\pgfsetlinewidth{0.000000pt}%
\definecolor{currentstroke}{rgb}{0.243113,0.292092,0.538516}%
\pgfsetstrokecolor{currentstroke}%
\pgfsetdash{}{0pt}%
\pgfpathmoveto{\pgfqpoint{4.551037in}{4.389232in}}%
\pgfpathlineto{\pgfqpoint{4.493544in}{4.454843in}}%
\pgfpathlineto{\pgfqpoint{4.414126in}{4.558993in}}%
\pgfpathclose%
\pgfusepath{fill}%
\end{pgfscope}%
\begin{pgfscope}%
\pgfpathrectangle{\pgfqpoint{0.539299in}{0.078740in}}{\pgfqpoint{7.842520in}{7.842520in}}%
\pgfusepath{clip}%
\pgfsetbuttcap%
\pgfsetroundjoin%
\definecolor{currentfill}{rgb}{0.274952,0.037752,0.364543}%
\pgfsetfillcolor{currentfill}%
\pgfsetlinewidth{0.000000pt}%
\definecolor{currentstroke}{rgb}{0.241237,0.296485,0.539709}%
\pgfsetstrokecolor{currentstroke}%
\pgfsetdash{}{0pt}%
\pgfpathmoveto{\pgfqpoint{7.262707in}{2.788802in}}%
\pgfpathlineto{\pgfqpoint{7.192040in}{2.737188in}}%
\pgfpathlineto{\pgfqpoint{7.331215in}{2.641797in}}%
\pgfpathclose%
\pgfusepath{fill}%
\end{pgfscope}%
\begin{pgfscope}%
\pgfpathrectangle{\pgfqpoint{0.539299in}{0.078740in}}{\pgfqpoint{7.842520in}{7.842520in}}%
\pgfusepath{clip}%
\pgfsetbuttcap%
\pgfsetroundjoin%
\definecolor{currentfill}{rgb}{0.135066,0.544853,0.554029}%
\pgfsetfillcolor{currentfill}%
\pgfsetlinewidth{0.000000pt}%
\definecolor{currentstroke}{rgb}{0.239346,0.300855,0.540844}%
\pgfsetstrokecolor{currentstroke}%
\pgfsetdash{}{0pt}%
\pgfpathmoveto{\pgfqpoint{4.277193in}{4.723467in}}%
\pgfpathlineto{\pgfqpoint{4.414126in}{4.558993in}}%
\pgfpathlineto{\pgfqpoint{4.493544in}{4.454843in}}%
\pgfpathclose%
\pgfusepath{fill}%
\end{pgfscope}%
\begin{pgfscope}%
\pgfpathrectangle{\pgfqpoint{0.539299in}{0.078740in}}{\pgfqpoint{7.842520in}{7.842520in}}%
\pgfusepath{clip}%
\pgfsetbuttcap%
\pgfsetroundjoin%
\definecolor{currentfill}{rgb}{0.123463,0.581687,0.547445}%
\pgfsetfillcolor{currentfill}%
\pgfsetlinewidth{0.000000pt}%
\definecolor{currentstroke}{rgb}{0.237441,0.305202,0.541921}%
\pgfsetstrokecolor{currentstroke}%
\pgfsetdash{}{0pt}%
\pgfpathmoveto{\pgfqpoint{4.277193in}{4.723467in}}%
\pgfpathlineto{\pgfqpoint{4.357325in}{4.615401in}}%
\pgfpathlineto{\pgfqpoint{4.140312in}{4.874931in}}%
\pgfpathclose%
\pgfusepath{fill}%
\end{pgfscope}%
\begin{pgfscope}%
\pgfpathrectangle{\pgfqpoint{0.539299in}{0.078740in}}{\pgfqpoint{7.842520in}{7.842520in}}%
\pgfusepath{clip}%
\pgfsetbuttcap%
\pgfsetroundjoin%
\definecolor{currentfill}{rgb}{0.210503,0.363727,0.552206}%
\pgfsetfillcolor{currentfill}%
\pgfsetlinewidth{0.000000pt}%
\definecolor{currentstroke}{rgb}{0.235526,0.309527,0.542944}%
\pgfsetstrokecolor{currentstroke}%
\pgfsetdash{}{0pt}%
\pgfpathmoveto{\pgfqpoint{4.961637in}{3.904256in}}%
\pgfpathlineto{\pgfqpoint{5.098620in}{3.762206in}}%
\pgfpathlineto{\pgfqpoint{5.175033in}{3.721001in}}%
\pgfpathclose%
\pgfusepath{fill}%
\end{pgfscope}%
\begin{pgfscope}%
\pgfpathrectangle{\pgfqpoint{0.539299in}{0.078740in}}{\pgfqpoint{7.842520in}{7.842520in}}%
\pgfusepath{clip}%
\pgfsetbuttcap%
\pgfsetroundjoin%
\definecolor{currentfill}{rgb}{0.214298,0.355619,0.551184}%
\pgfsetfillcolor{currentfill}%
\pgfsetlinewidth{0.000000pt}%
\definecolor{currentstroke}{rgb}{0.233603,0.313828,0.543914}%
\pgfsetstrokecolor{currentstroke}%
\pgfsetdash{}{0pt}%
\pgfpathmoveto{\pgfqpoint{2.320596in}{4.102334in}}%
\pgfpathlineto{\pgfqpoint{2.290658in}{3.630794in}}%
\pgfpathlineto{\pgfqpoint{2.205440in}{3.565187in}}%
\pgfpathclose%
\pgfusepath{fill}%
\end{pgfscope}%
\begin{pgfscope}%
\pgfpathrectangle{\pgfqpoint{0.539299in}{0.078740in}}{\pgfqpoint{7.842520in}{7.842520in}}%
\pgfusepath{clip}%
\pgfsetbuttcap%
\pgfsetroundjoin%
\definecolor{currentfill}{rgb}{0.267968,0.223549,0.512008}%
\pgfsetfillcolor{currentfill}%
\pgfsetlinewidth{0.000000pt}%
\definecolor{currentstroke}{rgb}{0.231674,0.318106,0.544834}%
\pgfsetstrokecolor{currentstroke}%
\pgfsetdash{}{0pt}%
\pgfpathmoveto{\pgfqpoint{5.724096in}{3.331845in}}%
\pgfpathlineto{\pgfqpoint{5.862282in}{3.253769in}}%
\pgfpathlineto{\pgfqpoint{5.936847in}{3.290164in}}%
\pgfpathclose%
\pgfusepath{fill}%
\end{pgfscope}%
\begin{pgfscope}%
\pgfpathrectangle{\pgfqpoint{0.539299in}{0.078740in}}{\pgfqpoint{7.842520in}{7.842520in}}%
\pgfusepath{clip}%
\pgfsetbuttcap%
\pgfsetroundjoin%
\definecolor{currentfill}{rgb}{0.278012,0.180367,0.486697}%
\pgfsetfillcolor{currentfill}%
\pgfsetlinewidth{0.000000pt}%
\definecolor{currentstroke}{rgb}{0.229739,0.322361,0.545706}%
\pgfsetstrokecolor{currentstroke}%
\pgfsetdash{}{0pt}%
\pgfpathmoveto{\pgfqpoint{6.287425in}{3.199853in}}%
\pgfpathlineto{\pgfqpoint{6.213837in}{3.153582in}}%
\pgfpathlineto{\pgfqpoint{6.352735in}{3.080591in}}%
\pgfpathclose%
\pgfusepath{fill}%
\end{pgfscope}%
\begin{pgfscope}%
\pgfpathrectangle{\pgfqpoint{0.539299in}{0.078740in}}{\pgfqpoint{7.842520in}{7.842520in}}%
\pgfusepath{clip}%
\pgfsetbuttcap%
\pgfsetroundjoin%
\definecolor{currentfill}{rgb}{0.248629,0.278775,0.534556}%
\pgfsetfillcolor{currentfill}%
\pgfsetlinewidth{0.000000pt}%
\definecolor{currentstroke}{rgb}{0.227802,0.326594,0.546532}%
\pgfsetstrokecolor{currentstroke}%
\pgfsetdash{}{0pt}%
\pgfpathmoveto{\pgfqpoint{5.586291in}{3.415267in}}%
\pgfpathlineto{\pgfqpoint{5.448865in}{3.506545in}}%
\pgfpathlineto{\pgfqpoint{5.373172in}{3.514648in}}%
\pgfpathclose%
\pgfusepath{fill}%
\end{pgfscope}%
\begin{pgfscope}%
\pgfpathrectangle{\pgfqpoint{0.539299in}{0.078740in}}{\pgfqpoint{7.842520in}{7.842520in}}%
\pgfusepath{clip}%
\pgfsetbuttcap%
\pgfsetroundjoin%
\definecolor{currentfill}{rgb}{0.282656,0.100196,0.422160}%
\pgfsetfillcolor{currentfill}%
\pgfsetlinewidth{0.000000pt}%
\definecolor{currentstroke}{rgb}{0.225863,0.330805,0.547314}%
\pgfsetstrokecolor{currentstroke}%
\pgfsetdash{}{0pt}%
\pgfpathmoveto{\pgfqpoint{7.053093in}{2.832020in}}%
\pgfpathlineto{\pgfqpoint{6.914313in}{2.924367in}}%
\pgfpathlineto{\pgfqpoint{6.842661in}{2.878816in}}%
\pgfpathclose%
\pgfusepath{fill}%
\end{pgfscope}%
\begin{pgfscope}%
\pgfpathrectangle{\pgfqpoint{0.539299in}{0.078740in}}{\pgfqpoint{7.842520in}{7.842520in}}%
\pgfusepath{clip}%
\pgfsetbuttcap%
\pgfsetroundjoin%
\definecolor{currentfill}{rgb}{0.221989,0.339161,0.548752}%
\pgfsetfillcolor{currentfill}%
\pgfsetlinewidth{0.000000pt}%
\definecolor{currentstroke}{rgb}{0.223925,0.334994,0.548053}%
\pgfsetstrokecolor{currentstroke}%
\pgfsetdash{}{0pt}%
\pgfpathmoveto{\pgfqpoint{5.098620in}{3.762206in}}%
\pgfpathlineto{\pgfqpoint{5.235775in}{3.632407in}}%
\pgfpathlineto{\pgfqpoint{5.175033in}{3.721001in}}%
\pgfpathclose%
\pgfusepath{fill}%
\end{pgfscope}%
\begin{pgfscope}%
\pgfpathrectangle{\pgfqpoint{0.539299in}{0.078740in}}{\pgfqpoint{7.842520in}{7.842520in}}%
\pgfusepath{clip}%
\pgfsetbuttcap%
\pgfsetroundjoin%
\definecolor{currentfill}{rgb}{0.271828,0.209303,0.504434}%
\pgfsetfillcolor{currentfill}%
\pgfsetlinewidth{0.000000pt}%
\definecolor{currentstroke}{rgb}{0.221989,0.339161,0.548752}%
\pgfsetstrokecolor{currentstroke}%
\pgfsetdash{}{0pt}%
\pgfpathmoveto{\pgfqpoint{6.075188in}{3.222781in}}%
\pgfpathlineto{\pgfqpoint{5.936847in}{3.290164in}}%
\pgfpathlineto{\pgfqpoint{6.000832in}{3.178458in}}%
\pgfpathclose%
\pgfusepath{fill}%
\end{pgfscope}%
\begin{pgfscope}%
\pgfpathrectangle{\pgfqpoint{0.539299in}{0.078740in}}{\pgfqpoint{7.842520in}{7.842520in}}%
\pgfusepath{clip}%
\pgfsetbuttcap%
\pgfsetroundjoin%
\definecolor{currentfill}{rgb}{0.280255,0.165693,0.476498}%
\pgfsetfillcolor{currentfill}%
\pgfsetlinewidth{0.000000pt}%
\definecolor{currentstroke}{rgb}{0.220057,0.343307,0.549413}%
\pgfsetstrokecolor{currentstroke}%
\pgfsetdash{}{0pt}%
\pgfpathmoveto{\pgfqpoint{6.352735in}{3.080591in}}%
\pgfpathlineto{\pgfqpoint{6.564795in}{3.052267in}}%
\pgfpathlineto{\pgfqpoint{6.426028in}{3.129202in}}%
\pgfpathclose%
\pgfusepath{fill}%
\end{pgfscope}%
\begin{pgfscope}%
\pgfpathrectangle{\pgfqpoint{0.539299in}{0.078740in}}{\pgfqpoint{7.842520in}{7.842520in}}%
\pgfusepath{clip}%
\pgfsetbuttcap%
\pgfsetroundjoin%
\definecolor{currentfill}{rgb}{0.119738,0.603785,0.541400}%
\pgfsetfillcolor{currentfill}%
\pgfsetlinewidth{0.000000pt}%
\definecolor{currentstroke}{rgb}{0.218130,0.347432,0.550038}%
\pgfsetstrokecolor{currentstroke}%
\pgfsetdash{}{0pt}%
\pgfpathmoveto{\pgfqpoint{2.869477in}{4.725189in}}%
\pgfpathlineto{\pgfqpoint{2.996415in}{5.025431in}}%
\pgfpathlineto{\pgfqpoint{2.954036in}{4.737238in}}%
\pgfpathclose%
\pgfusepath{fill}%
\end{pgfscope}%
\begin{pgfscope}%
\pgfpathrectangle{\pgfqpoint{0.539299in}{0.078740in}}{\pgfqpoint{7.842520in}{7.842520in}}%
\pgfusepath{clip}%
\pgfsetbuttcap%
\pgfsetroundjoin%
\definecolor{currentfill}{rgb}{0.187231,0.414746,0.556547}%
\pgfsetfillcolor{currentfill}%
\pgfsetlinewidth{0.000000pt}%
\definecolor{currentstroke}{rgb}{0.216210,0.351535,0.550627}%
\pgfsetstrokecolor{currentstroke}%
\pgfsetdash{}{0pt}%
\pgfpathmoveto{\pgfqpoint{2.491969in}{4.215044in}}%
\pgfpathlineto{\pgfqpoint{2.290658in}{3.630794in}}%
\pgfpathlineto{\pgfqpoint{2.406429in}{4.162525in}}%
\pgfpathclose%
\pgfusepath{fill}%
\end{pgfscope}%
\begin{pgfscope}%
\pgfpathrectangle{\pgfqpoint{0.539299in}{0.078740in}}{\pgfqpoint{7.842520in}{7.842520in}}%
\pgfusepath{clip}%
\pgfsetbuttcap%
\pgfsetroundjoin%
\definecolor{currentfill}{rgb}{0.283197,0.115680,0.436115}%
\pgfsetfillcolor{currentfill}%
\pgfsetlinewidth{0.000000pt}%
\definecolor{currentstroke}{rgb}{0.214298,0.355619,0.551184}%
\pgfsetstrokecolor{currentstroke}%
\pgfsetdash{}{0pt}%
\pgfpathmoveto{\pgfqpoint{6.703683in}{2.968722in}}%
\pgfpathlineto{\pgfqpoint{6.842661in}{2.878816in}}%
\pgfpathlineto{\pgfqpoint{6.914313in}{2.924367in}}%
\pgfpathclose%
\pgfusepath{fill}%
\end{pgfscope}%
\begin{pgfscope}%
\pgfpathrectangle{\pgfqpoint{0.539299in}{0.078740in}}{\pgfqpoint{7.842520in}{7.842520in}}%
\pgfusepath{clip}%
\pgfsetbuttcap%
\pgfsetroundjoin%
\definecolor{currentfill}{rgb}{0.235526,0.309527,0.542944}%
\pgfsetfillcolor{currentfill}%
\pgfsetlinewidth{0.000000pt}%
\definecolor{currentstroke}{rgb}{0.212395,0.359683,0.551710}%
\pgfsetstrokecolor{currentstroke}%
\pgfsetdash{}{0pt}%
\pgfpathmoveto{\pgfqpoint{5.235775in}{3.632407in}}%
\pgfpathlineto{\pgfqpoint{5.373172in}{3.514648in}}%
\pgfpathlineto{\pgfqpoint{5.311793in}{3.607895in}}%
\pgfpathclose%
\pgfusepath{fill}%
\end{pgfscope}%
\begin{pgfscope}%
\pgfpathrectangle{\pgfqpoint{0.539299in}{0.078740in}}{\pgfqpoint{7.842520in}{7.842520in}}%
\pgfusepath{clip}%
\pgfsetbuttcap%
\pgfsetroundjoin%
\definecolor{currentfill}{rgb}{0.137339,0.662252,0.515571}%
\pgfsetfillcolor{currentfill}%
\pgfsetlinewidth{0.000000pt}%
\definecolor{currentstroke}{rgb}{0.210503,0.363727,0.552206}%
\pgfsetstrokecolor{currentstroke}%
\pgfsetdash{}{0pt}%
\pgfpathmoveto{\pgfqpoint{3.296607in}{5.181085in}}%
\pgfpathlineto{\pgfqpoint{3.165894in}{5.004406in}}%
\pgfpathlineto{\pgfqpoint{3.081416in}{5.017996in}}%
\pgfpathclose%
\pgfusepath{fill}%
\end{pgfscope}%
\begin{pgfscope}%
\pgfpathrectangle{\pgfqpoint{0.539299in}{0.078740in}}{\pgfqpoint{7.842520in}{7.842520in}}%
\pgfusepath{clip}%
\pgfsetbuttcap%
\pgfsetroundjoin%
\definecolor{currentfill}{rgb}{0.270595,0.214069,0.507052}%
\pgfsetfillcolor{currentfill}%
\pgfsetlinewidth{0.000000pt}%
\definecolor{currentstroke}{rgb}{0.208623,0.367752,0.552675}%
\pgfsetstrokecolor{currentstroke}%
\pgfsetdash{}{0pt}%
\pgfpathmoveto{\pgfqpoint{6.000832in}{3.178458in}}%
\pgfpathlineto{\pgfqpoint{5.936847in}{3.290164in}}%
\pgfpathlineto{\pgfqpoint{5.862282in}{3.253769in}}%
\pgfpathclose%
\pgfusepath{fill}%
\end{pgfscope}%
\begin{pgfscope}%
\pgfpathrectangle{\pgfqpoint{0.539299in}{0.078740in}}{\pgfqpoint{7.842520in}{7.842520in}}%
\pgfusepath{clip}%
\pgfsetbuttcap%
\pgfsetroundjoin%
\definecolor{currentfill}{rgb}{0.255645,0.260703,0.528312}%
\pgfsetfillcolor{currentfill}%
\pgfsetlinewidth{0.000000pt}%
\definecolor{currentstroke}{rgb}{0.206756,0.371758,0.553117}%
\pgfsetstrokecolor{currentstroke}%
\pgfsetdash{}{0pt}%
\pgfpathmoveto{\pgfqpoint{5.586291in}{3.415267in}}%
\pgfpathlineto{\pgfqpoint{5.510865in}{3.407873in}}%
\pgfpathlineto{\pgfqpoint{5.724096in}{3.331845in}}%
\pgfpathclose%
\pgfusepath{fill}%
\end{pgfscope}%
\begin{pgfscope}%
\pgfpathrectangle{\pgfqpoint{0.539299in}{0.078740in}}{\pgfqpoint{7.842520in}{7.842520in}}%
\pgfusepath{clip}%
\pgfsetbuttcap%
\pgfsetroundjoin%
\definecolor{currentfill}{rgb}{0.280267,0.073417,0.397163}%
\pgfsetfillcolor{currentfill}%
\pgfsetlinewidth{0.000000pt}%
\definecolor{currentstroke}{rgb}{0.204903,0.375746,0.553533}%
\pgfsetstrokecolor{currentstroke}%
\pgfsetdash{}{0pt}%
\pgfpathmoveto{\pgfqpoint{7.192040in}{2.737188in}}%
\pgfpathlineto{\pgfqpoint{7.053093in}{2.832020in}}%
\pgfpathlineto{\pgfqpoint{6.981719in}{2.783278in}}%
\pgfpathclose%
\pgfusepath{fill}%
\end{pgfscope}%
\begin{pgfscope}%
\pgfpathrectangle{\pgfqpoint{0.539299in}{0.078740in}}{\pgfqpoint{7.842520in}{7.842520in}}%
\pgfusepath{clip}%
\pgfsetbuttcap%
\pgfsetroundjoin%
\definecolor{currentfill}{rgb}{0.274128,0.199721,0.498911}%
\pgfsetfillcolor{currentfill}%
\pgfsetlinewidth{0.000000pt}%
\definecolor{currentstroke}{rgb}{0.203063,0.379716,0.553925}%
\pgfsetstrokecolor{currentstroke}%
\pgfsetdash{}{0pt}%
\pgfpathmoveto{\pgfqpoint{6.000832in}{3.178458in}}%
\pgfpathlineto{\pgfqpoint{6.213837in}{3.153582in}}%
\pgfpathlineto{\pgfqpoint{6.075188in}{3.222781in}}%
\pgfpathclose%
\pgfusepath{fill}%
\end{pgfscope}%
\begin{pgfscope}%
\pgfpathrectangle{\pgfqpoint{0.539299in}{0.078740in}}{\pgfqpoint{7.842520in}{7.842520in}}%
\pgfusepath{clip}%
\pgfsetbuttcap%
\pgfsetroundjoin%
\definecolor{currentfill}{rgb}{0.122312,0.633153,0.530398}%
\pgfsetfillcolor{currentfill}%
\pgfsetlinewidth{0.000000pt}%
\definecolor{currentstroke}{rgb}{0.201239,0.383670,0.554294}%
\pgfsetstrokecolor{currentstroke}%
\pgfsetdash{}{0pt}%
\pgfpathmoveto{\pgfqpoint{2.954036in}{4.737238in}}%
\pgfpathlineto{\pgfqpoint{2.996415in}{5.025431in}}%
\pgfpathlineto{\pgfqpoint{3.081416in}{5.017996in}}%
\pgfpathclose%
\pgfusepath{fill}%
\end{pgfscope}%
\begin{pgfscope}%
\pgfpathrectangle{\pgfqpoint{0.539299in}{0.078740in}}{\pgfqpoint{7.842520in}{7.842520in}}%
\pgfusepath{clip}%
\pgfsetbuttcap%
\pgfsetroundjoin%
\definecolor{currentfill}{rgb}{0.281887,0.150881,0.465405}%
\pgfsetfillcolor{currentfill}%
\pgfsetlinewidth{0.000000pt}%
\definecolor{currentstroke}{rgb}{0.199430,0.387607,0.554642}%
\pgfsetstrokecolor{currentstroke}%
\pgfsetdash{}{0pt}%
\pgfpathmoveto{\pgfqpoint{6.703683in}{2.968722in}}%
\pgfpathlineto{\pgfqpoint{6.564795in}{3.052267in}}%
\pgfpathlineto{\pgfqpoint{6.491826in}{3.002343in}}%
\pgfpathclose%
\pgfusepath{fill}%
\end{pgfscope}%
\begin{pgfscope}%
\pgfpathrectangle{\pgfqpoint{0.539299in}{0.078740in}}{\pgfqpoint{7.842520in}{7.842520in}}%
\pgfusepath{clip}%
\pgfsetbuttcap%
\pgfsetroundjoin%
\definecolor{currentfill}{rgb}{0.175707,0.697900,0.491033}%
\pgfsetfillcolor{currentfill}%
\pgfsetlinewidth{0.000000pt}%
\definecolor{currentstroke}{rgb}{0.197636,0.391528,0.554969}%
\pgfsetstrokecolor{currentstroke}%
\pgfsetdash{}{0pt}%
\pgfpathmoveto{\pgfqpoint{3.648554in}{5.237108in}}%
\pgfpathlineto{\pgfqpoint{3.731742in}{5.157127in}}%
\pgfpathlineto{\pgfqpoint{3.513793in}{5.222466in}}%
\pgfpathclose%
\pgfusepath{fill}%
\end{pgfscope}%
\begin{pgfscope}%
\pgfpathrectangle{\pgfqpoint{0.539299in}{0.078740in}}{\pgfqpoint{7.842520in}{7.842520in}}%
\pgfusepath{clip}%
\pgfsetbuttcap%
\pgfsetroundjoin%
\definecolor{currentfill}{rgb}{0.282327,0.094955,0.417331}%
\pgfsetfillcolor{currentfill}%
\pgfsetlinewidth{0.000000pt}%
\definecolor{currentstroke}{rgb}{0.195860,0.395433,0.555276}%
\pgfsetstrokecolor{currentstroke}%
\pgfsetdash{}{0pt}%
\pgfpathmoveto{\pgfqpoint{6.842661in}{2.878816in}}%
\pgfpathlineto{\pgfqpoint{6.981719in}{2.783278in}}%
\pgfpathlineto{\pgfqpoint{7.053093in}{2.832020in}}%
\pgfpathclose%
\pgfusepath{fill}%
\end{pgfscope}%
\begin{pgfscope}%
\pgfpathrectangle{\pgfqpoint{0.539299in}{0.078740in}}{\pgfqpoint{7.842520in}{7.842520in}}%
\pgfusepath{clip}%
\pgfsetbuttcap%
\pgfsetroundjoin%
\definecolor{currentfill}{rgb}{0.157851,0.683765,0.501686}%
\pgfsetfillcolor{currentfill}%
\pgfsetlinewidth{0.000000pt}%
\definecolor{currentstroke}{rgb}{0.194100,0.399323,0.555565}%
\pgfsetstrokecolor{currentstroke}%
\pgfsetdash{}{0pt}%
\pgfpathmoveto{\pgfqpoint{3.784535in}{5.195377in}}%
\pgfpathlineto{\pgfqpoint{3.867334in}{5.102013in}}%
\pgfpathlineto{\pgfqpoint{3.731742in}{5.157127in}}%
\pgfpathclose%
\pgfusepath{fill}%
\end{pgfscope}%
\begin{pgfscope}%
\pgfpathrectangle{\pgfqpoint{0.539299in}{0.078740in}}{\pgfqpoint{7.842520in}{7.842520in}}%
\pgfusepath{clip}%
\pgfsetbuttcap%
\pgfsetroundjoin%
\definecolor{currentfill}{rgb}{0.248629,0.278775,0.534556}%
\pgfsetfillcolor{currentfill}%
\pgfsetlinewidth{0.000000pt}%
\definecolor{currentstroke}{rgb}{0.192357,0.403199,0.555836}%
\pgfsetstrokecolor{currentstroke}%
\pgfsetdash{}{0pt}%
\pgfpathmoveto{\pgfqpoint{5.373172in}{3.514648in}}%
\pgfpathlineto{\pgfqpoint{5.510865in}{3.407873in}}%
\pgfpathlineto{\pgfqpoint{5.586291in}{3.415267in}}%
\pgfpathclose%
\pgfusepath{fill}%
\end{pgfscope}%
\begin{pgfscope}%
\pgfpathrectangle{\pgfqpoint{0.539299in}{0.078740in}}{\pgfqpoint{7.842520in}{7.842520in}}%
\pgfusepath{clip}%
\pgfsetbuttcap%
\pgfsetroundjoin%
\definecolor{currentfill}{rgb}{0.280868,0.160771,0.472899}%
\pgfsetfillcolor{currentfill}%
\pgfsetlinewidth{0.000000pt}%
\definecolor{currentstroke}{rgb}{0.190631,0.407061,0.556089}%
\pgfsetstrokecolor{currentstroke}%
\pgfsetdash{}{0pt}%
\pgfpathmoveto{\pgfqpoint{6.491826in}{3.002343in}}%
\pgfpathlineto{\pgfqpoint{6.564795in}{3.052267in}}%
\pgfpathlineto{\pgfqpoint{6.352735in}{3.080591in}}%
\pgfpathclose%
\pgfusepath{fill}%
\end{pgfscope}%
\begin{pgfscope}%
\pgfpathrectangle{\pgfqpoint{0.539299in}{0.078740in}}{\pgfqpoint{7.842520in}{7.842520in}}%
\pgfusepath{clip}%
\pgfsetbuttcap%
\pgfsetroundjoin%
\definecolor{currentfill}{rgb}{0.180653,0.701402,0.488189}%
\pgfsetfillcolor{currentfill}%
\pgfsetlinewidth{0.000000pt}%
\definecolor{currentstroke}{rgb}{0.188923,0.410910,0.556326}%
\pgfsetstrokecolor{currentstroke}%
\pgfsetdash{}{0pt}%
\pgfpathmoveto{\pgfqpoint{3.380711in}{5.141597in}}%
\pgfpathlineto{\pgfqpoint{3.429695in}{5.281129in}}%
\pgfpathlineto{\pgfqpoint{3.513793in}{5.222466in}}%
\pgfpathclose%
\pgfusepath{fill}%
\end{pgfscope}%
\begin{pgfscope}%
\pgfpathrectangle{\pgfqpoint{0.539299in}{0.078740in}}{\pgfqpoint{7.842520in}{7.842520in}}%
\pgfusepath{clip}%
\pgfsetbuttcap%
\pgfsetroundjoin%
\definecolor{currentfill}{rgb}{0.276022,0.044167,0.370164}%
\pgfsetfillcolor{currentfill}%
\pgfsetlinewidth{0.000000pt}%
\definecolor{currentstroke}{rgb}{0.187231,0.414746,0.556547}%
\pgfsetstrokecolor{currentstroke}%
\pgfsetdash{}{0pt}%
\pgfpathmoveto{\pgfqpoint{7.331215in}{2.641797in}}%
\pgfpathlineto{\pgfqpoint{7.192040in}{2.737188in}}%
\pgfpathlineto{\pgfqpoint{7.120863in}{2.683197in}}%
\pgfpathclose%
\pgfusepath{fill}%
\end{pgfscope}%
\begin{pgfscope}%
\pgfpathrectangle{\pgfqpoint{0.539299in}{0.078740in}}{\pgfqpoint{7.842520in}{7.842520in}}%
\pgfusepath{clip}%
\pgfsetbuttcap%
\pgfsetroundjoin%
\definecolor{currentfill}{rgb}{0.262138,0.242286,0.520837}%
\pgfsetfillcolor{currentfill}%
\pgfsetlinewidth{0.000000pt}%
\definecolor{currentstroke}{rgb}{0.185556,0.418570,0.556753}%
\pgfsetstrokecolor{currentstroke}%
\pgfsetdash{}{0pt}%
\pgfpathmoveto{\pgfqpoint{5.862282in}{3.253769in}}%
\pgfpathlineto{\pgfqpoint{5.724096in}{3.331845in}}%
\pgfpathlineto{\pgfqpoint{5.648894in}{3.310389in}}%
\pgfpathclose%
\pgfusepath{fill}%
\end{pgfscope}%
\begin{pgfscope}%
\pgfpathrectangle{\pgfqpoint{0.539299in}{0.078740in}}{\pgfqpoint{7.842520in}{7.842520in}}%
\pgfusepath{clip}%
\pgfsetbuttcap%
\pgfsetroundjoin%
\definecolor{currentfill}{rgb}{0.147607,0.511733,0.557049}%
\pgfsetfillcolor{currentfill}%
\pgfsetlinewidth{0.000000pt}%
\definecolor{currentstroke}{rgb}{0.183898,0.422383,0.556944}%
\pgfsetstrokecolor{currentstroke}%
\pgfsetdash{}{0pt}%
\pgfpathmoveto{\pgfqpoint{2.699021in}{4.682479in}}%
\pgfpathlineto{\pgfqpoint{2.577164in}{4.260977in}}%
\pgfpathlineto{\pgfqpoint{2.491969in}{4.215044in}}%
\pgfpathclose%
\pgfusepath{fill}%
\end{pgfscope}%
\begin{pgfscope}%
\pgfpathrectangle{\pgfqpoint{0.539299in}{0.078740in}}{\pgfqpoint{7.842520in}{7.842520in}}%
\pgfusepath{clip}%
\pgfsetbuttcap%
\pgfsetroundjoin%
\definecolor{currentfill}{rgb}{0.175707,0.697900,0.491033}%
\pgfsetfillcolor{currentfill}%
\pgfsetlinewidth{0.000000pt}%
\definecolor{currentstroke}{rgb}{0.182256,0.426184,0.557120}%
\pgfsetstrokecolor{currentstroke}%
\pgfsetdash{}{0pt}%
\pgfpathmoveto{\pgfqpoint{3.296607in}{5.181085in}}%
\pgfpathlineto{\pgfqpoint{3.429695in}{5.281129in}}%
\pgfpathlineto{\pgfqpoint{3.380711in}{5.141597in}}%
\pgfpathclose%
\pgfusepath{fill}%
\end{pgfscope}%
\begin{pgfscope}%
\pgfpathrectangle{\pgfqpoint{0.539299in}{0.078740in}}{\pgfqpoint{7.842520in}{7.842520in}}%
\pgfusepath{clip}%
\pgfsetbuttcap%
\pgfsetroundjoin%
\definecolor{currentfill}{rgb}{0.277134,0.185228,0.489898}%
\pgfsetfillcolor{currentfill}%
\pgfsetlinewidth{0.000000pt}%
\definecolor{currentstroke}{rgb}{0.180629,0.429975,0.557282}%
\pgfsetstrokecolor{currentstroke}%
\pgfsetdash{}{0pt}%
\pgfpathmoveto{\pgfqpoint{6.352735in}{3.080591in}}%
\pgfpathlineto{\pgfqpoint{6.213837in}{3.153582in}}%
\pgfpathlineto{\pgfqpoint{6.139708in}{3.103444in}}%
\pgfpathclose%
\pgfusepath{fill}%
\end{pgfscope}%
\begin{pgfscope}%
\pgfpathrectangle{\pgfqpoint{0.539299in}{0.078740in}}{\pgfqpoint{7.842520in}{7.842520in}}%
\pgfusepath{clip}%
\pgfsetbuttcap%
\pgfsetroundjoin%
\definecolor{currentfill}{rgb}{0.185556,0.418570,0.556753}%
\pgfsetfillcolor{currentfill}%
\pgfsetlinewidth{0.000000pt}%
\definecolor{currentstroke}{rgb}{0.179019,0.433756,0.557430}%
\pgfsetstrokecolor{currentstroke}%
\pgfsetdash{}{0pt}%
\pgfpathmoveto{\pgfqpoint{2.406429in}{4.162525in}}%
\pgfpathlineto{\pgfqpoint{2.290658in}{3.630794in}}%
\pgfpathlineto{\pgfqpoint{2.320596in}{4.102334in}}%
\pgfpathclose%
\pgfusepath{fill}%
\end{pgfscope}%
\begin{pgfscope}%
\pgfpathrectangle{\pgfqpoint{0.539299in}{0.078740in}}{\pgfqpoint{7.842520in}{7.842520in}}%
\pgfusepath{clip}%
\pgfsetbuttcap%
\pgfsetroundjoin%
\definecolor{currentfill}{rgb}{0.279566,0.067836,0.391917}%
\pgfsetfillcolor{currentfill}%
\pgfsetlinewidth{0.000000pt}%
\definecolor{currentstroke}{rgb}{0.177423,0.437527,0.557565}%
\pgfsetstrokecolor{currentstroke}%
\pgfsetdash{}{0pt}%
\pgfpathmoveto{\pgfqpoint{7.120863in}{2.683197in}}%
\pgfpathlineto{\pgfqpoint{7.192040in}{2.737188in}}%
\pgfpathlineto{\pgfqpoint{6.981719in}{2.783278in}}%
\pgfpathclose%
\pgfusepath{fill}%
\end{pgfscope}%
\begin{pgfscope}%
\pgfpathrectangle{\pgfqpoint{0.539299in}{0.078740in}}{\pgfqpoint{7.842520in}{7.842520in}}%
\pgfusepath{clip}%
\pgfsetbuttcap%
\pgfsetroundjoin%
\definecolor{currentfill}{rgb}{0.274128,0.199721,0.498911}%
\pgfsetfillcolor{currentfill}%
\pgfsetlinewidth{0.000000pt}%
\definecolor{currentstroke}{rgb}{0.175841,0.441290,0.557685}%
\pgfsetstrokecolor{currentstroke}%
\pgfsetdash{}{0pt}%
\pgfpathmoveto{\pgfqpoint{6.139708in}{3.103444in}}%
\pgfpathlineto{\pgfqpoint{6.213837in}{3.153582in}}%
\pgfpathlineto{\pgfqpoint{6.000832in}{3.178458in}}%
\pgfpathclose%
\pgfusepath{fill}%
\end{pgfscope}%
\begin{pgfscope}%
\pgfpathrectangle{\pgfqpoint{0.539299in}{0.078740in}}{\pgfqpoint{7.842520in}{7.842520in}}%
\pgfusepath{clip}%
\pgfsetbuttcap%
\pgfsetroundjoin%
\definecolor{currentfill}{rgb}{0.132444,0.552216,0.553018}%
\pgfsetfillcolor{currentfill}%
\pgfsetlinewidth{0.000000pt}%
\definecolor{currentstroke}{rgb}{0.174274,0.445044,0.557792}%
\pgfsetstrokecolor{currentstroke}%
\pgfsetdash{}{0pt}%
\pgfpathmoveto{\pgfqpoint{2.577164in}{4.260977in}}%
\pgfpathlineto{\pgfqpoint{2.699021in}{4.682479in}}%
\pgfpathlineto{\pgfqpoint{2.784458in}{4.707293in}}%
\pgfpathclose%
\pgfusepath{fill}%
\end{pgfscope}%
\begin{pgfscope}%
\pgfpathrectangle{\pgfqpoint{0.539299in}{0.078740in}}{\pgfqpoint{7.842520in}{7.842520in}}%
\pgfusepath{clip}%
\pgfsetbuttcap%
\pgfsetroundjoin%
\definecolor{currentfill}{rgb}{0.255645,0.260703,0.528312}%
\pgfsetfillcolor{currentfill}%
\pgfsetlinewidth{0.000000pt}%
\definecolor{currentstroke}{rgb}{0.172719,0.448791,0.557885}%
\pgfsetstrokecolor{currentstroke}%
\pgfsetdash{}{0pt}%
\pgfpathmoveto{\pgfqpoint{5.724096in}{3.331845in}}%
\pgfpathlineto{\pgfqpoint{5.510865in}{3.407873in}}%
\pgfpathlineto{\pgfqpoint{5.648894in}{3.310389in}}%
\pgfpathclose%
\pgfusepath{fill}%
\end{pgfscope}%
\begin{pgfscope}%
\pgfpathrectangle{\pgfqpoint{0.539299in}{0.078740in}}{\pgfqpoint{7.842520in}{7.842520in}}%
\pgfusepath{clip}%
\pgfsetbuttcap%
\pgfsetroundjoin%
\definecolor{currentfill}{rgb}{0.146616,0.673050,0.508936}%
\pgfsetfillcolor{currentfill}%
\pgfsetlinewidth{0.000000pt}%
\definecolor{currentstroke}{rgb}{0.171176,0.452530,0.557965}%
\pgfsetstrokecolor{currentstroke}%
\pgfsetdash{}{0pt}%
\pgfpathmoveto{\pgfqpoint{4.003621in}{5.004426in}}%
\pgfpathlineto{\pgfqpoint{3.867334in}{5.102013in}}%
\pgfpathlineto{\pgfqpoint{3.921341in}{5.107578in}}%
\pgfpathclose%
\pgfusepath{fill}%
\end{pgfscope}%
\begin{pgfscope}%
\pgfpathrectangle{\pgfqpoint{0.539299in}{0.078740in}}{\pgfqpoint{7.842520in}{7.842520in}}%
\pgfusepath{clip}%
\pgfsetbuttcap%
\pgfsetroundjoin%
\definecolor{currentfill}{rgb}{0.119483,0.614817,0.537692}%
\pgfsetfillcolor{currentfill}%
\pgfsetlinewidth{0.000000pt}%
\definecolor{currentstroke}{rgb}{0.169646,0.456262,0.558030}%
\pgfsetstrokecolor{currentstroke}%
\pgfsetdash{}{0pt}%
\pgfpathmoveto{\pgfqpoint{2.784458in}{4.707293in}}%
\pgfpathlineto{\pgfqpoint{2.996415in}{5.025431in}}%
\pgfpathlineto{\pgfqpoint{2.869477in}{4.725189in}}%
\pgfpathclose%
\pgfusepath{fill}%
\end{pgfscope}%
\begin{pgfscope}%
\pgfpathrectangle{\pgfqpoint{0.539299in}{0.078740in}}{\pgfqpoint{7.842520in}{7.842520in}}%
\pgfusepath{clip}%
\pgfsetbuttcap%
\pgfsetroundjoin%
\definecolor{currentfill}{rgb}{0.283229,0.120777,0.440584}%
\pgfsetfillcolor{currentfill}%
\pgfsetlinewidth{0.000000pt}%
\definecolor{currentstroke}{rgb}{0.168126,0.459988,0.558082}%
\pgfsetstrokecolor{currentstroke}%
\pgfsetdash{}{0pt}%
\pgfpathmoveto{\pgfqpoint{6.770368in}{2.826882in}}%
\pgfpathlineto{\pgfqpoint{6.842661in}{2.878816in}}%
\pgfpathlineto{\pgfqpoint{6.703683in}{2.968722in}}%
\pgfpathclose%
\pgfusepath{fill}%
\end{pgfscope}%
\begin{pgfscope}%
\pgfpathrectangle{\pgfqpoint{0.539299in}{0.078740in}}{\pgfqpoint{7.842520in}{7.842520in}}%
\pgfusepath{clip}%
\pgfsetbuttcap%
\pgfsetroundjoin%
\definecolor{currentfill}{rgb}{0.180653,0.701402,0.488189}%
\pgfsetfillcolor{currentfill}%
\pgfsetlinewidth{0.000000pt}%
\definecolor{currentstroke}{rgb}{0.166617,0.463708,0.558119}%
\pgfsetstrokecolor{currentstroke}%
\pgfsetdash{}{0pt}%
\pgfpathmoveto{\pgfqpoint{3.784535in}{5.195377in}}%
\pgfpathlineto{\pgfqpoint{3.731742in}{5.157127in}}%
\pgfpathlineto{\pgfqpoint{3.648554in}{5.237108in}}%
\pgfpathclose%
\pgfusepath{fill}%
\end{pgfscope}%
\begin{pgfscope}%
\pgfpathrectangle{\pgfqpoint{0.539299in}{0.078740in}}{\pgfqpoint{7.842520in}{7.842520in}}%
\pgfusepath{clip}%
\pgfsetbuttcap%
\pgfsetroundjoin%
\definecolor{currentfill}{rgb}{0.282290,0.145912,0.461510}%
\pgfsetfillcolor{currentfill}%
\pgfsetlinewidth{0.000000pt}%
\definecolor{currentstroke}{rgb}{0.165117,0.467423,0.558141}%
\pgfsetstrokecolor{currentstroke}%
\pgfsetdash{}{0pt}%
\pgfpathmoveto{\pgfqpoint{6.491826in}{3.002343in}}%
\pgfpathlineto{\pgfqpoint{6.631052in}{2.917912in}}%
\pgfpathlineto{\pgfqpoint{6.703683in}{2.968722in}}%
\pgfpathclose%
\pgfusepath{fill}%
\end{pgfscope}%
\begin{pgfscope}%
\pgfpathrectangle{\pgfqpoint{0.539299in}{0.078740in}}{\pgfqpoint{7.842520in}{7.842520in}}%
\pgfusepath{clip}%
\pgfsetbuttcap%
\pgfsetroundjoin%
\definecolor{currentfill}{rgb}{0.183898,0.422383,0.556944}%
\pgfsetfillcolor{currentfill}%
\pgfsetlinewidth{0.000000pt}%
\definecolor{currentstroke}{rgb}{0.163625,0.471133,0.558148}%
\pgfsetstrokecolor{currentstroke}%
\pgfsetdash{}{0pt}%
\pgfpathmoveto{\pgfqpoint{4.961637in}{3.904256in}}%
\pgfpathlineto{\pgfqpoint{4.824755in}{4.057714in}}%
\pgfpathlineto{\pgfqpoint{4.884097in}{3.971926in}}%
\pgfpathclose%
\pgfusepath{fill}%
\end{pgfscope}%
\begin{pgfscope}%
\pgfpathrectangle{\pgfqpoint{0.539299in}{0.078740in}}{\pgfqpoint{7.842520in}{7.842520in}}%
\pgfusepath{clip}%
\pgfsetbuttcap%
\pgfsetroundjoin%
\definecolor{currentfill}{rgb}{0.194100,0.399323,0.555565}%
\pgfsetfillcolor{currentfill}%
\pgfsetlinewidth{0.000000pt}%
\definecolor{currentstroke}{rgb}{0.162142,0.474838,0.558140}%
\pgfsetstrokecolor{currentstroke}%
\pgfsetdash{}{0pt}%
\pgfpathmoveto{\pgfqpoint{4.884097in}{3.971926in}}%
\pgfpathlineto{\pgfqpoint{5.098620in}{3.762206in}}%
\pgfpathlineto{\pgfqpoint{4.961637in}{3.904256in}}%
\pgfpathclose%
\pgfusepath{fill}%
\end{pgfscope}%
\begin{pgfscope}%
\pgfpathrectangle{\pgfqpoint{0.539299in}{0.078740in}}{\pgfqpoint{7.842520in}{7.842520in}}%
\pgfusepath{clip}%
\pgfsetbuttcap%
\pgfsetroundjoin%
\definecolor{currentfill}{rgb}{0.162142,0.474838,0.558140}%
\pgfsetfillcolor{currentfill}%
\pgfsetlinewidth{0.000000pt}%
\definecolor{currentstroke}{rgb}{0.160665,0.478540,0.558115}%
\pgfsetstrokecolor{currentstroke}%
\pgfsetdash{}{0pt}%
\pgfpathmoveto{\pgfqpoint{4.609098in}{4.314481in}}%
\pgfpathlineto{\pgfqpoint{4.824755in}{4.057714in}}%
\pgfpathlineto{\pgfqpoint{4.687905in}{4.220515in}}%
\pgfpathclose%
\pgfusepath{fill}%
\end{pgfscope}%
\begin{pgfscope}%
\pgfpathrectangle{\pgfqpoint{0.539299in}{0.078740in}}{\pgfqpoint{7.842520in}{7.842520in}}%
\pgfusepath{clip}%
\pgfsetbuttcap%
\pgfsetroundjoin%
\definecolor{currentfill}{rgb}{0.214298,0.355619,0.551184}%
\pgfsetfillcolor{currentfill}%
\pgfsetlinewidth{0.000000pt}%
\definecolor{currentstroke}{rgb}{0.159194,0.482237,0.558073}%
\pgfsetstrokecolor{currentstroke}%
\pgfsetdash{}{0pt}%
\pgfpathmoveto{\pgfqpoint{2.205440in}{3.565187in}}%
\pgfpathlineto{\pgfqpoint{2.119879in}{3.494959in}}%
\pgfpathlineto{\pgfqpoint{2.234531in}{4.033237in}}%
\pgfpathclose%
\pgfusepath{fill}%
\end{pgfscope}%
\begin{pgfscope}%
\pgfpathrectangle{\pgfqpoint{0.539299in}{0.078740in}}{\pgfqpoint{7.842520in}{7.842520in}}%
\pgfusepath{clip}%
\pgfsetbuttcap%
\pgfsetroundjoin%
\definecolor{currentfill}{rgb}{0.151918,0.500685,0.557587}%
\pgfsetfillcolor{currentfill}%
\pgfsetlinewidth{0.000000pt}%
\definecolor{currentstroke}{rgb}{0.157729,0.485932,0.558013}%
\pgfsetstrokecolor{currentstroke}%
\pgfsetdash{}{0pt}%
\pgfpathmoveto{\pgfqpoint{4.687905in}{4.220515in}}%
\pgfpathlineto{\pgfqpoint{4.551037in}{4.389232in}}%
\pgfpathlineto{\pgfqpoint{4.609098in}{4.314481in}}%
\pgfpathclose%
\pgfusepath{fill}%
\end{pgfscope}%
\begin{pgfscope}%
\pgfpathrectangle{\pgfqpoint{0.539299in}{0.078740in}}{\pgfqpoint{7.842520in}{7.842520in}}%
\pgfusepath{clip}%
\pgfsetbuttcap%
\pgfsetroundjoin%
\definecolor{currentfill}{rgb}{0.134692,0.658636,0.517649}%
\pgfsetfillcolor{currentfill}%
\pgfsetlinewidth{0.000000pt}%
\definecolor{currentstroke}{rgb}{0.156270,0.489624,0.557936}%
\pgfsetstrokecolor{currentstroke}%
\pgfsetdash{}{0pt}%
\pgfpathmoveto{\pgfqpoint{3.921341in}{5.107578in}}%
\pgfpathlineto{\pgfqpoint{4.140312in}{4.874931in}}%
\pgfpathlineto{\pgfqpoint{4.003621in}{5.004426in}}%
\pgfpathclose%
\pgfusepath{fill}%
\end{pgfscope}%
\begin{pgfscope}%
\pgfpathrectangle{\pgfqpoint{0.539299in}{0.078740in}}{\pgfqpoint{7.842520in}{7.842520in}}%
\pgfusepath{clip}%
\pgfsetbuttcap%
\pgfsetroundjoin%
\definecolor{currentfill}{rgb}{0.267968,0.223549,0.512008}%
\pgfsetfillcolor{currentfill}%
\pgfsetlinewidth{0.000000pt}%
\definecolor{currentstroke}{rgb}{0.154815,0.493313,0.557840}%
\pgfsetstrokecolor{currentstroke}%
\pgfsetdash{}{0pt}%
\pgfpathmoveto{\pgfqpoint{6.000832in}{3.178458in}}%
\pgfpathlineto{\pgfqpoint{5.862282in}{3.253769in}}%
\pgfpathlineto{\pgfqpoint{5.787278in}{3.220081in}}%
\pgfpathclose%
\pgfusepath{fill}%
\end{pgfscope}%
\begin{pgfscope}%
\pgfpathrectangle{\pgfqpoint{0.539299in}{0.078740in}}{\pgfqpoint{7.842520in}{7.842520in}}%
\pgfusepath{clip}%
\pgfsetbuttcap%
\pgfsetroundjoin%
\definecolor{currentfill}{rgb}{0.214298,0.355619,0.551184}%
\pgfsetfillcolor{currentfill}%
\pgfsetlinewidth{0.000000pt}%
\definecolor{currentstroke}{rgb}{0.153364,0.497000,0.557724}%
\pgfsetstrokecolor{currentstroke}%
\pgfsetdash{}{0pt}%
\pgfpathmoveto{\pgfqpoint{5.159216in}{3.667393in}}%
\pgfpathlineto{\pgfqpoint{5.235775in}{3.632407in}}%
\pgfpathlineto{\pgfqpoint{5.098620in}{3.762206in}}%
\pgfpathclose%
\pgfusepath{fill}%
\end{pgfscope}%
\begin{pgfscope}%
\pgfpathrectangle{\pgfqpoint{0.539299in}{0.078740in}}{\pgfqpoint{7.842520in}{7.842520in}}%
\pgfusepath{clip}%
\pgfsetbuttcap%
\pgfsetroundjoin%
\definecolor{currentfill}{rgb}{0.274952,0.037752,0.364543}%
\pgfsetfillcolor{currentfill}%
\pgfsetlinewidth{0.000000pt}%
\definecolor{currentstroke}{rgb}{0.151918,0.500685,0.557587}%
\pgfsetstrokecolor{currentstroke}%
\pgfsetdash{}{0pt}%
\pgfpathmoveto{\pgfqpoint{7.120863in}{2.683197in}}%
\pgfpathlineto{\pgfqpoint{7.260119in}{2.579903in}}%
\pgfpathlineto{\pgfqpoint{7.331215in}{2.641797in}}%
\pgfpathclose%
\pgfusepath{fill}%
\end{pgfscope}%
\begin{pgfscope}%
\pgfpathrectangle{\pgfqpoint{0.539299in}{0.078740in}}{\pgfqpoint{7.842520in}{7.842520in}}%
\pgfusepath{clip}%
\pgfsetbuttcap%
\pgfsetroundjoin%
\definecolor{currentfill}{rgb}{0.131172,0.555899,0.552459}%
\pgfsetfillcolor{currentfill}%
\pgfsetlinewidth{0.000000pt}%
\definecolor{currentstroke}{rgb}{0.150476,0.504369,0.557430}%
\pgfsetstrokecolor{currentstroke}%
\pgfsetdash{}{0pt}%
\pgfpathmoveto{\pgfqpoint{4.333879in}{4.667896in}}%
\pgfpathlineto{\pgfqpoint{4.551037in}{4.389232in}}%
\pgfpathlineto{\pgfqpoint{4.414126in}{4.558993in}}%
\pgfpathclose%
\pgfusepath{fill}%
\end{pgfscope}%
\begin{pgfscope}%
\pgfpathrectangle{\pgfqpoint{0.539299in}{0.078740in}}{\pgfqpoint{7.842520in}{7.842520in}}%
\pgfusepath{clip}%
\pgfsetbuttcap%
\pgfsetroundjoin%
\definecolor{currentfill}{rgb}{0.123463,0.581687,0.547445}%
\pgfsetfillcolor{currentfill}%
\pgfsetlinewidth{0.000000pt}%
\definecolor{currentstroke}{rgb}{0.149039,0.508051,0.557250}%
\pgfsetstrokecolor{currentstroke}%
\pgfsetdash{}{0pt}%
\pgfpathmoveto{\pgfqpoint{4.333879in}{4.667896in}}%
\pgfpathlineto{\pgfqpoint{4.414126in}{4.558993in}}%
\pgfpathlineto{\pgfqpoint{4.277193in}{4.723467in}}%
\pgfpathclose%
\pgfusepath{fill}%
\end{pgfscope}%
\begin{pgfscope}%
\pgfpathrectangle{\pgfqpoint{0.539299in}{0.078740in}}{\pgfqpoint{7.842520in}{7.842520in}}%
\pgfusepath{clip}%
\pgfsetbuttcap%
\pgfsetroundjoin%
\definecolor{currentfill}{rgb}{0.119699,0.618490,0.536347}%
\pgfsetfillcolor{currentfill}%
\pgfsetlinewidth{0.000000pt}%
\definecolor{currentstroke}{rgb}{0.147607,0.511733,0.557049}%
\pgfsetstrokecolor{currentstroke}%
\pgfsetdash{}{0pt}%
\pgfpathmoveto{\pgfqpoint{4.140312in}{4.874931in}}%
\pgfpathlineto{\pgfqpoint{4.196221in}{4.834423in}}%
\pgfpathlineto{\pgfqpoint{4.277193in}{4.723467in}}%
\pgfpathclose%
\pgfusepath{fill}%
\end{pgfscope}%
\begin{pgfscope}%
\pgfpathrectangle{\pgfqpoint{0.539299in}{0.078740in}}{\pgfqpoint{7.842520in}{7.842520in}}%
\pgfusepath{clip}%
\pgfsetbuttcap%
\pgfsetroundjoin%
\definecolor{currentfill}{rgb}{0.283072,0.130895,0.449241}%
\pgfsetfillcolor{currentfill}%
\pgfsetlinewidth{0.000000pt}%
\definecolor{currentstroke}{rgb}{0.146180,0.515413,0.556823}%
\pgfsetstrokecolor{currentstroke}%
\pgfsetdash{}{0pt}%
\pgfpathmoveto{\pgfqpoint{6.703683in}{2.968722in}}%
\pgfpathlineto{\pgfqpoint{6.631052in}{2.917912in}}%
\pgfpathlineto{\pgfqpoint{6.770368in}{2.826882in}}%
\pgfpathclose%
\pgfusepath{fill}%
\end{pgfscope}%
\begin{pgfscope}%
\pgfpathrectangle{\pgfqpoint{0.539299in}{0.078740in}}{\pgfqpoint{7.842520in}{7.842520in}}%
\pgfusepath{clip}%
\pgfsetbuttcap%
\pgfsetroundjoin%
\definecolor{currentfill}{rgb}{0.223925,0.334994,0.548053}%
\pgfsetfillcolor{currentfill}%
\pgfsetlinewidth{0.000000pt}%
\definecolor{currentstroke}{rgb}{0.144759,0.519093,0.556572}%
\pgfsetstrokecolor{currentstroke}%
\pgfsetdash{}{0pt}%
\pgfpathmoveto{\pgfqpoint{5.235775in}{3.632407in}}%
\pgfpathlineto{\pgfqpoint{5.159216in}{3.667393in}}%
\pgfpathlineto{\pgfqpoint{5.373172in}{3.514648in}}%
\pgfpathclose%
\pgfusepath{fill}%
\end{pgfscope}%
\begin{pgfscope}%
\pgfpathrectangle{\pgfqpoint{0.539299in}{0.078740in}}{\pgfqpoint{7.842520in}{7.842520in}}%
\pgfusepath{clip}%
\pgfsetbuttcap%
\pgfsetroundjoin%
\definecolor{currentfill}{rgb}{0.262138,0.242286,0.520837}%
\pgfsetfillcolor{currentfill}%
\pgfsetlinewidth{0.000000pt}%
\definecolor{currentstroke}{rgb}{0.143343,0.522773,0.556295}%
\pgfsetstrokecolor{currentstroke}%
\pgfsetdash{}{0pt}%
\pgfpathmoveto{\pgfqpoint{5.648894in}{3.310389in}}%
\pgfpathlineto{\pgfqpoint{5.787278in}{3.220081in}}%
\pgfpathlineto{\pgfqpoint{5.862282in}{3.253769in}}%
\pgfpathclose%
\pgfusepath{fill}%
\end{pgfscope}%
\begin{pgfscope}%
\pgfpathrectangle{\pgfqpoint{0.539299in}{0.078740in}}{\pgfqpoint{7.842520in}{7.842520in}}%
\pgfusepath{clip}%
\pgfsetbuttcap%
\pgfsetroundjoin%
\definecolor{currentfill}{rgb}{0.282656,0.100196,0.422160}%
\pgfsetfillcolor{currentfill}%
\pgfsetlinewidth{0.000000pt}%
\definecolor{currentstroke}{rgb}{0.141935,0.526453,0.555991}%
\pgfsetstrokecolor{currentstroke}%
\pgfsetdash{}{0pt}%
\pgfpathmoveto{\pgfqpoint{6.842661in}{2.878816in}}%
\pgfpathlineto{\pgfqpoint{6.909736in}{2.729276in}}%
\pgfpathlineto{\pgfqpoint{6.981719in}{2.783278in}}%
\pgfpathclose%
\pgfusepath{fill}%
\end{pgfscope}%
\begin{pgfscope}%
\pgfpathrectangle{\pgfqpoint{0.539299in}{0.078740in}}{\pgfqpoint{7.842520in}{7.842520in}}%
\pgfusepath{clip}%
\pgfsetbuttcap%
\pgfsetroundjoin%
\definecolor{currentfill}{rgb}{0.170948,0.694384,0.493803}%
\pgfsetfillcolor{currentfill}%
\pgfsetlinewidth{0.000000pt}%
\definecolor{currentstroke}{rgb}{0.140536,0.530132,0.555659}%
\pgfsetstrokecolor{currentstroke}%
\pgfsetdash{}{0pt}%
\pgfpathmoveto{\pgfqpoint{3.081416in}{5.017996in}}%
\pgfpathlineto{\pgfqpoint{3.211903in}{5.214787in}}%
\pgfpathlineto{\pgfqpoint{3.296607in}{5.181085in}}%
\pgfpathclose%
\pgfusepath{fill}%
\end{pgfscope}%
\begin{pgfscope}%
\pgfpathrectangle{\pgfqpoint{0.539299in}{0.078740in}}{\pgfqpoint{7.842520in}{7.842520in}}%
\pgfusepath{clip}%
\pgfsetbuttcap%
\pgfsetroundjoin%
\definecolor{currentfill}{rgb}{0.277134,0.185228,0.489898}%
\pgfsetfillcolor{currentfill}%
\pgfsetlinewidth{0.000000pt}%
\definecolor{currentstroke}{rgb}{0.139147,0.533812,0.555298}%
\pgfsetstrokecolor{currentstroke}%
\pgfsetdash{}{0pt}%
\pgfpathmoveto{\pgfqpoint{6.352735in}{3.080591in}}%
\pgfpathlineto{\pgfqpoint{6.139708in}{3.103444in}}%
\pgfpathlineto{\pgfqpoint{6.278864in}{3.026523in}}%
\pgfpathclose%
\pgfusepath{fill}%
\end{pgfscope}%
\begin{pgfscope}%
\pgfpathrectangle{\pgfqpoint{0.539299in}{0.078740in}}{\pgfqpoint{7.842520in}{7.842520in}}%
\pgfusepath{clip}%
\pgfsetbuttcap%
\pgfsetroundjoin%
\definecolor{currentfill}{rgb}{0.208030,0.718701,0.472873}%
\pgfsetfillcolor{currentfill}%
\pgfsetlinewidth{0.000000pt}%
\definecolor{currentstroke}{rgb}{0.137770,0.537492,0.554906}%
\pgfsetstrokecolor{currentstroke}%
\pgfsetdash{}{0pt}%
\pgfpathmoveto{\pgfqpoint{3.648554in}{5.237108in}}%
\pgfpathlineto{\pgfqpoint{3.513793in}{5.222466in}}%
\pgfpathlineto{\pgfqpoint{3.429695in}{5.281129in}}%
\pgfpathclose%
\pgfusepath{fill}%
\end{pgfscope}%
\begin{pgfscope}%
\pgfpathrectangle{\pgfqpoint{0.539299in}{0.078740in}}{\pgfqpoint{7.842520in}{7.842520in}}%
\pgfusepath{clip}%
\pgfsetbuttcap%
\pgfsetroundjoin%
\definecolor{currentfill}{rgb}{0.280255,0.165693,0.476498}%
\pgfsetfillcolor{currentfill}%
\pgfsetlinewidth{0.000000pt}%
\definecolor{currentstroke}{rgb}{0.136408,0.541173,0.554483}%
\pgfsetstrokecolor{currentstroke}%
\pgfsetdash{}{0pt}%
\pgfpathmoveto{\pgfqpoint{6.491826in}{3.002343in}}%
\pgfpathlineto{\pgfqpoint{6.352735in}{3.080591in}}%
\pgfpathlineto{\pgfqpoint{6.418242in}{2.945854in}}%
\pgfpathclose%
\pgfusepath{fill}%
\end{pgfscope}%
\begin{pgfscope}%
\pgfpathrectangle{\pgfqpoint{0.539299in}{0.078740in}}{\pgfqpoint{7.842520in}{7.842520in}}%
\pgfusepath{clip}%
\pgfsetbuttcap%
\pgfsetroundjoin%
\definecolor{currentfill}{rgb}{0.170948,0.694384,0.493803}%
\pgfsetfillcolor{currentfill}%
\pgfsetlinewidth{0.000000pt}%
\definecolor{currentstroke}{rgb}{0.135066,0.544853,0.554029}%
\pgfsetstrokecolor{currentstroke}%
\pgfsetdash{}{0pt}%
\pgfpathmoveto{\pgfqpoint{3.867334in}{5.102013in}}%
\pgfpathlineto{\pgfqpoint{3.784535in}{5.195377in}}%
\pgfpathlineto{\pgfqpoint{3.921341in}{5.107578in}}%
\pgfpathclose%
\pgfusepath{fill}%
\end{pgfscope}%
\begin{pgfscope}%
\pgfpathrectangle{\pgfqpoint{0.539299in}{0.078740in}}{\pgfqpoint{7.842520in}{7.842520in}}%
\pgfusepath{clip}%
\pgfsetbuttcap%
\pgfsetroundjoin%
\definecolor{currentfill}{rgb}{0.281446,0.084320,0.407414}%
\pgfsetfillcolor{currentfill}%
\pgfsetlinewidth{0.000000pt}%
\definecolor{currentstroke}{rgb}{0.133743,0.548535,0.553541}%
\pgfsetstrokecolor{currentstroke}%
\pgfsetdash{}{0pt}%
\pgfpathmoveto{\pgfqpoint{7.120863in}{2.683197in}}%
\pgfpathlineto{\pgfqpoint{6.981719in}{2.783278in}}%
\pgfpathlineto{\pgfqpoint{6.909736in}{2.729276in}}%
\pgfpathclose%
\pgfusepath{fill}%
\end{pgfscope}%
\begin{pgfscope}%
\pgfpathrectangle{\pgfqpoint{0.539299in}{0.078740in}}{\pgfqpoint{7.842520in}{7.842520in}}%
\pgfusepath{clip}%
\pgfsetbuttcap%
\pgfsetroundjoin%
\definecolor{currentfill}{rgb}{0.241237,0.296485,0.539709}%
\pgfsetfillcolor{currentfill}%
\pgfsetlinewidth{0.000000pt}%
\definecolor{currentstroke}{rgb}{0.132444,0.552216,0.553018}%
\pgfsetstrokecolor{currentstroke}%
\pgfsetdash{}{0pt}%
\pgfpathmoveto{\pgfqpoint{5.373172in}{3.514648in}}%
\pgfpathlineto{\pgfqpoint{5.434988in}{3.409236in}}%
\pgfpathlineto{\pgfqpoint{5.510865in}{3.407873in}}%
\pgfpathclose%
\pgfusepath{fill}%
\end{pgfscope}%
\begin{pgfscope}%
\pgfpathrectangle{\pgfqpoint{0.539299in}{0.078740in}}{\pgfqpoint{7.842520in}{7.842520in}}%
\pgfusepath{clip}%
\pgfsetbuttcap%
\pgfsetroundjoin%
\definecolor{currentfill}{rgb}{0.283091,0.110553,0.431554}%
\pgfsetfillcolor{currentfill}%
\pgfsetlinewidth{0.000000pt}%
\definecolor{currentstroke}{rgb}{0.131172,0.555899,0.552459}%
\pgfsetstrokecolor{currentstroke}%
\pgfsetdash{}{0pt}%
\pgfpathmoveto{\pgfqpoint{6.770368in}{2.826882in}}%
\pgfpathlineto{\pgfqpoint{6.909736in}{2.729276in}}%
\pgfpathlineto{\pgfqpoint{6.842661in}{2.878816in}}%
\pgfpathclose%
\pgfusepath{fill}%
\end{pgfscope}%
\begin{pgfscope}%
\pgfpathrectangle{\pgfqpoint{0.539299in}{0.078740in}}{\pgfqpoint{7.842520in}{7.842520in}}%
\pgfusepath{clip}%
\pgfsetbuttcap%
\pgfsetroundjoin%
\definecolor{currentfill}{rgb}{0.278826,0.175490,0.483397}%
\pgfsetfillcolor{currentfill}%
\pgfsetlinewidth{0.000000pt}%
\definecolor{currentstroke}{rgb}{0.129933,0.559582,0.551864}%
\pgfsetstrokecolor{currentstroke}%
\pgfsetdash{}{0pt}%
\pgfpathmoveto{\pgfqpoint{6.418242in}{2.945854in}}%
\pgfpathlineto{\pgfqpoint{6.352735in}{3.080591in}}%
\pgfpathlineto{\pgfqpoint{6.278864in}{3.026523in}}%
\pgfpathclose%
\pgfusepath{fill}%
\end{pgfscope}%
\begin{pgfscope}%
\pgfpathrectangle{\pgfqpoint{0.539299in}{0.078740in}}{\pgfqpoint{7.842520in}{7.842520in}}%
\pgfusepath{clip}%
\pgfsetbuttcap%
\pgfsetroundjoin%
\definecolor{currentfill}{rgb}{0.171176,0.452530,0.557965}%
\pgfsetfillcolor{currentfill}%
\pgfsetlinewidth{0.000000pt}%
\definecolor{currentstroke}{rgb}{0.128729,0.563265,0.551229}%
\pgfsetstrokecolor{currentstroke}%
\pgfsetdash{}{0pt}%
\pgfpathmoveto{\pgfqpoint{4.884097in}{3.971926in}}%
\pgfpathlineto{\pgfqpoint{4.824755in}{4.057714in}}%
\pgfpathlineto{\pgfqpoint{4.746612in}{4.139665in}}%
\pgfpathclose%
\pgfusepath{fill}%
\end{pgfscope}%
\begin{pgfscope}%
\pgfpathrectangle{\pgfqpoint{0.539299in}{0.078740in}}{\pgfqpoint{7.842520in}{7.842520in}}%
\pgfusepath{clip}%
\pgfsetbuttcap%
\pgfsetroundjoin%
\definecolor{currentfill}{rgb}{0.281412,0.155834,0.469201}%
\pgfsetfillcolor{currentfill}%
\pgfsetlinewidth{0.000000pt}%
\definecolor{currentstroke}{rgb}{0.127568,0.566949,0.550556}%
\pgfsetstrokecolor{currentstroke}%
\pgfsetdash{}{0pt}%
\pgfpathmoveto{\pgfqpoint{6.491826in}{3.002343in}}%
\pgfpathlineto{\pgfqpoint{6.418242in}{2.945854in}}%
\pgfpathlineto{\pgfqpoint{6.631052in}{2.917912in}}%
\pgfpathclose%
\pgfusepath{fill}%
\end{pgfscope}%
\begin{pgfscope}%
\pgfpathrectangle{\pgfqpoint{0.539299in}{0.078740in}}{\pgfqpoint{7.842520in}{7.842520in}}%
\pgfusepath{clip}%
\pgfsetbuttcap%
\pgfsetroundjoin%
\definecolor{currentfill}{rgb}{0.192357,0.403199,0.555836}%
\pgfsetfillcolor{currentfill}%
\pgfsetlinewidth{0.000000pt}%
\definecolor{currentstroke}{rgb}{0.126453,0.570633,0.549841}%
\pgfsetstrokecolor{currentstroke}%
\pgfsetdash{}{0pt}%
\pgfpathmoveto{\pgfqpoint{5.021609in}{3.813991in}}%
\pgfpathlineto{\pgfqpoint{5.098620in}{3.762206in}}%
\pgfpathlineto{\pgfqpoint{4.884097in}{3.971926in}}%
\pgfpathclose%
\pgfusepath{fill}%
\end{pgfscope}%
\begin{pgfscope}%
\pgfpathrectangle{\pgfqpoint{0.539299in}{0.078740in}}{\pgfqpoint{7.842520in}{7.842520in}}%
\pgfusepath{clip}%
\pgfsetbuttcap%
\pgfsetroundjoin%
\definecolor{currentfill}{rgb}{0.160665,0.478540,0.558115}%
\pgfsetfillcolor{currentfill}%
\pgfsetlinewidth{0.000000pt}%
\definecolor{currentstroke}{rgb}{0.125394,0.574318,0.549086}%
\pgfsetstrokecolor{currentstroke}%
\pgfsetdash{}{0pt}%
\pgfpathmoveto{\pgfqpoint{4.746612in}{4.139665in}}%
\pgfpathlineto{\pgfqpoint{4.824755in}{4.057714in}}%
\pgfpathlineto{\pgfqpoint{4.609098in}{4.314481in}}%
\pgfpathclose%
\pgfusepath{fill}%
\end{pgfscope}%
\begin{pgfscope}%
\pgfpathrectangle{\pgfqpoint{0.539299in}{0.078740in}}{\pgfqpoint{7.842520in}{7.842520in}}%
\pgfusepath{clip}%
\pgfsetbuttcap%
\pgfsetroundjoin%
\definecolor{currentfill}{rgb}{0.203063,0.379716,0.553925}%
\pgfsetfillcolor{currentfill}%
\pgfsetlinewidth{0.000000pt}%
\definecolor{currentstroke}{rgb}{0.124395,0.578002,0.548287}%
\pgfsetstrokecolor{currentstroke}%
\pgfsetdash{}{0pt}%
\pgfpathmoveto{\pgfqpoint{5.159216in}{3.667393in}}%
\pgfpathlineto{\pgfqpoint{5.098620in}{3.762206in}}%
\pgfpathlineto{\pgfqpoint{5.021609in}{3.813991in}}%
\pgfpathclose%
\pgfusepath{fill}%
\end{pgfscope}%
\begin{pgfscope}%
\pgfpathrectangle{\pgfqpoint{0.539299in}{0.078740in}}{\pgfqpoint{7.842520in}{7.842520in}}%
\pgfusepath{clip}%
\pgfsetbuttcap%
\pgfsetroundjoin%
\definecolor{currentfill}{rgb}{0.267968,0.223549,0.512008}%
\pgfsetfillcolor{currentfill}%
\pgfsetlinewidth{0.000000pt}%
\definecolor{currentstroke}{rgb}{0.123463,0.581687,0.547445}%
\pgfsetstrokecolor{currentstroke}%
\pgfsetdash{}{0pt}%
\pgfpathmoveto{\pgfqpoint{5.787278in}{3.220081in}}%
\pgfpathlineto{\pgfqpoint{5.926016in}{3.134613in}}%
\pgfpathlineto{\pgfqpoint{6.000832in}{3.178458in}}%
\pgfpathclose%
\pgfusepath{fill}%
\end{pgfscope}%
\begin{pgfscope}%
\pgfpathrectangle{\pgfqpoint{0.539299in}{0.078740in}}{\pgfqpoint{7.842520in}{7.842520in}}%
\pgfusepath{clip}%
\pgfsetbuttcap%
\pgfsetroundjoin%
\definecolor{currentfill}{rgb}{0.273006,0.204520,0.501721}%
\pgfsetfillcolor{currentfill}%
\pgfsetlinewidth{0.000000pt}%
\definecolor{currentstroke}{rgb}{0.122606,0.585371,0.546557}%
\pgfsetstrokecolor{currentstroke}%
\pgfsetdash{}{0pt}%
\pgfpathmoveto{\pgfqpoint{6.065090in}{3.051608in}}%
\pgfpathlineto{\pgfqpoint{6.139708in}{3.103444in}}%
\pgfpathlineto{\pgfqpoint{6.000832in}{3.178458in}}%
\pgfpathclose%
\pgfusepath{fill}%
\end{pgfscope}%
\begin{pgfscope}%
\pgfpathrectangle{\pgfqpoint{0.539299in}{0.078740in}}{\pgfqpoint{7.842520in}{7.842520in}}%
\pgfusepath{clip}%
\pgfsetbuttcap%
\pgfsetroundjoin%
\definecolor{currentfill}{rgb}{0.162016,0.687316,0.499129}%
\pgfsetfillcolor{currentfill}%
\pgfsetlinewidth{0.000000pt}%
\definecolor{currentstroke}{rgb}{0.121831,0.589055,0.545623}%
\pgfsetstrokecolor{currentstroke}%
\pgfsetdash{}{0pt}%
\pgfpathmoveto{\pgfqpoint{2.996415in}{5.025431in}}%
\pgfpathlineto{\pgfqpoint{3.211903in}{5.214787in}}%
\pgfpathlineto{\pgfqpoint{3.081416in}{5.017996in}}%
\pgfpathclose%
\pgfusepath{fill}%
\end{pgfscope}%
\begin{pgfscope}%
\pgfpathrectangle{\pgfqpoint{0.539299in}{0.078740in}}{\pgfqpoint{7.842520in}{7.842520in}}%
\pgfusepath{clip}%
\pgfsetbuttcap%
\pgfsetroundjoin%
\definecolor{currentfill}{rgb}{0.246811,0.283237,0.535941}%
\pgfsetfillcolor{currentfill}%
\pgfsetlinewidth{0.000000pt}%
\definecolor{currentstroke}{rgb}{0.121148,0.592739,0.544641}%
\pgfsetstrokecolor{currentstroke}%
\pgfsetdash{}{0pt}%
\pgfpathmoveto{\pgfqpoint{5.648894in}{3.310389in}}%
\pgfpathlineto{\pgfqpoint{5.510865in}{3.407873in}}%
\pgfpathlineto{\pgfqpoint{5.434988in}{3.409236in}}%
\pgfpathclose%
\pgfusepath{fill}%
\end{pgfscope}%
\begin{pgfscope}%
\pgfpathrectangle{\pgfqpoint{0.539299in}{0.078740in}}{\pgfqpoint{7.842520in}{7.842520in}}%
\pgfusepath{clip}%
\pgfsetbuttcap%
\pgfsetroundjoin%
\definecolor{currentfill}{rgb}{0.140536,0.530132,0.555659}%
\pgfsetfillcolor{currentfill}%
\pgfsetlinewidth{0.000000pt}%
\definecolor{currentstroke}{rgb}{0.120565,0.596422,0.543611}%
\pgfsetstrokecolor{currentstroke}%
\pgfsetdash{}{0pt}%
\pgfpathmoveto{\pgfqpoint{4.609098in}{4.314481in}}%
\pgfpathlineto{\pgfqpoint{4.551037in}{4.389232in}}%
\pgfpathlineto{\pgfqpoint{4.471520in}{4.492348in}}%
\pgfpathclose%
\pgfusepath{fill}%
\end{pgfscope}%
\begin{pgfscope}%
\pgfpathrectangle{\pgfqpoint{0.539299in}{0.078740in}}{\pgfqpoint{7.842520in}{7.842520in}}%
\pgfusepath{clip}%
\pgfsetbuttcap%
\pgfsetroundjoin%
\definecolor{currentfill}{rgb}{0.214298,0.355619,0.551184}%
\pgfsetfillcolor{currentfill}%
\pgfsetlinewidth{0.000000pt}%
\definecolor{currentstroke}{rgb}{0.120092,0.600104,0.542530}%
\pgfsetstrokecolor{currentstroke}%
\pgfsetdash{}{0pt}%
\pgfpathmoveto{\pgfqpoint{2.234531in}{4.033237in}}%
\pgfpathlineto{\pgfqpoint{2.119879in}{3.494959in}}%
\pgfpathlineto{\pgfqpoint{2.034011in}{3.419410in}}%
\pgfpathclose%
\pgfusepath{fill}%
\end{pgfscope}%
\begin{pgfscope}%
\pgfpathrectangle{\pgfqpoint{0.539299in}{0.078740in}}{\pgfqpoint{7.842520in}{7.842520in}}%
\pgfusepath{clip}%
\pgfsetbuttcap%
\pgfsetroundjoin%
\definecolor{currentfill}{rgb}{0.140210,0.665859,0.513427}%
\pgfsetfillcolor{currentfill}%
\pgfsetlinewidth{0.000000pt}%
\definecolor{currentstroke}{rgb}{0.119738,0.603785,0.541400}%
\pgfsetstrokecolor{currentstroke}%
\pgfsetdash{}{0pt}%
\pgfpathmoveto{\pgfqpoint{4.058651in}{4.983988in}}%
\pgfpathlineto{\pgfqpoint{4.140312in}{4.874931in}}%
\pgfpathlineto{\pgfqpoint{3.921341in}{5.107578in}}%
\pgfpathclose%
\pgfusepath{fill}%
\end{pgfscope}%
\begin{pgfscope}%
\pgfpathrectangle{\pgfqpoint{0.539299in}{0.078740in}}{\pgfqpoint{7.842520in}{7.842520in}}%
\pgfusepath{clip}%
\pgfsetbuttcap%
\pgfsetroundjoin%
\definecolor{currentfill}{rgb}{0.277941,0.056324,0.381191}%
\pgfsetfillcolor{currentfill}%
\pgfsetlinewidth{0.000000pt}%
\definecolor{currentstroke}{rgb}{0.119512,0.607464,0.540218}%
\pgfsetstrokecolor{currentstroke}%
\pgfsetdash{}{0pt}%
\pgfpathmoveto{\pgfqpoint{7.049132in}{2.625447in}}%
\pgfpathlineto{\pgfqpoint{7.260119in}{2.579903in}}%
\pgfpathlineto{\pgfqpoint{7.120863in}{2.683197in}}%
\pgfpathclose%
\pgfusepath{fill}%
\end{pgfscope}%
\begin{pgfscope}%
\pgfpathrectangle{\pgfqpoint{0.539299in}{0.078740in}}{\pgfqpoint{7.842520in}{7.842520in}}%
\pgfusepath{clip}%
\pgfsetbuttcap%
\pgfsetroundjoin%
\definecolor{currentfill}{rgb}{0.129933,0.559582,0.551864}%
\pgfsetfillcolor{currentfill}%
\pgfsetlinewidth{0.000000pt}%
\definecolor{currentstroke}{rgb}{0.119423,0.611141,0.538982}%
\pgfsetstrokecolor{currentstroke}%
\pgfsetdash{}{0pt}%
\pgfpathmoveto{\pgfqpoint{4.471520in}{4.492348in}}%
\pgfpathlineto{\pgfqpoint{4.551037in}{4.389232in}}%
\pgfpathlineto{\pgfqpoint{4.333879in}{4.667896in}}%
\pgfpathclose%
\pgfusepath{fill}%
\end{pgfscope}%
\begin{pgfscope}%
\pgfpathrectangle{\pgfqpoint{0.539299in}{0.078740in}}{\pgfqpoint{7.842520in}{7.842520in}}%
\pgfusepath{clip}%
\pgfsetbuttcap%
\pgfsetroundjoin%
\definecolor{currentfill}{rgb}{0.208030,0.718701,0.472873}%
\pgfsetfillcolor{currentfill}%
\pgfsetlinewidth{0.000000pt}%
\definecolor{currentstroke}{rgb}{0.119483,0.614817,0.537692}%
\pgfsetstrokecolor{currentstroke}%
\pgfsetdash{}{0pt}%
\pgfpathmoveto{\pgfqpoint{3.429695in}{5.281129in}}%
\pgfpathlineto{\pgfqpoint{3.296607in}{5.181085in}}%
\pgfpathlineto{\pgfqpoint{3.211903in}{5.214787in}}%
\pgfpathclose%
\pgfusepath{fill}%
\end{pgfscope}%
\begin{pgfscope}%
\pgfpathrectangle{\pgfqpoint{0.539299in}{0.078740in}}{\pgfqpoint{7.842520in}{7.842520in}}%
\pgfusepath{clip}%
\pgfsetbuttcap%
\pgfsetroundjoin%
\definecolor{currentfill}{rgb}{0.223925,0.334994,0.548053}%
\pgfsetfillcolor{currentfill}%
\pgfsetlinewidth{0.000000pt}%
\definecolor{currentstroke}{rgb}{0.119699,0.618490,0.536347}%
\pgfsetstrokecolor{currentstroke}%
\pgfsetdash{}{0pt}%
\pgfpathmoveto{\pgfqpoint{5.373172in}{3.514648in}}%
\pgfpathlineto{\pgfqpoint{5.159216in}{3.667393in}}%
\pgfpathlineto{\pgfqpoint{5.296988in}{3.532612in}}%
\pgfpathclose%
\pgfusepath{fill}%
\end{pgfscope}%
\begin{pgfscope}%
\pgfpathrectangle{\pgfqpoint{0.539299in}{0.078740in}}{\pgfqpoint{7.842520in}{7.842520in}}%
\pgfusepath{clip}%
\pgfsetbuttcap%
\pgfsetroundjoin%
\definecolor{currentfill}{rgb}{0.128087,0.647749,0.523491}%
\pgfsetfillcolor{currentfill}%
\pgfsetlinewidth{0.000000pt}%
\definecolor{currentstroke}{rgb}{0.120081,0.622161,0.534946}%
\pgfsetstrokecolor{currentstroke}%
\pgfsetdash{}{0pt}%
\pgfpathmoveto{\pgfqpoint{4.140312in}{4.874931in}}%
\pgfpathlineto{\pgfqpoint{4.058651in}{4.983988in}}%
\pgfpathlineto{\pgfqpoint{4.196221in}{4.834423in}}%
\pgfpathclose%
\pgfusepath{fill}%
\end{pgfscope}%
\begin{pgfscope}%
\pgfpathrectangle{\pgfqpoint{0.539299in}{0.078740in}}{\pgfqpoint{7.842520in}{7.842520in}}%
\pgfusepath{clip}%
\pgfsetbuttcap%
\pgfsetroundjoin%
\definecolor{currentfill}{rgb}{0.119423,0.611141,0.538982}%
\pgfsetfillcolor{currentfill}%
\pgfsetlinewidth{0.000000pt}%
\definecolor{currentstroke}{rgb}{0.120638,0.625828,0.533488}%
\pgfsetstrokecolor{currentstroke}%
\pgfsetdash{}{0pt}%
\pgfpathmoveto{\pgfqpoint{4.277193in}{4.723467in}}%
\pgfpathlineto{\pgfqpoint{4.196221in}{4.834423in}}%
\pgfpathlineto{\pgfqpoint{4.333879in}{4.667896in}}%
\pgfpathclose%
\pgfusepath{fill}%
\end{pgfscope}%
\begin{pgfscope}%
\pgfpathrectangle{\pgfqpoint{0.539299in}{0.078740in}}{\pgfqpoint{7.842520in}{7.842520in}}%
\pgfusepath{clip}%
\pgfsetbuttcap%
\pgfsetroundjoin%
\definecolor{currentfill}{rgb}{0.185556,0.418570,0.556753}%
\pgfsetfillcolor{currentfill}%
\pgfsetlinewidth{0.000000pt}%
\definecolor{currentstroke}{rgb}{0.121380,0.629492,0.531973}%
\pgfsetstrokecolor{currentstroke}%
\pgfsetdash{}{0pt}%
\pgfpathmoveto{\pgfqpoint{2.234531in}{4.033237in}}%
\pgfpathlineto{\pgfqpoint{2.320596in}{4.102334in}}%
\pgfpathlineto{\pgfqpoint{2.205440in}{3.565187in}}%
\pgfpathclose%
\pgfusepath{fill}%
\end{pgfscope}%
\begin{pgfscope}%
\pgfpathrectangle{\pgfqpoint{0.539299in}{0.078740in}}{\pgfqpoint{7.842520in}{7.842520in}}%
\pgfusepath{clip}%
\pgfsetbuttcap%
\pgfsetroundjoin%
\definecolor{currentfill}{rgb}{0.270595,0.214069,0.507052}%
\pgfsetfillcolor{currentfill}%
\pgfsetlinewidth{0.000000pt}%
\definecolor{currentstroke}{rgb}{0.122312,0.633153,0.530398}%
\pgfsetstrokecolor{currentstroke}%
\pgfsetdash{}{0pt}%
\pgfpathmoveto{\pgfqpoint{6.000832in}{3.178458in}}%
\pgfpathlineto{\pgfqpoint{5.926016in}{3.134613in}}%
\pgfpathlineto{\pgfqpoint{6.065090in}{3.051608in}}%
\pgfpathclose%
\pgfusepath{fill}%
\end{pgfscope}%
\begin{pgfscope}%
\pgfpathrectangle{\pgfqpoint{0.539299in}{0.078740in}}{\pgfqpoint{7.842520in}{7.842520in}}%
\pgfusepath{clip}%
\pgfsetbuttcap%
\pgfsetroundjoin%
\definecolor{currentfill}{rgb}{0.233603,0.313828,0.543914}%
\pgfsetfillcolor{currentfill}%
\pgfsetlinewidth{0.000000pt}%
\definecolor{currentstroke}{rgb}{0.123444,0.636809,0.528763}%
\pgfsetstrokecolor{currentstroke}%
\pgfsetdash{}{0pt}%
\pgfpathmoveto{\pgfqpoint{5.296988in}{3.532612in}}%
\pgfpathlineto{\pgfqpoint{5.434988in}{3.409236in}}%
\pgfpathlineto{\pgfqpoint{5.373172in}{3.514648in}}%
\pgfpathclose%
\pgfusepath{fill}%
\end{pgfscope}%
\begin{pgfscope}%
\pgfpathrectangle{\pgfqpoint{0.539299in}{0.078740in}}{\pgfqpoint{7.842520in}{7.842520in}}%
\pgfusepath{clip}%
\pgfsetbuttcap%
\pgfsetroundjoin%
\definecolor{currentfill}{rgb}{0.280894,0.078907,0.402329}%
\pgfsetfillcolor{currentfill}%
\pgfsetlinewidth{0.000000pt}%
\definecolor{currentstroke}{rgb}{0.124780,0.640461,0.527068}%
\pgfsetstrokecolor{currentstroke}%
\pgfsetdash{}{0pt}%
\pgfpathmoveto{\pgfqpoint{6.909736in}{2.729276in}}%
\pgfpathlineto{\pgfqpoint{7.049132in}{2.625447in}}%
\pgfpathlineto{\pgfqpoint{7.120863in}{2.683197in}}%
\pgfpathclose%
\pgfusepath{fill}%
\end{pgfscope}%
\begin{pgfscope}%
\pgfpathrectangle{\pgfqpoint{0.539299in}{0.078740in}}{\pgfqpoint{7.842520in}{7.842520in}}%
\pgfusepath{clip}%
\pgfsetbuttcap%
\pgfsetroundjoin%
\definecolor{currentfill}{rgb}{0.276194,0.190074,0.493001}%
\pgfsetfillcolor{currentfill}%
\pgfsetlinewidth{0.000000pt}%
\definecolor{currentstroke}{rgb}{0.126326,0.644107,0.525311}%
\pgfsetstrokecolor{currentstroke}%
\pgfsetdash{}{0pt}%
\pgfpathmoveto{\pgfqpoint{6.278864in}{3.026523in}}%
\pgfpathlineto{\pgfqpoint{6.139708in}{3.103444in}}%
\pgfpathlineto{\pgfqpoint{6.204464in}{2.968792in}}%
\pgfpathclose%
\pgfusepath{fill}%
\end{pgfscope}%
\begin{pgfscope}%
\pgfpathrectangle{\pgfqpoint{0.539299in}{0.078740in}}{\pgfqpoint{7.842520in}{7.842520in}}%
\pgfusepath{clip}%
\pgfsetbuttcap%
\pgfsetroundjoin%
\definecolor{currentfill}{rgb}{0.283072,0.130895,0.449241}%
\pgfsetfillcolor{currentfill}%
\pgfsetlinewidth{0.000000pt}%
\definecolor{currentstroke}{rgb}{0.128087,0.647749,0.523491}%
\pgfsetstrokecolor{currentstroke}%
\pgfsetdash{}{0pt}%
\pgfpathmoveto{\pgfqpoint{6.770368in}{2.826882in}}%
\pgfpathlineto{\pgfqpoint{6.631052in}{2.917912in}}%
\pgfpathlineto{\pgfqpoint{6.697430in}{2.767965in}}%
\pgfpathclose%
\pgfusepath{fill}%
\end{pgfscope}%
\begin{pgfscope}%
\pgfpathrectangle{\pgfqpoint{0.539299in}{0.078740in}}{\pgfqpoint{7.842520in}{7.842520in}}%
\pgfusepath{clip}%
\pgfsetbuttcap%
\pgfsetroundjoin%
\definecolor{currentfill}{rgb}{0.128729,0.563265,0.551229}%
\pgfsetfillcolor{currentfill}%
\pgfsetlinewidth{0.000000pt}%
\definecolor{currentstroke}{rgb}{0.130067,0.651384,0.521608}%
\pgfsetstrokecolor{currentstroke}%
\pgfsetdash{}{0pt}%
\pgfpathmoveto{\pgfqpoint{2.699021in}{4.682479in}}%
\pgfpathlineto{\pgfqpoint{2.491969in}{4.215044in}}%
\pgfpathlineto{\pgfqpoint{2.613215in}{4.649492in}}%
\pgfpathclose%
\pgfusepath{fill}%
\end{pgfscope}%
\begin{pgfscope}%
\pgfpathrectangle{\pgfqpoint{0.539299in}{0.078740in}}{\pgfqpoint{7.842520in}{7.842520in}}%
\pgfusepath{clip}%
\pgfsetbuttcap%
\pgfsetroundjoin%
\definecolor{currentfill}{rgb}{0.258965,0.251537,0.524736}%
\pgfsetfillcolor{currentfill}%
\pgfsetlinewidth{0.000000pt}%
\definecolor{currentstroke}{rgb}{0.132268,0.655014,0.519661}%
\pgfsetstrokecolor{currentstroke}%
\pgfsetdash{}{0pt}%
\pgfpathmoveto{\pgfqpoint{5.711861in}{3.191718in}}%
\pgfpathlineto{\pgfqpoint{5.787278in}{3.220081in}}%
\pgfpathlineto{\pgfqpoint{5.648894in}{3.310389in}}%
\pgfpathclose%
\pgfusepath{fill}%
\end{pgfscope}%
\begin{pgfscope}%
\pgfpathrectangle{\pgfqpoint{0.539299in}{0.078740in}}{\pgfqpoint{7.842520in}{7.842520in}}%
\pgfusepath{clip}%
\pgfsetbuttcap%
\pgfsetroundjoin%
\definecolor{currentfill}{rgb}{0.281887,0.150881,0.465405}%
\pgfsetfillcolor{currentfill}%
\pgfsetlinewidth{0.000000pt}%
\definecolor{currentstroke}{rgb}{0.134692,0.658636,0.517649}%
\pgfsetstrokecolor{currentstroke}%
\pgfsetdash{}{0pt}%
\pgfpathmoveto{\pgfqpoint{6.631052in}{2.917912in}}%
\pgfpathlineto{\pgfqpoint{6.418242in}{2.945854in}}%
\pgfpathlineto{\pgfqpoint{6.557783in}{2.860007in}}%
\pgfpathclose%
\pgfusepath{fill}%
\end{pgfscope}%
\begin{pgfscope}%
\pgfpathrectangle{\pgfqpoint{0.539299in}{0.078740in}}{\pgfqpoint{7.842520in}{7.842520in}}%
\pgfusepath{clip}%
\pgfsetbuttcap%
\pgfsetroundjoin%
\definecolor{currentfill}{rgb}{0.132268,0.655014,0.519661}%
\pgfsetfillcolor{currentfill}%
\pgfsetlinewidth{0.000000pt}%
\definecolor{currentstroke}{rgb}{0.137339,0.662252,0.515571}%
\pgfsetstrokecolor{currentstroke}%
\pgfsetdash{}{0pt}%
\pgfpathmoveto{\pgfqpoint{2.910930in}{5.025499in}}%
\pgfpathlineto{\pgfqpoint{2.996415in}{5.025431in}}%
\pgfpathlineto{\pgfqpoint{2.784458in}{4.707293in}}%
\pgfpathclose%
\pgfusepath{fill}%
\end{pgfscope}%
\begin{pgfscope}%
\pgfpathrectangle{\pgfqpoint{0.539299in}{0.078740in}}{\pgfqpoint{7.842520in}{7.842520in}}%
\pgfusepath{clip}%
\pgfsetbuttcap%
\pgfsetroundjoin%
\definecolor{currentfill}{rgb}{0.239374,0.735588,0.455688}%
\pgfsetfillcolor{currentfill}%
\pgfsetlinewidth{0.000000pt}%
\definecolor{currentstroke}{rgb}{0.140210,0.665859,0.513427}%
\pgfsetstrokecolor{currentstroke}%
\pgfsetdash{}{0pt}%
\pgfpathmoveto{\pgfqpoint{3.429695in}{5.281129in}}%
\pgfpathlineto{\pgfqpoint{3.564639in}{5.312794in}}%
\pgfpathlineto{\pgfqpoint{3.648554in}{5.237108in}}%
\pgfpathclose%
\pgfusepath{fill}%
\end{pgfscope}%
\begin{pgfscope}%
\pgfpathrectangle{\pgfqpoint{0.539299in}{0.078740in}}{\pgfqpoint{7.842520in}{7.842520in}}%
\pgfusepath{clip}%
\pgfsetbuttcap%
\pgfsetroundjoin%
\definecolor{currentfill}{rgb}{0.276022,0.044167,0.370164}%
\pgfsetfillcolor{currentfill}%
\pgfsetlinewidth{0.000000pt}%
\definecolor{currentstroke}{rgb}{0.143303,0.669459,0.511215}%
\pgfsetstrokecolor{currentstroke}%
\pgfsetdash{}{0pt}%
\pgfpathmoveto{\pgfqpoint{7.049132in}{2.625447in}}%
\pgfpathlineto{\pgfqpoint{7.188539in}{2.515946in}}%
\pgfpathlineto{\pgfqpoint{7.260119in}{2.579903in}}%
\pgfpathclose%
\pgfusepath{fill}%
\end{pgfscope}%
\begin{pgfscope}%
\pgfpathrectangle{\pgfqpoint{0.539299in}{0.078740in}}{\pgfqpoint{7.842520in}{7.842520in}}%
\pgfusepath{clip}%
\pgfsetbuttcap%
\pgfsetroundjoin%
\definecolor{currentfill}{rgb}{0.246811,0.283237,0.535941}%
\pgfsetfillcolor{currentfill}%
\pgfsetlinewidth{0.000000pt}%
\definecolor{currentstroke}{rgb}{0.146616,0.673050,0.508936}%
\pgfsetstrokecolor{currentstroke}%
\pgfsetdash{}{0pt}%
\pgfpathmoveto{\pgfqpoint{5.434988in}{3.409236in}}%
\pgfpathlineto{\pgfqpoint{5.573268in}{3.296150in}}%
\pgfpathlineto{\pgfqpoint{5.648894in}{3.310389in}}%
\pgfpathclose%
\pgfusepath{fill}%
\end{pgfscope}%
\begin{pgfscope}%
\pgfpathrectangle{\pgfqpoint{0.539299in}{0.078740in}}{\pgfqpoint{7.842520in}{7.842520in}}%
\pgfusepath{clip}%
\pgfsetbuttcap%
\pgfsetroundjoin%
\definecolor{currentfill}{rgb}{0.143343,0.522773,0.556295}%
\pgfsetfillcolor{currentfill}%
\pgfsetlinewidth{0.000000pt}%
\definecolor{currentstroke}{rgb}{0.150148,0.676631,0.506589}%
\pgfsetstrokecolor{currentstroke}%
\pgfsetdash{}{0pt}%
\pgfpathmoveto{\pgfqpoint{2.527103in}{4.606886in}}%
\pgfpathlineto{\pgfqpoint{2.491969in}{4.215044in}}%
\pgfpathlineto{\pgfqpoint{2.406429in}{4.162525in}}%
\pgfpathclose%
\pgfusepath{fill}%
\end{pgfscope}%
\begin{pgfscope}%
\pgfpathrectangle{\pgfqpoint{0.539299in}{0.078740in}}{\pgfqpoint{7.842520in}{7.842520in}}%
\pgfusepath{clip}%
\pgfsetbuttcap%
\pgfsetroundjoin%
\definecolor{currentfill}{rgb}{0.275191,0.194905,0.496005}%
\pgfsetfillcolor{currentfill}%
\pgfsetlinewidth{0.000000pt}%
\definecolor{currentstroke}{rgb}{0.153894,0.680203,0.504172}%
\pgfsetstrokecolor{currentstroke}%
\pgfsetdash{}{0pt}%
\pgfpathmoveto{\pgfqpoint{6.204464in}{2.968792in}}%
\pgfpathlineto{\pgfqpoint{6.139708in}{3.103444in}}%
\pgfpathlineto{\pgfqpoint{6.065090in}{3.051608in}}%
\pgfpathclose%
\pgfusepath{fill}%
\end{pgfscope}%
\begin{pgfscope}%
\pgfpathrectangle{\pgfqpoint{0.539299in}{0.078740in}}{\pgfqpoint{7.842520in}{7.842520in}}%
\pgfusepath{clip}%
\pgfsetbuttcap%
\pgfsetroundjoin%
\definecolor{currentfill}{rgb}{0.283197,0.115680,0.436115}%
\pgfsetfillcolor{currentfill}%
\pgfsetlinewidth{0.000000pt}%
\definecolor{currentstroke}{rgb}{0.157851,0.683765,0.501686}%
\pgfsetstrokecolor{currentstroke}%
\pgfsetdash{}{0pt}%
\pgfpathmoveto{\pgfqpoint{6.770368in}{2.826882in}}%
\pgfpathlineto{\pgfqpoint{6.697430in}{2.767965in}}%
\pgfpathlineto{\pgfqpoint{6.909736in}{2.729276in}}%
\pgfpathclose%
\pgfusepath{fill}%
\end{pgfscope}%
\begin{pgfscope}%
\pgfpathrectangle{\pgfqpoint{0.539299in}{0.078740in}}{\pgfqpoint{7.842520in}{7.842520in}}%
\pgfusepath{clip}%
\pgfsetbuttcap%
\pgfsetroundjoin%
\definecolor{currentfill}{rgb}{0.278012,0.180367,0.486697}%
\pgfsetfillcolor{currentfill}%
\pgfsetlinewidth{0.000000pt}%
\definecolor{currentstroke}{rgb}{0.162016,0.687316,0.499129}%
\pgfsetstrokecolor{currentstroke}%
\pgfsetdash{}{0pt}%
\pgfpathmoveto{\pgfqpoint{6.204464in}{2.968792in}}%
\pgfpathlineto{\pgfqpoint{6.418242in}{2.945854in}}%
\pgfpathlineto{\pgfqpoint{6.278864in}{3.026523in}}%
\pgfpathclose%
\pgfusepath{fill}%
\end{pgfscope}%
\begin{pgfscope}%
\pgfpathrectangle{\pgfqpoint{0.539299in}{0.078740in}}{\pgfqpoint{7.842520in}{7.842520in}}%
\pgfusepath{clip}%
\pgfsetbuttcap%
\pgfsetroundjoin%
\definecolor{currentfill}{rgb}{0.263663,0.237631,0.518762}%
\pgfsetfillcolor{currentfill}%
\pgfsetlinewidth{0.000000pt}%
\definecolor{currentstroke}{rgb}{0.166383,0.690856,0.496502}%
\pgfsetstrokecolor{currentstroke}%
\pgfsetdash{}{0pt}%
\pgfpathmoveto{\pgfqpoint{5.711861in}{3.191718in}}%
\pgfpathlineto{\pgfqpoint{5.926016in}{3.134613in}}%
\pgfpathlineto{\pgfqpoint{5.787278in}{3.220081in}}%
\pgfpathclose%
\pgfusepath{fill}%
\end{pgfscope}%
\begin{pgfscope}%
\pgfpathrectangle{\pgfqpoint{0.539299in}{0.078740in}}{\pgfqpoint{7.842520in}{7.842520in}}%
\pgfusepath{clip}%
\pgfsetbuttcap%
\pgfsetroundjoin%
\definecolor{currentfill}{rgb}{0.282623,0.140926,0.457517}%
\pgfsetfillcolor{currentfill}%
\pgfsetlinewidth{0.000000pt}%
\definecolor{currentstroke}{rgb}{0.170948,0.694384,0.493803}%
\pgfsetstrokecolor{currentstroke}%
\pgfsetdash{}{0pt}%
\pgfpathmoveto{\pgfqpoint{6.697430in}{2.767965in}}%
\pgfpathlineto{\pgfqpoint{6.631052in}{2.917912in}}%
\pgfpathlineto{\pgfqpoint{6.557783in}{2.860007in}}%
\pgfpathclose%
\pgfusepath{fill}%
\end{pgfscope}%
\begin{pgfscope}%
\pgfpathrectangle{\pgfqpoint{0.539299in}{0.078740in}}{\pgfqpoint{7.842520in}{7.842520in}}%
\pgfusepath{clip}%
\pgfsetbuttcap%
\pgfsetroundjoin%
\definecolor{currentfill}{rgb}{0.226397,0.728888,0.462789}%
\pgfsetfillcolor{currentfill}%
\pgfsetlinewidth{0.000000pt}%
\definecolor{currentstroke}{rgb}{0.175707,0.697900,0.491033}%
\pgfsetstrokecolor{currentstroke}%
\pgfsetdash{}{0pt}%
\pgfpathmoveto{\pgfqpoint{3.700962in}{5.285351in}}%
\pgfpathlineto{\pgfqpoint{3.784535in}{5.195377in}}%
\pgfpathlineto{\pgfqpoint{3.648554in}{5.237108in}}%
\pgfpathclose%
\pgfusepath{fill}%
\end{pgfscope}%
\begin{pgfscope}%
\pgfpathrectangle{\pgfqpoint{0.539299in}{0.078740in}}{\pgfqpoint{7.842520in}{7.842520in}}%
\pgfusepath{clip}%
\pgfsetbuttcap%
\pgfsetroundjoin%
\definecolor{currentfill}{rgb}{0.253935,0.265254,0.529983}%
\pgfsetfillcolor{currentfill}%
\pgfsetlinewidth{0.000000pt}%
\definecolor{currentstroke}{rgb}{0.180653,0.701402,0.488189}%
\pgfsetstrokecolor{currentstroke}%
\pgfsetdash{}{0pt}%
\pgfpathmoveto{\pgfqpoint{5.648894in}{3.310389in}}%
\pgfpathlineto{\pgfqpoint{5.573268in}{3.296150in}}%
\pgfpathlineto{\pgfqpoint{5.711861in}{3.191718in}}%
\pgfpathclose%
\pgfusepath{fill}%
\end{pgfscope}%
\begin{pgfscope}%
\pgfpathrectangle{\pgfqpoint{0.539299in}{0.078740in}}{\pgfqpoint{7.842520in}{7.842520in}}%
\pgfusepath{clip}%
\pgfsetbuttcap%
\pgfsetroundjoin%
\definecolor{currentfill}{rgb}{0.220124,0.725509,0.466226}%
\pgfsetfillcolor{currentfill}%
\pgfsetlinewidth{0.000000pt}%
\definecolor{currentstroke}{rgb}{0.185783,0.704891,0.485273}%
\pgfsetstrokecolor{currentstroke}%
\pgfsetdash{}{0pt}%
\pgfpathmoveto{\pgfqpoint{3.700962in}{5.285351in}}%
\pgfpathlineto{\pgfqpoint{3.921341in}{5.107578in}}%
\pgfpathlineto{\pgfqpoint{3.784535in}{5.195377in}}%
\pgfpathclose%
\pgfusepath{fill}%
\end{pgfscope}%
\begin{pgfscope}%
\pgfpathrectangle{\pgfqpoint{0.539299in}{0.078740in}}{\pgfqpoint{7.842520in}{7.842520in}}%
\pgfusepath{clip}%
\pgfsetbuttcap%
\pgfsetroundjoin%
\definecolor{currentfill}{rgb}{0.123444,0.636809,0.528763}%
\pgfsetfillcolor{currentfill}%
\pgfsetlinewidth{0.000000pt}%
\definecolor{currentstroke}{rgb}{0.191090,0.708366,0.482284}%
\pgfsetstrokecolor{currentstroke}%
\pgfsetdash{}{0pt}%
\pgfpathmoveto{\pgfqpoint{2.825015in}{5.016747in}}%
\pgfpathlineto{\pgfqpoint{2.784458in}{4.707293in}}%
\pgfpathlineto{\pgfqpoint{2.699021in}{4.682479in}}%
\pgfpathclose%
\pgfusepath{fill}%
\end{pgfscope}%
\begin{pgfscope}%
\pgfpathrectangle{\pgfqpoint{0.539299in}{0.078740in}}{\pgfqpoint{7.842520in}{7.842520in}}%
\pgfusepath{clip}%
\pgfsetbuttcap%
\pgfsetroundjoin%
\definecolor{currentfill}{rgb}{0.280894,0.078907,0.402329}%
\pgfsetfillcolor{currentfill}%
\pgfsetlinewidth{0.000000pt}%
\definecolor{currentstroke}{rgb}{0.196571,0.711827,0.479221}%
\pgfsetstrokecolor{currentstroke}%
\pgfsetdash{}{0pt}%
\pgfpathmoveto{\pgfqpoint{7.049132in}{2.625447in}}%
\pgfpathlineto{\pgfqpoint{6.909736in}{2.729276in}}%
\pgfpathlineto{\pgfqpoint{6.976816in}{2.562893in}}%
\pgfpathclose%
\pgfusepath{fill}%
\end{pgfscope}%
\begin{pgfscope}%
\pgfpathrectangle{\pgfqpoint{0.539299in}{0.078740in}}{\pgfqpoint{7.842520in}{7.842520in}}%
\pgfusepath{clip}%
\pgfsetbuttcap%
\pgfsetroundjoin%
\definecolor{currentfill}{rgb}{0.280868,0.160771,0.472899}%
\pgfsetfillcolor{currentfill}%
\pgfsetlinewidth{0.000000pt}%
\definecolor{currentstroke}{rgb}{0.202219,0.715272,0.476084}%
\pgfsetstrokecolor{currentstroke}%
\pgfsetdash{}{0pt}%
\pgfpathmoveto{\pgfqpoint{6.557783in}{2.860007in}}%
\pgfpathlineto{\pgfqpoint{6.418242in}{2.945854in}}%
\pgfpathlineto{\pgfqpoint{6.344091in}{2.884105in}}%
\pgfpathclose%
\pgfusepath{fill}%
\end{pgfscope}%
\begin{pgfscope}%
\pgfpathrectangle{\pgfqpoint{0.539299in}{0.078740in}}{\pgfqpoint{7.842520in}{7.842520in}}%
\pgfusepath{clip}%
\pgfsetbuttcap%
\pgfsetroundjoin%
\definecolor{currentfill}{rgb}{0.216210,0.351535,0.550627}%
\pgfsetfillcolor{currentfill}%
\pgfsetlinewidth{0.000000pt}%
\definecolor{currentstroke}{rgb}{0.208030,0.718701,0.472873}%
\pgfsetstrokecolor{currentstroke}%
\pgfsetdash{}{0pt}%
\pgfpathmoveto{\pgfqpoint{2.034011in}{3.419410in}}%
\pgfpathlineto{\pgfqpoint{1.947878in}{3.337740in}}%
\pgfpathlineto{\pgfqpoint{2.148306in}{3.953837in}}%
\pgfpathclose%
\pgfusepath{fill}%
\end{pgfscope}%
\begin{pgfscope}%
\pgfpathrectangle{\pgfqpoint{0.539299in}{0.078740in}}{\pgfqpoint{7.842520in}{7.842520in}}%
\pgfusepath{clip}%
\pgfsetbuttcap%
\pgfsetroundjoin%
\definecolor{currentfill}{rgb}{0.252899,0.742211,0.448284}%
\pgfsetfillcolor{currentfill}%
\pgfsetlinewidth{0.000000pt}%
\definecolor{currentstroke}{rgb}{0.214000,0.722114,0.469588}%
\pgfsetstrokecolor{currentstroke}%
\pgfsetdash{}{0pt}%
\pgfpathmoveto{\pgfqpoint{3.648554in}{5.237108in}}%
\pgfpathlineto{\pgfqpoint{3.564639in}{5.312794in}}%
\pgfpathlineto{\pgfqpoint{3.700962in}{5.285351in}}%
\pgfpathclose%
\pgfusepath{fill}%
\end{pgfscope}%
\begin{pgfscope}%
\pgfpathrectangle{\pgfqpoint{0.539299in}{0.078740in}}{\pgfqpoint{7.842520in}{7.842520in}}%
\pgfusepath{clip}%
\pgfsetbuttcap%
\pgfsetroundjoin%
\definecolor{currentfill}{rgb}{0.190631,0.407061,0.556089}%
\pgfsetfillcolor{currentfill}%
\pgfsetlinewidth{0.000000pt}%
\definecolor{currentstroke}{rgb}{0.220124,0.725509,0.466226}%
\pgfsetstrokecolor{currentstroke}%
\pgfsetdash{}{0pt}%
\pgfpathmoveto{\pgfqpoint{5.021609in}{3.813991in}}%
\pgfpathlineto{\pgfqpoint{4.943959in}{3.876474in}}%
\pgfpathlineto{\pgfqpoint{5.159216in}{3.667393in}}%
\pgfpathclose%
\pgfusepath{fill}%
\end{pgfscope}%
\begin{pgfscope}%
\pgfpathrectangle{\pgfqpoint{0.539299in}{0.078740in}}{\pgfqpoint{7.842520in}{7.842520in}}%
\pgfusepath{clip}%
\pgfsetbuttcap%
\pgfsetroundjoin%
\definecolor{currentfill}{rgb}{0.175841,0.441290,0.557685}%
\pgfsetfillcolor{currentfill}%
\pgfsetlinewidth{0.000000pt}%
\definecolor{currentstroke}{rgb}{0.226397,0.728888,0.462789}%
\pgfsetstrokecolor{currentstroke}%
\pgfsetdash{}{0pt}%
\pgfpathmoveto{\pgfqpoint{5.021609in}{3.813991in}}%
\pgfpathlineto{\pgfqpoint{4.884097in}{3.971926in}}%
\pgfpathlineto{\pgfqpoint{4.805857in}{4.049409in}}%
\pgfpathclose%
\pgfusepath{fill}%
\end{pgfscope}%
\begin{pgfscope}%
\pgfpathrectangle{\pgfqpoint{0.539299in}{0.078740in}}{\pgfqpoint{7.842520in}{7.842520in}}%
\pgfusepath{clip}%
\pgfsetbuttcap%
\pgfsetroundjoin%
\definecolor{currentfill}{rgb}{0.252899,0.742211,0.448284}%
\pgfsetfillcolor{currentfill}%
\pgfsetlinewidth{0.000000pt}%
\definecolor{currentstroke}{rgb}{0.232815,0.732247,0.459277}%
\pgfsetstrokecolor{currentstroke}%
\pgfsetdash{}{0pt}%
\pgfpathmoveto{\pgfqpoint{3.429695in}{5.281129in}}%
\pgfpathlineto{\pgfqpoint{3.211903in}{5.214787in}}%
\pgfpathlineto{\pgfqpoint{3.344948in}{5.333814in}}%
\pgfpathclose%
\pgfusepath{fill}%
\end{pgfscope}%
\begin{pgfscope}%
\pgfpathrectangle{\pgfqpoint{0.539299in}{0.078740in}}{\pgfqpoint{7.842520in}{7.842520in}}%
\pgfusepath{clip}%
\pgfsetbuttcap%
\pgfsetroundjoin%
\definecolor{currentfill}{rgb}{0.165117,0.467423,0.558141}%
\pgfsetfillcolor{currentfill}%
\pgfsetlinewidth{0.000000pt}%
\definecolor{currentstroke}{rgb}{0.239374,0.735588,0.455688}%
\pgfsetstrokecolor{currentstroke}%
\pgfsetdash{}{0pt}%
\pgfpathmoveto{\pgfqpoint{4.805857in}{4.049409in}}%
\pgfpathlineto{\pgfqpoint{4.884097in}{3.971926in}}%
\pgfpathlineto{\pgfqpoint{4.746612in}{4.139665in}}%
\pgfpathclose%
\pgfusepath{fill}%
\end{pgfscope}%
\begin{pgfscope}%
\pgfpathrectangle{\pgfqpoint{0.539299in}{0.078740in}}{\pgfqpoint{7.842520in}{7.842520in}}%
\pgfusepath{clip}%
\pgfsetbuttcap%
\pgfsetroundjoin%
\definecolor{currentfill}{rgb}{0.208030,0.718701,0.472873}%
\pgfsetfillcolor{currentfill}%
\pgfsetlinewidth{0.000000pt}%
\definecolor{currentstroke}{rgb}{0.246070,0.738910,0.452024}%
\pgfsetstrokecolor{currentstroke}%
\pgfsetdash{}{0pt}%
\pgfpathmoveto{\pgfqpoint{3.126635in}{5.241459in}}%
\pgfpathlineto{\pgfqpoint{3.211903in}{5.214787in}}%
\pgfpathlineto{\pgfqpoint{2.996415in}{5.025431in}}%
\pgfpathclose%
\pgfusepath{fill}%
\end{pgfscope}%
\begin{pgfscope}%
\pgfpathrectangle{\pgfqpoint{0.539299in}{0.078740in}}{\pgfqpoint{7.842520in}{7.842520in}}%
\pgfusepath{clip}%
\pgfsetbuttcap%
\pgfsetroundjoin%
\definecolor{currentfill}{rgb}{0.283091,0.110553,0.431554}%
\pgfsetfillcolor{currentfill}%
\pgfsetlinewidth{0.000000pt}%
\definecolor{currentstroke}{rgb}{0.252899,0.742211,0.448284}%
\pgfsetstrokecolor{currentstroke}%
\pgfsetdash{}{0pt}%
\pgfpathmoveto{\pgfqpoint{6.909736in}{2.729276in}}%
\pgfpathlineto{\pgfqpoint{6.697430in}{2.767965in}}%
\pgfpathlineto{\pgfqpoint{6.837125in}{2.669064in}}%
\pgfpathclose%
\pgfusepath{fill}%
\end{pgfscope}%
\begin{pgfscope}%
\pgfpathrectangle{\pgfqpoint{0.539299in}{0.078740in}}{\pgfqpoint{7.842520in}{7.842520in}}%
\pgfusepath{clip}%
\pgfsetbuttcap%
\pgfsetroundjoin%
\definecolor{currentfill}{rgb}{0.270595,0.214069,0.507052}%
\pgfsetfillcolor{currentfill}%
\pgfsetlinewidth{0.000000pt}%
\definecolor{currentstroke}{rgb}{0.259857,0.745492,0.444467}%
\pgfsetstrokecolor{currentstroke}%
\pgfsetdash{}{0pt}%
\pgfpathmoveto{\pgfqpoint{6.065090in}{3.051608in}}%
\pgfpathlineto{\pgfqpoint{5.926016in}{3.134613in}}%
\pgfpathlineto{\pgfqpoint{5.990036in}{3.000736in}}%
\pgfpathclose%
\pgfusepath{fill}%
\end{pgfscope}%
\begin{pgfscope}%
\pgfpathrectangle{\pgfqpoint{0.539299in}{0.078740in}}{\pgfqpoint{7.842520in}{7.842520in}}%
\pgfusepath{clip}%
\pgfsetbuttcap%
\pgfsetroundjoin%
\definecolor{currentfill}{rgb}{0.278826,0.175490,0.483397}%
\pgfsetfillcolor{currentfill}%
\pgfsetlinewidth{0.000000pt}%
\definecolor{currentstroke}{rgb}{0.266941,0.748751,0.440573}%
\pgfsetstrokecolor{currentstroke}%
\pgfsetdash{}{0pt}%
\pgfpathmoveto{\pgfqpoint{6.204464in}{2.968792in}}%
\pgfpathlineto{\pgfqpoint{6.344091in}{2.884105in}}%
\pgfpathlineto{\pgfqpoint{6.418242in}{2.945854in}}%
\pgfpathclose%
\pgfusepath{fill}%
\end{pgfscope}%
\begin{pgfscope}%
\pgfpathrectangle{\pgfqpoint{0.539299in}{0.078740in}}{\pgfqpoint{7.842520in}{7.842520in}}%
\pgfusepath{clip}%
\pgfsetbuttcap%
\pgfsetroundjoin%
\definecolor{currentfill}{rgb}{0.212395,0.359683,0.551710}%
\pgfsetfillcolor{currentfill}%
\pgfsetlinewidth{0.000000pt}%
\definecolor{currentstroke}{rgb}{0.274149,0.751988,0.436601}%
\pgfsetstrokecolor{currentstroke}%
\pgfsetdash{}{0pt}%
\pgfpathmoveto{\pgfqpoint{5.159216in}{3.667393in}}%
\pgfpathlineto{\pgfqpoint{5.220286in}{3.561627in}}%
\pgfpathlineto{\pgfqpoint{5.296988in}{3.532612in}}%
\pgfpathclose%
\pgfusepath{fill}%
\end{pgfscope}%
\begin{pgfscope}%
\pgfpathrectangle{\pgfqpoint{0.539299in}{0.078740in}}{\pgfqpoint{7.842520in}{7.842520in}}%
\pgfusepath{clip}%
\pgfsetbuttcap%
\pgfsetroundjoin%
\definecolor{currentfill}{rgb}{0.149039,0.508051,0.557250}%
\pgfsetfillcolor{currentfill}%
\pgfsetlinewidth{0.000000pt}%
\definecolor{currentstroke}{rgb}{0.281477,0.755203,0.432552}%
\pgfsetstrokecolor{currentstroke}%
\pgfsetdash{}{0pt}%
\pgfpathmoveto{\pgfqpoint{4.746612in}{4.139665in}}%
\pgfpathlineto{\pgfqpoint{4.609098in}{4.314481in}}%
\pgfpathlineto{\pgfqpoint{4.667716in}{4.230148in}}%
\pgfpathclose%
\pgfusepath{fill}%
\end{pgfscope}%
\begin{pgfscope}%
\pgfpathrectangle{\pgfqpoint{0.539299in}{0.078740in}}{\pgfqpoint{7.842520in}{7.842520in}}%
\pgfusepath{clip}%
\pgfsetbuttcap%
\pgfsetroundjoin%
\definecolor{currentfill}{rgb}{0.263663,0.237631,0.518762}%
\pgfsetfillcolor{currentfill}%
\pgfsetlinewidth{0.000000pt}%
\definecolor{currentstroke}{rgb}{0.288921,0.758394,0.428426}%
\pgfsetstrokecolor{currentstroke}%
\pgfsetdash{}{0pt}%
\pgfpathmoveto{\pgfqpoint{5.711861in}{3.191718in}}%
\pgfpathlineto{\pgfqpoint{5.850784in}{3.093964in}}%
\pgfpathlineto{\pgfqpoint{5.926016in}{3.134613in}}%
\pgfpathclose%
\pgfusepath{fill}%
\end{pgfscope}%
\begin{pgfscope}%
\pgfpathrectangle{\pgfqpoint{0.539299in}{0.078740in}}{\pgfqpoint{7.842520in}{7.842520in}}%
\pgfusepath{clip}%
\pgfsetbuttcap%
\pgfsetroundjoin%
\definecolor{currentfill}{rgb}{0.278791,0.062145,0.386592}%
\pgfsetfillcolor{currentfill}%
\pgfsetlinewidth{0.000000pt}%
\definecolor{currentstroke}{rgb}{0.296479,0.761561,0.424223}%
\pgfsetstrokecolor{currentstroke}%
\pgfsetdash{}{0pt}%
\pgfpathmoveto{\pgfqpoint{7.188539in}{2.515946in}}%
\pgfpathlineto{\pgfqpoint{7.049132in}{2.625447in}}%
\pgfpathlineto{\pgfqpoint{6.976816in}{2.562893in}}%
\pgfpathclose%
\pgfusepath{fill}%
\end{pgfscope}%
\begin{pgfscope}%
\pgfpathrectangle{\pgfqpoint{0.539299in}{0.078740in}}{\pgfqpoint{7.842520in}{7.842520in}}%
\pgfusepath{clip}%
\pgfsetbuttcap%
\pgfsetroundjoin%
\definecolor{currentfill}{rgb}{0.126453,0.570633,0.549841}%
\pgfsetfillcolor{currentfill}%
\pgfsetlinewidth{0.000000pt}%
\definecolor{currentstroke}{rgb}{0.304148,0.764704,0.419943}%
\pgfsetstrokecolor{currentstroke}%
\pgfsetdash{}{0pt}%
\pgfpathmoveto{\pgfqpoint{2.613215in}{4.649492in}}%
\pgfpathlineto{\pgfqpoint{2.491969in}{4.215044in}}%
\pgfpathlineto{\pgfqpoint{2.527103in}{4.606886in}}%
\pgfpathclose%
\pgfusepath{fill}%
\end{pgfscope}%
\begin{pgfscope}%
\pgfpathrectangle{\pgfqpoint{0.539299in}{0.078740in}}{\pgfqpoint{7.842520in}{7.842520in}}%
\pgfusepath{clip}%
\pgfsetbuttcap%
\pgfsetroundjoin%
\definecolor{currentfill}{rgb}{0.227802,0.326594,0.546532}%
\pgfsetfillcolor{currentfill}%
\pgfsetlinewidth{0.000000pt}%
\definecolor{currentstroke}{rgb}{0.311925,0.767822,0.415586}%
\pgfsetstrokecolor{currentstroke}%
\pgfsetdash{}{0pt}%
\pgfpathmoveto{\pgfqpoint{5.358646in}{3.421072in}}%
\pgfpathlineto{\pgfqpoint{5.434988in}{3.409236in}}%
\pgfpathlineto{\pgfqpoint{5.296988in}{3.532612in}}%
\pgfpathclose%
\pgfusepath{fill}%
\end{pgfscope}%
\begin{pgfscope}%
\pgfpathrectangle{\pgfqpoint{0.539299in}{0.078740in}}{\pgfqpoint{7.842520in}{7.842520in}}%
\pgfusepath{clip}%
\pgfsetbuttcap%
\pgfsetroundjoin%
\definecolor{currentfill}{rgb}{0.191090,0.708366,0.482284}%
\pgfsetfillcolor{currentfill}%
\pgfsetlinewidth{0.000000pt}%
\definecolor{currentstroke}{rgb}{0.319809,0.770914,0.411152}%
\pgfsetstrokecolor{currentstroke}%
\pgfsetdash{}{0pt}%
\pgfpathmoveto{\pgfqpoint{4.058651in}{4.983988in}}%
\pgfpathlineto{\pgfqpoint{3.921341in}{5.107578in}}%
\pgfpathlineto{\pgfqpoint{3.838249in}{5.208623in}}%
\pgfpathclose%
\pgfusepath{fill}%
\end{pgfscope}%
\begin{pgfscope}%
\pgfpathrectangle{\pgfqpoint{0.539299in}{0.078740in}}{\pgfqpoint{7.842520in}{7.842520in}}%
\pgfusepath{clip}%
\pgfsetbuttcap%
\pgfsetroundjoin%
\definecolor{currentfill}{rgb}{0.281924,0.089666,0.412415}%
\pgfsetfillcolor{currentfill}%
\pgfsetlinewidth{0.000000pt}%
\definecolor{currentstroke}{rgb}{0.327796,0.773980,0.406640}%
\pgfsetstrokecolor{currentstroke}%
\pgfsetdash{}{0pt}%
\pgfpathmoveto{\pgfqpoint{6.976816in}{2.562893in}}%
\pgfpathlineto{\pgfqpoint{6.909736in}{2.729276in}}%
\pgfpathlineto{\pgfqpoint{6.837125in}{2.669064in}}%
\pgfpathclose%
\pgfusepath{fill}%
\end{pgfscope}%
\begin{pgfscope}%
\pgfpathrectangle{\pgfqpoint{0.539299in}{0.078740in}}{\pgfqpoint{7.842520in}{7.842520in}}%
\pgfusepath{clip}%
\pgfsetbuttcap%
\pgfsetroundjoin%
\definecolor{currentfill}{rgb}{0.127568,0.566949,0.550556}%
\pgfsetfillcolor{currentfill}%
\pgfsetlinewidth{0.000000pt}%
\definecolor{currentstroke}{rgb}{0.335885,0.777018,0.402049}%
\pgfsetstrokecolor{currentstroke}%
\pgfsetdash{}{0pt}%
\pgfpathmoveto{\pgfqpoint{4.391174in}{4.600585in}}%
\pgfpathlineto{\pgfqpoint{4.609098in}{4.314481in}}%
\pgfpathlineto{\pgfqpoint{4.471520in}{4.492348in}}%
\pgfpathclose%
\pgfusepath{fill}%
\end{pgfscope}%
\begin{pgfscope}%
\pgfpathrectangle{\pgfqpoint{0.539299in}{0.078740in}}{\pgfqpoint{7.842520in}{7.842520in}}%
\pgfusepath{clip}%
\pgfsetbuttcap%
\pgfsetroundjoin%
\definecolor{currentfill}{rgb}{0.121148,0.592739,0.544641}%
\pgfsetfillcolor{currentfill}%
\pgfsetlinewidth{0.000000pt}%
\definecolor{currentstroke}{rgb}{0.344074,0.780029,0.397381}%
\pgfsetstrokecolor{currentstroke}%
\pgfsetdash{}{0pt}%
\pgfpathmoveto{\pgfqpoint{4.471520in}{4.492348in}}%
\pgfpathlineto{\pgfqpoint{4.333879in}{4.667896in}}%
\pgfpathlineto{\pgfqpoint{4.391174in}{4.600585in}}%
\pgfpathclose%
\pgfusepath{fill}%
\end{pgfscope}%
\begin{pgfscope}%
\pgfpathrectangle{\pgfqpoint{0.539299in}{0.078740in}}{\pgfqpoint{7.842520in}{7.842520in}}%
\pgfusepath{clip}%
\pgfsetbuttcap%
\pgfsetroundjoin%
\definecolor{currentfill}{rgb}{0.143303,0.669459,0.511215}%
\pgfsetfillcolor{currentfill}%
\pgfsetlinewidth{0.000000pt}%
\definecolor{currentstroke}{rgb}{0.352360,0.783011,0.392636}%
\pgfsetstrokecolor{currentstroke}%
\pgfsetdash{}{0pt}%
\pgfpathmoveto{\pgfqpoint{2.910930in}{5.025499in}}%
\pgfpathlineto{\pgfqpoint{2.784458in}{4.707293in}}%
\pgfpathlineto{\pgfqpoint{2.825015in}{5.016747in}}%
\pgfpathclose%
\pgfusepath{fill}%
\end{pgfscope}%
\begin{pgfscope}%
\pgfpathrectangle{\pgfqpoint{0.539299in}{0.078740in}}{\pgfqpoint{7.842520in}{7.842520in}}%
\pgfusepath{clip}%
\pgfsetbuttcap%
\pgfsetroundjoin%
\definecolor{currentfill}{rgb}{0.273006,0.204520,0.501721}%
\pgfsetfillcolor{currentfill}%
\pgfsetlinewidth{0.000000pt}%
\definecolor{currentstroke}{rgb}{0.360741,0.785964,0.387814}%
\pgfsetstrokecolor{currentstroke}%
\pgfsetdash{}{0pt}%
\pgfpathmoveto{\pgfqpoint{6.065090in}{3.051608in}}%
\pgfpathlineto{\pgfqpoint{5.990036in}{3.000736in}}%
\pgfpathlineto{\pgfqpoint{6.204464in}{2.968792in}}%
\pgfpathclose%
\pgfusepath{fill}%
\end{pgfscope}%
\begin{pgfscope}%
\pgfpathrectangle{\pgfqpoint{0.539299in}{0.078740in}}{\pgfqpoint{7.842520in}{7.842520in}}%
\pgfusepath{clip}%
\pgfsetbuttcap%
\pgfsetroundjoin%
\definecolor{currentfill}{rgb}{0.267968,0.223549,0.512008}%
\pgfsetfillcolor{currentfill}%
\pgfsetlinewidth{0.000000pt}%
\definecolor{currentstroke}{rgb}{0.369214,0.788888,0.382914}%
\pgfsetstrokecolor{currentstroke}%
\pgfsetdash{}{0pt}%
\pgfpathmoveto{\pgfqpoint{5.926016in}{3.134613in}}%
\pgfpathlineto{\pgfqpoint{5.850784in}{3.093964in}}%
\pgfpathlineto{\pgfqpoint{5.990036in}{3.000736in}}%
\pgfpathclose%
\pgfusepath{fill}%
\end{pgfscope}%
\begin{pgfscope}%
\pgfpathrectangle{\pgfqpoint{0.539299in}{0.078740in}}{\pgfqpoint{7.842520in}{7.842520in}}%
\pgfusepath{clip}%
\pgfsetbuttcap%
\pgfsetroundjoin%
\definecolor{currentfill}{rgb}{0.282623,0.140926,0.457517}%
\pgfsetfillcolor{currentfill}%
\pgfsetlinewidth{0.000000pt}%
\definecolor{currentstroke}{rgb}{0.377779,0.791781,0.377939}%
\pgfsetstrokecolor{currentstroke}%
\pgfsetdash{}{0pt}%
\pgfpathmoveto{\pgfqpoint{6.557783in}{2.860007in}}%
\pgfpathlineto{\pgfqpoint{6.483913in}{2.795763in}}%
\pgfpathlineto{\pgfqpoint{6.697430in}{2.767965in}}%
\pgfpathclose%
\pgfusepath{fill}%
\end{pgfscope}%
\begin{pgfscope}%
\pgfpathrectangle{\pgfqpoint{0.539299in}{0.078740in}}{\pgfqpoint{7.842520in}{7.842520in}}%
\pgfusepath{clip}%
\pgfsetbuttcap%
\pgfsetroundjoin%
\definecolor{currentfill}{rgb}{0.122312,0.633153,0.530398}%
\pgfsetfillcolor{currentfill}%
\pgfsetlinewidth{0.000000pt}%
\definecolor{currentstroke}{rgb}{0.386433,0.794644,0.372886}%
\pgfsetstrokecolor{currentstroke}%
\pgfsetdash{}{0pt}%
\pgfpathmoveto{\pgfqpoint{4.252784in}{4.779999in}}%
\pgfpathlineto{\pgfqpoint{4.333879in}{4.667896in}}%
\pgfpathlineto{\pgfqpoint{4.196221in}{4.834423in}}%
\pgfpathclose%
\pgfusepath{fill}%
\end{pgfscope}%
\begin{pgfscope}%
\pgfpathrectangle{\pgfqpoint{0.539299in}{0.078740in}}{\pgfqpoint{7.842520in}{7.842520in}}%
\pgfusepath{clip}%
\pgfsetbuttcap%
\pgfsetroundjoin%
\definecolor{currentfill}{rgb}{0.143303,0.669459,0.511215}%
\pgfsetfillcolor{currentfill}%
\pgfsetlinewidth{0.000000pt}%
\definecolor{currentstroke}{rgb}{0.395174,0.797475,0.367757}%
\pgfsetstrokecolor{currentstroke}%
\pgfsetdash{}{0pt}%
\pgfpathmoveto{\pgfqpoint{4.196221in}{4.834423in}}%
\pgfpathlineto{\pgfqpoint{4.058651in}{4.983988in}}%
\pgfpathlineto{\pgfqpoint{4.114399in}{4.946663in}}%
\pgfpathclose%
\pgfusepath{fill}%
\end{pgfscope}%
\begin{pgfscope}%
\pgfpathrectangle{\pgfqpoint{0.539299in}{0.078740in}}{\pgfqpoint{7.842520in}{7.842520in}}%
\pgfusepath{clip}%
\pgfsetbuttcap%
\pgfsetroundjoin%
\definecolor{currentfill}{rgb}{0.241237,0.296485,0.539709}%
\pgfsetfillcolor{currentfill}%
\pgfsetlinewidth{0.000000pt}%
\definecolor{currentstroke}{rgb}{0.404001,0.800275,0.362552}%
\pgfsetstrokecolor{currentstroke}%
\pgfsetdash{}{0pt}%
\pgfpathmoveto{\pgfqpoint{5.573268in}{3.296150in}}%
\pgfpathlineto{\pgfqpoint{5.434988in}{3.409236in}}%
\pgfpathlineto{\pgfqpoint{5.497218in}{3.291318in}}%
\pgfpathclose%
\pgfusepath{fill}%
\end{pgfscope}%
\begin{pgfscope}%
\pgfpathrectangle{\pgfqpoint{0.539299in}{0.078740in}}{\pgfqpoint{7.842520in}{7.842520in}}%
\pgfusepath{clip}%
\pgfsetbuttcap%
\pgfsetroundjoin%
\definecolor{currentfill}{rgb}{0.281412,0.155834,0.469201}%
\pgfsetfillcolor{currentfill}%
\pgfsetlinewidth{0.000000pt}%
\definecolor{currentstroke}{rgb}{0.412913,0.803041,0.357269}%
\pgfsetstrokecolor{currentstroke}%
\pgfsetdash{}{0pt}%
\pgfpathmoveto{\pgfqpoint{6.344091in}{2.884105in}}%
\pgfpathlineto{\pgfqpoint{6.483913in}{2.795763in}}%
\pgfpathlineto{\pgfqpoint{6.557783in}{2.860007in}}%
\pgfpathclose%
\pgfusepath{fill}%
\end{pgfscope}%
\begin{pgfscope}%
\pgfpathrectangle{\pgfqpoint{0.539299in}{0.078740in}}{\pgfqpoint{7.842520in}{7.842520in}}%
\pgfusepath{clip}%
\pgfsetbuttcap%
\pgfsetroundjoin%
\definecolor{currentfill}{rgb}{0.185556,0.418570,0.556753}%
\pgfsetfillcolor{currentfill}%
\pgfsetlinewidth{0.000000pt}%
\definecolor{currentstroke}{rgb}{0.421908,0.805774,0.351910}%
\pgfsetstrokecolor{currentstroke}%
\pgfsetdash{}{0pt}%
\pgfpathmoveto{\pgfqpoint{2.034011in}{3.419410in}}%
\pgfpathlineto{\pgfqpoint{2.148306in}{3.953837in}}%
\pgfpathlineto{\pgfqpoint{2.234531in}{4.033237in}}%
\pgfpathclose%
\pgfusepath{fill}%
\end{pgfscope}%
\begin{pgfscope}%
\pgfpathrectangle{\pgfqpoint{0.539299in}{0.078740in}}{\pgfqpoint{7.842520in}{7.842520in}}%
\pgfusepath{clip}%
\pgfsetbuttcap%
\pgfsetroundjoin%
\definecolor{currentfill}{rgb}{0.248629,0.278775,0.534556}%
\pgfsetfillcolor{currentfill}%
\pgfsetlinewidth{0.000000pt}%
\definecolor{currentstroke}{rgb}{0.430983,0.808473,0.346476}%
\pgfsetstrokecolor{currentstroke}%
\pgfsetdash{}{0pt}%
\pgfpathmoveto{\pgfqpoint{5.573268in}{3.296150in}}%
\pgfpathlineto{\pgfqpoint{5.497218in}{3.291318in}}%
\pgfpathlineto{\pgfqpoint{5.711861in}{3.191718in}}%
\pgfpathclose%
\pgfusepath{fill}%
\end{pgfscope}%
\begin{pgfscope}%
\pgfpathrectangle{\pgfqpoint{0.539299in}{0.078740in}}{\pgfqpoint{7.842520in}{7.842520in}}%
\pgfusepath{clip}%
\pgfsetbuttcap%
\pgfsetroundjoin%
\definecolor{currentfill}{rgb}{0.277018,0.050344,0.375715}%
\pgfsetfillcolor{currentfill}%
\pgfsetlinewidth{0.000000pt}%
\definecolor{currentstroke}{rgb}{0.440137,0.811138,0.340967}%
\pgfsetstrokecolor{currentstroke}%
\pgfsetdash{}{0pt}%
\pgfpathmoveto{\pgfqpoint{7.116450in}{2.449156in}}%
\pgfpathlineto{\pgfqpoint{7.188539in}{2.515946in}}%
\pgfpathlineto{\pgfqpoint{6.976816in}{2.562893in}}%
\pgfpathclose%
\pgfusepath{fill}%
\end{pgfscope}%
\begin{pgfscope}%
\pgfpathrectangle{\pgfqpoint{0.539299in}{0.078740in}}{\pgfqpoint{7.842520in}{7.842520in}}%
\pgfusepath{clip}%
\pgfsetbuttcap%
\pgfsetroundjoin%
\definecolor{currentfill}{rgb}{0.190631,0.407061,0.556089}%
\pgfsetfillcolor{currentfill}%
\pgfsetlinewidth{0.000000pt}%
\definecolor{currentstroke}{rgb}{0.449368,0.813768,0.335384}%
\pgfsetstrokecolor{currentstroke}%
\pgfsetdash{}{0pt}%
\pgfpathmoveto{\pgfqpoint{4.943959in}{3.876474in}}%
\pgfpathlineto{\pgfqpoint{5.082079in}{3.713504in}}%
\pgfpathlineto{\pgfqpoint{5.159216in}{3.667393in}}%
\pgfpathclose%
\pgfusepath{fill}%
\end{pgfscope}%
\begin{pgfscope}%
\pgfpathrectangle{\pgfqpoint{0.539299in}{0.078740in}}{\pgfqpoint{7.842520in}{7.842520in}}%
\pgfusepath{clip}%
\pgfsetbuttcap%
\pgfsetroundjoin%
\definecolor{currentfill}{rgb}{0.203063,0.379716,0.553925}%
\pgfsetfillcolor{currentfill}%
\pgfsetlinewidth{0.000000pt}%
\definecolor{currentstroke}{rgb}{0.458674,0.816363,0.329727}%
\pgfsetstrokecolor{currentstroke}%
\pgfsetdash{}{0pt}%
\pgfpathmoveto{\pgfqpoint{5.159216in}{3.667393in}}%
\pgfpathlineto{\pgfqpoint{5.082079in}{3.713504in}}%
\pgfpathlineto{\pgfqpoint{5.220286in}{3.561627in}}%
\pgfpathclose%
\pgfusepath{fill}%
\end{pgfscope}%
\begin{pgfscope}%
\pgfpathrectangle{\pgfqpoint{0.539299in}{0.078740in}}{\pgfqpoint{7.842520in}{7.842520in}}%
\pgfusepath{clip}%
\pgfsetbuttcap%
\pgfsetroundjoin%
\definecolor{currentfill}{rgb}{0.174274,0.445044,0.557792}%
\pgfsetfillcolor{currentfill}%
\pgfsetlinewidth{0.000000pt}%
\definecolor{currentstroke}{rgb}{0.468053,0.818921,0.323998}%
\pgfsetstrokecolor{currentstroke}%
\pgfsetdash{}{0pt}%
\pgfpathmoveto{\pgfqpoint{4.805857in}{4.049409in}}%
\pgfpathlineto{\pgfqpoint{4.943959in}{3.876474in}}%
\pgfpathlineto{\pgfqpoint{5.021609in}{3.813991in}}%
\pgfpathclose%
\pgfusepath{fill}%
\end{pgfscope}%
\begin{pgfscope}%
\pgfpathrectangle{\pgfqpoint{0.539299in}{0.078740in}}{\pgfqpoint{7.842520in}{7.842520in}}%
\pgfusepath{clip}%
\pgfsetbuttcap%
\pgfsetroundjoin%
\definecolor{currentfill}{rgb}{0.288921,0.758394,0.428426}%
\pgfsetfillcolor{currentfill}%
\pgfsetlinewidth{0.000000pt}%
\definecolor{currentstroke}{rgb}{0.477504,0.821444,0.318195}%
\pgfsetstrokecolor{currentstroke}%
\pgfsetdash{}{0pt}%
\pgfpathmoveto{\pgfqpoint{3.429695in}{5.281129in}}%
\pgfpathlineto{\pgfqpoint{3.480023in}{5.382736in}}%
\pgfpathlineto{\pgfqpoint{3.564639in}{5.312794in}}%
\pgfpathclose%
\pgfusepath{fill}%
\end{pgfscope}%
\begin{pgfscope}%
\pgfpathrectangle{\pgfqpoint{0.539299in}{0.078740in}}{\pgfqpoint{7.842520in}{7.842520in}}%
\pgfusepath{clip}%
\pgfsetbuttcap%
\pgfsetroundjoin%
\definecolor{currentfill}{rgb}{0.141935,0.526453,0.555991}%
\pgfsetfillcolor{currentfill}%
\pgfsetlinewidth{0.000000pt}%
\definecolor{currentstroke}{rgb}{0.487026,0.823929,0.312321}%
\pgfsetstrokecolor{currentstroke}%
\pgfsetdash{}{0pt}%
\pgfpathmoveto{\pgfqpoint{2.406429in}{4.162525in}}%
\pgfpathlineto{\pgfqpoint{2.320596in}{4.102334in}}%
\pgfpathlineto{\pgfqpoint{2.440760in}{4.553021in}}%
\pgfpathclose%
\pgfusepath{fill}%
\end{pgfscope}%
\begin{pgfscope}%
\pgfpathrectangle{\pgfqpoint{0.539299in}{0.078740in}}{\pgfqpoint{7.842520in}{7.842520in}}%
\pgfusepath{clip}%
\pgfsetbuttcap%
\pgfsetroundjoin%
\definecolor{currentfill}{rgb}{0.239374,0.735588,0.455688}%
\pgfsetfillcolor{currentfill}%
\pgfsetlinewidth{0.000000pt}%
\definecolor{currentstroke}{rgb}{0.496615,0.826376,0.306377}%
\pgfsetstrokecolor{currentstroke}%
\pgfsetdash{}{0pt}%
\pgfpathmoveto{\pgfqpoint{3.700962in}{5.285351in}}%
\pgfpathlineto{\pgfqpoint{3.838249in}{5.208623in}}%
\pgfpathlineto{\pgfqpoint{3.921341in}{5.107578in}}%
\pgfpathclose%
\pgfusepath{fill}%
\end{pgfscope}%
\begin{pgfscope}%
\pgfpathrectangle{\pgfqpoint{0.539299in}{0.078740in}}{\pgfqpoint{7.842520in}{7.842520in}}%
\pgfusepath{clip}%
\pgfsetbuttcap%
\pgfsetroundjoin%
\definecolor{currentfill}{rgb}{0.153364,0.497000,0.557724}%
\pgfsetfillcolor{currentfill}%
\pgfsetlinewidth{0.000000pt}%
\definecolor{currentstroke}{rgb}{0.506271,0.828786,0.300362}%
\pgfsetstrokecolor{currentstroke}%
\pgfsetdash{}{0pt}%
\pgfpathmoveto{\pgfqpoint{4.746612in}{4.139665in}}%
\pgfpathlineto{\pgfqpoint{4.667716in}{4.230148in}}%
\pgfpathlineto{\pgfqpoint{4.805857in}{4.049409in}}%
\pgfpathclose%
\pgfusepath{fill}%
\end{pgfscope}%
\begin{pgfscope}%
\pgfpathrectangle{\pgfqpoint{0.539299in}{0.078740in}}{\pgfqpoint{7.842520in}{7.842520in}}%
\pgfusepath{clip}%
\pgfsetbuttcap%
\pgfsetroundjoin%
\definecolor{currentfill}{rgb}{0.218130,0.347432,0.550038}%
\pgfsetfillcolor{currentfill}%
\pgfsetlinewidth{0.000000pt}%
\definecolor{currentstroke}{rgb}{0.515992,0.831158,0.294279}%
\pgfsetstrokecolor{currentstroke}%
\pgfsetdash{}{0pt}%
\pgfpathmoveto{\pgfqpoint{5.296988in}{3.532612in}}%
\pgfpathlineto{\pgfqpoint{5.220286in}{3.561627in}}%
\pgfpathlineto{\pgfqpoint{5.358646in}{3.421072in}}%
\pgfpathclose%
\pgfusepath{fill}%
\end{pgfscope}%
\begin{pgfscope}%
\pgfpathrectangle{\pgfqpoint{0.539299in}{0.078740in}}{\pgfqpoint{7.842520in}{7.842520in}}%
\pgfusepath{clip}%
\pgfsetbuttcap%
\pgfsetroundjoin%
\definecolor{currentfill}{rgb}{0.126326,0.644107,0.525311}%
\pgfsetfillcolor{currentfill}%
\pgfsetlinewidth{0.000000pt}%
\definecolor{currentstroke}{rgb}{0.525776,0.833491,0.288127}%
\pgfsetstrokecolor{currentstroke}%
\pgfsetdash{}{0pt}%
\pgfpathmoveto{\pgfqpoint{2.825015in}{5.016747in}}%
\pgfpathlineto{\pgfqpoint{2.699021in}{4.682479in}}%
\pgfpathlineto{\pgfqpoint{2.613215in}{4.649492in}}%
\pgfpathclose%
\pgfusepath{fill}%
\end{pgfscope}%
\begin{pgfscope}%
\pgfpathrectangle{\pgfqpoint{0.539299in}{0.078740in}}{\pgfqpoint{7.842520in}{7.842520in}}%
\pgfusepath{clip}%
\pgfsetbuttcap%
\pgfsetroundjoin%
\definecolor{currentfill}{rgb}{0.283091,0.110553,0.431554}%
\pgfsetfillcolor{currentfill}%
\pgfsetlinewidth{0.000000pt}%
\definecolor{currentstroke}{rgb}{0.535621,0.835785,0.281908}%
\pgfsetstrokecolor{currentstroke}%
\pgfsetdash{}{0pt}%
\pgfpathmoveto{\pgfqpoint{6.697430in}{2.767965in}}%
\pgfpathlineto{\pgfqpoint{6.763891in}{2.602375in}}%
\pgfpathlineto{\pgfqpoint{6.837125in}{2.669064in}}%
\pgfpathclose%
\pgfusepath{fill}%
\end{pgfscope}%
\begin{pgfscope}%
\pgfpathrectangle{\pgfqpoint{0.539299in}{0.078740in}}{\pgfqpoint{7.842520in}{7.842520in}}%
\pgfusepath{clip}%
\pgfsetbuttcap%
\pgfsetroundjoin%
\definecolor{currentfill}{rgb}{0.136408,0.541173,0.554483}%
\pgfsetfillcolor{currentfill}%
\pgfsetlinewidth{0.000000pt}%
\definecolor{currentstroke}{rgb}{0.545524,0.838039,0.275626}%
\pgfsetstrokecolor{currentstroke}%
\pgfsetdash{}{0pt}%
\pgfpathmoveto{\pgfqpoint{4.667716in}{4.230148in}}%
\pgfpathlineto{\pgfqpoint{4.609098in}{4.314481in}}%
\pgfpathlineto{\pgfqpoint{4.529493in}{4.415381in}}%
\pgfpathclose%
\pgfusepath{fill}%
\end{pgfscope}%
\begin{pgfscope}%
\pgfpathrectangle{\pgfqpoint{0.539299in}{0.078740in}}{\pgfqpoint{7.842520in}{7.842520in}}%
\pgfusepath{clip}%
\pgfsetbuttcap%
\pgfsetroundjoin%
\definecolor{currentfill}{rgb}{0.282884,0.135920,0.453427}%
\pgfsetfillcolor{currentfill}%
\pgfsetlinewidth{0.000000pt}%
\definecolor{currentstroke}{rgb}{0.555484,0.840254,0.269281}%
\pgfsetstrokecolor{currentstroke}%
\pgfsetdash{}{0pt}%
\pgfpathmoveto{\pgfqpoint{6.697430in}{2.767965in}}%
\pgfpathlineto{\pgfqpoint{6.483913in}{2.795763in}}%
\pgfpathlineto{\pgfqpoint{6.623869in}{2.702268in}}%
\pgfpathclose%
\pgfusepath{fill}%
\end{pgfscope}%
\begin{pgfscope}%
\pgfpathrectangle{\pgfqpoint{0.539299in}{0.078740in}}{\pgfqpoint{7.842520in}{7.842520in}}%
\pgfusepath{clip}%
\pgfsetbuttcap%
\pgfsetroundjoin%
\definecolor{currentfill}{rgb}{0.278012,0.180367,0.486697}%
\pgfsetfillcolor{currentfill}%
\pgfsetlinewidth{0.000000pt}%
\definecolor{currentstroke}{rgb}{0.565498,0.842430,0.262877}%
\pgfsetstrokecolor{currentstroke}%
\pgfsetdash{}{0pt}%
\pgfpathmoveto{\pgfqpoint{6.269433in}{2.819189in}}%
\pgfpathlineto{\pgfqpoint{6.344091in}{2.884105in}}%
\pgfpathlineto{\pgfqpoint{6.204464in}{2.968792in}}%
\pgfpathclose%
\pgfusepath{fill}%
\end{pgfscope}%
\begin{pgfscope}%
\pgfpathrectangle{\pgfqpoint{0.539299in}{0.078740in}}{\pgfqpoint{7.842520in}{7.842520in}}%
\pgfusepath{clip}%
\pgfsetbuttcap%
\pgfsetroundjoin%
\definecolor{currentfill}{rgb}{0.233603,0.313828,0.543914}%
\pgfsetfillcolor{currentfill}%
\pgfsetlinewidth{0.000000pt}%
\definecolor{currentstroke}{rgb}{0.575563,0.844566,0.256415}%
\pgfsetstrokecolor{currentstroke}%
\pgfsetdash{}{0pt}%
\pgfpathmoveto{\pgfqpoint{5.434988in}{3.409236in}}%
\pgfpathlineto{\pgfqpoint{5.358646in}{3.421072in}}%
\pgfpathlineto{\pgfqpoint{5.497218in}{3.291318in}}%
\pgfpathclose%
\pgfusepath{fill}%
\end{pgfscope}%
\begin{pgfscope}%
\pgfpathrectangle{\pgfqpoint{0.539299in}{0.078740in}}{\pgfqpoint{7.842520in}{7.842520in}}%
\pgfusepath{clip}%
\pgfsetbuttcap%
\pgfsetroundjoin%
\definecolor{currentfill}{rgb}{0.274128,0.199721,0.498911}%
\pgfsetfillcolor{currentfill}%
\pgfsetlinewidth{0.000000pt}%
\definecolor{currentstroke}{rgb}{0.585678,0.846661,0.249897}%
\pgfsetstrokecolor{currentstroke}%
\pgfsetdash{}{0pt}%
\pgfpathmoveto{\pgfqpoint{5.990036in}{3.000736in}}%
\pgfpathlineto{\pgfqpoint{6.129597in}{2.909848in}}%
\pgfpathlineto{\pgfqpoint{6.204464in}{2.968792in}}%
\pgfpathclose%
\pgfusepath{fill}%
\end{pgfscope}%
\begin{pgfscope}%
\pgfpathrectangle{\pgfqpoint{0.539299in}{0.078740in}}{\pgfqpoint{7.842520in}{7.842520in}}%
\pgfusepath{clip}%
\pgfsetbuttcap%
\pgfsetroundjoin%
\definecolor{currentfill}{rgb}{0.304148,0.764704,0.419943}%
\pgfsetfillcolor{currentfill}%
\pgfsetlinewidth{0.000000pt}%
\definecolor{currentstroke}{rgb}{0.595839,0.848717,0.243329}%
\pgfsetstrokecolor{currentstroke}%
\pgfsetdash{}{0pt}%
\pgfpathmoveto{\pgfqpoint{3.344948in}{5.333814in}}%
\pgfpathlineto{\pgfqpoint{3.480023in}{5.382736in}}%
\pgfpathlineto{\pgfqpoint{3.429695in}{5.281129in}}%
\pgfpathclose%
\pgfusepath{fill}%
\end{pgfscope}%
\begin{pgfscope}%
\pgfpathrectangle{\pgfqpoint{0.539299in}{0.078740in}}{\pgfqpoint{7.842520in}{7.842520in}}%
\pgfusepath{clip}%
\pgfsetbuttcap%
\pgfsetroundjoin%
\definecolor{currentfill}{rgb}{0.126453,0.570633,0.549841}%
\pgfsetfillcolor{currentfill}%
\pgfsetlinewidth{0.000000pt}%
\definecolor{currentstroke}{rgb}{0.606045,0.850733,0.236712}%
\pgfsetstrokecolor{currentstroke}%
\pgfsetdash{}{0pt}%
\pgfpathmoveto{\pgfqpoint{4.529493in}{4.415381in}}%
\pgfpathlineto{\pgfqpoint{4.609098in}{4.314481in}}%
\pgfpathlineto{\pgfqpoint{4.391174in}{4.600585in}}%
\pgfpathclose%
\pgfusepath{fill}%
\end{pgfscope}%
\begin{pgfscope}%
\pgfpathrectangle{\pgfqpoint{0.539299in}{0.078740in}}{\pgfqpoint{7.842520in}{7.842520in}}%
\pgfusepath{clip}%
\pgfsetbuttcap%
\pgfsetroundjoin%
\definecolor{currentfill}{rgb}{0.202219,0.715272,0.476084}%
\pgfsetfillcolor{currentfill}%
\pgfsetlinewidth{0.000000pt}%
\definecolor{currentstroke}{rgb}{0.616293,0.852709,0.230052}%
\pgfsetstrokecolor{currentstroke}%
\pgfsetdash{}{0pt}%
\pgfpathmoveto{\pgfqpoint{2.910930in}{5.025499in}}%
\pgfpathlineto{\pgfqpoint{3.040850in}{5.259550in}}%
\pgfpathlineto{\pgfqpoint{2.996415in}{5.025431in}}%
\pgfpathclose%
\pgfusepath{fill}%
\end{pgfscope}%
\begin{pgfscope}%
\pgfpathrectangle{\pgfqpoint{0.539299in}{0.078740in}}{\pgfqpoint{7.842520in}{7.842520in}}%
\pgfusepath{clip}%
\pgfsetbuttcap%
\pgfsetroundjoin%
\definecolor{currentfill}{rgb}{0.202219,0.715272,0.476084}%
\pgfsetfillcolor{currentfill}%
\pgfsetlinewidth{0.000000pt}%
\definecolor{currentstroke}{rgb}{0.626579,0.854645,0.223353}%
\pgfsetstrokecolor{currentstroke}%
\pgfsetdash{}{0pt}%
\pgfpathmoveto{\pgfqpoint{3.976153in}{5.092524in}}%
\pgfpathlineto{\pgfqpoint{4.058651in}{4.983988in}}%
\pgfpathlineto{\pgfqpoint{3.838249in}{5.208623in}}%
\pgfpathclose%
\pgfusepath{fill}%
\end{pgfscope}%
\begin{pgfscope}%
\pgfpathrectangle{\pgfqpoint{0.539299in}{0.078740in}}{\pgfqpoint{7.842520in}{7.842520in}}%
\pgfusepath{clip}%
\pgfsetbuttcap%
\pgfsetroundjoin%
\definecolor{currentfill}{rgb}{0.260571,0.246922,0.522828}%
\pgfsetfillcolor{currentfill}%
\pgfsetlinewidth{0.000000pt}%
\definecolor{currentstroke}{rgb}{0.636902,0.856542,0.216620}%
\pgfsetstrokecolor{currentstroke}%
\pgfsetdash{}{0pt}%
\pgfpathmoveto{\pgfqpoint{5.775167in}{3.059358in}}%
\pgfpathlineto{\pgfqpoint{5.850784in}{3.093964in}}%
\pgfpathlineto{\pgfqpoint{5.711861in}{3.191718in}}%
\pgfpathclose%
\pgfusepath{fill}%
\end{pgfscope}%
\begin{pgfscope}%
\pgfpathrectangle{\pgfqpoint{0.539299in}{0.078740in}}{\pgfqpoint{7.842520in}{7.842520in}}%
\pgfusepath{clip}%
\pgfsetbuttcap%
\pgfsetroundjoin%
\definecolor{currentfill}{rgb}{0.218130,0.347432,0.550038}%
\pgfsetfillcolor{currentfill}%
\pgfsetlinewidth{0.000000pt}%
\definecolor{currentstroke}{rgb}{0.647257,0.858400,0.209861}%
\pgfsetstrokecolor{currentstroke}%
\pgfsetdash{}{0pt}%
\pgfpathmoveto{\pgfqpoint{2.062008in}{3.862537in}}%
\pgfpathlineto{\pgfqpoint{1.947878in}{3.337740in}}%
\pgfpathlineto{\pgfqpoint{1.861534in}{3.249013in}}%
\pgfpathclose%
\pgfusepath{fill}%
\end{pgfscope}%
\begin{pgfscope}%
\pgfpathrectangle{\pgfqpoint{0.539299in}{0.078740in}}{\pgfqpoint{7.842520in}{7.842520in}}%
\pgfusepath{clip}%
\pgfsetbuttcap%
\pgfsetroundjoin%
\definecolor{currentfill}{rgb}{0.282327,0.094955,0.417331}%
\pgfsetfillcolor{currentfill}%
\pgfsetlinewidth{0.000000pt}%
\definecolor{currentstroke}{rgb}{0.657642,0.860219,0.203082}%
\pgfsetstrokecolor{currentstroke}%
\pgfsetdash{}{0pt}%
\pgfpathmoveto{\pgfqpoint{6.837125in}{2.669064in}}%
\pgfpathlineto{\pgfqpoint{6.763891in}{2.602375in}}%
\pgfpathlineto{\pgfqpoint{6.976816in}{2.562893in}}%
\pgfpathclose%
\pgfusepath{fill}%
\end{pgfscope}%
\begin{pgfscope}%
\pgfpathrectangle{\pgfqpoint{0.539299in}{0.078740in}}{\pgfqpoint{7.842520in}{7.842520in}}%
\pgfusepath{clip}%
\pgfsetbuttcap%
\pgfsetroundjoin%
\definecolor{currentfill}{rgb}{0.120638,0.625828,0.533488}%
\pgfsetfillcolor{currentfill}%
\pgfsetlinewidth{0.000000pt}%
\definecolor{currentstroke}{rgb}{0.668054,0.861999,0.196293}%
\pgfsetstrokecolor{currentstroke}%
\pgfsetdash{}{0pt}%
\pgfpathmoveto{\pgfqpoint{4.333879in}{4.667896in}}%
\pgfpathlineto{\pgfqpoint{4.252784in}{4.779999in}}%
\pgfpathlineto{\pgfqpoint{4.391174in}{4.600585in}}%
\pgfpathclose%
\pgfusepath{fill}%
\end{pgfscope}%
\begin{pgfscope}%
\pgfpathrectangle{\pgfqpoint{0.539299in}{0.078740in}}{\pgfqpoint{7.842520in}{7.842520in}}%
\pgfusepath{clip}%
\pgfsetbuttcap%
\pgfsetroundjoin%
\definecolor{currentfill}{rgb}{0.175707,0.697900,0.491033}%
\pgfsetfillcolor{currentfill}%
\pgfsetlinewidth{0.000000pt}%
\definecolor{currentstroke}{rgb}{0.678489,0.863742,0.189503}%
\pgfsetstrokecolor{currentstroke}%
\pgfsetdash{}{0pt}%
\pgfpathmoveto{\pgfqpoint{4.114399in}{4.946663in}}%
\pgfpathlineto{\pgfqpoint{4.058651in}{4.983988in}}%
\pgfpathlineto{\pgfqpoint{3.976153in}{5.092524in}}%
\pgfpathclose%
\pgfusepath{fill}%
\end{pgfscope}%
\begin{pgfscope}%
\pgfpathrectangle{\pgfqpoint{0.539299in}{0.078740in}}{\pgfqpoint{7.842520in}{7.842520in}}%
\pgfusepath{clip}%
\pgfsetbuttcap%
\pgfsetroundjoin%
\definecolor{currentfill}{rgb}{0.239374,0.735588,0.455688}%
\pgfsetfillcolor{currentfill}%
\pgfsetlinewidth{0.000000pt}%
\definecolor{currentstroke}{rgb}{0.688944,0.865448,0.182725}%
\pgfsetstrokecolor{currentstroke}%
\pgfsetdash{}{0pt}%
\pgfpathmoveto{\pgfqpoint{2.996415in}{5.025431in}}%
\pgfpathlineto{\pgfqpoint{3.040850in}{5.259550in}}%
\pgfpathlineto{\pgfqpoint{3.126635in}{5.241459in}}%
\pgfpathclose%
\pgfusepath{fill}%
\end{pgfscope}%
\begin{pgfscope}%
\pgfpathrectangle{\pgfqpoint{0.539299in}{0.078740in}}{\pgfqpoint{7.842520in}{7.842520in}}%
\pgfusepath{clip}%
\pgfsetbuttcap%
\pgfsetroundjoin%
\definecolor{currentfill}{rgb}{0.283229,0.120777,0.440584}%
\pgfsetfillcolor{currentfill}%
\pgfsetlinewidth{0.000000pt}%
\definecolor{currentstroke}{rgb}{0.699415,0.867117,0.175971}%
\pgfsetstrokecolor{currentstroke}%
\pgfsetdash{}{0pt}%
\pgfpathmoveto{\pgfqpoint{6.623869in}{2.702268in}}%
\pgfpathlineto{\pgfqpoint{6.763891in}{2.602375in}}%
\pgfpathlineto{\pgfqpoint{6.697430in}{2.767965in}}%
\pgfpathclose%
\pgfusepath{fill}%
\end{pgfscope}%
\begin{pgfscope}%
\pgfpathrectangle{\pgfqpoint{0.539299in}{0.078740in}}{\pgfqpoint{7.842520in}{7.842520in}}%
\pgfusepath{clip}%
\pgfsetbuttcap%
\pgfsetroundjoin%
\definecolor{currentfill}{rgb}{0.140210,0.665859,0.513427}%
\pgfsetfillcolor{currentfill}%
\pgfsetlinewidth{0.000000pt}%
\definecolor{currentstroke}{rgb}{0.709898,0.868751,0.169257}%
\pgfsetstrokecolor{currentstroke}%
\pgfsetdash{}{0pt}%
\pgfpathmoveto{\pgfqpoint{4.196221in}{4.834423in}}%
\pgfpathlineto{\pgfqpoint{4.114399in}{4.946663in}}%
\pgfpathlineto{\pgfqpoint{4.252784in}{4.779999in}}%
\pgfpathclose%
\pgfusepath{fill}%
\end{pgfscope}%
\begin{pgfscope}%
\pgfpathrectangle{\pgfqpoint{0.539299in}{0.078740in}}{\pgfqpoint{7.842520in}{7.842520in}}%
\pgfusepath{clip}%
\pgfsetbuttcap%
\pgfsetroundjoin%
\definecolor{currentfill}{rgb}{0.277134,0.185228,0.489898}%
\pgfsetfillcolor{currentfill}%
\pgfsetlinewidth{0.000000pt}%
\definecolor{currentstroke}{rgb}{0.720391,0.870350,0.162603}%
\pgfsetstrokecolor{currentstroke}%
\pgfsetdash{}{0pt}%
\pgfpathmoveto{\pgfqpoint{6.204464in}{2.968792in}}%
\pgfpathlineto{\pgfqpoint{6.129597in}{2.909848in}}%
\pgfpathlineto{\pgfqpoint{6.269433in}{2.819189in}}%
\pgfpathclose%
\pgfusepath{fill}%
\end{pgfscope}%
\begin{pgfscope}%
\pgfpathrectangle{\pgfqpoint{0.539299in}{0.078740in}}{\pgfqpoint{7.842520in}{7.842520in}}%
\pgfusepath{clip}%
\pgfsetbuttcap%
\pgfsetroundjoin%
\definecolor{currentfill}{rgb}{0.296479,0.761561,0.424223}%
\pgfsetfillcolor{currentfill}%
\pgfsetlinewidth{0.000000pt}%
\definecolor{currentstroke}{rgb}{0.730889,0.871916,0.156029}%
\pgfsetstrokecolor{currentstroke}%
\pgfsetdash{}{0pt}%
\pgfpathmoveto{\pgfqpoint{3.344948in}{5.333814in}}%
\pgfpathlineto{\pgfqpoint{3.211903in}{5.214787in}}%
\pgfpathlineto{\pgfqpoint{3.259589in}{5.378978in}}%
\pgfpathclose%
\pgfusepath{fill}%
\end{pgfscope}%
\begin{pgfscope}%
\pgfpathrectangle{\pgfqpoint{0.539299in}{0.078740in}}{\pgfqpoint{7.842520in}{7.842520in}}%
\pgfusepath{clip}%
\pgfsetbuttcap%
\pgfsetroundjoin%
\definecolor{currentfill}{rgb}{0.248629,0.278775,0.534556}%
\pgfsetfillcolor{currentfill}%
\pgfsetlinewidth{0.000000pt}%
\definecolor{currentstroke}{rgb}{0.741388,0.873449,0.149561}%
\pgfsetstrokecolor{currentstroke}%
\pgfsetdash{}{0pt}%
\pgfpathmoveto{\pgfqpoint{5.711861in}{3.191718in}}%
\pgfpathlineto{\pgfqpoint{5.497218in}{3.291318in}}%
\pgfpathlineto{\pgfqpoint{5.636047in}{3.171258in}}%
\pgfpathclose%
\pgfusepath{fill}%
\end{pgfscope}%
\begin{pgfscope}%
\pgfpathrectangle{\pgfqpoint{0.539299in}{0.078740in}}{\pgfqpoint{7.842520in}{7.842520in}}%
\pgfusepath{clip}%
\pgfsetbuttcap%
\pgfsetroundjoin%
\definecolor{currentfill}{rgb}{0.304148,0.764704,0.419943}%
\pgfsetfillcolor{currentfill}%
\pgfsetlinewidth{0.000000pt}%
\definecolor{currentstroke}{rgb}{0.751884,0.874951,0.143228}%
\pgfsetstrokecolor{currentstroke}%
\pgfsetdash{}{0pt}%
\pgfpathmoveto{\pgfqpoint{3.564639in}{5.312794in}}%
\pgfpathlineto{\pgfqpoint{3.480023in}{5.382736in}}%
\pgfpathlineto{\pgfqpoint{3.700962in}{5.285351in}}%
\pgfpathclose%
\pgfusepath{fill}%
\end{pgfscope}%
\begin{pgfscope}%
\pgfpathrectangle{\pgfqpoint{0.539299in}{0.078740in}}{\pgfqpoint{7.842520in}{7.842520in}}%
\pgfusepath{clip}%
\pgfsetbuttcap%
\pgfsetroundjoin%
\definecolor{currentfill}{rgb}{0.281477,0.755203,0.432552}%
\pgfsetfillcolor{currentfill}%
\pgfsetlinewidth{0.000000pt}%
\definecolor{currentstroke}{rgb}{0.762373,0.876424,0.137064}%
\pgfsetstrokecolor{currentstroke}%
\pgfsetdash{}{0pt}%
\pgfpathmoveto{\pgfqpoint{3.259589in}{5.378978in}}%
\pgfpathlineto{\pgfqpoint{3.211903in}{5.214787in}}%
\pgfpathlineto{\pgfqpoint{3.126635in}{5.241459in}}%
\pgfpathclose%
\pgfusepath{fill}%
\end{pgfscope}%
\begin{pgfscope}%
\pgfpathrectangle{\pgfqpoint{0.539299in}{0.078740in}}{\pgfqpoint{7.842520in}{7.842520in}}%
\pgfusepath{clip}%
\pgfsetbuttcap%
\pgfsetroundjoin%
\definecolor{currentfill}{rgb}{0.280868,0.160771,0.472899}%
\pgfsetfillcolor{currentfill}%
\pgfsetlinewidth{0.000000pt}%
\definecolor{currentstroke}{rgb}{0.772852,0.877868,0.131109}%
\pgfsetstrokecolor{currentstroke}%
\pgfsetdash{}{0pt}%
\pgfpathmoveto{\pgfqpoint{6.409497in}{2.726792in}}%
\pgfpathlineto{\pgfqpoint{6.483913in}{2.795763in}}%
\pgfpathlineto{\pgfqpoint{6.344091in}{2.884105in}}%
\pgfpathclose%
\pgfusepath{fill}%
\end{pgfscope}%
\begin{pgfscope}%
\pgfpathrectangle{\pgfqpoint{0.539299in}{0.078740in}}{\pgfqpoint{7.842520in}{7.842520in}}%
\pgfusepath{clip}%
\pgfsetbuttcap%
\pgfsetroundjoin%
\definecolor{currentfill}{rgb}{0.267968,0.223549,0.512008}%
\pgfsetfillcolor{currentfill}%
\pgfsetlinewidth{0.000000pt}%
\definecolor{currentstroke}{rgb}{0.783315,0.879285,0.125405}%
\pgfsetstrokecolor{currentstroke}%
\pgfsetdash{}{0pt}%
\pgfpathmoveto{\pgfqpoint{5.914592in}{2.953804in}}%
\pgfpathlineto{\pgfqpoint{5.990036in}{3.000736in}}%
\pgfpathlineto{\pgfqpoint{5.850784in}{3.093964in}}%
\pgfpathclose%
\pgfusepath{fill}%
\end{pgfscope}%
\begin{pgfscope}%
\pgfpathrectangle{\pgfqpoint{0.539299in}{0.078740in}}{\pgfqpoint{7.842520in}{7.842520in}}%
\pgfusepath{clip}%
\pgfsetbuttcap%
\pgfsetroundjoin%
\definecolor{currentfill}{rgb}{0.255645,0.260703,0.528312}%
\pgfsetfillcolor{currentfill}%
\pgfsetlinewidth{0.000000pt}%
\definecolor{currentstroke}{rgb}{0.793760,0.880678,0.120005}%
\pgfsetstrokecolor{currentstroke}%
\pgfsetdash{}{0pt}%
\pgfpathmoveto{\pgfqpoint{5.636047in}{3.171258in}}%
\pgfpathlineto{\pgfqpoint{5.775167in}{3.059358in}}%
\pgfpathlineto{\pgfqpoint{5.711861in}{3.191718in}}%
\pgfpathclose%
\pgfusepath{fill}%
\end{pgfscope}%
\begin{pgfscope}%
\pgfpathrectangle{\pgfqpoint{0.539299in}{0.078740in}}{\pgfqpoint{7.842520in}{7.842520in}}%
\pgfusepath{clip}%
\pgfsetbuttcap%
\pgfsetroundjoin%
\definecolor{currentfill}{rgb}{0.125394,0.574318,0.549086}%
\pgfsetfillcolor{currentfill}%
\pgfsetlinewidth{0.000000pt}%
\definecolor{currentstroke}{rgb}{0.804182,0.882046,0.114965}%
\pgfsetstrokecolor{currentstroke}%
\pgfsetdash{}{0pt}%
\pgfpathmoveto{\pgfqpoint{2.440760in}{4.553021in}}%
\pgfpathlineto{\pgfqpoint{2.527103in}{4.606886in}}%
\pgfpathlineto{\pgfqpoint{2.406429in}{4.162525in}}%
\pgfpathclose%
\pgfusepath{fill}%
\end{pgfscope}%
\begin{pgfscope}%
\pgfpathrectangle{\pgfqpoint{0.539299in}{0.078740in}}{\pgfqpoint{7.842520in}{7.842520in}}%
\pgfusepath{clip}%
\pgfsetbuttcap%
\pgfsetroundjoin%
\definecolor{currentfill}{rgb}{0.277018,0.050344,0.375715}%
\pgfsetfillcolor{currentfill}%
\pgfsetlinewidth{0.000000pt}%
\definecolor{currentstroke}{rgb}{0.814576,0.883393,0.110347}%
\pgfsetstrokecolor{currentstroke}%
\pgfsetdash{}{0pt}%
\pgfpathmoveto{\pgfqpoint{6.976816in}{2.562893in}}%
\pgfpathlineto{\pgfqpoint{7.043838in}{2.379025in}}%
\pgfpathlineto{\pgfqpoint{7.116450in}{2.449156in}}%
\pgfpathclose%
\pgfusepath{fill}%
\end{pgfscope}%
\begin{pgfscope}%
\pgfpathrectangle{\pgfqpoint{0.539299in}{0.078740in}}{\pgfqpoint{7.842520in}{7.842520in}}%
\pgfusepath{clip}%
\pgfsetbuttcap%
\pgfsetroundjoin%
\definecolor{currentfill}{rgb}{0.141935,0.526453,0.555991}%
\pgfsetfillcolor{currentfill}%
\pgfsetlinewidth{0.000000pt}%
\definecolor{currentstroke}{rgb}{0.824940,0.884720,0.106217}%
\pgfsetstrokecolor{currentstroke}%
\pgfsetdash{}{0pt}%
\pgfpathmoveto{\pgfqpoint{2.320596in}{4.102334in}}%
\pgfpathlineto{\pgfqpoint{2.234531in}{4.033237in}}%
\pgfpathlineto{\pgfqpoint{2.440760in}{4.553021in}}%
\pgfpathclose%
\pgfusepath{fill}%
\end{pgfscope}%
\begin{pgfscope}%
\pgfpathrectangle{\pgfqpoint{0.539299in}{0.078740in}}{\pgfqpoint{7.842520in}{7.842520in}}%
\pgfusepath{clip}%
\pgfsetbuttcap%
\pgfsetroundjoin%
\definecolor{currentfill}{rgb}{0.281446,0.084320,0.407414}%
\pgfsetfillcolor{currentfill}%
\pgfsetlinewidth{0.000000pt}%
\definecolor{currentstroke}{rgb}{0.835270,0.886029,0.102646}%
\pgfsetstrokecolor{currentstroke}%
\pgfsetdash{}{0pt}%
\pgfpathmoveto{\pgfqpoint{6.976816in}{2.562893in}}%
\pgfpathlineto{\pgfqpoint{6.763891in}{2.602375in}}%
\pgfpathlineto{\pgfqpoint{6.903908in}{2.494991in}}%
\pgfpathclose%
\pgfusepath{fill}%
\end{pgfscope}%
\begin{pgfscope}%
\pgfpathrectangle{\pgfqpoint{0.539299in}{0.078740in}}{\pgfqpoint{7.842520in}{7.842520in}}%
\pgfusepath{clip}%
\pgfsetbuttcap%
\pgfsetroundjoin%
\definecolor{currentfill}{rgb}{0.280255,0.165693,0.476498}%
\pgfsetfillcolor{currentfill}%
\pgfsetlinewidth{0.000000pt}%
\definecolor{currentstroke}{rgb}{0.845561,0.887322,0.099702}%
\pgfsetstrokecolor{currentstroke}%
\pgfsetdash{}{0pt}%
\pgfpathmoveto{\pgfqpoint{6.344091in}{2.884105in}}%
\pgfpathlineto{\pgfqpoint{6.269433in}{2.819189in}}%
\pgfpathlineto{\pgfqpoint{6.409497in}{2.726792in}}%
\pgfpathclose%
\pgfusepath{fill}%
\end{pgfscope}%
\begin{pgfscope}%
\pgfpathrectangle{\pgfqpoint{0.539299in}{0.078740in}}{\pgfqpoint{7.842520in}{7.842520in}}%
\pgfusepath{clip}%
\pgfsetbuttcap%
\pgfsetroundjoin%
\definecolor{currentfill}{rgb}{0.271828,0.209303,0.504434}%
\pgfsetfillcolor{currentfill}%
\pgfsetlinewidth{0.000000pt}%
\definecolor{currentstroke}{rgb}{0.855810,0.888601,0.097452}%
\pgfsetstrokecolor{currentstroke}%
\pgfsetdash{}{0pt}%
\pgfpathmoveto{\pgfqpoint{5.990036in}{3.000736in}}%
\pgfpathlineto{\pgfqpoint{5.914592in}{2.953804in}}%
\pgfpathlineto{\pgfqpoint{6.129597in}{2.909848in}}%
\pgfpathclose%
\pgfusepath{fill}%
\end{pgfscope}%
\begin{pgfscope}%
\pgfpathrectangle{\pgfqpoint{0.539299in}{0.078740in}}{\pgfqpoint{7.842520in}{7.842520in}}%
\pgfusepath{clip}%
\pgfsetbuttcap%
\pgfsetroundjoin%
\definecolor{currentfill}{rgb}{0.188923,0.410910,0.556326}%
\pgfsetfillcolor{currentfill}%
\pgfsetlinewidth{0.000000pt}%
\definecolor{currentstroke}{rgb}{0.866013,0.889868,0.095953}%
\pgfsetstrokecolor{currentstroke}%
\pgfsetdash{}{0pt}%
\pgfpathmoveto{\pgfqpoint{1.947878in}{3.337740in}}%
\pgfpathlineto{\pgfqpoint{2.062008in}{3.862537in}}%
\pgfpathlineto{\pgfqpoint{2.148306in}{3.953837in}}%
\pgfpathclose%
\pgfusepath{fill}%
\end{pgfscope}%
\begin{pgfscope}%
\pgfpathrectangle{\pgfqpoint{0.539299in}{0.078740in}}{\pgfqpoint{7.842520in}{7.842520in}}%
\pgfusepath{clip}%
\pgfsetbuttcap%
\pgfsetroundjoin%
\definecolor{currentfill}{rgb}{0.282884,0.135920,0.453427}%
\pgfsetfillcolor{currentfill}%
\pgfsetlinewidth{0.000000pt}%
\definecolor{currentstroke}{rgb}{0.876168,0.891125,0.095250}%
\pgfsetstrokecolor{currentstroke}%
\pgfsetdash{}{0pt}%
\pgfpathmoveto{\pgfqpoint{6.549730in}{2.630860in}}%
\pgfpathlineto{\pgfqpoint{6.623869in}{2.702268in}}%
\pgfpathlineto{\pgfqpoint{6.483913in}{2.795763in}}%
\pgfpathclose%
\pgfusepath{fill}%
\end{pgfscope}%
\begin{pgfscope}%
\pgfpathrectangle{\pgfqpoint{0.539299in}{0.078740in}}{\pgfqpoint{7.842520in}{7.842520in}}%
\pgfusepath{clip}%
\pgfsetbuttcap%
\pgfsetroundjoin%
\definecolor{currentfill}{rgb}{0.263663,0.237631,0.518762}%
\pgfsetfillcolor{currentfill}%
\pgfsetlinewidth{0.000000pt}%
\definecolor{currentstroke}{rgb}{0.886271,0.892374,0.095374}%
\pgfsetstrokecolor{currentstroke}%
\pgfsetdash{}{0pt}%
\pgfpathmoveto{\pgfqpoint{5.914592in}{2.953804in}}%
\pgfpathlineto{\pgfqpoint{5.850784in}{3.093964in}}%
\pgfpathlineto{\pgfqpoint{5.775167in}{3.059358in}}%
\pgfpathclose%
\pgfusepath{fill}%
\end{pgfscope}%
\begin{pgfscope}%
\pgfpathrectangle{\pgfqpoint{0.539299in}{0.078740in}}{\pgfqpoint{7.842520in}{7.842520in}}%
\pgfusepath{clip}%
\pgfsetbuttcap%
\pgfsetroundjoin%
\definecolor{currentfill}{rgb}{0.278791,0.062145,0.386592}%
\pgfsetfillcolor{currentfill}%
\pgfsetlinewidth{0.000000pt}%
\definecolor{currentstroke}{rgb}{0.896320,0.893616,0.096335}%
\pgfsetstrokecolor{currentstroke}%
\pgfsetdash{}{0pt}%
\pgfpathmoveto{\pgfqpoint{6.903908in}{2.494991in}}%
\pgfpathlineto{\pgfqpoint{7.043838in}{2.379025in}}%
\pgfpathlineto{\pgfqpoint{6.976816in}{2.562893in}}%
\pgfpathclose%
\pgfusepath{fill}%
\end{pgfscope}%
\begin{pgfscope}%
\pgfpathrectangle{\pgfqpoint{0.539299in}{0.078740in}}{\pgfqpoint{7.842520in}{7.842520in}}%
\pgfusepath{clip}%
\pgfsetbuttcap%
\pgfsetroundjoin%
\definecolor{currentfill}{rgb}{0.195860,0.395433,0.555276}%
\pgfsetfillcolor{currentfill}%
\pgfsetlinewidth{0.000000pt}%
\definecolor{currentstroke}{rgb}{0.906311,0.894855,0.098125}%
\pgfsetstrokecolor{currentstroke}%
\pgfsetdash{}{0pt}%
\pgfpathmoveto{\pgfqpoint{5.220286in}{3.561627in}}%
\pgfpathlineto{\pgfqpoint{5.082079in}{3.713504in}}%
\pgfpathlineto{\pgfqpoint{5.143030in}{3.602315in}}%
\pgfpathclose%
\pgfusepath{fill}%
\end{pgfscope}%
\begin{pgfscope}%
\pgfpathrectangle{\pgfqpoint{0.539299in}{0.078740in}}{\pgfqpoint{7.842520in}{7.842520in}}%
\pgfusepath{clip}%
\pgfsetbuttcap%
\pgfsetroundjoin%
\definecolor{currentfill}{rgb}{0.174274,0.445044,0.557792}%
\pgfsetfillcolor{currentfill}%
\pgfsetlinewidth{0.000000pt}%
\definecolor{currentstroke}{rgb}{0.916242,0.896091,0.100717}%
\pgfsetstrokecolor{currentstroke}%
\pgfsetdash{}{0pt}%
\pgfpathmoveto{\pgfqpoint{5.082079in}{3.713504in}}%
\pgfpathlineto{\pgfqpoint{4.943959in}{3.876474in}}%
\pgfpathlineto{\pgfqpoint{4.865625in}{3.949141in}}%
\pgfpathclose%
\pgfusepath{fill}%
\end{pgfscope}%
\begin{pgfscope}%
\pgfpathrectangle{\pgfqpoint{0.539299in}{0.078740in}}{\pgfqpoint{7.842520in}{7.842520in}}%
\pgfusepath{clip}%
\pgfsetbuttcap%
\pgfsetroundjoin%
\definecolor{currentfill}{rgb}{0.206756,0.371758,0.553117}%
\pgfsetfillcolor{currentfill}%
\pgfsetlinewidth{0.000000pt}%
\definecolor{currentstroke}{rgb}{0.926106,0.897330,0.104071}%
\pgfsetstrokecolor{currentstroke}%
\pgfsetdash{}{0pt}%
\pgfpathmoveto{\pgfqpoint{5.143030in}{3.602315in}}%
\pgfpathlineto{\pgfqpoint{5.358646in}{3.421072in}}%
\pgfpathlineto{\pgfqpoint{5.220286in}{3.561627in}}%
\pgfpathclose%
\pgfusepath{fill}%
\end{pgfscope}%
\begin{pgfscope}%
\pgfpathrectangle{\pgfqpoint{0.539299in}{0.078740in}}{\pgfqpoint{7.842520in}{7.842520in}}%
\pgfusepath{clip}%
\pgfsetbuttcap%
\pgfsetroundjoin%
\definecolor{currentfill}{rgb}{0.163625,0.471133,0.558148}%
\pgfsetfillcolor{currentfill}%
\pgfsetlinewidth{0.000000pt}%
\definecolor{currentstroke}{rgb}{0.935904,0.898570,0.108131}%
\pgfsetstrokecolor{currentstroke}%
\pgfsetdash{}{0pt}%
\pgfpathmoveto{\pgfqpoint{4.865625in}{3.949141in}}%
\pgfpathlineto{\pgfqpoint{4.943959in}{3.876474in}}%
\pgfpathlineto{\pgfqpoint{4.805857in}{4.049409in}}%
\pgfpathclose%
\pgfusepath{fill}%
\end{pgfscope}%
\begin{pgfscope}%
\pgfpathrectangle{\pgfqpoint{0.539299in}{0.078740in}}{\pgfqpoint{7.842520in}{7.842520in}}%
\pgfusepath{clip}%
\pgfsetbuttcap%
\pgfsetroundjoin%
\definecolor{currentfill}{rgb}{0.226397,0.728888,0.462789}%
\pgfsetfillcolor{currentfill}%
\pgfsetlinewidth{0.000000pt}%
\definecolor{currentstroke}{rgb}{0.945636,0.899815,0.112838}%
\pgfsetstrokecolor{currentstroke}%
\pgfsetdash{}{0pt}%
\pgfpathmoveto{\pgfqpoint{2.825015in}{5.016747in}}%
\pgfpathlineto{\pgfqpoint{3.040850in}{5.259550in}}%
\pgfpathlineto{\pgfqpoint{2.910930in}{5.025499in}}%
\pgfpathclose%
\pgfusepath{fill}%
\end{pgfscope}%
\begin{pgfscope}%
\pgfpathrectangle{\pgfqpoint{0.539299in}{0.078740in}}{\pgfqpoint{7.842520in}{7.842520in}}%
\pgfusepath{clip}%
\pgfsetbuttcap%
\pgfsetroundjoin%
\definecolor{currentfill}{rgb}{0.282290,0.145912,0.461510}%
\pgfsetfillcolor{currentfill}%
\pgfsetlinewidth{0.000000pt}%
\definecolor{currentstroke}{rgb}{0.955300,0.901065,0.118128}%
\pgfsetstrokecolor{currentstroke}%
\pgfsetdash{}{0pt}%
\pgfpathmoveto{\pgfqpoint{6.483913in}{2.795763in}}%
\pgfpathlineto{\pgfqpoint{6.409497in}{2.726792in}}%
\pgfpathlineto{\pgfqpoint{6.549730in}{2.630860in}}%
\pgfpathclose%
\pgfusepath{fill}%
\end{pgfscope}%
\begin{pgfscope}%
\pgfpathrectangle{\pgfqpoint{0.539299in}{0.078740in}}{\pgfqpoint{7.842520in}{7.842520in}}%
\pgfusepath{clip}%
\pgfsetbuttcap%
\pgfsetroundjoin%
\definecolor{currentfill}{rgb}{0.223925,0.334994,0.548053}%
\pgfsetfillcolor{currentfill}%
\pgfsetlinewidth{0.000000pt}%
\definecolor{currentstroke}{rgb}{0.964894,0.902323,0.123941}%
\pgfsetstrokecolor{currentstroke}%
\pgfsetdash{}{0pt}%
\pgfpathmoveto{\pgfqpoint{5.497218in}{3.291318in}}%
\pgfpathlineto{\pgfqpoint{5.358646in}{3.421072in}}%
\pgfpathlineto{\pgfqpoint{5.281811in}{3.444596in}}%
\pgfpathclose%
\pgfusepath{fill}%
\end{pgfscope}%
\begin{pgfscope}%
\pgfpathrectangle{\pgfqpoint{0.539299in}{0.078740in}}{\pgfqpoint{7.842520in}{7.842520in}}%
\pgfusepath{clip}%
\pgfsetbuttcap%
\pgfsetroundjoin%
\definecolor{currentfill}{rgb}{0.283197,0.115680,0.436115}%
\pgfsetfillcolor{currentfill}%
\pgfsetlinewidth{0.000000pt}%
\definecolor{currentstroke}{rgb}{0.974417,0.903590,0.130215}%
\pgfsetstrokecolor{currentstroke}%
\pgfsetdash{}{0pt}%
\pgfpathmoveto{\pgfqpoint{6.690065in}{2.529737in}}%
\pgfpathlineto{\pgfqpoint{6.763891in}{2.602375in}}%
\pgfpathlineto{\pgfqpoint{6.623869in}{2.702268in}}%
\pgfpathclose%
\pgfusepath{fill}%
\end{pgfscope}%
\begin{pgfscope}%
\pgfpathrectangle{\pgfqpoint{0.539299in}{0.078740in}}{\pgfqpoint{7.842520in}{7.842520in}}%
\pgfusepath{clip}%
\pgfsetbuttcap%
\pgfsetroundjoin%
\definecolor{currentfill}{rgb}{0.223925,0.334994,0.548053}%
\pgfsetfillcolor{currentfill}%
\pgfsetlinewidth{0.000000pt}%
\definecolor{currentstroke}{rgb}{0.983868,0.904867,0.136897}%
\pgfsetstrokecolor{currentstroke}%
\pgfsetdash{}{0pt}%
\pgfpathmoveto{\pgfqpoint{1.975740in}{3.757494in}}%
\pgfpathlineto{\pgfqpoint{1.861534in}{3.249013in}}%
\pgfpathlineto{\pgfqpoint{1.775040in}{3.152116in}}%
\pgfpathclose%
\pgfusepath{fill}%
\end{pgfscope}%
\begin{pgfscope}%
\pgfpathrectangle{\pgfqpoint{0.539299in}{0.078740in}}{\pgfqpoint{7.842520in}{7.842520in}}%
\pgfusepath{clip}%
\pgfsetbuttcap%
\pgfsetroundjoin%
\definecolor{currentfill}{rgb}{0.304148,0.764704,0.419943}%
\pgfsetfillcolor{currentfill}%
\pgfsetlinewidth{0.000000pt}%
\definecolor{currentstroke}{rgb}{0.993248,0.906157,0.143936}%
\pgfsetstrokecolor{currentstroke}%
\pgfsetdash{}{0pt}%
\pgfpathmoveto{\pgfqpoint{3.754366in}{5.305678in}}%
\pgfpathlineto{\pgfqpoint{3.838249in}{5.208623in}}%
\pgfpathlineto{\pgfqpoint{3.700962in}{5.285351in}}%
\pgfpathclose%
\pgfusepath{fill}%
\end{pgfscope}%
\begin{pgfscope}%
\pgfpathrectangle{\pgfqpoint{0.539299in}{0.078740in}}{\pgfqpoint{7.842520in}{7.842520in}}%
\pgfusepath{clip}%
\pgfsetbuttcap%
\pgfsetroundjoin%
\definecolor{currentfill}{rgb}{0.140536,0.530132,0.555659}%
\pgfsetfillcolor{currentfill}%
\pgfsetlinewidth{0.000000pt}%
\definecolor{currentstroke}{rgb}{0.267004,0.004874,0.329415}%
\pgfsetstrokecolor{currentstroke}%
\pgfsetdash{}{0pt}%
\pgfpathmoveto{\pgfqpoint{4.805857in}{4.049409in}}%
\pgfpathlineto{\pgfqpoint{4.667716in}{4.230148in}}%
\pgfpathlineto{\pgfqpoint{4.588029in}{4.327753in}}%
\pgfpathclose%
\pgfusepath{fill}%
\end{pgfscope}%
\begin{pgfscope}%
\pgfpathrectangle{\pgfqpoint{0.539299in}{0.078740in}}{\pgfqpoint{7.842520in}{7.842520in}}%
\pgfusepath{clip}%
\pgfsetbuttcap%
\pgfsetroundjoin%
\definecolor{currentfill}{rgb}{0.278012,0.180367,0.486697}%
\pgfsetfillcolor{currentfill}%
\pgfsetlinewidth{0.000000pt}%
\definecolor{currentstroke}{rgb}{0.268510,0.009605,0.335427}%
\pgfsetstrokecolor{currentstroke}%
\pgfsetdash{}{0pt}%
\pgfpathmoveto{\pgfqpoint{6.194335in}{2.753858in}}%
\pgfpathlineto{\pgfqpoint{6.269433in}{2.819189in}}%
\pgfpathlineto{\pgfqpoint{6.129597in}{2.909848in}}%
\pgfpathclose%
\pgfusepath{fill}%
\end{pgfscope}%
\begin{pgfscope}%
\pgfpathrectangle{\pgfqpoint{0.539299in}{0.078740in}}{\pgfqpoint{7.842520in}{7.842520in}}%
\pgfusepath{clip}%
\pgfsetbuttcap%
\pgfsetroundjoin%
\definecolor{currentfill}{rgb}{0.352360,0.783011,0.392636}%
\pgfsetfillcolor{currentfill}%
\pgfsetlinewidth{0.000000pt}%
\definecolor{currentstroke}{rgb}{0.269944,0.014625,0.341379}%
\pgfsetstrokecolor{currentstroke}%
\pgfsetdash{}{0pt}%
\pgfpathmoveto{\pgfqpoint{3.344948in}{5.333814in}}%
\pgfpathlineto{\pgfqpoint{3.259589in}{5.378978in}}%
\pgfpathlineto{\pgfqpoint{3.480023in}{5.382736in}}%
\pgfpathclose%
\pgfusepath{fill}%
\end{pgfscope}%
\begin{pgfscope}%
\pgfpathrectangle{\pgfqpoint{0.539299in}{0.078740in}}{\pgfqpoint{7.842520in}{7.842520in}}%
\pgfusepath{clip}%
\pgfsetbuttcap%
\pgfsetroundjoin%
\definecolor{currentfill}{rgb}{0.344074,0.780029,0.397381}%
\pgfsetfillcolor{currentfill}%
\pgfsetlinewidth{0.000000pt}%
\definecolor{currentstroke}{rgb}{0.271305,0.019942,0.347269}%
\pgfsetstrokecolor{currentstroke}%
\pgfsetdash{}{0pt}%
\pgfpathmoveto{\pgfqpoint{3.700962in}{5.285351in}}%
\pgfpathlineto{\pgfqpoint{3.480023in}{5.382736in}}%
\pgfpathlineto{\pgfqpoint{3.616640in}{5.370253in}}%
\pgfpathclose%
\pgfusepath{fill}%
\end{pgfscope}%
\begin{pgfscope}%
\pgfpathrectangle{\pgfqpoint{0.539299in}{0.078740in}}{\pgfqpoint{7.842520in}{7.842520in}}%
\pgfusepath{clip}%
\pgfsetbuttcap%
\pgfsetroundjoin%
\definecolor{currentfill}{rgb}{0.273006,0.204520,0.501721}%
\pgfsetfillcolor{currentfill}%
\pgfsetlinewidth{0.000000pt}%
\definecolor{currentstroke}{rgb}{0.272594,0.025563,0.353093}%
\pgfsetstrokecolor{currentstroke}%
\pgfsetdash{}{0pt}%
\pgfpathmoveto{\pgfqpoint{6.129597in}{2.909848in}}%
\pgfpathlineto{\pgfqpoint{5.914592in}{2.953804in}}%
\pgfpathlineto{\pgfqpoint{6.054321in}{2.852638in}}%
\pgfpathclose%
\pgfusepath{fill}%
\end{pgfscope}%
\begin{pgfscope}%
\pgfpathrectangle{\pgfqpoint{0.539299in}{0.078740in}}{\pgfqpoint{7.842520in}{7.842520in}}%
\pgfusepath{clip}%
\pgfsetbuttcap%
\pgfsetroundjoin%
\definecolor{currentfill}{rgb}{0.132268,0.655014,0.519661}%
\pgfsetfillcolor{currentfill}%
\pgfsetlinewidth{0.000000pt}%
\definecolor{currentstroke}{rgb}{0.273809,0.031497,0.358853}%
\pgfsetstrokecolor{currentstroke}%
\pgfsetdash{}{0pt}%
\pgfpathmoveto{\pgfqpoint{2.613215in}{4.649492in}}%
\pgfpathlineto{\pgfqpoint{2.527103in}{4.606886in}}%
\pgfpathlineto{\pgfqpoint{2.738735in}{4.997476in}}%
\pgfpathclose%
\pgfusepath{fill}%
\end{pgfscope}%
\begin{pgfscope}%
\pgfpathrectangle{\pgfqpoint{0.539299in}{0.078740in}}{\pgfqpoint{7.842520in}{7.842520in}}%
\pgfusepath{clip}%
\pgfsetbuttcap%
\pgfsetroundjoin%
\definecolor{currentfill}{rgb}{0.243113,0.292092,0.538516}%
\pgfsetfillcolor{currentfill}%
\pgfsetlinewidth{0.000000pt}%
\definecolor{currentstroke}{rgb}{0.274952,0.037752,0.364543}%
\pgfsetstrokecolor{currentstroke}%
\pgfsetdash{}{0pt}%
\pgfpathmoveto{\pgfqpoint{5.497218in}{3.291318in}}%
\pgfpathlineto{\pgfqpoint{5.559838in}{3.161010in}}%
\pgfpathlineto{\pgfqpoint{5.636047in}{3.171258in}}%
\pgfpathclose%
\pgfusepath{fill}%
\end{pgfscope}%
\begin{pgfscope}%
\pgfpathrectangle{\pgfqpoint{0.539299in}{0.078740in}}{\pgfqpoint{7.842520in}{7.842520in}}%
\pgfusepath{clip}%
\pgfsetbuttcap%
\pgfsetroundjoin%
\definecolor{currentfill}{rgb}{0.162016,0.687316,0.499129}%
\pgfsetfillcolor{currentfill}%
\pgfsetlinewidth{0.000000pt}%
\definecolor{currentstroke}{rgb}{0.276022,0.044167,0.370164}%
\pgfsetstrokecolor{currentstroke}%
\pgfsetdash{}{0pt}%
\pgfpathmoveto{\pgfqpoint{2.613215in}{4.649492in}}%
\pgfpathlineto{\pgfqpoint{2.738735in}{4.997476in}}%
\pgfpathlineto{\pgfqpoint{2.825015in}{5.016747in}}%
\pgfpathclose%
\pgfusepath{fill}%
\end{pgfscope}%
\begin{pgfscope}%
\pgfpathrectangle{\pgfqpoint{0.539299in}{0.078740in}}{\pgfqpoint{7.842520in}{7.842520in}}%
\pgfusepath{clip}%
\pgfsetbuttcap%
\pgfsetroundjoin%
\definecolor{currentfill}{rgb}{0.282327,0.094955,0.417331}%
\pgfsetfillcolor{currentfill}%
\pgfsetlinewidth{0.000000pt}%
\definecolor{currentstroke}{rgb}{0.277018,0.050344,0.375715}%
\pgfsetstrokecolor{currentstroke}%
\pgfsetdash{}{0pt}%
\pgfpathmoveto{\pgfqpoint{6.903908in}{2.494991in}}%
\pgfpathlineto{\pgfqpoint{6.763891in}{2.602375in}}%
\pgfpathlineto{\pgfqpoint{6.690065in}{2.529737in}}%
\pgfpathclose%
\pgfusepath{fill}%
\end{pgfscope}%
\begin{pgfscope}%
\pgfpathrectangle{\pgfqpoint{0.539299in}{0.078740in}}{\pgfqpoint{7.842520in}{7.842520in}}%
\pgfusepath{clip}%
\pgfsetbuttcap%
\pgfsetroundjoin%
\definecolor{currentfill}{rgb}{0.125394,0.574318,0.549086}%
\pgfsetfillcolor{currentfill}%
\pgfsetlinewidth{0.000000pt}%
\definecolor{currentstroke}{rgb}{0.277941,0.056324,0.381191}%
\pgfsetstrokecolor{currentstroke}%
\pgfsetdash{}{0pt}%
\pgfpathmoveto{\pgfqpoint{4.667716in}{4.230148in}}%
\pgfpathlineto{\pgfqpoint{4.529493in}{4.415381in}}%
\pgfpathlineto{\pgfqpoint{4.449060in}{4.521490in}}%
\pgfpathclose%
\pgfusepath{fill}%
\end{pgfscope}%
\begin{pgfscope}%
\pgfpathrectangle{\pgfqpoint{0.539299in}{0.078740in}}{\pgfqpoint{7.842520in}{7.842520in}}%
\pgfusepath{clip}%
\pgfsetbuttcap%
\pgfsetroundjoin%
\definecolor{currentfill}{rgb}{0.283229,0.120777,0.440584}%
\pgfsetfillcolor{currentfill}%
\pgfsetlinewidth{0.000000pt}%
\definecolor{currentstroke}{rgb}{0.278791,0.062145,0.386592}%
\pgfsetstrokecolor{currentstroke}%
\pgfsetdash{}{0pt}%
\pgfpathmoveto{\pgfqpoint{6.690065in}{2.529737in}}%
\pgfpathlineto{\pgfqpoint{6.623869in}{2.702268in}}%
\pgfpathlineto{\pgfqpoint{6.549730in}{2.630860in}}%
\pgfpathclose%
\pgfusepath{fill}%
\end{pgfscope}%
\begin{pgfscope}%
\pgfpathrectangle{\pgfqpoint{0.539299in}{0.078740in}}{\pgfqpoint{7.842520in}{7.842520in}}%
\pgfusepath{clip}%
\pgfsetbuttcap%
\pgfsetroundjoin%
\definecolor{currentfill}{rgb}{0.119738,0.603785,0.541400}%
\pgfsetfillcolor{currentfill}%
\pgfsetlinewidth{0.000000pt}%
\definecolor{currentstroke}{rgb}{0.279566,0.067836,0.391917}%
\pgfsetstrokecolor{currentstroke}%
\pgfsetdash{}{0pt}%
\pgfpathmoveto{\pgfqpoint{4.449060in}{4.521490in}}%
\pgfpathlineto{\pgfqpoint{4.529493in}{4.415381in}}%
\pgfpathlineto{\pgfqpoint{4.391174in}{4.600585in}}%
\pgfpathclose%
\pgfusepath{fill}%
\end{pgfscope}%
\begin{pgfscope}%
\pgfpathrectangle{\pgfqpoint{0.539299in}{0.078740in}}{\pgfqpoint{7.842520in}{7.842520in}}%
\pgfusepath{clip}%
\pgfsetbuttcap%
\pgfsetroundjoin%
\definecolor{currentfill}{rgb}{0.319809,0.770914,0.411152}%
\pgfsetfillcolor{currentfill}%
\pgfsetlinewidth{0.000000pt}%
\definecolor{currentstroke}{rgb}{0.280267,0.073417,0.397163}%
\pgfsetstrokecolor{currentstroke}%
\pgfsetdash{}{0pt}%
\pgfpathmoveto{\pgfqpoint{3.126635in}{5.241459in}}%
\pgfpathlineto{\pgfqpoint{3.040850in}{5.259550in}}%
\pgfpathlineto{\pgfqpoint{3.259589in}{5.378978in}}%
\pgfpathclose%
\pgfusepath{fill}%
\end{pgfscope}%
\begin{pgfscope}%
\pgfpathrectangle{\pgfqpoint{0.539299in}{0.078740in}}{\pgfqpoint{7.842520in}{7.842520in}}%
\pgfusepath{clip}%
\pgfsetbuttcap%
\pgfsetroundjoin%
\definecolor{currentfill}{rgb}{0.276194,0.190074,0.493001}%
\pgfsetfillcolor{currentfill}%
\pgfsetlinewidth{0.000000pt}%
\definecolor{currentstroke}{rgb}{0.280894,0.078907,0.402329}%
\pgfsetstrokecolor{currentstroke}%
\pgfsetdash{}{0pt}%
\pgfpathmoveto{\pgfqpoint{6.054321in}{2.852638in}}%
\pgfpathlineto{\pgfqpoint{6.194335in}{2.753858in}}%
\pgfpathlineto{\pgfqpoint{6.129597in}{2.909848in}}%
\pgfpathclose%
\pgfusepath{fill}%
\end{pgfscope}%
\begin{pgfscope}%
\pgfpathrectangle{\pgfqpoint{0.539299in}{0.078740in}}{\pgfqpoint{7.842520in}{7.842520in}}%
\pgfusepath{clip}%
\pgfsetbuttcap%
\pgfsetroundjoin%
\definecolor{currentfill}{rgb}{0.266941,0.748751,0.440573}%
\pgfsetfillcolor{currentfill}%
\pgfsetlinewidth{0.000000pt}%
\definecolor{currentstroke}{rgb}{0.281446,0.084320,0.407414}%
\pgfsetstrokecolor{currentstroke}%
\pgfsetdash{}{0pt}%
\pgfpathmoveto{\pgfqpoint{3.838249in}{5.208623in}}%
\pgfpathlineto{\pgfqpoint{3.892829in}{5.198517in}}%
\pgfpathlineto{\pgfqpoint{3.976153in}{5.092524in}}%
\pgfpathclose%
\pgfusepath{fill}%
\end{pgfscope}%
\begin{pgfscope}%
\pgfpathrectangle{\pgfqpoint{0.539299in}{0.078740in}}{\pgfqpoint{7.842520in}{7.842520in}}%
\pgfusepath{clip}%
\pgfsetbuttcap%
\pgfsetroundjoin%
\definecolor{currentfill}{rgb}{0.253935,0.265254,0.529983}%
\pgfsetfillcolor{currentfill}%
\pgfsetlinewidth{0.000000pt}%
\definecolor{currentstroke}{rgb}{0.281924,0.089666,0.412415}%
\pgfsetstrokecolor{currentstroke}%
\pgfsetdash{}{0pt}%
\pgfpathmoveto{\pgfqpoint{5.699184in}{3.033538in}}%
\pgfpathlineto{\pgfqpoint{5.775167in}{3.059358in}}%
\pgfpathlineto{\pgfqpoint{5.636047in}{3.171258in}}%
\pgfpathclose%
\pgfusepath{fill}%
\end{pgfscope}%
\begin{pgfscope}%
\pgfpathrectangle{\pgfqpoint{0.539299in}{0.078740in}}{\pgfqpoint{7.842520in}{7.842520in}}%
\pgfusepath{clip}%
\pgfsetbuttcap%
\pgfsetroundjoin%
\definecolor{currentfill}{rgb}{0.185556,0.418570,0.556753}%
\pgfsetfillcolor{currentfill}%
\pgfsetlinewidth{0.000000pt}%
\definecolor{currentstroke}{rgb}{0.282327,0.094955,0.417331}%
\pgfsetstrokecolor{currentstroke}%
\pgfsetdash{}{0pt}%
\pgfpathmoveto{\pgfqpoint{5.143030in}{3.602315in}}%
\pgfpathlineto{\pgfqpoint{5.082079in}{3.713504in}}%
\pgfpathlineto{\pgfqpoint{5.004323in}{3.770773in}}%
\pgfpathclose%
\pgfusepath{fill}%
\end{pgfscope}%
\begin{pgfscope}%
\pgfpathrectangle{\pgfqpoint{0.539299in}{0.078740in}}{\pgfqpoint{7.842520in}{7.842520in}}%
\pgfusepath{clip}%
\pgfsetbuttcap%
\pgfsetroundjoin%
\definecolor{currentfill}{rgb}{0.280868,0.160771,0.472899}%
\pgfsetfillcolor{currentfill}%
\pgfsetlinewidth{0.000000pt}%
\definecolor{currentstroke}{rgb}{0.282656,0.100196,0.422160}%
\pgfsetstrokecolor{currentstroke}%
\pgfsetdash{}{0pt}%
\pgfpathmoveto{\pgfqpoint{6.269433in}{2.819189in}}%
\pgfpathlineto{\pgfqpoint{6.334603in}{2.655498in}}%
\pgfpathlineto{\pgfqpoint{6.409497in}{2.726792in}}%
\pgfpathclose%
\pgfusepath{fill}%
\end{pgfscope}%
\begin{pgfscope}%
\pgfpathrectangle{\pgfqpoint{0.539299in}{0.078740in}}{\pgfqpoint{7.842520in}{7.842520in}}%
\pgfusepath{clip}%
\pgfsetbuttcap%
\pgfsetroundjoin%
\definecolor{currentfill}{rgb}{0.206756,0.371758,0.553117}%
\pgfsetfillcolor{currentfill}%
\pgfsetlinewidth{0.000000pt}%
\definecolor{currentstroke}{rgb}{0.282910,0.105393,0.426902}%
\pgfsetstrokecolor{currentstroke}%
\pgfsetdash{}{0pt}%
\pgfpathmoveto{\pgfqpoint{5.281811in}{3.444596in}}%
\pgfpathlineto{\pgfqpoint{5.358646in}{3.421072in}}%
\pgfpathlineto{\pgfqpoint{5.143030in}{3.602315in}}%
\pgfpathclose%
\pgfusepath{fill}%
\end{pgfscope}%
\begin{pgfscope}%
\pgfpathrectangle{\pgfqpoint{0.539299in}{0.078740in}}{\pgfqpoint{7.842520in}{7.842520in}}%
\pgfusepath{clip}%
\pgfsetbuttcap%
\pgfsetroundjoin%
\definecolor{currentfill}{rgb}{0.174274,0.445044,0.557792}%
\pgfsetfillcolor{currentfill}%
\pgfsetlinewidth{0.000000pt}%
\definecolor{currentstroke}{rgb}{0.283091,0.110553,0.431554}%
\pgfsetstrokecolor{currentstroke}%
\pgfsetdash{}{0pt}%
\pgfpathmoveto{\pgfqpoint{4.865625in}{3.949141in}}%
\pgfpathlineto{\pgfqpoint{5.004323in}{3.770773in}}%
\pgfpathlineto{\pgfqpoint{5.082079in}{3.713504in}}%
\pgfpathclose%
\pgfusepath{fill}%
\end{pgfscope}%
\begin{pgfscope}%
\pgfpathrectangle{\pgfqpoint{0.539299in}{0.078740in}}{\pgfqpoint{7.842520in}{7.842520in}}%
\pgfusepath{clip}%
\pgfsetbuttcap%
\pgfsetroundjoin%
\definecolor{currentfill}{rgb}{0.279566,0.067836,0.391917}%
\pgfsetfillcolor{currentfill}%
\pgfsetlinewidth{0.000000pt}%
\definecolor{currentstroke}{rgb}{0.283197,0.115680,0.436115}%
\pgfsetstrokecolor{currentstroke}%
\pgfsetdash{}{0pt}%
\pgfpathmoveto{\pgfqpoint{7.043838in}{2.379025in}}%
\pgfpathlineto{\pgfqpoint{6.903908in}{2.494991in}}%
\pgfpathlineto{\pgfqpoint{6.830421in}{2.421820in}}%
\pgfpathclose%
\pgfusepath{fill}%
\end{pgfscope}%
\begin{pgfscope}%
\pgfpathrectangle{\pgfqpoint{0.539299in}{0.078740in}}{\pgfqpoint{7.842520in}{7.842520in}}%
\pgfusepath{clip}%
\pgfsetbuttcap%
\pgfsetroundjoin%
\definecolor{currentfill}{rgb}{0.137339,0.662252,0.515571}%
\pgfsetfillcolor{currentfill}%
\pgfsetlinewidth{0.000000pt}%
\definecolor{currentstroke}{rgb}{0.283229,0.120777,0.440584}%
\pgfsetstrokecolor{currentstroke}%
\pgfsetdash{}{0pt}%
\pgfpathmoveto{\pgfqpoint{4.391174in}{4.600585in}}%
\pgfpathlineto{\pgfqpoint{4.252784in}{4.779999in}}%
\pgfpathlineto{\pgfqpoint{4.170832in}{4.893238in}}%
\pgfpathclose%
\pgfusepath{fill}%
\end{pgfscope}%
\begin{pgfscope}%
\pgfpathrectangle{\pgfqpoint{0.539299in}{0.078740in}}{\pgfqpoint{7.842520in}{7.842520in}}%
\pgfusepath{clip}%
\pgfsetbuttcap%
\pgfsetroundjoin%
\definecolor{currentfill}{rgb}{0.232815,0.732247,0.459277}%
\pgfsetfillcolor{currentfill}%
\pgfsetlinewidth{0.000000pt}%
\definecolor{currentstroke}{rgb}{0.283187,0.125848,0.444960}%
\pgfsetstrokecolor{currentstroke}%
\pgfsetdash{}{0pt}%
\pgfpathmoveto{\pgfqpoint{3.976153in}{5.092524in}}%
\pgfpathlineto{\pgfqpoint{3.892829in}{5.198517in}}%
\pgfpathlineto{\pgfqpoint{4.114399in}{4.946663in}}%
\pgfpathclose%
\pgfusepath{fill}%
\end{pgfscope}%
\begin{pgfscope}%
\pgfpathrectangle{\pgfqpoint{0.539299in}{0.078740in}}{\pgfqpoint{7.842520in}{7.842520in}}%
\pgfusepath{clip}%
\pgfsetbuttcap%
\pgfsetroundjoin%
\definecolor{currentfill}{rgb}{0.175707,0.697900,0.491033}%
\pgfsetfillcolor{currentfill}%
\pgfsetlinewidth{0.000000pt}%
\definecolor{currentstroke}{rgb}{0.283072,0.130895,0.449241}%
\pgfsetstrokecolor{currentstroke}%
\pgfsetdash{}{0pt}%
\pgfpathmoveto{\pgfqpoint{4.252784in}{4.779999in}}%
\pgfpathlineto{\pgfqpoint{4.114399in}{4.946663in}}%
\pgfpathlineto{\pgfqpoint{4.031728in}{5.058097in}}%
\pgfpathclose%
\pgfusepath{fill}%
\end{pgfscope}%
\begin{pgfscope}%
\pgfpathrectangle{\pgfqpoint{0.539299in}{0.078740in}}{\pgfqpoint{7.842520in}{7.842520in}}%
\pgfusepath{clip}%
\pgfsetbuttcap%
\pgfsetroundjoin%
\definecolor{currentfill}{rgb}{0.223925,0.334994,0.548053}%
\pgfsetfillcolor{currentfill}%
\pgfsetlinewidth{0.000000pt}%
\definecolor{currentstroke}{rgb}{0.282884,0.135920,0.453427}%
\pgfsetstrokecolor{currentstroke}%
\pgfsetdash{}{0pt}%
\pgfpathmoveto{\pgfqpoint{5.281811in}{3.444596in}}%
\pgfpathlineto{\pgfqpoint{5.420728in}{3.297681in}}%
\pgfpathlineto{\pgfqpoint{5.497218in}{3.291318in}}%
\pgfpathclose%
\pgfusepath{fill}%
\end{pgfscope}%
\begin{pgfscope}%
\pgfpathrectangle{\pgfqpoint{0.539299in}{0.078740in}}{\pgfqpoint{7.842520in}{7.842520in}}%
\pgfusepath{clip}%
\pgfsetbuttcap%
\pgfsetroundjoin%
\definecolor{currentfill}{rgb}{0.344074,0.780029,0.397381}%
\pgfsetfillcolor{currentfill}%
\pgfsetlinewidth{0.000000pt}%
\definecolor{currentstroke}{rgb}{0.282623,0.140926,0.457517}%
\pgfsetstrokecolor{currentstroke}%
\pgfsetdash{}{0pt}%
\pgfpathmoveto{\pgfqpoint{3.616640in}{5.370253in}}%
\pgfpathlineto{\pgfqpoint{3.754366in}{5.305678in}}%
\pgfpathlineto{\pgfqpoint{3.700962in}{5.285351in}}%
\pgfpathclose%
\pgfusepath{fill}%
\end{pgfscope}%
\begin{pgfscope}%
\pgfpathrectangle{\pgfqpoint{0.539299in}{0.078740in}}{\pgfqpoint{7.842520in}{7.842520in}}%
\pgfusepath{clip}%
\pgfsetbuttcap%
\pgfsetroundjoin%
\definecolor{currentfill}{rgb}{0.260571,0.246922,0.522828}%
\pgfsetfillcolor{currentfill}%
\pgfsetlinewidth{0.000000pt}%
\definecolor{currentstroke}{rgb}{0.282290,0.145912,0.461510}%
\pgfsetstrokecolor{currentstroke}%
\pgfsetdash{}{0pt}%
\pgfpathmoveto{\pgfqpoint{5.775167in}{3.059358in}}%
\pgfpathlineto{\pgfqpoint{5.699184in}{3.033538in}}%
\pgfpathlineto{\pgfqpoint{5.914592in}{2.953804in}}%
\pgfpathclose%
\pgfusepath{fill}%
\end{pgfscope}%
\begin{pgfscope}%
\pgfpathrectangle{\pgfqpoint{0.539299in}{0.078740in}}{\pgfqpoint{7.842520in}{7.842520in}}%
\pgfusepath{clip}%
\pgfsetbuttcap%
\pgfsetroundjoin%
\definecolor{currentfill}{rgb}{0.151918,0.500685,0.557587}%
\pgfsetfillcolor{currentfill}%
\pgfsetlinewidth{0.000000pt}%
\definecolor{currentstroke}{rgb}{0.281887,0.150881,0.465405}%
\pgfsetstrokecolor{currentstroke}%
\pgfsetdash{}{0pt}%
\pgfpathmoveto{\pgfqpoint{4.805857in}{4.049409in}}%
\pgfpathlineto{\pgfqpoint{4.726876in}{4.135706in}}%
\pgfpathlineto{\pgfqpoint{4.865625in}{3.949141in}}%
\pgfpathclose%
\pgfusepath{fill}%
\end{pgfscope}%
\begin{pgfscope}%
\pgfpathrectangle{\pgfqpoint{0.539299in}{0.078740in}}{\pgfqpoint{7.842520in}{7.842520in}}%
\pgfusepath{clip}%
\pgfsetbuttcap%
\pgfsetroundjoin%
\definecolor{currentfill}{rgb}{0.192357,0.403199,0.555836}%
\pgfsetfillcolor{currentfill}%
\pgfsetlinewidth{0.000000pt}%
\definecolor{currentstroke}{rgb}{0.281412,0.155834,0.469201}%
\pgfsetstrokecolor{currentstroke}%
\pgfsetdash{}{0pt}%
\pgfpathmoveto{\pgfqpoint{2.062008in}{3.862537in}}%
\pgfpathlineto{\pgfqpoint{1.861534in}{3.249013in}}%
\pgfpathlineto{\pgfqpoint{1.975740in}{3.757494in}}%
\pgfpathclose%
\pgfusepath{fill}%
\end{pgfscope}%
\begin{pgfscope}%
\pgfpathrectangle{\pgfqpoint{0.539299in}{0.078740in}}{\pgfqpoint{7.842520in}{7.842520in}}%
\pgfusepath{clip}%
\pgfsetbuttcap%
\pgfsetroundjoin%
\definecolor{currentfill}{rgb}{0.235526,0.309527,0.542944}%
\pgfsetfillcolor{currentfill}%
\pgfsetlinewidth{0.000000pt}%
\definecolor{currentstroke}{rgb}{0.280868,0.160771,0.472899}%
\pgfsetstrokecolor{currentstroke}%
\pgfsetdash{}{0pt}%
\pgfpathmoveto{\pgfqpoint{5.497218in}{3.291318in}}%
\pgfpathlineto{\pgfqpoint{5.420728in}{3.297681in}}%
\pgfpathlineto{\pgfqpoint{5.559838in}{3.161010in}}%
\pgfpathclose%
\pgfusepath{fill}%
\end{pgfscope}%
\begin{pgfscope}%
\pgfpathrectangle{\pgfqpoint{0.539299in}{0.078740in}}{\pgfqpoint{7.842520in}{7.842520in}}%
\pgfusepath{clip}%
\pgfsetbuttcap%
\pgfsetroundjoin%
\definecolor{currentfill}{rgb}{0.281446,0.084320,0.407414}%
\pgfsetfillcolor{currentfill}%
\pgfsetlinewidth{0.000000pt}%
\definecolor{currentstroke}{rgb}{0.280255,0.165693,0.476498}%
\pgfsetstrokecolor{currentstroke}%
\pgfsetdash{}{0pt}%
\pgfpathmoveto{\pgfqpoint{6.830421in}{2.421820in}}%
\pgfpathlineto{\pgfqpoint{6.903908in}{2.494991in}}%
\pgfpathlineto{\pgfqpoint{6.690065in}{2.529737in}}%
\pgfpathclose%
\pgfusepath{fill}%
\end{pgfscope}%
\begin{pgfscope}%
\pgfpathrectangle{\pgfqpoint{0.539299in}{0.078740in}}{\pgfqpoint{7.842520in}{7.842520in}}%
\pgfusepath{clip}%
\pgfsetbuttcap%
\pgfsetroundjoin%
\definecolor{currentfill}{rgb}{0.282290,0.145912,0.461510}%
\pgfsetfillcolor{currentfill}%
\pgfsetlinewidth{0.000000pt}%
\definecolor{currentstroke}{rgb}{0.279574,0.170599,0.479997}%
\pgfsetstrokecolor{currentstroke}%
\pgfsetdash{}{0pt}%
\pgfpathmoveto{\pgfqpoint{6.549730in}{2.630860in}}%
\pgfpathlineto{\pgfqpoint{6.409497in}{2.726792in}}%
\pgfpathlineto{\pgfqpoint{6.334603in}{2.655498in}}%
\pgfpathclose%
\pgfusepath{fill}%
\end{pgfscope}%
\begin{pgfscope}%
\pgfpathrectangle{\pgfqpoint{0.539299in}{0.078740in}}{\pgfqpoint{7.842520in}{7.842520in}}%
\pgfusepath{clip}%
\pgfsetbuttcap%
\pgfsetroundjoin%
\definecolor{currentfill}{rgb}{0.140536,0.530132,0.555659}%
\pgfsetfillcolor{currentfill}%
\pgfsetlinewidth{0.000000pt}%
\definecolor{currentstroke}{rgb}{0.278826,0.175490,0.483397}%
\pgfsetstrokecolor{currentstroke}%
\pgfsetdash{}{0pt}%
\pgfpathmoveto{\pgfqpoint{4.588029in}{4.327753in}}%
\pgfpathlineto{\pgfqpoint{4.726876in}{4.135706in}}%
\pgfpathlineto{\pgfqpoint{4.805857in}{4.049409in}}%
\pgfpathclose%
\pgfusepath{fill}%
\end{pgfscope}%
\begin{pgfscope}%
\pgfpathrectangle{\pgfqpoint{0.539299in}{0.078740in}}{\pgfqpoint{7.842520in}{7.842520in}}%
\pgfusepath{clip}%
\pgfsetbuttcap%
\pgfsetroundjoin%
\definecolor{currentfill}{rgb}{0.279574,0.170599,0.479997}%
\pgfsetfillcolor{currentfill}%
\pgfsetlinewidth{0.000000pt}%
\definecolor{currentstroke}{rgb}{0.278012,0.180367,0.486697}%
\pgfsetstrokecolor{currentstroke}%
\pgfsetdash{}{0pt}%
\pgfpathmoveto{\pgfqpoint{6.269433in}{2.819189in}}%
\pgfpathlineto{\pgfqpoint{6.194335in}{2.753858in}}%
\pgfpathlineto{\pgfqpoint{6.334603in}{2.655498in}}%
\pgfpathclose%
\pgfusepath{fill}%
\end{pgfscope}%
\begin{pgfscope}%
\pgfpathrectangle{\pgfqpoint{0.539299in}{0.078740in}}{\pgfqpoint{7.842520in}{7.842520in}}%
\pgfusepath{clip}%
\pgfsetbuttcap%
\pgfsetroundjoin%
\definecolor{currentfill}{rgb}{0.248629,0.278775,0.534556}%
\pgfsetfillcolor{currentfill}%
\pgfsetlinewidth{0.000000pt}%
\definecolor{currentstroke}{rgb}{0.277134,0.185228,0.489898}%
\pgfsetstrokecolor{currentstroke}%
\pgfsetdash{}{0pt}%
\pgfpathmoveto{\pgfqpoint{5.636047in}{3.171258in}}%
\pgfpathlineto{\pgfqpoint{5.559838in}{3.161010in}}%
\pgfpathlineto{\pgfqpoint{5.699184in}{3.033538in}}%
\pgfpathclose%
\pgfusepath{fill}%
\end{pgfscope}%
\begin{pgfscope}%
\pgfpathrectangle{\pgfqpoint{0.539299in}{0.078740in}}{\pgfqpoint{7.842520in}{7.842520in}}%
\pgfusepath{clip}%
\pgfsetbuttcap%
\pgfsetroundjoin%
\definecolor{currentfill}{rgb}{0.124395,0.578002,0.548287}%
\pgfsetfillcolor{currentfill}%
\pgfsetlinewidth{0.000000pt}%
\definecolor{currentstroke}{rgb}{0.276194,0.190074,0.493001}%
\pgfsetstrokecolor{currentstroke}%
\pgfsetdash{}{0pt}%
\pgfpathmoveto{\pgfqpoint{4.449060in}{4.521490in}}%
\pgfpathlineto{\pgfqpoint{4.588029in}{4.327753in}}%
\pgfpathlineto{\pgfqpoint{4.667716in}{4.230148in}}%
\pgfpathclose%
\pgfusepath{fill}%
\end{pgfscope}%
\begin{pgfscope}%
\pgfpathrectangle{\pgfqpoint{0.539299in}{0.078740in}}{\pgfqpoint{7.842520in}{7.842520in}}%
\pgfusepath{clip}%
\pgfsetbuttcap%
\pgfsetroundjoin%
\definecolor{currentfill}{rgb}{0.141935,0.526453,0.555991}%
\pgfsetfillcolor{currentfill}%
\pgfsetlinewidth{0.000000pt}%
\definecolor{currentstroke}{rgb}{0.275191,0.194905,0.496005}%
\pgfsetstrokecolor{currentstroke}%
\pgfsetdash{}{0pt}%
\pgfpathmoveto{\pgfqpoint{2.234531in}{4.033237in}}%
\pgfpathlineto{\pgfqpoint{2.148306in}{3.953837in}}%
\pgfpathlineto{\pgfqpoint{2.267756in}{4.403873in}}%
\pgfpathclose%
\pgfusepath{fill}%
\end{pgfscope}%
\begin{pgfscope}%
\pgfpathrectangle{\pgfqpoint{0.539299in}{0.078740in}}{\pgfqpoint{7.842520in}{7.842520in}}%
\pgfusepath{clip}%
\pgfsetbuttcap%
\pgfsetroundjoin%
\definecolor{currentfill}{rgb}{0.277941,0.056324,0.381191}%
\pgfsetfillcolor{currentfill}%
\pgfsetlinewidth{0.000000pt}%
\definecolor{currentstroke}{rgb}{0.274128,0.199721,0.498911}%
\pgfsetstrokecolor{currentstroke}%
\pgfsetdash{}{0pt}%
\pgfpathmoveto{\pgfqpoint{6.830421in}{2.421820in}}%
\pgfpathlineto{\pgfqpoint{6.970702in}{2.305402in}}%
\pgfpathlineto{\pgfqpoint{7.043838in}{2.379025in}}%
\pgfpathclose%
\pgfusepath{fill}%
\end{pgfscope}%
\begin{pgfscope}%
\pgfpathrectangle{\pgfqpoint{0.539299in}{0.078740in}}{\pgfqpoint{7.842520in}{7.842520in}}%
\pgfusepath{clip}%
\pgfsetbuttcap%
\pgfsetroundjoin%
\definecolor{currentfill}{rgb}{0.124395,0.578002,0.548287}%
\pgfsetfillcolor{currentfill}%
\pgfsetlinewidth{0.000000pt}%
\definecolor{currentstroke}{rgb}{0.273006,0.204520,0.501721}%
\pgfsetstrokecolor{currentstroke}%
\pgfsetdash{}{0pt}%
\pgfpathmoveto{\pgfqpoint{2.440760in}{4.553021in}}%
\pgfpathlineto{\pgfqpoint{2.234531in}{4.033237in}}%
\pgfpathlineto{\pgfqpoint{2.354275in}{4.486045in}}%
\pgfpathclose%
\pgfusepath{fill}%
\end{pgfscope}%
\begin{pgfscope}%
\pgfpathrectangle{\pgfqpoint{0.539299in}{0.078740in}}{\pgfqpoint{7.842520in}{7.842520in}}%
\pgfusepath{clip}%
\pgfsetbuttcap%
\pgfsetroundjoin%
\definecolor{currentfill}{rgb}{0.271828,0.209303,0.504434}%
\pgfsetfillcolor{currentfill}%
\pgfsetlinewidth{0.000000pt}%
\definecolor{currentstroke}{rgb}{0.271828,0.209303,0.504434}%
\pgfsetstrokecolor{currentstroke}%
\pgfsetdash{}{0pt}%
\pgfpathmoveto{\pgfqpoint{5.914592in}{2.953804in}}%
\pgfpathlineto{\pgfqpoint{5.978689in}{2.800383in}}%
\pgfpathlineto{\pgfqpoint{6.054321in}{2.852638in}}%
\pgfpathclose%
\pgfusepath{fill}%
\end{pgfscope}%
\begin{pgfscope}%
\pgfpathrectangle{\pgfqpoint{0.539299in}{0.078740in}}{\pgfqpoint{7.842520in}{7.842520in}}%
\pgfusepath{clip}%
\pgfsetbuttcap%
\pgfsetroundjoin%
\definecolor{currentfill}{rgb}{0.311925,0.767822,0.415586}%
\pgfsetfillcolor{currentfill}%
\pgfsetlinewidth{0.000000pt}%
\definecolor{currentstroke}{rgb}{0.270595,0.214069,0.507052}%
\pgfsetstrokecolor{currentstroke}%
\pgfsetdash{}{0pt}%
\pgfpathmoveto{\pgfqpoint{3.838249in}{5.208623in}}%
\pgfpathlineto{\pgfqpoint{3.754366in}{5.305678in}}%
\pgfpathlineto{\pgfqpoint{3.892829in}{5.198517in}}%
\pgfpathclose%
\pgfusepath{fill}%
\end{pgfscope}%
\begin{pgfscope}%
\pgfpathrectangle{\pgfqpoint{0.539299in}{0.078740in}}{\pgfqpoint{7.842520in}{7.842520in}}%
\pgfusepath{clip}%
\pgfsetbuttcap%
\pgfsetroundjoin%
\definecolor{currentfill}{rgb}{0.283229,0.120777,0.440584}%
\pgfsetfillcolor{currentfill}%
\pgfsetlinewidth{0.000000pt}%
\definecolor{currentstroke}{rgb}{0.269308,0.218818,0.509577}%
\pgfsetstrokecolor{currentstroke}%
\pgfsetdash{}{0pt}%
\pgfpathmoveto{\pgfqpoint{6.549730in}{2.630860in}}%
\pgfpathlineto{\pgfqpoint{6.475077in}{2.555660in}}%
\pgfpathlineto{\pgfqpoint{6.690065in}{2.529737in}}%
\pgfpathclose%
\pgfusepath{fill}%
\end{pgfscope}%
\begin{pgfscope}%
\pgfpathrectangle{\pgfqpoint{0.539299in}{0.078740in}}{\pgfqpoint{7.842520in}{7.842520in}}%
\pgfusepath{clip}%
\pgfsetbuttcap%
\pgfsetroundjoin%
\definecolor{currentfill}{rgb}{0.123444,0.636809,0.528763}%
\pgfsetfillcolor{currentfill}%
\pgfsetlinewidth{0.000000pt}%
\definecolor{currentstroke}{rgb}{0.267968,0.223549,0.512008}%
\pgfsetstrokecolor{currentstroke}%
\pgfsetdash{}{0pt}%
\pgfpathmoveto{\pgfqpoint{4.391174in}{4.600585in}}%
\pgfpathlineto{\pgfqpoint{4.309978in}{4.711999in}}%
\pgfpathlineto{\pgfqpoint{4.449060in}{4.521490in}}%
\pgfpathclose%
\pgfusepath{fill}%
\end{pgfscope}%
\begin{pgfscope}%
\pgfpathrectangle{\pgfqpoint{0.539299in}{0.078740in}}{\pgfqpoint{7.842520in}{7.842520in}}%
\pgfusepath{clip}%
\pgfsetbuttcap%
\pgfsetroundjoin%
\definecolor{currentfill}{rgb}{0.404001,0.800275,0.362552}%
\pgfsetfillcolor{currentfill}%
\pgfsetlinewidth{0.000000pt}%
\definecolor{currentstroke}{rgb}{0.266580,0.228262,0.514349}%
\pgfsetstrokecolor{currentstroke}%
\pgfsetdash{}{0pt}%
\pgfpathmoveto{\pgfqpoint{3.480023in}{5.382736in}}%
\pgfpathlineto{\pgfqpoint{3.259589in}{5.378978in}}%
\pgfpathlineto{\pgfqpoint{3.394745in}{5.445102in}}%
\pgfpathclose%
\pgfusepath{fill}%
\end{pgfscope}%
\begin{pgfscope}%
\pgfpathrectangle{\pgfqpoint{0.539299in}{0.078740in}}{\pgfqpoint{7.842520in}{7.842520in}}%
\pgfusepath{clip}%
\pgfsetbuttcap%
\pgfsetroundjoin%
\definecolor{currentfill}{rgb}{0.262138,0.242286,0.520837}%
\pgfsetfillcolor{currentfill}%
\pgfsetlinewidth{0.000000pt}%
\definecolor{currentstroke}{rgb}{0.265145,0.232956,0.516599}%
\pgfsetstrokecolor{currentstroke}%
\pgfsetdash{}{0pt}%
\pgfpathmoveto{\pgfqpoint{5.914592in}{2.953804in}}%
\pgfpathlineto{\pgfqpoint{5.699184in}{3.033538in}}%
\pgfpathlineto{\pgfqpoint{5.838796in}{2.913871in}}%
\pgfpathclose%
\pgfusepath{fill}%
\end{pgfscope}%
\begin{pgfscope}%
\pgfpathrectangle{\pgfqpoint{0.539299in}{0.078740in}}{\pgfqpoint{7.842520in}{7.842520in}}%
\pgfusepath{clip}%
\pgfsetbuttcap%
\pgfsetroundjoin%
\definecolor{currentfill}{rgb}{0.288921,0.758394,0.428426}%
\pgfsetfillcolor{currentfill}%
\pgfsetlinewidth{0.000000pt}%
\definecolor{currentstroke}{rgb}{0.263663,0.237631,0.518762}%
\pgfsetstrokecolor{currentstroke}%
\pgfsetdash{}{0pt}%
\pgfpathmoveto{\pgfqpoint{2.825015in}{5.016747in}}%
\pgfpathlineto{\pgfqpoint{2.954612in}{5.267216in}}%
\pgfpathlineto{\pgfqpoint{3.040850in}{5.259550in}}%
\pgfpathclose%
\pgfusepath{fill}%
\end{pgfscope}%
\begin{pgfscope}%
\pgfpathrectangle{\pgfqpoint{0.539299in}{0.078740in}}{\pgfqpoint{7.842520in}{7.842520in}}%
\pgfusepath{clip}%
\pgfsetbuttcap%
\pgfsetroundjoin%
\definecolor{currentfill}{rgb}{0.140210,0.665859,0.513427}%
\pgfsetfillcolor{currentfill}%
\pgfsetlinewidth{0.000000pt}%
\definecolor{currentstroke}{rgb}{0.262138,0.242286,0.520837}%
\pgfsetstrokecolor{currentstroke}%
\pgfsetdash{}{0pt}%
\pgfpathmoveto{\pgfqpoint{4.170832in}{4.893238in}}%
\pgfpathlineto{\pgfqpoint{4.309978in}{4.711999in}}%
\pgfpathlineto{\pgfqpoint{4.391174in}{4.600585in}}%
\pgfpathclose%
\pgfusepath{fill}%
\end{pgfscope}%
\begin{pgfscope}%
\pgfpathrectangle{\pgfqpoint{0.539299in}{0.078740in}}{\pgfqpoint{7.842520in}{7.842520in}}%
\pgfusepath{clip}%
\pgfsetbuttcap%
\pgfsetroundjoin%
\definecolor{currentfill}{rgb}{0.275191,0.194905,0.496005}%
\pgfsetfillcolor{currentfill}%
\pgfsetlinewidth{0.000000pt}%
\definecolor{currentstroke}{rgb}{0.260571,0.246922,0.522828}%
\pgfsetstrokecolor{currentstroke}%
\pgfsetdash{}{0pt}%
\pgfpathmoveto{\pgfqpoint{5.978689in}{2.800383in}}%
\pgfpathlineto{\pgfqpoint{6.194335in}{2.753858in}}%
\pgfpathlineto{\pgfqpoint{6.054321in}{2.852638in}}%
\pgfpathclose%
\pgfusepath{fill}%
\end{pgfscope}%
\begin{pgfscope}%
\pgfpathrectangle{\pgfqpoint{0.539299in}{0.078740in}}{\pgfqpoint{7.842520in}{7.842520in}}%
\pgfusepath{clip}%
\pgfsetbuttcap%
\pgfsetroundjoin%
\definecolor{currentfill}{rgb}{0.282623,0.140926,0.457517}%
\pgfsetfillcolor{currentfill}%
\pgfsetlinewidth{0.000000pt}%
\definecolor{currentstroke}{rgb}{0.258965,0.251537,0.524736}%
\pgfsetstrokecolor{currentstroke}%
\pgfsetdash{}{0pt}%
\pgfpathmoveto{\pgfqpoint{6.334603in}{2.655498in}}%
\pgfpathlineto{\pgfqpoint{6.475077in}{2.555660in}}%
\pgfpathlineto{\pgfqpoint{6.549730in}{2.630860in}}%
\pgfpathclose%
\pgfusepath{fill}%
\end{pgfscope}%
\begin{pgfscope}%
\pgfpathrectangle{\pgfqpoint{0.539299in}{0.078740in}}{\pgfqpoint{7.842520in}{7.842520in}}%
\pgfusepath{clip}%
\pgfsetbuttcap%
\pgfsetroundjoin%
\definecolor{currentfill}{rgb}{0.246070,0.738910,0.452024}%
\pgfsetfillcolor{currentfill}%
\pgfsetlinewidth{0.000000pt}%
\definecolor{currentstroke}{rgb}{0.257322,0.256130,0.526563}%
\pgfsetstrokecolor{currentstroke}%
\pgfsetdash{}{0pt}%
\pgfpathmoveto{\pgfqpoint{4.114399in}{4.946663in}}%
\pgfpathlineto{\pgfqpoint{3.892829in}{5.198517in}}%
\pgfpathlineto{\pgfqpoint{4.031728in}{5.058097in}}%
\pgfpathclose%
\pgfusepath{fill}%
\end{pgfscope}%
\begin{pgfscope}%
\pgfpathrectangle{\pgfqpoint{0.539299in}{0.078740in}}{\pgfqpoint{7.842520in}{7.842520in}}%
\pgfusepath{clip}%
\pgfsetbuttcap%
\pgfsetroundjoin%
\definecolor{currentfill}{rgb}{0.231674,0.318106,0.544834}%
\pgfsetfillcolor{currentfill}%
\pgfsetlinewidth{0.000000pt}%
\definecolor{currentstroke}{rgb}{0.255645,0.260703,0.528312}%
\pgfsetstrokecolor{currentstroke}%
\pgfsetdash{}{0pt}%
\pgfpathmoveto{\pgfqpoint{1.775040in}{3.152116in}}%
\pgfpathlineto{\pgfqpoint{1.688478in}{3.045686in}}%
\pgfpathlineto{\pgfqpoint{1.803826in}{3.497087in}}%
\pgfpathclose%
\pgfusepath{fill}%
\end{pgfscope}%
\begin{pgfscope}%
\pgfpathrectangle{\pgfqpoint{0.539299in}{0.078740in}}{\pgfqpoint{7.842520in}{7.842520in}}%
\pgfusepath{clip}%
\pgfsetbuttcap%
\pgfsetroundjoin%
\definecolor{currentfill}{rgb}{0.185783,0.704891,0.485273}%
\pgfsetfillcolor{currentfill}%
\pgfsetlinewidth{0.000000pt}%
\definecolor{currentstroke}{rgb}{0.253935,0.265254,0.529983}%
\pgfsetstrokecolor{currentstroke}%
\pgfsetdash{}{0pt}%
\pgfpathmoveto{\pgfqpoint{4.031728in}{5.058097in}}%
\pgfpathlineto{\pgfqpoint{4.170832in}{4.893238in}}%
\pgfpathlineto{\pgfqpoint{4.252784in}{4.779999in}}%
\pgfpathclose%
\pgfusepath{fill}%
\end{pgfscope}%
\begin{pgfscope}%
\pgfpathrectangle{\pgfqpoint{0.539299in}{0.078740in}}{\pgfqpoint{7.842520in}{7.842520in}}%
\pgfusepath{clip}%
\pgfsetbuttcap%
\pgfsetroundjoin%
\definecolor{currentfill}{rgb}{0.267968,0.223549,0.512008}%
\pgfsetfillcolor{currentfill}%
\pgfsetlinewidth{0.000000pt}%
\definecolor{currentstroke}{rgb}{0.252194,0.269783,0.531579}%
\pgfsetstrokecolor{currentstroke}%
\pgfsetdash{}{0pt}%
\pgfpathmoveto{\pgfqpoint{5.838796in}{2.913871in}}%
\pgfpathlineto{\pgfqpoint{5.978689in}{2.800383in}}%
\pgfpathlineto{\pgfqpoint{5.914592in}{2.953804in}}%
\pgfpathclose%
\pgfusepath{fill}%
\end{pgfscope}%
\begin{pgfscope}%
\pgfpathrectangle{\pgfqpoint{0.539299in}{0.078740in}}{\pgfqpoint{7.842520in}{7.842520in}}%
\pgfusepath{clip}%
\pgfsetbuttcap%
\pgfsetroundjoin%
\definecolor{currentfill}{rgb}{0.412913,0.803041,0.357269}%
\pgfsetfillcolor{currentfill}%
\pgfsetlinewidth{0.000000pt}%
\definecolor{currentstroke}{rgb}{0.250425,0.274290,0.533103}%
\pgfsetstrokecolor{currentstroke}%
\pgfsetdash{}{0pt}%
\pgfpathmoveto{\pgfqpoint{3.480023in}{5.382736in}}%
\pgfpathlineto{\pgfqpoint{3.531606in}{5.447979in}}%
\pgfpathlineto{\pgfqpoint{3.616640in}{5.370253in}}%
\pgfpathclose%
\pgfusepath{fill}%
\end{pgfscope}%
\begin{pgfscope}%
\pgfpathrectangle{\pgfqpoint{0.539299in}{0.078740in}}{\pgfqpoint{7.842520in}{7.842520in}}%
\pgfusepath{clip}%
\pgfsetbuttcap%
\pgfsetroundjoin%
\definecolor{currentfill}{rgb}{0.386433,0.794644,0.372886}%
\pgfsetfillcolor{currentfill}%
\pgfsetlinewidth{0.000000pt}%
\definecolor{currentstroke}{rgb}{0.248629,0.278775,0.534556}%
\pgfsetstrokecolor{currentstroke}%
\pgfsetdash{}{0pt}%
\pgfpathmoveto{\pgfqpoint{3.259589in}{5.378978in}}%
\pgfpathlineto{\pgfqpoint{3.040850in}{5.259550in}}%
\pgfpathlineto{\pgfqpoint{3.173672in}{5.414731in}}%
\pgfpathclose%
\pgfusepath{fill}%
\end{pgfscope}%
\begin{pgfscope}%
\pgfpathrectangle{\pgfqpoint{0.539299in}{0.078740in}}{\pgfqpoint{7.842520in}{7.842520in}}%
\pgfusepath{clip}%
\pgfsetbuttcap%
\pgfsetroundjoin%
\definecolor{currentfill}{rgb}{0.137339,0.662252,0.515571}%
\pgfsetfillcolor{currentfill}%
\pgfsetlinewidth{0.000000pt}%
\definecolor{currentstroke}{rgb}{0.246811,0.283237,0.535941}%
\pgfsetstrokecolor{currentstroke}%
\pgfsetdash{}{0pt}%
\pgfpathmoveto{\pgfqpoint{2.652170in}{4.965747in}}%
\pgfpathlineto{\pgfqpoint{2.527103in}{4.606886in}}%
\pgfpathlineto{\pgfqpoint{2.440760in}{4.553021in}}%
\pgfpathclose%
\pgfusepath{fill}%
\end{pgfscope}%
\begin{pgfscope}%
\pgfpathrectangle{\pgfqpoint{0.539299in}{0.078740in}}{\pgfqpoint{7.842520in}{7.842520in}}%
\pgfusepath{clip}%
\pgfsetbuttcap%
\pgfsetroundjoin%
\definecolor{currentfill}{rgb}{0.281446,0.084320,0.407414}%
\pgfsetfillcolor{currentfill}%
\pgfsetlinewidth{0.000000pt}%
\definecolor{currentstroke}{rgb}{0.244972,0.287675,0.537260}%
\pgfsetstrokecolor{currentstroke}%
\pgfsetdash{}{0pt}%
\pgfpathmoveto{\pgfqpoint{6.690065in}{2.529737in}}%
\pgfpathlineto{\pgfqpoint{6.756390in}{2.344144in}}%
\pgfpathlineto{\pgfqpoint{6.830421in}{2.421820in}}%
\pgfpathclose%
\pgfusepath{fill}%
\end{pgfscope}%
\begin{pgfscope}%
\pgfpathrectangle{\pgfqpoint{0.539299in}{0.078740in}}{\pgfqpoint{7.842520in}{7.842520in}}%
\pgfusepath{clip}%
\pgfsetbuttcap%
\pgfsetroundjoin%
\definecolor{currentfill}{rgb}{0.283197,0.115680,0.436115}%
\pgfsetfillcolor{currentfill}%
\pgfsetlinewidth{0.000000pt}%
\definecolor{currentstroke}{rgb}{0.243113,0.292092,0.538516}%
\pgfsetstrokecolor{currentstroke}%
\pgfsetdash{}{0pt}%
\pgfpathmoveto{\pgfqpoint{6.475077in}{2.555660in}}%
\pgfpathlineto{\pgfqpoint{6.615697in}{2.452498in}}%
\pgfpathlineto{\pgfqpoint{6.690065in}{2.529737in}}%
\pgfpathclose%
\pgfusepath{fill}%
\end{pgfscope}%
\begin{pgfscope}%
\pgfpathrectangle{\pgfqpoint{0.539299in}{0.078740in}}{\pgfqpoint{7.842520in}{7.842520in}}%
\pgfusepath{clip}%
\pgfsetbuttcap%
\pgfsetroundjoin%
\definecolor{currentfill}{rgb}{0.175707,0.697900,0.491033}%
\pgfsetfillcolor{currentfill}%
\pgfsetlinewidth{0.000000pt}%
\definecolor{currentstroke}{rgb}{0.241237,0.296485,0.539709}%
\pgfsetstrokecolor{currentstroke}%
\pgfsetdash{}{0pt}%
\pgfpathmoveto{\pgfqpoint{2.738735in}{4.997476in}}%
\pgfpathlineto{\pgfqpoint{2.527103in}{4.606886in}}%
\pgfpathlineto{\pgfqpoint{2.652170in}{4.965747in}}%
\pgfpathclose%
\pgfusepath{fill}%
\end{pgfscope}%
\begin{pgfscope}%
\pgfpathrectangle{\pgfqpoint{0.539299in}{0.078740in}}{\pgfqpoint{7.842520in}{7.842520in}}%
\pgfusepath{clip}%
\pgfsetbuttcap%
\pgfsetroundjoin%
\definecolor{currentfill}{rgb}{0.197636,0.391528,0.554969}%
\pgfsetfillcolor{currentfill}%
\pgfsetlinewidth{0.000000pt}%
\definecolor{currentstroke}{rgb}{0.239346,0.300855,0.540844}%
\pgfsetstrokecolor{currentstroke}%
\pgfsetdash{}{0pt}%
\pgfpathmoveto{\pgfqpoint{1.775040in}{3.152116in}}%
\pgfpathlineto{\pgfqpoint{1.889628in}{3.636546in}}%
\pgfpathlineto{\pgfqpoint{1.975740in}{3.757494in}}%
\pgfpathclose%
\pgfusepath{fill}%
\end{pgfscope}%
\begin{pgfscope}%
\pgfpathrectangle{\pgfqpoint{0.539299in}{0.078740in}}{\pgfqpoint{7.842520in}{7.842520in}}%
\pgfusepath{clip}%
\pgfsetbuttcap%
\pgfsetroundjoin%
\definecolor{currentfill}{rgb}{0.197636,0.391528,0.554969}%
\pgfsetfillcolor{currentfill}%
\pgfsetlinewidth{0.000000pt}%
\definecolor{currentstroke}{rgb}{0.237441,0.305202,0.541921}%
\pgfsetstrokecolor{currentstroke}%
\pgfsetdash{}{0pt}%
\pgfpathmoveto{\pgfqpoint{5.143030in}{3.602315in}}%
\pgfpathlineto{\pgfqpoint{5.204442in}{3.480326in}}%
\pgfpathlineto{\pgfqpoint{5.281811in}{3.444596in}}%
\pgfpathclose%
\pgfusepath{fill}%
\end{pgfscope}%
\begin{pgfscope}%
\pgfpathrectangle{\pgfqpoint{0.539299in}{0.078740in}}{\pgfqpoint{7.842520in}{7.842520in}}%
\pgfusepath{clip}%
\pgfsetbuttcap%
\pgfsetroundjoin%
\definecolor{currentfill}{rgb}{0.280868,0.160771,0.472899}%
\pgfsetfillcolor{currentfill}%
\pgfsetlinewidth{0.000000pt}%
\definecolor{currentstroke}{rgb}{0.235526,0.309527,0.542944}%
\pgfsetstrokecolor{currentstroke}%
\pgfsetdash{}{0pt}%
\pgfpathmoveto{\pgfqpoint{6.259304in}{2.584902in}}%
\pgfpathlineto{\pgfqpoint{6.334603in}{2.655498in}}%
\pgfpathlineto{\pgfqpoint{6.194335in}{2.753858in}}%
\pgfpathclose%
\pgfusepath{fill}%
\end{pgfscope}%
\begin{pgfscope}%
\pgfpathrectangle{\pgfqpoint{0.539299in}{0.078740in}}{\pgfqpoint{7.842520in}{7.842520in}}%
\pgfusepath{clip}%
\pgfsetbuttcap%
\pgfsetroundjoin%
\definecolor{currentfill}{rgb}{0.208623,0.367752,0.552675}%
\pgfsetfillcolor{currentfill}%
\pgfsetlinewidth{0.000000pt}%
\definecolor{currentstroke}{rgb}{0.233603,0.313828,0.543914}%
\pgfsetstrokecolor{currentstroke}%
\pgfsetdash{}{0pt}%
\pgfpathmoveto{\pgfqpoint{5.204442in}{3.480326in}}%
\pgfpathlineto{\pgfqpoint{5.420728in}{3.297681in}}%
\pgfpathlineto{\pgfqpoint{5.281811in}{3.444596in}}%
\pgfpathclose%
\pgfusepath{fill}%
\end{pgfscope}%
\begin{pgfscope}%
\pgfpathrectangle{\pgfqpoint{0.539299in}{0.078740in}}{\pgfqpoint{7.842520in}{7.842520in}}%
\pgfusepath{clip}%
\pgfsetbuttcap%
\pgfsetroundjoin%
\definecolor{currentfill}{rgb}{0.174274,0.445044,0.557792}%
\pgfsetfillcolor{currentfill}%
\pgfsetlinewidth{0.000000pt}%
\definecolor{currentstroke}{rgb}{0.231674,0.318106,0.544834}%
\pgfsetstrokecolor{currentstroke}%
\pgfsetdash{}{0pt}%
\pgfpathmoveto{\pgfqpoint{4.925899in}{3.838472in}}%
\pgfpathlineto{\pgfqpoint{5.143030in}{3.602315in}}%
\pgfpathlineto{\pgfqpoint{5.004323in}{3.770773in}}%
\pgfpathclose%
\pgfusepath{fill}%
\end{pgfscope}%
\begin{pgfscope}%
\pgfpathrectangle{\pgfqpoint{0.539299in}{0.078740in}}{\pgfqpoint{7.842520in}{7.842520in}}%
\pgfusepath{clip}%
\pgfsetbuttcap%
\pgfsetroundjoin%
\definecolor{currentfill}{rgb}{0.125394,0.574318,0.549086}%
\pgfsetfillcolor{currentfill}%
\pgfsetlinewidth{0.000000pt}%
\definecolor{currentstroke}{rgb}{0.229739,0.322361,0.545706}%
\pgfsetstrokecolor{currentstroke}%
\pgfsetdash{}{0pt}%
\pgfpathmoveto{\pgfqpoint{2.267756in}{4.403873in}}%
\pgfpathlineto{\pgfqpoint{2.354275in}{4.486045in}}%
\pgfpathlineto{\pgfqpoint{2.234531in}{4.033237in}}%
\pgfpathclose%
\pgfusepath{fill}%
\end{pgfscope}%
\begin{pgfscope}%
\pgfpathrectangle{\pgfqpoint{0.539299in}{0.078740in}}{\pgfqpoint{7.842520in}{7.842520in}}%
\pgfusepath{clip}%
\pgfsetbuttcap%
\pgfsetroundjoin%
\definecolor{currentfill}{rgb}{0.276194,0.190074,0.493001}%
\pgfsetfillcolor{currentfill}%
\pgfsetlinewidth{0.000000pt}%
\definecolor{currentstroke}{rgb}{0.227802,0.326594,0.546532}%
\pgfsetstrokecolor{currentstroke}%
\pgfsetdash{}{0pt}%
\pgfpathmoveto{\pgfqpoint{6.118863in}{2.691325in}}%
\pgfpathlineto{\pgfqpoint{6.194335in}{2.753858in}}%
\pgfpathlineto{\pgfqpoint{5.978689in}{2.800383in}}%
\pgfpathclose%
\pgfusepath{fill}%
\end{pgfscope}%
\begin{pgfscope}%
\pgfpathrectangle{\pgfqpoint{0.539299in}{0.078740in}}{\pgfqpoint{7.842520in}{7.842520in}}%
\pgfusepath{clip}%
\pgfsetbuttcap%
\pgfsetroundjoin%
\definecolor{currentfill}{rgb}{0.277941,0.056324,0.381191}%
\pgfsetfillcolor{currentfill}%
\pgfsetlinewidth{0.000000pt}%
\definecolor{currentstroke}{rgb}{0.225863,0.330805,0.547314}%
\pgfsetstrokecolor{currentstroke}%
\pgfsetdash{}{0pt}%
\pgfpathmoveto{\pgfqpoint{6.897057in}{2.228556in}}%
\pgfpathlineto{\pgfqpoint{6.970702in}{2.305402in}}%
\pgfpathlineto{\pgfqpoint{6.830421in}{2.421820in}}%
\pgfpathclose%
\pgfusepath{fill}%
\end{pgfscope}%
\begin{pgfscope}%
\pgfpathrectangle{\pgfqpoint{0.539299in}{0.078740in}}{\pgfqpoint{7.842520in}{7.842520in}}%
\pgfusepath{clip}%
\pgfsetbuttcap%
\pgfsetroundjoin%
\definecolor{currentfill}{rgb}{0.231674,0.318106,0.544834}%
\pgfsetfillcolor{currentfill}%
\pgfsetlinewidth{0.000000pt}%
\definecolor{currentstroke}{rgb}{0.223925,0.334994,0.548053}%
\pgfsetstrokecolor{currentstroke}%
\pgfsetdash{}{0pt}%
\pgfpathmoveto{\pgfqpoint{5.483218in}{3.162772in}}%
\pgfpathlineto{\pgfqpoint{5.559838in}{3.161010in}}%
\pgfpathlineto{\pgfqpoint{5.420728in}{3.297681in}}%
\pgfpathclose%
\pgfusepath{fill}%
\end{pgfscope}%
\begin{pgfscope}%
\pgfpathrectangle{\pgfqpoint{0.539299in}{0.078740in}}{\pgfqpoint{7.842520in}{7.842520in}}%
\pgfusepath{clip}%
\pgfsetbuttcap%
\pgfsetroundjoin%
\definecolor{currentfill}{rgb}{0.282327,0.094955,0.417331}%
\pgfsetfillcolor{currentfill}%
\pgfsetlinewidth{0.000000pt}%
\definecolor{currentstroke}{rgb}{0.221989,0.339161,0.548752}%
\pgfsetstrokecolor{currentstroke}%
\pgfsetdash{}{0pt}%
\pgfpathmoveto{\pgfqpoint{6.615697in}{2.452498in}}%
\pgfpathlineto{\pgfqpoint{6.756390in}{2.344144in}}%
\pgfpathlineto{\pgfqpoint{6.690065in}{2.529737in}}%
\pgfpathclose%
\pgfusepath{fill}%
\end{pgfscope}%
\begin{pgfscope}%
\pgfpathrectangle{\pgfqpoint{0.539299in}{0.078740in}}{\pgfqpoint{7.842520in}{7.842520in}}%
\pgfusepath{clip}%
\pgfsetbuttcap%
\pgfsetroundjoin%
\definecolor{currentfill}{rgb}{0.162142,0.474838,0.558140}%
\pgfsetfillcolor{currentfill}%
\pgfsetlinewidth{0.000000pt}%
\definecolor{currentstroke}{rgb}{0.220057,0.343307,0.549413}%
\pgfsetstrokecolor{currentstroke}%
\pgfsetdash{}{0pt}%
\pgfpathmoveto{\pgfqpoint{4.925899in}{3.838472in}}%
\pgfpathlineto{\pgfqpoint{5.004323in}{3.770773in}}%
\pgfpathlineto{\pgfqpoint{4.865625in}{3.949141in}}%
\pgfpathclose%
\pgfusepath{fill}%
\end{pgfscope}%
\begin{pgfscope}%
\pgfpathrectangle{\pgfqpoint{0.539299in}{0.078740in}}{\pgfqpoint{7.842520in}{7.842520in}}%
\pgfusepath{clip}%
\pgfsetbuttcap%
\pgfsetroundjoin%
\definecolor{currentfill}{rgb}{0.430983,0.808473,0.346476}%
\pgfsetfillcolor{currentfill}%
\pgfsetlinewidth{0.000000pt}%
\definecolor{currentstroke}{rgb}{0.218130,0.347432,0.550038}%
\pgfsetstrokecolor{currentstroke}%
\pgfsetdash{}{0pt}%
\pgfpathmoveto{\pgfqpoint{3.394745in}{5.445102in}}%
\pgfpathlineto{\pgfqpoint{3.531606in}{5.447979in}}%
\pgfpathlineto{\pgfqpoint{3.480023in}{5.382736in}}%
\pgfpathclose%
\pgfusepath{fill}%
\end{pgfscope}%
\begin{pgfscope}%
\pgfpathrectangle{\pgfqpoint{0.539299in}{0.078740in}}{\pgfqpoint{7.842520in}{7.842520in}}%
\pgfusepath{clip}%
\pgfsetbuttcap%
\pgfsetroundjoin%
\definecolor{currentfill}{rgb}{0.237441,0.305202,0.541921}%
\pgfsetfillcolor{currentfill}%
\pgfsetlinewidth{0.000000pt}%
\definecolor{currentstroke}{rgb}{0.216210,0.351535,0.550627}%
\pgfsetstrokecolor{currentstroke}%
\pgfsetdash{}{0pt}%
\pgfpathmoveto{\pgfqpoint{1.803826in}{3.497087in}}%
\pgfpathlineto{\pgfqpoint{1.688478in}{3.045686in}}%
\pgfpathlineto{\pgfqpoint{1.601948in}{2.927986in}}%
\pgfpathclose%
\pgfusepath{fill}%
\end{pgfscope}%
\begin{pgfscope}%
\pgfpathrectangle{\pgfqpoint{0.539299in}{0.078740in}}{\pgfqpoint{7.842520in}{7.842520in}}%
\pgfusepath{clip}%
\pgfsetbuttcap%
\pgfsetroundjoin%
\definecolor{currentfill}{rgb}{0.246811,0.283237,0.535941}%
\pgfsetfillcolor{currentfill}%
\pgfsetlinewidth{0.000000pt}%
\definecolor{currentstroke}{rgb}{0.214298,0.355619,0.551184}%
\pgfsetstrokecolor{currentstroke}%
\pgfsetdash{}{0pt}%
\pgfpathmoveto{\pgfqpoint{5.622835in}{3.018878in}}%
\pgfpathlineto{\pgfqpoint{5.699184in}{3.033538in}}%
\pgfpathlineto{\pgfqpoint{5.559838in}{3.161010in}}%
\pgfpathclose%
\pgfusepath{fill}%
\end{pgfscope}%
\begin{pgfscope}%
\pgfpathrectangle{\pgfqpoint{0.539299in}{0.078740in}}{\pgfqpoint{7.842520in}{7.842520in}}%
\pgfusepath{clip}%
\pgfsetbuttcap%
\pgfsetroundjoin%
\definecolor{currentfill}{rgb}{0.412913,0.803041,0.357269}%
\pgfsetfillcolor{currentfill}%
\pgfsetlinewidth{0.000000pt}%
\definecolor{currentstroke}{rgb}{0.212395,0.359683,0.551710}%
\pgfsetstrokecolor{currentstroke}%
\pgfsetdash{}{0pt}%
\pgfpathmoveto{\pgfqpoint{3.616640in}{5.370253in}}%
\pgfpathlineto{\pgfqpoint{3.531606in}{5.447979in}}%
\pgfpathlineto{\pgfqpoint{3.754366in}{5.305678in}}%
\pgfpathclose%
\pgfusepath{fill}%
\end{pgfscope}%
\begin{pgfscope}%
\pgfpathrectangle{\pgfqpoint{0.539299in}{0.078740in}}{\pgfqpoint{7.842520in}{7.842520in}}%
\pgfusepath{clip}%
\pgfsetbuttcap%
\pgfsetroundjoin%
\definecolor{currentfill}{rgb}{0.144759,0.519093,0.556572}%
\pgfsetfillcolor{currentfill}%
\pgfsetlinewidth{0.000000pt}%
\definecolor{currentstroke}{rgb}{0.210503,0.363727,0.552206}%
\pgfsetstrokecolor{currentstroke}%
\pgfsetdash{}{0pt}%
\pgfpathmoveto{\pgfqpoint{4.726876in}{4.135706in}}%
\pgfpathlineto{\pgfqpoint{4.786562in}{4.030722in}}%
\pgfpathlineto{\pgfqpoint{4.865625in}{3.949141in}}%
\pgfpathclose%
\pgfusepath{fill}%
\end{pgfscope}%
\begin{pgfscope}%
\pgfpathrectangle{\pgfqpoint{0.539299in}{0.078740in}}{\pgfqpoint{7.842520in}{7.842520in}}%
\pgfusepath{clip}%
\pgfsetbuttcap%
\pgfsetroundjoin%
\definecolor{currentfill}{rgb}{0.266941,0.748751,0.440573}%
\pgfsetfillcolor{currentfill}%
\pgfsetlinewidth{0.000000pt}%
\definecolor{currentstroke}{rgb}{0.208623,0.367752,0.552675}%
\pgfsetstrokecolor{currentstroke}%
\pgfsetdash{}{0pt}%
\pgfpathmoveto{\pgfqpoint{2.738735in}{4.997476in}}%
\pgfpathlineto{\pgfqpoint{2.867996in}{5.262317in}}%
\pgfpathlineto{\pgfqpoint{2.825015in}{5.016747in}}%
\pgfpathclose%
\pgfusepath{fill}%
\end{pgfscope}%
\begin{pgfscope}%
\pgfpathrectangle{\pgfqpoint{0.539299in}{0.078740in}}{\pgfqpoint{7.842520in}{7.842520in}}%
\pgfusepath{clip}%
\pgfsetbuttcap%
\pgfsetroundjoin%
\definecolor{currentfill}{rgb}{0.279574,0.170599,0.479997}%
\pgfsetfillcolor{currentfill}%
\pgfsetlinewidth{0.000000pt}%
\definecolor{currentstroke}{rgb}{0.206756,0.371758,0.553117}%
\pgfsetstrokecolor{currentstroke}%
\pgfsetdash{}{0pt}%
\pgfpathmoveto{\pgfqpoint{6.194335in}{2.753858in}}%
\pgfpathlineto{\pgfqpoint{6.118863in}{2.691325in}}%
\pgfpathlineto{\pgfqpoint{6.259304in}{2.584902in}}%
\pgfpathclose%
\pgfusepath{fill}%
\end{pgfscope}%
\begin{pgfscope}%
\pgfpathrectangle{\pgfqpoint{0.539299in}{0.078740in}}{\pgfqpoint{7.842520in}{7.842520in}}%
\pgfusepath{clip}%
\pgfsetbuttcap%
\pgfsetroundjoin%
\definecolor{currentfill}{rgb}{0.377779,0.791781,0.377939}%
\pgfsetfillcolor{currentfill}%
\pgfsetlinewidth{0.000000pt}%
\definecolor{currentstroke}{rgb}{0.204903,0.375746,0.553533}%
\pgfsetstrokecolor{currentstroke}%
\pgfsetdash{}{0pt}%
\pgfpathmoveto{\pgfqpoint{3.173672in}{5.414731in}}%
\pgfpathlineto{\pgfqpoint{3.040850in}{5.259550in}}%
\pgfpathlineto{\pgfqpoint{2.954612in}{5.267216in}}%
\pgfpathclose%
\pgfusepath{fill}%
\end{pgfscope}%
\begin{pgfscope}%
\pgfpathrectangle{\pgfqpoint{0.539299in}{0.078740in}}{\pgfqpoint{7.842520in}{7.842520in}}%
\pgfusepath{clip}%
\pgfsetbuttcap%
\pgfsetroundjoin%
\definecolor{currentfill}{rgb}{0.282884,0.135920,0.453427}%
\pgfsetfillcolor{currentfill}%
\pgfsetlinewidth{0.000000pt}%
\definecolor{currentstroke}{rgb}{0.203063,0.379716,0.553925}%
\pgfsetstrokecolor{currentstroke}%
\pgfsetdash{}{0pt}%
\pgfpathmoveto{\pgfqpoint{6.475077in}{2.555660in}}%
\pgfpathlineto{\pgfqpoint{6.334603in}{2.655498in}}%
\pgfpathlineto{\pgfqpoint{6.399983in}{2.479315in}}%
\pgfpathclose%
\pgfusepath{fill}%
\end{pgfscope}%
\begin{pgfscope}%
\pgfpathrectangle{\pgfqpoint{0.539299in}{0.078740in}}{\pgfqpoint{7.842520in}{7.842520in}}%
\pgfusepath{clip}%
\pgfsetbuttcap%
\pgfsetroundjoin%
\definecolor{currentfill}{rgb}{0.144759,0.519093,0.556572}%
\pgfsetfillcolor{currentfill}%
\pgfsetlinewidth{0.000000pt}%
\definecolor{currentstroke}{rgb}{0.201239,0.383670,0.554294}%
\pgfsetstrokecolor{currentstroke}%
\pgfsetdash{}{0pt}%
\pgfpathmoveto{\pgfqpoint{2.181328in}{4.304148in}}%
\pgfpathlineto{\pgfqpoint{2.148306in}{3.953837in}}%
\pgfpathlineto{\pgfqpoint{2.062008in}{3.862537in}}%
\pgfpathclose%
\pgfusepath{fill}%
\end{pgfscope}%
\begin{pgfscope}%
\pgfpathrectangle{\pgfqpoint{0.539299in}{0.078740in}}{\pgfqpoint{7.842520in}{7.842520in}}%
\pgfusepath{clip}%
\pgfsetbuttcap%
\pgfsetroundjoin%
\definecolor{currentfill}{rgb}{0.253935,0.265254,0.529983}%
\pgfsetfillcolor{currentfill}%
\pgfsetlinewidth{0.000000pt}%
\definecolor{currentstroke}{rgb}{0.199430,0.387607,0.554642}%
\pgfsetstrokecolor{currentstroke}%
\pgfsetdash{}{0pt}%
\pgfpathmoveto{\pgfqpoint{5.838796in}{2.913871in}}%
\pgfpathlineto{\pgfqpoint{5.699184in}{3.033538in}}%
\pgfpathlineto{\pgfqpoint{5.622835in}{3.018878in}}%
\pgfpathclose%
\pgfusepath{fill}%
\end{pgfscope}%
\begin{pgfscope}%
\pgfpathrectangle{\pgfqpoint{0.539299in}{0.078740in}}{\pgfqpoint{7.842520in}{7.842520in}}%
\pgfusepath{clip}%
\pgfsetbuttcap%
\pgfsetroundjoin%
\definecolor{currentfill}{rgb}{0.278791,0.062145,0.386592}%
\pgfsetfillcolor{currentfill}%
\pgfsetlinewidth{0.000000pt}%
\definecolor{currentstroke}{rgb}{0.197636,0.391528,0.554969}%
\pgfsetstrokecolor{currentstroke}%
\pgfsetdash{}{0pt}%
\pgfpathmoveto{\pgfqpoint{6.830421in}{2.421820in}}%
\pgfpathlineto{\pgfqpoint{6.756390in}{2.344144in}}%
\pgfpathlineto{\pgfqpoint{6.897057in}{2.228556in}}%
\pgfpathclose%
\pgfusepath{fill}%
\end{pgfscope}%
\begin{pgfscope}%
\pgfpathrectangle{\pgfqpoint{0.539299in}{0.078740in}}{\pgfqpoint{7.842520in}{7.842520in}}%
\pgfusepath{clip}%
\pgfsetbuttcap%
\pgfsetroundjoin%
\definecolor{currentfill}{rgb}{0.203063,0.379716,0.553925}%
\pgfsetfillcolor{currentfill}%
\pgfsetlinewidth{0.000000pt}%
\definecolor{currentstroke}{rgb}{0.195860,0.395433,0.555276}%
\pgfsetstrokecolor{currentstroke}%
\pgfsetdash{}{0pt}%
\pgfpathmoveto{\pgfqpoint{1.803826in}{3.497087in}}%
\pgfpathlineto{\pgfqpoint{1.889628in}{3.636546in}}%
\pgfpathlineto{\pgfqpoint{1.775040in}{3.152116in}}%
\pgfpathclose%
\pgfusepath{fill}%
\end{pgfscope}%
\begin{pgfscope}%
\pgfpathrectangle{\pgfqpoint{0.539299in}{0.078740in}}{\pgfqpoint{7.842520in}{7.842520in}}%
\pgfusepath{clip}%
\pgfsetbuttcap%
\pgfsetroundjoin%
\definecolor{currentfill}{rgb}{0.123463,0.581687,0.547445}%
\pgfsetfillcolor{currentfill}%
\pgfsetlinewidth{0.000000pt}%
\definecolor{currentstroke}{rgb}{0.194100,0.399323,0.555565}%
\pgfsetstrokecolor{currentstroke}%
\pgfsetdash{}{0pt}%
\pgfpathmoveto{\pgfqpoint{4.726876in}{4.135706in}}%
\pgfpathlineto{\pgfqpoint{4.588029in}{4.327753in}}%
\pgfpathlineto{\pgfqpoint{4.507522in}{4.430367in}}%
\pgfpathclose%
\pgfusepath{fill}%
\end{pgfscope}%
\begin{pgfscope}%
\pgfpathrectangle{\pgfqpoint{0.539299in}{0.078740in}}{\pgfqpoint{7.842520in}{7.842520in}}%
\pgfusepath{clip}%
\pgfsetbuttcap%
\pgfsetroundjoin%
\definecolor{currentfill}{rgb}{0.319809,0.770914,0.411152}%
\pgfsetfillcolor{currentfill}%
\pgfsetlinewidth{0.000000pt}%
\definecolor{currentstroke}{rgb}{0.192357,0.403199,0.555836}%
\pgfsetstrokecolor{currentstroke}%
\pgfsetdash{}{0pt}%
\pgfpathmoveto{\pgfqpoint{2.867996in}{5.262317in}}%
\pgfpathlineto{\pgfqpoint{2.954612in}{5.267216in}}%
\pgfpathlineto{\pgfqpoint{2.825015in}{5.016747in}}%
\pgfpathclose%
\pgfusepath{fill}%
\end{pgfscope}%
\begin{pgfscope}%
\pgfpathrectangle{\pgfqpoint{0.539299in}{0.078740in}}{\pgfqpoint{7.842520in}{7.842520in}}%
\pgfusepath{clip}%
\pgfsetbuttcap%
\pgfsetroundjoin%
\definecolor{currentfill}{rgb}{0.283229,0.120777,0.440584}%
\pgfsetfillcolor{currentfill}%
\pgfsetlinewidth{0.000000pt}%
\definecolor{currentstroke}{rgb}{0.190631,0.407061,0.556089}%
\pgfsetstrokecolor{currentstroke}%
\pgfsetdash{}{0pt}%
\pgfpathmoveto{\pgfqpoint{6.399983in}{2.479315in}}%
\pgfpathlineto{\pgfqpoint{6.615697in}{2.452498in}}%
\pgfpathlineto{\pgfqpoint{6.475077in}{2.555660in}}%
\pgfpathclose%
\pgfusepath{fill}%
\end{pgfscope}%
\begin{pgfscope}%
\pgfpathrectangle{\pgfqpoint{0.539299in}{0.078740in}}{\pgfqpoint{7.842520in}{7.842520in}}%
\pgfusepath{clip}%
\pgfsetbuttcap%
\pgfsetroundjoin%
\definecolor{currentfill}{rgb}{0.265145,0.232956,0.516599}%
\pgfsetfillcolor{currentfill}%
\pgfsetlinewidth{0.000000pt}%
\definecolor{currentstroke}{rgb}{0.188923,0.410910,0.556326}%
\pgfsetstrokecolor{currentstroke}%
\pgfsetdash{}{0pt}%
\pgfpathmoveto{\pgfqpoint{5.838796in}{2.913871in}}%
\pgfpathlineto{\pgfqpoint{5.762666in}{2.883788in}}%
\pgfpathlineto{\pgfqpoint{5.978689in}{2.800383in}}%
\pgfpathclose%
\pgfusepath{fill}%
\end{pgfscope}%
\begin{pgfscope}%
\pgfpathrectangle{\pgfqpoint{0.539299in}{0.078740in}}{\pgfqpoint{7.842520in}{7.842520in}}%
\pgfusepath{clip}%
\pgfsetbuttcap%
\pgfsetroundjoin%
\definecolor{currentfill}{rgb}{0.208623,0.367752,0.552675}%
\pgfsetfillcolor{currentfill}%
\pgfsetlinewidth{0.000000pt}%
\definecolor{currentstroke}{rgb}{0.187231,0.414746,0.556547}%
\pgfsetstrokecolor{currentstroke}%
\pgfsetdash{}{0pt}%
\pgfpathmoveto{\pgfqpoint{5.343771in}{3.316408in}}%
\pgfpathlineto{\pgfqpoint{5.420728in}{3.297681in}}%
\pgfpathlineto{\pgfqpoint{5.204442in}{3.480326in}}%
\pgfpathclose%
\pgfusepath{fill}%
\end{pgfscope}%
\begin{pgfscope}%
\pgfpathrectangle{\pgfqpoint{0.539299in}{0.078740in}}{\pgfqpoint{7.842520in}{7.842520in}}%
\pgfusepath{clip}%
\pgfsetbuttcap%
\pgfsetroundjoin%
\definecolor{currentfill}{rgb}{0.185556,0.418570,0.556753}%
\pgfsetfillcolor{currentfill}%
\pgfsetlinewidth{0.000000pt}%
\definecolor{currentstroke}{rgb}{0.185556,0.418570,0.556753}%
\pgfsetstrokecolor{currentstroke}%
\pgfsetdash{}{0pt}%
\pgfpathmoveto{\pgfqpoint{5.143030in}{3.602315in}}%
\pgfpathlineto{\pgfqpoint{5.065172in}{3.654551in}}%
\pgfpathlineto{\pgfqpoint{5.204442in}{3.480326in}}%
\pgfpathclose%
\pgfusepath{fill}%
\end{pgfscope}%
\begin{pgfscope}%
\pgfpathrectangle{\pgfqpoint{0.539299in}{0.078740in}}{\pgfqpoint{7.842520in}{7.842520in}}%
\pgfusepath{clip}%
\pgfsetbuttcap%
\pgfsetroundjoin%
\definecolor{currentfill}{rgb}{0.221989,0.339161,0.548752}%
\pgfsetfillcolor{currentfill}%
\pgfsetlinewidth{0.000000pt}%
\definecolor{currentstroke}{rgb}{0.183898,0.422383,0.556944}%
\pgfsetstrokecolor{currentstroke}%
\pgfsetdash{}{0pt}%
\pgfpathmoveto{\pgfqpoint{5.420728in}{3.297681in}}%
\pgfpathlineto{\pgfqpoint{5.343771in}{3.316408in}}%
\pgfpathlineto{\pgfqpoint{5.483218in}{3.162772in}}%
\pgfpathclose%
\pgfusepath{fill}%
\end{pgfscope}%
\begin{pgfscope}%
\pgfpathrectangle{\pgfqpoint{0.539299in}{0.078740in}}{\pgfqpoint{7.842520in}{7.842520in}}%
\pgfusepath{clip}%
\pgfsetbuttcap%
\pgfsetroundjoin%
\definecolor{currentfill}{rgb}{0.282290,0.145912,0.461510}%
\pgfsetfillcolor{currentfill}%
\pgfsetlinewidth{0.000000pt}%
\definecolor{currentstroke}{rgb}{0.182256,0.426184,0.557120}%
\pgfsetstrokecolor{currentstroke}%
\pgfsetdash{}{0pt}%
\pgfpathmoveto{\pgfqpoint{6.399983in}{2.479315in}}%
\pgfpathlineto{\pgfqpoint{6.334603in}{2.655498in}}%
\pgfpathlineto{\pgfqpoint{6.259304in}{2.584902in}}%
\pgfpathclose%
\pgfusepath{fill}%
\end{pgfscope}%
\begin{pgfscope}%
\pgfpathrectangle{\pgfqpoint{0.539299in}{0.078740in}}{\pgfqpoint{7.842520in}{7.842520in}}%
\pgfusepath{clip}%
\pgfsetbuttcap%
\pgfsetroundjoin%
\definecolor{currentfill}{rgb}{0.458674,0.816363,0.329727}%
\pgfsetfillcolor{currentfill}%
\pgfsetlinewidth{0.000000pt}%
\definecolor{currentstroke}{rgb}{0.180629,0.429975,0.557282}%
\pgfsetstrokecolor{currentstroke}%
\pgfsetdash{}{0pt}%
\pgfpathmoveto{\pgfqpoint{3.308861in}{5.497668in}}%
\pgfpathlineto{\pgfqpoint{3.394745in}{5.445102in}}%
\pgfpathlineto{\pgfqpoint{3.259589in}{5.378978in}}%
\pgfpathclose%
\pgfusepath{fill}%
\end{pgfscope}%
\begin{pgfscope}%
\pgfpathrectangle{\pgfqpoint{0.539299in}{0.078740in}}{\pgfqpoint{7.842520in}{7.842520in}}%
\pgfusepath{clip}%
\pgfsetbuttcap%
\pgfsetroundjoin%
\definecolor{currentfill}{rgb}{0.174274,0.445044,0.557792}%
\pgfsetfillcolor{currentfill}%
\pgfsetlinewidth{0.000000pt}%
\definecolor{currentstroke}{rgb}{0.179019,0.433756,0.557430}%
\pgfsetstrokecolor{currentstroke}%
\pgfsetdash{}{0pt}%
\pgfpathmoveto{\pgfqpoint{5.065172in}{3.654551in}}%
\pgfpathlineto{\pgfqpoint{5.143030in}{3.602315in}}%
\pgfpathlineto{\pgfqpoint{4.925899in}{3.838472in}}%
\pgfpathclose%
\pgfusepath{fill}%
\end{pgfscope}%
\begin{pgfscope}%
\pgfpathrectangle{\pgfqpoint{0.539299in}{0.078740in}}{\pgfqpoint{7.842520in}{7.842520in}}%
\pgfusepath{clip}%
\pgfsetbuttcap%
\pgfsetroundjoin%
\definecolor{currentfill}{rgb}{0.237441,0.305202,0.541921}%
\pgfsetfillcolor{currentfill}%
\pgfsetlinewidth{0.000000pt}%
\definecolor{currentstroke}{rgb}{0.177423,0.437527,0.557565}%
\pgfsetstrokecolor{currentstroke}%
\pgfsetdash{}{0pt}%
\pgfpathmoveto{\pgfqpoint{5.622835in}{3.018878in}}%
\pgfpathlineto{\pgfqpoint{5.559838in}{3.161010in}}%
\pgfpathlineto{\pgfqpoint{5.483218in}{3.162772in}}%
\pgfpathclose%
\pgfusepath{fill}%
\end{pgfscope}%
\begin{pgfscope}%
\pgfpathrectangle{\pgfqpoint{0.539299in}{0.078740in}}{\pgfqpoint{7.842520in}{7.842520in}}%
\pgfusepath{clip}%
\pgfsetbuttcap%
\pgfsetroundjoin%
\definecolor{currentfill}{rgb}{0.120638,0.625828,0.533488}%
\pgfsetfillcolor{currentfill}%
\pgfsetlinewidth{0.000000pt}%
\definecolor{currentstroke}{rgb}{0.175841,0.441290,0.557685}%
\pgfsetstrokecolor{currentstroke}%
\pgfsetdash{}{0pt}%
\pgfpathmoveto{\pgfqpoint{4.449060in}{4.521490in}}%
\pgfpathlineto{\pgfqpoint{4.367781in}{4.630418in}}%
\pgfpathlineto{\pgfqpoint{4.588029in}{4.327753in}}%
\pgfpathclose%
\pgfusepath{fill}%
\end{pgfscope}%
\begin{pgfscope}%
\pgfpathrectangle{\pgfqpoint{0.539299in}{0.078740in}}{\pgfqpoint{7.842520in}{7.842520in}}%
\pgfusepath{clip}%
\pgfsetbuttcap%
\pgfsetroundjoin%
\definecolor{currentfill}{rgb}{0.132268,0.655014,0.519661}%
\pgfsetfillcolor{currentfill}%
\pgfsetlinewidth{0.000000pt}%
\definecolor{currentstroke}{rgb}{0.174274,0.445044,0.557792}%
\pgfsetstrokecolor{currentstroke}%
\pgfsetdash{}{0pt}%
\pgfpathmoveto{\pgfqpoint{4.309978in}{4.711999in}}%
\pgfpathlineto{\pgfqpoint{4.367781in}{4.630418in}}%
\pgfpathlineto{\pgfqpoint{4.449060in}{4.521490in}}%
\pgfpathclose%
\pgfusepath{fill}%
\end{pgfscope}%
\begin{pgfscope}%
\pgfpathrectangle{\pgfqpoint{0.539299in}{0.078740in}}{\pgfqpoint{7.842520in}{7.842520in}}%
\pgfusepath{clip}%
\pgfsetbuttcap%
\pgfsetroundjoin%
\definecolor{currentfill}{rgb}{0.150476,0.504369,0.557430}%
\pgfsetfillcolor{currentfill}%
\pgfsetlinewidth{0.000000pt}%
\definecolor{currentstroke}{rgb}{0.172719,0.448791,0.557885}%
\pgfsetstrokecolor{currentstroke}%
\pgfsetdash{}{0pt}%
\pgfpathmoveto{\pgfqpoint{4.865625in}{3.949141in}}%
\pgfpathlineto{\pgfqpoint{4.786562in}{4.030722in}}%
\pgfpathlineto{\pgfqpoint{4.925899in}{3.838472in}}%
\pgfpathclose%
\pgfusepath{fill}%
\end{pgfscope}%
\begin{pgfscope}%
\pgfpathrectangle{\pgfqpoint{0.539299in}{0.078740in}}{\pgfqpoint{7.842520in}{7.842520in}}%
\pgfusepath{clip}%
\pgfsetbuttcap%
\pgfsetroundjoin%
\definecolor{currentfill}{rgb}{0.458674,0.816363,0.329727}%
\pgfsetfillcolor{currentfill}%
\pgfsetlinewidth{0.000000pt}%
\definecolor{currentstroke}{rgb}{0.171176,0.452530,0.557965}%
\pgfsetstrokecolor{currentstroke}%
\pgfsetdash{}{0pt}%
\pgfpathmoveto{\pgfqpoint{3.173672in}{5.414731in}}%
\pgfpathlineto{\pgfqpoint{3.308861in}{5.497668in}}%
\pgfpathlineto{\pgfqpoint{3.259589in}{5.378978in}}%
\pgfpathclose%
\pgfusepath{fill}%
\end{pgfscope}%
\begin{pgfscope}%
\pgfpathrectangle{\pgfqpoint{0.539299in}{0.078740in}}{\pgfqpoint{7.842520in}{7.842520in}}%
\pgfusepath{clip}%
\pgfsetbuttcap%
\pgfsetroundjoin%
\definecolor{currentfill}{rgb}{0.369214,0.788888,0.382914}%
\pgfsetfillcolor{currentfill}%
\pgfsetlinewidth{0.000000pt}%
\definecolor{currentstroke}{rgb}{0.169646,0.456262,0.558030}%
\pgfsetstrokecolor{currentstroke}%
\pgfsetdash{}{0pt}%
\pgfpathmoveto{\pgfqpoint{3.808702in}{5.299454in}}%
\pgfpathlineto{\pgfqpoint{3.892829in}{5.198517in}}%
\pgfpathlineto{\pgfqpoint{3.754366in}{5.305678in}}%
\pgfpathclose%
\pgfusepath{fill}%
\end{pgfscope}%
\begin{pgfscope}%
\pgfpathrectangle{\pgfqpoint{0.539299in}{0.078740in}}{\pgfqpoint{7.842520in}{7.842520in}}%
\pgfusepath{clip}%
\pgfsetbuttcap%
\pgfsetroundjoin%
\definecolor{currentfill}{rgb}{0.282327,0.094955,0.417331}%
\pgfsetfillcolor{currentfill}%
\pgfsetlinewidth{0.000000pt}%
\definecolor{currentstroke}{rgb}{0.168126,0.459988,0.558082}%
\pgfsetstrokecolor{currentstroke}%
\pgfsetdash{}{0pt}%
\pgfpathmoveto{\pgfqpoint{6.756390in}{2.344144in}}%
\pgfpathlineto{\pgfqpoint{6.615697in}{2.452498in}}%
\pgfpathlineto{\pgfqpoint{6.540856in}{2.372768in}}%
\pgfpathclose%
\pgfusepath{fill}%
\end{pgfscope}%
\begin{pgfscope}%
\pgfpathrectangle{\pgfqpoint{0.539299in}{0.078740in}}{\pgfqpoint{7.842520in}{7.842520in}}%
\pgfusepath{clip}%
\pgfsetbuttcap%
\pgfsetroundjoin%
\definecolor{currentfill}{rgb}{0.275191,0.194905,0.496005}%
\pgfsetfillcolor{currentfill}%
\pgfsetlinewidth{0.000000pt}%
\definecolor{currentstroke}{rgb}{0.166617,0.463708,0.558119}%
\pgfsetstrokecolor{currentstroke}%
\pgfsetdash{}{0pt}%
\pgfpathmoveto{\pgfqpoint{6.043073in}{2.634987in}}%
\pgfpathlineto{\pgfqpoint{6.118863in}{2.691325in}}%
\pgfpathlineto{\pgfqpoint{5.978689in}{2.800383in}}%
\pgfpathclose%
\pgfusepath{fill}%
\end{pgfscope}%
\begin{pgfscope}%
\pgfpathrectangle{\pgfqpoint{0.539299in}{0.078740in}}{\pgfqpoint{7.842520in}{7.842520in}}%
\pgfusepath{clip}%
\pgfsetbuttcap%
\pgfsetroundjoin%
\definecolor{currentfill}{rgb}{0.255645,0.260703,0.528312}%
\pgfsetfillcolor{currentfill}%
\pgfsetlinewidth{0.000000pt}%
\definecolor{currentstroke}{rgb}{0.165117,0.467423,0.558141}%
\pgfsetstrokecolor{currentstroke}%
\pgfsetdash{}{0pt}%
\pgfpathmoveto{\pgfqpoint{5.622835in}{3.018878in}}%
\pgfpathlineto{\pgfqpoint{5.762666in}{2.883788in}}%
\pgfpathlineto{\pgfqpoint{5.838796in}{2.913871in}}%
\pgfpathclose%
\pgfusepath{fill}%
\end{pgfscope}%
\begin{pgfscope}%
\pgfpathrectangle{\pgfqpoint{0.539299in}{0.078740in}}{\pgfqpoint{7.842520in}{7.842520in}}%
\pgfusepath{clip}%
\pgfsetbuttcap%
\pgfsetroundjoin%
\definecolor{currentfill}{rgb}{0.133743,0.548535,0.553541}%
\pgfsetfillcolor{currentfill}%
\pgfsetlinewidth{0.000000pt}%
\definecolor{currentstroke}{rgb}{0.163625,0.471133,0.558148}%
\pgfsetstrokecolor{currentstroke}%
\pgfsetdash{}{0pt}%
\pgfpathmoveto{\pgfqpoint{4.647114in}{4.229077in}}%
\pgfpathlineto{\pgfqpoint{4.786562in}{4.030722in}}%
\pgfpathlineto{\pgfqpoint{4.726876in}{4.135706in}}%
\pgfpathclose%
\pgfusepath{fill}%
\end{pgfscope}%
\begin{pgfscope}%
\pgfpathrectangle{\pgfqpoint{0.539299in}{0.078740in}}{\pgfqpoint{7.842520in}{7.842520in}}%
\pgfusepath{clip}%
\pgfsetbuttcap%
\pgfsetroundjoin%
\definecolor{currentfill}{rgb}{0.196571,0.711827,0.479221}%
\pgfsetfillcolor{currentfill}%
\pgfsetlinewidth{0.000000pt}%
\definecolor{currentstroke}{rgb}{0.162142,0.474838,0.558140}%
\pgfsetstrokecolor{currentstroke}%
\pgfsetdash{}{0pt}%
\pgfpathmoveto{\pgfqpoint{4.309978in}{4.711999in}}%
\pgfpathlineto{\pgfqpoint{4.170832in}{4.893238in}}%
\pgfpathlineto{\pgfqpoint{4.088028in}{5.004974in}}%
\pgfpathclose%
\pgfusepath{fill}%
\end{pgfscope}%
\begin{pgfscope}%
\pgfpathrectangle{\pgfqpoint{0.539299in}{0.078740in}}{\pgfqpoint{7.842520in}{7.842520in}}%
\pgfusepath{clip}%
\pgfsetbuttcap%
\pgfsetroundjoin%
\definecolor{currentfill}{rgb}{0.311925,0.767822,0.415586}%
\pgfsetfillcolor{currentfill}%
\pgfsetlinewidth{0.000000pt}%
\definecolor{currentstroke}{rgb}{0.160665,0.478540,0.558115}%
\pgfsetstrokecolor{currentstroke}%
\pgfsetdash{}{0pt}%
\pgfpathmoveto{\pgfqpoint{4.031728in}{5.058097in}}%
\pgfpathlineto{\pgfqpoint{3.892829in}{5.198517in}}%
\pgfpathlineto{\pgfqpoint{3.948223in}{5.166109in}}%
\pgfpathclose%
\pgfusepath{fill}%
\end{pgfscope}%
\begin{pgfscope}%
\pgfpathrectangle{\pgfqpoint{0.539299in}{0.078740in}}{\pgfqpoint{7.842520in}{7.842520in}}%
\pgfusepath{clip}%
\pgfsetbuttcap%
\pgfsetroundjoin%
\definecolor{currentfill}{rgb}{0.146180,0.515413,0.556823}%
\pgfsetfillcolor{currentfill}%
\pgfsetlinewidth{0.000000pt}%
\definecolor{currentstroke}{rgb}{0.159194,0.482237,0.558073}%
\pgfsetstrokecolor{currentstroke}%
\pgfsetdash{}{0pt}%
\pgfpathmoveto{\pgfqpoint{2.181328in}{4.304148in}}%
\pgfpathlineto{\pgfqpoint{2.062008in}{3.862537in}}%
\pgfpathlineto{\pgfqpoint{1.975740in}{3.757494in}}%
\pgfpathclose%
\pgfusepath{fill}%
\end{pgfscope}%
\begin{pgfscope}%
\pgfpathrectangle{\pgfqpoint{0.539299in}{0.078740in}}{\pgfqpoint{7.842520in}{7.842520in}}%
\pgfusepath{clip}%
\pgfsetbuttcap%
\pgfsetroundjoin%
\definecolor{currentfill}{rgb}{0.283091,0.110553,0.431554}%
\pgfsetfillcolor{currentfill}%
\pgfsetlinewidth{0.000000pt}%
\definecolor{currentstroke}{rgb}{0.157729,0.485932,0.558013}%
\pgfsetstrokecolor{currentstroke}%
\pgfsetdash{}{0pt}%
\pgfpathmoveto{\pgfqpoint{6.540856in}{2.372768in}}%
\pgfpathlineto{\pgfqpoint{6.615697in}{2.452498in}}%
\pgfpathlineto{\pgfqpoint{6.399983in}{2.479315in}}%
\pgfpathclose%
\pgfusepath{fill}%
\end{pgfscope}%
\begin{pgfscope}%
\pgfpathrectangle{\pgfqpoint{0.539299in}{0.078740in}}{\pgfqpoint{7.842520in}{7.842520in}}%
\pgfusepath{clip}%
\pgfsetbuttcap%
\pgfsetroundjoin%
\definecolor{currentfill}{rgb}{0.266941,0.748751,0.440573}%
\pgfsetfillcolor{currentfill}%
\pgfsetlinewidth{0.000000pt}%
\definecolor{currentstroke}{rgb}{0.156270,0.489624,0.557936}%
\pgfsetstrokecolor{currentstroke}%
\pgfsetdash{}{0pt}%
\pgfpathmoveto{\pgfqpoint{4.031728in}{5.058097in}}%
\pgfpathlineto{\pgfqpoint{3.948223in}{5.166109in}}%
\pgfpathlineto{\pgfqpoint{4.170832in}{4.893238in}}%
\pgfpathclose%
\pgfusepath{fill}%
\end{pgfscope}%
\begin{pgfscope}%
\pgfpathrectangle{\pgfqpoint{0.539299in}{0.078740in}}{\pgfqpoint{7.842520in}{7.842520in}}%
\pgfusepath{clip}%
\pgfsetbuttcap%
\pgfsetroundjoin%
\definecolor{currentfill}{rgb}{0.440137,0.811138,0.340967}%
\pgfsetfillcolor{currentfill}%
\pgfsetlinewidth{0.000000pt}%
\definecolor{currentstroke}{rgb}{0.154815,0.493313,0.557840}%
\pgfsetstrokecolor{currentstroke}%
\pgfsetdash{}{0pt}%
\pgfpathmoveto{\pgfqpoint{3.754366in}{5.305678in}}%
\pgfpathlineto{\pgfqpoint{3.531606in}{5.447979in}}%
\pgfpathlineto{\pgfqpoint{3.669721in}{5.396407in}}%
\pgfpathclose%
\pgfusepath{fill}%
\end{pgfscope}%
\begin{pgfscope}%
\pgfpathrectangle{\pgfqpoint{0.539299in}{0.078740in}}{\pgfqpoint{7.842520in}{7.842520in}}%
\pgfusepath{clip}%
\pgfsetbuttcap%
\pgfsetroundjoin%
\definecolor{currentfill}{rgb}{0.266580,0.228262,0.514349}%
\pgfsetfillcolor{currentfill}%
\pgfsetlinewidth{0.000000pt}%
\definecolor{currentstroke}{rgb}{0.153364,0.497000,0.557724}%
\pgfsetstrokecolor{currentstroke}%
\pgfsetdash{}{0pt}%
\pgfpathmoveto{\pgfqpoint{5.762666in}{2.883788in}}%
\pgfpathlineto{\pgfqpoint{5.902739in}{2.756289in}}%
\pgfpathlineto{\pgfqpoint{5.978689in}{2.800383in}}%
\pgfpathclose%
\pgfusepath{fill}%
\end{pgfscope}%
\begin{pgfscope}%
\pgfpathrectangle{\pgfqpoint{0.539299in}{0.078740in}}{\pgfqpoint{7.842520in}{7.842520in}}%
\pgfusepath{clip}%
\pgfsetbuttcap%
\pgfsetroundjoin%
\definecolor{currentfill}{rgb}{0.123463,0.581687,0.547445}%
\pgfsetfillcolor{currentfill}%
\pgfsetlinewidth{0.000000pt}%
\definecolor{currentstroke}{rgb}{0.151918,0.500685,0.557587}%
\pgfsetstrokecolor{currentstroke}%
\pgfsetdash{}{0pt}%
\pgfpathmoveto{\pgfqpoint{4.726876in}{4.135706in}}%
\pgfpathlineto{\pgfqpoint{4.507522in}{4.430367in}}%
\pgfpathlineto{\pgfqpoint{4.647114in}{4.229077in}}%
\pgfpathclose%
\pgfusepath{fill}%
\end{pgfscope}%
\begin{pgfscope}%
\pgfpathrectangle{\pgfqpoint{0.539299in}{0.078740in}}{\pgfqpoint{7.842520in}{7.842520in}}%
\pgfusepath{clip}%
\pgfsetbuttcap%
\pgfsetroundjoin%
\definecolor{currentfill}{rgb}{0.126453,0.570633,0.549841}%
\pgfsetfillcolor{currentfill}%
\pgfsetlinewidth{0.000000pt}%
\definecolor{currentstroke}{rgb}{0.150476,0.504369,0.557430}%
\pgfsetstrokecolor{currentstroke}%
\pgfsetdash{}{0pt}%
\pgfpathmoveto{\pgfqpoint{2.267756in}{4.403873in}}%
\pgfpathlineto{\pgfqpoint{2.148306in}{3.953837in}}%
\pgfpathlineto{\pgfqpoint{2.181328in}{4.304148in}}%
\pgfpathclose%
\pgfusepath{fill}%
\end{pgfscope}%
\begin{pgfscope}%
\pgfpathrectangle{\pgfqpoint{0.539299in}{0.078740in}}{\pgfqpoint{7.842520in}{7.842520in}}%
\pgfusepath{clip}%
\pgfsetbuttcap%
\pgfsetroundjoin%
\definecolor{currentfill}{rgb}{0.278791,0.062145,0.386592}%
\pgfsetfillcolor{currentfill}%
\pgfsetlinewidth{0.000000pt}%
\definecolor{currentstroke}{rgb}{0.149039,0.508051,0.557250}%
\pgfsetstrokecolor{currentstroke}%
\pgfsetdash{}{0pt}%
\pgfpathmoveto{\pgfqpoint{6.897057in}{2.228556in}}%
\pgfpathlineto{\pgfqpoint{6.756390in}{2.344144in}}%
\pgfpathlineto{\pgfqpoint{6.681865in}{2.263405in}}%
\pgfpathclose%
\pgfusepath{fill}%
\end{pgfscope}%
\begin{pgfscope}%
\pgfpathrectangle{\pgfqpoint{0.539299in}{0.078740in}}{\pgfqpoint{7.842520in}{7.842520in}}%
\pgfusepath{clip}%
\pgfsetbuttcap%
\pgfsetroundjoin%
\definecolor{currentfill}{rgb}{0.288921,0.758394,0.428426}%
\pgfsetfillcolor{currentfill}%
\pgfsetlinewidth{0.000000pt}%
\definecolor{currentstroke}{rgb}{0.147607,0.511733,0.557049}%
\pgfsetstrokecolor{currentstroke}%
\pgfsetdash{}{0pt}%
\pgfpathmoveto{\pgfqpoint{2.867996in}{5.262317in}}%
\pgfpathlineto{\pgfqpoint{2.738735in}{4.997476in}}%
\pgfpathlineto{\pgfqpoint{2.652170in}{4.965747in}}%
\pgfpathclose%
\pgfusepath{fill}%
\end{pgfscope}%
\begin{pgfscope}%
\pgfpathrectangle{\pgfqpoint{0.539299in}{0.078740in}}{\pgfqpoint{7.842520in}{7.842520in}}%
\pgfusepath{clip}%
\pgfsetbuttcap%
\pgfsetroundjoin%
\definecolor{currentfill}{rgb}{0.252194,0.269783,0.531579}%
\pgfsetfillcolor{currentfill}%
\pgfsetlinewidth{0.000000pt}%
\definecolor{currentstroke}{rgb}{0.146180,0.515413,0.556823}%
\pgfsetstrokecolor{currentstroke}%
\pgfsetdash{}{0pt}%
\pgfpathmoveto{\pgfqpoint{1.633975in}{3.148721in}}%
\pgfpathlineto{\pgfqpoint{1.601948in}{2.927986in}}%
\pgfpathlineto{\pgfqpoint{1.515589in}{2.796677in}}%
\pgfpathclose%
\pgfusepath{fill}%
\end{pgfscope}%
\begin{pgfscope}%
\pgfpathrectangle{\pgfqpoint{0.539299in}{0.078740in}}{\pgfqpoint{7.842520in}{7.842520in}}%
\pgfusepath{clip}%
\pgfsetbuttcap%
\pgfsetroundjoin%
\definecolor{currentfill}{rgb}{0.185783,0.704891,0.485273}%
\pgfsetfillcolor{currentfill}%
\pgfsetlinewidth{0.000000pt}%
\definecolor{currentstroke}{rgb}{0.144759,0.519093,0.556572}%
\pgfsetstrokecolor{currentstroke}%
\pgfsetdash{}{0pt}%
\pgfpathmoveto{\pgfqpoint{2.440760in}{4.553021in}}%
\pgfpathlineto{\pgfqpoint{2.565417in}{4.919366in}}%
\pgfpathlineto{\pgfqpoint{2.652170in}{4.965747in}}%
\pgfpathclose%
\pgfusepath{fill}%
\end{pgfscope}%
\begin{pgfscope}%
\pgfpathrectangle{\pgfqpoint{0.539299in}{0.078740in}}{\pgfqpoint{7.842520in}{7.842520in}}%
\pgfusepath{clip}%
\pgfsetbuttcap%
\pgfsetroundjoin%
\definecolor{currentfill}{rgb}{0.280255,0.165693,0.476498}%
\pgfsetfillcolor{currentfill}%
\pgfsetlinewidth{0.000000pt}%
\definecolor{currentstroke}{rgb}{0.143343,0.522773,0.556295}%
\pgfsetstrokecolor{currentstroke}%
\pgfsetdash{}{0pt}%
\pgfpathmoveto{\pgfqpoint{6.118863in}{2.691325in}}%
\pgfpathlineto{\pgfqpoint{6.183670in}{2.518396in}}%
\pgfpathlineto{\pgfqpoint{6.259304in}{2.584902in}}%
\pgfpathclose%
\pgfusepath{fill}%
\end{pgfscope}%
\begin{pgfscope}%
\pgfpathrectangle{\pgfqpoint{0.539299in}{0.078740in}}{\pgfqpoint{7.842520in}{7.842520in}}%
\pgfusepath{clip}%
\pgfsetbuttcap%
\pgfsetroundjoin%
\definecolor{currentfill}{rgb}{0.273006,0.204520,0.501721}%
\pgfsetfillcolor{currentfill}%
\pgfsetlinewidth{0.000000pt}%
\definecolor{currentstroke}{rgb}{0.141935,0.526453,0.555991}%
\pgfsetstrokecolor{currentstroke}%
\pgfsetdash{}{0pt}%
\pgfpathmoveto{\pgfqpoint{5.978689in}{2.800383in}}%
\pgfpathlineto{\pgfqpoint{5.902739in}{2.756289in}}%
\pgfpathlineto{\pgfqpoint{6.043073in}{2.634987in}}%
\pgfpathclose%
\pgfusepath{fill}%
\end{pgfscope}%
\begin{pgfscope}%
\pgfpathrectangle{\pgfqpoint{0.539299in}{0.078740in}}{\pgfqpoint{7.842520in}{7.842520in}}%
\pgfusepath{clip}%
\pgfsetbuttcap%
\pgfsetroundjoin%
\definecolor{currentfill}{rgb}{0.281446,0.084320,0.407414}%
\pgfsetfillcolor{currentfill}%
\pgfsetlinewidth{0.000000pt}%
\definecolor{currentstroke}{rgb}{0.140536,0.530132,0.555659}%
\pgfsetstrokecolor{currentstroke}%
\pgfsetdash{}{0pt}%
\pgfpathmoveto{\pgfqpoint{6.540856in}{2.372768in}}%
\pgfpathlineto{\pgfqpoint{6.681865in}{2.263405in}}%
\pgfpathlineto{\pgfqpoint{6.756390in}{2.344144in}}%
\pgfpathclose%
\pgfusepath{fill}%
\end{pgfscope}%
\begin{pgfscope}%
\pgfpathrectangle{\pgfqpoint{0.539299in}{0.078740in}}{\pgfqpoint{7.842520in}{7.842520in}}%
\pgfusepath{clip}%
\pgfsetbuttcap%
\pgfsetroundjoin%
\definecolor{currentfill}{rgb}{0.121380,0.629492,0.531973}%
\pgfsetfillcolor{currentfill}%
\pgfsetlinewidth{0.000000pt}%
\definecolor{currentstroke}{rgb}{0.139147,0.533812,0.555298}%
\pgfsetstrokecolor{currentstroke}%
\pgfsetdash{}{0pt}%
\pgfpathmoveto{\pgfqpoint{4.588029in}{4.327753in}}%
\pgfpathlineto{\pgfqpoint{4.367781in}{4.630418in}}%
\pgfpathlineto{\pgfqpoint{4.507522in}{4.430367in}}%
\pgfpathclose%
\pgfusepath{fill}%
\end{pgfscope}%
\begin{pgfscope}%
\pgfpathrectangle{\pgfqpoint{0.539299in}{0.078740in}}{\pgfqpoint{7.842520in}{7.842520in}}%
\pgfusepath{clip}%
\pgfsetbuttcap%
\pgfsetroundjoin%
\definecolor{currentfill}{rgb}{0.496615,0.826376,0.306377}%
\pgfsetfillcolor{currentfill}%
\pgfsetlinewidth{0.000000pt}%
\definecolor{currentstroke}{rgb}{0.137770,0.537492,0.554906}%
\pgfsetstrokecolor{currentstroke}%
\pgfsetdash{}{0pt}%
\pgfpathmoveto{\pgfqpoint{3.531606in}{5.447979in}}%
\pgfpathlineto{\pgfqpoint{3.394745in}{5.445102in}}%
\pgfpathlineto{\pgfqpoint{3.308861in}{5.497668in}}%
\pgfpathclose%
\pgfusepath{fill}%
\end{pgfscope}%
\begin{pgfscope}%
\pgfpathrectangle{\pgfqpoint{0.539299in}{0.078740in}}{\pgfqpoint{7.842520in}{7.842520in}}%
\pgfusepath{clip}%
\pgfsetbuttcap%
\pgfsetroundjoin%
\definecolor{currentfill}{rgb}{0.282290,0.145912,0.461510}%
\pgfsetfillcolor{currentfill}%
\pgfsetlinewidth{0.000000pt}%
\definecolor{currentstroke}{rgb}{0.136408,0.541173,0.554483}%
\pgfsetstrokecolor{currentstroke}%
\pgfsetdash{}{0pt}%
\pgfpathmoveto{\pgfqpoint{6.259304in}{2.584902in}}%
\pgfpathlineto{\pgfqpoint{6.183670in}{2.518396in}}%
\pgfpathlineto{\pgfqpoint{6.399983in}{2.479315in}}%
\pgfpathclose%
\pgfusepath{fill}%
\end{pgfscope}%
\begin{pgfscope}%
\pgfpathrectangle{\pgfqpoint{0.539299in}{0.078740in}}{\pgfqpoint{7.842520in}{7.842520in}}%
\pgfusepath{clip}%
\pgfsetbuttcap%
\pgfsetroundjoin%
\definecolor{currentfill}{rgb}{0.421908,0.805774,0.351910}%
\pgfsetfillcolor{currentfill}%
\pgfsetlinewidth{0.000000pt}%
\definecolor{currentstroke}{rgb}{0.135066,0.544853,0.554029}%
\pgfsetstrokecolor{currentstroke}%
\pgfsetdash{}{0pt}%
\pgfpathmoveto{\pgfqpoint{3.754366in}{5.305678in}}%
\pgfpathlineto{\pgfqpoint{3.669721in}{5.396407in}}%
\pgfpathlineto{\pgfqpoint{3.808702in}{5.299454in}}%
\pgfpathclose%
\pgfusepath{fill}%
\end{pgfscope}%
\begin{pgfscope}%
\pgfpathrectangle{\pgfqpoint{0.539299in}{0.078740in}}{\pgfqpoint{7.842520in}{7.842520in}}%
\pgfusepath{clip}%
\pgfsetbuttcap%
\pgfsetroundjoin%
\definecolor{currentfill}{rgb}{0.140210,0.665859,0.513427}%
\pgfsetfillcolor{currentfill}%
\pgfsetlinewidth{0.000000pt}%
\definecolor{currentstroke}{rgb}{0.133743,0.548535,0.553541}%
\pgfsetstrokecolor{currentstroke}%
\pgfsetdash{}{0pt}%
\pgfpathmoveto{\pgfqpoint{2.440760in}{4.553021in}}%
\pgfpathlineto{\pgfqpoint{2.354275in}{4.486045in}}%
\pgfpathlineto{\pgfqpoint{2.478591in}{4.855876in}}%
\pgfpathclose%
\pgfusepath{fill}%
\end{pgfscope}%
\begin{pgfscope}%
\pgfpathrectangle{\pgfqpoint{0.539299in}{0.078740in}}{\pgfqpoint{7.842520in}{7.842520in}}%
\pgfusepath{clip}%
\pgfsetbuttcap%
\pgfsetroundjoin%
\definecolor{currentfill}{rgb}{0.449368,0.813768,0.335384}%
\pgfsetfillcolor{currentfill}%
\pgfsetlinewidth{0.000000pt}%
\definecolor{currentstroke}{rgb}{0.132444,0.552216,0.553018}%
\pgfsetstrokecolor{currentstroke}%
\pgfsetdash{}{0pt}%
\pgfpathmoveto{\pgfqpoint{2.954612in}{5.267216in}}%
\pgfpathlineto{\pgfqpoint{3.087266in}{5.438839in}}%
\pgfpathlineto{\pgfqpoint{3.173672in}{5.414731in}}%
\pgfpathclose%
\pgfusepath{fill}%
\end{pgfscope}%
\begin{pgfscope}%
\pgfpathrectangle{\pgfqpoint{0.539299in}{0.078740in}}{\pgfqpoint{7.842520in}{7.842520in}}%
\pgfusepath{clip}%
\pgfsetbuttcap%
\pgfsetroundjoin%
\definecolor{currentfill}{rgb}{0.278826,0.175490,0.483397}%
\pgfsetfillcolor{currentfill}%
\pgfsetlinewidth{0.000000pt}%
\definecolor{currentstroke}{rgb}{0.131172,0.555899,0.552459}%
\pgfsetstrokecolor{currentstroke}%
\pgfsetdash{}{0pt}%
\pgfpathmoveto{\pgfqpoint{6.043073in}{2.634987in}}%
\pgfpathlineto{\pgfqpoint{6.183670in}{2.518396in}}%
\pgfpathlineto{\pgfqpoint{6.118863in}{2.691325in}}%
\pgfpathclose%
\pgfusepath{fill}%
\end{pgfscope}%
\begin{pgfscope}%
\pgfpathrectangle{\pgfqpoint{0.539299in}{0.078740in}}{\pgfqpoint{7.842520in}{7.842520in}}%
\pgfusepath{clip}%
\pgfsetbuttcap%
\pgfsetroundjoin%
\definecolor{currentfill}{rgb}{0.214298,0.355619,0.551184}%
\pgfsetfillcolor{currentfill}%
\pgfsetlinewidth{0.000000pt}%
\definecolor{currentstroke}{rgb}{0.129933,0.559582,0.551864}%
\pgfsetstrokecolor{currentstroke}%
\pgfsetdash{}{0pt}%
\pgfpathmoveto{\pgfqpoint{1.601948in}{2.927986in}}%
\pgfpathlineto{\pgfqpoint{1.718526in}{3.335879in}}%
\pgfpathlineto{\pgfqpoint{1.803826in}{3.497087in}}%
\pgfpathclose%
\pgfusepath{fill}%
\end{pgfscope}%
\begin{pgfscope}%
\pgfpathrectangle{\pgfqpoint{0.539299in}{0.078740in}}{\pgfqpoint{7.842520in}{7.842520in}}%
\pgfusepath{clip}%
\pgfsetbuttcap%
\pgfsetroundjoin%
\definecolor{currentfill}{rgb}{0.162016,0.687316,0.499129}%
\pgfsetfillcolor{currentfill}%
\pgfsetlinewidth{0.000000pt}%
\definecolor{currentstroke}{rgb}{0.128729,0.563265,0.551229}%
\pgfsetstrokecolor{currentstroke}%
\pgfsetdash{}{0pt}%
\pgfpathmoveto{\pgfqpoint{4.227924in}{4.824025in}}%
\pgfpathlineto{\pgfqpoint{4.367781in}{4.630418in}}%
\pgfpathlineto{\pgfqpoint{4.309978in}{4.711999in}}%
\pgfpathclose%
\pgfusepath{fill}%
\end{pgfscope}%
\begin{pgfscope}%
\pgfpathrectangle{\pgfqpoint{0.539299in}{0.078740in}}{\pgfqpoint{7.842520in}{7.842520in}}%
\pgfusepath{clip}%
\pgfsetbuttcap%
\pgfsetroundjoin%
\definecolor{currentfill}{rgb}{0.277018,0.050344,0.375715}%
\pgfsetfillcolor{currentfill}%
\pgfsetlinewidth{0.000000pt}%
\definecolor{currentstroke}{rgb}{0.127568,0.566949,0.550556}%
\pgfsetstrokecolor{currentstroke}%
\pgfsetdash{}{0pt}%
\pgfpathmoveto{\pgfqpoint{6.681865in}{2.263405in}}%
\pgfpathlineto{\pgfqpoint{6.822930in}{2.149191in}}%
\pgfpathlineto{\pgfqpoint{6.897057in}{2.228556in}}%
\pgfpathclose%
\pgfusepath{fill}%
\end{pgfscope}%
\begin{pgfscope}%
\pgfpathrectangle{\pgfqpoint{0.539299in}{0.078740in}}{\pgfqpoint{7.842520in}{7.842520in}}%
\pgfusepath{clip}%
\pgfsetbuttcap%
\pgfsetroundjoin%
\definecolor{currentfill}{rgb}{0.369214,0.788888,0.382914}%
\pgfsetfillcolor{currentfill}%
\pgfsetlinewidth{0.000000pt}%
\definecolor{currentstroke}{rgb}{0.126453,0.570633,0.549841}%
\pgfsetstrokecolor{currentstroke}%
\pgfsetdash{}{0pt}%
\pgfpathmoveto{\pgfqpoint{3.948223in}{5.166109in}}%
\pgfpathlineto{\pgfqpoint{3.892829in}{5.198517in}}%
\pgfpathlineto{\pgfqpoint{3.808702in}{5.299454in}}%
\pgfpathclose%
\pgfusepath{fill}%
\end{pgfscope}%
\begin{pgfscope}%
\pgfpathrectangle{\pgfqpoint{0.539299in}{0.078740in}}{\pgfqpoint{7.842520in}{7.842520in}}%
\pgfusepath{clip}%
\pgfsetbuttcap%
\pgfsetroundjoin%
\definecolor{currentfill}{rgb}{0.202219,0.715272,0.476084}%
\pgfsetfillcolor{currentfill}%
\pgfsetlinewidth{0.000000pt}%
\definecolor{currentstroke}{rgb}{0.125394,0.574318,0.549086}%
\pgfsetstrokecolor{currentstroke}%
\pgfsetdash{}{0pt}%
\pgfpathmoveto{\pgfqpoint{4.309978in}{4.711999in}}%
\pgfpathlineto{\pgfqpoint{4.088028in}{5.004974in}}%
\pgfpathlineto{\pgfqpoint{4.227924in}{4.824025in}}%
\pgfpathclose%
\pgfusepath{fill}%
\end{pgfscope}%
\begin{pgfscope}%
\pgfpathrectangle{\pgfqpoint{0.539299in}{0.078740in}}{\pgfqpoint{7.842520in}{7.842520in}}%
\pgfusepath{clip}%
\pgfsetbuttcap%
\pgfsetroundjoin%
\definecolor{currentfill}{rgb}{0.258965,0.251537,0.524736}%
\pgfsetfillcolor{currentfill}%
\pgfsetlinewidth{0.000000pt}%
\definecolor{currentstroke}{rgb}{0.124395,0.578002,0.548287}%
\pgfsetstrokecolor{currentstroke}%
\pgfsetdash{}{0pt}%
\pgfpathmoveto{\pgfqpoint{1.515589in}{2.796677in}}%
\pgfpathlineto{\pgfqpoint{1.429595in}{2.648382in}}%
\pgfpathlineto{\pgfqpoint{1.633975in}{3.148721in}}%
\pgfpathclose%
\pgfusepath{fill}%
\end{pgfscope}%
\begin{pgfscope}%
\pgfpathrectangle{\pgfqpoint{0.539299in}{0.078740in}}{\pgfqpoint{7.842520in}{7.842520in}}%
\pgfusepath{clip}%
\pgfsetbuttcap%
\pgfsetroundjoin%
\definecolor{currentfill}{rgb}{0.199430,0.387607,0.554642}%
\pgfsetfillcolor{currentfill}%
\pgfsetlinewidth{0.000000pt}%
\definecolor{currentstroke}{rgb}{0.123463,0.581687,0.547445}%
\pgfsetstrokecolor{currentstroke}%
\pgfsetdash{}{0pt}%
\pgfpathmoveto{\pgfqpoint{5.343771in}{3.316408in}}%
\pgfpathlineto{\pgfqpoint{5.204442in}{3.480326in}}%
\pgfpathlineto{\pgfqpoint{5.266300in}{3.347822in}}%
\pgfpathclose%
\pgfusepath{fill}%
\end{pgfscope}%
\begin{pgfscope}%
\pgfpathrectangle{\pgfqpoint{0.539299in}{0.078740in}}{\pgfqpoint{7.842520in}{7.842520in}}%
\pgfusepath{clip}%
\pgfsetbuttcap%
\pgfsetroundjoin%
\definecolor{currentfill}{rgb}{0.210503,0.363727,0.552206}%
\pgfsetfillcolor{currentfill}%
\pgfsetlinewidth{0.000000pt}%
\definecolor{currentstroke}{rgb}{0.122606,0.585371,0.546557}%
\pgfsetstrokecolor{currentstroke}%
\pgfsetdash{}{0pt}%
\pgfpathmoveto{\pgfqpoint{5.483218in}{3.162772in}}%
\pgfpathlineto{\pgfqpoint{5.343771in}{3.316408in}}%
\pgfpathlineto{\pgfqpoint{5.266300in}{3.347822in}}%
\pgfpathclose%
\pgfusepath{fill}%
\end{pgfscope}%
\begin{pgfscope}%
\pgfpathrectangle{\pgfqpoint{0.539299in}{0.078740in}}{\pgfqpoint{7.842520in}{7.842520in}}%
\pgfusepath{clip}%
\pgfsetbuttcap%
\pgfsetroundjoin%
\definecolor{currentfill}{rgb}{0.274149,0.751988,0.436601}%
\pgfsetfillcolor{currentfill}%
\pgfsetlinewidth{0.000000pt}%
\definecolor{currentstroke}{rgb}{0.121831,0.589055,0.545623}%
\pgfsetstrokecolor{currentstroke}%
\pgfsetdash{}{0pt}%
\pgfpathmoveto{\pgfqpoint{4.170832in}{4.893238in}}%
\pgfpathlineto{\pgfqpoint{3.948223in}{5.166109in}}%
\pgfpathlineto{\pgfqpoint{4.088028in}{5.004974in}}%
\pgfpathclose%
\pgfusepath{fill}%
\end{pgfscope}%
\begin{pgfscope}%
\pgfpathrectangle{\pgfqpoint{0.539299in}{0.078740in}}{\pgfqpoint{7.842520in}{7.842520in}}%
\pgfusepath{clip}%
\pgfsetbuttcap%
\pgfsetroundjoin%
\definecolor{currentfill}{rgb}{0.229739,0.322361,0.545706}%
\pgfsetfillcolor{currentfill}%
\pgfsetlinewidth{0.000000pt}%
\definecolor{currentstroke}{rgb}{0.121148,0.592739,0.544641}%
\pgfsetstrokecolor{currentstroke}%
\pgfsetdash{}{0pt}%
\pgfpathmoveto{\pgfqpoint{5.483218in}{3.162772in}}%
\pgfpathlineto{\pgfqpoint{5.406152in}{3.177575in}}%
\pgfpathlineto{\pgfqpoint{5.622835in}{3.018878in}}%
\pgfpathclose%
\pgfusepath{fill}%
\end{pgfscope}%
\begin{pgfscope}%
\pgfpathrectangle{\pgfqpoint{0.539299in}{0.078740in}}{\pgfqpoint{7.842520in}{7.842520in}}%
\pgfusepath{clip}%
\pgfsetbuttcap%
\pgfsetroundjoin%
\definecolor{currentfill}{rgb}{0.283091,0.110553,0.431554}%
\pgfsetfillcolor{currentfill}%
\pgfsetlinewidth{0.000000pt}%
\definecolor{currentstroke}{rgb}{0.120565,0.596422,0.543611}%
\pgfsetstrokecolor{currentstroke}%
\pgfsetdash{}{0pt}%
\pgfpathmoveto{\pgfqpoint{6.540856in}{2.372768in}}%
\pgfpathlineto{\pgfqpoint{6.399983in}{2.479315in}}%
\pgfpathlineto{\pgfqpoint{6.465616in}{2.293258in}}%
\pgfpathclose%
\pgfusepath{fill}%
\end{pgfscope}%
\begin{pgfscope}%
\pgfpathrectangle{\pgfqpoint{0.539299in}{0.078740in}}{\pgfqpoint{7.842520in}{7.842520in}}%
\pgfusepath{clip}%
\pgfsetbuttcap%
\pgfsetroundjoin%
\definecolor{currentfill}{rgb}{0.180629,0.429975,0.557282}%
\pgfsetfillcolor{currentfill}%
\pgfsetlinewidth{0.000000pt}%
\definecolor{currentstroke}{rgb}{0.120092,0.600104,0.542530}%
\pgfsetstrokecolor{currentstroke}%
\pgfsetdash{}{0pt}%
\pgfpathmoveto{\pgfqpoint{5.126490in}{3.527887in}}%
\pgfpathlineto{\pgfqpoint{5.204442in}{3.480326in}}%
\pgfpathlineto{\pgfqpoint{5.065172in}{3.654551in}}%
\pgfpathclose%
\pgfusepath{fill}%
\end{pgfscope}%
\begin{pgfscope}%
\pgfpathrectangle{\pgfqpoint{0.539299in}{0.078740in}}{\pgfqpoint{7.842520in}{7.842520in}}%
\pgfusepath{clip}%
\pgfsetbuttcap%
\pgfsetroundjoin%
\definecolor{currentfill}{rgb}{0.252194,0.269783,0.531579}%
\pgfsetfillcolor{currentfill}%
\pgfsetlinewidth{0.000000pt}%
\definecolor{currentstroke}{rgb}{0.119738,0.603785,0.541400}%
\pgfsetstrokecolor{currentstroke}%
\pgfsetdash{}{0pt}%
\pgfpathmoveto{\pgfqpoint{5.686202in}{2.865893in}}%
\pgfpathlineto{\pgfqpoint{5.762666in}{2.883788in}}%
\pgfpathlineto{\pgfqpoint{5.622835in}{3.018878in}}%
\pgfpathclose%
\pgfusepath{fill}%
\end{pgfscope}%
\begin{pgfscope}%
\pgfpathrectangle{\pgfqpoint{0.539299in}{0.078740in}}{\pgfqpoint{7.842520in}{7.842520in}}%
\pgfusepath{clip}%
\pgfsetbuttcap%
\pgfsetroundjoin%
\definecolor{currentfill}{rgb}{0.282623,0.140926,0.457517}%
\pgfsetfillcolor{currentfill}%
\pgfsetlinewidth{0.000000pt}%
\definecolor{currentstroke}{rgb}{0.119512,0.607464,0.540218}%
\pgfsetstrokecolor{currentstroke}%
\pgfsetdash{}{0pt}%
\pgfpathmoveto{\pgfqpoint{6.399983in}{2.479315in}}%
\pgfpathlineto{\pgfqpoint{6.183670in}{2.518396in}}%
\pgfpathlineto{\pgfqpoint{6.324525in}{2.404996in}}%
\pgfpathclose%
\pgfusepath{fill}%
\end{pgfscope}%
\begin{pgfscope}%
\pgfpathrectangle{\pgfqpoint{0.539299in}{0.078740in}}{\pgfqpoint{7.842520in}{7.842520in}}%
\pgfusepath{clip}%
\pgfsetbuttcap%
\pgfsetroundjoin%
\definecolor{currentfill}{rgb}{0.163625,0.471133,0.558148}%
\pgfsetfillcolor{currentfill}%
\pgfsetlinewidth{0.000000pt}%
\definecolor{currentstroke}{rgb}{0.119423,0.611141,0.538982}%
\pgfsetstrokecolor{currentstroke}%
\pgfsetdash{}{0pt}%
\pgfpathmoveto{\pgfqpoint{4.986662in}{3.717303in}}%
\pgfpathlineto{\pgfqpoint{5.065172in}{3.654551in}}%
\pgfpathlineto{\pgfqpoint{4.925899in}{3.838472in}}%
\pgfpathclose%
\pgfusepath{fill}%
\end{pgfscope}%
\begin{pgfscope}%
\pgfpathrectangle{\pgfqpoint{0.539299in}{0.078740in}}{\pgfqpoint{7.842520in}{7.842520in}}%
\pgfusepath{clip}%
\pgfsetbuttcap%
\pgfsetroundjoin%
\definecolor{currentfill}{rgb}{0.151918,0.500685,0.557587}%
\pgfsetfillcolor{currentfill}%
\pgfsetlinewidth{0.000000pt}%
\definecolor{currentstroke}{rgb}{0.119483,0.614817,0.537692}%
\pgfsetstrokecolor{currentstroke}%
\pgfsetdash{}{0pt}%
\pgfpathmoveto{\pgfqpoint{1.975740in}{3.757494in}}%
\pgfpathlineto{\pgfqpoint{1.889628in}{3.636546in}}%
\pgfpathlineto{\pgfqpoint{2.095141in}{4.184186in}}%
\pgfpathclose%
\pgfusepath{fill}%
\end{pgfscope}%
\begin{pgfscope}%
\pgfpathrectangle{\pgfqpoint{0.539299in}{0.078740in}}{\pgfqpoint{7.842520in}{7.842520in}}%
\pgfusepath{clip}%
\pgfsetbuttcap%
\pgfsetroundjoin%
\definecolor{currentfill}{rgb}{0.506271,0.828786,0.300362}%
\pgfsetfillcolor{currentfill}%
\pgfsetlinewidth{0.000000pt}%
\definecolor{currentstroke}{rgb}{0.119699,0.618490,0.536347}%
\pgfsetstrokecolor{currentstroke}%
\pgfsetdash{}{0pt}%
\pgfpathmoveto{\pgfqpoint{3.087266in}{5.438839in}}%
\pgfpathlineto{\pgfqpoint{3.308861in}{5.497668in}}%
\pgfpathlineto{\pgfqpoint{3.173672in}{5.414731in}}%
\pgfpathclose%
\pgfusepath{fill}%
\end{pgfscope}%
\begin{pgfscope}%
\pgfpathrectangle{\pgfqpoint{0.539299in}{0.078740in}}{\pgfqpoint{7.842520in}{7.842520in}}%
\pgfusepath{clip}%
\pgfsetbuttcap%
\pgfsetroundjoin%
\definecolor{currentfill}{rgb}{0.225863,0.330805,0.547314}%
\pgfsetfillcolor{currentfill}%
\pgfsetlinewidth{0.000000pt}%
\definecolor{currentstroke}{rgb}{0.120081,0.622161,0.534946}%
\pgfsetstrokecolor{currentstroke}%
\pgfsetdash{}{0pt}%
\pgfpathmoveto{\pgfqpoint{1.718526in}{3.335879in}}%
\pgfpathlineto{\pgfqpoint{1.601948in}{2.927986in}}%
\pgfpathlineto{\pgfqpoint{1.633975in}{3.148721in}}%
\pgfpathclose%
\pgfusepath{fill}%
\end{pgfscope}%
\begin{pgfscope}%
\pgfpathrectangle{\pgfqpoint{0.539299in}{0.078740in}}{\pgfqpoint{7.842520in}{7.842520in}}%
\pgfusepath{clip}%
\pgfsetbuttcap%
\pgfsetroundjoin%
\definecolor{currentfill}{rgb}{0.258965,0.251537,0.524736}%
\pgfsetfillcolor{currentfill}%
\pgfsetlinewidth{0.000000pt}%
\definecolor{currentstroke}{rgb}{0.120638,0.625828,0.533488}%
\pgfsetstrokecolor{currentstroke}%
\pgfsetdash{}{0pt}%
\pgfpathmoveto{\pgfqpoint{5.902739in}{2.756289in}}%
\pgfpathlineto{\pgfqpoint{5.762666in}{2.883788in}}%
\pgfpathlineto{\pgfqpoint{5.686202in}{2.865893in}}%
\pgfpathclose%
\pgfusepath{fill}%
\end{pgfscope}%
\begin{pgfscope}%
\pgfpathrectangle{\pgfqpoint{0.539299in}{0.078740in}}{\pgfqpoint{7.842520in}{7.842520in}}%
\pgfusepath{clip}%
\pgfsetbuttcap%
\pgfsetroundjoin%
\definecolor{currentfill}{rgb}{0.281924,0.089666,0.412415}%
\pgfsetfillcolor{currentfill}%
\pgfsetlinewidth{0.000000pt}%
\definecolor{currentstroke}{rgb}{0.121380,0.629492,0.531973}%
\pgfsetstrokecolor{currentstroke}%
\pgfsetdash{}{0pt}%
\pgfpathmoveto{\pgfqpoint{6.465616in}{2.293258in}}%
\pgfpathlineto{\pgfqpoint{6.681865in}{2.263405in}}%
\pgfpathlineto{\pgfqpoint{6.540856in}{2.372768in}}%
\pgfpathclose%
\pgfusepath{fill}%
\end{pgfscope}%
\begin{pgfscope}%
\pgfpathrectangle{\pgfqpoint{0.539299in}{0.078740in}}{\pgfqpoint{7.842520in}{7.842520in}}%
\pgfusepath{clip}%
\pgfsetbuttcap%
\pgfsetroundjoin%
\definecolor{currentfill}{rgb}{0.185783,0.704891,0.485273}%
\pgfsetfillcolor{currentfill}%
\pgfsetlinewidth{0.000000pt}%
\definecolor{currentstroke}{rgb}{0.122312,0.633153,0.530398}%
\pgfsetstrokecolor{currentstroke}%
\pgfsetdash{}{0pt}%
\pgfpathmoveto{\pgfqpoint{2.478591in}{4.855876in}}%
\pgfpathlineto{\pgfqpoint{2.565417in}{4.919366in}}%
\pgfpathlineto{\pgfqpoint{2.440760in}{4.553021in}}%
\pgfpathclose%
\pgfusepath{fill}%
\end{pgfscope}%
\begin{pgfscope}%
\pgfpathrectangle{\pgfqpoint{0.539299in}{0.078740in}}{\pgfqpoint{7.842520in}{7.842520in}}%
\pgfusepath{clip}%
\pgfsetbuttcap%
\pgfsetroundjoin%
\definecolor{currentfill}{rgb}{0.283229,0.120777,0.440584}%
\pgfsetfillcolor{currentfill}%
\pgfsetlinewidth{0.000000pt}%
\definecolor{currentstroke}{rgb}{0.123444,0.636809,0.528763}%
\pgfsetstrokecolor{currentstroke}%
\pgfsetdash{}{0pt}%
\pgfpathmoveto{\pgfqpoint{6.465616in}{2.293258in}}%
\pgfpathlineto{\pgfqpoint{6.399983in}{2.479315in}}%
\pgfpathlineto{\pgfqpoint{6.324525in}{2.404996in}}%
\pgfpathclose%
\pgfusepath{fill}%
\end{pgfscope}%
\begin{pgfscope}%
\pgfpathrectangle{\pgfqpoint{0.539299in}{0.078740in}}{\pgfqpoint{7.842520in}{7.842520in}}%
\pgfusepath{clip}%
\pgfsetbuttcap%
\pgfsetroundjoin%
\definecolor{currentfill}{rgb}{0.139147,0.533812,0.555298}%
\pgfsetfillcolor{currentfill}%
\pgfsetlinewidth{0.000000pt}%
\definecolor{currentstroke}{rgb}{0.124780,0.640461,0.527068}%
\pgfsetstrokecolor{currentstroke}%
\pgfsetdash{}{0pt}%
\pgfpathmoveto{\pgfqpoint{4.925899in}{3.838472in}}%
\pgfpathlineto{\pgfqpoint{4.786562in}{4.030722in}}%
\pgfpathlineto{\pgfqpoint{4.706729in}{4.119068in}}%
\pgfpathclose%
\pgfusepath{fill}%
\end{pgfscope}%
\begin{pgfscope}%
\pgfpathrectangle{\pgfqpoint{0.539299in}{0.078740in}}{\pgfqpoint{7.842520in}{7.842520in}}%
\pgfusepath{clip}%
\pgfsetbuttcap%
\pgfsetroundjoin%
\definecolor{currentfill}{rgb}{0.273006,0.204520,0.501721}%
\pgfsetfillcolor{currentfill}%
\pgfsetlinewidth{0.000000pt}%
\definecolor{currentstroke}{rgb}{0.126326,0.644107,0.525311}%
\pgfsetstrokecolor{currentstroke}%
\pgfsetdash{}{0pt}%
\pgfpathmoveto{\pgfqpoint{5.902739in}{2.756289in}}%
\pgfpathlineto{\pgfqpoint{5.967002in}{2.588084in}}%
\pgfpathlineto{\pgfqpoint{6.043073in}{2.634987in}}%
\pgfpathclose%
\pgfusepath{fill}%
\end{pgfscope}%
\begin{pgfscope}%
\pgfpathrectangle{\pgfqpoint{0.539299in}{0.078740in}}{\pgfqpoint{7.842520in}{7.842520in}}%
\pgfusepath{clip}%
\pgfsetbuttcap%
\pgfsetroundjoin%
\definecolor{currentfill}{rgb}{0.271828,0.209303,0.504434}%
\pgfsetfillcolor{currentfill}%
\pgfsetlinewidth{0.000000pt}%
\definecolor{currentstroke}{rgb}{0.128087,0.647749,0.523491}%
\pgfsetstrokecolor{currentstroke}%
\pgfsetdash{}{0pt}%
\pgfpathmoveto{\pgfqpoint{1.429595in}{2.648382in}}%
\pgfpathlineto{\pgfqpoint{1.344269in}{2.477731in}}%
\pgfpathlineto{\pgfqpoint{1.550510in}{2.929884in}}%
\pgfpathclose%
\pgfusepath{fill}%
\end{pgfscope}%
\begin{pgfscope}%
\pgfpathrectangle{\pgfqpoint{0.539299in}{0.078740in}}{\pgfqpoint{7.842520in}{7.842520in}}%
\pgfusepath{clip}%
\pgfsetbuttcap%
\pgfsetroundjoin%
\definecolor{currentfill}{rgb}{0.545524,0.838039,0.275626}%
\pgfsetfillcolor{currentfill}%
\pgfsetlinewidth{0.000000pt}%
\definecolor{currentstroke}{rgb}{0.130067,0.651384,0.521608}%
\pgfsetstrokecolor{currentstroke}%
\pgfsetdash{}{0pt}%
\pgfpathmoveto{\pgfqpoint{3.531606in}{5.447979in}}%
\pgfpathlineto{\pgfqpoint{3.308861in}{5.497668in}}%
\pgfpathlineto{\pgfqpoint{3.445912in}{5.515993in}}%
\pgfpathclose%
\pgfusepath{fill}%
\end{pgfscope}%
\begin{pgfscope}%
\pgfpathrectangle{\pgfqpoint{0.539299in}{0.078740in}}{\pgfqpoint{7.842520in}{7.842520in}}%
\pgfusepath{clip}%
\pgfsetbuttcap%
\pgfsetroundjoin%
\definecolor{currentfill}{rgb}{0.212395,0.359683,0.551710}%
\pgfsetfillcolor{currentfill}%
\pgfsetlinewidth{0.000000pt}%
\definecolor{currentstroke}{rgb}{0.132268,0.655014,0.519661}%
\pgfsetstrokecolor{currentstroke}%
\pgfsetdash{}{0pt}%
\pgfpathmoveto{\pgfqpoint{5.266300in}{3.347822in}}%
\pgfpathlineto{\pgfqpoint{5.406152in}{3.177575in}}%
\pgfpathlineto{\pgfqpoint{5.483218in}{3.162772in}}%
\pgfpathclose%
\pgfusepath{fill}%
\end{pgfscope}%
\begin{pgfscope}%
\pgfpathrectangle{\pgfqpoint{0.539299in}{0.078740in}}{\pgfqpoint{7.842520in}{7.842520in}}%
\pgfusepath{clip}%
\pgfsetbuttcap%
\pgfsetroundjoin%
\definecolor{currentfill}{rgb}{0.231674,0.318106,0.544834}%
\pgfsetfillcolor{currentfill}%
\pgfsetlinewidth{0.000000pt}%
\definecolor{currentstroke}{rgb}{0.134692,0.658636,0.517649}%
\pgfsetstrokecolor{currentstroke}%
\pgfsetdash{}{0pt}%
\pgfpathmoveto{\pgfqpoint{5.622835in}{3.018878in}}%
\pgfpathlineto{\pgfqpoint{5.406152in}{3.177575in}}%
\pgfpathlineto{\pgfqpoint{5.546103in}{3.017088in}}%
\pgfpathclose%
\pgfusepath{fill}%
\end{pgfscope}%
\begin{pgfscope}%
\pgfpathrectangle{\pgfqpoint{0.539299in}{0.078740in}}{\pgfqpoint{7.842520in}{7.842520in}}%
\pgfusepath{clip}%
\pgfsetbuttcap%
\pgfsetroundjoin%
\definecolor{currentfill}{rgb}{0.243113,0.292092,0.538516}%
\pgfsetfillcolor{currentfill}%
\pgfsetlinewidth{0.000000pt}%
\definecolor{currentstroke}{rgb}{0.137339,0.662252,0.515571}%
\pgfsetstrokecolor{currentstroke}%
\pgfsetdash{}{0pt}%
\pgfpathmoveto{\pgfqpoint{5.622835in}{3.018878in}}%
\pgfpathlineto{\pgfqpoint{5.546103in}{3.017088in}}%
\pgfpathlineto{\pgfqpoint{5.686202in}{2.865893in}}%
\pgfpathclose%
\pgfusepath{fill}%
\end{pgfscope}%
\begin{pgfscope}%
\pgfpathrectangle{\pgfqpoint{0.539299in}{0.078740in}}{\pgfqpoint{7.842520in}{7.842520in}}%
\pgfusepath{clip}%
\pgfsetbuttcap%
\pgfsetroundjoin%
\definecolor{currentfill}{rgb}{0.187231,0.414746,0.556547}%
\pgfsetfillcolor{currentfill}%
\pgfsetlinewidth{0.000000pt}%
\definecolor{currentstroke}{rgb}{0.140210,0.665859,0.513427}%
\pgfsetstrokecolor{currentstroke}%
\pgfsetdash{}{0pt}%
\pgfpathmoveto{\pgfqpoint{5.266300in}{3.347822in}}%
\pgfpathlineto{\pgfqpoint{5.204442in}{3.480326in}}%
\pgfpathlineto{\pgfqpoint{5.126490in}{3.527887in}}%
\pgfpathclose%
\pgfusepath{fill}%
\end{pgfscope}%
\begin{pgfscope}%
\pgfpathrectangle{\pgfqpoint{0.539299in}{0.078740in}}{\pgfqpoint{7.842520in}{7.842520in}}%
\pgfusepath{clip}%
\pgfsetbuttcap%
\pgfsetroundjoin%
\definecolor{currentfill}{rgb}{0.277018,0.050344,0.375715}%
\pgfsetfillcolor{currentfill}%
\pgfsetlinewidth{0.000000pt}%
\definecolor{currentstroke}{rgb}{0.143303,0.669459,0.511215}%
\pgfsetstrokecolor{currentstroke}%
\pgfsetdash{}{0pt}%
\pgfpathmoveto{\pgfqpoint{6.748358in}{2.068409in}}%
\pgfpathlineto{\pgfqpoint{6.822930in}{2.149191in}}%
\pgfpathlineto{\pgfqpoint{6.681865in}{2.263405in}}%
\pgfpathclose%
\pgfusepath{fill}%
\end{pgfscope}%
\begin{pgfscope}%
\pgfpathrectangle{\pgfqpoint{0.539299in}{0.078740in}}{\pgfqpoint{7.842520in}{7.842520in}}%
\pgfusepath{clip}%
\pgfsetbuttcap%
\pgfsetroundjoin%
\definecolor{currentfill}{rgb}{0.278012,0.180367,0.486697}%
\pgfsetfillcolor{currentfill}%
\pgfsetlinewidth{0.000000pt}%
\definecolor{currentstroke}{rgb}{0.146616,0.673050,0.508936}%
\pgfsetstrokecolor{currentstroke}%
\pgfsetdash{}{0pt}%
\pgfpathmoveto{\pgfqpoint{6.043073in}{2.634987in}}%
\pgfpathlineto{\pgfqpoint{5.967002in}{2.588084in}}%
\pgfpathlineto{\pgfqpoint{6.183670in}{2.518396in}}%
\pgfpathclose%
\pgfusepath{fill}%
\end{pgfscope}%
\begin{pgfscope}%
\pgfpathrectangle{\pgfqpoint{0.539299in}{0.078740in}}{\pgfqpoint{7.842520in}{7.842520in}}%
\pgfusepath{clip}%
\pgfsetbuttcap%
\pgfsetroundjoin%
\definecolor{currentfill}{rgb}{0.123463,0.581687,0.547445}%
\pgfsetfillcolor{currentfill}%
\pgfsetlinewidth{0.000000pt}%
\definecolor{currentstroke}{rgb}{0.150148,0.676631,0.506589}%
\pgfsetstrokecolor{currentstroke}%
\pgfsetdash{}{0pt}%
\pgfpathmoveto{\pgfqpoint{4.786562in}{4.030722in}}%
\pgfpathlineto{\pgfqpoint{4.647114in}{4.229077in}}%
\pgfpathlineto{\pgfqpoint{4.566539in}{4.326950in}}%
\pgfpathclose%
\pgfusepath{fill}%
\end{pgfscope}%
\begin{pgfscope}%
\pgfpathrectangle{\pgfqpoint{0.539299in}{0.078740in}}{\pgfqpoint{7.842520in}{7.842520in}}%
\pgfusepath{clip}%
\pgfsetbuttcap%
\pgfsetroundjoin%
\definecolor{currentfill}{rgb}{0.449368,0.813768,0.335384}%
\pgfsetfillcolor{currentfill}%
\pgfsetlinewidth{0.000000pt}%
\definecolor{currentstroke}{rgb}{0.153894,0.680203,0.504172}%
\pgfsetstrokecolor{currentstroke}%
\pgfsetdash{}{0pt}%
\pgfpathmoveto{\pgfqpoint{2.954612in}{5.267216in}}%
\pgfpathlineto{\pgfqpoint{2.867996in}{5.262317in}}%
\pgfpathlineto{\pgfqpoint{3.000456in}{5.448722in}}%
\pgfpathclose%
\pgfusepath{fill}%
\end{pgfscope}%
\begin{pgfscope}%
\pgfpathrectangle{\pgfqpoint{0.539299in}{0.078740in}}{\pgfqpoint{7.842520in}{7.842520in}}%
\pgfusepath{clip}%
\pgfsetbuttcap%
\pgfsetroundjoin%
\definecolor{currentfill}{rgb}{0.515992,0.831158,0.294279}%
\pgfsetfillcolor{currentfill}%
\pgfsetlinewidth{0.000000pt}%
\definecolor{currentstroke}{rgb}{0.157851,0.683765,0.501686}%
\pgfsetstrokecolor{currentstroke}%
\pgfsetdash{}{0pt}%
\pgfpathmoveto{\pgfqpoint{3.669721in}{5.396407in}}%
\pgfpathlineto{\pgfqpoint{3.531606in}{5.447979in}}%
\pgfpathlineto{\pgfqpoint{3.584365in}{5.478002in}}%
\pgfpathclose%
\pgfusepath{fill}%
\end{pgfscope}%
\begin{pgfscope}%
\pgfpathrectangle{\pgfqpoint{0.539299in}{0.078740in}}{\pgfqpoint{7.842520in}{7.842520in}}%
\pgfusepath{clip}%
\pgfsetbuttcap%
\pgfsetroundjoin%
\definecolor{currentfill}{rgb}{0.360741,0.785964,0.387814}%
\pgfsetfillcolor{currentfill}%
\pgfsetlinewidth{0.000000pt}%
\definecolor{currentstroke}{rgb}{0.162016,0.687316,0.499129}%
\pgfsetstrokecolor{currentstroke}%
\pgfsetdash{}{0pt}%
\pgfpathmoveto{\pgfqpoint{2.867996in}{5.262317in}}%
\pgfpathlineto{\pgfqpoint{2.652170in}{4.965747in}}%
\pgfpathlineto{\pgfqpoint{2.781099in}{5.242417in}}%
\pgfpathclose%
\pgfusepath{fill}%
\end{pgfscope}%
\begin{pgfscope}%
\pgfpathrectangle{\pgfqpoint{0.539299in}{0.078740in}}{\pgfqpoint{7.842520in}{7.842520in}}%
\pgfusepath{clip}%
\pgfsetbuttcap%
\pgfsetroundjoin%
\definecolor{currentfill}{rgb}{0.131172,0.555899,0.552459}%
\pgfsetfillcolor{currentfill}%
\pgfsetlinewidth{0.000000pt}%
\definecolor{currentstroke}{rgb}{0.166383,0.690856,0.496502}%
\pgfsetstrokecolor{currentstroke}%
\pgfsetdash{}{0pt}%
\pgfpathmoveto{\pgfqpoint{1.975740in}{3.757494in}}%
\pgfpathlineto{\pgfqpoint{2.095141in}{4.184186in}}%
\pgfpathlineto{\pgfqpoint{2.181328in}{4.304148in}}%
\pgfpathclose%
\pgfusepath{fill}%
\end{pgfscope}%
\begin{pgfscope}%
\pgfpathrectangle{\pgfqpoint{0.539299in}{0.078740in}}{\pgfqpoint{7.842520in}{7.842520in}}%
\pgfusepath{clip}%
\pgfsetbuttcap%
\pgfsetroundjoin%
\definecolor{currentfill}{rgb}{0.169646,0.456262,0.558030}%
\pgfsetfillcolor{currentfill}%
\pgfsetlinewidth{0.000000pt}%
\definecolor{currentstroke}{rgb}{0.170948,0.694384,0.493803}%
\pgfsetstrokecolor{currentstroke}%
\pgfsetdash{}{0pt}%
\pgfpathmoveto{\pgfqpoint{5.065172in}{3.654551in}}%
\pgfpathlineto{\pgfqpoint{4.986662in}{3.717303in}}%
\pgfpathlineto{\pgfqpoint{5.126490in}{3.527887in}}%
\pgfpathclose%
\pgfusepath{fill}%
\end{pgfscope}%
\begin{pgfscope}%
\pgfpathrectangle{\pgfqpoint{0.539299in}{0.078740in}}{\pgfqpoint{7.842520in}{7.842520in}}%
\pgfusepath{clip}%
\pgfsetbuttcap%
\pgfsetroundjoin%
\definecolor{currentfill}{rgb}{0.280894,0.078907,0.402329}%
\pgfsetfillcolor{currentfill}%
\pgfsetlinewidth{0.000000pt}%
\definecolor{currentstroke}{rgb}{0.175707,0.697900,0.491033}%
\pgfsetstrokecolor{currentstroke}%
\pgfsetdash{}{0pt}%
\pgfpathmoveto{\pgfqpoint{6.465616in}{2.293258in}}%
\pgfpathlineto{\pgfqpoint{6.606910in}{2.181621in}}%
\pgfpathlineto{\pgfqpoint{6.681865in}{2.263405in}}%
\pgfpathclose%
\pgfusepath{fill}%
\end{pgfscope}%
\begin{pgfscope}%
\pgfpathrectangle{\pgfqpoint{0.539299in}{0.078740in}}{\pgfqpoint{7.842520in}{7.842520in}}%
\pgfusepath{clip}%
\pgfsetbuttcap%
\pgfsetroundjoin%
\definecolor{currentfill}{rgb}{0.262138,0.242286,0.520837}%
\pgfsetfillcolor{currentfill}%
\pgfsetlinewidth{0.000000pt}%
\definecolor{currentstroke}{rgb}{0.180653,0.701402,0.488189}%
\pgfsetstrokecolor{currentstroke}%
\pgfsetdash{}{0pt}%
\pgfpathmoveto{\pgfqpoint{5.686202in}{2.865893in}}%
\pgfpathlineto{\pgfqpoint{5.826491in}{2.723216in}}%
\pgfpathlineto{\pgfqpoint{5.902739in}{2.756289in}}%
\pgfpathclose%
\pgfusepath{fill}%
\end{pgfscope}%
\begin{pgfscope}%
\pgfpathrectangle{\pgfqpoint{0.539299in}{0.078740in}}{\pgfqpoint{7.842520in}{7.842520in}}%
\pgfusepath{clip}%
\pgfsetbuttcap%
\pgfsetroundjoin%
\definecolor{currentfill}{rgb}{0.487026,0.823929,0.312321}%
\pgfsetfillcolor{currentfill}%
\pgfsetlinewidth{0.000000pt}%
\definecolor{currentstroke}{rgb}{0.185783,0.704891,0.485273}%
\pgfsetstrokecolor{currentstroke}%
\pgfsetdash{}{0pt}%
\pgfpathmoveto{\pgfqpoint{3.000456in}{5.448722in}}%
\pgfpathlineto{\pgfqpoint{3.087266in}{5.438839in}}%
\pgfpathlineto{\pgfqpoint{2.954612in}{5.267216in}}%
\pgfpathclose%
\pgfusepath{fill}%
\end{pgfscope}%
\begin{pgfscope}%
\pgfpathrectangle{\pgfqpoint{0.539299in}{0.078740in}}{\pgfqpoint{7.842520in}{7.842520in}}%
\pgfusepath{clip}%
\pgfsetbuttcap%
\pgfsetroundjoin%
\definecolor{currentfill}{rgb}{0.140210,0.665859,0.513427}%
\pgfsetfillcolor{currentfill}%
\pgfsetlinewidth{0.000000pt}%
\definecolor{currentstroke}{rgb}{0.191090,0.708366,0.482284}%
\pgfsetstrokecolor{currentstroke}%
\pgfsetdash{}{0pt}%
\pgfpathmoveto{\pgfqpoint{2.267756in}{4.403873in}}%
\pgfpathlineto{\pgfqpoint{2.391827in}{4.772531in}}%
\pgfpathlineto{\pgfqpoint{2.354275in}{4.486045in}}%
\pgfpathclose%
\pgfusepath{fill}%
\end{pgfscope}%
\begin{pgfscope}%
\pgfpathrectangle{\pgfqpoint{0.539299in}{0.078740in}}{\pgfqpoint{7.842520in}{7.842520in}}%
\pgfusepath{clip}%
\pgfsetbuttcap%
\pgfsetroundjoin%
\definecolor{currentfill}{rgb}{0.269308,0.218818,0.509577}%
\pgfsetfillcolor{currentfill}%
\pgfsetlinewidth{0.000000pt}%
\definecolor{currentstroke}{rgb}{0.196571,0.711827,0.479221}%
\pgfsetstrokecolor{currentstroke}%
\pgfsetdash{}{0pt}%
\pgfpathmoveto{\pgfqpoint{5.826491in}{2.723216in}}%
\pgfpathlineto{\pgfqpoint{5.967002in}{2.588084in}}%
\pgfpathlineto{\pgfqpoint{5.902739in}{2.756289in}}%
\pgfpathclose%
\pgfusepath{fill}%
\end{pgfscope}%
\begin{pgfscope}%
\pgfpathrectangle{\pgfqpoint{0.539299in}{0.078740in}}{\pgfqpoint{7.842520in}{7.842520in}}%
\pgfusepath{clip}%
\pgfsetbuttcap%
\pgfsetroundjoin%
\definecolor{currentfill}{rgb}{0.121380,0.629492,0.531973}%
\pgfsetfillcolor{currentfill}%
\pgfsetlinewidth{0.000000pt}%
\definecolor{currentstroke}{rgb}{0.202219,0.715272,0.476084}%
\pgfsetstrokecolor{currentstroke}%
\pgfsetdash{}{0pt}%
\pgfpathmoveto{\pgfqpoint{4.647114in}{4.229077in}}%
\pgfpathlineto{\pgfqpoint{4.507522in}{4.430367in}}%
\pgfpathlineto{\pgfqpoint{4.426173in}{4.535091in}}%
\pgfpathclose%
\pgfusepath{fill}%
\end{pgfscope}%
\begin{pgfscope}%
\pgfpathrectangle{\pgfqpoint{0.539299in}{0.078740in}}{\pgfqpoint{7.842520in}{7.842520in}}%
\pgfusepath{clip}%
\pgfsetbuttcap%
\pgfsetroundjoin%
\definecolor{currentfill}{rgb}{0.283187,0.125848,0.444960}%
\pgfsetfillcolor{currentfill}%
\pgfsetlinewidth{0.000000pt}%
\definecolor{currentstroke}{rgb}{0.208030,0.718701,0.472873}%
\pgfsetstrokecolor{currentstroke}%
\pgfsetdash{}{0pt}%
\pgfpathmoveto{\pgfqpoint{1.388975in}{2.359780in}}%
\pgfpathlineto{\pgfqpoint{1.344269in}{2.477731in}}%
\pgfpathlineto{\pgfqpoint{1.260140in}{2.275047in}}%
\pgfpathclose%
\pgfusepath{fill}%
\end{pgfscope}%
\begin{pgfscope}%
\pgfpathrectangle{\pgfqpoint{0.539299in}{0.078740in}}{\pgfqpoint{7.842520in}{7.842520in}}%
\pgfusepath{clip}%
\pgfsetbuttcap%
\pgfsetroundjoin%
\definecolor{currentfill}{rgb}{0.151918,0.500685,0.557587}%
\pgfsetfillcolor{currentfill}%
\pgfsetlinewidth{0.000000pt}%
\definecolor{currentstroke}{rgb}{0.214000,0.722114,0.469588}%
\pgfsetstrokecolor{currentstroke}%
\pgfsetdash{}{0pt}%
\pgfpathmoveto{\pgfqpoint{4.846758in}{3.914972in}}%
\pgfpathlineto{\pgfqpoint{4.986662in}{3.717303in}}%
\pgfpathlineto{\pgfqpoint{4.925899in}{3.838472in}}%
\pgfpathclose%
\pgfusepath{fill}%
\end{pgfscope}%
\begin{pgfscope}%
\pgfpathrectangle{\pgfqpoint{0.539299in}{0.078740in}}{\pgfqpoint{7.842520in}{7.842520in}}%
\pgfusepath{clip}%
\pgfsetbuttcap%
\pgfsetroundjoin%
\definecolor{currentfill}{rgb}{0.278791,0.062145,0.386592}%
\pgfsetfillcolor{currentfill}%
\pgfsetlinewidth{0.000000pt}%
\definecolor{currentstroke}{rgb}{0.220124,0.725509,0.466226}%
\pgfsetstrokecolor{currentstroke}%
\pgfsetdash{}{0pt}%
\pgfpathmoveto{\pgfqpoint{6.681865in}{2.263405in}}%
\pgfpathlineto{\pgfqpoint{6.606910in}{2.181621in}}%
\pgfpathlineto{\pgfqpoint{6.748358in}{2.068409in}}%
\pgfpathclose%
\pgfusepath{fill}%
\end{pgfscope}%
\begin{pgfscope}%
\pgfpathrectangle{\pgfqpoint{0.539299in}{0.078740in}}{\pgfqpoint{7.842520in}{7.842520in}}%
\pgfusepath{clip}%
\pgfsetbuttcap%
\pgfsetroundjoin%
\definecolor{currentfill}{rgb}{0.282623,0.140926,0.457517}%
\pgfsetfillcolor{currentfill}%
\pgfsetlinewidth{0.000000pt}%
\definecolor{currentstroke}{rgb}{0.226397,0.728888,0.462789}%
\pgfsetstrokecolor{currentstroke}%
\pgfsetdash{}{0pt}%
\pgfpathmoveto{\pgfqpoint{6.324525in}{2.404996in}}%
\pgfpathlineto{\pgfqpoint{6.183670in}{2.518396in}}%
\pgfpathlineto{\pgfqpoint{6.248772in}{2.336093in}}%
\pgfpathclose%
\pgfusepath{fill}%
\end{pgfscope}%
\begin{pgfscope}%
\pgfpathrectangle{\pgfqpoint{0.539299in}{0.078740in}}{\pgfqpoint{7.842520in}{7.842520in}}%
\pgfusepath{clip}%
\pgfsetbuttcap%
\pgfsetroundjoin%
\definecolor{currentfill}{rgb}{0.496615,0.826376,0.306377}%
\pgfsetfillcolor{currentfill}%
\pgfsetlinewidth{0.000000pt}%
\definecolor{currentstroke}{rgb}{0.232815,0.732247,0.459277}%
\pgfsetstrokecolor{currentstroke}%
\pgfsetdash{}{0pt}%
\pgfpathmoveto{\pgfqpoint{3.808702in}{5.299454in}}%
\pgfpathlineto{\pgfqpoint{3.669721in}{5.396407in}}%
\pgfpathlineto{\pgfqpoint{3.584365in}{5.478002in}}%
\pgfpathclose%
\pgfusepath{fill}%
\end{pgfscope}%
\begin{pgfscope}%
\pgfpathrectangle{\pgfqpoint{0.539299in}{0.078740in}}{\pgfqpoint{7.842520in}{7.842520in}}%
\pgfusepath{clip}%
\pgfsetbuttcap%
\pgfsetroundjoin%
\definecolor{currentfill}{rgb}{0.159194,0.482237,0.558073}%
\pgfsetfillcolor{currentfill}%
\pgfsetlinewidth{0.000000pt}%
\definecolor{currentstroke}{rgb}{0.239374,0.735588,0.455688}%
\pgfsetstrokecolor{currentstroke}%
\pgfsetdash{}{0pt}%
\pgfpathmoveto{\pgfqpoint{2.009371in}{4.040892in}}%
\pgfpathlineto{\pgfqpoint{1.889628in}{3.636546in}}%
\pgfpathlineto{\pgfqpoint{1.803826in}{3.497087in}}%
\pgfpathclose%
\pgfusepath{fill}%
\end{pgfscope}%
\begin{pgfscope}%
\pgfpathrectangle{\pgfqpoint{0.539299in}{0.078740in}}{\pgfqpoint{7.842520in}{7.842520in}}%
\pgfusepath{clip}%
\pgfsetbuttcap%
\pgfsetroundjoin%
\definecolor{currentfill}{rgb}{0.137339,0.662252,0.515571}%
\pgfsetfillcolor{currentfill}%
\pgfsetlinewidth{0.000000pt}%
\definecolor{currentstroke}{rgb}{0.246070,0.738910,0.452024}%
\pgfsetstrokecolor{currentstroke}%
\pgfsetdash{}{0pt}%
\pgfpathmoveto{\pgfqpoint{4.507522in}{4.430367in}}%
\pgfpathlineto{\pgfqpoint{4.367781in}{4.630418in}}%
\pgfpathlineto{\pgfqpoint{4.426173in}{4.535091in}}%
\pgfpathclose%
\pgfusepath{fill}%
\end{pgfscope}%
\begin{pgfscope}%
\pgfpathrectangle{\pgfqpoint{0.539299in}{0.078740in}}{\pgfqpoint{7.842520in}{7.842520in}}%
\pgfusepath{clip}%
\pgfsetbuttcap%
\pgfsetroundjoin%
\definecolor{currentfill}{rgb}{0.139147,0.533812,0.555298}%
\pgfsetfillcolor{currentfill}%
\pgfsetlinewidth{0.000000pt}%
\definecolor{currentstroke}{rgb}{0.252899,0.742211,0.448284}%
\pgfsetstrokecolor{currentstroke}%
\pgfsetdash{}{0pt}%
\pgfpathmoveto{\pgfqpoint{4.925899in}{3.838472in}}%
\pgfpathlineto{\pgfqpoint{4.706729in}{4.119068in}}%
\pgfpathlineto{\pgfqpoint{4.846758in}{3.914972in}}%
\pgfpathclose%
\pgfusepath{fill}%
\end{pgfscope}%
\begin{pgfscope}%
\pgfpathrectangle{\pgfqpoint{0.539299in}{0.078740in}}{\pgfqpoint{7.842520in}{7.842520in}}%
\pgfusepath{clip}%
\pgfsetbuttcap%
\pgfsetroundjoin%
\definecolor{currentfill}{rgb}{0.246811,0.283237,0.535941}%
\pgfsetfillcolor{currentfill}%
\pgfsetlinewidth{0.000000pt}%
\definecolor{currentstroke}{rgb}{0.259857,0.745492,0.444467}%
\pgfsetstrokecolor{currentstroke}%
\pgfsetdash{}{0pt}%
\pgfpathmoveto{\pgfqpoint{1.429595in}{2.648382in}}%
\pgfpathlineto{\pgfqpoint{1.550510in}{2.929884in}}%
\pgfpathlineto{\pgfqpoint{1.633975in}{3.148721in}}%
\pgfpathclose%
\pgfusepath{fill}%
\end{pgfscope}%
\begin{pgfscope}%
\pgfpathrectangle{\pgfqpoint{0.539299in}{0.078740in}}{\pgfqpoint{7.842520in}{7.842520in}}%
\pgfusepath{clip}%
\pgfsetbuttcap%
\pgfsetroundjoin%
\definecolor{currentfill}{rgb}{0.555484,0.840254,0.269281}%
\pgfsetfillcolor{currentfill}%
\pgfsetlinewidth{0.000000pt}%
\definecolor{currentstroke}{rgb}{0.266941,0.748751,0.440573}%
\pgfsetstrokecolor{currentstroke}%
\pgfsetdash{}{0pt}%
\pgfpathmoveto{\pgfqpoint{3.584365in}{5.478002in}}%
\pgfpathlineto{\pgfqpoint{3.531606in}{5.447979in}}%
\pgfpathlineto{\pgfqpoint{3.445912in}{5.515993in}}%
\pgfpathclose%
\pgfusepath{fill}%
\end{pgfscope}%
\begin{pgfscope}%
\pgfpathrectangle{\pgfqpoint{0.539299in}{0.078740in}}{\pgfqpoint{7.842520in}{7.842520in}}%
\pgfusepath{clip}%
\pgfsetbuttcap%
\pgfsetroundjoin%
\definecolor{currentfill}{rgb}{0.283229,0.120777,0.440584}%
\pgfsetfillcolor{currentfill}%
\pgfsetlinewidth{0.000000pt}%
\definecolor{currentstroke}{rgb}{0.274149,0.751988,0.436601}%
\pgfsetstrokecolor{currentstroke}%
\pgfsetdash{}{0pt}%
\pgfpathmoveto{\pgfqpoint{6.324525in}{2.404996in}}%
\pgfpathlineto{\pgfqpoint{6.248772in}{2.336093in}}%
\pgfpathlineto{\pgfqpoint{6.465616in}{2.293258in}}%
\pgfpathclose%
\pgfusepath{fill}%
\end{pgfscope}%
\begin{pgfscope}%
\pgfpathrectangle{\pgfqpoint{0.539299in}{0.078740in}}{\pgfqpoint{7.842520in}{7.842520in}}%
\pgfusepath{clip}%
\pgfsetbuttcap%
\pgfsetroundjoin%
\definecolor{currentfill}{rgb}{0.278826,0.175490,0.483397}%
\pgfsetfillcolor{currentfill}%
\pgfsetlinewidth{0.000000pt}%
\definecolor{currentstroke}{rgb}{0.281477,0.755203,0.432552}%
\pgfsetstrokecolor{currentstroke}%
\pgfsetdash{}{0pt}%
\pgfpathmoveto{\pgfqpoint{6.183670in}{2.518396in}}%
\pgfpathlineto{\pgfqpoint{5.967002in}{2.588084in}}%
\pgfpathlineto{\pgfqpoint{6.107758in}{2.459413in}}%
\pgfpathclose%
\pgfusepath{fill}%
\end{pgfscope}%
\begin{pgfscope}%
\pgfpathrectangle{\pgfqpoint{0.539299in}{0.078740in}}{\pgfqpoint{7.842520in}{7.842520in}}%
\pgfusepath{clip}%
\pgfsetbuttcap%
\pgfsetroundjoin%
\definecolor{currentfill}{rgb}{0.565498,0.842430,0.262877}%
\pgfsetfillcolor{currentfill}%
\pgfsetlinewidth{0.000000pt}%
\definecolor{currentstroke}{rgb}{0.288921,0.758394,0.428426}%
\pgfsetstrokecolor{currentstroke}%
\pgfsetdash{}{0pt}%
\pgfpathmoveto{\pgfqpoint{3.222442in}{5.537820in}}%
\pgfpathlineto{\pgfqpoint{3.308861in}{5.497668in}}%
\pgfpathlineto{\pgfqpoint{3.087266in}{5.438839in}}%
\pgfpathclose%
\pgfusepath{fill}%
\end{pgfscope}%
\begin{pgfscope}%
\pgfpathrectangle{\pgfqpoint{0.539299in}{0.078740in}}{\pgfqpoint{7.842520in}{7.842520in}}%
\pgfusepath{clip}%
\pgfsetbuttcap%
\pgfsetroundjoin%
\definecolor{currentfill}{rgb}{0.311925,0.767822,0.415586}%
\pgfsetfillcolor{currentfill}%
\pgfsetlinewidth{0.000000pt}%
\definecolor{currentstroke}{rgb}{0.296479,0.761561,0.424223}%
\pgfsetstrokecolor{currentstroke}%
\pgfsetdash{}{0pt}%
\pgfpathmoveto{\pgfqpoint{2.652170in}{4.965747in}}%
\pgfpathlineto{\pgfqpoint{2.565417in}{4.919366in}}%
\pgfpathlineto{\pgfqpoint{2.694035in}{5.204781in}}%
\pgfpathclose%
\pgfusepath{fill}%
\end{pgfscope}%
\begin{pgfscope}%
\pgfpathrectangle{\pgfqpoint{0.539299in}{0.078740in}}{\pgfqpoint{7.842520in}{7.842520in}}%
\pgfusepath{clip}%
\pgfsetbuttcap%
\pgfsetroundjoin%
\definecolor{currentfill}{rgb}{0.123463,0.581687,0.547445}%
\pgfsetfillcolor{currentfill}%
\pgfsetlinewidth{0.000000pt}%
\definecolor{currentstroke}{rgb}{0.304148,0.764704,0.419943}%
\pgfsetstrokecolor{currentstroke}%
\pgfsetdash{}{0pt}%
\pgfpathmoveto{\pgfqpoint{4.566539in}{4.326950in}}%
\pgfpathlineto{\pgfqpoint{4.706729in}{4.119068in}}%
\pgfpathlineto{\pgfqpoint{4.786562in}{4.030722in}}%
\pgfpathclose%
\pgfusepath{fill}%
\end{pgfscope}%
\begin{pgfscope}%
\pgfpathrectangle{\pgfqpoint{0.539299in}{0.078740in}}{\pgfqpoint{7.842520in}{7.842520in}}%
\pgfusepath{clip}%
\pgfsetbuttcap%
\pgfsetroundjoin%
\definecolor{currentfill}{rgb}{0.281887,0.150881,0.465405}%
\pgfsetfillcolor{currentfill}%
\pgfsetlinewidth{0.000000pt}%
\definecolor{currentstroke}{rgb}{0.311925,0.767822,0.415586}%
\pgfsetstrokecolor{currentstroke}%
\pgfsetdash{}{0pt}%
\pgfpathmoveto{\pgfqpoint{6.248772in}{2.336093in}}%
\pgfpathlineto{\pgfqpoint{6.183670in}{2.518396in}}%
\pgfpathlineto{\pgfqpoint{6.107758in}{2.459413in}}%
\pgfpathclose%
\pgfusepath{fill}%
\end{pgfscope}%
\begin{pgfscope}%
\pgfpathrectangle{\pgfqpoint{0.539299in}{0.078740in}}{\pgfqpoint{7.842520in}{7.842520in}}%
\pgfusepath{clip}%
\pgfsetbuttcap%
\pgfsetroundjoin%
\definecolor{currentfill}{rgb}{0.214000,0.722114,0.469588}%
\pgfsetfillcolor{currentfill}%
\pgfsetlinewidth{0.000000pt}%
\definecolor{currentstroke}{rgb}{0.319809,0.770914,0.411152}%
\pgfsetstrokecolor{currentstroke}%
\pgfsetdash{}{0pt}%
\pgfpathmoveto{\pgfqpoint{4.145020in}{4.933425in}}%
\pgfpathlineto{\pgfqpoint{4.367781in}{4.630418in}}%
\pgfpathlineto{\pgfqpoint{4.227924in}{4.824025in}}%
\pgfpathclose%
\pgfusepath{fill}%
\end{pgfscope}%
\begin{pgfscope}%
\pgfpathrectangle{\pgfqpoint{0.539299in}{0.078740in}}{\pgfqpoint{7.842520in}{7.842520in}}%
\pgfusepath{clip}%
\pgfsetbuttcap%
\pgfsetroundjoin%
\definecolor{currentfill}{rgb}{0.449368,0.813768,0.335384}%
\pgfsetfillcolor{currentfill}%
\pgfsetlinewidth{0.000000pt}%
\definecolor{currentstroke}{rgb}{0.327796,0.773980,0.406640}%
\pgfsetstrokecolor{currentstroke}%
\pgfsetdash{}{0pt}%
\pgfpathmoveto{\pgfqpoint{3.723815in}{5.392306in}}%
\pgfpathlineto{\pgfqpoint{3.948223in}{5.166109in}}%
\pgfpathlineto{\pgfqpoint{3.808702in}{5.299454in}}%
\pgfpathclose%
\pgfusepath{fill}%
\end{pgfscope}%
\begin{pgfscope}%
\pgfpathrectangle{\pgfqpoint{0.539299in}{0.078740in}}{\pgfqpoint{7.842520in}{7.842520in}}%
\pgfusepath{clip}%
\pgfsetbuttcap%
\pgfsetroundjoin%
\definecolor{currentfill}{rgb}{0.278826,0.175490,0.483397}%
\pgfsetfillcolor{currentfill}%
\pgfsetlinewidth{0.000000pt}%
\definecolor{currentstroke}{rgb}{0.335885,0.777018,0.402049}%
\pgfsetstrokecolor{currentstroke}%
\pgfsetdash{}{0pt}%
\pgfpathmoveto{\pgfqpoint{1.468606in}{2.671104in}}%
\pgfpathlineto{\pgfqpoint{1.344269in}{2.477731in}}%
\pgfpathlineto{\pgfqpoint{1.388975in}{2.359780in}}%
\pgfpathclose%
\pgfusepath{fill}%
\end{pgfscope}%
\begin{pgfscope}%
\pgfpathrectangle{\pgfqpoint{0.539299in}{0.078740in}}{\pgfqpoint{7.842520in}{7.842520in}}%
\pgfusepath{clip}%
\pgfsetbuttcap%
\pgfsetroundjoin%
\definecolor{currentfill}{rgb}{0.296479,0.761561,0.424223}%
\pgfsetfillcolor{currentfill}%
\pgfsetlinewidth{0.000000pt}%
\definecolor{currentstroke}{rgb}{0.344074,0.780029,0.397381}%
\pgfsetstrokecolor{currentstroke}%
\pgfsetdash{}{0pt}%
\pgfpathmoveto{\pgfqpoint{4.227924in}{4.824025in}}%
\pgfpathlineto{\pgfqpoint{4.088028in}{5.004974in}}%
\pgfpathlineto{\pgfqpoint{4.004391in}{5.111952in}}%
\pgfpathclose%
\pgfusepath{fill}%
\end{pgfscope}%
\begin{pgfscope}%
\pgfpathrectangle{\pgfqpoint{0.539299in}{0.078740in}}{\pgfqpoint{7.842520in}{7.842520in}}%
\pgfusepath{clip}%
\pgfsetbuttcap%
\pgfsetroundjoin%
\definecolor{currentfill}{rgb}{0.266580,0.228262,0.514349}%
\pgfsetfillcolor{currentfill}%
\pgfsetlinewidth{0.000000pt}%
\definecolor{currentstroke}{rgb}{0.352360,0.783011,0.392636}%
\pgfsetstrokecolor{currentstroke}%
\pgfsetdash{}{0pt}%
\pgfpathmoveto{\pgfqpoint{1.550510in}{2.929884in}}%
\pgfpathlineto{\pgfqpoint{1.344269in}{2.477731in}}%
\pgfpathlineto{\pgfqpoint{1.468606in}{2.671104in}}%
\pgfpathclose%
\pgfusepath{fill}%
\end{pgfscope}%
\begin{pgfscope}%
\pgfpathrectangle{\pgfqpoint{0.539299in}{0.078740in}}{\pgfqpoint{7.842520in}{7.842520in}}%
\pgfusepath{clip}%
\pgfsetbuttcap%
\pgfsetroundjoin%
\definecolor{currentfill}{rgb}{0.185783,0.704891,0.485273}%
\pgfsetfillcolor{currentfill}%
\pgfsetlinewidth{0.000000pt}%
\definecolor{currentstroke}{rgb}{0.360741,0.785964,0.387814}%
\pgfsetstrokecolor{currentstroke}%
\pgfsetdash{}{0pt}%
\pgfpathmoveto{\pgfqpoint{2.354275in}{4.486045in}}%
\pgfpathlineto{\pgfqpoint{2.391827in}{4.772531in}}%
\pgfpathlineto{\pgfqpoint{2.478591in}{4.855876in}}%
\pgfpathclose%
\pgfusepath{fill}%
\end{pgfscope}%
\begin{pgfscope}%
\pgfpathrectangle{\pgfqpoint{0.539299in}{0.078740in}}{\pgfqpoint{7.842520in}{7.842520in}}%
\pgfusepath{clip}%
\pgfsetbuttcap%
\pgfsetroundjoin%
\definecolor{currentfill}{rgb}{0.352360,0.783011,0.392636}%
\pgfsetfillcolor{currentfill}%
\pgfsetlinewidth{0.000000pt}%
\definecolor{currentstroke}{rgb}{0.369214,0.788888,0.382914}%
\pgfsetstrokecolor{currentstroke}%
\pgfsetdash{}{0pt}%
\pgfpathmoveto{\pgfqpoint{4.088028in}{5.004974in}}%
\pgfpathlineto{\pgfqpoint{3.948223in}{5.166109in}}%
\pgfpathlineto{\pgfqpoint{4.004391in}{5.111952in}}%
\pgfpathclose%
\pgfusepath{fill}%
\end{pgfscope}%
\begin{pgfscope}%
\pgfpathrectangle{\pgfqpoint{0.539299in}{0.078740in}}{\pgfqpoint{7.842520in}{7.842520in}}%
\pgfusepath{clip}%
\pgfsetbuttcap%
\pgfsetroundjoin%
\definecolor{currentfill}{rgb}{0.280894,0.078907,0.402329}%
\pgfsetfillcolor{currentfill}%
\pgfsetlinewidth{0.000000pt}%
\definecolor{currentstroke}{rgb}{0.377779,0.791781,0.377939}%
\pgfsetstrokecolor{currentstroke}%
\pgfsetdash{}{0pt}%
\pgfpathmoveto{\pgfqpoint{6.465616in}{2.293258in}}%
\pgfpathlineto{\pgfqpoint{6.531589in}{2.101217in}}%
\pgfpathlineto{\pgfqpoint{6.606910in}{2.181621in}}%
\pgfpathclose%
\pgfusepath{fill}%
\end{pgfscope}%
\begin{pgfscope}%
\pgfpathrectangle{\pgfqpoint{0.539299in}{0.078740in}}{\pgfqpoint{7.842520in}{7.842520in}}%
\pgfusepath{clip}%
\pgfsetbuttcap%
\pgfsetroundjoin%
\definecolor{currentfill}{rgb}{0.121380,0.629492,0.531973}%
\pgfsetfillcolor{currentfill}%
\pgfsetlinewidth{0.000000pt}%
\definecolor{currentstroke}{rgb}{0.386433,0.794644,0.372886}%
\pgfsetstrokecolor{currentstroke}%
\pgfsetdash{}{0pt}%
\pgfpathmoveto{\pgfqpoint{4.566539in}{4.326950in}}%
\pgfpathlineto{\pgfqpoint{4.647114in}{4.229077in}}%
\pgfpathlineto{\pgfqpoint{4.426173in}{4.535091in}}%
\pgfpathclose%
\pgfusepath{fill}%
\end{pgfscope}%
\begin{pgfscope}%
\pgfpathrectangle{\pgfqpoint{0.539299in}{0.078740in}}{\pgfqpoint{7.842520in}{7.842520in}}%
\pgfusepath{clip}%
\pgfsetbuttcap%
\pgfsetroundjoin%
\definecolor{currentfill}{rgb}{0.134692,0.658636,0.517649}%
\pgfsetfillcolor{currentfill}%
\pgfsetlinewidth{0.000000pt}%
\definecolor{currentstroke}{rgb}{0.395174,0.797475,0.367757}%
\pgfsetstrokecolor{currentstroke}%
\pgfsetdash{}{0pt}%
\pgfpathmoveto{\pgfqpoint{2.181328in}{4.304148in}}%
\pgfpathlineto{\pgfqpoint{2.391827in}{4.772531in}}%
\pgfpathlineto{\pgfqpoint{2.267756in}{4.403873in}}%
\pgfpathclose%
\pgfusepath{fill}%
\end{pgfscope}%
\begin{pgfscope}%
\pgfpathrectangle{\pgfqpoint{0.539299in}{0.078740in}}{\pgfqpoint{7.842520in}{7.842520in}}%
\pgfusepath{clip}%
\pgfsetbuttcap%
\pgfsetroundjoin%
\definecolor{currentfill}{rgb}{0.136408,0.541173,0.554483}%
\pgfsetfillcolor{currentfill}%
\pgfsetlinewidth{0.000000pt}%
\definecolor{currentstroke}{rgb}{0.404001,0.800275,0.362552}%
\pgfsetstrokecolor{currentstroke}%
\pgfsetdash{}{0pt}%
\pgfpathmoveto{\pgfqpoint{1.889628in}{3.636546in}}%
\pgfpathlineto{\pgfqpoint{2.009371in}{4.040892in}}%
\pgfpathlineto{\pgfqpoint{2.095141in}{4.184186in}}%
\pgfpathclose%
\pgfusepath{fill}%
\end{pgfscope}%
\begin{pgfscope}%
\pgfpathrectangle{\pgfqpoint{0.539299in}{0.078740in}}{\pgfqpoint{7.842520in}{7.842520in}}%
\pgfusepath{clip}%
\pgfsetbuttcap%
\pgfsetroundjoin%
\definecolor{currentfill}{rgb}{0.283091,0.110553,0.431554}%
\pgfsetfillcolor{currentfill}%
\pgfsetlinewidth{0.000000pt}%
\definecolor{currentstroke}{rgb}{0.412913,0.803041,0.357269}%
\pgfsetstrokecolor{currentstroke}%
\pgfsetdash{}{0pt}%
\pgfpathmoveto{\pgfqpoint{6.465616in}{2.293258in}}%
\pgfpathlineto{\pgfqpoint{6.248772in}{2.336093in}}%
\pgfpathlineto{\pgfqpoint{6.390049in}{2.217039in}}%
\pgfpathclose%
\pgfusepath{fill}%
\end{pgfscope}%
\begin{pgfscope}%
\pgfpathrectangle{\pgfqpoint{0.539299in}{0.078740in}}{\pgfqpoint{7.842520in}{7.842520in}}%
\pgfusepath{clip}%
\pgfsetbuttcap%
\pgfsetroundjoin%
\definecolor{currentfill}{rgb}{0.377779,0.791781,0.377939}%
\pgfsetfillcolor{currentfill}%
\pgfsetlinewidth{0.000000pt}%
\definecolor{currentstroke}{rgb}{0.421908,0.805774,0.351910}%
\pgfsetstrokecolor{currentstroke}%
\pgfsetdash{}{0pt}%
\pgfpathmoveto{\pgfqpoint{2.781099in}{5.242417in}}%
\pgfpathlineto{\pgfqpoint{2.652170in}{4.965747in}}%
\pgfpathlineto{\pgfqpoint{2.694035in}{5.204781in}}%
\pgfpathclose%
\pgfusepath{fill}%
\end{pgfscope}%
\begin{pgfscope}%
\pgfpathrectangle{\pgfqpoint{0.539299in}{0.078740in}}{\pgfqpoint{7.842520in}{7.842520in}}%
\pgfusepath{clip}%
\pgfsetbuttcap%
\pgfsetroundjoin%
\definecolor{currentfill}{rgb}{0.241237,0.296485,0.539709}%
\pgfsetfillcolor{currentfill}%
\pgfsetlinewidth{0.000000pt}%
\definecolor{currentstroke}{rgb}{0.430983,0.808473,0.346476}%
\pgfsetstrokecolor{currentstroke}%
\pgfsetdash{}{0pt}%
\pgfpathmoveto{\pgfqpoint{5.686202in}{2.865893in}}%
\pgfpathlineto{\pgfqpoint{5.546103in}{3.017088in}}%
\pgfpathlineto{\pgfqpoint{5.609380in}{2.861680in}}%
\pgfpathclose%
\pgfusepath{fill}%
\end{pgfscope}%
\begin{pgfscope}%
\pgfpathrectangle{\pgfqpoint{0.539299in}{0.078740in}}{\pgfqpoint{7.842520in}{7.842520in}}%
\pgfusepath{clip}%
\pgfsetbuttcap%
\pgfsetroundjoin%
\definecolor{currentfill}{rgb}{0.277941,0.056324,0.381191}%
\pgfsetfillcolor{currentfill}%
\pgfsetlinewidth{0.000000pt}%
\definecolor{currentstroke}{rgb}{0.440137,0.811138,0.340967}%
\pgfsetstrokecolor{currentstroke}%
\pgfsetdash{}{0pt}%
\pgfpathmoveto{\pgfqpoint{6.606910in}{2.181621in}}%
\pgfpathlineto{\pgfqpoint{6.531589in}{2.101217in}}%
\pgfpathlineto{\pgfqpoint{6.748358in}{2.068409in}}%
\pgfpathclose%
\pgfusepath{fill}%
\end{pgfscope}%
\begin{pgfscope}%
\pgfpathrectangle{\pgfqpoint{0.539299in}{0.078740in}}{\pgfqpoint{7.842520in}{7.842520in}}%
\pgfusepath{clip}%
\pgfsetbuttcap%
\pgfsetroundjoin%
\definecolor{currentfill}{rgb}{0.216210,0.351535,0.550627}%
\pgfsetfillcolor{currentfill}%
\pgfsetlinewidth{0.000000pt}%
\definecolor{currentstroke}{rgb}{0.449368,0.813768,0.335384}%
\pgfsetstrokecolor{currentstroke}%
\pgfsetdash{}{0pt}%
\pgfpathmoveto{\pgfqpoint{5.546103in}{3.017088in}}%
\pgfpathlineto{\pgfqpoint{5.406152in}{3.177575in}}%
\pgfpathlineto{\pgfqpoint{5.328594in}{3.205423in}}%
\pgfpathclose%
\pgfusepath{fill}%
\end{pgfscope}%
\begin{pgfscope}%
\pgfpathrectangle{\pgfqpoint{0.539299in}{0.078740in}}{\pgfqpoint{7.842520in}{7.842520in}}%
\pgfusepath{clip}%
\pgfsetbuttcap%
\pgfsetroundjoin%
\definecolor{currentfill}{rgb}{0.203063,0.379716,0.553925}%
\pgfsetfillcolor{currentfill}%
\pgfsetlinewidth{0.000000pt}%
\definecolor{currentstroke}{rgb}{0.458674,0.816363,0.329727}%
\pgfsetstrokecolor{currentstroke}%
\pgfsetdash{}{0pt}%
\pgfpathmoveto{\pgfqpoint{5.406152in}{3.177575in}}%
\pgfpathlineto{\pgfqpoint{5.266300in}{3.347822in}}%
\pgfpathlineto{\pgfqpoint{5.328594in}{3.205423in}}%
\pgfpathclose%
\pgfusepath{fill}%
\end{pgfscope}%
\begin{pgfscope}%
\pgfpathrectangle{\pgfqpoint{0.539299in}{0.078740in}}{\pgfqpoint{7.842520in}{7.842520in}}%
\pgfusepath{clip}%
\pgfsetbuttcap%
\pgfsetroundjoin%
\definecolor{currentfill}{rgb}{0.252194,0.269783,0.531579}%
\pgfsetfillcolor{currentfill}%
\pgfsetlinewidth{0.000000pt}%
\definecolor{currentstroke}{rgb}{0.468053,0.818921,0.323998}%
\pgfsetstrokecolor{currentstroke}%
\pgfsetdash{}{0pt}%
\pgfpathmoveto{\pgfqpoint{5.609380in}{2.861680in}}%
\pgfpathlineto{\pgfqpoint{5.826491in}{2.723216in}}%
\pgfpathlineto{\pgfqpoint{5.686202in}{2.865893in}}%
\pgfpathclose%
\pgfusepath{fill}%
\end{pgfscope}%
\begin{pgfscope}%
\pgfpathrectangle{\pgfqpoint{0.539299in}{0.078740in}}{\pgfqpoint{7.842520in}{7.842520in}}%
\pgfusepath{clip}%
\pgfsetbuttcap%
\pgfsetroundjoin%
\definecolor{currentfill}{rgb}{0.281924,0.089666,0.412415}%
\pgfsetfillcolor{currentfill}%
\pgfsetlinewidth{0.000000pt}%
\definecolor{currentstroke}{rgb}{0.477504,0.821444,0.318195}%
\pgfsetstrokecolor{currentstroke}%
\pgfsetdash{}{0pt}%
\pgfpathmoveto{\pgfqpoint{6.390049in}{2.217039in}}%
\pgfpathlineto{\pgfqpoint{6.531589in}{2.101217in}}%
\pgfpathlineto{\pgfqpoint{6.465616in}{2.293258in}}%
\pgfpathclose%
\pgfusepath{fill}%
\end{pgfscope}%
\begin{pgfscope}%
\pgfpathrectangle{\pgfqpoint{0.539299in}{0.078740in}}{\pgfqpoint{7.842520in}{7.842520in}}%
\pgfusepath{clip}%
\pgfsetbuttcap%
\pgfsetroundjoin%
\definecolor{currentfill}{rgb}{0.183898,0.422383,0.556944}%
\pgfsetfillcolor{currentfill}%
\pgfsetlinewidth{0.000000pt}%
\definecolor{currentstroke}{rgb}{0.487026,0.823929,0.312321}%
\pgfsetstrokecolor{currentstroke}%
\pgfsetdash{}{0pt}%
\pgfpathmoveto{\pgfqpoint{5.188263in}{3.391183in}}%
\pgfpathlineto{\pgfqpoint{5.266300in}{3.347822in}}%
\pgfpathlineto{\pgfqpoint{5.126490in}{3.527887in}}%
\pgfpathclose%
\pgfusepath{fill}%
\end{pgfscope}%
\begin{pgfscope}%
\pgfpathrectangle{\pgfqpoint{0.539299in}{0.078740in}}{\pgfqpoint{7.842520in}{7.842520in}}%
\pgfusepath{clip}%
\pgfsetbuttcap%
\pgfsetroundjoin%
\definecolor{currentfill}{rgb}{0.525776,0.833491,0.288127}%
\pgfsetfillcolor{currentfill}%
\pgfsetlinewidth{0.000000pt}%
\definecolor{currentstroke}{rgb}{0.496615,0.826376,0.306377}%
\pgfsetstrokecolor{currentstroke}%
\pgfsetdash{}{0pt}%
\pgfpathmoveto{\pgfqpoint{3.584365in}{5.478002in}}%
\pgfpathlineto{\pgfqpoint{3.723815in}{5.392306in}}%
\pgfpathlineto{\pgfqpoint{3.808702in}{5.299454in}}%
\pgfpathclose%
\pgfusepath{fill}%
\end{pgfscope}%
\begin{pgfscope}%
\pgfpathrectangle{\pgfqpoint{0.539299in}{0.078740in}}{\pgfqpoint{7.842520in}{7.842520in}}%
\pgfusepath{clip}%
\pgfsetbuttcap%
\pgfsetroundjoin%
\definecolor{currentfill}{rgb}{0.170948,0.694384,0.493803}%
\pgfsetfillcolor{currentfill}%
\pgfsetlinewidth{0.000000pt}%
\definecolor{currentstroke}{rgb}{0.506271,0.828786,0.300362}%
\pgfsetstrokecolor{currentstroke}%
\pgfsetdash{}{0pt}%
\pgfpathmoveto{\pgfqpoint{4.426173in}{4.535091in}}%
\pgfpathlineto{\pgfqpoint{4.367781in}{4.630418in}}%
\pgfpathlineto{\pgfqpoint{4.285648in}{4.739044in}}%
\pgfpathclose%
\pgfusepath{fill}%
\end{pgfscope}%
\begin{pgfscope}%
\pgfpathrectangle{\pgfqpoint{0.539299in}{0.078740in}}{\pgfqpoint{7.842520in}{7.842520in}}%
\pgfusepath{clip}%
\pgfsetbuttcap%
\pgfsetroundjoin%
\definecolor{currentfill}{rgb}{0.606045,0.850733,0.236712}%
\pgfsetfillcolor{currentfill}%
\pgfsetlinewidth{0.000000pt}%
\definecolor{currentstroke}{rgb}{0.515992,0.831158,0.294279}%
\pgfsetstrokecolor{currentstroke}%
\pgfsetdash{}{0pt}%
\pgfpathmoveto{\pgfqpoint{3.445912in}{5.515993in}}%
\pgfpathlineto{\pgfqpoint{3.308861in}{5.497668in}}%
\pgfpathlineto{\pgfqpoint{3.359631in}{5.571327in}}%
\pgfpathclose%
\pgfusepath{fill}%
\end{pgfscope}%
\begin{pgfscope}%
\pgfpathrectangle{\pgfqpoint{0.539299in}{0.078740in}}{\pgfqpoint{7.842520in}{7.842520in}}%
\pgfusepath{clip}%
\pgfsetbuttcap%
\pgfsetroundjoin%
\definecolor{currentfill}{rgb}{0.265145,0.232956,0.516599}%
\pgfsetfillcolor{currentfill}%
\pgfsetlinewidth{0.000000pt}%
\definecolor{currentstroke}{rgb}{0.525776,0.833491,0.288127}%
\pgfsetstrokecolor{currentstroke}%
\pgfsetdash{}{0pt}%
\pgfpathmoveto{\pgfqpoint{5.826491in}{2.723216in}}%
\pgfpathlineto{\pgfqpoint{5.749939in}{2.703326in}}%
\pgfpathlineto{\pgfqpoint{5.967002in}{2.588084in}}%
\pgfpathclose%
\pgfusepath{fill}%
\end{pgfscope}%
\begin{pgfscope}%
\pgfpathrectangle{\pgfqpoint{0.539299in}{0.078740in}}{\pgfqpoint{7.842520in}{7.842520in}}%
\pgfusepath{clip}%
\pgfsetbuttcap%
\pgfsetroundjoin%
\definecolor{currentfill}{rgb}{0.175841,0.441290,0.557685}%
\pgfsetfillcolor{currentfill}%
\pgfsetlinewidth{0.000000pt}%
\definecolor{currentstroke}{rgb}{0.535621,0.835785,0.281908}%
\pgfsetstrokecolor{currentstroke}%
\pgfsetdash{}{0pt}%
\pgfpathmoveto{\pgfqpoint{1.803826in}{3.497087in}}%
\pgfpathlineto{\pgfqpoint{1.718526in}{3.335879in}}%
\pgfpathlineto{\pgfqpoint{1.839990in}{3.669004in}}%
\pgfpathclose%
\pgfusepath{fill}%
\end{pgfscope}%
\begin{pgfscope}%
\pgfpathrectangle{\pgfqpoint{0.539299in}{0.078740in}}{\pgfqpoint{7.842520in}{7.842520in}}%
\pgfusepath{clip}%
\pgfsetbuttcap%
\pgfsetroundjoin%
\definecolor{currentfill}{rgb}{0.220124,0.725509,0.466226}%
\pgfsetfillcolor{currentfill}%
\pgfsetlinewidth{0.000000pt}%
\definecolor{currentstroke}{rgb}{0.545524,0.838039,0.275626}%
\pgfsetstrokecolor{currentstroke}%
\pgfsetdash{}{0pt}%
\pgfpathmoveto{\pgfqpoint{4.285648in}{4.739044in}}%
\pgfpathlineto{\pgfqpoint{4.367781in}{4.630418in}}%
\pgfpathlineto{\pgfqpoint{4.145020in}{4.933425in}}%
\pgfpathclose%
\pgfusepath{fill}%
\end{pgfscope}%
\begin{pgfscope}%
\pgfpathrectangle{\pgfqpoint{0.539299in}{0.078740in}}{\pgfqpoint{7.842520in}{7.842520in}}%
\pgfusepath{clip}%
\pgfsetbuttcap%
\pgfsetroundjoin%
\definecolor{currentfill}{rgb}{0.278012,0.180367,0.486697}%
\pgfsetfillcolor{currentfill}%
\pgfsetlinewidth{0.000000pt}%
\definecolor{currentstroke}{rgb}{0.555484,0.840254,0.269281}%
\pgfsetstrokecolor{currentstroke}%
\pgfsetdash{}{0pt}%
\pgfpathmoveto{\pgfqpoint{5.967002in}{2.588084in}}%
\pgfpathlineto{\pgfqpoint{6.031600in}{2.411053in}}%
\pgfpathlineto{\pgfqpoint{6.107758in}{2.459413in}}%
\pgfpathclose%
\pgfusepath{fill}%
\end{pgfscope}%
\begin{pgfscope}%
\pgfpathrectangle{\pgfqpoint{0.539299in}{0.078740in}}{\pgfqpoint{7.842520in}{7.842520in}}%
\pgfusepath{clip}%
\pgfsetbuttcap%
\pgfsetroundjoin%
\definecolor{currentfill}{rgb}{0.157729,0.485932,0.558013}%
\pgfsetfillcolor{currentfill}%
\pgfsetlinewidth{0.000000pt}%
\definecolor{currentstroke}{rgb}{0.565498,0.842430,0.262877}%
\pgfsetstrokecolor{currentstroke}%
\pgfsetdash{}{0pt}%
\pgfpathmoveto{\pgfqpoint{5.126490in}{3.527887in}}%
\pgfpathlineto{\pgfqpoint{4.986662in}{3.717303in}}%
\pgfpathlineto{\pgfqpoint{4.907447in}{3.788480in}}%
\pgfpathclose%
\pgfusepath{fill}%
\end{pgfscope}%
\begin{pgfscope}%
\pgfpathrectangle{\pgfqpoint{0.539299in}{0.078740in}}{\pgfqpoint{7.842520in}{7.842520in}}%
\pgfusepath{clip}%
\pgfsetbuttcap%
\pgfsetroundjoin%
\definecolor{currentfill}{rgb}{0.468053,0.818921,0.323998}%
\pgfsetfillcolor{currentfill}%
\pgfsetlinewidth{0.000000pt}%
\definecolor{currentstroke}{rgb}{0.575563,0.844566,0.256415}%
\pgfsetstrokecolor{currentstroke}%
\pgfsetdash{}{0pt}%
\pgfpathmoveto{\pgfqpoint{3.863917in}{5.267515in}}%
\pgfpathlineto{\pgfqpoint{3.948223in}{5.166109in}}%
\pgfpathlineto{\pgfqpoint{3.723815in}{5.392306in}}%
\pgfpathclose%
\pgfusepath{fill}%
\end{pgfscope}%
\begin{pgfscope}%
\pgfpathrectangle{\pgfqpoint{0.539299in}{0.078740in}}{\pgfqpoint{7.842520in}{7.842520in}}%
\pgfusepath{clip}%
\pgfsetbuttcap%
\pgfsetroundjoin%
\definecolor{currentfill}{rgb}{0.626579,0.854645,0.223353}%
\pgfsetfillcolor{currentfill}%
\pgfsetlinewidth{0.000000pt}%
\definecolor{currentstroke}{rgb}{0.585678,0.846661,0.249897}%
\pgfsetstrokecolor{currentstroke}%
\pgfsetdash{}{0pt}%
\pgfpathmoveto{\pgfqpoint{3.359631in}{5.571327in}}%
\pgfpathlineto{\pgfqpoint{3.308861in}{5.497668in}}%
\pgfpathlineto{\pgfqpoint{3.222442in}{5.537820in}}%
\pgfpathclose%
\pgfusepath{fill}%
\end{pgfscope}%
\begin{pgfscope}%
\pgfpathrectangle{\pgfqpoint{0.539299in}{0.078740in}}{\pgfqpoint{7.842520in}{7.842520in}}%
\pgfusepath{clip}%
\pgfsetbuttcap%
\pgfsetroundjoin%
\definecolor{currentfill}{rgb}{0.319809,0.770914,0.411152}%
\pgfsetfillcolor{currentfill}%
\pgfsetlinewidth{0.000000pt}%
\definecolor{currentstroke}{rgb}{0.595839,0.848717,0.243329}%
\pgfsetstrokecolor{currentstroke}%
\pgfsetdash{}{0pt}%
\pgfpathmoveto{\pgfqpoint{2.694035in}{5.204781in}}%
\pgfpathlineto{\pgfqpoint{2.565417in}{4.919366in}}%
\pgfpathlineto{\pgfqpoint{2.478591in}{4.855876in}}%
\pgfpathclose%
\pgfusepath{fill}%
\end{pgfscope}%
\begin{pgfscope}%
\pgfpathrectangle{\pgfqpoint{0.539299in}{0.078740in}}{\pgfqpoint{7.842520in}{7.842520in}}%
\pgfusepath{clip}%
\pgfsetbuttcap%
\pgfsetroundjoin%
\definecolor{currentfill}{rgb}{0.304148,0.764704,0.419943}%
\pgfsetfillcolor{currentfill}%
\pgfsetlinewidth{0.000000pt}%
\definecolor{currentstroke}{rgb}{0.606045,0.850733,0.236712}%
\pgfsetstrokecolor{currentstroke}%
\pgfsetdash{}{0pt}%
\pgfpathmoveto{\pgfqpoint{4.004391in}{5.111952in}}%
\pgfpathlineto{\pgfqpoint{4.145020in}{4.933425in}}%
\pgfpathlineto{\pgfqpoint{4.227924in}{4.824025in}}%
\pgfpathclose%
\pgfusepath{fill}%
\end{pgfscope}%
\begin{pgfscope}%
\pgfpathrectangle{\pgfqpoint{0.539299in}{0.078740in}}{\pgfqpoint{7.842520in}{7.842520in}}%
\pgfusepath{clip}%
\pgfsetbuttcap%
\pgfsetroundjoin%
\definecolor{currentfill}{rgb}{0.281412,0.155834,0.469201}%
\pgfsetfillcolor{currentfill}%
\pgfsetlinewidth{0.000000pt}%
\definecolor{currentstroke}{rgb}{0.616293,0.852709,0.230052}%
\pgfsetstrokecolor{currentstroke}%
\pgfsetdash{}{0pt}%
\pgfpathmoveto{\pgfqpoint{6.107758in}{2.459413in}}%
\pgfpathlineto{\pgfqpoint{6.031600in}{2.411053in}}%
\pgfpathlineto{\pgfqpoint{6.248772in}{2.336093in}}%
\pgfpathclose%
\pgfusepath{fill}%
\end{pgfscope}%
\begin{pgfscope}%
\pgfpathrectangle{\pgfqpoint{0.539299in}{0.078740in}}{\pgfqpoint{7.842520in}{7.842520in}}%
\pgfusepath{clip}%
\pgfsetbuttcap%
\pgfsetroundjoin%
\definecolor{currentfill}{rgb}{0.146180,0.515413,0.556823}%
\pgfsetfillcolor{currentfill}%
\pgfsetlinewidth{0.000000pt}%
\definecolor{currentstroke}{rgb}{0.626579,0.854645,0.223353}%
\pgfsetstrokecolor{currentstroke}%
\pgfsetdash{}{0pt}%
\pgfpathmoveto{\pgfqpoint{4.907447in}{3.788480in}}%
\pgfpathlineto{\pgfqpoint{4.986662in}{3.717303in}}%
\pgfpathlineto{\pgfqpoint{4.846758in}{3.914972in}}%
\pgfpathclose%
\pgfusepath{fill}%
\end{pgfscope}%
\begin{pgfscope}%
\pgfpathrectangle{\pgfqpoint{0.539299in}{0.078740in}}{\pgfqpoint{7.842520in}{7.842520in}}%
\pgfusepath{clip}%
\pgfsetbuttcap%
\pgfsetroundjoin%
\definecolor{currentfill}{rgb}{0.421908,0.805774,0.351910}%
\pgfsetfillcolor{currentfill}%
\pgfsetlinewidth{0.000000pt}%
\definecolor{currentstroke}{rgb}{0.636902,0.856542,0.216620}%
\pgfsetstrokecolor{currentstroke}%
\pgfsetdash{}{0pt}%
\pgfpathmoveto{\pgfqpoint{4.004391in}{5.111952in}}%
\pgfpathlineto{\pgfqpoint{3.948223in}{5.166109in}}%
\pgfpathlineto{\pgfqpoint{3.863917in}{5.267515in}}%
\pgfpathclose%
\pgfusepath{fill}%
\end{pgfscope}%
\begin{pgfscope}%
\pgfpathrectangle{\pgfqpoint{0.539299in}{0.078740in}}{\pgfqpoint{7.842520in}{7.842520in}}%
\pgfusepath{clip}%
\pgfsetbuttcap%
\pgfsetroundjoin%
\definecolor{currentfill}{rgb}{0.276022,0.044167,0.370164}%
\pgfsetfillcolor{currentfill}%
\pgfsetlinewidth{0.000000pt}%
\definecolor{currentstroke}{rgb}{0.647257,0.858400,0.209861}%
\pgfsetstrokecolor{currentstroke}%
\pgfsetdash{}{0pt}%
\pgfpathmoveto{\pgfqpoint{6.748358in}{2.068409in}}%
\pgfpathlineto{\pgfqpoint{6.531589in}{2.101217in}}%
\pgfpathlineto{\pgfqpoint{6.673382in}{1.987622in}}%
\pgfpathclose%
\pgfusepath{fill}%
\end{pgfscope}%
\begin{pgfscope}%
\pgfpathrectangle{\pgfqpoint{0.539299in}{0.078740in}}{\pgfqpoint{7.842520in}{7.842520in}}%
\pgfusepath{clip}%
\pgfsetbuttcap%
\pgfsetroundjoin%
\definecolor{currentfill}{rgb}{0.496615,0.826376,0.306377}%
\pgfsetfillcolor{currentfill}%
\pgfsetlinewidth{0.000000pt}%
\definecolor{currentstroke}{rgb}{0.657642,0.860219,0.203082}%
\pgfsetstrokecolor{currentstroke}%
\pgfsetdash{}{0pt}%
\pgfpathmoveto{\pgfqpoint{2.913349in}{5.441460in}}%
\pgfpathlineto{\pgfqpoint{2.867996in}{5.262317in}}%
\pgfpathlineto{\pgfqpoint{2.781099in}{5.242417in}}%
\pgfpathclose%
\pgfusepath{fill}%
\end{pgfscope}%
\begin{pgfscope}%
\pgfpathrectangle{\pgfqpoint{0.539299in}{0.078740in}}{\pgfqpoint{7.842520in}{7.842520in}}%
\pgfusepath{clip}%
\pgfsetbuttcap%
\pgfsetroundjoin%
\definecolor{currentfill}{rgb}{0.231674,0.318106,0.544834}%
\pgfsetfillcolor{currentfill}%
\pgfsetlinewidth{0.000000pt}%
\definecolor{currentstroke}{rgb}{0.668054,0.861999,0.196293}%
\pgfsetstrokecolor{currentstroke}%
\pgfsetdash{}{0pt}%
\pgfpathmoveto{\pgfqpoint{5.609380in}{2.861680in}}%
\pgfpathlineto{\pgfqpoint{5.546103in}{3.017088in}}%
\pgfpathlineto{\pgfqpoint{5.468948in}{3.028932in}}%
\pgfpathclose%
\pgfusepath{fill}%
\end{pgfscope}%
\begin{pgfscope}%
\pgfpathrectangle{\pgfqpoint{0.539299in}{0.078740in}}{\pgfqpoint{7.842520in}{7.842520in}}%
\pgfusepath{clip}%
\pgfsetbuttcap%
\pgfsetroundjoin%
\definecolor{currentfill}{rgb}{0.218130,0.347432,0.550038}%
\pgfsetfillcolor{currentfill}%
\pgfsetlinewidth{0.000000pt}%
\definecolor{currentstroke}{rgb}{0.678489,0.863742,0.189503}%
\pgfsetstrokecolor{currentstroke}%
\pgfsetdash{}{0pt}%
\pgfpathmoveto{\pgfqpoint{5.468948in}{3.028932in}}%
\pgfpathlineto{\pgfqpoint{5.546103in}{3.017088in}}%
\pgfpathlineto{\pgfqpoint{5.328594in}{3.205423in}}%
\pgfpathclose%
\pgfusepath{fill}%
\end{pgfscope}%
\begin{pgfscope}%
\pgfpathrectangle{\pgfqpoint{0.539299in}{0.078740in}}{\pgfqpoint{7.842520in}{7.842520in}}%
\pgfusepath{clip}%
\pgfsetbuttcap%
\pgfsetroundjoin%
\definecolor{currentfill}{rgb}{0.253935,0.265254,0.529983}%
\pgfsetfillcolor{currentfill}%
\pgfsetlinewidth{0.000000pt}%
\definecolor{currentstroke}{rgb}{0.688944,0.865448,0.182725}%
\pgfsetstrokecolor{currentstroke}%
\pgfsetdash{}{0pt}%
\pgfpathmoveto{\pgfqpoint{5.609380in}{2.861680in}}%
\pgfpathlineto{\pgfqpoint{5.749939in}{2.703326in}}%
\pgfpathlineto{\pgfqpoint{5.826491in}{2.723216in}}%
\pgfpathclose%
\pgfusepath{fill}%
\end{pgfscope}%
\begin{pgfscope}%
\pgfpathrectangle{\pgfqpoint{0.539299in}{0.078740in}}{\pgfqpoint{7.842520in}{7.842520in}}%
\pgfusepath{clip}%
\pgfsetbuttcap%
\pgfsetroundjoin%
\definecolor{currentfill}{rgb}{0.146180,0.515413,0.556823}%
\pgfsetfillcolor{currentfill}%
\pgfsetlinewidth{0.000000pt}%
\definecolor{currentstroke}{rgb}{0.699415,0.867117,0.175971}%
\pgfsetstrokecolor{currentstroke}%
\pgfsetdash{}{0pt}%
\pgfpathmoveto{\pgfqpoint{1.803826in}{3.497087in}}%
\pgfpathlineto{\pgfqpoint{1.924234in}{3.870625in}}%
\pgfpathlineto{\pgfqpoint{2.009371in}{4.040892in}}%
\pgfpathclose%
\pgfusepath{fill}%
\end{pgfscope}%
\begin{pgfscope}%
\pgfpathrectangle{\pgfqpoint{0.539299in}{0.078740in}}{\pgfqpoint{7.842520in}{7.842520in}}%
\pgfusepath{clip}%
\pgfsetbuttcap%
\pgfsetroundjoin%
\definecolor{currentfill}{rgb}{0.535621,0.835785,0.281908}%
\pgfsetfillcolor{currentfill}%
\pgfsetlinewidth{0.000000pt}%
\definecolor{currentstroke}{rgb}{0.709898,0.868751,0.169257}%
\pgfsetstrokecolor{currentstroke}%
\pgfsetdash{}{0pt}%
\pgfpathmoveto{\pgfqpoint{3.000456in}{5.448722in}}%
\pgfpathlineto{\pgfqpoint{2.867996in}{5.262317in}}%
\pgfpathlineto{\pgfqpoint{2.913349in}{5.441460in}}%
\pgfpathclose%
\pgfusepath{fill}%
\end{pgfscope}%
\begin{pgfscope}%
\pgfpathrectangle{\pgfqpoint{0.539299in}{0.078740in}}{\pgfqpoint{7.842520in}{7.842520in}}%
\pgfusepath{clip}%
\pgfsetbuttcap%
\pgfsetroundjoin%
\definecolor{currentfill}{rgb}{0.192357,0.403199,0.555836}%
\pgfsetfillcolor{currentfill}%
\pgfsetlinewidth{0.000000pt}%
\definecolor{currentstroke}{rgb}{0.720391,0.870350,0.162603}%
\pgfsetstrokecolor{currentstroke}%
\pgfsetdash{}{0pt}%
\pgfpathmoveto{\pgfqpoint{1.718526in}{3.335879in}}%
\pgfpathlineto{\pgfqpoint{1.633975in}{3.148721in}}%
\pgfpathlineto{\pgfqpoint{1.756965in}{3.430597in}}%
\pgfpathclose%
\pgfusepath{fill}%
\end{pgfscope}%
\begin{pgfscope}%
\pgfpathrectangle{\pgfqpoint{0.539299in}{0.078740in}}{\pgfqpoint{7.842520in}{7.842520in}}%
\pgfusepath{clip}%
\pgfsetbuttcap%
\pgfsetroundjoin%
\definecolor{currentfill}{rgb}{0.190631,0.407061,0.556089}%
\pgfsetfillcolor{currentfill}%
\pgfsetlinewidth{0.000000pt}%
\definecolor{currentstroke}{rgb}{0.730889,0.871916,0.156029}%
\pgfsetstrokecolor{currentstroke}%
\pgfsetdash{}{0pt}%
\pgfpathmoveto{\pgfqpoint{5.266300in}{3.347822in}}%
\pgfpathlineto{\pgfqpoint{5.188263in}{3.391183in}}%
\pgfpathlineto{\pgfqpoint{5.328594in}{3.205423in}}%
\pgfpathclose%
\pgfusepath{fill}%
\end{pgfscope}%
\begin{pgfscope}%
\pgfpathrectangle{\pgfqpoint{0.539299in}{0.078740in}}{\pgfqpoint{7.842520in}{7.842520in}}%
\pgfusepath{clip}%
\pgfsetbuttcap%
\pgfsetroundjoin%
\definecolor{currentfill}{rgb}{0.267968,0.223549,0.512008}%
\pgfsetfillcolor{currentfill}%
\pgfsetlinewidth{0.000000pt}%
\definecolor{currentstroke}{rgb}{0.741388,0.873449,0.149561}%
\pgfsetstrokecolor{currentstroke}%
\pgfsetdash{}{0pt}%
\pgfpathmoveto{\pgfqpoint{5.749939in}{2.703326in}}%
\pgfpathlineto{\pgfqpoint{5.890666in}{2.553330in}}%
\pgfpathlineto{\pgfqpoint{5.967002in}{2.588084in}}%
\pgfpathclose%
\pgfusepath{fill}%
\end{pgfscope}%
\begin{pgfscope}%
\pgfpathrectangle{\pgfqpoint{0.539299in}{0.078740in}}{\pgfqpoint{7.842520in}{7.842520in}}%
\pgfusepath{clip}%
\pgfsetbuttcap%
\pgfsetroundjoin%
\definecolor{currentfill}{rgb}{0.626579,0.854645,0.223353}%
\pgfsetfillcolor{currentfill}%
\pgfsetlinewidth{0.000000pt}%
\definecolor{currentstroke}{rgb}{0.751884,0.874951,0.143228}%
\pgfsetstrokecolor{currentstroke}%
\pgfsetdash{}{0pt}%
\pgfpathmoveto{\pgfqpoint{3.445912in}{5.515993in}}%
\pgfpathlineto{\pgfqpoint{3.359631in}{5.571327in}}%
\pgfpathlineto{\pgfqpoint{3.584365in}{5.478002in}}%
\pgfpathclose%
\pgfusepath{fill}%
\end{pgfscope}%
\begin{pgfscope}%
\pgfpathrectangle{\pgfqpoint{0.539299in}{0.078740in}}{\pgfqpoint{7.842520in}{7.842520in}}%
\pgfusepath{clip}%
\pgfsetbuttcap%
\pgfsetroundjoin%
\definecolor{currentfill}{rgb}{0.616293,0.852709,0.230052}%
\pgfsetfillcolor{currentfill}%
\pgfsetlinewidth{0.000000pt}%
\definecolor{currentstroke}{rgb}{0.762373,0.876424,0.137064}%
\pgfsetstrokecolor{currentstroke}%
\pgfsetdash{}{0pt}%
\pgfpathmoveto{\pgfqpoint{3.135579in}{5.562556in}}%
\pgfpathlineto{\pgfqpoint{3.222442in}{5.537820in}}%
\pgfpathlineto{\pgfqpoint{3.087266in}{5.438839in}}%
\pgfpathclose%
\pgfusepath{fill}%
\end{pgfscope}%
\begin{pgfscope}%
\pgfpathrectangle{\pgfqpoint{0.539299in}{0.078740in}}{\pgfqpoint{7.842520in}{7.842520in}}%
\pgfusepath{clip}%
\pgfsetbuttcap%
\pgfsetroundjoin%
\definecolor{currentfill}{rgb}{0.275191,0.194905,0.496005}%
\pgfsetfillcolor{currentfill}%
\pgfsetlinewidth{0.000000pt}%
\definecolor{currentstroke}{rgb}{0.772852,0.877868,0.131109}%
\pgfsetstrokecolor{currentstroke}%
\pgfsetdash{}{0pt}%
\pgfpathmoveto{\pgfqpoint{5.890666in}{2.553330in}}%
\pgfpathlineto{\pgfqpoint{6.031600in}{2.411053in}}%
\pgfpathlineto{\pgfqpoint{5.967002in}{2.588084in}}%
\pgfpathclose%
\pgfusepath{fill}%
\end{pgfscope}%
\begin{pgfscope}%
\pgfpathrectangle{\pgfqpoint{0.539299in}{0.078740in}}{\pgfqpoint{7.842520in}{7.842520in}}%
\pgfusepath{clip}%
\pgfsetbuttcap%
\pgfsetroundjoin%
\definecolor{currentfill}{rgb}{0.283091,0.110553,0.431554}%
\pgfsetfillcolor{currentfill}%
\pgfsetlinewidth{0.000000pt}%
\definecolor{currentstroke}{rgb}{0.783315,0.879285,0.125405}%
\pgfsetstrokecolor{currentstroke}%
\pgfsetdash{}{0pt}%
\pgfpathmoveto{\pgfqpoint{6.248772in}{2.336093in}}%
\pgfpathlineto{\pgfqpoint{6.314217in}{2.147221in}}%
\pgfpathlineto{\pgfqpoint{6.390049in}{2.217039in}}%
\pgfpathclose%
\pgfusepath{fill}%
\end{pgfscope}%
\begin{pgfscope}%
\pgfpathrectangle{\pgfqpoint{0.539299in}{0.078740in}}{\pgfqpoint{7.842520in}{7.842520in}}%
\pgfusepath{clip}%
\pgfsetbuttcap%
\pgfsetroundjoin%
\definecolor{currentfill}{rgb}{0.606045,0.850733,0.236712}%
\pgfsetfillcolor{currentfill}%
\pgfsetlinewidth{0.000000pt}%
\definecolor{currentstroke}{rgb}{0.793760,0.880678,0.120005}%
\pgfsetstrokecolor{currentstroke}%
\pgfsetdash{}{0pt}%
\pgfpathmoveto{\pgfqpoint{3.087266in}{5.438839in}}%
\pgfpathlineto{\pgfqpoint{3.000456in}{5.448722in}}%
\pgfpathlineto{\pgfqpoint{3.135579in}{5.562556in}}%
\pgfpathclose%
\pgfusepath{fill}%
\end{pgfscope}%
\begin{pgfscope}%
\pgfpathrectangle{\pgfqpoint{0.539299in}{0.078740in}}{\pgfqpoint{7.842520in}{7.842520in}}%
\pgfusepath{clip}%
\pgfsetbuttcap%
\pgfsetroundjoin%
\definecolor{currentfill}{rgb}{0.123463,0.581687,0.547445}%
\pgfsetfillcolor{currentfill}%
\pgfsetlinewidth{0.000000pt}%
\definecolor{currentstroke}{rgb}{0.804182,0.882046,0.114965}%
\pgfsetstrokecolor{currentstroke}%
\pgfsetdash{}{0pt}%
\pgfpathmoveto{\pgfqpoint{4.846758in}{3.914972in}}%
\pgfpathlineto{\pgfqpoint{4.706729in}{4.119068in}}%
\pgfpathlineto{\pgfqpoint{4.626092in}{4.211058in}}%
\pgfpathclose%
\pgfusepath{fill}%
\end{pgfscope}%
\begin{pgfscope}%
\pgfpathrectangle{\pgfqpoint{0.539299in}{0.078740in}}{\pgfqpoint{7.842520in}{7.842520in}}%
\pgfusepath{clip}%
\pgfsetbuttcap%
\pgfsetroundjoin%
\definecolor{currentfill}{rgb}{0.171176,0.452530,0.557965}%
\pgfsetfillcolor{currentfill}%
\pgfsetlinewidth{0.000000pt}%
\definecolor{currentstroke}{rgb}{0.814576,0.883393,0.110347}%
\pgfsetstrokecolor{currentstroke}%
\pgfsetdash{}{0pt}%
\pgfpathmoveto{\pgfqpoint{5.126490in}{3.527887in}}%
\pgfpathlineto{\pgfqpoint{5.047899in}{3.585834in}}%
\pgfpathlineto{\pgfqpoint{5.188263in}{3.391183in}}%
\pgfpathclose%
\pgfusepath{fill}%
\end{pgfscope}%
\begin{pgfscope}%
\pgfpathrectangle{\pgfqpoint{0.539299in}{0.078740in}}{\pgfqpoint{7.842520in}{7.842520in}}%
\pgfusepath{clip}%
\pgfsetbuttcap%
\pgfsetroundjoin%
\definecolor{currentfill}{rgb}{0.159194,0.482237,0.558073}%
\pgfsetfillcolor{currentfill}%
\pgfsetlinewidth{0.000000pt}%
\definecolor{currentstroke}{rgb}{0.824940,0.884720,0.106217}%
\pgfsetstrokecolor{currentstroke}%
\pgfsetdash{}{0pt}%
\pgfpathmoveto{\pgfqpoint{4.907447in}{3.788480in}}%
\pgfpathlineto{\pgfqpoint{5.047899in}{3.585834in}}%
\pgfpathlineto{\pgfqpoint{5.126490in}{3.527887in}}%
\pgfpathclose%
\pgfusepath{fill}%
\end{pgfscope}%
\begin{pgfscope}%
\pgfpathrectangle{\pgfqpoint{0.539299in}{0.078740in}}{\pgfqpoint{7.842520in}{7.842520in}}%
\pgfusepath{clip}%
\pgfsetbuttcap%
\pgfsetroundjoin%
\definecolor{currentfill}{rgb}{0.204903,0.375746,0.553533}%
\pgfsetfillcolor{currentfill}%
\pgfsetlinewidth{0.000000pt}%
\definecolor{currentstroke}{rgb}{0.835270,0.886029,0.102646}%
\pgfsetstrokecolor{currentstroke}%
\pgfsetdash{}{0pt}%
\pgfpathmoveto{\pgfqpoint{1.756965in}{3.430597in}}%
\pgfpathlineto{\pgfqpoint{1.633975in}{3.148721in}}%
\pgfpathlineto{\pgfqpoint{1.550510in}{2.929884in}}%
\pgfpathclose%
\pgfusepath{fill}%
\end{pgfscope}%
\begin{pgfscope}%
\pgfpathrectangle{\pgfqpoint{0.539299in}{0.078740in}}{\pgfqpoint{7.842520in}{7.842520in}}%
\pgfusepath{clip}%
\pgfsetbuttcap%
\pgfsetroundjoin%
\definecolor{currentfill}{rgb}{0.154815,0.493313,0.557840}%
\pgfsetfillcolor{currentfill}%
\pgfsetlinewidth{0.000000pt}%
\definecolor{currentstroke}{rgb}{0.845561,0.887322,0.099702}%
\pgfsetstrokecolor{currentstroke}%
\pgfsetdash{}{0pt}%
\pgfpathmoveto{\pgfqpoint{1.839990in}{3.669004in}}%
\pgfpathlineto{\pgfqpoint{1.924234in}{3.870625in}}%
\pgfpathlineto{\pgfqpoint{1.803826in}{3.497087in}}%
\pgfpathclose%
\pgfusepath{fill}%
\end{pgfscope}%
\begin{pgfscope}%
\pgfpathrectangle{\pgfqpoint{0.539299in}{0.078740in}}{\pgfqpoint{7.842520in}{7.842520in}}%
\pgfusepath{clip}%
\pgfsetbuttcap%
\pgfsetroundjoin%
\definecolor{currentfill}{rgb}{0.281887,0.150881,0.465405}%
\pgfsetfillcolor{currentfill}%
\pgfsetlinewidth{0.000000pt}%
\definecolor{currentstroke}{rgb}{0.855810,0.888601,0.097452}%
\pgfsetstrokecolor{currentstroke}%
\pgfsetdash{}{0pt}%
\pgfpathmoveto{\pgfqpoint{6.248772in}{2.336093in}}%
\pgfpathlineto{\pgfqpoint{6.031600in}{2.411053in}}%
\pgfpathlineto{\pgfqpoint{6.172774in}{2.275861in}}%
\pgfpathclose%
\pgfusepath{fill}%
\end{pgfscope}%
\begin{pgfscope}%
\pgfpathrectangle{\pgfqpoint{0.539299in}{0.078740in}}{\pgfqpoint{7.842520in}{7.842520in}}%
\pgfusepath{clip}%
\pgfsetbuttcap%
\pgfsetroundjoin%
\definecolor{currentfill}{rgb}{0.266580,0.228262,0.514349}%
\pgfsetfillcolor{currentfill}%
\pgfsetlinewidth{0.000000pt}%
\definecolor{currentstroke}{rgb}{0.866013,0.889868,0.095953}%
\pgfsetstrokecolor{currentstroke}%
\pgfsetdash{}{0pt}%
\pgfpathmoveto{\pgfqpoint{1.468606in}{2.671104in}}%
\pgfpathlineto{\pgfqpoint{1.388975in}{2.359780in}}%
\pgfpathlineto{\pgfqpoint{1.596362in}{2.813436in}}%
\pgfpathclose%
\pgfusepath{fill}%
\end{pgfscope}%
\begin{pgfscope}%
\pgfpathrectangle{\pgfqpoint{0.539299in}{0.078740in}}{\pgfqpoint{7.842520in}{7.842520in}}%
\pgfusepath{clip}%
\pgfsetbuttcap%
\pgfsetroundjoin%
\definecolor{currentfill}{rgb}{0.280894,0.078907,0.402329}%
\pgfsetfillcolor{currentfill}%
\pgfsetlinewidth{0.000000pt}%
\definecolor{currentstroke}{rgb}{0.876168,0.891125,0.095250}%
\pgfsetstrokecolor{currentstroke}%
\pgfsetdash{}{0pt}%
\pgfpathmoveto{\pgfqpoint{6.455962in}{2.024776in}}%
\pgfpathlineto{\pgfqpoint{6.531589in}{2.101217in}}%
\pgfpathlineto{\pgfqpoint{6.390049in}{2.217039in}}%
\pgfpathclose%
\pgfusepath{fill}%
\end{pgfscope}%
\begin{pgfscope}%
\pgfpathrectangle{\pgfqpoint{0.539299in}{0.078740in}}{\pgfqpoint{7.842520in}{7.842520in}}%
\pgfusepath{clip}%
\pgfsetbuttcap%
\pgfsetroundjoin%
\definecolor{currentfill}{rgb}{0.121380,0.629492,0.531973}%
\pgfsetfillcolor{currentfill}%
\pgfsetlinewidth{0.000000pt}%
\definecolor{currentstroke}{rgb}{0.886271,0.892374,0.095374}%
\pgfsetstrokecolor{currentstroke}%
\pgfsetdash{}{0pt}%
\pgfpathmoveto{\pgfqpoint{4.485130in}{4.425835in}}%
\pgfpathlineto{\pgfqpoint{4.706729in}{4.119068in}}%
\pgfpathlineto{\pgfqpoint{4.566539in}{4.326950in}}%
\pgfpathclose%
\pgfusepath{fill}%
\end{pgfscope}%
\begin{pgfscope}%
\pgfpathrectangle{\pgfqpoint{0.539299in}{0.078740in}}{\pgfqpoint{7.842520in}{7.842520in}}%
\pgfusepath{clip}%
\pgfsetbuttcap%
\pgfsetroundjoin%
\definecolor{currentfill}{rgb}{0.124780,0.640461,0.527068}%
\pgfsetfillcolor{currentfill}%
\pgfsetlinewidth{0.000000pt}%
\definecolor{currentstroke}{rgb}{0.896320,0.893616,0.096335}%
\pgfsetstrokecolor{currentstroke}%
\pgfsetdash{}{0pt}%
\pgfpathmoveto{\pgfqpoint{2.095141in}{4.184186in}}%
\pgfpathlineto{\pgfqpoint{2.219156in}{4.533644in}}%
\pgfpathlineto{\pgfqpoint{2.181328in}{4.304148in}}%
\pgfpathclose%
\pgfusepath{fill}%
\end{pgfscope}%
\begin{pgfscope}%
\pgfpathrectangle{\pgfqpoint{0.539299in}{0.078740in}}{\pgfqpoint{7.842520in}{7.842520in}}%
\pgfusepath{clip}%
\pgfsetbuttcap%
\pgfsetroundjoin%
\definecolor{currentfill}{rgb}{0.283187,0.125848,0.444960}%
\pgfsetfillcolor{currentfill}%
\pgfsetlinewidth{0.000000pt}%
\definecolor{currentstroke}{rgb}{0.906311,0.894855,0.098125}%
\pgfsetstrokecolor{currentstroke}%
\pgfsetdash{}{0pt}%
\pgfpathmoveto{\pgfqpoint{6.172774in}{2.275861in}}%
\pgfpathlineto{\pgfqpoint{6.314217in}{2.147221in}}%
\pgfpathlineto{\pgfqpoint{6.248772in}{2.336093in}}%
\pgfpathclose%
\pgfusepath{fill}%
\end{pgfscope}%
\begin{pgfscope}%
\pgfpathrectangle{\pgfqpoint{0.539299in}{0.078740in}}{\pgfqpoint{7.842520in}{7.842520in}}%
\pgfusepath{clip}%
\pgfsetbuttcap%
\pgfsetroundjoin%
\definecolor{currentfill}{rgb}{0.233603,0.313828,0.543914}%
\pgfsetfillcolor{currentfill}%
\pgfsetlinewidth{0.000000pt}%
\definecolor{currentstroke}{rgb}{0.916242,0.896091,0.100717}%
\pgfsetstrokecolor{currentstroke}%
\pgfsetdash{}{0pt}%
\pgfpathmoveto{\pgfqpoint{1.675575in}{3.148455in}}%
\pgfpathlineto{\pgfqpoint{1.550510in}{2.929884in}}%
\pgfpathlineto{\pgfqpoint{1.468606in}{2.671104in}}%
\pgfpathclose%
\pgfusepath{fill}%
\end{pgfscope}%
\begin{pgfscope}%
\pgfpathrectangle{\pgfqpoint{0.539299in}{0.078740in}}{\pgfqpoint{7.842520in}{7.842520in}}%
\pgfusepath{clip}%
\pgfsetbuttcap%
\pgfsetroundjoin%
\definecolor{currentfill}{rgb}{0.170948,0.694384,0.493803}%
\pgfsetfillcolor{currentfill}%
\pgfsetlinewidth{0.000000pt}%
\definecolor{currentstroke}{rgb}{0.926106,0.897330,0.104071}%
\pgfsetstrokecolor{currentstroke}%
\pgfsetdash{}{0pt}%
\pgfpathmoveto{\pgfqpoint{2.305286in}{4.666266in}}%
\pgfpathlineto{\pgfqpoint{2.391827in}{4.772531in}}%
\pgfpathlineto{\pgfqpoint{2.181328in}{4.304148in}}%
\pgfpathclose%
\pgfusepath{fill}%
\end{pgfscope}%
\begin{pgfscope}%
\pgfpathrectangle{\pgfqpoint{0.539299in}{0.078740in}}{\pgfqpoint{7.842520in}{7.842520in}}%
\pgfusepath{clip}%
\pgfsetbuttcap%
\pgfsetroundjoin%
\definecolor{currentfill}{rgb}{0.277941,0.056324,0.381191}%
\pgfsetfillcolor{currentfill}%
\pgfsetlinewidth{0.000000pt}%
\definecolor{currentstroke}{rgb}{0.935904,0.898570,0.108131}%
\pgfsetstrokecolor{currentstroke}%
\pgfsetdash{}{0pt}%
\pgfpathmoveto{\pgfqpoint{6.673382in}{1.987622in}}%
\pgfpathlineto{\pgfqpoint{6.531589in}{2.101217in}}%
\pgfpathlineto{\pgfqpoint{6.455962in}{2.024776in}}%
\pgfpathclose%
\pgfusepath{fill}%
\end{pgfscope}%
\begin{pgfscope}%
\pgfpathrectangle{\pgfqpoint{0.539299in}{0.078740in}}{\pgfqpoint{7.842520in}{7.842520in}}%
\pgfusepath{clip}%
\pgfsetbuttcap%
\pgfsetroundjoin%
\definecolor{currentfill}{rgb}{0.135066,0.544853,0.554029}%
\pgfsetfillcolor{currentfill}%
\pgfsetlinewidth{0.000000pt}%
\definecolor{currentstroke}{rgb}{0.945636,0.899815,0.112838}%
\pgfsetstrokecolor{currentstroke}%
\pgfsetdash{}{0pt}%
\pgfpathmoveto{\pgfqpoint{4.846758in}{3.914972in}}%
\pgfpathlineto{\pgfqpoint{4.766857in}{3.997620in}}%
\pgfpathlineto{\pgfqpoint{4.907447in}{3.788480in}}%
\pgfpathclose%
\pgfusepath{fill}%
\end{pgfscope}%
\begin{pgfscope}%
\pgfpathrectangle{\pgfqpoint{0.539299in}{0.078740in}}{\pgfqpoint{7.842520in}{7.842520in}}%
\pgfusepath{clip}%
\pgfsetbuttcap%
\pgfsetroundjoin%
\definecolor{currentfill}{rgb}{0.137339,0.662252,0.515571}%
\pgfsetfillcolor{currentfill}%
\pgfsetlinewidth{0.000000pt}%
\definecolor{currentstroke}{rgb}{0.955300,0.901065,0.118128}%
\pgfsetstrokecolor{currentstroke}%
\pgfsetdash{}{0pt}%
\pgfpathmoveto{\pgfqpoint{4.426173in}{4.535091in}}%
\pgfpathlineto{\pgfqpoint{4.485130in}{4.425835in}}%
\pgfpathlineto{\pgfqpoint{4.566539in}{4.326950in}}%
\pgfpathclose%
\pgfusepath{fill}%
\end{pgfscope}%
\begin{pgfscope}%
\pgfpathrectangle{\pgfqpoint{0.539299in}{0.078740in}}{\pgfqpoint{7.842520in}{7.842520in}}%
\pgfusepath{clip}%
\pgfsetbuttcap%
\pgfsetroundjoin%
\definecolor{currentfill}{rgb}{0.281924,0.089666,0.412415}%
\pgfsetfillcolor{currentfill}%
\pgfsetlinewidth{0.000000pt}%
\definecolor{currentstroke}{rgb}{0.964894,0.902323,0.123941}%
\pgfsetstrokecolor{currentstroke}%
\pgfsetdash{}{0pt}%
\pgfpathmoveto{\pgfqpoint{6.390049in}{2.217039in}}%
\pgfpathlineto{\pgfqpoint{6.314217in}{2.147221in}}%
\pgfpathlineto{\pgfqpoint{6.455962in}{2.024776in}}%
\pgfpathclose%
\pgfusepath{fill}%
\end{pgfscope}%
\begin{pgfscope}%
\pgfpathrectangle{\pgfqpoint{0.539299in}{0.078740in}}{\pgfqpoint{7.842520in}{7.842520in}}%
\pgfusepath{clip}%
\pgfsetbuttcap%
\pgfsetroundjoin%
\definecolor{currentfill}{rgb}{0.271828,0.209303,0.504434}%
\pgfsetfillcolor{currentfill}%
\pgfsetlinewidth{0.000000pt}%
\definecolor{currentstroke}{rgb}{0.974417,0.903590,0.130215}%
\pgfsetstrokecolor{currentstroke}%
\pgfsetdash{}{0pt}%
\pgfpathmoveto{\pgfqpoint{1.596362in}{2.813436in}}%
\pgfpathlineto{\pgfqpoint{1.388975in}{2.359780in}}%
\pgfpathlineto{\pgfqpoint{1.520051in}{2.413317in}}%
\pgfpathclose%
\pgfusepath{fill}%
\end{pgfscope}%
\begin{pgfscope}%
\pgfpathrectangle{\pgfqpoint{0.539299in}{0.078740in}}{\pgfqpoint{7.842520in}{7.842520in}}%
\pgfusepath{clip}%
\pgfsetbuttcap%
\pgfsetroundjoin%
\definecolor{currentfill}{rgb}{0.123463,0.581687,0.547445}%
\pgfsetfillcolor{currentfill}%
\pgfsetlinewidth{0.000000pt}%
\definecolor{currentstroke}{rgb}{0.983868,0.904867,0.136897}%
\pgfsetstrokecolor{currentstroke}%
\pgfsetdash{}{0pt}%
\pgfpathmoveto{\pgfqpoint{4.626092in}{4.211058in}}%
\pgfpathlineto{\pgfqpoint{4.766857in}{3.997620in}}%
\pgfpathlineto{\pgfqpoint{4.846758in}{3.914972in}}%
\pgfpathclose%
\pgfusepath{fill}%
\end{pgfscope}%
\begin{pgfscope}%
\pgfpathrectangle{\pgfqpoint{0.539299in}{0.078740in}}{\pgfqpoint{7.842520in}{7.842520in}}%
\pgfusepath{clip}%
\pgfsetbuttcap%
\pgfsetroundjoin%
\definecolor{currentfill}{rgb}{0.171176,0.452530,0.557965}%
\pgfsetfillcolor{currentfill}%
\pgfsetlinewidth{0.000000pt}%
\definecolor{currentstroke}{rgb}{0.993248,0.906157,0.143936}%
\pgfsetstrokecolor{currentstroke}%
\pgfsetdash{}{0pt}%
\pgfpathmoveto{\pgfqpoint{1.756965in}{3.430597in}}%
\pgfpathlineto{\pgfqpoint{1.839990in}{3.669004in}}%
\pgfpathlineto{\pgfqpoint{1.718526in}{3.335879in}}%
\pgfpathclose%
\pgfusepath{fill}%
\end{pgfscope}%
\begin{pgfscope}%
\pgfpathrectangle{\pgfqpoint{0.539299in}{0.078740in}}{\pgfqpoint{7.842520in}{7.842520in}}%
\pgfusepath{clip}%
\pgfsetbuttcap%
\pgfsetroundjoin%
\definecolor{currentfill}{rgb}{0.515992,0.831158,0.294279}%
\pgfsetfillcolor{currentfill}%
\pgfsetlinewidth{0.000000pt}%
\definecolor{currentstroke}{rgb}{0.267004,0.004874,0.329415}%
\pgfsetstrokecolor{currentstroke}%
\pgfsetdash{}{0pt}%
\pgfpathmoveto{\pgfqpoint{2.694035in}{5.204781in}}%
\pgfpathlineto{\pgfqpoint{2.913349in}{5.441460in}}%
\pgfpathlineto{\pgfqpoint{2.781099in}{5.242417in}}%
\pgfpathclose%
\pgfusepath{fill}%
\end{pgfscope}%
\begin{pgfscope}%
\pgfpathrectangle{\pgfqpoint{0.539299in}{0.078740in}}{\pgfqpoint{7.842520in}{7.842520in}}%
\pgfusepath{clip}%
\pgfsetbuttcap%
\pgfsetroundjoin%
\definecolor{currentfill}{rgb}{0.606045,0.850733,0.236712}%
\pgfsetfillcolor{currentfill}%
\pgfsetlinewidth{0.000000pt}%
\definecolor{currentstroke}{rgb}{0.268510,0.009605,0.335427}%
\pgfsetstrokecolor{currentstroke}%
\pgfsetdash{}{0pt}%
\pgfpathmoveto{\pgfqpoint{3.584365in}{5.478002in}}%
\pgfpathlineto{\pgfqpoint{3.638229in}{5.473510in}}%
\pgfpathlineto{\pgfqpoint{3.723815in}{5.392306in}}%
\pgfpathclose%
\pgfusepath{fill}%
\end{pgfscope}%
\begin{pgfscope}%
\pgfpathrectangle{\pgfqpoint{0.539299in}{0.078740in}}{\pgfqpoint{7.842520in}{7.842520in}}%
\pgfusepath{clip}%
\pgfsetbuttcap%
\pgfsetroundjoin%
\definecolor{currentfill}{rgb}{0.678489,0.863742,0.189503}%
\pgfsetfillcolor{currentfill}%
\pgfsetlinewidth{0.000000pt}%
\definecolor{currentstroke}{rgb}{0.269944,0.014625,0.341379}%
\pgfsetstrokecolor{currentstroke}%
\pgfsetdash{}{0pt}%
\pgfpathmoveto{\pgfqpoint{3.359631in}{5.571327in}}%
\pgfpathlineto{\pgfqpoint{3.222442in}{5.537820in}}%
\pgfpathlineto{\pgfqpoint{3.135579in}{5.562556in}}%
\pgfpathclose%
\pgfusepath{fill}%
\end{pgfscope}%
\begin{pgfscope}%
\pgfpathrectangle{\pgfqpoint{0.539299in}{0.078740in}}{\pgfqpoint{7.842520in}{7.842520in}}%
\pgfusepath{clip}%
\pgfsetbuttcap%
\pgfsetroundjoin%
\definecolor{currentfill}{rgb}{0.668054,0.861999,0.196293}%
\pgfsetfillcolor{currentfill}%
\pgfsetlinewidth{0.000000pt}%
\definecolor{currentstroke}{rgb}{0.271305,0.019942,0.347269}%
\pgfsetstrokecolor{currentstroke}%
\pgfsetdash{}{0pt}%
\pgfpathmoveto{\pgfqpoint{3.359631in}{5.571327in}}%
\pgfpathlineto{\pgfqpoint{3.498366in}{5.547173in}}%
\pgfpathlineto{\pgfqpoint{3.584365in}{5.478002in}}%
\pgfpathclose%
\pgfusepath{fill}%
\end{pgfscope}%
\begin{pgfscope}%
\pgfpathrectangle{\pgfqpoint{0.539299in}{0.078740in}}{\pgfqpoint{7.842520in}{7.842520in}}%
\pgfusepath{clip}%
\pgfsetbuttcap%
\pgfsetroundjoin%
\definecolor{currentfill}{rgb}{0.220124,0.725509,0.466226}%
\pgfsetfillcolor{currentfill}%
\pgfsetlinewidth{0.000000pt}%
\definecolor{currentstroke}{rgb}{0.272594,0.025563,0.353093}%
\pgfsetstrokecolor{currentstroke}%
\pgfsetdash{}{0pt}%
\pgfpathmoveto{\pgfqpoint{4.426173in}{4.535091in}}%
\pgfpathlineto{\pgfqpoint{4.285648in}{4.739044in}}%
\pgfpathlineto{\pgfqpoint{4.202671in}{4.843466in}}%
\pgfpathclose%
\pgfusepath{fill}%
\end{pgfscope}%
\begin{pgfscope}%
\pgfpathrectangle{\pgfqpoint{0.539299in}{0.078740in}}{\pgfqpoint{7.842520in}{7.842520in}}%
\pgfusepath{clip}%
\pgfsetbuttcap%
\pgfsetroundjoin%
\definecolor{currentfill}{rgb}{0.121380,0.629492,0.531973}%
\pgfsetfillcolor{currentfill}%
\pgfsetlinewidth{0.000000pt}%
\definecolor{currentstroke}{rgb}{0.273809,0.031497,0.358853}%
\pgfsetstrokecolor{currentstroke}%
\pgfsetdash{}{0pt}%
\pgfpathmoveto{\pgfqpoint{2.219156in}{4.533644in}}%
\pgfpathlineto{\pgfqpoint{2.095141in}{4.184186in}}%
\pgfpathlineto{\pgfqpoint{2.009371in}{4.040892in}}%
\pgfpathclose%
\pgfusepath{fill}%
\end{pgfscope}%
\begin{pgfscope}%
\pgfpathrectangle{\pgfqpoint{0.539299in}{0.078740in}}{\pgfqpoint{7.842520in}{7.842520in}}%
\pgfusepath{clip}%
\pgfsetbuttcap%
\pgfsetroundjoin%
\definecolor{currentfill}{rgb}{0.395174,0.797475,0.367757}%
\pgfsetfillcolor{currentfill}%
\pgfsetlinewidth{0.000000pt}%
\definecolor{currentstroke}{rgb}{0.274952,0.037752,0.364543}%
\pgfsetstrokecolor{currentstroke}%
\pgfsetdash{}{0pt}%
\pgfpathmoveto{\pgfqpoint{2.478591in}{4.855876in}}%
\pgfpathlineto{\pgfqpoint{2.606940in}{5.146364in}}%
\pgfpathlineto{\pgfqpoint{2.694035in}{5.204781in}}%
\pgfpathclose%
\pgfusepath{fill}%
\end{pgfscope}%
\begin{pgfscope}%
\pgfpathrectangle{\pgfqpoint{0.539299in}{0.078740in}}{\pgfqpoint{7.842520in}{7.842520in}}%
\pgfusepath{clip}%
\pgfsetbuttcap%
\pgfsetroundjoin%
\definecolor{currentfill}{rgb}{0.241237,0.296485,0.539709}%
\pgfsetfillcolor{currentfill}%
\pgfsetlinewidth{0.000000pt}%
\definecolor{currentstroke}{rgb}{0.276022,0.044167,0.370164}%
\pgfsetstrokecolor{currentstroke}%
\pgfsetdash{}{0pt}%
\pgfpathmoveto{\pgfqpoint{5.749939in}{2.703326in}}%
\pgfpathlineto{\pgfqpoint{5.609380in}{2.861680in}}%
\pgfpathlineto{\pgfqpoint{5.532155in}{2.871486in}}%
\pgfpathclose%
\pgfusepath{fill}%
\end{pgfscope}%
\begin{pgfscope}%
\pgfpathrectangle{\pgfqpoint{0.539299in}{0.078740in}}{\pgfqpoint{7.842520in}{7.842520in}}%
\pgfusepath{clip}%
\pgfsetbuttcap%
\pgfsetroundjoin%
\definecolor{currentfill}{rgb}{0.229739,0.322361,0.545706}%
\pgfsetfillcolor{currentfill}%
\pgfsetlinewidth{0.000000pt}%
\definecolor{currentstroke}{rgb}{0.277018,0.050344,0.375715}%
\pgfsetstrokecolor{currentstroke}%
\pgfsetdash{}{0pt}%
\pgfpathmoveto{\pgfqpoint{5.532155in}{2.871486in}}%
\pgfpathlineto{\pgfqpoint{5.609380in}{2.861680in}}%
\pgfpathlineto{\pgfqpoint{5.468948in}{3.028932in}}%
\pgfpathclose%
\pgfusepath{fill}%
\end{pgfscope}%
\begin{pgfscope}%
\pgfpathrectangle{\pgfqpoint{0.539299in}{0.078740in}}{\pgfqpoint{7.842520in}{7.842520in}}%
\pgfusepath{clip}%
\pgfsetbuttcap%
\pgfsetroundjoin%
\definecolor{currentfill}{rgb}{0.263663,0.237631,0.518762}%
\pgfsetfillcolor{currentfill}%
\pgfsetlinewidth{0.000000pt}%
\definecolor{currentstroke}{rgb}{0.277941,0.056324,0.381191}%
\pgfsetstrokecolor{currentstroke}%
\pgfsetdash{}{0pt}%
\pgfpathmoveto{\pgfqpoint{5.814052in}{2.532529in}}%
\pgfpathlineto{\pgfqpoint{5.890666in}{2.553330in}}%
\pgfpathlineto{\pgfqpoint{5.749939in}{2.703326in}}%
\pgfpathclose%
\pgfusepath{fill}%
\end{pgfscope}%
\begin{pgfscope}%
\pgfpathrectangle{\pgfqpoint{0.539299in}{0.078740in}}{\pgfqpoint{7.842520in}{7.842520in}}%
\pgfusepath{clip}%
\pgfsetbuttcap%
\pgfsetroundjoin%
\definecolor{currentfill}{rgb}{0.276022,0.044167,0.370164}%
\pgfsetfillcolor{currentfill}%
\pgfsetlinewidth{0.000000pt}%
\definecolor{currentstroke}{rgb}{0.278791,0.062145,0.386592}%
\pgfsetstrokecolor{currentstroke}%
\pgfsetdash{}{0pt}%
\pgfpathmoveto{\pgfqpoint{6.455962in}{2.024776in}}%
\pgfpathlineto{\pgfqpoint{6.598042in}{1.908413in}}%
\pgfpathlineto{\pgfqpoint{6.673382in}{1.987622in}}%
\pgfpathclose%
\pgfusepath{fill}%
\end{pgfscope}%
\begin{pgfscope}%
\pgfpathrectangle{\pgfqpoint{0.539299in}{0.078740in}}{\pgfqpoint{7.842520in}{7.842520in}}%
\pgfusepath{clip}%
\pgfsetbuttcap%
\pgfsetroundjoin%
\definecolor{currentfill}{rgb}{0.121380,0.629492,0.531973}%
\pgfsetfillcolor{currentfill}%
\pgfsetlinewidth{0.000000pt}%
\definecolor{currentstroke}{rgb}{0.279566,0.067836,0.391917}%
\pgfsetstrokecolor{currentstroke}%
\pgfsetdash{}{0pt}%
\pgfpathmoveto{\pgfqpoint{4.485130in}{4.425835in}}%
\pgfpathlineto{\pgfqpoint{4.626092in}{4.211058in}}%
\pgfpathlineto{\pgfqpoint{4.706729in}{4.119068in}}%
\pgfpathclose%
\pgfusepath{fill}%
\end{pgfscope}%
\begin{pgfscope}%
\pgfpathrectangle{\pgfqpoint{0.539299in}{0.078740in}}{\pgfqpoint{7.842520in}{7.842520in}}%
\pgfusepath{clip}%
\pgfsetbuttcap%
\pgfsetroundjoin%
\definecolor{currentfill}{rgb}{0.281477,0.755203,0.432552}%
\pgfsetfillcolor{currentfill}%
\pgfsetlinewidth{0.000000pt}%
\definecolor{currentstroke}{rgb}{0.280267,0.073417,0.397163}%
\pgfsetstrokecolor{currentstroke}%
\pgfsetdash{}{0pt}%
\pgfpathmoveto{\pgfqpoint{4.202671in}{4.843466in}}%
\pgfpathlineto{\pgfqpoint{4.285648in}{4.739044in}}%
\pgfpathlineto{\pgfqpoint{4.145020in}{4.933425in}}%
\pgfpathclose%
\pgfusepath{fill}%
\end{pgfscope}%
\begin{pgfscope}%
\pgfpathrectangle{\pgfqpoint{0.539299in}{0.078740in}}{\pgfqpoint{7.842520in}{7.842520in}}%
\pgfusepath{clip}%
\pgfsetbuttcap%
\pgfsetroundjoin%
\definecolor{currentfill}{rgb}{0.275191,0.194905,0.496005}%
\pgfsetfillcolor{currentfill}%
\pgfsetlinewidth{0.000000pt}%
\definecolor{currentstroke}{rgb}{0.280894,0.078907,0.402329}%
\pgfsetstrokecolor{currentstroke}%
\pgfsetdash{}{0pt}%
\pgfpathmoveto{\pgfqpoint{5.955205in}{2.375680in}}%
\pgfpathlineto{\pgfqpoint{6.031600in}{2.411053in}}%
\pgfpathlineto{\pgfqpoint{5.890666in}{2.553330in}}%
\pgfpathclose%
\pgfusepath{fill}%
\end{pgfscope}%
\begin{pgfscope}%
\pgfpathrectangle{\pgfqpoint{0.539299in}{0.078740in}}{\pgfqpoint{7.842520in}{7.842520in}}%
\pgfusepath{clip}%
\pgfsetbuttcap%
\pgfsetroundjoin%
\definecolor{currentfill}{rgb}{0.233603,0.313828,0.543914}%
\pgfsetfillcolor{currentfill}%
\pgfsetlinewidth{0.000000pt}%
\definecolor{currentstroke}{rgb}{0.281446,0.084320,0.407414}%
\pgfsetstrokecolor{currentstroke}%
\pgfsetdash{}{0pt}%
\pgfpathmoveto{\pgfqpoint{1.468606in}{2.671104in}}%
\pgfpathlineto{\pgfqpoint{1.596362in}{2.813436in}}%
\pgfpathlineto{\pgfqpoint{1.675575in}{3.148455in}}%
\pgfpathclose%
\pgfusepath{fill}%
\end{pgfscope}%
\begin{pgfscope}%
\pgfpathrectangle{\pgfqpoint{0.539299in}{0.078740in}}{\pgfqpoint{7.842520in}{7.842520in}}%
\pgfusepath{clip}%
\pgfsetbuttcap%
\pgfsetroundjoin%
\definecolor{currentfill}{rgb}{0.162016,0.687316,0.499129}%
\pgfsetfillcolor{currentfill}%
\pgfsetlinewidth{0.000000pt}%
\definecolor{currentstroke}{rgb}{0.281924,0.089666,0.412415}%
\pgfsetstrokecolor{currentstroke}%
\pgfsetdash{}{0pt}%
\pgfpathmoveto{\pgfqpoint{2.181328in}{4.304148in}}%
\pgfpathlineto{\pgfqpoint{2.219156in}{4.533644in}}%
\pgfpathlineto{\pgfqpoint{2.305286in}{4.666266in}}%
\pgfpathclose%
\pgfusepath{fill}%
\end{pgfscope}%
\begin{pgfscope}%
\pgfpathrectangle{\pgfqpoint{0.539299in}{0.078740in}}{\pgfqpoint{7.842520in}{7.842520in}}%
\pgfusepath{clip}%
\pgfsetbuttcap%
\pgfsetroundjoin%
\definecolor{currentfill}{rgb}{0.201239,0.383670,0.554294}%
\pgfsetfillcolor{currentfill}%
\pgfsetlinewidth{0.000000pt}%
\definecolor{currentstroke}{rgb}{0.282327,0.094955,0.417331}%
\pgfsetstrokecolor{currentstroke}%
\pgfsetdash{}{0pt}%
\pgfpathmoveto{\pgfqpoint{5.468948in}{3.028932in}}%
\pgfpathlineto{\pgfqpoint{5.328594in}{3.205423in}}%
\pgfpathlineto{\pgfqpoint{5.250481in}{3.245064in}}%
\pgfpathclose%
\pgfusepath{fill}%
\end{pgfscope}%
\begin{pgfscope}%
\pgfpathrectangle{\pgfqpoint{0.539299in}{0.078740in}}{\pgfqpoint{7.842520in}{7.842520in}}%
\pgfusepath{clip}%
\pgfsetbuttcap%
\pgfsetroundjoin%
\definecolor{currentfill}{rgb}{0.555484,0.840254,0.269281}%
\pgfsetfillcolor{currentfill}%
\pgfsetlinewidth{0.000000pt}%
\definecolor{currentstroke}{rgb}{0.282656,0.100196,0.422160}%
\pgfsetstrokecolor{currentstroke}%
\pgfsetdash{}{0pt}%
\pgfpathmoveto{\pgfqpoint{3.863917in}{5.267515in}}%
\pgfpathlineto{\pgfqpoint{3.723815in}{5.392306in}}%
\pgfpathlineto{\pgfqpoint{3.778860in}{5.358543in}}%
\pgfpathclose%
\pgfusepath{fill}%
\end{pgfscope}%
\begin{pgfscope}%
\pgfpathrectangle{\pgfqpoint{0.539299in}{0.078740in}}{\pgfqpoint{7.842520in}{7.842520in}}%
\pgfusepath{clip}%
\pgfsetbuttcap%
\pgfsetroundjoin%
\definecolor{currentfill}{rgb}{0.199430,0.387607,0.554642}%
\pgfsetfillcolor{currentfill}%
\pgfsetlinewidth{0.000000pt}%
\definecolor{currentstroke}{rgb}{0.282910,0.105393,0.426902}%
\pgfsetstrokecolor{currentstroke}%
\pgfsetdash{}{0pt}%
\pgfpathmoveto{\pgfqpoint{1.550510in}{2.929884in}}%
\pgfpathlineto{\pgfqpoint{1.675575in}{3.148455in}}%
\pgfpathlineto{\pgfqpoint{1.756965in}{3.430597in}}%
\pgfpathclose%
\pgfusepath{fill}%
\end{pgfscope}%
\begin{pgfscope}%
\pgfpathrectangle{\pgfqpoint{0.539299in}{0.078740in}}{\pgfqpoint{7.842520in}{7.842520in}}%
\pgfusepath{clip}%
\pgfsetbuttcap%
\pgfsetroundjoin%
\definecolor{currentfill}{rgb}{0.636902,0.856542,0.216620}%
\pgfsetfillcolor{currentfill}%
\pgfsetlinewidth{0.000000pt}%
\definecolor{currentstroke}{rgb}{0.283091,0.110553,0.431554}%
\pgfsetstrokecolor{currentstroke}%
\pgfsetdash{}{0pt}%
\pgfpathmoveto{\pgfqpoint{3.135579in}{5.562556in}}%
\pgfpathlineto{\pgfqpoint{3.000456in}{5.448722in}}%
\pgfpathlineto{\pgfqpoint{2.913349in}{5.441460in}}%
\pgfpathclose%
\pgfusepath{fill}%
\end{pgfscope}%
\begin{pgfscope}%
\pgfpathrectangle{\pgfqpoint{0.539299in}{0.078740in}}{\pgfqpoint{7.842520in}{7.842520in}}%
\pgfusepath{clip}%
\pgfsetbuttcap%
\pgfsetroundjoin%
\definecolor{currentfill}{rgb}{0.281412,0.155834,0.469201}%
\pgfsetfillcolor{currentfill}%
\pgfsetlinewidth{0.000000pt}%
\definecolor{currentstroke}{rgb}{0.283197,0.115680,0.436115}%
\pgfsetstrokecolor{currentstroke}%
\pgfsetdash{}{0pt}%
\pgfpathmoveto{\pgfqpoint{6.172774in}{2.275861in}}%
\pgfpathlineto{\pgfqpoint{6.031600in}{2.411053in}}%
\pgfpathlineto{\pgfqpoint{6.096558in}{2.227023in}}%
\pgfpathclose%
\pgfusepath{fill}%
\end{pgfscope}%
\begin{pgfscope}%
\pgfpathrectangle{\pgfqpoint{0.539299in}{0.078740in}}{\pgfqpoint{7.842520in}{7.842520in}}%
\pgfusepath{clip}%
\pgfsetbuttcap%
\pgfsetroundjoin%
\definecolor{currentfill}{rgb}{0.187231,0.414746,0.556547}%
\pgfsetfillcolor{currentfill}%
\pgfsetlinewidth{0.000000pt}%
\definecolor{currentstroke}{rgb}{0.283229,0.120777,0.440584}%
\pgfsetstrokecolor{currentstroke}%
\pgfsetdash{}{0pt}%
\pgfpathmoveto{\pgfqpoint{5.250481in}{3.245064in}}%
\pgfpathlineto{\pgfqpoint{5.328594in}{3.205423in}}%
\pgfpathlineto{\pgfqpoint{5.188263in}{3.391183in}}%
\pgfpathclose%
\pgfusepath{fill}%
\end{pgfscope}%
\begin{pgfscope}%
\pgfpathrectangle{\pgfqpoint{0.539299in}{0.078740in}}{\pgfqpoint{7.842520in}{7.842520in}}%
\pgfusepath{clip}%
\pgfsetbuttcap%
\pgfsetroundjoin%
\definecolor{currentfill}{rgb}{0.377779,0.791781,0.377939}%
\pgfsetfillcolor{currentfill}%
\pgfsetlinewidth{0.000000pt}%
\definecolor{currentstroke}{rgb}{0.283187,0.125848,0.444960}%
\pgfsetstrokecolor{currentstroke}%
\pgfsetdash{}{0pt}%
\pgfpathmoveto{\pgfqpoint{4.061290in}{5.036247in}}%
\pgfpathlineto{\pgfqpoint{4.145020in}{4.933425in}}%
\pgfpathlineto{\pgfqpoint{4.004391in}{5.111952in}}%
\pgfpathclose%
\pgfusepath{fill}%
\end{pgfscope}%
\begin{pgfscope}%
\pgfpathrectangle{\pgfqpoint{0.539299in}{0.078740in}}{\pgfqpoint{7.842520in}{7.842520in}}%
\pgfusepath{clip}%
\pgfsetbuttcap%
\pgfsetroundjoin%
\definecolor{currentfill}{rgb}{0.496615,0.826376,0.306377}%
\pgfsetfillcolor{currentfill}%
\pgfsetlinewidth{0.000000pt}%
\definecolor{currentstroke}{rgb}{0.283072,0.130895,0.449241}%
\pgfsetstrokecolor{currentstroke}%
\pgfsetdash{}{0pt}%
\pgfpathmoveto{\pgfqpoint{3.778860in}{5.358543in}}%
\pgfpathlineto{\pgfqpoint{4.004391in}{5.111952in}}%
\pgfpathlineto{\pgfqpoint{3.863917in}{5.267515in}}%
\pgfpathclose%
\pgfusepath{fill}%
\end{pgfscope}%
\begin{pgfscope}%
\pgfpathrectangle{\pgfqpoint{0.539299in}{0.078740in}}{\pgfqpoint{7.842520in}{7.842520in}}%
\pgfusepath{clip}%
\pgfsetbuttcap%
\pgfsetroundjoin%
\definecolor{currentfill}{rgb}{0.311925,0.767822,0.415586}%
\pgfsetfillcolor{currentfill}%
\pgfsetlinewidth{0.000000pt}%
\definecolor{currentstroke}{rgb}{0.282884,0.135920,0.453427}%
\pgfsetstrokecolor{currentstroke}%
\pgfsetdash{}{0pt}%
\pgfpathmoveto{\pgfqpoint{2.478591in}{4.855876in}}%
\pgfpathlineto{\pgfqpoint{2.391827in}{4.772531in}}%
\pgfpathlineto{\pgfqpoint{2.519975in}{5.063797in}}%
\pgfpathclose%
\pgfusepath{fill}%
\end{pgfscope}%
\begin{pgfscope}%
\pgfpathrectangle{\pgfqpoint{0.539299in}{0.078740in}}{\pgfqpoint{7.842520in}{7.842520in}}%
\pgfusepath{clip}%
\pgfsetbuttcap%
\pgfsetroundjoin%
\definecolor{currentfill}{rgb}{0.283072,0.130895,0.449241}%
\pgfsetfillcolor{currentfill}%
\pgfsetlinewidth{0.000000pt}%
\definecolor{currentstroke}{rgb}{0.282623,0.140926,0.457517}%
\pgfsetstrokecolor{currentstroke}%
\pgfsetdash{}{0pt}%
\pgfpathmoveto{\pgfqpoint{6.096558in}{2.227023in}}%
\pgfpathlineto{\pgfqpoint{6.314217in}{2.147221in}}%
\pgfpathlineto{\pgfqpoint{6.172774in}{2.275861in}}%
\pgfpathclose%
\pgfusepath{fill}%
\end{pgfscope}%
\begin{pgfscope}%
\pgfpathrectangle{\pgfqpoint{0.539299in}{0.078740in}}{\pgfqpoint{7.842520in}{7.842520in}}%
\pgfusepath{clip}%
\pgfsetbuttcap%
\pgfsetroundjoin%
\definecolor{currentfill}{rgb}{0.657642,0.860219,0.203082}%
\pgfsetfillcolor{currentfill}%
\pgfsetlinewidth{0.000000pt}%
\definecolor{currentstroke}{rgb}{0.282290,0.145912,0.461510}%
\pgfsetstrokecolor{currentstroke}%
\pgfsetdash{}{0pt}%
\pgfpathmoveto{\pgfqpoint{3.498366in}{5.547173in}}%
\pgfpathlineto{\pgfqpoint{3.638229in}{5.473510in}}%
\pgfpathlineto{\pgfqpoint{3.584365in}{5.478002in}}%
\pgfpathclose%
\pgfusepath{fill}%
\end{pgfscope}%
\begin{pgfscope}%
\pgfpathrectangle{\pgfqpoint{0.539299in}{0.078740in}}{\pgfqpoint{7.842520in}{7.842520in}}%
\pgfusepath{clip}%
\pgfsetbuttcap%
\pgfsetroundjoin%
\definecolor{currentfill}{rgb}{0.175707,0.697900,0.491033}%
\pgfsetfillcolor{currentfill}%
\pgfsetlinewidth{0.000000pt}%
\definecolor{currentstroke}{rgb}{0.281887,0.150881,0.465405}%
\pgfsetstrokecolor{currentstroke}%
\pgfsetdash{}{0pt}%
\pgfpathmoveto{\pgfqpoint{4.343978in}{4.638176in}}%
\pgfpathlineto{\pgfqpoint{4.485130in}{4.425835in}}%
\pgfpathlineto{\pgfqpoint{4.426173in}{4.535091in}}%
\pgfpathclose%
\pgfusepath{fill}%
\end{pgfscope}%
\begin{pgfscope}%
\pgfpathrectangle{\pgfqpoint{0.539299in}{0.078740in}}{\pgfqpoint{7.842520in}{7.842520in}}%
\pgfusepath{clip}%
\pgfsetbuttcap%
\pgfsetroundjoin%
\definecolor{currentfill}{rgb}{0.160665,0.478540,0.558115}%
\pgfsetfillcolor{currentfill}%
\pgfsetlinewidth{0.000000pt}%
\definecolor{currentstroke}{rgb}{0.281412,0.155834,0.469201}%
\pgfsetstrokecolor{currentstroke}%
\pgfsetdash{}{0pt}%
\pgfpathmoveto{\pgfqpoint{4.968612in}{3.651484in}}%
\pgfpathlineto{\pgfqpoint{5.188263in}{3.391183in}}%
\pgfpathlineto{\pgfqpoint{5.047899in}{3.585834in}}%
\pgfpathclose%
\pgfusepath{fill}%
\end{pgfscope}%
\begin{pgfscope}%
\pgfpathrectangle{\pgfqpoint{0.539299in}{0.078740in}}{\pgfqpoint{7.842520in}{7.842520in}}%
\pgfusepath{clip}%
\pgfsetbuttcap%
\pgfsetroundjoin%
\definecolor{currentfill}{rgb}{0.282327,0.094955,0.417331}%
\pgfsetfillcolor{currentfill}%
\pgfsetlinewidth{0.000000pt}%
\definecolor{currentstroke}{rgb}{0.280868,0.160771,0.472899}%
\pgfsetstrokecolor{currentstroke}%
\pgfsetdash{}{0pt}%
\pgfpathmoveto{\pgfqpoint{6.314217in}{2.147221in}}%
\pgfpathlineto{\pgfqpoint{6.238160in}{2.086590in}}%
\pgfpathlineto{\pgfqpoint{6.455962in}{2.024776in}}%
\pgfpathclose%
\pgfusepath{fill}%
\end{pgfscope}%
\begin{pgfscope}%
\pgfpathrectangle{\pgfqpoint{0.539299in}{0.078740in}}{\pgfqpoint{7.842520in}{7.842520in}}%
\pgfusepath{clip}%
\pgfsetbuttcap%
\pgfsetroundjoin%
\definecolor{currentfill}{rgb}{0.255645,0.260703,0.528312}%
\pgfsetfillcolor{currentfill}%
\pgfsetlinewidth{0.000000pt}%
\definecolor{currentstroke}{rgb}{0.280255,0.165693,0.476498}%
\pgfsetstrokecolor{currentstroke}%
\pgfsetdash{}{0pt}%
\pgfpathmoveto{\pgfqpoint{5.814052in}{2.532529in}}%
\pgfpathlineto{\pgfqpoint{5.749939in}{2.703326in}}%
\pgfpathlineto{\pgfqpoint{5.673051in}{2.697726in}}%
\pgfpathclose%
\pgfusepath{fill}%
\end{pgfscope}%
\begin{pgfscope}%
\pgfpathrectangle{\pgfqpoint{0.539299in}{0.078740in}}{\pgfqpoint{7.842520in}{7.842520in}}%
\pgfusepath{clip}%
\pgfsetbuttcap%
\pgfsetroundjoin%
\definecolor{currentfill}{rgb}{0.243113,0.292092,0.538516}%
\pgfsetfillcolor{currentfill}%
\pgfsetlinewidth{0.000000pt}%
\definecolor{currentstroke}{rgb}{0.279574,0.170599,0.479997}%
\pgfsetstrokecolor{currentstroke}%
\pgfsetdash{}{0pt}%
\pgfpathmoveto{\pgfqpoint{5.673051in}{2.697726in}}%
\pgfpathlineto{\pgfqpoint{5.749939in}{2.703326in}}%
\pgfpathlineto{\pgfqpoint{5.532155in}{2.871486in}}%
\pgfpathclose%
\pgfusepath{fill}%
\end{pgfscope}%
\begin{pgfscope}%
\pgfpathrectangle{\pgfqpoint{0.539299in}{0.078740in}}{\pgfqpoint{7.842520in}{7.842520in}}%
\pgfusepath{clip}%
\pgfsetbuttcap%
\pgfsetroundjoin%
\definecolor{currentfill}{rgb}{0.270595,0.214069,0.507052}%
\pgfsetfillcolor{currentfill}%
\pgfsetlinewidth{0.000000pt}%
\definecolor{currentstroke}{rgb}{0.278826,0.175490,0.483397}%
\pgfsetstrokecolor{currentstroke}%
\pgfsetdash{}{0pt}%
\pgfpathmoveto{\pgfqpoint{5.890666in}{2.553330in}}%
\pgfpathlineto{\pgfqpoint{5.814052in}{2.532529in}}%
\pgfpathlineto{\pgfqpoint{5.955205in}{2.375680in}}%
\pgfpathclose%
\pgfusepath{fill}%
\end{pgfscope}%
\begin{pgfscope}%
\pgfpathrectangle{\pgfqpoint{0.539299in}{0.078740in}}{\pgfqpoint{7.842520in}{7.842520in}}%
\pgfusepath{clip}%
\pgfsetbuttcap%
\pgfsetroundjoin%
\definecolor{currentfill}{rgb}{0.226397,0.728888,0.462789}%
\pgfsetfillcolor{currentfill}%
\pgfsetlinewidth{0.000000pt}%
\definecolor{currentstroke}{rgb}{0.278012,0.180367,0.486697}%
\pgfsetstrokecolor{currentstroke}%
\pgfsetdash{}{0pt}%
\pgfpathmoveto{\pgfqpoint{4.202671in}{4.843466in}}%
\pgfpathlineto{\pgfqpoint{4.343978in}{4.638176in}}%
\pgfpathlineto{\pgfqpoint{4.426173in}{4.535091in}}%
\pgfpathclose%
\pgfusepath{fill}%
\end{pgfscope}%
\begin{pgfscope}%
\pgfpathrectangle{\pgfqpoint{0.539299in}{0.078740in}}{\pgfqpoint{7.842520in}{7.842520in}}%
\pgfusepath{clip}%
\pgfsetbuttcap%
\pgfsetroundjoin%
\definecolor{currentfill}{rgb}{0.216210,0.351535,0.550627}%
\pgfsetfillcolor{currentfill}%
\pgfsetlinewidth{0.000000pt}%
\definecolor{currentstroke}{rgb}{0.277134,0.185228,0.489898}%
\pgfsetstrokecolor{currentstroke}%
\pgfsetdash{}{0pt}%
\pgfpathmoveto{\pgfqpoint{5.468948in}{3.028932in}}%
\pgfpathlineto{\pgfqpoint{5.391316in}{3.053947in}}%
\pgfpathlineto{\pgfqpoint{5.532155in}{2.871486in}}%
\pgfpathclose%
\pgfusepath{fill}%
\end{pgfscope}%
\begin{pgfscope}%
\pgfpathrectangle{\pgfqpoint{0.539299in}{0.078740in}}{\pgfqpoint{7.842520in}{7.842520in}}%
\pgfusepath{clip}%
\pgfsetbuttcap%
\pgfsetroundjoin%
\definecolor{currentfill}{rgb}{0.149039,0.508051,0.557250}%
\pgfsetfillcolor{currentfill}%
\pgfsetlinewidth{0.000000pt}%
\definecolor{currentstroke}{rgb}{0.276194,0.190074,0.493001}%
\pgfsetstrokecolor{currentstroke}%
\pgfsetdash{}{0pt}%
\pgfpathmoveto{\pgfqpoint{5.047899in}{3.585834in}}%
\pgfpathlineto{\pgfqpoint{4.907447in}{3.788480in}}%
\pgfpathlineto{\pgfqpoint{4.968612in}{3.651484in}}%
\pgfpathclose%
\pgfusepath{fill}%
\end{pgfscope}%
\begin{pgfscope}%
\pgfpathrectangle{\pgfqpoint{0.539299in}{0.078740in}}{\pgfqpoint{7.842520in}{7.842520in}}%
\pgfusepath{clip}%
\pgfsetbuttcap%
\pgfsetroundjoin%
\definecolor{currentfill}{rgb}{0.279574,0.170599,0.479997}%
\pgfsetfillcolor{currentfill}%
\pgfsetlinewidth{0.000000pt}%
\definecolor{currentstroke}{rgb}{0.275191,0.194905,0.496005}%
\pgfsetstrokecolor{currentstroke}%
\pgfsetdash{}{0pt}%
\pgfpathmoveto{\pgfqpoint{6.031600in}{2.411053in}}%
\pgfpathlineto{\pgfqpoint{5.955205in}{2.375680in}}%
\pgfpathlineto{\pgfqpoint{6.096558in}{2.227023in}}%
\pgfpathclose%
\pgfusepath{fill}%
\end{pgfscope}%
\begin{pgfscope}%
\pgfpathrectangle{\pgfqpoint{0.539299in}{0.078740in}}{\pgfqpoint{7.842520in}{7.842520in}}%
\pgfusepath{clip}%
\pgfsetbuttcap%
\pgfsetroundjoin%
\definecolor{currentfill}{rgb}{0.203063,0.379716,0.553925}%
\pgfsetfillcolor{currentfill}%
\pgfsetlinewidth{0.000000pt}%
\definecolor{currentstroke}{rgb}{0.274128,0.199721,0.498911}%
\pgfsetstrokecolor{currentstroke}%
\pgfsetdash{}{0pt}%
\pgfpathmoveto{\pgfqpoint{5.250481in}{3.245064in}}%
\pgfpathlineto{\pgfqpoint{5.391316in}{3.053947in}}%
\pgfpathlineto{\pgfqpoint{5.468948in}{3.028932in}}%
\pgfpathclose%
\pgfusepath{fill}%
\end{pgfscope}%
\begin{pgfscope}%
\pgfpathrectangle{\pgfqpoint{0.539299in}{0.078740in}}{\pgfqpoint{7.842520in}{7.842520in}}%
\pgfusepath{clip}%
\pgfsetbuttcap%
\pgfsetroundjoin%
\definecolor{currentfill}{rgb}{0.616293,0.852709,0.230052}%
\pgfsetfillcolor{currentfill}%
\pgfsetlinewidth{0.000000pt}%
\definecolor{currentstroke}{rgb}{0.273006,0.204520,0.501721}%
\pgfsetstrokecolor{currentstroke}%
\pgfsetdash{}{0pt}%
\pgfpathmoveto{\pgfqpoint{3.723815in}{5.392306in}}%
\pgfpathlineto{\pgfqpoint{3.638229in}{5.473510in}}%
\pgfpathlineto{\pgfqpoint{3.778860in}{5.358543in}}%
\pgfpathclose%
\pgfusepath{fill}%
\end{pgfscope}%
\begin{pgfscope}%
\pgfpathrectangle{\pgfqpoint{0.539299in}{0.078740in}}{\pgfqpoint{7.842520in}{7.842520in}}%
\pgfusepath{clip}%
\pgfsetbuttcap%
\pgfsetroundjoin%
\definecolor{currentfill}{rgb}{0.395174,0.797475,0.367757}%
\pgfsetfillcolor{currentfill}%
\pgfsetlinewidth{0.000000pt}%
\definecolor{currentstroke}{rgb}{0.271828,0.209303,0.504434}%
\pgfsetstrokecolor{currentstroke}%
\pgfsetdash{}{0pt}%
\pgfpathmoveto{\pgfqpoint{2.519975in}{5.063797in}}%
\pgfpathlineto{\pgfqpoint{2.606940in}{5.146364in}}%
\pgfpathlineto{\pgfqpoint{2.478591in}{4.855876in}}%
\pgfpathclose%
\pgfusepath{fill}%
\end{pgfscope}%
\begin{pgfscope}%
\pgfpathrectangle{\pgfqpoint{0.539299in}{0.078740in}}{\pgfqpoint{7.842520in}{7.842520in}}%
\pgfusepath{clip}%
\pgfsetbuttcap%
\pgfsetroundjoin%
\definecolor{currentfill}{rgb}{0.276022,0.044167,0.370164}%
\pgfsetfillcolor{currentfill}%
\pgfsetlinewidth{0.000000pt}%
\definecolor{currentstroke}{rgb}{0.270595,0.214069,0.507052}%
\pgfsetstrokecolor{currentstroke}%
\pgfsetdash{}{0pt}%
\pgfpathmoveto{\pgfqpoint{6.522373in}{1.832361in}}%
\pgfpathlineto{\pgfqpoint{6.598042in}{1.908413in}}%
\pgfpathlineto{\pgfqpoint{6.455962in}{2.024776in}}%
\pgfpathclose%
\pgfusepath{fill}%
\end{pgfscope}%
\begin{pgfscope}%
\pgfpathrectangle{\pgfqpoint{0.539299in}{0.078740in}}{\pgfqpoint{7.842520in}{7.842520in}}%
\pgfusepath{clip}%
\pgfsetbuttcap%
\pgfsetroundjoin%
\definecolor{currentfill}{rgb}{0.283229,0.120777,0.440584}%
\pgfsetfillcolor{currentfill}%
\pgfsetlinewidth{0.000000pt}%
\definecolor{currentstroke}{rgb}{0.269308,0.218818,0.509577}%
\pgfsetstrokecolor{currentstroke}%
\pgfsetdash{}{0pt}%
\pgfpathmoveto{\pgfqpoint{6.096558in}{2.227023in}}%
\pgfpathlineto{\pgfqpoint{6.238160in}{2.086590in}}%
\pgfpathlineto{\pgfqpoint{6.314217in}{2.147221in}}%
\pgfpathclose%
\pgfusepath{fill}%
\end{pgfscope}%
\begin{pgfscope}%
\pgfpathrectangle{\pgfqpoint{0.539299in}{0.078740in}}{\pgfqpoint{7.842520in}{7.842520in}}%
\pgfusepath{clip}%
\pgfsetbuttcap%
\pgfsetroundjoin%
\definecolor{currentfill}{rgb}{0.352360,0.783011,0.392636}%
\pgfsetfillcolor{currentfill}%
\pgfsetlinewidth{0.000000pt}%
\definecolor{currentstroke}{rgb}{0.267968,0.223549,0.512008}%
\pgfsetstrokecolor{currentstroke}%
\pgfsetdash{}{0pt}%
\pgfpathmoveto{\pgfqpoint{4.145020in}{4.933425in}}%
\pgfpathlineto{\pgfqpoint{4.061290in}{5.036247in}}%
\pgfpathlineto{\pgfqpoint{4.202671in}{4.843466in}}%
\pgfpathclose%
\pgfusepath{fill}%
\end{pgfscope}%
\begin{pgfscope}%
\pgfpathrectangle{\pgfqpoint{0.539299in}{0.078740in}}{\pgfqpoint{7.842520in}{7.842520in}}%
\pgfusepath{clip}%
\pgfsetbuttcap%
\pgfsetroundjoin%
\definecolor{currentfill}{rgb}{0.121148,0.592739,0.544641}%
\pgfsetfillcolor{currentfill}%
\pgfsetlinewidth{0.000000pt}%
\definecolor{currentstroke}{rgb}{0.266580,0.228262,0.514349}%
\pgfsetstrokecolor{currentstroke}%
\pgfsetdash{}{0pt}%
\pgfpathmoveto{\pgfqpoint{2.049048in}{4.173228in}}%
\pgfpathlineto{\pgfqpoint{2.009371in}{4.040892in}}%
\pgfpathlineto{\pgfqpoint{1.924234in}{3.870625in}}%
\pgfpathclose%
\pgfusepath{fill}%
\end{pgfscope}%
\begin{pgfscope}%
\pgfpathrectangle{\pgfqpoint{0.539299in}{0.078740in}}{\pgfqpoint{7.842520in}{7.842520in}}%
\pgfusepath{clip}%
\pgfsetbuttcap%
\pgfsetroundjoin%
\definecolor{currentfill}{rgb}{0.246811,0.283237,0.535941}%
\pgfsetfillcolor{currentfill}%
\pgfsetlinewidth{0.000000pt}%
\definecolor{currentstroke}{rgb}{0.265145,0.232956,0.516599}%
\pgfsetstrokecolor{currentstroke}%
\pgfsetdash{}{0pt}%
\pgfpathmoveto{\pgfqpoint{1.726778in}{2.916957in}}%
\pgfpathlineto{\pgfqpoint{1.596362in}{2.813436in}}%
\pgfpathlineto{\pgfqpoint{1.520051in}{2.413317in}}%
\pgfpathclose%
\pgfusepath{fill}%
\end{pgfscope}%
\begin{pgfscope}%
\pgfpathrectangle{\pgfqpoint{0.539299in}{0.078740in}}{\pgfqpoint{7.842520in}{7.842520in}}%
\pgfusepath{clip}%
\pgfsetbuttcap%
\pgfsetroundjoin%
\definecolor{currentfill}{rgb}{0.129933,0.559582,0.551864}%
\pgfsetfillcolor{currentfill}%
\pgfsetlinewidth{0.000000pt}%
\definecolor{currentstroke}{rgb}{0.263663,0.237631,0.518762}%
\pgfsetstrokecolor{currentstroke}%
\pgfsetdash{}{0pt}%
\pgfpathmoveto{\pgfqpoint{4.827480in}{3.864804in}}%
\pgfpathlineto{\pgfqpoint{4.907447in}{3.788480in}}%
\pgfpathlineto{\pgfqpoint{4.766857in}{3.997620in}}%
\pgfpathclose%
\pgfusepath{fill}%
\end{pgfscope}%
\begin{pgfscope}%
\pgfpathrectangle{\pgfqpoint{0.539299in}{0.078740in}}{\pgfqpoint{7.842520in}{7.842520in}}%
\pgfusepath{clip}%
\pgfsetbuttcap%
\pgfsetroundjoin%
\definecolor{currentfill}{rgb}{0.730889,0.871916,0.156029}%
\pgfsetfillcolor{currentfill}%
\pgfsetlinewidth{0.000000pt}%
\definecolor{currentstroke}{rgb}{0.262138,0.242286,0.520837}%
\pgfsetstrokecolor{currentstroke}%
\pgfsetdash{}{0pt}%
\pgfpathmoveto{\pgfqpoint{3.135579in}{5.562556in}}%
\pgfpathlineto{\pgfqpoint{3.272857in}{5.610573in}}%
\pgfpathlineto{\pgfqpoint{3.359631in}{5.571327in}}%
\pgfpathclose%
\pgfusepath{fill}%
\end{pgfscope}%
\begin{pgfscope}%
\pgfpathrectangle{\pgfqpoint{0.539299in}{0.078740in}}{\pgfqpoint{7.842520in}{7.842520in}}%
\pgfusepath{clip}%
\pgfsetbuttcap%
\pgfsetroundjoin%
\definecolor{currentfill}{rgb}{0.175841,0.441290,0.557685}%
\pgfsetfillcolor{currentfill}%
\pgfsetlinewidth{0.000000pt}%
\definecolor{currentstroke}{rgb}{0.260571,0.246922,0.522828}%
\pgfsetstrokecolor{currentstroke}%
\pgfsetdash{}{0pt}%
\pgfpathmoveto{\pgfqpoint{5.188263in}{3.391183in}}%
\pgfpathlineto{\pgfqpoint{5.109597in}{3.444492in}}%
\pgfpathlineto{\pgfqpoint{5.250481in}{3.245064in}}%
\pgfpathclose%
\pgfusepath{fill}%
\end{pgfscope}%
\begin{pgfscope}%
\pgfpathrectangle{\pgfqpoint{0.539299in}{0.078740in}}{\pgfqpoint{7.842520in}{7.842520in}}%
\pgfusepath{clip}%
\pgfsetbuttcap%
\pgfsetroundjoin%
\definecolor{currentfill}{rgb}{0.730889,0.871916,0.156029}%
\pgfsetfillcolor{currentfill}%
\pgfsetlinewidth{0.000000pt}%
\definecolor{currentstroke}{rgb}{0.258965,0.251537,0.524736}%
\pgfsetstrokecolor{currentstroke}%
\pgfsetdash{}{0pt}%
\pgfpathmoveto{\pgfqpoint{3.498366in}{5.547173in}}%
\pgfpathlineto{\pgfqpoint{3.359631in}{5.571327in}}%
\pgfpathlineto{\pgfqpoint{3.272857in}{5.610573in}}%
\pgfpathclose%
\pgfusepath{fill}%
\end{pgfscope}%
\begin{pgfscope}%
\pgfpathrectangle{\pgfqpoint{0.539299in}{0.078740in}}{\pgfqpoint{7.842520in}{7.842520in}}%
\pgfusepath{clip}%
\pgfsetbuttcap%
\pgfsetroundjoin%
\definecolor{currentfill}{rgb}{0.585678,0.846661,0.249897}%
\pgfsetfillcolor{currentfill}%
\pgfsetlinewidth{0.000000pt}%
\definecolor{currentstroke}{rgb}{0.257322,0.256130,0.526563}%
\pgfsetstrokecolor{currentstroke}%
\pgfsetdash{}{0pt}%
\pgfpathmoveto{\pgfqpoint{2.694035in}{5.204781in}}%
\pgfpathlineto{\pgfqpoint{2.826074in}{5.413789in}}%
\pgfpathlineto{\pgfqpoint{2.913349in}{5.441460in}}%
\pgfpathclose%
\pgfusepath{fill}%
\end{pgfscope}%
\begin{pgfscope}%
\pgfpathrectangle{\pgfqpoint{0.539299in}{0.078740in}}{\pgfqpoint{7.842520in}{7.842520in}}%
\pgfusepath{clip}%
\pgfsetbuttcap%
\pgfsetroundjoin%
\definecolor{currentfill}{rgb}{0.506271,0.828786,0.300362}%
\pgfsetfillcolor{currentfill}%
\pgfsetlinewidth{0.000000pt}%
\definecolor{currentstroke}{rgb}{0.255645,0.260703,0.528312}%
\pgfsetstrokecolor{currentstroke}%
\pgfsetdash{}{0pt}%
\pgfpathmoveto{\pgfqpoint{3.919960in}{5.210264in}}%
\pgfpathlineto{\pgfqpoint{4.004391in}{5.111952in}}%
\pgfpathlineto{\pgfqpoint{3.778860in}{5.358543in}}%
\pgfpathclose%
\pgfusepath{fill}%
\end{pgfscope}%
\begin{pgfscope}%
\pgfpathrectangle{\pgfqpoint{0.539299in}{0.078740in}}{\pgfqpoint{7.842520in}{7.842520in}}%
\pgfusepath{clip}%
\pgfsetbuttcap%
\pgfsetroundjoin%
\definecolor{currentfill}{rgb}{0.449368,0.813768,0.335384}%
\pgfsetfillcolor{currentfill}%
\pgfsetlinewidth{0.000000pt}%
\definecolor{currentstroke}{rgb}{0.253935,0.265254,0.529983}%
\pgfsetstrokecolor{currentstroke}%
\pgfsetdash{}{0pt}%
\pgfpathmoveto{\pgfqpoint{3.919960in}{5.210264in}}%
\pgfpathlineto{\pgfqpoint{4.061290in}{5.036247in}}%
\pgfpathlineto{\pgfqpoint{4.004391in}{5.111952in}}%
\pgfpathclose%
\pgfusepath{fill}%
\end{pgfscope}%
\begin{pgfscope}%
\pgfpathrectangle{\pgfqpoint{0.539299in}{0.078740in}}{\pgfqpoint{7.842520in}{7.842520in}}%
\pgfusepath{clip}%
\pgfsetbuttcap%
\pgfsetroundjoin%
\definecolor{currentfill}{rgb}{0.281446,0.084320,0.407414}%
\pgfsetfillcolor{currentfill}%
\pgfsetlinewidth{0.000000pt}%
\definecolor{currentstroke}{rgb}{0.252194,0.269783,0.531579}%
\pgfsetstrokecolor{currentstroke}%
\pgfsetdash{}{0pt}%
\pgfpathmoveto{\pgfqpoint{6.455962in}{2.024776in}}%
\pgfpathlineto{\pgfqpoint{6.238160in}{2.086590in}}%
\pgfpathlineto{\pgfqpoint{6.380073in}{1.954750in}}%
\pgfpathclose%
\pgfusepath{fill}%
\end{pgfscope}%
\begin{pgfscope}%
\pgfpathrectangle{\pgfqpoint{0.539299in}{0.078740in}}{\pgfqpoint{7.842520in}{7.842520in}}%
\pgfusepath{clip}%
\pgfsetbuttcap%
\pgfsetroundjoin%
\definecolor{currentfill}{rgb}{0.258965,0.251537,0.524736}%
\pgfsetfillcolor{currentfill}%
\pgfsetlinewidth{0.000000pt}%
\definecolor{currentstroke}{rgb}{0.250425,0.274290,0.533103}%
\pgfsetstrokecolor{currentstroke}%
\pgfsetdash{}{0pt}%
\pgfpathmoveto{\pgfqpoint{1.520051in}{2.413317in}}%
\pgfpathlineto{\pgfqpoint{1.652813in}{2.445145in}}%
\pgfpathlineto{\pgfqpoint{1.726778in}{2.916957in}}%
\pgfpathclose%
\pgfusepath{fill}%
\end{pgfscope}%
\begin{pgfscope}%
\pgfpathrectangle{\pgfqpoint{0.539299in}{0.078740in}}{\pgfqpoint{7.842520in}{7.842520in}}%
\pgfusepath{clip}%
\pgfsetbuttcap%
\pgfsetroundjoin%
\definecolor{currentfill}{rgb}{0.304148,0.764704,0.419943}%
\pgfsetfillcolor{currentfill}%
\pgfsetlinewidth{0.000000pt}%
\definecolor{currentstroke}{rgb}{0.248629,0.278775,0.534556}%
\pgfsetstrokecolor{currentstroke}%
\pgfsetdash{}{0pt}%
\pgfpathmoveto{\pgfqpoint{2.519975in}{5.063797in}}%
\pgfpathlineto{\pgfqpoint{2.391827in}{4.772531in}}%
\pgfpathlineto{\pgfqpoint{2.305286in}{4.666266in}}%
\pgfpathclose%
\pgfusepath{fill}%
\end{pgfscope}%
\begin{pgfscope}%
\pgfpathrectangle{\pgfqpoint{0.539299in}{0.078740in}}{\pgfqpoint{7.842520in}{7.842520in}}%
\pgfusepath{clip}%
\pgfsetbuttcap%
\pgfsetroundjoin%
\definecolor{currentfill}{rgb}{0.162142,0.474838,0.558140}%
\pgfsetfillcolor{currentfill}%
\pgfsetlinewidth{0.000000pt}%
\definecolor{currentstroke}{rgb}{0.246811,0.283237,0.535941}%
\pgfsetstrokecolor{currentstroke}%
\pgfsetdash{}{0pt}%
\pgfpathmoveto{\pgfqpoint{4.968612in}{3.651484in}}%
\pgfpathlineto{\pgfqpoint{5.109597in}{3.444492in}}%
\pgfpathlineto{\pgfqpoint{5.188263in}{3.391183in}}%
\pgfpathclose%
\pgfusepath{fill}%
\end{pgfscope}%
\begin{pgfscope}%
\pgfpathrectangle{\pgfqpoint{0.539299in}{0.078740in}}{\pgfqpoint{7.842520in}{7.842520in}}%
\pgfusepath{clip}%
\pgfsetbuttcap%
\pgfsetroundjoin%
\definecolor{currentfill}{rgb}{0.277941,0.056324,0.381191}%
\pgfsetfillcolor{currentfill}%
\pgfsetlinewidth{0.000000pt}%
\definecolor{currentstroke}{rgb}{0.244972,0.287675,0.537260}%
\pgfsetstrokecolor{currentstroke}%
\pgfsetdash{}{0pt}%
\pgfpathmoveto{\pgfqpoint{6.380073in}{1.954750in}}%
\pgfpathlineto{\pgfqpoint{6.522373in}{1.832361in}}%
\pgfpathlineto{\pgfqpoint{6.455962in}{2.024776in}}%
\pgfpathclose%
\pgfusepath{fill}%
\end{pgfscope}%
\begin{pgfscope}%
\pgfpathrectangle{\pgfqpoint{0.539299in}{0.078740in}}{\pgfqpoint{7.842520in}{7.842520in}}%
\pgfusepath{clip}%
\pgfsetbuttcap%
\pgfsetroundjoin%
\definecolor{currentfill}{rgb}{0.134692,0.658636,0.517649}%
\pgfsetfillcolor{currentfill}%
\pgfsetlinewidth{0.000000pt}%
\definecolor{currentstroke}{rgb}{0.243113,0.292092,0.538516}%
\pgfsetstrokecolor{currentstroke}%
\pgfsetdash{}{0pt}%
\pgfpathmoveto{\pgfqpoint{2.219156in}{4.533644in}}%
\pgfpathlineto{\pgfqpoint{2.009371in}{4.040892in}}%
\pgfpathlineto{\pgfqpoint{2.133656in}{4.370779in}}%
\pgfpathclose%
\pgfusepath{fill}%
\end{pgfscope}%
\begin{pgfscope}%
\pgfpathrectangle{\pgfqpoint{0.539299in}{0.078740in}}{\pgfqpoint{7.842520in}{7.842520in}}%
\pgfusepath{clip}%
\pgfsetbuttcap%
\pgfsetroundjoin%
\definecolor{currentfill}{rgb}{0.216210,0.351535,0.550627}%
\pgfsetfillcolor{currentfill}%
\pgfsetlinewidth{0.000000pt}%
\definecolor{currentstroke}{rgb}{0.241237,0.296485,0.539709}%
\pgfsetstrokecolor{currentstroke}%
\pgfsetdash{}{0pt}%
\pgfpathmoveto{\pgfqpoint{1.726778in}{2.916957in}}%
\pgfpathlineto{\pgfqpoint{1.675575in}{3.148455in}}%
\pgfpathlineto{\pgfqpoint{1.596362in}{2.813436in}}%
\pgfpathclose%
\pgfusepath{fill}%
\end{pgfscope}%
\begin{pgfscope}%
\pgfpathrectangle{\pgfqpoint{0.539299in}{0.078740in}}{\pgfqpoint{7.842520in}{7.842520in}}%
\pgfusepath{clip}%
\pgfsetbuttcap%
\pgfsetroundjoin%
\definecolor{currentfill}{rgb}{0.699415,0.867117,0.175971}%
\pgfsetfillcolor{currentfill}%
\pgfsetlinewidth{0.000000pt}%
\definecolor{currentstroke}{rgb}{0.239346,0.300855,0.540844}%
\pgfsetstrokecolor{currentstroke}%
\pgfsetdash{}{0pt}%
\pgfpathmoveto{\pgfqpoint{2.913349in}{5.441460in}}%
\pgfpathlineto{\pgfqpoint{3.048388in}{5.568482in}}%
\pgfpathlineto{\pgfqpoint{3.135579in}{5.562556in}}%
\pgfpathclose%
\pgfusepath{fill}%
\end{pgfscope}%
\begin{pgfscope}%
\pgfpathrectangle{\pgfqpoint{0.539299in}{0.078740in}}{\pgfqpoint{7.842520in}{7.842520in}}%
\pgfusepath{clip}%
\pgfsetbuttcap%
\pgfsetroundjoin%
\definecolor{currentfill}{rgb}{0.127568,0.566949,0.550556}%
\pgfsetfillcolor{currentfill}%
\pgfsetlinewidth{0.000000pt}%
\definecolor{currentstroke}{rgb}{0.237441,0.305202,0.541921}%
\pgfsetstrokecolor{currentstroke}%
\pgfsetdash{}{0pt}%
\pgfpathmoveto{\pgfqpoint{1.839990in}{3.669004in}}%
\pgfpathlineto{\pgfqpoint{2.049048in}{4.173228in}}%
\pgfpathlineto{\pgfqpoint{1.924234in}{3.870625in}}%
\pgfpathclose%
\pgfusepath{fill}%
\end{pgfscope}%
\begin{pgfscope}%
\pgfpathrectangle{\pgfqpoint{0.539299in}{0.078740in}}{\pgfqpoint{7.842520in}{7.842520in}}%
\pgfusepath{clip}%
\pgfsetbuttcap%
\pgfsetroundjoin%
\definecolor{currentfill}{rgb}{0.121380,0.629492,0.531973}%
\pgfsetfillcolor{currentfill}%
\pgfsetlinewidth{0.000000pt}%
\definecolor{currentstroke}{rgb}{0.235526,0.309527,0.542944}%
\pgfsetstrokecolor{currentstroke}%
\pgfsetdash{}{0pt}%
\pgfpathmoveto{\pgfqpoint{4.766857in}{3.997620in}}%
\pgfpathlineto{\pgfqpoint{4.626092in}{4.211058in}}%
\pgfpathlineto{\pgfqpoint{4.544630in}{4.302516in}}%
\pgfpathclose%
\pgfusepath{fill}%
\end{pgfscope}%
\begin{pgfscope}%
\pgfpathrectangle{\pgfqpoint{0.539299in}{0.078740in}}{\pgfqpoint{7.842520in}{7.842520in}}%
\pgfusepath{clip}%
\pgfsetbuttcap%
\pgfsetroundjoin%
\definecolor{currentfill}{rgb}{0.136408,0.541173,0.554483}%
\pgfsetfillcolor{currentfill}%
\pgfsetlinewidth{0.000000pt}%
\definecolor{currentstroke}{rgb}{0.233603,0.313828,0.543914}%
\pgfsetstrokecolor{currentstroke}%
\pgfsetdash{}{0pt}%
\pgfpathmoveto{\pgfqpoint{4.907447in}{3.788480in}}%
\pgfpathlineto{\pgfqpoint{4.827480in}{3.864804in}}%
\pgfpathlineto{\pgfqpoint{4.968612in}{3.651484in}}%
\pgfpathclose%
\pgfusepath{fill}%
\end{pgfscope}%
\begin{pgfscope}%
\pgfpathrectangle{\pgfqpoint{0.539299in}{0.078740in}}{\pgfqpoint{7.842520in}{7.842520in}}%
\pgfusepath{clip}%
\pgfsetbuttcap%
\pgfsetroundjoin%
\definecolor{currentfill}{rgb}{0.137339,0.662252,0.515571}%
\pgfsetfillcolor{currentfill}%
\pgfsetlinewidth{0.000000pt}%
\definecolor{currentstroke}{rgb}{0.231674,0.318106,0.544834}%
\pgfsetstrokecolor{currentstroke}%
\pgfsetdash{}{0pt}%
\pgfpathmoveto{\pgfqpoint{4.626092in}{4.211058in}}%
\pgfpathlineto{\pgfqpoint{4.485130in}{4.425835in}}%
\pgfpathlineto{\pgfqpoint{4.544630in}{4.302516in}}%
\pgfpathclose%
\pgfusepath{fill}%
\end{pgfscope}%
\begin{pgfscope}%
\pgfpathrectangle{\pgfqpoint{0.539299in}{0.078740in}}{\pgfqpoint{7.842520in}{7.842520in}}%
\pgfusepath{clip}%
\pgfsetbuttcap%
\pgfsetroundjoin%
\definecolor{currentfill}{rgb}{0.177423,0.437527,0.557565}%
\pgfsetfillcolor{currentfill}%
\pgfsetlinewidth{0.000000pt}%
\definecolor{currentstroke}{rgb}{0.229739,0.322361,0.545706}%
\pgfsetstrokecolor{currentstroke}%
\pgfsetdash{}{0pt}%
\pgfpathmoveto{\pgfqpoint{1.756965in}{3.430597in}}%
\pgfpathlineto{\pgfqpoint{1.675575in}{3.148455in}}%
\pgfpathlineto{\pgfqpoint{1.804002in}{3.315816in}}%
\pgfpathclose%
\pgfusepath{fill}%
\end{pgfscope}%
\begin{pgfscope}%
\pgfpathrectangle{\pgfqpoint{0.539299in}{0.078740in}}{\pgfqpoint{7.842520in}{7.842520in}}%
\pgfusepath{clip}%
\pgfsetbuttcap%
\pgfsetroundjoin%
\definecolor{currentfill}{rgb}{0.143343,0.522773,0.556295}%
\pgfsetfillcolor{currentfill}%
\pgfsetlinewidth{0.000000pt}%
\definecolor{currentstroke}{rgb}{0.227802,0.326594,0.546532}%
\pgfsetstrokecolor{currentstroke}%
\pgfsetdash{}{0pt}%
\pgfpathmoveto{\pgfqpoint{1.965641in}{3.935827in}}%
\pgfpathlineto{\pgfqpoint{1.839990in}{3.669004in}}%
\pgfpathlineto{\pgfqpoint{1.756965in}{3.430597in}}%
\pgfpathclose%
\pgfusepath{fill}%
\end{pgfscope}%
\begin{pgfscope}%
\pgfpathrectangle{\pgfqpoint{0.539299in}{0.078740in}}{\pgfqpoint{7.842520in}{7.842520in}}%
\pgfusepath{clip}%
\pgfsetbuttcap%
\pgfsetroundjoin%
\definecolor{currentfill}{rgb}{0.535621,0.835785,0.281908}%
\pgfsetfillcolor{currentfill}%
\pgfsetlinewidth{0.000000pt}%
\definecolor{currentstroke}{rgb}{0.225863,0.330805,0.547314}%
\pgfsetstrokecolor{currentstroke}%
\pgfsetdash{}{0pt}%
\pgfpathmoveto{\pgfqpoint{2.694035in}{5.204781in}}%
\pgfpathlineto{\pgfqpoint{2.606940in}{5.146364in}}%
\pgfpathlineto{\pgfqpoint{2.738785in}{5.362101in}}%
\pgfpathclose%
\pgfusepath{fill}%
\end{pgfscope}%
\begin{pgfscope}%
\pgfpathrectangle{\pgfqpoint{0.539299in}{0.078740in}}{\pgfqpoint{7.842520in}{7.842520in}}%
\pgfusepath{clip}%
\pgfsetbuttcap%
\pgfsetroundjoin%
\definecolor{currentfill}{rgb}{0.123444,0.636809,0.528763}%
\pgfsetfillcolor{currentfill}%
\pgfsetlinewidth{0.000000pt}%
\definecolor{currentstroke}{rgb}{0.223925,0.334994,0.548053}%
\pgfsetstrokecolor{currentstroke}%
\pgfsetdash{}{0pt}%
\pgfpathmoveto{\pgfqpoint{2.133656in}{4.370779in}}%
\pgfpathlineto{\pgfqpoint{2.009371in}{4.040892in}}%
\pgfpathlineto{\pgfqpoint{2.049048in}{4.173228in}}%
\pgfpathclose%
\pgfusepath{fill}%
\end{pgfscope}%
\begin{pgfscope}%
\pgfpathrectangle{\pgfqpoint{0.539299in}{0.078740in}}{\pgfqpoint{7.842520in}{7.842520in}}%
\pgfusepath{clip}%
\pgfsetbuttcap%
\pgfsetroundjoin%
\definecolor{currentfill}{rgb}{0.255645,0.260703,0.528312}%
\pgfsetfillcolor{currentfill}%
\pgfsetlinewidth{0.000000pt}%
\definecolor{currentstroke}{rgb}{0.221989,0.339161,0.548752}%
\pgfsetstrokecolor{currentstroke}%
\pgfsetdash{}{0pt}%
\pgfpathmoveto{\pgfqpoint{5.737117in}{2.526203in}}%
\pgfpathlineto{\pgfqpoint{5.814052in}{2.532529in}}%
\pgfpathlineto{\pgfqpoint{5.673051in}{2.697726in}}%
\pgfpathclose%
\pgfusepath{fill}%
\end{pgfscope}%
\begin{pgfscope}%
\pgfpathrectangle{\pgfqpoint{0.539299in}{0.078740in}}{\pgfqpoint{7.842520in}{7.842520in}}%
\pgfusepath{clip}%
\pgfsetbuttcap%
\pgfsetroundjoin%
\definecolor{currentfill}{rgb}{0.265145,0.232956,0.516599}%
\pgfsetfillcolor{currentfill}%
\pgfsetlinewidth{0.000000pt}%
\definecolor{currentstroke}{rgb}{0.220057,0.343307,0.549413}%
\pgfsetstrokecolor{currentstroke}%
\pgfsetdash{}{0pt}%
\pgfpathmoveto{\pgfqpoint{5.737117in}{2.526203in}}%
\pgfpathlineto{\pgfqpoint{5.955205in}{2.375680in}}%
\pgfpathlineto{\pgfqpoint{5.814052in}{2.532529in}}%
\pgfpathclose%
\pgfusepath{fill}%
\end{pgfscope}%
\begin{pgfscope}%
\pgfpathrectangle{\pgfqpoint{0.539299in}{0.078740in}}{\pgfqpoint{7.842520in}{7.842520in}}%
\pgfusepath{clip}%
\pgfsetbuttcap%
\pgfsetroundjoin%
\definecolor{currentfill}{rgb}{0.751884,0.874951,0.143228}%
\pgfsetfillcolor{currentfill}%
\pgfsetlinewidth{0.000000pt}%
\definecolor{currentstroke}{rgb}{0.218130,0.347432,0.550038}%
\pgfsetstrokecolor{currentstroke}%
\pgfsetdash{}{0pt}%
\pgfpathmoveto{\pgfqpoint{3.048388in}{5.568482in}}%
\pgfpathlineto{\pgfqpoint{3.272857in}{5.610573in}}%
\pgfpathlineto{\pgfqpoint{3.135579in}{5.562556in}}%
\pgfpathclose%
\pgfusepath{fill}%
\end{pgfscope}%
\begin{pgfscope}%
\pgfpathrectangle{\pgfqpoint{0.539299in}{0.078740in}}{\pgfqpoint{7.842520in}{7.842520in}}%
\pgfusepath{clip}%
\pgfsetbuttcap%
\pgfsetroundjoin%
\definecolor{currentfill}{rgb}{0.277134,0.185228,0.489898}%
\pgfsetfillcolor{currentfill}%
\pgfsetlinewidth{0.000000pt}%
\definecolor{currentstroke}{rgb}{0.216210,0.351535,0.550627}%
\pgfsetstrokecolor{currentstroke}%
\pgfsetdash{}{0pt}%
\pgfpathmoveto{\pgfqpoint{6.096558in}{2.227023in}}%
\pgfpathlineto{\pgfqpoint{5.955205in}{2.375680in}}%
\pgfpathlineto{\pgfqpoint{5.878547in}{2.354530in}}%
\pgfpathclose%
\pgfusepath{fill}%
\end{pgfscope}%
\begin{pgfscope}%
\pgfpathrectangle{\pgfqpoint{0.539299in}{0.078740in}}{\pgfqpoint{7.842520in}{7.842520in}}%
\pgfusepath{clip}%
\pgfsetbuttcap%
\pgfsetroundjoin%
\definecolor{currentfill}{rgb}{0.730889,0.871916,0.156029}%
\pgfsetfillcolor{currentfill}%
\pgfsetlinewidth{0.000000pt}%
\definecolor{currentstroke}{rgb}{0.214298,0.355619,0.551184}%
\pgfsetstrokecolor{currentstroke}%
\pgfsetdash{}{0pt}%
\pgfpathmoveto{\pgfqpoint{3.638229in}{5.473510in}}%
\pgfpathlineto{\pgfqpoint{3.498366in}{5.547173in}}%
\pgfpathlineto{\pgfqpoint{3.411816in}{5.600141in}}%
\pgfpathclose%
\pgfusepath{fill}%
\end{pgfscope}%
\begin{pgfscope}%
\pgfpathrectangle{\pgfqpoint{0.539299in}{0.078740in}}{\pgfqpoint{7.842520in}{7.842520in}}%
\pgfusepath{clip}%
\pgfsetbuttcap%
\pgfsetroundjoin%
\definecolor{currentfill}{rgb}{0.235526,0.309527,0.542944}%
\pgfsetfillcolor{currentfill}%
\pgfsetlinewidth{0.000000pt}%
\definecolor{currentstroke}{rgb}{0.212395,0.359683,0.551710}%
\pgfsetstrokecolor{currentstroke}%
\pgfsetdash{}{0pt}%
\pgfpathmoveto{\pgfqpoint{5.532155in}{2.871486in}}%
\pgfpathlineto{\pgfqpoint{5.595772in}{2.706138in}}%
\pgfpathlineto{\pgfqpoint{5.673051in}{2.697726in}}%
\pgfpathclose%
\pgfusepath{fill}%
\end{pgfscope}%
\begin{pgfscope}%
\pgfpathrectangle{\pgfqpoint{0.539299in}{0.078740in}}{\pgfqpoint{7.842520in}{7.842520in}}%
\pgfusepath{clip}%
\pgfsetbuttcap%
\pgfsetroundjoin%
\definecolor{currentfill}{rgb}{0.121148,0.592739,0.544641}%
\pgfsetfillcolor{currentfill}%
\pgfsetlinewidth{0.000000pt}%
\definecolor{currentstroke}{rgb}{0.210503,0.363727,0.552206}%
\pgfsetstrokecolor{currentstroke}%
\pgfsetdash{}{0pt}%
\pgfpathmoveto{\pgfqpoint{4.827480in}{3.864804in}}%
\pgfpathlineto{\pgfqpoint{4.766857in}{3.997620in}}%
\pgfpathlineto{\pgfqpoint{4.686160in}{4.082636in}}%
\pgfpathclose%
\pgfusepath{fill}%
\end{pgfscope}%
\begin{pgfscope}%
\pgfpathrectangle{\pgfqpoint{0.539299in}{0.078740in}}{\pgfqpoint{7.842520in}{7.842520in}}%
\pgfusepath{clip}%
\pgfsetbuttcap%
\pgfsetroundjoin%
\definecolor{currentfill}{rgb}{0.688944,0.865448,0.182725}%
\pgfsetfillcolor{currentfill}%
\pgfsetlinewidth{0.000000pt}%
\definecolor{currentstroke}{rgb}{0.208623,0.367752,0.552675}%
\pgfsetstrokecolor{currentstroke}%
\pgfsetdash{}{0pt}%
\pgfpathmoveto{\pgfqpoint{3.048388in}{5.568482in}}%
\pgfpathlineto{\pgfqpoint{2.913349in}{5.441460in}}%
\pgfpathlineto{\pgfqpoint{2.826074in}{5.413789in}}%
\pgfpathclose%
\pgfusepath{fill}%
\end{pgfscope}%
\begin{pgfscope}%
\pgfpathrectangle{\pgfqpoint{0.539299in}{0.078740in}}{\pgfqpoint{7.842520in}{7.842520in}}%
\pgfusepath{clip}%
\pgfsetbuttcap%
\pgfsetroundjoin%
\definecolor{currentfill}{rgb}{0.283187,0.125848,0.444960}%
\pgfsetfillcolor{currentfill}%
\pgfsetlinewidth{0.000000pt}%
\definecolor{currentstroke}{rgb}{0.206756,0.371758,0.553117}%
\pgfsetstrokecolor{currentstroke}%
\pgfsetdash{}{0pt}%
\pgfpathmoveto{\pgfqpoint{6.161891in}{2.037228in}}%
\pgfpathlineto{\pgfqpoint{6.238160in}{2.086590in}}%
\pgfpathlineto{\pgfqpoint{6.096558in}{2.227023in}}%
\pgfpathclose%
\pgfusepath{fill}%
\end{pgfscope}%
\begin{pgfscope}%
\pgfpathrectangle{\pgfqpoint{0.539299in}{0.078740in}}{\pgfqpoint{7.842520in}{7.842520in}}%
\pgfusepath{clip}%
\pgfsetbuttcap%
\pgfsetroundjoin%
\definecolor{currentfill}{rgb}{0.252194,0.269783,0.531579}%
\pgfsetfillcolor{currentfill}%
\pgfsetlinewidth{0.000000pt}%
\definecolor{currentstroke}{rgb}{0.204903,0.375746,0.553533}%
\pgfsetstrokecolor{currentstroke}%
\pgfsetdash{}{0pt}%
\pgfpathmoveto{\pgfqpoint{1.726778in}{2.916957in}}%
\pgfpathlineto{\pgfqpoint{1.652813in}{2.445145in}}%
\pgfpathlineto{\pgfqpoint{1.786943in}{2.460503in}}%
\pgfpathclose%
\pgfusepath{fill}%
\end{pgfscope}%
\begin{pgfscope}%
\pgfpathrectangle{\pgfqpoint{0.539299in}{0.078740in}}{\pgfqpoint{7.842520in}{7.842520in}}%
\pgfusepath{clip}%
\pgfsetbuttcap%
\pgfsetroundjoin%
\definecolor{currentfill}{rgb}{0.762373,0.876424,0.137064}%
\pgfsetfillcolor{currentfill}%
\pgfsetlinewidth{0.000000pt}%
\definecolor{currentstroke}{rgb}{0.203063,0.379716,0.553925}%
\pgfsetstrokecolor{currentstroke}%
\pgfsetdash{}{0pt}%
\pgfpathmoveto{\pgfqpoint{3.498366in}{5.547173in}}%
\pgfpathlineto{\pgfqpoint{3.272857in}{5.610573in}}%
\pgfpathlineto{\pgfqpoint{3.411816in}{5.600141in}}%
\pgfpathclose%
\pgfusepath{fill}%
\end{pgfscope}%
\begin{pgfscope}%
\pgfpathrectangle{\pgfqpoint{0.539299in}{0.078740in}}{\pgfqpoint{7.842520in}{7.842520in}}%
\pgfusepath{clip}%
\pgfsetbuttcap%
\pgfsetroundjoin%
\definecolor{currentfill}{rgb}{0.206756,0.371758,0.553117}%
\pgfsetfillcolor{currentfill}%
\pgfsetlinewidth{0.000000pt}%
\definecolor{currentstroke}{rgb}{0.201239,0.383670,0.554294}%
\pgfsetstrokecolor{currentstroke}%
\pgfsetdash{}{0pt}%
\pgfpathmoveto{\pgfqpoint{5.532155in}{2.871486in}}%
\pgfpathlineto{\pgfqpoint{5.391316in}{3.053947in}}%
\pgfpathlineto{\pgfqpoint{5.313135in}{3.090200in}}%
\pgfpathclose%
\pgfusepath{fill}%
\end{pgfscope}%
\begin{pgfscope}%
\pgfpathrectangle{\pgfqpoint{0.539299in}{0.078740in}}{\pgfqpoint{7.842520in}{7.842520in}}%
\pgfusepath{clip}%
\pgfsetbuttcap%
\pgfsetroundjoin%
\definecolor{currentfill}{rgb}{0.606045,0.850733,0.236712}%
\pgfsetfillcolor{currentfill}%
\pgfsetlinewidth{0.000000pt}%
\definecolor{currentstroke}{rgb}{0.199430,0.387607,0.554642}%
\pgfsetstrokecolor{currentstroke}%
\pgfsetdash{}{0pt}%
\pgfpathmoveto{\pgfqpoint{2.738785in}{5.362101in}}%
\pgfpathlineto{\pgfqpoint{2.826074in}{5.413789in}}%
\pgfpathlineto{\pgfqpoint{2.694035in}{5.204781in}}%
\pgfpathclose%
\pgfusepath{fill}%
\end{pgfscope}%
\begin{pgfscope}%
\pgfpathrectangle{\pgfqpoint{0.539299in}{0.078740in}}{\pgfqpoint{7.842520in}{7.842520in}}%
\pgfusepath{clip}%
\pgfsetbuttcap%
\pgfsetroundjoin%
\definecolor{currentfill}{rgb}{0.281446,0.084320,0.407414}%
\pgfsetfillcolor{currentfill}%
\pgfsetlinewidth{0.000000pt}%
\definecolor{currentstroke}{rgb}{0.197636,0.391528,0.554969}%
\pgfsetstrokecolor{currentstroke}%
\pgfsetdash{}{0pt}%
\pgfpathmoveto{\pgfqpoint{6.238160in}{2.086590in}}%
\pgfpathlineto{\pgfqpoint{6.303947in}{1.893136in}}%
\pgfpathlineto{\pgfqpoint{6.380073in}{1.954750in}}%
\pgfpathclose%
\pgfusepath{fill}%
\end{pgfscope}%
\begin{pgfscope}%
\pgfpathrectangle{\pgfqpoint{0.539299in}{0.078740in}}{\pgfqpoint{7.842520in}{7.842520in}}%
\pgfusepath{clip}%
\pgfsetbuttcap%
\pgfsetroundjoin%
\definecolor{currentfill}{rgb}{0.226397,0.728888,0.462789}%
\pgfsetfillcolor{currentfill}%
\pgfsetlinewidth{0.000000pt}%
\definecolor{currentstroke}{rgb}{0.195860,0.395433,0.555276}%
\pgfsetstrokecolor{currentstroke}%
\pgfsetdash{}{0pt}%
\pgfpathmoveto{\pgfqpoint{4.260950in}{4.734983in}}%
\pgfpathlineto{\pgfqpoint{4.485130in}{4.425835in}}%
\pgfpathlineto{\pgfqpoint{4.343978in}{4.638176in}}%
\pgfpathclose%
\pgfusepath{fill}%
\end{pgfscope}%
\begin{pgfscope}%
\pgfpathrectangle{\pgfqpoint{0.539299in}{0.078740in}}{\pgfqpoint{7.842520in}{7.842520in}}%
\pgfusepath{clip}%
\pgfsetbuttcap%
\pgfsetroundjoin%
\definecolor{currentfill}{rgb}{0.120638,0.625828,0.533488}%
\pgfsetfillcolor{currentfill}%
\pgfsetlinewidth{0.000000pt}%
\definecolor{currentstroke}{rgb}{0.194100,0.399323,0.555565}%
\pgfsetstrokecolor{currentstroke}%
\pgfsetdash{}{0pt}%
\pgfpathmoveto{\pgfqpoint{4.686160in}{4.082636in}}%
\pgfpathlineto{\pgfqpoint{4.766857in}{3.997620in}}%
\pgfpathlineto{\pgfqpoint{4.544630in}{4.302516in}}%
\pgfpathclose%
\pgfusepath{fill}%
\end{pgfscope}%
\begin{pgfscope}%
\pgfpathrectangle{\pgfqpoint{0.539299in}{0.078740in}}{\pgfqpoint{7.842520in}{7.842520in}}%
\pgfusepath{clip}%
\pgfsetbuttcap%
\pgfsetroundjoin%
\definecolor{currentfill}{rgb}{0.192357,0.403199,0.555836}%
\pgfsetfillcolor{currentfill}%
\pgfsetlinewidth{0.000000pt}%
\definecolor{currentstroke}{rgb}{0.192357,0.403199,0.555836}%
\pgfsetstrokecolor{currentstroke}%
\pgfsetdash{}{0pt}%
\pgfpathmoveto{\pgfqpoint{5.313135in}{3.090200in}}%
\pgfpathlineto{\pgfqpoint{5.391316in}{3.053947in}}%
\pgfpathlineto{\pgfqpoint{5.250481in}{3.245064in}}%
\pgfpathclose%
\pgfusepath{fill}%
\end{pgfscope}%
\begin{pgfscope}%
\pgfpathrectangle{\pgfqpoint{0.539299in}{0.078740in}}{\pgfqpoint{7.842520in}{7.842520in}}%
\pgfusepath{clip}%
\pgfsetbuttcap%
\pgfsetroundjoin%
\definecolor{currentfill}{rgb}{0.192357,0.403199,0.555836}%
\pgfsetfillcolor{currentfill}%
\pgfsetlinewidth{0.000000pt}%
\definecolor{currentstroke}{rgb}{0.190631,0.407061,0.556089}%
\pgfsetstrokecolor{currentstroke}%
\pgfsetdash{}{0pt}%
\pgfpathmoveto{\pgfqpoint{1.804002in}{3.315816in}}%
\pgfpathlineto{\pgfqpoint{1.675575in}{3.148455in}}%
\pgfpathlineto{\pgfqpoint{1.726778in}{2.916957in}}%
\pgfpathclose%
\pgfusepath{fill}%
\end{pgfscope}%
\begin{pgfscope}%
\pgfpathrectangle{\pgfqpoint{0.539299in}{0.078740in}}{\pgfqpoint{7.842520in}{7.842520in}}%
\pgfusepath{clip}%
\pgfsetbuttcap%
\pgfsetroundjoin%
\definecolor{currentfill}{rgb}{0.288921,0.758394,0.428426}%
\pgfsetfillcolor{currentfill}%
\pgfsetlinewidth{0.000000pt}%
\definecolor{currentstroke}{rgb}{0.188923,0.410910,0.556326}%
\pgfsetstrokecolor{currentstroke}%
\pgfsetdash{}{0pt}%
\pgfpathmoveto{\pgfqpoint{4.260950in}{4.734983in}}%
\pgfpathlineto{\pgfqpoint{4.343978in}{4.638176in}}%
\pgfpathlineto{\pgfqpoint{4.202671in}{4.843466in}}%
\pgfpathclose%
\pgfusepath{fill}%
\end{pgfscope}%
\begin{pgfscope}%
\pgfpathrectangle{\pgfqpoint{0.539299in}{0.078740in}}{\pgfqpoint{7.842520in}{7.842520in}}%
\pgfusepath{clip}%
\pgfsetbuttcap%
\pgfsetroundjoin%
\definecolor{currentfill}{rgb}{0.276022,0.044167,0.370164}%
\pgfsetfillcolor{currentfill}%
\pgfsetlinewidth{0.000000pt}%
\definecolor{currentstroke}{rgb}{0.187231,0.414746,0.556547}%
\pgfsetstrokecolor{currentstroke}%
\pgfsetdash{}{0pt}%
\pgfpathmoveto{\pgfqpoint{6.446395in}{1.760841in}}%
\pgfpathlineto{\pgfqpoint{6.522373in}{1.832361in}}%
\pgfpathlineto{\pgfqpoint{6.380073in}{1.954750in}}%
\pgfpathclose%
\pgfusepath{fill}%
\end{pgfscope}%
\begin{pgfscope}%
\pgfpathrectangle{\pgfqpoint{0.539299in}{0.078740in}}{\pgfqpoint{7.842520in}{7.842520in}}%
\pgfusepath{clip}%
\pgfsetbuttcap%
\pgfsetroundjoin%
\definecolor{currentfill}{rgb}{0.688944,0.865448,0.182725}%
\pgfsetfillcolor{currentfill}%
\pgfsetlinewidth{0.000000pt}%
\definecolor{currentstroke}{rgb}{0.185556,0.418570,0.556753}%
\pgfsetstrokecolor{currentstroke}%
\pgfsetdash{}{0pt}%
\pgfpathmoveto{\pgfqpoint{3.778860in}{5.358543in}}%
\pgfpathlineto{\pgfqpoint{3.638229in}{5.473510in}}%
\pgfpathlineto{\pgfqpoint{3.552031in}{5.538964in}}%
\pgfpathclose%
\pgfusepath{fill}%
\end{pgfscope}%
\begin{pgfscope}%
\pgfpathrectangle{\pgfqpoint{0.539299in}{0.078740in}}{\pgfqpoint{7.842520in}{7.842520in}}%
\pgfusepath{clip}%
\pgfsetbuttcap%
\pgfsetroundjoin%
\definecolor{currentfill}{rgb}{0.267968,0.223549,0.512008}%
\pgfsetfillcolor{currentfill}%
\pgfsetlinewidth{0.000000pt}%
\definecolor{currentstroke}{rgb}{0.183898,0.422383,0.556944}%
\pgfsetstrokecolor{currentstroke}%
\pgfsetdash{}{0pt}%
\pgfpathmoveto{\pgfqpoint{5.878547in}{2.354530in}}%
\pgfpathlineto{\pgfqpoint{5.955205in}{2.375680in}}%
\pgfpathlineto{\pgfqpoint{5.737117in}{2.526203in}}%
\pgfpathclose%
\pgfusepath{fill}%
\end{pgfscope}%
\begin{pgfscope}%
\pgfpathrectangle{\pgfqpoint{0.539299in}{0.078740in}}{\pgfqpoint{7.842520in}{7.842520in}}%
\pgfusepath{clip}%
\pgfsetbuttcap%
\pgfsetroundjoin%
\definecolor{currentfill}{rgb}{0.278826,0.175490,0.483397}%
\pgfsetfillcolor{currentfill}%
\pgfsetlinewidth{0.000000pt}%
\definecolor{currentstroke}{rgb}{0.182256,0.426184,0.557120}%
\pgfsetstrokecolor{currentstroke}%
\pgfsetdash{}{0pt}%
\pgfpathmoveto{\pgfqpoint{6.096558in}{2.227023in}}%
\pgfpathlineto{\pgfqpoint{5.878547in}{2.354530in}}%
\pgfpathlineto{\pgfqpoint{6.020118in}{2.191364in}}%
\pgfpathclose%
\pgfusepath{fill}%
\end{pgfscope}%
\begin{pgfscope}%
\pgfpathrectangle{\pgfqpoint{0.539299in}{0.078740in}}{\pgfqpoint{7.842520in}{7.842520in}}%
\pgfusepath{clip}%
\pgfsetbuttcap%
\pgfsetroundjoin%
\definecolor{currentfill}{rgb}{0.282290,0.145912,0.461510}%
\pgfsetfillcolor{currentfill}%
\pgfsetlinewidth{0.000000pt}%
\definecolor{currentstroke}{rgb}{0.180629,0.429975,0.557282}%
\pgfsetstrokecolor{currentstroke}%
\pgfsetdash{}{0pt}%
\pgfpathmoveto{\pgfqpoint{6.020118in}{2.191364in}}%
\pgfpathlineto{\pgfqpoint{6.161891in}{2.037228in}}%
\pgfpathlineto{\pgfqpoint{6.096558in}{2.227023in}}%
\pgfpathclose%
\pgfusepath{fill}%
\end{pgfscope}%
\begin{pgfscope}%
\pgfpathrectangle{\pgfqpoint{0.539299in}{0.078740in}}{\pgfqpoint{7.842520in}{7.842520in}}%
\pgfusepath{clip}%
\pgfsetbuttcap%
\pgfsetroundjoin%
\definecolor{currentfill}{rgb}{0.244972,0.287675,0.537260}%
\pgfsetfillcolor{currentfill}%
\pgfsetlinewidth{0.000000pt}%
\definecolor{currentstroke}{rgb}{0.179019,0.433756,0.557430}%
\pgfsetstrokecolor{currentstroke}%
\pgfsetdash{}{0pt}%
\pgfpathmoveto{\pgfqpoint{5.673051in}{2.697726in}}%
\pgfpathlineto{\pgfqpoint{5.595772in}{2.706138in}}%
\pgfpathlineto{\pgfqpoint{5.737117in}{2.526203in}}%
\pgfpathclose%
\pgfusepath{fill}%
\end{pgfscope}%
\begin{pgfscope}%
\pgfpathrectangle{\pgfqpoint{0.539299in}{0.078740in}}{\pgfqpoint{7.842520in}{7.842520in}}%
\pgfusepath{clip}%
\pgfsetbuttcap%
\pgfsetroundjoin%
\definecolor{currentfill}{rgb}{0.122606,0.585371,0.546557}%
\pgfsetfillcolor{currentfill}%
\pgfsetlinewidth{0.000000pt}%
\definecolor{currentstroke}{rgb}{0.177423,0.437527,0.557565}%
\pgfsetstrokecolor{currentstroke}%
\pgfsetdash{}{0pt}%
\pgfpathmoveto{\pgfqpoint{1.965641in}{3.935827in}}%
\pgfpathlineto{\pgfqpoint{2.049048in}{4.173228in}}%
\pgfpathlineto{\pgfqpoint{1.839990in}{3.669004in}}%
\pgfpathclose%
\pgfusepath{fill}%
\end{pgfscope}%
\begin{pgfscope}%
\pgfpathrectangle{\pgfqpoint{0.539299in}{0.078740in}}{\pgfqpoint{7.842520in}{7.842520in}}%
\pgfusepath{clip}%
\pgfsetbuttcap%
\pgfsetroundjoin%
\definecolor{currentfill}{rgb}{0.165117,0.467423,0.558141}%
\pgfsetfillcolor{currentfill}%
\pgfsetlinewidth{0.000000pt}%
\definecolor{currentstroke}{rgb}{0.175841,0.441290,0.557685}%
\pgfsetstrokecolor{currentstroke}%
\pgfsetdash{}{0pt}%
\pgfpathmoveto{\pgfqpoint{5.250481in}{3.245064in}}%
\pgfpathlineto{\pgfqpoint{5.109597in}{3.444492in}}%
\pgfpathlineto{\pgfqpoint{5.030239in}{3.504331in}}%
\pgfpathclose%
\pgfusepath{fill}%
\end{pgfscope}%
\begin{pgfscope}%
\pgfpathrectangle{\pgfqpoint{0.539299in}{0.078740in}}{\pgfqpoint{7.842520in}{7.842520in}}%
\pgfusepath{clip}%
\pgfsetbuttcap%
\pgfsetroundjoin%
\definecolor{currentfill}{rgb}{0.369214,0.788888,0.382914}%
\pgfsetfillcolor{currentfill}%
\pgfsetlinewidth{0.000000pt}%
\definecolor{currentstroke}{rgb}{0.174274,0.445044,0.557792}%
\pgfsetstrokecolor{currentstroke}%
\pgfsetdash{}{0pt}%
\pgfpathmoveto{\pgfqpoint{2.305286in}{4.666266in}}%
\pgfpathlineto{\pgfqpoint{2.433328in}{4.953363in}}%
\pgfpathlineto{\pgfqpoint{2.519975in}{5.063797in}}%
\pgfpathclose%
\pgfusepath{fill}%
\end{pgfscope}%
\begin{pgfscope}%
\pgfpathrectangle{\pgfqpoint{0.539299in}{0.078740in}}{\pgfqpoint{7.842520in}{7.842520in}}%
\pgfusepath{clip}%
\pgfsetbuttcap%
\pgfsetroundjoin%
\definecolor{currentfill}{rgb}{0.259857,0.745492,0.444467}%
\pgfsetfillcolor{currentfill}%
\pgfsetlinewidth{0.000000pt}%
\definecolor{currentstroke}{rgb}{0.172719,0.448791,0.557885}%
\pgfsetstrokecolor{currentstroke}%
\pgfsetdash{}{0pt}%
\pgfpathmoveto{\pgfqpoint{2.305286in}{4.666266in}}%
\pgfpathlineto{\pgfqpoint{2.219156in}{4.533644in}}%
\pgfpathlineto{\pgfqpoint{2.347219in}{4.810962in}}%
\pgfpathclose%
\pgfusepath{fill}%
\end{pgfscope}%
\begin{pgfscope}%
\pgfpathrectangle{\pgfqpoint{0.539299in}{0.078740in}}{\pgfqpoint{7.842520in}{7.842520in}}%
\pgfusepath{clip}%
\pgfsetbuttcap%
\pgfsetroundjoin%
\definecolor{currentfill}{rgb}{0.386433,0.794644,0.372886}%
\pgfsetfillcolor{currentfill}%
\pgfsetlinewidth{0.000000pt}%
\definecolor{currentstroke}{rgb}{0.171176,0.452530,0.557965}%
\pgfsetstrokecolor{currentstroke}%
\pgfsetdash{}{0pt}%
\pgfpathmoveto{\pgfqpoint{4.061290in}{5.036247in}}%
\pgfpathlineto{\pgfqpoint{4.118884in}{4.938971in}}%
\pgfpathlineto{\pgfqpoint{4.202671in}{4.843466in}}%
\pgfpathclose%
\pgfusepath{fill}%
\end{pgfscope}%
\begin{pgfscope}%
\pgfpathrectangle{\pgfqpoint{0.539299in}{0.078740in}}{\pgfqpoint{7.842520in}{7.842520in}}%
\pgfusepath{clip}%
\pgfsetbuttcap%
\pgfsetroundjoin%
\definecolor{currentfill}{rgb}{0.282910,0.105393,0.426902}%
\pgfsetfillcolor{currentfill}%
\pgfsetlinewidth{0.000000pt}%
\definecolor{currentstroke}{rgb}{0.169646,0.456262,0.558030}%
\pgfsetstrokecolor{currentstroke}%
\pgfsetdash{}{0pt}%
\pgfpathmoveto{\pgfqpoint{6.238160in}{2.086590in}}%
\pgfpathlineto{\pgfqpoint{6.161891in}{2.037228in}}%
\pgfpathlineto{\pgfqpoint{6.303947in}{1.893136in}}%
\pgfpathclose%
\pgfusepath{fill}%
\end{pgfscope}%
\begin{pgfscope}%
\pgfpathrectangle{\pgfqpoint{0.539299in}{0.078740in}}{\pgfqpoint{7.842520in}{7.842520in}}%
\pgfusepath{clip}%
\pgfsetbuttcap%
\pgfsetroundjoin%
\definecolor{currentfill}{rgb}{0.223925,0.334994,0.548053}%
\pgfsetfillcolor{currentfill}%
\pgfsetlinewidth{0.000000pt}%
\definecolor{currentstroke}{rgb}{0.168126,0.459988,0.558082}%
\pgfsetstrokecolor{currentstroke}%
\pgfsetdash{}{0pt}%
\pgfpathmoveto{\pgfqpoint{5.532155in}{2.871486in}}%
\pgfpathlineto{\pgfqpoint{5.454461in}{2.894198in}}%
\pgfpathlineto{\pgfqpoint{5.595772in}{2.706138in}}%
\pgfpathclose%
\pgfusepath{fill}%
\end{pgfscope}%
\begin{pgfscope}%
\pgfpathrectangle{\pgfqpoint{0.539299in}{0.078740in}}{\pgfqpoint{7.842520in}{7.842520in}}%
\pgfusepath{clip}%
\pgfsetbuttcap%
\pgfsetroundjoin%
\definecolor{currentfill}{rgb}{0.157729,0.485932,0.558013}%
\pgfsetfillcolor{currentfill}%
\pgfsetlinewidth{0.000000pt}%
\definecolor{currentstroke}{rgb}{0.166617,0.463708,0.558119}%
\pgfsetstrokecolor{currentstroke}%
\pgfsetdash{}{0pt}%
\pgfpathmoveto{\pgfqpoint{1.804002in}{3.315816in}}%
\pgfpathlineto{\pgfqpoint{1.883808in}{3.652469in}}%
\pgfpathlineto{\pgfqpoint{1.756965in}{3.430597in}}%
\pgfpathclose%
\pgfusepath{fill}%
\end{pgfscope}%
\begin{pgfscope}%
\pgfpathrectangle{\pgfqpoint{0.539299in}{0.078740in}}{\pgfqpoint{7.842520in}{7.842520in}}%
\pgfusepath{clip}%
\pgfsetbuttcap%
\pgfsetroundjoin%
\definecolor{currentfill}{rgb}{0.170948,0.694384,0.493803}%
\pgfsetfillcolor{currentfill}%
\pgfsetlinewidth{0.000000pt}%
\definecolor{currentstroke}{rgb}{0.165117,0.467423,0.558141}%
\pgfsetstrokecolor{currentstroke}%
\pgfsetdash{}{0pt}%
\pgfpathmoveto{\pgfqpoint{4.544630in}{4.302516in}}%
\pgfpathlineto{\pgfqpoint{4.485130in}{4.425835in}}%
\pgfpathlineto{\pgfqpoint{4.402885in}{4.521271in}}%
\pgfpathclose%
\pgfusepath{fill}%
\end{pgfscope}%
\begin{pgfscope}%
\pgfpathrectangle{\pgfqpoint{0.539299in}{0.078740in}}{\pgfqpoint{7.842520in}{7.842520in}}%
\pgfusepath{clip}%
\pgfsetbuttcap%
\pgfsetroundjoin%
\definecolor{currentfill}{rgb}{0.151918,0.500685,0.557587}%
\pgfsetfillcolor{currentfill}%
\pgfsetlinewidth{0.000000pt}%
\definecolor{currentstroke}{rgb}{0.163625,0.471133,0.558148}%
\pgfsetstrokecolor{currentstroke}%
\pgfsetdash{}{0pt}%
\pgfpathmoveto{\pgfqpoint{5.030239in}{3.504331in}}%
\pgfpathlineto{\pgfqpoint{5.109597in}{3.444492in}}%
\pgfpathlineto{\pgfqpoint{4.968612in}{3.651484in}}%
\pgfpathclose%
\pgfusepath{fill}%
\end{pgfscope}%
\begin{pgfscope}%
\pgfpathrectangle{\pgfqpoint{0.539299in}{0.078740in}}{\pgfqpoint{7.842520in}{7.842520in}}%
\pgfusepath{clip}%
\pgfsetbuttcap%
\pgfsetroundjoin%
\definecolor{currentfill}{rgb}{0.535621,0.835785,0.281908}%
\pgfsetfillcolor{currentfill}%
\pgfsetlinewidth{0.000000pt}%
\definecolor{currentstroke}{rgb}{0.162142,0.474838,0.558140}%
\pgfsetstrokecolor{currentstroke}%
\pgfsetdash{}{0pt}%
\pgfpathmoveto{\pgfqpoint{2.738785in}{5.362101in}}%
\pgfpathlineto{\pgfqpoint{2.606940in}{5.146364in}}%
\pgfpathlineto{\pgfqpoint{2.519975in}{5.063797in}}%
\pgfpathclose%
\pgfusepath{fill}%
\end{pgfscope}%
\begin{pgfscope}%
\pgfpathrectangle{\pgfqpoint{0.539299in}{0.078740in}}{\pgfqpoint{7.842520in}{7.842520in}}%
\pgfusepath{clip}%
\pgfsetbuttcap%
\pgfsetroundjoin%
\definecolor{currentfill}{rgb}{0.595839,0.848717,0.243329}%
\pgfsetfillcolor{currentfill}%
\pgfsetlinewidth{0.000000pt}%
\definecolor{currentstroke}{rgb}{0.160665,0.478540,0.558115}%
\pgfsetstrokecolor{currentstroke}%
\pgfsetdash{}{0pt}%
\pgfpathmoveto{\pgfqpoint{3.834801in}{5.295343in}}%
\pgfpathlineto{\pgfqpoint{3.919960in}{5.210264in}}%
\pgfpathlineto{\pgfqpoint{3.778860in}{5.358543in}}%
\pgfpathclose%
\pgfusepath{fill}%
\end{pgfscope}%
\begin{pgfscope}%
\pgfpathrectangle{\pgfqpoint{0.539299in}{0.078740in}}{\pgfqpoint{7.842520in}{7.842520in}}%
\pgfusepath{clip}%
\pgfsetbuttcap%
\pgfsetroundjoin%
\definecolor{currentfill}{rgb}{0.525776,0.833491,0.288127}%
\pgfsetfillcolor{currentfill}%
\pgfsetlinewidth{0.000000pt}%
\definecolor{currentstroke}{rgb}{0.159194,0.482237,0.558073}%
\pgfsetstrokecolor{currentstroke}%
\pgfsetdash{}{0pt}%
\pgfpathmoveto{\pgfqpoint{4.061290in}{5.036247in}}%
\pgfpathlineto{\pgfqpoint{3.919960in}{5.210264in}}%
\pgfpathlineto{\pgfqpoint{3.834801in}{5.295343in}}%
\pgfpathclose%
\pgfusepath{fill}%
\end{pgfscope}%
\begin{pgfscope}%
\pgfpathrectangle{\pgfqpoint{0.539299in}{0.078740in}}{\pgfqpoint{7.842520in}{7.842520in}}%
\pgfusepath{clip}%
\pgfsetbuttcap%
\pgfsetroundjoin%
\definecolor{currentfill}{rgb}{0.208623,0.367752,0.552675}%
\pgfsetfillcolor{currentfill}%
\pgfsetlinewidth{0.000000pt}%
\definecolor{currentstroke}{rgb}{0.157729,0.485932,0.558013}%
\pgfsetstrokecolor{currentstroke}%
\pgfsetdash{}{0pt}%
\pgfpathmoveto{\pgfqpoint{5.313135in}{3.090200in}}%
\pgfpathlineto{\pgfqpoint{5.454461in}{2.894198in}}%
\pgfpathlineto{\pgfqpoint{5.532155in}{2.871486in}}%
\pgfpathclose%
\pgfusepath{fill}%
\end{pgfscope}%
\begin{pgfscope}%
\pgfpathrectangle{\pgfqpoint{0.539299in}{0.078740in}}{\pgfqpoint{7.842520in}{7.842520in}}%
\pgfusepath{clip}%
\pgfsetbuttcap%
\pgfsetroundjoin%
\definecolor{currentfill}{rgb}{0.139147,0.533812,0.555298}%
\pgfsetfillcolor{currentfill}%
\pgfsetlinewidth{0.000000pt}%
\definecolor{currentstroke}{rgb}{0.156270,0.489624,0.557936}%
\pgfsetstrokecolor{currentstroke}%
\pgfsetdash{}{0pt}%
\pgfpathmoveto{\pgfqpoint{1.756965in}{3.430597in}}%
\pgfpathlineto{\pgfqpoint{1.883808in}{3.652469in}}%
\pgfpathlineto{\pgfqpoint{1.965641in}{3.935827in}}%
\pgfpathclose%
\pgfusepath{fill}%
\end{pgfscope}%
\begin{pgfscope}%
\pgfpathrectangle{\pgfqpoint{0.539299in}{0.078740in}}{\pgfqpoint{7.842520in}{7.842520in}}%
\pgfusepath{clip}%
\pgfsetbuttcap%
\pgfsetroundjoin%
\definecolor{currentfill}{rgb}{0.278791,0.062145,0.386592}%
\pgfsetfillcolor{currentfill}%
\pgfsetlinewidth{0.000000pt}%
\definecolor{currentstroke}{rgb}{0.154815,0.493313,0.557840}%
\pgfsetstrokecolor{currentstroke}%
\pgfsetdash{}{0pt}%
\pgfpathmoveto{\pgfqpoint{6.380073in}{1.954750in}}%
\pgfpathlineto{\pgfqpoint{6.303947in}{1.893136in}}%
\pgfpathlineto{\pgfqpoint{6.446395in}{1.760841in}}%
\pgfpathclose%
\pgfusepath{fill}%
\end{pgfscope}%
\begin{pgfscope}%
\pgfpathrectangle{\pgfqpoint{0.539299in}{0.078740in}}{\pgfqpoint{7.842520in}{7.842520in}}%
\pgfusepath{clip}%
\pgfsetbuttcap%
\pgfsetroundjoin%
\definecolor{currentfill}{rgb}{0.751884,0.874951,0.143228}%
\pgfsetfillcolor{currentfill}%
\pgfsetlinewidth{0.000000pt}%
\definecolor{currentstroke}{rgb}{0.153364,0.497000,0.557724}%
\pgfsetstrokecolor{currentstroke}%
\pgfsetdash{}{0pt}%
\pgfpathmoveto{\pgfqpoint{3.411816in}{5.600141in}}%
\pgfpathlineto{\pgfqpoint{3.552031in}{5.538964in}}%
\pgfpathlineto{\pgfqpoint{3.638229in}{5.473510in}}%
\pgfpathclose%
\pgfusepath{fill}%
\end{pgfscope}%
\begin{pgfscope}%
\pgfpathrectangle{\pgfqpoint{0.539299in}{0.078740in}}{\pgfqpoint{7.842520in}{7.842520in}}%
\pgfusepath{clip}%
\pgfsetbuttcap%
\pgfsetroundjoin%
\definecolor{currentfill}{rgb}{0.220124,0.725509,0.466226}%
\pgfsetfillcolor{currentfill}%
\pgfsetlinewidth{0.000000pt}%
\definecolor{currentstroke}{rgb}{0.151918,0.500685,0.557587}%
\pgfsetstrokecolor{currentstroke}%
\pgfsetdash{}{0pt}%
\pgfpathmoveto{\pgfqpoint{4.402885in}{4.521271in}}%
\pgfpathlineto{\pgfqpoint{4.485130in}{4.425835in}}%
\pgfpathlineto{\pgfqpoint{4.260950in}{4.734983in}}%
\pgfpathclose%
\pgfusepath{fill}%
\end{pgfscope}%
\begin{pgfscope}%
\pgfpathrectangle{\pgfqpoint{0.539299in}{0.078740in}}{\pgfqpoint{7.842520in}{7.842520in}}%
\pgfusepath{clip}%
\pgfsetbuttcap%
\pgfsetroundjoin%
\definecolor{currentfill}{rgb}{0.225863,0.330805,0.547314}%
\pgfsetfillcolor{currentfill}%
\pgfsetlinewidth{0.000000pt}%
\definecolor{currentstroke}{rgb}{0.150476,0.504369,0.557430}%
\pgfsetstrokecolor{currentstroke}%
\pgfsetdash{}{0pt}%
\pgfpathmoveto{\pgfqpoint{1.786943in}{2.460503in}}%
\pgfpathlineto{\pgfqpoint{1.859345in}{2.989662in}}%
\pgfpathlineto{\pgfqpoint{1.726778in}{2.916957in}}%
\pgfpathclose%
\pgfusepath{fill}%
\end{pgfscope}%
\begin{pgfscope}%
\pgfpathrectangle{\pgfqpoint{0.539299in}{0.078740in}}{\pgfqpoint{7.842520in}{7.842520in}}%
\pgfusepath{clip}%
\pgfsetbuttcap%
\pgfsetroundjoin%
\definecolor{currentfill}{rgb}{0.804182,0.882046,0.114965}%
\pgfsetfillcolor{currentfill}%
\pgfsetlinewidth{0.000000pt}%
\definecolor{currentstroke}{rgb}{0.149039,0.508051,0.557250}%
\pgfsetstrokecolor{currentstroke}%
\pgfsetdash{}{0pt}%
\pgfpathmoveto{\pgfqpoint{3.185707in}{5.629892in}}%
\pgfpathlineto{\pgfqpoint{3.272857in}{5.610573in}}%
\pgfpathlineto{\pgfqpoint{3.048388in}{5.568482in}}%
\pgfpathclose%
\pgfusepath{fill}%
\end{pgfscope}%
\begin{pgfscope}%
\pgfpathrectangle{\pgfqpoint{0.539299in}{0.078740in}}{\pgfqpoint{7.842520in}{7.842520in}}%
\pgfusepath{clip}%
\pgfsetbuttcap%
\pgfsetroundjoin%
\definecolor{currentfill}{rgb}{0.180629,0.429975,0.557282}%
\pgfsetfillcolor{currentfill}%
\pgfsetlinewidth{0.000000pt}%
\definecolor{currentstroke}{rgb}{0.147607,0.511733,0.557049}%
\pgfsetstrokecolor{currentstroke}%
\pgfsetdash{}{0pt}%
\pgfpathmoveto{\pgfqpoint{5.171743in}{3.293790in}}%
\pgfpathlineto{\pgfqpoint{5.313135in}{3.090200in}}%
\pgfpathlineto{\pgfqpoint{5.250481in}{3.245064in}}%
\pgfpathclose%
\pgfusepath{fill}%
\end{pgfscope}%
\begin{pgfscope}%
\pgfpathrectangle{\pgfqpoint{0.539299in}{0.078740in}}{\pgfqpoint{7.842520in}{7.842520in}}%
\pgfusepath{clip}%
\pgfsetbuttcap%
\pgfsetroundjoin%
\definecolor{currentfill}{rgb}{0.751884,0.874951,0.143228}%
\pgfsetfillcolor{currentfill}%
\pgfsetlinewidth{0.000000pt}%
\definecolor{currentstroke}{rgb}{0.146180,0.515413,0.556823}%
\pgfsetstrokecolor{currentstroke}%
\pgfsetdash{}{0pt}%
\pgfpathmoveto{\pgfqpoint{2.826074in}{5.413789in}}%
\pgfpathlineto{\pgfqpoint{2.961006in}{5.551827in}}%
\pgfpathlineto{\pgfqpoint{3.048388in}{5.568482in}}%
\pgfpathclose%
\pgfusepath{fill}%
\end{pgfscope}%
\begin{pgfscope}%
\pgfpathrectangle{\pgfqpoint{0.539299in}{0.078740in}}{\pgfqpoint{7.842520in}{7.842520in}}%
\pgfusepath{clip}%
\pgfsetbuttcap%
\pgfsetroundjoin%
\definecolor{currentfill}{rgb}{0.232815,0.732247,0.459277}%
\pgfsetfillcolor{currentfill}%
\pgfsetlinewidth{0.000000pt}%
\definecolor{currentstroke}{rgb}{0.144759,0.519093,0.556572}%
\pgfsetstrokecolor{currentstroke}%
\pgfsetdash{}{0pt}%
\pgfpathmoveto{\pgfqpoint{2.347219in}{4.810962in}}%
\pgfpathlineto{\pgfqpoint{2.219156in}{4.533644in}}%
\pgfpathlineto{\pgfqpoint{2.133656in}{4.370779in}}%
\pgfpathclose%
\pgfusepath{fill}%
\end{pgfscope}%
\begin{pgfscope}%
\pgfpathrectangle{\pgfqpoint{0.539299in}{0.078740in}}{\pgfqpoint{7.842520in}{7.842520in}}%
\pgfusepath{clip}%
\pgfsetbuttcap%
\pgfsetroundjoin%
\definecolor{currentfill}{rgb}{0.126453,0.570633,0.549841}%
\pgfsetfillcolor{currentfill}%
\pgfsetlinewidth{0.000000pt}%
\definecolor{currentstroke}{rgb}{0.143343,0.522773,0.556295}%
\pgfsetstrokecolor{currentstroke}%
\pgfsetdash{}{0pt}%
\pgfpathmoveto{\pgfqpoint{4.968612in}{3.651484in}}%
\pgfpathlineto{\pgfqpoint{4.827480in}{3.864804in}}%
\pgfpathlineto{\pgfqpoint{4.746722in}{3.941721in}}%
\pgfpathclose%
\pgfusepath{fill}%
\end{pgfscope}%
\begin{pgfscope}%
\pgfpathrectangle{\pgfqpoint{0.539299in}{0.078740in}}{\pgfqpoint{7.842520in}{7.842520in}}%
\pgfusepath{clip}%
\pgfsetbuttcap%
\pgfsetroundjoin%
\definecolor{currentfill}{rgb}{0.166617,0.463708,0.558119}%
\pgfsetfillcolor{currentfill}%
\pgfsetlinewidth{0.000000pt}%
\definecolor{currentstroke}{rgb}{0.141935,0.526453,0.555991}%
\pgfsetstrokecolor{currentstroke}%
\pgfsetdash{}{0pt}%
\pgfpathmoveto{\pgfqpoint{5.250481in}{3.245064in}}%
\pgfpathlineto{\pgfqpoint{5.030239in}{3.504331in}}%
\pgfpathlineto{\pgfqpoint{5.171743in}{3.293790in}}%
\pgfpathclose%
\pgfusepath{fill}%
\end{pgfscope}%
\begin{pgfscope}%
\pgfpathrectangle{\pgfqpoint{0.539299in}{0.078740in}}{\pgfqpoint{7.842520in}{7.842520in}}%
\pgfusepath{clip}%
\pgfsetbuttcap%
\pgfsetroundjoin%
\definecolor{currentfill}{rgb}{0.344074,0.780029,0.397381}%
\pgfsetfillcolor{currentfill}%
\pgfsetlinewidth{0.000000pt}%
\definecolor{currentstroke}{rgb}{0.140536,0.530132,0.555659}%
\pgfsetstrokecolor{currentstroke}%
\pgfsetdash{}{0pt}%
\pgfpathmoveto{\pgfqpoint{2.347219in}{4.810962in}}%
\pgfpathlineto{\pgfqpoint{2.433328in}{4.953363in}}%
\pgfpathlineto{\pgfqpoint{2.305286in}{4.666266in}}%
\pgfpathclose%
\pgfusepath{fill}%
\end{pgfscope}%
\begin{pgfscope}%
\pgfpathrectangle{\pgfqpoint{0.539299in}{0.078740in}}{\pgfqpoint{7.842520in}{7.842520in}}%
\pgfusepath{clip}%
\pgfsetbuttcap%
\pgfsetroundjoin%
\definecolor{currentfill}{rgb}{0.352360,0.783011,0.392636}%
\pgfsetfillcolor{currentfill}%
\pgfsetlinewidth{0.000000pt}%
\definecolor{currentstroke}{rgb}{0.139147,0.533812,0.555298}%
\pgfsetstrokecolor{currentstroke}%
\pgfsetdash{}{0pt}%
\pgfpathmoveto{\pgfqpoint{4.202671in}{4.843466in}}%
\pgfpathlineto{\pgfqpoint{4.118884in}{4.938971in}}%
\pgfpathlineto{\pgfqpoint{4.260950in}{4.734983in}}%
\pgfpathclose%
\pgfusepath{fill}%
\end{pgfscope}%
\begin{pgfscope}%
\pgfpathrectangle{\pgfqpoint{0.539299in}{0.078740in}}{\pgfqpoint{7.842520in}{7.842520in}}%
\pgfusepath{clip}%
\pgfsetbuttcap%
\pgfsetroundjoin%
\definecolor{currentfill}{rgb}{0.709898,0.868751,0.169257}%
\pgfsetfillcolor{currentfill}%
\pgfsetlinewidth{0.000000pt}%
\definecolor{currentstroke}{rgb}{0.137770,0.537492,0.554906}%
\pgfsetstrokecolor{currentstroke}%
\pgfsetdash{}{0pt}%
\pgfpathmoveto{\pgfqpoint{3.552031in}{5.538964in}}%
\pgfpathlineto{\pgfqpoint{3.693131in}{5.434822in}}%
\pgfpathlineto{\pgfqpoint{3.778860in}{5.358543in}}%
\pgfpathclose%
\pgfusepath{fill}%
\end{pgfscope}%
\begin{pgfscope}%
\pgfpathrectangle{\pgfqpoint{0.539299in}{0.078740in}}{\pgfqpoint{7.842520in}{7.842520in}}%
\pgfusepath{clip}%
\pgfsetbuttcap%
\pgfsetroundjoin%
\definecolor{currentfill}{rgb}{0.241237,0.296485,0.539709}%
\pgfsetfillcolor{currentfill}%
\pgfsetlinewidth{0.000000pt}%
\definecolor{currentstroke}{rgb}{0.136408,0.541173,0.554483}%
\pgfsetstrokecolor{currentstroke}%
\pgfsetdash{}{0pt}%
\pgfpathmoveto{\pgfqpoint{1.786943in}{2.460503in}}%
\pgfpathlineto{\pgfqpoint{1.922231in}{2.462670in}}%
\pgfpathlineto{\pgfqpoint{1.859345in}{2.989662in}}%
\pgfpathclose%
\pgfusepath{fill}%
\end{pgfscope}%
\begin{pgfscope}%
\pgfpathrectangle{\pgfqpoint{0.539299in}{0.078740in}}{\pgfqpoint{7.842520in}{7.842520in}}%
\pgfusepath{clip}%
\pgfsetbuttcap%
\pgfsetroundjoin%
\definecolor{currentfill}{rgb}{0.468053,0.818921,0.323998}%
\pgfsetfillcolor{currentfill}%
\pgfsetlinewidth{0.000000pt}%
\definecolor{currentstroke}{rgb}{0.135066,0.544853,0.554029}%
\pgfsetstrokecolor{currentstroke}%
\pgfsetdash{}{0pt}%
\pgfpathmoveto{\pgfqpoint{4.061290in}{5.036247in}}%
\pgfpathlineto{\pgfqpoint{3.976785in}{5.127804in}}%
\pgfpathlineto{\pgfqpoint{4.118884in}{4.938971in}}%
\pgfpathclose%
\pgfusepath{fill}%
\end{pgfscope}%
\begin{pgfscope}%
\pgfpathrectangle{\pgfqpoint{0.539299in}{0.078740in}}{\pgfqpoint{7.842520in}{7.842520in}}%
\pgfusepath{clip}%
\pgfsetbuttcap%
\pgfsetroundjoin%
\definecolor{currentfill}{rgb}{0.668054,0.861999,0.196293}%
\pgfsetfillcolor{currentfill}%
\pgfsetlinewidth{0.000000pt}%
\definecolor{currentstroke}{rgb}{0.133743,0.548535,0.553541}%
\pgfsetstrokecolor{currentstroke}%
\pgfsetdash{}{0pt}%
\pgfpathmoveto{\pgfqpoint{3.778860in}{5.358543in}}%
\pgfpathlineto{\pgfqpoint{3.693131in}{5.434822in}}%
\pgfpathlineto{\pgfqpoint{3.834801in}{5.295343in}}%
\pgfpathclose%
\pgfusepath{fill}%
\end{pgfscope}%
\begin{pgfscope}%
\pgfpathrectangle{\pgfqpoint{0.539299in}{0.078740in}}{\pgfqpoint{7.842520in}{7.842520in}}%
\pgfusepath{clip}%
\pgfsetbuttcap%
\pgfsetroundjoin%
\definecolor{currentfill}{rgb}{0.824940,0.884720,0.106217}%
\pgfsetfillcolor{currentfill}%
\pgfsetlinewidth{0.000000pt}%
\definecolor{currentstroke}{rgb}{0.132444,0.552216,0.553018}%
\pgfsetstrokecolor{currentstroke}%
\pgfsetdash{}{0pt}%
\pgfpathmoveto{\pgfqpoint{3.411816in}{5.600141in}}%
\pgfpathlineto{\pgfqpoint{3.272857in}{5.610573in}}%
\pgfpathlineto{\pgfqpoint{3.324832in}{5.632651in}}%
\pgfpathclose%
\pgfusepath{fill}%
\end{pgfscope}%
\begin{pgfscope}%
\pgfpathrectangle{\pgfqpoint{0.539299in}{0.078740in}}{\pgfqpoint{7.842520in}{7.842520in}}%
\pgfusepath{clip}%
\pgfsetbuttcap%
\pgfsetroundjoin%
\definecolor{currentfill}{rgb}{0.535621,0.835785,0.281908}%
\pgfsetfillcolor{currentfill}%
\pgfsetlinewidth{0.000000pt}%
\definecolor{currentstroke}{rgb}{0.131172,0.555899,0.552459}%
\pgfsetstrokecolor{currentstroke}%
\pgfsetdash{}{0pt}%
\pgfpathmoveto{\pgfqpoint{3.834801in}{5.295343in}}%
\pgfpathlineto{\pgfqpoint{3.976785in}{5.127804in}}%
\pgfpathlineto{\pgfqpoint{4.061290in}{5.036247in}}%
\pgfpathclose%
\pgfusepath{fill}%
\end{pgfscope}%
\begin{pgfscope}%
\pgfpathrectangle{\pgfqpoint{0.539299in}{0.078740in}}{\pgfqpoint{7.842520in}{7.842520in}}%
\pgfusepath{clip}%
\pgfsetbuttcap%
\pgfsetroundjoin%
\definecolor{currentfill}{rgb}{0.120081,0.622161,0.534946}%
\pgfsetfillcolor{currentfill}%
\pgfsetlinewidth{0.000000pt}%
\definecolor{currentstroke}{rgb}{0.129933,0.559582,0.551864}%
\pgfsetstrokecolor{currentstroke}%
\pgfsetdash{}{0pt}%
\pgfpathmoveto{\pgfqpoint{4.686160in}{4.082636in}}%
\pgfpathlineto{\pgfqpoint{4.604647in}{4.165055in}}%
\pgfpathlineto{\pgfqpoint{4.827480in}{3.864804in}}%
\pgfpathclose%
\pgfusepath{fill}%
\end{pgfscope}%
\begin{pgfscope}%
\pgfpathrectangle{\pgfqpoint{0.539299in}{0.078740in}}{\pgfqpoint{7.842520in}{7.842520in}}%
\pgfusepath{clip}%
\pgfsetbuttcap%
\pgfsetroundjoin%
\definecolor{currentfill}{rgb}{0.140536,0.530132,0.555659}%
\pgfsetfillcolor{currentfill}%
\pgfsetlinewidth{0.000000pt}%
\definecolor{currentstroke}{rgb}{0.128729,0.563265,0.551229}%
\pgfsetstrokecolor{currentstroke}%
\pgfsetdash{}{0pt}%
\pgfpathmoveto{\pgfqpoint{4.888578in}{3.720803in}}%
\pgfpathlineto{\pgfqpoint{5.030239in}{3.504331in}}%
\pgfpathlineto{\pgfqpoint{4.968612in}{3.651484in}}%
\pgfpathclose%
\pgfusepath{fill}%
\end{pgfscope}%
\begin{pgfscope}%
\pgfpathrectangle{\pgfqpoint{0.539299in}{0.078740in}}{\pgfqpoint{7.842520in}{7.842520in}}%
\pgfusepath{clip}%
\pgfsetbuttcap%
\pgfsetroundjoin%
\definecolor{currentfill}{rgb}{0.175841,0.441290,0.557685}%
\pgfsetfillcolor{currentfill}%
\pgfsetlinewidth{0.000000pt}%
\definecolor{currentstroke}{rgb}{0.127568,0.566949,0.550556}%
\pgfsetstrokecolor{currentstroke}%
\pgfsetdash{}{0pt}%
\pgfpathmoveto{\pgfqpoint{1.804002in}{3.315816in}}%
\pgfpathlineto{\pgfqpoint{1.726778in}{2.916957in}}%
\pgfpathlineto{\pgfqpoint{1.935199in}{3.440713in}}%
\pgfpathclose%
\pgfusepath{fill}%
\end{pgfscope}%
\begin{pgfscope}%
\pgfpathrectangle{\pgfqpoint{0.539299in}{0.078740in}}{\pgfqpoint{7.842520in}{7.842520in}}%
\pgfusepath{clip}%
\pgfsetbuttcap%
\pgfsetroundjoin%
\definecolor{currentfill}{rgb}{0.280868,0.160771,0.472899}%
\pgfsetfillcolor{currentfill}%
\pgfsetlinewidth{0.000000pt}%
\definecolor{currentstroke}{rgb}{0.126453,0.570633,0.549841}%
\pgfsetstrokecolor{currentstroke}%
\pgfsetdash{}{0pt}%
\pgfpathmoveto{\pgfqpoint{6.020118in}{2.191364in}}%
\pgfpathlineto{\pgfqpoint{5.943414in}{2.169358in}}%
\pgfpathlineto{\pgfqpoint{6.161891in}{2.037228in}}%
\pgfpathclose%
\pgfusepath{fill}%
\end{pgfscope}%
\begin{pgfscope}%
\pgfpathrectangle{\pgfqpoint{0.539299in}{0.078740in}}{\pgfqpoint{7.842520in}{7.842520in}}%
\pgfusepath{clip}%
\pgfsetbuttcap%
\pgfsetroundjoin%
\definecolor{currentfill}{rgb}{0.271828,0.209303,0.504434}%
\pgfsetfillcolor{currentfill}%
\pgfsetlinewidth{0.000000pt}%
\definecolor{currentstroke}{rgb}{0.125394,0.574318,0.549086}%
\pgfsetstrokecolor{currentstroke}%
\pgfsetdash{}{0pt}%
\pgfpathmoveto{\pgfqpoint{6.020118in}{2.191364in}}%
\pgfpathlineto{\pgfqpoint{5.878547in}{2.354530in}}%
\pgfpathlineto{\pgfqpoint{5.801571in}{2.347348in}}%
\pgfpathclose%
\pgfusepath{fill}%
\end{pgfscope}%
\begin{pgfscope}%
\pgfpathrectangle{\pgfqpoint{0.539299in}{0.078740in}}{\pgfqpoint{7.842520in}{7.842520in}}%
\pgfusepath{clip}%
\pgfsetbuttcap%
\pgfsetroundjoin%
\definecolor{currentfill}{rgb}{0.263663,0.237631,0.518762}%
\pgfsetfillcolor{currentfill}%
\pgfsetlinewidth{0.000000pt}%
\definecolor{currentstroke}{rgb}{0.124395,0.578002,0.548287}%
\pgfsetstrokecolor{currentstroke}%
\pgfsetdash{}{0pt}%
\pgfpathmoveto{\pgfqpoint{5.737117in}{2.526203in}}%
\pgfpathlineto{\pgfqpoint{5.801571in}{2.347348in}}%
\pgfpathlineto{\pgfqpoint{5.878547in}{2.354530in}}%
\pgfpathclose%
\pgfusepath{fill}%
\end{pgfscope}%
\begin{pgfscope}%
\pgfpathrectangle{\pgfqpoint{0.539299in}{0.078740in}}{\pgfqpoint{7.842520in}{7.842520in}}%
\pgfusepath{clip}%
\pgfsetbuttcap%
\pgfsetroundjoin%
\definecolor{currentfill}{rgb}{0.282327,0.094955,0.417331}%
\pgfsetfillcolor{currentfill}%
\pgfsetlinewidth{0.000000pt}%
\definecolor{currentstroke}{rgb}{0.123463,0.581687,0.547445}%
\pgfsetstrokecolor{currentstroke}%
\pgfsetdash{}{0pt}%
\pgfpathmoveto{\pgfqpoint{6.303947in}{1.893136in}}%
\pgfpathlineto{\pgfqpoint{6.161891in}{2.037228in}}%
\pgfpathlineto{\pgfqpoint{6.227580in}{1.841168in}}%
\pgfpathclose%
\pgfusepath{fill}%
\end{pgfscope}%
\begin{pgfscope}%
\pgfpathrectangle{\pgfqpoint{0.539299in}{0.078740in}}{\pgfqpoint{7.842520in}{7.842520in}}%
\pgfusepath{clip}%
\pgfsetbuttcap%
\pgfsetroundjoin%
\definecolor{currentfill}{rgb}{0.814576,0.883393,0.110347}%
\pgfsetfillcolor{currentfill}%
\pgfsetlinewidth{0.000000pt}%
\definecolor{currentstroke}{rgb}{0.122606,0.585371,0.546557}%
\pgfsetstrokecolor{currentstroke}%
\pgfsetdash{}{0pt}%
\pgfpathmoveto{\pgfqpoint{3.185707in}{5.629892in}}%
\pgfpathlineto{\pgfqpoint{3.048388in}{5.568482in}}%
\pgfpathlineto{\pgfqpoint{2.961006in}{5.551827in}}%
\pgfpathclose%
\pgfusepath{fill}%
\end{pgfscope}%
\begin{pgfscope}%
\pgfpathrectangle{\pgfqpoint{0.539299in}{0.078740in}}{\pgfqpoint{7.842520in}{7.842520in}}%
\pgfusepath{clip}%
\pgfsetbuttcap%
\pgfsetroundjoin%
\definecolor{currentfill}{rgb}{0.132268,0.655014,0.519661}%
\pgfsetfillcolor{currentfill}%
\pgfsetlinewidth{0.000000pt}%
\definecolor{currentstroke}{rgb}{0.121831,0.589055,0.545623}%
\pgfsetstrokecolor{currentstroke}%
\pgfsetdash{}{0pt}%
\pgfpathmoveto{\pgfqpoint{4.544630in}{4.302516in}}%
\pgfpathlineto{\pgfqpoint{4.604647in}{4.165055in}}%
\pgfpathlineto{\pgfqpoint{4.686160in}{4.082636in}}%
\pgfpathclose%
\pgfusepath{fill}%
\end{pgfscope}%
\begin{pgfscope}%
\pgfpathrectangle{\pgfqpoint{0.539299in}{0.078740in}}{\pgfqpoint{7.842520in}{7.842520in}}%
\pgfusepath{clip}%
\pgfsetbuttcap%
\pgfsetroundjoin%
\definecolor{currentfill}{rgb}{0.279566,0.067836,0.391917}%
\pgfsetfillcolor{currentfill}%
\pgfsetlinewidth{0.000000pt}%
\definecolor{currentstroke}{rgb}{0.121148,0.592739,0.544641}%
\pgfsetstrokecolor{currentstroke}%
\pgfsetdash{}{0pt}%
\pgfpathmoveto{\pgfqpoint{6.227580in}{1.841168in}}%
\pgfpathlineto{\pgfqpoint{6.446395in}{1.760841in}}%
\pgfpathlineto{\pgfqpoint{6.303947in}{1.893136in}}%
\pgfpathclose%
\pgfusepath{fill}%
\end{pgfscope}%
\begin{pgfscope}%
\pgfpathrectangle{\pgfqpoint{0.539299in}{0.078740in}}{\pgfqpoint{7.842520in}{7.842520in}}%
\pgfusepath{clip}%
\pgfsetbuttcap%
\pgfsetroundjoin%
\definecolor{currentfill}{rgb}{0.835270,0.886029,0.102646}%
\pgfsetfillcolor{currentfill}%
\pgfsetlinewidth{0.000000pt}%
\definecolor{currentstroke}{rgb}{0.120565,0.596422,0.543611}%
\pgfsetstrokecolor{currentstroke}%
\pgfsetdash{}{0pt}%
\pgfpathmoveto{\pgfqpoint{3.324832in}{5.632651in}}%
\pgfpathlineto{\pgfqpoint{3.272857in}{5.610573in}}%
\pgfpathlineto{\pgfqpoint{3.185707in}{5.629892in}}%
\pgfpathclose%
\pgfusepath{fill}%
\end{pgfscope}%
\begin{pgfscope}%
\pgfpathrectangle{\pgfqpoint{0.539299in}{0.078740in}}{\pgfqpoint{7.842520in}{7.842520in}}%
\pgfusepath{clip}%
\pgfsetbuttcap%
\pgfsetroundjoin%
\definecolor{currentfill}{rgb}{0.235526,0.309527,0.542944}%
\pgfsetfillcolor{currentfill}%
\pgfsetlinewidth{0.000000pt}%
\definecolor{currentstroke}{rgb}{0.120092,0.600104,0.542530}%
\pgfsetstrokecolor{currentstroke}%
\pgfsetdash{}{0pt}%
\pgfpathmoveto{\pgfqpoint{5.595772in}{2.706138in}}%
\pgfpathlineto{\pgfqpoint{5.518023in}{2.726615in}}%
\pgfpathlineto{\pgfqpoint{5.737117in}{2.526203in}}%
\pgfpathclose%
\pgfusepath{fill}%
\end{pgfscope}%
\begin{pgfscope}%
\pgfpathrectangle{\pgfqpoint{0.539299in}{0.078740in}}{\pgfqpoint{7.842520in}{7.842520in}}%
\pgfusepath{clip}%
\pgfsetbuttcap%
\pgfsetroundjoin%
\definecolor{currentfill}{rgb}{0.127568,0.566949,0.550556}%
\pgfsetfillcolor{currentfill}%
\pgfsetlinewidth{0.000000pt}%
\definecolor{currentstroke}{rgb}{0.119738,0.603785,0.541400}%
\pgfsetstrokecolor{currentstroke}%
\pgfsetdash{}{0pt}%
\pgfpathmoveto{\pgfqpoint{4.746722in}{3.941721in}}%
\pgfpathlineto{\pgfqpoint{4.888578in}{3.720803in}}%
\pgfpathlineto{\pgfqpoint{4.968612in}{3.651484in}}%
\pgfpathclose%
\pgfusepath{fill}%
\end{pgfscope}%
\begin{pgfscope}%
\pgfpathrectangle{\pgfqpoint{0.539299in}{0.078740in}}{\pgfqpoint{7.842520in}{7.842520in}}%
\pgfusepath{clip}%
\pgfsetbuttcap%
\pgfsetroundjoin%
\definecolor{currentfill}{rgb}{0.180653,0.701402,0.488189}%
\pgfsetfillcolor{currentfill}%
\pgfsetlinewidth{0.000000pt}%
\definecolor{currentstroke}{rgb}{0.119512,0.607464,0.540218}%
\pgfsetstrokecolor{currentstroke}%
\pgfsetdash{}{0pt}%
\pgfpathmoveto{\pgfqpoint{2.133656in}{4.370779in}}%
\pgfpathlineto{\pgfqpoint{2.049048in}{4.173228in}}%
\pgfpathlineto{\pgfqpoint{2.261900in}{4.632062in}}%
\pgfpathclose%
\pgfusepath{fill}%
\end{pgfscope}%
\begin{pgfscope}%
\pgfpathrectangle{\pgfqpoint{0.539299in}{0.078740in}}{\pgfqpoint{7.842520in}{7.842520in}}%
\pgfusepath{clip}%
\pgfsetbuttcap%
\pgfsetroundjoin%
\definecolor{currentfill}{rgb}{0.183898,0.422383,0.556944}%
\pgfsetfillcolor{currentfill}%
\pgfsetlinewidth{0.000000pt}%
\definecolor{currentstroke}{rgb}{0.119423,0.611141,0.538982}%
\pgfsetstrokecolor{currentstroke}%
\pgfsetdash{}{0pt}%
\pgfpathmoveto{\pgfqpoint{1.935199in}{3.440713in}}%
\pgfpathlineto{\pgfqpoint{1.726778in}{2.916957in}}%
\pgfpathlineto{\pgfqpoint{1.859345in}{2.989662in}}%
\pgfpathclose%
\pgfusepath{fill}%
\end{pgfscope}%
\begin{pgfscope}%
\pgfpathrectangle{\pgfqpoint{0.539299in}{0.078740in}}{\pgfqpoint{7.842520in}{7.842520in}}%
\pgfusepath{clip}%
\pgfsetbuttcap%
\pgfsetroundjoin%
\definecolor{currentfill}{rgb}{0.595839,0.848717,0.243329}%
\pgfsetfillcolor{currentfill}%
\pgfsetlinewidth{0.000000pt}%
\definecolor{currentstroke}{rgb}{0.119483,0.614817,0.537692}%
\pgfsetstrokecolor{currentstroke}%
\pgfsetdash{}{0pt}%
\pgfpathmoveto{\pgfqpoint{2.519975in}{5.063797in}}%
\pgfpathlineto{\pgfqpoint{2.651662in}{5.282437in}}%
\pgfpathlineto{\pgfqpoint{2.738785in}{5.362101in}}%
\pgfpathclose%
\pgfusepath{fill}%
\end{pgfscope}%
\begin{pgfscope}%
\pgfpathrectangle{\pgfqpoint{0.539299in}{0.078740in}}{\pgfqpoint{7.842520in}{7.842520in}}%
\pgfusepath{clip}%
\pgfsetbuttcap%
\pgfsetroundjoin%
\definecolor{currentfill}{rgb}{0.151918,0.500685,0.557587}%
\pgfsetfillcolor{currentfill}%
\pgfsetlinewidth{0.000000pt}%
\definecolor{currentstroke}{rgb}{0.119699,0.618490,0.536347}%
\pgfsetstrokecolor{currentstroke}%
\pgfsetdash{}{0pt}%
\pgfpathmoveto{\pgfqpoint{1.804002in}{3.315816in}}%
\pgfpathlineto{\pgfqpoint{1.935199in}{3.440713in}}%
\pgfpathlineto{\pgfqpoint{1.883808in}{3.652469in}}%
\pgfpathclose%
\pgfusepath{fill}%
\end{pgfscope}%
\begin{pgfscope}%
\pgfpathrectangle{\pgfqpoint{0.539299in}{0.078740in}}{\pgfqpoint{7.842520in}{7.842520in}}%
\pgfusepath{clip}%
\pgfsetbuttcap%
\pgfsetroundjoin%
\definecolor{currentfill}{rgb}{0.214298,0.355619,0.551184}%
\pgfsetfillcolor{currentfill}%
\pgfsetlinewidth{0.000000pt}%
\definecolor{currentstroke}{rgb}{0.120081,0.622161,0.534946}%
\pgfsetstrokecolor{currentstroke}%
\pgfsetdash{}{0pt}%
\pgfpathmoveto{\pgfqpoint{5.595772in}{2.706138in}}%
\pgfpathlineto{\pgfqpoint{5.454461in}{2.894198in}}%
\pgfpathlineto{\pgfqpoint{5.376216in}{2.927023in}}%
\pgfpathclose%
\pgfusepath{fill}%
\end{pgfscope}%
\begin{pgfscope}%
\pgfpathrectangle{\pgfqpoint{0.539299in}{0.078740in}}{\pgfqpoint{7.842520in}{7.842520in}}%
\pgfusepath{clip}%
\pgfsetbuttcap%
\pgfsetroundjoin%
\definecolor{currentfill}{rgb}{0.824940,0.884720,0.106217}%
\pgfsetfillcolor{currentfill}%
\pgfsetlinewidth{0.000000pt}%
\definecolor{currentstroke}{rgb}{0.120638,0.625828,0.533488}%
\pgfsetstrokecolor{currentstroke}%
\pgfsetdash{}{0pt}%
\pgfpathmoveto{\pgfqpoint{3.324832in}{5.632651in}}%
\pgfpathlineto{\pgfqpoint{3.552031in}{5.538964in}}%
\pgfpathlineto{\pgfqpoint{3.411816in}{5.600141in}}%
\pgfpathclose%
\pgfusepath{fill}%
\end{pgfscope}%
\begin{pgfscope}%
\pgfpathrectangle{\pgfqpoint{0.539299in}{0.078740in}}{\pgfqpoint{7.842520in}{7.842520in}}%
\pgfusepath{clip}%
\pgfsetbuttcap%
\pgfsetroundjoin%
\definecolor{currentfill}{rgb}{0.720391,0.870350,0.162603}%
\pgfsetfillcolor{currentfill}%
\pgfsetlinewidth{0.000000pt}%
\definecolor{currentstroke}{rgb}{0.121380,0.629492,0.531973}%
\pgfsetstrokecolor{currentstroke}%
\pgfsetdash{}{0pt}%
\pgfpathmoveto{\pgfqpoint{2.826074in}{5.413789in}}%
\pgfpathlineto{\pgfqpoint{2.738785in}{5.362101in}}%
\pgfpathlineto{\pgfqpoint{2.873601in}{5.508440in}}%
\pgfpathclose%
\pgfusepath{fill}%
\end{pgfscope}%
\begin{pgfscope}%
\pgfpathrectangle{\pgfqpoint{0.539299in}{0.078740in}}{\pgfqpoint{7.842520in}{7.842520in}}%
\pgfusepath{clip}%
\pgfsetbuttcap%
\pgfsetroundjoin%
\definecolor{currentfill}{rgb}{0.197636,0.391528,0.554969}%
\pgfsetfillcolor{currentfill}%
\pgfsetlinewidth{0.000000pt}%
\definecolor{currentstroke}{rgb}{0.122312,0.633153,0.530398}%
\pgfsetstrokecolor{currentstroke}%
\pgfsetdash{}{0pt}%
\pgfpathmoveto{\pgfqpoint{5.376216in}{2.927023in}}%
\pgfpathlineto{\pgfqpoint{5.454461in}{2.894198in}}%
\pgfpathlineto{\pgfqpoint{5.313135in}{3.090200in}}%
\pgfpathclose%
\pgfusepath{fill}%
\end{pgfscope}%
\begin{pgfscope}%
\pgfpathrectangle{\pgfqpoint{0.539299in}{0.078740in}}{\pgfqpoint{7.842520in}{7.842520in}}%
\pgfusepath{clip}%
\pgfsetbuttcap%
\pgfsetroundjoin%
\definecolor{currentfill}{rgb}{0.119699,0.618490,0.536347}%
\pgfsetfillcolor{currentfill}%
\pgfsetlinewidth{0.000000pt}%
\definecolor{currentstroke}{rgb}{0.123444,0.636809,0.528763}%
\pgfsetstrokecolor{currentstroke}%
\pgfsetdash{}{0pt}%
\pgfpathmoveto{\pgfqpoint{4.604647in}{4.165055in}}%
\pgfpathlineto{\pgfqpoint{4.746722in}{3.941721in}}%
\pgfpathlineto{\pgfqpoint{4.827480in}{3.864804in}}%
\pgfpathclose%
\pgfusepath{fill}%
\end{pgfscope}%
\begin{pgfscope}%
\pgfpathrectangle{\pgfqpoint{0.539299in}{0.078740in}}{\pgfqpoint{7.842520in}{7.842520in}}%
\pgfusepath{clip}%
\pgfsetbuttcap%
\pgfsetroundjoin%
\definecolor{currentfill}{rgb}{0.283197,0.115680,0.436115}%
\pgfsetfillcolor{currentfill}%
\pgfsetlinewidth{0.000000pt}%
\definecolor{currentstroke}{rgb}{0.124780,0.640461,0.527068}%
\pgfsetstrokecolor{currentstroke}%
\pgfsetdash{}{0pt}%
\pgfpathmoveto{\pgfqpoint{6.227580in}{1.841168in}}%
\pgfpathlineto{\pgfqpoint{6.161891in}{2.037228in}}%
\pgfpathlineto{\pgfqpoint{6.085388in}{2.000146in}}%
\pgfpathclose%
\pgfusepath{fill}%
\end{pgfscope}%
\begin{pgfscope}%
\pgfpathrectangle{\pgfqpoint{0.539299in}{0.078740in}}{\pgfqpoint{7.842520in}{7.842520in}}%
\pgfusepath{clip}%
\pgfsetbuttcap%
\pgfsetroundjoin%
\definecolor{currentfill}{rgb}{0.281887,0.150881,0.465405}%
\pgfsetfillcolor{currentfill}%
\pgfsetlinewidth{0.000000pt}%
\definecolor{currentstroke}{rgb}{0.126326,0.644107,0.525311}%
\pgfsetstrokecolor{currentstroke}%
\pgfsetdash{}{0pt}%
\pgfpathmoveto{\pgfqpoint{6.161891in}{2.037228in}}%
\pgfpathlineto{\pgfqpoint{5.943414in}{2.169358in}}%
\pgfpathlineto{\pgfqpoint{6.085388in}{2.000146in}}%
\pgfpathclose%
\pgfusepath{fill}%
\end{pgfscope}%
\begin{pgfscope}%
\pgfpathrectangle{\pgfqpoint{0.539299in}{0.078740in}}{\pgfqpoint{7.842520in}{7.842520in}}%
\pgfusepath{clip}%
\pgfsetbuttcap%
\pgfsetroundjoin%
\definecolor{currentfill}{rgb}{0.274128,0.199721,0.498911}%
\pgfsetfillcolor{currentfill}%
\pgfsetlinewidth{0.000000pt}%
\definecolor{currentstroke}{rgb}{0.128087,0.647749,0.523491}%
\pgfsetstrokecolor{currentstroke}%
\pgfsetdash{}{0pt}%
\pgfpathmoveto{\pgfqpoint{5.801571in}{2.347348in}}%
\pgfpathlineto{\pgfqpoint{5.943414in}{2.169358in}}%
\pgfpathlineto{\pgfqpoint{6.020118in}{2.191364in}}%
\pgfpathclose%
\pgfusepath{fill}%
\end{pgfscope}%
\begin{pgfscope}%
\pgfpathrectangle{\pgfqpoint{0.539299in}{0.078740in}}{\pgfqpoint{7.842520in}{7.842520in}}%
\pgfusepath{clip}%
\pgfsetbuttcap%
\pgfsetroundjoin%
\definecolor{currentfill}{rgb}{0.772852,0.877868,0.131109}%
\pgfsetfillcolor{currentfill}%
\pgfsetlinewidth{0.000000pt}%
\definecolor{currentstroke}{rgb}{0.130067,0.651384,0.521608}%
\pgfsetstrokecolor{currentstroke}%
\pgfsetdash{}{0pt}%
\pgfpathmoveto{\pgfqpoint{2.873601in}{5.508440in}}%
\pgfpathlineto{\pgfqpoint{2.961006in}{5.551827in}}%
\pgfpathlineto{\pgfqpoint{2.826074in}{5.413789in}}%
\pgfpathclose%
\pgfusepath{fill}%
\end{pgfscope}%
\begin{pgfscope}%
\pgfpathrectangle{\pgfqpoint{0.539299in}{0.078740in}}{\pgfqpoint{7.842520in}{7.842520in}}%
\pgfusepath{clip}%
\pgfsetbuttcap%
\pgfsetroundjoin%
\definecolor{currentfill}{rgb}{0.220124,0.725509,0.466226}%
\pgfsetfillcolor{currentfill}%
\pgfsetlinewidth{0.000000pt}%
\definecolor{currentstroke}{rgb}{0.132268,0.655014,0.519661}%
\pgfsetstrokecolor{currentstroke}%
\pgfsetdash{}{0pt}%
\pgfpathmoveto{\pgfqpoint{4.319824in}{4.607800in}}%
\pgfpathlineto{\pgfqpoint{4.544630in}{4.302516in}}%
\pgfpathlineto{\pgfqpoint{4.402885in}{4.521271in}}%
\pgfpathclose%
\pgfusepath{fill}%
\end{pgfscope}%
\begin{pgfscope}%
\pgfpathrectangle{\pgfqpoint{0.539299in}{0.078740in}}{\pgfqpoint{7.842520in}{7.842520in}}%
\pgfusepath{clip}%
\pgfsetbuttcap%
\pgfsetroundjoin%
\definecolor{currentfill}{rgb}{0.124780,0.640461,0.527068}%
\pgfsetfillcolor{currentfill}%
\pgfsetlinewidth{0.000000pt}%
\definecolor{currentstroke}{rgb}{0.134692,0.658636,0.517649}%
\pgfsetstrokecolor{currentstroke}%
\pgfsetdash{}{0pt}%
\pgfpathmoveto{\pgfqpoint{2.094863in}{4.143974in}}%
\pgfpathlineto{\pgfqpoint{2.049048in}{4.173228in}}%
\pgfpathlineto{\pgfqpoint{1.965641in}{3.935827in}}%
\pgfpathclose%
\pgfusepath{fill}%
\end{pgfscope}%
\begin{pgfscope}%
\pgfpathrectangle{\pgfqpoint{0.539299in}{0.078740in}}{\pgfqpoint{7.842520in}{7.842520in}}%
\pgfusepath{clip}%
\pgfsetbuttcap%
\pgfsetroundjoin%
\definecolor{currentfill}{rgb}{0.253935,0.265254,0.529983}%
\pgfsetfillcolor{currentfill}%
\pgfsetlinewidth{0.000000pt}%
\definecolor{currentstroke}{rgb}{0.137339,0.662252,0.515571}%
\pgfsetstrokecolor{currentstroke}%
\pgfsetdash{}{0pt}%
\pgfpathmoveto{\pgfqpoint{5.659792in}{2.533268in}}%
\pgfpathlineto{\pgfqpoint{5.801571in}{2.347348in}}%
\pgfpathlineto{\pgfqpoint{5.737117in}{2.526203in}}%
\pgfpathclose%
\pgfusepath{fill}%
\end{pgfscope}%
\begin{pgfscope}%
\pgfpathrectangle{\pgfqpoint{0.539299in}{0.078740in}}{\pgfqpoint{7.842520in}{7.842520in}}%
\pgfusepath{clip}%
\pgfsetbuttcap%
\pgfsetroundjoin%
\definecolor{currentfill}{rgb}{0.212395,0.359683,0.551710}%
\pgfsetfillcolor{currentfill}%
\pgfsetlinewidth{0.000000pt}%
\definecolor{currentstroke}{rgb}{0.140210,0.665859,0.513427}%
\pgfsetstrokecolor{currentstroke}%
\pgfsetdash{}{0pt}%
\pgfpathmoveto{\pgfqpoint{1.859345in}{2.989662in}}%
\pgfpathlineto{\pgfqpoint{1.922231in}{2.462670in}}%
\pgfpathlineto{\pgfqpoint{1.993699in}{3.037069in}}%
\pgfpathclose%
\pgfusepath{fill}%
\end{pgfscope}%
\begin{pgfscope}%
\pgfpathrectangle{\pgfqpoint{0.539299in}{0.078740in}}{\pgfqpoint{7.842520in}{7.842520in}}%
\pgfusepath{clip}%
\pgfsetbuttcap%
\pgfsetroundjoin%
\definecolor{currentfill}{rgb}{0.506271,0.828786,0.300362}%
\pgfsetfillcolor{currentfill}%
\pgfsetlinewidth{0.000000pt}%
\definecolor{currentstroke}{rgb}{0.143303,0.669459,0.511215}%
\pgfsetstrokecolor{currentstroke}%
\pgfsetdash{}{0pt}%
\pgfpathmoveto{\pgfqpoint{2.519975in}{5.063797in}}%
\pgfpathlineto{\pgfqpoint{2.433328in}{4.953363in}}%
\pgfpathlineto{\pgfqpoint{2.564916in}{5.170485in}}%
\pgfpathclose%
\pgfusepath{fill}%
\end{pgfscope}%
\begin{pgfscope}%
\pgfpathrectangle{\pgfqpoint{0.539299in}{0.078740in}}{\pgfqpoint{7.842520in}{7.842520in}}%
\pgfusepath{clip}%
\pgfsetbuttcap%
\pgfsetroundjoin%
\definecolor{currentfill}{rgb}{0.274149,0.751988,0.436601}%
\pgfsetfillcolor{currentfill}%
\pgfsetlinewidth{0.000000pt}%
\definecolor{currentstroke}{rgb}{0.146616,0.673050,0.508936}%
\pgfsetstrokecolor{currentstroke}%
\pgfsetdash{}{0pt}%
\pgfpathmoveto{\pgfqpoint{2.133656in}{4.370779in}}%
\pgfpathlineto{\pgfqpoint{2.261900in}{4.632062in}}%
\pgfpathlineto{\pgfqpoint{2.347219in}{4.810962in}}%
\pgfpathclose%
\pgfusepath{fill}%
\end{pgfscope}%
\begin{pgfscope}%
\pgfpathrectangle{\pgfqpoint{0.539299in}{0.078740in}}{\pgfqpoint{7.842520in}{7.842520in}}%
\pgfusepath{clip}%
\pgfsetbuttcap%
\pgfsetroundjoin%
\definecolor{currentfill}{rgb}{0.277941,0.056324,0.381191}%
\pgfsetfillcolor{currentfill}%
\pgfsetlinewidth{0.000000pt}%
\definecolor{currentstroke}{rgb}{0.150148,0.676631,0.506589}%
\pgfsetstrokecolor{currentstroke}%
\pgfsetdash{}{0pt}%
\pgfpathmoveto{\pgfqpoint{6.370114in}{1.694823in}}%
\pgfpathlineto{\pgfqpoint{6.446395in}{1.760841in}}%
\pgfpathlineto{\pgfqpoint{6.227580in}{1.841168in}}%
\pgfpathclose%
\pgfusepath{fill}%
\end{pgfscope}%
\begin{pgfscope}%
\pgfpathrectangle{\pgfqpoint{0.539299in}{0.078740in}}{\pgfqpoint{7.842520in}{7.842520in}}%
\pgfusepath{clip}%
\pgfsetbuttcap%
\pgfsetroundjoin%
\definecolor{currentfill}{rgb}{0.119738,0.603785,0.541400}%
\pgfsetfillcolor{currentfill}%
\pgfsetlinewidth{0.000000pt}%
\definecolor{currentstroke}{rgb}{0.153894,0.680203,0.504172}%
\pgfsetstrokecolor{currentstroke}%
\pgfsetdash{}{0pt}%
\pgfpathmoveto{\pgfqpoint{1.883808in}{3.652469in}}%
\pgfpathlineto{\pgfqpoint{2.094863in}{4.143974in}}%
\pgfpathlineto{\pgfqpoint{1.965641in}{3.935827in}}%
\pgfpathclose%
\pgfusepath{fill}%
\end{pgfscope}%
\begin{pgfscope}%
\pgfpathrectangle{\pgfqpoint{0.539299in}{0.078740in}}{\pgfqpoint{7.842520in}{7.842520in}}%
\pgfusepath{clip}%
\pgfsetbuttcap%
\pgfsetroundjoin%
\definecolor{currentfill}{rgb}{0.239346,0.300855,0.540844}%
\pgfsetfillcolor{currentfill}%
\pgfsetlinewidth{0.000000pt}%
\definecolor{currentstroke}{rgb}{0.157851,0.683765,0.501686}%
\pgfsetstrokecolor{currentstroke}%
\pgfsetdash{}{0pt}%
\pgfpathmoveto{\pgfqpoint{5.737117in}{2.526203in}}%
\pgfpathlineto{\pgfqpoint{5.518023in}{2.726615in}}%
\pgfpathlineto{\pgfqpoint{5.659792in}{2.533268in}}%
\pgfpathclose%
\pgfusepath{fill}%
\end{pgfscope}%
\begin{pgfscope}%
\pgfpathrectangle{\pgfqpoint{0.539299in}{0.078740in}}{\pgfqpoint{7.842520in}{7.842520in}}%
\pgfusepath{clip}%
\pgfsetbuttcap%
\pgfsetroundjoin%
\definecolor{currentfill}{rgb}{0.169646,0.456262,0.558030}%
\pgfsetfillcolor{currentfill}%
\pgfsetlinewidth{0.000000pt}%
\definecolor{currentstroke}{rgb}{0.162016,0.687316,0.499129}%
\pgfsetstrokecolor{currentstroke}%
\pgfsetdash{}{0pt}%
\pgfpathmoveto{\pgfqpoint{5.313135in}{3.090200in}}%
\pgfpathlineto{\pgfqpoint{5.171743in}{3.293790in}}%
\pgfpathlineto{\pgfqpoint{5.092313in}{3.347287in}}%
\pgfpathclose%
\pgfusepath{fill}%
\end{pgfscope}%
\begin{pgfscope}%
\pgfpathrectangle{\pgfqpoint{0.539299in}{0.078740in}}{\pgfqpoint{7.842520in}{7.842520in}}%
\pgfusepath{clip}%
\pgfsetbuttcap%
\pgfsetroundjoin%
\definecolor{currentfill}{rgb}{0.281477,0.755203,0.432552}%
\pgfsetfillcolor{currentfill}%
\pgfsetlinewidth{0.000000pt}%
\definecolor{currentstroke}{rgb}{0.166383,0.690856,0.496502}%
\pgfsetstrokecolor{currentstroke}%
\pgfsetdash{}{0pt}%
\pgfpathmoveto{\pgfqpoint{4.402885in}{4.521271in}}%
\pgfpathlineto{\pgfqpoint{4.260950in}{4.734983in}}%
\pgfpathlineto{\pgfqpoint{4.319824in}{4.607800in}}%
\pgfpathclose%
\pgfusepath{fill}%
\end{pgfscope}%
\begin{pgfscope}%
\pgfpathrectangle{\pgfqpoint{0.539299in}{0.078740in}}{\pgfqpoint{7.842520in}{7.842520in}}%
\pgfusepath{clip}%
\pgfsetbuttcap%
\pgfsetroundjoin%
\definecolor{currentfill}{rgb}{0.231674,0.318106,0.544834}%
\pgfsetfillcolor{currentfill}%
\pgfsetlinewidth{0.000000pt}%
\definecolor{currentstroke}{rgb}{0.170948,0.694384,0.493803}%
\pgfsetstrokecolor{currentstroke}%
\pgfsetdash{}{0pt}%
\pgfpathmoveto{\pgfqpoint{1.993699in}{3.037069in}}%
\pgfpathlineto{\pgfqpoint{1.922231in}{2.462670in}}%
\pgfpathlineto{\pgfqpoint{2.058527in}{2.453906in}}%
\pgfpathclose%
\pgfusepath{fill}%
\end{pgfscope}%
\begin{pgfscope}%
\pgfpathrectangle{\pgfqpoint{0.539299in}{0.078740in}}{\pgfqpoint{7.842520in}{7.842520in}}%
\pgfusepath{clip}%
\pgfsetbuttcap%
\pgfsetroundjoin%
\definecolor{currentfill}{rgb}{0.216210,0.351535,0.550627}%
\pgfsetfillcolor{currentfill}%
\pgfsetlinewidth{0.000000pt}%
\definecolor{currentstroke}{rgb}{0.175707,0.697900,0.491033}%
\pgfsetstrokecolor{currentstroke}%
\pgfsetdash{}{0pt}%
\pgfpathmoveto{\pgfqpoint{5.376216in}{2.927023in}}%
\pgfpathlineto{\pgfqpoint{5.518023in}{2.726615in}}%
\pgfpathlineto{\pgfqpoint{5.595772in}{2.706138in}}%
\pgfpathclose%
\pgfusepath{fill}%
\end{pgfscope}%
\begin{pgfscope}%
\pgfpathrectangle{\pgfqpoint{0.539299in}{0.078740in}}{\pgfqpoint{7.842520in}{7.842520in}}%
\pgfusepath{clip}%
\pgfsetbuttcap%
\pgfsetroundjoin%
\definecolor{currentfill}{rgb}{0.793760,0.880678,0.120005}%
\pgfsetfillcolor{currentfill}%
\pgfsetlinewidth{0.000000pt}%
\definecolor{currentstroke}{rgb}{0.180653,0.701402,0.488189}%
\pgfsetstrokecolor{currentstroke}%
\pgfsetdash{}{0pt}%
\pgfpathmoveto{\pgfqpoint{3.693131in}{5.434822in}}%
\pgfpathlineto{\pgfqpoint{3.552031in}{5.538964in}}%
\pgfpathlineto{\pgfqpoint{3.465333in}{5.584033in}}%
\pgfpathclose%
\pgfusepath{fill}%
\end{pgfscope}%
\begin{pgfscope}%
\pgfpathrectangle{\pgfqpoint{0.539299in}{0.078740in}}{\pgfqpoint{7.842520in}{7.842520in}}%
\pgfusepath{clip}%
\pgfsetbuttcap%
\pgfsetroundjoin%
\definecolor{currentfill}{rgb}{0.156270,0.489624,0.557936}%
\pgfsetfillcolor{currentfill}%
\pgfsetlinewidth{0.000000pt}%
\definecolor{currentstroke}{rgb}{0.185783,0.704891,0.485273}%
\pgfsetstrokecolor{currentstroke}%
\pgfsetdash{}{0pt}%
\pgfpathmoveto{\pgfqpoint{5.092313in}{3.347287in}}%
\pgfpathlineto{\pgfqpoint{5.171743in}{3.293790in}}%
\pgfpathlineto{\pgfqpoint{5.030239in}{3.504331in}}%
\pgfpathclose%
\pgfusepath{fill}%
\end{pgfscope}%
\begin{pgfscope}%
\pgfpathrectangle{\pgfqpoint{0.539299in}{0.078740in}}{\pgfqpoint{7.842520in}{7.842520in}}%
\pgfusepath{clip}%
\pgfsetbuttcap%
\pgfsetroundjoin%
\definecolor{currentfill}{rgb}{0.162016,0.687316,0.499129}%
\pgfsetfillcolor{currentfill}%
\pgfsetlinewidth{0.000000pt}%
\definecolor{currentstroke}{rgb}{0.191090,0.708366,0.482284}%
\pgfsetstrokecolor{currentstroke}%
\pgfsetdash{}{0pt}%
\pgfpathmoveto{\pgfqpoint{4.544630in}{4.302516in}}%
\pgfpathlineto{\pgfqpoint{4.462343in}{4.388175in}}%
\pgfpathlineto{\pgfqpoint{4.604647in}{4.165055in}}%
\pgfpathclose%
\pgfusepath{fill}%
\end{pgfscope}%
\begin{pgfscope}%
\pgfpathrectangle{\pgfqpoint{0.539299in}{0.078740in}}{\pgfqpoint{7.842520in}{7.842520in}}%
\pgfusepath{clip}%
\pgfsetbuttcap%
\pgfsetroundjoin%
\definecolor{currentfill}{rgb}{0.855810,0.888601,0.097452}%
\pgfsetfillcolor{currentfill}%
\pgfsetlinewidth{0.000000pt}%
\definecolor{currentstroke}{rgb}{0.196571,0.711827,0.479221}%
\pgfsetstrokecolor{currentstroke}%
\pgfsetdash{}{0pt}%
\pgfpathmoveto{\pgfqpoint{2.961006in}{5.551827in}}%
\pgfpathlineto{\pgfqpoint{3.098325in}{5.625023in}}%
\pgfpathlineto{\pgfqpoint{3.185707in}{5.629892in}}%
\pgfpathclose%
\pgfusepath{fill}%
\end{pgfscope}%
\begin{pgfscope}%
\pgfpathrectangle{\pgfqpoint{0.539299in}{0.078740in}}{\pgfqpoint{7.842520in}{7.842520in}}%
\pgfusepath{clip}%
\pgfsetbuttcap%
\pgfsetroundjoin%
\definecolor{currentfill}{rgb}{0.585678,0.846661,0.249897}%
\pgfsetfillcolor{currentfill}%
\pgfsetlinewidth{0.000000pt}%
\definecolor{currentstroke}{rgb}{0.202219,0.715272,0.476084}%
\pgfsetstrokecolor{currentstroke}%
\pgfsetdash{}{0pt}%
\pgfpathmoveto{\pgfqpoint{2.564916in}{5.170485in}}%
\pgfpathlineto{\pgfqpoint{2.651662in}{5.282437in}}%
\pgfpathlineto{\pgfqpoint{2.519975in}{5.063797in}}%
\pgfpathclose%
\pgfusepath{fill}%
\end{pgfscope}%
\begin{pgfscope}%
\pgfpathrectangle{\pgfqpoint{0.539299in}{0.078740in}}{\pgfqpoint{7.842520in}{7.842520in}}%
\pgfusepath{clip}%
\pgfsetbuttcap%
\pgfsetroundjoin%
\definecolor{currentfill}{rgb}{0.133743,0.548535,0.553541}%
\pgfsetfillcolor{currentfill}%
\pgfsetlinewidth{0.000000pt}%
\definecolor{currentstroke}{rgb}{0.208030,0.718701,0.472873}%
\pgfsetstrokecolor{currentstroke}%
\pgfsetdash{}{0pt}%
\pgfpathmoveto{\pgfqpoint{1.883808in}{3.652469in}}%
\pgfpathlineto{\pgfqpoint{1.935199in}{3.440713in}}%
\pgfpathlineto{\pgfqpoint{2.013886in}{3.822753in}}%
\pgfpathclose%
\pgfusepath{fill}%
\end{pgfscope}%
\begin{pgfscope}%
\pgfpathrectangle{\pgfqpoint{0.539299in}{0.078740in}}{\pgfqpoint{7.842520in}{7.842520in}}%
\pgfusepath{clip}%
\pgfsetbuttcap%
\pgfsetroundjoin%
\definecolor{currentfill}{rgb}{0.430983,0.808473,0.346476}%
\pgfsetfillcolor{currentfill}%
\pgfsetlinewidth{0.000000pt}%
\definecolor{currentstroke}{rgb}{0.214000,0.722114,0.469588}%
\pgfsetstrokecolor{currentstroke}%
\pgfsetdash{}{0pt}%
\pgfpathmoveto{\pgfqpoint{4.034345in}{5.020034in}}%
\pgfpathlineto{\pgfqpoint{4.260950in}{4.734983in}}%
\pgfpathlineto{\pgfqpoint{4.118884in}{4.938971in}}%
\pgfpathclose%
\pgfusepath{fill}%
\end{pgfscope}%
\begin{pgfscope}%
\pgfpathrectangle{\pgfqpoint{0.539299in}{0.078740in}}{\pgfqpoint{7.842520in}{7.842520in}}%
\pgfusepath{clip}%
\pgfsetbuttcap%
\pgfsetroundjoin%
\definecolor{currentfill}{rgb}{0.175841,0.441290,0.557685}%
\pgfsetfillcolor{currentfill}%
\pgfsetlinewidth{0.000000pt}%
\definecolor{currentstroke}{rgb}{0.220124,0.725509,0.466226}%
\pgfsetstrokecolor{currentstroke}%
\pgfsetdash{}{0pt}%
\pgfpathmoveto{\pgfqpoint{1.935199in}{3.440713in}}%
\pgfpathlineto{\pgfqpoint{1.859345in}{2.989662in}}%
\pgfpathlineto{\pgfqpoint{1.993699in}{3.037069in}}%
\pgfpathclose%
\pgfusepath{fill}%
\end{pgfscope}%
\begin{pgfscope}%
\pgfpathrectangle{\pgfqpoint{0.539299in}{0.078740in}}{\pgfqpoint{7.842520in}{7.842520in}}%
\pgfusepath{clip}%
\pgfsetbuttcap%
\pgfsetroundjoin%
\definecolor{currentfill}{rgb}{0.202219,0.715272,0.476084}%
\pgfsetfillcolor{currentfill}%
\pgfsetlinewidth{0.000000pt}%
\definecolor{currentstroke}{rgb}{0.226397,0.728888,0.462789}%
\pgfsetstrokecolor{currentstroke}%
\pgfsetdash{}{0pt}%
\pgfpathmoveto{\pgfqpoint{2.261900in}{4.632062in}}%
\pgfpathlineto{\pgfqpoint{2.049048in}{4.173228in}}%
\pgfpathlineto{\pgfqpoint{2.177668in}{4.411616in}}%
\pgfpathclose%
\pgfusepath{fill}%
\end{pgfscope}%
\begin{pgfscope}%
\pgfpathrectangle{\pgfqpoint{0.539299in}{0.078740in}}{\pgfqpoint{7.842520in}{7.842520in}}%
\pgfusepath{clip}%
\pgfsetbuttcap%
\pgfsetroundjoin%
\definecolor{currentfill}{rgb}{0.185556,0.418570,0.556753}%
\pgfsetfillcolor{currentfill}%
\pgfsetlinewidth{0.000000pt}%
\definecolor{currentstroke}{rgb}{0.232815,0.732247,0.459277}%
\pgfsetstrokecolor{currentstroke}%
\pgfsetdash{}{0pt}%
\pgfpathmoveto{\pgfqpoint{5.376216in}{2.927023in}}%
\pgfpathlineto{\pgfqpoint{5.313135in}{3.090200in}}%
\pgfpathlineto{\pgfqpoint{5.234327in}{3.134099in}}%
\pgfpathclose%
\pgfusepath{fill}%
\end{pgfscope}%
\begin{pgfscope}%
\pgfpathrectangle{\pgfqpoint{0.539299in}{0.078740in}}{\pgfqpoint{7.842520in}{7.842520in}}%
\pgfusepath{clip}%
\pgfsetbuttcap%
\pgfsetroundjoin%
\definecolor{currentfill}{rgb}{0.214000,0.722114,0.469588}%
\pgfsetfillcolor{currentfill}%
\pgfsetlinewidth{0.000000pt}%
\definecolor{currentstroke}{rgb}{0.239374,0.735588,0.455688}%
\pgfsetstrokecolor{currentstroke}%
\pgfsetdash{}{0pt}%
\pgfpathmoveto{\pgfqpoint{4.462343in}{4.388175in}}%
\pgfpathlineto{\pgfqpoint{4.544630in}{4.302516in}}%
\pgfpathlineto{\pgfqpoint{4.319824in}{4.607800in}}%
\pgfpathclose%
\pgfusepath{fill}%
\end{pgfscope}%
\begin{pgfscope}%
\pgfpathrectangle{\pgfqpoint{0.539299in}{0.078740in}}{\pgfqpoint{7.842520in}{7.842520in}}%
\pgfusepath{clip}%
\pgfsetbuttcap%
\pgfsetroundjoin%
\definecolor{currentfill}{rgb}{0.845561,0.887322,0.099702}%
\pgfsetfillcolor{currentfill}%
\pgfsetlinewidth{0.000000pt}%
\definecolor{currentstroke}{rgb}{0.246070,0.738910,0.452024}%
\pgfsetstrokecolor{currentstroke}%
\pgfsetdash{}{0pt}%
\pgfpathmoveto{\pgfqpoint{3.465333in}{5.584033in}}%
\pgfpathlineto{\pgfqpoint{3.552031in}{5.538964in}}%
\pgfpathlineto{\pgfqpoint{3.324832in}{5.632651in}}%
\pgfpathclose%
\pgfusepath{fill}%
\end{pgfscope}%
\begin{pgfscope}%
\pgfpathrectangle{\pgfqpoint{0.539299in}{0.078740in}}{\pgfqpoint{7.842520in}{7.842520in}}%
\pgfusepath{clip}%
\pgfsetbuttcap%
\pgfsetroundjoin%
\definecolor{currentfill}{rgb}{0.506271,0.828786,0.300362}%
\pgfsetfillcolor{currentfill}%
\pgfsetlinewidth{0.000000pt}%
\definecolor{currentstroke}{rgb}{0.252899,0.742211,0.448284}%
\pgfsetstrokecolor{currentstroke}%
\pgfsetdash{}{0pt}%
\pgfpathmoveto{\pgfqpoint{4.118884in}{4.938971in}}%
\pgfpathlineto{\pgfqpoint{3.976785in}{5.127804in}}%
\pgfpathlineto{\pgfqpoint{4.034345in}{5.020034in}}%
\pgfpathclose%
\pgfusepath{fill}%
\end{pgfscope}%
\begin{pgfscope}%
\pgfpathrectangle{\pgfqpoint{0.539299in}{0.078740in}}{\pgfqpoint{7.842520in}{7.842520in}}%
\pgfusepath{clip}%
\pgfsetbuttcap%
\pgfsetroundjoin%
\definecolor{currentfill}{rgb}{0.477504,0.821444,0.318195}%
\pgfsetfillcolor{currentfill}%
\pgfsetlinewidth{0.000000pt}%
\definecolor{currentstroke}{rgb}{0.259857,0.745492,0.444467}%
\pgfsetstrokecolor{currentstroke}%
\pgfsetdash{}{0pt}%
\pgfpathmoveto{\pgfqpoint{2.564916in}{5.170485in}}%
\pgfpathlineto{\pgfqpoint{2.433328in}{4.953363in}}%
\pgfpathlineto{\pgfqpoint{2.347219in}{4.810962in}}%
\pgfpathclose%
\pgfusepath{fill}%
\end{pgfscope}%
\begin{pgfscope}%
\pgfpathrectangle{\pgfqpoint{0.539299in}{0.078740in}}{\pgfqpoint{7.842520in}{7.842520in}}%
\pgfusepath{clip}%
\pgfsetbuttcap%
\pgfsetroundjoin%
\definecolor{currentfill}{rgb}{0.720391,0.870350,0.162603}%
\pgfsetfillcolor{currentfill}%
\pgfsetlinewidth{0.000000pt}%
\definecolor{currentstroke}{rgb}{0.266941,0.748751,0.440573}%
\pgfsetstrokecolor{currentstroke}%
\pgfsetdash{}{0pt}%
\pgfpathmoveto{\pgfqpoint{2.873601in}{5.508440in}}%
\pgfpathlineto{\pgfqpoint{2.738785in}{5.362101in}}%
\pgfpathlineto{\pgfqpoint{2.651662in}{5.282437in}}%
\pgfpathclose%
\pgfusepath{fill}%
\end{pgfscope}%
\begin{pgfscope}%
\pgfpathrectangle{\pgfqpoint{0.539299in}{0.078740in}}{\pgfqpoint{7.842520in}{7.842520in}}%
\pgfusepath{clip}%
\pgfsetbuttcap%
\pgfsetroundjoin%
\definecolor{currentfill}{rgb}{0.709898,0.868751,0.169257}%
\pgfsetfillcolor{currentfill}%
\pgfsetlinewidth{0.000000pt}%
\definecolor{currentstroke}{rgb}{0.274149,0.751988,0.436601}%
\pgfsetstrokecolor{currentstroke}%
\pgfsetdash{}{0pt}%
\pgfpathmoveto{\pgfqpoint{3.749006in}{5.361964in}}%
\pgfpathlineto{\pgfqpoint{3.834801in}{5.295343in}}%
\pgfpathlineto{\pgfqpoint{3.693131in}{5.434822in}}%
\pgfpathclose%
\pgfusepath{fill}%
\end{pgfscope}%
\begin{pgfscope}%
\pgfpathrectangle{\pgfqpoint{0.539299in}{0.078740in}}{\pgfqpoint{7.842520in}{7.842520in}}%
\pgfusepath{clip}%
\pgfsetbuttcap%
\pgfsetroundjoin%
\definecolor{currentfill}{rgb}{0.876168,0.891125,0.095250}%
\pgfsetfillcolor{currentfill}%
\pgfsetlinewidth{0.000000pt}%
\definecolor{currentstroke}{rgb}{0.281477,0.755203,0.432552}%
\pgfsetstrokecolor{currentstroke}%
\pgfsetdash{}{0pt}%
\pgfpathmoveto{\pgfqpoint{3.185707in}{5.629892in}}%
\pgfpathlineto{\pgfqpoint{3.098325in}{5.625023in}}%
\pgfpathlineto{\pgfqpoint{3.324832in}{5.632651in}}%
\pgfpathclose%
\pgfusepath{fill}%
\end{pgfscope}%
\begin{pgfscope}%
\pgfpathrectangle{\pgfqpoint{0.539299in}{0.078740in}}{\pgfqpoint{7.842520in}{7.842520in}}%
\pgfusepath{clip}%
\pgfsetbuttcap%
\pgfsetroundjoin%
\definecolor{currentfill}{rgb}{0.157851,0.683765,0.501686}%
\pgfsetfillcolor{currentfill}%
\pgfsetlinewidth{0.000000pt}%
\definecolor{currentstroke}{rgb}{0.288921,0.758394,0.428426}%
\pgfsetstrokecolor{currentstroke}%
\pgfsetdash{}{0pt}%
\pgfpathmoveto{\pgfqpoint{2.177668in}{4.411616in}}%
\pgfpathlineto{\pgfqpoint{2.049048in}{4.173228in}}%
\pgfpathlineto{\pgfqpoint{2.094863in}{4.143974in}}%
\pgfpathclose%
\pgfusepath{fill}%
\end{pgfscope}%
\begin{pgfscope}%
\pgfpathrectangle{\pgfqpoint{0.539299in}{0.078740in}}{\pgfqpoint{7.842520in}{7.842520in}}%
\pgfusepath{clip}%
\pgfsetbuttcap%
\pgfsetroundjoin%
\definecolor{currentfill}{rgb}{0.616293,0.852709,0.230052}%
\pgfsetfillcolor{currentfill}%
\pgfsetlinewidth{0.000000pt}%
\definecolor{currentstroke}{rgb}{0.296479,0.761561,0.424223}%
\pgfsetstrokecolor{currentstroke}%
\pgfsetdash{}{0pt}%
\pgfpathmoveto{\pgfqpoint{3.891582in}{5.202680in}}%
\pgfpathlineto{\pgfqpoint{3.976785in}{5.127804in}}%
\pgfpathlineto{\pgfqpoint{3.834801in}{5.295343in}}%
\pgfpathclose%
\pgfusepath{fill}%
\end{pgfscope}%
\begin{pgfscope}%
\pgfpathrectangle{\pgfqpoint{0.539299in}{0.078740in}}{\pgfqpoint{7.842520in}{7.842520in}}%
\pgfusepath{clip}%
\pgfsetbuttcap%
\pgfsetroundjoin%
\definecolor{currentfill}{rgb}{0.171176,0.452530,0.557965}%
\pgfsetfillcolor{currentfill}%
\pgfsetlinewidth{0.000000pt}%
\definecolor{currentstroke}{rgb}{0.304148,0.764704,0.419943}%
\pgfsetstrokecolor{currentstroke}%
\pgfsetdash{}{0pt}%
\pgfpathmoveto{\pgfqpoint{5.092313in}{3.347287in}}%
\pgfpathlineto{\pgfqpoint{5.234327in}{3.134099in}}%
\pgfpathlineto{\pgfqpoint{5.313135in}{3.090200in}}%
\pgfpathclose%
\pgfusepath{fill}%
\end{pgfscope}%
\begin{pgfscope}%
\pgfpathrectangle{\pgfqpoint{0.539299in}{0.078740in}}{\pgfqpoint{7.842520in}{7.842520in}}%
\pgfusepath{clip}%
\pgfsetbuttcap%
\pgfsetroundjoin%
\definecolor{currentfill}{rgb}{0.119738,0.603785,0.541400}%
\pgfsetfillcolor{currentfill}%
\pgfsetlinewidth{0.000000pt}%
\definecolor{currentstroke}{rgb}{0.311925,0.767822,0.415586}%
\pgfsetstrokecolor{currentstroke}%
\pgfsetdash{}{0pt}%
\pgfpathmoveto{\pgfqpoint{2.013886in}{3.822753in}}%
\pgfpathlineto{\pgfqpoint{2.094863in}{4.143974in}}%
\pgfpathlineto{\pgfqpoint{1.883808in}{3.652469in}}%
\pgfpathclose%
\pgfusepath{fill}%
\end{pgfscope}%
\begin{pgfscope}%
\pgfpathrectangle{\pgfqpoint{0.539299in}{0.078740in}}{\pgfqpoint{7.842520in}{7.842520in}}%
\pgfusepath{clip}%
\pgfsetbuttcap%
\pgfsetroundjoin%
\definecolor{currentfill}{rgb}{0.129933,0.559582,0.551864}%
\pgfsetfillcolor{currentfill}%
\pgfsetlinewidth{0.000000pt}%
\definecolor{currentstroke}{rgb}{0.319809,0.770914,0.411152}%
\pgfsetstrokecolor{currentstroke}%
\pgfsetdash{}{0pt}%
\pgfpathmoveto{\pgfqpoint{5.030239in}{3.504331in}}%
\pgfpathlineto{\pgfqpoint{4.888578in}{3.720803in}}%
\pgfpathlineto{\pgfqpoint{4.807756in}{3.788327in}}%
\pgfpathclose%
\pgfusepath{fill}%
\end{pgfscope}%
\begin{pgfscope}%
\pgfpathrectangle{\pgfqpoint{0.539299in}{0.078740in}}{\pgfqpoint{7.842520in}{7.842520in}}%
\pgfusepath{clip}%
\pgfsetbuttcap%
\pgfsetroundjoin%
\definecolor{currentfill}{rgb}{0.283072,0.130895,0.449241}%
\pgfsetfillcolor{currentfill}%
\pgfsetlinewidth{0.000000pt}%
\definecolor{currentstroke}{rgb}{0.327796,0.773980,0.406640}%
\pgfsetstrokecolor{currentstroke}%
\pgfsetdash{}{0pt}%
\pgfpathmoveto{\pgfqpoint{6.085388in}{2.000146in}}%
\pgfpathlineto{\pgfqpoint{6.008600in}{1.974963in}}%
\pgfpathlineto{\pgfqpoint{6.227580in}{1.841168in}}%
\pgfpathclose%
\pgfusepath{fill}%
\end{pgfscope}%
\begin{pgfscope}%
\pgfpathrectangle{\pgfqpoint{0.539299in}{0.078740in}}{\pgfqpoint{7.842520in}{7.842520in}}%
\pgfusepath{clip}%
\pgfsetbuttcap%
\pgfsetroundjoin%
\definecolor{currentfill}{rgb}{0.280868,0.160771,0.472899}%
\pgfsetfillcolor{currentfill}%
\pgfsetlinewidth{0.000000pt}%
\definecolor{currentstroke}{rgb}{0.335885,0.777018,0.402049}%
\pgfsetstrokecolor{currentstroke}%
\pgfsetdash{}{0pt}%
\pgfpathmoveto{\pgfqpoint{5.943414in}{2.169358in}}%
\pgfpathlineto{\pgfqpoint{6.008600in}{1.974963in}}%
\pgfpathlineto{\pgfqpoint{6.085388in}{2.000146in}}%
\pgfpathclose%
\pgfusepath{fill}%
\end{pgfscope}%
\begin{pgfscope}%
\pgfpathrectangle{\pgfqpoint{0.539299in}{0.078740in}}{\pgfqpoint{7.842520in}{7.842520in}}%
\pgfusepath{clip}%
\pgfsetbuttcap%
\pgfsetroundjoin%
\definecolor{currentfill}{rgb}{0.278791,0.062145,0.386592}%
\pgfsetfillcolor{currentfill}%
\pgfsetlinewidth{0.000000pt}%
\definecolor{currentstroke}{rgb}{0.344074,0.780029,0.397381}%
\pgfsetstrokecolor{currentstroke}%
\pgfsetdash{}{0pt}%
\pgfpathmoveto{\pgfqpoint{6.227580in}{1.841168in}}%
\pgfpathlineto{\pgfqpoint{6.293516in}{1.634687in}}%
\pgfpathlineto{\pgfqpoint{6.370114in}{1.694823in}}%
\pgfpathclose%
\pgfusepath{fill}%
\end{pgfscope}%
\begin{pgfscope}%
\pgfpathrectangle{\pgfqpoint{0.539299in}{0.078740in}}{\pgfqpoint{7.842520in}{7.842520in}}%
\pgfusepath{clip}%
\pgfsetbuttcap%
\pgfsetroundjoin%
\definecolor{currentfill}{rgb}{0.352360,0.783011,0.392636}%
\pgfsetfillcolor{currentfill}%
\pgfsetlinewidth{0.000000pt}%
\definecolor{currentstroke}{rgb}{0.352360,0.783011,0.392636}%
\pgfsetstrokecolor{currentstroke}%
\pgfsetdash{}{0pt}%
\pgfpathmoveto{\pgfqpoint{4.319824in}{4.607800in}}%
\pgfpathlineto{\pgfqpoint{4.260950in}{4.734983in}}%
\pgfpathlineto{\pgfqpoint{4.177132in}{4.819968in}}%
\pgfpathclose%
\pgfusepath{fill}%
\end{pgfscope}%
\begin{pgfscope}%
\pgfpathrectangle{\pgfqpoint{0.539299in}{0.078740in}}{\pgfqpoint{7.842520in}{7.842520in}}%
\pgfusepath{clip}%
\pgfsetbuttcap%
\pgfsetroundjoin%
\definecolor{currentfill}{rgb}{0.265145,0.232956,0.516599}%
\pgfsetfillcolor{currentfill}%
\pgfsetlinewidth{0.000000pt}%
\definecolor{currentstroke}{rgb}{0.360741,0.785964,0.387814}%
\pgfsetstrokecolor{currentstroke}%
\pgfsetdash{}{0pt}%
\pgfpathmoveto{\pgfqpoint{5.943414in}{2.169358in}}%
\pgfpathlineto{\pgfqpoint{5.801571in}{2.347348in}}%
\pgfpathlineto{\pgfqpoint{5.724190in}{2.352097in}}%
\pgfpathclose%
\pgfusepath{fill}%
\end{pgfscope}%
\begin{pgfscope}%
\pgfpathrectangle{\pgfqpoint{0.539299in}{0.078740in}}{\pgfqpoint{7.842520in}{7.842520in}}%
\pgfusepath{clip}%
\pgfsetbuttcap%
\pgfsetroundjoin%
\definecolor{currentfill}{rgb}{0.814576,0.883393,0.110347}%
\pgfsetfillcolor{currentfill}%
\pgfsetlinewidth{0.000000pt}%
\definecolor{currentstroke}{rgb}{0.369214,0.788888,0.382914}%
\pgfsetstrokecolor{currentstroke}%
\pgfsetdash{}{0pt}%
\pgfpathmoveto{\pgfqpoint{3.465333in}{5.584033in}}%
\pgfpathlineto{\pgfqpoint{3.606833in}{5.491393in}}%
\pgfpathlineto{\pgfqpoint{3.693131in}{5.434822in}}%
\pgfpathclose%
\pgfusepath{fill}%
\end{pgfscope}%
\begin{pgfscope}%
\pgfpathrectangle{\pgfqpoint{0.539299in}{0.078740in}}{\pgfqpoint{7.842520in}{7.842520in}}%
\pgfusepath{clip}%
\pgfsetbuttcap%
\pgfsetroundjoin%
\definecolor{currentfill}{rgb}{0.119423,0.611141,0.538982}%
\pgfsetfillcolor{currentfill}%
\pgfsetlinewidth{0.000000pt}%
\definecolor{currentstroke}{rgb}{0.377779,0.791781,0.377939}%
\pgfsetstrokecolor{currentstroke}%
\pgfsetdash{}{0pt}%
\pgfpathmoveto{\pgfqpoint{4.888578in}{3.720803in}}%
\pgfpathlineto{\pgfqpoint{4.746722in}{3.941721in}}%
\pgfpathlineto{\pgfqpoint{4.665156in}{4.013355in}}%
\pgfpathclose%
\pgfusepath{fill}%
\end{pgfscope}%
\begin{pgfscope}%
\pgfpathrectangle{\pgfqpoint{0.539299in}{0.078740in}}{\pgfqpoint{7.842520in}{7.842520in}}%
\pgfusepath{clip}%
\pgfsetbuttcap%
\pgfsetroundjoin%
\definecolor{currentfill}{rgb}{0.144759,0.519093,0.556572}%
\pgfsetfillcolor{currentfill}%
\pgfsetlinewidth{0.000000pt}%
\definecolor{currentstroke}{rgb}{0.386433,0.794644,0.372886}%
\pgfsetstrokecolor{currentstroke}%
\pgfsetdash{}{0pt}%
\pgfpathmoveto{\pgfqpoint{5.030239in}{3.504331in}}%
\pgfpathlineto{\pgfqpoint{4.950134in}{3.565760in}}%
\pgfpathlineto{\pgfqpoint{5.092313in}{3.347287in}}%
\pgfpathclose%
\pgfusepath{fill}%
\end{pgfscope}%
\begin{pgfscope}%
\pgfpathrectangle{\pgfqpoint{0.539299in}{0.078740in}}{\pgfqpoint{7.842520in}{7.842520in}}%
\pgfusepath{clip}%
\pgfsetbuttcap%
\pgfsetroundjoin%
\definecolor{currentfill}{rgb}{0.203063,0.379716,0.553925}%
\pgfsetfillcolor{currentfill}%
\pgfsetlinewidth{0.000000pt}%
\definecolor{currentstroke}{rgb}{0.395174,0.797475,0.367757}%
\pgfsetstrokecolor{currentstroke}%
\pgfsetdash{}{0pt}%
\pgfpathmoveto{\pgfqpoint{2.058527in}{2.453906in}}%
\pgfpathlineto{\pgfqpoint{2.129568in}{3.063212in}}%
\pgfpathlineto{\pgfqpoint{1.993699in}{3.037069in}}%
\pgfpathclose%
\pgfusepath{fill}%
\end{pgfscope}%
\begin{pgfscope}%
\pgfpathrectangle{\pgfqpoint{0.539299in}{0.078740in}}{\pgfqpoint{7.842520in}{7.842520in}}%
\pgfusepath{clip}%
\pgfsetbuttcap%
\pgfsetroundjoin%
\definecolor{currentfill}{rgb}{0.421908,0.805774,0.351910}%
\pgfsetfillcolor{currentfill}%
\pgfsetlinewidth{0.000000pt}%
\definecolor{currentstroke}{rgb}{0.404001,0.800275,0.362552}%
\pgfsetstrokecolor{currentstroke}%
\pgfsetdash{}{0pt}%
\pgfpathmoveto{\pgfqpoint{4.177132in}{4.819968in}}%
\pgfpathlineto{\pgfqpoint{4.260950in}{4.734983in}}%
\pgfpathlineto{\pgfqpoint{4.034345in}{5.020034in}}%
\pgfpathclose%
\pgfusepath{fill}%
\end{pgfscope}%
\begin{pgfscope}%
\pgfpathrectangle{\pgfqpoint{0.539299in}{0.078740in}}{\pgfqpoint{7.842520in}{7.842520in}}%
\pgfusepath{clip}%
\pgfsetbuttcap%
\pgfsetroundjoin%
\definecolor{currentfill}{rgb}{0.244972,0.287675,0.537260}%
\pgfsetfillcolor{currentfill}%
\pgfsetlinewidth{0.000000pt}%
\definecolor{currentstroke}{rgb}{0.412913,0.803041,0.357269}%
\pgfsetstrokecolor{currentstroke}%
\pgfsetdash{}{0pt}%
\pgfpathmoveto{\pgfqpoint{5.581983in}{2.550780in}}%
\pgfpathlineto{\pgfqpoint{5.801571in}{2.347348in}}%
\pgfpathlineto{\pgfqpoint{5.659792in}{2.533268in}}%
\pgfpathclose%
\pgfusepath{fill}%
\end{pgfscope}%
\begin{pgfscope}%
\pgfpathrectangle{\pgfqpoint{0.539299in}{0.078740in}}{\pgfqpoint{7.842520in}{7.842520in}}%
\pgfusepath{clip}%
\pgfsetbuttcap%
\pgfsetroundjoin%
\definecolor{currentfill}{rgb}{0.772852,0.877868,0.131109}%
\pgfsetfillcolor{currentfill}%
\pgfsetlinewidth{0.000000pt}%
\definecolor{currentstroke}{rgb}{0.421908,0.805774,0.351910}%
\pgfsetstrokecolor{currentstroke}%
\pgfsetdash{}{0pt}%
\pgfpathmoveto{\pgfqpoint{3.693131in}{5.434822in}}%
\pgfpathlineto{\pgfqpoint{3.606833in}{5.491393in}}%
\pgfpathlineto{\pgfqpoint{3.749006in}{5.361964in}}%
\pgfpathclose%
\pgfusepath{fill}%
\end{pgfscope}%
\begin{pgfscope}%
\pgfpathrectangle{\pgfqpoint{0.539299in}{0.078740in}}{\pgfqpoint{7.842520in}{7.842520in}}%
\pgfusepath{clip}%
\pgfsetbuttcap%
\pgfsetroundjoin%
\definecolor{currentfill}{rgb}{0.229739,0.322361,0.545706}%
\pgfsetfillcolor{currentfill}%
\pgfsetlinewidth{0.000000pt}%
\definecolor{currentstroke}{rgb}{0.430983,0.808473,0.346476}%
\pgfsetstrokecolor{currentstroke}%
\pgfsetdash{}{0pt}%
\pgfpathmoveto{\pgfqpoint{5.659792in}{2.533268in}}%
\pgfpathlineto{\pgfqpoint{5.518023in}{2.726615in}}%
\pgfpathlineto{\pgfqpoint{5.581983in}{2.550780in}}%
\pgfpathclose%
\pgfusepath{fill}%
\end{pgfscope}%
\begin{pgfscope}%
\pgfpathrectangle{\pgfqpoint{0.539299in}{0.078740in}}{\pgfqpoint{7.842520in}{7.842520in}}%
\pgfusepath{clip}%
\pgfsetbuttcap%
\pgfsetroundjoin%
\definecolor{currentfill}{rgb}{0.131172,0.555899,0.552459}%
\pgfsetfillcolor{currentfill}%
\pgfsetlinewidth{0.000000pt}%
\definecolor{currentstroke}{rgb}{0.440137,0.811138,0.340967}%
\pgfsetstrokecolor{currentstroke}%
\pgfsetdash{}{0pt}%
\pgfpathmoveto{\pgfqpoint{5.030239in}{3.504331in}}%
\pgfpathlineto{\pgfqpoint{4.807756in}{3.788327in}}%
\pgfpathlineto{\pgfqpoint{4.950134in}{3.565760in}}%
\pgfpathclose%
\pgfusepath{fill}%
\end{pgfscope}%
\begin{pgfscope}%
\pgfpathrectangle{\pgfqpoint{0.539299in}{0.078740in}}{\pgfqpoint{7.842520in}{7.842520in}}%
\pgfusepath{clip}%
\pgfsetbuttcap%
\pgfsetroundjoin%
\definecolor{currentfill}{rgb}{0.154815,0.493313,0.557840}%
\pgfsetfillcolor{currentfill}%
\pgfsetlinewidth{0.000000pt}%
\definecolor{currentstroke}{rgb}{0.449368,0.813768,0.335384}%
\pgfsetstrokecolor{currentstroke}%
\pgfsetdash{}{0pt}%
\pgfpathmoveto{\pgfqpoint{1.993699in}{3.037069in}}%
\pgfpathlineto{\pgfqpoint{2.068708in}{3.529775in}}%
\pgfpathlineto{\pgfqpoint{1.935199in}{3.440713in}}%
\pgfpathclose%
\pgfusepath{fill}%
\end{pgfscope}%
\begin{pgfscope}%
\pgfpathrectangle{\pgfqpoint{0.539299in}{0.078740in}}{\pgfqpoint{7.842520in}{7.842520in}}%
\pgfusepath{clip}%
\pgfsetbuttcap%
\pgfsetroundjoin%
\definecolor{currentfill}{rgb}{0.585678,0.846661,0.249897}%
\pgfsetfillcolor{currentfill}%
\pgfsetlinewidth{0.000000pt}%
\definecolor{currentstroke}{rgb}{0.458674,0.816363,0.329727}%
\pgfsetstrokecolor{currentstroke}%
\pgfsetdash{}{0pt}%
\pgfpathmoveto{\pgfqpoint{3.976785in}{5.127804in}}%
\pgfpathlineto{\pgfqpoint{3.891582in}{5.202680in}}%
\pgfpathlineto{\pgfqpoint{4.034345in}{5.020034in}}%
\pgfpathclose%
\pgfusepath{fill}%
\end{pgfscope}%
\begin{pgfscope}%
\pgfpathrectangle{\pgfqpoint{0.539299in}{0.078740in}}{\pgfqpoint{7.842520in}{7.842520in}}%
\pgfusepath{clip}%
\pgfsetbuttcap%
\pgfsetroundjoin%
\definecolor{currentfill}{rgb}{0.688944,0.865448,0.182725}%
\pgfsetfillcolor{currentfill}%
\pgfsetlinewidth{0.000000pt}%
\definecolor{currentstroke}{rgb}{0.468053,0.818921,0.323998}%
\pgfsetstrokecolor{currentstroke}%
\pgfsetdash{}{0pt}%
\pgfpathmoveto{\pgfqpoint{3.891582in}{5.202680in}}%
\pgfpathlineto{\pgfqpoint{3.834801in}{5.295343in}}%
\pgfpathlineto{\pgfqpoint{3.749006in}{5.361964in}}%
\pgfpathclose%
\pgfusepath{fill}%
\end{pgfscope}%
\begin{pgfscope}%
\pgfpathrectangle{\pgfqpoint{0.539299in}{0.078740in}}{\pgfqpoint{7.842520in}{7.842520in}}%
\pgfusepath{clip}%
\pgfsetbuttcap%
\pgfsetroundjoin%
\definecolor{currentfill}{rgb}{0.281446,0.084320,0.407414}%
\pgfsetfillcolor{currentfill}%
\pgfsetlinewidth{0.000000pt}%
\definecolor{currentstroke}{rgb}{0.477504,0.821444,0.318195}%
\pgfsetstrokecolor{currentstroke}%
\pgfsetdash{}{0pt}%
\pgfpathmoveto{\pgfqpoint{6.150940in}{1.799026in}}%
\pgfpathlineto{\pgfqpoint{6.293516in}{1.634687in}}%
\pgfpathlineto{\pgfqpoint{6.227580in}{1.841168in}}%
\pgfpathclose%
\pgfusepath{fill}%
\end{pgfscope}%
\begin{pgfscope}%
\pgfpathrectangle{\pgfqpoint{0.539299in}{0.078740in}}{\pgfqpoint{7.842520in}{7.842520in}}%
\pgfusepath{clip}%
\pgfsetbuttcap%
\pgfsetroundjoin%
\definecolor{currentfill}{rgb}{0.283197,0.115680,0.436115}%
\pgfsetfillcolor{currentfill}%
\pgfsetlinewidth{0.000000pt}%
\definecolor{currentstroke}{rgb}{0.487026,0.823929,0.312321}%
\pgfsetstrokecolor{currentstroke}%
\pgfsetdash{}{0pt}%
\pgfpathmoveto{\pgfqpoint{6.227580in}{1.841168in}}%
\pgfpathlineto{\pgfqpoint{6.008600in}{1.974963in}}%
\pgfpathlineto{\pgfqpoint{6.150940in}{1.799026in}}%
\pgfpathclose%
\pgfusepath{fill}%
\end{pgfscope}%
\begin{pgfscope}%
\pgfpathrectangle{\pgfqpoint{0.539299in}{0.078740in}}{\pgfqpoint{7.842520in}{7.842520in}}%
\pgfusepath{clip}%
\pgfsetbuttcap%
\pgfsetroundjoin%
\definecolor{currentfill}{rgb}{0.140210,0.665859,0.513427}%
\pgfsetfillcolor{currentfill}%
\pgfsetlinewidth{0.000000pt}%
\definecolor{currentstroke}{rgb}{0.496615,0.826376,0.306377}%
\pgfsetstrokecolor{currentstroke}%
\pgfsetdash{}{0pt}%
\pgfpathmoveto{\pgfqpoint{4.604647in}{4.165055in}}%
\pgfpathlineto{\pgfqpoint{4.522320in}{4.238703in}}%
\pgfpathlineto{\pgfqpoint{4.746722in}{3.941721in}}%
\pgfpathclose%
\pgfusepath{fill}%
\end{pgfscope}%
\begin{pgfscope}%
\pgfpathrectangle{\pgfqpoint{0.539299in}{0.078740in}}{\pgfqpoint{7.842520in}{7.842520in}}%
\pgfusepath{clip}%
\pgfsetbuttcap%
\pgfsetroundjoin%
\definecolor{currentfill}{rgb}{0.876168,0.891125,0.095250}%
\pgfsetfillcolor{currentfill}%
\pgfsetlinewidth{0.000000pt}%
\definecolor{currentstroke}{rgb}{0.506271,0.828786,0.300362}%
\pgfsetstrokecolor{currentstroke}%
\pgfsetdash{}{0pt}%
\pgfpathmoveto{\pgfqpoint{3.010887in}{5.591305in}}%
\pgfpathlineto{\pgfqpoint{3.098325in}{5.625023in}}%
\pgfpathlineto{\pgfqpoint{2.961006in}{5.551827in}}%
\pgfpathclose%
\pgfusepath{fill}%
\end{pgfscope}%
\begin{pgfscope}%
\pgfpathrectangle{\pgfqpoint{0.539299in}{0.078740in}}{\pgfqpoint{7.842520in}{7.842520in}}%
\pgfusepath{clip}%
\pgfsetbuttcap%
\pgfsetroundjoin%
\definecolor{currentfill}{rgb}{0.855810,0.888601,0.097452}%
\pgfsetfillcolor{currentfill}%
\pgfsetlinewidth{0.000000pt}%
\definecolor{currentstroke}{rgb}{0.515992,0.831158,0.294279}%
\pgfsetstrokecolor{currentstroke}%
\pgfsetdash{}{0pt}%
\pgfpathmoveto{\pgfqpoint{2.961006in}{5.551827in}}%
\pgfpathlineto{\pgfqpoint{2.873601in}{5.508440in}}%
\pgfpathlineto{\pgfqpoint{3.010887in}{5.591305in}}%
\pgfpathclose%
\pgfusepath{fill}%
\end{pgfscope}%
\begin{pgfscope}%
\pgfpathrectangle{\pgfqpoint{0.539299in}{0.078740in}}{\pgfqpoint{7.842520in}{7.842520in}}%
\pgfusepath{clip}%
\pgfsetbuttcap%
\pgfsetroundjoin%
\definecolor{currentfill}{rgb}{0.277134,0.185228,0.489898}%
\pgfsetfillcolor{currentfill}%
\pgfsetlinewidth{0.000000pt}%
\definecolor{currentstroke}{rgb}{0.525776,0.833491,0.288127}%
\pgfsetstrokecolor{currentstroke}%
\pgfsetdash{}{0pt}%
\pgfpathmoveto{\pgfqpoint{5.866377in}{2.159848in}}%
\pgfpathlineto{\pgfqpoint{6.008600in}{1.974963in}}%
\pgfpathlineto{\pgfqpoint{5.943414in}{2.169358in}}%
\pgfpathclose%
\pgfusepath{fill}%
\end{pgfscope}%
\begin{pgfscope}%
\pgfpathrectangle{\pgfqpoint{0.539299in}{0.078740in}}{\pgfqpoint{7.842520in}{7.842520in}}%
\pgfusepath{clip}%
\pgfsetbuttcap%
\pgfsetroundjoin%
\definecolor{currentfill}{rgb}{0.223925,0.334994,0.548053}%
\pgfsetfillcolor{currentfill}%
\pgfsetlinewidth{0.000000pt}%
\definecolor{currentstroke}{rgb}{0.535621,0.835785,0.281908}%
\pgfsetstrokecolor{currentstroke}%
\pgfsetdash{}{0pt}%
\pgfpathmoveto{\pgfqpoint{2.058527in}{2.453906in}}%
\pgfpathlineto{\pgfqpoint{2.195719in}{2.435884in}}%
\pgfpathlineto{\pgfqpoint{2.266737in}{3.071192in}}%
\pgfpathclose%
\pgfusepath{fill}%
\end{pgfscope}%
\begin{pgfscope}%
\pgfpathrectangle{\pgfqpoint{0.539299in}{0.078740in}}{\pgfqpoint{7.842520in}{7.842520in}}%
\pgfusepath{clip}%
\pgfsetbuttcap%
\pgfsetroundjoin%
\definecolor{currentfill}{rgb}{0.906311,0.894855,0.098125}%
\pgfsetfillcolor{currentfill}%
\pgfsetlinewidth{0.000000pt}%
\definecolor{currentstroke}{rgb}{0.545524,0.838039,0.275626}%
\pgfsetstrokecolor{currentstroke}%
\pgfsetdash{}{0pt}%
\pgfpathmoveto{\pgfqpoint{3.324832in}{5.632651in}}%
\pgfpathlineto{\pgfqpoint{3.098325in}{5.625023in}}%
\pgfpathlineto{\pgfqpoint{3.237559in}{5.639984in}}%
\pgfpathclose%
\pgfusepath{fill}%
\end{pgfscope}%
\begin{pgfscope}%
\pgfpathrectangle{\pgfqpoint{0.539299in}{0.078740in}}{\pgfqpoint{7.842520in}{7.842520in}}%
\pgfusepath{clip}%
\pgfsetbuttcap%
\pgfsetroundjoin%
\definecolor{currentfill}{rgb}{0.386433,0.794644,0.372886}%
\pgfsetfillcolor{currentfill}%
\pgfsetlinewidth{0.000000pt}%
\definecolor{currentstroke}{rgb}{0.555484,0.840254,0.269281}%
\pgfsetstrokecolor{currentstroke}%
\pgfsetdash{}{0pt}%
\pgfpathmoveto{\pgfqpoint{2.347219in}{4.810962in}}%
\pgfpathlineto{\pgfqpoint{2.261900in}{4.632062in}}%
\pgfpathlineto{\pgfqpoint{2.393562in}{4.830606in}}%
\pgfpathclose%
\pgfusepath{fill}%
\end{pgfscope}%
\begin{pgfscope}%
\pgfpathrectangle{\pgfqpoint{0.539299in}{0.078740in}}{\pgfqpoint{7.842520in}{7.842520in}}%
\pgfusepath{clip}%
\pgfsetbuttcap%
\pgfsetroundjoin%
\definecolor{currentfill}{rgb}{0.267968,0.223549,0.512008}%
\pgfsetfillcolor{currentfill}%
\pgfsetlinewidth{0.000000pt}%
\definecolor{currentstroke}{rgb}{0.565498,0.842430,0.262877}%
\pgfsetstrokecolor{currentstroke}%
\pgfsetdash{}{0pt}%
\pgfpathmoveto{\pgfqpoint{5.866377in}{2.159848in}}%
\pgfpathlineto{\pgfqpoint{5.943414in}{2.169358in}}%
\pgfpathlineto{\pgfqpoint{5.724190in}{2.352097in}}%
\pgfpathclose%
\pgfusepath{fill}%
\end{pgfscope}%
\begin{pgfscope}%
\pgfpathrectangle{\pgfqpoint{0.539299in}{0.078740in}}{\pgfqpoint{7.842520in}{7.842520in}}%
\pgfusepath{clip}%
\pgfsetbuttcap%
\pgfsetroundjoin%
\definecolor{currentfill}{rgb}{0.197636,0.391528,0.554969}%
\pgfsetfillcolor{currentfill}%
\pgfsetlinewidth{0.000000pt}%
\definecolor{currentstroke}{rgb}{0.575563,0.844566,0.256415}%
\pgfsetstrokecolor{currentstroke}%
\pgfsetdash{}{0pt}%
\pgfpathmoveto{\pgfqpoint{5.518023in}{2.726615in}}%
\pgfpathlineto{\pgfqpoint{5.376216in}{2.927023in}}%
\pgfpathlineto{\pgfqpoint{5.297333in}{2.965317in}}%
\pgfpathclose%
\pgfusepath{fill}%
\end{pgfscope}%
\begin{pgfscope}%
\pgfpathrectangle{\pgfqpoint{0.539299in}{0.078740in}}{\pgfqpoint{7.842520in}{7.842520in}}%
\pgfusepath{clip}%
\pgfsetbuttcap%
\pgfsetroundjoin%
\definecolor{currentfill}{rgb}{0.525776,0.833491,0.288127}%
\pgfsetfillcolor{currentfill}%
\pgfsetlinewidth{0.000000pt}%
\definecolor{currentstroke}{rgb}{0.585678,0.846661,0.249897}%
\pgfsetstrokecolor{currentstroke}%
\pgfsetdash{}{0pt}%
\pgfpathmoveto{\pgfqpoint{2.347219in}{4.810962in}}%
\pgfpathlineto{\pgfqpoint{2.478789in}{5.021565in}}%
\pgfpathlineto{\pgfqpoint{2.564916in}{5.170485in}}%
\pgfpathclose%
\pgfusepath{fill}%
\end{pgfscope}%
\begin{pgfscope}%
\pgfpathrectangle{\pgfqpoint{0.539299in}{0.078740in}}{\pgfqpoint{7.842520in}{7.842520in}}%
\pgfusepath{clip}%
\pgfsetbuttcap%
\pgfsetroundjoin%
\definecolor{currentfill}{rgb}{0.772852,0.877868,0.131109}%
\pgfsetfillcolor{currentfill}%
\pgfsetlinewidth{0.000000pt}%
\definecolor{currentstroke}{rgb}{0.595839,0.848717,0.243329}%
\pgfsetstrokecolor{currentstroke}%
\pgfsetdash{}{0pt}%
\pgfpathmoveto{\pgfqpoint{2.651662in}{5.282437in}}%
\pgfpathlineto{\pgfqpoint{2.786367in}{5.433811in}}%
\pgfpathlineto{\pgfqpoint{2.873601in}{5.508440in}}%
\pgfpathclose%
\pgfusepath{fill}%
\end{pgfscope}%
\begin{pgfscope}%
\pgfpathrectangle{\pgfqpoint{0.539299in}{0.078740in}}{\pgfqpoint{7.842520in}{7.842520in}}%
\pgfusepath{clip}%
\pgfsetbuttcap%
\pgfsetroundjoin%
\definecolor{currentfill}{rgb}{0.119512,0.607464,0.540218}%
\pgfsetfillcolor{currentfill}%
\pgfsetlinewidth{0.000000pt}%
\definecolor{currentstroke}{rgb}{0.606045,0.850733,0.236712}%
\pgfsetstrokecolor{currentstroke}%
\pgfsetdash{}{0pt}%
\pgfpathmoveto{\pgfqpoint{4.665156in}{4.013355in}}%
\pgfpathlineto{\pgfqpoint{4.807756in}{3.788327in}}%
\pgfpathlineto{\pgfqpoint{4.888578in}{3.720803in}}%
\pgfpathclose%
\pgfusepath{fill}%
\end{pgfscope}%
\begin{pgfscope}%
\pgfpathrectangle{\pgfqpoint{0.539299in}{0.078740in}}{\pgfqpoint{7.842520in}{7.842520in}}%
\pgfusepath{clip}%
\pgfsetbuttcap%
\pgfsetroundjoin%
\definecolor{currentfill}{rgb}{0.122606,0.585371,0.546557}%
\pgfsetfillcolor{currentfill}%
\pgfsetlinewidth{0.000000pt}%
\definecolor{currentstroke}{rgb}{0.616293,0.852709,0.230052}%
\pgfsetstrokecolor{currentstroke}%
\pgfsetdash{}{0pt}%
\pgfpathmoveto{\pgfqpoint{2.013886in}{3.822753in}}%
\pgfpathlineto{\pgfqpoint{1.935199in}{3.440713in}}%
\pgfpathlineto{\pgfqpoint{2.146689in}{3.948369in}}%
\pgfpathclose%
\pgfusepath{fill}%
\end{pgfscope}%
\begin{pgfscope}%
\pgfpathrectangle{\pgfqpoint{0.539299in}{0.078740in}}{\pgfqpoint{7.842520in}{7.842520in}}%
\pgfusepath{clip}%
\pgfsetbuttcap%
\pgfsetroundjoin%
\definecolor{currentfill}{rgb}{0.248629,0.278775,0.534556}%
\pgfsetfillcolor{currentfill}%
\pgfsetlinewidth{0.000000pt}%
\definecolor{currentstroke}{rgb}{0.626579,0.854645,0.223353}%
\pgfsetstrokecolor{currentstroke}%
\pgfsetdash{}{0pt}%
\pgfpathmoveto{\pgfqpoint{5.581983in}{2.550780in}}%
\pgfpathlineto{\pgfqpoint{5.724190in}{2.352097in}}%
\pgfpathlineto{\pgfqpoint{5.801571in}{2.347348in}}%
\pgfpathclose%
\pgfusepath{fill}%
\end{pgfscope}%
\begin{pgfscope}%
\pgfpathrectangle{\pgfqpoint{0.539299in}{0.078740in}}{\pgfqpoint{7.842520in}{7.842520in}}%
\pgfusepath{clip}%
\pgfsetbuttcap%
\pgfsetroundjoin%
\definecolor{currentfill}{rgb}{0.182256,0.426184,0.557120}%
\pgfsetfillcolor{currentfill}%
\pgfsetlinewidth{0.000000pt}%
\definecolor{currentstroke}{rgb}{0.636902,0.856542,0.216620}%
\pgfsetstrokecolor{currentstroke}%
\pgfsetdash{}{0pt}%
\pgfpathmoveto{\pgfqpoint{5.297333in}{2.965317in}}%
\pgfpathlineto{\pgfqpoint{5.376216in}{2.927023in}}%
\pgfpathlineto{\pgfqpoint{5.234327in}{3.134099in}}%
\pgfpathclose%
\pgfusepath{fill}%
\end{pgfscope}%
\begin{pgfscope}%
\pgfpathrectangle{\pgfqpoint{0.539299in}{0.078740in}}{\pgfqpoint{7.842520in}{7.842520in}}%
\pgfusepath{clip}%
\pgfsetbuttcap%
\pgfsetroundjoin%
\definecolor{currentfill}{rgb}{0.202219,0.715272,0.476084}%
\pgfsetfillcolor{currentfill}%
\pgfsetlinewidth{0.000000pt}%
\definecolor{currentstroke}{rgb}{0.647257,0.858400,0.209861}%
\pgfsetstrokecolor{currentstroke}%
\pgfsetdash{}{0pt}%
\pgfpathmoveto{\pgfqpoint{4.604647in}{4.165055in}}%
\pgfpathlineto{\pgfqpoint{4.462343in}{4.388175in}}%
\pgfpathlineto{\pgfqpoint{4.379257in}{4.461675in}}%
\pgfpathclose%
\pgfusepath{fill}%
\end{pgfscope}%
\begin{pgfscope}%
\pgfpathrectangle{\pgfqpoint{0.539299in}{0.078740in}}{\pgfqpoint{7.842520in}{7.842520in}}%
\pgfusepath{clip}%
\pgfsetbuttcap%
\pgfsetroundjoin%
\definecolor{currentfill}{rgb}{0.163625,0.471133,0.558148}%
\pgfsetfillcolor{currentfill}%
\pgfsetlinewidth{0.000000pt}%
\definecolor{currentstroke}{rgb}{0.657642,0.860219,0.203082}%
\pgfsetstrokecolor{currentstroke}%
\pgfsetdash{}{0pt}%
\pgfpathmoveto{\pgfqpoint{1.993699in}{3.037069in}}%
\pgfpathlineto{\pgfqpoint{2.129568in}{3.063212in}}%
\pgfpathlineto{\pgfqpoint{2.068708in}{3.529775in}}%
\pgfpathclose%
\pgfusepath{fill}%
\end{pgfscope}%
\begin{pgfscope}%
\pgfpathrectangle{\pgfqpoint{0.539299in}{0.078740in}}{\pgfqpoint{7.842520in}{7.842520in}}%
\pgfusepath{clip}%
\pgfsetbuttcap%
\pgfsetroundjoin%
\definecolor{currentfill}{rgb}{0.896320,0.893616,0.096335}%
\pgfsetfillcolor{currentfill}%
\pgfsetlinewidth{0.000000pt}%
\definecolor{currentstroke}{rgb}{0.668054,0.861999,0.196293}%
\pgfsetstrokecolor{currentstroke}%
\pgfsetdash{}{0pt}%
\pgfpathmoveto{\pgfqpoint{3.465333in}{5.584033in}}%
\pgfpathlineto{\pgfqpoint{3.324832in}{5.632651in}}%
\pgfpathlineto{\pgfqpoint{3.378279in}{5.603577in}}%
\pgfpathclose%
\pgfusepath{fill}%
\end{pgfscope}%
\begin{pgfscope}%
\pgfpathrectangle{\pgfqpoint{0.539299in}{0.078740in}}{\pgfqpoint{7.842520in}{7.842520in}}%
\pgfusepath{clip}%
\pgfsetbuttcap%
\pgfsetroundjoin%
\definecolor{currentfill}{rgb}{0.197636,0.391528,0.554969}%
\pgfsetfillcolor{currentfill}%
\pgfsetlinewidth{0.000000pt}%
\definecolor{currentstroke}{rgb}{0.678489,0.863742,0.189503}%
\pgfsetstrokecolor{currentstroke}%
\pgfsetdash{}{0pt}%
\pgfpathmoveto{\pgfqpoint{2.266737in}{3.071192in}}%
\pgfpathlineto{\pgfqpoint{2.129568in}{3.063212in}}%
\pgfpathlineto{\pgfqpoint{2.058527in}{2.453906in}}%
\pgfpathclose%
\pgfusepath{fill}%
\end{pgfscope}%
\begin{pgfscope}%
\pgfpathrectangle{\pgfqpoint{0.539299in}{0.078740in}}{\pgfqpoint{7.842520in}{7.842520in}}%
\pgfusepath{clip}%
\pgfsetbuttcap%
\pgfsetroundjoin%
\definecolor{currentfill}{rgb}{0.216210,0.351535,0.550627}%
\pgfsetfillcolor{currentfill}%
\pgfsetlinewidth{0.000000pt}%
\definecolor{currentstroke}{rgb}{0.688944,0.865448,0.182725}%
\pgfsetstrokecolor{currentstroke}%
\pgfsetdash{}{0pt}%
\pgfpathmoveto{\pgfqpoint{5.518023in}{2.726615in}}%
\pgfpathlineto{\pgfqpoint{5.439710in}{2.755331in}}%
\pgfpathlineto{\pgfqpoint{5.581983in}{2.550780in}}%
\pgfpathclose%
\pgfusepath{fill}%
\end{pgfscope}%
\begin{pgfscope}%
\pgfpathrectangle{\pgfqpoint{0.539299in}{0.078740in}}{\pgfqpoint{7.842520in}{7.842520in}}%
\pgfusepath{clip}%
\pgfsetbuttcap%
\pgfsetroundjoin%
\definecolor{currentfill}{rgb}{0.296479,0.761561,0.424223}%
\pgfsetfillcolor{currentfill}%
\pgfsetlinewidth{0.000000pt}%
\definecolor{currentstroke}{rgb}{0.699415,0.867117,0.175971}%
\pgfsetstrokecolor{currentstroke}%
\pgfsetdash{}{0pt}%
\pgfpathmoveto{\pgfqpoint{2.177668in}{4.411616in}}%
\pgfpathlineto{\pgfqpoint{2.309553in}{4.592125in}}%
\pgfpathlineto{\pgfqpoint{2.261900in}{4.632062in}}%
\pgfpathclose%
\pgfusepath{fill}%
\end{pgfscope}%
\begin{pgfscope}%
\pgfpathrectangle{\pgfqpoint{0.539299in}{0.078740in}}{\pgfqpoint{7.842520in}{7.842520in}}%
\pgfusepath{clip}%
\pgfsetbuttcap%
\pgfsetroundjoin%
\definecolor{currentfill}{rgb}{0.266941,0.748751,0.440573}%
\pgfsetfillcolor{currentfill}%
\pgfsetlinewidth{0.000000pt}%
\definecolor{currentstroke}{rgb}{0.709898,0.868751,0.169257}%
\pgfsetstrokecolor{currentstroke}%
\pgfsetdash{}{0pt}%
\pgfpathmoveto{\pgfqpoint{4.319824in}{4.607800in}}%
\pgfpathlineto{\pgfqpoint{4.379257in}{4.461675in}}%
\pgfpathlineto{\pgfqpoint{4.462343in}{4.388175in}}%
\pgfpathclose%
\pgfusepath{fill}%
\end{pgfscope}%
\begin{pgfscope}%
\pgfpathrectangle{\pgfqpoint{0.539299in}{0.078740in}}{\pgfqpoint{7.842520in}{7.842520in}}%
\pgfusepath{clip}%
\pgfsetbuttcap%
\pgfsetroundjoin%
\definecolor{currentfill}{rgb}{0.137339,0.662252,0.515571}%
\pgfsetfillcolor{currentfill}%
\pgfsetlinewidth{0.000000pt}%
\definecolor{currentstroke}{rgb}{0.720391,0.870350,0.162603}%
\pgfsetstrokecolor{currentstroke}%
\pgfsetdash{}{0pt}%
\pgfpathmoveto{\pgfqpoint{4.746722in}{3.941721in}}%
\pgfpathlineto{\pgfqpoint{4.522320in}{4.238703in}}%
\pgfpathlineto{\pgfqpoint{4.665156in}{4.013355in}}%
\pgfpathclose%
\pgfusepath{fill}%
\end{pgfscope}%
\begin{pgfscope}%
\pgfpathrectangle{\pgfqpoint{0.539299in}{0.078740in}}{\pgfqpoint{7.842520in}{7.842520in}}%
\pgfusepath{clip}%
\pgfsetbuttcap%
\pgfsetroundjoin%
\definecolor{currentfill}{rgb}{0.127568,0.566949,0.550556}%
\pgfsetfillcolor{currentfill}%
\pgfsetlinewidth{0.000000pt}%
\definecolor{currentstroke}{rgb}{0.730889,0.871916,0.156029}%
\pgfsetstrokecolor{currentstroke}%
\pgfsetdash{}{0pt}%
\pgfpathmoveto{\pgfqpoint{2.146689in}{3.948369in}}%
\pgfpathlineto{\pgfqpoint{1.935199in}{3.440713in}}%
\pgfpathlineto{\pgfqpoint{2.068708in}{3.529775in}}%
\pgfpathclose%
\pgfusepath{fill}%
\end{pgfscope}%
\begin{pgfscope}%
\pgfpathrectangle{\pgfqpoint{0.539299in}{0.078740in}}{\pgfqpoint{7.842520in}{7.842520in}}%
\pgfusepath{clip}%
\pgfsetbuttcap%
\pgfsetroundjoin%
\definecolor{currentfill}{rgb}{0.162142,0.474838,0.558140}%
\pgfsetfillcolor{currentfill}%
\pgfsetlinewidth{0.000000pt}%
\definecolor{currentstroke}{rgb}{0.741388,0.873449,0.149561}%
\pgfsetstrokecolor{currentstroke}%
\pgfsetdash{}{0pt}%
\pgfpathmoveto{\pgfqpoint{5.154816in}{3.180272in}}%
\pgfpathlineto{\pgfqpoint{5.234327in}{3.134099in}}%
\pgfpathlineto{\pgfqpoint{5.092313in}{3.347287in}}%
\pgfpathclose%
\pgfusepath{fill}%
\end{pgfscope}%
\begin{pgfscope}%
\pgfpathrectangle{\pgfqpoint{0.539299in}{0.078740in}}{\pgfqpoint{7.842520in}{7.842520in}}%
\pgfusepath{clip}%
\pgfsetbuttcap%
\pgfsetroundjoin%
\definecolor{currentfill}{rgb}{0.468053,0.818921,0.323998}%
\pgfsetfillcolor{currentfill}%
\pgfsetlinewidth{0.000000pt}%
\definecolor{currentstroke}{rgb}{0.751884,0.874951,0.143228}%
\pgfsetstrokecolor{currentstroke}%
\pgfsetdash{}{0pt}%
\pgfpathmoveto{\pgfqpoint{2.393562in}{4.830606in}}%
\pgfpathlineto{\pgfqpoint{2.478789in}{5.021565in}}%
\pgfpathlineto{\pgfqpoint{2.347219in}{4.810962in}}%
\pgfpathclose%
\pgfusepath{fill}%
\end{pgfscope}%
\begin{pgfscope}%
\pgfpathrectangle{\pgfqpoint{0.539299in}{0.078740in}}{\pgfqpoint{7.842520in}{7.842520in}}%
\pgfusepath{clip}%
\pgfsetbuttcap%
\pgfsetroundjoin%
\definecolor{currentfill}{rgb}{0.143303,0.669459,0.511215}%
\pgfsetfillcolor{currentfill}%
\pgfsetlinewidth{0.000000pt}%
\definecolor{currentstroke}{rgb}{0.762373,0.876424,0.137064}%
\pgfsetstrokecolor{currentstroke}%
\pgfsetdash{}{0pt}%
\pgfpathmoveto{\pgfqpoint{2.227125in}{4.300185in}}%
\pgfpathlineto{\pgfqpoint{2.094863in}{4.143974in}}%
\pgfpathlineto{\pgfqpoint{2.013886in}{3.822753in}}%
\pgfpathclose%
\pgfusepath{fill}%
\end{pgfscope}%
\begin{pgfscope}%
\pgfpathrectangle{\pgfqpoint{0.539299in}{0.078740in}}{\pgfqpoint{7.842520in}{7.842520in}}%
\pgfusepath{clip}%
\pgfsetbuttcap%
\pgfsetroundjoin%
\definecolor{currentfill}{rgb}{0.699415,0.867117,0.175971}%
\pgfsetfillcolor{currentfill}%
\pgfsetlinewidth{0.000000pt}%
\definecolor{currentstroke}{rgb}{0.772852,0.877868,0.131109}%
\pgfsetstrokecolor{currentstroke}%
\pgfsetdash{}{0pt}%
\pgfpathmoveto{\pgfqpoint{2.699533in}{5.323077in}}%
\pgfpathlineto{\pgfqpoint{2.651662in}{5.282437in}}%
\pgfpathlineto{\pgfqpoint{2.564916in}{5.170485in}}%
\pgfpathclose%
\pgfusepath{fill}%
\end{pgfscope}%
\begin{pgfscope}%
\pgfpathrectangle{\pgfqpoint{0.539299in}{0.078740in}}{\pgfqpoint{7.842520in}{7.842520in}}%
\pgfusepath{clip}%
\pgfsetbuttcap%
\pgfsetroundjoin%
\definecolor{currentfill}{rgb}{0.191090,0.708366,0.482284}%
\pgfsetfillcolor{currentfill}%
\pgfsetlinewidth{0.000000pt}%
\definecolor{currentstroke}{rgb}{0.783315,0.879285,0.125405}%
\pgfsetstrokecolor{currentstroke}%
\pgfsetdash{}{0pt}%
\pgfpathmoveto{\pgfqpoint{2.177668in}{4.411616in}}%
\pgfpathlineto{\pgfqpoint{2.094863in}{4.143974in}}%
\pgfpathlineto{\pgfqpoint{2.227125in}{4.300185in}}%
\pgfpathclose%
\pgfusepath{fill}%
\end{pgfscope}%
\begin{pgfscope}%
\pgfpathrectangle{\pgfqpoint{0.539299in}{0.078740in}}{\pgfqpoint{7.842520in}{7.842520in}}%
\pgfusepath{clip}%
\pgfsetbuttcap%
\pgfsetroundjoin%
\definecolor{currentfill}{rgb}{0.926106,0.897330,0.104071}%
\pgfsetfillcolor{currentfill}%
\pgfsetlinewidth{0.000000pt}%
\definecolor{currentstroke}{rgb}{0.793760,0.880678,0.120005}%
\pgfsetstrokecolor{currentstroke}%
\pgfsetdash{}{0pt}%
\pgfpathmoveto{\pgfqpoint{3.378279in}{5.603577in}}%
\pgfpathlineto{\pgfqpoint{3.324832in}{5.632651in}}%
\pgfpathlineto{\pgfqpoint{3.237559in}{5.639984in}}%
\pgfpathclose%
\pgfusepath{fill}%
\end{pgfscope}%
\begin{pgfscope}%
\pgfpathrectangle{\pgfqpoint{0.539299in}{0.078740in}}{\pgfqpoint{7.842520in}{7.842520in}}%
\pgfusepath{clip}%
\pgfsetbuttcap%
\pgfsetroundjoin%
\definecolor{currentfill}{rgb}{0.199430,0.387607,0.554642}%
\pgfsetfillcolor{currentfill}%
\pgfsetlinewidth{0.000000pt}%
\definecolor{currentstroke}{rgb}{0.804182,0.882046,0.114965}%
\pgfsetstrokecolor{currentstroke}%
\pgfsetdash{}{0pt}%
\pgfpathmoveto{\pgfqpoint{5.297333in}{2.965317in}}%
\pgfpathlineto{\pgfqpoint{5.439710in}{2.755331in}}%
\pgfpathlineto{\pgfqpoint{5.518023in}{2.726615in}}%
\pgfpathclose%
\pgfusepath{fill}%
\end{pgfscope}%
\begin{pgfscope}%
\pgfpathrectangle{\pgfqpoint{0.539299in}{0.078740in}}{\pgfqpoint{7.842520in}{7.842520in}}%
\pgfusepath{clip}%
\pgfsetbuttcap%
\pgfsetroundjoin%
\definecolor{currentfill}{rgb}{0.886271,0.892374,0.095374}%
\pgfsetfillcolor{currentfill}%
\pgfsetlinewidth{0.000000pt}%
\definecolor{currentstroke}{rgb}{0.814576,0.883393,0.110347}%
\pgfsetstrokecolor{currentstroke}%
\pgfsetdash{}{0pt}%
\pgfpathmoveto{\pgfqpoint{3.378279in}{5.603577in}}%
\pgfpathlineto{\pgfqpoint{3.606833in}{5.491393in}}%
\pgfpathlineto{\pgfqpoint{3.465333in}{5.584033in}}%
\pgfpathclose%
\pgfusepath{fill}%
\end{pgfscope}%
\begin{pgfscope}%
\pgfpathrectangle{\pgfqpoint{0.539299in}{0.078740in}}{\pgfqpoint{7.842520in}{7.842520in}}%
\pgfusepath{clip}%
\pgfsetbuttcap%
\pgfsetroundjoin%
\definecolor{currentfill}{rgb}{0.855810,0.888601,0.097452}%
\pgfsetfillcolor{currentfill}%
\pgfsetlinewidth{0.000000pt}%
\definecolor{currentstroke}{rgb}{0.824940,0.884720,0.106217}%
\pgfsetstrokecolor{currentstroke}%
\pgfsetdash{}{0pt}%
\pgfpathmoveto{\pgfqpoint{3.010887in}{5.591305in}}%
\pgfpathlineto{\pgfqpoint{2.873601in}{5.508440in}}%
\pgfpathlineto{\pgfqpoint{2.786367in}{5.433811in}}%
\pgfpathclose%
\pgfusepath{fill}%
\end{pgfscope}%
\begin{pgfscope}%
\pgfpathrectangle{\pgfqpoint{0.539299in}{0.078740in}}{\pgfqpoint{7.842520in}{7.842520in}}%
\pgfusepath{clip}%
\pgfsetbuttcap%
\pgfsetroundjoin%
\definecolor{currentfill}{rgb}{0.196571,0.711827,0.479221}%
\pgfsetfillcolor{currentfill}%
\pgfsetlinewidth{0.000000pt}%
\definecolor{currentstroke}{rgb}{0.835270,0.886029,0.102646}%
\pgfsetstrokecolor{currentstroke}%
\pgfsetdash{}{0pt}%
\pgfpathmoveto{\pgfqpoint{4.379257in}{4.461675in}}%
\pgfpathlineto{\pgfqpoint{4.522320in}{4.238703in}}%
\pgfpathlineto{\pgfqpoint{4.604647in}{4.165055in}}%
\pgfpathclose%
\pgfusepath{fill}%
\end{pgfscope}%
\begin{pgfscope}%
\pgfpathrectangle{\pgfqpoint{0.539299in}{0.078740in}}{\pgfqpoint{7.842520in}{7.842520in}}%
\pgfusepath{clip}%
\pgfsetbuttcap%
\pgfsetroundjoin%
\definecolor{currentfill}{rgb}{0.412913,0.803041,0.357269}%
\pgfsetfillcolor{currentfill}%
\pgfsetlinewidth{0.000000pt}%
\definecolor{currentstroke}{rgb}{0.845561,0.887322,0.099702}%
\pgfsetstrokecolor{currentstroke}%
\pgfsetdash{}{0pt}%
\pgfpathmoveto{\pgfqpoint{4.177132in}{4.819968in}}%
\pgfpathlineto{\pgfqpoint{4.092594in}{4.886710in}}%
\pgfpathlineto{\pgfqpoint{4.319824in}{4.607800in}}%
\pgfpathclose%
\pgfusepath{fill}%
\end{pgfscope}%
\begin{pgfscope}%
\pgfpathrectangle{\pgfqpoint{0.539299in}{0.078740in}}{\pgfqpoint{7.842520in}{7.842520in}}%
\pgfusepath{clip}%
\pgfsetbuttcap%
\pgfsetroundjoin%
\definecolor{currentfill}{rgb}{0.218130,0.347432,0.550038}%
\pgfsetfillcolor{currentfill}%
\pgfsetlinewidth{0.000000pt}%
\definecolor{currentstroke}{rgb}{0.855810,0.888601,0.097452}%
\pgfsetstrokecolor{currentstroke}%
\pgfsetdash{}{0pt}%
\pgfpathmoveto{\pgfqpoint{2.195719in}{2.435884in}}%
\pgfpathlineto{\pgfqpoint{2.333716in}{2.409916in}}%
\pgfpathlineto{\pgfqpoint{2.405032in}{3.063501in}}%
\pgfpathclose%
\pgfusepath{fill}%
\end{pgfscope}%
\begin{pgfscope}%
\pgfpathrectangle{\pgfqpoint{0.539299in}{0.078740in}}{\pgfqpoint{7.842520in}{7.842520in}}%
\pgfusepath{clip}%
\pgfsetbuttcap%
\pgfsetroundjoin%
\definecolor{currentfill}{rgb}{0.762373,0.876424,0.137064}%
\pgfsetfillcolor{currentfill}%
\pgfsetlinewidth{0.000000pt}%
\definecolor{currentstroke}{rgb}{0.866013,0.889868,0.095953}%
\pgfsetstrokecolor{currentstroke}%
\pgfsetdash{}{0pt}%
\pgfpathmoveto{\pgfqpoint{2.786367in}{5.433811in}}%
\pgfpathlineto{\pgfqpoint{2.651662in}{5.282437in}}%
\pgfpathlineto{\pgfqpoint{2.699533in}{5.323077in}}%
\pgfpathclose%
\pgfusepath{fill}%
\end{pgfscope}%
\begin{pgfscope}%
\pgfpathrectangle{\pgfqpoint{0.539299in}{0.078740in}}{\pgfqpoint{7.842520in}{7.842520in}}%
\pgfusepath{clip}%
\pgfsetbuttcap%
\pgfsetroundjoin%
\definecolor{currentfill}{rgb}{0.282327,0.094955,0.417331}%
\pgfsetfillcolor{currentfill}%
\pgfsetlinewidth{0.000000pt}%
\definecolor{currentstroke}{rgb}{0.876168,0.891125,0.095250}%
\pgfsetstrokecolor{currentstroke}%
\pgfsetdash{}{0pt}%
\pgfpathmoveto{\pgfqpoint{6.073966in}{1.765621in}}%
\pgfpathlineto{\pgfqpoint{6.293516in}{1.634687in}}%
\pgfpathlineto{\pgfqpoint{6.150940in}{1.799026in}}%
\pgfpathclose%
\pgfusepath{fill}%
\end{pgfscope}%
\begin{pgfscope}%
\pgfpathrectangle{\pgfqpoint{0.539299in}{0.078740in}}{\pgfqpoint{7.842520in}{7.842520in}}%
\pgfusepath{clip}%
\pgfsetbuttcap%
\pgfsetroundjoin%
\definecolor{currentfill}{rgb}{0.283072,0.130895,0.449241}%
\pgfsetfillcolor{currentfill}%
\pgfsetlinewidth{0.000000pt}%
\definecolor{currentstroke}{rgb}{0.886271,0.892374,0.095374}%
\pgfsetstrokecolor{currentstroke}%
\pgfsetdash{}{0pt}%
\pgfpathmoveto{\pgfqpoint{6.150940in}{1.799026in}}%
\pgfpathlineto{\pgfqpoint{6.008600in}{1.974963in}}%
\pgfpathlineto{\pgfqpoint{6.073966in}{1.765621in}}%
\pgfpathclose%
\pgfusepath{fill}%
\end{pgfscope}%
\begin{pgfscope}%
\pgfpathrectangle{\pgfqpoint{0.539299in}{0.078740in}}{\pgfqpoint{7.842520in}{7.842520in}}%
\pgfusepath{clip}%
\pgfsetbuttcap%
\pgfsetroundjoin%
\definecolor{currentfill}{rgb}{0.135066,0.544853,0.554029}%
\pgfsetfillcolor{currentfill}%
\pgfsetlinewidth{0.000000pt}%
\definecolor{currentstroke}{rgb}{0.896320,0.893616,0.096335}%
\pgfsetstrokecolor{currentstroke}%
\pgfsetdash{}{0pt}%
\pgfpathmoveto{\pgfqpoint{5.092313in}{3.347287in}}%
\pgfpathlineto{\pgfqpoint{4.950134in}{3.565760in}}%
\pgfpathlineto{\pgfqpoint{4.869240in}{3.622273in}}%
\pgfpathclose%
\pgfusepath{fill}%
\end{pgfscope}%
\begin{pgfscope}%
\pgfpathrectangle{\pgfqpoint{0.539299in}{0.078740in}}{\pgfqpoint{7.842520in}{7.842520in}}%
\pgfusepath{clip}%
\pgfsetbuttcap%
\pgfsetroundjoin%
\definecolor{currentfill}{rgb}{0.171176,0.452530,0.557965}%
\pgfsetfillcolor{currentfill}%
\pgfsetlinewidth{0.000000pt}%
\definecolor{currentstroke}{rgb}{0.906311,0.894855,0.098125}%
\pgfsetstrokecolor{currentstroke}%
\pgfsetdash{}{0pt}%
\pgfpathmoveto{\pgfqpoint{5.234327in}{3.134099in}}%
\pgfpathlineto{\pgfqpoint{5.154816in}{3.180272in}}%
\pgfpathlineto{\pgfqpoint{5.297333in}{2.965317in}}%
\pgfpathclose%
\pgfusepath{fill}%
\end{pgfscope}%
\begin{pgfscope}%
\pgfpathrectangle{\pgfqpoint{0.539299in}{0.078740in}}{\pgfqpoint{7.842520in}{7.842520in}}%
\pgfusepath{clip}%
\pgfsetbuttcap%
\pgfsetroundjoin%
\definecolor{currentfill}{rgb}{0.496615,0.826376,0.306377}%
\pgfsetfillcolor{currentfill}%
\pgfsetlinewidth{0.000000pt}%
\definecolor{currentstroke}{rgb}{0.916242,0.896091,0.100717}%
\pgfsetstrokecolor{currentstroke}%
\pgfsetdash{}{0pt}%
\pgfpathmoveto{\pgfqpoint{4.034345in}{5.020034in}}%
\pgfpathlineto{\pgfqpoint{4.092594in}{4.886710in}}%
\pgfpathlineto{\pgfqpoint{4.177132in}{4.819968in}}%
\pgfpathclose%
\pgfusepath{fill}%
\end{pgfscope}%
\begin{pgfscope}%
\pgfpathrectangle{\pgfqpoint{0.539299in}{0.078740in}}{\pgfqpoint{7.842520in}{7.842520in}}%
\pgfusepath{clip}%
\pgfsetbuttcap%
\pgfsetroundjoin%
\definecolor{currentfill}{rgb}{0.277134,0.185228,0.489898}%
\pgfsetfillcolor{currentfill}%
\pgfsetlinewidth{0.000000pt}%
\definecolor{currentstroke}{rgb}{0.926106,0.897330,0.104071}%
\pgfsetstrokecolor{currentstroke}%
\pgfsetdash{}{0pt}%
\pgfpathmoveto{\pgfqpoint{5.931443in}{1.959659in}}%
\pgfpathlineto{\pgfqpoint{6.008600in}{1.974963in}}%
\pgfpathlineto{\pgfqpoint{5.866377in}{2.159848in}}%
\pgfpathclose%
\pgfusepath{fill}%
\end{pgfscope}%
\begin{pgfscope}%
\pgfpathrectangle{\pgfqpoint{0.539299in}{0.078740in}}{\pgfqpoint{7.842520in}{7.842520in}}%
\pgfusepath{clip}%
\pgfsetbuttcap%
\pgfsetroundjoin%
\definecolor{currentfill}{rgb}{0.377779,0.791781,0.377939}%
\pgfsetfillcolor{currentfill}%
\pgfsetlinewidth{0.000000pt}%
\definecolor{currentstroke}{rgb}{0.935904,0.898570,0.108131}%
\pgfsetstrokecolor{currentstroke}%
\pgfsetdash{}{0pt}%
\pgfpathmoveto{\pgfqpoint{2.261900in}{4.632062in}}%
\pgfpathlineto{\pgfqpoint{2.309553in}{4.592125in}}%
\pgfpathlineto{\pgfqpoint{2.393562in}{4.830606in}}%
\pgfpathclose%
\pgfusepath{fill}%
\end{pgfscope}%
\begin{pgfscope}%
\pgfpathrectangle{\pgfqpoint{0.539299in}{0.078740in}}{\pgfqpoint{7.842520in}{7.842520in}}%
\pgfusepath{clip}%
\pgfsetbuttcap%
\pgfsetroundjoin%
\definecolor{currentfill}{rgb}{0.134692,0.658636,0.517649}%
\pgfsetfillcolor{currentfill}%
\pgfsetlinewidth{0.000000pt}%
\definecolor{currentstroke}{rgb}{0.945636,0.899815,0.112838}%
\pgfsetstrokecolor{currentstroke}%
\pgfsetdash{}{0pt}%
\pgfpathmoveto{\pgfqpoint{2.013886in}{3.822753in}}%
\pgfpathlineto{\pgfqpoint{2.146689in}{3.948369in}}%
\pgfpathlineto{\pgfqpoint{2.227125in}{4.300185in}}%
\pgfpathclose%
\pgfusepath{fill}%
\end{pgfscope}%
\begin{pgfscope}%
\pgfpathrectangle{\pgfqpoint{0.539299in}{0.078740in}}{\pgfqpoint{7.842520in}{7.842520in}}%
\pgfusepath{clip}%
\pgfsetbuttcap%
\pgfsetroundjoin%
\definecolor{currentfill}{rgb}{0.935904,0.898570,0.108131}%
\pgfsetfillcolor{currentfill}%
\pgfsetlinewidth{0.000000pt}%
\definecolor{currentstroke}{rgb}{0.955300,0.901065,0.118128}%
\pgfsetstrokecolor{currentstroke}%
\pgfsetdash{}{0pt}%
\pgfpathmoveto{\pgfqpoint{3.237559in}{5.639984in}}%
\pgfpathlineto{\pgfqpoint{3.098325in}{5.625023in}}%
\pgfpathlineto{\pgfqpoint{3.150177in}{5.616991in}}%
\pgfpathclose%
\pgfusepath{fill}%
\end{pgfscope}%
\begin{pgfscope}%
\pgfpathrectangle{\pgfqpoint{0.539299in}{0.078740in}}{\pgfqpoint{7.842520in}{7.842520in}}%
\pgfusepath{clip}%
\pgfsetbuttcap%
\pgfsetroundjoin%
\definecolor{currentfill}{rgb}{0.814576,0.883393,0.110347}%
\pgfsetfillcolor{currentfill}%
\pgfsetlinewidth{0.000000pt}%
\definecolor{currentstroke}{rgb}{0.964894,0.902323,0.123941}%
\pgfsetstrokecolor{currentstroke}%
\pgfsetdash{}{0pt}%
\pgfpathmoveto{\pgfqpoint{3.749006in}{5.361964in}}%
\pgfpathlineto{\pgfqpoint{3.606833in}{5.491393in}}%
\pgfpathlineto{\pgfqpoint{3.662699in}{5.404278in}}%
\pgfpathclose%
\pgfusepath{fill}%
\end{pgfscope}%
\begin{pgfscope}%
\pgfpathrectangle{\pgfqpoint{0.539299in}{0.078740in}}{\pgfqpoint{7.842520in}{7.842520in}}%
\pgfusepath{clip}%
\pgfsetbuttcap%
\pgfsetroundjoin%
\definecolor{currentfill}{rgb}{0.143343,0.522773,0.556295}%
\pgfsetfillcolor{currentfill}%
\pgfsetlinewidth{0.000000pt}%
\definecolor{currentstroke}{rgb}{0.974417,0.903590,0.130215}%
\pgfsetstrokecolor{currentstroke}%
\pgfsetdash{}{0pt}%
\pgfpathmoveto{\pgfqpoint{2.068708in}{3.529775in}}%
\pgfpathlineto{\pgfqpoint{2.129568in}{3.063212in}}%
\pgfpathlineto{\pgfqpoint{2.204165in}{3.588171in}}%
\pgfpathclose%
\pgfusepath{fill}%
\end{pgfscope}%
\begin{pgfscope}%
\pgfpathrectangle{\pgfqpoint{0.539299in}{0.078740in}}{\pgfqpoint{7.842520in}{7.842520in}}%
\pgfusepath{clip}%
\pgfsetbuttcap%
\pgfsetroundjoin%
\definecolor{currentfill}{rgb}{0.751884,0.874951,0.143228}%
\pgfsetfillcolor{currentfill}%
\pgfsetlinewidth{0.000000pt}%
\definecolor{currentstroke}{rgb}{0.983868,0.904867,0.136897}%
\pgfsetstrokecolor{currentstroke}%
\pgfsetdash{}{0pt}%
\pgfpathmoveto{\pgfqpoint{3.662699in}{5.404278in}}%
\pgfpathlineto{\pgfqpoint{3.891582in}{5.202680in}}%
\pgfpathlineto{\pgfqpoint{3.749006in}{5.361964in}}%
\pgfpathclose%
\pgfusepath{fill}%
\end{pgfscope}%
\begin{pgfscope}%
\pgfpathrectangle{\pgfqpoint{0.539299in}{0.078740in}}{\pgfqpoint{7.842520in}{7.842520in}}%
\pgfusepath{clip}%
\pgfsetbuttcap%
\pgfsetroundjoin%
\definecolor{currentfill}{rgb}{0.657642,0.860219,0.203082}%
\pgfsetfillcolor{currentfill}%
\pgfsetlinewidth{0.000000pt}%
\definecolor{currentstroke}{rgb}{0.993248,0.906157,0.143936}%
\pgfsetstrokecolor{currentstroke}%
\pgfsetdash{}{0pt}%
\pgfpathmoveto{\pgfqpoint{4.034345in}{5.020034in}}%
\pgfpathlineto{\pgfqpoint{3.891582in}{5.202680in}}%
\pgfpathlineto{\pgfqpoint{3.805790in}{5.254759in}}%
\pgfpathclose%
\pgfusepath{fill}%
\end{pgfscope}%
\begin{pgfscope}%
\pgfpathrectangle{\pgfqpoint{0.539299in}{0.078740in}}{\pgfqpoint{7.842520in}{7.842520in}}%
\pgfusepath{clip}%
\pgfsetbuttcap%
\pgfsetroundjoin%
\definecolor{currentfill}{rgb}{0.262138,0.242286,0.520837}%
\pgfsetfillcolor{currentfill}%
\pgfsetlinewidth{0.000000pt}%
\definecolor{currentstroke}{rgb}{0.267004,0.004874,0.329415}%
\pgfsetstrokecolor{currentstroke}%
\pgfsetdash{}{0pt}%
\pgfpathmoveto{\pgfqpoint{5.788906in}{2.159810in}}%
\pgfpathlineto{\pgfqpoint{5.866377in}{2.159848in}}%
\pgfpathlineto{\pgfqpoint{5.724190in}{2.352097in}}%
\pgfpathclose%
\pgfusepath{fill}%
\end{pgfscope}%
\begin{pgfscope}%
\pgfpathrectangle{\pgfqpoint{0.539299in}{0.078740in}}{\pgfqpoint{7.842520in}{7.842520in}}%
\pgfusepath{clip}%
\pgfsetbuttcap%
\pgfsetroundjoin%
\definecolor{currentfill}{rgb}{0.657642,0.860219,0.203082}%
\pgfsetfillcolor{currentfill}%
\pgfsetlinewidth{0.000000pt}%
\definecolor{currentstroke}{rgb}{0.268510,0.009605,0.335427}%
\pgfsetstrokecolor{currentstroke}%
\pgfsetdash{}{0pt}%
\pgfpathmoveto{\pgfqpoint{2.564916in}{5.170485in}}%
\pgfpathlineto{\pgfqpoint{2.478789in}{5.021565in}}%
\pgfpathlineto{\pgfqpoint{2.699533in}{5.323077in}}%
\pgfpathclose%
\pgfusepath{fill}%
\end{pgfscope}%
\begin{pgfscope}%
\pgfpathrectangle{\pgfqpoint{0.539299in}{0.078740in}}{\pgfqpoint{7.842520in}{7.842520in}}%
\pgfusepath{clip}%
\pgfsetbuttcap%
\pgfsetroundjoin%
\definecolor{currentfill}{rgb}{0.935904,0.898570,0.108131}%
\pgfsetfillcolor{currentfill}%
\pgfsetlinewidth{0.000000pt}%
\definecolor{currentstroke}{rgb}{0.269944,0.014625,0.341379}%
\pgfsetstrokecolor{currentstroke}%
\pgfsetdash{}{0pt}%
\pgfpathmoveto{\pgfqpoint{3.150177in}{5.616991in}}%
\pgfpathlineto{\pgfqpoint{3.098325in}{5.625023in}}%
\pgfpathlineto{\pgfqpoint{3.010887in}{5.591305in}}%
\pgfpathclose%
\pgfusepath{fill}%
\end{pgfscope}%
\begin{pgfscope}%
\pgfpathrectangle{\pgfqpoint{0.539299in}{0.078740in}}{\pgfqpoint{7.842520in}{7.842520in}}%
\pgfusepath{clip}%
\pgfsetbuttcap%
\pgfsetroundjoin%
\definecolor{currentfill}{rgb}{0.266941,0.748751,0.440573}%
\pgfsetfillcolor{currentfill}%
\pgfsetlinewidth{0.000000pt}%
\definecolor{currentstroke}{rgb}{0.271305,0.019942,0.347269}%
\pgfsetstrokecolor{currentstroke}%
\pgfsetdash{}{0pt}%
\pgfpathmoveto{\pgfqpoint{2.227125in}{4.300185in}}%
\pgfpathlineto{\pgfqpoint{2.309553in}{4.592125in}}%
\pgfpathlineto{\pgfqpoint{2.177668in}{4.411616in}}%
\pgfpathclose%
\pgfusepath{fill}%
\end{pgfscope}%
\begin{pgfscope}%
\pgfpathrectangle{\pgfqpoint{0.539299in}{0.078740in}}{\pgfqpoint{7.842520in}{7.842520in}}%
\pgfusepath{clip}%
\pgfsetbuttcap%
\pgfsetroundjoin%
\definecolor{currentfill}{rgb}{0.327796,0.773980,0.406640}%
\pgfsetfillcolor{currentfill}%
\pgfsetlinewidth{0.000000pt}%
\definecolor{currentstroke}{rgb}{0.272594,0.025563,0.353093}%
\pgfsetstrokecolor{currentstroke}%
\pgfsetdash{}{0pt}%
\pgfpathmoveto{\pgfqpoint{4.235995in}{4.678979in}}%
\pgfpathlineto{\pgfqpoint{4.379257in}{4.461675in}}%
\pgfpathlineto{\pgfqpoint{4.319824in}{4.607800in}}%
\pgfpathclose%
\pgfusepath{fill}%
\end{pgfscope}%
\begin{pgfscope}%
\pgfpathrectangle{\pgfqpoint{0.539299in}{0.078740in}}{\pgfqpoint{7.842520in}{7.842520in}}%
\pgfusepath{clip}%
\pgfsetbuttcap%
\pgfsetroundjoin%
\definecolor{currentfill}{rgb}{0.120092,0.600104,0.542530}%
\pgfsetfillcolor{currentfill}%
\pgfsetlinewidth{0.000000pt}%
\definecolor{currentstroke}{rgb}{0.273809,0.031497,0.358853}%
\pgfsetstrokecolor{currentstroke}%
\pgfsetdash{}{0pt}%
\pgfpathmoveto{\pgfqpoint{4.950134in}{3.565760in}}%
\pgfpathlineto{\pgfqpoint{4.807756in}{3.788327in}}%
\pgfpathlineto{\pgfqpoint{4.726130in}{3.847156in}}%
\pgfpathclose%
\pgfusepath{fill}%
\end{pgfscope}%
\begin{pgfscope}%
\pgfpathrectangle{\pgfqpoint{0.539299in}{0.078740in}}{\pgfqpoint{7.842520in}{7.842520in}}%
\pgfusepath{clip}%
\pgfsetbuttcap%
\pgfsetroundjoin%
\definecolor{currentfill}{rgb}{0.192357,0.403199,0.555836}%
\pgfsetfillcolor{currentfill}%
\pgfsetlinewidth{0.000000pt}%
\definecolor{currentstroke}{rgb}{0.274952,0.037752,0.364543}%
\pgfsetstrokecolor{currentstroke}%
\pgfsetdash{}{0pt}%
\pgfpathmoveto{\pgfqpoint{2.266737in}{3.071192in}}%
\pgfpathlineto{\pgfqpoint{2.195719in}{2.435884in}}%
\pgfpathlineto{\pgfqpoint{2.405032in}{3.063501in}}%
\pgfpathclose%
\pgfusepath{fill}%
\end{pgfscope}%
\begin{pgfscope}%
\pgfpathrectangle{\pgfqpoint{0.539299in}{0.078740in}}{\pgfqpoint{7.842520in}{7.842520in}}%
\pgfusepath{clip}%
\pgfsetbuttcap%
\pgfsetroundjoin%
\definecolor{currentfill}{rgb}{0.150476,0.504369,0.557430}%
\pgfsetfillcolor{currentfill}%
\pgfsetlinewidth{0.000000pt}%
\definecolor{currentstroke}{rgb}{0.276022,0.044167,0.370164}%
\pgfsetstrokecolor{currentstroke}%
\pgfsetdash{}{0pt}%
\pgfpathmoveto{\pgfqpoint{5.092313in}{3.347287in}}%
\pgfpathlineto{\pgfqpoint{5.012128in}{3.399559in}}%
\pgfpathlineto{\pgfqpoint{5.154816in}{3.180272in}}%
\pgfpathclose%
\pgfusepath{fill}%
\end{pgfscope}%
\begin{pgfscope}%
\pgfpathrectangle{\pgfqpoint{0.539299in}{0.078740in}}{\pgfqpoint{7.842520in}{7.842520in}}%
\pgfusepath{clip}%
\pgfsetbuttcap%
\pgfsetroundjoin%
\definecolor{currentfill}{rgb}{0.280894,0.078907,0.402329}%
\pgfsetfillcolor{currentfill}%
\pgfsetlinewidth{0.000000pt}%
\definecolor{currentstroke}{rgb}{0.277018,0.050344,0.375715}%
\pgfsetstrokecolor{currentstroke}%
\pgfsetdash{}{0pt}%
\pgfpathmoveto{\pgfqpoint{6.216568in}{1.580071in}}%
\pgfpathlineto{\pgfqpoint{6.293516in}{1.634687in}}%
\pgfpathlineto{\pgfqpoint{6.073966in}{1.765621in}}%
\pgfpathclose%
\pgfusepath{fill}%
\end{pgfscope}%
\begin{pgfscope}%
\pgfpathrectangle{\pgfqpoint{0.539299in}{0.078740in}}{\pgfqpoint{7.842520in}{7.842520in}}%
\pgfusepath{clip}%
\pgfsetbuttcap%
\pgfsetroundjoin%
\definecolor{currentfill}{rgb}{0.896320,0.893616,0.096335}%
\pgfsetfillcolor{currentfill}%
\pgfsetlinewidth{0.000000pt}%
\definecolor{currentstroke}{rgb}{0.277941,0.056324,0.381191}%
\pgfsetstrokecolor{currentstroke}%
\pgfsetdash{}{0pt}%
\pgfpathmoveto{\pgfqpoint{3.520103in}{5.522735in}}%
\pgfpathlineto{\pgfqpoint{3.606833in}{5.491393in}}%
\pgfpathlineto{\pgfqpoint{3.378279in}{5.603577in}}%
\pgfpathclose%
\pgfusepath{fill}%
\end{pgfscope}%
\begin{pgfscope}%
\pgfpathrectangle{\pgfqpoint{0.539299in}{0.078740in}}{\pgfqpoint{7.842520in}{7.842520in}}%
\pgfusepath{clip}%
\pgfsetbuttcap%
\pgfsetroundjoin%
\definecolor{currentfill}{rgb}{0.239346,0.300855,0.540844}%
\pgfsetfillcolor{currentfill}%
\pgfsetlinewidth{0.000000pt}%
\definecolor{currentstroke}{rgb}{0.278791,0.062145,0.386592}%
\pgfsetstrokecolor{currentstroke}%
\pgfsetdash{}{0pt}%
\pgfpathmoveto{\pgfqpoint{5.646299in}{2.364748in}}%
\pgfpathlineto{\pgfqpoint{5.724190in}{2.352097in}}%
\pgfpathlineto{\pgfqpoint{5.581983in}{2.550780in}}%
\pgfpathclose%
\pgfusepath{fill}%
\end{pgfscope}%
\begin{pgfscope}%
\pgfpathrectangle{\pgfqpoint{0.539299in}{0.078740in}}{\pgfqpoint{7.842520in}{7.842520in}}%
\pgfusepath{clip}%
\pgfsetbuttcap%
\pgfsetroundjoin%
\definecolor{currentfill}{rgb}{0.218130,0.347432,0.550038}%
\pgfsetfillcolor{currentfill}%
\pgfsetlinewidth{0.000000pt}%
\definecolor{currentstroke}{rgb}{0.279566,0.067836,0.391917}%
\pgfsetstrokecolor{currentstroke}%
\pgfsetdash{}{0pt}%
\pgfpathmoveto{\pgfqpoint{2.405032in}{3.063501in}}%
\pgfpathlineto{\pgfqpoint{2.333716in}{2.409916in}}%
\pgfpathlineto{\pgfqpoint{2.472446in}{2.377069in}}%
\pgfpathclose%
\pgfusepath{fill}%
\end{pgfscope}%
\begin{pgfscope}%
\pgfpathrectangle{\pgfqpoint{0.539299in}{0.078740in}}{\pgfqpoint{7.842520in}{7.842520in}}%
\pgfusepath{clip}%
\pgfsetbuttcap%
\pgfsetroundjoin%
\definecolor{currentfill}{rgb}{0.121148,0.592739,0.544641}%
\pgfsetfillcolor{currentfill}%
\pgfsetlinewidth{0.000000pt}%
\definecolor{currentstroke}{rgb}{0.280267,0.073417,0.397163}%
\pgfsetstrokecolor{currentstroke}%
\pgfsetdash{}{0pt}%
\pgfpathmoveto{\pgfqpoint{2.146689in}{3.948369in}}%
\pgfpathlineto{\pgfqpoint{2.068708in}{3.529775in}}%
\pgfpathlineto{\pgfqpoint{2.204165in}{3.588171in}}%
\pgfpathclose%
\pgfusepath{fill}%
\end{pgfscope}%
\begin{pgfscope}%
\pgfpathrectangle{\pgfqpoint{0.539299in}{0.078740in}}{\pgfqpoint{7.842520in}{7.842520in}}%
\pgfusepath{clip}%
\pgfsetbuttcap%
\pgfsetroundjoin%
\definecolor{currentfill}{rgb}{0.404001,0.800275,0.362552}%
\pgfsetfillcolor{currentfill}%
\pgfsetlinewidth{0.000000pt}%
\definecolor{currentstroke}{rgb}{0.280894,0.078907,0.402329}%
\pgfsetstrokecolor{currentstroke}%
\pgfsetdash{}{0pt}%
\pgfpathmoveto{\pgfqpoint{4.319824in}{4.607800in}}%
\pgfpathlineto{\pgfqpoint{4.092594in}{4.886710in}}%
\pgfpathlineto{\pgfqpoint{4.235995in}{4.678979in}}%
\pgfpathclose%
\pgfusepath{fill}%
\end{pgfscope}%
\begin{pgfscope}%
\pgfpathrectangle{\pgfqpoint{0.539299in}{0.078740in}}{\pgfqpoint{7.842520in}{7.842520in}}%
\pgfusepath{clip}%
\pgfsetbuttcap%
\pgfsetroundjoin%
\definecolor{currentfill}{rgb}{0.281412,0.155834,0.469201}%
\pgfsetfillcolor{currentfill}%
\pgfsetlinewidth{0.000000pt}%
\definecolor{currentstroke}{rgb}{0.281446,0.084320,0.407414}%
\pgfsetstrokecolor{currentstroke}%
\pgfsetdash{}{0pt}%
\pgfpathmoveto{\pgfqpoint{6.008600in}{1.974963in}}%
\pgfpathlineto{\pgfqpoint{5.931443in}{1.959659in}}%
\pgfpathlineto{\pgfqpoint{6.073966in}{1.765621in}}%
\pgfpathclose%
\pgfusepath{fill}%
\end{pgfscope}%
\begin{pgfscope}%
\pgfpathrectangle{\pgfqpoint{0.539299in}{0.078740in}}{\pgfqpoint{7.842520in}{7.842520in}}%
\pgfusepath{clip}%
\pgfsetbuttcap%
\pgfsetroundjoin%
\definecolor{currentfill}{rgb}{0.136408,0.541173,0.554483}%
\pgfsetfillcolor{currentfill}%
\pgfsetlinewidth{0.000000pt}%
\definecolor{currentstroke}{rgb}{0.281924,0.089666,0.412415}%
\pgfsetstrokecolor{currentstroke}%
\pgfsetdash{}{0pt}%
\pgfpathmoveto{\pgfqpoint{4.869240in}{3.622273in}}%
\pgfpathlineto{\pgfqpoint{5.012128in}{3.399559in}}%
\pgfpathlineto{\pgfqpoint{5.092313in}{3.347287in}}%
\pgfpathclose%
\pgfusepath{fill}%
\end{pgfscope}%
\begin{pgfscope}%
\pgfpathrectangle{\pgfqpoint{0.539299in}{0.078740in}}{\pgfqpoint{7.842520in}{7.842520in}}%
\pgfusepath{clip}%
\pgfsetbuttcap%
\pgfsetroundjoin%
\definecolor{currentfill}{rgb}{0.122312,0.633153,0.530398}%
\pgfsetfillcolor{currentfill}%
\pgfsetlinewidth{0.000000pt}%
\definecolor{currentstroke}{rgb}{0.282327,0.094955,0.417331}%
\pgfsetstrokecolor{currentstroke}%
\pgfsetdash{}{0pt}%
\pgfpathmoveto{\pgfqpoint{4.726130in}{3.847156in}}%
\pgfpathlineto{\pgfqpoint{4.807756in}{3.788327in}}%
\pgfpathlineto{\pgfqpoint{4.665156in}{4.013355in}}%
\pgfpathclose%
\pgfusepath{fill}%
\end{pgfscope}%
\begin{pgfscope}%
\pgfpathrectangle{\pgfqpoint{0.539299in}{0.078740in}}{\pgfqpoint{7.842520in}{7.842520in}}%
\pgfusepath{clip}%
\pgfsetbuttcap%
\pgfsetroundjoin%
\definecolor{currentfill}{rgb}{0.270595,0.214069,0.507052}%
\pgfsetfillcolor{currentfill}%
\pgfsetlinewidth{0.000000pt}%
\definecolor{currentstroke}{rgb}{0.282656,0.100196,0.422160}%
\pgfsetstrokecolor{currentstroke}%
\pgfsetdash{}{0pt}%
\pgfpathmoveto{\pgfqpoint{5.931443in}{1.959659in}}%
\pgfpathlineto{\pgfqpoint{5.866377in}{2.159848in}}%
\pgfpathlineto{\pgfqpoint{5.788906in}{2.159810in}}%
\pgfpathclose%
\pgfusepath{fill}%
\end{pgfscope}%
\begin{pgfscope}%
\pgfpathrectangle{\pgfqpoint{0.539299in}{0.078740in}}{\pgfqpoint{7.842520in}{7.842520in}}%
\pgfusepath{clip}%
\pgfsetbuttcap%
\pgfsetroundjoin%
\definecolor{currentfill}{rgb}{0.866013,0.889868,0.095953}%
\pgfsetfillcolor{currentfill}%
\pgfsetlinewidth{0.000000pt}%
\definecolor{currentstroke}{rgb}{0.282910,0.105393,0.426902}%
\pgfsetstrokecolor{currentstroke}%
\pgfsetdash{}{0pt}%
\pgfpathmoveto{\pgfqpoint{3.662699in}{5.404278in}}%
\pgfpathlineto{\pgfqpoint{3.606833in}{5.491393in}}%
\pgfpathlineto{\pgfqpoint{3.520103in}{5.522735in}}%
\pgfpathclose%
\pgfusepath{fill}%
\end{pgfscope}%
\begin{pgfscope}%
\pgfpathrectangle{\pgfqpoint{0.539299in}{0.078740in}}{\pgfqpoint{7.842520in}{7.842520in}}%
\pgfusepath{clip}%
\pgfsetbuttcap%
\pgfsetroundjoin%
\definecolor{currentfill}{rgb}{0.955300,0.901065,0.118128}%
\pgfsetfillcolor{currentfill}%
\pgfsetlinewidth{0.000000pt}%
\definecolor{currentstroke}{rgb}{0.283091,0.110553,0.431554}%
\pgfsetstrokecolor{currentstroke}%
\pgfsetdash{}{0pt}%
\pgfpathmoveto{\pgfqpoint{3.237559in}{5.639984in}}%
\pgfpathlineto{\pgfqpoint{3.150177in}{5.616991in}}%
\pgfpathlineto{\pgfqpoint{3.378279in}{5.603577in}}%
\pgfpathclose%
\pgfusepath{fill}%
\end{pgfscope}%
\begin{pgfscope}%
\pgfpathrectangle{\pgfqpoint{0.539299in}{0.078740in}}{\pgfqpoint{7.842520in}{7.842520in}}%
\pgfusepath{clip}%
\pgfsetbuttcap%
\pgfsetroundjoin%
\definecolor{currentfill}{rgb}{0.896320,0.893616,0.096335}%
\pgfsetfillcolor{currentfill}%
\pgfsetlinewidth{0.000000pt}%
\definecolor{currentstroke}{rgb}{0.283197,0.115680,0.436115}%
\pgfsetstrokecolor{currentstroke}%
\pgfsetdash{}{0pt}%
\pgfpathmoveto{\pgfqpoint{2.786367in}{5.433811in}}%
\pgfpathlineto{\pgfqpoint{2.923598in}{5.523700in}}%
\pgfpathlineto{\pgfqpoint{3.010887in}{5.591305in}}%
\pgfpathclose%
\pgfusepath{fill}%
\end{pgfscope}%
\begin{pgfscope}%
\pgfpathrectangle{\pgfqpoint{0.539299in}{0.078740in}}{\pgfqpoint{7.842520in}{7.842520in}}%
\pgfusepath{clip}%
\pgfsetbuttcap%
\pgfsetroundjoin%
\definecolor{currentfill}{rgb}{0.206756,0.371758,0.553117}%
\pgfsetfillcolor{currentfill}%
\pgfsetlinewidth{0.000000pt}%
\definecolor{currentstroke}{rgb}{0.283229,0.120777,0.440584}%
\pgfsetstrokecolor{currentstroke}%
\pgfsetdash{}{0pt}%
\pgfpathmoveto{\pgfqpoint{5.439710in}{2.755331in}}%
\pgfpathlineto{\pgfqpoint{5.360735in}{2.786461in}}%
\pgfpathlineto{\pgfqpoint{5.581983in}{2.550780in}}%
\pgfpathclose%
\pgfusepath{fill}%
\end{pgfscope}%
\begin{pgfscope}%
\pgfpathrectangle{\pgfqpoint{0.539299in}{0.078740in}}{\pgfqpoint{7.842520in}{7.842520in}}%
\pgfusepath{clip}%
\pgfsetbuttcap%
\pgfsetroundjoin%
\definecolor{currentfill}{rgb}{0.575563,0.844566,0.256415}%
\pgfsetfillcolor{currentfill}%
\pgfsetlinewidth{0.000000pt}%
\definecolor{currentstroke}{rgb}{0.283187,0.125848,0.444960}%
\pgfsetstrokecolor{currentstroke}%
\pgfsetdash{}{0pt}%
\pgfpathmoveto{\pgfqpoint{3.949148in}{5.080344in}}%
\pgfpathlineto{\pgfqpoint{4.092594in}{4.886710in}}%
\pgfpathlineto{\pgfqpoint{4.034345in}{5.020034in}}%
\pgfpathclose%
\pgfusepath{fill}%
\end{pgfscope}%
\begin{pgfscope}%
\pgfpathrectangle{\pgfqpoint{0.539299in}{0.078740in}}{\pgfqpoint{7.842520in}{7.842520in}}%
\pgfusepath{clip}%
\pgfsetbuttcap%
\pgfsetroundjoin%
\definecolor{currentfill}{rgb}{0.647257,0.858400,0.209861}%
\pgfsetfillcolor{currentfill}%
\pgfsetlinewidth{0.000000pt}%
\definecolor{currentstroke}{rgb}{0.283072,0.130895,0.449241}%
\pgfsetstrokecolor{currentstroke}%
\pgfsetdash{}{0pt}%
\pgfpathmoveto{\pgfqpoint{4.034345in}{5.020034in}}%
\pgfpathlineto{\pgfqpoint{3.805790in}{5.254759in}}%
\pgfpathlineto{\pgfqpoint{3.949148in}{5.080344in}}%
\pgfpathclose%
\pgfusepath{fill}%
\end{pgfscope}%
\begin{pgfscope}%
\pgfpathrectangle{\pgfqpoint{0.539299in}{0.078740in}}{\pgfqpoint{7.842520in}{7.842520in}}%
\pgfusepath{clip}%
\pgfsetbuttcap%
\pgfsetroundjoin%
\definecolor{currentfill}{rgb}{0.121148,0.592739,0.544641}%
\pgfsetfillcolor{currentfill}%
\pgfsetlinewidth{0.000000pt}%
\definecolor{currentstroke}{rgb}{0.282884,0.135920,0.453427}%
\pgfsetstrokecolor{currentstroke}%
\pgfsetdash{}{0pt}%
\pgfpathmoveto{\pgfqpoint{4.726130in}{3.847156in}}%
\pgfpathlineto{\pgfqpoint{4.869240in}{3.622273in}}%
\pgfpathlineto{\pgfqpoint{4.950134in}{3.565760in}}%
\pgfpathclose%
\pgfusepath{fill}%
\end{pgfscope}%
\begin{pgfscope}%
\pgfpathrectangle{\pgfqpoint{0.539299in}{0.078740in}}{\pgfqpoint{7.842520in}{7.842520in}}%
\pgfusepath{clip}%
\pgfsetbuttcap%
\pgfsetroundjoin%
\definecolor{currentfill}{rgb}{0.151918,0.500685,0.557587}%
\pgfsetfillcolor{currentfill}%
\pgfsetlinewidth{0.000000pt}%
\definecolor{currentstroke}{rgb}{0.282623,0.140926,0.457517}%
\pgfsetstrokecolor{currentstroke}%
\pgfsetdash{}{0pt}%
\pgfpathmoveto{\pgfqpoint{2.341276in}{3.620065in}}%
\pgfpathlineto{\pgfqpoint{2.129568in}{3.063212in}}%
\pgfpathlineto{\pgfqpoint{2.266737in}{3.071192in}}%
\pgfpathclose%
\pgfusepath{fill}%
\end{pgfscope}%
\begin{pgfscope}%
\pgfpathrectangle{\pgfqpoint{0.539299in}{0.078740in}}{\pgfqpoint{7.842520in}{7.842520in}}%
\pgfusepath{clip}%
\pgfsetbuttcap%
\pgfsetroundjoin%
\definecolor{currentfill}{rgb}{0.751884,0.874951,0.143228}%
\pgfsetfillcolor{currentfill}%
\pgfsetlinewidth{0.000000pt}%
\definecolor{currentstroke}{rgb}{0.282290,0.145912,0.461510}%
\pgfsetstrokecolor{currentstroke}%
\pgfsetdash{}{0pt}%
\pgfpathmoveto{\pgfqpoint{3.805790in}{5.254759in}}%
\pgfpathlineto{\pgfqpoint{3.891582in}{5.202680in}}%
\pgfpathlineto{\pgfqpoint{3.662699in}{5.404278in}}%
\pgfpathclose%
\pgfusepath{fill}%
\end{pgfscope}%
\begin{pgfscope}%
\pgfpathrectangle{\pgfqpoint{0.539299in}{0.078740in}}{\pgfqpoint{7.842520in}{7.842520in}}%
\pgfusepath{clip}%
\pgfsetbuttcap%
\pgfsetroundjoin%
\definecolor{currentfill}{rgb}{0.250425,0.274290,0.533103}%
\pgfsetfillcolor{currentfill}%
\pgfsetlinewidth{0.000000pt}%
\definecolor{currentstroke}{rgb}{0.281887,0.150881,0.465405}%
\pgfsetstrokecolor{currentstroke}%
\pgfsetdash{}{0pt}%
\pgfpathmoveto{\pgfqpoint{5.724190in}{2.352097in}}%
\pgfpathlineto{\pgfqpoint{5.646299in}{2.364748in}}%
\pgfpathlineto{\pgfqpoint{5.788906in}{2.159810in}}%
\pgfpathclose%
\pgfusepath{fill}%
\end{pgfscope}%
\begin{pgfscope}%
\pgfpathrectangle{\pgfqpoint{0.539299in}{0.078740in}}{\pgfqpoint{7.842520in}{7.842520in}}%
\pgfusepath{clip}%
\pgfsetbuttcap%
\pgfsetroundjoin%
\definecolor{currentfill}{rgb}{0.190631,0.407061,0.556089}%
\pgfsetfillcolor{currentfill}%
\pgfsetlinewidth{0.000000pt}%
\definecolor{currentstroke}{rgb}{0.281412,0.155834,0.469201}%
\pgfsetstrokecolor{currentstroke}%
\pgfsetdash{}{0pt}%
\pgfpathmoveto{\pgfqpoint{5.297333in}{2.965317in}}%
\pgfpathlineto{\pgfqpoint{5.360735in}{2.786461in}}%
\pgfpathlineto{\pgfqpoint{5.439710in}{2.755331in}}%
\pgfpathclose%
\pgfusepath{fill}%
\end{pgfscope}%
\begin{pgfscope}%
\pgfpathrectangle{\pgfqpoint{0.539299in}{0.078740in}}{\pgfqpoint{7.842520in}{7.842520in}}%
\pgfusepath{clip}%
\pgfsetbuttcap%
\pgfsetroundjoin%
\definecolor{currentfill}{rgb}{0.845561,0.887322,0.099702}%
\pgfsetfillcolor{currentfill}%
\pgfsetlinewidth{0.000000pt}%
\definecolor{currentstroke}{rgb}{0.280868,0.160771,0.472899}%
\pgfsetstrokecolor{currentstroke}%
\pgfsetdash{}{0pt}%
\pgfpathmoveto{\pgfqpoint{2.923598in}{5.523700in}}%
\pgfpathlineto{\pgfqpoint{2.786367in}{5.433811in}}%
\pgfpathlineto{\pgfqpoint{2.699533in}{5.323077in}}%
\pgfpathclose%
\pgfusepath{fill}%
\end{pgfscope}%
\begin{pgfscope}%
\pgfpathrectangle{\pgfqpoint{0.539299in}{0.078740in}}{\pgfqpoint{7.842520in}{7.842520in}}%
\pgfusepath{clip}%
\pgfsetbuttcap%
\pgfsetroundjoin%
\definecolor{currentfill}{rgb}{0.136408,0.541173,0.554483}%
\pgfsetfillcolor{currentfill}%
\pgfsetlinewidth{0.000000pt}%
\definecolor{currentstroke}{rgb}{0.280255,0.165693,0.476498}%
\pgfsetstrokecolor{currentstroke}%
\pgfsetdash{}{0pt}%
\pgfpathmoveto{\pgfqpoint{2.204165in}{3.588171in}}%
\pgfpathlineto{\pgfqpoint{2.129568in}{3.063212in}}%
\pgfpathlineto{\pgfqpoint{2.341276in}{3.620065in}}%
\pgfpathclose%
\pgfusepath{fill}%
\end{pgfscope}%
\begin{pgfscope}%
\pgfpathrectangle{\pgfqpoint{0.539299in}{0.078740in}}{\pgfqpoint{7.842520in}{7.842520in}}%
\pgfusepath{clip}%
\pgfsetbuttcap%
\pgfsetroundjoin%
\definecolor{currentfill}{rgb}{0.185783,0.704891,0.485273}%
\pgfsetfillcolor{currentfill}%
\pgfsetlinewidth{0.000000pt}%
\definecolor{currentstroke}{rgb}{0.279574,0.170599,0.479997}%
\pgfsetstrokecolor{currentstroke}%
\pgfsetdash{}{0pt}%
\pgfpathmoveto{\pgfqpoint{4.665156in}{4.013355in}}%
\pgfpathlineto{\pgfqpoint{4.522320in}{4.238703in}}%
\pgfpathlineto{\pgfqpoint{4.439213in}{4.296234in}}%
\pgfpathclose%
\pgfusepath{fill}%
\end{pgfscope}%
\begin{pgfscope}%
\pgfpathrectangle{\pgfqpoint{0.539299in}{0.078740in}}{\pgfqpoint{7.842520in}{7.842520in}}%
\pgfusepath{clip}%
\pgfsetbuttcap%
\pgfsetroundjoin%
\definecolor{currentfill}{rgb}{0.688944,0.865448,0.182725}%
\pgfsetfillcolor{currentfill}%
\pgfsetlinewidth{0.000000pt}%
\definecolor{currentstroke}{rgb}{0.278826,0.175490,0.483397}%
\pgfsetstrokecolor{currentstroke}%
\pgfsetdash{}{0pt}%
\pgfpathmoveto{\pgfqpoint{2.699533in}{5.323077in}}%
\pgfpathlineto{\pgfqpoint{2.478789in}{5.021565in}}%
\pgfpathlineto{\pgfqpoint{2.613359in}{5.171039in}}%
\pgfpathclose%
\pgfusepath{fill}%
\end{pgfscope}%
\begin{pgfscope}%
\pgfpathrectangle{\pgfqpoint{0.539299in}{0.078740in}}{\pgfqpoint{7.842520in}{7.842520in}}%
\pgfusepath{clip}%
\pgfsetbuttcap%
\pgfsetroundjoin%
\definecolor{currentfill}{rgb}{0.545524,0.838039,0.275626}%
\pgfsetfillcolor{currentfill}%
\pgfsetlinewidth{0.000000pt}%
\definecolor{currentstroke}{rgb}{0.278012,0.180367,0.486697}%
\pgfsetstrokecolor{currentstroke}%
\pgfsetdash{}{0pt}%
\pgfpathmoveto{\pgfqpoint{2.393562in}{4.830606in}}%
\pgfpathlineto{\pgfqpoint{2.528143in}{4.972174in}}%
\pgfpathlineto{\pgfqpoint{2.478789in}{5.021565in}}%
\pgfpathclose%
\pgfusepath{fill}%
\end{pgfscope}%
\begin{pgfscope}%
\pgfpathrectangle{\pgfqpoint{0.539299in}{0.078740in}}{\pgfqpoint{7.842520in}{7.842520in}}%
\pgfusepath{clip}%
\pgfsetbuttcap%
\pgfsetroundjoin%
\definecolor{currentfill}{rgb}{0.225863,0.330805,0.547314}%
\pgfsetfillcolor{currentfill}%
\pgfsetlinewidth{0.000000pt}%
\definecolor{currentstroke}{rgb}{0.277134,0.185228,0.489898}%
\pgfsetstrokecolor{currentstroke}%
\pgfsetdash{}{0pt}%
\pgfpathmoveto{\pgfqpoint{5.503585in}{2.573766in}}%
\pgfpathlineto{\pgfqpoint{5.646299in}{2.364748in}}%
\pgfpathlineto{\pgfqpoint{5.581983in}{2.550780in}}%
\pgfpathclose%
\pgfusepath{fill}%
\end{pgfscope}%
\begin{pgfscope}%
\pgfpathrectangle{\pgfqpoint{0.539299in}{0.078740in}}{\pgfqpoint{7.842520in}{7.842520in}}%
\pgfusepath{clip}%
\pgfsetbuttcap%
\pgfsetroundjoin%
\definecolor{currentfill}{rgb}{0.162016,0.687316,0.499129}%
\pgfsetfillcolor{currentfill}%
\pgfsetlinewidth{0.000000pt}%
\definecolor{currentstroke}{rgb}{0.276194,0.190074,0.493001}%
\pgfsetstrokecolor{currentstroke}%
\pgfsetdash{}{0pt}%
\pgfpathmoveto{\pgfqpoint{2.227125in}{4.300185in}}%
\pgfpathlineto{\pgfqpoint{2.146689in}{3.948369in}}%
\pgfpathlineto{\pgfqpoint{2.281795in}{4.035068in}}%
\pgfpathclose%
\pgfusepath{fill}%
\end{pgfscope}%
\begin{pgfscope}%
\pgfpathrectangle{\pgfqpoint{0.539299in}{0.078740in}}{\pgfqpoint{7.842520in}{7.842520in}}%
\pgfusepath{clip}%
\pgfsetbuttcap%
\pgfsetroundjoin%
\definecolor{currentfill}{rgb}{0.945636,0.899815,0.112838}%
\pgfsetfillcolor{currentfill}%
\pgfsetlinewidth{0.000000pt}%
\definecolor{currentstroke}{rgb}{0.275191,0.194905,0.496005}%
\pgfsetstrokecolor{currentstroke}%
\pgfsetdash{}{0pt}%
\pgfpathmoveto{\pgfqpoint{3.010887in}{5.591305in}}%
\pgfpathlineto{\pgfqpoint{2.923598in}{5.523700in}}%
\pgfpathlineto{\pgfqpoint{3.150177in}{5.616991in}}%
\pgfpathclose%
\pgfusepath{fill}%
\end{pgfscope}%
\begin{pgfscope}%
\pgfpathrectangle{\pgfqpoint{0.539299in}{0.078740in}}{\pgfqpoint{7.842520in}{7.842520in}}%
\pgfusepath{clip}%
\pgfsetbuttcap%
\pgfsetroundjoin%
\definecolor{currentfill}{rgb}{0.123444,0.636809,0.528763}%
\pgfsetfillcolor{currentfill}%
\pgfsetlinewidth{0.000000pt}%
\definecolor{currentstroke}{rgb}{0.274128,0.199721,0.498911}%
\pgfsetstrokecolor{currentstroke}%
\pgfsetdash{}{0pt}%
\pgfpathmoveto{\pgfqpoint{2.204165in}{3.588171in}}%
\pgfpathlineto{\pgfqpoint{2.281795in}{4.035068in}}%
\pgfpathlineto{\pgfqpoint{2.146689in}{3.948369in}}%
\pgfpathclose%
\pgfusepath{fill}%
\end{pgfscope}%
\begin{pgfscope}%
\pgfpathrectangle{\pgfqpoint{0.539299in}{0.078740in}}{\pgfqpoint{7.842520in}{7.842520in}}%
\pgfusepath{clip}%
\pgfsetbuttcap%
\pgfsetroundjoin%
\definecolor{currentfill}{rgb}{0.246070,0.738910,0.452024}%
\pgfsetfillcolor{currentfill}%
\pgfsetlinewidth{0.000000pt}%
\definecolor{currentstroke}{rgb}{0.273006,0.204520,0.501721}%
\pgfsetstrokecolor{currentstroke}%
\pgfsetdash{}{0pt}%
\pgfpathmoveto{\pgfqpoint{4.522320in}{4.238703in}}%
\pgfpathlineto{\pgfqpoint{4.379257in}{4.461675in}}%
\pgfpathlineto{\pgfqpoint{4.439213in}{4.296234in}}%
\pgfpathclose%
\pgfusepath{fill}%
\end{pgfscope}%
\begin{pgfscope}%
\pgfpathrectangle{\pgfqpoint{0.539299in}{0.078740in}}{\pgfqpoint{7.842520in}{7.842520in}}%
\pgfusepath{clip}%
\pgfsetbuttcap%
\pgfsetroundjoin%
\definecolor{currentfill}{rgb}{0.208623,0.367752,0.552675}%
\pgfsetfillcolor{currentfill}%
\pgfsetlinewidth{0.000000pt}%
\definecolor{currentstroke}{rgb}{0.271828,0.209303,0.504434}%
\pgfsetstrokecolor{currentstroke}%
\pgfsetdash{}{0pt}%
\pgfpathmoveto{\pgfqpoint{5.581983in}{2.550780in}}%
\pgfpathlineto{\pgfqpoint{5.360735in}{2.786461in}}%
\pgfpathlineto{\pgfqpoint{5.503585in}{2.573766in}}%
\pgfpathclose%
\pgfusepath{fill}%
\end{pgfscope}%
\begin{pgfscope}%
\pgfpathrectangle{\pgfqpoint{0.539299in}{0.078740in}}{\pgfqpoint{7.842520in}{7.842520in}}%
\pgfusepath{clip}%
\pgfsetbuttcap%
\pgfsetroundjoin%
\definecolor{currentfill}{rgb}{0.160665,0.478540,0.558115}%
\pgfsetfillcolor{currentfill}%
\pgfsetlinewidth{0.000000pt}%
\definecolor{currentstroke}{rgb}{0.270595,0.214069,0.507052}%
\pgfsetstrokecolor{currentstroke}%
\pgfsetdash{}{0pt}%
\pgfpathmoveto{\pgfqpoint{5.154816in}{3.180272in}}%
\pgfpathlineto{\pgfqpoint{5.074536in}{3.221540in}}%
\pgfpathlineto{\pgfqpoint{5.297333in}{2.965317in}}%
\pgfpathclose%
\pgfusepath{fill}%
\end{pgfscope}%
\begin{pgfscope}%
\pgfpathrectangle{\pgfqpoint{0.539299in}{0.078740in}}{\pgfqpoint{7.842520in}{7.842520in}}%
\pgfusepath{clip}%
\pgfsetbuttcap%
\pgfsetroundjoin%
\definecolor{currentfill}{rgb}{0.281924,0.089666,0.412415}%
\pgfsetfillcolor{currentfill}%
\pgfsetlinewidth{0.000000pt}%
\definecolor{currentstroke}{rgb}{0.269308,0.218818,0.509577}%
\pgfsetstrokecolor{currentstroke}%
\pgfsetdash{}{0pt}%
\pgfpathmoveto{\pgfqpoint{6.073966in}{1.765621in}}%
\pgfpathlineto{\pgfqpoint{6.139221in}{1.529769in}}%
\pgfpathlineto{\pgfqpoint{6.216568in}{1.580071in}}%
\pgfpathclose%
\pgfusepath{fill}%
\end{pgfscope}%
\begin{pgfscope}%
\pgfpathrectangle{\pgfqpoint{0.539299in}{0.078740in}}{\pgfqpoint{7.842520in}{7.842520in}}%
\pgfusepath{clip}%
\pgfsetbuttcap%
\pgfsetroundjoin%
\definecolor{currentfill}{rgb}{0.140210,0.665859,0.513427}%
\pgfsetfillcolor{currentfill}%
\pgfsetlinewidth{0.000000pt}%
\definecolor{currentstroke}{rgb}{0.267968,0.223549,0.512008}%
\pgfsetstrokecolor{currentstroke}%
\pgfsetdash{}{0pt}%
\pgfpathmoveto{\pgfqpoint{4.665156in}{4.013355in}}%
\pgfpathlineto{\pgfqpoint{4.582787in}{4.072527in}}%
\pgfpathlineto{\pgfqpoint{4.726130in}{3.847156in}}%
\pgfpathclose%
\pgfusepath{fill}%
\end{pgfscope}%
\begin{pgfscope}%
\pgfpathrectangle{\pgfqpoint{0.539299in}{0.078740in}}{\pgfqpoint{7.842520in}{7.842520in}}%
\pgfusepath{clip}%
\pgfsetbuttcap%
\pgfsetroundjoin%
\definecolor{currentfill}{rgb}{0.430983,0.808473,0.346476}%
\pgfsetfillcolor{currentfill}%
\pgfsetlinewidth{0.000000pt}%
\definecolor{currentstroke}{rgb}{0.266580,0.228262,0.514349}%
\pgfsetstrokecolor{currentstroke}%
\pgfsetdash{}{0pt}%
\pgfpathmoveto{\pgfqpoint{2.393562in}{4.830606in}}%
\pgfpathlineto{\pgfqpoint{2.309553in}{4.592125in}}%
\pgfpathlineto{\pgfqpoint{2.444220in}{4.720662in}}%
\pgfpathclose%
\pgfusepath{fill}%
\end{pgfscope}%
\begin{pgfscope}%
\pgfpathrectangle{\pgfqpoint{0.539299in}{0.078740in}}{\pgfqpoint{7.842520in}{7.842520in}}%
\pgfusepath{clip}%
\pgfsetbuttcap%
\pgfsetroundjoin%
\definecolor{currentfill}{rgb}{0.187231,0.414746,0.556547}%
\pgfsetfillcolor{currentfill}%
\pgfsetlinewidth{0.000000pt}%
\definecolor{currentstroke}{rgb}{0.265145,0.232956,0.516599}%
\pgfsetstrokecolor{currentstroke}%
\pgfsetdash{}{0pt}%
\pgfpathmoveto{\pgfqpoint{2.405032in}{3.063501in}}%
\pgfpathlineto{\pgfqpoint{2.472446in}{2.377069in}}%
\pgfpathlineto{\pgfqpoint{2.544311in}{3.042210in}}%
\pgfpathclose%
\pgfusepath{fill}%
\end{pgfscope}%
\begin{pgfscope}%
\pgfpathrectangle{\pgfqpoint{0.539299in}{0.078740in}}{\pgfqpoint{7.842520in}{7.842520in}}%
\pgfusepath{clip}%
\pgfsetbuttcap%
\pgfsetroundjoin%
\definecolor{currentfill}{rgb}{0.964894,0.902323,0.123941}%
\pgfsetfillcolor{currentfill}%
\pgfsetlinewidth{0.000000pt}%
\definecolor{currentstroke}{rgb}{0.263663,0.237631,0.518762}%
\pgfsetstrokecolor{currentstroke}%
\pgfsetdash{}{0pt}%
\pgfpathmoveto{\pgfqpoint{3.378279in}{5.603577in}}%
\pgfpathlineto{\pgfqpoint{3.150177in}{5.616991in}}%
\pgfpathlineto{\pgfqpoint{3.291045in}{5.591989in}}%
\pgfpathclose%
\pgfusepath{fill}%
\end{pgfscope}%
\begin{pgfscope}%
\pgfpathrectangle{\pgfqpoint{0.539299in}{0.078740in}}{\pgfqpoint{7.842520in}{7.842520in}}%
\pgfusepath{clip}%
\pgfsetbuttcap%
\pgfsetroundjoin%
\definecolor{currentfill}{rgb}{0.214298,0.355619,0.551184}%
\pgfsetfillcolor{currentfill}%
\pgfsetlinewidth{0.000000pt}%
\definecolor{currentstroke}{rgb}{0.262138,0.242286,0.520837}%
\pgfsetstrokecolor{currentstroke}%
\pgfsetdash{}{0pt}%
\pgfpathmoveto{\pgfqpoint{2.472446in}{2.377069in}}%
\pgfpathlineto{\pgfqpoint{2.611847in}{2.338243in}}%
\pgfpathlineto{\pgfqpoint{2.684453in}{3.009085in}}%
\pgfpathclose%
\pgfusepath{fill}%
\end{pgfscope}%
\begin{pgfscope}%
\pgfpathrectangle{\pgfqpoint{0.539299in}{0.078740in}}{\pgfqpoint{7.842520in}{7.842520in}}%
\pgfusepath{clip}%
\pgfsetbuttcap%
\pgfsetroundjoin%
\definecolor{currentfill}{rgb}{0.636902,0.856542,0.216620}%
\pgfsetfillcolor{currentfill}%
\pgfsetlinewidth{0.000000pt}%
\definecolor{currentstroke}{rgb}{0.260571,0.246922,0.522828}%
\pgfsetstrokecolor{currentstroke}%
\pgfsetdash{}{0pt}%
\pgfpathmoveto{\pgfqpoint{2.478789in}{5.021565in}}%
\pgfpathlineto{\pgfqpoint{2.528143in}{4.972174in}}%
\pgfpathlineto{\pgfqpoint{2.613359in}{5.171039in}}%
\pgfpathclose%
\pgfusepath{fill}%
\end{pgfscope}%
\begin{pgfscope}%
\pgfpathrectangle{\pgfqpoint{0.539299in}{0.078740in}}{\pgfqpoint{7.842520in}{7.842520in}}%
\pgfusepath{clip}%
\pgfsetbuttcap%
\pgfsetroundjoin%
\definecolor{currentfill}{rgb}{0.149039,0.508051,0.557250}%
\pgfsetfillcolor{currentfill}%
\pgfsetlinewidth{0.000000pt}%
\definecolor{currentstroke}{rgb}{0.258965,0.251537,0.524736}%
\pgfsetstrokecolor{currentstroke}%
\pgfsetdash{}{0pt}%
\pgfpathmoveto{\pgfqpoint{2.405032in}{3.063501in}}%
\pgfpathlineto{\pgfqpoint{2.341276in}{3.620065in}}%
\pgfpathlineto{\pgfqpoint{2.266737in}{3.071192in}}%
\pgfpathclose%
\pgfusepath{fill}%
\end{pgfscope}%
\begin{pgfscope}%
\pgfpathrectangle{\pgfqpoint{0.539299in}{0.078740in}}{\pgfqpoint{7.842520in}{7.842520in}}%
\pgfusepath{clip}%
\pgfsetbuttcap%
\pgfsetroundjoin%
\definecolor{currentfill}{rgb}{0.344074,0.780029,0.397381}%
\pgfsetfillcolor{currentfill}%
\pgfsetlinewidth{0.000000pt}%
\definecolor{currentstroke}{rgb}{0.257322,0.256130,0.526563}%
\pgfsetstrokecolor{currentstroke}%
\pgfsetdash{}{0pt}%
\pgfpathmoveto{\pgfqpoint{2.227125in}{4.300185in}}%
\pgfpathlineto{\pgfqpoint{2.444220in}{4.720662in}}%
\pgfpathlineto{\pgfqpoint{2.309553in}{4.592125in}}%
\pgfpathclose%
\pgfusepath{fill}%
\end{pgfscope}%
\begin{pgfscope}%
\pgfpathrectangle{\pgfqpoint{0.539299in}{0.078740in}}{\pgfqpoint{7.842520in}{7.842520in}}%
\pgfusepath{clip}%
\pgfsetbuttcap%
\pgfsetroundjoin%
\definecolor{currentfill}{rgb}{0.175707,0.697900,0.491033}%
\pgfsetfillcolor{currentfill}%
\pgfsetlinewidth{0.000000pt}%
\definecolor{currentstroke}{rgb}{0.255645,0.260703,0.528312}%
\pgfsetstrokecolor{currentstroke}%
\pgfsetdash{}{0pt}%
\pgfpathmoveto{\pgfqpoint{4.439213in}{4.296234in}}%
\pgfpathlineto{\pgfqpoint{4.582787in}{4.072527in}}%
\pgfpathlineto{\pgfqpoint{4.665156in}{4.013355in}}%
\pgfpathclose%
\pgfusepath{fill}%
\end{pgfscope}%
\begin{pgfscope}%
\pgfpathrectangle{\pgfqpoint{0.539299in}{0.078740in}}{\pgfqpoint{7.842520in}{7.842520in}}%
\pgfusepath{clip}%
\pgfsetbuttcap%
\pgfsetroundjoin%
\definecolor{currentfill}{rgb}{0.935904,0.898570,0.108131}%
\pgfsetfillcolor{currentfill}%
\pgfsetlinewidth{0.000000pt}%
\definecolor{currentstroke}{rgb}{0.253935,0.265254,0.529983}%
\pgfsetstrokecolor{currentstroke}%
\pgfsetdash{}{0pt}%
\pgfpathmoveto{\pgfqpoint{3.433110in}{5.522817in}}%
\pgfpathlineto{\pgfqpoint{3.520103in}{5.522735in}}%
\pgfpathlineto{\pgfqpoint{3.378279in}{5.603577in}}%
\pgfpathclose%
\pgfusepath{fill}%
\end{pgfscope}%
\begin{pgfscope}%
\pgfpathrectangle{\pgfqpoint{0.539299in}{0.078740in}}{\pgfqpoint{7.842520in}{7.842520in}}%
\pgfusepath{clip}%
\pgfsetbuttcap%
\pgfsetroundjoin%
\definecolor{currentfill}{rgb}{0.278826,0.175490,0.483397}%
\pgfsetfillcolor{currentfill}%
\pgfsetlinewidth{0.000000pt}%
\definecolor{currentstroke}{rgb}{0.252194,0.269783,0.531579}%
\pgfsetstrokecolor{currentstroke}%
\pgfsetdash{}{0pt}%
\pgfpathmoveto{\pgfqpoint{6.073966in}{1.765621in}}%
\pgfpathlineto{\pgfqpoint{5.931443in}{1.959659in}}%
\pgfpathlineto{\pgfqpoint{5.853809in}{1.950429in}}%
\pgfpathclose%
\pgfusepath{fill}%
\end{pgfscope}%
\begin{pgfscope}%
\pgfpathrectangle{\pgfqpoint{0.539299in}{0.078740in}}{\pgfqpoint{7.842520in}{7.842520in}}%
\pgfusepath{clip}%
\pgfsetbuttcap%
\pgfsetroundjoin%
\definecolor{currentfill}{rgb}{0.140536,0.530132,0.555659}%
\pgfsetfillcolor{currentfill}%
\pgfsetlinewidth{0.000000pt}%
\definecolor{currentstroke}{rgb}{0.250425,0.274290,0.533103}%
\pgfsetstrokecolor{currentstroke}%
\pgfsetdash{}{0pt}%
\pgfpathmoveto{\pgfqpoint{4.931146in}{3.442939in}}%
\pgfpathlineto{\pgfqpoint{5.154816in}{3.180272in}}%
\pgfpathlineto{\pgfqpoint{5.012128in}{3.399559in}}%
\pgfpathclose%
\pgfusepath{fill}%
\end{pgfscope}%
\begin{pgfscope}%
\pgfpathrectangle{\pgfqpoint{0.539299in}{0.078740in}}{\pgfqpoint{7.842520in}{7.842520in}}%
\pgfusepath{clip}%
\pgfsetbuttcap%
\pgfsetroundjoin%
\definecolor{currentfill}{rgb}{0.386433,0.794644,0.372886}%
\pgfsetfillcolor{currentfill}%
\pgfsetlinewidth{0.000000pt}%
\definecolor{currentstroke}{rgb}{0.248629,0.278775,0.534556}%
\pgfsetstrokecolor{currentstroke}%
\pgfsetdash{}{0pt}%
\pgfpathmoveto{\pgfqpoint{4.235995in}{4.678979in}}%
\pgfpathlineto{\pgfqpoint{4.151482in}{4.727445in}}%
\pgfpathlineto{\pgfqpoint{4.379257in}{4.461675in}}%
\pgfpathclose%
\pgfusepath{fill}%
\end{pgfscope}%
\begin{pgfscope}%
\pgfpathrectangle{\pgfqpoint{0.539299in}{0.078740in}}{\pgfqpoint{7.842520in}{7.842520in}}%
\pgfusepath{clip}%
\pgfsetbuttcap%
\pgfsetroundjoin%
\definecolor{currentfill}{rgb}{0.177423,0.437527,0.557565}%
\pgfsetfillcolor{currentfill}%
\pgfsetlinewidth{0.000000pt}%
\definecolor{currentstroke}{rgb}{0.246811,0.283237,0.535941}%
\pgfsetstrokecolor{currentstroke}%
\pgfsetdash{}{0pt}%
\pgfpathmoveto{\pgfqpoint{5.217726in}{3.002518in}}%
\pgfpathlineto{\pgfqpoint{5.360735in}{2.786461in}}%
\pgfpathlineto{\pgfqpoint{5.297333in}{2.965317in}}%
\pgfpathclose%
\pgfusepath{fill}%
\end{pgfscope}%
\begin{pgfscope}%
\pgfpathrectangle{\pgfqpoint{0.539299in}{0.078740in}}{\pgfqpoint{7.842520in}{7.842520in}}%
\pgfusepath{clip}%
\pgfsetbuttcap%
\pgfsetroundjoin%
\definecolor{currentfill}{rgb}{0.120081,0.622161,0.534946}%
\pgfsetfillcolor{currentfill}%
\pgfsetlinewidth{0.000000pt}%
\definecolor{currentstroke}{rgb}{0.244972,0.287675,0.537260}%
\pgfsetstrokecolor{currentstroke}%
\pgfsetdash{}{0pt}%
\pgfpathmoveto{\pgfqpoint{2.204165in}{3.588171in}}%
\pgfpathlineto{\pgfqpoint{2.341276in}{3.620065in}}%
\pgfpathlineto{\pgfqpoint{2.281795in}{4.035068in}}%
\pgfpathclose%
\pgfusepath{fill}%
\end{pgfscope}%
\begin{pgfscope}%
\pgfpathrectangle{\pgfqpoint{0.539299in}{0.078740in}}{\pgfqpoint{7.842520in}{7.842520in}}%
\pgfusepath{clip}%
\pgfsetbuttcap%
\pgfsetroundjoin%
\definecolor{currentfill}{rgb}{0.265145,0.232956,0.516599}%
\pgfsetfillcolor{currentfill}%
\pgfsetlinewidth{0.000000pt}%
\definecolor{currentstroke}{rgb}{0.243113,0.292092,0.538516}%
\pgfsetstrokecolor{currentstroke}%
\pgfsetdash{}{0pt}%
\pgfpathmoveto{\pgfqpoint{5.788906in}{2.159810in}}%
\pgfpathlineto{\pgfqpoint{5.710883in}{2.164223in}}%
\pgfpathlineto{\pgfqpoint{5.931443in}{1.959659in}}%
\pgfpathclose%
\pgfusepath{fill}%
\end{pgfscope}%
\begin{pgfscope}%
\pgfpathrectangle{\pgfqpoint{0.539299in}{0.078740in}}{\pgfqpoint{7.842520in}{7.842520in}}%
\pgfusepath{clip}%
\pgfsetbuttcap%
\pgfsetroundjoin%
\definecolor{currentfill}{rgb}{0.866013,0.889868,0.095953}%
\pgfsetfillcolor{currentfill}%
\pgfsetlinewidth{0.000000pt}%
\definecolor{currentstroke}{rgb}{0.241237,0.296485,0.539709}%
\pgfsetstrokecolor{currentstroke}%
\pgfsetdash{}{0pt}%
\pgfpathmoveto{\pgfqpoint{2.836696in}{5.416830in}}%
\pgfpathlineto{\pgfqpoint{2.923598in}{5.523700in}}%
\pgfpathlineto{\pgfqpoint{2.699533in}{5.323077in}}%
\pgfpathclose%
\pgfusepath{fill}%
\end{pgfscope}%
\begin{pgfscope}%
\pgfpathrectangle{\pgfqpoint{0.539299in}{0.078740in}}{\pgfqpoint{7.842520in}{7.842520in}}%
\pgfusepath{clip}%
\pgfsetbuttcap%
\pgfsetroundjoin%
\definecolor{currentfill}{rgb}{0.226397,0.728888,0.462789}%
\pgfsetfillcolor{currentfill}%
\pgfsetlinewidth{0.000000pt}%
\definecolor{currentstroke}{rgb}{0.239346,0.300855,0.540844}%
\pgfsetstrokecolor{currentstroke}%
\pgfsetdash{}{0pt}%
\pgfpathmoveto{\pgfqpoint{2.281795in}{4.035068in}}%
\pgfpathlineto{\pgfqpoint{2.361965in}{4.410404in}}%
\pgfpathlineto{\pgfqpoint{2.227125in}{4.300185in}}%
\pgfpathclose%
\pgfusepath{fill}%
\end{pgfscope}%
\begin{pgfscope}%
\pgfpathrectangle{\pgfqpoint{0.539299in}{0.078740in}}{\pgfqpoint{7.842520in}{7.842520in}}%
\pgfusepath{clip}%
\pgfsetbuttcap%
\pgfsetroundjoin%
\definecolor{currentfill}{rgb}{0.906311,0.894855,0.098125}%
\pgfsetfillcolor{currentfill}%
\pgfsetlinewidth{0.000000pt}%
\definecolor{currentstroke}{rgb}{0.237441,0.305202,0.541921}%
\pgfsetstrokecolor{currentstroke}%
\pgfsetdash{}{0pt}%
\pgfpathmoveto{\pgfqpoint{3.662699in}{5.404278in}}%
\pgfpathlineto{\pgfqpoint{3.520103in}{5.522735in}}%
\pgfpathlineto{\pgfqpoint{3.433110in}{5.522817in}}%
\pgfpathclose%
\pgfusepath{fill}%
\end{pgfscope}%
\begin{pgfscope}%
\pgfpathrectangle{\pgfqpoint{0.539299in}{0.078740in}}{\pgfqpoint{7.842520in}{7.842520in}}%
\pgfusepath{clip}%
\pgfsetbuttcap%
\pgfsetroundjoin%
\definecolor{currentfill}{rgb}{0.128729,0.563265,0.551229}%
\pgfsetfillcolor{currentfill}%
\pgfsetlinewidth{0.000000pt}%
\definecolor{currentstroke}{rgb}{0.235526,0.309527,0.542944}%
\pgfsetstrokecolor{currentstroke}%
\pgfsetdash{}{0pt}%
\pgfpathmoveto{\pgfqpoint{5.012128in}{3.399559in}}%
\pgfpathlineto{\pgfqpoint{4.869240in}{3.622273in}}%
\pgfpathlineto{\pgfqpoint{4.931146in}{3.442939in}}%
\pgfpathclose%
\pgfusepath{fill}%
\end{pgfscope}%
\begin{pgfscope}%
\pgfpathrectangle{\pgfqpoint{0.539299in}{0.078740in}}{\pgfqpoint{7.842520in}{7.842520in}}%
\pgfusepath{clip}%
\pgfsetbuttcap%
\pgfsetroundjoin%
\definecolor{currentfill}{rgb}{0.163625,0.471133,0.558148}%
\pgfsetfillcolor{currentfill}%
\pgfsetlinewidth{0.000000pt}%
\definecolor{currentstroke}{rgb}{0.233603,0.313828,0.543914}%
\pgfsetstrokecolor{currentstroke}%
\pgfsetdash{}{0pt}%
\pgfpathmoveto{\pgfqpoint{5.297333in}{2.965317in}}%
\pgfpathlineto{\pgfqpoint{5.074536in}{3.221540in}}%
\pgfpathlineto{\pgfqpoint{5.217726in}{3.002518in}}%
\pgfpathclose%
\pgfusepath{fill}%
\end{pgfscope}%
\begin{pgfscope}%
\pgfpathrectangle{\pgfqpoint{0.539299in}{0.078740in}}{\pgfqpoint{7.842520in}{7.842520in}}%
\pgfusepath{clip}%
\pgfsetbuttcap%
\pgfsetroundjoin%
\definecolor{currentfill}{rgb}{0.283197,0.115680,0.436115}%
\pgfsetfillcolor{currentfill}%
\pgfsetlinewidth{0.000000pt}%
\definecolor{currentstroke}{rgb}{0.231674,0.318106,0.544834}%
\pgfsetstrokecolor{currentstroke}%
\pgfsetdash{}{0pt}%
\pgfpathmoveto{\pgfqpoint{6.073966in}{1.765621in}}%
\pgfpathlineto{\pgfqpoint{5.996574in}{1.738442in}}%
\pgfpathlineto{\pgfqpoint{6.139221in}{1.529769in}}%
\pgfpathclose%
\pgfusepath{fill}%
\end{pgfscope}%
\begin{pgfscope}%
\pgfpathrectangle{\pgfqpoint{0.539299in}{0.078740in}}{\pgfqpoint{7.842520in}{7.842520in}}%
\pgfusepath{clip}%
\pgfsetbuttcap%
\pgfsetroundjoin%
\definecolor{currentfill}{rgb}{0.506271,0.828786,0.300362}%
\pgfsetfillcolor{currentfill}%
\pgfsetlinewidth{0.000000pt}%
\definecolor{currentstroke}{rgb}{0.229739,0.322361,0.545706}%
\pgfsetstrokecolor{currentstroke}%
\pgfsetdash{}{0pt}%
\pgfpathmoveto{\pgfqpoint{4.235995in}{4.678979in}}%
\pgfpathlineto{\pgfqpoint{4.092594in}{4.886710in}}%
\pgfpathlineto{\pgfqpoint{4.007443in}{4.927974in}}%
\pgfpathclose%
\pgfusepath{fill}%
\end{pgfscope}%
\begin{pgfscope}%
\pgfpathrectangle{\pgfqpoint{0.539299in}{0.078740in}}{\pgfqpoint{7.842520in}{7.842520in}}%
\pgfusepath{clip}%
\pgfsetbuttcap%
\pgfsetroundjoin%
\definecolor{currentfill}{rgb}{0.515992,0.831158,0.294279}%
\pgfsetfillcolor{currentfill}%
\pgfsetlinewidth{0.000000pt}%
\definecolor{currentstroke}{rgb}{0.227802,0.326594,0.546532}%
\pgfsetstrokecolor{currentstroke}%
\pgfsetdash{}{0pt}%
\pgfpathmoveto{\pgfqpoint{2.444220in}{4.720662in}}%
\pgfpathlineto{\pgfqpoint{2.528143in}{4.972174in}}%
\pgfpathlineto{\pgfqpoint{2.393562in}{4.830606in}}%
\pgfpathclose%
\pgfusepath{fill}%
\end{pgfscope}%
\begin{pgfscope}%
\pgfpathrectangle{\pgfqpoint{0.539299in}{0.078740in}}{\pgfqpoint{7.842520in}{7.842520in}}%
\pgfusepath{clip}%
\pgfsetbuttcap%
\pgfsetroundjoin%
\definecolor{currentfill}{rgb}{0.964894,0.902323,0.123941}%
\pgfsetfillcolor{currentfill}%
\pgfsetlinewidth{0.000000pt}%
\definecolor{currentstroke}{rgb}{0.225863,0.330805,0.547314}%
\pgfsetstrokecolor{currentstroke}%
\pgfsetdash{}{0pt}%
\pgfpathmoveto{\pgfqpoint{3.291045in}{5.591989in}}%
\pgfpathlineto{\pgfqpoint{3.433110in}{5.522817in}}%
\pgfpathlineto{\pgfqpoint{3.378279in}{5.603577in}}%
\pgfpathclose%
\pgfusepath{fill}%
\end{pgfscope}%
\begin{pgfscope}%
\pgfpathrectangle{\pgfqpoint{0.539299in}{0.078740in}}{\pgfqpoint{7.842520in}{7.842520in}}%
\pgfusepath{clip}%
\pgfsetbuttcap%
\pgfsetroundjoin%
\definecolor{currentfill}{rgb}{0.955300,0.901065,0.118128}%
\pgfsetfillcolor{currentfill}%
\pgfsetlinewidth{0.000000pt}%
\definecolor{currentstroke}{rgb}{0.223925,0.334994,0.548053}%
\pgfsetstrokecolor{currentstroke}%
\pgfsetdash{}{0pt}%
\pgfpathmoveto{\pgfqpoint{2.923598in}{5.523700in}}%
\pgfpathlineto{\pgfqpoint{3.062895in}{5.558135in}}%
\pgfpathlineto{\pgfqpoint{3.150177in}{5.616991in}}%
\pgfpathclose%
\pgfusepath{fill}%
\end{pgfscope}%
\begin{pgfscope}%
\pgfpathrectangle{\pgfqpoint{0.539299in}{0.078740in}}{\pgfqpoint{7.842520in}{7.842520in}}%
\pgfusepath{clip}%
\pgfsetbuttcap%
\pgfsetroundjoin%
\definecolor{currentfill}{rgb}{0.214298,0.355619,0.551184}%
\pgfsetfillcolor{currentfill}%
\pgfsetlinewidth{0.000000pt}%
\definecolor{currentstroke}{rgb}{0.221989,0.339161,0.548752}%
\pgfsetstrokecolor{currentstroke}%
\pgfsetdash{}{0pt}%
\pgfpathmoveto{\pgfqpoint{2.611847in}{2.338243in}}%
\pgfpathlineto{\pgfqpoint{2.751870in}{2.294211in}}%
\pgfpathlineto{\pgfqpoint{2.684453in}{3.009085in}}%
\pgfpathclose%
\pgfusepath{fill}%
\end{pgfscope}%
\begin{pgfscope}%
\pgfpathrectangle{\pgfqpoint{0.539299in}{0.078740in}}{\pgfqpoint{7.842520in}{7.842520in}}%
\pgfusepath{clip}%
\pgfsetbuttcap%
\pgfsetroundjoin%
\definecolor{currentfill}{rgb}{0.185556,0.418570,0.556753}%
\pgfsetfillcolor{currentfill}%
\pgfsetlinewidth{0.000000pt}%
\definecolor{currentstroke}{rgb}{0.220057,0.343307,0.549413}%
\pgfsetstrokecolor{currentstroke}%
\pgfsetdash{}{0pt}%
\pgfpathmoveto{\pgfqpoint{2.544311in}{3.042210in}}%
\pgfpathlineto{\pgfqpoint{2.472446in}{2.377069in}}%
\pgfpathlineto{\pgfqpoint{2.684453in}{3.009085in}}%
\pgfpathclose%
\pgfusepath{fill}%
\end{pgfscope}%
\begin{pgfscope}%
\pgfpathrectangle{\pgfqpoint{0.539299in}{0.078740in}}{\pgfqpoint{7.842520in}{7.842520in}}%
\pgfusepath{clip}%
\pgfsetbuttcap%
\pgfsetroundjoin%
\definecolor{currentfill}{rgb}{0.304148,0.764704,0.419943}%
\pgfsetfillcolor{currentfill}%
\pgfsetlinewidth{0.000000pt}%
\definecolor{currentstroke}{rgb}{0.218130,0.347432,0.550038}%
\pgfsetstrokecolor{currentstroke}%
\pgfsetdash{}{0pt}%
\pgfpathmoveto{\pgfqpoint{4.439213in}{4.296234in}}%
\pgfpathlineto{\pgfqpoint{4.379257in}{4.461675in}}%
\pgfpathlineto{\pgfqpoint{4.295430in}{4.515609in}}%
\pgfpathclose%
\pgfusepath{fill}%
\end{pgfscope}%
\begin{pgfscope}%
\pgfpathrectangle{\pgfqpoint{0.539299in}{0.078740in}}{\pgfqpoint{7.842520in}{7.842520in}}%
\pgfusepath{clip}%
\pgfsetbuttcap%
\pgfsetroundjoin%
\definecolor{currentfill}{rgb}{0.595839,0.848717,0.243329}%
\pgfsetfillcolor{currentfill}%
\pgfsetlinewidth{0.000000pt}%
\definecolor{currentstroke}{rgb}{0.216210,0.351535,0.550627}%
\pgfsetstrokecolor{currentstroke}%
\pgfsetdash{}{0pt}%
\pgfpathmoveto{\pgfqpoint{4.007443in}{4.927974in}}%
\pgfpathlineto{\pgfqpoint{4.092594in}{4.886710in}}%
\pgfpathlineto{\pgfqpoint{3.949148in}{5.080344in}}%
\pgfpathclose%
\pgfusepath{fill}%
\end{pgfscope}%
\begin{pgfscope}%
\pgfpathrectangle{\pgfqpoint{0.539299in}{0.078740in}}{\pgfqpoint{7.842520in}{7.842520in}}%
\pgfusepath{clip}%
\pgfsetbuttcap%
\pgfsetroundjoin%
\definecolor{currentfill}{rgb}{0.327796,0.773980,0.406640}%
\pgfsetfillcolor{currentfill}%
\pgfsetlinewidth{0.000000pt}%
\definecolor{currentstroke}{rgb}{0.214298,0.355619,0.551184}%
\pgfsetstrokecolor{currentstroke}%
\pgfsetdash{}{0pt}%
\pgfpathmoveto{\pgfqpoint{2.227125in}{4.300185in}}%
\pgfpathlineto{\pgfqpoint{2.361965in}{4.410404in}}%
\pgfpathlineto{\pgfqpoint{2.444220in}{4.720662in}}%
\pgfpathclose%
\pgfusepath{fill}%
\end{pgfscope}%
\begin{pgfscope}%
\pgfpathrectangle{\pgfqpoint{0.539299in}{0.078740in}}{\pgfqpoint{7.842520in}{7.842520in}}%
\pgfusepath{clip}%
\pgfsetbuttcap%
\pgfsetroundjoin%
\definecolor{currentfill}{rgb}{0.280868,0.160771,0.472899}%
\pgfsetfillcolor{currentfill}%
\pgfsetlinewidth{0.000000pt}%
\definecolor{currentstroke}{rgb}{0.212395,0.359683,0.551710}%
\pgfsetstrokecolor{currentstroke}%
\pgfsetdash{}{0pt}%
\pgfpathmoveto{\pgfqpoint{5.853809in}{1.950429in}}%
\pgfpathlineto{\pgfqpoint{5.996574in}{1.738442in}}%
\pgfpathlineto{\pgfqpoint{6.073966in}{1.765621in}}%
\pgfpathclose%
\pgfusepath{fill}%
\end{pgfscope}%
\begin{pgfscope}%
\pgfpathrectangle{\pgfqpoint{0.539299in}{0.078740in}}{\pgfqpoint{7.842520in}{7.842520in}}%
\pgfusepath{clip}%
\pgfsetbuttcap%
\pgfsetroundjoin%
\definecolor{currentfill}{rgb}{0.241237,0.296485,0.539709}%
\pgfsetfillcolor{currentfill}%
\pgfsetlinewidth{0.000000pt}%
\definecolor{currentstroke}{rgb}{0.210503,0.363727,0.552206}%
\pgfsetstrokecolor{currentstroke}%
\pgfsetdash{}{0pt}%
\pgfpathmoveto{\pgfqpoint{5.788906in}{2.159810in}}%
\pgfpathlineto{\pgfqpoint{5.646299in}{2.364748in}}%
\pgfpathlineto{\pgfqpoint{5.567779in}{2.379196in}}%
\pgfpathclose%
\pgfusepath{fill}%
\end{pgfscope}%
\begin{pgfscope}%
\pgfpathrectangle{\pgfqpoint{0.539299in}{0.078740in}}{\pgfqpoint{7.842520in}{7.842520in}}%
\pgfusepath{clip}%
\pgfsetbuttcap%
\pgfsetroundjoin%
\definecolor{currentfill}{rgb}{0.814576,0.883393,0.110347}%
\pgfsetfillcolor{currentfill}%
\pgfsetlinewidth{0.000000pt}%
\definecolor{currentstroke}{rgb}{0.208623,0.367752,0.552675}%
\pgfsetstrokecolor{currentstroke}%
\pgfsetdash{}{0pt}%
\pgfpathmoveto{\pgfqpoint{3.719555in}{5.277291in}}%
\pgfpathlineto{\pgfqpoint{3.805790in}{5.254759in}}%
\pgfpathlineto{\pgfqpoint{3.662699in}{5.404278in}}%
\pgfpathclose%
\pgfusepath{fill}%
\end{pgfscope}%
\begin{pgfscope}%
\pgfpathrectangle{\pgfqpoint{0.539299in}{0.078740in}}{\pgfqpoint{7.842520in}{7.842520in}}%
\pgfusepath{clip}%
\pgfsetbuttcap%
\pgfsetroundjoin%
\definecolor{currentfill}{rgb}{0.709898,0.868751,0.169257}%
\pgfsetfillcolor{currentfill}%
\pgfsetlinewidth{0.000000pt}%
\definecolor{currentstroke}{rgb}{0.206756,0.371758,0.553117}%
\pgfsetstrokecolor{currentstroke}%
\pgfsetdash{}{0pt}%
\pgfpathmoveto{\pgfqpoint{3.949148in}{5.080344in}}%
\pgfpathlineto{\pgfqpoint{3.805790in}{5.254759in}}%
\pgfpathlineto{\pgfqpoint{3.863419in}{5.112875in}}%
\pgfpathclose%
\pgfusepath{fill}%
\end{pgfscope}%
\begin{pgfscope}%
\pgfpathrectangle{\pgfqpoint{0.539299in}{0.078740in}}{\pgfqpoint{7.842520in}{7.842520in}}%
\pgfusepath{clip}%
\pgfsetbuttcap%
\pgfsetroundjoin%
\definecolor{currentfill}{rgb}{0.143343,0.522773,0.556295}%
\pgfsetfillcolor{currentfill}%
\pgfsetlinewidth{0.000000pt}%
\definecolor{currentstroke}{rgb}{0.204903,0.375746,0.553533}%
\pgfsetstrokecolor{currentstroke}%
\pgfsetdash{}{0pt}%
\pgfpathmoveto{\pgfqpoint{4.931146in}{3.442939in}}%
\pgfpathlineto{\pgfqpoint{5.074536in}{3.221540in}}%
\pgfpathlineto{\pgfqpoint{5.154816in}{3.180272in}}%
\pgfpathclose%
\pgfusepath{fill}%
\end{pgfscope}%
\begin{pgfscope}%
\pgfpathrectangle{\pgfqpoint{0.539299in}{0.078740in}}{\pgfqpoint{7.842520in}{7.842520in}}%
\pgfusepath{clip}%
\pgfsetbuttcap%
\pgfsetroundjoin%
\definecolor{currentfill}{rgb}{0.119699,0.618490,0.536347}%
\pgfsetfillcolor{currentfill}%
\pgfsetlinewidth{0.000000pt}%
\definecolor{currentstroke}{rgb}{0.203063,0.379716,0.553925}%
\pgfsetstrokecolor{currentstroke}%
\pgfsetdash{}{0pt}%
\pgfpathmoveto{\pgfqpoint{4.869240in}{3.622273in}}%
\pgfpathlineto{\pgfqpoint{4.726130in}{3.847156in}}%
\pgfpathlineto{\pgfqpoint{4.787540in}{3.665841in}}%
\pgfpathclose%
\pgfusepath{fill}%
\end{pgfscope}%
\begin{pgfscope}%
\pgfpathrectangle{\pgfqpoint{0.539299in}{0.078740in}}{\pgfqpoint{7.842520in}{7.842520in}}%
\pgfusepath{clip}%
\pgfsetbuttcap%
\pgfsetroundjoin%
\definecolor{currentfill}{rgb}{0.783315,0.879285,0.125405}%
\pgfsetfillcolor{currentfill}%
\pgfsetlinewidth{0.000000pt}%
\definecolor{currentstroke}{rgb}{0.201239,0.383670,0.554294}%
\pgfsetstrokecolor{currentstroke}%
\pgfsetdash{}{0pt}%
\pgfpathmoveto{\pgfqpoint{2.699533in}{5.323077in}}%
\pgfpathlineto{\pgfqpoint{2.613359in}{5.171039in}}%
\pgfpathlineto{\pgfqpoint{2.750456in}{5.265020in}}%
\pgfpathclose%
\pgfusepath{fill}%
\end{pgfscope}%
\begin{pgfscope}%
\pgfpathrectangle{\pgfqpoint{0.539299in}{0.078740in}}{\pgfqpoint{7.842520in}{7.842520in}}%
\pgfusepath{clip}%
\pgfsetbuttcap%
\pgfsetroundjoin%
\definecolor{currentfill}{rgb}{0.377779,0.791781,0.377939}%
\pgfsetfillcolor{currentfill}%
\pgfsetlinewidth{0.000000pt}%
\definecolor{currentstroke}{rgb}{0.199430,0.387607,0.554642}%
\pgfsetstrokecolor{currentstroke}%
\pgfsetdash{}{0pt}%
\pgfpathmoveto{\pgfqpoint{4.379257in}{4.461675in}}%
\pgfpathlineto{\pgfqpoint{4.151482in}{4.727445in}}%
\pgfpathlineto{\pgfqpoint{4.295430in}{4.515609in}}%
\pgfpathclose%
\pgfusepath{fill}%
\end{pgfscope}%
\begin{pgfscope}%
\pgfpathrectangle{\pgfqpoint{0.539299in}{0.078740in}}{\pgfqpoint{7.842520in}{7.842520in}}%
\pgfusepath{clip}%
\pgfsetbuttcap%
\pgfsetroundjoin%
\definecolor{currentfill}{rgb}{0.267968,0.223549,0.512008}%
\pgfsetfillcolor{currentfill}%
\pgfsetlinewidth{0.000000pt}%
\definecolor{currentstroke}{rgb}{0.197636,0.391528,0.554969}%
\pgfsetstrokecolor{currentstroke}%
\pgfsetdash{}{0pt}%
\pgfpathmoveto{\pgfqpoint{5.710883in}{2.164223in}}%
\pgfpathlineto{\pgfqpoint{5.853809in}{1.950429in}}%
\pgfpathlineto{\pgfqpoint{5.931443in}{1.959659in}}%
\pgfpathclose%
\pgfusepath{fill}%
\end{pgfscope}%
\begin{pgfscope}%
\pgfpathrectangle{\pgfqpoint{0.539299in}{0.078740in}}{\pgfqpoint{7.842520in}{7.842520in}}%
\pgfusepath{clip}%
\pgfsetbuttcap%
\pgfsetroundjoin%
\definecolor{currentfill}{rgb}{0.216210,0.351535,0.550627}%
\pgfsetfillcolor{currentfill}%
\pgfsetlinewidth{0.000000pt}%
\definecolor{currentstroke}{rgb}{0.195860,0.395433,0.555276}%
\pgfsetstrokecolor{currentstroke}%
\pgfsetdash{}{0pt}%
\pgfpathmoveto{\pgfqpoint{5.424487in}{2.595182in}}%
\pgfpathlineto{\pgfqpoint{5.646299in}{2.364748in}}%
\pgfpathlineto{\pgfqpoint{5.503585in}{2.573766in}}%
\pgfpathclose%
\pgfusepath{fill}%
\end{pgfscope}%
\begin{pgfscope}%
\pgfpathrectangle{\pgfqpoint{0.539299in}{0.078740in}}{\pgfqpoint{7.842520in}{7.842520in}}%
\pgfusepath{clip}%
\pgfsetbuttcap%
\pgfsetroundjoin%
\definecolor{currentfill}{rgb}{0.935904,0.898570,0.108131}%
\pgfsetfillcolor{currentfill}%
\pgfsetlinewidth{0.000000pt}%
\definecolor{currentstroke}{rgb}{0.194100,0.399323,0.555565}%
\pgfsetstrokecolor{currentstroke}%
\pgfsetdash{}{0pt}%
\pgfpathmoveto{\pgfqpoint{3.062895in}{5.558135in}}%
\pgfpathlineto{\pgfqpoint{2.923598in}{5.523700in}}%
\pgfpathlineto{\pgfqpoint{2.836696in}{5.416830in}}%
\pgfpathclose%
\pgfusepath{fill}%
\end{pgfscope}%
\begin{pgfscope}%
\pgfpathrectangle{\pgfqpoint{0.539299in}{0.078740in}}{\pgfqpoint{7.842520in}{7.842520in}}%
\pgfusepath{clip}%
\pgfsetbuttcap%
\pgfsetroundjoin%
\definecolor{currentfill}{rgb}{0.129933,0.559582,0.551864}%
\pgfsetfillcolor{currentfill}%
\pgfsetlinewidth{0.000000pt}%
\definecolor{currentstroke}{rgb}{0.192357,0.403199,0.555836}%
\pgfsetstrokecolor{currentstroke}%
\pgfsetdash{}{0pt}%
\pgfpathmoveto{\pgfqpoint{2.479796in}{3.628915in}}%
\pgfpathlineto{\pgfqpoint{2.341276in}{3.620065in}}%
\pgfpathlineto{\pgfqpoint{2.405032in}{3.063501in}}%
\pgfpathclose%
\pgfusepath{fill}%
\end{pgfscope}%
\begin{pgfscope}%
\pgfpathrectangle{\pgfqpoint{0.539299in}{0.078740in}}{\pgfqpoint{7.842520in}{7.842520in}}%
\pgfusepath{clip}%
\pgfsetbuttcap%
\pgfsetroundjoin%
\definecolor{currentfill}{rgb}{0.835270,0.886029,0.102646}%
\pgfsetfillcolor{currentfill}%
\pgfsetlinewidth{0.000000pt}%
\definecolor{currentstroke}{rgb}{0.190631,0.407061,0.556089}%
\pgfsetstrokecolor{currentstroke}%
\pgfsetdash{}{0pt}%
\pgfpathmoveto{\pgfqpoint{2.699533in}{5.323077in}}%
\pgfpathlineto{\pgfqpoint{2.750456in}{5.265020in}}%
\pgfpathlineto{\pgfqpoint{2.836696in}{5.416830in}}%
\pgfpathclose%
\pgfusepath{fill}%
\end{pgfscope}%
\begin{pgfscope}%
\pgfpathrectangle{\pgfqpoint{0.539299in}{0.078740in}}{\pgfqpoint{7.842520in}{7.842520in}}%
\pgfusepath{clip}%
\pgfsetbuttcap%
\pgfsetroundjoin%
\definecolor{currentfill}{rgb}{0.496615,0.826376,0.306377}%
\pgfsetfillcolor{currentfill}%
\pgfsetlinewidth{0.000000pt}%
\definecolor{currentstroke}{rgb}{0.188923,0.410910,0.556326}%
\pgfsetstrokecolor{currentstroke}%
\pgfsetdash{}{0pt}%
\pgfpathmoveto{\pgfqpoint{4.007443in}{4.927974in}}%
\pgfpathlineto{\pgfqpoint{4.151482in}{4.727445in}}%
\pgfpathlineto{\pgfqpoint{4.235995in}{4.678979in}}%
\pgfpathclose%
\pgfusepath{fill}%
\end{pgfscope}%
\begin{pgfscope}%
\pgfpathrectangle{\pgfqpoint{0.539299in}{0.078740in}}{\pgfqpoint{7.842520in}{7.842520in}}%
\pgfusepath{clip}%
\pgfsetbuttcap%
\pgfsetroundjoin%
\definecolor{currentfill}{rgb}{0.983868,0.904867,0.136897}%
\pgfsetfillcolor{currentfill}%
\pgfsetlinewidth{0.000000pt}%
\definecolor{currentstroke}{rgb}{0.187231,0.414746,0.556547}%
\pgfsetstrokecolor{currentstroke}%
\pgfsetdash{}{0pt}%
\pgfpathmoveto{\pgfqpoint{3.150177in}{5.616991in}}%
\pgfpathlineto{\pgfqpoint{3.203841in}{5.543255in}}%
\pgfpathlineto{\pgfqpoint{3.291045in}{5.591989in}}%
\pgfpathclose%
\pgfusepath{fill}%
\end{pgfscope}%
\begin{pgfscope}%
\pgfpathrectangle{\pgfqpoint{0.539299in}{0.078740in}}{\pgfqpoint{7.842520in}{7.842520in}}%
\pgfusepath{clip}%
\pgfsetbuttcap%
\pgfsetroundjoin%
\definecolor{currentfill}{rgb}{0.906311,0.894855,0.098125}%
\pgfsetfillcolor{currentfill}%
\pgfsetlinewidth{0.000000pt}%
\definecolor{currentstroke}{rgb}{0.185556,0.418570,0.556753}%
\pgfsetstrokecolor{currentstroke}%
\pgfsetdash{}{0pt}%
\pgfpathmoveto{\pgfqpoint{3.433110in}{5.522817in}}%
\pgfpathlineto{\pgfqpoint{3.576041in}{5.415869in}}%
\pgfpathlineto{\pgfqpoint{3.662699in}{5.404278in}}%
\pgfpathclose%
\pgfusepath{fill}%
\end{pgfscope}%
\begin{pgfscope}%
\pgfpathrectangle{\pgfqpoint{0.539299in}{0.078740in}}{\pgfqpoint{7.842520in}{7.842520in}}%
\pgfusepath{clip}%
\pgfsetbuttcap%
\pgfsetroundjoin%
\definecolor{currentfill}{rgb}{0.199430,0.387607,0.554642}%
\pgfsetfillcolor{currentfill}%
\pgfsetlinewidth{0.000000pt}%
\definecolor{currentstroke}{rgb}{0.183898,0.422383,0.556944}%
\pgfsetstrokecolor{currentstroke}%
\pgfsetdash{}{0pt}%
\pgfpathmoveto{\pgfqpoint{5.503585in}{2.573766in}}%
\pgfpathlineto{\pgfqpoint{5.360735in}{2.786461in}}%
\pgfpathlineto{\pgfqpoint{5.424487in}{2.595182in}}%
\pgfpathclose%
\pgfusepath{fill}%
\end{pgfscope}%
\begin{pgfscope}%
\pgfpathrectangle{\pgfqpoint{0.539299in}{0.078740in}}{\pgfqpoint{7.842520in}{7.842520in}}%
\pgfusepath{clip}%
\pgfsetbuttcap%
\pgfsetroundjoin%
\definecolor{currentfill}{rgb}{0.140210,0.665859,0.513427}%
\pgfsetfillcolor{currentfill}%
\pgfsetlinewidth{0.000000pt}%
\definecolor{currentstroke}{rgb}{0.182256,0.426184,0.557120}%
\pgfsetstrokecolor{currentstroke}%
\pgfsetdash{}{0pt}%
\pgfpathmoveto{\pgfqpoint{2.281795in}{4.035068in}}%
\pgfpathlineto{\pgfqpoint{2.341276in}{3.620065in}}%
\pgfpathlineto{\pgfqpoint{2.418849in}{4.087714in}}%
\pgfpathclose%
\pgfusepath{fill}%
\end{pgfscope}%
\begin{pgfscope}%
\pgfpathrectangle{\pgfqpoint{0.539299in}{0.078740in}}{\pgfqpoint{7.842520in}{7.842520in}}%
\pgfusepath{clip}%
\pgfsetbuttcap%
\pgfsetroundjoin%
\definecolor{currentfill}{rgb}{0.246811,0.283237,0.535941}%
\pgfsetfillcolor{currentfill}%
\pgfsetlinewidth{0.000000pt}%
\definecolor{currentstroke}{rgb}{0.180629,0.429975,0.557282}%
\pgfsetstrokecolor{currentstroke}%
\pgfsetdash{}{0pt}%
\pgfpathmoveto{\pgfqpoint{5.567779in}{2.379196in}}%
\pgfpathlineto{\pgfqpoint{5.710883in}{2.164223in}}%
\pgfpathlineto{\pgfqpoint{5.788906in}{2.159810in}}%
\pgfpathclose%
\pgfusepath{fill}%
\end{pgfscope}%
\begin{pgfscope}%
\pgfpathrectangle{\pgfqpoint{0.539299in}{0.078740in}}{\pgfqpoint{7.842520in}{7.842520in}}%
\pgfusepath{clip}%
\pgfsetbuttcap%
\pgfsetroundjoin%
\definecolor{currentfill}{rgb}{0.121148,0.592739,0.544641}%
\pgfsetfillcolor{currentfill}%
\pgfsetlinewidth{0.000000pt}%
\definecolor{currentstroke}{rgb}{0.179019,0.433756,0.557430}%
\pgfsetstrokecolor{currentstroke}%
\pgfsetdash{}{0pt}%
\pgfpathmoveto{\pgfqpoint{4.869240in}{3.622273in}}%
\pgfpathlineto{\pgfqpoint{4.787540in}{3.665841in}}%
\pgfpathlineto{\pgfqpoint{4.931146in}{3.442939in}}%
\pgfpathclose%
\pgfusepath{fill}%
\end{pgfscope}%
\begin{pgfscope}%
\pgfpathrectangle{\pgfqpoint{0.539299in}{0.078740in}}{\pgfqpoint{7.842520in}{7.842520in}}%
\pgfusepath{clip}%
\pgfsetbuttcap%
\pgfsetroundjoin%
\definecolor{currentfill}{rgb}{0.214000,0.722114,0.469588}%
\pgfsetfillcolor{currentfill}%
\pgfsetlinewidth{0.000000pt}%
\definecolor{currentstroke}{rgb}{0.177423,0.437527,0.557565}%
\pgfsetstrokecolor{currentstroke}%
\pgfsetdash{}{0pt}%
\pgfpathmoveto{\pgfqpoint{2.361965in}{4.410404in}}%
\pgfpathlineto{\pgfqpoint{2.281795in}{4.035068in}}%
\pgfpathlineto{\pgfqpoint{2.418849in}{4.087714in}}%
\pgfpathclose%
\pgfusepath{fill}%
\end{pgfscope}%
\begin{pgfscope}%
\pgfpathrectangle{\pgfqpoint{0.539299in}{0.078740in}}{\pgfqpoint{7.842520in}{7.842520in}}%
\pgfusepath{clip}%
\pgfsetbuttcap%
\pgfsetroundjoin%
\definecolor{currentfill}{rgb}{0.720391,0.870350,0.162603}%
\pgfsetfillcolor{currentfill}%
\pgfsetlinewidth{0.000000pt}%
\definecolor{currentstroke}{rgb}{0.175841,0.441290,0.557685}%
\pgfsetstrokecolor{currentstroke}%
\pgfsetdash{}{0pt}%
\pgfpathmoveto{\pgfqpoint{2.750456in}{5.265020in}}%
\pgfpathlineto{\pgfqpoint{2.613359in}{5.171039in}}%
\pgfpathlineto{\pgfqpoint{2.528143in}{4.972174in}}%
\pgfpathclose%
\pgfusepath{fill}%
\end{pgfscope}%
\begin{pgfscope}%
\pgfpathrectangle{\pgfqpoint{0.539299in}{0.078740in}}{\pgfqpoint{7.842520in}{7.842520in}}%
\pgfusepath{clip}%
\pgfsetbuttcap%
\pgfsetroundjoin%
\definecolor{currentfill}{rgb}{0.983868,0.904867,0.136897}%
\pgfsetfillcolor{currentfill}%
\pgfsetlinewidth{0.000000pt}%
\definecolor{currentstroke}{rgb}{0.174274,0.445044,0.557792}%
\pgfsetstrokecolor{currentstroke}%
\pgfsetdash{}{0pt}%
\pgfpathmoveto{\pgfqpoint{3.062895in}{5.558135in}}%
\pgfpathlineto{\pgfqpoint{3.203841in}{5.543255in}}%
\pgfpathlineto{\pgfqpoint{3.150177in}{5.616991in}}%
\pgfpathclose%
\pgfusepath{fill}%
\end{pgfscope}%
\begin{pgfscope}%
\pgfpathrectangle{\pgfqpoint{0.539299in}{0.078740in}}{\pgfqpoint{7.842520in}{7.842520in}}%
\pgfusepath{clip}%
\pgfsetbuttcap%
\pgfsetroundjoin%
\definecolor{currentfill}{rgb}{0.162016,0.687316,0.499129}%
\pgfsetfillcolor{currentfill}%
\pgfsetlinewidth{0.000000pt}%
\definecolor{currentstroke}{rgb}{0.172719,0.448791,0.557885}%
\pgfsetstrokecolor{currentstroke}%
\pgfsetdash{}{0pt}%
\pgfpathmoveto{\pgfqpoint{4.499656in}{4.110842in}}%
\pgfpathlineto{\pgfqpoint{4.726130in}{3.847156in}}%
\pgfpathlineto{\pgfqpoint{4.582787in}{4.072527in}}%
\pgfpathclose%
\pgfusepath{fill}%
\end{pgfscope}%
\begin{pgfscope}%
\pgfpathrectangle{\pgfqpoint{0.539299in}{0.078740in}}{\pgfqpoint{7.842520in}{7.842520in}}%
\pgfusepath{clip}%
\pgfsetbuttcap%
\pgfsetroundjoin%
\definecolor{currentfill}{rgb}{0.866013,0.889868,0.095953}%
\pgfsetfillcolor{currentfill}%
\pgfsetlinewidth{0.000000pt}%
\definecolor{currentstroke}{rgb}{0.171176,0.452530,0.557965}%
\pgfsetstrokecolor{currentstroke}%
\pgfsetdash{}{0pt}%
\pgfpathmoveto{\pgfqpoint{3.662699in}{5.404278in}}%
\pgfpathlineto{\pgfqpoint{3.576041in}{5.415869in}}%
\pgfpathlineto{\pgfqpoint{3.719555in}{5.277291in}}%
\pgfpathclose%
\pgfusepath{fill}%
\end{pgfscope}%
\begin{pgfscope}%
\pgfpathrectangle{\pgfqpoint{0.539299in}{0.078740in}}{\pgfqpoint{7.842520in}{7.842520in}}%
\pgfusepath{clip}%
\pgfsetbuttcap%
\pgfsetroundjoin%
\definecolor{currentfill}{rgb}{0.668054,0.861999,0.196293}%
\pgfsetfillcolor{currentfill}%
\pgfsetlinewidth{0.000000pt}%
\definecolor{currentstroke}{rgb}{0.169646,0.456262,0.558030}%
\pgfsetstrokecolor{currentstroke}%
\pgfsetdash{}{0pt}%
\pgfpathmoveto{\pgfqpoint{3.949148in}{5.080344in}}%
\pgfpathlineto{\pgfqpoint{3.863419in}{5.112875in}}%
\pgfpathlineto{\pgfqpoint{4.007443in}{4.927974in}}%
\pgfpathclose%
\pgfusepath{fill}%
\end{pgfscope}%
\begin{pgfscope}%
\pgfpathrectangle{\pgfqpoint{0.539299in}{0.078740in}}{\pgfqpoint{7.842520in}{7.842520in}}%
\pgfusepath{clip}%
\pgfsetbuttcap%
\pgfsetroundjoin%
\definecolor{currentfill}{rgb}{0.772852,0.877868,0.131109}%
\pgfsetfillcolor{currentfill}%
\pgfsetlinewidth{0.000000pt}%
\definecolor{currentstroke}{rgb}{0.168126,0.459988,0.558082}%
\pgfsetstrokecolor{currentstroke}%
\pgfsetdash{}{0pt}%
\pgfpathmoveto{\pgfqpoint{3.863419in}{5.112875in}}%
\pgfpathlineto{\pgfqpoint{3.805790in}{5.254759in}}%
\pgfpathlineto{\pgfqpoint{3.719555in}{5.277291in}}%
\pgfpathclose%
\pgfusepath{fill}%
\end{pgfscope}%
\begin{pgfscope}%
\pgfpathrectangle{\pgfqpoint{0.539299in}{0.078740in}}{\pgfqpoint{7.842520in}{7.842520in}}%
\pgfusepath{clip}%
\pgfsetbuttcap%
\pgfsetroundjoin%
\definecolor{currentfill}{rgb}{0.143343,0.522773,0.556295}%
\pgfsetfillcolor{currentfill}%
\pgfsetlinewidth{0.000000pt}%
\definecolor{currentstroke}{rgb}{0.166617,0.463708,0.558119}%
\pgfsetstrokecolor{currentstroke}%
\pgfsetdash{}{0pt}%
\pgfpathmoveto{\pgfqpoint{2.619520in}{3.617668in}}%
\pgfpathlineto{\pgfqpoint{2.405032in}{3.063501in}}%
\pgfpathlineto{\pgfqpoint{2.544311in}{3.042210in}}%
\pgfpathclose%
\pgfusepath{fill}%
\end{pgfscope}%
\begin{pgfscope}%
\pgfpathrectangle{\pgfqpoint{0.539299in}{0.078740in}}{\pgfqpoint{7.842520in}{7.842520in}}%
\pgfusepath{clip}%
\pgfsetbuttcap%
\pgfsetroundjoin%
\definecolor{currentfill}{rgb}{0.282910,0.105393,0.426902}%
\pgfsetfillcolor{currentfill}%
\pgfsetlinewidth{0.000000pt}%
\definecolor{currentstroke}{rgb}{0.165117,0.467423,0.558141}%
\pgfsetstrokecolor{currentstroke}%
\pgfsetdash{}{0pt}%
\pgfpathmoveto{\pgfqpoint{6.061409in}{1.481685in}}%
\pgfpathlineto{\pgfqpoint{6.139221in}{1.529769in}}%
\pgfpathlineto{\pgfqpoint{5.996574in}{1.738442in}}%
\pgfpathclose%
\pgfusepath{fill}%
\end{pgfscope}%
\begin{pgfscope}%
\pgfpathrectangle{\pgfqpoint{0.539299in}{0.078740in}}{\pgfqpoint{7.842520in}{7.842520in}}%
\pgfusepath{clip}%
\pgfsetbuttcap%
\pgfsetroundjoin%
\definecolor{currentfill}{rgb}{0.220057,0.343307,0.549413}%
\pgfsetfillcolor{currentfill}%
\pgfsetlinewidth{0.000000pt}%
\definecolor{currentstroke}{rgb}{0.163625,0.471133,0.558148}%
\pgfsetstrokecolor{currentstroke}%
\pgfsetdash{}{0pt}%
\pgfpathmoveto{\pgfqpoint{5.424487in}{2.595182in}}%
\pgfpathlineto{\pgfqpoint{5.567779in}{2.379196in}}%
\pgfpathlineto{\pgfqpoint{5.646299in}{2.364748in}}%
\pgfpathclose%
\pgfusepath{fill}%
\end{pgfscope}%
\begin{pgfscope}%
\pgfpathrectangle{\pgfqpoint{0.539299in}{0.078740in}}{\pgfqpoint{7.842520in}{7.842520in}}%
\pgfusepath{clip}%
\pgfsetbuttcap%
\pgfsetroundjoin%
\definecolor{currentfill}{rgb}{0.214000,0.722114,0.469588}%
\pgfsetfillcolor{currentfill}%
\pgfsetlinewidth{0.000000pt}%
\definecolor{currentstroke}{rgb}{0.162142,0.474838,0.558140}%
\pgfsetstrokecolor{currentstroke}%
\pgfsetdash{}{0pt}%
\pgfpathmoveto{\pgfqpoint{4.582787in}{4.072527in}}%
\pgfpathlineto{\pgfqpoint{4.439213in}{4.296234in}}%
\pgfpathlineto{\pgfqpoint{4.499656in}{4.110842in}}%
\pgfpathclose%
\pgfusepath{fill}%
\end{pgfscope}%
\begin{pgfscope}%
\pgfpathrectangle{\pgfqpoint{0.539299in}{0.078740in}}{\pgfqpoint{7.842520in}{7.842520in}}%
\pgfusepath{clip}%
\pgfsetbuttcap%
\pgfsetroundjoin%
\definecolor{currentfill}{rgb}{0.168126,0.459988,0.558082}%
\pgfsetfillcolor{currentfill}%
\pgfsetlinewidth{0.000000pt}%
\definecolor{currentstroke}{rgb}{0.160665,0.478540,0.558115}%
\pgfsetstrokecolor{currentstroke}%
\pgfsetdash{}{0pt}%
\pgfpathmoveto{\pgfqpoint{5.217726in}{3.002518in}}%
\pgfpathlineto{\pgfqpoint{5.137323in}{3.030184in}}%
\pgfpathlineto{\pgfqpoint{5.360735in}{2.786461in}}%
\pgfpathclose%
\pgfusepath{fill}%
\end{pgfscope}%
\begin{pgfscope}%
\pgfpathrectangle{\pgfqpoint{0.539299in}{0.078740in}}{\pgfqpoint{7.842520in}{7.842520in}}%
\pgfusepath{clip}%
\pgfsetbuttcap%
\pgfsetroundjoin%
\definecolor{currentfill}{rgb}{0.983868,0.904867,0.136897}%
\pgfsetfillcolor{currentfill}%
\pgfsetlinewidth{0.000000pt}%
\definecolor{currentstroke}{rgb}{0.159194,0.482237,0.558073}%
\pgfsetstrokecolor{currentstroke}%
\pgfsetdash{}{0pt}%
\pgfpathmoveto{\pgfqpoint{3.203841in}{5.543255in}}%
\pgfpathlineto{\pgfqpoint{3.433110in}{5.522817in}}%
\pgfpathlineto{\pgfqpoint{3.291045in}{5.591989in}}%
\pgfpathclose%
\pgfusepath{fill}%
\end{pgfscope}%
\begin{pgfscope}%
\pgfpathrectangle{\pgfqpoint{0.539299in}{0.078740in}}{\pgfqpoint{7.842520in}{7.842520in}}%
\pgfusepath{clip}%
\pgfsetbuttcap%
\pgfsetroundjoin%
\definecolor{currentfill}{rgb}{0.214298,0.355619,0.551184}%
\pgfsetfillcolor{currentfill}%
\pgfsetlinewidth{0.000000pt}%
\definecolor{currentstroke}{rgb}{0.157729,0.485932,0.558013}%
\pgfsetstrokecolor{currentstroke}%
\pgfsetdash{}{0pt}%
\pgfpathmoveto{\pgfqpoint{2.751870in}{2.294211in}}%
\pgfpathlineto{\pgfqpoint{2.892470in}{2.245647in}}%
\pgfpathlineto{\pgfqpoint{2.966934in}{2.913287in}}%
\pgfpathclose%
\pgfusepath{fill}%
\end{pgfscope}%
\begin{pgfscope}%
\pgfpathrectangle{\pgfqpoint{0.539299in}{0.078740in}}{\pgfqpoint{7.842520in}{7.842520in}}%
\pgfusepath{clip}%
\pgfsetbuttcap%
\pgfsetroundjoin%
\definecolor{currentfill}{rgb}{0.128087,0.647749,0.523491}%
\pgfsetfillcolor{currentfill}%
\pgfsetlinewidth{0.000000pt}%
\definecolor{currentstroke}{rgb}{0.156270,0.489624,0.557936}%
\pgfsetstrokecolor{currentstroke}%
\pgfsetdash{}{0pt}%
\pgfpathmoveto{\pgfqpoint{4.787540in}{3.665841in}}%
\pgfpathlineto{\pgfqpoint{4.726130in}{3.847156in}}%
\pgfpathlineto{\pgfqpoint{4.643709in}{3.889018in}}%
\pgfpathclose%
\pgfusepath{fill}%
\end{pgfscope}%
\begin{pgfscope}%
\pgfpathrectangle{\pgfqpoint{0.539299in}{0.078740in}}{\pgfqpoint{7.842520in}{7.842520in}}%
\pgfusepath{clip}%
\pgfsetbuttcap%
\pgfsetroundjoin%
\definecolor{currentfill}{rgb}{0.185556,0.418570,0.556753}%
\pgfsetfillcolor{currentfill}%
\pgfsetlinewidth{0.000000pt}%
\definecolor{currentstroke}{rgb}{0.154815,0.493313,0.557840}%
\pgfsetstrokecolor{currentstroke}%
\pgfsetdash{}{0pt}%
\pgfpathmoveto{\pgfqpoint{2.684453in}{3.009085in}}%
\pgfpathlineto{\pgfqpoint{2.751870in}{2.294211in}}%
\pgfpathlineto{\pgfqpoint{2.825356in}{2.965660in}}%
\pgfpathclose%
\pgfusepath{fill}%
\end{pgfscope}%
\begin{pgfscope}%
\pgfpathrectangle{\pgfqpoint{0.539299in}{0.078740in}}{\pgfqpoint{7.842520in}{7.842520in}}%
\pgfusepath{clip}%
\pgfsetbuttcap%
\pgfsetroundjoin%
\definecolor{currentfill}{rgb}{0.595839,0.848717,0.243329}%
\pgfsetfillcolor{currentfill}%
\pgfsetlinewidth{0.000000pt}%
\definecolor{currentstroke}{rgb}{0.153364,0.497000,0.557724}%
\pgfsetstrokecolor{currentstroke}%
\pgfsetdash{}{0pt}%
\pgfpathmoveto{\pgfqpoint{2.665187in}{5.062348in}}%
\pgfpathlineto{\pgfqpoint{2.528143in}{4.972174in}}%
\pgfpathlineto{\pgfqpoint{2.444220in}{4.720662in}}%
\pgfpathclose%
\pgfusepath{fill}%
\end{pgfscope}%
\begin{pgfscope}%
\pgfpathrectangle{\pgfqpoint{0.539299in}{0.078740in}}{\pgfqpoint{7.842520in}{7.842520in}}%
\pgfusepath{clip}%
\pgfsetbuttcap%
\pgfsetroundjoin%
\definecolor{currentfill}{rgb}{0.127568,0.566949,0.550556}%
\pgfsetfillcolor{currentfill}%
\pgfsetlinewidth{0.000000pt}%
\definecolor{currentstroke}{rgb}{0.151918,0.500685,0.557587}%
\pgfsetstrokecolor{currentstroke}%
\pgfsetdash{}{0pt}%
\pgfpathmoveto{\pgfqpoint{2.479796in}{3.628915in}}%
\pgfpathlineto{\pgfqpoint{2.405032in}{3.063501in}}%
\pgfpathlineto{\pgfqpoint{2.619520in}{3.617668in}}%
\pgfpathclose%
\pgfusepath{fill}%
\end{pgfscope}%
\begin{pgfscope}%
\pgfpathrectangle{\pgfqpoint{0.539299in}{0.078740in}}{\pgfqpoint{7.842520in}{7.842520in}}%
\pgfusepath{clip}%
\pgfsetbuttcap%
\pgfsetroundjoin%
\definecolor{currentfill}{rgb}{0.945636,0.899815,0.112838}%
\pgfsetfillcolor{currentfill}%
\pgfsetlinewidth{0.000000pt}%
\definecolor{currentstroke}{rgb}{0.150476,0.504369,0.557430}%
\pgfsetstrokecolor{currentstroke}%
\pgfsetdash{}{0pt}%
\pgfpathmoveto{\pgfqpoint{3.062895in}{5.558135in}}%
\pgfpathlineto{\pgfqpoint{2.836696in}{5.416830in}}%
\pgfpathlineto{\pgfqpoint{2.975959in}{5.457547in}}%
\pgfpathclose%
\pgfusepath{fill}%
\end{pgfscope}%
\begin{pgfscope}%
\pgfpathrectangle{\pgfqpoint{0.539299in}{0.078740in}}{\pgfqpoint{7.842520in}{7.842520in}}%
\pgfusepath{clip}%
\pgfsetbuttcap%
\pgfsetroundjoin%
\definecolor{currentfill}{rgb}{0.440137,0.811138,0.340967}%
\pgfsetfillcolor{currentfill}%
\pgfsetlinewidth{0.000000pt}%
\definecolor{currentstroke}{rgb}{0.149039,0.508051,0.557250}%
\pgfsetstrokecolor{currentstroke}%
\pgfsetdash{}{0pt}%
\pgfpathmoveto{\pgfqpoint{2.581236in}{4.802712in}}%
\pgfpathlineto{\pgfqpoint{2.444220in}{4.720662in}}%
\pgfpathlineto{\pgfqpoint{2.361965in}{4.410404in}}%
\pgfpathclose%
\pgfusepath{fill}%
\end{pgfscope}%
\begin{pgfscope}%
\pgfpathrectangle{\pgfqpoint{0.539299in}{0.078740in}}{\pgfqpoint{7.842520in}{7.842520in}}%
\pgfusepath{clip}%
\pgfsetbuttcap%
\pgfsetroundjoin%
\definecolor{currentfill}{rgb}{0.154815,0.493313,0.557840}%
\pgfsetfillcolor{currentfill}%
\pgfsetlinewidth{0.000000pt}%
\definecolor{currentstroke}{rgb}{0.147607,0.511733,0.557049}%
\pgfsetstrokecolor{currentstroke}%
\pgfsetdash{}{0pt}%
\pgfpathmoveto{\pgfqpoint{5.217726in}{3.002518in}}%
\pgfpathlineto{\pgfqpoint{5.074536in}{3.221540in}}%
\pgfpathlineto{\pgfqpoint{5.137323in}{3.030184in}}%
\pgfpathclose%
\pgfusepath{fill}%
\end{pgfscope}%
\begin{pgfscope}%
\pgfpathrectangle{\pgfqpoint{0.539299in}{0.078740in}}{\pgfqpoint{7.842520in}{7.842520in}}%
\pgfusepath{clip}%
\pgfsetbuttcap%
\pgfsetroundjoin%
\definecolor{currentfill}{rgb}{0.153894,0.680203,0.504172}%
\pgfsetfillcolor{currentfill}%
\pgfsetlinewidth{0.000000pt}%
\definecolor{currentstroke}{rgb}{0.146180,0.515413,0.556823}%
\pgfsetstrokecolor{currentstroke}%
\pgfsetdash{}{0pt}%
\pgfpathmoveto{\pgfqpoint{4.643709in}{3.889018in}}%
\pgfpathlineto{\pgfqpoint{4.726130in}{3.847156in}}%
\pgfpathlineto{\pgfqpoint{4.499656in}{4.110842in}}%
\pgfpathclose%
\pgfusepath{fill}%
\end{pgfscope}%
\begin{pgfscope}%
\pgfpathrectangle{\pgfqpoint{0.539299in}{0.078740in}}{\pgfqpoint{7.842520in}{7.842520in}}%
\pgfusepath{clip}%
\pgfsetbuttcap%
\pgfsetroundjoin%
\definecolor{currentfill}{rgb}{0.187231,0.414746,0.556547}%
\pgfsetfillcolor{currentfill}%
\pgfsetlinewidth{0.000000pt}%
\definecolor{currentstroke}{rgb}{0.144759,0.519093,0.556572}%
\pgfsetstrokecolor{currentstroke}%
\pgfsetdash{}{0pt}%
\pgfpathmoveto{\pgfqpoint{5.424487in}{2.595182in}}%
\pgfpathlineto{\pgfqpoint{5.360735in}{2.786461in}}%
\pgfpathlineto{\pgfqpoint{5.281003in}{2.812185in}}%
\pgfpathclose%
\pgfusepath{fill}%
\end{pgfscope}%
\begin{pgfscope}%
\pgfpathrectangle{\pgfqpoint{0.539299in}{0.078740in}}{\pgfqpoint{7.842520in}{7.842520in}}%
\pgfusepath{clip}%
\pgfsetbuttcap%
\pgfsetroundjoin%
\definecolor{currentfill}{rgb}{0.274128,0.199721,0.498911}%
\pgfsetfillcolor{currentfill}%
\pgfsetlinewidth{0.000000pt}%
\definecolor{currentstroke}{rgb}{0.143343,0.522773,0.556295}%
\pgfsetstrokecolor{currentstroke}%
\pgfsetdash{}{0pt}%
\pgfpathmoveto{\pgfqpoint{5.996574in}{1.738442in}}%
\pgfpathlineto{\pgfqpoint{5.853809in}{1.950429in}}%
\pgfpathlineto{\pgfqpoint{5.775575in}{1.941647in}}%
\pgfpathclose%
\pgfusepath{fill}%
\end{pgfscope}%
\begin{pgfscope}%
\pgfpathrectangle{\pgfqpoint{0.539299in}{0.078740in}}{\pgfqpoint{7.842520in}{7.842520in}}%
\pgfusepath{clip}%
\pgfsetbuttcap%
\pgfsetroundjoin%
\definecolor{currentfill}{rgb}{0.288921,0.758394,0.428426}%
\pgfsetfillcolor{currentfill}%
\pgfsetlinewidth{0.000000pt}%
\definecolor{currentstroke}{rgb}{0.141935,0.526453,0.555991}%
\pgfsetstrokecolor{currentstroke}%
\pgfsetdash{}{0pt}%
\pgfpathmoveto{\pgfqpoint{2.418849in}{4.087714in}}%
\pgfpathlineto{\pgfqpoint{2.498985in}{4.479903in}}%
\pgfpathlineto{\pgfqpoint{2.361965in}{4.410404in}}%
\pgfpathclose%
\pgfusepath{fill}%
\end{pgfscope}%
\begin{pgfscope}%
\pgfpathrectangle{\pgfqpoint{0.539299in}{0.078740in}}{\pgfqpoint{7.842520in}{7.842520in}}%
\pgfusepath{clip}%
\pgfsetbuttcap%
\pgfsetroundjoin%
\definecolor{currentfill}{rgb}{0.352360,0.783011,0.392636}%
\pgfsetfillcolor{currentfill}%
\pgfsetlinewidth{0.000000pt}%
\definecolor{currentstroke}{rgb}{0.140536,0.530132,0.555659}%
\pgfsetstrokecolor{currentstroke}%
\pgfsetdash{}{0pt}%
\pgfpathmoveto{\pgfqpoint{4.295430in}{4.515609in}}%
\pgfpathlineto{\pgfqpoint{4.210962in}{4.541645in}}%
\pgfpathlineto{\pgfqpoint{4.439213in}{4.296234in}}%
\pgfpathclose%
\pgfusepath{fill}%
\end{pgfscope}%
\begin{pgfscope}%
\pgfpathrectangle{\pgfqpoint{0.539299in}{0.078740in}}{\pgfqpoint{7.842520in}{7.842520in}}%
\pgfusepath{clip}%
\pgfsetbuttcap%
\pgfsetroundjoin%
\definecolor{currentfill}{rgb}{0.720391,0.870350,0.162603}%
\pgfsetfillcolor{currentfill}%
\pgfsetlinewidth{0.000000pt}%
\definecolor{currentstroke}{rgb}{0.139147,0.533812,0.555298}%
\pgfsetstrokecolor{currentstroke}%
\pgfsetdash{}{0pt}%
\pgfpathmoveto{\pgfqpoint{2.528143in}{4.972174in}}%
\pgfpathlineto{\pgfqpoint{2.665187in}{5.062348in}}%
\pgfpathlineto{\pgfqpoint{2.750456in}{5.265020in}}%
\pgfpathclose%
\pgfusepath{fill}%
\end{pgfscope}%
\begin{pgfscope}%
\pgfpathrectangle{\pgfqpoint{0.539299in}{0.078740in}}{\pgfqpoint{7.842520in}{7.842520in}}%
\pgfusepath{clip}%
\pgfsetbuttcap%
\pgfsetroundjoin%
\definecolor{currentfill}{rgb}{0.128087,0.647749,0.523491}%
\pgfsetfillcolor{currentfill}%
\pgfsetlinewidth{0.000000pt}%
\definecolor{currentstroke}{rgb}{0.137770,0.537492,0.554906}%
\pgfsetstrokecolor{currentstroke}%
\pgfsetdash{}{0pt}%
\pgfpathmoveto{\pgfqpoint{2.479796in}{3.628915in}}%
\pgfpathlineto{\pgfqpoint{2.557549in}{4.110498in}}%
\pgfpathlineto{\pgfqpoint{2.341276in}{3.620065in}}%
\pgfpathclose%
\pgfusepath{fill}%
\end{pgfscope}%
\begin{pgfscope}%
\pgfpathrectangle{\pgfqpoint{0.539299in}{0.078740in}}{\pgfqpoint{7.842520in}{7.842520in}}%
\pgfusepath{clip}%
\pgfsetbuttcap%
\pgfsetroundjoin%
\definecolor{currentfill}{rgb}{0.141935,0.526453,0.555991}%
\pgfsetfillcolor{currentfill}%
\pgfsetlinewidth{0.000000pt}%
\definecolor{currentstroke}{rgb}{0.136408,0.541173,0.554483}%
\pgfsetstrokecolor{currentstroke}%
\pgfsetdash{}{0pt}%
\pgfpathmoveto{\pgfqpoint{2.684453in}{3.009085in}}%
\pgfpathlineto{\pgfqpoint{2.619520in}{3.617668in}}%
\pgfpathlineto{\pgfqpoint{2.544311in}{3.042210in}}%
\pgfpathclose%
\pgfusepath{fill}%
\end{pgfscope}%
\begin{pgfscope}%
\pgfpathrectangle{\pgfqpoint{0.539299in}{0.078740in}}{\pgfqpoint{7.842520in}{7.842520in}}%
\pgfusepath{clip}%
\pgfsetbuttcap%
\pgfsetroundjoin%
\definecolor{currentfill}{rgb}{0.153894,0.680203,0.504172}%
\pgfsetfillcolor{currentfill}%
\pgfsetlinewidth{0.000000pt}%
\definecolor{currentstroke}{rgb}{0.135066,0.544853,0.554029}%
\pgfsetstrokecolor{currentstroke}%
\pgfsetdash{}{0pt}%
\pgfpathmoveto{\pgfqpoint{2.341276in}{3.620065in}}%
\pgfpathlineto{\pgfqpoint{2.557549in}{4.110498in}}%
\pgfpathlineto{\pgfqpoint{2.418849in}{4.087714in}}%
\pgfpathclose%
\pgfusepath{fill}%
\end{pgfscope}%
\begin{pgfscope}%
\pgfpathrectangle{\pgfqpoint{0.539299in}{0.078740in}}{\pgfqpoint{7.842520in}{7.842520in}}%
\pgfusepath{clip}%
\pgfsetbuttcap%
\pgfsetroundjoin%
\definecolor{currentfill}{rgb}{0.216210,0.351535,0.550627}%
\pgfsetfillcolor{currentfill}%
\pgfsetlinewidth{0.000000pt}%
\definecolor{currentstroke}{rgb}{0.133743,0.548535,0.553541}%
\pgfsetstrokecolor{currentstroke}%
\pgfsetdash{}{0pt}%
\pgfpathmoveto{\pgfqpoint{2.966934in}{2.913287in}}%
\pgfpathlineto{\pgfqpoint{2.892470in}{2.245647in}}%
\pgfpathlineto{\pgfqpoint{3.033612in}{2.193144in}}%
\pgfpathclose%
\pgfusepath{fill}%
\end{pgfscope}%
\begin{pgfscope}%
\pgfpathrectangle{\pgfqpoint{0.539299in}{0.078740in}}{\pgfqpoint{7.842520in}{7.842520in}}%
\pgfusepath{clip}%
\pgfsetbuttcap%
\pgfsetroundjoin%
\definecolor{currentfill}{rgb}{0.282884,0.135920,0.453427}%
\pgfsetfillcolor{currentfill}%
\pgfsetlinewidth{0.000000pt}%
\definecolor{currentstroke}{rgb}{0.132444,0.552216,0.553018}%
\pgfsetstrokecolor{currentstroke}%
\pgfsetdash{}{0pt}%
\pgfpathmoveto{\pgfqpoint{5.918659in}{1.713508in}}%
\pgfpathlineto{\pgfqpoint{6.061409in}{1.481685in}}%
\pgfpathlineto{\pgfqpoint{5.996574in}{1.738442in}}%
\pgfpathclose%
\pgfusepath{fill}%
\end{pgfscope}%
\begin{pgfscope}%
\pgfpathrectangle{\pgfqpoint{0.539299in}{0.078740in}}{\pgfqpoint{7.842520in}{7.842520in}}%
\pgfusepath{clip}%
\pgfsetbuttcap%
\pgfsetroundjoin%
\definecolor{currentfill}{rgb}{0.171176,0.452530,0.557965}%
\pgfsetfillcolor{currentfill}%
\pgfsetlinewidth{0.000000pt}%
\definecolor{currentstroke}{rgb}{0.131172,0.555899,0.552459}%
\pgfsetstrokecolor{currentstroke}%
\pgfsetdash{}{0pt}%
\pgfpathmoveto{\pgfqpoint{5.360735in}{2.786461in}}%
\pgfpathlineto{\pgfqpoint{5.137323in}{3.030184in}}%
\pgfpathlineto{\pgfqpoint{5.281003in}{2.812185in}}%
\pgfpathclose%
\pgfusepath{fill}%
\end{pgfscope}%
\begin{pgfscope}%
\pgfpathrectangle{\pgfqpoint{0.539299in}{0.078740in}}{\pgfqpoint{7.842520in}{7.842520in}}%
\pgfusepath{clip}%
\pgfsetbuttcap%
\pgfsetroundjoin%
\definecolor{currentfill}{rgb}{0.185556,0.418570,0.556753}%
\pgfsetfillcolor{currentfill}%
\pgfsetlinewidth{0.000000pt}%
\definecolor{currentstroke}{rgb}{0.129933,0.559582,0.551864}%
\pgfsetstrokecolor{currentstroke}%
\pgfsetdash{}{0pt}%
\pgfpathmoveto{\pgfqpoint{2.966934in}{2.913287in}}%
\pgfpathlineto{\pgfqpoint{2.825356in}{2.965660in}}%
\pgfpathlineto{\pgfqpoint{2.751870in}{2.294211in}}%
\pgfpathclose%
\pgfusepath{fill}%
\end{pgfscope}%
\begin{pgfscope}%
\pgfpathrectangle{\pgfqpoint{0.539299in}{0.078740in}}{\pgfqpoint{7.842520in}{7.842520in}}%
\pgfusepath{clip}%
\pgfsetbuttcap%
\pgfsetroundjoin%
\definecolor{currentfill}{rgb}{0.430983,0.808473,0.346476}%
\pgfsetfillcolor{currentfill}%
\pgfsetlinewidth{0.000000pt}%
\definecolor{currentstroke}{rgb}{0.128729,0.563265,0.551229}%
\pgfsetstrokecolor{currentstroke}%
\pgfsetdash{}{0pt}%
\pgfpathmoveto{\pgfqpoint{4.295430in}{4.515609in}}%
\pgfpathlineto{\pgfqpoint{4.151482in}{4.727445in}}%
\pgfpathlineto{\pgfqpoint{4.210962in}{4.541645in}}%
\pgfpathclose%
\pgfusepath{fill}%
\end{pgfscope}%
\begin{pgfscope}%
\pgfpathrectangle{\pgfqpoint{0.539299in}{0.078740in}}{\pgfqpoint{7.842520in}{7.842520in}}%
\pgfusepath{clip}%
\pgfsetbuttcap%
\pgfsetroundjoin%
\definecolor{currentfill}{rgb}{0.412913,0.803041,0.357269}%
\pgfsetfillcolor{currentfill}%
\pgfsetlinewidth{0.000000pt}%
\definecolor{currentstroke}{rgb}{0.127568,0.566949,0.550556}%
\pgfsetstrokecolor{currentstroke}%
\pgfsetdash{}{0pt}%
\pgfpathmoveto{\pgfqpoint{2.361965in}{4.410404in}}%
\pgfpathlineto{\pgfqpoint{2.498985in}{4.479903in}}%
\pgfpathlineto{\pgfqpoint{2.581236in}{4.802712in}}%
\pgfpathclose%
\pgfusepath{fill}%
\end{pgfscope}%
\begin{pgfscope}%
\pgfpathrectangle{\pgfqpoint{0.539299in}{0.078740in}}{\pgfqpoint{7.842520in}{7.842520in}}%
\pgfusepath{clip}%
\pgfsetbuttcap%
\pgfsetroundjoin%
\definecolor{currentfill}{rgb}{0.128729,0.563265,0.551229}%
\pgfsetfillcolor{currentfill}%
\pgfsetlinewidth{0.000000pt}%
\definecolor{currentstroke}{rgb}{0.126453,0.570633,0.549841}%
\pgfsetstrokecolor{currentstroke}%
\pgfsetdash{}{0pt}%
\pgfpathmoveto{\pgfqpoint{5.074536in}{3.221540in}}%
\pgfpathlineto{\pgfqpoint{4.931146in}{3.442939in}}%
\pgfpathlineto{\pgfqpoint{4.849351in}{3.468201in}}%
\pgfpathclose%
\pgfusepath{fill}%
\end{pgfscope}%
\begin{pgfscope}%
\pgfpathrectangle{\pgfqpoint{0.539299in}{0.078740in}}{\pgfqpoint{7.842520in}{7.842520in}}%
\pgfusepath{clip}%
\pgfsetbuttcap%
\pgfsetroundjoin%
\definecolor{currentfill}{rgb}{0.253935,0.265254,0.529983}%
\pgfsetfillcolor{currentfill}%
\pgfsetlinewidth{0.000000pt}%
\definecolor{currentstroke}{rgb}{0.125394,0.574318,0.549086}%
\pgfsetstrokecolor{currentstroke}%
\pgfsetdash{}{0pt}%
\pgfpathmoveto{\pgfqpoint{5.710883in}{2.164223in}}%
\pgfpathlineto{\pgfqpoint{5.632182in}{2.166073in}}%
\pgfpathlineto{\pgfqpoint{5.853809in}{1.950429in}}%
\pgfpathclose%
\pgfusepath{fill}%
\end{pgfscope}%
\begin{pgfscope}%
\pgfpathrectangle{\pgfqpoint{0.539299in}{0.078740in}}{\pgfqpoint{7.842520in}{7.842520in}}%
\pgfusepath{clip}%
\pgfsetbuttcap%
\pgfsetroundjoin%
\definecolor{currentfill}{rgb}{0.935904,0.898570,0.108131}%
\pgfsetfillcolor{currentfill}%
\pgfsetlinewidth{0.000000pt}%
\definecolor{currentstroke}{rgb}{0.124395,0.578002,0.548287}%
\pgfsetstrokecolor{currentstroke}%
\pgfsetdash{}{0pt}%
\pgfpathmoveto{\pgfqpoint{3.489230in}{5.389845in}}%
\pgfpathlineto{\pgfqpoint{3.576041in}{5.415869in}}%
\pgfpathlineto{\pgfqpoint{3.433110in}{5.522817in}}%
\pgfpathclose%
\pgfusepath{fill}%
\end{pgfscope}%
\begin{pgfscope}%
\pgfpathrectangle{\pgfqpoint{0.539299in}{0.078740in}}{\pgfqpoint{7.842520in}{7.842520in}}%
\pgfusepath{clip}%
\pgfsetbuttcap%
\pgfsetroundjoin%
\definecolor{currentfill}{rgb}{0.585678,0.846661,0.249897}%
\pgfsetfillcolor{currentfill}%
\pgfsetlinewidth{0.000000pt}%
\definecolor{currentstroke}{rgb}{0.123463,0.581687,0.547445}%
\pgfsetstrokecolor{currentstroke}%
\pgfsetdash{}{0pt}%
\pgfpathmoveto{\pgfqpoint{2.581236in}{4.802712in}}%
\pgfpathlineto{\pgfqpoint{2.665187in}{5.062348in}}%
\pgfpathlineto{\pgfqpoint{2.444220in}{4.720662in}}%
\pgfpathclose%
\pgfusepath{fill}%
\end{pgfscope}%
\begin{pgfscope}%
\pgfpathrectangle{\pgfqpoint{0.539299in}{0.078740in}}{\pgfqpoint{7.842520in}{7.842520in}}%
\pgfusepath{clip}%
\pgfsetbuttcap%
\pgfsetroundjoin%
\definecolor{currentfill}{rgb}{0.983868,0.904867,0.136897}%
\pgfsetfillcolor{currentfill}%
\pgfsetlinewidth{0.000000pt}%
\definecolor{currentstroke}{rgb}{0.122606,0.585371,0.546557}%
\pgfsetstrokecolor{currentstroke}%
\pgfsetdash{}{0pt}%
\pgfpathmoveto{\pgfqpoint{3.346062in}{5.485171in}}%
\pgfpathlineto{\pgfqpoint{3.433110in}{5.522817in}}%
\pgfpathlineto{\pgfqpoint{3.203841in}{5.543255in}}%
\pgfpathclose%
\pgfusepath{fill}%
\end{pgfscope}%
\begin{pgfscope}%
\pgfpathrectangle{\pgfqpoint{0.539299in}{0.078740in}}{\pgfqpoint{7.842520in}{7.842520in}}%
\pgfusepath{clip}%
\pgfsetbuttcap%
\pgfsetroundjoin%
\definecolor{currentfill}{rgb}{0.876168,0.891125,0.095250}%
\pgfsetfillcolor{currentfill}%
\pgfsetlinewidth{0.000000pt}%
\definecolor{currentstroke}{rgb}{0.121831,0.589055,0.545623}%
\pgfsetstrokecolor{currentstroke}%
\pgfsetdash{}{0pt}%
\pgfpathmoveto{\pgfqpoint{2.836696in}{5.416830in}}%
\pgfpathlineto{\pgfqpoint{2.750456in}{5.265020in}}%
\pgfpathlineto{\pgfqpoint{2.889650in}{5.309100in}}%
\pgfpathclose%
\pgfusepath{fill}%
\end{pgfscope}%
\begin{pgfscope}%
\pgfpathrectangle{\pgfqpoint{0.539299in}{0.078740in}}{\pgfqpoint{7.842520in}{7.842520in}}%
\pgfusepath{clip}%
\pgfsetbuttcap%
\pgfsetroundjoin%
\definecolor{currentfill}{rgb}{0.259857,0.745492,0.444467}%
\pgfsetfillcolor{currentfill}%
\pgfsetlinewidth{0.000000pt}%
\definecolor{currentstroke}{rgb}{0.121148,0.592739,0.544641}%
\pgfsetstrokecolor{currentstroke}%
\pgfsetdash{}{0pt}%
\pgfpathmoveto{\pgfqpoint{4.499656in}{4.110842in}}%
\pgfpathlineto{\pgfqpoint{4.439213in}{4.296234in}}%
\pgfpathlineto{\pgfqpoint{4.355396in}{4.329235in}}%
\pgfpathclose%
\pgfusepath{fill}%
\end{pgfscope}%
\begin{pgfscope}%
\pgfpathrectangle{\pgfqpoint{0.539299in}{0.078740in}}{\pgfqpoint{7.842520in}{7.842520in}}%
\pgfusepath{clip}%
\pgfsetbuttcap%
\pgfsetroundjoin%
\definecolor{currentfill}{rgb}{0.277134,0.185228,0.489898}%
\pgfsetfillcolor{currentfill}%
\pgfsetlinewidth{0.000000pt}%
\definecolor{currentstroke}{rgb}{0.120565,0.596422,0.543611}%
\pgfsetstrokecolor{currentstroke}%
\pgfsetdash{}{0pt}%
\pgfpathmoveto{\pgfqpoint{5.996574in}{1.738442in}}%
\pgfpathlineto{\pgfqpoint{5.775575in}{1.941647in}}%
\pgfpathlineto{\pgfqpoint{5.918659in}{1.713508in}}%
\pgfpathclose%
\pgfusepath{fill}%
\end{pgfscope}%
\begin{pgfscope}%
\pgfpathrectangle{\pgfqpoint{0.539299in}{0.078740in}}{\pgfqpoint{7.842520in}{7.842520in}}%
\pgfusepath{clip}%
\pgfsetbuttcap%
\pgfsetroundjoin%
\definecolor{currentfill}{rgb}{0.555484,0.840254,0.269281}%
\pgfsetfillcolor{currentfill}%
\pgfsetlinewidth{0.000000pt}%
\definecolor{currentstroke}{rgb}{0.120092,0.600104,0.542530}%
\pgfsetstrokecolor{currentstroke}%
\pgfsetdash{}{0pt}%
\pgfpathmoveto{\pgfqpoint{4.066411in}{4.745025in}}%
\pgfpathlineto{\pgfqpoint{4.151482in}{4.727445in}}%
\pgfpathlineto{\pgfqpoint{4.007443in}{4.927974in}}%
\pgfpathclose%
\pgfusepath{fill}%
\end{pgfscope}%
\begin{pgfscope}%
\pgfpathrectangle{\pgfqpoint{0.539299in}{0.078740in}}{\pgfqpoint{7.842520in}{7.842520in}}%
\pgfusepath{clip}%
\pgfsetbuttcap%
\pgfsetroundjoin%
\definecolor{currentfill}{rgb}{0.916242,0.896091,0.100717}%
\pgfsetfillcolor{currentfill}%
\pgfsetlinewidth{0.000000pt}%
\definecolor{currentstroke}{rgb}{0.119738,0.603785,0.541400}%
\pgfsetstrokecolor{currentstroke}%
\pgfsetdash{}{0pt}%
\pgfpathmoveto{\pgfqpoint{2.975959in}{5.457547in}}%
\pgfpathlineto{\pgfqpoint{2.836696in}{5.416830in}}%
\pgfpathlineto{\pgfqpoint{2.889650in}{5.309100in}}%
\pgfpathclose%
\pgfusepath{fill}%
\end{pgfscope}%
\begin{pgfscope}%
\pgfpathrectangle{\pgfqpoint{0.539299in}{0.078740in}}{\pgfqpoint{7.842520in}{7.842520in}}%
\pgfusepath{clip}%
\pgfsetbuttcap%
\pgfsetroundjoin%
\definecolor{currentfill}{rgb}{0.896320,0.893616,0.096335}%
\pgfsetfillcolor{currentfill}%
\pgfsetlinewidth{0.000000pt}%
\definecolor{currentstroke}{rgb}{0.119512,0.607464,0.540218}%
\pgfsetstrokecolor{currentstroke}%
\pgfsetdash{}{0pt}%
\pgfpathmoveto{\pgfqpoint{3.719555in}{5.277291in}}%
\pgfpathlineto{\pgfqpoint{3.576041in}{5.415869in}}%
\pgfpathlineto{\pgfqpoint{3.489230in}{5.389845in}}%
\pgfpathclose%
\pgfusepath{fill}%
\end{pgfscope}%
\begin{pgfscope}%
\pgfpathrectangle{\pgfqpoint{0.539299in}{0.078740in}}{\pgfqpoint{7.842520in}{7.842520in}}%
\pgfusepath{clip}%
\pgfsetbuttcap%
\pgfsetroundjoin%
\definecolor{currentfill}{rgb}{0.983868,0.904867,0.136897}%
\pgfsetfillcolor{currentfill}%
\pgfsetlinewidth{0.000000pt}%
\definecolor{currentstroke}{rgb}{0.119423,0.611141,0.538982}%
\pgfsetstrokecolor{currentstroke}%
\pgfsetdash{}{0pt}%
\pgfpathmoveto{\pgfqpoint{3.116915in}{5.451034in}}%
\pgfpathlineto{\pgfqpoint{3.203841in}{5.543255in}}%
\pgfpathlineto{\pgfqpoint{3.062895in}{5.558135in}}%
\pgfpathclose%
\pgfusepath{fill}%
\end{pgfscope}%
\begin{pgfscope}%
\pgfpathrectangle{\pgfqpoint{0.539299in}{0.078740in}}{\pgfqpoint{7.842520in}{7.842520in}}%
\pgfusepath{clip}%
\pgfsetbuttcap%
\pgfsetroundjoin%
\definecolor{currentfill}{rgb}{0.237441,0.305202,0.541921}%
\pgfsetfillcolor{currentfill}%
\pgfsetlinewidth{0.000000pt}%
\definecolor{currentstroke}{rgb}{0.119483,0.614817,0.537692}%
\pgfsetstrokecolor{currentstroke}%
\pgfsetdash{}{0pt}%
\pgfpathmoveto{\pgfqpoint{5.567779in}{2.379196in}}%
\pgfpathlineto{\pgfqpoint{5.632182in}{2.166073in}}%
\pgfpathlineto{\pgfqpoint{5.710883in}{2.164223in}}%
\pgfpathclose%
\pgfusepath{fill}%
\end{pgfscope}%
\begin{pgfscope}%
\pgfpathrectangle{\pgfqpoint{0.539299in}{0.078740in}}{\pgfqpoint{7.842520in}{7.842520in}}%
\pgfusepath{clip}%
\pgfsetbuttcap%
\pgfsetroundjoin%
\definecolor{currentfill}{rgb}{0.120565,0.596422,0.543611}%
\pgfsetfillcolor{currentfill}%
\pgfsetlinewidth{0.000000pt}%
\definecolor{currentstroke}{rgb}{0.119699,0.618490,0.536347}%
\pgfsetstrokecolor{currentstroke}%
\pgfsetdash{}{0pt}%
\pgfpathmoveto{\pgfqpoint{4.849351in}{3.468201in}}%
\pgfpathlineto{\pgfqpoint{4.931146in}{3.442939in}}%
\pgfpathlineto{\pgfqpoint{4.787540in}{3.665841in}}%
\pgfpathclose%
\pgfusepath{fill}%
\end{pgfscope}%
\begin{pgfscope}%
\pgfpathrectangle{\pgfqpoint{0.539299in}{0.078740in}}{\pgfqpoint{7.842520in}{7.842520in}}%
\pgfusepath{clip}%
\pgfsetbuttcap%
\pgfsetroundjoin%
\definecolor{currentfill}{rgb}{0.335885,0.777018,0.402049}%
\pgfsetfillcolor{currentfill}%
\pgfsetlinewidth{0.000000pt}%
\definecolor{currentstroke}{rgb}{0.120081,0.622161,0.534946}%
\pgfsetstrokecolor{currentstroke}%
\pgfsetdash{}{0pt}%
\pgfpathmoveto{\pgfqpoint{4.439213in}{4.296234in}}%
\pgfpathlineto{\pgfqpoint{4.210962in}{4.541645in}}%
\pgfpathlineto{\pgfqpoint{4.355396in}{4.329235in}}%
\pgfpathclose%
\pgfusepath{fill}%
\end{pgfscope}%
\begin{pgfscope}%
\pgfpathrectangle{\pgfqpoint{0.539299in}{0.078740in}}{\pgfqpoint{7.842520in}{7.842520in}}%
\pgfusepath{clip}%
\pgfsetbuttcap%
\pgfsetroundjoin%
\definecolor{currentfill}{rgb}{0.974417,0.903590,0.130215}%
\pgfsetfillcolor{currentfill}%
\pgfsetlinewidth{0.000000pt}%
\definecolor{currentstroke}{rgb}{0.120638,0.625828,0.533488}%
\pgfsetstrokecolor{currentstroke}%
\pgfsetdash{}{0pt}%
\pgfpathmoveto{\pgfqpoint{3.116915in}{5.451034in}}%
\pgfpathlineto{\pgfqpoint{3.062895in}{5.558135in}}%
\pgfpathlineto{\pgfqpoint{2.975959in}{5.457547in}}%
\pgfpathclose%
\pgfusepath{fill}%
\end{pgfscope}%
\begin{pgfscope}%
\pgfpathrectangle{\pgfqpoint{0.539299in}{0.078740in}}{\pgfqpoint{7.842520in}{7.842520in}}%
\pgfusepath{clip}%
\pgfsetbuttcap%
\pgfsetroundjoin%
\definecolor{currentfill}{rgb}{0.144759,0.519093,0.556572}%
\pgfsetfillcolor{currentfill}%
\pgfsetlinewidth{0.000000pt}%
\definecolor{currentstroke}{rgb}{0.121380,0.629492,0.531973}%
\pgfsetstrokecolor{currentstroke}%
\pgfsetdash{}{0pt}%
\pgfpathmoveto{\pgfqpoint{5.137323in}{3.030184in}}%
\pgfpathlineto{\pgfqpoint{5.074536in}{3.221540in}}%
\pgfpathlineto{\pgfqpoint{4.993441in}{3.248999in}}%
\pgfpathclose%
\pgfusepath{fill}%
\end{pgfscope}%
\begin{pgfscope}%
\pgfpathrectangle{\pgfqpoint{0.539299in}{0.078740in}}{\pgfqpoint{7.842520in}{7.842520in}}%
\pgfusepath{clip}%
\pgfsetbuttcap%
\pgfsetroundjoin%
\definecolor{currentfill}{rgb}{0.709898,0.868751,0.169257}%
\pgfsetfillcolor{currentfill}%
\pgfsetlinewidth{0.000000pt}%
\definecolor{currentstroke}{rgb}{0.122312,0.633153,0.530398}%
\pgfsetstrokecolor{currentstroke}%
\pgfsetdash{}{0pt}%
\pgfpathmoveto{\pgfqpoint{3.863419in}{5.112875in}}%
\pgfpathlineto{\pgfqpoint{3.777327in}{5.109990in}}%
\pgfpathlineto{\pgfqpoint{4.007443in}{4.927974in}}%
\pgfpathclose%
\pgfusepath{fill}%
\end{pgfscope}%
\begin{pgfscope}%
\pgfpathrectangle{\pgfqpoint{0.539299in}{0.078740in}}{\pgfqpoint{7.842520in}{7.842520in}}%
\pgfusepath{clip}%
\pgfsetbuttcap%
\pgfsetroundjoin%
\definecolor{currentfill}{rgb}{0.783315,0.879285,0.125405}%
\pgfsetfillcolor{currentfill}%
\pgfsetlinewidth{0.000000pt}%
\definecolor{currentstroke}{rgb}{0.123444,0.636809,0.528763}%
\pgfsetstrokecolor{currentstroke}%
\pgfsetdash{}{0pt}%
\pgfpathmoveto{\pgfqpoint{3.777327in}{5.109990in}}%
\pgfpathlineto{\pgfqpoint{3.863419in}{5.112875in}}%
\pgfpathlineto{\pgfqpoint{3.719555in}{5.277291in}}%
\pgfpathclose%
\pgfusepath{fill}%
\end{pgfscope}%
\begin{pgfscope}%
\pgfpathrectangle{\pgfqpoint{0.539299in}{0.078740in}}{\pgfqpoint{7.842520in}{7.842520in}}%
\pgfusepath{clip}%
\pgfsetbuttcap%
\pgfsetroundjoin%
\definecolor{currentfill}{rgb}{0.964894,0.902323,0.123941}%
\pgfsetfillcolor{currentfill}%
\pgfsetlinewidth{0.000000pt}%
\definecolor{currentstroke}{rgb}{0.124780,0.640461,0.527068}%
\pgfsetstrokecolor{currentstroke}%
\pgfsetdash{}{0pt}%
\pgfpathmoveto{\pgfqpoint{3.489230in}{5.389845in}}%
\pgfpathlineto{\pgfqpoint{3.433110in}{5.522817in}}%
\pgfpathlineto{\pgfqpoint{3.346062in}{5.485171in}}%
\pgfpathclose%
\pgfusepath{fill}%
\end{pgfscope}%
\begin{pgfscope}%
\pgfpathrectangle{\pgfqpoint{0.539299in}{0.078740in}}{\pgfqpoint{7.842520in}{7.842520in}}%
\pgfusepath{clip}%
\pgfsetbuttcap%
\pgfsetroundjoin%
\definecolor{currentfill}{rgb}{0.132268,0.655014,0.519661}%
\pgfsetfillcolor{currentfill}%
\pgfsetlinewidth{0.000000pt}%
\definecolor{currentstroke}{rgb}{0.126326,0.644107,0.525311}%
\pgfsetstrokecolor{currentstroke}%
\pgfsetdash{}{0pt}%
\pgfpathmoveto{\pgfqpoint{2.619520in}{3.617668in}}%
\pgfpathlineto{\pgfqpoint{2.557549in}{4.110498in}}%
\pgfpathlineto{\pgfqpoint{2.479796in}{3.628915in}}%
\pgfpathclose%
\pgfusepath{fill}%
\end{pgfscope}%
\begin{pgfscope}%
\pgfpathrectangle{\pgfqpoint{0.539299in}{0.078740in}}{\pgfqpoint{7.842520in}{7.842520in}}%
\pgfusepath{clip}%
\pgfsetbuttcap%
\pgfsetroundjoin%
\definecolor{currentfill}{rgb}{0.258965,0.251537,0.524736}%
\pgfsetfillcolor{currentfill}%
\pgfsetlinewidth{0.000000pt}%
\definecolor{currentstroke}{rgb}{0.128087,0.647749,0.523491}%
\pgfsetstrokecolor{currentstroke}%
\pgfsetdash{}{0pt}%
\pgfpathmoveto{\pgfqpoint{5.853809in}{1.950429in}}%
\pgfpathlineto{\pgfqpoint{5.632182in}{2.166073in}}%
\pgfpathlineto{\pgfqpoint{5.775575in}{1.941647in}}%
\pgfpathclose%
\pgfusepath{fill}%
\end{pgfscope}%
\begin{pgfscope}%
\pgfpathrectangle{\pgfqpoint{0.539299in}{0.078740in}}{\pgfqpoint{7.842520in}{7.842520in}}%
\pgfusepath{clip}%
\pgfsetbuttcap%
\pgfsetroundjoin%
\definecolor{currentfill}{rgb}{0.814576,0.883393,0.110347}%
\pgfsetfillcolor{currentfill}%
\pgfsetlinewidth{0.000000pt}%
\definecolor{currentstroke}{rgb}{0.130067,0.651384,0.521608}%
\pgfsetstrokecolor{currentstroke}%
\pgfsetdash{}{0pt}%
\pgfpathmoveto{\pgfqpoint{2.750456in}{5.265020in}}%
\pgfpathlineto{\pgfqpoint{2.665187in}{5.062348in}}%
\pgfpathlineto{\pgfqpoint{2.889650in}{5.309100in}}%
\pgfpathclose%
\pgfusepath{fill}%
\end{pgfscope}%
\begin{pgfscope}%
\pgfpathrectangle{\pgfqpoint{0.539299in}{0.078740in}}{\pgfqpoint{7.842520in}{7.842520in}}%
\pgfusepath{clip}%
\pgfsetbuttcap%
\pgfsetroundjoin%
\definecolor{currentfill}{rgb}{0.210503,0.363727,0.552206}%
\pgfsetfillcolor{currentfill}%
\pgfsetlinewidth{0.000000pt}%
\definecolor{currentstroke}{rgb}{0.132268,0.655014,0.519661}%
\pgfsetstrokecolor{currentstroke}%
\pgfsetdash{}{0pt}%
\pgfpathmoveto{\pgfqpoint{5.488510in}{2.387296in}}%
\pgfpathlineto{\pgfqpoint{5.567779in}{2.379196in}}%
\pgfpathlineto{\pgfqpoint{5.424487in}{2.595182in}}%
\pgfpathclose%
\pgfusepath{fill}%
\end{pgfscope}%
\begin{pgfscope}%
\pgfpathrectangle{\pgfqpoint{0.539299in}{0.078740in}}{\pgfqpoint{7.842520in}{7.842520in}}%
\pgfusepath{clip}%
\pgfsetbuttcap%
\pgfsetroundjoin%
\definecolor{currentfill}{rgb}{0.131172,0.555899,0.552459}%
\pgfsetfillcolor{currentfill}%
\pgfsetlinewidth{0.000000pt}%
\definecolor{currentstroke}{rgb}{0.134692,0.658636,0.517649}%
\pgfsetstrokecolor{currentstroke}%
\pgfsetdash{}{0pt}%
\pgfpathmoveto{\pgfqpoint{4.993441in}{3.248999in}}%
\pgfpathlineto{\pgfqpoint{5.074536in}{3.221540in}}%
\pgfpathlineto{\pgfqpoint{4.849351in}{3.468201in}}%
\pgfpathclose%
\pgfusepath{fill}%
\end{pgfscope}%
\begin{pgfscope}%
\pgfpathrectangle{\pgfqpoint{0.539299in}{0.078740in}}{\pgfqpoint{7.842520in}{7.842520in}}%
\pgfusepath{clip}%
\pgfsetbuttcap%
\pgfsetroundjoin%
\definecolor{currentfill}{rgb}{0.130067,0.651384,0.521608}%
\pgfsetfillcolor{currentfill}%
\pgfsetlinewidth{0.000000pt}%
\definecolor{currentstroke}{rgb}{0.137339,0.662252,0.515571}%
\pgfsetstrokecolor{currentstroke}%
\pgfsetdash{}{0pt}%
\pgfpathmoveto{\pgfqpoint{4.705051in}{3.687048in}}%
\pgfpathlineto{\pgfqpoint{4.787540in}{3.665841in}}%
\pgfpathlineto{\pgfqpoint{4.643709in}{3.889018in}}%
\pgfpathclose%
\pgfusepath{fill}%
\end{pgfscope}%
\begin{pgfscope}%
\pgfpathrectangle{\pgfqpoint{0.539299in}{0.078740in}}{\pgfqpoint{7.842520in}{7.842520in}}%
\pgfusepath{clip}%
\pgfsetbuttcap%
\pgfsetroundjoin%
\definecolor{currentfill}{rgb}{0.327796,0.773980,0.406640}%
\pgfsetfillcolor{currentfill}%
\pgfsetlinewidth{0.000000pt}%
\definecolor{currentstroke}{rgb}{0.140210,0.665859,0.513427}%
\pgfsetstrokecolor{currentstroke}%
\pgfsetdash{}{0pt}%
\pgfpathmoveto{\pgfqpoint{2.637836in}{4.513404in}}%
\pgfpathlineto{\pgfqpoint{2.498985in}{4.479903in}}%
\pgfpathlineto{\pgfqpoint{2.418849in}{4.087714in}}%
\pgfpathclose%
\pgfusepath{fill}%
\end{pgfscope}%
\begin{pgfscope}%
\pgfpathrectangle{\pgfqpoint{0.539299in}{0.078740in}}{\pgfqpoint{7.842520in}{7.842520in}}%
\pgfusepath{clip}%
\pgfsetbuttcap%
\pgfsetroundjoin%
\definecolor{currentfill}{rgb}{0.487026,0.823929,0.312321}%
\pgfsetfillcolor{currentfill}%
\pgfsetlinewidth{0.000000pt}%
\definecolor{currentstroke}{rgb}{0.143303,0.669459,0.511215}%
\pgfsetstrokecolor{currentstroke}%
\pgfsetdash{}{0pt}%
\pgfpathmoveto{\pgfqpoint{4.210962in}{4.541645in}}%
\pgfpathlineto{\pgfqpoint{4.151482in}{4.727445in}}%
\pgfpathlineto{\pgfqpoint{4.066411in}{4.745025in}}%
\pgfpathclose%
\pgfusepath{fill}%
\end{pgfscope}%
\begin{pgfscope}%
\pgfpathrectangle{\pgfqpoint{0.539299in}{0.078740in}}{\pgfqpoint{7.842520in}{7.842520in}}%
\pgfusepath{clip}%
\pgfsetbuttcap%
\pgfsetroundjoin%
\definecolor{currentfill}{rgb}{0.125394,0.574318,0.549086}%
\pgfsetfillcolor{currentfill}%
\pgfsetlinewidth{0.000000pt}%
\definecolor{currentstroke}{rgb}{0.146616,0.673050,0.508936}%
\pgfsetstrokecolor{currentstroke}%
\pgfsetdash{}{0pt}%
\pgfpathmoveto{\pgfqpoint{2.760273in}{3.588885in}}%
\pgfpathlineto{\pgfqpoint{2.619520in}{3.617668in}}%
\pgfpathlineto{\pgfqpoint{2.684453in}{3.009085in}}%
\pgfpathclose%
\pgfusepath{fill}%
\end{pgfscope}%
\begin{pgfscope}%
\pgfpathrectangle{\pgfqpoint{0.539299in}{0.078740in}}{\pgfqpoint{7.842520in}{7.842520in}}%
\pgfusepath{clip}%
\pgfsetbuttcap%
\pgfsetroundjoin%
\definecolor{currentfill}{rgb}{0.218130,0.347432,0.550038}%
\pgfsetfillcolor{currentfill}%
\pgfsetlinewidth{0.000000pt}%
\definecolor{currentstroke}{rgb}{0.150148,0.676631,0.506589}%
\pgfsetstrokecolor{currentstroke}%
\pgfsetdash{}{0pt}%
\pgfpathmoveto{\pgfqpoint{3.175266in}{2.137225in}}%
\pgfpathlineto{\pgfqpoint{3.251833in}{2.786336in}}%
\pgfpathlineto{\pgfqpoint{3.033612in}{2.193144in}}%
\pgfpathclose%
\pgfusepath{fill}%
\end{pgfscope}%
\begin{pgfscope}%
\pgfpathrectangle{\pgfqpoint{0.539299in}{0.078740in}}{\pgfqpoint{7.842520in}{7.842520in}}%
\pgfusepath{clip}%
\pgfsetbuttcap%
\pgfsetroundjoin%
\definecolor{currentfill}{rgb}{0.274149,0.751988,0.436601}%
\pgfsetfillcolor{currentfill}%
\pgfsetlinewidth{0.000000pt}%
\definecolor{currentstroke}{rgb}{0.153894,0.680203,0.504172}%
\pgfsetstrokecolor{currentstroke}%
\pgfsetdash{}{0pt}%
\pgfpathmoveto{\pgfqpoint{2.418849in}{4.087714in}}%
\pgfpathlineto{\pgfqpoint{2.557549in}{4.110498in}}%
\pgfpathlineto{\pgfqpoint{2.637836in}{4.513404in}}%
\pgfpathclose%
\pgfusepath{fill}%
\end{pgfscope}%
\begin{pgfscope}%
\pgfpathrectangle{\pgfqpoint{0.539299in}{0.078740in}}{\pgfqpoint{7.842520in}{7.842520in}}%
\pgfusepath{clip}%
\pgfsetbuttcap%
\pgfsetroundjoin%
\definecolor{currentfill}{rgb}{0.283229,0.120777,0.440584}%
\pgfsetfillcolor{currentfill}%
\pgfsetlinewidth{0.000000pt}%
\definecolor{currentstroke}{rgb}{0.157851,0.683765,0.501686}%
\pgfsetstrokecolor{currentstroke}%
\pgfsetdash{}{0pt}%
\pgfpathmoveto{\pgfqpoint{5.983060in}{1.432854in}}%
\pgfpathlineto{\pgfqpoint{6.061409in}{1.481685in}}%
\pgfpathlineto{\pgfqpoint{5.918659in}{1.713508in}}%
\pgfpathclose%
\pgfusepath{fill}%
\end{pgfscope}%
\begin{pgfscope}%
\pgfpathrectangle{\pgfqpoint{0.539299in}{0.078740in}}{\pgfqpoint{7.842520in}{7.842520in}}%
\pgfusepath{clip}%
\pgfsetbuttcap%
\pgfsetroundjoin%
\definecolor{currentfill}{rgb}{0.886271,0.892374,0.095374}%
\pgfsetfillcolor{currentfill}%
\pgfsetlinewidth{0.000000pt}%
\definecolor{currentstroke}{rgb}{0.162016,0.687316,0.499129}%
\pgfsetstrokecolor{currentstroke}%
\pgfsetdash{}{0pt}%
\pgfpathmoveto{\pgfqpoint{3.489230in}{5.389845in}}%
\pgfpathlineto{\pgfqpoint{3.633064in}{5.262991in}}%
\pgfpathlineto{\pgfqpoint{3.719555in}{5.277291in}}%
\pgfpathclose%
\pgfusepath{fill}%
\end{pgfscope}%
\begin{pgfscope}%
\pgfpathrectangle{\pgfqpoint{0.539299in}{0.078740in}}{\pgfqpoint{7.842520in}{7.842520in}}%
\pgfusepath{clip}%
\pgfsetbuttcap%
\pgfsetroundjoin%
\definecolor{currentfill}{rgb}{0.141935,0.526453,0.555991}%
\pgfsetfillcolor{currentfill}%
\pgfsetlinewidth{0.000000pt}%
\definecolor{currentstroke}{rgb}{0.166383,0.690856,0.496502}%
\pgfsetstrokecolor{currentstroke}%
\pgfsetdash{}{0pt}%
\pgfpathmoveto{\pgfqpoint{2.684453in}{3.009085in}}%
\pgfpathlineto{\pgfqpoint{2.825356in}{2.965660in}}%
\pgfpathlineto{\pgfqpoint{2.901907in}{3.544823in}}%
\pgfpathclose%
\pgfusepath{fill}%
\end{pgfscope}%
\begin{pgfscope}%
\pgfpathrectangle{\pgfqpoint{0.539299in}{0.078740in}}{\pgfqpoint{7.842520in}{7.842520in}}%
\pgfusepath{clip}%
\pgfsetbuttcap%
\pgfsetroundjoin%
\definecolor{currentfill}{rgb}{0.616293,0.852709,0.230052}%
\pgfsetfillcolor{currentfill}%
\pgfsetlinewidth{0.000000pt}%
\definecolor{currentstroke}{rgb}{0.170948,0.694384,0.493803}%
\pgfsetstrokecolor{currentstroke}%
\pgfsetdash{}{0pt}%
\pgfpathmoveto{\pgfqpoint{4.007443in}{4.927974in}}%
\pgfpathlineto{\pgfqpoint{3.921827in}{4.935824in}}%
\pgfpathlineto{\pgfqpoint{4.066411in}{4.745025in}}%
\pgfpathclose%
\pgfusepath{fill}%
\end{pgfscope}%
\begin{pgfscope}%
\pgfpathrectangle{\pgfqpoint{0.539299in}{0.078740in}}{\pgfqpoint{7.842520in}{7.842520in}}%
\pgfusepath{clip}%
\pgfsetbuttcap%
\pgfsetroundjoin%
\definecolor{currentfill}{rgb}{0.187231,0.414746,0.556547}%
\pgfsetfillcolor{currentfill}%
\pgfsetlinewidth{0.000000pt}%
\definecolor{currentstroke}{rgb}{0.175707,0.697900,0.491033}%
\pgfsetstrokecolor{currentstroke}%
\pgfsetdash{}{0pt}%
\pgfpathmoveto{\pgfqpoint{3.033612in}{2.193144in}}%
\pgfpathlineto{\pgfqpoint{3.109114in}{2.853159in}}%
\pgfpathlineto{\pgfqpoint{2.966934in}{2.913287in}}%
\pgfpathclose%
\pgfusepath{fill}%
\end{pgfscope}%
\begin{pgfscope}%
\pgfpathrectangle{\pgfqpoint{0.539299in}{0.078740in}}{\pgfqpoint{7.842520in}{7.842520in}}%
\pgfusepath{clip}%
\pgfsetbuttcap%
\pgfsetroundjoin%
\definecolor{currentfill}{rgb}{0.458674,0.816363,0.329727}%
\pgfsetfillcolor{currentfill}%
\pgfsetlinewidth{0.000000pt}%
\definecolor{currentstroke}{rgb}{0.180653,0.701402,0.488189}%
\pgfsetstrokecolor{currentstroke}%
\pgfsetdash{}{0pt}%
\pgfpathmoveto{\pgfqpoint{2.581236in}{4.802712in}}%
\pgfpathlineto{\pgfqpoint{2.498985in}{4.479903in}}%
\pgfpathlineto{\pgfqpoint{2.637836in}{4.513404in}}%
\pgfpathclose%
\pgfusepath{fill}%
\end{pgfscope}%
\begin{pgfscope}%
\pgfpathrectangle{\pgfqpoint{0.539299in}{0.078740in}}{\pgfqpoint{7.842520in}{7.842520in}}%
\pgfusepath{clip}%
\pgfsetbuttcap%
\pgfsetroundjoin%
\definecolor{currentfill}{rgb}{0.688944,0.865448,0.182725}%
\pgfsetfillcolor{currentfill}%
\pgfsetlinewidth{0.000000pt}%
\definecolor{currentstroke}{rgb}{0.185783,0.704891,0.485273}%
\pgfsetstrokecolor{currentstroke}%
\pgfsetdash{}{0pt}%
\pgfpathmoveto{\pgfqpoint{3.777327in}{5.109990in}}%
\pgfpathlineto{\pgfqpoint{3.921827in}{4.935824in}}%
\pgfpathlineto{\pgfqpoint{4.007443in}{4.927974in}}%
\pgfpathclose%
\pgfusepath{fill}%
\end{pgfscope}%
\begin{pgfscope}%
\pgfpathrectangle{\pgfqpoint{0.539299in}{0.078740in}}{\pgfqpoint{7.842520in}{7.842520in}}%
\pgfusepath{clip}%
\pgfsetbuttcap%
\pgfsetroundjoin%
\definecolor{currentfill}{rgb}{0.180653,0.701402,0.488189}%
\pgfsetfillcolor{currentfill}%
\pgfsetlinewidth{0.000000pt}%
\definecolor{currentstroke}{rgb}{0.191090,0.708366,0.482284}%
\pgfsetstrokecolor{currentstroke}%
\pgfsetdash{}{0pt}%
\pgfpathmoveto{\pgfqpoint{4.499656in}{4.110842in}}%
\pgfpathlineto{\pgfqpoint{4.560545in}{3.904435in}}%
\pgfpathlineto{\pgfqpoint{4.643709in}{3.889018in}}%
\pgfpathclose%
\pgfusepath{fill}%
\end{pgfscope}%
\begin{pgfscope}%
\pgfpathrectangle{\pgfqpoint{0.539299in}{0.078740in}}{\pgfqpoint{7.842520in}{7.842520in}}%
\pgfusepath{clip}%
\pgfsetbuttcap%
\pgfsetroundjoin%
\definecolor{currentfill}{rgb}{0.835270,0.886029,0.102646}%
\pgfsetfillcolor{currentfill}%
\pgfsetlinewidth{0.000000pt}%
\definecolor{currentstroke}{rgb}{0.196571,0.711827,0.479221}%
\pgfsetstrokecolor{currentstroke}%
\pgfsetdash{}{0pt}%
\pgfpathmoveto{\pgfqpoint{3.719555in}{5.277291in}}%
\pgfpathlineto{\pgfqpoint{3.633064in}{5.262991in}}%
\pgfpathlineto{\pgfqpoint{3.777327in}{5.109990in}}%
\pgfpathclose%
\pgfusepath{fill}%
\end{pgfscope}%
\begin{pgfscope}%
\pgfpathrectangle{\pgfqpoint{0.539299in}{0.078740in}}{\pgfqpoint{7.842520in}{7.842520in}}%
\pgfusepath{clip}%
\pgfsetbuttcap%
\pgfsetroundjoin%
\definecolor{currentfill}{rgb}{0.223925,0.334994,0.548053}%
\pgfsetfillcolor{currentfill}%
\pgfsetlinewidth{0.000000pt}%
\definecolor{currentstroke}{rgb}{0.202219,0.715272,0.476084}%
\pgfsetstrokecolor{currentstroke}%
\pgfsetdash{}{0pt}%
\pgfpathmoveto{\pgfqpoint{5.567779in}{2.379196in}}%
\pgfpathlineto{\pgfqpoint{5.488510in}{2.387296in}}%
\pgfpathlineto{\pgfqpoint{5.632182in}{2.166073in}}%
\pgfpathclose%
\pgfusepath{fill}%
\end{pgfscope}%
\begin{pgfscope}%
\pgfpathrectangle{\pgfqpoint{0.539299in}{0.078740in}}{\pgfqpoint{7.842520in}{7.842520in}}%
\pgfusepath{clip}%
\pgfsetbuttcap%
\pgfsetroundjoin%
\definecolor{currentfill}{rgb}{0.945636,0.899815,0.112838}%
\pgfsetfillcolor{currentfill}%
\pgfsetlinewidth{0.000000pt}%
\definecolor{currentstroke}{rgb}{0.208030,0.718701,0.472873}%
\pgfsetstrokecolor{currentstroke}%
\pgfsetdash{}{0pt}%
\pgfpathmoveto{\pgfqpoint{2.975959in}{5.457547in}}%
\pgfpathlineto{\pgfqpoint{2.889650in}{5.309100in}}%
\pgfpathlineto{\pgfqpoint{3.116915in}{5.451034in}}%
\pgfpathclose%
\pgfusepath{fill}%
\end{pgfscope}%
\begin{pgfscope}%
\pgfpathrectangle{\pgfqpoint{0.539299in}{0.078740in}}{\pgfqpoint{7.842520in}{7.842520in}}%
\pgfusepath{clip}%
\pgfsetbuttcap%
\pgfsetroundjoin%
\definecolor{currentfill}{rgb}{0.179019,0.433756,0.557430}%
\pgfsetfillcolor{currentfill}%
\pgfsetlinewidth{0.000000pt}%
\definecolor{currentstroke}{rgb}{0.214000,0.722114,0.469588}%
\pgfsetstrokecolor{currentstroke}%
\pgfsetdash{}{0pt}%
\pgfpathmoveto{\pgfqpoint{5.281003in}{2.812185in}}%
\pgfpathlineto{\pgfqpoint{5.200440in}{2.822813in}}%
\pgfpathlineto{\pgfqpoint{5.424487in}{2.595182in}}%
\pgfpathclose%
\pgfusepath{fill}%
\end{pgfscope}%
\begin{pgfscope}%
\pgfpathrectangle{\pgfqpoint{0.539299in}{0.078740in}}{\pgfqpoint{7.842520in}{7.842520in}}%
\pgfusepath{clip}%
\pgfsetbuttcap%
\pgfsetroundjoin%
\definecolor{currentfill}{rgb}{0.636902,0.856542,0.216620}%
\pgfsetfillcolor{currentfill}%
\pgfsetlinewidth{0.000000pt}%
\definecolor{currentstroke}{rgb}{0.220124,0.725509,0.466226}%
\pgfsetstrokecolor{currentstroke}%
\pgfsetdash{}{0pt}%
\pgfpathmoveto{\pgfqpoint{2.720220in}{4.843367in}}%
\pgfpathlineto{\pgfqpoint{2.665187in}{5.062348in}}%
\pgfpathlineto{\pgfqpoint{2.581236in}{4.802712in}}%
\pgfpathclose%
\pgfusepath{fill}%
\end{pgfscope}%
\begin{pgfscope}%
\pgfpathrectangle{\pgfqpoint{0.539299in}{0.078740in}}{\pgfqpoint{7.842520in}{7.842520in}}%
\pgfusepath{clip}%
\pgfsetbuttcap%
\pgfsetroundjoin%
\definecolor{currentfill}{rgb}{0.983868,0.904867,0.136897}%
\pgfsetfillcolor{currentfill}%
\pgfsetlinewidth{0.000000pt}%
\definecolor{currentstroke}{rgb}{0.226397,0.728888,0.462789}%
\pgfsetstrokecolor{currentstroke}%
\pgfsetdash{}{0pt}%
\pgfpathmoveto{\pgfqpoint{3.346062in}{5.485171in}}%
\pgfpathlineto{\pgfqpoint{3.203841in}{5.543255in}}%
\pgfpathlineto{\pgfqpoint{3.259203in}{5.402997in}}%
\pgfpathclose%
\pgfusepath{fill}%
\end{pgfscope}%
\begin{pgfscope}%
\pgfpathrectangle{\pgfqpoint{0.539299in}{0.078740in}}{\pgfqpoint{7.842520in}{7.842520in}}%
\pgfusepath{clip}%
\pgfsetbuttcap%
\pgfsetroundjoin%
\definecolor{currentfill}{rgb}{0.120638,0.625828,0.533488}%
\pgfsetfillcolor{currentfill}%
\pgfsetlinewidth{0.000000pt}%
\definecolor{currentstroke}{rgb}{0.232815,0.732247,0.459277}%
\pgfsetstrokecolor{currentstroke}%
\pgfsetdash{}{0pt}%
\pgfpathmoveto{\pgfqpoint{4.787540in}{3.665841in}}%
\pgfpathlineto{\pgfqpoint{4.705051in}{3.687048in}}%
\pgfpathlineto{\pgfqpoint{4.849351in}{3.468201in}}%
\pgfpathclose%
\pgfusepath{fill}%
\end{pgfscope}%
\begin{pgfscope}%
\pgfpathrectangle{\pgfqpoint{0.539299in}{0.078740in}}{\pgfqpoint{7.842520in}{7.842520in}}%
\pgfusepath{clip}%
\pgfsetbuttcap%
\pgfsetroundjoin%
\definecolor{currentfill}{rgb}{0.804182,0.882046,0.114965}%
\pgfsetfillcolor{currentfill}%
\pgfsetlinewidth{0.000000pt}%
\definecolor{currentstroke}{rgb}{0.239374,0.735588,0.455688}%
\pgfsetstrokecolor{currentstroke}%
\pgfsetdash{}{0pt}%
\pgfpathmoveto{\pgfqpoint{2.889650in}{5.309100in}}%
\pgfpathlineto{\pgfqpoint{2.665187in}{5.062348in}}%
\pgfpathlineto{\pgfqpoint{2.804286in}{5.106496in}}%
\pgfpathclose%
\pgfusepath{fill}%
\end{pgfscope}%
\begin{pgfscope}%
\pgfpathrectangle{\pgfqpoint{0.539299in}{0.078740in}}{\pgfqpoint{7.842520in}{7.842520in}}%
\pgfusepath{clip}%
\pgfsetbuttcap%
\pgfsetroundjoin%
\definecolor{currentfill}{rgb}{0.983868,0.904867,0.136897}%
\pgfsetfillcolor{currentfill}%
\pgfsetlinewidth{0.000000pt}%
\definecolor{currentstroke}{rgb}{0.246070,0.738910,0.452024}%
\pgfsetstrokecolor{currentstroke}%
\pgfsetdash{}{0pt}%
\pgfpathmoveto{\pgfqpoint{3.203841in}{5.543255in}}%
\pgfpathlineto{\pgfqpoint{3.116915in}{5.451034in}}%
\pgfpathlineto{\pgfqpoint{3.259203in}{5.402997in}}%
\pgfpathclose%
\pgfusepath{fill}%
\end{pgfscope}%
\begin{pgfscope}%
\pgfpathrectangle{\pgfqpoint{0.539299in}{0.078740in}}{\pgfqpoint{7.842520in}{7.842520in}}%
\pgfusepath{clip}%
\pgfsetbuttcap%
\pgfsetroundjoin%
\definecolor{currentfill}{rgb}{0.175707,0.697900,0.491033}%
\pgfsetfillcolor{currentfill}%
\pgfsetlinewidth{0.000000pt}%
\definecolor{currentstroke}{rgb}{0.252899,0.742211,0.448284}%
\pgfsetstrokecolor{currentstroke}%
\pgfsetdash{}{0pt}%
\pgfpathmoveto{\pgfqpoint{2.619520in}{3.617668in}}%
\pgfpathlineto{\pgfqpoint{2.697635in}{4.107094in}}%
\pgfpathlineto{\pgfqpoint{2.557549in}{4.110498in}}%
\pgfpathclose%
\pgfusepath{fill}%
\end{pgfscope}%
\begin{pgfscope}%
\pgfpathrectangle{\pgfqpoint{0.539299in}{0.078740in}}{\pgfqpoint{7.842520in}{7.842520in}}%
\pgfusepath{clip}%
\pgfsetbuttcap%
\pgfsetroundjoin%
\definecolor{currentfill}{rgb}{0.163625,0.471133,0.558148}%
\pgfsetfillcolor{currentfill}%
\pgfsetlinewidth{0.000000pt}%
\definecolor{currentstroke}{rgb}{0.259857,0.745492,0.444467}%
\pgfsetstrokecolor{currentstroke}%
\pgfsetdash{}{0pt}%
\pgfpathmoveto{\pgfqpoint{5.137323in}{3.030184in}}%
\pgfpathlineto{\pgfqpoint{5.200440in}{2.822813in}}%
\pgfpathlineto{\pgfqpoint{5.281003in}{2.812185in}}%
\pgfpathclose%
\pgfusepath{fill}%
\end{pgfscope}%
\begin{pgfscope}%
\pgfpathrectangle{\pgfqpoint{0.539299in}{0.078740in}}{\pgfqpoint{7.842520in}{7.842520in}}%
\pgfusepath{clip}%
\pgfsetbuttcap%
\pgfsetroundjoin%
\definecolor{currentfill}{rgb}{0.124395,0.578002,0.548287}%
\pgfsetfillcolor{currentfill}%
\pgfsetlinewidth{0.000000pt}%
\definecolor{currentstroke}{rgb}{0.266941,0.748751,0.440573}%
\pgfsetstrokecolor{currentstroke}%
\pgfsetdash{}{0pt}%
\pgfpathmoveto{\pgfqpoint{2.684453in}{3.009085in}}%
\pgfpathlineto{\pgfqpoint{2.901907in}{3.544823in}}%
\pgfpathlineto{\pgfqpoint{2.760273in}{3.588885in}}%
\pgfpathclose%
\pgfusepath{fill}%
\end{pgfscope}%
\begin{pgfscope}%
\pgfpathrectangle{\pgfqpoint{0.539299in}{0.078740in}}{\pgfqpoint{7.842520in}{7.842520in}}%
\pgfusepath{clip}%
\pgfsetbuttcap%
\pgfsetroundjoin%
\definecolor{currentfill}{rgb}{0.197636,0.391528,0.554969}%
\pgfsetfillcolor{currentfill}%
\pgfsetlinewidth{0.000000pt}%
\definecolor{currentstroke}{rgb}{0.274149,0.751988,0.436601}%
\pgfsetstrokecolor{currentstroke}%
\pgfsetdash{}{0pt}%
\pgfpathmoveto{\pgfqpoint{5.424487in}{2.595182in}}%
\pgfpathlineto{\pgfqpoint{5.344589in}{2.606000in}}%
\pgfpathlineto{\pgfqpoint{5.488510in}{2.387296in}}%
\pgfpathclose%
\pgfusepath{fill}%
\end{pgfscope}%
\begin{pgfscope}%
\pgfpathrectangle{\pgfqpoint{0.539299in}{0.078740in}}{\pgfqpoint{7.842520in}{7.842520in}}%
\pgfusepath{clip}%
\pgfsetbuttcap%
\pgfsetroundjoin%
\definecolor{currentfill}{rgb}{0.188923,0.410910,0.556326}%
\pgfsetfillcolor{currentfill}%
\pgfsetlinewidth{0.000000pt}%
\definecolor{currentstroke}{rgb}{0.281477,0.755203,0.432552}%
\pgfsetstrokecolor{currentstroke}%
\pgfsetdash{}{0pt}%
\pgfpathmoveto{\pgfqpoint{3.033612in}{2.193144in}}%
\pgfpathlineto{\pgfqpoint{3.251833in}{2.786336in}}%
\pgfpathlineto{\pgfqpoint{3.109114in}{2.853159in}}%
\pgfpathclose%
\pgfusepath{fill}%
\end{pgfscope}%
\begin{pgfscope}%
\pgfpathrectangle{\pgfqpoint{0.539299in}{0.078740in}}{\pgfqpoint{7.842520in}{7.842520in}}%
\pgfusepath{clip}%
\pgfsetbuttcap%
\pgfsetroundjoin%
\definecolor{currentfill}{rgb}{0.296479,0.761561,0.424223}%
\pgfsetfillcolor{currentfill}%
\pgfsetlinewidth{0.000000pt}%
\definecolor{currentstroke}{rgb}{0.288921,0.758394,0.428426}%
\pgfsetstrokecolor{currentstroke}%
\pgfsetdash{}{0pt}%
\pgfpathmoveto{\pgfqpoint{4.355396in}{4.329235in}}%
\pgfpathlineto{\pgfqpoint{4.270980in}{4.328400in}}%
\pgfpathlineto{\pgfqpoint{4.499656in}{4.110842in}}%
\pgfpathclose%
\pgfusepath{fill}%
\end{pgfscope}%
\begin{pgfscope}%
\pgfpathrectangle{\pgfqpoint{0.539299in}{0.078740in}}{\pgfqpoint{7.842520in}{7.842520in}}%
\pgfusepath{clip}%
\pgfsetbuttcap%
\pgfsetroundjoin%
\definecolor{currentfill}{rgb}{0.150148,0.676631,0.506589}%
\pgfsetfillcolor{currentfill}%
\pgfsetlinewidth{0.000000pt}%
\definecolor{currentstroke}{rgb}{0.296479,0.761561,0.424223}%
\pgfsetstrokecolor{currentstroke}%
\pgfsetdash{}{0pt}%
\pgfpathmoveto{\pgfqpoint{4.643709in}{3.889018in}}%
\pgfpathlineto{\pgfqpoint{4.560545in}{3.904435in}}%
\pgfpathlineto{\pgfqpoint{4.705051in}{3.687048in}}%
\pgfpathclose%
\pgfusepath{fill}%
\end{pgfscope}%
\begin{pgfscope}%
\pgfpathrectangle{\pgfqpoint{0.539299in}{0.078740in}}{\pgfqpoint{7.842520in}{7.842520in}}%
\pgfusepath{clip}%
\pgfsetbuttcap%
\pgfsetroundjoin%
\definecolor{currentfill}{rgb}{0.709898,0.868751,0.169257}%
\pgfsetfillcolor{currentfill}%
\pgfsetlinewidth{0.000000pt}%
\definecolor{currentstroke}{rgb}{0.304148,0.764704,0.419943}%
\pgfsetstrokecolor{currentstroke}%
\pgfsetdash{}{0pt}%
\pgfpathmoveto{\pgfqpoint{2.804286in}{5.106496in}}%
\pgfpathlineto{\pgfqpoint{2.665187in}{5.062348in}}%
\pgfpathlineto{\pgfqpoint{2.720220in}{4.843367in}}%
\pgfpathclose%
\pgfusepath{fill}%
\end{pgfscope}%
\begin{pgfscope}%
\pgfpathrectangle{\pgfqpoint{0.539299in}{0.078740in}}{\pgfqpoint{7.842520in}{7.842520in}}%
\pgfusepath{clip}%
\pgfsetbuttcap%
\pgfsetroundjoin%
\definecolor{currentfill}{rgb}{0.964894,0.902323,0.123941}%
\pgfsetfillcolor{currentfill}%
\pgfsetlinewidth{0.000000pt}%
\definecolor{currentstroke}{rgb}{0.311925,0.767822,0.415586}%
\pgfsetstrokecolor{currentstroke}%
\pgfsetdash{}{0pt}%
\pgfpathmoveto{\pgfqpoint{3.259203in}{5.402997in}}%
\pgfpathlineto{\pgfqpoint{3.489230in}{5.389845in}}%
\pgfpathlineto{\pgfqpoint{3.346062in}{5.485171in}}%
\pgfpathclose%
\pgfusepath{fill}%
\end{pgfscope}%
\begin{pgfscope}%
\pgfpathrectangle{\pgfqpoint{0.539299in}{0.078740in}}{\pgfqpoint{7.842520in}{7.842520in}}%
\pgfusepath{clip}%
\pgfsetbuttcap%
\pgfsetroundjoin%
\definecolor{currentfill}{rgb}{0.545524,0.838039,0.275626}%
\pgfsetfillcolor{currentfill}%
\pgfsetlinewidth{0.000000pt}%
\definecolor{currentstroke}{rgb}{0.319809,0.770914,0.411152}%
\pgfsetstrokecolor{currentstroke}%
\pgfsetdash{}{0pt}%
\pgfpathmoveto{\pgfqpoint{2.637836in}{4.513404in}}%
\pgfpathlineto{\pgfqpoint{2.720220in}{4.843367in}}%
\pgfpathlineto{\pgfqpoint{2.581236in}{4.802712in}}%
\pgfpathclose%
\pgfusepath{fill}%
\end{pgfscope}%
\begin{pgfscope}%
\pgfpathrectangle{\pgfqpoint{0.539299in}{0.078740in}}{\pgfqpoint{7.842520in}{7.842520in}}%
\pgfusepath{clip}%
\pgfsetbuttcap%
\pgfsetroundjoin%
\definecolor{currentfill}{rgb}{0.221989,0.339161,0.548752}%
\pgfsetfillcolor{currentfill}%
\pgfsetlinewidth{0.000000pt}%
\definecolor{currentstroke}{rgb}{0.327796,0.773980,0.406640}%
\pgfsetstrokecolor{currentstroke}%
\pgfsetdash{}{0pt}%
\pgfpathmoveto{\pgfqpoint{3.395040in}{2.713747in}}%
\pgfpathlineto{\pgfqpoint{3.175266in}{2.137225in}}%
\pgfpathlineto{\pgfqpoint{3.317407in}{2.078347in}}%
\pgfpathclose%
\pgfusepath{fill}%
\end{pgfscope}%
\begin{pgfscope}%
\pgfpathrectangle{\pgfqpoint{0.539299in}{0.078740in}}{\pgfqpoint{7.842520in}{7.842520in}}%
\pgfusepath{clip}%
\pgfsetbuttcap%
\pgfsetroundjoin%
\definecolor{currentfill}{rgb}{0.267968,0.223549,0.512008}%
\pgfsetfillcolor{currentfill}%
\pgfsetlinewidth{0.000000pt}%
\definecolor{currentstroke}{rgb}{0.335885,0.777018,0.402049}%
\pgfsetstrokecolor{currentstroke}%
\pgfsetdash{}{0pt}%
\pgfpathmoveto{\pgfqpoint{5.918659in}{1.713508in}}%
\pgfpathlineto{\pgfqpoint{5.775575in}{1.941647in}}%
\pgfpathlineto{\pgfqpoint{5.696616in}{1.925940in}}%
\pgfpathclose%
\pgfusepath{fill}%
\end{pgfscope}%
\begin{pgfscope}%
\pgfpathrectangle{\pgfqpoint{0.539299in}{0.078740in}}{\pgfqpoint{7.842520in}{7.842520in}}%
\pgfusepath{clip}%
\pgfsetbuttcap%
\pgfsetroundjoin%
\definecolor{currentfill}{rgb}{0.182256,0.426184,0.557120}%
\pgfsetfillcolor{currentfill}%
\pgfsetlinewidth{0.000000pt}%
\definecolor{currentstroke}{rgb}{0.344074,0.780029,0.397381}%
\pgfsetstrokecolor{currentstroke}%
\pgfsetdash{}{0pt}%
\pgfpathmoveto{\pgfqpoint{5.200440in}{2.822813in}}%
\pgfpathlineto{\pgfqpoint{5.344589in}{2.606000in}}%
\pgfpathlineto{\pgfqpoint{5.424487in}{2.595182in}}%
\pgfpathclose%
\pgfusepath{fill}%
\end{pgfscope}%
\begin{pgfscope}%
\pgfpathrectangle{\pgfqpoint{0.539299in}{0.078740in}}{\pgfqpoint{7.842520in}{7.842520in}}%
\pgfusepath{clip}%
\pgfsetbuttcap%
\pgfsetroundjoin%
\definecolor{currentfill}{rgb}{0.296479,0.761561,0.424223}%
\pgfsetfillcolor{currentfill}%
\pgfsetlinewidth{0.000000pt}%
\definecolor{currentstroke}{rgb}{0.352360,0.783011,0.392636}%
\pgfsetstrokecolor{currentstroke}%
\pgfsetdash{}{0pt}%
\pgfpathmoveto{\pgfqpoint{2.557549in}{4.110498in}}%
\pgfpathlineto{\pgfqpoint{2.697635in}{4.107094in}}%
\pgfpathlineto{\pgfqpoint{2.637836in}{4.513404in}}%
\pgfpathclose%
\pgfusepath{fill}%
\end{pgfscope}%
\begin{pgfscope}%
\pgfpathrectangle{\pgfqpoint{0.539299in}{0.078740in}}{\pgfqpoint{7.842520in}{7.842520in}}%
\pgfusepath{clip}%
\pgfsetbuttcap%
\pgfsetroundjoin%
\definecolor{currentfill}{rgb}{0.281887,0.150881,0.465405}%
\pgfsetfillcolor{currentfill}%
\pgfsetlinewidth{0.000000pt}%
\definecolor{currentstroke}{rgb}{0.360741,0.785964,0.387814}%
\pgfsetstrokecolor{currentstroke}%
\pgfsetdash{}{0pt}%
\pgfpathmoveto{\pgfqpoint{5.918659in}{1.713508in}}%
\pgfpathlineto{\pgfqpoint{5.840111in}{1.685406in}}%
\pgfpathlineto{\pgfqpoint{5.983060in}{1.432854in}}%
\pgfpathclose%
\pgfusepath{fill}%
\end{pgfscope}%
\begin{pgfscope}%
\pgfpathrectangle{\pgfqpoint{0.539299in}{0.078740in}}{\pgfqpoint{7.842520in}{7.842520in}}%
\pgfusepath{clip}%
\pgfsetbuttcap%
\pgfsetroundjoin%
\definecolor{currentfill}{rgb}{0.369214,0.788888,0.382914}%
\pgfsetfillcolor{currentfill}%
\pgfsetlinewidth{0.000000pt}%
\definecolor{currentstroke}{rgb}{0.369214,0.788888,0.382914}%
\pgfsetstrokecolor{currentstroke}%
\pgfsetdash{}{0pt}%
\pgfpathmoveto{\pgfqpoint{4.210962in}{4.541645in}}%
\pgfpathlineto{\pgfqpoint{4.270980in}{4.328400in}}%
\pgfpathlineto{\pgfqpoint{4.355396in}{4.329235in}}%
\pgfpathclose%
\pgfusepath{fill}%
\end{pgfscope}%
\begin{pgfscope}%
\pgfpathrectangle{\pgfqpoint{0.539299in}{0.078740in}}{\pgfqpoint{7.842520in}{7.842520in}}%
\pgfusepath{clip}%
\pgfsetbuttcap%
\pgfsetroundjoin%
\definecolor{currentfill}{rgb}{0.141935,0.526453,0.555991}%
\pgfsetfillcolor{currentfill}%
\pgfsetlinewidth{0.000000pt}%
\definecolor{currentstroke}{rgb}{0.377779,0.791781,0.377939}%
\pgfsetstrokecolor{currentstroke}%
\pgfsetdash{}{0pt}%
\pgfpathmoveto{\pgfqpoint{3.044293in}{3.487491in}}%
\pgfpathlineto{\pgfqpoint{2.825356in}{2.965660in}}%
\pgfpathlineto{\pgfqpoint{2.966934in}{2.913287in}}%
\pgfpathclose%
\pgfusepath{fill}%
\end{pgfscope}%
\begin{pgfscope}%
\pgfpathrectangle{\pgfqpoint{0.539299in}{0.078740in}}{\pgfqpoint{7.842520in}{7.842520in}}%
\pgfusepath{clip}%
\pgfsetbuttcap%
\pgfsetroundjoin%
\definecolor{currentfill}{rgb}{0.935904,0.898570,0.108131}%
\pgfsetfillcolor{currentfill}%
\pgfsetlinewidth{0.000000pt}%
\definecolor{currentstroke}{rgb}{0.386433,0.794644,0.372886}%
\pgfsetstrokecolor{currentstroke}%
\pgfsetdash{}{0pt}%
\pgfpathmoveto{\pgfqpoint{3.116915in}{5.451034in}}%
\pgfpathlineto{\pgfqpoint{2.889650in}{5.309100in}}%
\pgfpathlineto{\pgfqpoint{3.030553in}{5.308756in}}%
\pgfpathclose%
\pgfusepath{fill}%
\end{pgfscope}%
\begin{pgfscope}%
\pgfpathrectangle{\pgfqpoint{0.539299in}{0.078740in}}{\pgfqpoint{7.842520in}{7.842520in}}%
\pgfusepath{clip}%
\pgfsetbuttcap%
\pgfsetroundjoin%
\definecolor{currentfill}{rgb}{0.137770,0.537492,0.554906}%
\pgfsetfillcolor{currentfill}%
\pgfsetlinewidth{0.000000pt}%
\definecolor{currentstroke}{rgb}{0.395174,0.797475,0.367757}%
\pgfsetstrokecolor{currentstroke}%
\pgfsetdash{}{0pt}%
\pgfpathmoveto{\pgfqpoint{4.993441in}{3.248999in}}%
\pgfpathlineto{\pgfqpoint{4.911519in}{3.252217in}}%
\pgfpathlineto{\pgfqpoint{5.137323in}{3.030184in}}%
\pgfpathclose%
\pgfusepath{fill}%
\end{pgfscope}%
\begin{pgfscope}%
\pgfpathrectangle{\pgfqpoint{0.539299in}{0.078740in}}{\pgfqpoint{7.842520in}{7.842520in}}%
\pgfusepath{clip}%
\pgfsetbuttcap%
\pgfsetroundjoin%
\definecolor{currentfill}{rgb}{0.220124,0.725509,0.466226}%
\pgfsetfillcolor{currentfill}%
\pgfsetlinewidth{0.000000pt}%
\definecolor{currentstroke}{rgb}{0.404001,0.800275,0.362552}%
\pgfsetstrokecolor{currentstroke}%
\pgfsetdash{}{0pt}%
\pgfpathmoveto{\pgfqpoint{4.499656in}{4.110842in}}%
\pgfpathlineto{\pgfqpoint{4.415846in}{4.118860in}}%
\pgfpathlineto{\pgfqpoint{4.560545in}{3.904435in}}%
\pgfpathclose%
\pgfusepath{fill}%
\end{pgfscope}%
\begin{pgfscope}%
\pgfpathrectangle{\pgfqpoint{0.539299in}{0.078740in}}{\pgfqpoint{7.842520in}{7.842520in}}%
\pgfusepath{clip}%
\pgfsetbuttcap%
\pgfsetroundjoin%
\definecolor{currentfill}{rgb}{0.252194,0.269783,0.531579}%
\pgfsetfillcolor{currentfill}%
\pgfsetlinewidth{0.000000pt}%
\definecolor{currentstroke}{rgb}{0.412913,0.803041,0.357269}%
\pgfsetstrokecolor{currentstroke}%
\pgfsetdash{}{0pt}%
\pgfpathmoveto{\pgfqpoint{5.775575in}{1.941647in}}%
\pgfpathlineto{\pgfqpoint{5.632182in}{2.166073in}}%
\pgfpathlineto{\pgfqpoint{5.696616in}{1.925940in}}%
\pgfpathclose%
\pgfusepath{fill}%
\end{pgfscope}%
\begin{pgfscope}%
\pgfpathrectangle{\pgfqpoint{0.539299in}{0.078740in}}{\pgfqpoint{7.842520in}{7.842520in}}%
\pgfusepath{clip}%
\pgfsetbuttcap%
\pgfsetroundjoin%
\definecolor{currentfill}{rgb}{0.140210,0.665859,0.513427}%
\pgfsetfillcolor{currentfill}%
\pgfsetlinewidth{0.000000pt}%
\definecolor{currentstroke}{rgb}{0.421908,0.805774,0.351910}%
\pgfsetstrokecolor{currentstroke}%
\pgfsetdash{}{0pt}%
\pgfpathmoveto{\pgfqpoint{2.838886in}{4.080766in}}%
\pgfpathlineto{\pgfqpoint{2.619520in}{3.617668in}}%
\pgfpathlineto{\pgfqpoint{2.760273in}{3.588885in}}%
\pgfpathclose%
\pgfusepath{fill}%
\end{pgfscope}%
\begin{pgfscope}%
\pgfpathrectangle{\pgfqpoint{0.539299in}{0.078740in}}{\pgfqpoint{7.842520in}{7.842520in}}%
\pgfusepath{clip}%
\pgfsetbuttcap%
\pgfsetroundjoin%
\definecolor{currentfill}{rgb}{0.185783,0.704891,0.485273}%
\pgfsetfillcolor{currentfill}%
\pgfsetlinewidth{0.000000pt}%
\definecolor{currentstroke}{rgb}{0.430983,0.808473,0.346476}%
\pgfsetstrokecolor{currentstroke}%
\pgfsetdash{}{0pt}%
\pgfpathmoveto{\pgfqpoint{2.697635in}{4.107094in}}%
\pgfpathlineto{\pgfqpoint{2.619520in}{3.617668in}}%
\pgfpathlineto{\pgfqpoint{2.838886in}{4.080766in}}%
\pgfpathclose%
\pgfusepath{fill}%
\end{pgfscope}%
\begin{pgfscope}%
\pgfpathrectangle{\pgfqpoint{0.539299in}{0.078740in}}{\pgfqpoint{7.842520in}{7.842520in}}%
\pgfusepath{clip}%
\pgfsetbuttcap%
\pgfsetroundjoin%
\definecolor{currentfill}{rgb}{0.126453,0.570633,0.549841}%
\pgfsetfillcolor{currentfill}%
\pgfsetlinewidth{0.000000pt}%
\definecolor{currentstroke}{rgb}{0.440137,0.811138,0.340967}%
\pgfsetstrokecolor{currentstroke}%
\pgfsetdash{}{0pt}%
\pgfpathmoveto{\pgfqpoint{4.849351in}{3.468201in}}%
\pgfpathlineto{\pgfqpoint{4.911519in}{3.252217in}}%
\pgfpathlineto{\pgfqpoint{4.993441in}{3.248999in}}%
\pgfpathclose%
\pgfusepath{fill}%
\end{pgfscope}%
\begin{pgfscope}%
\pgfpathrectangle{\pgfqpoint{0.539299in}{0.078740in}}{\pgfqpoint{7.842520in}{7.842520in}}%
\pgfusepath{clip}%
\pgfsetbuttcap%
\pgfsetroundjoin%
\definecolor{currentfill}{rgb}{0.525776,0.833491,0.288127}%
\pgfsetfillcolor{currentfill}%
\pgfsetlinewidth{0.000000pt}%
\definecolor{currentstroke}{rgb}{0.449368,0.813768,0.335384}%
\pgfsetstrokecolor{currentstroke}%
\pgfsetdash{}{0pt}%
\pgfpathmoveto{\pgfqpoint{4.066411in}{4.745025in}}%
\pgfpathlineto{\pgfqpoint{3.980947in}{4.722912in}}%
\pgfpathlineto{\pgfqpoint{4.210962in}{4.541645in}}%
\pgfpathclose%
\pgfusepath{fill}%
\end{pgfscope}%
\begin{pgfscope}%
\pgfpathrectangle{\pgfqpoint{0.539299in}{0.078740in}}{\pgfqpoint{7.842520in}{7.842520in}}%
\pgfusepath{clip}%
\pgfsetbuttcap%
\pgfsetroundjoin%
\definecolor{currentfill}{rgb}{0.274149,0.751988,0.436601}%
\pgfsetfillcolor{currentfill}%
\pgfsetlinewidth{0.000000pt}%
\definecolor{currentstroke}{rgb}{0.458674,0.816363,0.329727}%
\pgfsetstrokecolor{currentstroke}%
\pgfsetdash{}{0pt}%
\pgfpathmoveto{\pgfqpoint{4.270980in}{4.328400in}}%
\pgfpathlineto{\pgfqpoint{4.415846in}{4.118860in}}%
\pgfpathlineto{\pgfqpoint{4.499656in}{4.110842in}}%
\pgfpathclose%
\pgfusepath{fill}%
\end{pgfscope}%
\begin{pgfscope}%
\pgfpathrectangle{\pgfqpoint{0.539299in}{0.078740in}}{\pgfqpoint{7.842520in}{7.842520in}}%
\pgfusepath{clip}%
\pgfsetbuttcap%
\pgfsetroundjoin%
\definecolor{currentfill}{rgb}{0.896320,0.893616,0.096335}%
\pgfsetfillcolor{currentfill}%
\pgfsetlinewidth{0.000000pt}%
\definecolor{currentstroke}{rgb}{0.468053,0.818921,0.323998}%
\pgfsetstrokecolor{currentstroke}%
\pgfsetdash{}{0pt}%
\pgfpathmoveto{\pgfqpoint{3.489230in}{5.389845in}}%
\pgfpathlineto{\pgfqpoint{3.546541in}{5.204178in}}%
\pgfpathlineto{\pgfqpoint{3.633064in}{5.262991in}}%
\pgfpathclose%
\pgfusepath{fill}%
\end{pgfscope}%
\begin{pgfscope}%
\pgfpathrectangle{\pgfqpoint{0.539299in}{0.078740in}}{\pgfqpoint{7.842520in}{7.842520in}}%
\pgfusepath{clip}%
\pgfsetbuttcap%
\pgfsetroundjoin%
\definecolor{currentfill}{rgb}{0.273006,0.204520,0.501721}%
\pgfsetfillcolor{currentfill}%
\pgfsetlinewidth{0.000000pt}%
\definecolor{currentstroke}{rgb}{0.477504,0.821444,0.318195}%
\pgfsetstrokecolor{currentstroke}%
\pgfsetdash{}{0pt}%
\pgfpathmoveto{\pgfqpoint{5.696616in}{1.925940in}}%
\pgfpathlineto{\pgfqpoint{5.840111in}{1.685406in}}%
\pgfpathlineto{\pgfqpoint{5.918659in}{1.713508in}}%
\pgfpathclose%
\pgfusepath{fill}%
\end{pgfscope}%
\begin{pgfscope}%
\pgfpathrectangle{\pgfqpoint{0.539299in}{0.078740in}}{\pgfqpoint{7.842520in}{7.842520in}}%
\pgfusepath{clip}%
\pgfsetbuttcap%
\pgfsetroundjoin%
\definecolor{currentfill}{rgb}{0.955300,0.901065,0.118128}%
\pgfsetfillcolor{currentfill}%
\pgfsetlinewidth{0.000000pt}%
\definecolor{currentstroke}{rgb}{0.487026,0.823929,0.312321}%
\pgfsetstrokecolor{currentstroke}%
\pgfsetdash{}{0pt}%
\pgfpathmoveto{\pgfqpoint{3.259203in}{5.402997in}}%
\pgfpathlineto{\pgfqpoint{3.116915in}{5.451034in}}%
\pgfpathlineto{\pgfqpoint{3.030553in}{5.308756in}}%
\pgfpathclose%
\pgfusepath{fill}%
\end{pgfscope}%
\begin{pgfscope}%
\pgfpathrectangle{\pgfqpoint{0.539299in}{0.078740in}}{\pgfqpoint{7.842520in}{7.842520in}}%
\pgfusepath{clip}%
\pgfsetbuttcap%
\pgfsetroundjoin%
\definecolor{currentfill}{rgb}{0.223925,0.334994,0.548053}%
\pgfsetfillcolor{currentfill}%
\pgfsetlinewidth{0.000000pt}%
\definecolor{currentstroke}{rgb}{0.496615,0.826376,0.306377}%
\pgfsetstrokecolor{currentstroke}%
\pgfsetdash{}{0pt}%
\pgfpathmoveto{\pgfqpoint{3.460014in}{2.016906in}}%
\pgfpathlineto{\pgfqpoint{3.395040in}{2.713747in}}%
\pgfpathlineto{\pgfqpoint{3.317407in}{2.078347in}}%
\pgfpathclose%
\pgfusepath{fill}%
\end{pgfscope}%
\begin{pgfscope}%
\pgfpathrectangle{\pgfqpoint{0.539299in}{0.078740in}}{\pgfqpoint{7.842520in}{7.842520in}}%
\pgfusepath{clip}%
\pgfsetbuttcap%
\pgfsetroundjoin%
\definecolor{currentfill}{rgb}{0.835270,0.886029,0.102646}%
\pgfsetfillcolor{currentfill}%
\pgfsetlinewidth{0.000000pt}%
\definecolor{currentstroke}{rgb}{0.506271,0.828786,0.300362}%
\pgfsetstrokecolor{currentstroke}%
\pgfsetdash{}{0pt}%
\pgfpathmoveto{\pgfqpoint{2.945073in}{5.109745in}}%
\pgfpathlineto{\pgfqpoint{2.889650in}{5.309100in}}%
\pgfpathlineto{\pgfqpoint{2.804286in}{5.106496in}}%
\pgfpathclose%
\pgfusepath{fill}%
\end{pgfscope}%
\begin{pgfscope}%
\pgfpathrectangle{\pgfqpoint{0.539299in}{0.078740in}}{\pgfqpoint{7.842520in}{7.842520in}}%
\pgfusepath{clip}%
\pgfsetbuttcap%
\pgfsetroundjoin%
\definecolor{currentfill}{rgb}{0.153364,0.497000,0.557724}%
\pgfsetfillcolor{currentfill}%
\pgfsetlinewidth{0.000000pt}%
\definecolor{currentstroke}{rgb}{0.515992,0.831158,0.294279}%
\pgfsetstrokecolor{currentstroke}%
\pgfsetdash{}{0pt}%
\pgfpathmoveto{\pgfqpoint{5.056080in}{3.038167in}}%
\pgfpathlineto{\pgfqpoint{5.200440in}{2.822813in}}%
\pgfpathlineto{\pgfqpoint{5.137323in}{3.030184in}}%
\pgfpathclose%
\pgfusepath{fill}%
\end{pgfscope}%
\begin{pgfscope}%
\pgfpathrectangle{\pgfqpoint{0.539299in}{0.078740in}}{\pgfqpoint{7.842520in}{7.842520in}}%
\pgfusepath{clip}%
\pgfsetbuttcap%
\pgfsetroundjoin%
\definecolor{currentfill}{rgb}{0.955300,0.901065,0.118128}%
\pgfsetfillcolor{currentfill}%
\pgfsetlinewidth{0.000000pt}%
\definecolor{currentstroke}{rgb}{0.525776,0.833491,0.288127}%
\pgfsetstrokecolor{currentstroke}%
\pgfsetdash{}{0pt}%
\pgfpathmoveto{\pgfqpoint{3.402504in}{5.318957in}}%
\pgfpathlineto{\pgfqpoint{3.489230in}{5.389845in}}%
\pgfpathlineto{\pgfqpoint{3.259203in}{5.402997in}}%
\pgfpathclose%
\pgfusepath{fill}%
\end{pgfscope}%
\begin{pgfscope}%
\pgfpathrectangle{\pgfqpoint{0.539299in}{0.078740in}}{\pgfqpoint{7.842520in}{7.842520in}}%
\pgfusepath{clip}%
\pgfsetbuttcap%
\pgfsetroundjoin%
\definecolor{currentfill}{rgb}{0.125394,0.574318,0.549086}%
\pgfsetfillcolor{currentfill}%
\pgfsetlinewidth{0.000000pt}%
\definecolor{currentstroke}{rgb}{0.535621,0.835785,0.281908}%
\pgfsetstrokecolor{currentstroke}%
\pgfsetdash{}{0pt}%
\pgfpathmoveto{\pgfqpoint{2.901907in}{3.544823in}}%
\pgfpathlineto{\pgfqpoint{2.825356in}{2.965660in}}%
\pgfpathlineto{\pgfqpoint{3.044293in}{3.487491in}}%
\pgfpathclose%
\pgfusepath{fill}%
\end{pgfscope}%
\begin{pgfscope}%
\pgfpathrectangle{\pgfqpoint{0.539299in}{0.078740in}}{\pgfqpoint{7.842520in}{7.842520in}}%
\pgfusepath{clip}%
\pgfsetbuttcap%
\pgfsetroundjoin%
\definecolor{currentfill}{rgb}{0.606045,0.850733,0.236712}%
\pgfsetfillcolor{currentfill}%
\pgfsetlinewidth{0.000000pt}%
\definecolor{currentstroke}{rgb}{0.545524,0.838039,0.275626}%
\pgfsetstrokecolor{currentstroke}%
\pgfsetdash{}{0pt}%
\pgfpathmoveto{\pgfqpoint{3.921827in}{4.935824in}}%
\pgfpathlineto{\pgfqpoint{3.980947in}{4.722912in}}%
\pgfpathlineto{\pgfqpoint{4.066411in}{4.745025in}}%
\pgfpathclose%
\pgfusepath{fill}%
\end{pgfscope}%
\begin{pgfscope}%
\pgfpathrectangle{\pgfqpoint{0.539299in}{0.078740in}}{\pgfqpoint{7.842520in}{7.842520in}}%
\pgfusepath{clip}%
\pgfsetbuttcap%
\pgfsetroundjoin%
\definecolor{currentfill}{rgb}{0.192357,0.403199,0.555836}%
\pgfsetfillcolor{currentfill}%
\pgfsetlinewidth{0.000000pt}%
\definecolor{currentstroke}{rgb}{0.555484,0.840254,0.269281}%
\pgfsetstrokecolor{currentstroke}%
\pgfsetdash{}{0pt}%
\pgfpathmoveto{\pgfqpoint{3.395040in}{2.713747in}}%
\pgfpathlineto{\pgfqpoint{3.251833in}{2.786336in}}%
\pgfpathlineto{\pgfqpoint{3.175266in}{2.137225in}}%
\pgfpathclose%
\pgfusepath{fill}%
\end{pgfscope}%
\begin{pgfscope}%
\pgfpathrectangle{\pgfqpoint{0.539299in}{0.078740in}}{\pgfqpoint{7.842520in}{7.842520in}}%
\pgfusepath{clip}%
\pgfsetbuttcap%
\pgfsetroundjoin%
\definecolor{currentfill}{rgb}{0.845561,0.887322,0.099702}%
\pgfsetfillcolor{currentfill}%
\pgfsetlinewidth{0.000000pt}%
\definecolor{currentstroke}{rgb}{0.565498,0.842430,0.262877}%
\pgfsetstrokecolor{currentstroke}%
\pgfsetdash{}{0pt}%
\pgfpathmoveto{\pgfqpoint{3.633064in}{5.262991in}}%
\pgfpathlineto{\pgfqpoint{3.546541in}{5.204178in}}%
\pgfpathlineto{\pgfqpoint{3.777327in}{5.109990in}}%
\pgfpathclose%
\pgfusepath{fill}%
\end{pgfscope}%
\begin{pgfscope}%
\pgfpathrectangle{\pgfqpoint{0.539299in}{0.078740in}}{\pgfqpoint{7.842520in}{7.842520in}}%
\pgfusepath{clip}%
\pgfsetbuttcap%
\pgfsetroundjoin%
\definecolor{currentfill}{rgb}{0.751884,0.874951,0.143228}%
\pgfsetfillcolor{currentfill}%
\pgfsetlinewidth{0.000000pt}%
\definecolor{currentstroke}{rgb}{0.575563,0.844566,0.256415}%
\pgfsetstrokecolor{currentstroke}%
\pgfsetdash{}{0pt}%
\pgfpathmoveto{\pgfqpoint{3.921827in}{4.935824in}}%
\pgfpathlineto{\pgfqpoint{3.777327in}{5.109990in}}%
\pgfpathlineto{\pgfqpoint{3.691083in}{5.063599in}}%
\pgfpathclose%
\pgfusepath{fill}%
\end{pgfscope}%
\begin{pgfscope}%
\pgfpathrectangle{\pgfqpoint{0.539299in}{0.078740in}}{\pgfqpoint{7.842520in}{7.842520in}}%
\pgfusepath{clip}%
\pgfsetbuttcap%
\pgfsetroundjoin%
\definecolor{currentfill}{rgb}{0.143343,0.522773,0.556295}%
\pgfsetfillcolor{currentfill}%
\pgfsetlinewidth{0.000000pt}%
\definecolor{currentstroke}{rgb}{0.585678,0.846661,0.249897}%
\pgfsetstrokecolor{currentstroke}%
\pgfsetdash{}{0pt}%
\pgfpathmoveto{\pgfqpoint{2.966934in}{2.913287in}}%
\pgfpathlineto{\pgfqpoint{3.109114in}{2.853159in}}%
\pgfpathlineto{\pgfqpoint{3.044293in}{3.487491in}}%
\pgfpathclose%
\pgfusepath{fill}%
\end{pgfscope}%
\begin{pgfscope}%
\pgfpathrectangle{\pgfqpoint{0.539299in}{0.078740in}}{\pgfqpoint{7.842520in}{7.842520in}}%
\pgfusepath{clip}%
\pgfsetbuttcap%
\pgfsetroundjoin%
\definecolor{currentfill}{rgb}{0.225863,0.330805,0.547314}%
\pgfsetfillcolor{currentfill}%
\pgfsetlinewidth{0.000000pt}%
\definecolor{currentstroke}{rgb}{0.595839,0.848717,0.243329}%
\pgfsetstrokecolor{currentstroke}%
\pgfsetdash{}{0pt}%
\pgfpathmoveto{\pgfqpoint{5.552680in}{2.156474in}}%
\pgfpathlineto{\pgfqpoint{5.632182in}{2.166073in}}%
\pgfpathlineto{\pgfqpoint{5.488510in}{2.387296in}}%
\pgfpathclose%
\pgfusepath{fill}%
\end{pgfscope}%
\begin{pgfscope}%
\pgfpathrectangle{\pgfqpoint{0.539299in}{0.078740in}}{\pgfqpoint{7.842520in}{7.842520in}}%
\pgfusepath{clip}%
\pgfsetbuttcap%
\pgfsetroundjoin%
\definecolor{currentfill}{rgb}{0.886271,0.892374,0.095374}%
\pgfsetfillcolor{currentfill}%
\pgfsetlinewidth{0.000000pt}%
\definecolor{currentstroke}{rgb}{0.606045,0.850733,0.236712}%
\pgfsetstrokecolor{currentstroke}%
\pgfsetdash{}{0pt}%
\pgfpathmoveto{\pgfqpoint{3.030553in}{5.308756in}}%
\pgfpathlineto{\pgfqpoint{2.889650in}{5.309100in}}%
\pgfpathlineto{\pgfqpoint{2.945073in}{5.109745in}}%
\pgfpathclose%
\pgfusepath{fill}%
\end{pgfscope}%
\begin{pgfscope}%
\pgfpathrectangle{\pgfqpoint{0.539299in}{0.078740in}}{\pgfqpoint{7.842520in}{7.842520in}}%
\pgfusepath{clip}%
\pgfsetbuttcap%
\pgfsetroundjoin%
\definecolor{currentfill}{rgb}{0.377779,0.791781,0.377939}%
\pgfsetfillcolor{currentfill}%
\pgfsetlinewidth{0.000000pt}%
\definecolor{currentstroke}{rgb}{0.616293,0.852709,0.230052}%
\pgfsetstrokecolor{currentstroke}%
\pgfsetdash{}{0pt}%
\pgfpathmoveto{\pgfqpoint{2.637836in}{4.513404in}}%
\pgfpathlineto{\pgfqpoint{2.697635in}{4.107094in}}%
\pgfpathlineto{\pgfqpoint{2.778216in}{4.515166in}}%
\pgfpathclose%
\pgfusepath{fill}%
\end{pgfscope}%
\begin{pgfscope}%
\pgfpathrectangle{\pgfqpoint{0.539299in}{0.078740in}}{\pgfqpoint{7.842520in}{7.842520in}}%
\pgfusepath{clip}%
\pgfsetbuttcap%
\pgfsetroundjoin%
\definecolor{currentfill}{rgb}{0.140536,0.530132,0.555659}%
\pgfsetfillcolor{currentfill}%
\pgfsetlinewidth{0.000000pt}%
\definecolor{currentstroke}{rgb}{0.626579,0.854645,0.223353}%
\pgfsetstrokecolor{currentstroke}%
\pgfsetdash{}{0pt}%
\pgfpathmoveto{\pgfqpoint{4.911519in}{3.252217in}}%
\pgfpathlineto{\pgfqpoint{5.056080in}{3.038167in}}%
\pgfpathlineto{\pgfqpoint{5.137323in}{3.030184in}}%
\pgfpathclose%
\pgfusepath{fill}%
\end{pgfscope}%
\begin{pgfscope}%
\pgfpathrectangle{\pgfqpoint{0.539299in}{0.078740in}}{\pgfqpoint{7.842520in}{7.842520in}}%
\pgfusepath{clip}%
\pgfsetbuttcap%
\pgfsetroundjoin%
\definecolor{currentfill}{rgb}{0.575563,0.844566,0.256415}%
\pgfsetfillcolor{currentfill}%
\pgfsetlinewidth{0.000000pt}%
\definecolor{currentstroke}{rgb}{0.636902,0.856542,0.216620}%
\pgfsetstrokecolor{currentstroke}%
\pgfsetdash{}{0pt}%
\pgfpathmoveto{\pgfqpoint{2.720220in}{4.843367in}}%
\pgfpathlineto{\pgfqpoint{2.637836in}{4.513404in}}%
\pgfpathlineto{\pgfqpoint{2.860832in}{4.847363in}}%
\pgfpathclose%
\pgfusepath{fill}%
\end{pgfscope}%
\begin{pgfscope}%
\pgfpathrectangle{\pgfqpoint{0.539299in}{0.078740in}}{\pgfqpoint{7.842520in}{7.842520in}}%
\pgfusepath{clip}%
\pgfsetbuttcap%
\pgfsetroundjoin%
\definecolor{currentfill}{rgb}{0.699415,0.867117,0.175971}%
\pgfsetfillcolor{currentfill}%
\pgfsetlinewidth{0.000000pt}%
\definecolor{currentstroke}{rgb}{0.647257,0.858400,0.209861}%
\pgfsetstrokecolor{currentstroke}%
\pgfsetdash{}{0pt}%
\pgfpathmoveto{\pgfqpoint{2.804286in}{5.106496in}}%
\pgfpathlineto{\pgfqpoint{2.720220in}{4.843367in}}%
\pgfpathlineto{\pgfqpoint{2.860832in}{4.847363in}}%
\pgfpathclose%
\pgfusepath{fill}%
\end{pgfscope}%
\begin{pgfscope}%
\pgfpathrectangle{\pgfqpoint{0.539299in}{0.078740in}}{\pgfqpoint{7.842520in}{7.842520in}}%
\pgfusepath{clip}%
\pgfsetbuttcap%
\pgfsetroundjoin%
\definecolor{currentfill}{rgb}{0.421908,0.805774,0.351910}%
\pgfsetfillcolor{currentfill}%
\pgfsetlinewidth{0.000000pt}%
\definecolor{currentstroke}{rgb}{0.657642,0.860219,0.203082}%
\pgfsetstrokecolor{currentstroke}%
\pgfsetdash{}{0pt}%
\pgfpathmoveto{\pgfqpoint{4.125992in}{4.530691in}}%
\pgfpathlineto{\pgfqpoint{4.270980in}{4.328400in}}%
\pgfpathlineto{\pgfqpoint{4.210962in}{4.541645in}}%
\pgfpathclose%
\pgfusepath{fill}%
\end{pgfscope}%
\begin{pgfscope}%
\pgfpathrectangle{\pgfqpoint{0.539299in}{0.078740in}}{\pgfqpoint{7.842520in}{7.842520in}}%
\pgfusepath{clip}%
\pgfsetbuttcap%
\pgfsetroundjoin%
\definecolor{currentfill}{rgb}{0.926106,0.897330,0.104071}%
\pgfsetfillcolor{currentfill}%
\pgfsetlinewidth{0.000000pt}%
\definecolor{currentstroke}{rgb}{0.668054,0.861999,0.196293}%
\pgfsetstrokecolor{currentstroke}%
\pgfsetdash{}{0pt}%
\pgfpathmoveto{\pgfqpoint{3.402504in}{5.318957in}}%
\pgfpathlineto{\pgfqpoint{3.546541in}{5.204178in}}%
\pgfpathlineto{\pgfqpoint{3.489230in}{5.389845in}}%
\pgfpathclose%
\pgfusepath{fill}%
\end{pgfscope}%
\begin{pgfscope}%
\pgfpathrectangle{\pgfqpoint{0.539299in}{0.078740in}}{\pgfqpoint{7.842520in}{7.842520in}}%
\pgfusepath{clip}%
\pgfsetbuttcap%
\pgfsetroundjoin%
\definecolor{currentfill}{rgb}{0.124780,0.640461,0.527068}%
\pgfsetfillcolor{currentfill}%
\pgfsetlinewidth{0.000000pt}%
\definecolor{currentstroke}{rgb}{0.678489,0.863742,0.189503}%
\pgfsetstrokecolor{currentstroke}%
\pgfsetdash{}{0pt}%
\pgfpathmoveto{\pgfqpoint{4.849351in}{3.468201in}}%
\pgfpathlineto{\pgfqpoint{4.705051in}{3.687048in}}%
\pgfpathlineto{\pgfqpoint{4.621833in}{3.675341in}}%
\pgfpathclose%
\pgfusepath{fill}%
\end{pgfscope}%
\begin{pgfscope}%
\pgfpathrectangle{\pgfqpoint{0.539299in}{0.078740in}}{\pgfqpoint{7.842520in}{7.842520in}}%
\pgfusepath{clip}%
\pgfsetbuttcap%
\pgfsetroundjoin%
\definecolor{currentfill}{rgb}{0.140210,0.665859,0.513427}%
\pgfsetfillcolor{currentfill}%
\pgfsetlinewidth{0.000000pt}%
\definecolor{currentstroke}{rgb}{0.688944,0.865448,0.182725}%
\pgfsetstrokecolor{currentstroke}%
\pgfsetdash{}{0pt}%
\pgfpathmoveto{\pgfqpoint{2.760273in}{3.588885in}}%
\pgfpathlineto{\pgfqpoint{2.901907in}{3.544823in}}%
\pgfpathlineto{\pgfqpoint{2.838886in}{4.080766in}}%
\pgfpathclose%
\pgfusepath{fill}%
\end{pgfscope}%
\begin{pgfscope}%
\pgfpathrectangle{\pgfqpoint{0.539299in}{0.078740in}}{\pgfqpoint{7.842520in}{7.842520in}}%
\pgfusepath{clip}%
\pgfsetbuttcap%
\pgfsetroundjoin%
\definecolor{currentfill}{rgb}{0.496615,0.826376,0.306377}%
\pgfsetfillcolor{currentfill}%
\pgfsetlinewidth{0.000000pt}%
\definecolor{currentstroke}{rgb}{0.699415,0.867117,0.175971}%
\pgfsetstrokecolor{currentstroke}%
\pgfsetdash{}{0pt}%
\pgfpathmoveto{\pgfqpoint{4.210962in}{4.541645in}}%
\pgfpathlineto{\pgfqpoint{3.980947in}{4.722912in}}%
\pgfpathlineto{\pgfqpoint{4.125992in}{4.530691in}}%
\pgfpathclose%
\pgfusepath{fill}%
\end{pgfscope}%
\begin{pgfscope}%
\pgfpathrectangle{\pgfqpoint{0.539299in}{0.078740in}}{\pgfqpoint{7.842520in}{7.842520in}}%
\pgfusepath{clip}%
\pgfsetbuttcap%
\pgfsetroundjoin%
\definecolor{currentfill}{rgb}{0.199430,0.387607,0.554642}%
\pgfsetfillcolor{currentfill}%
\pgfsetlinewidth{0.000000pt}%
\definecolor{currentstroke}{rgb}{0.709898,0.868751,0.169257}%
\pgfsetstrokecolor{currentstroke}%
\pgfsetdash{}{0pt}%
\pgfpathmoveto{\pgfqpoint{5.408389in}{2.379042in}}%
\pgfpathlineto{\pgfqpoint{5.488510in}{2.387296in}}%
\pgfpathlineto{\pgfqpoint{5.344589in}{2.606000in}}%
\pgfpathclose%
\pgfusepath{fill}%
\end{pgfscope}%
\begin{pgfscope}%
\pgfpathrectangle{\pgfqpoint{0.539299in}{0.078740in}}{\pgfqpoint{7.842520in}{7.842520in}}%
\pgfusepath{clip}%
\pgfsetbuttcap%
\pgfsetroundjoin%
\definecolor{currentfill}{rgb}{0.241237,0.296485,0.539709}%
\pgfsetfillcolor{currentfill}%
\pgfsetlinewidth{0.000000pt}%
\definecolor{currentstroke}{rgb}{0.720391,0.870350,0.162603}%
\pgfsetstrokecolor{currentstroke}%
\pgfsetdash{}{0pt}%
\pgfpathmoveto{\pgfqpoint{5.696616in}{1.925940in}}%
\pgfpathlineto{\pgfqpoint{5.632182in}{2.166073in}}%
\pgfpathlineto{\pgfqpoint{5.552680in}{2.156474in}}%
\pgfpathclose%
\pgfusepath{fill}%
\end{pgfscope}%
\begin{pgfscope}%
\pgfpathrectangle{\pgfqpoint{0.539299in}{0.078740in}}{\pgfqpoint{7.842520in}{7.842520in}}%
\pgfusepath{clip}%
\pgfsetbuttcap%
\pgfsetroundjoin%
\definecolor{currentfill}{rgb}{0.515992,0.831158,0.294279}%
\pgfsetfillcolor{currentfill}%
\pgfsetlinewidth{0.000000pt}%
\definecolor{currentstroke}{rgb}{0.730889,0.871916,0.156029}%
\pgfsetstrokecolor{currentstroke}%
\pgfsetdash{}{0pt}%
\pgfpathmoveto{\pgfqpoint{2.860832in}{4.847363in}}%
\pgfpathlineto{\pgfqpoint{2.637836in}{4.513404in}}%
\pgfpathlineto{\pgfqpoint{2.778216in}{4.515166in}}%
\pgfpathclose%
\pgfusepath{fill}%
\end{pgfscope}%
\begin{pgfscope}%
\pgfpathrectangle{\pgfqpoint{0.539299in}{0.078740in}}{\pgfqpoint{7.842520in}{7.842520in}}%
\pgfusepath{clip}%
\pgfsetbuttcap%
\pgfsetroundjoin%
\definecolor{currentfill}{rgb}{0.945636,0.899815,0.112838}%
\pgfsetfillcolor{currentfill}%
\pgfsetlinewidth{0.000000pt}%
\definecolor{currentstroke}{rgb}{0.741388,0.873449,0.149561}%
\pgfsetstrokecolor{currentstroke}%
\pgfsetdash{}{0pt}%
\pgfpathmoveto{\pgfqpoint{3.030553in}{5.308756in}}%
\pgfpathlineto{\pgfqpoint{3.172817in}{5.269283in}}%
\pgfpathlineto{\pgfqpoint{3.259203in}{5.402997in}}%
\pgfpathclose%
\pgfusepath{fill}%
\end{pgfscope}%
\begin{pgfscope}%
\pgfpathrectangle{\pgfqpoint{0.539299in}{0.078740in}}{\pgfqpoint{7.842520in}{7.842520in}}%
\pgfusepath{clip}%
\pgfsetbuttcap%
\pgfsetroundjoin%
\definecolor{currentfill}{rgb}{0.311925,0.767822,0.415586}%
\pgfsetfillcolor{currentfill}%
\pgfsetlinewidth{0.000000pt}%
\definecolor{currentstroke}{rgb}{0.751884,0.874951,0.143228}%
\pgfsetstrokecolor{currentstroke}%
\pgfsetdash{}{0pt}%
\pgfpathmoveto{\pgfqpoint{2.778216in}{4.515166in}}%
\pgfpathlineto{\pgfqpoint{2.697635in}{4.107094in}}%
\pgfpathlineto{\pgfqpoint{2.838886in}{4.080766in}}%
\pgfpathclose%
\pgfusepath{fill}%
\end{pgfscope}%
\begin{pgfscope}%
\pgfpathrectangle{\pgfqpoint{0.539299in}{0.078740in}}{\pgfqpoint{7.842520in}{7.842520in}}%
\pgfusepath{clip}%
\pgfsetbuttcap%
\pgfsetroundjoin%
\definecolor{currentfill}{rgb}{0.762373,0.876424,0.137064}%
\pgfsetfillcolor{currentfill}%
\pgfsetlinewidth{0.000000pt}%
\definecolor{currentstroke}{rgb}{0.762373,0.876424,0.137064}%
\pgfsetstrokecolor{currentstroke}%
\pgfsetdash{}{0pt}%
\pgfpathmoveto{\pgfqpoint{2.860832in}{4.847363in}}%
\pgfpathlineto{\pgfqpoint{2.945073in}{5.109745in}}%
\pgfpathlineto{\pgfqpoint{2.804286in}{5.106496in}}%
\pgfpathclose%
\pgfusepath{fill}%
\end{pgfscope}%
\begin{pgfscope}%
\pgfpathrectangle{\pgfqpoint{0.539299in}{0.078740in}}{\pgfqpoint{7.842520in}{7.842520in}}%
\pgfusepath{clip}%
\pgfsetbuttcap%
\pgfsetroundjoin%
\definecolor{currentfill}{rgb}{0.824940,0.884720,0.106217}%
\pgfsetfillcolor{currentfill}%
\pgfsetlinewidth{0.000000pt}%
\definecolor{currentstroke}{rgb}{0.772852,0.877868,0.131109}%
\pgfsetstrokecolor{currentstroke}%
\pgfsetdash{}{0pt}%
\pgfpathmoveto{\pgfqpoint{3.777327in}{5.109990in}}%
\pgfpathlineto{\pgfqpoint{3.546541in}{5.204178in}}%
\pgfpathlineto{\pgfqpoint{3.691083in}{5.063599in}}%
\pgfpathclose%
\pgfusepath{fill}%
\end{pgfscope}%
\begin{pgfscope}%
\pgfpathrectangle{\pgfqpoint{0.539299in}{0.078740in}}{\pgfqpoint{7.842520in}{7.842520in}}%
\pgfusepath{clip}%
\pgfsetbuttcap%
\pgfsetroundjoin%
\definecolor{currentfill}{rgb}{0.162016,0.687316,0.499129}%
\pgfsetfillcolor{currentfill}%
\pgfsetlinewidth{0.000000pt}%
\definecolor{currentstroke}{rgb}{0.783315,0.879285,0.125405}%
\pgfsetstrokecolor{currentstroke}%
\pgfsetdash{}{0pt}%
\pgfpathmoveto{\pgfqpoint{4.705051in}{3.687048in}}%
\pgfpathlineto{\pgfqpoint{4.560545in}{3.904435in}}%
\pgfpathlineto{\pgfqpoint{4.476731in}{3.882966in}}%
\pgfpathclose%
\pgfusepath{fill}%
\end{pgfscope}%
\begin{pgfscope}%
\pgfpathrectangle{\pgfqpoint{0.539299in}{0.078740in}}{\pgfqpoint{7.842520in}{7.842520in}}%
\pgfusepath{clip}%
\pgfsetbuttcap%
\pgfsetroundjoin%
\definecolor{currentfill}{rgb}{0.657642,0.860219,0.203082}%
\pgfsetfillcolor{currentfill}%
\pgfsetlinewidth{0.000000pt}%
\definecolor{currentstroke}{rgb}{0.793760,0.880678,0.120005}%
\pgfsetstrokecolor{currentstroke}%
\pgfsetdash{}{0pt}%
\pgfpathmoveto{\pgfqpoint{3.835937in}{4.901790in}}%
\pgfpathlineto{\pgfqpoint{3.980947in}{4.722912in}}%
\pgfpathlineto{\pgfqpoint{3.921827in}{4.935824in}}%
\pgfpathclose%
\pgfusepath{fill}%
\end{pgfscope}%
\begin{pgfscope}%
\pgfpathrectangle{\pgfqpoint{0.539299in}{0.078740in}}{\pgfqpoint{7.842520in}{7.842520in}}%
\pgfusepath{clip}%
\pgfsetbuttcap%
\pgfsetroundjoin%
\definecolor{currentfill}{rgb}{0.120565,0.596422,0.543611}%
\pgfsetfillcolor{currentfill}%
\pgfsetlinewidth{0.000000pt}%
\definecolor{currentstroke}{rgb}{0.804182,0.882046,0.114965}%
\pgfsetstrokecolor{currentstroke}%
\pgfsetdash{}{0pt}%
\pgfpathmoveto{\pgfqpoint{4.766767in}{3.464787in}}%
\pgfpathlineto{\pgfqpoint{4.911519in}{3.252217in}}%
\pgfpathlineto{\pgfqpoint{4.849351in}{3.468201in}}%
\pgfpathclose%
\pgfusepath{fill}%
\end{pgfscope}%
\begin{pgfscope}%
\pgfpathrectangle{\pgfqpoint{0.539299in}{0.078740in}}{\pgfqpoint{7.842520in}{7.842520in}}%
\pgfusepath{clip}%
\pgfsetbuttcap%
\pgfsetroundjoin%
\definecolor{currentfill}{rgb}{0.730889,0.871916,0.156029}%
\pgfsetfillcolor{currentfill}%
\pgfsetlinewidth{0.000000pt}%
\definecolor{currentstroke}{rgb}{0.814576,0.883393,0.110347}%
\pgfsetstrokecolor{currentstroke}%
\pgfsetdash{}{0pt}%
\pgfpathmoveto{\pgfqpoint{3.691083in}{5.063599in}}%
\pgfpathlineto{\pgfqpoint{3.835937in}{4.901790in}}%
\pgfpathlineto{\pgfqpoint{3.921827in}{4.935824in}}%
\pgfpathclose%
\pgfusepath{fill}%
\end{pgfscope}%
\begin{pgfscope}%
\pgfpathrectangle{\pgfqpoint{0.539299in}{0.078740in}}{\pgfqpoint{7.842520in}{7.842520in}}%
\pgfusepath{clip}%
\pgfsetbuttcap%
\pgfsetroundjoin%
\definecolor{currentfill}{rgb}{0.280255,0.165693,0.476498}%
\pgfsetfillcolor{currentfill}%
\pgfsetlinewidth{0.000000pt}%
\definecolor{currentstroke}{rgb}{0.824940,0.884720,0.106217}%
\pgfsetstrokecolor{currentstroke}%
\pgfsetdash{}{0pt}%
\pgfpathmoveto{\pgfqpoint{5.983060in}{1.432854in}}%
\pgfpathlineto{\pgfqpoint{5.840111in}{1.685406in}}%
\pgfpathlineto{\pgfqpoint{5.760821in}{1.647418in}}%
\pgfpathclose%
\pgfusepath{fill}%
\end{pgfscope}%
\begin{pgfscope}%
\pgfpathrectangle{\pgfqpoint{0.539299in}{0.078740in}}{\pgfqpoint{7.842520in}{7.842520in}}%
\pgfusepath{clip}%
\pgfsetbuttcap%
\pgfsetroundjoin%
\definecolor{currentfill}{rgb}{0.227802,0.326594,0.546532}%
\pgfsetfillcolor{currentfill}%
\pgfsetlinewidth{0.000000pt}%
\definecolor{currentstroke}{rgb}{0.835270,0.886029,0.102646}%
\pgfsetstrokecolor{currentstroke}%
\pgfsetdash{}{0pt}%
\pgfpathmoveto{\pgfqpoint{3.460014in}{2.016906in}}%
\pgfpathlineto{\pgfqpoint{3.603070in}{1.953238in}}%
\pgfpathlineto{\pgfqpoint{3.682755in}{2.554388in}}%
\pgfpathclose%
\pgfusepath{fill}%
\end{pgfscope}%
\begin{pgfscope}%
\pgfpathrectangle{\pgfqpoint{0.539299in}{0.078740in}}{\pgfqpoint{7.842520in}{7.842520in}}%
\pgfusepath{clip}%
\pgfsetbuttcap%
\pgfsetroundjoin%
\definecolor{currentfill}{rgb}{0.121380,0.629492,0.531973}%
\pgfsetfillcolor{currentfill}%
\pgfsetlinewidth{0.000000pt}%
\definecolor{currentstroke}{rgb}{0.845561,0.887322,0.099702}%
\pgfsetstrokecolor{currentstroke}%
\pgfsetdash{}{0pt}%
\pgfpathmoveto{\pgfqpoint{4.849351in}{3.468201in}}%
\pgfpathlineto{\pgfqpoint{4.621833in}{3.675341in}}%
\pgfpathlineto{\pgfqpoint{4.766767in}{3.464787in}}%
\pgfpathclose%
\pgfusepath{fill}%
\end{pgfscope}%
\begin{pgfscope}%
\pgfpathrectangle{\pgfqpoint{0.539299in}{0.078740in}}{\pgfqpoint{7.842520in}{7.842520in}}%
\pgfusepath{clip}%
\pgfsetbuttcap%
\pgfsetroundjoin%
\definecolor{currentfill}{rgb}{0.935904,0.898570,0.108131}%
\pgfsetfillcolor{currentfill}%
\pgfsetlinewidth{0.000000pt}%
\definecolor{currentstroke}{rgb}{0.855810,0.888601,0.097452}%
\pgfsetstrokecolor{currentstroke}%
\pgfsetdash{}{0pt}%
\pgfpathmoveto{\pgfqpoint{3.259203in}{5.402997in}}%
\pgfpathlineto{\pgfqpoint{3.316139in}{5.195751in}}%
\pgfpathlineto{\pgfqpoint{3.402504in}{5.318957in}}%
\pgfpathclose%
\pgfusepath{fill}%
\end{pgfscope}%
\begin{pgfscope}%
\pgfpathrectangle{\pgfqpoint{0.539299in}{0.078740in}}{\pgfqpoint{7.842520in}{7.842520in}}%
\pgfusepath{clip}%
\pgfsetbuttcap%
\pgfsetroundjoin%
\definecolor{currentfill}{rgb}{0.208030,0.718701,0.472873}%
\pgfsetfillcolor{currentfill}%
\pgfsetlinewidth{0.000000pt}%
\definecolor{currentstroke}{rgb}{0.866013,0.889868,0.095953}%
\pgfsetstrokecolor{currentstroke}%
\pgfsetdash{}{0pt}%
\pgfpathmoveto{\pgfqpoint{4.476731in}{3.882966in}}%
\pgfpathlineto{\pgfqpoint{4.560545in}{3.904435in}}%
\pgfpathlineto{\pgfqpoint{4.415846in}{4.118860in}}%
\pgfpathclose%
\pgfusepath{fill}%
\end{pgfscope}%
\begin{pgfscope}%
\pgfpathrectangle{\pgfqpoint{0.539299in}{0.078740in}}{\pgfqpoint{7.842520in}{7.842520in}}%
\pgfusepath{clip}%
\pgfsetbuttcap%
\pgfsetroundjoin%
\definecolor{currentfill}{rgb}{0.212395,0.359683,0.551710}%
\pgfsetfillcolor{currentfill}%
\pgfsetlinewidth{0.000000pt}%
\definecolor{currentstroke}{rgb}{0.876168,0.891125,0.095250}%
\pgfsetstrokecolor{currentstroke}%
\pgfsetdash{}{0pt}%
\pgfpathmoveto{\pgfqpoint{5.488510in}{2.387296in}}%
\pgfpathlineto{\pgfqpoint{5.408389in}{2.379042in}}%
\pgfpathlineto{\pgfqpoint{5.552680in}{2.156474in}}%
\pgfpathclose%
\pgfusepath{fill}%
\end{pgfscope}%
\begin{pgfscope}%
\pgfpathrectangle{\pgfqpoint{0.539299in}{0.078740in}}{\pgfqpoint{7.842520in}{7.842520in}}%
\pgfusepath{clip}%
\pgfsetbuttcap%
\pgfsetroundjoin%
\definecolor{currentfill}{rgb}{0.195860,0.395433,0.555276}%
\pgfsetfillcolor{currentfill}%
\pgfsetlinewidth{0.000000pt}%
\definecolor{currentstroke}{rgb}{0.886271,0.892374,0.095374}%
\pgfsetstrokecolor{currentstroke}%
\pgfsetdash{}{0pt}%
\pgfpathmoveto{\pgfqpoint{3.538693in}{2.636201in}}%
\pgfpathlineto{\pgfqpoint{3.395040in}{2.713747in}}%
\pgfpathlineto{\pgfqpoint{3.460014in}{2.016906in}}%
\pgfpathclose%
\pgfusepath{fill}%
\end{pgfscope}%
\begin{pgfscope}%
\pgfpathrectangle{\pgfqpoint{0.539299in}{0.078740in}}{\pgfqpoint{7.842520in}{7.842520in}}%
\pgfusepath{clip}%
\pgfsetbuttcap%
\pgfsetroundjoin%
\definecolor{currentfill}{rgb}{0.168126,0.459988,0.558082}%
\pgfsetfillcolor{currentfill}%
\pgfsetlinewidth{0.000000pt}%
\definecolor{currentstroke}{rgb}{0.896320,0.893616,0.096335}%
\pgfsetstrokecolor{currentstroke}%
\pgfsetdash{}{0pt}%
\pgfpathmoveto{\pgfqpoint{5.344589in}{2.606000in}}%
\pgfpathlineto{\pgfqpoint{5.200440in}{2.822813in}}%
\pgfpathlineto{\pgfqpoint{5.119001in}{2.807062in}}%
\pgfpathclose%
\pgfusepath{fill}%
\end{pgfscope}%
\begin{pgfscope}%
\pgfpathrectangle{\pgfqpoint{0.539299in}{0.078740in}}{\pgfqpoint{7.842520in}{7.842520in}}%
\pgfusepath{clip}%
\pgfsetbuttcap%
\pgfsetroundjoin%
\definecolor{currentfill}{rgb}{0.126453,0.570633,0.549841}%
\pgfsetfillcolor{currentfill}%
\pgfsetlinewidth{0.000000pt}%
\definecolor{currentstroke}{rgb}{0.906311,0.894855,0.098125}%
\pgfsetstrokecolor{currentstroke}%
\pgfsetdash{}{0pt}%
\pgfpathmoveto{\pgfqpoint{3.109114in}{2.853159in}}%
\pgfpathlineto{\pgfqpoint{3.187325in}{3.418679in}}%
\pgfpathlineto{\pgfqpoint{3.044293in}{3.487491in}}%
\pgfpathclose%
\pgfusepath{fill}%
\end{pgfscope}%
\begin{pgfscope}%
\pgfpathrectangle{\pgfqpoint{0.539299in}{0.078740in}}{\pgfqpoint{7.842520in}{7.842520in}}%
\pgfusepath{clip}%
\pgfsetbuttcap%
\pgfsetroundjoin%
\definecolor{currentfill}{rgb}{0.855810,0.888601,0.097452}%
\pgfsetfillcolor{currentfill}%
\pgfsetlinewidth{0.000000pt}%
\definecolor{currentstroke}{rgb}{0.916242,0.896091,0.100717}%
\pgfsetstrokecolor{currentstroke}%
\pgfsetdash{}{0pt}%
\pgfpathmoveto{\pgfqpoint{3.030553in}{5.308756in}}%
\pgfpathlineto{\pgfqpoint{2.945073in}{5.109745in}}%
\pgfpathlineto{\pgfqpoint{3.087223in}{5.076965in}}%
\pgfpathclose%
\pgfusepath{fill}%
\end{pgfscope}%
\begin{pgfscope}%
\pgfpathrectangle{\pgfqpoint{0.539299in}{0.078740in}}{\pgfqpoint{7.842520in}{7.842520in}}%
\pgfusepath{clip}%
\pgfsetbuttcap%
\pgfsetroundjoin%
\definecolor{currentfill}{rgb}{0.147607,0.511733,0.557049}%
\pgfsetfillcolor{currentfill}%
\pgfsetlinewidth{0.000000pt}%
\definecolor{currentstroke}{rgb}{0.926106,0.897330,0.104071}%
\pgfsetstrokecolor{currentstroke}%
\pgfsetdash{}{0pt}%
\pgfpathmoveto{\pgfqpoint{3.109114in}{2.853159in}}%
\pgfpathlineto{\pgfqpoint{3.251833in}{2.786336in}}%
\pgfpathlineto{\pgfqpoint{3.330909in}{3.339983in}}%
\pgfpathclose%
\pgfusepath{fill}%
\end{pgfscope}%
\begin{pgfscope}%
\pgfpathrectangle{\pgfqpoint{0.539299in}{0.078740in}}{\pgfqpoint{7.842520in}{7.842520in}}%
\pgfusepath{clip}%
\pgfsetbuttcap%
\pgfsetroundjoin%
\definecolor{currentfill}{rgb}{0.935904,0.898570,0.108131}%
\pgfsetfillcolor{currentfill}%
\pgfsetlinewidth{0.000000pt}%
\definecolor{currentstroke}{rgb}{0.935904,0.898570,0.108131}%
\pgfsetstrokecolor{currentstroke}%
\pgfsetdash{}{0pt}%
\pgfpathmoveto{\pgfqpoint{3.172817in}{5.269283in}}%
\pgfpathlineto{\pgfqpoint{3.316139in}{5.195751in}}%
\pgfpathlineto{\pgfqpoint{3.259203in}{5.402997in}}%
\pgfpathclose%
\pgfusepath{fill}%
\end{pgfscope}%
\begin{pgfscope}%
\pgfpathrectangle{\pgfqpoint{0.539299in}{0.078740in}}{\pgfqpoint{7.842520in}{7.842520in}}%
\pgfusepath{clip}%
\pgfsetbuttcap%
\pgfsetroundjoin%
\definecolor{currentfill}{rgb}{0.185783,0.704891,0.485273}%
\pgfsetfillcolor{currentfill}%
\pgfsetlinewidth{0.000000pt}%
\definecolor{currentstroke}{rgb}{0.945636,0.899815,0.112838}%
\pgfsetstrokecolor{currentstroke}%
\pgfsetdash{}{0pt}%
\pgfpathmoveto{\pgfqpoint{2.901907in}{3.544823in}}%
\pgfpathlineto{\pgfqpoint{2.981110in}{4.034439in}}%
\pgfpathlineto{\pgfqpoint{2.838886in}{4.080766in}}%
\pgfpathclose%
\pgfusepath{fill}%
\end{pgfscope}%
\begin{pgfscope}%
\pgfpathrectangle{\pgfqpoint{0.539299in}{0.078740in}}{\pgfqpoint{7.842520in}{7.842520in}}%
\pgfusepath{clip}%
\pgfsetbuttcap%
\pgfsetroundjoin%
\definecolor{currentfill}{rgb}{0.896320,0.893616,0.096335}%
\pgfsetfillcolor{currentfill}%
\pgfsetlinewidth{0.000000pt}%
\definecolor{currentstroke}{rgb}{0.955300,0.901065,0.118128}%
\pgfsetstrokecolor{currentstroke}%
\pgfsetdash{}{0pt}%
\pgfpathmoveto{\pgfqpoint{3.087223in}{5.076965in}}%
\pgfpathlineto{\pgfqpoint{3.172817in}{5.269283in}}%
\pgfpathlineto{\pgfqpoint{3.030553in}{5.308756in}}%
\pgfpathclose%
\pgfusepath{fill}%
\end{pgfscope}%
\begin{pgfscope}%
\pgfpathrectangle{\pgfqpoint{0.539299in}{0.078740in}}{\pgfqpoint{7.842520in}{7.842520in}}%
\pgfusepath{clip}%
\pgfsetbuttcap%
\pgfsetroundjoin%
\definecolor{currentfill}{rgb}{0.282623,0.140926,0.457517}%
\pgfsetfillcolor{currentfill}%
\pgfsetlinewidth{0.000000pt}%
\definecolor{currentstroke}{rgb}{0.964894,0.902323,0.123941}%
\pgfsetstrokecolor{currentstroke}%
\pgfsetdash{}{0pt}%
\pgfpathmoveto{\pgfqpoint{5.760821in}{1.647418in}}%
\pgfpathlineto{\pgfqpoint{5.904099in}{1.379502in}}%
\pgfpathlineto{\pgfqpoint{5.983060in}{1.432854in}}%
\pgfpathclose%
\pgfusepath{fill}%
\end{pgfscope}%
\begin{pgfscope}%
\pgfpathrectangle{\pgfqpoint{0.539299in}{0.078740in}}{\pgfqpoint{7.842520in}{7.842520in}}%
\pgfusepath{clip}%
\pgfsetbuttcap%
\pgfsetroundjoin%
\definecolor{currentfill}{rgb}{0.269308,0.218818,0.509577}%
\pgfsetfillcolor{currentfill}%
\pgfsetlinewidth{0.000000pt}%
\definecolor{currentstroke}{rgb}{0.974417,0.903590,0.130215}%
\pgfsetstrokecolor{currentstroke}%
\pgfsetdash{}{0pt}%
\pgfpathmoveto{\pgfqpoint{5.760821in}{1.647418in}}%
\pgfpathlineto{\pgfqpoint{5.840111in}{1.685406in}}%
\pgfpathlineto{\pgfqpoint{5.696616in}{1.925940in}}%
\pgfpathclose%
\pgfusepath{fill}%
\end{pgfscope}%
\begin{pgfscope}%
\pgfpathrectangle{\pgfqpoint{0.539299in}{0.078740in}}{\pgfqpoint{7.842520in}{7.842520in}}%
\pgfusepath{clip}%
\pgfsetbuttcap%
\pgfsetroundjoin%
\definecolor{currentfill}{rgb}{0.140210,0.665859,0.513427}%
\pgfsetfillcolor{currentfill}%
\pgfsetlinewidth{0.000000pt}%
\definecolor{currentstroke}{rgb}{0.983868,0.904867,0.136897}%
\pgfsetstrokecolor{currentstroke}%
\pgfsetdash{}{0pt}%
\pgfpathmoveto{\pgfqpoint{3.044293in}{3.487491in}}%
\pgfpathlineto{\pgfqpoint{2.981110in}{4.034439in}}%
\pgfpathlineto{\pgfqpoint{2.901907in}{3.544823in}}%
\pgfpathclose%
\pgfusepath{fill}%
\end{pgfscope}%
\begin{pgfscope}%
\pgfpathrectangle{\pgfqpoint{0.539299in}{0.078740in}}{\pgfqpoint{7.842520in}{7.842520in}}%
\pgfusepath{clip}%
\pgfsetbuttcap%
\pgfsetroundjoin%
\definecolor{currentfill}{rgb}{0.150148,0.676631,0.506589}%
\pgfsetfillcolor{currentfill}%
\pgfsetlinewidth{0.000000pt}%
\definecolor{currentstroke}{rgb}{0.993248,0.906157,0.143936}%
\pgfsetstrokecolor{currentstroke}%
\pgfsetdash{}{0pt}%
\pgfpathmoveto{\pgfqpoint{4.476731in}{3.882966in}}%
\pgfpathlineto{\pgfqpoint{4.621833in}{3.675341in}}%
\pgfpathlineto{\pgfqpoint{4.705051in}{3.687048in}}%
\pgfpathclose%
\pgfusepath{fill}%
\end{pgfscope}%
\begin{pgfscope}%
\pgfpathrectangle{\pgfqpoint{0.539299in}{0.078740in}}{\pgfqpoint{7.842520in}{7.842520in}}%
\pgfusepath{clip}%
\pgfsetbuttcap%
\pgfsetroundjoin%
\definecolor{currentfill}{rgb}{0.906311,0.894855,0.098125}%
\pgfsetfillcolor{currentfill}%
\pgfsetlinewidth{0.000000pt}%
\definecolor{currentstroke}{rgb}{0.267004,0.004874,0.329415}%
\pgfsetstrokecolor{currentstroke}%
\pgfsetdash{}{0pt}%
\pgfpathmoveto{\pgfqpoint{3.316139in}{5.195751in}}%
\pgfpathlineto{\pgfqpoint{3.546541in}{5.204178in}}%
\pgfpathlineto{\pgfqpoint{3.402504in}{5.318957in}}%
\pgfpathclose%
\pgfusepath{fill}%
\end{pgfscope}%
\begin{pgfscope}%
\pgfpathrectangle{\pgfqpoint{0.539299in}{0.078740in}}{\pgfqpoint{7.842520in}{7.842520in}}%
\pgfusepath{clip}%
\pgfsetbuttcap%
\pgfsetroundjoin%
\definecolor{currentfill}{rgb}{0.395174,0.797475,0.367757}%
\pgfsetfillcolor{currentfill}%
\pgfsetlinewidth{0.000000pt}%
\definecolor{currentstroke}{rgb}{0.268510,0.009605,0.335427}%
\pgfsetstrokecolor{currentstroke}%
\pgfsetdash{}{0pt}%
\pgfpathmoveto{\pgfqpoint{2.838886in}{4.080766in}}%
\pgfpathlineto{\pgfqpoint{2.919862in}{4.489063in}}%
\pgfpathlineto{\pgfqpoint{2.778216in}{4.515166in}}%
\pgfpathclose%
\pgfusepath{fill}%
\end{pgfscope}%
\begin{pgfscope}%
\pgfpathrectangle{\pgfqpoint{0.539299in}{0.078740in}}{\pgfqpoint{7.842520in}{7.842520in}}%
\pgfusepath{clip}%
\pgfsetbuttcap%
\pgfsetroundjoin%
\definecolor{currentfill}{rgb}{0.335885,0.777018,0.402049}%
\pgfsetfillcolor{currentfill}%
\pgfsetlinewidth{0.000000pt}%
\definecolor{currentstroke}{rgb}{0.269944,0.014625,0.341379}%
\pgfsetstrokecolor{currentstroke}%
\pgfsetdash{}{0pt}%
\pgfpathmoveto{\pgfqpoint{4.415846in}{4.118860in}}%
\pgfpathlineto{\pgfqpoint{4.270980in}{4.328400in}}%
\pgfpathlineto{\pgfqpoint{4.186126in}{4.283787in}}%
\pgfpathclose%
\pgfusepath{fill}%
\end{pgfscope}%
\begin{pgfscope}%
\pgfpathrectangle{\pgfqpoint{0.539299in}{0.078740in}}{\pgfqpoint{7.842520in}{7.842520in}}%
\pgfusepath{clip}%
\pgfsetbuttcap%
\pgfsetroundjoin%
\definecolor{currentfill}{rgb}{0.535621,0.835785,0.281908}%
\pgfsetfillcolor{currentfill}%
\pgfsetlinewidth{0.000000pt}%
\definecolor{currentstroke}{rgb}{0.271305,0.019942,0.347269}%
\pgfsetstrokecolor{currentstroke}%
\pgfsetdash{}{0pt}%
\pgfpathmoveto{\pgfqpoint{2.778216in}{4.515166in}}%
\pgfpathlineto{\pgfqpoint{2.919862in}{4.489063in}}%
\pgfpathlineto{\pgfqpoint{2.860832in}{4.847363in}}%
\pgfpathclose%
\pgfusepath{fill}%
\end{pgfscope}%
\begin{pgfscope}%
\pgfpathrectangle{\pgfqpoint{0.539299in}{0.078740in}}{\pgfqpoint{7.842520in}{7.842520in}}%
\pgfusepath{clip}%
\pgfsetbuttcap%
\pgfsetroundjoin%
\definecolor{currentfill}{rgb}{0.772852,0.877868,0.131109}%
\pgfsetfillcolor{currentfill}%
\pgfsetlinewidth{0.000000pt}%
\definecolor{currentstroke}{rgb}{0.272594,0.025563,0.353093}%
\pgfsetstrokecolor{currentstroke}%
\pgfsetdash{}{0pt}%
\pgfpathmoveto{\pgfqpoint{2.945073in}{5.109745in}}%
\pgfpathlineto{\pgfqpoint{2.860832in}{4.847363in}}%
\pgfpathlineto{\pgfqpoint{3.087223in}{5.076965in}}%
\pgfpathclose%
\pgfusepath{fill}%
\end{pgfscope}%
\begin{pgfscope}%
\pgfpathrectangle{\pgfqpoint{0.539299in}{0.078740in}}{\pgfqpoint{7.842520in}{7.842520in}}%
\pgfusepath{clip}%
\pgfsetbuttcap%
\pgfsetroundjoin%
\definecolor{currentfill}{rgb}{0.187231,0.414746,0.556547}%
\pgfsetfillcolor{currentfill}%
\pgfsetlinewidth{0.000000pt}%
\definecolor{currentstroke}{rgb}{0.273809,0.031497,0.358853}%
\pgfsetstrokecolor{currentstroke}%
\pgfsetdash{}{0pt}%
\pgfpathmoveto{\pgfqpoint{5.263813in}{2.595433in}}%
\pgfpathlineto{\pgfqpoint{5.408389in}{2.379042in}}%
\pgfpathlineto{\pgfqpoint{5.344589in}{2.606000in}}%
\pgfpathclose%
\pgfusepath{fill}%
\end{pgfscope}%
\begin{pgfscope}%
\pgfpathrectangle{\pgfqpoint{0.539299in}{0.078740in}}{\pgfqpoint{7.842520in}{7.842520in}}%
\pgfusepath{clip}%
\pgfsetbuttcap%
\pgfsetroundjoin%
\definecolor{currentfill}{rgb}{0.149039,0.508051,0.557250}%
\pgfsetfillcolor{currentfill}%
\pgfsetlinewidth{0.000000pt}%
\definecolor{currentstroke}{rgb}{0.274952,0.037752,0.364543}%
\pgfsetstrokecolor{currentstroke}%
\pgfsetdash{}{0pt}%
\pgfpathmoveto{\pgfqpoint{5.200440in}{2.822813in}}%
\pgfpathlineto{\pgfqpoint{5.056080in}{3.038167in}}%
\pgfpathlineto{\pgfqpoint{4.973989in}{3.014923in}}%
\pgfpathclose%
\pgfusepath{fill}%
\end{pgfscope}%
\begin{pgfscope}%
\pgfpathrectangle{\pgfqpoint{0.539299in}{0.078740in}}{\pgfqpoint{7.842520in}{7.842520in}}%
\pgfusepath{clip}%
\pgfsetbuttcap%
\pgfsetroundjoin%
\definecolor{currentfill}{rgb}{0.229739,0.322361,0.545706}%
\pgfsetfillcolor{currentfill}%
\pgfsetlinewidth{0.000000pt}%
\definecolor{currentstroke}{rgb}{0.276022,0.044167,0.370164}%
\pgfsetstrokecolor{currentstroke}%
\pgfsetdash{}{0pt}%
\pgfpathmoveto{\pgfqpoint{3.682755in}{2.554388in}}%
\pgfpathlineto{\pgfqpoint{3.603070in}{1.953238in}}%
\pgfpathlineto{\pgfqpoint{3.746562in}{1.887616in}}%
\pgfpathclose%
\pgfusepath{fill}%
\end{pgfscope}%
\begin{pgfscope}%
\pgfpathrectangle{\pgfqpoint{0.539299in}{0.078740in}}{\pgfqpoint{7.842520in}{7.842520in}}%
\pgfusepath{clip}%
\pgfsetbuttcap%
\pgfsetroundjoin%
\definecolor{currentfill}{rgb}{0.199430,0.387607,0.554642}%
\pgfsetfillcolor{currentfill}%
\pgfsetlinewidth{0.000000pt}%
\definecolor{currentstroke}{rgb}{0.277018,0.050344,0.375715}%
\pgfsetstrokecolor{currentstroke}%
\pgfsetdash{}{0pt}%
\pgfpathmoveto{\pgfqpoint{3.682755in}{2.554388in}}%
\pgfpathlineto{\pgfqpoint{3.538693in}{2.636201in}}%
\pgfpathlineto{\pgfqpoint{3.460014in}{2.016906in}}%
\pgfpathclose%
\pgfusepath{fill}%
\end{pgfscope}%
\begin{pgfscope}%
\pgfpathrectangle{\pgfqpoint{0.539299in}{0.078740in}}{\pgfqpoint{7.842520in}{7.842520in}}%
\pgfusepath{clip}%
\pgfsetbuttcap%
\pgfsetroundjoin%
\definecolor{currentfill}{rgb}{0.128729,0.563265,0.551229}%
\pgfsetfillcolor{currentfill}%
\pgfsetlinewidth{0.000000pt}%
\definecolor{currentstroke}{rgb}{0.277941,0.056324,0.381191}%
\pgfsetstrokecolor{currentstroke}%
\pgfsetdash{}{0pt}%
\pgfpathmoveto{\pgfqpoint{3.330909in}{3.339983in}}%
\pgfpathlineto{\pgfqpoint{3.187325in}{3.418679in}}%
\pgfpathlineto{\pgfqpoint{3.109114in}{2.853159in}}%
\pgfpathclose%
\pgfusepath{fill}%
\end{pgfscope}%
\begin{pgfscope}%
\pgfpathrectangle{\pgfqpoint{0.539299in}{0.078740in}}{\pgfqpoint{7.842520in}{7.842520in}}%
\pgfusepath{clip}%
\pgfsetbuttcap%
\pgfsetroundjoin%
\definecolor{currentfill}{rgb}{0.440137,0.811138,0.340967}%
\pgfsetfillcolor{currentfill}%
\pgfsetlinewidth{0.000000pt}%
\definecolor{currentstroke}{rgb}{0.278791,0.062145,0.386592}%
\pgfsetstrokecolor{currentstroke}%
\pgfsetdash{}{0pt}%
\pgfpathmoveto{\pgfqpoint{4.270980in}{4.328400in}}%
\pgfpathlineto{\pgfqpoint{4.125992in}{4.530691in}}%
\pgfpathlineto{\pgfqpoint{4.040710in}{4.473147in}}%
\pgfpathclose%
\pgfusepath{fill}%
\end{pgfscope}%
\begin{pgfscope}%
\pgfpathrectangle{\pgfqpoint{0.539299in}{0.078740in}}{\pgfqpoint{7.842520in}{7.842520in}}%
\pgfusepath{clip}%
\pgfsetbuttcap%
\pgfsetroundjoin%
\definecolor{currentfill}{rgb}{0.149039,0.508051,0.557250}%
\pgfsetfillcolor{currentfill}%
\pgfsetlinewidth{0.000000pt}%
\definecolor{currentstroke}{rgb}{0.279566,0.067836,0.391917}%
\pgfsetstrokecolor{currentstroke}%
\pgfsetdash{}{0pt}%
\pgfpathmoveto{\pgfqpoint{3.251833in}{2.786336in}}%
\pgfpathlineto{\pgfqpoint{3.395040in}{2.713747in}}%
\pgfpathlineto{\pgfqpoint{3.330909in}{3.339983in}}%
\pgfpathclose%
\pgfusepath{fill}%
\end{pgfscope}%
\begin{pgfscope}%
\pgfpathrectangle{\pgfqpoint{0.539299in}{0.078740in}}{\pgfqpoint{7.842520in}{7.842520in}}%
\pgfusepath{clip}%
\pgfsetbuttcap%
\pgfsetroundjoin%
\definecolor{currentfill}{rgb}{0.172719,0.448791,0.557885}%
\pgfsetfillcolor{currentfill}%
\pgfsetlinewidth{0.000000pt}%
\definecolor{currentstroke}{rgb}{0.280267,0.073417,0.397163}%
\pgfsetstrokecolor{currentstroke}%
\pgfsetdash{}{0pt}%
\pgfpathmoveto{\pgfqpoint{5.344589in}{2.606000in}}%
\pgfpathlineto{\pgfqpoint{5.119001in}{2.807062in}}%
\pgfpathlineto{\pgfqpoint{5.263813in}{2.595433in}}%
\pgfpathclose%
\pgfusepath{fill}%
\end{pgfscope}%
\begin{pgfscope}%
\pgfpathrectangle{\pgfqpoint{0.539299in}{0.078740in}}{\pgfqpoint{7.842520in}{7.842520in}}%
\pgfusepath{clip}%
\pgfsetbuttcap%
\pgfsetroundjoin%
\definecolor{currentfill}{rgb}{0.246070,0.738910,0.452024}%
\pgfsetfillcolor{currentfill}%
\pgfsetlinewidth{0.000000pt}%
\definecolor{currentstroke}{rgb}{0.280894,0.078907,0.402329}%
\pgfsetstrokecolor{currentstroke}%
\pgfsetdash{}{0pt}%
\pgfpathmoveto{\pgfqpoint{4.415846in}{4.118860in}}%
\pgfpathlineto{\pgfqpoint{4.331483in}{4.086352in}}%
\pgfpathlineto{\pgfqpoint{4.476731in}{3.882966in}}%
\pgfpathclose%
\pgfusepath{fill}%
\end{pgfscope}%
\begin{pgfscope}%
\pgfpathrectangle{\pgfqpoint{0.539299in}{0.078740in}}{\pgfqpoint{7.842520in}{7.842520in}}%
\pgfusepath{clip}%
\pgfsetbuttcap%
\pgfsetroundjoin%
\definecolor{currentfill}{rgb}{0.824940,0.884720,0.106217}%
\pgfsetfillcolor{currentfill}%
\pgfsetlinewidth{0.000000pt}%
\definecolor{currentstroke}{rgb}{0.281446,0.084320,0.407414}%
\pgfsetstrokecolor{currentstroke}%
\pgfsetdash{}{0pt}%
\pgfpathmoveto{\pgfqpoint{3.546541in}{5.204178in}}%
\pgfpathlineto{\pgfqpoint{3.604941in}{4.965350in}}%
\pgfpathlineto{\pgfqpoint{3.691083in}{5.063599in}}%
\pgfpathclose%
\pgfusepath{fill}%
\end{pgfscope}%
\begin{pgfscope}%
\pgfpathrectangle{\pgfqpoint{0.539299in}{0.078740in}}{\pgfqpoint{7.842520in}{7.842520in}}%
\pgfusepath{clip}%
\pgfsetbuttcap%
\pgfsetroundjoin%
\definecolor{currentfill}{rgb}{0.515992,0.831158,0.294279}%
\pgfsetfillcolor{currentfill}%
\pgfsetlinewidth{0.000000pt}%
\definecolor{currentstroke}{rgb}{0.281924,0.089666,0.412415}%
\pgfsetstrokecolor{currentstroke}%
\pgfsetdash{}{0pt}%
\pgfpathmoveto{\pgfqpoint{4.040710in}{4.473147in}}%
\pgfpathlineto{\pgfqpoint{4.125992in}{4.530691in}}%
\pgfpathlineto{\pgfqpoint{3.980947in}{4.722912in}}%
\pgfpathclose%
\pgfusepath{fill}%
\end{pgfscope}%
\begin{pgfscope}%
\pgfpathrectangle{\pgfqpoint{0.539299in}{0.078740in}}{\pgfqpoint{7.842520in}{7.842520in}}%
\pgfusepath{clip}%
\pgfsetbuttcap%
\pgfsetroundjoin%
\definecolor{currentfill}{rgb}{0.129933,0.559582,0.551864}%
\pgfsetfillcolor{currentfill}%
\pgfsetlinewidth{0.000000pt}%
\definecolor{currentstroke}{rgb}{0.282327,0.094955,0.417331}%
\pgfsetstrokecolor{currentstroke}%
\pgfsetdash{}{0pt}%
\pgfpathmoveto{\pgfqpoint{4.828804in}{3.219577in}}%
\pgfpathlineto{\pgfqpoint{5.056080in}{3.038167in}}%
\pgfpathlineto{\pgfqpoint{4.911519in}{3.252217in}}%
\pgfpathclose%
\pgfusepath{fill}%
\end{pgfscope}%
\begin{pgfscope}%
\pgfpathrectangle{\pgfqpoint{0.539299in}{0.078740in}}{\pgfqpoint{7.842520in}{7.842520in}}%
\pgfusepath{clip}%
\pgfsetbuttcap%
\pgfsetroundjoin%
\definecolor{currentfill}{rgb}{0.886271,0.892374,0.095374}%
\pgfsetfillcolor{currentfill}%
\pgfsetlinewidth{0.000000pt}%
\definecolor{currentstroke}{rgb}{0.282656,0.100196,0.422160}%
\pgfsetstrokecolor{currentstroke}%
\pgfsetdash{}{0pt}%
\pgfpathmoveto{\pgfqpoint{3.087223in}{5.076965in}}%
\pgfpathlineto{\pgfqpoint{3.316139in}{5.195751in}}%
\pgfpathlineto{\pgfqpoint{3.172817in}{5.269283in}}%
\pgfpathclose%
\pgfusepath{fill}%
\end{pgfscope}%
\begin{pgfscope}%
\pgfpathrectangle{\pgfqpoint{0.539299in}{0.078740in}}{\pgfqpoint{7.842520in}{7.842520in}}%
\pgfusepath{clip}%
\pgfsetbuttcap%
\pgfsetroundjoin%
\definecolor{currentfill}{rgb}{0.606045,0.850733,0.236712}%
\pgfsetfillcolor{currentfill}%
\pgfsetlinewidth{0.000000pt}%
\definecolor{currentstroke}{rgb}{0.282910,0.105393,0.426902}%
\pgfsetstrokecolor{currentstroke}%
\pgfsetdash{}{0pt}%
\pgfpathmoveto{\pgfqpoint{2.919862in}{4.489063in}}%
\pgfpathlineto{\pgfqpoint{3.002777in}{4.819100in}}%
\pgfpathlineto{\pgfqpoint{2.860832in}{4.847363in}}%
\pgfpathclose%
\pgfusepath{fill}%
\end{pgfscope}%
\begin{pgfscope}%
\pgfpathrectangle{\pgfqpoint{0.539299in}{0.078740in}}{\pgfqpoint{7.842520in}{7.842520in}}%
\pgfusepath{clip}%
\pgfsetbuttcap%
\pgfsetroundjoin%
\definecolor{currentfill}{rgb}{0.762373,0.876424,0.137064}%
\pgfsetfillcolor{currentfill}%
\pgfsetlinewidth{0.000000pt}%
\definecolor{currentstroke}{rgb}{0.283091,0.110553,0.431554}%
\pgfsetstrokecolor{currentstroke}%
\pgfsetdash{}{0pt}%
\pgfpathmoveto{\pgfqpoint{3.604941in}{4.965350in}}%
\pgfpathlineto{\pgfqpoint{3.835937in}{4.901790in}}%
\pgfpathlineto{\pgfqpoint{3.691083in}{5.063599in}}%
\pgfpathclose%
\pgfusepath{fill}%
\end{pgfscope}%
\begin{pgfscope}%
\pgfpathrectangle{\pgfqpoint{0.539299in}{0.078740in}}{\pgfqpoint{7.842520in}{7.842520in}}%
\pgfusepath{clip}%
\pgfsetbuttcap%
\pgfsetroundjoin%
\definecolor{currentfill}{rgb}{0.720391,0.870350,0.162603}%
\pgfsetfillcolor{currentfill}%
\pgfsetlinewidth{0.000000pt}%
\definecolor{currentstroke}{rgb}{0.283197,0.115680,0.436115}%
\pgfsetstrokecolor{currentstroke}%
\pgfsetdash{}{0pt}%
\pgfpathmoveto{\pgfqpoint{3.087223in}{5.076965in}}%
\pgfpathlineto{\pgfqpoint{2.860832in}{4.847363in}}%
\pgfpathlineto{\pgfqpoint{3.002777in}{4.819100in}}%
\pgfpathclose%
\pgfusepath{fill}%
\end{pgfscope}%
\begin{pgfscope}%
\pgfpathrectangle{\pgfqpoint{0.539299in}{0.078740in}}{\pgfqpoint{7.842520in}{7.842520in}}%
\pgfusepath{clip}%
\pgfsetbuttcap%
\pgfsetroundjoin%
\definecolor{currentfill}{rgb}{0.668054,0.861999,0.196293}%
\pgfsetfillcolor{currentfill}%
\pgfsetlinewidth{0.000000pt}%
\definecolor{currentstroke}{rgb}{0.283229,0.120777,0.440584}%
\pgfsetstrokecolor{currentstroke}%
\pgfsetdash{}{0pt}%
\pgfpathmoveto{\pgfqpoint{3.980947in}{4.722912in}}%
\pgfpathlineto{\pgfqpoint{3.835937in}{4.901790in}}%
\pgfpathlineto{\pgfqpoint{3.750009in}{4.817083in}}%
\pgfpathclose%
\pgfusepath{fill}%
\end{pgfscope}%
\begin{pgfscope}%
\pgfpathrectangle{\pgfqpoint{0.539299in}{0.078740in}}{\pgfqpoint{7.842520in}{7.842520in}}%
\pgfusepath{clip}%
\pgfsetbuttcap%
\pgfsetroundjoin%
\definecolor{currentfill}{rgb}{0.876168,0.891125,0.095250}%
\pgfsetfillcolor{currentfill}%
\pgfsetlinewidth{0.000000pt}%
\definecolor{currentstroke}{rgb}{0.283187,0.125848,0.444960}%
\pgfsetstrokecolor{currentstroke}%
\pgfsetdash{}{0pt}%
\pgfpathmoveto{\pgfqpoint{3.460255in}{5.092948in}}%
\pgfpathlineto{\pgfqpoint{3.546541in}{5.204178in}}%
\pgfpathlineto{\pgfqpoint{3.316139in}{5.195751in}}%
\pgfpathclose%
\pgfusepath{fill}%
\end{pgfscope}%
\begin{pgfscope}%
\pgfpathrectangle{\pgfqpoint{0.539299in}{0.078740in}}{\pgfqpoint{7.842520in}{7.842520in}}%
\pgfusepath{clip}%
\pgfsetbuttcap%
\pgfsetroundjoin%
\definecolor{currentfill}{rgb}{0.304148,0.764704,0.419943}%
\pgfsetfillcolor{currentfill}%
\pgfsetlinewidth{0.000000pt}%
\definecolor{currentstroke}{rgb}{0.283072,0.130895,0.449241}%
\pgfsetstrokecolor{currentstroke}%
\pgfsetdash{}{0pt}%
\pgfpathmoveto{\pgfqpoint{4.186126in}{4.283787in}}%
\pgfpathlineto{\pgfqpoint{4.331483in}{4.086352in}}%
\pgfpathlineto{\pgfqpoint{4.415846in}{4.118860in}}%
\pgfpathclose%
\pgfusepath{fill}%
\end{pgfscope}%
\begin{pgfscope}%
\pgfpathrectangle{\pgfqpoint{0.539299in}{0.078740in}}{\pgfqpoint{7.842520in}{7.842520in}}%
\pgfusepath{clip}%
\pgfsetbuttcap%
\pgfsetroundjoin%
\definecolor{currentfill}{rgb}{0.395174,0.797475,0.367757}%
\pgfsetfillcolor{currentfill}%
\pgfsetlinewidth{0.000000pt}%
\definecolor{currentstroke}{rgb}{0.282884,0.135920,0.453427}%
\pgfsetstrokecolor{currentstroke}%
\pgfsetdash{}{0pt}%
\pgfpathmoveto{\pgfqpoint{2.838886in}{4.080766in}}%
\pgfpathlineto{\pgfqpoint{3.062545in}{4.438625in}}%
\pgfpathlineto{\pgfqpoint{2.919862in}{4.489063in}}%
\pgfpathclose%
\pgfusepath{fill}%
\end{pgfscope}%
\begin{pgfscope}%
\pgfpathrectangle{\pgfqpoint{0.539299in}{0.078740in}}{\pgfqpoint{7.842520in}{7.842520in}}%
\pgfusepath{clip}%
\pgfsetbuttcap%
\pgfsetroundjoin%
\definecolor{currentfill}{rgb}{0.235526,0.309527,0.542944}%
\pgfsetfillcolor{currentfill}%
\pgfsetlinewidth{0.000000pt}%
\definecolor{currentstroke}{rgb}{0.282623,0.140926,0.457517}%
\pgfsetstrokecolor{currentstroke}%
\pgfsetdash{}{0pt}%
\pgfpathmoveto{\pgfqpoint{5.472274in}{2.124924in}}%
\pgfpathlineto{\pgfqpoint{5.696616in}{1.925940in}}%
\pgfpathlineto{\pgfqpoint{5.552680in}{2.156474in}}%
\pgfpathclose%
\pgfusepath{fill}%
\end{pgfscope}%
\begin{pgfscope}%
\pgfpathrectangle{\pgfqpoint{0.539299in}{0.078740in}}{\pgfqpoint{7.842520in}{7.842520in}}%
\pgfusepath{clip}%
\pgfsetbuttcap%
\pgfsetroundjoin%
\definecolor{currentfill}{rgb}{0.319809,0.770914,0.411152}%
\pgfsetfillcolor{currentfill}%
\pgfsetlinewidth{0.000000pt}%
\definecolor{currentstroke}{rgb}{0.282290,0.145912,0.461510}%
\pgfsetstrokecolor{currentstroke}%
\pgfsetdash{}{0pt}%
\pgfpathmoveto{\pgfqpoint{2.981110in}{4.034439in}}%
\pgfpathlineto{\pgfqpoint{3.062545in}{4.438625in}}%
\pgfpathlineto{\pgfqpoint{2.838886in}{4.080766in}}%
\pgfpathclose%
\pgfusepath{fill}%
\end{pgfscope}%
\begin{pgfscope}%
\pgfpathrectangle{\pgfqpoint{0.539299in}{0.078740in}}{\pgfqpoint{7.842520in}{7.842520in}}%
\pgfusepath{clip}%
\pgfsetbuttcap%
\pgfsetroundjoin%
\definecolor{currentfill}{rgb}{0.153364,0.497000,0.557724}%
\pgfsetfillcolor{currentfill}%
\pgfsetlinewidth{0.000000pt}%
\definecolor{currentstroke}{rgb}{0.281887,0.150881,0.465405}%
\pgfsetstrokecolor{currentstroke}%
\pgfsetdash{}{0pt}%
\pgfpathmoveto{\pgfqpoint{4.973989in}{3.014923in}}%
\pgfpathlineto{\pgfqpoint{5.119001in}{2.807062in}}%
\pgfpathlineto{\pgfqpoint{5.200440in}{2.822813in}}%
\pgfpathclose%
\pgfusepath{fill}%
\end{pgfscope}%
\begin{pgfscope}%
\pgfpathrectangle{\pgfqpoint{0.539299in}{0.078740in}}{\pgfqpoint{7.842520in}{7.842520in}}%
\pgfusepath{clip}%
\pgfsetbuttcap%
\pgfsetroundjoin%
\definecolor{currentfill}{rgb}{0.180653,0.701402,0.488189}%
\pgfsetfillcolor{currentfill}%
\pgfsetlinewidth{0.000000pt}%
\definecolor{currentstroke}{rgb}{0.281412,0.155834,0.469201}%
\pgfsetstrokecolor{currentstroke}%
\pgfsetdash{}{0pt}%
\pgfpathmoveto{\pgfqpoint{3.044293in}{3.487491in}}%
\pgfpathlineto{\pgfqpoint{3.124144in}{3.970746in}}%
\pgfpathlineto{\pgfqpoint{2.981110in}{4.034439in}}%
\pgfpathclose%
\pgfusepath{fill}%
\end{pgfscope}%
\begin{pgfscope}%
\pgfpathrectangle{\pgfqpoint{0.539299in}{0.078740in}}{\pgfqpoint{7.842520in}{7.842520in}}%
\pgfusepath{clip}%
\pgfsetbuttcap%
\pgfsetroundjoin%
\definecolor{currentfill}{rgb}{0.119512,0.607464,0.540218}%
\pgfsetfillcolor{currentfill}%
\pgfsetlinewidth{0.000000pt}%
\definecolor{currentstroke}{rgb}{0.280868,0.160771,0.472899}%
\pgfsetstrokecolor{currentstroke}%
\pgfsetdash{}{0pt}%
\pgfpathmoveto{\pgfqpoint{4.683466in}{3.421156in}}%
\pgfpathlineto{\pgfqpoint{4.911519in}{3.252217in}}%
\pgfpathlineto{\pgfqpoint{4.766767in}{3.464787in}}%
\pgfpathclose%
\pgfusepath{fill}%
\end{pgfscope}%
\begin{pgfscope}%
\pgfpathrectangle{\pgfqpoint{0.539299in}{0.078740in}}{\pgfqpoint{7.842520in}{7.842520in}}%
\pgfusepath{clip}%
\pgfsetbuttcap%
\pgfsetroundjoin%
\definecolor{currentfill}{rgb}{0.260571,0.246922,0.522828}%
\pgfsetfillcolor{currentfill}%
\pgfsetlinewidth{0.000000pt}%
\definecolor{currentstroke}{rgb}{0.280255,0.165693,0.476498}%
\pgfsetstrokecolor{currentstroke}%
\pgfsetdash{}{0pt}%
\pgfpathmoveto{\pgfqpoint{5.696616in}{1.925940in}}%
\pgfpathlineto{\pgfqpoint{5.616816in}{1.894378in}}%
\pgfpathlineto{\pgfqpoint{5.760821in}{1.647418in}}%
\pgfpathclose%
\pgfusepath{fill}%
\end{pgfscope}%
\begin{pgfscope}%
\pgfpathrectangle{\pgfqpoint{0.539299in}{0.078740in}}{\pgfqpoint{7.842520in}{7.842520in}}%
\pgfusepath{clip}%
\pgfsetbuttcap%
\pgfsetroundjoin%
\definecolor{currentfill}{rgb}{0.835270,0.886029,0.102646}%
\pgfsetfillcolor{currentfill}%
\pgfsetlinewidth{0.000000pt}%
\definecolor{currentstroke}{rgb}{0.279574,0.170599,0.479997}%
\pgfsetstrokecolor{currentstroke}%
\pgfsetdash{}{0pt}%
\pgfpathmoveto{\pgfqpoint{3.460255in}{5.092948in}}%
\pgfpathlineto{\pgfqpoint{3.604941in}{4.965350in}}%
\pgfpathlineto{\pgfqpoint{3.546541in}{5.204178in}}%
\pgfpathclose%
\pgfusepath{fill}%
\end{pgfscope}%
\begin{pgfscope}%
\pgfpathrectangle{\pgfqpoint{0.539299in}{0.078740in}}{\pgfqpoint{7.842520in}{7.842520in}}%
\pgfusepath{clip}%
\pgfsetbuttcap%
\pgfsetroundjoin%
\definecolor{currentfill}{rgb}{0.412913,0.803041,0.357269}%
\pgfsetfillcolor{currentfill}%
\pgfsetlinewidth{0.000000pt}%
\definecolor{currentstroke}{rgb}{0.278826,0.175490,0.483397}%
\pgfsetstrokecolor{currentstroke}%
\pgfsetdash{}{0pt}%
\pgfpathmoveto{\pgfqpoint{4.040710in}{4.473147in}}%
\pgfpathlineto{\pgfqpoint{4.186126in}{4.283787in}}%
\pgfpathlineto{\pgfqpoint{4.270980in}{4.328400in}}%
\pgfpathclose%
\pgfusepath{fill}%
\end{pgfscope}%
\begin{pgfscope}%
\pgfpathrectangle{\pgfqpoint{0.539299in}{0.078740in}}{\pgfqpoint{7.842520in}{7.842520in}}%
\pgfusepath{clip}%
\pgfsetbuttcap%
\pgfsetroundjoin%
\definecolor{currentfill}{rgb}{0.845561,0.887322,0.099702}%
\pgfsetfillcolor{currentfill}%
\pgfsetlinewidth{0.000000pt}%
\definecolor{currentstroke}{rgb}{0.278012,0.180367,0.486697}%
\pgfsetstrokecolor{currentstroke}%
\pgfsetdash{}{0pt}%
\pgfpathmoveto{\pgfqpoint{3.230452in}{5.012748in}}%
\pgfpathlineto{\pgfqpoint{3.316139in}{5.195751in}}%
\pgfpathlineto{\pgfqpoint{3.087223in}{5.076965in}}%
\pgfpathclose%
\pgfusepath{fill}%
\end{pgfscope}%
\begin{pgfscope}%
\pgfpathrectangle{\pgfqpoint{0.539299in}{0.078740in}}{\pgfqpoint{7.842520in}{7.842520in}}%
\pgfusepath{clip}%
\pgfsetbuttcap%
\pgfsetroundjoin%
\definecolor{currentfill}{rgb}{0.216210,0.351535,0.550627}%
\pgfsetfillcolor{currentfill}%
\pgfsetlinewidth{0.000000pt}%
\definecolor{currentstroke}{rgb}{0.277134,0.185228,0.489898}%
\pgfsetstrokecolor{currentstroke}%
\pgfsetdash{}{0pt}%
\pgfpathmoveto{\pgfqpoint{5.552680in}{2.156474in}}%
\pgfpathlineto{\pgfqpoint{5.408389in}{2.379042in}}%
\pgfpathlineto{\pgfqpoint{5.472274in}{2.124924in}}%
\pgfpathclose%
\pgfusepath{fill}%
\end{pgfscope}%
\begin{pgfscope}%
\pgfpathrectangle{\pgfqpoint{0.539299in}{0.078740in}}{\pgfqpoint{7.842520in}{7.842520in}}%
\pgfusepath{clip}%
\pgfsetbuttcap%
\pgfsetroundjoin%
\definecolor{currentfill}{rgb}{0.132268,0.655014,0.519661}%
\pgfsetfillcolor{currentfill}%
\pgfsetlinewidth{0.000000pt}%
\definecolor{currentstroke}{rgb}{0.276194,0.190074,0.493001}%
\pgfsetstrokecolor{currentstroke}%
\pgfsetdash{}{0pt}%
\pgfpathmoveto{\pgfqpoint{3.267846in}{3.892059in}}%
\pgfpathlineto{\pgfqpoint{3.044293in}{3.487491in}}%
\pgfpathlineto{\pgfqpoint{3.187325in}{3.418679in}}%
\pgfpathclose%
\pgfusepath{fill}%
\end{pgfscope}%
\begin{pgfscope}%
\pgfpathrectangle{\pgfqpoint{0.539299in}{0.078740in}}{\pgfqpoint{7.842520in}{7.842520in}}%
\pgfusepath{clip}%
\pgfsetbuttcap%
\pgfsetroundjoin%
\definecolor{currentfill}{rgb}{0.135066,0.544853,0.554029}%
\pgfsetfillcolor{currentfill}%
\pgfsetlinewidth{0.000000pt}%
\definecolor{currentstroke}{rgb}{0.275191,0.194905,0.496005}%
\pgfsetstrokecolor{currentstroke}%
\pgfsetdash{}{0pt}%
\pgfpathmoveto{\pgfqpoint{4.973989in}{3.014923in}}%
\pgfpathlineto{\pgfqpoint{5.056080in}{3.038167in}}%
\pgfpathlineto{\pgfqpoint{4.828804in}{3.219577in}}%
\pgfpathclose%
\pgfusepath{fill}%
\end{pgfscope}%
\begin{pgfscope}%
\pgfpathrectangle{\pgfqpoint{0.539299in}{0.078740in}}{\pgfqpoint{7.842520in}{7.842520in}}%
\pgfusepath{clip}%
\pgfsetbuttcap%
\pgfsetroundjoin%
\definecolor{currentfill}{rgb}{0.535621,0.835785,0.281908}%
\pgfsetfillcolor{currentfill}%
\pgfsetlinewidth{0.000000pt}%
\definecolor{currentstroke}{rgb}{0.274128,0.199721,0.498911}%
\pgfsetstrokecolor{currentstroke}%
\pgfsetdash{}{0pt}%
\pgfpathmoveto{\pgfqpoint{2.919862in}{4.489063in}}%
\pgfpathlineto{\pgfqpoint{3.062545in}{4.438625in}}%
\pgfpathlineto{\pgfqpoint{3.002777in}{4.819100in}}%
\pgfpathclose%
\pgfusepath{fill}%
\end{pgfscope}%
\begin{pgfscope}%
\pgfpathrectangle{\pgfqpoint{0.539299in}{0.078740in}}{\pgfqpoint{7.842520in}{7.842520in}}%
\pgfusepath{clip}%
\pgfsetbuttcap%
\pgfsetroundjoin%
\definecolor{currentfill}{rgb}{0.730889,0.871916,0.156029}%
\pgfsetfillcolor{currentfill}%
\pgfsetlinewidth{0.000000pt}%
\definecolor{currentstroke}{rgb}{0.273006,0.204520,0.501721}%
\pgfsetstrokecolor{currentstroke}%
\pgfsetdash{}{0pt}%
\pgfpathmoveto{\pgfqpoint{3.750009in}{4.817083in}}%
\pgfpathlineto{\pgfqpoint{3.835937in}{4.901790in}}%
\pgfpathlineto{\pgfqpoint{3.604941in}{4.965350in}}%
\pgfpathclose%
\pgfusepath{fill}%
\end{pgfscope}%
\begin{pgfscope}%
\pgfpathrectangle{\pgfqpoint{0.539299in}{0.078740in}}{\pgfqpoint{7.842520in}{7.842520in}}%
\pgfusepath{clip}%
\pgfsetbuttcap%
\pgfsetroundjoin%
\definecolor{currentfill}{rgb}{0.555484,0.840254,0.269281}%
\pgfsetfillcolor{currentfill}%
\pgfsetlinewidth{0.000000pt}%
\definecolor{currentstroke}{rgb}{0.271828,0.209303,0.504434}%
\pgfsetstrokecolor{currentstroke}%
\pgfsetdash{}{0pt}%
\pgfpathmoveto{\pgfqpoint{3.895306in}{4.651896in}}%
\pgfpathlineto{\pgfqpoint{4.040710in}{4.473147in}}%
\pgfpathlineto{\pgfqpoint{3.980947in}{4.722912in}}%
\pgfpathclose%
\pgfusepath{fill}%
\end{pgfscope}%
\begin{pgfscope}%
\pgfpathrectangle{\pgfqpoint{0.539299in}{0.078740in}}{\pgfqpoint{7.842520in}{7.842520in}}%
\pgfusepath{clip}%
\pgfsetbuttcap%
\pgfsetroundjoin%
\definecolor{currentfill}{rgb}{0.132268,0.655014,0.519661}%
\pgfsetfillcolor{currentfill}%
\pgfsetlinewidth{0.000000pt}%
\definecolor{currentstroke}{rgb}{0.270595,0.214069,0.507052}%
\pgfsetstrokecolor{currentstroke}%
\pgfsetdash{}{0pt}%
\pgfpathmoveto{\pgfqpoint{4.766767in}{3.464787in}}%
\pgfpathlineto{\pgfqpoint{4.621833in}{3.675341in}}%
\pgfpathlineto{\pgfqpoint{4.537995in}{3.619381in}}%
\pgfpathclose%
\pgfusepath{fill}%
\end{pgfscope}%
\begin{pgfscope}%
\pgfpathrectangle{\pgfqpoint{0.539299in}{0.078740in}}{\pgfqpoint{7.842520in}{7.842520in}}%
\pgfusepath{clip}%
\pgfsetbuttcap%
\pgfsetroundjoin%
\definecolor{currentfill}{rgb}{0.626579,0.854645,0.223353}%
\pgfsetfillcolor{currentfill}%
\pgfsetlinewidth{0.000000pt}%
\definecolor{currentstroke}{rgb}{0.269308,0.218818,0.509577}%
\pgfsetstrokecolor{currentstroke}%
\pgfsetdash{}{0pt}%
\pgfpathmoveto{\pgfqpoint{3.980947in}{4.722912in}}%
\pgfpathlineto{\pgfqpoint{3.750009in}{4.817083in}}%
\pgfpathlineto{\pgfqpoint{3.895306in}{4.651896in}}%
\pgfpathclose%
\pgfusepath{fill}%
\end{pgfscope}%
\begin{pgfscope}%
\pgfpathrectangle{\pgfqpoint{0.539299in}{0.078740in}}{\pgfqpoint{7.842520in}{7.842520in}}%
\pgfusepath{clip}%
\pgfsetbuttcap%
\pgfsetroundjoin%
\definecolor{currentfill}{rgb}{0.237441,0.305202,0.541921}%
\pgfsetfillcolor{currentfill}%
\pgfsetlinewidth{0.000000pt}%
\definecolor{currentstroke}{rgb}{0.267968,0.223549,0.512008}%
\pgfsetstrokecolor{currentstroke}%
\pgfsetdash{}{0pt}%
\pgfpathmoveto{\pgfqpoint{3.746562in}{1.887616in}}%
\pgfpathlineto{\pgfqpoint{3.890478in}{1.820244in}}%
\pgfpathlineto{\pgfqpoint{3.971998in}{2.380078in}}%
\pgfpathclose%
\pgfusepath{fill}%
\end{pgfscope}%
\begin{pgfscope}%
\pgfpathrectangle{\pgfqpoint{0.539299in}{0.078740in}}{\pgfqpoint{7.842520in}{7.842520in}}%
\pgfusepath{clip}%
\pgfsetbuttcap%
\pgfsetroundjoin%
\definecolor{currentfill}{rgb}{0.282290,0.145912,0.461510}%
\pgfsetfillcolor{currentfill}%
\pgfsetlinewidth{0.000000pt}%
\definecolor{currentstroke}{rgb}{0.266580,0.228262,0.514349}%
\pgfsetstrokecolor{currentstroke}%
\pgfsetdash{}{0pt}%
\pgfpathmoveto{\pgfqpoint{5.824454in}{1.317143in}}%
\pgfpathlineto{\pgfqpoint{5.904099in}{1.379502in}}%
\pgfpathlineto{\pgfqpoint{5.760821in}{1.647418in}}%
\pgfpathclose%
\pgfusepath{fill}%
\end{pgfscope}%
\begin{pgfscope}%
\pgfpathrectangle{\pgfqpoint{0.539299in}{0.078740in}}{\pgfqpoint{7.842520in}{7.842520in}}%
\pgfusepath{clip}%
\pgfsetbuttcap%
\pgfsetroundjoin%
\definecolor{currentfill}{rgb}{0.720391,0.870350,0.162603}%
\pgfsetfillcolor{currentfill}%
\pgfsetlinewidth{0.000000pt}%
\definecolor{currentstroke}{rgb}{0.265145,0.232956,0.516599}%
\pgfsetstrokecolor{currentstroke}%
\pgfsetdash{}{0pt}%
\pgfpathmoveto{\pgfqpoint{3.087223in}{5.076965in}}%
\pgfpathlineto{\pgfqpoint{3.002777in}{4.819100in}}%
\pgfpathlineto{\pgfqpoint{3.145794in}{4.762662in}}%
\pgfpathclose%
\pgfusepath{fill}%
\end{pgfscope}%
\begin{pgfscope}%
\pgfpathrectangle{\pgfqpoint{0.539299in}{0.078740in}}{\pgfqpoint{7.842520in}{7.842520in}}%
\pgfusepath{clip}%
\pgfsetbuttcap%
\pgfsetroundjoin%
\definecolor{currentfill}{rgb}{0.241237,0.296485,0.539709}%
\pgfsetfillcolor{currentfill}%
\pgfsetlinewidth{0.000000pt}%
\definecolor{currentstroke}{rgb}{0.263663,0.237631,0.518762}%
\pgfsetstrokecolor{currentstroke}%
\pgfsetdash{}{0pt}%
\pgfpathmoveto{\pgfqpoint{5.616816in}{1.894378in}}%
\pgfpathlineto{\pgfqpoint{5.696616in}{1.925940in}}%
\pgfpathlineto{\pgfqpoint{5.472274in}{2.124924in}}%
\pgfpathclose%
\pgfusepath{fill}%
\end{pgfscope}%
\begin{pgfscope}%
\pgfpathrectangle{\pgfqpoint{0.539299in}{0.078740in}}{\pgfqpoint{7.842520in}{7.842520in}}%
\pgfusepath{clip}%
\pgfsetbuttcap%
\pgfsetroundjoin%
\definecolor{currentfill}{rgb}{0.204903,0.375746,0.553533}%
\pgfsetfillcolor{currentfill}%
\pgfsetlinewidth{0.000000pt}%
\definecolor{currentstroke}{rgb}{0.262138,0.242286,0.520837}%
\pgfsetstrokecolor{currentstroke}%
\pgfsetdash{}{0pt}%
\pgfpathmoveto{\pgfqpoint{3.746562in}{1.887616in}}%
\pgfpathlineto{\pgfqpoint{3.827198in}{2.468870in}}%
\pgfpathlineto{\pgfqpoint{3.682755in}{2.554388in}}%
\pgfpathclose%
\pgfusepath{fill}%
\end{pgfscope}%
\begin{pgfscope}%
\pgfpathrectangle{\pgfqpoint{0.539299in}{0.078740in}}{\pgfqpoint{7.842520in}{7.842520in}}%
\pgfusepath{clip}%
\pgfsetbuttcap%
\pgfsetroundjoin%
\definecolor{currentfill}{rgb}{0.154815,0.493313,0.557840}%
\pgfsetfillcolor{currentfill}%
\pgfsetlinewidth{0.000000pt}%
\definecolor{currentstroke}{rgb}{0.260571,0.246922,0.522828}%
\pgfsetstrokecolor{currentstroke}%
\pgfsetdash{}{0pt}%
\pgfpathmoveto{\pgfqpoint{3.619444in}{3.158415in}}%
\pgfpathlineto{\pgfqpoint{3.395040in}{2.713747in}}%
\pgfpathlineto{\pgfqpoint{3.538693in}{2.636201in}}%
\pgfpathclose%
\pgfusepath{fill}%
\end{pgfscope}%
\begin{pgfscope}%
\pgfpathrectangle{\pgfqpoint{0.539299in}{0.078740in}}{\pgfqpoint{7.842520in}{7.842520in}}%
\pgfusepath{clip}%
\pgfsetbuttcap%
\pgfsetroundjoin%
\definecolor{currentfill}{rgb}{0.175707,0.697900,0.491033}%
\pgfsetfillcolor{currentfill}%
\pgfsetlinewidth{0.000000pt}%
\definecolor{currentstroke}{rgb}{0.258965,0.251537,0.524736}%
\pgfsetstrokecolor{currentstroke}%
\pgfsetdash{}{0pt}%
\pgfpathmoveto{\pgfqpoint{3.124144in}{3.970746in}}%
\pgfpathlineto{\pgfqpoint{3.044293in}{3.487491in}}%
\pgfpathlineto{\pgfqpoint{3.267846in}{3.892059in}}%
\pgfpathclose%
\pgfusepath{fill}%
\end{pgfscope}%
\begin{pgfscope}%
\pgfpathrectangle{\pgfqpoint{0.539299in}{0.078740in}}{\pgfqpoint{7.842520in}{7.842520in}}%
\pgfusepath{clip}%
\pgfsetbuttcap%
\pgfsetroundjoin%
\definecolor{currentfill}{rgb}{0.132444,0.552216,0.553018}%
\pgfsetfillcolor{currentfill}%
\pgfsetlinewidth{0.000000pt}%
\definecolor{currentstroke}{rgb}{0.257322,0.256130,0.526563}%
\pgfsetstrokecolor{currentstroke}%
\pgfsetdash{}{0pt}%
\pgfpathmoveto{\pgfqpoint{3.330909in}{3.339983in}}%
\pgfpathlineto{\pgfqpoint{3.395040in}{2.713747in}}%
\pgfpathlineto{\pgfqpoint{3.474971in}{3.252817in}}%
\pgfpathclose%
\pgfusepath{fill}%
\end{pgfscope}%
\begin{pgfscope}%
\pgfpathrectangle{\pgfqpoint{0.539299in}{0.078740in}}{\pgfqpoint{7.842520in}{7.842520in}}%
\pgfusepath{clip}%
\pgfsetbuttcap%
\pgfsetroundjoin%
\definecolor{currentfill}{rgb}{0.311925,0.767822,0.415586}%
\pgfsetfillcolor{currentfill}%
\pgfsetlinewidth{0.000000pt}%
\definecolor{currentstroke}{rgb}{0.255645,0.260703,0.528312}%
\pgfsetstrokecolor{currentstroke}%
\pgfsetdash{}{0pt}%
\pgfpathmoveto{\pgfqpoint{3.124144in}{3.970746in}}%
\pgfpathlineto{\pgfqpoint{3.062545in}{4.438625in}}%
\pgfpathlineto{\pgfqpoint{2.981110in}{4.034439in}}%
\pgfpathclose%
\pgfusepath{fill}%
\end{pgfscope}%
\begin{pgfscope}%
\pgfpathrectangle{\pgfqpoint{0.539299in}{0.078740in}}{\pgfqpoint{7.842520in}{7.842520in}}%
\pgfusepath{clip}%
\pgfsetbuttcap%
\pgfsetroundjoin%
\definecolor{currentfill}{rgb}{0.157851,0.683765,0.501686}%
\pgfsetfillcolor{currentfill}%
\pgfsetlinewidth{0.000000pt}%
\definecolor{currentstroke}{rgb}{0.253935,0.265254,0.529983}%
\pgfsetstrokecolor{currentstroke}%
\pgfsetdash{}{0pt}%
\pgfpathmoveto{\pgfqpoint{4.537995in}{3.619381in}}%
\pgfpathlineto{\pgfqpoint{4.621833in}{3.675341in}}%
\pgfpathlineto{\pgfqpoint{4.476731in}{3.882966in}}%
\pgfpathclose%
\pgfusepath{fill}%
\end{pgfscope}%
\begin{pgfscope}%
\pgfpathrectangle{\pgfqpoint{0.539299in}{0.078740in}}{\pgfqpoint{7.842520in}{7.842520in}}%
\pgfusepath{clip}%
\pgfsetbuttcap%
\pgfsetroundjoin%
\definecolor{currentfill}{rgb}{0.835270,0.886029,0.102646}%
\pgfsetfillcolor{currentfill}%
\pgfsetlinewidth{0.000000pt}%
\definecolor{currentstroke}{rgb}{0.252194,0.269783,0.531579}%
\pgfsetstrokecolor{currentstroke}%
\pgfsetdash{}{0pt}%
\pgfpathmoveto{\pgfqpoint{3.316139in}{5.195751in}}%
\pgfpathlineto{\pgfqpoint{3.374513in}{4.921390in}}%
\pgfpathlineto{\pgfqpoint{3.460255in}{5.092948in}}%
\pgfpathclose%
\pgfusepath{fill}%
\end{pgfscope}%
\begin{pgfscope}%
\pgfpathrectangle{\pgfqpoint{0.539299in}{0.078740in}}{\pgfqpoint{7.842520in}{7.842520in}}%
\pgfusepath{clip}%
\pgfsetbuttcap%
\pgfsetroundjoin%
\definecolor{currentfill}{rgb}{0.121148,0.592739,0.544641}%
\pgfsetfillcolor{currentfill}%
\pgfsetlinewidth{0.000000pt}%
\definecolor{currentstroke}{rgb}{0.250425,0.274290,0.533103}%
\pgfsetstrokecolor{currentstroke}%
\pgfsetdash{}{0pt}%
\pgfpathmoveto{\pgfqpoint{4.683466in}{3.421156in}}%
\pgfpathlineto{\pgfqpoint{4.828804in}{3.219577in}}%
\pgfpathlineto{\pgfqpoint{4.911519in}{3.252217in}}%
\pgfpathclose%
\pgfusepath{fill}%
\end{pgfscope}%
\begin{pgfscope}%
\pgfpathrectangle{\pgfqpoint{0.539299in}{0.078740in}}{\pgfqpoint{7.842520in}{7.842520in}}%
\pgfusepath{clip}%
\pgfsetbuttcap%
\pgfsetroundjoin%
\definecolor{currentfill}{rgb}{0.772852,0.877868,0.131109}%
\pgfsetfillcolor{currentfill}%
\pgfsetlinewidth{0.000000pt}%
\definecolor{currentstroke}{rgb}{0.248629,0.278775,0.534556}%
\pgfsetstrokecolor{currentstroke}%
\pgfsetdash{}{0pt}%
\pgfpathmoveto{\pgfqpoint{3.087223in}{5.076965in}}%
\pgfpathlineto{\pgfqpoint{3.145794in}{4.762662in}}%
\pgfpathlineto{\pgfqpoint{3.230452in}{5.012748in}}%
\pgfpathclose%
\pgfusepath{fill}%
\end{pgfscope}%
\begin{pgfscope}%
\pgfpathrectangle{\pgfqpoint{0.539299in}{0.078740in}}{\pgfqpoint{7.842520in}{7.842520in}}%
\pgfusepath{clip}%
\pgfsetbuttcap%
\pgfsetroundjoin%
\definecolor{currentfill}{rgb}{0.190631,0.407061,0.556089}%
\pgfsetfillcolor{currentfill}%
\pgfsetlinewidth{0.000000pt}%
\definecolor{currentstroke}{rgb}{0.246811,0.283237,0.535941}%
\pgfsetstrokecolor{currentstroke}%
\pgfsetdash{}{0pt}%
\pgfpathmoveto{\pgfqpoint{5.327340in}{2.342890in}}%
\pgfpathlineto{\pgfqpoint{5.408389in}{2.379042in}}%
\pgfpathlineto{\pgfqpoint{5.263813in}{2.595433in}}%
\pgfpathclose%
\pgfusepath{fill}%
\end{pgfscope}%
\begin{pgfscope}%
\pgfpathrectangle{\pgfqpoint{0.539299in}{0.078740in}}{\pgfqpoint{7.842520in}{7.842520in}}%
\pgfusepath{clip}%
\pgfsetbuttcap%
\pgfsetroundjoin%
\definecolor{currentfill}{rgb}{0.824940,0.884720,0.106217}%
\pgfsetfillcolor{currentfill}%
\pgfsetlinewidth{0.000000pt}%
\definecolor{currentstroke}{rgb}{0.244972,0.287675,0.537260}%
\pgfsetstrokecolor{currentstroke}%
\pgfsetdash{}{0pt}%
\pgfpathmoveto{\pgfqpoint{3.230452in}{5.012748in}}%
\pgfpathlineto{\pgfqpoint{3.374513in}{4.921390in}}%
\pgfpathlineto{\pgfqpoint{3.316139in}{5.195751in}}%
\pgfpathclose%
\pgfusepath{fill}%
\end{pgfscope}%
\begin{pgfscope}%
\pgfpathrectangle{\pgfqpoint{0.539299in}{0.078740in}}{\pgfqpoint{7.842520in}{7.842520in}}%
\pgfusepath{clip}%
\pgfsetbuttcap%
\pgfsetroundjoin%
\definecolor{currentfill}{rgb}{0.606045,0.850733,0.236712}%
\pgfsetfillcolor{currentfill}%
\pgfsetlinewidth{0.000000pt}%
\definecolor{currentstroke}{rgb}{0.243113,0.292092,0.538516}%
\pgfsetstrokecolor{currentstroke}%
\pgfsetdash{}{0pt}%
\pgfpathmoveto{\pgfqpoint{3.002777in}{4.819100in}}%
\pgfpathlineto{\pgfqpoint{3.062545in}{4.438625in}}%
\pgfpathlineto{\pgfqpoint{3.145794in}{4.762662in}}%
\pgfpathclose%
\pgfusepath{fill}%
\end{pgfscope}%
\begin{pgfscope}%
\pgfpathrectangle{\pgfqpoint{0.539299in}{0.078740in}}{\pgfqpoint{7.842520in}{7.842520in}}%
\pgfusepath{clip}%
\pgfsetbuttcap%
\pgfsetroundjoin%
\definecolor{currentfill}{rgb}{0.226397,0.728888,0.462789}%
\pgfsetfillcolor{currentfill}%
\pgfsetlinewidth{0.000000pt}%
\definecolor{currentstroke}{rgb}{0.241237,0.296485,0.539709}%
\pgfsetstrokecolor{currentstroke}%
\pgfsetdash{}{0pt}%
\pgfpathmoveto{\pgfqpoint{4.331483in}{4.086352in}}%
\pgfpathlineto{\pgfqpoint{4.392412in}{3.813568in}}%
\pgfpathlineto{\pgfqpoint{4.476731in}{3.882966in}}%
\pgfpathclose%
\pgfusepath{fill}%
\end{pgfscope}%
\begin{pgfscope}%
\pgfpathrectangle{\pgfqpoint{0.539299in}{0.078740in}}{\pgfqpoint{7.842520in}{7.842520in}}%
\pgfusepath{clip}%
\pgfsetbuttcap%
\pgfsetroundjoin%
\definecolor{currentfill}{rgb}{0.804182,0.882046,0.114965}%
\pgfsetfillcolor{currentfill}%
\pgfsetlinewidth{0.000000pt}%
\definecolor{currentstroke}{rgb}{0.239346,0.300855,0.540844}%
\pgfsetstrokecolor{currentstroke}%
\pgfsetdash{}{0pt}%
\pgfpathmoveto{\pgfqpoint{3.460255in}{5.092948in}}%
\pgfpathlineto{\pgfqpoint{3.374513in}{4.921390in}}%
\pgfpathlineto{\pgfqpoint{3.604941in}{4.965350in}}%
\pgfpathclose%
\pgfusepath{fill}%
\end{pgfscope}%
\begin{pgfscope}%
\pgfpathrectangle{\pgfqpoint{0.539299in}{0.078740in}}{\pgfqpoint{7.842520in}{7.842520in}}%
\pgfusepath{clip}%
\pgfsetbuttcap%
\pgfsetroundjoin%
\definecolor{currentfill}{rgb}{0.210503,0.363727,0.552206}%
\pgfsetfillcolor{currentfill}%
\pgfsetlinewidth{0.000000pt}%
\definecolor{currentstroke}{rgb}{0.237441,0.305202,0.541921}%
\pgfsetstrokecolor{currentstroke}%
\pgfsetdash{}{0pt}%
\pgfpathmoveto{\pgfqpoint{3.971998in}{2.380078in}}%
\pgfpathlineto{\pgfqpoint{3.827198in}{2.468870in}}%
\pgfpathlineto{\pgfqpoint{3.746562in}{1.887616in}}%
\pgfpathclose%
\pgfusepath{fill}%
\end{pgfscope}%
\begin{pgfscope}%
\pgfpathrectangle{\pgfqpoint{0.539299in}{0.078740in}}{\pgfqpoint{7.842520in}{7.842520in}}%
\pgfusepath{clip}%
\pgfsetbuttcap%
\pgfsetroundjoin%
\definecolor{currentfill}{rgb}{0.124780,0.640461,0.527068}%
\pgfsetfillcolor{currentfill}%
\pgfsetlinewidth{0.000000pt}%
\definecolor{currentstroke}{rgb}{0.235526,0.309527,0.542944}%
\pgfsetstrokecolor{currentstroke}%
\pgfsetdash{}{0pt}%
\pgfpathmoveto{\pgfqpoint{4.537995in}{3.619381in}}%
\pgfpathlineto{\pgfqpoint{4.683466in}{3.421156in}}%
\pgfpathlineto{\pgfqpoint{4.766767in}{3.464787in}}%
\pgfpathclose%
\pgfusepath{fill}%
\end{pgfscope}%
\begin{pgfscope}%
\pgfpathrectangle{\pgfqpoint{0.539299in}{0.078740in}}{\pgfqpoint{7.842520in}{7.842520in}}%
\pgfusepath{clip}%
\pgfsetbuttcap%
\pgfsetroundjoin%
\definecolor{currentfill}{rgb}{0.136408,0.541173,0.554483}%
\pgfsetfillcolor{currentfill}%
\pgfsetlinewidth{0.000000pt}%
\definecolor{currentstroke}{rgb}{0.233603,0.313828,0.543914}%
\pgfsetstrokecolor{currentstroke}%
\pgfsetdash{}{0pt}%
\pgfpathmoveto{\pgfqpoint{3.474971in}{3.252817in}}%
\pgfpathlineto{\pgfqpoint{3.395040in}{2.713747in}}%
\pgfpathlineto{\pgfqpoint{3.619444in}{3.158415in}}%
\pgfpathclose%
\pgfusepath{fill}%
\end{pgfscope}%
\begin{pgfscope}%
\pgfpathrectangle{\pgfqpoint{0.539299in}{0.078740in}}{\pgfqpoint{7.842520in}{7.842520in}}%
\pgfusepath{clip}%
\pgfsetbuttcap%
\pgfsetroundjoin%
\definecolor{currentfill}{rgb}{0.239346,0.300855,0.540844}%
\pgfsetfillcolor{currentfill}%
\pgfsetlinewidth{0.000000pt}%
\definecolor{currentstroke}{rgb}{0.231674,0.318106,0.544834}%
\pgfsetstrokecolor{currentstroke}%
\pgfsetdash{}{0pt}%
\pgfpathmoveto{\pgfqpoint{3.971998in}{2.380078in}}%
\pgfpathlineto{\pgfqpoint{3.890478in}{1.820244in}}%
\pgfpathlineto{\pgfqpoint{4.034810in}{1.751253in}}%
\pgfpathclose%
\pgfusepath{fill}%
\end{pgfscope}%
\begin{pgfscope}%
\pgfpathrectangle{\pgfqpoint{0.539299in}{0.078740in}}{\pgfqpoint{7.842520in}{7.842520in}}%
\pgfusepath{clip}%
\pgfsetbuttcap%
\pgfsetroundjoin%
\definecolor{currentfill}{rgb}{0.126326,0.644107,0.525311}%
\pgfsetfillcolor{currentfill}%
\pgfsetlinewidth{0.000000pt}%
\definecolor{currentstroke}{rgb}{0.229739,0.322361,0.545706}%
\pgfsetstrokecolor{currentstroke}%
\pgfsetdash{}{0pt}%
\pgfpathmoveto{\pgfqpoint{3.187325in}{3.418679in}}%
\pgfpathlineto{\pgfqpoint{3.330909in}{3.339983in}}%
\pgfpathlineto{\pgfqpoint{3.412099in}{3.800510in}}%
\pgfpathclose%
\pgfusepath{fill}%
\end{pgfscope}%
\begin{pgfscope}%
\pgfpathrectangle{\pgfqpoint{0.539299in}{0.078740in}}{\pgfqpoint{7.842520in}{7.842520in}}%
\pgfusepath{clip}%
\pgfsetbuttcap%
\pgfsetroundjoin%
\definecolor{currentfill}{rgb}{0.377779,0.791781,0.377939}%
\pgfsetfillcolor{currentfill}%
\pgfsetlinewidth{0.000000pt}%
\definecolor{currentstroke}{rgb}{0.227802,0.326594,0.546532}%
\pgfsetstrokecolor{currentstroke}%
\pgfsetdash{}{0pt}%
\pgfpathmoveto{\pgfqpoint{3.206070in}{4.367077in}}%
\pgfpathlineto{\pgfqpoint{3.062545in}{4.438625in}}%
\pgfpathlineto{\pgfqpoint{3.124144in}{3.970746in}}%
\pgfpathclose%
\pgfusepath{fill}%
\end{pgfscope}%
\begin{pgfscope}%
\pgfpathrectangle{\pgfqpoint{0.539299in}{0.078740in}}{\pgfqpoint{7.842520in}{7.842520in}}%
\pgfusepath{clip}%
\pgfsetbuttcap%
\pgfsetroundjoin%
\definecolor{currentfill}{rgb}{0.204903,0.375746,0.553533}%
\pgfsetfillcolor{currentfill}%
\pgfsetlinewidth{0.000000pt}%
\definecolor{currentstroke}{rgb}{0.225863,0.330805,0.547314}%
\pgfsetstrokecolor{currentstroke}%
\pgfsetdash{}{0pt}%
\pgfpathmoveto{\pgfqpoint{5.408389in}{2.379042in}}%
\pgfpathlineto{\pgfqpoint{5.327340in}{2.342890in}}%
\pgfpathlineto{\pgfqpoint{5.472274in}{2.124924in}}%
\pgfpathclose%
\pgfusepath{fill}%
\end{pgfscope}%
\begin{pgfscope}%
\pgfpathrectangle{\pgfqpoint{0.539299in}{0.078740in}}{\pgfqpoint{7.842520in}{7.842520in}}%
\pgfusepath{clip}%
\pgfsetbuttcap%
\pgfsetroundjoin%
\definecolor{currentfill}{rgb}{0.162142,0.474838,0.558140}%
\pgfsetfillcolor{currentfill}%
\pgfsetlinewidth{0.000000pt}%
\definecolor{currentstroke}{rgb}{0.223925,0.334994,0.548053}%
\pgfsetstrokecolor{currentstroke}%
\pgfsetdash{}{0pt}%
\pgfpathmoveto{\pgfqpoint{5.263813in}{2.595433in}}%
\pgfpathlineto{\pgfqpoint{5.119001in}{2.807062in}}%
\pgfpathlineto{\pgfqpoint{5.036687in}{2.752478in}}%
\pgfpathclose%
\pgfusepath{fill}%
\end{pgfscope}%
\begin{pgfscope}%
\pgfpathrectangle{\pgfqpoint{0.539299in}{0.078740in}}{\pgfqpoint{7.842520in}{7.842520in}}%
\pgfusepath{clip}%
\pgfsetbuttcap%
\pgfsetroundjoin%
\definecolor{currentfill}{rgb}{0.344074,0.780029,0.397381}%
\pgfsetfillcolor{currentfill}%
\pgfsetlinewidth{0.000000pt}%
\definecolor{currentstroke}{rgb}{0.221989,0.339161,0.548752}%
\pgfsetstrokecolor{currentstroke}%
\pgfsetdash{}{0pt}%
\pgfpathmoveto{\pgfqpoint{4.186126in}{4.283787in}}%
\pgfpathlineto{\pgfqpoint{4.101044in}{4.185140in}}%
\pgfpathlineto{\pgfqpoint{4.331483in}{4.086352in}}%
\pgfpathclose%
\pgfusepath{fill}%
\end{pgfscope}%
\begin{pgfscope}%
\pgfpathrectangle{\pgfqpoint{0.539299in}{0.078740in}}{\pgfqpoint{7.842520in}{7.842520in}}%
\pgfusepath{clip}%
\pgfsetbuttcap%
\pgfsetroundjoin%
\definecolor{currentfill}{rgb}{0.159194,0.482237,0.558073}%
\pgfsetfillcolor{currentfill}%
\pgfsetlinewidth{0.000000pt}%
\definecolor{currentstroke}{rgb}{0.220057,0.343307,0.549413}%
\pgfsetstrokecolor{currentstroke}%
\pgfsetdash{}{0pt}%
\pgfpathmoveto{\pgfqpoint{3.764278in}{3.057833in}}%
\pgfpathlineto{\pgfqpoint{3.538693in}{2.636201in}}%
\pgfpathlineto{\pgfqpoint{3.682755in}{2.554388in}}%
\pgfpathclose%
\pgfusepath{fill}%
\end{pgfscope}%
\begin{pgfscope}%
\pgfpathrectangle{\pgfqpoint{0.539299in}{0.078740in}}{\pgfqpoint{7.842520in}{7.842520in}}%
\pgfusepath{clip}%
\pgfsetbuttcap%
\pgfsetroundjoin%
\definecolor{currentfill}{rgb}{0.162016,0.687316,0.499129}%
\pgfsetfillcolor{currentfill}%
\pgfsetlinewidth{0.000000pt}%
\definecolor{currentstroke}{rgb}{0.218130,0.347432,0.550038}%
\pgfsetstrokecolor{currentstroke}%
\pgfsetdash{}{0pt}%
\pgfpathmoveto{\pgfqpoint{3.412099in}{3.800510in}}%
\pgfpathlineto{\pgfqpoint{3.267846in}{3.892059in}}%
\pgfpathlineto{\pgfqpoint{3.187325in}{3.418679in}}%
\pgfpathclose%
\pgfusepath{fill}%
\end{pgfscope}%
\begin{pgfscope}%
\pgfpathrectangle{\pgfqpoint{0.539299in}{0.078740in}}{\pgfqpoint{7.842520in}{7.842520in}}%
\pgfusepath{clip}%
\pgfsetbuttcap%
\pgfsetroundjoin%
\definecolor{currentfill}{rgb}{0.699415,0.867117,0.175971}%
\pgfsetfillcolor{currentfill}%
\pgfsetlinewidth{0.000000pt}%
\definecolor{currentstroke}{rgb}{0.216210,0.351535,0.550627}%
\pgfsetstrokecolor{currentstroke}%
\pgfsetdash{}{0pt}%
\pgfpathmoveto{\pgfqpoint{3.230452in}{5.012748in}}%
\pgfpathlineto{\pgfqpoint{3.145794in}{4.762662in}}%
\pgfpathlineto{\pgfqpoint{3.289659in}{4.681825in}}%
\pgfpathclose%
\pgfusepath{fill}%
\end{pgfscope}%
\begin{pgfscope}%
\pgfpathrectangle{\pgfqpoint{0.539299in}{0.078740in}}{\pgfqpoint{7.842520in}{7.842520in}}%
\pgfusepath{clip}%
\pgfsetbuttcap%
\pgfsetroundjoin%
\definecolor{currentfill}{rgb}{0.585678,0.846661,0.249897}%
\pgfsetfillcolor{currentfill}%
\pgfsetlinewidth{0.000000pt}%
\definecolor{currentstroke}{rgb}{0.214298,0.355619,0.551184}%
\pgfsetstrokecolor{currentstroke}%
\pgfsetdash{}{0pt}%
\pgfpathmoveto{\pgfqpoint{3.062545in}{4.438625in}}%
\pgfpathlineto{\pgfqpoint{3.289659in}{4.681825in}}%
\pgfpathlineto{\pgfqpoint{3.145794in}{4.762662in}}%
\pgfpathclose%
\pgfusepath{fill}%
\end{pgfscope}%
\begin{pgfscope}%
\pgfpathrectangle{\pgfqpoint{0.539299in}{0.078740in}}{\pgfqpoint{7.842520in}{7.842520in}}%
\pgfusepath{clip}%
\pgfsetbuttcap%
\pgfsetroundjoin%
\definecolor{currentfill}{rgb}{0.412913,0.803041,0.357269}%
\pgfsetfillcolor{currentfill}%
\pgfsetlinewidth{0.000000pt}%
\definecolor{currentstroke}{rgb}{0.212395,0.359683,0.551710}%
\pgfsetstrokecolor{currentstroke}%
\pgfsetdash{}{0pt}%
\pgfpathmoveto{\pgfqpoint{4.101044in}{4.185140in}}%
\pgfpathlineto{\pgfqpoint{4.186126in}{4.283787in}}%
\pgfpathlineto{\pgfqpoint{4.040710in}{4.473147in}}%
\pgfpathclose%
\pgfusepath{fill}%
\end{pgfscope}%
\begin{pgfscope}%
\pgfpathrectangle{\pgfqpoint{0.539299in}{0.078740in}}{\pgfqpoint{7.842520in}{7.842520in}}%
\pgfusepath{clip}%
\pgfsetbuttcap%
\pgfsetroundjoin%
\definecolor{currentfill}{rgb}{0.699415,0.867117,0.175971}%
\pgfsetfillcolor{currentfill}%
\pgfsetlinewidth{0.000000pt}%
\definecolor{currentstroke}{rgb}{0.210503,0.363727,0.552206}%
\pgfsetstrokecolor{currentstroke}%
\pgfsetdash{}{0pt}%
\pgfpathmoveto{\pgfqpoint{3.664322in}{4.672860in}}%
\pgfpathlineto{\pgfqpoint{3.750009in}{4.817083in}}%
\pgfpathlineto{\pgfqpoint{3.604941in}{4.965350in}}%
\pgfpathclose%
\pgfusepath{fill}%
\end{pgfscope}%
\begin{pgfscope}%
\pgfpathrectangle{\pgfqpoint{0.539299in}{0.078740in}}{\pgfqpoint{7.842520in}{7.842520in}}%
\pgfusepath{clip}%
\pgfsetbuttcap%
\pgfsetroundjoin%
\definecolor{currentfill}{rgb}{0.180653,0.701402,0.488189}%
\pgfsetfillcolor{currentfill}%
\pgfsetlinewidth{0.000000pt}%
\definecolor{currentstroke}{rgb}{0.208623,0.367752,0.552675}%
\pgfsetstrokecolor{currentstroke}%
\pgfsetdash{}{0pt}%
\pgfpathmoveto{\pgfqpoint{4.476731in}{3.882966in}}%
\pgfpathlineto{\pgfqpoint{4.392412in}{3.813568in}}%
\pgfpathlineto{\pgfqpoint{4.537995in}{3.619381in}}%
\pgfpathclose%
\pgfusepath{fill}%
\end{pgfscope}%
\begin{pgfscope}%
\pgfpathrectangle{\pgfqpoint{0.539299in}{0.078740in}}{\pgfqpoint{7.842520in}{7.842520in}}%
\pgfusepath{clip}%
\pgfsetbuttcap%
\pgfsetroundjoin%
\definecolor{currentfill}{rgb}{0.751884,0.874951,0.143228}%
\pgfsetfillcolor{currentfill}%
\pgfsetlinewidth{0.000000pt}%
\definecolor{currentstroke}{rgb}{0.206756,0.371758,0.553117}%
\pgfsetstrokecolor{currentstroke}%
\pgfsetdash{}{0pt}%
\pgfpathmoveto{\pgfqpoint{3.604941in}{4.965350in}}%
\pgfpathlineto{\pgfqpoint{3.374513in}{4.921390in}}%
\pgfpathlineto{\pgfqpoint{3.519195in}{4.806876in}}%
\pgfpathclose%
\pgfusepath{fill}%
\end{pgfscope}%
\begin{pgfscope}%
\pgfpathrectangle{\pgfqpoint{0.539299in}{0.078740in}}{\pgfqpoint{7.842520in}{7.842520in}}%
\pgfusepath{clip}%
\pgfsetbuttcap%
\pgfsetroundjoin%
\definecolor{currentfill}{rgb}{0.741388,0.873449,0.149561}%
\pgfsetfillcolor{currentfill}%
\pgfsetlinewidth{0.000000pt}%
\definecolor{currentstroke}{rgb}{0.204903,0.375746,0.553533}%
\pgfsetstrokecolor{currentstroke}%
\pgfsetdash{}{0pt}%
\pgfpathmoveto{\pgfqpoint{3.289659in}{4.681825in}}%
\pgfpathlineto{\pgfqpoint{3.374513in}{4.921390in}}%
\pgfpathlineto{\pgfqpoint{3.230452in}{5.012748in}}%
\pgfpathclose%
\pgfusepath{fill}%
\end{pgfscope}%
\begin{pgfscope}%
\pgfpathrectangle{\pgfqpoint{0.539299in}{0.078740in}}{\pgfqpoint{7.842520in}{7.842520in}}%
\pgfusepath{clip}%
\pgfsetbuttcap%
\pgfsetroundjoin%
\definecolor{currentfill}{rgb}{0.278012,0.180367,0.486697}%
\pgfsetfillcolor{currentfill}%
\pgfsetlinewidth{0.000000pt}%
\definecolor{currentstroke}{rgb}{0.203063,0.379716,0.553925}%
\pgfsetstrokecolor{currentstroke}%
\pgfsetdash{}{0pt}%
\pgfpathmoveto{\pgfqpoint{5.760821in}{1.647418in}}%
\pgfpathlineto{\pgfqpoint{5.680690in}{1.591684in}}%
\pgfpathlineto{\pgfqpoint{5.824454in}{1.317143in}}%
\pgfpathclose%
\pgfusepath{fill}%
\end{pgfscope}%
\begin{pgfscope}%
\pgfpathrectangle{\pgfqpoint{0.539299in}{0.078740in}}{\pgfqpoint{7.842520in}{7.842520in}}%
\pgfusepath{clip}%
\pgfsetbuttcap%
\pgfsetroundjoin%
\definecolor{currentfill}{rgb}{0.150476,0.504369,0.557430}%
\pgfsetfillcolor{currentfill}%
\pgfsetlinewidth{0.000000pt}%
\definecolor{currentstroke}{rgb}{0.201239,0.383670,0.554294}%
\pgfsetstrokecolor{currentstroke}%
\pgfsetdash{}{0pt}%
\pgfpathmoveto{\pgfqpoint{5.036687in}{2.752478in}}%
\pgfpathlineto{\pgfqpoint{5.119001in}{2.807062in}}%
\pgfpathlineto{\pgfqpoint{4.973989in}{3.014923in}}%
\pgfpathclose%
\pgfusepath{fill}%
\end{pgfscope}%
\begin{pgfscope}%
\pgfpathrectangle{\pgfqpoint{0.539299in}{0.078740in}}{\pgfqpoint{7.842520in}{7.842520in}}%
\pgfusepath{clip}%
\pgfsetbuttcap%
\pgfsetroundjoin%
\definecolor{currentfill}{rgb}{0.545524,0.838039,0.275626}%
\pgfsetfillcolor{currentfill}%
\pgfsetlinewidth{0.000000pt}%
\definecolor{currentstroke}{rgb}{0.199430,0.387607,0.554642}%
\pgfsetstrokecolor{currentstroke}%
\pgfsetdash{}{0pt}%
\pgfpathmoveto{\pgfqpoint{3.895306in}{4.651896in}}%
\pgfpathlineto{\pgfqpoint{3.809748in}{4.522655in}}%
\pgfpathlineto{\pgfqpoint{4.040710in}{4.473147in}}%
\pgfpathclose%
\pgfusepath{fill}%
\end{pgfscope}%
\begin{pgfscope}%
\pgfpathrectangle{\pgfqpoint{0.539299in}{0.078740in}}{\pgfqpoint{7.842520in}{7.842520in}}%
\pgfusepath{clip}%
\pgfsetbuttcap%
\pgfsetroundjoin%
\definecolor{currentfill}{rgb}{0.606045,0.850733,0.236712}%
\pgfsetfillcolor{currentfill}%
\pgfsetlinewidth{0.000000pt}%
\definecolor{currentstroke}{rgb}{0.197636,0.391528,0.554969}%
\pgfsetstrokecolor{currentstroke}%
\pgfsetdash{}{0pt}%
\pgfpathmoveto{\pgfqpoint{3.895306in}{4.651896in}}%
\pgfpathlineto{\pgfqpoint{3.750009in}{4.817083in}}%
\pgfpathlineto{\pgfqpoint{3.809748in}{4.522655in}}%
\pgfpathclose%
\pgfusepath{fill}%
\end{pgfscope}%
\begin{pgfscope}%
\pgfpathrectangle{\pgfqpoint{0.539299in}{0.078740in}}{\pgfqpoint{7.842520in}{7.842520in}}%
\pgfusepath{clip}%
\pgfsetbuttcap%
\pgfsetroundjoin%
\definecolor{currentfill}{rgb}{0.123444,0.636809,0.528763}%
\pgfsetfillcolor{currentfill}%
\pgfsetlinewidth{0.000000pt}%
\definecolor{currentstroke}{rgb}{0.195860,0.395433,0.555276}%
\pgfsetstrokecolor{currentstroke}%
\pgfsetdash{}{0pt}%
\pgfpathmoveto{\pgfqpoint{3.412099in}{3.800510in}}%
\pgfpathlineto{\pgfqpoint{3.330909in}{3.339983in}}%
\pgfpathlineto{\pgfqpoint{3.474971in}{3.252817in}}%
\pgfpathclose%
\pgfusepath{fill}%
\end{pgfscope}%
\begin{pgfscope}%
\pgfpathrectangle{\pgfqpoint{0.539299in}{0.078740in}}{\pgfqpoint{7.842520in}{7.842520in}}%
\pgfusepath{clip}%
\pgfsetbuttcap%
\pgfsetroundjoin%
\definecolor{currentfill}{rgb}{0.257322,0.256130,0.526563}%
\pgfsetfillcolor{currentfill}%
\pgfsetlinewidth{0.000000pt}%
\definecolor{currentstroke}{rgb}{0.194100,0.399323,0.555565}%
\pgfsetstrokecolor{currentstroke}%
\pgfsetdash{}{0pt}%
\pgfpathmoveto{\pgfqpoint{5.760821in}{1.647418in}}%
\pgfpathlineto{\pgfqpoint{5.616816in}{1.894378in}}%
\pgfpathlineto{\pgfqpoint{5.536081in}{1.836733in}}%
\pgfpathclose%
\pgfusepath{fill}%
\end{pgfscope}%
\begin{pgfscope}%
\pgfpathrectangle{\pgfqpoint{0.539299in}{0.078740in}}{\pgfqpoint{7.842520in}{7.842520in}}%
\pgfusepath{clip}%
\pgfsetbuttcap%
\pgfsetroundjoin%
\definecolor{currentfill}{rgb}{0.515992,0.831158,0.294279}%
\pgfsetfillcolor{currentfill}%
\pgfsetlinewidth{0.000000pt}%
\definecolor{currentstroke}{rgb}{0.192357,0.403199,0.555836}%
\pgfsetstrokecolor{currentstroke}%
\pgfsetdash{}{0pt}%
\pgfpathmoveto{\pgfqpoint{3.206070in}{4.367077in}}%
\pgfpathlineto{\pgfqpoint{3.289659in}{4.681825in}}%
\pgfpathlineto{\pgfqpoint{3.062545in}{4.438625in}}%
\pgfpathclose%
\pgfusepath{fill}%
\end{pgfscope}%
\begin{pgfscope}%
\pgfpathrectangle{\pgfqpoint{0.539299in}{0.078740in}}{\pgfqpoint{7.842520in}{7.842520in}}%
\pgfusepath{clip}%
\pgfsetbuttcap%
\pgfsetroundjoin%
\definecolor{currentfill}{rgb}{0.180629,0.429975,0.557282}%
\pgfsetfillcolor{currentfill}%
\pgfsetlinewidth{0.000000pt}%
\definecolor{currentstroke}{rgb}{0.190631,0.407061,0.556089}%
\pgfsetstrokecolor{currentstroke}%
\pgfsetdash{}{0pt}%
\pgfpathmoveto{\pgfqpoint{5.263813in}{2.595433in}}%
\pgfpathlineto{\pgfqpoint{5.182120in}{2.551317in}}%
\pgfpathlineto{\pgfqpoint{5.327340in}{2.342890in}}%
\pgfpathclose%
\pgfusepath{fill}%
\end{pgfscope}%
\begin{pgfscope}%
\pgfpathrectangle{\pgfqpoint{0.539299in}{0.078740in}}{\pgfqpoint{7.842520in}{7.842520in}}%
\pgfusepath{clip}%
\pgfsetbuttcap%
\pgfsetroundjoin%
\definecolor{currentfill}{rgb}{0.288921,0.758394,0.428426}%
\pgfsetfillcolor{currentfill}%
\pgfsetlinewidth{0.000000pt}%
\definecolor{currentstroke}{rgb}{0.188923,0.410910,0.556326}%
\pgfsetstrokecolor{currentstroke}%
\pgfsetdash{}{0pt}%
\pgfpathmoveto{\pgfqpoint{3.124144in}{3.970746in}}%
\pgfpathlineto{\pgfqpoint{3.267846in}{3.892059in}}%
\pgfpathlineto{\pgfqpoint{3.350268in}{4.277351in}}%
\pgfpathclose%
\pgfusepath{fill}%
\end{pgfscope}%
\begin{pgfscope}%
\pgfpathrectangle{\pgfqpoint{0.539299in}{0.078740in}}{\pgfqpoint{7.842520in}{7.842520in}}%
\pgfusepath{clip}%
\pgfsetbuttcap%
\pgfsetroundjoin%
\definecolor{currentfill}{rgb}{0.252899,0.742211,0.448284}%
\pgfsetfillcolor{currentfill}%
\pgfsetlinewidth{0.000000pt}%
\definecolor{currentstroke}{rgb}{0.187231,0.414746,0.556547}%
\pgfsetstrokecolor{currentstroke}%
\pgfsetdash{}{0pt}%
\pgfpathmoveto{\pgfqpoint{4.246749in}{4.002642in}}%
\pgfpathlineto{\pgfqpoint{4.392412in}{3.813568in}}%
\pgfpathlineto{\pgfqpoint{4.331483in}{4.086352in}}%
\pgfpathclose%
\pgfusepath{fill}%
\end{pgfscope}%
\begin{pgfscope}%
\pgfpathrectangle{\pgfqpoint{0.539299in}{0.078740in}}{\pgfqpoint{7.842520in}{7.842520in}}%
\pgfusepath{clip}%
\pgfsetbuttcap%
\pgfsetroundjoin%
\definecolor{currentfill}{rgb}{0.141935,0.526453,0.555991}%
\pgfsetfillcolor{currentfill}%
\pgfsetlinewidth{0.000000pt}%
\definecolor{currentstroke}{rgb}{0.185556,0.418570,0.556753}%
\pgfsetstrokecolor{currentstroke}%
\pgfsetdash{}{0pt}%
\pgfpathmoveto{\pgfqpoint{3.764278in}{3.057833in}}%
\pgfpathlineto{\pgfqpoint{3.619444in}{3.158415in}}%
\pgfpathlineto{\pgfqpoint{3.538693in}{2.636201in}}%
\pgfpathclose%
\pgfusepath{fill}%
\end{pgfscope}%
\begin{pgfscope}%
\pgfpathrectangle{\pgfqpoint{0.539299in}{0.078740in}}{\pgfqpoint{7.842520in}{7.842520in}}%
\pgfusepath{clip}%
\pgfsetbuttcap%
\pgfsetroundjoin%
\definecolor{currentfill}{rgb}{0.360741,0.785964,0.387814}%
\pgfsetfillcolor{currentfill}%
\pgfsetlinewidth{0.000000pt}%
\definecolor{currentstroke}{rgb}{0.183898,0.422383,0.556944}%
\pgfsetstrokecolor{currentstroke}%
\pgfsetdash{}{0pt}%
\pgfpathmoveto{\pgfqpoint{3.206070in}{4.367077in}}%
\pgfpathlineto{\pgfqpoint{3.124144in}{3.970746in}}%
\pgfpathlineto{\pgfqpoint{3.350268in}{4.277351in}}%
\pgfpathclose%
\pgfusepath{fill}%
\end{pgfscope}%
\begin{pgfscope}%
\pgfpathrectangle{\pgfqpoint{0.539299in}{0.078740in}}{\pgfqpoint{7.842520in}{7.842520in}}%
\pgfusepath{clip}%
\pgfsetbuttcap%
\pgfsetroundjoin%
\definecolor{currentfill}{rgb}{0.709898,0.868751,0.169257}%
\pgfsetfillcolor{currentfill}%
\pgfsetlinewidth{0.000000pt}%
\definecolor{currentstroke}{rgb}{0.182256,0.426184,0.557120}%
\pgfsetstrokecolor{currentstroke}%
\pgfsetdash{}{0pt}%
\pgfpathmoveto{\pgfqpoint{3.519195in}{4.806876in}}%
\pgfpathlineto{\pgfqpoint{3.664322in}{4.672860in}}%
\pgfpathlineto{\pgfqpoint{3.604941in}{4.965350in}}%
\pgfpathclose%
\pgfusepath{fill}%
\end{pgfscope}%
\begin{pgfscope}%
\pgfpathrectangle{\pgfqpoint{0.539299in}{0.078740in}}{\pgfqpoint{7.842520in}{7.842520in}}%
\pgfusepath{clip}%
\pgfsetbuttcap%
\pgfsetroundjoin%
\definecolor{currentfill}{rgb}{0.304148,0.764704,0.419943}%
\pgfsetfillcolor{currentfill}%
\pgfsetlinewidth{0.000000pt}%
\definecolor{currentstroke}{rgb}{0.180629,0.429975,0.557282}%
\pgfsetstrokecolor{currentstroke}%
\pgfsetdash{}{0pt}%
\pgfpathmoveto{\pgfqpoint{4.331483in}{4.086352in}}%
\pgfpathlineto{\pgfqpoint{4.101044in}{4.185140in}}%
\pgfpathlineto{\pgfqpoint{4.246749in}{4.002642in}}%
\pgfpathclose%
\pgfusepath{fill}%
\end{pgfscope}%
\begin{pgfscope}%
\pgfpathrectangle{\pgfqpoint{0.539299in}{0.078740in}}{\pgfqpoint{7.842520in}{7.842520in}}%
\pgfusepath{clip}%
\pgfsetbuttcap%
\pgfsetroundjoin%
\definecolor{currentfill}{rgb}{0.132444,0.552216,0.553018}%
\pgfsetfillcolor{currentfill}%
\pgfsetlinewidth{0.000000pt}%
\definecolor{currentstroke}{rgb}{0.179019,0.433756,0.557430}%
\pgfsetstrokecolor{currentstroke}%
\pgfsetdash{}{0pt}%
\pgfpathmoveto{\pgfqpoint{4.828804in}{3.219577in}}%
\pgfpathlineto{\pgfqpoint{4.891094in}{2.947957in}}%
\pgfpathlineto{\pgfqpoint{4.973989in}{3.014923in}}%
\pgfpathclose%
\pgfusepath{fill}%
\end{pgfscope}%
\begin{pgfscope}%
\pgfpathrectangle{\pgfqpoint{0.539299in}{0.078740in}}{\pgfqpoint{7.842520in}{7.842520in}}%
\pgfusepath{clip}%
\pgfsetbuttcap%
\pgfsetroundjoin%
\definecolor{currentfill}{rgb}{0.168126,0.459988,0.558082}%
\pgfsetfillcolor{currentfill}%
\pgfsetlinewidth{0.000000pt}%
\definecolor{currentstroke}{rgb}{0.177423,0.437527,0.557565}%
\pgfsetstrokecolor{currentstroke}%
\pgfsetdash{}{0pt}%
\pgfpathmoveto{\pgfqpoint{5.036687in}{2.752478in}}%
\pgfpathlineto{\pgfqpoint{5.182120in}{2.551317in}}%
\pgfpathlineto{\pgfqpoint{5.263813in}{2.595433in}}%
\pgfpathclose%
\pgfusepath{fill}%
\end{pgfscope}%
\begin{pgfscope}%
\pgfpathrectangle{\pgfqpoint{0.539299in}{0.078740in}}{\pgfqpoint{7.842520in}{7.842520in}}%
\pgfusepath{clip}%
\pgfsetbuttcap%
\pgfsetroundjoin%
\definecolor{currentfill}{rgb}{0.699415,0.867117,0.175971}%
\pgfsetfillcolor{currentfill}%
\pgfsetlinewidth{0.000000pt}%
\definecolor{currentstroke}{rgb}{0.175841,0.441290,0.557685}%
\pgfsetstrokecolor{currentstroke}%
\pgfsetdash{}{0pt}%
\pgfpathmoveto{\pgfqpoint{3.519195in}{4.806876in}}%
\pgfpathlineto{\pgfqpoint{3.374513in}{4.921390in}}%
\pgfpathlineto{\pgfqpoint{3.289659in}{4.681825in}}%
\pgfpathclose%
\pgfusepath{fill}%
\end{pgfscope}%
\begin{pgfscope}%
\pgfpathrectangle{\pgfqpoint{0.539299in}{0.078740in}}{\pgfqpoint{7.842520in}{7.842520in}}%
\pgfusepath{clip}%
\pgfsetbuttcap%
\pgfsetroundjoin%
\definecolor{currentfill}{rgb}{0.626579,0.854645,0.223353}%
\pgfsetfillcolor{currentfill}%
\pgfsetlinewidth{0.000000pt}%
\definecolor{currentstroke}{rgb}{0.174274,0.445044,0.557792}%
\pgfsetstrokecolor{currentstroke}%
\pgfsetdash{}{0pt}%
\pgfpathmoveto{\pgfqpoint{3.809748in}{4.522655in}}%
\pgfpathlineto{\pgfqpoint{3.750009in}{4.817083in}}%
\pgfpathlineto{\pgfqpoint{3.664322in}{4.672860in}}%
\pgfpathclose%
\pgfusepath{fill}%
\end{pgfscope}%
\begin{pgfscope}%
\pgfpathrectangle{\pgfqpoint{0.539299in}{0.078740in}}{\pgfqpoint{7.842520in}{7.842520in}}%
\pgfusepath{clip}%
\pgfsetbuttcap%
\pgfsetroundjoin%
\definecolor{currentfill}{rgb}{0.165117,0.467423,0.558141}%
\pgfsetfillcolor{currentfill}%
\pgfsetlinewidth{0.000000pt}%
\definecolor{currentstroke}{rgb}{0.172719,0.448791,0.557885}%
\pgfsetstrokecolor{currentstroke}%
\pgfsetdash{}{0pt}%
\pgfpathmoveto{\pgfqpoint{3.682755in}{2.554388in}}%
\pgfpathlineto{\pgfqpoint{3.827198in}{2.468870in}}%
\pgfpathlineto{\pgfqpoint{3.909430in}{2.951944in}}%
\pgfpathclose%
\pgfusepath{fill}%
\end{pgfscope}%
\begin{pgfscope}%
\pgfpathrectangle{\pgfqpoint{0.539299in}{0.078740in}}{\pgfqpoint{7.842520in}{7.842520in}}%
\pgfusepath{clip}%
\pgfsetbuttcap%
\pgfsetroundjoin%
\definecolor{currentfill}{rgb}{0.430983,0.808473,0.346476}%
\pgfsetfillcolor{currentfill}%
\pgfsetlinewidth{0.000000pt}%
\definecolor{currentstroke}{rgb}{0.171176,0.452530,0.557965}%
\pgfsetstrokecolor{currentstroke}%
\pgfsetdash{}{0pt}%
\pgfpathmoveto{\pgfqpoint{4.101044in}{4.185140in}}%
\pgfpathlineto{\pgfqpoint{4.040710in}{4.473147in}}%
\pgfpathlineto{\pgfqpoint{3.955353in}{4.359218in}}%
\pgfpathclose%
\pgfusepath{fill}%
\end{pgfscope}%
\begin{pgfscope}%
\pgfpathrectangle{\pgfqpoint{0.539299in}{0.078740in}}{\pgfqpoint{7.842520in}{7.842520in}}%
\pgfusepath{clip}%
\pgfsetbuttcap%
\pgfsetroundjoin%
\definecolor{currentfill}{rgb}{0.496615,0.826376,0.306377}%
\pgfsetfillcolor{currentfill}%
\pgfsetlinewidth{0.000000pt}%
\definecolor{currentstroke}{rgb}{0.169646,0.456262,0.558030}%
\pgfsetstrokecolor{currentstroke}%
\pgfsetdash{}{0pt}%
\pgfpathmoveto{\pgfqpoint{4.040710in}{4.473147in}}%
\pgfpathlineto{\pgfqpoint{3.809748in}{4.522655in}}%
\pgfpathlineto{\pgfqpoint{3.955353in}{4.359218in}}%
\pgfpathclose%
\pgfusepath{fill}%
\end{pgfscope}%
\begin{pgfscope}%
\pgfpathrectangle{\pgfqpoint{0.539299in}{0.078740in}}{\pgfqpoint{7.842520in}{7.842520in}}%
\pgfusepath{clip}%
\pgfsetbuttcap%
\pgfsetroundjoin%
\definecolor{currentfill}{rgb}{0.237441,0.305202,0.541921}%
\pgfsetfillcolor{currentfill}%
\pgfsetlinewidth{0.000000pt}%
\definecolor{currentstroke}{rgb}{0.168126,0.459988,0.558082}%
\pgfsetstrokecolor{currentstroke}%
\pgfsetdash{}{0pt}%
\pgfpathmoveto{\pgfqpoint{5.472274in}{2.124924in}}%
\pgfpathlineto{\pgfqpoint{5.536081in}{1.836733in}}%
\pgfpathlineto{\pgfqpoint{5.616816in}{1.894378in}}%
\pgfpathclose%
\pgfusepath{fill}%
\end{pgfscope}%
\begin{pgfscope}%
\pgfpathrectangle{\pgfqpoint{0.539299in}{0.078740in}}{\pgfqpoint{7.842520in}{7.842520in}}%
\pgfusepath{clip}%
\pgfsetbuttcap%
\pgfsetroundjoin%
\definecolor{currentfill}{rgb}{0.265145,0.232956,0.516599}%
\pgfsetfillcolor{currentfill}%
\pgfsetlinewidth{0.000000pt}%
\definecolor{currentstroke}{rgb}{0.166617,0.463708,0.558119}%
\pgfsetstrokecolor{currentstroke}%
\pgfsetdash{}{0pt}%
\pgfpathmoveto{\pgfqpoint{5.536081in}{1.836733in}}%
\pgfpathlineto{\pgfqpoint{5.680690in}{1.591684in}}%
\pgfpathlineto{\pgfqpoint{5.760821in}{1.647418in}}%
\pgfpathclose%
\pgfusepath{fill}%
\end{pgfscope}%
\begin{pgfscope}%
\pgfpathrectangle{\pgfqpoint{0.539299in}{0.078740in}}{\pgfqpoint{7.842520in}{7.842520in}}%
\pgfusepath{clip}%
\pgfsetbuttcap%
\pgfsetroundjoin%
\definecolor{currentfill}{rgb}{0.216210,0.351535,0.550627}%
\pgfsetfillcolor{currentfill}%
\pgfsetlinewidth{0.000000pt}%
\definecolor{currentstroke}{rgb}{0.165117,0.467423,0.558141}%
\pgfsetstrokecolor{currentstroke}%
\pgfsetdash{}{0pt}%
\pgfpathmoveto{\pgfqpoint{4.034810in}{1.751253in}}%
\pgfpathlineto{\pgfqpoint{4.117136in}{2.288294in}}%
\pgfpathlineto{\pgfqpoint{3.971998in}{2.380078in}}%
\pgfpathclose%
\pgfusepath{fill}%
\end{pgfscope}%
\begin{pgfscope}%
\pgfpathrectangle{\pgfqpoint{0.539299in}{0.078740in}}{\pgfqpoint{7.842520in}{7.842520in}}%
\pgfusepath{clip}%
\pgfsetbuttcap%
\pgfsetroundjoin%
\definecolor{currentfill}{rgb}{0.246811,0.283237,0.535941}%
\pgfsetfillcolor{currentfill}%
\pgfsetlinewidth{0.000000pt}%
\definecolor{currentstroke}{rgb}{0.163625,0.471133,0.558148}%
\pgfsetstrokecolor{currentstroke}%
\pgfsetdash{}{0pt}%
\pgfpathmoveto{\pgfqpoint{4.262595in}{2.193629in}}%
\pgfpathlineto{\pgfqpoint{4.034810in}{1.751253in}}%
\pgfpathlineto{\pgfqpoint{4.179552in}{1.680684in}}%
\pgfpathclose%
\pgfusepath{fill}%
\end{pgfscope}%
\begin{pgfscope}%
\pgfpathrectangle{\pgfqpoint{0.539299in}{0.078740in}}{\pgfqpoint{7.842520in}{7.842520in}}%
\pgfusepath{clip}%
\pgfsetbuttcap%
\pgfsetroundjoin%
\definecolor{currentfill}{rgb}{0.120565,0.596422,0.543611}%
\pgfsetfillcolor{currentfill}%
\pgfsetlinewidth{0.000000pt}%
\definecolor{currentstroke}{rgb}{0.162142,0.474838,0.558140}%
\pgfsetstrokecolor{currentstroke}%
\pgfsetdash{}{0pt}%
\pgfpathmoveto{\pgfqpoint{4.745380in}{3.138731in}}%
\pgfpathlineto{\pgfqpoint{4.828804in}{3.219577in}}%
\pgfpathlineto{\pgfqpoint{4.683466in}{3.421156in}}%
\pgfpathclose%
\pgfusepath{fill}%
\end{pgfscope}%
\begin{pgfscope}%
\pgfpathrectangle{\pgfqpoint{0.539299in}{0.078740in}}{\pgfqpoint{7.842520in}{7.842520in}}%
\pgfusepath{clip}%
\pgfsetbuttcap%
\pgfsetroundjoin%
\definecolor{currentfill}{rgb}{0.496615,0.826376,0.306377}%
\pgfsetfillcolor{currentfill}%
\pgfsetlinewidth{0.000000pt}%
\definecolor{currentstroke}{rgb}{0.160665,0.478540,0.558115}%
\pgfsetstrokecolor{currentstroke}%
\pgfsetdash{}{0pt}%
\pgfpathmoveto{\pgfqpoint{3.350268in}{4.277351in}}%
\pgfpathlineto{\pgfqpoint{3.289659in}{4.681825in}}%
\pgfpathlineto{\pgfqpoint{3.206070in}{4.367077in}}%
\pgfpathclose%
\pgfusepath{fill}%
\end{pgfscope}%
\begin{pgfscope}%
\pgfpathrectangle{\pgfqpoint{0.539299in}{0.078740in}}{\pgfqpoint{7.842520in}{7.842520in}}%
\pgfusepath{clip}%
\pgfsetbuttcap%
\pgfsetroundjoin%
\definecolor{currentfill}{rgb}{0.636902,0.856542,0.216620}%
\pgfsetfillcolor{currentfill}%
\pgfsetlinewidth{0.000000pt}%
\definecolor{currentstroke}{rgb}{0.159194,0.482237,0.558073}%
\pgfsetstrokecolor{currentstroke}%
\pgfsetdash{}{0pt}%
\pgfpathmoveto{\pgfqpoint{3.289659in}{4.681825in}}%
\pgfpathlineto{\pgfqpoint{3.434182in}{4.580061in}}%
\pgfpathlineto{\pgfqpoint{3.519195in}{4.806876in}}%
\pgfpathclose%
\pgfusepath{fill}%
\end{pgfscope}%
\begin{pgfscope}%
\pgfpathrectangle{\pgfqpoint{0.539299in}{0.078740in}}{\pgfqpoint{7.842520in}{7.842520in}}%
\pgfusepath{clip}%
\pgfsetbuttcap%
\pgfsetroundjoin%
\definecolor{currentfill}{rgb}{0.143303,0.669459,0.511215}%
\pgfsetfillcolor{currentfill}%
\pgfsetlinewidth{0.000000pt}%
\definecolor{currentstroke}{rgb}{0.157729,0.485932,0.558013}%
\pgfsetstrokecolor{currentstroke}%
\pgfsetdash{}{0pt}%
\pgfpathmoveto{\pgfqpoint{3.474971in}{3.252817in}}%
\pgfpathlineto{\pgfqpoint{3.556804in}{3.697996in}}%
\pgfpathlineto{\pgfqpoint{3.412099in}{3.800510in}}%
\pgfpathclose%
\pgfusepath{fill}%
\end{pgfscope}%
\begin{pgfscope}%
\pgfpathrectangle{\pgfqpoint{0.539299in}{0.078740in}}{\pgfqpoint{7.842520in}{7.842520in}}%
\pgfusepath{clip}%
\pgfsetbuttcap%
\pgfsetroundjoin%
\definecolor{currentfill}{rgb}{0.143343,0.522773,0.556295}%
\pgfsetfillcolor{currentfill}%
\pgfsetlinewidth{0.000000pt}%
\definecolor{currentstroke}{rgb}{0.156270,0.489624,0.557936}%
\pgfsetstrokecolor{currentstroke}%
\pgfsetdash{}{0pt}%
\pgfpathmoveto{\pgfqpoint{4.973989in}{3.014923in}}%
\pgfpathlineto{\pgfqpoint{4.891094in}{2.947957in}}%
\pgfpathlineto{\pgfqpoint{5.036687in}{2.752478in}}%
\pgfpathclose%
\pgfusepath{fill}%
\end{pgfscope}%
\begin{pgfscope}%
\pgfpathrectangle{\pgfqpoint{0.539299in}{0.078740in}}{\pgfqpoint{7.842520in}{7.842520in}}%
\pgfusepath{clip}%
\pgfsetbuttcap%
\pgfsetroundjoin%
\definecolor{currentfill}{rgb}{0.124780,0.640461,0.527068}%
\pgfsetfillcolor{currentfill}%
\pgfsetlinewidth{0.000000pt}%
\definecolor{currentstroke}{rgb}{0.154815,0.493313,0.557840}%
\pgfsetstrokecolor{currentstroke}%
\pgfsetdash{}{0pt}%
\pgfpathmoveto{\pgfqpoint{4.599574in}{3.325252in}}%
\pgfpathlineto{\pgfqpoint{4.683466in}{3.421156in}}%
\pgfpathlineto{\pgfqpoint{4.537995in}{3.619381in}}%
\pgfpathclose%
\pgfusepath{fill}%
\end{pgfscope}%
\begin{pgfscope}%
\pgfpathrectangle{\pgfqpoint{0.539299in}{0.078740in}}{\pgfqpoint{7.842520in}{7.842520in}}%
\pgfusepath{clip}%
\pgfsetbuttcap%
\pgfsetroundjoin%
\definecolor{currentfill}{rgb}{0.259857,0.745492,0.444467}%
\pgfsetfillcolor{currentfill}%
\pgfsetlinewidth{0.000000pt}%
\definecolor{currentstroke}{rgb}{0.153364,0.497000,0.557724}%
\pgfsetstrokecolor{currentstroke}%
\pgfsetdash{}{0pt}%
\pgfpathmoveto{\pgfqpoint{3.267846in}{3.892059in}}%
\pgfpathlineto{\pgfqpoint{3.412099in}{3.800510in}}%
\pgfpathlineto{\pgfqpoint{3.494996in}{4.172102in}}%
\pgfpathclose%
\pgfusepath{fill}%
\end{pgfscope}%
\begin{pgfscope}%
\pgfpathrectangle{\pgfqpoint{0.539299in}{0.078740in}}{\pgfqpoint{7.842520in}{7.842520in}}%
\pgfusepath{clip}%
\pgfsetbuttcap%
\pgfsetroundjoin%
\definecolor{currentfill}{rgb}{0.147607,0.511733,0.557049}%
\pgfsetfillcolor{currentfill}%
\pgfsetlinewidth{0.000000pt}%
\definecolor{currentstroke}{rgb}{0.151918,0.500685,0.557587}%
\pgfsetstrokecolor{currentstroke}%
\pgfsetdash{}{0pt}%
\pgfpathmoveto{\pgfqpoint{3.909430in}{2.951944in}}%
\pgfpathlineto{\pgfqpoint{3.764278in}{3.057833in}}%
\pgfpathlineto{\pgfqpoint{3.682755in}{2.554388in}}%
\pgfpathclose%
\pgfusepath{fill}%
\end{pgfscope}%
\begin{pgfscope}%
\pgfpathrectangle{\pgfqpoint{0.539299in}{0.078740in}}{\pgfqpoint{7.842520in}{7.842520in}}%
\pgfusepath{clip}%
\pgfsetbuttcap%
\pgfsetroundjoin%
\definecolor{currentfill}{rgb}{0.535621,0.835785,0.281908}%
\pgfsetfillcolor{currentfill}%
\pgfsetlinewidth{0.000000pt}%
\definecolor{currentstroke}{rgb}{0.150476,0.504369,0.557430}%
\pgfsetstrokecolor{currentstroke}%
\pgfsetdash{}{0pt}%
\pgfpathmoveto{\pgfqpoint{3.434182in}{4.580061in}}%
\pgfpathlineto{\pgfqpoint{3.289659in}{4.681825in}}%
\pgfpathlineto{\pgfqpoint{3.350268in}{4.277351in}}%
\pgfpathclose%
\pgfusepath{fill}%
\end{pgfscope}%
\begin{pgfscope}%
\pgfpathrectangle{\pgfqpoint{0.539299in}{0.078740in}}{\pgfqpoint{7.842520in}{7.842520in}}%
\pgfusepath{clip}%
\pgfsetbuttcap%
\pgfsetroundjoin%
\definecolor{currentfill}{rgb}{0.119483,0.614817,0.537692}%
\pgfsetfillcolor{currentfill}%
\pgfsetlinewidth{0.000000pt}%
\definecolor{currentstroke}{rgb}{0.149039,0.508051,0.557250}%
\pgfsetstrokecolor{currentstroke}%
\pgfsetdash{}{0pt}%
\pgfpathmoveto{\pgfqpoint{3.474971in}{3.252817in}}%
\pgfpathlineto{\pgfqpoint{3.619444in}{3.158415in}}%
\pgfpathlineto{\pgfqpoint{3.701876in}{3.586183in}}%
\pgfpathclose%
\pgfusepath{fill}%
\end{pgfscope}%
\begin{pgfscope}%
\pgfpathrectangle{\pgfqpoint{0.539299in}{0.078740in}}{\pgfqpoint{7.842520in}{7.842520in}}%
\pgfusepath{clip}%
\pgfsetbuttcap%
\pgfsetroundjoin%
\definecolor{currentfill}{rgb}{0.327796,0.773980,0.406640}%
\pgfsetfillcolor{currentfill}%
\pgfsetlinewidth{0.000000pt}%
\definecolor{currentstroke}{rgb}{0.147607,0.511733,0.557049}%
\pgfsetstrokecolor{currentstroke}%
\pgfsetdash{}{0pt}%
\pgfpathmoveto{\pgfqpoint{3.494996in}{4.172102in}}%
\pgfpathlineto{\pgfqpoint{3.350268in}{4.277351in}}%
\pgfpathlineto{\pgfqpoint{3.267846in}{3.892059in}}%
\pgfpathclose%
\pgfusepath{fill}%
\end{pgfscope}%
\begin{pgfscope}%
\pgfpathrectangle{\pgfqpoint{0.539299in}{0.078740in}}{\pgfqpoint{7.842520in}{7.842520in}}%
\pgfusepath{clip}%
\pgfsetbuttcap%
\pgfsetroundjoin%
\definecolor{currentfill}{rgb}{0.616293,0.852709,0.230052}%
\pgfsetfillcolor{currentfill}%
\pgfsetlinewidth{0.000000pt}%
\definecolor{currentstroke}{rgb}{0.146180,0.515413,0.556823}%
\pgfsetstrokecolor{currentstroke}%
\pgfsetdash{}{0pt}%
\pgfpathmoveto{\pgfqpoint{3.579199in}{4.460532in}}%
\pgfpathlineto{\pgfqpoint{3.664322in}{4.672860in}}%
\pgfpathlineto{\pgfqpoint{3.519195in}{4.806876in}}%
\pgfpathclose%
\pgfusepath{fill}%
\end{pgfscope}%
\begin{pgfscope}%
\pgfpathrectangle{\pgfqpoint{0.539299in}{0.078740in}}{\pgfqpoint{7.842520in}{7.842520in}}%
\pgfusepath{clip}%
\pgfsetbuttcap%
\pgfsetroundjoin%
\definecolor{currentfill}{rgb}{0.168126,0.459988,0.558082}%
\pgfsetfillcolor{currentfill}%
\pgfsetlinewidth{0.000000pt}%
\definecolor{currentstroke}{rgb}{0.144759,0.519093,0.556572}%
\pgfsetstrokecolor{currentstroke}%
\pgfsetdash{}{0pt}%
\pgfpathmoveto{\pgfqpoint{3.909430in}{2.951944in}}%
\pgfpathlineto{\pgfqpoint{3.827198in}{2.468870in}}%
\pgfpathlineto{\pgfqpoint{3.971998in}{2.380078in}}%
\pgfpathclose%
\pgfusepath{fill}%
\end{pgfscope}%
\begin{pgfscope}%
\pgfpathrectangle{\pgfqpoint{0.539299in}{0.078740in}}{\pgfqpoint{7.842520in}{7.842520in}}%
\pgfusepath{clip}%
\pgfsetbuttcap%
\pgfsetroundjoin%
\definecolor{currentfill}{rgb}{0.221989,0.339161,0.548752}%
\pgfsetfillcolor{currentfill}%
\pgfsetlinewidth{0.000000pt}%
\definecolor{currentstroke}{rgb}{0.143343,0.522773,0.556295}%
\pgfsetstrokecolor{currentstroke}%
\pgfsetdash{}{0pt}%
\pgfpathmoveto{\pgfqpoint{4.117136in}{2.288294in}}%
\pgfpathlineto{\pgfqpoint{4.034810in}{1.751253in}}%
\pgfpathlineto{\pgfqpoint{4.262595in}{2.193629in}}%
\pgfpathclose%
\pgfusepath{fill}%
\end{pgfscope}%
\begin{pgfscope}%
\pgfpathrectangle{\pgfqpoint{0.539299in}{0.078740in}}{\pgfqpoint{7.842520in}{7.842520in}}%
\pgfusepath{clip}%
\pgfsetbuttcap%
\pgfsetroundjoin%
\definecolor{currentfill}{rgb}{0.606045,0.850733,0.236712}%
\pgfsetfillcolor{currentfill}%
\pgfsetlinewidth{0.000000pt}%
\definecolor{currentstroke}{rgb}{0.141935,0.526453,0.555991}%
\pgfsetstrokecolor{currentstroke}%
\pgfsetdash{}{0pt}%
\pgfpathmoveto{\pgfqpoint{3.519195in}{4.806876in}}%
\pgfpathlineto{\pgfqpoint{3.434182in}{4.580061in}}%
\pgfpathlineto{\pgfqpoint{3.579199in}{4.460532in}}%
\pgfpathclose%
\pgfusepath{fill}%
\end{pgfscope}%
\begin{pgfscope}%
\pgfpathrectangle{\pgfqpoint{0.539299in}{0.078740in}}{\pgfqpoint{7.842520in}{7.842520in}}%
\pgfusepath{clip}%
\pgfsetbuttcap%
\pgfsetroundjoin%
\definecolor{currentfill}{rgb}{0.127568,0.566949,0.550556}%
\pgfsetfillcolor{currentfill}%
\pgfsetlinewidth{0.000000pt}%
\definecolor{currentstroke}{rgb}{0.140536,0.530132,0.555659}%
\pgfsetstrokecolor{currentstroke}%
\pgfsetdash{}{0pt}%
\pgfpathmoveto{\pgfqpoint{4.745380in}{3.138731in}}%
\pgfpathlineto{\pgfqpoint{4.891094in}{2.947957in}}%
\pgfpathlineto{\pgfqpoint{4.828804in}{3.219577in}}%
\pgfpathclose%
\pgfusepath{fill}%
\end{pgfscope}%
\begin{pgfscope}%
\pgfpathrectangle{\pgfqpoint{0.539299in}{0.078740in}}{\pgfqpoint{7.842520in}{7.842520in}}%
\pgfusepath{clip}%
\pgfsetbuttcap%
\pgfsetroundjoin%
\definecolor{currentfill}{rgb}{0.175707,0.697900,0.491033}%
\pgfsetfillcolor{currentfill}%
\pgfsetlinewidth{0.000000pt}%
\definecolor{currentstroke}{rgb}{0.139147,0.533812,0.555298}%
\pgfsetstrokecolor{currentstroke}%
\pgfsetdash{}{0pt}%
\pgfpathmoveto{\pgfqpoint{4.537995in}{3.619381in}}%
\pgfpathlineto{\pgfqpoint{4.392412in}{3.813568in}}%
\pgfpathlineto{\pgfqpoint{4.307790in}{3.685037in}}%
\pgfpathclose%
\pgfusepath{fill}%
\end{pgfscope}%
\begin{pgfscope}%
\pgfpathrectangle{\pgfqpoint{0.539299in}{0.078740in}}{\pgfqpoint{7.842520in}{7.842520in}}%
\pgfusepath{clip}%
\pgfsetbuttcap%
\pgfsetroundjoin%
\definecolor{currentfill}{rgb}{0.565498,0.842430,0.262877}%
\pgfsetfillcolor{currentfill}%
\pgfsetlinewidth{0.000000pt}%
\definecolor{currentstroke}{rgb}{0.137770,0.537492,0.554906}%
\pgfsetstrokecolor{currentstroke}%
\pgfsetdash{}{0pt}%
\pgfpathmoveto{\pgfqpoint{3.579199in}{4.460532in}}%
\pgfpathlineto{\pgfqpoint{3.809748in}{4.522655in}}%
\pgfpathlineto{\pgfqpoint{3.664322in}{4.672860in}}%
\pgfpathclose%
\pgfusepath{fill}%
\end{pgfscope}%
\begin{pgfscope}%
\pgfpathrectangle{\pgfqpoint{0.539299in}{0.078740in}}{\pgfqpoint{7.842520in}{7.842520in}}%
\pgfusepath{clip}%
\pgfsetbuttcap%
\pgfsetroundjoin%
\definecolor{currentfill}{rgb}{0.132268,0.655014,0.519661}%
\pgfsetfillcolor{currentfill}%
\pgfsetlinewidth{0.000000pt}%
\definecolor{currentstroke}{rgb}{0.136408,0.541173,0.554483}%
\pgfsetstrokecolor{currentstroke}%
\pgfsetdash{}{0pt}%
\pgfpathmoveto{\pgfqpoint{3.701876in}{3.586183in}}%
\pgfpathlineto{\pgfqpoint{3.556804in}{3.697996in}}%
\pgfpathlineto{\pgfqpoint{3.474971in}{3.252817in}}%
\pgfpathclose%
\pgfusepath{fill}%
\end{pgfscope}%
\begin{pgfscope}%
\pgfpathrectangle{\pgfqpoint{0.539299in}{0.078740in}}{\pgfqpoint{7.842520in}{7.842520in}}%
\pgfusepath{clip}%
\pgfsetbuttcap%
\pgfsetroundjoin%
\definecolor{currentfill}{rgb}{0.203063,0.379716,0.553925}%
\pgfsetfillcolor{currentfill}%
\pgfsetlinewidth{0.000000pt}%
\definecolor{currentstroke}{rgb}{0.135066,0.544853,0.554029}%
\pgfsetstrokecolor{currentstroke}%
\pgfsetdash{}{0pt}%
\pgfpathmoveto{\pgfqpoint{5.472274in}{2.124924in}}%
\pgfpathlineto{\pgfqpoint{5.327340in}{2.342890in}}%
\pgfpathlineto{\pgfqpoint{5.245331in}{2.266200in}}%
\pgfpathclose%
\pgfusepath{fill}%
\end{pgfscope}%
\begin{pgfscope}%
\pgfpathrectangle{\pgfqpoint{0.539299in}{0.078740in}}{\pgfqpoint{7.842520in}{7.842520in}}%
\pgfusepath{clip}%
\pgfsetbuttcap%
\pgfsetroundjoin%
\definecolor{currentfill}{rgb}{0.248629,0.278775,0.534556}%
\pgfsetfillcolor{currentfill}%
\pgfsetlinewidth{0.000000pt}%
\definecolor{currentstroke}{rgb}{0.133743,0.548535,0.553541}%
\pgfsetstrokecolor{currentstroke}%
\pgfsetdash{}{0pt}%
\pgfpathmoveto{\pgfqpoint{4.179552in}{1.680684in}}%
\pgfpathlineto{\pgfqpoint{4.324695in}{1.608470in}}%
\pgfpathlineto{\pgfqpoint{4.262595in}{2.193629in}}%
\pgfpathclose%
\pgfusepath{fill}%
\end{pgfscope}%
\begin{pgfscope}%
\pgfpathrectangle{\pgfqpoint{0.539299in}{0.078740in}}{\pgfqpoint{7.842520in}{7.842520in}}%
\pgfusepath{clip}%
\pgfsetbuttcap%
\pgfsetroundjoin%
\definecolor{currentfill}{rgb}{0.449368,0.813768,0.335384}%
\pgfsetfillcolor{currentfill}%
\pgfsetlinewidth{0.000000pt}%
\definecolor{currentstroke}{rgb}{0.132444,0.552216,0.553018}%
\pgfsetstrokecolor{currentstroke}%
\pgfsetdash{}{0pt}%
\pgfpathmoveto{\pgfqpoint{3.434182in}{4.580061in}}%
\pgfpathlineto{\pgfqpoint{3.350268in}{4.277351in}}%
\pgfpathlineto{\pgfqpoint{3.494996in}{4.172102in}}%
\pgfpathclose%
\pgfusepath{fill}%
\end{pgfscope}%
\begin{pgfscope}%
\pgfpathrectangle{\pgfqpoint{0.539299in}{0.078740in}}{\pgfqpoint{7.842520in}{7.842520in}}%
\pgfusepath{clip}%
\pgfsetbuttcap%
\pgfsetroundjoin%
\definecolor{currentfill}{rgb}{0.239374,0.735588,0.455688}%
\pgfsetfillcolor{currentfill}%
\pgfsetlinewidth{0.000000pt}%
\definecolor{currentstroke}{rgb}{0.131172,0.555899,0.552459}%
\pgfsetstrokecolor{currentstroke}%
\pgfsetdash{}{0pt}%
\pgfpathmoveto{\pgfqpoint{4.161873in}{3.857033in}}%
\pgfpathlineto{\pgfqpoint{4.392412in}{3.813568in}}%
\pgfpathlineto{\pgfqpoint{4.246749in}{4.002642in}}%
\pgfpathclose%
\pgfusepath{fill}%
\end{pgfscope}%
\begin{pgfscope}%
\pgfpathrectangle{\pgfqpoint{0.539299in}{0.078740in}}{\pgfqpoint{7.842520in}{7.842520in}}%
\pgfusepath{clip}%
\pgfsetbuttcap%
\pgfsetroundjoin%
\definecolor{currentfill}{rgb}{0.119423,0.611141,0.538982}%
\pgfsetfillcolor{currentfill}%
\pgfsetlinewidth{0.000000pt}%
\definecolor{currentstroke}{rgb}{0.129933,0.559582,0.551864}%
\pgfsetstrokecolor{currentstroke}%
\pgfsetdash{}{0pt}%
\pgfpathmoveto{\pgfqpoint{4.683466in}{3.421156in}}%
\pgfpathlineto{\pgfqpoint{4.599574in}{3.325252in}}%
\pgfpathlineto{\pgfqpoint{4.745380in}{3.138731in}}%
\pgfpathclose%
\pgfusepath{fill}%
\end{pgfscope}%
\begin{pgfscope}%
\pgfpathrectangle{\pgfqpoint{0.539299in}{0.078740in}}{\pgfqpoint{7.842520in}{7.842520in}}%
\pgfusepath{clip}%
\pgfsetbuttcap%
\pgfsetroundjoin%
\definecolor{currentfill}{rgb}{0.239374,0.735588,0.455688}%
\pgfsetfillcolor{currentfill}%
\pgfsetlinewidth{0.000000pt}%
\definecolor{currentstroke}{rgb}{0.128729,0.563265,0.551229}%
\pgfsetstrokecolor{currentstroke}%
\pgfsetdash{}{0pt}%
\pgfpathmoveto{\pgfqpoint{3.494996in}{4.172102in}}%
\pgfpathlineto{\pgfqpoint{3.412099in}{3.800510in}}%
\pgfpathlineto{\pgfqpoint{3.556804in}{3.697996in}}%
\pgfpathclose%
\pgfusepath{fill}%
\end{pgfscope}%
\begin{pgfscope}%
\pgfpathrectangle{\pgfqpoint{0.539299in}{0.078740in}}{\pgfqpoint{7.842520in}{7.842520in}}%
\pgfusepath{clip}%
\pgfsetbuttcap%
\pgfsetroundjoin%
\definecolor{currentfill}{rgb}{0.311925,0.767822,0.415586}%
\pgfsetfillcolor{currentfill}%
\pgfsetlinewidth{0.000000pt}%
\definecolor{currentstroke}{rgb}{0.127568,0.566949,0.550556}%
\pgfsetstrokecolor{currentstroke}%
\pgfsetdash{}{0pt}%
\pgfpathmoveto{\pgfqpoint{4.246749in}{4.002642in}}%
\pgfpathlineto{\pgfqpoint{4.101044in}{4.185140in}}%
\pgfpathlineto{\pgfqpoint{4.015993in}{4.022293in}}%
\pgfpathclose%
\pgfusepath{fill}%
\end{pgfscope}%
\begin{pgfscope}%
\pgfpathrectangle{\pgfqpoint{0.539299in}{0.078740in}}{\pgfqpoint{7.842520in}{7.842520in}}%
\pgfusepath{clip}%
\pgfsetbuttcap%
\pgfsetroundjoin%
\definecolor{currentfill}{rgb}{0.119738,0.603785,0.541400}%
\pgfsetfillcolor{currentfill}%
\pgfsetlinewidth{0.000000pt}%
\definecolor{currentstroke}{rgb}{0.126453,0.570633,0.549841}%
\pgfsetstrokecolor{currentstroke}%
\pgfsetdash{}{0pt}%
\pgfpathmoveto{\pgfqpoint{3.619444in}{3.158415in}}%
\pgfpathlineto{\pgfqpoint{3.764278in}{3.057833in}}%
\pgfpathlineto{\pgfqpoint{3.701876in}{3.586183in}}%
\pgfpathclose%
\pgfusepath{fill}%
\end{pgfscope}%
\begin{pgfscope}%
\pgfpathrectangle{\pgfqpoint{0.539299in}{0.078740in}}{\pgfqpoint{7.842520in}{7.842520in}}%
\pgfusepath{clip}%
\pgfsetbuttcap%
\pgfsetroundjoin%
\definecolor{currentfill}{rgb}{0.458674,0.816363,0.329727}%
\pgfsetfillcolor{currentfill}%
\pgfsetlinewidth{0.000000pt}%
\definecolor{currentstroke}{rgb}{0.125394,0.574318,0.549086}%
\pgfsetstrokecolor{currentstroke}%
\pgfsetdash{}{0pt}%
\pgfpathmoveto{\pgfqpoint{3.809748in}{4.522655in}}%
\pgfpathlineto{\pgfqpoint{3.870205in}{4.179274in}}%
\pgfpathlineto{\pgfqpoint{3.955353in}{4.359218in}}%
\pgfpathclose%
\pgfusepath{fill}%
\end{pgfscope}%
\begin{pgfscope}%
\pgfpathrectangle{\pgfqpoint{0.539299in}{0.078740in}}{\pgfqpoint{7.842520in}{7.842520in}}%
\pgfusepath{clip}%
\pgfsetbuttcap%
\pgfsetroundjoin%
\definecolor{currentfill}{rgb}{0.278012,0.180367,0.486697}%
\pgfsetfillcolor{currentfill}%
\pgfsetlinewidth{0.000000pt}%
\definecolor{currentstroke}{rgb}{0.124395,0.578002,0.548287}%
\pgfsetstrokecolor{currentstroke}%
\pgfsetdash{}{0pt}%
\pgfpathmoveto{\pgfqpoint{5.680690in}{1.591684in}}%
\pgfpathlineto{\pgfqpoint{5.599643in}{1.509341in}}%
\pgfpathlineto{\pgfqpoint{5.824454in}{1.317143in}}%
\pgfpathclose%
\pgfusepath{fill}%
\end{pgfscope}%
\begin{pgfscope}%
\pgfpathrectangle{\pgfqpoint{0.539299in}{0.078740in}}{\pgfqpoint{7.842520in}{7.842520in}}%
\pgfusepath{clip}%
\pgfsetbuttcap%
\pgfsetroundjoin%
\definecolor{currentfill}{rgb}{0.377779,0.791781,0.377939}%
\pgfsetfillcolor{currentfill}%
\pgfsetlinewidth{0.000000pt}%
\definecolor{currentstroke}{rgb}{0.123463,0.581687,0.547445}%
\pgfsetstrokecolor{currentstroke}%
\pgfsetdash{}{0pt}%
\pgfpathmoveto{\pgfqpoint{4.015993in}{4.022293in}}%
\pgfpathlineto{\pgfqpoint{4.101044in}{4.185140in}}%
\pgfpathlineto{\pgfqpoint{3.955353in}{4.359218in}}%
\pgfpathclose%
\pgfusepath{fill}%
\end{pgfscope}%
\begin{pgfscope}%
\pgfpathrectangle{\pgfqpoint{0.539299in}{0.078740in}}{\pgfqpoint{7.842520in}{7.842520in}}%
\pgfusepath{clip}%
\pgfsetbuttcap%
\pgfsetroundjoin%
\definecolor{currentfill}{rgb}{0.487026,0.823929,0.312321}%
\pgfsetfillcolor{currentfill}%
\pgfsetlinewidth{0.000000pt}%
\definecolor{currentstroke}{rgb}{0.122606,0.585371,0.546557}%
\pgfsetstrokecolor{currentstroke}%
\pgfsetdash{}{0pt}%
\pgfpathmoveto{\pgfqpoint{3.494996in}{4.172102in}}%
\pgfpathlineto{\pgfqpoint{3.579199in}{4.460532in}}%
\pgfpathlineto{\pgfqpoint{3.434182in}{4.580061in}}%
\pgfpathclose%
\pgfusepath{fill}%
\end{pgfscope}%
\begin{pgfscope}%
\pgfpathrectangle{\pgfqpoint{0.539299in}{0.078740in}}{\pgfqpoint{7.842520in}{7.842520in}}%
\pgfusepath{clip}%
\pgfsetbuttcap%
\pgfsetroundjoin%
\definecolor{currentfill}{rgb}{0.227802,0.326594,0.546532}%
\pgfsetfillcolor{currentfill}%
\pgfsetlinewidth{0.000000pt}%
\definecolor{currentstroke}{rgb}{0.121831,0.589055,0.545623}%
\pgfsetstrokecolor{currentstroke}%
\pgfsetdash{}{0pt}%
\pgfpathmoveto{\pgfqpoint{5.390896in}{2.059660in}}%
\pgfpathlineto{\pgfqpoint{5.536081in}{1.836733in}}%
\pgfpathlineto{\pgfqpoint{5.472274in}{2.124924in}}%
\pgfpathclose%
\pgfusepath{fill}%
\end{pgfscope}%
\begin{pgfscope}%
\pgfpathrectangle{\pgfqpoint{0.539299in}{0.078740in}}{\pgfqpoint{7.842520in}{7.842520in}}%
\pgfusepath{clip}%
\pgfsetbuttcap%
\pgfsetroundjoin%
\definecolor{currentfill}{rgb}{0.506271,0.828786,0.300362}%
\pgfsetfillcolor{currentfill}%
\pgfsetlinewidth{0.000000pt}%
\definecolor{currentstroke}{rgb}{0.121148,0.592739,0.544641}%
\pgfsetstrokecolor{currentstroke}%
\pgfsetdash{}{0pt}%
\pgfpathmoveto{\pgfqpoint{3.724577in}{4.326091in}}%
\pgfpathlineto{\pgfqpoint{3.809748in}{4.522655in}}%
\pgfpathlineto{\pgfqpoint{3.579199in}{4.460532in}}%
\pgfpathclose%
\pgfusepath{fill}%
\end{pgfscope}%
\begin{pgfscope}%
\pgfpathrectangle{\pgfqpoint{0.539299in}{0.078740in}}{\pgfqpoint{7.842520in}{7.842520in}}%
\pgfusepath{clip}%
\pgfsetbuttcap%
\pgfsetroundjoin%
\definecolor{currentfill}{rgb}{0.130067,0.651384,0.521608}%
\pgfsetfillcolor{currentfill}%
\pgfsetlinewidth{0.000000pt}%
\definecolor{currentstroke}{rgb}{0.120565,0.596422,0.543611}%
\pgfsetstrokecolor{currentstroke}%
\pgfsetdash{}{0pt}%
\pgfpathmoveto{\pgfqpoint{4.537995in}{3.619381in}}%
\pgfpathlineto{\pgfqpoint{4.453701in}{3.507500in}}%
\pgfpathlineto{\pgfqpoint{4.599574in}{3.325252in}}%
\pgfpathclose%
\pgfusepath{fill}%
\end{pgfscope}%
\begin{pgfscope}%
\pgfpathrectangle{\pgfqpoint{0.539299in}{0.078740in}}{\pgfqpoint{7.842520in}{7.842520in}}%
\pgfusepath{clip}%
\pgfsetbuttcap%
\pgfsetroundjoin%
\definecolor{currentfill}{rgb}{0.180629,0.429975,0.557282}%
\pgfsetfillcolor{currentfill}%
\pgfsetlinewidth{0.000000pt}%
\definecolor{currentstroke}{rgb}{0.120092,0.600104,0.542530}%
\pgfsetstrokecolor{currentstroke}%
\pgfsetdash{}{0pt}%
\pgfpathmoveto{\pgfqpoint{5.327340in}{2.342890in}}%
\pgfpathlineto{\pgfqpoint{5.182120in}{2.551317in}}%
\pgfpathlineto{\pgfqpoint{5.099519in}{2.460623in}}%
\pgfpathclose%
\pgfusepath{fill}%
\end{pgfscope}%
\begin{pgfscope}%
\pgfpathrectangle{\pgfqpoint{0.539299in}{0.078740in}}{\pgfqpoint{7.842520in}{7.842520in}}%
\pgfusepath{clip}%
\pgfsetbuttcap%
\pgfsetroundjoin%
\definecolor{currentfill}{rgb}{0.153894,0.680203,0.504172}%
\pgfsetfillcolor{currentfill}%
\pgfsetlinewidth{0.000000pt}%
\definecolor{currentstroke}{rgb}{0.119738,0.603785,0.541400}%
\pgfsetstrokecolor{currentstroke}%
\pgfsetdash{}{0pt}%
\pgfpathmoveto{\pgfqpoint{4.307790in}{3.685037in}}%
\pgfpathlineto{\pgfqpoint{4.453701in}{3.507500in}}%
\pgfpathlineto{\pgfqpoint{4.537995in}{3.619381in}}%
\pgfpathclose%
\pgfusepath{fill}%
\end{pgfscope}%
\begin{pgfscope}%
\pgfpathrectangle{\pgfqpoint{0.539299in}{0.078740in}}{\pgfqpoint{7.842520in}{7.842520in}}%
\pgfusepath{clip}%
\pgfsetbuttcap%
\pgfsetroundjoin%
\definecolor{currentfill}{rgb}{0.458674,0.816363,0.329727}%
\pgfsetfillcolor{currentfill}%
\pgfsetlinewidth{0.000000pt}%
\definecolor{currentstroke}{rgb}{0.119512,0.607464,0.540218}%
\pgfsetstrokecolor{currentstroke}%
\pgfsetdash{}{0pt}%
\pgfpathmoveto{\pgfqpoint{3.724577in}{4.326091in}}%
\pgfpathlineto{\pgfqpoint{3.870205in}{4.179274in}}%
\pgfpathlineto{\pgfqpoint{3.809748in}{4.522655in}}%
\pgfpathclose%
\pgfusepath{fill}%
\end{pgfscope}%
\begin{pgfscope}%
\pgfpathrectangle{\pgfqpoint{0.539299in}{0.078740in}}{\pgfqpoint{7.842520in}{7.842520in}}%
\pgfusepath{clip}%
\pgfsetbuttcap%
\pgfsetroundjoin%
\definecolor{currentfill}{rgb}{0.449368,0.813768,0.335384}%
\pgfsetfillcolor{currentfill}%
\pgfsetlinewidth{0.000000pt}%
\definecolor{currentstroke}{rgb}{0.119423,0.611141,0.538982}%
\pgfsetstrokecolor{currentstroke}%
\pgfsetdash{}{0pt}%
\pgfpathmoveto{\pgfqpoint{3.724577in}{4.326091in}}%
\pgfpathlineto{\pgfqpoint{3.579199in}{4.460532in}}%
\pgfpathlineto{\pgfqpoint{3.494996in}{4.172102in}}%
\pgfpathclose%
\pgfusepath{fill}%
\end{pgfscope}%
\begin{pgfscope}%
\pgfpathrectangle{\pgfqpoint{0.539299in}{0.078740in}}{\pgfqpoint{7.842520in}{7.842520in}}%
\pgfusepath{clip}%
\pgfsetbuttcap%
\pgfsetroundjoin%
\definecolor{currentfill}{rgb}{0.208030,0.718701,0.472873}%
\pgfsetfillcolor{currentfill}%
\pgfsetlinewidth{0.000000pt}%
\definecolor{currentstroke}{rgb}{0.119483,0.614817,0.537692}%
\pgfsetstrokecolor{currentstroke}%
\pgfsetdash{}{0pt}%
\pgfpathmoveto{\pgfqpoint{4.307790in}{3.685037in}}%
\pgfpathlineto{\pgfqpoint{4.392412in}{3.813568in}}%
\pgfpathlineto{\pgfqpoint{4.161873in}{3.857033in}}%
\pgfpathclose%
\pgfusepath{fill}%
\end{pgfscope}%
\begin{pgfscope}%
\pgfpathrectangle{\pgfqpoint{0.539299in}{0.078740in}}{\pgfqpoint{7.842520in}{7.842520in}}%
\pgfusepath{clip}%
\pgfsetbuttcap%
\pgfsetroundjoin%
\definecolor{currentfill}{rgb}{0.281477,0.755203,0.432552}%
\pgfsetfillcolor{currentfill}%
\pgfsetlinewidth{0.000000pt}%
\definecolor{currentstroke}{rgb}{0.119699,0.618490,0.536347}%
\pgfsetstrokecolor{currentstroke}%
\pgfsetdash{}{0pt}%
\pgfpathmoveto{\pgfqpoint{3.556804in}{3.697996in}}%
\pgfpathlineto{\pgfqpoint{3.640137in}{4.053707in}}%
\pgfpathlineto{\pgfqpoint{3.494996in}{4.172102in}}%
\pgfpathclose%
\pgfusepath{fill}%
\end{pgfscope}%
\begin{pgfscope}%
\pgfpathrectangle{\pgfqpoint{0.539299in}{0.078740in}}{\pgfqpoint{7.842520in}{7.842520in}}%
\pgfusepath{clip}%
\pgfsetbuttcap%
\pgfsetroundjoin%
\definecolor{currentfill}{rgb}{0.210503,0.363727,0.552206}%
\pgfsetfillcolor{currentfill}%
\pgfsetlinewidth{0.000000pt}%
\definecolor{currentstroke}{rgb}{0.120081,0.622161,0.534946}%
\pgfsetstrokecolor{currentstroke}%
\pgfsetdash{}{0pt}%
\pgfpathmoveto{\pgfqpoint{5.245331in}{2.266200in}}%
\pgfpathlineto{\pgfqpoint{5.390896in}{2.059660in}}%
\pgfpathlineto{\pgfqpoint{5.472274in}{2.124924in}}%
\pgfpathclose%
\pgfusepath{fill}%
\end{pgfscope}%
\begin{pgfscope}%
\pgfpathrectangle{\pgfqpoint{0.539299in}{0.078740in}}{\pgfqpoint{7.842520in}{7.842520in}}%
\pgfusepath{clip}%
\pgfsetbuttcap%
\pgfsetroundjoin%
\definecolor{currentfill}{rgb}{0.281887,0.150881,0.465405}%
\pgfsetfillcolor{currentfill}%
\pgfsetlinewidth{0.000000pt}%
\definecolor{currentstroke}{rgb}{0.120638,0.625828,0.533488}%
\pgfsetstrokecolor{currentstroke}%
\pgfsetdash{}{0pt}%
\pgfpathmoveto{\pgfqpoint{5.824454in}{1.317143in}}%
\pgfpathlineto{\pgfqpoint{5.599643in}{1.509341in}}%
\pgfpathlineto{\pgfqpoint{5.744062in}{1.240624in}}%
\pgfpathclose%
\pgfusepath{fill}%
\end{pgfscope}%
\begin{pgfscope}%
\pgfpathrectangle{\pgfqpoint{0.539299in}{0.078740in}}{\pgfqpoint{7.842520in}{7.842520in}}%
\pgfusepath{clip}%
\pgfsetbuttcap%
\pgfsetroundjoin%
\definecolor{currentfill}{rgb}{0.274149,0.751988,0.436601}%
\pgfsetfillcolor{currentfill}%
\pgfsetlinewidth{0.000000pt}%
\definecolor{currentstroke}{rgb}{0.121380,0.629492,0.531973}%
\pgfsetstrokecolor{currentstroke}%
\pgfsetdash{}{0pt}%
\pgfpathmoveto{\pgfqpoint{4.015993in}{4.022293in}}%
\pgfpathlineto{\pgfqpoint{4.161873in}{3.857033in}}%
\pgfpathlineto{\pgfqpoint{4.246749in}{4.002642in}}%
\pgfpathclose%
\pgfusepath{fill}%
\end{pgfscope}%
\begin{pgfscope}%
\pgfpathrectangle{\pgfqpoint{0.539299in}{0.078740in}}{\pgfqpoint{7.842520in}{7.842520in}}%
\pgfusepath{clip}%
\pgfsetbuttcap%
\pgfsetroundjoin%
\definecolor{currentfill}{rgb}{0.154815,0.493313,0.557840}%
\pgfsetfillcolor{currentfill}%
\pgfsetlinewidth{0.000000pt}%
\definecolor{currentstroke}{rgb}{0.122312,0.633153,0.530398}%
\pgfsetstrokecolor{currentstroke}%
\pgfsetdash{}{0pt}%
\pgfpathmoveto{\pgfqpoint{3.971998in}{2.380078in}}%
\pgfpathlineto{\pgfqpoint{4.054864in}{2.841426in}}%
\pgfpathlineto{\pgfqpoint{3.909430in}{2.951944in}}%
\pgfpathclose%
\pgfusepath{fill}%
\end{pgfscope}%
\begin{pgfscope}%
\pgfpathrectangle{\pgfqpoint{0.539299in}{0.078740in}}{\pgfqpoint{7.842520in}{7.842520in}}%
\pgfusepath{clip}%
\pgfsetbuttcap%
\pgfsetroundjoin%
\definecolor{currentfill}{rgb}{0.377779,0.791781,0.377939}%
\pgfsetfillcolor{currentfill}%
\pgfsetlinewidth{0.000000pt}%
\definecolor{currentstroke}{rgb}{0.123444,0.636809,0.528763}%
\pgfsetstrokecolor{currentstroke}%
\pgfsetdash{}{0pt}%
\pgfpathmoveto{\pgfqpoint{3.955353in}{4.359218in}}%
\pgfpathlineto{\pgfqpoint{3.870205in}{4.179274in}}%
\pgfpathlineto{\pgfqpoint{4.015993in}{4.022293in}}%
\pgfpathclose%
\pgfusepath{fill}%
\end{pgfscope}%
\begin{pgfscope}%
\pgfpathrectangle{\pgfqpoint{0.539299in}{0.078740in}}{\pgfqpoint{7.842520in}{7.842520in}}%
\pgfusepath{clip}%
\pgfsetbuttcap%
\pgfsetroundjoin%
\definecolor{currentfill}{rgb}{0.162142,0.474838,0.558140}%
\pgfsetfillcolor{currentfill}%
\pgfsetlinewidth{0.000000pt}%
\definecolor{currentstroke}{rgb}{0.124780,0.640461,0.527068}%
\pgfsetstrokecolor{currentstroke}%
\pgfsetdash{}{0pt}%
\pgfpathmoveto{\pgfqpoint{5.182120in}{2.551317in}}%
\pgfpathlineto{\pgfqpoint{5.036687in}{2.752478in}}%
\pgfpathlineto{\pgfqpoint{4.953553in}{2.645992in}}%
\pgfpathclose%
\pgfusepath{fill}%
\end{pgfscope}%
\begin{pgfscope}%
\pgfpathrectangle{\pgfqpoint{0.539299in}{0.078740in}}{\pgfqpoint{7.842520in}{7.842520in}}%
\pgfusepath{clip}%
\pgfsetbuttcap%
\pgfsetroundjoin%
\definecolor{currentfill}{rgb}{0.177423,0.437527,0.557565}%
\pgfsetfillcolor{currentfill}%
\pgfsetlinewidth{0.000000pt}%
\definecolor{currentstroke}{rgb}{0.126326,0.644107,0.525311}%
\pgfsetstrokecolor{currentstroke}%
\pgfsetdash{}{0pt}%
\pgfpathmoveto{\pgfqpoint{3.971998in}{2.380078in}}%
\pgfpathlineto{\pgfqpoint{4.117136in}{2.288294in}}%
\pgfpathlineto{\pgfqpoint{4.200551in}{2.726746in}}%
\pgfpathclose%
\pgfusepath{fill}%
\end{pgfscope}%
\begin{pgfscope}%
\pgfpathrectangle{\pgfqpoint{0.539299in}{0.078740in}}{\pgfqpoint{7.842520in}{7.842520in}}%
\pgfusepath{clip}%
\pgfsetbuttcap%
\pgfsetroundjoin%
\definecolor{currentfill}{rgb}{0.202219,0.715272,0.476084}%
\pgfsetfillcolor{currentfill}%
\pgfsetlinewidth{0.000000pt}%
\definecolor{currentstroke}{rgb}{0.128087,0.647749,0.523491}%
\pgfsetstrokecolor{currentstroke}%
\pgfsetdash{}{0pt}%
\pgfpathmoveto{\pgfqpoint{3.701876in}{3.586183in}}%
\pgfpathlineto{\pgfqpoint{3.640137in}{4.053707in}}%
\pgfpathlineto{\pgfqpoint{3.556804in}{3.697996in}}%
\pgfpathclose%
\pgfusepath{fill}%
\end{pgfscope}%
\begin{pgfscope}%
\pgfpathrectangle{\pgfqpoint{0.539299in}{0.078740in}}{\pgfqpoint{7.842520in}{7.842520in}}%
\pgfusepath{clip}%
\pgfsetbuttcap%
\pgfsetroundjoin%
\definecolor{currentfill}{rgb}{0.377779,0.791781,0.377939}%
\pgfsetfillcolor{currentfill}%
\pgfsetlinewidth{0.000000pt}%
\definecolor{currentstroke}{rgb}{0.130067,0.651384,0.521608}%
\pgfsetstrokecolor{currentstroke}%
\pgfsetdash{}{0pt}%
\pgfpathmoveto{\pgfqpoint{3.494996in}{4.172102in}}%
\pgfpathlineto{\pgfqpoint{3.640137in}{4.053707in}}%
\pgfpathlineto{\pgfqpoint{3.724577in}{4.326091in}}%
\pgfpathclose%
\pgfusepath{fill}%
\end{pgfscope}%
\begin{pgfscope}%
\pgfpathrectangle{\pgfqpoint{0.539299in}{0.078740in}}{\pgfqpoint{7.842520in}{7.842520in}}%
\pgfusepath{clip}%
\pgfsetbuttcap%
\pgfsetroundjoin%
\definecolor{currentfill}{rgb}{0.122606,0.585371,0.546557}%
\pgfsetfillcolor{currentfill}%
\pgfsetlinewidth{0.000000pt}%
\definecolor{currentstroke}{rgb}{0.132268,0.655014,0.519661}%
\pgfsetstrokecolor{currentstroke}%
\pgfsetdash{}{0pt}%
\pgfpathmoveto{\pgfqpoint{3.847250in}{3.466509in}}%
\pgfpathlineto{\pgfqpoint{3.764278in}{3.057833in}}%
\pgfpathlineto{\pgfqpoint{3.909430in}{2.951944in}}%
\pgfpathclose%
\pgfusepath{fill}%
\end{pgfscope}%
\begin{pgfscope}%
\pgfpathrectangle{\pgfqpoint{0.539299in}{0.078740in}}{\pgfqpoint{7.842520in}{7.842520in}}%
\pgfusepath{clip}%
\pgfsetbuttcap%
\pgfsetroundjoin%
\definecolor{currentfill}{rgb}{0.122312,0.633153,0.530398}%
\pgfsetfillcolor{currentfill}%
\pgfsetlinewidth{0.000000pt}%
\definecolor{currentstroke}{rgb}{0.134692,0.658636,0.517649}%
\pgfsetstrokecolor{currentstroke}%
\pgfsetdash{}{0pt}%
\pgfpathmoveto{\pgfqpoint{3.701876in}{3.586183in}}%
\pgfpathlineto{\pgfqpoint{3.764278in}{3.057833in}}%
\pgfpathlineto{\pgfqpoint{3.847250in}{3.466509in}}%
\pgfpathclose%
\pgfusepath{fill}%
\end{pgfscope}%
\begin{pgfscope}%
\pgfpathrectangle{\pgfqpoint{0.539299in}{0.078740in}}{\pgfqpoint{7.842520in}{7.842520in}}%
\pgfusepath{clip}%
\pgfsetbuttcap%
\pgfsetroundjoin%
\definecolor{currentfill}{rgb}{0.263663,0.237631,0.518762}%
\pgfsetfillcolor{currentfill}%
\pgfsetlinewidth{0.000000pt}%
\definecolor{currentstroke}{rgb}{0.137339,0.662252,0.515571}%
\pgfsetstrokecolor{currentstroke}%
\pgfsetdash{}{0pt}%
\pgfpathmoveto{\pgfqpoint{5.536081in}{1.836733in}}%
\pgfpathlineto{\pgfqpoint{5.599643in}{1.509341in}}%
\pgfpathlineto{\pgfqpoint{5.680690in}{1.591684in}}%
\pgfpathclose%
\pgfusepath{fill}%
\end{pgfscope}%
\begin{pgfscope}%
\pgfpathrectangle{\pgfqpoint{0.539299in}{0.078740in}}{\pgfqpoint{7.842520in}{7.842520in}}%
\pgfusepath{clip}%
\pgfsetbuttcap%
\pgfsetroundjoin%
\definecolor{currentfill}{rgb}{0.150476,0.504369,0.557430}%
\pgfsetfillcolor{currentfill}%
\pgfsetlinewidth{0.000000pt}%
\definecolor{currentstroke}{rgb}{0.140210,0.665859,0.513427}%
\pgfsetstrokecolor{currentstroke}%
\pgfsetdash{}{0pt}%
\pgfpathmoveto{\pgfqpoint{4.953553in}{2.645992in}}%
\pgfpathlineto{\pgfqpoint{5.036687in}{2.752478in}}%
\pgfpathlineto{\pgfqpoint{4.891094in}{2.947957in}}%
\pgfpathclose%
\pgfusepath{fill}%
\end{pgfscope}%
\begin{pgfscope}%
\pgfpathrectangle{\pgfqpoint{0.539299in}{0.078740in}}{\pgfqpoint{7.842520in}{7.842520in}}%
\pgfusepath{clip}%
\pgfsetbuttcap%
\pgfsetroundjoin%
\definecolor{currentfill}{rgb}{0.386433,0.794644,0.372886}%
\pgfsetfillcolor{currentfill}%
\pgfsetlinewidth{0.000000pt}%
\definecolor{currentstroke}{rgb}{0.143303,0.669459,0.511215}%
\pgfsetstrokecolor{currentstroke}%
\pgfsetdash{}{0pt}%
\pgfpathmoveto{\pgfqpoint{3.640137in}{4.053707in}}%
\pgfpathlineto{\pgfqpoint{3.870205in}{4.179274in}}%
\pgfpathlineto{\pgfqpoint{3.724577in}{4.326091in}}%
\pgfpathclose%
\pgfusepath{fill}%
\end{pgfscope}%
\begin{pgfscope}%
\pgfpathrectangle{\pgfqpoint{0.539299in}{0.078740in}}{\pgfqpoint{7.842520in}{7.842520in}}%
\pgfusepath{clip}%
\pgfsetbuttcap%
\pgfsetroundjoin%
\definecolor{currentfill}{rgb}{0.253935,0.265254,0.529983}%
\pgfsetfillcolor{currentfill}%
\pgfsetlinewidth{0.000000pt}%
\definecolor{currentstroke}{rgb}{0.146616,0.673050,0.508936}%
\pgfsetstrokecolor{currentstroke}%
\pgfsetdash{}{0pt}%
\pgfpathmoveto{\pgfqpoint{4.408361in}{2.095987in}}%
\pgfpathlineto{\pgfqpoint{4.324695in}{1.608470in}}%
\pgfpathlineto{\pgfqpoint{4.470234in}{1.534400in}}%
\pgfpathclose%
\pgfusepath{fill}%
\end{pgfscope}%
\begin{pgfscope}%
\pgfpathrectangle{\pgfqpoint{0.539299in}{0.078740in}}{\pgfqpoint{7.842520in}{7.842520in}}%
\pgfusepath{clip}%
\pgfsetbuttcap%
\pgfsetroundjoin%
\definecolor{currentfill}{rgb}{0.187231,0.414746,0.556547}%
\pgfsetfillcolor{currentfill}%
\pgfsetlinewidth{0.000000pt}%
\definecolor{currentstroke}{rgb}{0.150148,0.676631,0.506589}%
\pgfsetstrokecolor{currentstroke}%
\pgfsetdash{}{0pt}%
\pgfpathmoveto{\pgfqpoint{5.327340in}{2.342890in}}%
\pgfpathlineto{\pgfqpoint{5.099519in}{2.460623in}}%
\pgfpathlineto{\pgfqpoint{5.245331in}{2.266200in}}%
\pgfpathclose%
\pgfusepath{fill}%
\end{pgfscope}%
\begin{pgfscope}%
\pgfpathrectangle{\pgfqpoint{0.539299in}{0.078740in}}{\pgfqpoint{7.842520in}{7.842520in}}%
\pgfusepath{clip}%
\pgfsetbuttcap%
\pgfsetroundjoin%
\definecolor{currentfill}{rgb}{0.160665,0.478540,0.558115}%
\pgfsetfillcolor{currentfill}%
\pgfsetlinewidth{0.000000pt}%
\definecolor{currentstroke}{rgb}{0.153894,0.680203,0.504172}%
\pgfsetstrokecolor{currentstroke}%
\pgfsetdash{}{0pt}%
\pgfpathmoveto{\pgfqpoint{4.200551in}{2.726746in}}%
\pgfpathlineto{\pgfqpoint{4.054864in}{2.841426in}}%
\pgfpathlineto{\pgfqpoint{3.971998in}{2.380078in}}%
\pgfpathclose%
\pgfusepath{fill}%
\end{pgfscope}%
\begin{pgfscope}%
\pgfpathrectangle{\pgfqpoint{0.539299in}{0.078740in}}{\pgfqpoint{7.842520in}{7.842520in}}%
\pgfusepath{clip}%
\pgfsetbuttcap%
\pgfsetroundjoin%
\definecolor{currentfill}{rgb}{0.229739,0.322361,0.545706}%
\pgfsetfillcolor{currentfill}%
\pgfsetlinewidth{0.000000pt}%
\definecolor{currentstroke}{rgb}{0.157851,0.683765,0.501686}%
\pgfsetstrokecolor{currentstroke}%
\pgfsetdash{}{0pt}%
\pgfpathmoveto{\pgfqpoint{4.324695in}{1.608470in}}%
\pgfpathlineto{\pgfqpoint{4.408361in}{2.095987in}}%
\pgfpathlineto{\pgfqpoint{4.262595in}{2.193629in}}%
\pgfpathclose%
\pgfusepath{fill}%
\end{pgfscope}%
\begin{pgfscope}%
\pgfpathrectangle{\pgfqpoint{0.539299in}{0.078740in}}{\pgfqpoint{7.842520in}{7.842520in}}%
\pgfusepath{clip}%
\pgfsetbuttcap%
\pgfsetroundjoin%
\definecolor{currentfill}{rgb}{0.239374,0.735588,0.455688}%
\pgfsetfillcolor{currentfill}%
\pgfsetlinewidth{0.000000pt}%
\definecolor{currentstroke}{rgb}{0.162016,0.687316,0.499129}%
\pgfsetstrokecolor{currentstroke}%
\pgfsetdash{}{0pt}%
\pgfpathmoveto{\pgfqpoint{3.785592in}{3.924272in}}%
\pgfpathlineto{\pgfqpoint{3.640137in}{4.053707in}}%
\pgfpathlineto{\pgfqpoint{3.701876in}{3.586183in}}%
\pgfpathclose%
\pgfusepath{fill}%
\end{pgfscope}%
\begin{pgfscope}%
\pgfpathrectangle{\pgfqpoint{0.539299in}{0.078740in}}{\pgfqpoint{7.842520in}{7.842520in}}%
\pgfusepath{clip}%
\pgfsetbuttcap%
\pgfsetroundjoin%
\definecolor{currentfill}{rgb}{0.319809,0.770914,0.411152}%
\pgfsetfillcolor{currentfill}%
\pgfsetlinewidth{0.000000pt}%
\definecolor{currentstroke}{rgb}{0.166383,0.690856,0.496502}%
\pgfsetstrokecolor{currentstroke}%
\pgfsetdash{}{0pt}%
\pgfpathmoveto{\pgfqpoint{3.640137in}{4.053707in}}%
\pgfpathlineto{\pgfqpoint{3.785592in}{3.924272in}}%
\pgfpathlineto{\pgfqpoint{3.870205in}{4.179274in}}%
\pgfpathclose%
\pgfusepath{fill}%
\end{pgfscope}%
\begin{pgfscope}%
\pgfpathrectangle{\pgfqpoint{0.539299in}{0.078740in}}{\pgfqpoint{7.842520in}{7.842520in}}%
\pgfusepath{clip}%
\pgfsetbuttcap%
\pgfsetroundjoin%
\definecolor{currentfill}{rgb}{0.133743,0.548535,0.553541}%
\pgfsetfillcolor{currentfill}%
\pgfsetlinewidth{0.000000pt}%
\definecolor{currentstroke}{rgb}{0.170948,0.694384,0.493803}%
\pgfsetstrokecolor{currentstroke}%
\pgfsetdash{}{0pt}%
\pgfpathmoveto{\pgfqpoint{4.807495in}{2.824405in}}%
\pgfpathlineto{\pgfqpoint{4.891094in}{2.947957in}}%
\pgfpathlineto{\pgfqpoint{4.745380in}{3.138731in}}%
\pgfpathclose%
\pgfusepath{fill}%
\end{pgfscope}%
\begin{pgfscope}%
\pgfpathrectangle{\pgfqpoint{0.539299in}{0.078740in}}{\pgfqpoint{7.842520in}{7.842520in}}%
\pgfusepath{clip}%
\pgfsetbuttcap%
\pgfsetroundjoin%
\definecolor{currentfill}{rgb}{0.319809,0.770914,0.411152}%
\pgfsetfillcolor{currentfill}%
\pgfsetlinewidth{0.000000pt}%
\definecolor{currentstroke}{rgb}{0.175707,0.697900,0.491033}%
\pgfsetstrokecolor{currentstroke}%
\pgfsetdash{}{0pt}%
\pgfpathmoveto{\pgfqpoint{3.870205in}{4.179274in}}%
\pgfpathlineto{\pgfqpoint{3.785592in}{3.924272in}}%
\pgfpathlineto{\pgfqpoint{4.015993in}{4.022293in}}%
\pgfpathclose%
\pgfusepath{fill}%
\end{pgfscope}%
\begin{pgfscope}%
\pgfpathrectangle{\pgfqpoint{0.539299in}{0.078740in}}{\pgfqpoint{7.842520in}{7.842520in}}%
\pgfusepath{clip}%
\pgfsetbuttcap%
\pgfsetroundjoin%
\definecolor{currentfill}{rgb}{0.182256,0.426184,0.557120}%
\pgfsetfillcolor{currentfill}%
\pgfsetlinewidth{0.000000pt}%
\definecolor{currentstroke}{rgb}{0.180653,0.701402,0.488189}%
\pgfsetstrokecolor{currentstroke}%
\pgfsetdash{}{0pt}%
\pgfpathmoveto{\pgfqpoint{4.117136in}{2.288294in}}%
\pgfpathlineto{\pgfqpoint{4.262595in}{2.193629in}}%
\pgfpathlineto{\pgfqpoint{4.200551in}{2.726746in}}%
\pgfpathclose%
\pgfusepath{fill}%
\end{pgfscope}%
\begin{pgfscope}%
\pgfpathrectangle{\pgfqpoint{0.539299in}{0.078740in}}{\pgfqpoint{7.842520in}{7.842520in}}%
\pgfusepath{clip}%
\pgfsetbuttcap%
\pgfsetroundjoin%
\definecolor{currentfill}{rgb}{0.121831,0.589055,0.545623}%
\pgfsetfillcolor{currentfill}%
\pgfsetlinewidth{0.000000pt}%
\definecolor{currentstroke}{rgb}{0.185783,0.704891,0.485273}%
\pgfsetstrokecolor{currentstroke}%
\pgfsetdash{}{0pt}%
\pgfpathmoveto{\pgfqpoint{4.599574in}{3.325252in}}%
\pgfpathlineto{\pgfqpoint{4.661391in}{2.997194in}}%
\pgfpathlineto{\pgfqpoint{4.745380in}{3.138731in}}%
\pgfpathclose%
\pgfusepath{fill}%
\end{pgfscope}%
\begin{pgfscope}%
\pgfpathrectangle{\pgfqpoint{0.539299in}{0.078740in}}{\pgfqpoint{7.842520in}{7.842520in}}%
\pgfusepath{clip}%
\pgfsetbuttcap%
\pgfsetroundjoin%
\definecolor{currentfill}{rgb}{0.168126,0.459988,0.558082}%
\pgfsetfillcolor{currentfill}%
\pgfsetlinewidth{0.000000pt}%
\definecolor{currentstroke}{rgb}{0.191090,0.708366,0.482284}%
\pgfsetstrokecolor{currentstroke}%
\pgfsetdash{}{0pt}%
\pgfpathmoveto{\pgfqpoint{5.099519in}{2.460623in}}%
\pgfpathlineto{\pgfqpoint{5.182120in}{2.551317in}}%
\pgfpathlineto{\pgfqpoint{4.953553in}{2.645992in}}%
\pgfpathclose%
\pgfusepath{fill}%
\end{pgfscope}%
\begin{pgfscope}%
\pgfpathrectangle{\pgfqpoint{0.539299in}{0.078740in}}{\pgfqpoint{7.842520in}{7.842520in}}%
\pgfusepath{clip}%
\pgfsetbuttcap%
\pgfsetroundjoin%
\definecolor{currentfill}{rgb}{0.208030,0.718701,0.472873}%
\pgfsetfillcolor{currentfill}%
\pgfsetlinewidth{0.000000pt}%
\definecolor{currentstroke}{rgb}{0.196571,0.711827,0.479221}%
\pgfsetstrokecolor{currentstroke}%
\pgfsetdash{}{0pt}%
\pgfpathmoveto{\pgfqpoint{3.931280in}{3.785621in}}%
\pgfpathlineto{\pgfqpoint{3.785592in}{3.924272in}}%
\pgfpathlineto{\pgfqpoint{3.701876in}{3.586183in}}%
\pgfpathclose%
\pgfusepath{fill}%
\end{pgfscope}%
\begin{pgfscope}%
\pgfpathrectangle{\pgfqpoint{0.539299in}{0.078740in}}{\pgfqpoint{7.842520in}{7.842520in}}%
\pgfusepath{clip}%
\pgfsetbuttcap%
\pgfsetroundjoin%
\definecolor{currentfill}{rgb}{0.162016,0.687316,0.499129}%
\pgfsetfillcolor{currentfill}%
\pgfsetlinewidth{0.000000pt}%
\definecolor{currentstroke}{rgb}{0.202219,0.715272,0.476084}%
\pgfsetstrokecolor{currentstroke}%
\pgfsetdash{}{0pt}%
\pgfpathmoveto{\pgfqpoint{3.701876in}{3.586183in}}%
\pgfpathlineto{\pgfqpoint{3.847250in}{3.466509in}}%
\pgfpathlineto{\pgfqpoint{3.931280in}{3.785621in}}%
\pgfpathclose%
\pgfusepath{fill}%
\end{pgfscope}%
\begin{pgfscope}%
\pgfpathrectangle{\pgfqpoint{0.539299in}{0.078740in}}{\pgfqpoint{7.842520in}{7.842520in}}%
\pgfusepath{clip}%
\pgfsetbuttcap%
\pgfsetroundjoin%
\definecolor{currentfill}{rgb}{0.119423,0.611141,0.538982}%
\pgfsetfillcolor{currentfill}%
\pgfsetlinewidth{0.000000pt}%
\definecolor{currentstroke}{rgb}{0.208030,0.718701,0.472873}%
\pgfsetstrokecolor{currentstroke}%
\pgfsetdash{}{0pt}%
\pgfpathmoveto{\pgfqpoint{3.909430in}{2.951944in}}%
\pgfpathlineto{\pgfqpoint{3.992871in}{3.340173in}}%
\pgfpathlineto{\pgfqpoint{3.847250in}{3.466509in}}%
\pgfpathclose%
\pgfusepath{fill}%
\end{pgfscope}%
\begin{pgfscope}%
\pgfpathrectangle{\pgfqpoint{0.539299in}{0.078740in}}{\pgfqpoint{7.842520in}{7.842520in}}%
\pgfusepath{clip}%
\pgfsetbuttcap%
\pgfsetroundjoin%
\definecolor{currentfill}{rgb}{0.121380,0.629492,0.531973}%
\pgfsetfillcolor{currentfill}%
\pgfsetlinewidth{0.000000pt}%
\definecolor{currentstroke}{rgb}{0.214000,0.722114,0.469588}%
\pgfsetstrokecolor{currentstroke}%
\pgfsetdash{}{0pt}%
\pgfpathmoveto{\pgfqpoint{4.515273in}{3.165074in}}%
\pgfpathlineto{\pgfqpoint{4.599574in}{3.325252in}}%
\pgfpathlineto{\pgfqpoint{4.453701in}{3.507500in}}%
\pgfpathclose%
\pgfusepath{fill}%
\end{pgfscope}%
\begin{pgfscope}%
\pgfpathrectangle{\pgfqpoint{0.539299in}{0.078740in}}{\pgfqpoint{7.842520in}{7.842520in}}%
\pgfusepath{clip}%
\pgfsetbuttcap%
\pgfsetroundjoin%
\definecolor{currentfill}{rgb}{0.259857,0.745492,0.444467}%
\pgfsetfillcolor{currentfill}%
\pgfsetlinewidth{0.000000pt}%
\definecolor{currentstroke}{rgb}{0.220124,0.725509,0.466226}%
\pgfsetstrokecolor{currentstroke}%
\pgfsetdash{}{0pt}%
\pgfpathmoveto{\pgfqpoint{4.015993in}{4.022293in}}%
\pgfpathlineto{\pgfqpoint{3.785592in}{3.924272in}}%
\pgfpathlineto{\pgfqpoint{3.931280in}{3.785621in}}%
\pgfpathclose%
\pgfusepath{fill}%
\end{pgfscope}%
\begin{pgfscope}%
\pgfpathrectangle{\pgfqpoint{0.539299in}{0.078740in}}{\pgfqpoint{7.842520in}{7.842520in}}%
\pgfusepath{clip}%
\pgfsetbuttcap%
\pgfsetroundjoin%
\definecolor{currentfill}{rgb}{0.232815,0.732247,0.459277}%
\pgfsetfillcolor{currentfill}%
\pgfsetlinewidth{0.000000pt}%
\definecolor{currentstroke}{rgb}{0.226397,0.728888,0.462789}%
\pgfsetstrokecolor{currentstroke}%
\pgfsetdash{}{0pt}%
\pgfpathmoveto{\pgfqpoint{4.077139in}{3.639292in}}%
\pgfpathlineto{\pgfqpoint{4.161873in}{3.857033in}}%
\pgfpathlineto{\pgfqpoint{4.015993in}{4.022293in}}%
\pgfpathclose%
\pgfusepath{fill}%
\end{pgfscope}%
\begin{pgfscope}%
\pgfpathrectangle{\pgfqpoint{0.539299in}{0.078740in}}{\pgfqpoint{7.842520in}{7.842520in}}%
\pgfusepath{clip}%
\pgfsetbuttcap%
\pgfsetroundjoin%
\definecolor{currentfill}{rgb}{0.150148,0.676631,0.506589}%
\pgfsetfillcolor{currentfill}%
\pgfsetlinewidth{0.000000pt}%
\definecolor{currentstroke}{rgb}{0.232815,0.732247,0.459277}%
\pgfsetstrokecolor{currentstroke}%
\pgfsetdash{}{0pt}%
\pgfpathmoveto{\pgfqpoint{4.307790in}{3.685037in}}%
\pgfpathlineto{\pgfqpoint{4.223116in}{3.486528in}}%
\pgfpathlineto{\pgfqpoint{4.453701in}{3.507500in}}%
\pgfpathclose%
\pgfusepath{fill}%
\end{pgfscope}%
\begin{pgfscope}%
\pgfpathrectangle{\pgfqpoint{0.539299in}{0.078740in}}{\pgfqpoint{7.842520in}{7.842520in}}%
\pgfusepath{clip}%
\pgfsetbuttcap%
\pgfsetroundjoin%
\definecolor{currentfill}{rgb}{0.180653,0.701402,0.488189}%
\pgfsetfillcolor{currentfill}%
\pgfsetlinewidth{0.000000pt}%
\definecolor{currentstroke}{rgb}{0.239374,0.735588,0.455688}%
\pgfsetstrokecolor{currentstroke}%
\pgfsetdash{}{0pt}%
\pgfpathmoveto{\pgfqpoint{4.161873in}{3.857033in}}%
\pgfpathlineto{\pgfqpoint{4.223116in}{3.486528in}}%
\pgfpathlineto{\pgfqpoint{4.307790in}{3.685037in}}%
\pgfpathclose%
\pgfusepath{fill}%
\end{pgfscope}%
\begin{pgfscope}%
\pgfpathrectangle{\pgfqpoint{0.539299in}{0.078740in}}{\pgfqpoint{7.842520in}{7.842520in}}%
\pgfusepath{clip}%
\pgfsetbuttcap%
\pgfsetroundjoin%
\definecolor{currentfill}{rgb}{0.131172,0.555899,0.552459}%
\pgfsetfillcolor{currentfill}%
\pgfsetlinewidth{0.000000pt}%
\definecolor{currentstroke}{rgb}{0.246070,0.738910,0.452024}%
\pgfsetstrokecolor{currentstroke}%
\pgfsetdash{}{0pt}%
\pgfpathmoveto{\pgfqpoint{3.909430in}{2.951944in}}%
\pgfpathlineto{\pgfqpoint{4.054864in}{2.841426in}}%
\pgfpathlineto{\pgfqpoint{4.138693in}{3.208126in}}%
\pgfpathclose%
\pgfusepath{fill}%
\end{pgfscope}%
\begin{pgfscope}%
\pgfpathrectangle{\pgfqpoint{0.539299in}{0.078740in}}{\pgfqpoint{7.842520in}{7.842520in}}%
\pgfusepath{clip}%
\pgfsetbuttcap%
\pgfsetroundjoin%
\definecolor{currentfill}{rgb}{0.226397,0.728888,0.462789}%
\pgfsetfillcolor{currentfill}%
\pgfsetlinewidth{0.000000pt}%
\definecolor{currentstroke}{rgb}{0.252899,0.742211,0.448284}%
\pgfsetstrokecolor{currentstroke}%
\pgfsetdash{}{0pt}%
\pgfpathmoveto{\pgfqpoint{3.931280in}{3.785621in}}%
\pgfpathlineto{\pgfqpoint{4.077139in}{3.639292in}}%
\pgfpathlineto{\pgfqpoint{4.015993in}{4.022293in}}%
\pgfpathclose%
\pgfusepath{fill}%
\end{pgfscope}%
\begin{pgfscope}%
\pgfpathrectangle{\pgfqpoint{0.539299in}{0.078740in}}{\pgfqpoint{7.842520in}{7.842520in}}%
\pgfusepath{clip}%
\pgfsetbuttcap%
\pgfsetroundjoin%
\definecolor{currentfill}{rgb}{0.146180,0.515413,0.556823}%
\pgfsetfillcolor{currentfill}%
\pgfsetlinewidth{0.000000pt}%
\definecolor{currentstroke}{rgb}{0.259857,0.745492,0.444467}%
\pgfsetstrokecolor{currentstroke}%
\pgfsetdash{}{0pt}%
\pgfpathmoveto{\pgfqpoint{4.891094in}{2.947957in}}%
\pgfpathlineto{\pgfqpoint{4.807495in}{2.824405in}}%
\pgfpathlineto{\pgfqpoint{4.953553in}{2.645992in}}%
\pgfpathclose%
\pgfusepath{fill}%
\end{pgfscope}%
\begin{pgfscope}%
\pgfpathrectangle{\pgfqpoint{0.539299in}{0.078740in}}{\pgfqpoint{7.842520in}{7.842520in}}%
\pgfusepath{clip}%
\pgfsetbuttcap%
\pgfsetroundjoin%
\definecolor{currentfill}{rgb}{0.121148,0.592739,0.544641}%
\pgfsetfillcolor{currentfill}%
\pgfsetlinewidth{0.000000pt}%
\definecolor{currentstroke}{rgb}{0.266941,0.748751,0.440573}%
\pgfsetstrokecolor{currentstroke}%
\pgfsetdash{}{0pt}%
\pgfpathmoveto{\pgfqpoint{4.138693in}{3.208126in}}%
\pgfpathlineto{\pgfqpoint{3.992871in}{3.340173in}}%
\pgfpathlineto{\pgfqpoint{3.909430in}{2.951944in}}%
\pgfpathclose%
\pgfusepath{fill}%
\end{pgfscope}%
\begin{pgfscope}%
\pgfpathrectangle{\pgfqpoint{0.539299in}{0.078740in}}{\pgfqpoint{7.842520in}{7.842520in}}%
\pgfusepath{clip}%
\pgfsetbuttcap%
\pgfsetroundjoin%
\definecolor{currentfill}{rgb}{0.146616,0.673050,0.508936}%
\pgfsetfillcolor{currentfill}%
\pgfsetlinewidth{0.000000pt}%
\definecolor{currentstroke}{rgb}{0.274149,0.751988,0.436601}%
\pgfsetstrokecolor{currentstroke}%
\pgfsetdash{}{0pt}%
\pgfpathmoveto{\pgfqpoint{3.931280in}{3.785621in}}%
\pgfpathlineto{\pgfqpoint{3.847250in}{3.466509in}}%
\pgfpathlineto{\pgfqpoint{3.992871in}{3.340173in}}%
\pgfpathclose%
\pgfusepath{fill}%
\end{pgfscope}%
\begin{pgfscope}%
\pgfpathrectangle{\pgfqpoint{0.539299in}{0.078740in}}{\pgfqpoint{7.842520in}{7.842520in}}%
\pgfusepath{clip}%
\pgfsetbuttcap%
\pgfsetroundjoin%
\definecolor{currentfill}{rgb}{0.229739,0.322361,0.545706}%
\pgfsetfillcolor{currentfill}%
\pgfsetlinewidth{0.000000pt}%
\definecolor{currentstroke}{rgb}{0.281477,0.755203,0.432552}%
\pgfsetstrokecolor{currentstroke}%
\pgfsetdash{}{0pt}%
\pgfpathmoveto{\pgfqpoint{5.390896in}{2.059660in}}%
\pgfpathlineto{\pgfqpoint{5.308523in}{1.948082in}}%
\pgfpathlineto{\pgfqpoint{5.536081in}{1.836733in}}%
\pgfpathclose%
\pgfusepath{fill}%
\end{pgfscope}%
\begin{pgfscope}%
\pgfpathrectangle{\pgfqpoint{0.539299in}{0.078740in}}{\pgfqpoint{7.842520in}{7.842520in}}%
\pgfusepath{clip}%
\pgfsetbuttcap%
\pgfsetroundjoin%
\definecolor{currentfill}{rgb}{0.175707,0.697900,0.491033}%
\pgfsetfillcolor{currentfill}%
\pgfsetlinewidth{0.000000pt}%
\definecolor{currentstroke}{rgb}{0.288921,0.758394,0.428426}%
\pgfsetstrokecolor{currentstroke}%
\pgfsetdash{}{0pt}%
\pgfpathmoveto{\pgfqpoint{4.161873in}{3.857033in}}%
\pgfpathlineto{\pgfqpoint{4.077139in}{3.639292in}}%
\pgfpathlineto{\pgfqpoint{4.223116in}{3.486528in}}%
\pgfpathclose%
\pgfusepath{fill}%
\end{pgfscope}%
\begin{pgfscope}%
\pgfpathrectangle{\pgfqpoint{0.539299in}{0.078740in}}{\pgfqpoint{7.842520in}{7.842520in}}%
\pgfusepath{clip}%
\pgfsetbuttcap%
\pgfsetroundjoin%
\definecolor{currentfill}{rgb}{0.131172,0.555899,0.552459}%
\pgfsetfillcolor{currentfill}%
\pgfsetlinewidth{0.000000pt}%
\definecolor{currentstroke}{rgb}{0.296479,0.761561,0.424223}%
\pgfsetstrokecolor{currentstroke}%
\pgfsetdash{}{0pt}%
\pgfpathmoveto{\pgfqpoint{4.745380in}{3.138731in}}%
\pgfpathlineto{\pgfqpoint{4.661391in}{2.997194in}}%
\pgfpathlineto{\pgfqpoint{4.807495in}{2.824405in}}%
\pgfpathclose%
\pgfusepath{fill}%
\end{pgfscope}%
\begin{pgfscope}%
\pgfpathrectangle{\pgfqpoint{0.539299in}{0.078740in}}{\pgfqpoint{7.842520in}{7.842520in}}%
\pgfusepath{clip}%
\pgfsetbuttcap%
\pgfsetroundjoin%
\definecolor{currentfill}{rgb}{0.132268,0.655014,0.519661}%
\pgfsetfillcolor{currentfill}%
\pgfsetlinewidth{0.000000pt}%
\definecolor{currentstroke}{rgb}{0.304148,0.764704,0.419943}%
\pgfsetstrokecolor{currentstroke}%
\pgfsetdash{}{0pt}%
\pgfpathmoveto{\pgfqpoint{4.223116in}{3.486528in}}%
\pgfpathlineto{\pgfqpoint{4.369172in}{3.328258in}}%
\pgfpathlineto{\pgfqpoint{4.453701in}{3.507500in}}%
\pgfpathclose%
\pgfusepath{fill}%
\end{pgfscope}%
\begin{pgfscope}%
\pgfpathrectangle{\pgfqpoint{0.539299in}{0.078740in}}{\pgfqpoint{7.842520in}{7.842520in}}%
\pgfusepath{clip}%
\pgfsetbuttcap%
\pgfsetroundjoin%
\definecolor{currentfill}{rgb}{0.121148,0.592739,0.544641}%
\pgfsetfillcolor{currentfill}%
\pgfsetlinewidth{0.000000pt}%
\definecolor{currentstroke}{rgb}{0.311925,0.767822,0.415586}%
\pgfsetstrokecolor{currentstroke}%
\pgfsetdash{}{0pt}%
\pgfpathmoveto{\pgfqpoint{4.515273in}{3.165074in}}%
\pgfpathlineto{\pgfqpoint{4.661391in}{2.997194in}}%
\pgfpathlineto{\pgfqpoint{4.599574in}{3.325252in}}%
\pgfpathclose%
\pgfusepath{fill}%
\end{pgfscope}%
\begin{pgfscope}%
\pgfpathrectangle{\pgfqpoint{0.539299in}{0.078740in}}{\pgfqpoint{7.842520in}{7.842520in}}%
\pgfusepath{clip}%
\pgfsetbuttcap%
\pgfsetroundjoin%
\definecolor{currentfill}{rgb}{0.157851,0.683765,0.501686}%
\pgfsetfillcolor{currentfill}%
\pgfsetlinewidth{0.000000pt}%
\definecolor{currentstroke}{rgb}{0.319809,0.770914,0.411152}%
\pgfsetstrokecolor{currentstroke}%
\pgfsetdash{}{0pt}%
\pgfpathmoveto{\pgfqpoint{3.992871in}{3.340173in}}%
\pgfpathlineto{\pgfqpoint{4.077139in}{3.639292in}}%
\pgfpathlineto{\pgfqpoint{3.931280in}{3.785621in}}%
\pgfpathclose%
\pgfusepath{fill}%
\end{pgfscope}%
\begin{pgfscope}%
\pgfpathrectangle{\pgfqpoint{0.539299in}{0.078740in}}{\pgfqpoint{7.842520in}{7.842520in}}%
\pgfusepath{clip}%
\pgfsetbuttcap%
\pgfsetroundjoin%
\definecolor{currentfill}{rgb}{0.122312,0.633153,0.530398}%
\pgfsetfillcolor{currentfill}%
\pgfsetlinewidth{0.000000pt}%
\definecolor{currentstroke}{rgb}{0.327796,0.773980,0.406640}%
\pgfsetstrokecolor{currentstroke}%
\pgfsetdash{}{0pt}%
\pgfpathmoveto{\pgfqpoint{4.453701in}{3.507500in}}%
\pgfpathlineto{\pgfqpoint{4.369172in}{3.328258in}}%
\pgfpathlineto{\pgfqpoint{4.515273in}{3.165074in}}%
\pgfpathclose%
\pgfusepath{fill}%
\end{pgfscope}%
\begin{pgfscope}%
\pgfpathrectangle{\pgfqpoint{0.539299in}{0.078740in}}{\pgfqpoint{7.842520in}{7.842520in}}%
\pgfusepath{clip}%
\pgfsetbuttcap%
\pgfsetroundjoin%
\definecolor{currentfill}{rgb}{0.258965,0.251537,0.524736}%
\pgfsetfillcolor{currentfill}%
\pgfsetlinewidth{0.000000pt}%
\definecolor{currentstroke}{rgb}{0.335885,0.777018,0.402049}%
\pgfsetstrokecolor{currentstroke}%
\pgfsetdash{}{0pt}%
\pgfpathmoveto{\pgfqpoint{4.554418in}{1.995003in}}%
\pgfpathlineto{\pgfqpoint{4.470234in}{1.534400in}}%
\pgfpathlineto{\pgfqpoint{4.616159in}{1.458069in}}%
\pgfpathclose%
\pgfusepath{fill}%
\end{pgfscope}%
\begin{pgfscope}%
\pgfpathrectangle{\pgfqpoint{0.539299in}{0.078740in}}{\pgfqpoint{7.842520in}{7.842520in}}%
\pgfusepath{clip}%
\pgfsetbuttcap%
\pgfsetroundjoin%
\definecolor{currentfill}{rgb}{0.255645,0.260703,0.528312}%
\pgfsetfillcolor{currentfill}%
\pgfsetlinewidth{0.000000pt}%
\definecolor{currentstroke}{rgb}{0.344074,0.780029,0.397381}%
\pgfsetstrokecolor{currentstroke}%
\pgfsetdash{}{0pt}%
\pgfpathmoveto{\pgfqpoint{5.536081in}{1.836733in}}%
\pgfpathlineto{\pgfqpoint{5.454351in}{1.741746in}}%
\pgfpathlineto{\pgfqpoint{5.599643in}{1.509341in}}%
\pgfpathclose%
\pgfusepath{fill}%
\end{pgfscope}%
\begin{pgfscope}%
\pgfpathrectangle{\pgfqpoint{0.539299in}{0.078740in}}{\pgfqpoint{7.842520in}{7.842520in}}%
\pgfusepath{clip}%
\pgfsetbuttcap%
\pgfsetroundjoin%
\definecolor{currentfill}{rgb}{0.237441,0.305202,0.541921}%
\pgfsetfillcolor{currentfill}%
\pgfsetlinewidth{0.000000pt}%
\definecolor{currentstroke}{rgb}{0.352360,0.783011,0.392636}%
\pgfsetstrokecolor{currentstroke}%
\pgfsetdash{}{0pt}%
\pgfpathmoveto{\pgfqpoint{4.408361in}{2.095987in}}%
\pgfpathlineto{\pgfqpoint{4.470234in}{1.534400in}}%
\pgfpathlineto{\pgfqpoint{4.554418in}{1.995003in}}%
\pgfpathclose%
\pgfusepath{fill}%
\end{pgfscope}%
\begin{pgfscope}%
\pgfpathrectangle{\pgfqpoint{0.539299in}{0.078740in}}{\pgfqpoint{7.842520in}{7.842520in}}%
\pgfusepath{clip}%
\pgfsetbuttcap%
\pgfsetroundjoin%
\definecolor{currentfill}{rgb}{0.169646,0.456262,0.558030}%
\pgfsetfillcolor{currentfill}%
\pgfsetlinewidth{0.000000pt}%
\definecolor{currentstroke}{rgb}{0.360741,0.785964,0.387814}%
\pgfsetstrokecolor{currentstroke}%
\pgfsetdash{}{0pt}%
\pgfpathmoveto{\pgfqpoint{4.200551in}{2.726746in}}%
\pgfpathlineto{\pgfqpoint{4.262595in}{2.193629in}}%
\pgfpathlineto{\pgfqpoint{4.346469in}{2.608126in}}%
\pgfpathclose%
\pgfusepath{fill}%
\end{pgfscope}%
\begin{pgfscope}%
\pgfpathrectangle{\pgfqpoint{0.539299in}{0.078740in}}{\pgfqpoint{7.842520in}{7.842520in}}%
\pgfusepath{clip}%
\pgfsetbuttcap%
\pgfsetroundjoin%
\definecolor{currentfill}{rgb}{0.140210,0.665859,0.513427}%
\pgfsetfillcolor{currentfill}%
\pgfsetlinewidth{0.000000pt}%
\definecolor{currentstroke}{rgb}{0.369214,0.788888,0.382914}%
\pgfsetstrokecolor{currentstroke}%
\pgfsetdash{}{0pt}%
\pgfpathmoveto{\pgfqpoint{4.223116in}{3.486528in}}%
\pgfpathlineto{\pgfqpoint{4.077139in}{3.639292in}}%
\pgfpathlineto{\pgfqpoint{3.992871in}{3.340173in}}%
\pgfpathclose%
\pgfusepath{fill}%
\end{pgfscope}%
\begin{pgfscope}%
\pgfpathrectangle{\pgfqpoint{0.539299in}{0.078740in}}{\pgfqpoint{7.842520in}{7.842520in}}%
\pgfusepath{clip}%
\pgfsetbuttcap%
\pgfsetroundjoin%
\definecolor{currentfill}{rgb}{0.139147,0.533812,0.555298}%
\pgfsetfillcolor{currentfill}%
\pgfsetlinewidth{0.000000pt}%
\definecolor{currentstroke}{rgb}{0.377779,0.791781,0.377939}%
\pgfsetstrokecolor{currentstroke}%
\pgfsetdash{}{0pt}%
\pgfpathmoveto{\pgfqpoint{4.054864in}{2.841426in}}%
\pgfpathlineto{\pgfqpoint{4.200551in}{2.726746in}}%
\pgfpathlineto{\pgfqpoint{4.284682in}{3.071051in}}%
\pgfpathclose%
\pgfusepath{fill}%
\end{pgfscope}%
\begin{pgfscope}%
\pgfpathrectangle{\pgfqpoint{0.539299in}{0.078740in}}{\pgfqpoint{7.842520in}{7.842520in}}%
\pgfusepath{clip}%
\pgfsetbuttcap%
\pgfsetroundjoin%
\definecolor{currentfill}{rgb}{0.212395,0.359683,0.551710}%
\pgfsetfillcolor{currentfill}%
\pgfsetlinewidth{0.000000pt}%
\definecolor{currentstroke}{rgb}{0.386433,0.794644,0.372886}%
\pgfsetstrokecolor{currentstroke}%
\pgfsetdash{}{0pt}%
\pgfpathmoveto{\pgfqpoint{5.308523in}{1.948082in}}%
\pgfpathlineto{\pgfqpoint{5.390896in}{2.059660in}}%
\pgfpathlineto{\pgfqpoint{5.245331in}{2.266200in}}%
\pgfpathclose%
\pgfusepath{fill}%
\end{pgfscope}%
\begin{pgfscope}%
\pgfpathrectangle{\pgfqpoint{0.539299in}{0.078740in}}{\pgfqpoint{7.842520in}{7.842520in}}%
\pgfusepath{clip}%
\pgfsetbuttcap%
\pgfsetroundjoin%
\definecolor{currentfill}{rgb}{0.123444,0.636809,0.528763}%
\pgfsetfillcolor{currentfill}%
\pgfsetlinewidth{0.000000pt}%
\definecolor{currentstroke}{rgb}{0.395174,0.797475,0.367757}%
\pgfsetstrokecolor{currentstroke}%
\pgfsetdash{}{0pt}%
\pgfpathmoveto{\pgfqpoint{3.992871in}{3.340173in}}%
\pgfpathlineto{\pgfqpoint{4.138693in}{3.208126in}}%
\pgfpathlineto{\pgfqpoint{4.223116in}{3.486528in}}%
\pgfpathclose%
\pgfusepath{fill}%
\end{pgfscope}%
\begin{pgfscope}%
\pgfpathrectangle{\pgfqpoint{0.539299in}{0.078740in}}{\pgfqpoint{7.842520in}{7.842520in}}%
\pgfusepath{clip}%
\pgfsetbuttcap%
\pgfsetroundjoin%
\definecolor{currentfill}{rgb}{0.126453,0.570633,0.549841}%
\pgfsetfillcolor{currentfill}%
\pgfsetlinewidth{0.000000pt}%
\definecolor{currentstroke}{rgb}{0.404001,0.800275,0.362552}%
\pgfsetstrokecolor{currentstroke}%
\pgfsetdash{}{0pt}%
\pgfpathmoveto{\pgfqpoint{4.138693in}{3.208126in}}%
\pgfpathlineto{\pgfqpoint{4.054864in}{2.841426in}}%
\pgfpathlineto{\pgfqpoint{4.284682in}{3.071051in}}%
\pgfpathclose%
\pgfusepath{fill}%
\end{pgfscope}%
\begin{pgfscope}%
\pgfpathrectangle{\pgfqpoint{0.539299in}{0.078740in}}{\pgfqpoint{7.842520in}{7.842520in}}%
\pgfusepath{clip}%
\pgfsetbuttcap%
\pgfsetroundjoin%
\definecolor{currentfill}{rgb}{0.281887,0.150881,0.465405}%
\pgfsetfillcolor{currentfill}%
\pgfsetlinewidth{0.000000pt}%
\definecolor{currentstroke}{rgb}{0.412913,0.803041,0.357269}%
\pgfsetstrokecolor{currentstroke}%
\pgfsetdash{}{0pt}%
\pgfpathmoveto{\pgfqpoint{5.599643in}{1.509341in}}%
\pgfpathlineto{\pgfqpoint{5.662874in}{1.144010in}}%
\pgfpathlineto{\pgfqpoint{5.744062in}{1.240624in}}%
\pgfpathclose%
\pgfusepath{fill}%
\end{pgfscope}%
\begin{pgfscope}%
\pgfpathrectangle{\pgfqpoint{0.539299in}{0.078740in}}{\pgfqpoint{7.842520in}{7.842520in}}%
\pgfusepath{clip}%
\pgfsetbuttcap%
\pgfsetroundjoin%
\definecolor{currentfill}{rgb}{0.194100,0.399323,0.555565}%
\pgfsetfillcolor{currentfill}%
\pgfsetlinewidth{0.000000pt}%
\definecolor{currentstroke}{rgb}{0.421908,0.805774,0.351910}%
\pgfsetstrokecolor{currentstroke}%
\pgfsetdash{}{0pt}%
\pgfpathmoveto{\pgfqpoint{4.492593in}{2.485503in}}%
\pgfpathlineto{\pgfqpoint{4.262595in}{2.193629in}}%
\pgfpathlineto{\pgfqpoint{4.408361in}{2.095987in}}%
\pgfpathclose%
\pgfusepath{fill}%
\end{pgfscope}%
\begin{pgfscope}%
\pgfpathrectangle{\pgfqpoint{0.539299in}{0.078740in}}{\pgfqpoint{7.842520in}{7.842520in}}%
\pgfusepath{clip}%
\pgfsetbuttcap%
\pgfsetroundjoin%
\definecolor{currentfill}{rgb}{0.239346,0.300855,0.540844}%
\pgfsetfillcolor{currentfill}%
\pgfsetlinewidth{0.000000pt}%
\definecolor{currentstroke}{rgb}{0.430983,0.808473,0.346476}%
\pgfsetstrokecolor{currentstroke}%
\pgfsetdash{}{0pt}%
\pgfpathmoveto{\pgfqpoint{5.536081in}{1.836733in}}%
\pgfpathlineto{\pgfqpoint{5.308523in}{1.948082in}}%
\pgfpathlineto{\pgfqpoint{5.454351in}{1.741746in}}%
\pgfpathclose%
\pgfusepath{fill}%
\end{pgfscope}%
\begin{pgfscope}%
\pgfpathrectangle{\pgfqpoint{0.539299in}{0.078740in}}{\pgfqpoint{7.842520in}{7.842520in}}%
\pgfusepath{clip}%
\pgfsetbuttcap%
\pgfsetroundjoin%
\definecolor{currentfill}{rgb}{0.175841,0.441290,0.557685}%
\pgfsetfillcolor{currentfill}%
\pgfsetlinewidth{0.000000pt}%
\definecolor{currentstroke}{rgb}{0.440137,0.811138,0.340967}%
\pgfsetstrokecolor{currentstroke}%
\pgfsetdash{}{0pt}%
\pgfpathmoveto{\pgfqpoint{4.346469in}{2.608126in}}%
\pgfpathlineto{\pgfqpoint{4.262595in}{2.193629in}}%
\pgfpathlineto{\pgfqpoint{4.492593in}{2.485503in}}%
\pgfpathclose%
\pgfusepath{fill}%
\end{pgfscope}%
\begin{pgfscope}%
\pgfpathrectangle{\pgfqpoint{0.539299in}{0.078740in}}{\pgfqpoint{7.842520in}{7.842520in}}%
\pgfusepath{clip}%
\pgfsetbuttcap%
\pgfsetroundjoin%
\definecolor{currentfill}{rgb}{0.120638,0.625828,0.533488}%
\pgfsetfillcolor{currentfill}%
\pgfsetlinewidth{0.000000pt}%
\definecolor{currentstroke}{rgb}{0.449368,0.813768,0.335384}%
\pgfsetstrokecolor{currentstroke}%
\pgfsetdash{}{0pt}%
\pgfpathmoveto{\pgfqpoint{4.284682in}{3.071051in}}%
\pgfpathlineto{\pgfqpoint{4.369172in}{3.328258in}}%
\pgfpathlineto{\pgfqpoint{4.223116in}{3.486528in}}%
\pgfpathclose%
\pgfusepath{fill}%
\end{pgfscope}%
\begin{pgfscope}%
\pgfpathrectangle{\pgfqpoint{0.539299in}{0.078740in}}{\pgfqpoint{7.842520in}{7.842520in}}%
\pgfusepath{clip}%
\pgfsetbuttcap%
\pgfsetroundjoin%
\definecolor{currentfill}{rgb}{0.119699,0.618490,0.536347}%
\pgfsetfillcolor{currentfill}%
\pgfsetlinewidth{0.000000pt}%
\definecolor{currentstroke}{rgb}{0.458674,0.816363,0.329727}%
\pgfsetstrokecolor{currentstroke}%
\pgfsetdash{}{0pt}%
\pgfpathmoveto{\pgfqpoint{4.223116in}{3.486528in}}%
\pgfpathlineto{\pgfqpoint{4.138693in}{3.208126in}}%
\pgfpathlineto{\pgfqpoint{4.284682in}{3.071051in}}%
\pgfpathclose%
\pgfusepath{fill}%
\end{pgfscope}%
\begin{pgfscope}%
\pgfpathrectangle{\pgfqpoint{0.539299in}{0.078740in}}{\pgfqpoint{7.842520in}{7.842520in}}%
\pgfusepath{clip}%
\pgfsetbuttcap%
\pgfsetroundjoin%
\definecolor{currentfill}{rgb}{0.119738,0.603785,0.541400}%
\pgfsetfillcolor{currentfill}%
\pgfsetlinewidth{0.000000pt}%
\definecolor{currentstroke}{rgb}{0.468053,0.818921,0.323998}%
\pgfsetstrokecolor{currentstroke}%
\pgfsetdash{}{0pt}%
\pgfpathmoveto{\pgfqpoint{4.515273in}{3.165074in}}%
\pgfpathlineto{\pgfqpoint{4.369172in}{3.328258in}}%
\pgfpathlineto{\pgfqpoint{4.284682in}{3.071051in}}%
\pgfpathclose%
\pgfusepath{fill}%
\end{pgfscope}%
\begin{pgfscope}%
\pgfpathrectangle{\pgfqpoint{0.539299in}{0.078740in}}{\pgfqpoint{7.842520in}{7.842520in}}%
\pgfusepath{clip}%
\pgfsetbuttcap%
\pgfsetroundjoin%
\definecolor{currentfill}{rgb}{0.183898,0.422383,0.556944}%
\pgfsetfillcolor{currentfill}%
\pgfsetlinewidth{0.000000pt}%
\definecolor{currentstroke}{rgb}{0.477504,0.821444,0.318195}%
\pgfsetstrokecolor{currentstroke}%
\pgfsetdash{}{0pt}%
\pgfpathmoveto{\pgfqpoint{5.245331in}{2.266200in}}%
\pgfpathlineto{\pgfqpoint{5.099519in}{2.460623in}}%
\pgfpathlineto{\pgfqpoint{5.016082in}{2.310084in}}%
\pgfpathclose%
\pgfusepath{fill}%
\end{pgfscope}%
\begin{pgfscope}%
\pgfpathrectangle{\pgfqpoint{0.539299in}{0.078740in}}{\pgfqpoint{7.842520in}{7.842520in}}%
\pgfusepath{clip}%
\pgfsetbuttcap%
\pgfsetroundjoin%
\definecolor{currentfill}{rgb}{0.143343,0.522773,0.556295}%
\pgfsetfillcolor{currentfill}%
\pgfsetlinewidth{0.000000pt}%
\definecolor{currentstroke}{rgb}{0.487026,0.823929,0.312321}%
\pgfsetstrokecolor{currentstroke}%
\pgfsetdash{}{0pt}%
\pgfpathmoveto{\pgfqpoint{4.284682in}{3.071051in}}%
\pgfpathlineto{\pgfqpoint{4.200551in}{2.726746in}}%
\pgfpathlineto{\pgfqpoint{4.346469in}{2.608126in}}%
\pgfpathclose%
\pgfusepath{fill}%
\end{pgfscope}%
\begin{pgfscope}%
\pgfpathrectangle{\pgfqpoint{0.539299in}{0.078740in}}{\pgfqpoint{7.842520in}{7.842520in}}%
\pgfusepath{clip}%
\pgfsetbuttcap%
\pgfsetroundjoin%
\definecolor{currentfill}{rgb}{0.171176,0.452530,0.557965}%
\pgfsetfillcolor{currentfill}%
\pgfsetlinewidth{0.000000pt}%
\definecolor{currentstroke}{rgb}{0.496615,0.826376,0.306377}%
\pgfsetstrokecolor{currentstroke}%
\pgfsetdash{}{0pt}%
\pgfpathmoveto{\pgfqpoint{5.099519in}{2.460623in}}%
\pgfpathlineto{\pgfqpoint{4.953553in}{2.645992in}}%
\pgfpathlineto{\pgfqpoint{5.016082in}{2.310084in}}%
\pgfpathclose%
\pgfusepath{fill}%
\end{pgfscope}%
\begin{pgfscope}%
\pgfpathrectangle{\pgfqpoint{0.539299in}{0.078740in}}{\pgfqpoint{7.842520in}{7.842520in}}%
\pgfusepath{clip}%
\pgfsetbuttcap%
\pgfsetroundjoin%
\definecolor{currentfill}{rgb}{0.137770,0.537492,0.554906}%
\pgfsetfillcolor{currentfill}%
\pgfsetlinewidth{0.000000pt}%
\definecolor{currentstroke}{rgb}{0.506271,0.828786,0.300362}%
\pgfsetstrokecolor{currentstroke}%
\pgfsetdash{}{0pt}%
\pgfpathmoveto{\pgfqpoint{4.661391in}{2.997194in}}%
\pgfpathlineto{\pgfqpoint{4.577041in}{2.783012in}}%
\pgfpathlineto{\pgfqpoint{4.807495in}{2.824405in}}%
\pgfpathclose%
\pgfusepath{fill}%
\end{pgfscope}%
\begin{pgfscope}%
\pgfpathrectangle{\pgfqpoint{0.539299in}{0.078740in}}{\pgfqpoint{7.842520in}{7.842520in}}%
\pgfusepath{clip}%
\pgfsetbuttcap%
\pgfsetroundjoin%
\definecolor{currentfill}{rgb}{0.129933,0.559582,0.551864}%
\pgfsetfillcolor{currentfill}%
\pgfsetlinewidth{0.000000pt}%
\definecolor{currentstroke}{rgb}{0.515992,0.831158,0.294279}%
\pgfsetstrokecolor{currentstroke}%
\pgfsetdash{}{0pt}%
\pgfpathmoveto{\pgfqpoint{4.577041in}{2.783012in}}%
\pgfpathlineto{\pgfqpoint{4.661391in}{2.997194in}}%
\pgfpathlineto{\pgfqpoint{4.515273in}{3.165074in}}%
\pgfpathclose%
\pgfusepath{fill}%
\end{pgfscope}%
\begin{pgfscope}%
\pgfpathrectangle{\pgfqpoint{0.539299in}{0.078740in}}{\pgfqpoint{7.842520in}{7.842520in}}%
\pgfusepath{clip}%
\pgfsetbuttcap%
\pgfsetroundjoin%
\definecolor{currentfill}{rgb}{0.151918,0.500685,0.557587}%
\pgfsetfillcolor{currentfill}%
\pgfsetlinewidth{0.000000pt}%
\definecolor{currentstroke}{rgb}{0.525776,0.833491,0.288127}%
\pgfsetstrokecolor{currentstroke}%
\pgfsetdash{}{0pt}%
\pgfpathmoveto{\pgfqpoint{4.953553in}{2.645992in}}%
\pgfpathlineto{\pgfqpoint{4.807495in}{2.824405in}}%
\pgfpathlineto{\pgfqpoint{4.723356in}{2.631725in}}%
\pgfpathclose%
\pgfusepath{fill}%
\end{pgfscope}%
\begin{pgfscope}%
\pgfpathrectangle{\pgfqpoint{0.539299in}{0.078740in}}{\pgfqpoint{7.842520in}{7.842520in}}%
\pgfusepath{clip}%
\pgfsetbuttcap%
\pgfsetroundjoin%
\definecolor{currentfill}{rgb}{0.124395,0.578002,0.548287}%
\pgfsetfillcolor{currentfill}%
\pgfsetlinewidth{0.000000pt}%
\definecolor{currentstroke}{rgb}{0.535621,0.835785,0.281908}%
\pgfsetstrokecolor{currentstroke}%
\pgfsetdash{}{0pt}%
\pgfpathmoveto{\pgfqpoint{4.284682in}{3.071051in}}%
\pgfpathlineto{\pgfqpoint{4.430807in}{2.929332in}}%
\pgfpathlineto{\pgfqpoint{4.515273in}{3.165074in}}%
\pgfpathclose%
\pgfusepath{fill}%
\end{pgfscope}%
\begin{pgfscope}%
\pgfpathrectangle{\pgfqpoint{0.539299in}{0.078740in}}{\pgfqpoint{7.842520in}{7.842520in}}%
\pgfusepath{clip}%
\pgfsetbuttcap%
\pgfsetroundjoin%
\definecolor{currentfill}{rgb}{0.199430,0.387607,0.554642}%
\pgfsetfillcolor{currentfill}%
\pgfsetlinewidth{0.000000pt}%
\definecolor{currentstroke}{rgb}{0.545524,0.838039,0.275626}%
\pgfsetstrokecolor{currentstroke}%
\pgfsetdash{}{0pt}%
\pgfpathmoveto{\pgfqpoint{4.492593in}{2.485503in}}%
\pgfpathlineto{\pgfqpoint{4.408361in}{2.095987in}}%
\pgfpathlineto{\pgfqpoint{4.554418in}{1.995003in}}%
\pgfpathclose%
\pgfusepath{fill}%
\end{pgfscope}%
\begin{pgfscope}%
\pgfpathrectangle{\pgfqpoint{0.539299in}{0.078740in}}{\pgfqpoint{7.842520in}{7.842520in}}%
\pgfusepath{clip}%
\pgfsetbuttcap%
\pgfsetroundjoin%
\definecolor{currentfill}{rgb}{0.137770,0.537492,0.554906}%
\pgfsetfillcolor{currentfill}%
\pgfsetlinewidth{0.000000pt}%
\definecolor{currentstroke}{rgb}{0.555484,0.840254,0.269281}%
\pgfsetstrokecolor{currentstroke}%
\pgfsetdash{}{0pt}%
\pgfpathmoveto{\pgfqpoint{4.346469in}{2.608126in}}%
\pgfpathlineto{\pgfqpoint{4.430807in}{2.929332in}}%
\pgfpathlineto{\pgfqpoint{4.284682in}{3.071051in}}%
\pgfpathclose%
\pgfusepath{fill}%
\end{pgfscope}%
\begin{pgfscope}%
\pgfpathrectangle{\pgfqpoint{0.539299in}{0.078740in}}{\pgfqpoint{7.842520in}{7.842520in}}%
\pgfusepath{clip}%
\pgfsetbuttcap%
\pgfsetroundjoin%
\definecolor{currentfill}{rgb}{0.243113,0.292092,0.538516}%
\pgfsetfillcolor{currentfill}%
\pgfsetlinewidth{0.000000pt}%
\definecolor{currentstroke}{rgb}{0.565498,0.842430,0.262877}%
\pgfsetstrokecolor{currentstroke}%
\pgfsetdash{}{0pt}%
\pgfpathmoveto{\pgfqpoint{4.616159in}{1.458069in}}%
\pgfpathlineto{\pgfqpoint{4.700745in}{1.889957in}}%
\pgfpathlineto{\pgfqpoint{4.554418in}{1.995003in}}%
\pgfpathclose%
\pgfusepath{fill}%
\end{pgfscope}%
\begin{pgfscope}%
\pgfpathrectangle{\pgfqpoint{0.539299in}{0.078740in}}{\pgfqpoint{7.842520in}{7.842520in}}%
\pgfusepath{clip}%
\pgfsetbuttcap%
\pgfsetroundjoin%
\definecolor{currentfill}{rgb}{0.131172,0.555899,0.552459}%
\pgfsetfillcolor{currentfill}%
\pgfsetlinewidth{0.000000pt}%
\definecolor{currentstroke}{rgb}{0.575563,0.844566,0.256415}%
\pgfsetstrokecolor{currentstroke}%
\pgfsetdash{}{0pt}%
\pgfpathmoveto{\pgfqpoint{4.515273in}{3.165074in}}%
\pgfpathlineto{\pgfqpoint{4.430807in}{2.929332in}}%
\pgfpathlineto{\pgfqpoint{4.577041in}{2.783012in}}%
\pgfpathclose%
\pgfusepath{fill}%
\end{pgfscope}%
\begin{pgfscope}%
\pgfpathrectangle{\pgfqpoint{0.539299in}{0.078740in}}{\pgfqpoint{7.842520in}{7.842520in}}%
\pgfusepath{clip}%
\pgfsetbuttcap%
\pgfsetroundjoin%
\definecolor{currentfill}{rgb}{0.206756,0.371758,0.553117}%
\pgfsetfillcolor{currentfill}%
\pgfsetlinewidth{0.000000pt}%
\definecolor{currentstroke}{rgb}{0.585678,0.846661,0.249897}%
\pgfsetstrokecolor{currentstroke}%
\pgfsetdash{}{0pt}%
\pgfpathmoveto{\pgfqpoint{5.245331in}{2.266200in}}%
\pgfpathlineto{\pgfqpoint{5.162383in}{2.135782in}}%
\pgfpathlineto{\pgfqpoint{5.308523in}{1.948082in}}%
\pgfpathclose%
\pgfusepath{fill}%
\end{pgfscope}%
\begin{pgfscope}%
\pgfpathrectangle{\pgfqpoint{0.539299in}{0.078740in}}{\pgfqpoint{7.842520in}{7.842520in}}%
\pgfusepath{clip}%
\pgfsetbuttcap%
\pgfsetroundjoin%
\definecolor{currentfill}{rgb}{0.153364,0.497000,0.557724}%
\pgfsetfillcolor{currentfill}%
\pgfsetlinewidth{0.000000pt}%
\definecolor{currentstroke}{rgb}{0.595839,0.848717,0.243329}%
\pgfsetstrokecolor{currentstroke}%
\pgfsetdash{}{0pt}%
\pgfpathmoveto{\pgfqpoint{4.492593in}{2.485503in}}%
\pgfpathlineto{\pgfqpoint{4.430807in}{2.929332in}}%
\pgfpathlineto{\pgfqpoint{4.346469in}{2.608126in}}%
\pgfpathclose%
\pgfusepath{fill}%
\end{pgfscope}%
\begin{pgfscope}%
\pgfpathrectangle{\pgfqpoint{0.539299in}{0.078740in}}{\pgfqpoint{7.842520in}{7.842520in}}%
\pgfusepath{clip}%
\pgfsetbuttcap%
\pgfsetroundjoin%
\definecolor{currentfill}{rgb}{0.192357,0.403199,0.555836}%
\pgfsetfillcolor{currentfill}%
\pgfsetlinewidth{0.000000pt}%
\definecolor{currentstroke}{rgb}{0.606045,0.850733,0.236712}%
\pgfsetstrokecolor{currentstroke}%
\pgfsetdash{}{0pt}%
\pgfpathmoveto{\pgfqpoint{5.016082in}{2.310084in}}%
\pgfpathlineto{\pgfqpoint{5.162383in}{2.135782in}}%
\pgfpathlineto{\pgfqpoint{5.245331in}{2.266200in}}%
\pgfpathclose%
\pgfusepath{fill}%
\end{pgfscope}%
\begin{pgfscope}%
\pgfpathrectangle{\pgfqpoint{0.539299in}{0.078740in}}{\pgfqpoint{7.842520in}{7.842520in}}%
\pgfusepath{clip}%
\pgfsetbuttcap%
\pgfsetroundjoin%
\definecolor{currentfill}{rgb}{0.147607,0.511733,0.557049}%
\pgfsetfillcolor{currentfill}%
\pgfsetlinewidth{0.000000pt}%
\definecolor{currentstroke}{rgb}{0.616293,0.852709,0.230052}%
\pgfsetstrokecolor{currentstroke}%
\pgfsetdash{}{0pt}%
\pgfpathmoveto{\pgfqpoint{4.807495in}{2.824405in}}%
\pgfpathlineto{\pgfqpoint{4.577041in}{2.783012in}}%
\pgfpathlineto{\pgfqpoint{4.723356in}{2.631725in}}%
\pgfpathclose%
\pgfusepath{fill}%
\end{pgfscope}%
\begin{pgfscope}%
\pgfpathrectangle{\pgfqpoint{0.539299in}{0.078740in}}{\pgfqpoint{7.842520in}{7.842520in}}%
\pgfusepath{clip}%
\pgfsetbuttcap%
\pgfsetroundjoin%
\definecolor{currentfill}{rgb}{0.160665,0.478540,0.558115}%
\pgfsetfillcolor{currentfill}%
\pgfsetlinewidth{0.000000pt}%
\definecolor{currentstroke}{rgb}{0.626579,0.854645,0.223353}%
\pgfsetstrokecolor{currentstroke}%
\pgfsetdash{}{0pt}%
\pgfpathmoveto{\pgfqpoint{4.723356in}{2.631725in}}%
\pgfpathlineto{\pgfqpoint{4.869718in}{2.474594in}}%
\pgfpathlineto{\pgfqpoint{4.953553in}{2.645992in}}%
\pgfpathclose%
\pgfusepath{fill}%
\end{pgfscope}%
\begin{pgfscope}%
\pgfpathrectangle{\pgfqpoint{0.539299in}{0.078740in}}{\pgfqpoint{7.842520in}{7.842520in}}%
\pgfusepath{clip}%
\pgfsetbuttcap%
\pgfsetroundjoin%
\definecolor{currentfill}{rgb}{0.266580,0.228262,0.514349}%
\pgfsetfillcolor{currentfill}%
\pgfsetlinewidth{0.000000pt}%
\definecolor{currentstroke}{rgb}{0.636902,0.856542,0.216620}%
\pgfsetstrokecolor{currentstroke}%
\pgfsetdash{}{0pt}%
\pgfpathmoveto{\pgfqpoint{4.616159in}{1.458069in}}%
\pgfpathlineto{\pgfqpoint{4.762457in}{1.378786in}}%
\pgfpathlineto{\pgfqpoint{4.847314in}{1.779621in}}%
\pgfpathclose%
\pgfusepath{fill}%
\end{pgfscope}%
\begin{pgfscope}%
\pgfpathrectangle{\pgfqpoint{0.539299in}{0.078740in}}{\pgfqpoint{7.842520in}{7.842520in}}%
\pgfusepath{clip}%
\pgfsetbuttcap%
\pgfsetroundjoin%
\definecolor{currentfill}{rgb}{0.266580,0.228262,0.514349}%
\pgfsetfillcolor{currentfill}%
\pgfsetlinewidth{0.000000pt}%
\definecolor{currentstroke}{rgb}{0.647257,0.858400,0.209861}%
\pgfsetstrokecolor{currentstroke}%
\pgfsetdash{}{0pt}%
\pgfpathmoveto{\pgfqpoint{5.517631in}{1.390496in}}%
\pgfpathlineto{\pgfqpoint{5.599643in}{1.509341in}}%
\pgfpathlineto{\pgfqpoint{5.454351in}{1.741746in}}%
\pgfpathclose%
\pgfusepath{fill}%
\end{pgfscope}%
\begin{pgfscope}%
\pgfpathrectangle{\pgfqpoint{0.539299in}{0.078740in}}{\pgfqpoint{7.842520in}{7.842520in}}%
\pgfusepath{clip}%
\pgfsetbuttcap%
\pgfsetroundjoin%
\definecolor{currentfill}{rgb}{0.169646,0.456262,0.558030}%
\pgfsetfillcolor{currentfill}%
\pgfsetlinewidth{0.000000pt}%
\definecolor{currentstroke}{rgb}{0.657642,0.860219,0.203082}%
\pgfsetstrokecolor{currentstroke}%
\pgfsetdash{}{0pt}%
\pgfpathmoveto{\pgfqpoint{5.016082in}{2.310084in}}%
\pgfpathlineto{\pgfqpoint{4.953553in}{2.645992in}}%
\pgfpathlineto{\pgfqpoint{4.869718in}{2.474594in}}%
\pgfpathclose%
\pgfusepath{fill}%
\end{pgfscope}%
\begin{pgfscope}%
\pgfpathrectangle{\pgfqpoint{0.539299in}{0.078740in}}{\pgfqpoint{7.842520in}{7.842520in}}%
\pgfusepath{clip}%
\pgfsetbuttcap%
\pgfsetroundjoin%
\definecolor{currentfill}{rgb}{0.147607,0.511733,0.557049}%
\pgfsetfillcolor{currentfill}%
\pgfsetlinewidth{0.000000pt}%
\definecolor{currentstroke}{rgb}{0.668054,0.861999,0.196293}%
\pgfsetstrokecolor{currentstroke}%
\pgfsetdash{}{0pt}%
\pgfpathmoveto{\pgfqpoint{4.492593in}{2.485503in}}%
\pgfpathlineto{\pgfqpoint{4.577041in}{2.783012in}}%
\pgfpathlineto{\pgfqpoint{4.430807in}{2.929332in}}%
\pgfpathclose%
\pgfusepath{fill}%
\end{pgfscope}%
\begin{pgfscope}%
\pgfpathrectangle{\pgfqpoint{0.539299in}{0.078740in}}{\pgfqpoint{7.842520in}{7.842520in}}%
\pgfusepath{clip}%
\pgfsetbuttcap%
\pgfsetroundjoin%
\definecolor{currentfill}{rgb}{0.278826,0.175490,0.483397}%
\pgfsetfillcolor{currentfill}%
\pgfsetlinewidth{0.000000pt}%
\definecolor{currentstroke}{rgb}{0.678489,0.863742,0.189503}%
\pgfsetstrokecolor{currentstroke}%
\pgfsetdash{}{0pt}%
\pgfpathmoveto{\pgfqpoint{5.517631in}{1.390496in}}%
\pgfpathlineto{\pgfqpoint{5.662874in}{1.144010in}}%
\pgfpathlineto{\pgfqpoint{5.599643in}{1.509341in}}%
\pgfpathclose%
\pgfusepath{fill}%
\end{pgfscope}%
\begin{pgfscope}%
\pgfpathrectangle{\pgfqpoint{0.539299in}{0.078740in}}{\pgfqpoint{7.842520in}{7.842520in}}%
\pgfusepath{clip}%
\pgfsetbuttcap%
\pgfsetroundjoin%
\definecolor{currentfill}{rgb}{0.188923,0.410910,0.556326}%
\pgfsetfillcolor{currentfill}%
\pgfsetlinewidth{0.000000pt}%
\definecolor{currentstroke}{rgb}{0.688944,0.865448,0.182725}%
\pgfsetstrokecolor{currentstroke}%
\pgfsetdash{}{0pt}%
\pgfpathmoveto{\pgfqpoint{4.554418in}{1.995003in}}%
\pgfpathlineto{\pgfqpoint{4.638902in}{2.358455in}}%
\pgfpathlineto{\pgfqpoint{4.492593in}{2.485503in}}%
\pgfpathclose%
\pgfusepath{fill}%
\end{pgfscope}%
\begin{pgfscope}%
\pgfpathrectangle{\pgfqpoint{0.539299in}{0.078740in}}{\pgfqpoint{7.842520in}{7.842520in}}%
\pgfusepath{clip}%
\pgfsetbuttcap%
\pgfsetroundjoin%
\definecolor{currentfill}{rgb}{0.250425,0.274290,0.533103}%
\pgfsetfillcolor{currentfill}%
\pgfsetlinewidth{0.000000pt}%
\definecolor{currentstroke}{rgb}{0.699415,0.867117,0.175971}%
\pgfsetstrokecolor{currentstroke}%
\pgfsetdash{}{0pt}%
\pgfpathmoveto{\pgfqpoint{4.847314in}{1.779621in}}%
\pgfpathlineto{\pgfqpoint{4.700745in}{1.889957in}}%
\pgfpathlineto{\pgfqpoint{4.616159in}{1.458069in}}%
\pgfpathclose%
\pgfusepath{fill}%
\end{pgfscope}%
\begin{pgfscope}%
\pgfpathrectangle{\pgfqpoint{0.539299in}{0.078740in}}{\pgfqpoint{7.842520in}{7.842520in}}%
\pgfusepath{clip}%
\pgfsetbuttcap%
\pgfsetroundjoin%
\definecolor{currentfill}{rgb}{0.156270,0.489624,0.557936}%
\pgfsetfillcolor{currentfill}%
\pgfsetlinewidth{0.000000pt}%
\definecolor{currentstroke}{rgb}{0.709898,0.868751,0.169257}%
\pgfsetstrokecolor{currentstroke}%
\pgfsetdash{}{0pt}%
\pgfpathmoveto{\pgfqpoint{4.723356in}{2.631725in}}%
\pgfpathlineto{\pgfqpoint{4.577041in}{2.783012in}}%
\pgfpathlineto{\pgfqpoint{4.492593in}{2.485503in}}%
\pgfpathclose%
\pgfusepath{fill}%
\end{pgfscope}%
\begin{pgfscope}%
\pgfpathrectangle{\pgfqpoint{0.539299in}{0.078740in}}{\pgfqpoint{7.842520in}{7.842520in}}%
\pgfusepath{clip}%
\pgfsetbuttcap%
\pgfsetroundjoin%
\definecolor{currentfill}{rgb}{0.168126,0.459988,0.558082}%
\pgfsetfillcolor{currentfill}%
\pgfsetlinewidth{0.000000pt}%
\definecolor{currentstroke}{rgb}{0.720391,0.870350,0.162603}%
\pgfsetstrokecolor{currentstroke}%
\pgfsetdash{}{0pt}%
\pgfpathmoveto{\pgfqpoint{4.492593in}{2.485503in}}%
\pgfpathlineto{\pgfqpoint{4.638902in}{2.358455in}}%
\pgfpathlineto{\pgfqpoint{4.723356in}{2.631725in}}%
\pgfpathclose%
\pgfusepath{fill}%
\end{pgfscope}%
\begin{pgfscope}%
\pgfpathrectangle{\pgfqpoint{0.539299in}{0.078740in}}{\pgfqpoint{7.842520in}{7.842520in}}%
\pgfusepath{clip}%
\pgfsetbuttcap%
\pgfsetroundjoin%
\definecolor{currentfill}{rgb}{0.212395,0.359683,0.551710}%
\pgfsetfillcolor{currentfill}%
\pgfsetlinewidth{0.000000pt}%
\definecolor{currentstroke}{rgb}{0.730889,0.871916,0.156029}%
\pgfsetstrokecolor{currentstroke}%
\pgfsetdash{}{0pt}%
\pgfpathmoveto{\pgfqpoint{4.554418in}{1.995003in}}%
\pgfpathlineto{\pgfqpoint{4.700745in}{1.889957in}}%
\pgfpathlineto{\pgfqpoint{4.785366in}{2.226099in}}%
\pgfpathclose%
\pgfusepath{fill}%
\end{pgfscope}%
\begin{pgfscope}%
\pgfpathrectangle{\pgfqpoint{0.539299in}{0.078740in}}{\pgfqpoint{7.842520in}{7.842520in}}%
\pgfusepath{clip}%
\pgfsetbuttcap%
\pgfsetroundjoin%
\definecolor{currentfill}{rgb}{0.195860,0.395433,0.555276}%
\pgfsetfillcolor{currentfill}%
\pgfsetlinewidth{0.000000pt}%
\definecolor{currentstroke}{rgb}{0.741388,0.873449,0.149561}%
\pgfsetstrokecolor{currentstroke}%
\pgfsetdash{}{0pt}%
\pgfpathmoveto{\pgfqpoint{4.785366in}{2.226099in}}%
\pgfpathlineto{\pgfqpoint{4.638902in}{2.358455in}}%
\pgfpathlineto{\pgfqpoint{4.554418in}{1.995003in}}%
\pgfpathclose%
\pgfusepath{fill}%
\end{pgfscope}%
\begin{pgfscope}%
\pgfpathrectangle{\pgfqpoint{0.539299in}{0.078740in}}{\pgfqpoint{7.842520in}{7.842520in}}%
\pgfusepath{clip}%
\pgfsetbuttcap%
\pgfsetroundjoin%
\definecolor{currentfill}{rgb}{0.168126,0.459988,0.558082}%
\pgfsetfillcolor{currentfill}%
\pgfsetlinewidth{0.000000pt}%
\definecolor{currentstroke}{rgb}{0.751884,0.874951,0.143228}%
\pgfsetstrokecolor{currentstroke}%
\pgfsetdash{}{0pt}%
\pgfpathmoveto{\pgfqpoint{4.723356in}{2.631725in}}%
\pgfpathlineto{\pgfqpoint{4.638902in}{2.358455in}}%
\pgfpathlineto{\pgfqpoint{4.869718in}{2.474594in}}%
\pgfpathclose%
\pgfusepath{fill}%
\end{pgfscope}%
\begin{pgfscope}%
\pgfpathrectangle{\pgfqpoint{0.539299in}{0.078740in}}{\pgfqpoint{7.842520in}{7.842520in}}%
\pgfusepath{clip}%
\pgfsetbuttcap%
\pgfsetroundjoin%
\definecolor{currentfill}{rgb}{0.243113,0.292092,0.538516}%
\pgfsetfillcolor{currentfill}%
\pgfsetlinewidth{0.000000pt}%
\definecolor{currentstroke}{rgb}{0.762373,0.876424,0.137064}%
\pgfsetstrokecolor{currentstroke}%
\pgfsetdash{}{0pt}%
\pgfpathmoveto{\pgfqpoint{5.371612in}{1.597343in}}%
\pgfpathlineto{\pgfqpoint{5.454351in}{1.741746in}}%
\pgfpathlineto{\pgfqpoint{5.308523in}{1.948082in}}%
\pgfpathclose%
\pgfusepath{fill}%
\end{pgfscope}%
\begin{pgfscope}%
\pgfpathrectangle{\pgfqpoint{0.539299in}{0.078740in}}{\pgfqpoint{7.842520in}{7.842520in}}%
\pgfusepath{clip}%
\pgfsetbuttcap%
\pgfsetroundjoin%
\definecolor{currentfill}{rgb}{0.269308,0.218818,0.509577}%
\pgfsetfillcolor{currentfill}%
\pgfsetlinewidth{0.000000pt}%
\definecolor{currentstroke}{rgb}{0.772852,0.877868,0.131109}%
\pgfsetstrokecolor{currentstroke}%
\pgfsetdash{}{0pt}%
\pgfpathmoveto{\pgfqpoint{4.847314in}{1.779621in}}%
\pgfpathlineto{\pgfqpoint{4.762457in}{1.378786in}}%
\pgfpathlineto{\pgfqpoint{4.909107in}{1.295428in}}%
\pgfpathclose%
\pgfusepath{fill}%
\end{pgfscope}%
\begin{pgfscope}%
\pgfpathrectangle{\pgfqpoint{0.539299in}{0.078740in}}{\pgfqpoint{7.842520in}{7.842520in}}%
\pgfusepath{clip}%
\pgfsetbuttcap%
\pgfsetroundjoin%
\definecolor{currentfill}{rgb}{0.180629,0.429975,0.557282}%
\pgfsetfillcolor{currentfill}%
\pgfsetlinewidth{0.000000pt}%
\definecolor{currentstroke}{rgb}{0.783315,0.879285,0.125405}%
\pgfsetstrokecolor{currentstroke}%
\pgfsetdash{}{0pt}%
\pgfpathmoveto{\pgfqpoint{5.016082in}{2.310084in}}%
\pgfpathlineto{\pgfqpoint{4.869718in}{2.474594in}}%
\pgfpathlineto{\pgfqpoint{4.785366in}{2.226099in}}%
\pgfpathclose%
\pgfusepath{fill}%
\end{pgfscope}%
\begin{pgfscope}%
\pgfpathrectangle{\pgfqpoint{0.539299in}{0.078740in}}{\pgfqpoint{7.842520in}{7.842520in}}%
\pgfusepath{clip}%
\pgfsetbuttcap%
\pgfsetroundjoin%
\definecolor{currentfill}{rgb}{0.179019,0.433756,0.557430}%
\pgfsetfillcolor{currentfill}%
\pgfsetlinewidth{0.000000pt}%
\definecolor{currentstroke}{rgb}{0.793760,0.880678,0.120005}%
\pgfsetstrokecolor{currentstroke}%
\pgfsetdash{}{0pt}%
\pgfpathmoveto{\pgfqpoint{4.638902in}{2.358455in}}%
\pgfpathlineto{\pgfqpoint{4.785366in}{2.226099in}}%
\pgfpathlineto{\pgfqpoint{4.869718in}{2.474594in}}%
\pgfpathclose%
\pgfusepath{fill}%
\end{pgfscope}%
\begin{pgfscope}%
\pgfpathrectangle{\pgfqpoint{0.539299in}{0.078740in}}{\pgfqpoint{7.842520in}{7.842520in}}%
\pgfusepath{clip}%
\pgfsetbuttcap%
\pgfsetroundjoin%
\definecolor{currentfill}{rgb}{0.214298,0.355619,0.551184}%
\pgfsetfillcolor{currentfill}%
\pgfsetlinewidth{0.000000pt}%
\definecolor{currentstroke}{rgb}{0.804182,0.882046,0.114965}%
\pgfsetstrokecolor{currentstroke}%
\pgfsetdash{}{0pt}%
\pgfpathmoveto{\pgfqpoint{5.308523in}{1.948082in}}%
\pgfpathlineto{\pgfqpoint{5.162383in}{2.135782in}}%
\pgfpathlineto{\pgfqpoint{5.078583in}{1.938515in}}%
\pgfpathclose%
\pgfusepath{fill}%
\end{pgfscope}%
\begin{pgfscope}%
\pgfpathrectangle{\pgfqpoint{0.539299in}{0.078740in}}{\pgfqpoint{7.842520in}{7.842520in}}%
\pgfusepath{clip}%
\pgfsetbuttcap%
\pgfsetroundjoin%
\definecolor{currentfill}{rgb}{0.260571,0.246922,0.522828}%
\pgfsetfillcolor{currentfill}%
\pgfsetlinewidth{0.000000pt}%
\definecolor{currentstroke}{rgb}{0.814576,0.883393,0.110347}%
\pgfsetstrokecolor{currentstroke}%
\pgfsetdash{}{0pt}%
\pgfpathmoveto{\pgfqpoint{5.454351in}{1.741746in}}%
\pgfpathlineto{\pgfqpoint{5.371612in}{1.597343in}}%
\pgfpathlineto{\pgfqpoint{5.517631in}{1.390496in}}%
\pgfpathclose%
\pgfusepath{fill}%
\end{pgfscope}%
\begin{pgfscope}%
\pgfpathrectangle{\pgfqpoint{0.539299in}{0.078740in}}{\pgfqpoint{7.842520in}{7.842520in}}%
\pgfusepath{clip}%
\pgfsetbuttcap%
\pgfsetroundjoin%
\definecolor{currentfill}{rgb}{0.201239,0.383670,0.554294}%
\pgfsetfillcolor{currentfill}%
\pgfsetlinewidth{0.000000pt}%
\definecolor{currentstroke}{rgb}{0.824940,0.884720,0.106217}%
\pgfsetstrokecolor{currentstroke}%
\pgfsetdash{}{0pt}%
\pgfpathmoveto{\pgfqpoint{5.078583in}{1.938515in}}%
\pgfpathlineto{\pgfqpoint{5.162383in}{2.135782in}}%
\pgfpathlineto{\pgfqpoint{5.016082in}{2.310084in}}%
\pgfpathclose%
\pgfusepath{fill}%
\end{pgfscope}%
\begin{pgfscope}%
\pgfpathrectangle{\pgfqpoint{0.539299in}{0.078740in}}{\pgfqpoint{7.842520in}{7.842520in}}%
\pgfusepath{clip}%
\pgfsetbuttcap%
\pgfsetroundjoin%
\definecolor{currentfill}{rgb}{0.206756,0.371758,0.553117}%
\pgfsetfillcolor{currentfill}%
\pgfsetlinewidth{0.000000pt}%
\definecolor{currentstroke}{rgb}{0.835270,0.886029,0.102646}%
\pgfsetstrokecolor{currentstroke}%
\pgfsetdash{}{0pt}%
\pgfpathmoveto{\pgfqpoint{4.931945in}{2.086920in}}%
\pgfpathlineto{\pgfqpoint{4.785366in}{2.226099in}}%
\pgfpathlineto{\pgfqpoint{4.700745in}{1.889957in}}%
\pgfpathclose%
\pgfusepath{fill}%
\end{pgfscope}%
\begin{pgfscope}%
\pgfpathrectangle{\pgfqpoint{0.539299in}{0.078740in}}{\pgfqpoint{7.842520in}{7.842520in}}%
\pgfusepath{clip}%
\pgfsetbuttcap%
\pgfsetroundjoin%
\definecolor{currentfill}{rgb}{0.192357,0.403199,0.555836}%
\pgfsetfillcolor{currentfill}%
\pgfsetlinewidth{0.000000pt}%
\definecolor{currentstroke}{rgb}{0.845561,0.887322,0.099702}%
\pgfsetstrokecolor{currentstroke}%
\pgfsetdash{}{0pt}%
\pgfpathmoveto{\pgfqpoint{4.785366in}{2.226099in}}%
\pgfpathlineto{\pgfqpoint{4.931945in}{2.086920in}}%
\pgfpathlineto{\pgfqpoint{5.016082in}{2.310084in}}%
\pgfpathclose%
\pgfusepath{fill}%
\end{pgfscope}%
\begin{pgfscope}%
\pgfpathrectangle{\pgfqpoint{0.539299in}{0.078740in}}{\pgfqpoint{7.842520in}{7.842520in}}%
\pgfusepath{clip}%
\pgfsetbuttcap%
\pgfsetroundjoin%
\definecolor{currentfill}{rgb}{0.223925,0.334994,0.548053}%
\pgfsetfillcolor{currentfill}%
\pgfsetlinewidth{0.000000pt}%
\definecolor{currentstroke}{rgb}{0.855810,0.888601,0.097452}%
\pgfsetstrokecolor{currentstroke}%
\pgfsetdash{}{0pt}%
\pgfpathmoveto{\pgfqpoint{4.931945in}{2.086920in}}%
\pgfpathlineto{\pgfqpoint{4.700745in}{1.889957in}}%
\pgfpathlineto{\pgfqpoint{4.847314in}{1.779621in}}%
\pgfpathclose%
\pgfusepath{fill}%
\end{pgfscope}%
\begin{pgfscope}%
\pgfpathrectangle{\pgfqpoint{0.539299in}{0.078740in}}{\pgfqpoint{7.842520in}{7.842520in}}%
\pgfusepath{clip}%
\pgfsetbuttcap%
\pgfsetroundjoin%
\definecolor{currentfill}{rgb}{0.201239,0.383670,0.554294}%
\pgfsetfillcolor{currentfill}%
\pgfsetlinewidth{0.000000pt}%
\definecolor{currentstroke}{rgb}{0.866013,0.889868,0.095953}%
\pgfsetstrokecolor{currentstroke}%
\pgfsetdash{}{0pt}%
\pgfpathmoveto{\pgfqpoint{5.016082in}{2.310084in}}%
\pgfpathlineto{\pgfqpoint{4.931945in}{2.086920in}}%
\pgfpathlineto{\pgfqpoint{5.078583in}{1.938515in}}%
\pgfpathclose%
\pgfusepath{fill}%
\end{pgfscope}%
\begin{pgfscope}%
\pgfpathrectangle{\pgfqpoint{0.539299in}{0.078740in}}{\pgfqpoint{7.842520in}{7.842520in}}%
\pgfusepath{clip}%
\pgfsetbuttcap%
\pgfsetroundjoin%
\definecolor{currentfill}{rgb}{0.239346,0.300855,0.540844}%
\pgfsetfillcolor{currentfill}%
\pgfsetlinewidth{0.000000pt}%
\definecolor{currentstroke}{rgb}{0.876168,0.891125,0.095250}%
\pgfsetstrokecolor{currentstroke}%
\pgfsetdash{}{0pt}%
\pgfpathmoveto{\pgfqpoint{5.225186in}{1.777193in}}%
\pgfpathlineto{\pgfqpoint{5.371612in}{1.597343in}}%
\pgfpathlineto{\pgfqpoint{5.308523in}{1.948082in}}%
\pgfpathclose%
\pgfusepath{fill}%
\end{pgfscope}%
\begin{pgfscope}%
\pgfpathrectangle{\pgfqpoint{0.539299in}{0.078740in}}{\pgfqpoint{7.842520in}{7.842520in}}%
\pgfusepath{clip}%
\pgfsetbuttcap%
\pgfsetroundjoin%
\definecolor{currentfill}{rgb}{0.225863,0.330805,0.547314}%
\pgfsetfillcolor{currentfill}%
\pgfsetlinewidth{0.000000pt}%
\definecolor{currentstroke}{rgb}{0.886271,0.892374,0.095374}%
\pgfsetstrokecolor{currentstroke}%
\pgfsetdash{}{0pt}%
\pgfpathmoveto{\pgfqpoint{5.308523in}{1.948082in}}%
\pgfpathlineto{\pgfqpoint{5.078583in}{1.938515in}}%
\pgfpathlineto{\pgfqpoint{5.225186in}{1.777193in}}%
\pgfpathclose%
\pgfusepath{fill}%
\end{pgfscope}%
\begin{pgfscope}%
\pgfpathrectangle{\pgfqpoint{0.539299in}{0.078740in}}{\pgfqpoint{7.842520in}{7.842520in}}%
\pgfusepath{clip}%
\pgfsetbuttcap%
\pgfsetroundjoin%
\definecolor{currentfill}{rgb}{0.258965,0.251537,0.524736}%
\pgfsetfillcolor{currentfill}%
\pgfsetlinewidth{0.000000pt}%
\definecolor{currentstroke}{rgb}{0.896320,0.893616,0.096335}%
\pgfsetstrokecolor{currentstroke}%
\pgfsetdash{}{0pt}%
\pgfpathmoveto{\pgfqpoint{4.909107in}{1.295428in}}%
\pgfpathlineto{\pgfqpoint{4.994083in}{1.662022in}}%
\pgfpathlineto{\pgfqpoint{4.847314in}{1.779621in}}%
\pgfpathclose%
\pgfusepath{fill}%
\end{pgfscope}%
\begin{pgfscope}%
\pgfpathrectangle{\pgfqpoint{0.539299in}{0.078740in}}{\pgfqpoint{7.842520in}{7.842520in}}%
\pgfusepath{clip}%
\pgfsetbuttcap%
\pgfsetroundjoin%
\definecolor{currentfill}{rgb}{0.280868,0.160771,0.472899}%
\pgfsetfillcolor{currentfill}%
\pgfsetlinewidth{0.000000pt}%
\definecolor{currentstroke}{rgb}{0.906311,0.894855,0.098125}%
\pgfsetstrokecolor{currentstroke}%
\pgfsetdash{}{0pt}%
\pgfpathmoveto{\pgfqpoint{5.434633in}{1.223811in}}%
\pgfpathlineto{\pgfqpoint{5.662874in}{1.144010in}}%
\pgfpathlineto{\pgfqpoint{5.517631in}{1.390496in}}%
\pgfpathclose%
\pgfusepath{fill}%
\end{pgfscope}%
\begin{pgfscope}%
\pgfpathrectangle{\pgfqpoint{0.539299in}{0.078740in}}{\pgfqpoint{7.842520in}{7.842520in}}%
\pgfusepath{clip}%
\pgfsetbuttcap%
\pgfsetroundjoin%
\definecolor{currentfill}{rgb}{0.231674,0.318106,0.544834}%
\pgfsetfillcolor{currentfill}%
\pgfsetlinewidth{0.000000pt}%
\definecolor{currentstroke}{rgb}{0.916242,0.896091,0.100717}%
\pgfsetstrokecolor{currentstroke}%
\pgfsetdash{}{0pt}%
\pgfpathmoveto{\pgfqpoint{4.847314in}{1.779621in}}%
\pgfpathlineto{\pgfqpoint{4.994083in}{1.662022in}}%
\pgfpathlineto{\pgfqpoint{4.931945in}{2.086920in}}%
\pgfpathclose%
\pgfusepath{fill}%
\end{pgfscope}%
\begin{pgfscope}%
\pgfpathrectangle{\pgfqpoint{0.539299in}{0.078740in}}{\pgfqpoint{7.842520in}{7.842520in}}%
\pgfusepath{clip}%
\pgfsetbuttcap%
\pgfsetroundjoin%
\definecolor{currentfill}{rgb}{0.223925,0.334994,0.548053}%
\pgfsetfillcolor{currentfill}%
\pgfsetlinewidth{0.000000pt}%
\definecolor{currentstroke}{rgb}{0.926106,0.897330,0.104071}%
\pgfsetstrokecolor{currentstroke}%
\pgfsetdash{}{0pt}%
\pgfpathmoveto{\pgfqpoint{4.994083in}{1.662022in}}%
\pgfpathlineto{\pgfqpoint{5.078583in}{1.938515in}}%
\pgfpathlineto{\pgfqpoint{4.931945in}{2.086920in}}%
\pgfpathclose%
\pgfusepath{fill}%
\end{pgfscope}%
\begin{pgfscope}%
\pgfpathrectangle{\pgfqpoint{0.539299in}{0.078740in}}{\pgfqpoint{7.842520in}{7.842520in}}%
\pgfusepath{clip}%
\pgfsetbuttcap%
\pgfsetroundjoin%
\definecolor{currentfill}{rgb}{0.265145,0.232956,0.516599}%
\pgfsetfillcolor{currentfill}%
\pgfsetlinewidth{0.000000pt}%
\definecolor{currentstroke}{rgb}{0.935904,0.898570,0.108131}%
\pgfsetstrokecolor{currentstroke}%
\pgfsetdash{}{0pt}%
\pgfpathmoveto{\pgfqpoint{4.909107in}{1.295428in}}%
\pgfpathlineto{\pgfqpoint{5.140982in}{1.534035in}}%
\pgfpathlineto{\pgfqpoint{4.994083in}{1.662022in}}%
\pgfpathclose%
\pgfusepath{fill}%
\end{pgfscope}%
\begin{pgfscope}%
\pgfpathrectangle{\pgfqpoint{0.539299in}{0.078740in}}{\pgfqpoint{7.842520in}{7.842520in}}%
\pgfusepath{clip}%
\pgfsetbuttcap%
\pgfsetroundjoin%
\definecolor{currentfill}{rgb}{0.283187,0.125848,0.444960}%
\pgfsetfillcolor{currentfill}%
\pgfsetlinewidth{0.000000pt}%
\definecolor{currentstroke}{rgb}{0.945636,0.899815,0.112838}%
\pgfsetstrokecolor{currentstroke}%
\pgfsetdash{}{0pt}%
\pgfpathmoveto{\pgfqpoint{5.434633in}{1.223811in}}%
\pgfpathlineto{\pgfqpoint{5.580846in}{1.019988in}}%
\pgfpathlineto{\pgfqpoint{5.662874in}{1.144010in}}%
\pgfpathclose%
\pgfusepath{fill}%
\end{pgfscope}%
\begin{pgfscope}%
\pgfpathrectangle{\pgfqpoint{0.539299in}{0.078740in}}{\pgfqpoint{7.842520in}{7.842520in}}%
\pgfusepath{clip}%
\pgfsetbuttcap%
\pgfsetroundjoin%
\definecolor{currentfill}{rgb}{0.276194,0.190074,0.493001}%
\pgfsetfillcolor{currentfill}%
\pgfsetlinewidth{0.000000pt}%
\definecolor{currentstroke}{rgb}{0.955300,0.901065,0.118128}%
\pgfsetstrokecolor{currentstroke}%
\pgfsetdash{}{0pt}%
\pgfpathmoveto{\pgfqpoint{5.056074in}{1.206164in}}%
\pgfpathlineto{\pgfqpoint{5.140982in}{1.534035in}}%
\pgfpathlineto{\pgfqpoint{4.909107in}{1.295428in}}%
\pgfpathclose%
\pgfusepath{fill}%
\end{pgfscope}%
\begin{pgfscope}%
\pgfpathrectangle{\pgfqpoint{0.539299in}{0.078740in}}{\pgfqpoint{7.842520in}{7.842520in}}%
\pgfusepath{clip}%
\pgfsetbuttcap%
\pgfsetroundjoin%
\definecolor{currentfill}{rgb}{0.267968,0.223549,0.512008}%
\pgfsetfillcolor{currentfill}%
\pgfsetlinewidth{0.000000pt}%
\definecolor{currentstroke}{rgb}{0.964894,0.902323,0.123941}%
\pgfsetstrokecolor{currentstroke}%
\pgfsetdash{}{0pt}%
\pgfpathmoveto{\pgfqpoint{5.517631in}{1.390496in}}%
\pgfpathlineto{\pgfqpoint{5.371612in}{1.597343in}}%
\pgfpathlineto{\pgfqpoint{5.287897in}{1.390668in}}%
\pgfpathclose%
\pgfusepath{fill}%
\end{pgfscope}%
\begin{pgfscope}%
\pgfpathrectangle{\pgfqpoint{0.539299in}{0.078740in}}{\pgfqpoint{7.842520in}{7.842520in}}%
\pgfusepath{clip}%
\pgfsetbuttcap%
\pgfsetroundjoin%
\definecolor{currentfill}{rgb}{0.239346,0.300855,0.540844}%
\pgfsetfillcolor{currentfill}%
\pgfsetlinewidth{0.000000pt}%
\definecolor{currentstroke}{rgb}{0.974417,0.903590,0.130215}%
\pgfsetstrokecolor{currentstroke}%
\pgfsetdash{}{0pt}%
\pgfpathmoveto{\pgfqpoint{5.078583in}{1.938515in}}%
\pgfpathlineto{\pgfqpoint{5.140982in}{1.534035in}}%
\pgfpathlineto{\pgfqpoint{5.225186in}{1.777193in}}%
\pgfpathclose%
\pgfusepath{fill}%
\end{pgfscope}%
\begin{pgfscope}%
\pgfpathrectangle{\pgfqpoint{0.539299in}{0.078740in}}{\pgfqpoint{7.842520in}{7.842520in}}%
\pgfusepath{clip}%
\pgfsetbuttcap%
\pgfsetroundjoin%
\definecolor{currentfill}{rgb}{0.243113,0.292092,0.538516}%
\pgfsetfillcolor{currentfill}%
\pgfsetlinewidth{0.000000pt}%
\definecolor{currentstroke}{rgb}{0.983868,0.904867,0.136897}%
\pgfsetstrokecolor{currentstroke}%
\pgfsetdash{}{0pt}%
\pgfpathmoveto{\pgfqpoint{4.994083in}{1.662022in}}%
\pgfpathlineto{\pgfqpoint{5.140982in}{1.534035in}}%
\pgfpathlineto{\pgfqpoint{5.078583in}{1.938515in}}%
\pgfpathclose%
\pgfusepath{fill}%
\end{pgfscope}%
\begin{pgfscope}%
\pgfpathrectangle{\pgfqpoint{0.539299in}{0.078740in}}{\pgfqpoint{7.842520in}{7.842520in}}%
\pgfusepath{clip}%
\pgfsetbuttcap%
\pgfsetroundjoin%
\definecolor{currentfill}{rgb}{0.255645,0.260703,0.528312}%
\pgfsetfillcolor{currentfill}%
\pgfsetlinewidth{0.000000pt}%
\definecolor{currentstroke}{rgb}{0.993248,0.906157,0.143936}%
\pgfsetstrokecolor{currentstroke}%
\pgfsetdash{}{0pt}%
\pgfpathmoveto{\pgfqpoint{5.287897in}{1.390668in}}%
\pgfpathlineto{\pgfqpoint{5.371612in}{1.597343in}}%
\pgfpathlineto{\pgfqpoint{5.225186in}{1.777193in}}%
\pgfpathclose%
\pgfusepath{fill}%
\end{pgfscope}%
\begin{pgfscope}%
\pgfpathrectangle{\pgfqpoint{0.539299in}{0.078740in}}{\pgfqpoint{7.842520in}{7.842520in}}%
\pgfusepath{clip}%
\pgfsetbuttcap%
\pgfsetroundjoin%
\definecolor{currentfill}{rgb}{0.276194,0.190074,0.493001}%
\pgfsetfillcolor{currentfill}%
\pgfsetlinewidth{0.000000pt}%
\definecolor{currentstroke}{rgb}{0.267004,0.004874,0.329415}%
\pgfsetstrokecolor{currentstroke}%
\pgfsetdash{}{0pt}%
\pgfpathmoveto{\pgfqpoint{5.287897in}{1.390668in}}%
\pgfpathlineto{\pgfqpoint{5.434633in}{1.223811in}}%
\pgfpathlineto{\pgfqpoint{5.517631in}{1.390496in}}%
\pgfpathclose%
\pgfusepath{fill}%
\end{pgfscope}%
\begin{pgfscope}%
\pgfpathrectangle{\pgfqpoint{0.539299in}{0.078740in}}{\pgfqpoint{7.842520in}{7.842520in}}%
\pgfusepath{clip}%
\pgfsetbuttcap%
\pgfsetroundjoin%
\definecolor{currentfill}{rgb}{0.257322,0.256130,0.526563}%
\pgfsetfillcolor{currentfill}%
\pgfsetlinewidth{0.000000pt}%
\definecolor{currentstroke}{rgb}{0.268510,0.009605,0.335427}%
\pgfsetstrokecolor{currentstroke}%
\pgfsetdash{}{0pt}%
\pgfpathmoveto{\pgfqpoint{5.225186in}{1.777193in}}%
\pgfpathlineto{\pgfqpoint{5.140982in}{1.534035in}}%
\pgfpathlineto{\pgfqpoint{5.287897in}{1.390668in}}%
\pgfpathclose%
\pgfusepath{fill}%
\end{pgfscope}%
\begin{pgfscope}%
\pgfpathrectangle{\pgfqpoint{0.539299in}{0.078740in}}{\pgfqpoint{7.842520in}{7.842520in}}%
\pgfusepath{clip}%
\pgfsetbuttcap%
\pgfsetroundjoin%
\definecolor{currentfill}{rgb}{0.273006,0.204520,0.501721}%
\pgfsetfillcolor{currentfill}%
\pgfsetlinewidth{0.000000pt}%
\definecolor{currentstroke}{rgb}{0.269944,0.014625,0.341379}%
\pgfsetstrokecolor{currentstroke}%
\pgfsetdash{}{0pt}%
\pgfpathmoveto{\pgfqpoint{5.287897in}{1.390668in}}%
\pgfpathlineto{\pgfqpoint{5.140982in}{1.534035in}}%
\pgfpathlineto{\pgfqpoint{5.056074in}{1.206164in}}%
\pgfpathclose%
\pgfusepath{fill}%
\end{pgfscope}%
\begin{pgfscope}%
\pgfpathrectangle{\pgfqpoint{0.539299in}{0.078740in}}{\pgfqpoint{7.842520in}{7.842520in}}%
\pgfusepath{clip}%
\pgfsetbuttcap%
\pgfsetroundjoin%
\definecolor{currentfill}{rgb}{0.280255,0.165693,0.476498}%
\pgfsetfillcolor{currentfill}%
\pgfsetlinewidth{0.000000pt}%
\definecolor{currentstroke}{rgb}{0.271305,0.019942,0.347269}%
\pgfsetstrokecolor{currentstroke}%
\pgfsetdash{}{0pt}%
\pgfpathmoveto{\pgfqpoint{5.056074in}{1.206164in}}%
\pgfpathlineto{\pgfqpoint{5.203294in}{1.107929in}}%
\pgfpathlineto{\pgfqpoint{5.287897in}{1.390668in}}%
\pgfpathclose%
\pgfusepath{fill}%
\end{pgfscope}%
\begin{pgfscope}%
\pgfpathrectangle{\pgfqpoint{0.539299in}{0.078740in}}{\pgfqpoint{7.842520in}{7.842520in}}%
\pgfusepath{clip}%
\pgfsetbuttcap%
\pgfsetroundjoin%
\definecolor{currentfill}{rgb}{0.280868,0.160771,0.472899}%
\pgfsetfillcolor{currentfill}%
\pgfsetlinewidth{0.000000pt}%
\definecolor{currentstroke}{rgb}{0.272594,0.025563,0.353093}%
\pgfsetstrokecolor{currentstroke}%
\pgfsetdash{}{0pt}%
\pgfpathmoveto{\pgfqpoint{5.350654in}{0.995313in}}%
\pgfpathlineto{\pgfqpoint{5.434633in}{1.223811in}}%
\pgfpathlineto{\pgfqpoint{5.287897in}{1.390668in}}%
\pgfpathclose%
\pgfusepath{fill}%
\end{pgfscope}%
\begin{pgfscope}%
\pgfpathrectangle{\pgfqpoint{0.539299in}{0.078740in}}{\pgfqpoint{7.842520in}{7.842520in}}%
\pgfusepath{clip}%
\pgfsetbuttcap%
\pgfsetroundjoin%
\definecolor{currentfill}{rgb}{0.283091,0.110553,0.431554}%
\pgfsetfillcolor{currentfill}%
\pgfsetlinewidth{0.000000pt}%
\definecolor{currentstroke}{rgb}{0.273809,0.031497,0.358853}%
\pgfsetstrokecolor{currentstroke}%
\pgfsetdash{}{0pt}%
\pgfpathmoveto{\pgfqpoint{5.497926in}{0.857909in}}%
\pgfpathlineto{\pgfqpoint{5.580846in}{1.019988in}}%
\pgfpathlineto{\pgfqpoint{5.434633in}{1.223811in}}%
\pgfpathclose%
\pgfusepath{fill}%
\end{pgfscope}%
\begin{pgfscope}%
\pgfpathrectangle{\pgfqpoint{0.539299in}{0.078740in}}{\pgfqpoint{7.842520in}{7.842520in}}%
\pgfusepath{clip}%
\pgfsetbuttcap%
\pgfsetroundjoin%
\definecolor{currentfill}{rgb}{0.281887,0.150881,0.465405}%
\pgfsetfillcolor{currentfill}%
\pgfsetlinewidth{0.000000pt}%
\definecolor{currentstroke}{rgb}{0.274952,0.037752,0.364543}%
\pgfsetstrokecolor{currentstroke}%
\pgfsetdash{}{0pt}%
\pgfpathmoveto{\pgfqpoint{5.203294in}{1.107929in}}%
\pgfpathlineto{\pgfqpoint{5.350654in}{0.995313in}}%
\pgfpathlineto{\pgfqpoint{5.287897in}{1.390668in}}%
\pgfpathclose%
\pgfusepath{fill}%
\end{pgfscope}%
\begin{pgfscope}%
\pgfpathrectangle{\pgfqpoint{0.539299in}{0.078740in}}{\pgfqpoint{7.842520in}{7.842520in}}%
\pgfusepath{clip}%
\pgfsetbuttcap%
\pgfsetroundjoin%
\definecolor{currentfill}{rgb}{0.283091,0.110553,0.431554}%
\pgfsetfillcolor{currentfill}%
\pgfsetlinewidth{0.000000pt}%
\definecolor{currentstroke}{rgb}{0.276022,0.044167,0.370164}%
\pgfsetstrokecolor{currentstroke}%
\pgfsetdash{}{0pt}%
\pgfpathmoveto{\pgfqpoint{5.434633in}{1.223811in}}%
\pgfpathlineto{\pgfqpoint{5.350654in}{0.995313in}}%
\pgfpathlineto{\pgfqpoint{5.497926in}{0.857909in}}%
\pgfpathclose%
\pgfusepath{fill}%
\end{pgfscope}%
\begin{pgfscope}%
\pgfsetbuttcap%
\pgfsetmiterjoin%
\definecolor{currentfill}{rgb}{1.000000,1.000000,1.000000}%
\pgfsetfillcolor{currentfill}%
\pgfsetlinewidth{0.000000pt}%
\definecolor{currentstroke}{rgb}{0.000000,0.000000,0.000000}%
\pgfsetstrokecolor{currentstroke}%
\pgfsetstrokeopacity{0.000000}%
\pgfsetdash{}{0pt}%
\pgfpathmoveto{\pgfqpoint{11.180860in}{0.157480in}}%
\pgfpathlineto{\pgfqpoint{11.495820in}{0.157480in}}%
\pgfpathlineto{\pgfqpoint{11.495820in}{7.842520in}}%
\pgfpathlineto{\pgfqpoint{11.180860in}{7.842520in}}%
\pgfpathclose%
\pgfusepath{fill}%
\end{pgfscope}%
\begin{pgfscope}%
\pgfpathrectangle{\pgfqpoint{11.180860in}{0.157480in}}{\pgfqpoint{0.314961in}{7.685039in}}%
\pgfusepath{clip}%
\pgfsetbuttcap%
\pgfsetmiterjoin%
\definecolor{currentfill}{rgb}{1.000000,1.000000,1.000000}%
\pgfsetfillcolor{currentfill}%
\pgfsetlinewidth{0.010037pt}%
\definecolor{currentstroke}{rgb}{1.000000,1.000000,1.000000}%
\pgfsetstrokecolor{currentstroke}%
\pgfsetdash{}{0pt}%
\pgfpathmoveto{\pgfqpoint{11.180860in}{0.157480in}}%
\pgfpathlineto{\pgfqpoint{11.180860in}{0.187500in}}%
\pgfpathlineto{\pgfqpoint{11.180860in}{7.812500in}}%
\pgfpathlineto{\pgfqpoint{11.180860in}{7.842520in}}%
\pgfpathlineto{\pgfqpoint{11.495820in}{7.842520in}}%
\pgfpathlineto{\pgfqpoint{11.495820in}{7.812500in}}%
\pgfpathlineto{\pgfqpoint{11.495820in}{0.187500in}}%
\pgfpathlineto{\pgfqpoint{11.495820in}{0.157480in}}%
\pgfpathlineto{\pgfqpoint{11.495820in}{0.157480in}}%
\pgfpathclose%
\pgfusepath{stroke,fill}%
\end{pgfscope}%
\begin{pgfscope}%
\pgfsys@transformshift{11.180000in}{0.160000in}%
\pgftext[left,bottom]{\includegraphics[interpolate=true,width=0.320000in,height=7.680000in]{unknown1-img0.png}}%
\end{pgfscope}%
\begin{pgfscope}%
\pgfsetbuttcap%
\pgfsetroundjoin%
\definecolor{currentfill}{rgb}{0.000000,0.000000,0.000000}%
\pgfsetfillcolor{currentfill}%
\pgfsetlinewidth{0.501875pt}%
\definecolor{currentstroke}{rgb}{0.000000,0.000000,0.000000}%
\pgfsetstrokecolor{currentstroke}%
\pgfsetdash{}{0pt}%
\pgfsys@defobject{currentmarker}{\pgfqpoint{-0.034722in}{0.000000in}}{\pgfqpoint{-0.000000in}{0.000000in}}{%
\pgfpathmoveto{\pgfqpoint{-0.000000in}{0.000000in}}%
\pgfpathlineto{\pgfqpoint{-0.034722in}{0.000000in}}%
\pgfusepath{stroke,fill}%
}%
\begin{pgfscope}%
\pgfsys@transformshift{11.495820in}{1.494658in}%
\pgfsys@useobject{currentmarker}{}%
\end{pgfscope}%
\end{pgfscope}%
\begin{pgfscope}%
\definecolor{textcolor}{rgb}{0.980392,0.811765,0.352941}%
\pgfsetstrokecolor{textcolor}%
\pgfsetfillcolor{textcolor}%
\pgftext[x=11.044431in, y=1.452449in, right, base]{\color{textcolor}\sffamily\fontsize{18.000000}{9.600000}\selectfont $\displaystyle 0.005$}%
\end{pgfscope}%
\begin{pgfscope}%
\pgfsetbuttcap%
\pgfsetroundjoin%
\definecolor{currentfill}{rgb}{0.000000,0.000000,0.000000}%
\pgfsetfillcolor{currentfill}%
\pgfsetlinewidth{0.501875pt}%
\definecolor{currentstroke}{rgb}{0.000000,0.000000,0.000000}%
\pgfsetstrokecolor{currentstroke}%
\pgfsetdash{}{0pt}%
\pgfsys@defobject{currentmarker}{\pgfqpoint{-0.034722in}{0.000000in}}{\pgfqpoint{-0.000000in}{0.000000in}}{%
\pgfpathmoveto{\pgfqpoint{-0.000000in}{0.000000in}}%
\pgfpathlineto{\pgfqpoint{-0.034722in}{0.000000in}}%
\pgfusepath{stroke,fill}%
}%
\begin{pgfscope}%
\pgfsys@transformshift{11.495820in}{2.832718in}%
\pgfsys@useobject{currentmarker}{}%
\end{pgfscope}%
\end{pgfscope}%
\begin{pgfscope}%
\definecolor{textcolor}{rgb}{0.980392,0.811765,0.352941}%
\pgfsetstrokecolor{textcolor}%
\pgfsetfillcolor{textcolor}%
\pgftext[x=11.044431in, y=2.790509in, right, base]{\color{textcolor}\sffamily\fontsize{18.000000}{9.600000}\selectfont $\displaystyle 0.010$}%
\end{pgfscope}%
\begin{pgfscope}%
\pgfsetbuttcap%
\pgfsetroundjoin%
\definecolor{currentfill}{rgb}{0.000000,0.000000,0.000000}%
\pgfsetfillcolor{currentfill}%
\pgfsetlinewidth{0.501875pt}%
\definecolor{currentstroke}{rgb}{0.000000,0.000000,0.000000}%
\pgfsetstrokecolor{currentstroke}%
\pgfsetdash{}{0pt}%
\pgfsys@defobject{currentmarker}{\pgfqpoint{-0.034722in}{0.000000in}}{\pgfqpoint{-0.000000in}{0.000000in}}{%
\pgfpathmoveto{\pgfqpoint{-0.000000in}{0.000000in}}%
\pgfpathlineto{\pgfqpoint{-0.034722in}{0.000000in}}%
\pgfusepath{stroke,fill}%
}%
\begin{pgfscope}%
\pgfsys@transformshift{11.495820in}{4.170779in}%
\pgfsys@useobject{currentmarker}{}%
\end{pgfscope}%
\end{pgfscope}%
\begin{pgfscope}%
\definecolor{textcolor}{rgb}{0.980392,0.811765,0.352941}%
\pgfsetstrokecolor{textcolor}%
\pgfsetfillcolor{textcolor}%
\pgftext[x=11.044431in, y=4.128569in, right, base]{\color{textcolor}\sffamily\fontsize{18.000000}{9.600000}\selectfont $\displaystyle 0.015$}%
\end{pgfscope}%
\begin{pgfscope}%
\pgfsetbuttcap%
\pgfsetroundjoin%
\definecolor{currentfill}{rgb}{0.000000,0.000000,0.000000}%
\pgfsetfillcolor{currentfill}%
\pgfsetlinewidth{0.501875pt}%
\definecolor{currentstroke}{rgb}{0.000000,0.000000,0.000000}%
\pgfsetstrokecolor{currentstroke}%
\pgfsetdash{}{0pt}%
\pgfsys@defobject{currentmarker}{\pgfqpoint{-0.034722in}{0.000000in}}{\pgfqpoint{-0.000000in}{0.000000in}}{%
\pgfpathmoveto{\pgfqpoint{-0.000000in}{0.000000in}}%
\pgfpathlineto{\pgfqpoint{-0.034722in}{0.000000in}}%
\pgfusepath{stroke,fill}%
}%
\begin{pgfscope}%
\pgfsys@transformshift{11.495820in}{5.508839in}%
\pgfsys@useobject{currentmarker}{}%
\end{pgfscope}%
\end{pgfscope}%
\begin{pgfscope}%
\definecolor{textcolor}{rgb}{0.980392,0.811765,0.352941}%
\pgfsetstrokecolor{textcolor}%
\pgfsetfillcolor{textcolor}%
\pgftext[x=11.044431in, y=5.466629in, right, base]{\color{textcolor}\sffamily\fontsize{18.000000}{9.600000}\selectfont $\displaystyle 0.020$}%
\end{pgfscope}%
\begin{pgfscope}%
\pgfsetbuttcap%
\pgfsetroundjoin%
\definecolor{currentfill}{rgb}{0.000000,0.000000,0.000000}%
\pgfsetfillcolor{currentfill}%
\pgfsetlinewidth{0.501875pt}%
\definecolor{currentstroke}{rgb}{0.000000,0.000000,0.000000}%
\pgfsetstrokecolor{currentstroke}%
\pgfsetdash{}{0pt}%
\pgfsys@defobject{currentmarker}{\pgfqpoint{-0.034722in}{0.000000in}}{\pgfqpoint{-0.000000in}{0.000000in}}{%
\pgfpathmoveto{\pgfqpoint{-0.000000in}{0.000000in}}%
\pgfpathlineto{\pgfqpoint{-0.034722in}{0.000000in}}%
\pgfusepath{stroke,fill}%
}%
\begin{pgfscope}%
\pgfsys@transformshift{11.495820in}{6.846899in}%
\pgfsys@useobject{currentmarker}{}%
\end{pgfscope}%
\end{pgfscope}%
\begin{pgfscope}%
\definecolor{textcolor}{rgb}{0.980392,0.811765,0.352941}%
\pgfsetstrokecolor{textcolor}%
\pgfsetfillcolor{textcolor}%
\pgftext[x=11.044431in, y=6.804689in, right, base]{\color{textcolor}\sffamily\fontsize{18.000000}{9.600000}\selectfont $\displaystyle 0.025$}%
\end{pgfscope}%
\begin{pgfscope}%
\pgfsetrectcap%
\pgfsetmiterjoin%
\pgfsetlinewidth{1.003750pt}%
\definecolor{currentstroke}{rgb}{0.000000,0.000000,0.000000}%
\pgfsetstrokecolor{currentstroke}%
\pgfsetdash{}{0pt}%
\pgfpathmoveto{\pgfqpoint{11.180860in}{0.157480in}}%
\pgfpathlineto{\pgfqpoint{11.180860in}{0.187500in}}%
\pgfpathlineto{\pgfqpoint{11.180860in}{7.812500in}}%
\pgfpathlineto{\pgfqpoint{11.180860in}{7.842520in}}%
\pgfpathlineto{\pgfqpoint{11.495820in}{7.842520in}}%
\pgfpathlineto{\pgfqpoint{11.495820in}{7.812500in}}%
\pgfpathlineto{\pgfqpoint{11.495820in}{0.187500in}}%
\pgfpathlineto{\pgfqpoint{11.495820in}{0.157480in}}%
\pgfpathclose%
\pgfusepath{stroke}%
\end{pgfscope}%
\end{pgfpicture}%
\makeatother%
\endgroup%
}
% 	\caption{$h = 2^{-5}$ 时差分逼近解}\label{fig:explicit2}
% \end{figure}

\subsection{隐式格式}

仍考虑问题 \eqref{solu}. 由隐式差分格式$\(\ref{2.2.44}\)$以及边值条件可得
\begin{subequations}
	\begin{empheq}[left = {\empheqlbrace}]{align}
		-\mu U_{j-1}^{m+1} +\(1+2\mu\) U_{j+1}^{m+1}-\mu U_{j+1}^{m+1} &=U_j^m& 1 &\leq j\leq N-1,& m&\geq 0,\nonumber\\
		U_j^0 &=\sin\(jh\pi\), &0 &\leq j\leq N,&&\nonumber\\
		U_0^m &=U_N^m=0, &&&m&\geq 1.\nonumber
	\end{empheq}
\end{subequations}
写为矩阵形式
\begin{equation*}
	AU^{m+1} = U^m, \quad m\geq 0, 
\end{equation*}
其中
\begin{equation*}
	A = \begin{pmatrix}
		1+2\mu&-\mu&&&\\
		-\mu&1+2\mu&-\mu &&\\
		&\ddots &\ddots&\ddots&\\
		&&-\mu &1+2\mu&-\mu \\
		&&&-\mu &1+2\mu
	\end{pmatrix},\quad U^m = \(U_1^m, \cdots, U_{N-1}^m\)^\T. 
\end{equation*}

取定$t_{\max} = 1.0$. 令$\mu = 0.25$, 利用 LU 分解求解方程, 求出误差阶如表 \ref{tab:imexpliciterr2} 所示.
\begin{table}[H]\centering\heiti\zihao{-5}
	\caption{隐式格式不同步长时的 $\mathbb{L}^2$, $\mathbb{L}^\infty$ 误差及收敛阶}\label{tab:imexpliciterr2}
	\begin{tabular}{|c|c|c|c|c|}\hline
		收敛阶	&	$\mathbb{L}^2$ 误差	&	$h$	&	$\mathbb{L}^\infty$ 误差		&	收敛阶\\\hline
			&	$7.7496 \times 10^{-3}$	&	$2^3$	&	$1.1624 \times 10^{-2}$	&	\\\hline
		$1.9409$	&	$2.0185 \times 10^{-3}$	&	$2^4$	&	$2.9424 \times 10^{-3}$	&	$1.9821$\\\hline
		$1.9739$	&	$5.1382 \times 10^{-4}$	&	$2^5$	&	$7.3792 \times 10^{-4}$	&	$1.9955$\\\hline
		$1.9879$	&	$1.2954 \times 10^{-4}$	&	$2^6$	&	$1.8463 \times 10^{-4}$	&	$1.9989$\\\hline
		$1.9942$	&	$3.2517 \times 10^{-5}$	&	$2^7$	&	$4.6165 \times 10^{-5}$	&	$1.9997$\\\hline
		$1.9971$	&	$8.1455 \times 10^{-6}$	&	$2^8$	&	$1.1542 \times 10^{-5}$	&	$1.9999$\\\hline
		$1.9986$	&	$2.0384 \times 10^{-6}$	&	$2^9$	&	$2.8855 \times 10^{-6}$	&	$2.0000$\\\hline
		$1.9992$	&	$5.0986 \times 10^{-7}$	&	$2^{10}$	&	$7.2141 \times 10^{-7}$	&	$2.0000$\\\hline
		$1.9989$	&	$1.2757 \times 10^{-7}$	&	$2^{11}$	&	$1.8045 \times 10^{-7}$	&	$1.9992$\\\hline
	\end{tabular}
\end{table}

由数值结果可以看出解序列逐步收敛到模型问题 \eqref{solu} 的解, 且收敛阶趋于$2$, 与理论结果相符.
%   数值解$U_{i,j}$与$u_{i,j}$在$t = t_{\max}$时刻图像如下所示, 

\subsection{Crank-Nicolson 格式}

仍考虑问题 \eqref{solu}. 由 Crank-Nicolson 格式以及边值条件可得
$$\lb\begin{aligned}
	\(1+\mu\)U_j^{m+1} &= \(1-\mu\)U_j^m &&&&\\
	&\quad +\frac{\mu}{2}\(U_{j-1}^m +U_{j+1}^m+U_{j-1}^{m+1}+U_{j+1}^{m+1}\),& 1&\leq j\leq N-1,& m&\geq 0,\\
	U_j^0 &=\sin\(jh\pi\), &0&\leq j\leq N,&&\\
	U_0^m &= U_N^m=0, &&&m&\geq 1.
\end{aligned}\rd$$
写为矩阵形式
\begin{equation*}
	AU^{m+1} = BU^m,\quad U^m = \(U_1^m, \cdots, U_{N-1}^m\)^\T, \quad m\geq 0, 
\end{equation*}
其中
\begin{equation*}
	A = \begin{pmatrix}
		1+\mu&-\frac{\mu}{2}&&&\\
		-\frac{\mu}{2}&1+\mu&-\frac{\mu}{2} &&\\
		&\ddots &\ddots&\ddots&\\
		&&-\frac{\mu}{2} &1+\mu&-\frac{\mu}{2} \\
		&&&-\frac{\mu}{2} &1+\mu
	\end{pmatrix},\quad B = \begin{pmatrix}
	1-\mu&\frac{\mu}{2}&&&\\
	\frac{\mu}{2}&1-\mu&\frac{\mu}{2} &&\\
	&\ddots &\ddots&\ddots&\\
	&&\frac{\mu}{2} &1-\mu&\frac{\mu}{2} \\
	&&&\frac{\mu}{2} &1-\mu
\end{pmatrix}.
\end{equation*}

\subsubsection{稳定性条件成立}

取 $t_{\max} =1$. 令$\mu = 0.25 \leq 1$, 满足$\mathbb{L}^{\infty}$, $\mathbb{L}^2$稳定性条件. 利用 LU 分解求解方程, 求出误差阶如表 \ref{tab:cnerr1} 所示.
\begin{table}[H]\centering\heiti\zihao{-5}
	\caption{Crank-Nicolson 格式不同步长时的 $\mathbb{L}^2$, $\mathbb{L}^\infty$ 误差及收敛阶}\label{tab:cnerr1}
	\begin{tabular}{|c|c|c|c|c|}\hline
		收敛阶	&	$\mathbb{L}^2$ 误差	&	$h$	&	$\mathbb{L}^\infty$ 误差		&	收敛阶\\\hline
		&	$3.1262 \times 10^{-3}$	&	$2^3$		&	$4.6892 \times 10^{-3}$	&	\\\hline
		$1.9500$	&	$8.0910 \times 10^{-4}$	&	$2^4$		&	$1.1795 \times 10^{-3}$	&	$1.9912$\\\hline
		$1.9762$	&	$2.0564 \times 10^{-4}$	&	$2^5$		&	$2.9532 \times 10^{-4}$	&	$1.9978$\\\hline
		$1.9884$	&	$5.1823 \times 10^{-5}$	&	$2^6$		&	$7.3860 \times 10^{-5}$	&	$1.9994$\\\hline
		$1.9943$	&	$1.3007 \times 10^{-5}$	&	$2^7$		&	$1.8467 \times 10^{-5}$	&	$1.9999$\\\hline
		$1.9972$	&	$3.2582 \times 10^{-6}$	&	$2^8$		&	$4.6168 \times 10^{-6}$	&	$2.0000$\\\hline
		$1.9986$	&	$8.1535 \times 10^{-7}$	&	$2^9$		&	$1.1542 \times 10^{-6}$	&	$2.0000$\\\hline
		$1.9992$	&	$2.0395 \times 10^{-7}$	&	$2^{10}$	&	$2.8857 \times 10^{-7}$	&	$1.9999$\\\hline
		$1.9983$	&	$5.1048 \times 10^{-8}$	&	$2^{11}$	&	$7.2210 \times 10^{-8}$	&	$1.9987$\\\hline
	\end{tabular}
\end{table}
可以看出解序列逐步收敛到模型问题 \eqref{solu} 的解, 且收敛阶趋于$2$, 与理论结果相符.
%   数值解$U_{i,j}$与$u_{i,j}$在$t = t_{\max}$时刻图像如下所示, 

\subsubsection{稳定性条件不成立}

在数值实验中, 取定$t_{\max} =1$. 令$\mu =10> 1$, 不满足$\mathbb{L}^{\infty}$稳定性条件, 满足$\mathbb{L}^2$稳定性条件. 利用 LU 分解求解方程, 求出误差阶如表 \ref{tab:cnerr2} 所示.
\begin{table}[H]\centering\heiti\zihao{-5}
	\caption{Crank-Nicolson 格式不同步长时的 $\mathbb{L}^2$, $\mathbb{L}^\infty$ 误差及收敛阶}\label{tab:cnerr2}
	\begin{tabular}{|c|c|c|c|c|}\hline
		收敛阶	&	$\mathbb{L}^2$ 误差	&	$h$	&	$\mathbb{L}^\infty$ 误差		&	收敛阶\\\hline
		&	$3.9675\times10^{-5}$	&	$2^6$	&	$5.6109\times10^{-5}$	&	\\\hline
		$1.6927$	&	$1.2273\times10^{-5}$	&	$2^7$	&	$1.7357\times10^{-5}$	&	$1.6927$\\\hline
		$1.9325$	&	$3.2154\times10^{-6}$	&	$2^8$	&	$4.5473\times10^{-6}$	&	$1.9325$\\\hline
		$1.9836$	&	$8.1308\times10^{-7}$	&	$2^9$	&	$1.1499\times10^{-6}$	&	$1.9836$\\\hline
		$1.9959$	&	$2.0384\times10^{-7}$	&	$2^{10}$	&	$2.8828\times10^{-7}$	&	$1.9959$\\\hline
		$1.9991$	&	$5.0992\times10^{-8}$	&	$2^{11}$	&	$7.2114\times10^{-8}$	&	$1.9991$\\\hline
	\end{tabular}
\end{table}
可以看出, 对于此问题模型, 解序列逐步收敛到模型问题 \eqref{solu} 的解, 且收敛阶趋于$2$.
%  数值解$U_{i,j}$与$u_{i,j}$在$t = t_{\max}$时刻图像如下所示, 

\section{二维抛物型方程 Crank-Nicolson 格式}

\subsection{模型问题}

考虑 $\Omega := (0, X) \times (0, Y)$ 上的热传导方程的 Dirichlet 初边值问题:
\begin{equation}\label{equ:2d_DirichletProblem}
	\lb\begin{aligned}
		u_t &= a\(u_{xx} + u_{yy}\),	&	(x, y) \in \Omega,&~t > 0,	\\
		u(x, y, 0) &= u^0(x, y),	&	0 \leq x \leq 1,&	\\
		u(x, y, t) &= 0,	&	&~t > 0.
	\end{aligned}\rd
\end{equation}
其中 $a > 0$ 是热扩散系数.

\subsection{差分逼近}

与一维情形类似, 在 $\Omega\times \R_+$ 上引入网格. 任取正整数 $N_x$ 和 $N_y$, 采用均匀时空网格:
\begin{align*}
	h_x &= \D x = \dfrac{X}{N_x},& h_y &= \D y = \dfrac{Y}{N_y},& \D t = \tau,\\
	x_j &= jh,& j &= 0, 1, \cdots, N_x,\\
	y &= kh,& k &= 0, 1, \cdots, N_y,\\
	t_m &= m\tau,& m &= 0, 1, \cdots,
\end{align*}
其网格节点集为
$$J := \lrb{\(x_j, y_k, t_m\):\quad j = 0, 1, \cdots, N_x,\quad k = 0, 1, \cdots, N_y,\quad m = 0, 1, \cdots},$$
常将 $\(x_j, y_k, t_m\)$ 简记为 $(j, k, m)$. 在网格上定义网格函数
$$U_{(h_x, h_y, \tau)} := \lrb{U_{jk}^m:\quad j = 0, 1, \cdots, N_x,\quad k = 0, 1, \cdots, N_y,\quad m = 0, 1, \cdots}$$
模型问题 \eqref{equ:2d_DirichletProblem} 的真解在 $(j, k, m)$ 上的取值记为 $u_{j,k}^m$. 下面使用 Crank-Nicolson 格式求解模型问题, 并分析稳定性, 收敛性和误差.

\begin{comment}
则该二维模型的显式差分格式为
\begin{equation*}
	\lb\begin{array}{r@{~}lr@{~}l}
		\frac{U_{jk}^{m + 1} - U_{jk}^m}{\tau} =& a \(\dfrac{U_{j + 1, k}^{m} - 2U_{jk}^{m} + U_{j - 1, k}^{m}}{h_x^2}\rd &&\\
		&\ld + \dfrac{U_{j, k + 1}^{m} - 2U_{jk}^{m} + U_{j, k - 1}^{m}}{h_y^2}\),&	1&\leq j \leq N_x - 1, 1 \leq k \leq N_y - 1, m\geq 0,\\
		U_{jk}^0 =& u_{jk}^0,&	0 &\leq j \leq N_x, 0 \leq k \leq N_y,\\
		U_{0k}^m =& U_{N_x k}^m = U_{j0}^m = U_{jN_y}^m,&	m&\geq 1.
	\end{array}\rd
\end{equation*}

设 $u$ 充分光滑, 由其在 $(x_j, y_k, t_m)$ 处 Taylor 展开知:
\begin{equation*}
	\lb\begin{array}{r@{~}l}
		\dfrac{u_{jk}^{m + 1} - u_{jk}^{m}}{\tau} &= (u_t)_{jk}^m + \dfrac{1}{2} (u_{tt})_{jk}^m\tau + \cdots,\\
		\dfrac{u_{j + 1, k}^{m} - 2u_{jk}^{m} + u_{j - 1, k}^{m}}{h_x} &= (u_{xx})_{jk}^m + \frac{1}{12} (u_{xxxx})_{jk}^m h_x^2 + \frac{1}{360} \(\frac{\pa^6 u}{\pa x^6}\)_{jk}^m h_x^4 + \cdots,
	\end{array}\rd
\end{equation*}
故局部截断误差为
$$Tu(x, y ,t) = \frac{1}{2}u_{tt} \tau - \frac{1}{12} au_{xxxx}h_x^2 - \frac{1}{12} au_{yyyy}h_y^2 + O(\tau^2 + h_x^4 + h_x^4),$$
即 $Tu(x, y ,t) = O(\tau + h_x^2 + h_x^2)$.

令 $\mu_x = \dfrac{a\tau}{h_x^2}$, $\mu_y = \dfrac{a\tau}{h_y^2}$, 则差分格式可以等价地写为
\begin{equation*}
	\lb\begin{array}{r@{~}l}
		U_{jk}^{m + 1} &= (1 - \mu_x - 2\mu_y)U_{jk}^{m} + \mu_x(U_{j + 1, k}^m + U_{j - 1, k}^m + \mu_y(U_{j, k + 1}^m + U_{j, k - 1}^m),\tag{$*$}\\
		e_{jk}^{m + 1} &= (1 - \mu_x - 2\mu_y)e_{jk}^{m} + \mu_x(e_{j + 1, k}^m + e_{j - 1, k}^m + \mu_y(e_{j, k + 1}^m + e_{j, k - 1}^m) - \tau T_{jk}^m,
	\end{array}\rd
\end{equation*}
易知, $(1 - \mu_x - 2\mu_y) \geq 0$, 即 $\mu_x + \mu_y \leq \dfrac{1}{2}$ 时格式满足最大值原理. 于是, 与一位情形类似地, 可知  $\mu_x + \mu_y \leq \dfrac{1}{2}$ 时格式满足最大值原理.

又将 Fourier 波形 $U_{jk}^m = \lambda_\a^m \e^{\i(\a_x x_j + \a_y y_k)}$ 代入 $*$ 式. 其中 $\a_x = \dfrac{l\pi}{X}$, $\a_y = \dfrac{l\pi}{Y}$, $\a = (\a_x, \a_y)$, $l = 1, \cdots, N_x$. 则
\begin{align*}
	\lambda_\a^{m + 1} \e^{\i(\a_x jh_x + \a_y kh_y)} =~& (1 - \mu_x - 2\mu_y) \lambda_\a^{m} \e^{\i(\a_x jh_x + \a_y kh_y)}\\
	&+ \mu_x \(\lambda_\a^{m} \e^{\i(\a_x (j + 1)h_x + \a_y kh_y)} + \lambda_\a^{m} \e^{\i(\a_x (j - 1)h_x + \a_y kh_y)}\)\\
	&+ \mu_y \(\lambda_\a^{m} \e^{\i(\a_x jh_x + \a_y (k + 1)h_y)} + \lambda_\a^{m} \e^{\i(\a_x jh_x + \a_y (k - 1)h_y)}\)
\end{align*}
得增长因子
$$\lambda_\a = 1 - 4\(\mu_x\sin^2\frac{\a_xh_x}{2} + \mu_y\sin^2\frac{\a_yh_y}{2}\),$$
故 $\mu_x + \mu_y \leq \dfrac{1}{2}$ 时, $|\lambda_\a| \leq 1$, 于是与一维情形类似地,  $\mu_x + \mu_y \leq \dfrac{1}{2}$ 时格式 $\mathbb{L}^2$ 稳定且收敛.

Crank-Nicolson

\end{comment}

\subsection{模型问题的 Crank-Nicolson 格式}

在点 $\(x, y, t + \frac{1}{2}\D t\)$ 用 $\frac{\de_t}{\D t}$, 替换 $\frac{\pa}{\pa t}$, 用 $(x, y, t + \D t)$, $(x, y, t)$ 两点的关于空间的二阶中心差商 $\frac{\de_x^2}{(\D x)^2} + \frac{\de_y^2}{(\D y)^2}$ 的平均值替换 $\frac{\pa^2}{\pa x^2}$, 得到问题 \eqref{equ:2d_DirichletProblem} 的 Crank-Nicolson 格式:
\begin{equation*}
	\lb\begin{array}{r@{~}l@{~}r@{~}l}
		\frac{U_{jk}^{m + 1} - U_{jk}^m}{\tau} &= \frac{a}{2} \(\frac{U_{j + 1, k}^{m} - 2U_{jk}^{m} + U_{j - 1, k}^{m}}{h_x^2} + \frac{U_{j, k + 1}^{m} - 2U_{jk}^{m} + U_{j, k - 1}^{m}}{h_y^2}\rd &	1&\leq j \leq N_x - 1, \\
		&\quad\ld + \frac{U_{j + 1, k}^{m + 1}  - 2U_{jk}^{m + 1}  + U_{j - 1, k}^{m + 1} }{h_x^2} + \frac{U_{j, k + 1}^{m + 1} - 2U_{jk}^{m + 1}  + U_{j, k - 1}^{m + 1} }{h_y^2}\),& 1 &\leq k \leq N_y - 1, m\geq 0,\\
		U_{jk}^0 &= u_{jk}^0,&	0 &\leq j \leq N_x, 0 \leq k \leq N_y,\\
		U_{0k}^m &= U_{N_x k}^m = U_{j0}^m = U_{jN_y}^m = 0,&	m&\geq 1.
	\end{array}\rd
\end{equation*}

\subsection{Crank-Nicolson 格式的截断误差}

引入截断误差算子
$$T_{h, \tau} := \(\dfrac{\de_t}{\tau} - a\dfrac{\de_x^2}{h_x^2} - a\dfrac{\de_y^2}{h_y^2}\) - \(\dfrac{\pa}{\pa t} - a\dfrac{\pa^2}{\pa x^2} - a\dfrac{\pa^2}{\pa y^2}\).$$
设 $u$ 充分光滑, 由其在 $(x_j, y_k, t_{m + \frac{1}{2}})$ 处 Taylor 展开知:
$$T_{jk}^{m + \frac{1}{2}} = -\frac{1}{12}(u_{ttt})_{jk}^{m + \frac{1}{2}} - \frac{1}{12}a (u_{xxxx})_{jk}^{m + \frac{1}{2}} h_x^2 - \frac{1}{12}a (u_{yyyy})_{jk}^{m + \frac{1}{2}} h_y^2 + O(\tau^4 + h_x^{4} + h_y^4),$$
故 $Tu(x, y, t) = O(\tau^2 + h_x^2 + h_y^2)$.

\subsection{Crank-Nicolson 格式的稳定性, 收敛性分析}

令 $\mu_x = \dfrac{a\tau}{h_x^2}$, $\mu_y = \dfrac{a\tau}{h_y^2}$, 则差分格式可等价地写为:
\begin{equation}\label{equ:2d_Dirichlet_CN_Scheme}
	\begin{aligned}
		(1 + \mu_x + 2\mu_y) U_{jk}^{m + 1} &= (1 - \mu_x - 2\mu_y)U_{jk}^{m} \\
		&\quad + \frac{1}{2}\mu_x(U_{j + 1, k}^m + U_{j - 1, k}^m + U_{j + 1, k}^{m + 1} + U_{j - 1, k}^{m + 1}),\\
		&\quad + \frac{1}{2}\mu_y(U_{j, k + 1}^m + U_{j, k - 1}^m + U_{j, k + 1}^{m + 1} + U_{j, k - 1}^{m + 1}),
	\end{aligned}
\end{equation}
故 $1 - \mu_x - \mu_y \geq 0$ 时,  即 $\mu_x + \mu_y \leq 1$ 时满足最大值原理, 于是 Crank-Nicolson 格式在 $\mu_x + \mu_y \leq 1$ 时 $\mathbb{L}^\infty$ 稳定且收敛.

将 Fourier 波形 $U_{jk}^m = \lambda_\a^m \e^{\i(\a_x x_j + \a_y y_k)}$ 代入差分格式 \eqref{equ:2d_Dirichlet_CN_Scheme}, 其中
$$\a_x = \dfrac{l\pi}{X},\quad \a_y = \dfrac{l\pi}{Y},\quad \a = (\a_x, \a_y),\quad l = 1, \cdots, N_x,$$
可得增长因子
$$\lambda_\a = \frac{1 - 2\(\mu_x\sin^2\frac{\a_xh_x}{2} + \mu_y\sin^2\frac{\a_yh_y}{2}\)}{1 + 2\(\mu_x\sin^2\frac{\a_xh_x}{2} + \mu_y\sin^2\frac{\a_yh_y}{2}\)},$$
即 $|\lambda_\a| \leq 1$ 对任意 $\a$ 成立, 即 Crank-Nicolson 格式无条件 $\mathbb{L}^2$ 稳定且收敛.

\begin{appendices}

	\href{https://github.com/Quejiahao/NumericalSolutionOfPartialDifferentialEquations.jl}{数值实验源码}

\end{appendices}

\end{document}
